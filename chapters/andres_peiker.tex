\index[ppl]{Peiker, Andres}

\question{Kuidas sina arvutite juurde said?}

Ilmselt 1984. aastal ja üsna juhuslikult.
Õppisin keskkoolis ja käisin lisaks Tartu 
Ülikoolis\index{Tartu Ülikool} füüsikaloengutel. Ühe loengu lõpus astus mees nimega Otto 
Teller\index[ppl]{Teller, Otto} auditooriumi ette ja ütles, et kes on 
arvutitest huvitatud, võivad natukeseks veel jääda. Üks 
seltskond jäigi ja Otto Teller viis meid Tähe 4 
õppehoonesse\index{Tartu Ülikool!Füüsikahoone}, kus oli kaks Nairi arvutit, vist
Nairi-K\index{Nairi!Nairi-K} ja Nairi-2\index{Nairi!Nairi-2}. 
Otto näitas neid ja ütles, et õhtustel 
aegadel on võimalik käia neid proovimas ja programmeerimas.

\question{Järeldan sellest, et sa oled Tartu poiss.}

Jah, esimesed kakskümmend viis eluaastat elasin ma Tartus. 

\question{Kui sa käisid keskkooliajal Tartu 
Ülikoolis loengutes, siis sul pidi olema reaalainete huvi?}

Ma õppisin Tartu 1. Keskkoolis\index{Tartu 1. Keskkool} 
matemaatika-füüsika eriklassis. Käisin küll olümpiaadidel, aga ei mäleta täpselt, 
kust too ülikooliloengute teema tuli. Füüsika tundus mulle kõige 
põnevam asi üldse, sellepärast saigi loengutes käidud.

\question{Mind paneb imestama, et keegi üldse selle loengu lõpus ära läks. 
Kogu sedalaadi rahvas tundus arvutite vastu huvi tundvat või ei olnud see nii?}

Ei, päris nii ei olnud, ikkagi pooled läksid ära. Esimesel korral, kui 
käisime arvuteid vaatamas, öeldi, millal saaks järgmine kord
tulla, aga siis tuli meid vähem ja lõpuks jäi 
kolm-neli inimest, kes hakkasid seal sageli käima.

\question{Kas see oli mingisugune ring või lihtsalt Otto isetegevus?}

Ma täpselt ei mäleta. Küllap Otto Teller\index[ppl]{Teller, 
Otto} juhendas alguses, mis ja kuidas. Kuidagi me 
tolle AP 
programmeerimiskeelega\index{AP keel}\sidenote{\begin{russian}АП 
(Автоматическое Программирование)\end{russian} oli Nairil kasutusel olnud 
programmeerimiskeel. Kuna Nairi-2 oli väga levinud, muutus AP keel ka 
universaalseks algoritmide kirjeldamise keeleks venekeelses teaduskirjanduses. Keel oli muu 
hulgas kuulus oma mõnevõrra ebasündsa konnotatsiooni omandanud võtmesõnade 
poolest.\label{sisu:apkeel}}, mis Nairide peal oli, tuttavaks saime. Küllap jah läbi 
Telleri.

\question{Kas raamatuid või muud materjali ei olnud abiks?}
Ei, seda ma küll ei mäleta, et oleks olnud.

\question{See on huvitav asjaolu. Ka teised ei suuda meenutada, kuidas nad täpselt
programmeerima õppisid, see oskus lihtsalt tuli. Mida te Nairidega tegite?}

Seal sai teha väga lihtsaid arvutusprogramme. Tollel 
arvutil ju kuvarit ei olnud, ainult elektrooniline kirjutusmasin. 
Kirjutasid programmi, too tuli paberi peale ja oli ainuke eksemplar, mida pidid alles hoidma, sest kui tahtsid programmi parandada, pidid vaatama toda prinditud paberit. Kõige kõvem asi, mille ma 
Nairi-K\index{Nairi!Nairi-K} peal valmis tegin, oli biorütmide arvutamise programm. Need olid tollal 
tähtsad asjad ja neid tehtigi Arvutuskeskuses\index{Tartu Ülikool!Arvutuskeskus}. Minu programm osutus nii
populaarseks, et tollest 
perfolindist tegi keegi koopia, mida lasti Tähe 4 
töötajatele usinasti välja, ilma et mina midagi teadnud oleksin.

\question{Kas need biorütmide algoritmid, mis liikusid, olid mõeldud käsitsi 
arvutamiseks? Kas kuidagi arvutati numbriliste meetoditega siinust?}

See oligi lihtne siinus: pidid sünniaja ütlema ja nendel emotsionaalsetel, füüsilistel, seksuaalsetel ja intellektuaalsetel biorütmidel oli siinuse 
lainepikkus erinev. Programm arvutas elatud päevade arvu ja tolle pealt 
joonistas nood neli siinust välja. Täiesti 
triviaalne asi iseenesest, asi oligi rohkem siinuse joonistamises paberile elektroonilise trükimasinaga. 

\question{Huvitav teema, mis oli toona oluline, aga täna ei hakka keegi inimestele 
biorütme joonistama\ldots}

See oli siis jah väga popp asi ja tundus tolle arvuti jaoks 
jõukohane ülesanne. Mälu oli vist neli kilobaiti ja arvuti ise oli sama suur kui köögimööbel.

\question{Milleks füüsikud seda kasutasid?}

Samamoodi arvutamiseks.

\question{Aga mida nad arvutasid?}

Sellest ei olnud juttu. Nairi-K\index{Nairi!Nairi-K} oli 
väiksem masin, põhiliselt kasutati teises toas olevat Nairi-2\index{Nairi!Nairi-2}. Meie sellele masinale eriti ligi ei saanud, too 
oli rohkem hõivatud ja ka rohkem \emph{advanced}, kuna sel 
olid lindiseadmed -- suured lindikapid, kus magnetlinti 
keerutati. Nairi-2-l ei olnud tavaline elektrooniline kirjutusmasin, 
vaid trumliga printer, mis suutis päris kiiresti paberit 
välja lasta.

\question{Kas arvutitega möllamine oli puhtalt põnev pusimine või 
tundus selle taga midagi sügavamat olema, et see on see, mida sa teha 
tahadki?}

Siis oli see kõik seotud tegelikult sellega, et kuna ma olin 
matemaatika-füüsika eriklassis, siis Andres Jaeger\index[ppl]{Jaeger, 
Andres} ülikoolist andis meile ka kolm aastat programmeerimist. See oli sisuliselt
küll ainult blokkskeemide joonistamine 
paberile, arvuti ligi me ei saanud. Tähe tänaval oli võimalus 
ise järele proovida seda, mida olid paberi peal teinud. 
Toda kooli õppeprogramm ei võimaldanud.

\question{Paberile skeemide joonistamine võib ju lihtsasti huvi ära tappa, kas 
sul ei tapnud?}

Ei, kindlasti mitte, ka blokkskeemide joonistamine 
oli huvitav. Kui Andres tolle ülesande andis, siis ütles, et kes 
suudab kolme \emph{if}'iga teha, on hea, kahe \emph{if}'iga on 
väga hea ja ühega juba kahtlaselt hea. Siis oli eesmärk olemas, et 
pidin ühega ära tegema -- too asi kõnetas mind. Sain õpetaja käest 
kiita ka: \enquote{Viis aastat tagasi oli 
meil ka üks õpilane, kes suutis selle algoritmi niimoodi ära teha, 
väga hea!}.

\question{Ühesõnaga sinu jaoks oli blokkskeemide joonistamine lihtsalt
ülesanne või pusle.}

Jah, absoluutselt. Loomulikult saab tolle ülesande ka jõuga ära teha, aga huvitav oli seda kõige optimaalsemalt lahendada.

\question{Huvitav, et teie koolil polnud tollal ühtegi arvutit. Tartu linna peal ju oli
arvuteid, miks kooliõpilased mõne juures ei käinud?}

Siis ei olnud veel kuskil arvuteid, ainult Tähe tänaval 
nood kaks Nairit. Ülikooli arvutuskeskuses\index{Tartu 
Ülikool!Arvutuskeskus} oli küll ES\index{ES EVM}, aga sellele ei saanud üldse 
ligi, sest operaatorid lõid programmi sisse. Füüsika 
instituudis\index{Füüsika instituut} Riia maantee lõpus oli vist ka 
üks PDP-11\index{PDP-11}.

Umbes tollal sebis Anne Villems\index[ppl]{Villems, Anne} 
ka Apple II-d\index{Apple II} Vanemuise 
tänavale\index{Tartu Ülikool!Vanemuise tänava õppehoone}. Sinna jõudsin ma 
järgmisena pärast Nairisid. 

Ka sinna viis meid ilmselt Otto Teller. Tundsin hiljem isegi
natuke piinlikkust, et tema näitas Tähe masinaid ja siis ma tegelikult 
hülgasin Tähe tänava ega käinud enam tema juures. Vahtisin ainult 
Vanemuise tänaval, sest Apple'id olid palju ägedamad -- ikkagi monitor ja 48
kilobaiti mälu, see oli ulmeliselt kiire.

\question{Kas Apple'il mängimine tuli ka teemaks?}

Jaa, absoluutselt. See oli kindlasti minu 
elu kõige suurem arvutimängude periood. Oleksime kahe klassivennaga peaaegu 
keemiaeksamile hiljaks jäänud, sest siis ei saanud seisu salvestada ja tuli nii kaugele mängida, kui sai. Pidime juba eksamile minema, aga tuli järgmine 
\emph{level}.

\question{Mis mängu te mängisite?}

Apple'il oli selline standardne mäng nagu \enquote{Pacman}\index{Pacman}, mis oli ka mänguautomaatides ja mujal. Apple'il nimetati seda \enquote{Super Puckmaniks}. Pean seda siiamaani kõige lahedamaks mänguks, mida olen kunagi 
mänginud. \enquote{Pacmani} oli kõigi teiste arvutite peal ka, aga tolles algoritmis, kuidas 
neli kolli liikusid, oli katastrofaalne erinevus: kõigi ülejäänud arvutite peal liikusid nad \emph{random}'iga, aga Apple'il oli neil oma kindel 
algoritm. Ja kui ise tegid täpselt ühtemoodi, siis 
situatsioon kordus mängust mängu. Meil olid esimese kuue 
\emph{level}i jaoks sammud algusest lõpuni välja töötatud. Teadsime täpselt, 
kuidas terve ekraan puhtaks mängida ja järgmisele \emph{level}'ile saada. 
Edasi oli paar avangut, mida sai erinevatel 
\emph{level}itel kasutada.

\question{Nii et \enquote{Pacman} oli põhimõtteliselt samasugune nagu male?}

Natuke jah. Ja seetõttu polnud võimalik seda teiste versioonide peal mängida, sest 
seal liikusid kollid \emph{random}'iga. \enquote{Super Pucman} oli jah kindlasti 
kõige olulisem mäng, kuigi oli ka teisi.

\question{Kas programmeerimine huvitas sind ka?}

Jah, muidugi. Alguses 
kirjutasin BASICus\index{BASIC}, pärast valdavalt 
assembleris\index{Assembler}, sest siis töötas programm 
tunduvalt kiiremini. 

\question{Kuidas BASICust assemblerisse hüppamine käis? BASICu võib tõesti 
suhteliselt lihtsasti üles korjata, aga assembleris peab täpselt teadma, 
mida teha.}

Ka BASICu puhul pidi arvuti arhitektuurist aru saama: kus 
tolles 48 kilobaidis paiknes tekstiekraan, kus 
graafiline ekraan, kus su programm ja kus opsüsteem. Tegelikult tekkis arusaam
arvuti arhitektuurist BASICu kõrvalt suhteliselt kiiresti, aga 
assembler tuli tänu sellele, et osa BASICu asju olid väga aeglased. 

Üks asi, 
mida ma seal tegin, oli orienteerumisneljapäevakute protokollid. Tollega 
alustas tegelikult Peep Abel\index[ppl]{Abel, Peep}, kes õppis ülikoolis 
rakendusmatemaatikat ja ülikooli lõpetades andis mulle kogu 
programmikomplekti üle. Aga see oli minu jaoks liiga aeglane: andmemaht 
oli 48 kilobaidi jaoks liiga suur ja oli ka 
mitme flopiga mängimist, et andmebaasid ära mahuksid. Nii ma kirjutansingi \emph{from scratch} kõik assembleris ja asi oli 
kohe palju kiirem.

Seejärel sai kogu Apple II opsüsteem disassembleeritud, kood ära 
kommenteeritud ja imestatud, et Steve Wozniak oli päris mitmes kohas
hämmastavaid trikke teinud. 
Kui hakkasin disassembleerima, siis ise arvasin, et tean assemblerit juba 
väga hästi, kuid paari asja puhul tekkis ikkagi 
vau-efekt. Assembleris olid ühe-, kahe- ja 
kolmebaidised käsud ning trikk seisnes selles, et ühte kolmebaidist käsku oli võimalik 
kasutada selliselt, et kui programm jooksis otse läbi, siis too 
käsk ei teinud midagi. Aga nende kolme baidi viimast 
kahte baiti sai kasutada nii, et kuskilt eespoolt hüppasid teise 
baidi peale, mis oli teine \emph{command}. Ühesõnaga, kolmebaidise käsu 
viimasesse kahte baiti paigutasid tegelikult teise assembleri käsu. Selliseid
elegantseid trikke oli tehtud! Pärast püüdsin ise ka mõnes kohas mõelda, 
kas saaksin toda nippi efektiivselt kasutada.

\question{Kust tekkis teadmine, et nii saab teha? Midagi pidi ju
selle kõige aluseks olema.}

Koodi disassembleerides pidi kogu 
algoritmist aru saama, mismoodi see töötab. Tegelikult oli asi selles, 
et kettaga suhtlemine oli suhteliselt aeglane ja ma tahtsin seda kiiremaks saada. 
\emph{Seek time} oli üks asi, mis palju mängis, ja asi lõppes sellega, et 
kirjutasin assembleris ise ketaste kopeerimise programmi, mis 
töötas opsüsteemist kümme korda kiiremini. 

\question{Kümme korda?!}

Ketta pea liikumine tuli lihtsalt ära optimeerida. Kui tahtsid kogu ketta 
kopeerida, siis tuli kas seestpoolt väljapoole või 
väljaspoolt sissepoole sõita. Ma tegin ühe liikumisega kirjutamise ära, 
mitte ei käinud standardselt edasi-tagasi\sidenote{Flopiseadmed, nagu elektromehaanilised kõvakettad siiani, 
kasutavad andmete lugemiseks ja kirjutamiseks pöörleva ketta pinna lähedal 
liikuvat tundlikku pead. Üks suuremaid pudelikaelu andmete 
liigutamisel flopilt oligi seadme võimekus pead ühest ketta servast teise 
viia.}.

\question{Jällegi, kust üldse tekkis arusaam, et kettaseadmega saab 
niisuguseid trikke teha? See vajab ju 
teadmist, julgust ja natuke ka arrogantsi, et mis see Woz ikka 
ketastest teab.}

Kuna opsüsteem oli universaalne, sai ketta 
kopeerimise programm tehtud \emph{dedicated} kujul ehk optimeerituna 
konkreetse masina jaoks. Tol ajal oli oluline, et kui kuskilt, kas või 
Moskvast, tuli mingi tüüp, kel oli kettaid kaasas, siis pidi 
suutma programmid kiiresti endale kopeerida, mitte tund aega jokutama. See oli praktiline vajadus -- muudmoodi polnud programme saada, internetti ei olnud. Käisime 
ühe klassivennaga kord isegi Moskvas puhtalt sellepärast, et arvutimänge saada.

\question{Moskva on suur linn\ldots}

Üks tüüp, kes käis Tartu Ülikoolis ja kellelt saime programme kopeerida, 
ütles, et kui Moskvasse satute, siis olete alati \emph{very welcome}. 
Ja ükskord läksimegi rongiga.

\question{Huvitav, mis see andmeside kiirus tuleb, kui arvestada, et sõidad 
rongiga edasi-tagasi ja kopeerid flopisid?}

Ei julge öelda, üks ketas oli vist 360 kilobaiti. Eks see oli 
ikkagi kõige kiirem viis andmeid transportida. Meil tuli alles
aastaid hiljem ja käis kord päevas -- helistasid modemiga 
sisse ja tõmbasid meilid ära.

\question{Mida sa pärast keskkooli õppima läksid?}

Tartu Ülikooli rakendusmatemaatikat\index{Tartu 
Ülikool!Matemaatikateaduskond!Rakendusmatemaatika}. 

\question{Kas sind vahepeal sõjaväkke ei võetud?}

Sõjaväest õnnestus ära viilida, aga ega ka õppimisest tegelikult palju välja ei tulnud, istusin Apple'ite juures edasi, nii nagu 
keskkooli ajal. Esimese kursuse tegin ära, kõik matemaatikaeksamid olid viied, aga inglise keelega kukkusin välja. Kuna ma keskkooli lõpetasin hõbemedaliga, siis uuesti sisseastumine oli väga lihtne, pidin ainult matemaatikaeksami tegema, mis oli triviaalne. Aga siis ma ei viitsinud enam üldse 
loengutesse minna, sest matemaatikaeksamid olid tehtud ja oleksin pidanud 
ainult inglise keele pärast uuesti esimesel kursusel käima.

Nii ma istusingi Apple'ite taga. 

Loomakasvatuse ja veterinaaria instituudi direktor Olkonen\sidenote{Instituudi 
täpne nimetus oli aastani 1994 (mil see liideti põllumajandusülikooliga) Eesti 
Loomakasvatuse ja Veterinaaria Teadusliku Uurimise Instituut. Ja 
piimandusteadlane Arvi Olkonen\index[ppl]{Olkonen, Arvi} ei olnud mitte terve 
instituudi direktor, vaid juhatas instituudi piimanduslaborit.} tegi 
doktoritööd ja tal oli terve bussitäis tädisid, kes olid valmis andmeid 
sisestama, aga tal ei olnud kohta, kuhu andmeid sisestada, ja seda, mis arvutab. 
Ma tegin talle programmi, mis võimaldas andmeid sisestada. Kõige olulisem oli luua selline \emph{user inteface}, et tädid kuidagi eksida ei 
saaks. Arvutuse osa oli tegelikult 
lihtne.

\question{Mida seal arvutada tuli? Kas statistikat?}

Tegu oli piimaproovidega, kus tuli märkida laktoosi protsent, valgu protsent ja hulk 
muid karakteristikud. Ma ei mäleta täpselt, Olkonen ütles ikkagi
algoritmid ette. Ma võisin matemaatilist nõu anda, kuid
üldiselt ta teadis ise, mida tegi. 

\question{Kas sa olid järelikult kusagil palgal?}

Ma olin jah Tartu Ülikooli arvutiklassis poole kohaga insenerina palgal. 
Kuna loomakasvatuse ja veterinaaria instituut ei saanud mulle 
ühekordselt maksta, siis mind võeti sinna tööle, aga ma ei käinud 
seal kunagi. Olin seal aasta või kaks tööl lihtsalt selleks, et tolle 
programmi eest tasu saada. Ma ei viitsinud isegi palka minna välja 
võtma -- pangakontosid siis veel ei olnud --, mispeale ülemus tuli mulle autoga järele ja 
viis palka välja võtma, sest ta ei jõudnud enam kassapidaja kisa ära 
kuulata.

\question{Tuldi autoga järele ja viidi raha saama. Programmeerija magus 
elu\ldots}

Jah, direktor Olkonen oli väga lõbus sell. Oma inimestega oli ta küll hirmus 
kuri -- alati kui me sinna läksime, siis kõigepealt sõimas kõigil näo täis. 
Samas oli ta väga ettevõtlik ja sihikindel tüüp. Mäletan, kuidas olime kord kodus isaga saunas ja 
ema tuli ütlema, et mingi mees otsib mind. See oli Olkonen, kellel
oli midagi kiirelt vaja. Ema sõnul oli ta enam-vähem ilma 
tutvustamata uksest sisse astunud, kohe elutuppa läinud ja maha istunud, et ta 
võib oodata, ei ole probleemi.

\question{Kui sa loengutesse ei jõudnud, siis mingi asi pidi sind arvutite 
juures kinni hoidma. Kas see oli assembleri ja pusimise huvi või midagi muud?}

Jah, ikka. Assembleris kirjutasin tekstiredaktori, kuhu sai 
ohtralt igasuguseid \emph{feature}'sid tehtud. Too oli kindlasti kõige 
keerulisem asi, mul peaks vist isegi kood paberil väljatrükituna 
alles olema -- viis või kuus tuhat rida assemblerit. 

\question{Seda ei ole üldse nii palju!}

Ikka üksjagu. 

\question{Assembleri koodi mõttes on seda palju, aga tekstiredaktor viie 
tuhande reaga pole paha!}

Aitasin ühel tütarlapsel, kes mulle väga meeldis, kursusetöid teha 
ja tolleks oligi tekstiredaktorit vaja. Muidu oleks pidanud 
kirjutusmasinal trükkima. Arvutis polnud ühtegi korralikku tekstiredaktorit, ainult mingisugused hädised asjad. Nii ma kirjutasingi ise,
et saaks teha suuri-väikseid tähti ja muud sellist.

\question{Jaan Tallinn kirjutas ka esimese asjana omale tekstiredaktori.\sidenote{Vt lk 
\pageref{sisu!jaani_tekstiredaktor}.} Kas internet on teinud hoopis karuteene? Vanasti, kui tahtsid 
tekstiredaktorit, pidid ise kirjutama, aga nüüd võtad millise iganes.}

Eks siis oli asi natuke ka selles, et programme ei olnud
kuskilt saada. Ameerikas olid Apple'i jaoks ilmselt kõik programmid olemas, aga Eestisse need ei jõudnud ja tuli ise teha. Aega oli ka palju\ldots

\question{Mis su ettekujutus tol ajal oli, et kuhu see viib? Istud järgmised 
20 aastat Vanemuise tänava klassis?}

Ausalt öeldes ei olnud mul konkreetset plaani, kuhu see 
viib.

Umbes 1990. aastal pandi Vanemuise tänava klassis meilindus käima ja 
samal ajal kutsus Taavi Talvik\index[ppl]{Talvik, Taavi} 
mind Postimehe\index{Postimees} toimetusse.

Ta oli seal SCO UNIXi\index{SCO UNIX} valmis pannud ja 
hulga terminale, mille kaudu ajakirjanikud oma artikleid 
sisestasid. Emacs oli vist tekstiredaktor ja eesmärk oli selle peale eesti keele 
õigekirjakontroll teha. Tollega ma seal tegelesingi. 

\question{Ühesõnaga sa läksid Vanemuise tänavalt Postimehesse?}

Jah, Postimehes tuli \emph{full time job}, aga käisin õhtuti ikkagi ka Vanemuise tänavalt läbi, sest nood 
tädikesed, kes arvutiklassi haldasid, ei olnud tehniliselt liiga võimekad ja andsin neile nõu. 

\question{Õigekirjakontrolli tegemine ei ole triviaalne asi, peab ju 
keelest ka aru saama.}

Ei ole tõesti triviaalne asi. Siis ma avastasingi, kui neetult keeruline eesti 
keel on. Miks peab iga teine sõna olema erand? Väga tüütu! Ega 
me toda tekstiredaktorit tegelikult valmis ei saanud.

\question{1993. aasta paiku hakkasid tekkima Filosoft\index{Filosoft} ja muud 
niisugused asjad, mis tegid Wordile eesti keele spelleri, aga selleks ajaks 
olid teadus ja arvutusvõimsus edasi läinud.}

Jah, absoluutselt, aga 1993. aastal tulin ma Tallinnasse 
Hansapanka\index{Hansapank} tööle.

\question{Kas sa tulidki otse Postimehest?}

Postimehe ja Hansapanga vahele jäi lühike periood, 
kui töötasin hulgifirmas\sidenote{Nõukogude ajal olid poed 
prestii\v{z}sed asutused, sest nende kaudu jagati valitutele defitsiitset 
Nõukogude kaupa. Ühtäkki aga hakati vastses vabariigis hulgifirmade 
kaudu jagama lääne kaupa.}. Neile kirjutasin ka üht programmi. 
Too hulgifirma oli väga kaootiline koht: bisnis läks 
neil hirmus hästi ja nood kolm kutti, kes olid omanikud, 
ostsid endale iga kuu uued BMWd ja tolmutasid nendega ümber maja sõita. Nii et see 
ei olnud eriti motiveeriv keskkond.

\question{Kust Taavi sind leidis?}

Ma täpselt ei mäleta, kus me Taaviga tuttavaks saime. 
Sel ajal kui mina Apple'ite taga istusin, oli Taavi sealsamas Tähe 
4 keldris, kus ma esimest korda Nairidega kokku puutusin ja kus tal oli üks
IBMi PC.

Võimalik, et Taavi tegi midagi Tartu Ülikooli raamatukogule\index{Tartu 
Ülikool!Raamatukogu} ja mina olin ka tollega kuidagi seotud ning äkki saime
raamatukogus PC taga kokku. Mina näitasin Taavile Apple'eid 
ja tema mulle PCd. Mängisime Taaviga Tähe tänaval sellist mängu nagu \enquote{King's 
Quest}\index{King's Quest}\sidenote{Sierra Entertainmenti seiklusmängude sari, mida peetakse oma 
valdkonnas klassikaks. Mängud ilmusid aastatel 1980--1998.} ja seeläbi saimegi paremini tuttavaks.

\question{\enquote{King's Quest} oli ju seiklusmäng.}

Just. Tollega läks ikka aega, et lõpuni mängida, istusime palju õhtuid 
seal.

\question{Kas Taavil oli ühel hetkel Postimehes abi vaja ja siis kutsuski 
sind?}

Täpselt. 

\question{Kuidas sa Hansapanka\index{Hansapank} sattusid?}

Hansapanka sattusin samuti tänu Taavi Talvikule\index[ppl]{Talvik, Taavi}. 
Taavi töötas tollal Valitsussides\index{Valitsusside} ja 
Rainer Nõlvak\index[ppl]{Nõlvak, Rainer} MicroLinkist\index{MicroLink} vist 
rääkis Taavile, et Tõnis Sildmäe\index[ppl]{Sildmäe, 
Tõnis} otsib kedagi, kes tunneks Unixit. Taavi ütles, et tema küll ei taha 
minna, ja küsis minu käest. Mõtlesin, et võin ju rääkida ja 
kuulata, mis teema on. Tulingi Tallinnasse Tõnis Sildmäega rääkima. 
Sildmäe küll jättis mulje, et tal on terve bussitäis Unixi-mehi ukse taga 
järjekorras, keda ta kõiki intervjueerib, aga tegelikult peale minu vist
ei olnudki kedagi. Igatahes ma sain panka tööle.

SCO UNIX\index{SCO UNIX} oli sinna juba ära installitud ja Tarmo 
Pajumets\index[ppl]{Pajumets, Tarmo} püüdis selle peale Oracle'it\index{Oracle} 
installida. Nad ei teadnud SCO UNIXist midagi, nii et esimese 
päeva lõunaks läksid nad konsooli tagant ära ja enam
tagasi ei tulnud, kui vaatasid, et ma vist tean natuke rohkem.

\question{Inimesed said oma teadmiste piiridest aru. Pank kui 
selline oli tolleks ajaks juba olemas, aga mis infosüsteemi peal see käis?}

Pank käis Paradoxi\index{Paradox} peal. Oracle'i andmebaasi majja 
toomine oli üks paljudest Hansapanga edu aluseks olevatest 
strateegilistest otsusest. Paradox 
töötas tol hetkel täiesti normaalselt, aga Tõnis 
Sildmäe\index[ppl]{Sildmäe, Tõnis} oli juba välja raalinud, et tegelikult me 
peaksime selle alla mõne tõsisema andmebaasimootori panema. 
Esialgu läks Oracle Novelli\index{Novell} peale, aga siis saime ka SCO 
nii kaugele, et migreerisime andmebaasi sinna.

\question{Räägi sellest lähemalt. Kellelgi, ilmselt Tõnisel 
oli arusaam, et arhitektuursed otsused võivad olla ärilise edu aluseks. 
Üheksakümnendate alguses see ei olnud väga levinud, kust tal see arusaam tuli?}

Panga seltskonnal oli ikkagi selge arusaam Paradoxi 
tehnoloogilistest piirangutest ja samal ajal ka visioon, kuhupoole 
pank liigub. Kui mina panka läksin, 
oli minu kõrvallaual juba vist esimene sularahaautomaat -- IBMi oma ja
suhteliselt pisikene, mahtus laua peale. Kaart ei käinud 
sisse, vaid magnetriba tuli läbi tõmmata. ATMid olid 
üks asi, mis Paradoxi andmebaasi piirangud välja tõi. Ja kuna 
klientide arv kasvas plahvatuslikult, nähti ilmselt ka tolle pealt, et kui selline kasv jätkub, ei suuda
Paradox kõiki ära teenindada.

\question{Samal ajal toimetati 
päris mitmes pangas valmis tarkvaraga, näiteks osteti Inglismaalt soft. Miks Hansapank tegi teistmoodi?}

Seda oskavad öelda need, kes olid päris alguses pangas. Ma ei 
tea, kuidas Spin Development\index{Spin Development} Hansapanka tuli. 
Nimigi viitab \emph{developer}'idele. Ja 
ilmselt esimene ülesanne oli mõni väike tükikene ja 
kui too läks hästi, hakkas asi edasi arenema. Ma täpsemalt ei oska öelda.

\question{Kas Spin Development oli Crebiti\index{Crebit} algus?}

Jah. Kui mina tööle läksin, siis esimese palga maksja oli tegelikult Spin 
Development, mis minu teada nimetati 
Crebitiks ümber. Mõni aeg hiljem ütles Londoni 
kindlustusfirma pangale, et te ei tea ITst mitte midagi ja kuidas te üldse oma riske juhite -- kogu 
asi on väljas, täiesti iseseisvas ettevõttes. Asi lõppes sellega, et Hansapank ostis Tõnise käest 
Crebiti aktsiad ära ja me tulime kõik Hansapanka tööle. Crebiti 
juriidiline keha jäi alles ja on kuni tänaseni Swedbank Support OÜ nime 
all olemas.

\question{Huvitav, et kultuur oli jätkuvalt Crebiti oma, sest kui mina pangast 
ära tulin aastal 2002, siis viimane särk, mille pank mulle andis, oli Crebiti 
logoga. Väga elujõuline asi!}

Kindlasti. Mitte ainult Crebit, vaid pank tervikuna oli 
äärmiselt elujõuline, vähemalt kuni 
Hoiupangaga\index{Hoiupank} liitumiseni. Siis toimus suur kultuuriline 
muutus, kui tuli palju teisi inimesi juurde.

\question{Kust see kultuur tuli?}

Ma olen toda mõelnud. Ilmselt ühelt poolt oli kõigil inimestel selge saavutusvajadus oma asja väga hästi teha. Seal 
isegi ei pidanud neid, kes ei \emph{perform}'inud 
piisavalt hästi, lahti laskma, vaid nad läksid ise ära. Samamoodi oli siis, kui 
Pajumets\index[ppl]{Pajumets, Tarmo} Oracle'it installis. 
Tegelikult oli tema kõrval veel üks mees, kes SCO UNIXi installis. 
Kui mina liitusin, siis too mees läks ise paari nädala pärast 
ära. Teda ei lastud lahti, vaid ta sai aru, et tal ei ole seal enam midagi teha. Ja selline kultuur 
oli absoluutselt kõigile ühesugune. Ei pidanud kaks korda kellelegi ütlema, 
vaid teadsid, et asi saab tehtud.

\question{Kui sa alustasid, siis sa kirjutasid assembleris koodi ehk tegelesid 
arendusega. Pangas aga läksid kohe asjade käigushoidmise peale. Kuidas ja 
miks see nihe toimus?}

See ei olnud teadlik valik. Töö tundus huvitav, 
ma ei olnud Oracle'i baasi varem näinud. Ma ise ei mõelnud, et olen 
programmeerija. Ka arvutiklassi Apple'ite puhul oli mu tööülesanne 
tegelikult kõigi inimeste assisteerimine, arvutite ülalhoidmine ja
probleemide lahendamine. Programmeerimine oli 
puhtalt hobi muu töö kõrval, kuigi alguse sai see Nairide peal
programmeerimisest. Tollal kõik lihtsalt tegutsesid, siis ei räägitud sellest, et ühed on arendajad ja teised ülalhoidjad.

\question{SCO jäi sulle külge Postimehest, aga Oracle?}

See jäigi külge Hansapangast. 

\question{Kas hakkasid lihtsalt otsast tegema? Aastaid hiljem oli 
tegu maailma ühe suurema Oracle'i koodi baasiga, Oracle'i konsultandid käisid 
majas ja rääkisid, et nad ei ole kuskil mujal midagi sellist näinud.}

Võis olla küll. Vilve Vene\index[ppl]{Vene, Vilve} ja Ruta 
Joost\index[ppl]{Joost, Ruta} kirjutasid usinasti PL/SQLi. Toda oli tõesti väga palju. Kasutajaliidese 
tehnoloogiaks kasutati Oracle Formsi\sidenote{Oracle Forms joonistas kliendi arvutisse \enquote{paksu} kliendi, mis pöördus otse andmebaasi protseduuride poole. Viimased sisaldasid kogu äriloogikat.}, 
sest ilmselt oli nii seda kõige efektiivsem teha. Baasi protseduurid käisid kõik kiiremini kui 
klient-server-lahendused.

\question{Kas sul ei tekkinud tunnet, et las Oracle käib siin edasi ja nüüd pigem 
programmeeriks?}

Pidin ikka programmeerima ka, kirjutasin skripte, 
mis kogu asja üleval hoidsid. Näiteks kuidas andmebaas käima pannakse, kui SCO 
UNIXiga masinat üles \emph{boot}'ida. Oracle'i installil ei olnud skripte, kirjutasin need ise, et baas ja \emph{listener}\sidenote{Oracle andmebaasi komponent, mis võtab vastu väliseid ühendusi andmebaasi külge.} käima panna. Ka 
\emph{backup}'i tegemiseks pidi skriptid kirjutama, lisaks kõik 
\emph{batch}-protsessid, mis olid Cs kirjutatud. 
Nii et tegelikult kirjutasin päris palju skripte.

\question{Sinu kirjeldus, kuidas
eri tehnoloogiatest tervik moodustus, kõlab päris keeruliselt. Kuidas see tervik tekkis ja kes 
seda juhtis? Kes oli arhitekt?}

Siis ei nimetatud kedagi arhitektiks. Julgen arvata, et 
tarkvara kontekstis oli arhitektiks Vilve Vene\index[ppl]{Vene, Vilve} -- kontseptsioon, kuidas kõik tarkvaraliselt kokku töötab, tuli 
ennekõike temalt. \emph{Non-functional requirements}'id tekitasin mina. 
Oluline oli, et iga C-programm, mis mõnd \emph{patch}'i tegi, ei 
oleks erinev. See tuli kuidagi ära standardiseerida -- ma pidin 
mittefunktsionaalsed nõuded esitama, et need kõik oleksid ühetaolised ja et 
saaks kasutada ühte skripti paljude asjade käivitamiseks.

\question{Kui Postimees käib ka siis, kui ajakirjanikud trükimasinaid 
kasutavad, siis pank enam trükimasina peal ei käi. \enquote{Pusime ja vaatame, 
kuidas Woz on teinud} pidi struktuursemaks muutuma -- kuidas see juhtus?}

Kuna kasv oli nii kiire, siis igaüks pidi vaatama mitte ainult 
seda, kuidas asi täna toimib, vaid ka seda, milline see 
aasta pärast välja näeks. Tõnis Sildmäe\index[ppl]{Sildmäe, Tõnis} 
soodustas kindlasti ka erinevate kontaktide teket, kes 
pakuksid uusi lahendusi. Nii arenesidki
asjad edasi. SCO UNIXile tulid ju samamoodi tehnilised piirangud 
ette. Aastal 1996 või 1997 sai see HP-UXi\index{HP-UX} vastu välja vahetatud. Enne 
vahetust olid mul nii HP kui ka Suni server laual, et võrrelda, kumb on
kiirem. Tolleks ajaks oli pank piisavalt suur ja oli selge, et me ei 
pane ühte masinat, vaid klastri. Rääkisime tarnijatega 
klastrilahendused läbi.

Sellist hüpet ei toimunud, et enne oli anarhia ja siis tehti kõik asjad 
korda. Kõik arenes evolutsiooniliselt, igal aastal vahetati lahendusi uute 
vastu välja. Teisti ei oleks üle elanud toda kümme aastat kestnud olukorda, 
kus iga üheksa kuu tagant klientide arv, käive ja
kasum kahekordistusid.

\question{Sellist kasvu ei kujuta tänapäeval enam väga ette, kui just 
kuskil Skype'i moodi kohas ei tööta.}

Nii kiiresti ja pikalt kasvavaid ettevõtteid ongi maailmas väga vähe olnud. See oli \emph{success story}.

\question{Sajandivahetuseks oli panka tekkinud üsna 
spetsialiseerunud tiim, kes kogu kupatust käigus hoidis. Kuidas see kolmik, 
mille peal kogu pangamaailm püsti seisis, tekkis?}

Aja jooksul, selles mõttes, et Madis Ollisaar\index[ppl]{Ollisaar, Madis} oli 
enne mind olemas. Ma ei teagi päris täpselt, mis tema roll päris alguses oli. 
Kui mina hakkasin Oracle'i baasiga toimetama, siis minu 
asi oli tehniline pool, et andmebaasi \emph{engine} töötaks, ja Madise 
asi oli luua uusi tabeleid, teha indekseid ja vaadata, et päringud hästi 
käivad. Toomas 
Suurmets\index[ppl]{Suurmets, Toomas} töötas Hoiupangas, aga 
ta tuli Hansapanka kaks aastat enne seda, kui Hoiupank ära osteti. 
Tolleks ajaks istus tema juba õigel pool lauda ja täiendas seda seltskonda. Kui 
Madis oli kõige ülemine, nii-öelda \emph{data layer} ja mina teadsin 
andmebaasi \emph{engine}'i osa, siis Toomas jagas hästi \emph{network}'i 
ja \emph{storage}'it. Kõik see kokku andiski tehnoloogilise 
\emph{stack}'i, et põhiasi töötaks.

\question{See tiim töötas väga hästi!}

Me istusime ühes toas ja olime kogu aeg samas infoväljas, nii et teadsime alati, mis toimub.

\question{Ma tean, et sul on Wagneri huvi. Kas see oli juba tol ajal? 
Teie toa kapi otsas oli makk, kust tuli aeg-ajalt eepilist klassikalist 
muusikat.}

Amazoni siis veel ei olnud, 
esimesed CDd ostsin selliselt veebilehelt nagu cdnow.com. 
Klassikalist muusikat sai mängitud jah -- tollal mitte küll 
Wagnerit, vaid põhiliselt Mozartit. 
Alguses ostsin mõned Enrico Caruso plaadid ja hiljem 
Mozartit ning neid me mängisime jah, teisi see ei seganud. Eks me tegime 
erinevaid asju. Oli ka periood, kui meile öeldi, et meie toas on iga päev tunda teatud lõhnasid. Ühel ajal oli meil tõesti alati 
konjakipudel kapis ja alustasime päeva pitsi konjakiga. Loomulikult 
mingit joomist ei olnud, aga eks lõhnaks piisas juba ühest pitsist ning klaas võis olla lõunani
laual. 
Ühel perioodil mängisime palju WRC rallit, mille vist Toomas püsti pani. Too tahtis väga palju \emph{network}'i, 
aga kuna Toomas oli \emph{network}'i põhjaga vend, siis ennekõike oligi toda 
\emph{bandwidth}'i meie toas.

\question{Kust su klassikahuvi pärineb?}

Klassikahuvi tuli Enrico Carusost. Vanematel oli kodus 
Vittorio Tortorelli raamat\sidenote{Enrico Caruso. Eesti 
Raamat 1968, tõlkija Õ. Karask.}, Tortorelli on itaallane ja kuna ta oli vist ka
Caruso kauge sugulane, siis see raamat 
oli küll ülimalt ülistav, aga huvitav ja jättis väga sügava mulje. 
Kui ükskord oli võimalik internetist tellida, siis ostsingi huvi pärast 
CDNow'st Caruso plaate. Teine asi draivis oli 1984. aasta 
Miloš Formani \enquote{Amadeus}, mida ma soovitan kindlasti kõigil vaadata, suurepärane 
film! Sealt tuli Mozarti huvi: tellisin neli-viis
raamatut Mozarti eluloost, mõni üle tuhande 
lehekülje paks. Edasi tulid Beethoven, Schubert, 
Schumann, Tšaikovski\ldots

\question{Mida sa praegu teed?}

Töötan G4Sis\index{G4S} baasteenuste arendusjuhina. 
Sisuliselt vastutan ülalhoiu eest, et kõik asjad oleksid püsti ja valvatud -- mitte ainult IT, vaid ka tehniline valve, sealhulgas 
üle-eestilise raadiosidevõrgu üle, et kõik signaalid jõuaksid keskele kokku.
