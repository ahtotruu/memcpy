\index[ppl]{Peiker, Andres}

\question{Kuidas ja umbes millal sa jõudsid arvutite juurde?}
Äkki äkki äkki?
Tere See siin on Memm copy ammu enne seda, kui Silicon Valley maailmale näitas, kuidas suuri infosüsteeme skaneeritakse oli Eestis seltskond inimesi, kelle hallatav infosüsteem kahekordistas oma ärimahte iga üheksa kuu tagant ja tegi nii kümme aastat järjest. See kõik oli väga suuresti meie tänaseid külalisi. Teine külas on Andres Peiker. Head kuulamist. Memm labi.
Tere.
Sinu nimi on Andres Raid. Raili. Alustame asjade algusest, sealt, kust kõik asjad on pihta hakanud. Kuidas sinu arvutite juurde said?
Ta oli mingi kaheksakümne neljas aastake ja osta, et.
Suht juhuslikult tegelikult selles mõttes, et ma õppisin siis keskkoolis ja ma käisin mingisugustel füüsika loengutel Tartu Ülikoolis ja ja ühe tolle loengu lõpus mees nimega Otto talle asendus auditooriumite ütles, et, et aga, et kes on arvutitest huvitatud, et võivad natukene siia veel jääda. Ja noh, siis mingisugune seltskond jäi, Otto Teller viis meid siis seal tähe neli olevas õppehoones. Seal oli kaks maili arvutit ma ka ja naine kaks vist olid nood viis näitas nõid kes ütles, et noh, et põhimõtteliselt siin nagu mingitel õhtustel aegadel on või on võimalik käia nagu programmeerida proovida asju.
Millest, millest ma kohe järeldad, sa Tartu boss absoluutselt esimesed kakskümmend viis eluaastat vanad ja, ja siis, millest ma järeldan, on, on see, et kui sa keskkooli ajal kuskil Tartu Ülikoolis mingites loengutes kestis, siis kui pidi olema mingi matemaatika, kui reaalainete või niisugune huvi
No ma õppisin Tartu esimesest keskkoolist tuli matemaatika-füüsika eriklass.
Noh, ma käisin olümpiaadidel või ei mäleta täpselt, kui Kustu ülikooli loengute teema üldse tuli, et noh, füüsikas tüli ja füüsika tundus mulle nagu kõige põnevam asi üldse, et et siis siis saigi seal käidud.
Ja seal fäski loengu lõpus. Mind paneb imestama, et keegi üldse nagu ära, eks, et selles mõttes, et kõik sõnavahed ja rahvas tundus nagu arvutite vastu huvi tundnud või siis nii olnud.
Ei, ikka ei olnud selles mõttes, et ma arvan, et ikkagi pooled läksid ära ja ja ta on grupp jäi tegelikult noh, esimesel korral käisime neid arvuteid vaatamas, siis öelda, et noh, et järgmine kord saaks nagu sel päeval tulla, siis tuli juba vähem inimesi ja lõpuks jäi mingisugune, ma arvan, mingi kolm-neli inimest võib-olla alles, kes on nagu rohkem käi käima hakkasid.
No see oli mingisugune ring või lihtsalt.
Ma enam nii täpselt ei mäleta, et, et ma arvan, et Otto Teller ikkagi seal natukene juhendas ka alguses, et, et mis, mis ja kuidas, et kuidagi me tolle AB programmeerimiskeelega, mis on mägede peal, oli kuidagi me tuttavaks saime, et ma arvan, et läbi läbi talle tema vahet
Mingeid raamatuid niisugust kraami.
Ei, seda ma küll ei mäleta, et oleks olnud.
See on huvitav asi. Tänapäeval me pöörame palju tähelepanu selleks, et õpetada inimesi programmeerima ja see on täiesti läbivalt, mitte keegi suuda meenutada, kurjasid, õpsid, programmeerib, kuidagi sündis täiesti lihtsalt tuli. Aga mida te tegite sinna nairidega?
Noh, see on, seal sai ikkagi teha väga lihtsaid mingisuguseid arvutust või samme selles mõttes, et tollel arvutil ju koha pealt tuli elektroonilise kirjutusmasinaga, nii et sa kirjutasid programmi ja ta tuli paberi peale ja ta oli ainukene eksemplar tollest programmist, mida, mida sa pidid siis alles hoidma tuleb, eks needki saab parandada, tahtsid, siis sa pidid vaatama today prinditud paberit. Et et noh, kõige kõvem asi, mille ma seal valmis tegin, olis dollareid, tähtsad asjad, biorütmid, et noid Arvutuskeskuses tehti ja, ja siis ma tegin, talle meeldib ka peale tegin ka biorütmide programmi, et.
Et ma saan auto, osutus seal populaarseks, et tollest minu perfolingvist keegi tegi koopia ja siis lasti toda seal talle Tähe neli töötajatele usinasti välja, ilma et ma midagi tean.
Biorütmide arvutamine ja see oli kuidagi naljakas, sest tõrvist need algoritmid, mis liikusid, olid vist mõeldud käsitsi arutamiseks ja seal olid minu meelest kuidagi arvutati Siimust.
Kui ta skandaali Taligi lihtne siinus selles mõttes, et lihtsalt sa pettur sünni sünniaja ütlema ja siis tolle elatud päevade arvu pealt noh, tosiinusega ka häbilainepikkus oli lihtsalt nendel emotsionaalne ja füüsiline ja seksuaalne, mis neljas oli, ma ei mäletagi, oli, oli lihtsalt erinev mustajoonistest olla neli siinus sisuliselt välja, tegelikult et noh, täiesti triviaalne asi iseenesest taheti, rohkem oligi too, et noh, kuidas paberi peale toda sinust joonistab, et elektroonilise trükimasinaga
Jah huvitav asi, mis oli nagu oluline meil tänaseks täiesti varaga biorütme hakata joonistama inimestele jumalaid.
Tuli siis mingisugune väga popp asi ja ja tundus tolle arvuti jaoks nagu niisugune jõukohane ülesanne pärast, et ma arvan, et see oli mingi, kas neli kilobaiti oli, oli mälu tollele arutelule ja noh, ta oli sama suur kui mul, ma ei tea, kodus köögimööbel.
Et selline suur asi, milleks ta tänu füüsikud kasutasid
Samamoodi kasutamiseks.
Aga mida mingid?
Ei tea sellest sellest nagu ei olnud juttu selles mõttes, et on ai kaali nagu väiksem masin teises toas oli null kaks. Põhiliselt ma saan aru, kasuti Duda meie sinna masinale nagu eriti pidi ei saanud. Et oli nagu rohkem hõivatud ja Toljad vaatasin ka selles mõttes, et lindiseadmed need suured lindikapid, kus siis seda magnetlinti keerutate ja tol ajal ei olnud mitte see tavaline elektrooniline kirjutusmasinaid, tol oli niisugune trummeliga printer. Noh, mis suutis ikkagi toda paberit nagu päris kiiresti välja lasta, et.
Printimise tehnoloogiaid rublale kuidagi väga oluline ka absoluutselt. Kas sealt see pusimine oli lihtsalt puhtalt niisugune nõu põne, arvutiga möllamine või seal mingisugune nagu sügavam asi ka tagavad, sulle tundus, et kuidagi, et seal mingisugune asi, mis seda teha tahan.
Siis tuli puhtalt seotud tegelikult sellega, et et kuna keskma olin matemaatika-füüsika eriklassis, siis meil oli seal Andres Jaeger, ülikoolist andis, andis programmeerimist ka kolm aastat, aga ta programmeerimine oli sisuliselt ainult mingite blokk skeemide joonistamine paberi peale, ühesõnaga noh, me arvuti ligi ei saanud ja, ja noh, siis too Tähe tänaval oli võimalus nagu ise järele proovida, siis seda, mida sa olid tegelikult nagu paberi peal teinud. See, et toda nagu tavakooli õppeprogramm ei võimaldanud kooliprogrammile joonestuspaberi peale, algas.
See paberi peal paks geimid joonistamine, see võis ju huvi ära tappa, aga sul või tapnud seal millegipärast läksid. Pärast seda loengut jäid sinna ja millega see käsitsi Emmy eid.
Ei, ta ei tapnud kindlasti tolle pärast, et, et ka too blokk, skeemide joonistamine, et näed, kui sa talle ülesande said, noh siis siis ta siis ta ütles ka, et umbes, et noh, et kes suudab nagu mingisuguse
Kolme Ifiga teha, et on, on hea kaheffiga on väga hea, et noh, et ühega noh ma ei tea, ja noh, siis siis sul oli nagu eesmärk olemas, et sa pidid ühe väga tegema talle tulemast, et noh, too asi kõnetas mind ja noh, siis sa said õpetaja käest kiita ka murde on tõesti, et noh, et ma mäletasin, võib-olla viis aastat tagasi oli meil ka üks õpilane, kes suutis nagu selle algoritmi nüüd selliselt ära teha, et ja väga hea
Ühesõnaga sinu jaoks ajalise oskab joonistada niisugune nagu ülesanne, kui naguniisugune pusle või.
Jah, absoluutselt selles mõttes, et noh, et ikkagi ütleme, et toob üles loomulikult jõuga, sa suudad tolle mängu lihtsalt ära teha, aga et noh, et nüüd, kuidas ta nagu kõige optimaalsem saaks kõige parem. Et, et noh, tooli huvita.
Okei ma ütlen, et, et selle koolikoolid olla ühtegi nagu arvutitel on Tartu linna peal ju arvuteid oli nagu küll tootjad.
Meie koolis ei olnud, ei olnud, siis ei olnud ikka mingisuguseid arvuteid kuskil selles mõttes, et.
Tähe tänaval oli, olid kunud kaks naid. Loomulikult seal ülikooli arvutuskeskuses oli jeeess. Lahkus, kes üldse nagu ise arvuti ligi saanud operaatorid lõid programmidesse. Ja siis oli füüsika instituudis Riia maantee lõpus, seal oli ka mine niisugune.
Pidipi üksteist äkki.
Ja ma arvan, et umbes tollel ajal kuskil Anne Villems sebis need Apple kahed ka tegelikult siis Vanemuise tänavale uhke, et noh, tooli nagu.
Asi, kuhu ma, kuhu me järgmisena jõudsin peale vaid naistega sõid? Ma arvan, et tuli ka Otto Teller, kes meid sinna viis ja ma isegi tundsin nagu pärast natukene piinlikkust, et, et tema näitas, no täplid. Ja siis ma tegelikult hülgas endale Tähe tänava ja ei käinud enam tema juures, ainult vahtisin seal Apple'it ei oleks käinud sellepärast et ta oli palju nagu ägeda rahast tol ajal oli ikkagi monitor ja, ja tal oli nelikümmend kaheksa kilogrammi sinu elu ja ja ta oli ikkagi nagu ulmeliselt kiire.
Päris päris asju teha jah. Kes mängimine kaaslasi papi peal tuli ülesse teemaks või?
Ja, ja absoluutselt tooli tooli päris hull selles mõttes, et too oli kindlasti minu elu kõige suurem arvutimänguperiood. Et ma oleks peaaegu kahe klassivennaga keemiaeksamile hiljaks jäänud, tavapärast, et noh siis salvestada seisu ei saanud, sa pidid lihtsalt nii kaugele mängima, kus said ja juhuslikult juhtus ta hästi nii minema, et oleks pidanud juba minema, aga tuli järgmine level ja pidid edasi hängima.
Apple'i peal sihuke standartne mäng on nagu Batman mis on mänguautomaati siis igal pool Apple'i peal nimetatud super Pukmeriks. Ja tuli ma siiamaani pean teda kõige lahedamaks mänguks, mida ma olen kunagi mänginud tolle pärast, et toda Pakmenit oli kõigi teiste arvutite peal ka. Aga, aga seal oli nagu mingi katastroof, haalne erinevust olles Algo Hitmis, kuidas tund neli kolli liikusid tolle pärast, et kõigi ülejäänud arvutite peal, nii palju kui mina olen mänginud liikusid rändumiga. Aga Apple'i peal oli neil oma kindel algoritm. Ja tulemuseks oli see, et kui sa ise tegid täpselt ühtemoodi siis tus, situatsioon kordusmänguks mängu ja me tegelikult tööd meil olid välja töötatud esimese kuue leveli jaoks tegelikult sisuliselt algusest lõpuni, et sa teadsid täpselt, kuidas sa terve tolle ekraani puhtaks mängisid ja jäime sellele noh, sealt edasi oli, oli sisuliselt mingisugused paar avangut, mida sai erinevatel serveritel kasutada.
Kuidas toimub põhimõttest Bäckman kui male?
Natukene natukene, absoluutselt selles mõttes ja, ja noh, selle tõttu olnud võimalik teiste peal mängida sellepärast et nad lihtsalt rändama ka liikusid. Et jah, arvutimängud jah, absoluutselt, no seal oli teisi teisi veel, aga super Uponor'i kindlasti.
Ja ja on programmeerimine.
Nojah, muidugi selles mõttes, et seal teisik. Jaa noh, ütleme, et ma alguses ma kirjutasin ikkagi teisikus, aga, aga pärast pärast sai ikkagi valdavalt Assembler kirjutatud tolle pärast, et noh, programm teeb su oluliselt kiiremini kui Assad, et noh, see on tõesti olnud.
No kuidas teisikustessembrisse hüppamine käis, sest Peisikust ma saan aru, et seal see käsud on inglise keeles, eks ole see korrektsele luks lihtsasti üles. Aga selleks sa pead teadma ikkagi väga täpselt, mida sa teed ja miks sa nii töötavad. Harilik protsessor, arhitektuurist ja nii edasi.
Noh, kapeisiku puhul sa pidid ikkagi tollest arvutiarhitektuurist aru saama, et,
Et noh, kust olles neljakümne kaheksas kilobaitides nüüd paiknes ekraan tekstiekraan, kus paiknes graafiline ekraan, kus paiknes see programm, kus oli opsüsteem et, et tegelikult talle arvutiarhitektuuris arusaamine tekkis tulle peesiku kõrvalt ka suhteliselt kiiresti.
Ja aga, aga noh, too Assembler tuli ikkagi tänu sellele, et et osad asjad olid väga aeglased. Et üks asi, mida ma seal tegin, oli orienteerumisneljapäevakute protokollid.
Tollega alustas tegelikult Peep Paabel, kes rakendusmatemaatikat ülikoolis õppida, aga ta lõpetas ülikooli ja siis ta andis mulle tolle kogudule programmi komplekti üle aga tuli minu jaoks liiga aeglane. Andmemaht oli tol ajal neljakümne kaheksa kilomeetri jaoks natuke liiga suur, seal oli ka mitmeflopiga mängimist, et, et need andmebaasid ära mahuksid ja mõtlesin, et teen, ma kirjutan from spetsiifika sõbraks. Ja siis ma kirjutasingi kõik Assembler start oli kõik palju kiirem vahe.
Ja noh, peale toda sai veel siis, siis sai kogu too opsüsteem tegelikult Ribes Insineeritud, tissassembleerituda sellega, et kogu too täna kommenteeritud siis sai imestatud, kus päris mitmes kohas Steve Wozniak oli hämmastavaid trikke teinud talle talle opsüsteemi kirjutamise juures, et nagu noh, tolleks ajaks kui hakkasin Tizzasolveerima, siis siis ise ka juba arvasid nõutud Assembler tean nagu väga hästi. Aga siis ikkagi paar sellist asja, mida avastasid, et Vao, kuidas teha saab, et et noh, nagu niisugune pisut Hipp Trek tegelikult. Et noh, et Assembler siis lihtsalt olid ühe poidilised, kahe maitside, kolme pallised käsud.
Ja see trikk oli see, et et ühte kolme potist käsk oli võimalik kasutada siis selliselt, et kui sul programmi oksis otsele, niisiis ta kolme baitine käske ei teinud midagi, aga sa said tolle kolme baidi viimast kahte potti kasutada selliselt, et saab kuskilt eespoolt hüppasid tolle teise baidi peale, mis oli siis teine Command, et sa sinna kolme baidise käsu viimasesse kahte Balti paigutasid tegelikult teise Assembleri käsu.
Et siis üks elegantsed tekkele tehtud ja siis pärast püüdsid ise ka nagu mõnes kohas mõelda, et kas ma saadada nagu efektiivselt kasutada.
Kust, kust see teadmine tuli, need nii teha saab, et vett saab, lisab, et saab lahti võtta selle noh, mingi teadmine pidi kuskilt jällegi lekkima, eks.
Ei, no kui me tolle koodina kudissassembleerisime, noh siis, siis sa pidid kogu tollest algoritmist arusaamad, mis, mismoodi ühtset töötab. No tegelikult oli, ta on opsust noh, mitte nüüd bioss selles mõttes, vaid vaid opsüsteem, et kettaga suhtluskett nagu suhteliselt aeglane. Ma tahtsin seda kiiremaks saada, mu püütud, siit taim oli seal üks asi, mis, mis välja mängisid lõppes sellega, et ma tegelikult kirjutasin Assembler ise nagu ketaste kopeerimise programmi, mis töötas siis nagu tollest opsüsteemist nagu mingi kümme korda kiiremini. Kümme kurd, noh sa pidid arvestab optimeerima, lihtsalt tuleb pealiikumised, kui sa tahtsid kogu ketta ära kopeerida, siis, siis too sa pidid, ma ei mäleta, kas seestpoolt väljapoole või väljaspoolt sissepoole sõitma, selleks et siis ta siis ta tegi ühe liikumisega kui kirjutamise ära, mitte ei käinud edasi-tagasi. Et muidu standardselt käski edasi-tagasi alatia.
Jällegi, et kust sa üldse niisugune arusaam, et kettaseadmega saab nii üksi trikke teha, et nagu võiks ette võtta siukse asja seda teadmist ja julgust, niisugust pealehakkamist, pluss natuke ekstravagantse ka, et nagu, mis see vastus ikka teab, kuidas ketast kopeerida?
Noh, noh, too opsüsteem on ikkagi universaalne tehtud, et, et selles mõttes see ketta kopeerimist programm sai tehtud nagu Detiteedid siis optimeerituna mingisuguse konkreetse asjaaegselt noh tol ajal oli oluline, et noh, et mingisugune kuskilt kas või Moskvast mingisugune tüüp tuli, tal oli mingeid kettaid, kassi pidid kiiresti suutma kopeelemmitse sai Jokute seal nagu tund aega kopeerida, vaid et sa saad nagu kiirelt endale ära tõmmata näoga Süüria rohkme munade, kust, kust sa neid programme saab Internetti jõudnud, et too liikus ikkagi nagu ma käisin isegi tegelikult koos ühe klassivennaga korra Moskvas puhtalt sellepärast, et et mingisuguseid arvutimänge saada Su.
Moskva suurlinn, kus huviline
Ei, lihtsalt vend käis ise Tartu Ülikoolis ja, ja me saime ta ju kokku, tal oli mingit rõmmyndokupeedisinud näha ja siis me saime temaga kontakti ja siis ta ütles, et noh, et ta, et umbes, et Moskvasse siis alati, et very welcome. Ja, ja siin me nüüd lihtsalt läksimegi.
On ju rongiga?
Kuid mitte, mis too andmeside kiirus siis tuleb, kui arvestada, et sa sõidad rongiga sinna, siis kupeedia flopid ära jäetud ära, siis see tuleb.
Ma ei julge öelda, mis kolmsada kuuskümmend kilobaiti oli üks ketas või? Nii-öelda päris päris soolikaid, ütles, et, aga, aga eks ta oli vast kõige kiirem viis ikkagi.
Et.
Iimil tuli ikkagi mingisugused aastat hiljem ja too käis ikkagi kord päevas, helistasid modemiga sisse ja tõmbasid meilida.
Aga ühel hetkel sai keskkool otsa siis oleks siit õppima midagi.
Tartu Ülikooli saatikat sõjaväkke võtma. Õnnestus ära viilida, noh, selge. Et Tartu Ülikooli rakendusmatemaatikat, aga tollest õppimisest tegelikult palju välja ei tulnud, tuleb jah, ma istusin ikkagi see lätete juures edasi, nii, nii nagu.
Kooli ajal, et.
Jah esimese kursuse ma.
Tegelikult tegin ära kõik matemaatikaeksamid olid viied, aga aga inglise keele ka kukkusin välja. Kuna hõbemedaliga lõpetasid siis siis sisseastumine uuesti väga lihtne pidi matemaatikaeksamitega, mis minule oli triviaalne. Aga, aga noh, siis ma enam ei viitsinud üldse loengutesse minna, sellepärast et noh, kõik matemaatika eksami tehtud, me oleks pidanud Ants inglise keelega seal esimese kursusega.
Ja siis.
Siis ma istusin seal äplitud, aga nüüd ma tegin loomakasvatuse ja veterinaariainstituudile mingisuguse dolla direktor olnud kolonn, tegi, tegi doktoritööd ja ja tal oli terve bussitäie tädisid, kes olid valmis andmeid sisestama, annan talle ei olnud kuhu neid andmeid sisestada, mis valmis arvuti. Ja siis ma tegin talle talle programmi, mis, mis nüüd on meil võimalus sisestada. Noh, seal oli siis oluline, kui uus NTFS teha, selline, et tädid eksida ei saaks kuidagi. Et noh, tooli kõige keerulisem kindlasti teha.
Ja noh, torudsus oli lihtne tegelikult.
See on vist midagi arvates seal mingisuguseid mingi statistikat lihtsalt.
Nad olid mingid piimaproovid kus siis laktoosi, valgu, igasuguseid hulk kahte eestikud ja noh, ma ei mäleta seal mingit korrelatsioonianalüüsi, tuli teha mingisuguseid noh, tema ütles ikkagi olnud algoritmid ette, mida tuleb teha selles mõttes, et oma või siis on matemaatiliselt nõu anda, aga aga üldiselt ta ikkagi teadis ise mida ta tegi. Mis tähendab seda, et siin kuskil palgal siis juba.
Ma olin poole kohaga palgal Tartu Ülikoolis, jah, seal arvutiklassis ma küll insenerina ja, ja piimaga see loomakasvatuse Veterinaaria Instituut. Kuna kuna ta mulle kuidagi nagu ühekordselt maksta ei saanud, siis mind võeti sinna tööle. Aga ma ei käinud seal kunagi lihtsalt selles mõttes, et ma olin seal mingi aasta või, või, või kaks olin tööl lihtsalt selleks, et saada nii-öelda tule programmi eest tasu, siis ma ei viitsinud palka ka minna välja võtma, noh siis pangakontosid eraldi. Ehk siis siis ta direktor tuli mulle tagajärgi ja viis mind sinna sellepärast, et ta ei olnud kassapidaja kisa ära kuulata.
Tuldi autoga järgi riigi raha saama. Täpselt noh, programmeerija magus elu.
Jah ei too oma vastuses, et direktor valdkonnad oli väga-väga nagu lõbus sell, et.
Et nende oma inimestega, ta oli hirmus kuhi. Alati kui me sinna läksime, siis ta kõigepealt sõimas kõigil näo täis, aga aga, aga väga ettevõtlik tüüp selles mõttes, et ma mäletan kunagi ma olin kodus isaga saunas. Siis ema tuli sõnul, et kui on mingi mees, tuli. Ja noh, sama olgu need siis tuli, tal oli midagi kiirelt vaja. Ja mul ema ütles, et ta ka nagu enam-vähem minna tutvust ta Montreali uksest sisse astunud ja läinud kohe elutuppa ja maha istunud. Teie eelarve vaadata, et kui on probleem, et vahet ei ole, kus, kus ma Alementaator.
Et noh, selles mõttes väga sihikindel
Aga see, et sa nagu loengutesse jõudnud, siis mingi asi võltsimisele arvutite juures kinni. See oligi see Assembler ja pusimise huvi või, või mis sa siin-seal võidis?
Nojah, selles mõttes, et ma tegin Assembler, siis ma kirjutasin tekstiredaktori.
Kuhu sai ikka päris ohtralt igasuguseid kitsesid. Tehtud too oli kindlasti kõige-kõige nagu keerulisem masin, mul peaks vist isegi too paberi peal väljatrükituna kood alles olema, tuli.
Kas kas kas viis tuhat või kuus tuhat, keda Assembler sa seda üldse nii palju, et üksjagu eco asembri koodi mõttes on seda palju, aga arvestada, et nagu tekstiredaktor viie tuhande reaga pole paha.
Et jah.
No maitsesin seal ühel tütarlapsel, kes mulle väga meeldis, hoidsin tal ka kursusetöid teha ja tuleks selle tekstiredaktorit nagu vaja. Et läheb ainult muidu, muidu oleks pidanud kirjutusmasinal trükkima. Et noh, et, aga noh, arvutis ühtegi kohalikku tekstiredaktorit ei olnud noh, oleks ka saanud üht või teistviisi teha seal mingisuguseid hädiseid, asjad olid aga aga noh, selleks, et kõik suured-väiksed, tähed, sellised asjad, noh.
Ei olnud lahenduste, siis ma keetsin.
See.
Jaan Tallinn kirjutas ka Prangli endale ühe esimese asjana kirjutatud tekstide tahtma.
Ka seminarist mis tekitab mõte, et kas see tähendab siis seda, et igasugune kuramuse interneedus ja muud niisugused asjad on teinud nagu hoopis karuteene. Et varsti, kui sa tahtsid niux tekstide traktorist ise kirjutama ja nüüd võtavad Internetist täpselt sellise nagu vaja võtta Tii või noh, mis iganes.
Noh, eks ta siis oli ka natukene lihtsalt see, et sul ei olnud neid programme kuskilt saada, eks, eks Ameerikas olid Apple'i jaoks ilmselt kõik programmid olemas. Aga, aga nad ei olnud lihtsalt Eestisse ja eksis.
Töö olnud ja siis tegi teiseneks.
Noh, aega ka oli ja.
Mis sul see tol ajal sinu ettekujutus oli? Et kuhu see kõik nagu viibekas istubki, nagu järgmised kakskümmend aastat nagu Apple'i Apple kahtede juures Vanemuise tänavas või?
Mul mul ei olnud mingisugust, väga konkreetset plaani küll ausalt öeldes, kuhu see viib, selles mõttes, et,
Noh, siis siis tulid, on mingisugune meilinduse käimapanek seal Vanemuise tänavas ja noh, ta oli kuskil üheksakümnes aasta siis tolle pärast, et siis siis Taavi Talvik kutsus mind Postimehe toimetusse.
Sinna ta oli mingisuguses kuu juuniks valmis pannud ja mingisuguse hulga terminale, mille kaudu siis sada ajakirjanikku artikleid sisestasid. Emaks oli, vist on tekstiredaktor saa. Ja, ja eesmärk oli siis teha eesti keele õigekirjakontrolli programm soo sinna peale ja tollega ma siis seal tegelesin, ühesugused läksid sealt loengus, tänavalt, Postimehesse jah. No no ma käisin seal Vanemuise tänaval ta ikkagi tolle pärast, et et no tädikesed, kes arvutiklassi seal nagu haldasid. Tehniliselt liiga võimekad ei olnud ja, ja noh, siis oli kõige kasulikuma sel õhtul läbi käisin ja meelega mingite asjadega nõu andsin, aga aga jah, siis tuli ikkagi Postimees nagu.
Mullu Taimsoo.
Ja see jällegi, et see eesti keeles Belleri või õigekirjakontrolli tegemine ei ole nagu triviaalne asi seal keelest ka ei ole, ei ole triviaalne asi.
Ja ja siis siis ma avastasingi, kui neetult keeruline see eesti keel ainult nagunii ka, et iga teine sõna veel mingi erand olevat. Väga tüütu oli, et ega me teda valmis ei saanud.
Tegelikult et me ei saanud teda valmis. Jah.
Sest umbes üheksakümne kolmandal aastal kuskil hakkas tekkima filosoftide niisugused asjad, et nad tegid vöödilise eesti keele spellerit. Ja see oli ka ikka päris suur tükk pusimist ja seal oli selleks ajaks riismeid sedasi läinud arvutusvõimsus ka, eks ole.
Jajah absoluutselt aga, aga noh, üheksakümmend kolm oli juba see aeg, kui ma tulin.
Härra Tallinnasse Anna kohta.
Sa tulidki otse Postimehest.
Et seal oli mingisugune lühikene periood Postimehe ja Hansapanga vahel ka tegelikult kus ma olin mingisuguse sulgi Birmas
Saan ma ka kirjutasin mingit programmifaile. Aga, aga ta oli nagu vägagi selline kaootiline koht selles mõttes, et too bisnis läks tollel hulgifirmal nagu hirmus hästi ja iga kuunud kolm kutti, kes, kes talle omanikud olid, ostsid, ostsid igaüks endale uue BMW. Tolmutasin Tõndega ümber tolle maja sõita, et et ei olnud liiga motiveeriv keskkond. Tegelikult.
Eks neid kahjuks, mis sa teed? Kuidas, kuidas see kuidas see sinna Postimehe kutsumine ees pidid siis tolle Taaviga kuidagi tuttav olema, kui kus tahes üles leidis.
Ega ma nüüd ei julge öelda, ausalt öeldes peast, kus, kus ma Taaviga tegelikult tuttavaks sain selles mõttes, et.
Sel ajal, kui mina äplite taga istusin, istus Taavi tegelikult sealsamas Tähe neli kus ma esimest korda nairidega kokku puutusin, istus Tähe neli keldris, kus oli mingisugune IBM PC.
Mugi.
Ja kas Taavi tegi midagi äkki Tartu Ülikooli Raamatukogule ja mina olin ka tolle kuidagi seotud ja kas me äkki seal Tartu Ülikooli raamatukogus tissid, aga saime kuidagi kokku?
Jaa.
Ja ja siis noh, vanetsi näplethaavile ja dominets mulle toda PC-d. Ja siis oli sihukene mände King's kvast ja ja siin me Taaviga mängisimetada Kings kosti seal Tähe tänaval
Ja noh, sealt me tuttavaks saime me tolle King's mästyle
Kings, kes on ju King's Quest, on ju metsakas. Just.
Et tollega tollega läks ikka aega, et lõpuni mängida, et me istusime ikka palju.
Niisiis ühel hetkel abi oli see Postimehes ja siis tal oli abi vaja, siis ta kutsus sind täpselt. Aga vaat nüüd seejuures Hansapanka sattusid. See on huvitav lugu.
Hansapanka ma sattusin suutnud aduda vedelikule. Taavi Talvik ütles valitsussides mahvan tol ajal. Ja mitra lingist Rainer Nõlvak. Mahvan oli see, kes küsis Taavi käest, et et Tõnis Sildmäe otsib kedagi, kes juuliks tunneks.
Panka.
Ta ütles, et tema küll ei taha minna. Ja küsis minu käest. Ja mõtlesin, et ah suva, et ma võin ju rääkida ja kuulata, et mis, mis seal siis. Deemon. Ma tulin Tallinnasse Tõnis silmaga rääkima, Sildmäe küll jättis mulje, et tal on terve bussitäis juunitsi mehi ukse taga. Jah, ega kass, keda kõiki integreerub, aga, aga vist tegelikult selle minu ühtegi ei olnudki. Igatahes igatahes ma sain nagu sinna tööle.
Jaa.
Ja siis tus, kool, juuniks oli sinna juba ära installitud ja, ja Tarmo Pajumets püüdis päädistada vaakled sinna skoobiale, inste skoori UNIXi peale installida. Aga ega nad tulles skoori unistust ka midagi ei teadnud, nii et esimese päeva lõunaks nad mõlemad läksid sealt konsoolid ära, täna ja rohkem sinna tagasi ei tulnud, et kui nad vaatasin, et ma, ma vist tean natuke rohkem.
Noore inimese hästi aru, kust nende teadmiste piirid on. Aga tol hetkel oli pank kui selline oli ju juba olemas. Ja, ja muidugi noh, mille peale käis nüüd, mis infosüsteem oli Oracle'i talle seal kaks paigaldama, siin?
Too käis paradoksi peal. Paradoks on siis oli paha, tooks andmebaas, aga aga eks too Oracle'i andmebaasi majja toomine oli nagu üks väga paljudest. Maa on nagu Hansapanga. Edu aluseks olevates strateegilistes otsusest, kuidas tollased juhid suutsid ette näha, vähem tooks, töötas tol hetkel täiesti normaalselt. Ei olnud häda midagi. Aga, aga juba oli Tõnis Sildmäe nagu välja raadio tegelikult me peaksime nagu mingisuguse tõsisema andmebaasi mootori sinna alla panema. Et esialgu tulekski novelli peal toovacle. Ja aga siis me saime tule Skujunitsiverda nii kaugele, et tsime, leidsime, turnisime skoojunud.
Ja vaat räägingi sellest korra lähemalt, et see tahab niukest. Ühesõnaga see tähendab seda, et kellegi teise peas siis ilmselt oli arusaam sellest, et ummik, arhitektuursed, otsuseid, info, arhitektuursed, otsused on kuidagi seotud nagu äriga või nagu äriedu aluseks. Üheksakümnete alguses see ei kõla nagu üldse, nagu triviaalne teadmine. Kusjuures oli.
No ma usun, et et too seltskond, kes seal oli tol ajal, oli ikkagi ka selge arusaamine tolle paradoksi tehnoloogilistest piirangutest ja samal ajal oli oli ka arusaamine, kuhu poole see pank liigub. Ehk siis ma arvan, et tolleks hetkeks, kui mina sinna tulin, äkki oli, oli seal minus kõrvallaua peal juba esimene sularahaautomaat tegelikult oli niisugune suhteliselt pisikene, mis mahtus laua peale IBMi oma kahte eile käinud sisse, vaid tuli magnetiga lihtsalt läbi tõmmata. Et noh, ma arvan, et Aadeeemmide asi oli üks ka, mis, mis nagu tolle paradoksi andmebaasi piirangud välja tõi noh, samamoodi kuna klientide arv kasvas plahvatuslikult noh, ilmselt ka tolle pealt nähti, et et too paradoks ei suuda tegelikult kui selline kasv jätkub, ära teenindada.
Teine asi, mis mind on ikka huvitanud, on see, et, et samal ajal toimetati päris mitmes pangas ka nagu valmis softiga vastati lihtsalt kuskilt Briti maad, mingisugune pangas oht ja tehti panka, tuleb, miks, miks, miks Hansa teistmoodi?
Ei oska öelda selles mõttes, et vabalt võib ka olla see, et et too seda oskavad need öelda, kes päris algusest olid tolle pärast, et etas. Kuid kuidas ta, kas ta.
Kuidas ta spinn Development sinna Hansapanka tuli? Et noh, nimi dispinda hakanud, et ilmselt oli seal siis mingeid Developer häid. Ja ilmselt ta esialgne ülesanne, mida teha tuli, oligi mingisugune väike tükikene ja kui oleks hästi, siis sealt hakkas asi arenema. Ma, ma ei oska öelda.
Udud sprindi lood, need on mingisugune Greriti algus.
Jah, selles mõttes, et ole, kui minagi tööle läksin, siis esimene palga maksta oli tegelikult spinn Development siis Sistus pin Development minu meelest nimetati lihtsalt grebitics ringi. Ja, ja mingi aeg hiljem siis ma saan aru, Londoni kindlustusfirma ütles pangale, et kuulge, et teine, kuid, ja IT-st mitte midagi, et kogu asi on nagu väljas mingisugusest täiesti iseseisvad ettevõtted, et kuidas ta nagu Magnitski ei huvita ennast lõppes sellega, et et Hansapank ostis siis Tõnise käest, need rebiti aktsiad ära ja kõik me tulime siis Hansapanka tööle, et Nabala Kerbitit jäi siis noh, ta juriidiline keha jäi alles ja kuni tänase päevani siis praeguseks swedbank Support OÜ nime all olemas.
Huvitav on see, et see kultuur oli ikka jätkuvalt nagu kleebiti oma, sest kui mina tulin pangast ära aastal kaks tuhat kaks ma pakun, siis mina sain viimase särgi, mis mulle väljastas, oli kleebiti looga. See oli oi kui jube elujõu vesi.
No kindlasti oli selles mõttes, et on tega ega ta noh, mitte ainult mitte ainult Kebit, vaid toob pank ise tervikuna oli tegelikult äärmiselt elujõuline.
Et noh
Ütleme, vähemalt kuni tolle hetkeni, kui kui Hansapank Hoiupangaga liituti et siis toimus ikkagi suundub tundeline muutus, esimene soojem kultuuridanud tuli lihtsalt väga palju teisi inimesi juurde.
No, mis ta ta kultuuri nagu püsti hoidis, kust, kust see tuli?
Ma ma olen Duda mõelnud, et ma ei tea, ilmselt ühelt poolt ilmselt oli kõigile inimestele, kes seal olid, oli ikkagi väga selge saavutusvajadus, et ikka oma asja väga hästi teha. Sellepärast, et seal isegi ei pidanud minu meelest need, kes võib-olla ei performinud piisavalt hästi lahti laskma, vaid nad läksid ise ära. Mil puhul, et, et noh, seesama et näed, kui ma ütlesin, et Pajumets seda oleks installis. Tegelikult oli üks mees seal veel kõrval kes tolles kuu juunis sinna Collins, Tallinn tegelikult kui ma tulin sinna siis ma saan aru, et too mees ise läks paari nädala pärast ära, tegelikult teda ei lasknud keegi lahti, et ta isegi ütles lahkumisavalduse oleks hea, sellepärast et ta sai aru, et, et noh, et noh, temal ei ole midagi teha selleni, et tatud Asko püüdis kuidagi midagi teha, aga kuskil võib-olla. Ja noh, tuli nagu absoluutselt kõigile olid ühesugune kultuur, et sai, sai seal pidanud kaks korda kellelegi ütlema, sa teadsid, et asi on tehtud.
Jaa. Üks asi, mis mind on, kui ma jutu puhul on veel huvitav, et alustasid pihta siis võid kirjutada Selveris PIN koodi, see on ju puhas nagu arendaja. Aga vallas sa läksid kuidagi kohe selle asja nii nagu opereerimise peale. Kuidas see toimus? Ja miks oskad sa?
Ja ega seal mingisugust nagu teadlikku valikut väga ei olnud, selles mõttes, et ta töö tundus huvitav aga Mogakla baasi olnud varem näinud ja selles mõttes, et noh
Ma ma kuidagi ei mõelnud, et ma olen programmeerija selles mõttes, et noh, tegelikult ju ka seal arvutiklassi säplite juures noh, ta tööülesanne oli tegelikult kõigi nende inimeste Assisteerimine, üllad, hoid, et nood arvutid töötaksid, et probleemid nende arvutitega oleksid lahendatud. Programmeerimine oli puhtalt nagu hobi tolle töö kõrval. Kui kuigi noh, kogu asi algas loomulikult programmeerimisdetailide programmeerime Apple'it meeldima. Et aga kuidagi ei mõelnud. Ja ma arvan, et tollal ka ju liiga palju konteksti, et need on arendajad ja need on ülal hoidnud hiljem et, et ma arvan, et, et ta tuli hiljem, et et siis kõik lihtsalt
Sest siis, kui siis kui mina uksest sisse saabusin, siis oli see juba ammu olemas.
Aga ana ja Saldseks muidugi.
Aga kuidas sulle too kuue uniks sai külge juba Postimehes kujunes see testimine ja Oracle'i, kuidas see sulle külge sai?
No see saigi külge sealt Hansapangast, selles mõttes, et toetlejali
Jah, lihtsalt tegema. Ja mitte väga palju aastaid hiljem oli see üks maailma suurimaid oraakli koodibaas, et minu meelest käis mingi praakspetsialisti, mulle meeldis see jutt sellest, kuidas nad seda ei ole kuskil näinud, et kellelgi on sihuke asi tehtud oraakli peal.
Võis olla küll sellepärast, et,
Vilve Vene auto ostja, kes seal kirjutasin, et ettuda Piia Sikk välja, kirjutati seal seal usinasti noh, toda oli tõesti väga võetud, osa oli väga palju noh, kuna too alati fookuses olid ju tehnoloogia vaid ikkagi selles mõttes, et ilmselt ilmselt oli nii teda kõige efektiivsem teha, jääks. No baasi protseduurid kestlik kiiremini kui mingisugune noh, klient-server asi, et.
Ja sul ei tekkinud seal tunnet, et no see, et, et see Oracle'i püstipanekul padjal püsti pandud, no las ta siis nüüd käib Promeeriks parem.
Noh, eks eks programmeerima pidi veel ikka natukene selles mõttes, et,
Skripte tuli kirjutada, mis mis siis kogutud asju üleval, et kas või seesama, et, et kuidas too, kui, kui sa tolle skool juuniksi masina üles poodi, et kuidas tuhat läheb, aeg käima pannakse. Ega tollel Oracle'i installi juures mingeid skripte ei olnud. Ainult et sa kirjutasid ise need skriptid, mis tollebaasi käima panid, vist enese käima panid. Kogu tolle asja tegid kogu päkapike tegemine tolleks pidiskepsid kirjutama, noh, lisaks sellele kõik need Päts protsessid, mis, mis olid tehtud siis kirjutatud kogu nende käivitamine, et oleks tulist skriptid teha. Et noh, nüüd metsa retk sai tegelikult kirjutatud ikkagi päris päris palju.
No see, mis sa kirjeldad, on päris nagu keeruline asi mis peab olema siis nii arenda, noh, arendamise mõttes, et see on mingi peeles kull, mingi kuramuse mindi, siis pätsid ja Oracle'i baasides nende seas see kõik kuidagi nagu terviku moodustama, et see koos püsiks ja oleks nagu tehtud praegu, kuidas te tervik tekkis, kes seda juhtivi?
Kes arhitekt oli?
Ega siis kedagi arhitektiks ei, ei nimetatud, aga.
Noh, ma julgeks siis arvata vast.
Vastu tulles tantsu kontekstis arhitekt oli ikkagi Vilve Vene. No vähemalt ütleme mulle selline mulje jäänud, et tema oli, tema oli siis tänapäeva mõistes arhitekti keegi selliste terminitega.
Ega tänapäevalgi väga-väga lihtne kasutada.
Et see kontseptsioon, kuidas, kui too kõik kokku töötab tarkvaraliselt, et ma arvan, et ta tuli ikkagi ennekõike ilmelt mingisugune ütleme, non saksa heli laiem võtsid need tekitasin mina. Et NATO samas skriptide asi, et noh, et iga too Siibäam, mis pingid Pätsi tegi, et too ei oleks erinev, et on kuidagi ära standardiseerida, siis ma pidin mingisugused, et mitte funktsionaalsed nõuded esitama, et et nad kõik oleksid ühetaolised, saaksin kasutada mingit ühte skripti paljude asjade käivitamisest.
Ja see on see jõud või mitte, funks aasta nõuet juurde sealt järgmine asi, et kuidas noh, Postimehes on ka ikkagi suur ajaleht, aga see ajaleht ilmub ka siis, kui need ajakirjanikud oma trükimasinaga kirjutavad, oma lood valmis. Aga pank trükimasina peal nagu enam ei käi. Mis tähendas seda, et ühel hetkel see naguniisugune pusin ise ja vaatan, kuidas mossel teinud pidi nagu maad andma sellele, et on mingisugune struktuur ja mingisugune niisugune formaalsemisel kuidas, kuidas see sündis, kas see oli mingisugune otsus, et nüüd hakkame nagu korralikuks või see sündis kuidagi sujuvalt?
Noh, tolle tolle igapäevase ülalhoiu kõrvalt sa pidid tegelikult ikkagi paratamatult noh, igaüks kuna kuna ta kasv oli nii kiire, siis igaüks pidi tegelikult vaatama, aga mitte ainult seda, kuidas see asi täna ära rullib, vaid ka, milline see asi nagu aasta pärast välja näeks. Ja, ja noh, kindlasti Tõnis silma ka Fassilteeris seda, et, et tuleksid igasugused erinevad kontaktid kes, kes mingisuguseid uusi lahendusi pakuksid ja, ja nendega sai, sai konsulteeritud ja äkitudeni. Nii need asjad arenesid edasi ka selles mõttes, et noh Cuzco juuniks ju samamoodi tollel tuli tehnilised piirangud ette.
Üheksakümmend kuus, üheksakümmend seitse kuskil sai ju happe uksi vastu välja vahetatud, enne toda sai, oli mul nii HP kui Sammy serblase laua peal ja sai võrreldud siis kumb, kumb nagu kiirem on noh, tolleks ajaks oli oli ta panka piisavalt suured, siis, siis oli selge, et, et me tegelikult ei pane ühte masinat, vaid me paneme klastri. Sain nende Vendoritega nad klastri lahendused läbi räägitud.
Jah.
Et ega, ega sellist, nagu et mingisugusel hetkel oleks mingisugune Mäppe toime, Tennoli on Ahja, siis, siis tehti kõik asjad korda. Et kõik, kõik see arenes tegelikult ikkagi evolutsiooniliselt, neid asju on lahendusi vahetati iga aasta välja, tegelikult tuttav on vastu tolle pärast, et testi ei oleks lihtsalt toda. Kümme aastat kestnud olukorda, kus, kus iga, ma ei tea, üheksa kuud on olnud kahekordsus, klientide arv.
Käive kasum mis iganes. Kõik numbrid kahekordsest kümme.
Jah, ega last kenasti sellist kasvu ei kujuta tänapäeval nagu väga ette enam kui sa just kuskil Skype moodi asja sõidelt.
Nojah, ega, ega nüüd ettevõtted ongi maailmas väga vähe vist, kes, kes nii kiiresti nii pikalt suudavad kasvada, noh, oli, oli mingisugune substes of Estoniat.
Ma mäletan, aastal just sajandi lõpuks oli sinna panka tekkinud mingisugune niisugune üsna ike spetsialiseeritud tiim nagu inimesi, kes opereeris seda kupatust seal. Kuidas te tiim tekkis, kes igalühel oli oma mingisugune valdkond, millega tegeles vahetanud ja kolmekesi moodustasid te niisuguse asja, millest Veeber seisneb kogu maailm püsti. Et kuidas tuli, kuidas ta kolmik tekkis.
Tohutu tekkis ka aja jooksul selles mõttes, et noh, Madis oli, saar oli enne mind olemas ja on olemas, muutmist, ma siis oli, seal on praegu ka olemas. Et.
Ma ei teagi päris täpselt, mis tema roll päris alguses oli, aga aga ikka ikkagi sel ajal, kui kui mina seal tolle Orkla Masingu toimetama hakkasin, siis minu asi oli nagu toit, tehniline pooletu, andmebaasi ilmsin, töötaks ja Madise asi oli siis luua sinna uusi tabeleid ja teha indekseid ja vaadata, et päringut hästi käivad nii-öelda see tagasi. Ja, ja noh, too, too hall jätkus tal edasi. Toomas Soomets tuli.
Ma pidin peaaegu ütlema, et ta tuli koos Hoiupangaga liitumisega ka tegelikult ei tulnud. Tegelikult tuli kaks aastat enne seda, kui ta töötas Hoiupangas, aga ta tuli kaks aastat enne seda, kui juba kärastati. Et, et tolleks ajaks, kui avastati, istus teha õigel pool lauda juba.
Et ja, ja noh, toomas toomas siis oli, oli selles mõttes nagu täijendastuda seltskonda, et kui Madis oli nagu kõige üle meie nii-öelda tahta, leiab mina, teadsin toda köögiandmebaasi Encini osa siis siis Toomas oli see mees, kes, kes nagu met vöögistest olid, siis täiesti jagas. Noh, siis siis too Kukkondiski nagu kogu tuletehnoloogilisest äkki, et põhjest tiim töötab hästi. Noh, mõistus minu arust toas ühes infoväljas kogu aeg alati on võimalik nagu öelda, mis toimub.
Kas sul oli juba tol ajal, ma tean, sul on Vaarmani huvi? No see oli juba tol ajal olemas. Sest ma mäletan, et kapi otsas oli makk ja sealt tuli aeg-ajalt tuli niukest eepilist klassikast muusikat.
Jajah see oli, seda ma ei oska öelda. Jää ei.
To klassic, jah, ausalt öeldes ma isegi päris A aastaarvu jälle julge öelda tolle pärast, et ei olnud veel omas on, et esimesed siin viibima Internetist ostsin tolle firma nimelist siidi, no punkt kommu. Et ja siis sai tunda klassikalist muusikat, seal mängitud? Jah, mitte küll vaagne, et põhiliselt tegelikult ma julgeks arvata Mozartit tol ajal põhiliselt Mozartit. Et jah, ma ostsin ka mingisugused päris alguses, ma ostsin mingisuguse siniHenrikuga Ruusa plaadid. Aga, aga kas ma ostsin Mozartit ja siis me mängisime seal meid neto ka kuidagi ise kanda? Me, eks me tegime erinevaid asju. Mingi periood oli, kus, kus välja tuldi.
Öeldi, et.
Teatud lõhnad on teie toas igapäevaselt tunda et.
Mingi periood oli tõesti, kus, kus meil oli alati konjakipudel kapis ja ja päeva sai alustatud difitsiga konjakiga. Et loomulikult mingit joomist ei olnud, aga, aga noh, eks tollest ühest pitsist juba noh, sul oli ka klaaslaua peal, võib olla kuni lõunani seal, et ega keegi ei joonud, aga lihtsalt natukene. Ja WRC ralli oli ka, mille Toomas siis püsti pandi, esindatud on FC hallit, mängisime ka seal mingisugune periood ikkagi, et noh, jälle, et ta tahtis väga palju network. Aga kuna Toomas selline löögi põhjal siis siis siis ta bänd vist kellelgi oli siis meie toas ennekõike.
Või akustilise klassiku pidule.
Klassiku huvi tuli sealt Enrico kohustus tegelikult, et mul oli vanematel kodus, oli Vituaalse salto hälli raamat Totu välja oli vist kaugelt sugulane ja Mehhikoga osales, ta kirjutas nagu Henrikuga ruudust raamatu noh loomulikult itaallane ja kuna ta oli sugulane veel, noh, siis ta on ülimad ülistav, aga, aga ta oli huvitav lugeda ja jättis väga sügava mulje. Ja siis, kui internet siis oli võimalik tellida, siis ma tellisin huvi pärast tule siit-sealt Sindi nafta, Edgar Ruusa plaate. Ja teine asi, mis häiris, oli ikkagi kaheksakümne viienda aasta Milos farmani Amadeus. Mida ma kindlasti soovitan kõigil vaadata, aga suurepärane film. Et Mozart ja sealt tuli ta Mozarti või. Ja noh, kui midagi sellist oleks, siis ma tellisin mingisuguseid raamatuid Mozarti eluloost mingi neli-viis ja mis mul on kodus üle tuhande lehekülje paks, et jah, sealt edasi. Ma ei tea, mis, nagu Beethoven, Schubert, Schumann, Tšaikovski. Mis projektid?
Töötan G4S-i, turvalise Everesti baasteenuste arendusjuht, noh sisuliselt siis vastutan õlahoiu eest. Et et kõik asjad oleksid püsti ja valvatud. Jah, selles mõttes, et noh see ongi, et mitte mitte siis ainult IT, vaid, vaid ka see tehniline valve, kuhu puutub, siis ka see raadiosidevõrk on meil hästi, on et kõik need signaalid jõuaksid siis siia keskele kokku.
Selge see ka huvitav ameti, nagu nagu Viksu need seiklused alates sellest tassemberist seal äppidel.
