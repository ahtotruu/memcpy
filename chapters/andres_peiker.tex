\index[ppl]{Peiker, Andres}

\question{Kuidas sina arvutite juurde said?}

Oli mingi 1984. aasta, äkki? Suht juhuslikult tegelikult selles mõttes, et ma 
õppisin siis keskkoolis ja käisin mingisugustel füüsika loengutel Tartu 
Ülikoolis\index{Tartu Ülikool}. Ühe tolle loengu lõpus mees nimega Otto 
Teller\index[ppl]{Teller, Otto} astus auditooriumi ette ja ütles, et kes on 
arvutitest huvitatud, võivad natukeseks veel  siia jääda. Ja noh, mingisugune 
seltskond jäi ja Otto Teller viis meid Tähe 4  
õppehoonesse\index{Tartu Ülikool!Füüsikahoone}, kus oli kaks Nairi arvutit, 
Nairi-K\index{Arvutid!Nairi!Nairi-K} ja Nairi-2\index{Arvutid!Nairi!Nairi-2} vist. 
Otto näitas noid ja ütles, et põhimõtteliselt siin nagu mingitel õhtustel 
aegadel on võimalik käia, programmeerida ja proovida asju.

\question{Millest ma kohe järeldan, et sa oled Tartu poiss?}

Jah, absoluutselt. Esimesed kakskümmend viis eluaastat ma elasin Tartus. 

\question{Ja mis ma veel järeldan on, et kui sa keskkooli ajal kuskil Tartu 
Ülikoolis loengutes käisid, sul pidi olema mingi reaalainete huvi?}

No ma õppisin Tartu 1. Keskkoolis\index{Koolid!Tartu 1. Keskkool}, meil oli 
matemaatika-füüsika eriklass. Käisin olümpiaadidel, ei mäleta täpselt, 
kust too ülikooli loengute teema üldse tuli. Füüsika tundus mulle nagu kõige 
põnevam asi üldse, et siis  saigi seal käidud.

\question{Selle loengu lõpus, mind paneb imestama, et keegi üldse ära läks. 
Kõik sedalaadi rahvas tundus arvutite vastu huvi tundvat, või siis ei olnud 
nii?}

Ei, ikka ei olnud. Ma arvan, et ikkagi pooled läksid ära. Esimesel korral 
käisime neid arvuteid vaatamas, siis öeldi, et järgmine kord saaks nagu sel 
päeval tulla, siis tuli juba vähem inimesi ja lõpuks jäi mingisugune, ma arvan, 
kolm-neli inimest võib-olla alles, kes seal nagu rohkem käima hakkasid.

\question{Oli see mingisugune ring või lihtsalt Otto isetegevus?}

Ma enam nii täpselt ei mäleta. Ma arvan, et Otto Teller\index[ppl]{Teller, 
Otto} ikkagi seal natukene juhendas ka alguses, et mis ja kuidas. Kuidagi me 
tolle AP 
programmeerimiskeelega\index{Keeled!AP}\sidenote[][-2cm]{\begin{russian}АП 
(Автоматическое Программирование)\end{russian} oli Nairil kasutusel olnud 
programmeerimiskeel. Kuna Nairi-2 oli väga levinud, muutus AP keel ka 
universaalseks algoritmide kirjeldamise keeleks venekeelses teaduskirjanduses. Keel oli muu 
hulgas kuulus ka oma mõnevõrra ebasündsa konnotatsiooni omandanud võtmesõnade 
poolest.}, mis seal Nairide  peal oli, tuttavaks saime. Ma arvan, et läbi 
Telleri, tema rääkis.

\question{Aga mingeid raamatuid või muud sellist?}
Ei, seda ma küll ei mäleta, et oleks olnud.

\question{See on huvitav asi. Läbivalt inimesed ei suuda meenutada, kuidas nad 
programmeerima õppisid, see lihtsalt tuli. Aga mida te tegite nonde Nairidega?}

Seal sai ikkagi teha väga lihtsaid  arvutusprogramme. Tollel 
arvutil ju kuvarit ei olnud, ta oli elektroonilise kirjutusmasinaga. Sa 
kirjutasid programmi, too tuli paberi peale ja  oli ainukene eksemplar  
programmist, seda sa pidid siis alles hoidma. Sest kui sa programmi parandada 
tahtsid, sa pidid vaatama toda prinditud paberit. Kõige kõvem asi, mille ma 
seal valmis tegin oli biorütmide arvutamise programm, tollal olid need 
tähtsad asjad. Noid Arvutuskeskuses\index{Tartu Ülikool!Arvutuskeskus} tehti 
ja  ma tegin Nairi-K\index{Arvutid!Nairi!Nairi-K} 
peale ka  biorütmide programmi. Ma saan aru, et too programm osutus seal 
populaarseks, et tollest 
minu perfolindist tegi keegi koopia ja siis lasti toda seal Tähe 4 
töötajatele usinasti välja, ilma et ma midagi teadnud oleksin.

\question{Need biorütmide algoritmid, mis liikusid, olid vist mõeldud käsitsi 
arvutamiseks? Kuidagi numbriliste meetoditega arvutati siinust vist.}

Ta oligi lihtne siinus,  lihtsalt sa pidid  sünniaja ütlema ja  nendel emotsionaalsel, füüsilisel,  seksuaalsel ja mis neljas oli, ma ei mäletagi, rütmil oli  siinuse 
lainepikkus lihtsalt erinev. Arvutas elatud päevade arvu ja tolle pealt 
joonistas nood neli siinust sisuliselt välja. Täiesti 
triviaalne asi iseenesest tegelikult, rohkem oligi too, et kuidas sa paberi 
peale toda siinust joonistad elektroonilise trükimasinaga. 

\question{Huvitav asi, mis oli toona oluline aga täna hakata inimestele 
biorütme joonistama\ldots}

Too siis mingisugune väga popp asi ja tundus tolle arvuti jaoks nagu niisugune 
jõukohane ülesanne. Ma arvan, et kas neli kilobaiti oli mälu tollele arvutil 
või? Ja ta oli sama suur kui mul, ma ei tea, kodus köögimööbel.

\question{Aga milleks füüsikud teda kasutasid?}

Samamoodi, ikka arvutamiseks.

\question{Aga mida nad arvutasid?}

Ei tea. Sellest nagu ei olnud juttu. Too Nairi-K\index{Arvutid!Nairi-K} oli 
nagu väiksem masin, teises toas oli Nairi-2\index{Arvutid!Nairi!Nairi-2}. Põhiliselt, 
ma saan aru, kasuti toda. Meie sinna masinale nagu eriti ligi ei saanud, too 
oli nagu rohkem hõivatud ja rohkem \emph{advanced} ka selles mõttes, et 
tal olid lindiseadmed. Need suured lindikapid, kus siis seda magnetlinti 
keerutati ja tollel ei olnud mitte see tavaline elektrooniline kirjutusmasin 
vaid sihuke trumliga printer. Mis suutis ikka päris kiiresti paberit  
välja lasta.

\question{Kas see arvutitega möllamine oli puhtalt selline põnev pusimine või 
tundus seal mingi sügavam asi ka taga olema, et see on see, mida sa teha 
tahadki?}

Siis oli see kõik puhtalt seotud tegelikult sellega, et kuna ma olin 
matemaatika-füüsika eriklassis, siis  Andres Jaeger\index[ppl]{Jaeger, 
Andres} ülikoolist andis meile programmeerimist ka kolm aastat. Aga too 
programmeerimine oli sisuliselt ainult mingite blokk-skeemide joonistamine 
paberi peale, me arvuti ligi ei saanud. Ja seal Tähe tänaval oli võimalus nagu 
ise järele proovida  seda, mida sa olid  paberi peal teinud. 
Toda  kooli õppeprogramm ei võimaldanud.

\question{Paberi peale skeemide joonistamine võib ju lihtsasti huvi ära tappa 
aga sul ei tapnud?}

Ei, ta ei tapnud kindlasti, ka too blokk-skeemide joonistamine 
oli huvitav. Kui sa tolle ülesande said, siis Andres ütles ka, et umbes, et kes 
suudab  kolme \emph{if}-iga teha on hea, kahe \emph{if}-iga on 
väga hea ja ühega juba kahtlane. Ja siis sul oli nagu eesmärk olemas, et sa 
pidid ühega ära tegema, too asi kõnetas mind. Ja siis sa said õpetaja käest 
kiita ka, et \enquote{Tõesti, ma mäletan siin võib-olla viis aastat tagasi oli 
meil ka üks õpilane, kes suutis selle algoritmi selliselt ära teha,  
väga hea!}.

\question{Ühesõnaga sinu jaoks oli too blokk-skeemide joonistamine niisugune 
ülesanne või pusle.}

Jah, absoluutselt. Loomulikult jõuga sa suudad tolle ülesande lihtsalt ära teha, aga et 
nüüd kuidas ta nagu kõige optimaalsem saaks, kõige parem. Too oli huvitav.

\question{Huvitav, et koolil tollal ühtegi arvutit ei olnud. Tartu linna peal 
arvuteid ju oli, miks koolist mõne juures ei käidud?}

Ei olnud, siis ei olnud ikka mingisuguseid arvuteid kuskil. Tähe tänaval olid 
nood kaks Nairit. Loomulikult seal ülikooli arvutuskeskuses\index{Tartu 
Ülikool!Arvutuskeskus} oli ES\index{Arvutid!ES EVM}, kus üldse nagu ise arvuti 
ligi ei saanud,  operaatorid lõid programmi sisse. Ja siis oli Füüsika 
Instituudis\index{Füüsika Instituut} Riia maantee lõpus, seal oli ka 
mingisugune PDP-11\index{Arvutid!PDP-11} äkki?

Ma arvan, et umbes tollel ajal kuskil Anne Villems\index[ppl]{Villems, Anne} 
sebis need Apple II-d\index{Arvutid!Apple II} ka tegelikult siis Vanemuise 
tänavale\index{Tartu Ülikool!Vanemuise tänava õppehoone}. Too oli koht, kuhu ma 
järgmisena jõudsin peale  Nairisid.  

Ma arvan, et see oli ka Otto Teller, kes meid sinna viis. Ma isegi tundsin nagu 
pärast natukene piinlikkust, et tema näitas noid masinaid ja siis ma tegelikult 
hülgasin tolle Tähe tänava ja ei käinud enam tema juures. Vahtisin ainult seal 
Vanemuise tänaval, Apple-d olid palju  ägedamad. Ikkagi monitor ja nelikümmend 
kaheksa kilobaiti mälu, oli ikkagi nagu ulmeliselt kiire.

\question{Kas Apple peal mängimine ka teemaks tuli?}

Jaa,  absoluutselt. Too oli päris hull selles mõttes, see oli kindlasti minu 
elu kõige suurem arvutimängude periood. Ma oleks peaaegu kahe klassivennaga 
keemia eksamile hiljaks jäänud. Sellepärast, et siis seisu salvestada  ei 
saanud, sa pidid lihtsalt nii kaugele mängima, kui said. Juhuslikult juhtus 
nii hästi minema, et oleks pidanud juba eksamile minema, aga tuli järgmine 
level ja pidid edasi mängima.

\question{Mis mängu te mängisite?}

Apple-i peal sihuke standardne mäng on nagu Pacman, mis on mänguautomaatides ja 
igal pool. Apple'i peal nimetati teda Super Puckman-iks\index{Mängud!Super 
Puckman}. Ma siiamaani pean teda kõige lahedamaks mänguks, mida ma olen kunagi 
mänginud. Toda Pacman-i oli kõigi teiste arvutite peal ka. Aga 
  oli  katastrofaalne  erinevust tolles algoritmis, kuidas nood 
neli kolli liikusid: kõigi ülejäänud arvutite peal, nii palju kui mina olen 
mänginud, liikusid nad \emph{random}-iga. Aga Apple-i peal oli neil oma kindel 
algoritm. Ja tulemuseks oli see, et kui sa ise tegid täpselt ühtemoodi siis 
situatsioon kordus mängust mängu. Meil olid esimese kuue 
leveli jaoks tegelikult sammud sisuliselt algusest lõpuni välja töötatud. Sa 
teadsid täpselt, 
kuidas sa terve ekraani puhtaks mängisid ja järgmisele levelile said. 
Sealt edasi oli sisuliselt mingisugused paar avangut, mida sai erinevatel 
levelitel kasutada.

\question{Põhimõtteliselt Pacman kui male?}

Natukene. Ja selle tõttu polnud võimalik teiste versioonide peal mängida, sest 
seal kollid liikusid lihtsalt  \emph{random}-iga. Arvutimängud jah, 
absoluutselt. No seal oli teisi teisi veel, aga Super Pucman kindlasti oli 
kõige olulisem.

\question{Ja programmeerimine ka kindlasti?}

Nojah, muidugi. Seal oli Basic\index{Keeled!Basic}. Ütleme, et alguses ma 
kirjutasin ikkagi Basicus, aga pärast sai ikkagi valdavalt 
assembleris\index{Keeled!Assembler} kirjutatud tolle pärast, et programm töötas 
oluliselt kiiremini kui sa ta assembleris tegid. 

\question{Kuidas Basicust assemblerisse hüppamine käis? Basicu võib tõesti 
suhtliselt lihtsasti üles korjata aga assembleris sa pead ikka täpselt teadma, 
mida sa teed?}

Ka Basicu puhul sa pidid ikkagi  arvuti arhitektuurist aru saama. Et kus 
tolles neljakümne kaheksas kilobaidis nüüd paiknes tekstiekraan, kus  
graafiline ekraan, kus  su programm, kus oli opsüsteem. Tegelikult 
arvuti arhitektuurist aru saamine tekkis Basicu kõrvalt suhteliselt kiiresti. Aga 
assembler tuli ikkagi tänu sellele, et osad asjad olid väga aeglased. 

Üks asi, 
mida ma seal tegin, oli orienteerumisneljapäevakute protokollid. Tollega 
alustas tegelikult Peep Abel\index[ppl]{Abel, Peep}, kes ülikoolis 
rakendusmatemaatikat õppis, aga ta lõpetas ülikooli ja andis mulle kogu 
tolle programmi komplekti üle. Aga see oli minu jaoks liiga aeglane. Andmemaht 
oli tolle  neljakümne kaheksa kilobaidi jaoks natuke liiga suur, seal oli ka 
mitme flopiga mängimist, et need andmebaasid ära mahuksid. Ja ma mõtlesin, 
et ma kirjutan \emph{from scratch} kõik assembleris. Kirjutasingi, kõik oli 
kohe palju kiirem. 

Seejärel sai kogu too Apple II opsüsteem tegelikult disassembleeritud, kood ära 
kommenteeritud ja  imestatud, et päris mitmes kohas Steve Wozniak oli 
hämmastavaid trikke teinud. Tolleks ajaks, 
kui hakkasin diassembleerima, siis ise ka arvasid et tead assemblerit juba 
väga hästi. Aga siis ikkagi oli paar sellist asja, mida avastasid, et 
\enquote{vau, kuidas teha saab!}, sihukene nagu  pisut \emph{hidden trick} 
tegelikult. Ühesõnaga, assembleris olid ühebaidised, kahebaidised ja 
kolmebaidised käsud. Ja trikk oli see, et ühte kolmebaidist käsku oli võimalik 
kasutada selliselt, et kui sul programm jooksis otse läbi, siis too 
kolmebaidine käsk ei teinud midagi. Aga sa said tolle kolme baidi viimast 
kahte baiti kasutada selliselt, et sa kuskilt eespoolt hüppasid tolle teise 
baidi peale, mis oli siis teine \emph{command}. Sinna kolmebaidise käsu 
viimasesse kahte baiti sa paigutasid tegelikult teise assembleri käsu. Sihukesi 
elegantseid trikke oli tehtud. Pärast püüdsid ise ka nagu mõnes kohas mõelda, 
et kas ma saaks toda nippi efektiivselt kasutada.

\question{Aga kust see teadmine tekkis, et nii saab teha, mingi teadmine oli ju 
selle kõige aluseks?}

Ei, no kui me tolle koodi disassembeerisime, siis sa pidid kogu tollest 
algoritmist aru saama, et mismoodi see asi töötab. Tegelikult oli asi selles, 
et kettaga suhtlemine oli suhteliselt aeglane, ma tahtsin seda kiiremaks saada. 
\emph{Seek time} oli seal üks asi, mis palju mängis ja asi lõppes sellega, et 
ma tegelikult kirjutasin assembleris ise ketaste kopeerimise programmi, mis 
töötas  opsüsteemist mingi kümme korda kiiremini. 

\question{Kümme korda!}

Sa pidid  optimeerima lihtsalt tolle pea liikumise. Kui sa tahtsid kogu ketta 
ära kopeerida, siis sa pidid, ma ei mäleta, kas seestpoolt väljapoole või 
väljaspoolt sissepoole sõitma. Tegid ühe liikumisega kirjutamise ära, 
mitte ei käinud edasi-tagasi. Standardselt käis ketta pea alati edasi-tagasi 
\sidenote{Flopiseadmed, nagu elektromehaanilised kõvakettad siiani, 
kasutavad andmete lugemiseks ja kirjutamiseks pöörleva ketta pinna lähedal 
liikuvat tundlikku pead. Ja üheks kõige suuremaks pudelikaelaks andmete 
liigutamisel flopilt oligi seadme võimekus toda pead ühest ketta servast teise 
viia.}.

\question{Jällegi, kust üldse tekkis niisugune arusaam, et kettaseadmega saab 
niisugusi trikke teha, et võiks ette võtta sihukese asja? See vajab ju 
teadmist, julgust ja natuke arrogantsi ka, et \enquote{mis see Woz ikka 
ketastest teab}?}

Too opsüsteem oli ikkagi  tehtud universaalne, selles mõttes see ketta 
kopeerimise programm sai tehtud nagu \emph{dedicated} ehk optimeerituna 
mingisuguse konkreetse masina jaoks. Tol ajal oli oluline, et kui kuskilt kas või 
Moskvast mingisugune tüüp tuli, tal oli kettaid kaasas, neid pidid 
kiiresti suutma kopeerida. Mitte, et sa ei jokuta seal nagu tund aega 
kopeerida, vaid et sa saad  nood programmid endale kiirelt ära tõmmata. 
Praktiline vajadus. Kust sa neid programme saad, internetti ei olnud. Ma käisin 
ise tegelikult koos ühe klassivennaga korra Moskvas puhtalt selle pärast, et 
mingisuguseid arvutimänge saada.

\question{Moskva on suur linn\ldots}

Lihtsalt too vend käis ise Tartu Ülikoolisja me saime temaga kokku, tal olid 
mingid programmid, me kopeerisime need ära. Nii me saime tema kontakti ja ta 
ütles umbes, et \enquote{kui Moskvasse satute, siis alati \emph{very welcome}}. 
Ja siis me ükskord lihtsalt läksimegi.

\question{Rongiga?}

Rongiga. 

\question{Huvitav, mis see andmeside kiirus tuleb, kui arvestada, et sõidad 
rongiga edasi-tagasi ja kopeerid flopisid?}

Ei julge öelda, kolmsada kuuskümmend kilobaiti oli üks ketas või? Eks see oli 
vast kõige kiirem viis ikkagi andmeid transportida. E-mail tuli ikkagi 
mingisugused aastad hiljem ja too käis  kord päevas, helistasid modemiga 
sisse ja tõmbasid meilid ära.

\question{Kui keskkool otsa sai, mida sa õppima läksid?}

Tartu Ülikooli rakendusmatemaatikat\index{Tartu 
Ülikool!Matemaatikateaduskond!Rakendusmatemaatika}. 

\question{Sõjaväkke ei võetud vahepeal?}

Õnnestus ära viilida. Aga õppimisest tegelikult palju välja ei tulnud 
tolle pärast, et ma istusin ikkagi seal Apple-te juures edasi, nii nagu 
keskkooli ajal. Esimese kursuse ma tegelikult tegin ära, kõik matemaatika 
eksamid olid viied, aga inglise keelega kukkusin välja. Kuna ma hõbemedaliga 
lõpetasin  siis uuesti sisse astumine oli väga lihtne, pidi ainult matemaatika 
eksami tegema, mis oli triviaalne. Aga siis ma enam ei viitsinud üldse 
loengutesse minna, sest  kõik matemaatika eksamid olid tehtud, me oleks pidanud 
ainult inglise keele pärast seal esimese kursusel käima.

Nii ma istusin seal Apple-te taga. 

Loomakasvatuse ja Veterinaaria instituudi direktor Olkonen\sidenote{Instituudi 
täpne nimetus oli aastani 1994 (mil ta liideti Põllumajandusülikooliga) Eesti 
Loomakasvatuse ja Veterinaaria Teadusliku Uurimise Instituut. Ja 
piimandusteadlane Arvi Olkonen\index[ppl]{Olkonen, Arvi} ei olnud mitte terve 
instituudi direktor, vaid juhatas instituudi piimanduse laborit.} tegi 
doktoritööd ja tal oli terve bussitäis tädisid, kes olid valmis andmeid 
sisestama. Aga tal ei olnud, kuhu neid andmeid sisestada ja seda, mis arvutab. 
Ma tegin talle  programmi, mis võimaldas andmeid sisestada. Seal oli siis 
oluline teha selline \emph{user inteface}, et tädid kuidagi eksida ei 
saaks, too oli kindlasti kõige keerulisem. Arvutuse osa oli  tegelikult 
lihtne.

\question{Mida seal arvutada tuli? Mingit statistikat?}

Need olid mingid piimaproovid, kus olid siis laktoosi protsent, valgu protsent ja hulk 
igasuguseid muid karakteristikud.  Ma ei mäleta täpselt, Olkonen ütles ikkagi 
nood algoritmid ette, mida tuleb teha. Ma võisin matemaatiliselt nõu anda, aga  
üldiselt ta ikkagi teadis ise, mida ta tegi. 

\question{Mis tähendab siis seda, et sa olid kusagil palgal?}

Ma olin seal arvutiklassis Tartu Ülikoolis poole kohaga palgal,  insenerina. 
Kuna too Loomakasvatuse ja Veterinaaria Instituut mulle kuidagi nagu 
ühekordselt maksta ei saanud, siis mind võeti sinna ka tööle. Aga ma ei käinud 
seal kunagi, ma olin seal mingi aasta või kaks tööl lihtsalt selleks, et tolle 
programmi eest nii-öelda tasu saada. Ma ei viitsinud palka ka minna välja 
võtma,  siis pangakontosid ei olnud. Mispeale ülemus tuli mulle autoga järgi ja 
viis palka välja võtma, sest ta ei jõudnud enam  kassapidaja kisa ära 
kuulata.

\question{Tuldi autoga järgi ja viidi raha saama. Programmeerija magus 
elu\ldots}

Jah, direktor Olkonen oli väga lõbus sell. Nende oma inimestega, ta oli hirmus 
kuri. Alati kui me sinna läksime, siis ta kõigepealt sõimas kõigil näo täis. 
Aga väga ettevõtlik tüüp. Ma mäletan, et kunagi ma olin kodus, isaga saunas. Ja 
ema tuli sauna, et \enquote{kuule, mingi mees tuli}. See oli siis Olkonen, tal 
oli midagi kiirelt vaja. Ja mul ema ütles, et ta oli nagu enam-vähem ilma 
tutvustamata uksest sisse astunud, läinud kohe elutuppa ja maha istunud. Et ta 
võib oodata, ei ole probleemi. Väga sihikindel. 

\question{Kui sa loengutesse ei jõudnud, siis mingi asi pidi sind seal arvutite 
juures kinni hoidma? Assembleri ja pusimise huvi või midagi muud?}

Nojah, ikka. Assembleris ma kirjutasin tekstiredaktori, kuhu sai ikka päris 
ohtralt igasuguseid \emph{feature}-sid tehtud. Too oli kindlasti kõige 
keerulisem asi, mul peaks vist isegi kood paberi peal väljatrükituna  
alles olema. Kas viis või kuus tuhat rida assemblerit. 

\question{Seda ei ole üldse nii palju}

Ikka üksjagu. 

\question{Assembleri koodi mõttes on seda palju, aga tekstiredaktor viie 
tuhande reaga pole paha!}

No ma aitasin seal ühel tütarlapsel, kes mulle väga meeldis,  kursusetöid teha 
ja tolleks oli seda tekstiredaktorit nagu vaja. Muidu oleks pidanud 
kirjutusmasinal trükkima. Aga arvutis ühtegi korralikku tekstiredaktorit polnud. No 
oleks kah saanud ühte või teistviisi teha,  mingisugused hädised asjad olid. 
Aga et kõik suured-väiksed tähed, sellised asjad, nende ei olnud lahendust ja siis 
ma kirjutasin.

\question{Jaan Tallinn kirjutas ka\sidenote{Vt. lk. 
\pageref{sisu!jaani_tekstiredaktor}} esimese asjana omale tekstiredaktori. Kas 
siis Internet on teinud hoopis karuteene? Vanasti, kui tahtsid 
tektsiredaktorit, pidid ise kirjutama. Nüüd võtad millise iganes.}

Eks ta siis oli ka natukene lihtsalt see, et sul ei olnud neid programme 
kuskilt saada. Ameerikas olid Apple'i jaoks ilmselt kõik programmid olemas, aga nad ei jõudnud lihtsalt Eestisse ja siis tegid ise. Aega ka oli ja\ldots

\question{Mis sul tol ajal ettekujutus oli, et kuhu see viib? Et istud järgmised 
20 aastat Vanemuise tänava klassis?}

Mul ausalt öeldes ei olnud küll mingisugust väga konkreetset plaani, kuhu see 
viib.

Kuskil üheksakümnendal aastal tuli meilinduse käimapanek seal Vanemuise tänavas, 
umbes sel ajal Taavi Talvik\index[ppl]{Talvik, Taavi} 
kutsus mind Postimehe\index{Postimees} toimetusse.

Sinna ta oli mingisuguses SCO UNIX'i\index{OS!SCO UNIX} valmis pannud ja 
mingisuguse hulga terminale, mille kaudu ajakirjanikud oma artikleid 
sisestasid. Emacs oli vist tekstiredaktor. Ja eesmärk oli siis sinna peale eesti keele 
õigekirjakontrolli teha. Tollega ma siis seal tegelesin. 

\question{Ühesõnaga sa läksid sealt Vanemuise tänavalt Postimehesse?}

Jah. Ma käisin seal Vanemuise tänaval ka ikkagi sest,  nood 
tädikesed, kes seda arvutiklassi seal haldasid, tehniliselt liiga võimekad ei 
olnud. Ja nii oli ikkagi kasulik, kui ma sel õhtul läbi käisin ja neile mingite 
asjadega nõu andsin. Aga jah, Postimehes tuli ikkagi \emph{full time job}.

\question{Õigekirjakontrolli tegemine ei ole triviaalne asi, seal peab ju 
keelest ka aru saama?}

Ei ole triviaalne asi. Siis ma avastasingi, kui neetult keeruline see eesti 
keel on. Miks kurat peab iga teine sõna mingi erand olema? Väga tüütu oli. Ega 
me toda valmis ei saanud, tegelikult.

\question{Umbes 93. aasta paiku hakkasid tekkima Filosoftid\index{Filosoft} ja 
niisugused asjad, nad tegid Wordile eesti keele spelleri. Aga selleks ajaks 
olid teadus ja arvutusvõimsus edasi läinud}

Jah, absoluutselt, aga  1993 oli juba see aeg, kui ma tulin ära Tallinnasse 
Hansapanka\index{Hansapank}.

\question{Sa tulidki otse Postimehest?}

Mingisugune lühikene periood oli Postimehe ja Hansapanga vahel ka tegelikult, 
kus ma olin  hulgifirmas\sidenote{Nõukogude ajal olid poed 
prestii\v{z}sed asutused, sest nende kaudu jagati valitutele defitsiitset 
nõukogude kaupa. Kujutage nüüd ette, kuidas vaadati vastses vabariigis hulgifirmadele, 
kelle kaudu ühtäkki jagati \emph{lääne} kaupa.}. Neile ma  kirjutasin ka mingit programmi. 
Aga too oli väga selline kaootiline koht selles mõttes, et  bisnis läks 
hulgifirmal nagu hirmus hästi ja iga kuu nood kolm kutti, kes ta omanikud olid, 
ostsid igaüks endale uue BMW ja tolmutasid nendega ümber tolle maja sõita. Et 
ei olnud liiga motiveeriv keskkond, tegelikult.

\question{Kust Taavi sind üles leidis?}

Ega ma nüüd ei julge öelda, ausalt öeldes peast, kus ma Taaviga tuttavaks sain. 
Sel ajal, kui mina Apple'te taga istusin, istus Taavi tegelikult sealsamas Tähe 
4, kus ma esimest korda Nairidega kokku puutusin, keldris, kus oli mingisugune 
IBM PC.

Ja kas Taavi tegi midagi äkki Tartu Ülikooli Raamatukogule\index{Tartu 
Ülikool!Raamatukogu} ja mina olin ka tollega kuidagi seotud ja kas me äkki seal 
Tartu Ülikooli Raamatukogus PC taga saime kuidagi kokku? Ma näitasin Apple'id 
Taavile ja tema näitas mulle toda PC-d. Oli sihukene mäng nagu King's 
Quest\index{Mängud!King's Quest}, mida me Taaviga mängisime seal Tähe tänaval. 
Ja noh, sealt me tuttavaks saime, et me tolle King's Questi\sidenote{King's 
Quest on Sierra Entertainmenti seiklusmängude sari, mida peetakse omas 
valdkonnas klassikaks. Mängud ilmusid aastatel 1980 kuni 1998.} läbi mängisime.

\question{King's Quest oli ju seiklusmäng?}

Just. Tollega läks ikka aega, et lõpuni mängida,  istusime ikka palju õhtuid 
seal.

\question{Eks Taavil oli ühel hetkel Postimehes abi vaja ja siis ta kutsus 
sind?}

Täpselt. 

\question{Aga kuidas sa Hansapanka\index{Hansapank} sattusid, see on huvitav 
lugu!}

Hansapanka ma sattusin samuti tänu Taavi Talvikule\index[ppl]{Talvik, Taavi}. 
Taavi töötas Valitsussides\index{Valitsusside} tol ajal ja 
Microlinkist\index{Microlink} Rainer Nõlvak\index[ppl]{Nõlvak, Rainer}, ma 
arvan, oli see, kes küsis Taavi käest,  et Tõnis Sildmäe\index[ppl]{Sildmäe, 
Tõnis} otsib kedagi, kes Unixit tunneks. Taavi ütles, et tema küll ei taha 
minna ja küsis minu käest. Mõtlesin, et ah, suva, et ma võin ju rääkida ja 
kuulata, et mis seal siis teema on. Tulin Tallinnasse Tõnis Sildmäega rääkima, 
Sildmäe küll jättis mulje, et tal on terve bussitäis Unixi-mehi ukse taga 
järjekorras, keda ta kõiki intervjueerib, aga vist tegelikult peale minu ühtegi 
ei olnudki. Igatahes ma sain panka tööle.

Too SCO UNIX\index{OS!SCO UNIX} oli sinna juba ära installitud ja Tarmo 
Pajumets\index[ppl]{Pajumets, Tarmo} püüdis  sinna peale Oracle't\index{Oracle} 
installida. Aga ega nad tollest SCO UNIX'ist midagi ei teadnud, nii et esimese 
päeva lõunaks nad mõlemad läksid sealt konsooli tagant ära ja rohkem  
tagasi ei tulnud, kui nad vaatasid, et ma vist tean natuke rohkem.

\question{Inimesed said oma teadmiste piiridest aru. Tollel hetkel pank kui 
selline oli juba olemas, mis infosüsteemi peal ta käis?}

Pank käis Paradox'i\index{Paradox} peal. Aga eks too Oracle'i andmebaasi majja 
toomine oli, ma arvan,  üks väga paljudest Hansapanga edu aluseks olevatest 
strateegilistest otsusest. Paradox 
töötas tol hetkel täiesti normaalselt, ei olnud häda midagi. Aga juba oli Tõnis 
Sildmäe\index[ppl]{Sildmäe, Tõnis} välja raalinud, et tegelikult me 
peaksime  mingisuguse tõsisema andmebaasi mootori sinna alla panema. 
Esialgu läks Oracle Novelli\index{Novell} peale  aga siis me saime tolle SCO ka 
nii kaugele, et migreerisime andmebaasi sinna.

\question{Räägi sellest palun korra lähemalt. Kellegi, ilmselt siis Tõnise, 
peas oli arusaam, et arhitektuursed otsused võivad olla ärilise edu aluseks. 
Üheksakümnendate alguses see ei olnud väga levinud asi, kust tal see tuli?}

Ma usun,  et panga seltskonnal oli ikkagi ka selge arusaamine Paradoxi 
tehnoloogilistest piirangutest ja samal ajal oli ka arusaamine, kuhu poole see 
pank liigub. Ma arvan, et tolleks hetkeks, kui mina sinna tulin, äkki 
oli seal minu kõrvallaua peal juba tegelikult esimene sularahaautomaat.  Oli 
sihuke suhteliselt pisikene,  mahtus laua peale, IBMi oma. Kaart ei käinud 
sisse, vaid tuli magnetriba lihtsalt läbi tõmmata. Ma arvan, et ATM-ide asi oli 
üks, mis  tolle Paradoxi andmebaasi piirangud välja tõi. Samamoodi, kuna 
klientide arv kasvas plahvatuslikult, ilmselt ka tolle pealt nähti, et too 
Paradox ei suuda tegelikult, kui selline kasv jätkub, ära teenindada.

\question{Teine asi, mis mind ikka on huvitanud on, et samal ajal toimetati 
päris mitmes pangas valmis tarkvaraga. Osteti Britimaalt pangasoft ja tehti 
sellega panka. Miks Hansapank teistmoodi tegi?}

Seda oskavad need öelda, kes päris alguses pangas olid. Ma ei 
tea, kuidas too Spin Development\index{Spin Development} sinna Hansapanka tuli 
Et noh, nimigi selline,  ilmselt oli seal siis mingeid \emph{developer}-e. Ja 
ilmselt esimene ülesanne, mida teha tuli, oligi mingisugune väike tükikene ja 
kui too läks hästi, siis sealt hakkas asi arenema. Ma ei oska öelda.

\question{Spin Development on siis Crebiti\index{Crebit} algus?}

Jah. Kui mina tööle läksin, siis esimese palga maksja oli tegelikult Spin 
Development, mis minu meelest nimetati lihtsalt 
Crebit-iks ringi. Ja mingi aeg hiljem siis, ma saan aru, Londoni 
kindlustusfirma ütles pangale, et kuulge, et te ei tea IT-st mitte midagi, kogu 
asi on väljas mingisuguses täiesti iseseisvas ettevõttes, et kuidas te oma 
riske juhite. Mis siis lõppes sellega, et Hansapank ostis Tõnise käest 
Crebiti aktsiad ära ja kõik me tulime  Hansapanka tööle. Crebiti 
juriidiline keha jäi alles ja on kuni tänase päevani Swedbank Support OÜ nime 
all olemas.

\question{Huvitav, et kultuur oli jätkuvalt Crebiti oma. Sest kui mina pangast 
ära tulin aastal 2002, siis viimane särk, mis pank mulle väljastas, oli Crebiti 
logoga. Jube elujõuline asi!}

No kindlasti. Mitte ainult Crebit, vaid toob pank ise tervikuna oli tegelikult 
äärmiselt elujõuline. Ütleme, vähemalt kuni tolle hetkeni, kui
Hoiupangaga\index{Hoiupank} liituti. Siis toimus ikkagi suur kultuuriline 
muutus, tuli lihtsalt väga palju teisi inimesi juurde.

\question{Kust see kultuur tuli?}

Ma olen toda mõelnud. Ilmselt ühelt poolt ilmselt oli kõigil inimestel, 
ikkagi väga selge saavutusvajadus oma asja väga hästi teha. Seal 
isegi ei pidanud minu meelest neid, kes võib-olla ei \emph{perform}-inud 
piisavalt hästi, lahti laskma, vaid nad läksid ise ära. Seesama lugu, kui  
Pajumets\index[ppl]{Pajumets, Tarmo} seda Oracle't installis. 
Tegelikult oli üks mees seal veel kõrval, kes tolle SCO UNIXi installis. 
Kui mina liitusin, siis ma saan aru, et too mees ise läks paari nädala pärast 
ära, tegelikult. Teda ei lasknud keegi lahti, ise kirjutas lahkumisavalduse, 
sest et ta sai aru, et tal ei ole seal enam midagi teha. Ja noh, too kultuur 
oli absoluutselt kõigile ühesugune. Sa ei pidanud kaks korda kellelegi ütlema, 
sa teadsid, et asi on tehtud.

\question{Kui sa alustasid, siis sa kirjutasid assembleris koodi, puhas 
arenduse värk. Aga pangas sa läksid kohe asjade opereerimise peale, kuidas ja 
miks see nihe toimus?}

Ega seal mingisugust  teadlikku valikut väga ei olnud. Töö tundus huvitav, ega 
ma Oracle baasi polnud varem näinud ju. Ma kuidagi ei mõelnud, et ma olen 
programmeerija. Tegelikult ju ka seal arvutiklassi Apple'te juures  tööülesanne 
oli tegelikult kõigi nende inimeste assisteerimine, ülalhoid, et nood arvutid 
töötaksid, et probleemid nendega oleksid lahendatud. Programmeerimine oli 
puhtalt hobi muu töö kõrval. Kuigi kogu asi algas loomulikult 
programmeerimisest Nairide peal. Ja ma arvan, et tollal ka ei olnud liiga palju 
konteksti, et \enquote{need on arendajad ja need on ülalhoidjad}. Ma arvan, et 
too tuli hiljem, siis kõik lihtsalt tegutsesid.

\question{Õige, see on nii pikalt olnud aga ei ole pidanud alati nii olema. SCO 
jäi sulle külge Postimehest, aga Oracle?}

No see saigi külge sealt Hansapangast. 

\question{Hakkasid lihtsalt tegema? Mitte väga palju aastaid hiljem oli 
tegemist ühe maailma suurima Oracle koodi baase, Oracle konsultandid käisid 
majas ja rääkisid, et nad ei ole kuskil maailmas midagi sellist näinud}

Võis olla küll. Vilve Vene\index[ppl]{Vene, Vilve} ja Juta 
Joost\index[ppl]{Joost, Juta} ja kes seal kirjutasid, kirjutasid PL/SQL-i ikka 
usinasti. Toda oli tõesti väga palju, väga väga palju. Oracle Forms 
oli tehnoloogiaks, et ilmselt oli nii teda kõige efektiivsem teha. Eks 
 baasi protseduurid käisid kõik kiiremini kui mingisugune, noh, 
klient-server asi.

\question{Sul ei tekkinud tunnet, et las see Oracle käib siin, et 
programmeeriks parem?}

Eks programmeerima pidi ikka natukene selles mõttes, et skripte tuli kirjutada, 
mis  siis kogu toda asja üleval hoidsid. Kas või seesama, et kui sa SCO 
UNIX-iga masina üles \emph{bood}-id, et kuidas andmebaas käima pannakse. Ega 
 Oracle installi juures mingeid skripte ei olnud, kirjutasid ise 
 skriptid, mis tolle baasi ja \emph{listener}-i käima panid. Kogu 
\emph{backup}-i tegemine, tolleks pidi skriptid kirjutama, lisaks sellele kõik 
need \emph{batch} protsessid, mis olid C-s kirjutatud. 
Noid skripte  sai tegelikult kirjutatud ikkagi päris päris palju.

\question{See, mida sa kirjeldad, on päris keeruline asi, mis kuidagi nendest 
eri tehnoloogiatest peaks terviku moodustama. Kuidas see tervik tekkis, kes 
seda juhtis? Kes arhitekt oli?}

Ega siis kedagi arhitektiks ei nimetatud. Ma julgeks ise arvata, et vast 
tarkvara kontekstis arhitekt oli ikkagi Vilve Vene\index[ppl]{Vene, Vilve}. 
Vähemalt  mulle on selline mulje jäänud, keegi sellist terminit ei kasutanud. 

See kontseptsioon, kuidas  kõik tarkvaraliselt kokku töötab, ma arvan, tuli 
ikkagi ennekõike Vilvelt. \emph{Non-functional requirements}-id tekitasin mina. 
See sama skriptide asi, et iga  C programm, mis mingit \emph{patch}-i tegi, ei 
oleks erinev. See tuli kuidagi ära standardiseerida,  ma pidin mingisugused  
mittefunktsionaalsed nõuded esitama, et nad kõik oleksid ühetaolised, et ma 
saaksin kasutada mingit ühte skripti paljude asjade käivitamiseks.

\question{Kui Postimees käib ka siis, kui ajakirjanikud trükimasinaid 
kasutavad, siis pank enam trükimasina peal ei käi. \enquote{Pusime ja vaatame, 
kuidas Woz on teinud} pidi struktuursemaks muutuma, kuidas see juhtus?}

Kuna  kasv oli nii kiire, siis igaüks pidi tegelikult vaatama  mitte ainult 
seda, kuidas see asi täna ära \emph{run}-ib, vaid ka seda, milline see asi nagu 
aasta pärast välja näeks. Ja kindlasti Tõnis Sildmäe\index[ppl]{Sildmäe, Tõnis} 
fasiliteeris ka seda, et tuleksid igasugused erinevad kontaktid, kes 
mingisuguseid uusi lahendusi pakuksid. Nii need 
asjad arenesid edasi ka. SCO UNIX-ile ju samamoodi  tulid tehnilised piirangud 
ette. 1996 või 1997 sai see HP-UX'i\index{OS!HP-UX} vastu välja vahetatud, enne 
vahetust olid mul nii HP kui SUN-i server laua peal ja sai võrreldud, kumb  
kiirem on. Tolleks ajaks oli pank ka piisavalt suur ja oli selge, et me  ei 
pane ühte masinat, vaid me paneme klastri. Sai nende \emph{vendor}-itega  
klastri lahendused läbi räägitud\ldots

Mingisugust hüpet ei toimunud, et enne oli anarhia ja siis  tehti kõik asjad 
korda. Kõik arenes evolutsiooniliselt, igal aastal vahetati lahendusi uute 
vastu välja. Teisti ei oleks lihtsalt üle elanud toda kümme aastat kestnud olukorda, 
kus iga, ma ei tea, üheksa kuu tagant  kahekordistusid klientide arv, käive 
kasum, mis iganes. Kõik numbrid kahekordistusid üheksa kuuga kümme aastat 
järjest.

\question{Sellist kasvu ei kujuta tänapäeval nagu väga ette enam, kui sa just 
kuskil Skype moodi kohas ei tööta}

Nojah, ega neid ettevõtted ongi maailmas väga vähe vist, kes nii kiiresti nii 
pikalt suudavad kasvada. Oli mingisugune \emph{success story}, jah.

\question{Ma mäletan, et sajandivahetuseks oli panka tekkinud üsna 
spetsialiseerunud tiim, kes kogu kupatust opereeris. Kuidas see kolmik, 
mille peal  kogu panga maailm püsti seisis, tekkis?}

Aja jooksul selles mõttes, et Madis Ollisaar\index[ppl]{Ollisaar, Madis} oli 
enne mind olemas. Ma ei teagi päris täpselt, mis tema roll päris alguses oli. 
Sel ajal, kui mina seal tolle Oracle baasiga toimetama hakkasin, siis minu 
asi oli tehniline pool, et andmebaasi \emph{engine} töötaks ja Madise 
asi oli luua uusi tabeleid ja teha indekseid ja vaadata, et päringud hästi 
käivad ja nii edasi-tagasi. Ja too roll jätkus tal edasi. Toomas 
Suurmets\index[ppl]{Suurmets, Toomas} tuli\ldots
Ma pidin peaaegu ütlema, et ta tuli koos Hoiupangaga liitumisega aga tegelikult 
ei tulnud. Ta töötas Hoiupangas, aga 
tegelikult ta tuli Hansapanka kaks aastat enne seda, kui Hoiupank ära osteti. 
Tolleks ajaks istus 
tema juba õigel pool lauda. Ja Toomas  nagu täiendas seda seltskonda. Kui 
Madis oli nagu kõige ülemine, nii-öelda \emph{data layer}, mina teadsin 
andmebaasi \emph{engine} osa, siis Toomas oli see mees, kes \emph{netowrkist} 
ja \emph{storage}-st hästi jagas.  Too kokku andiski kogu tehnoloogilise 
\emph{stack}-i, et põhiasi töötaks

\question{See tiim töötas jube hästi!}

No me istusime ühes toas. Ühes infoväljas kogu aeg, alati on võimalik öelda, 
mis toimub.

\question{Ma tean, sul on Wagneri huvi, kas see oli juba tol ajal? Ma mäletan, 
et teie toas kapi otsas oli makk, kust aegajalt tuli eepilist klassikalist 
muusikat?}

Ausalt öeldes ma isegi päris aastaarvu jälle julge öelda. Amazoni veel ei olnud, 
esimesed CD-d, mis ma Internetist ostsin, ostsin kohast nimega cdnow.com. 
Toda klassikalist muusikat sai mängitud jah. Mitte küll 
Wagnerit, põhiliselt tegelikult, ma julgeks arvata, Mozartit tol ajal. Päris 
alguses ma ostsin mingisugused Enrico Caruso plaadid. Aga pärast ma ostsin 
Mozartit ka ja mängisime seal neid, teisi see kuidagi ei seganud. Eks me tegime 
erinevaid asju. Mingi periood oli, kus, meile öeldi, et teatud lõhnad on teie 
toas igapäevaselt tunda. Mingi periood oli tõesti, kus meil oli alati 
konjakipudel kapis  ja päeva sai alustatudki pitsi konjakiga. Loomulikult 
mingit joomist ei olnud, aga eks lõhnaks piisas tollest ühest pitsist juba, sul oli ka klaas 
laua peal võib olla kuni lõunani. Ega keegi ei joonud, aga lihtsalt natukene. 
Ja WRC ralli oli ka, mille Toomas vist püsti pani, seda me 
mängisime ka  mingisugune periood. Too tahtis väga palju \emph{network}-i 
aga kuna Toomas oli \emph{network}-i põhjaga vend, siis kui toda 
\emph{bandwidth}-i kellelgi oli, siis meie toas ennekõike.

\question{Kust sul see klassika huvi tuleb?}

Klassika huvi tuli sealt Enrico Caruso-st tegelikult. Mul oli vanematel kodus 
Vittorio Tortorelli raamat\sidenote{Enrico Caruso. Vittorio Tortorelli. Eesti 
Raanat 1968, tõlkija Õ. Karask.}, Tortorelli  on itaallane ja kuna ta oli vist 
Carusole kaugelt sugulane ka, siis raamat 
on ülimalt ülistav, aga ta oli huvitav lugeda ja jättis väga sügava mulje. 
Ja kui ükskord Internetist  oli võimalik tellida, siis ma tellisin huvi pärast sealt 
CDNow-st neid Caruso plaate. Teine asi, mis draivis oli ikkagi 1984. aasta 
Miloš Forman'i Amadeus, mida ma kindlasti soovitan kõigil vaadata, suurepärane 
film. Sealt tuli Mozarti huvi, kuidagi selliselt ta läks,  tellisin 
mingisuguseid raamatuid Mozarti eluloost, mingi neli-viis, mõni on üle tuhande 
lehekülje paks. Sealt edasi on juba  lihtne. Beethoven, Schubert, 
Schumann, Tšaikovski \ldots

\question{Mis sa praegu teed?}

Töötan G4S-is\index{G4S}, turvalisem Eesti. Baasteenuste arendusjuht,  
sisuliselt  vastutan ülalhoiu eest, et kõik asjad oleksid püsti ja valvatud. 
Jah, mitte siis ainult IT vaid ka  tehniline valve, kuhu puutub, siis ka see 
raadiosidevõrk on meil üle Eesti on. Et kõik need signaalid jõuaksid  keskele 
kokku.
