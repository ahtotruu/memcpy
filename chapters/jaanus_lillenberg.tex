\index[ppl]{Lillenberg, Jaanus}
\question{Kuidas sina said arvutite juurde ja arvutid said sinu juurde?}

Ah nii algusest kohe. Ma arvan, et see alguse aasta oli 1983, kui  Tartu 
Ülikooli\index{Tartu Ülikool} tuli mingisuguse Nõukogude-Jaapani koostöö 
tulemusena personaalarvutite klass.

\question{Aastal 83, mis arvutid need olid?}

Need olid Yamaha MSXid\index{Arvutid!Yamaha MSX}. Yamaha MSX kuulub, ma 
ütleksin, samasse põlvkonda mis Commodore 64, mõned vihasemad Sinclairid ja 
siis võiks öelda, et ka Apple II. Äge oli see, et  nende arvutite peal jooksis 
tegelikult Microsofti  \emph{consumer} kasutajatele mõeldud operatsioonisüsteem.

\question{See oli Microsofti oma?}

\enquote{MSX} nagu Microsoft \emph{Extended}, ma täpselt ei mäleta\sidenote{Tundub, et 
ka asja juures olnud kodanikud ei mäleta enam täpselt, liikvel on mitmeid 
variante}. Ta nägi nii äge välja. Ma käisin koolis, mille peauksest see 
arvutiklass paiknes sõna otseses mõttes kümne sammu kaugusel  
keldris, keldrikorruse aken avanes täpselt selle kooli ukse ette. Sa läksid 
iga hommik ja iga õhtu mööda. Ühele üheteistaastasele 
poisile oli see vastupandamatu. Selles mõttes valikuvõimalusi tegelikult  
ei jäetud. 

\question{Sa lihtsalt pidid sealt uksest sisse minema}

Ei, ma \emph{literally} kõndisin selle aknast sisse lihtsalt mingi päev, sest 
aken oli tänavaga samal tasapinnal. Küsisin, kas võib tulla vaatama või midagi 
sellist. Ega ma ära ka ei läinud. Ja  umbes kolmandal päeval 
andis mulle keegi MSX BASICu\index{Keeled!BASIC!MSX BASIC} 
manuaali fotokopi fotokopi fotokopi. Vaatasin, mitte midagi 
inglise keelest aru ei saanud. Aga see oli nii huvitav kõik, et kuidas mänge saaks 
teha! See klass on tegelikult omaette peatükk, seal toimetamine kestis  
umbes kolm või neli aastat, mille seas ma olin ka vahepeal abiõpetaja. 
 Kirjutasin ise tekstiredaktoreid, mänge, siis loomulikult häkkisin 
lõputus koguses mingeid olemasolevaid mänge. Kirjutasin ka oma elu esimese 
viiruse, mis hävitas flopiketta põhimõtteliselt. 

\question{Aga mis kool see niisugune oli}

See oli Tartu 10. Keskkool\index{Koolid!Tartu 10. Keskkool} praegu on ta
Tartu Mart Reiniku Kool\index{Koolid!Tartu Mart Reiniku Kool|see{Koolid!Tartu 10. 
Keskkool}}. See klass paiknes siis  Vanemuise tänaval Vanemuise teatri vastas 
oleva õppehoone\index{Tartu Ülikool!Vanemuise tänava õppehoone} keldrikorrusel. 
Seal keldris oli tegelikult isegi kaks arvutiklassi. Teises klassis olid siis 
Ahhaadid või Agatid\index{Arvutid!Agat}, mis siis olid venelaste pihta pandud 
kas need olid Commodore  või Apple II koopia\sidenote{Agat kasutas küll sama 
6502 protsessorit, mis Commodore 64 ja Apple II ning oli suuresti viimasest 
inspireeritud, kuid erines disaini poolest mõlemast ja otsese koopiaga tegemist 
ei olnud}. Kusjuures mul läks rohkem kui aasta, et aru saada, et see tegelikult 
on Apple II koopia,  ta oli 
tegelikult nii teistsugune. Sealsamas klassis olid ka Apple 
II\index{Arvutid!Apple II} arvutid. Kuigi protsessori tasandil olid nad sarnased, 
oli  kõik seal peal olev nii erinev. 

\question{Jah, see nõukogude variant oli üsna industriaalse väljanägemisega}

Jah. Kas sa tead näiteks, et kui inglise keeles on klaviatuuril vasakult paremale lugedes 
qwerty aga Vene klaviatuuril on \enquote{pidev \emph{lag}}? 

\question{Kusjuures tol ajal ei teadnud keegi \emph{lag}ist midagi veel}

Aga ära ütle selles mõttes, et hüppame sealt mingisugune kaheksa-üheksa-kümme 
aastat edasi, siis olid mingid Tartu Ülikoolil arvuti- ja 
terminaliklassid. See oli too ajastu, kus oli võimas arvuti, millel oli siis 
niisugune hunnik terminale, see moodustas terminaliklassi. Siis oli juba 
väga palju võrgutegevust. Tollesama arvutiklassi kõrval oli siis ka mingisugune 
niisugune IBMi koopia või litsentsi alusel tehtud ES\index{Arvutid!ES 
EVM}\sidenote{ES EVM (\begin{russian}ЕС ЭВМ, Единая система электронных 
вычислительных машин\end{russian}) oli sari IBM 
System/360\index{Arvutid!System/360} ja System/370\index{Arvutid!System/370} 
kloone. Nende riistvara põhines küll IBMi omal kuid oli väheste eranditega 
siiski Nõukogudemaal välja töötatud. Tarkvara seevastu oli IBMi tarkvara 
lokaliseeritud ja väheste muutustega koopia. Neid masinaid nimetati eesti 
keeles hellitavalt \enquote{jessukesteks}}, nõukogude arvuti, kus olid need 
vägagi need klaviatuurid kasutusel ja tegelikult siis \enquote{pidev 
\emph{lag}} oli väga ilmne kontseptsioon.

\question{Kõik üheteistaastased, kes sealt mööda kõndisid, ometi ei roninud 
aknast sisse. Mingi tehnika või elektroonika huvi pidi sul ju siis all olema?}

Ei, ei olnud. Tegelikult pidin ma minema hoopis ratsutamistrenni sellel ajal ja 
läksin ka tegelikult. Aga ma ei oska öelda, millega sa seda võrdled? No mõni 
asi on kohe selline,  visuaalne, ta on uus, ta vastandub  
põhimõtteliselt kõigele, mis su ümber on, aga sa saad kohe pihta, et see on 
mingi väga lahe asi. Kujuta ette, kui sa läheksid kuskilt mööda, kus on 
lendamistrenn, kus inimest õpetataks lendama. Sa ei  hakka arutama, et ei tea, 
et ma pidin  minema õhtul  malet mängima või telekat vaatama, on ju.  
Lendamine on  mingi asi, mis on universaalset väga \emph{cool}. Ja kõigist 
teistest asjadest nagu kümme korda kõvem.

\question{Ja siis sa ei oskagi pärast väga hästi seletada, et miks sulle 
lendamise trenn meeldis ja miks sa ei läinud malet mängima?}

Lihtsalt kaldusid. Ükskõik, mis tee ees oli, ikka kaldusid kõrvale. 
Muidu ma läksin ratsutama ka, ratsutasin neli aastat, see ei seganud.

Klassivenna võtsin sinna kaasa ja, ma mäletan, oli selline eriti lahe  
arutelu. Me saime aru, et see arutelu ei baseeru nagu kogu infole, mis meil on, 
aga mõtlesime, et kuidas neid mänge ikkagi tehakse. Et see Assembler või 
masinkood näeb ikka nii \emph{random} välja. Et äkki on ikkagi 
võimalik selliseid \emph{random} kombinatsioone katsetades saada nagu päris 
lahedaid mänge. Ise mõtlesime ka, et see võib olla ikka päris tõsi ei ole, 
aga mõtle kui äge, kui nii saaks! Katsetad,  ma ei tea, kümme tuhat 
kombinatsiooni, kõikvõimalikud koodi variandid, ja vaatad,  milline läheb käima 
ja milline ei lähe. Õnneks mingi nädal aega hiljem oli juba  \emph{Hello 
World} ja asjad tulid juba. Kõik oli  väga äge. Sellest MSX 
BASICust\index{Keeled!BASIC!MSX BASIC}  kasvas hiljem välja Visual 
Basic\index{Keeled!Visual Basic} muide. Et Visual Basicu õppimine oli meil, 
\enquote{noh, \emph{what else is new}}.

\question{Kes seda asja vedas seal? Pidi ju olema keegi, kes sind aknast sisse 
lasi ja mitte kohe välja ei visanud?}

Mind visati sealt mitu korda välja. Aga nemad olid oma väljaviskamises 
tunduvalt vähem veenvad kui mina sisse ronimises. Ja ma nagu ei osanud ka  
kuidagi solvuda või kuidagimoodi nagu pahaks panna seda välja viskamist.  
Ma sain ju aru, et see ei ole minu jaoks tehtud. No näiteks, ma ronisin päris 
mitu korda sisse mingisuguste õpetajate täiendkoolituse juurde. Seal õpetajate 
täiendkoolitusel, nagu hiljem sain aru, ju tegelikult ei õpetatud 
arvuti kasutamist. Tegelikult õpetati neile seda, et maailm muutub. Ja arvuti õpe oli 
hea, niisugune käega katsutav, asi seda muutust kirjeldama. Selle visiooni taga oli üks 
väga-väga-väga vinge inimene, see on Anne Villems\index[ppl]{Villems, Anne}.

Anne Villems on selline\ldots  Ma arvan, et kõige defineerivam termin on, et ta 
on tohutult kirglik inimene. Ja tema kirg oli see, et kuidas maailm saaks 
paremaks. Aga  õpetame neid inimesi, kes õpetavad teisi! Näitama nendele 
inimestele, kes näitavad väikestele inimestele, mihuke see maailm võiks olla. 
Ja ma arvan, et need inimesed, kes sinna tema täienduskoolitustele jõudsid, 
need algatuseks olid juba mingis mõttes peaaegu paremad. Nad suutsid endale  
sõnastada, et ma peaksin sinna minema, äkki maailma muutub paremaks. Ja need, 
kes selle koolituse läbi käisid ja seal  omavahel suhtlesid, ma arvan, et see 
oli  üks väga suur rest kive selles vundamendis, mille peale see meie IT-riik on 
ehitatud.

\question{Sest sealt läks ju eksponenti pidi laiali, õpetajad õpetasid omakorda 
ja nendest said abiõpetajad ja nii ta laiali läks}

Bingo. Ma toon jumala konkreetse näite. Näiteks oli  seal 
üks õpetaja, kelle nimi mul ei ole küll meeles,  Tartu 
Kunstikoolist\index{Koolid!Tartu Kunstikool}, kes hiljem tõi oma lapsed 
arvutiklassi tundi pidama. Kunstikooli õpilased, neil oli mingi 
joonistusprogramm, 64 värvi, maa ja ilm. Ja nad päriselt tegid mingeid asju 
selle arvutiga. Olgugi, et printerit ei olnud, ja kahjuks see meedia jäi sinna, 
kus ta jäi,  mis nad valmis said. Aga vahet ei olnud. Täiesti ebaoluline.  
Nad said selle tunnetuse kontseptuaalselt täitsa uuest viisist kunsti teha. 

Isegi mina, kes ma mõtlesin, et kuidas ikkagi saaks mänge teha ja 
mängida ja  oleks nagu tore, niisugune lihtne ja primitiivne lähenemine, aga 
lõpuks jõudsin ma kuhugi välja. Aga nemad võtsid  
mingisuguse graafilise \emph{editor}'i. Ja lihtsalt tegid sellega mingeid 
sihukesi asju, et vot, sa saad aru\ldots See on sama trikk, ütleme, nagu siis, 
kui tuli iPhone. Et me ei tea veel, kuidas ta kõva on, aga ta kindlasti on kõva 
sajal moel. Ja siis kui need asjad hakkavad tulema, siis vaatad, et äge! Tollel 
hetkel oli see personaalarvutid, need MSXid olid samasugused asjad.

\question{Iga uudsus läheb ju lõpuks ikka üle, sinu jaoks arvutite oma ei 
läinud üle?}

Ta ei läinud üle. Või noh, läks ikka üle selle baastasandil. Ta läks 
mingisugusele järgmisele tasandile. Ma toon näite, et hüppame sealt mingi kaks 
aastat edasi. Tõravere Observatooriumi\index{Tõravere Observatoorium} 
astrofüüsikud, kes olid maailmaga  hoopis teistsuguses kontaktis kui 
mingisugused koolipoisid. Üks selline oli minu alumine naaber,  Enn 
Kasak\index[ppl]{Kasak, Enn}. Ühest küljest olid siis nii-öelda kontaktid 
teadusmaailmaga, aga teisest küljest maailm nagu ka sulas ja oli 
võimalik mingit bisnest teha. Nad  tõid endale (teadlased olid 
rääkinud teiste teadlastega, et on ägedad masinad) Amiga 
500\index{Arvutid!Amiga 500}\sidenote{Tuntud ka kui A500, oli Amiga 500 
koduarvuti-vaste samal 1987. aastal  Commodore poolt turule toodud professionaalsele Amiga 
2000-le. Tegu oli populaarseima Amiga mudeliga, eriti Euroopas},  mis oli nagu 
põhimõtteliselt \emph{next generation}  Commodore 64'st. Tal oli kümme 
korda võimsam protsessor juba,  graafika oli hoopis siis klassist ja nii edasi. 
Kui MSX-i Z80 protsessor  võimaldas  kolmehäälset muusikat teha, siis Amiga 
suutis teha  kuusteist kanalit. Tänapäeva mõistes oli 86. aastal täis MIDI 
lahendus  võimalik kodus püsti panna, mitte mingit probleemi polnud. Ja oi, kus 
ma seal nende Amigadega muusikat tegin. Täiesti nagu häbitult. Ööde kaupa 
võisin teha, kui ma sain.

\question{Kas sul muusikahuvi oli enne olemas või sigis koos Amigadega?}

Igal inimesel on mingisugune arvamus, kas talle  meeldib muusika või talle ei 
meeldi see, niisugune baastasandi vastus. Osaliselt on see vastus seotud sellega, et 
kas pead viisi või ei pea. No ja kui mind lastekoori ei võetud esimeses 
klassis, siis ma sain aru, et mulle meeldib muusika, sest ma olin väga kurb. 

Kui ma nende Yamahadega tegelesin, no ei olnud ainult mängimine. Ei olnud. 
Oli arvutiring, täiesti mitteametlik. Mingit ringi tegelikult polnud, mingid 
kutid vajusid lihtsalt iga päev pärast kooli kohale ja enne ära läinud, kui välja 
visati. Selle klassi juures olid tegelikult ju   
üliõpilased, mingi väike tiim, neli-viis tüüpi, kes  olid põhimõtteliselt 
lastesõbralikud. Mõned, Ain Sakk\index[ppl]{Sakk, Ain}, Alar Pandis\index[ppl]{Pandis, 
Alar} ja mõned teised kutid  
veel, ma arvan, jätkasid pärast ülikooli lõppu ka pedagoogidena. Ma usun 
vähemalt. Ei tea, mis neist saanud on. Ja nad tegelikult ka toetasid meid. Nad ei pidanud 
seda tegema. Aga milles oli trikk, kuidas see üldse sai käima minna, kuidas see 
mudel sai võimalikuks, oli see, et arvuteid oli klassis viisteist tükki. Aga 
täiendkoolitustel oli enamasti seitse kuni kümme inimest. 
Alati oli mingi \emph{spare} kuskil, alati olid mõned arvutid vabad. 
Ja põhimõtteliselt asi nägigi nii välja, et kutid tulid kohale pärast  kooli ja 
siis kes ees, see mees, onju. Ega kui sa arvuti said, sa seda ära ei 
andnud enam. Istusid seal põhimõtteliselt kõhuli peal ja tegid oma asju. 
Tundide ajal, kui oli täiendkoolitus, võis seal midagi teha, aga sa ei tohtinud 
mängida. Siis sa ootasid seal hambad ristis, mingi \emph{manual} oli 
kõrval, ikka proovisid sealt asju. 

Sealt tekkis see variant, et ka selle MSX BASICuga\index{Keeled!BASIC!MSX 
BASIC} sai teha muusikat. Sõna otseses mõttes panid noodid ritta, A, B, C, sättisid 
rütmi seal kiiremaks ja aeglasemaks, sai oktavi muuta. Aga said 
kolm erinevat muusikat  kokku panna, kui ma ei eksi. Ühetoonilist muusikat sai 
kindlasti teha, sellega ma tegin igasuguseid asju. Kuuled midagi, proovid 
järgi teha. See on väga \emph{addictive},  väga. Aga Amiga selle kõrval oli 
ikkagi\ldots\ See erinevus oli sama suur, kui sa nagu paned endale  papist 
tiivad külge ja jooksed ja mängid lennukit või lähedal päris lennuki peale. See 
hüpe oli nii kohutav. Kujuta ette, ühel juhul on sul olukord, kus sa 
paned teksti \emph{editor}'is  nooditähti paika ja mängid selle ette ette ja 
siis kuulad, eks. Teine variant, kus sul on  täisgraafiline muusika 
\emph{editor} koos nootidega,  koos digiklaveriga, mida sa saad arvuti 
klahvide peal mängida ja see \emph{record}'ib ja nii edasi. Põhimõtteliselt 
see asi, mis tänapäeval on. Sadu pille, mille seast sai valida. Mis 
tegelikult oli noh, totaalne digi-piiks, aga tollel hetkel need digi-piiksud 
oli ikka nii ära \emph{tune}'tud, et viiul ja klaver ikkagi kostusid kõrvale 
erinevalt. Kuigi kui sa neid ei teaks, et see nimi on juures, tunduks  üks 
 piiksamine puha. 

\question{Selleks, et suuta niimoodi kõrva järgi mingisuguseid asju järgi teha, 
selleks peab ikkagi kõrva olema. Sul mingisugune muusikaline kuulmine oli 
olemas?}

Mingi oli olemas, jah. Ega ma ei ole kindel, et need noodid kõik tegelikult ju õiged 
said, mis ma tegin. Aga see rõõm! Ütleme niimoodi, et iga kord, kui midagi 
natuke välja tuli, siis viskas puid alla ainult juurde, ja  
leek läks suurema hooga põlema. 

Tartus oli selline  võimas 
organisatsioon selle meie  vaadeldava  perioodi vaates nagu Tartu 
Tähetorn\index{Tartu Tähetorn}.
Tartu tähetorn on see, mis inimestele välja paistab. Aga sellises 
infotehnoloogilise ajaloo prismas on ta ainult väike ripats  Eesti  
Biokeskuse\index{Eesti Biokeskus}\sidenote{Eesti Biokeskus moodustati 1986. 
aastal Tartu Ülikooli\index{Tartu Ülikool} ja KBFI\index{KBFI} ühisasutusena} 
küljes, mis oli tähetorni kõrval üks väikene kuut, aga kus toimusid tegelikult 
ülisuured asjad. Tartu Tähetorn sattus nagu juhuse tahtel  heasse 
punkti, kuna tema katusele oli hea panna \enquote{sati pann}: 
 sealt paistis kaugele ja üldiselt  ei olnud puid ümber. Seal oli alati 
hea signaal ja 64 kilo sekundis, mis sealt kätte sai, oli ikka väga äge. Kahest Eesti 
esimesest internetiühendusest üks oli Tallinnas KBFI-s\index{KBFI} ja teine, Tartu oma, paikneski 
seal Tähetornis. Õigemini Eesti Biokeskuses, mille ruumid olid Tähetorni 
lähedal. Biokeskuses tegutses teine Villems, Richard Villems\index[ppl]{Villems, 
Richard}\sidenote{Richard Villems oli Eesti Biokeskuse juht selle asutamisest 
alates} kes koos Lippmaadega, nii palju kui mina olen aru saanud, üldse selle  
maailma avas.

Igatahes Amigad jõudsid sinna tähetorni. Nüüd, kujuta ette, 
meil oli juba järgmine asi: meil olid Amigad,  mis olid ühendatud  Internetti. 
Sest kõik, kes  tee peal olid, \emph{tap}'isid seda traati, mis enne Biokeskust katuselt 
alla tuli. 

\question{Tagasi minnes, sul pidi ikka distsipliini olema. Koolituse ajal 
kuskil taga nurgas istudes tuli ju vagusi olla}

Seal tekkisid kohe sellised sotsiaalsed surved. Ma toon sulle ühe näite. 
Ülikool ei jaksanud 24/7 ju  täienduskoolitusi teha. Seevastu 
üheteistaastastele kuttidele pole mitte mingi probleem 24/7 seal kohal olla, või 
kaheteistaastastele, eks ju. 

Ega suhteliselt ruttu mõtlesin ma sellise asja välja nagu võtmeluba. Ja 
laupäeval ja pühapäeval ju ei toimu koolitusi. Kell seitse ärkas 
valvur üles ja see ei ole nali. Neid võtmeloaga inimesi olime kas mina ja äkki 
võibolla üks inimene veel. Ja kui see võtmeluba saadi, siis kell kuus 
nelikümmend  kell helises laupäeval või pühapäeval, vahepeal siin laupäeval oli 
vene ajal kool. Ja siis kell seitse olime seal selle suure maja ukse taga, kus 
oli väga unine valvur, kes alguses ei uskunud, et mingi võtmeluba on.  
Tema ei tea midagi, võtmed antakse täiskasvanutele, minge minema. Aga kui 
sa oled juba kell seitse hommikul sinna kohale läinud, ega sa sealt ära ka ei 
lähe. Ja siis palusin helistada mingitele inimeste pühapäeva hommikul 
süüdimatult. Mul on see võtmeluba, mina olen enda õiguses kindel. Väga 
üksikud valvurid, kes olid uued, ei lasknud sisse, aga mingi paari-kolme kuuga 
olid nad kõik nii-öelda välja õpetatud.

\question{Sest jama ju ei tekkinud. Keegi ei läbustanud, ära midagi ei 
varastatud.}

Läbustamiseks polnud mitte mingit aega. Ainukene aeg, kus toimus mingi jama, 
oli siis, kui inimesed ootasid selle arvutiruumi ukse taga, et mingi koolitus 
lõppeks, et äkki järgmisel koolitusel on auk ja keegi saab sisse. 

\question{Aga miks just sulle anti see luba, kas sa paistsid silma kuidagi? 
Olid eriti tubli, korralik, pealetükkiv?}

Midagi kõigist neist. Aga selle sisuline põhjus oli see, et ma tänapäeva 
mõistes tegin vabatahtlikku tööd. Või noh, paneme selle  lapse vaatenurka. Ma 
nii tahtsin olla nende arvutite juures, et ma olin nõus tegema abiõpetaja tööd, 
kui toimus päris koolitus. Oma vabast ajast ilma rahata. 
Minu jaoks oli niisugune suva titekas kraam, on ju, et  
näidatakse, kuidas  arvuti käima läheb, mis see üldse on ja kuhu tuleb 
vajutada, mis see tähendab \emph{press any key}. Seal õpetati sellist  
ülilihtsa taseme \emph{skillset}'i, mille laps omandab paari-kolme 
päevaga. Ja mul ei olnud mingit probleemi näidata nendele tädidele, et näe, 
vajuta siia. Tädidel oli ka hea meel, et lapsed ka kuidagimoodi  oskavad seda 
asja. Ja Anne Villems\index[ppl]{Villems, Anne} ka mind välja ei visanud eriti. 

\question{\enquote{Eriti}}

Enda arust minul oli väga tore, ma ei tea kui palju sealpool oli seda, et me ei 
jaksa võidelda enam, ei ole mõtet välja visata. Ja neid tüüpe oli päris mitu, 
äkki ikkagi oli mingi kolm tüüpi, kellel see võtmeluba  oli. 

\question{Aga vähe, ikkagi}

Sest ega kõik ei jaksanud seal ka kogu aeg käia, kõik ei mahtunud ka. Eks need, 
kes  kes järgi ei jätnud, need lõpuks jäid. 

\question{See kestis sul kuni keskkooli ajani välja?}

See kestis mul kuni keskkoolini, jah. Keskkooli ma läksin teise kooli, 
Treffnerisse\index{Koolid!Hugo Treffneri Gümnaasium}. Treffneris olid  mingid 
oma arvutiklassid, aga siis juba  muutus see Tähetorn\index{Tartu Tähetorn} 
tolle aja peale selliseks\ldots\ ta oli tegelikult ikka hästi 
palju asju. Seal lindistasime  esimesed lood, ma käisin astronoomiaringis,
mu naaber töötas seal, seal olid Amigad, seal ma õppisin 
C-d\index{Keeled!C} kirjutama. Kaur Virunurm\index[ppl]{Virunurm, Kaur} õpetas. 
Ta oli ainuke tüüp, kes suutis, sest mis me seal tegime, on maailma kõige halvem 
õppimismeetod. Kujuta ette, sinu kõrval on inimene, kes tahab mingit asja 
õudselt osata ja ta ei oska mitte midagi ja ta ei viitsi \emph{manual}i 
lugeda. Põhimõtteliselt sind muudetakse selleks elavaks \emph{manual}iks, kus 
õppimise \emph{are we there yet} käib iga kahe-kolme minuti tagant.

Tartus oli teine keskus veel, füüsikahoone\index{Tartu 
Ülikool!Füüsikahoone}, mis oli Tähe tänava alguses. Taavi 
Talvik\index[ppl]{Talvik, Taavi} toimetas seal ja andis mulle ühe C 
\emph{manual}i, mis oli fotoaparaadiga üles pildistatud vist. Ta oli C 
mingisugune \emph{handbook}, äkki.

\question{Kas see võis olla see Richie kuulus sinise C-ga raamat\index{The C 
Programming Language}\label{sisu:richie}?}

Jah! Kuule, seal oli ainult must ja valge, sinist ei olnud seal midagi. Ja ma 
mäletan, et  isegi mingisugune kümme-viisteist aastat hiljem ma leidsin neid 
üksikuid fotokoopia lehti kuskilt ja lugesin \enquote{oo, päris hea kraam ju}. 
Ega ma väga palju asju C-s\index{Keeled!C} lõpuks ei kirjutanud, tollel hetkel 
(pärast kirjutasin kõvasti). Aga igatahes Kaur\index[ppl]{Virunurm, Kaur} oli 
see tüüp, kes nagu jaksas ära taluda selle tohutu huvi. 

Tol ajal olid juba esimesed XT-d\index{Arvutid!XT} ilmunud ja too 
Assembler\index{Keeled!Assembler} oli hoopiski teisest klassist kui Z80 Assembler, 
mis on ikka sihuke kodu-assembler. Vaatasin, et, kurat, see on ikka hoopis 
teine asi, kaheksa- ja mitte neljakohalised koodid! 

Siis mingil hetkel lõi selline murdeiga sisse ja tollel hetkel arvutid ei 
võtnud enam sada protsenti ajast, vaid seitsekümmend prossa ainult. 

\question{Tavaliselt keskkooli ajal tekib inimestel mingisugune kultuuriline 
kontekst, muusikat sa juba mainisid aga raamatud, veel midagi? Räägi palun 
sellest}

Väga hea, et sa selle välja tõid. Me peame aru saama sellest, millisele lavale  
need idud  kasvama läksid. 
Tartus oli see keskkond  ikkagi akadeemiline ja akadeemiline tähendas 
enamasti kõrget lugemust ning paremat kirjandusega kursis olekut.

Kasakul\index[ppl]{Kasak, Enn} oli tolle aja kohta täitsa OK raamatukogu, 
mu tädil oli ka üsna hea raamatukogu. Aga kui me vaatame neid autoreid või noh 
\emph{label}'eid nii-öelda, mis siis olid, siis Asimov\index{Isaac Asimov} oli 
kindlasti kõva. 

Mu enda vanaema oli tõlkija, tõlkis kuuekümnendatel näiteks \emph{\enquote{I, 
robot}} eesti keelde. Tegelikult ta vist tõlkis kellegagi koos terve Asimovi 
kogumiku. Teine väga tugev liin oli igasugused sõrmused ja nende 
isandad\index{\emph{Lord of the Rings}}. See oli täpselt sama kõva.
Võib-olla natukene mõtlen üle, aga Sõrmuste Isanda lugu  on ikkagi 
ju tegelikult lugu  mingist suurest sinust palju tugevamast kurjusest, mille 
vastu ei saa. Ja mõtle, mis aastad need olid, 1987-1989! See lootus! Need 
raamatud kuidagi õudselt hästi sobisid sinna aega.

\question{Kääbik oli eesti keeles olemas jah, aga mina sain teada, et see on osa 
suuremast loost, alles üheksakümnendate lõpus. See oli siis inglise keelne 
raamat?}

Jah, inglise keelses. Sa vaatasid neid raamatuid ja nad olid erilised. Nad olid 
niisuguse keskmise vene  papi-trüki kõrval läikivad ja ilusad. Inimesed olid 
pannud oma  raha sinna alla, et need saada.  Kuidas tunda ära   
inimest, kes tollel ajal nagu hästi tegev oli? Neil kõikidel on kodus kuskil 
ühes suhteliselt nähtavas kohas  täis-Tolkien. Hästi lihtne indikatsioon 
tegelikult. 

Ulme  oli siis  teine liin. Asimov, Bradbury\index{Ray Bradbury}, kõik 
see. Osad asjad olid tegelikult Mirabilia sarjas\sidenote{Mirabilia oli 
aastatel 1973 kuni 2012 ilmunud kirjastuse Eesti Raamat raamatusari, mis 
keskendus peamiselt ulme- ja kriminaalromaanidele. Omas ajas oli tegemist 
suurepärase võimalusega tutvuda (toonasele ajale kohaselt reeglina väga hästi 
tõlgitud) ulmekirjanduse klassikaga: ilmusid Simaki, Lemi, Strugatskite, 
Asimovi, Bradbury ja paljude teiste romaanid ja lühilugude kogumikud. Paljus 
kujundas just see sari terve põlvkonna ulme-huviliste maitse, Lääne klassikute 
hulka segati vahvasti ka Eesti, Soome ja näiteks ka Läti autorite loomingut} 
ilmunud ka. 

\question{Aga Strugatskid?}

Jaa, aga need olid sellised \enquote{kodukirjanukud}. Loomulikult nemad 
ka\index{Strugatskid} ja Stanisław Lem\index{Stanisław Herman 
Lem}\sidenote{Stanisław Herman Lem (1921-2006) oli Poola ulmekirjanik, kelle 
teosed sageli olid ühekorraga nii filosoofilised kui satiirilised ja 
humoorikad} ja teised. Oli selline ulmekirjanduse kogumik nagu \enquote{Lilled 
Algernonile}\sidenote{\enquote{Lilled Algernonile} (1976), sarjast \enquote{Ajast Aega}. Koostanud 
Ain Raitviir}, see oli ka suhteliselt kohustuslik kirjandus. Vaata, kui 
tütarlastel oli võib-olla Herman Hesse Stepihunt kotis, siis poisid raamatut 
kaasas ei kandnud, aga kuskil oli ikka natuke ära näritud nurkadega \enquote{Lilled 
Algernonile} või midagi sinnapoole. USA ulmekirjanduse sissevool tõepoolest 
moodustas sellise kultuurilise tagapõhja toimuvale.

Aga muusikaga oli jällegi teistmoodi. Suured inimesed kuulsid suurte inimeste 
muusikat, noored kuulasid noorte muusikat. Ja Tähetornis\index{Tartu Tähetorn}  
kuulati ikka palju muusikat maki pealt. Seal oli sellised naljakad 
\emph{setup}'id koos, et esiteks oli see Tähetorn ise, siis olid seal 
füüsikatudengid, kes pühendusid astronoomia suuna peale (näiteks Kaur 
Virunurm\index[ppl]{Virunurm, Kaur}). Teisest küljest olid seal Tähetorni 
direktori (ikkagi tähtis mees) lapsed, kes käisid Miina Härma 
Gümnaasiumis\index{Koolid!Miina Härma Gümnaasium} ja need lohistasid kohale 
mingeid omi koolivendi ja -õdesid, need tuiasid seal ka kogu aeg ringi. Aga 
Tartus tollel ajal oli kaks sellist kooli, mis  defineerisid, mis on äge. Need 
olid Treffner\index{Koolid!Hugo Treffneri Gümnaasium} ja Miina Härma, mõlemad 
arvasid, et nad on paremad kui teine. Tartu värk, onju. No ja Tähetornis olid 
siis ka mingid Treffneri tüübid. Ja vot siis tekkis sellise segu keskkoolist, 
ülikoolist, Internetist, millest mitte midagi peale plahvatuse tulla ei 
saanudki. Seal oli sihukesi tüüpe, kes kõik on tänapäeval Eestis mingisugused 
väga asjalikud tüübid.

\question{Keskkoolinoorena pidada astrofüüsikutega sammu ja originaalis 
Tolkieni lugeda (mis ei ole lihtne keel) ei ole lihtne. Sa pidid ikka nutikas 
inimene olema}

Väga konkreetselt oli sõnaraamat kõrval, kuni sa jõudsid nii kaugele, et seda 
enam vaja polnud. Ahjaa, meil oli koolis  niisugune huvitav õppeaine, nagu 
teaduse inglise keel. Treffneris\index{Koolid!Hugo Treffneri Gümnaasium} 
oli selline õppesuund gümnaasiumis nagu 
bioloogia-keemia suund. Meil olid seal sellised põnevad asjad nagu ladina keel. 
Tavaline inglise keel oli ka, aga oli ka teaduse inglise keel, mille 
kõrval tavaline inglise keel oli \emph{walk in the park}. Seda ainet andis 
meile muide bioloogia õpetaja\sidenote{Õpetaja Tago Sarapuu\index[ppl]{Sarapuu, Tago} 
õpetas ka Meelis Roosile\index[ppl]{Roos, Meelis} bioloogiat.}, kes ise oli hästi tugeva akadeemilise taustaga ja  
tegi hiljem pikalt akadeemilist karjääri. Sa 
mainisid seda sinise kaanega C õpikut. Kujuta ette, et sul on näiteks geneetika 
kohta sama asi ja siis sa võtad ja närid ennast sellest läbi lihtsalt. Paberit lendab 
kahele poole, aga sa tõlgid selle kõik ära. Sai ei tõlgi laule, 
seda õpib inglise keele tunnis. Ja see tegelikult õudselt aitas kaasa hiljem.

Ega meil oli saksa keel ka, ma sain saksa keele lõpueksamil kooli parima hinde. 
Aga miks? Sellepärast, et Kaur Virunurmel\index[ppl]{Virunurm, Kaur} oli samal 
ajal ülikoolis ka saksa keele eksam. Ja siis ta tõmbas sõnaraamatu arvutisse 
(ma arvan, et ta kuskilt pidi mingi faili saama), tegi baasi sellest ja siis oli 
võimalik skoorida, et õige vastus punkt, vale vastus miinuspunkt. Mina  
valmistusin saksa keele eksamiks konkreetselt niimoodi, et viimasel õhtu enne 
saksa keele eksamit ma mängisin punktide peale saksa keele sõnade tõlkimist. Ja 
see aitas mind  jubedalt. Ma arvan, et see oli üks esimesi kordasid, kus ma 
võin kindlalt väita, et infotehnoloogiline tööriist parandas minu sooritust 
hüppeliselt. Lõpuklassis vist ühel veerandil oli mul saksa keel ka kaks. Aga 
need olid sellised ealised iseärasused, hinded ja teadmised ei ole alati 
omavahel nagu lineaarses seoses.

Ja see kõik toimus sealsamas Tähetornis.

Miina Härma Gümnaasiumi\index{Koolid!Miina Härma Gümnaasium} kutid tõid sinna 
mingisugused Smith'id ja Cure'd ja 
samas ka vene muusika liinid. Ilmusid välja kitarrid, neid mängiti ja  
tegelikult isegi lindistati. Ma ei tea, mille peale, kassettidele kuskile. 

Samas oli Tähetornis selline  nõukogude-hõnguline teaduskultuur, mille juurde 
käis näiteks konjaki ja kohvi koos joomine. Mida küll  keskkooli õpilased ja 
üliõpilased ei saanud endale lubada, aga tubades  oli see hõng  üleval. Ma ei 
teagi, mis asjaoludel seda seal joodi. Mingeid  läbusid otseselt 
sellisel kujul ei toimunud, aga, kuidas öelda, see kõik tekitas atmosfääri.

\question{Seal ju keegi tegi teadust, ja mitte halba teadust}

Ja-jah, täiega. Ma selle kõige kõrvalt käisin isegi astronoomiaringis. Saadeti 
isegi mingi oma tööga kuskile Balti või isegi suuremale  
õpilaskonverentsidele esinema. Kõike sai teha.

\question{Kui see keskkool ükskord otsa sai, siis mis kooli sa läksid ja mida 
õppima?}

Ma pidin tegelikult minema ülikooli ajakirjandust õppima. Ülikoolis oli sihuke 
asi, et kui sa tahtsid ajakirjandust õppida, siis pidi olema portfoolio, sa 
pidid olema midagi avaldanud. Ülikooli mitteametlik \emph{statement} oli see, 
et kui sa tahad tulla ajakirjandust õppima, siis kuidagi veena meid, et sa 
tõesti tahad seda teha. Ajakirjanduse või üldse meedia õpetamine on 
 suhteliselt kallim tegevus kui näiteks keeleõpe. Kuidagimoodi aita 
meid, et me usume, et sa tõesti tahad ajakirjandust õppida, et sa lihtsalt ei  astu 
kuhugi juhuslikult sisse. Mulle tundus, et kultuuriajakirjanik on väga äge olla. Tolles vanuses, 
ma arvan, iga mees arvab, et  kultuuriajakirjanik on väga äge olla, sest 
kindlasti on olemas oma arvamus maailmast ja enda arvamuse mass-tiražeerimine  tundub 
nagu kuidagimoodi veidral kombel teiste aitamisena.

Ühesõnaga, ma tegin igasuguseid ettevalmistusi, avaldasin ja tõesti käisin 
kontsertidel, tegin intervjuusid, kirjutasin arvustusi ja kõik oli nagu OK. 
Aga ma käisin näiteringis ka samal ajal. Lõpuklassi kuskil veebruari-märtsikuus 
olid Viljandis\index{Viljandi} lavaka\index{Lavakas}\index{Eesti Muusika- ja 
Teatriakadeemia Lavakunstikool|see{Lavakas}} sisseastumiskatsete 
eelvoorud. Mõtlesin, et äge, läheme Viljandisse trallima ja pullima ja 
seiklema. No ja sealt eelvoorust ma sain edasi. Kool sai juba läbi ja  olin 
nagu hädas, et juunis on Lavaka asjad,  juulis peaks nagu eksamid 
ära tegema, mis ma siis teen. Läksin Lavaka sisseastumiskatsetele ja sain sisse.

Ma üldse ei stressanud, võib olla see ka aitas. Ma teadsin, et mul on
ajakirjandusega  niisugused plaanid. Head soovituskirjad olid ka 
toonaste Postimehe\index{Postimees} ajakirjanike poolt, kellega ma olin neid 
asju teinud, nad õpetasid mind ja igatepidi tore oli. No ja siis ma õppisin 
ligi aasta Lavakas\index{Lavakas}. Aga, no vot, seal oli selline lugu, et ma 
mingil hetkel sain aru, et see ei ole ikkagi see asi, mida ma teha tahan. Aga 
selle otsusele jõudmisel ikkagi  kõik see eelnev, mida ma kirjeldasin, oli 
tegelikult väga tugev mõjutaja. Ütleme,  see lendamise trenn tegelikult ikka aknast 
paistis kogu aeg. 

Kooli kõrvalt ma sattusin sellisesse ägedasse kohta nagu Riigikogu 
Kantselei\index{Riigikogu Kantselei}.

\question{Sel ajal olid seal juba ju võrgud ja BBSid ja asjad?}

Riigikogu Kantseleis oli täiesti adekvaatne kraam juba aastal 1992. 
Seal oli siis  sihukene  äge arvutivõrgumäng, 
põhimõtteliselt tolle aja Fortnite. Selle nimi oli 
MUME\sidenote{Üks populaarsemaid MUD-tüüpi mänge, mille nimi tulenebki fraasist 
\emph{Multi-Users in Middle-Earth}. Loodud varasele 1991. aastal loodud (siiani 
aktiivselt arendatavale) DikuMUD\index{Mängud!Muda!DikuMUD} mootorile. Vaata ka märkust \ref{sidenote!muda} 
leheküljel \pageref{sidenote!muda}.}, tegu oli Tolkieni ainetel loodud 
Mudaga\index{Mängud!Muda}. Ma tuletan meelde, need ringkonnad olid kõik väga 
tugevalt Tolkieni usku. MUME oli  põhimõtteliselt siis-tekstipõhise 
kasutajaliidesega võrgumäng. 

\question{Seal oli mingi server, eksole, kuhu mängijad külge läksid?}

Jah. Telneti pordi kaudu tõmmati sind külge,  kõik istusid oma 
\emph{socket}is, aga kõik korraga nägid, mida teised teevad. Ja kuna ta oli 
teksti baasil, siis tegelikult võrguühenduse kiirus ei olnud probleemiks.

\question{Aga kuidas sinna Riigikogu Kantseleisse\index{Riigikogu Kantselei} 
sai see adekvaatne kraam? 1992. aastal ei olnud Eesti Vabariik veel kuigi 
ägedal järjel ja oli muudki, mida korrastada?}

Väga hea küsimus. Mul ei ole kõikide kuttide nimesid meeles. Aga näiteks Tarvi 
Martens\index[ppl]{Martens, Tarvi} kindlasti teab sellest midagi. Siis sihuke 
kutt nagu Toomas Mölder\index[ppl]{Mölder, Toomas}, kes oli nlib'i, ehk siis 
tollase Rahvusraamatukogu\index{Rahvusraamatukogu} IT-juht  tänapäeva mõistes, ma arvan. Ta oli igatahes 
väga adekvaatne. Ja eks, ma arvan, KBFI\index{KBFI} rahvas ka aitas.

\question{Peale MUME mängimise tegite te seal ju kasulikke asju ka?}

Mina käisin tegelikult Lavakas tol ajal põhiliselt ja ma teadsin, et nad teevad 
mingeid väga vingeid asju. Aga tavaliselt, kui mina sinna saabusin, siis 
kuidagi töö lõppes ära, sest tuli \emph{orc}'ideks kehastuda ja minna 
\emph{whiteskin}'e tapma. MUME  adrenaliini tase absoluutselt ei jää alla 
tänapäeva arvutimängudele. Olid seal mingisugune haldjas, näiteks, ja tuleb sulle 
teade: \enquote{\emph{An orc enters the room}}. Selle peale ka täna teatud  
seltskonnal  lükkab vererõhu kakskümmend prossa ülespoole. Mudasid oli veel aga 
MUME oli üks esimesi selliseid mänge, mis kestis üle aastate. Esimesed 
eestlased, kes seal mängisid, tegid oma tegelased aastal 1991 või 1992 ja see 
kestis mingi kolm-neli aastat järjest. Pronto\index[ppl]{Pronto} mängis muide 
väga kõvasti MUMEt, Tanel Raja. 

\question{1993, ma mäletan, ikkagi mingid inimesed sisenesid Liivi tänaval 
VAXi klassi\index{Tartu Ülikool!Matemaatikateaduskond!Liivi 
õppehoone}\sidenote{\label{sidenote!vaks}Sõna \enquote{vask} mitmesuguste 
variatsioonidega  kutsutud ja ilmselt klassi toitnud arvuti tüübinime 
\enquote{VAX}\index{Arvutid!VAX} järgi 
nime saanud klass asus Matemaatikateaduskonna Liivi tänava 
õppehoone esimesel korrusel ja koosnes vask.ut.ee\index{Masinad!vask.ut.ee} 
külge ühendatud terminalidest} ja kui mina ükskord ülikooli lõpetasin, nad 
sealt väljunud ei olnud}

Jah, \enquote{pidev \emph{lag}} oli muide sealsamas VAXi klassi kõrval olevas 
ES-klassis, mis oli alati tühi, kuna need arvutid olid jamad\sidenote{Ilmselt 
peab Jaanus silmas Raua\index{Masinad!raud.ut.ee} klassi, mis käis päris IBMi 
raua ja mitte ESi peal. Raul Tölp\index[ppl]{Tölp, Raul} meenutab: Sattusin kas 
1996 või 1997 juhuslikult Liivi 2-te, kui mul paluti IBM-i esindajana raud.ut.ee 
serverile viimane \emph{power off} teha. Sinna juurde käis jutt, et masin küttis 
vesijahutusega tervet maja.}. Liivi tänava 
VAXi klass on omaette peatükk, sinna kohe varsti jõuame.

Aga igatahes kõik see lõppes nii, et ma tulin Lavakast esimese kursuse viimase veerandi 
pealt ära. Omavahel öeldes oli seal see, et kui sa oled üheksateistaastane, 
siis sa  võtad välismõjusid väga tugevalt sisse. Ja tolle aja  
näitleja amet erines sada protsenti sellest, mis ta täna on. Ma arvan, et  mul 
oli nädalaid, kus ma iga päev alla poole pudeli viina ei joonudki üldse. 
Organism oli nii tugev, vedas ilusti välja. Aga ma nägin ikkagi kutte, kes olid 
seda kümme aastat teinud. Ja mingil hetkel oli küsimus, et kas mina jaksaksin 
ja kas ma tahaksin ka niimoodi? \emph{Hell no}. See tegelikult  aitas ära 
tuleku otsust vormida, pluss see lendamistrenn akna taga. 

Siis ma läksin Tartusse\index{Tartu Ülikool} Eesti keelt, täpsemalt 
arvutuslingvistikat, õppima. See on ilgelt lahe ala, minu arust.

\question{Kes seda õpetas, see oli maailmaski tollal suhteliselt uus ala?}

Nüüd ma jään sulle vastuse võlgu. Seal oli üks hästi lahe lühike vanamees, kes 
oli  totaalne guru. Hästi viisakas, vaikne ja rahulik sell, nii palju kui 
mina temaga suhtlesin. Aga tema juurde ma jõudsin alles kolmandal aastal pärast 
spetsialiseerumist. Enne olime lihtsalt ühes väga toredas teaduskonnas, kus 
põhiliselt õppisid tüdrukud.

Selle taustal oli mul ikkagi tunne, et ma peaks ka mingit tööd tegema. Ja kõik, 
mis sul on, on ainult see lendamistrenn, kus oled natuke käinud. Ja kuidas sa 
sellest siis mingi asja teed. Aga mängu tuli seesama Vase klass.

Mängisime seal ju \emph{StackMUD}i\index{Mängud!Muda}. Stacken.kth.se\index{Masinad!stacken.kth.se} oli Rootsi 
Kuningliku Tehnikaülikooli\index{Rootsi Kuninglik Tehnikaülikool} mingi 
VAX\index{Arvutid!VAX}\sidenote{Arvutisari, mille töötas DEC välja
seitsmekümnendate keskel. Siiani üks kõige tuntumaid omalaadseid arhitektuure,
oli ta PDP-11\index{Arvutid!PDP-11} edasiarendus, peamiselt mälu virtuaalse
adresseerimise suunas. \emph{VAX - Virtual Address Extension}}, mille peal 
jooksis BSD\index{OS!BSD}, mille peal pandi käima Muda. 
Originaalne DikuMUD on tehtud Taanis, sealt Rootsi ei ole palju maad ja nad 
suhtlevad kogu aeg omavahel.

Selle mängimise tegi huvitavaks, et mängijad ei olnud  matemaatika üliõpilased, 
nagu oleks võinud arvata. Need olid eesti keele üliõpilased, usuteaduskonna 
üliõpilased. Näiteks praegune kirjanik ja usuteaduskonna õppejõud Meelis 
Friedenthal\index[ppl]{Friedenthal, Meelis} oli väga originaalne mudamängija. 
Oli teisigi tüüpe, kes  käisid seal tõesti väga-väga palju mängimas, mina 
sealhulgas. Ja selle seltskonnaga suhtlus ei  piirdunud ainult 
mängimisega, me ka ehitasime seda maailma. Ma olin mingi top kaks või kolm 
\emph{builder} kümnest üldse.

\question{Kuidas see käis? Kirjutasid mingit koodi või skripte kusagile või?}

Väga lihtne oli. Ma sain koodi koopia, mul oli andmebaasi struktuur, kus ma 
täitsin täitsin väljad ära. Kujuta ette, et sa saad andmebaasi \emph{dump}i ja 
siis võtad selle teksti \emph{editor}is lahti ja siis  näiteks ridadest sada 
kuni tuhat teed koopia ja  kirjutad sinna asjad teistmoodi. Üldse ei tundunud 
raske. Ja \emph{editor} oli vi\index{vi}\sidenote{Unixi spetsifikatsiooni osaks 
saanud 1976. aastal kirjutatud tekstiredaktor, mis on siiani teatud 
ringkondades (ka käesolev tekst sünnib osalt vi abil) väga populaarne. Siin 
kontekstis on oluline, et erinevalt tänapäevastest tekstiredaktoritest ei 
olnud vi mitte ainult tekstipõhine vaid ka suhtles kasutajaga ainult 
suhteliselt obskuursete käsu- ja klahvikombinatsioonide abil. Näiteks on 
legendaarsed  algajate kasutajate tulutud katsed redaktorist väljuda - 
selleks kasutatavad klahvikombinatsioonid \texttt{ZZ}, \texttt{:q!} ja veel kümmekond (!) 
samalaadset ei ole just intuitiivsed.} 
loomulikult. Kirjutamine käis tsoonide kaupa. Üks ala on üks tsoon, seal oli 
mingi sada ruumi, iga ruumi kohta kirjutasid ingliskeelse kirjelduse. Asju sai 
ruumis ka vist olla kuni sada või kuni 255, kolle sai olla ka kuni mingisugune 
kogus. Täitsid kõik statistika ära kõigi ruumide kohta, kirjeldused juurde, 
tegevused juurde ja postitasid. Kutid kompileerisid selle ära ja nii ta tuli. 

Me seime nende mängutegijatega tegelikult ka üsna hästi läbi. Ja no eks sa ikka 
vahel räägid olmest ka. Et \enquote{kurat, meil on siin ainult üks klass ja 
seegi  on kogu aeg pool-täis ja kui mingi ahv FTP-ga tont teab mida tõmbab, ei 
saa üldse mängida}. Ühesõnaga, niisugused inimlikud teemad. Ja siis nad ütlesid, 
et meil on siin üks arvuti üle, me saime uuema. VAX\index{Arvutid!VAX}, peal on 
BSD\index{OS!BSD}. Ma ütlesin, et ilgelt lahe, kas me seda kuidagi endale ei 
saaks? Kolme-nelja nädala pärast tuli kutt tagasi, et \enquote{no 
põhimõtteliselt saate}. 

\question{Aga see masin oli ju Rootsis?} \label{sisu!jaanus_liivi_tn}

Jah. Siis läksin mina Otto Telleri\index[ppl]{Teller, Otto} juurde, ta oli vist 
arvutiteaduse õppetooli juht\sidenote{Tõenäoliselt toimetas Otto siiski Tartu Ülikooli 
Arvutuskeskuses\index{Tartu Ülikool!Arvutuskeskus}, mis oli õppe-struktuurist 
eraldiseisev üksus}. Ütlesin, et te ilmselt mind ei usu, aga palun 
uskuge. Et Rootsis on üks arvuti, sellega tuleb kaasa 16 terminali, 
põhimõtteliselt me saaks teha ühe klassi, ma võin ise seda klassi 
administreerida (Kui sa muda sees mängid siis sa lihtsalt oled kaelani porine, 
ei ole midagi teha, süsteemide administreerimine tekkis  kuidagi iseenesest). 
Aga ma olen lihtsalt üks üliõpilane, ma ei tööta siin, palun aidake. Rääkisin 
veel Anne Villemsiga\index[ppl]{Villems, Anne} kelle põhiteene oli 
see, et ta ütles, et see on tegelikult \emph{legit} mees. Ta võib väga hullu 
juttu ajada aga ta kunagi ei valeta. Lõpuks Otto Teller ütles, et see on küll 
väga imelik kõik, aga olgu. Aga ise ajad kõik asjad korda.

Siis kirjutasin mingid kirjad igale poole, sain vastused. Ma ei tea, mida Otto 
Teller tegi, aga tegelikult hüppas lihtsalt pea ees tundmatusse mingi kahekümneaastase 
kuti kätega vehkimise peale. 1993. sügisel sõitsime Rootsi seda klassi 
üle vaatama ja  jaanuari- või veebruarikuus oli korraga mingisugune furgoon 
ukse taga Laial tänaval. Sellest tekkis Laia tänava klass\index{Tartu Ülikool!Laia tänava 
arvutiklass} ja mina sain tolle klassi adminiks. See oli minu 
esimene töökoht.

Tollel ajal admin ei olnud ainult  tehniline,  ta oli ka  
administratiivne tegelane, põhimõtteliselt kohalik jumal. Klassi administraator 
oligi äkki selle koha ametinimetus. Oleks mul mingisugune võimu-iha olnud, 
võinuksin seda väga hästi realiseerida. Aga ma tegin nii, et mingi hunnik tüüpe 
said võtmed ja ma palusin, et \enquote{ärge serveriruumi minge, eksju}. Kuskilt 
veeti mingisugused eraldi kaablid, Zyxeli\index{Zyxel}\sidenote{1988. aastal 
Taiwanil asutatud Zyxel Communications Corporation tootis omal ajal 
ülipopulaarseid ja hinnatud modemeid} modemid tegid üle \emph{leased line} 
internetiühenduse  ja seal me müttasime.

Jah, oleme väga ausad, tänapäeva mõistes oli see klass üks totaalne õnnetus. Nii 
madala SLA-ga asja ma  väga palju ei ole ka hiljem näinud. See arvuti oli väga 
vana, kui ta tuli. Ta läks väga tihti katki ja ma ei tundnud nii hästi seda 
rauda, see oli mingi väga proff raud. Selline legendaarne mees oli nagu Viljo 
Soo\index[ppl]{Soo, Viljo}, kes oli tänapäeva mõistes \emph{sysadmin} ja kes 
ikka  väga palju kordi aitas mul selle masina seal käima panna. Tollal tundus 
see liiga pikk aeg,  aga läks mingi kaks-kolm kuud ja me 
saime selle kõik käima ja see klass töötas. Töötas mingi neli või 
viis aastat äkki, palju aastaid. 

Teda kasutati  samal eesmärgil nagu vaske. Cure\index{Masinad!cure.ut.ee} 
oli selle klassi nimi, The Cure järgi. 

Nethack\index{Mängud!Nethack} oli kunagi selline mäng, mäletad? Sellest oli 
mingisugune naljakas kloon tehtud. Ma otsustasin, et see mäng on vaja eesti keelde 
tõlkida. Ja oli põhimõtteliselt samamoodi, et võtsid ikka selle C koodi lahti, 
hakkasid ülevalt reast number üks lugema.

\question{NetHacki lähtekood on niisamagi hea lugemine, see on üks kahest, 
mida ma olen oma elus lugemise eesmärgil välja trükkinud. Teine on Perl}

Ühesõnaga võtsin selle koodi ette ja alustasin reast üks, lõpetasin viimase 
reaga, kaasa arvatud kõik \emph{library}d ja mis tal seal kaasas oli, ja 
tõlkisin  kõik eesti keelde. Ma selle põhilise ekraani teadete osani jõudsin 
kell neli hommikul. Me kõik teame seda, et mingil hetkel tuleb  
väga suure väsimuse all niisugune veider pool-eufooriline meeleolu, kõik on 
seal olnud. Ja mul sattus see hetk tõlke peale ja nii juhtus, et see 
mäng oli väga naljakas\sidenote{Mängus tembutanud \enquote{mõõkhambulisi varblasi} meenutatakse siiani hea sõnaga}. Seda mängiti hästi palju seal klassis, seda enam, et 
võrguühendus alati ei töötanud aga Nethack oli kohalik. Senikaua, 
kuni keegi kuskil otsis, kus Viljo Soo\index[ppl]{Soo, Viljo} on, et ta meie 
modemitele restardi teeks, mängiti seda meie Nethack'i. Mina ei arvanud sest 
midagi, tegid ära, tehtud ja las ta olla. Kahjuks see kood läks koos Cure 
masinaga kaduma.

\question{Nutikal inimesel tol ajal oli tüüpiliselt kaks suunda, kuhu kiskus: 
Kas akadeemilisse poolde teadust tegema või äri suunas. Sind aga ei tõmmanud 
kumbki?}

Mind äri ei tõmmanud, sest ma olen äärmiselt vaesest perekonnast pärit. Raha ei 
olnud mingi asi, mul ei olnud seda lihtsalt kunagi. Sai ja piim oli asi, mis 
läks kuude kaupa. Ja kui sul raha ei ole, siis ei teki sul temaga ka lähedasi 
suhteid. Akadeemilise maailmaga oli see, et me räägime ju ajast, mil ma olin 
alles esimesel kursusel.

Aga sealt Cure klassi tegemisest mõned kuud edasi toimus üks selline, ma 
ütleksin, tüvikursus. Pildile ilmus tagasi Anne Villems\index[ppl]{Villems, 
Anne} ja korraldas 1994. aasta alguses Eesti esimesed veebmasterite kursused. 
Liivi tänaval olid kuulutused üleval.

Oma arust olin ma ikka juba kõva käpp siis. 
Gopher\index{Gopher}\sidenote{Gopher oli varajane hüpertekstiprotokoll, WWW 
protokollistiku eellane. Erinevalt suhteliselt lõdvalt struktureeritud veebist, 
surus Gopher sisu küllalt rangesse hierarhiasse ning oli navigeeritav 
menüüsüsteemi abil.} oli tol ajal teema. Ei pidanud üldse palju lugema, et 
Gopherist aru saada. Ta oli nii \emph{simple},  tegelikult  HTML 1.0 standardi 
eelkäija. Sul oli klient, server ja mingisugune \emph{markup language}, mille 
spetsifikatsiooni võis kümne minutiga üle käia ja põhimõttest aru saada. Ta ei 
olnud nii keeruline, et sinna  oleks väga palju aega pannud. Siis tuli HTML, 
mis oli nagu Gopher, aga jutt oli natuke pikem. 

Need veebmasterite kursused ikkagi tulid, ja kursusel selgus, et ikka natuke 
rohkem kui kümme minutit läheb aru saamisega. See seltskond, kes sellel kursustel käis, oli 
suhteliselt kirju. Tänapäeval on ikkagi see, et sa satud oma \emph{dedicated} 
silossse, sellesse kohta, kus sa pead sattuma ja kui sa tolles silos ei ole, sa 
sinna ei satu. Aga nendele kursustele sattus igasuguseid karvaseid ja sulelisi, 
no täiesti igasugust rahvast teaduskonnaüleselt. Ma ei imestaks, kui isegi 
mingeid usuteadlased kohal oleksid olnud. Aga see oli OK.

Anto Veldre\index[ppl]{Veldre, Anto} oli näiteks seal. Tollal oli ta hoopis 
teine mees kui täna, aga sealt asjad hakkasid. Tal oli juba mingi huvi seal 
ees, ta tegi mingeid teisi asju, aga minu arust oli ta ka tollel kursusel. Aga 
ma ei mäleta, kas õpetaja või õpilasena.

\question{Nojah, mis seal siis nii väga palju vahet}

Tollal ei olnud vahet. Üks oli selle asja läbi lugenud ja rääkis teistele 
edasi. Aga Anne Villems\index[ppl]{Villems, Anne} oli selle asja ikka 
tegelikult väga hästi ette valmistanud. Kõik käisid kursustel ära ja otsekohe mitte 
midagi otseselt ei juhtunud. Aga kuskil jäid need nimekirjad alles. 

EENet\index{EENet} tegi endale veebilehe ja oli kuulnud, et on sellised 
inimesed nagu veebmasterid, kes veebi oskavad teha ja et tekib nagu mingisuguse 
avaliku infohalduse funktsioon üldse organisatsioonis. Nüüd ma oskan seda nii 
hästi seletada, tollel hetkel me pigem panime asju Internetti. Kõva sõna, 
võtame Eesti kaardi ja kui sinna peale vajutada, siis juhtub midagi. See 
\emph{skill} oli kõik olemas. Nii EENet otsustas, et ta teeb  endale täie 
kohaga veebmasteri positsiooni. Äkki võis olla nii, et keegi ülimalt \emph{smart} 
sekretär oli seda seni teinud, aga ta pidi natuke teisi asju tegema. Või tegi Marek 
Tiits\index[ppl]{Tiits, Marek} seda, kes seal tol ajal oli? Ei mäleta. Igatahes 
asi lõppes sellega, et mingi pool aastat pärast noid kursusi  kutsus EENet mind 
veebmasteriks. Palk oli lihtsalt kolm korda kõrgem. Jube raske oli ei öelda. 
Leidsime siis kellegi teise, kes toda klassi haldas. 

Ja siis noh, siis läks  nagu väga ägedaks. Me saime Tarvi 
Martensiga\index[ppl]{Martens, Tarvi} tuttavaks ja Toomas 
Mölderiga\index[ppl]{Mölder, Toomas} ja käisime Eestit esindamas mingitel 
asjadel juba ja\ldots\~ Tegime EENetile korraliku veebi. Tollal oli ta korralik, 
pärast muidugi tehti palju-palju ägedamad. Sinna tuli hiljem Pille 
Pruulmann-Vengerfeldt\index[ppl]{Pruulmann-Vengerfeldt, Pille}, kes on 
meediaprofessor äkki praegu kuskil Rootsi ülikoolis\sidenote{Aastal 2020 on 
Pille Malmö ülikoolis meedia ja kommunikatsiooni professor} ja kes on ka 
ERR-i\index{Eesti Rahvusringhääling} nõukogu 
liige. Seal ma puutusin Marek Tiitsu\index[ppl]{Tiits, Marek} kaudu esimest 
korda kokku europrojektidega. See oli tollal, kui eurot veel ei olnud, aga oli 
selline rahaühik nagu eküü\sidenote{Selle valuuta tähiseks oli ECU: 
\emph{European Currency Unit}}. 

\question{Marek oli tol ajal ju see võlur, kes valdas unikaalset teadmist, 
kuidas fondidest raha saab}

Ja mina aitasin tal neid asju teha. Ma arvan küll, et see aitamine tegelikult 
oli  umbes niimoodi, et mina tulin mingite veidrate ja mitte väga reaalsusega 
kokkupuutes ideedega ja tema siis  kasutas nendest hullustest 
mingisugust väikest osa, millel oli mingi point, projektide juures ära.

\question{Praeguseks on ära unustatud, aga tol ajal käis tohutult palju igasugu 
põnevaid ja kasulikke asju EENeti ja IBSi\index{IBS|see{Institute of Baltic 
Studies}}\index{IBS} purkides}

Seal jooksis igasugu naljakaid teenuseid. Aga see polnud kõik, millega tegeleti. 
Näiteks minu tööarvutiks oli Marek suutnud tekitada  Silicon 
Graphicsi\index{Arvutid!Silicon Graphics}\sidenote{1990. aastal asutatud ja 
2009. aastal pankrotistunud Silicon Graphics oli peamiselt 3D-graafikale 
keskendunud riist- ja tarkvaratootja. Mitmed varased arvuti abi kasutanud 
filmid, näiteks 1993. aastal linastunud Jurassic Park, kasutasid just Silicon 
Graphicsi tööriistu. Ettevõttele tegi lõpu odavate laiatarbe x86-arvutite 
võimsuse kiire kasv} masina. Silicon Graphics oli nii kõva asi, et tänapäeval 
ei ole tõenäoliselt \emph{consumer} arvutit, mis oleks nii palju tavalisest 
kõvem. Võib-olla on ta võrreldav selle Mac Proga. Ta oli ulmeline aparaat, 
selle disain oli juba selline. Korpused olid värvilised! Võtad lahti, kõvaketas 
käib lahti kangiga. Sa said aru, et see on nagu nii kõva, ta oli nagu autode 
Bugatti või Porsche 911  mingi eriti vihane tuuning. Täiesti \emph{over the 
edge}. Ja sa saad aru, miks ma selle töö hea meelega vastu võtsin, olles tulnud  
totaalselt vananenud VAXi klassi administraatori koha pealt. See oli tõeliselt 
äge!

Ja kõige selle taga oli selline, ma ütleksin  pikk, aga  heledat positiivset 
värvi, Richard Villemsi\index[ppl]{Villems, Richard} vari. Tema seda kõike 
tegelikult püsti hoidis.

\question{Hoidis püsti, aga vist ka natuke nagu joonte sees, et inimesed päris 
hullustega ei tegeleks?}

Jaa. Richard Villemsi naine on muide Anne Villems\index[ppl]{Villems, Anne}. 
See tegelikult on  üks ülivõimas perekond. Tänu Richard Villemsi suurele 
mõjule toimusid EENetis\index{EENet} igasugused väga kõvade asjade arutelud, 
mis üldse tegelikult ei olnud selle organisatsiooni põhikirjajärgne tegevus. Ja 
sealsamas oli ka ülikool,  sealsamas oli see sama arvutiteaduse 
instituut\index{Tartu Ülikool!Matemaatikateaduskond!Arvutiteaduse Instituut}  
Liivi tänaval. Siis oli seal füüsikamaja. Seal nagu seda püssirohtu oli, ja 
tekkis  igasuguseid initsiatiive.

\question{Kas kogu selles maailmas BBSid ka kuidagi figureerisid?}

BBSid olid kogu aeg taustal ja mingites BBSides mul isegi olid kasutajad, ma 
isegi käisin seal. Aga BBSi tehnoloogiline \emph{carrier} oli modemiga üle
telefoniliini ühendumine serverisse. Mina sain väga varakult väga kiire 
Interneti juurde, 64 kilo sekundis on väga kiire. Ja kui sa oled \emph{text 
based}, siis ta tegelikult on  tuntavalt kiire. 

Ja sealt tulid juba väga kiiresti IRC'ud\sidenote{\emph{Internet Relay Chat 
(IRC)} on klient-server arhitektuuril põhinev tekstipõhise kommunikatsiooni protokoll. 
IRC on peamiselt disainitud suhtluseks suuremates gruppides, kuid 
võimaldab ka üks-ühele suhtlust} ja võtsid BBSide funktsiooni üle. 

EENetis\index{EENet} toimus väga palju lahedaid asju, mõnes mõttes on ta nagu 
ebaproportsionaalselt nähtav organisatsioon. Mis on  väga hea, ma arvan, sest 
tegelikult nad aitasid ikka selle korraldada, et koolid said Interneti kätte ja 
see oli ülikõva.

\question{Lisaks sellele, et keegi kuskil poliitilise otsuse teeb, peab keegi 
suutma ja viitsima neid otsuseid ka reaalselt ellu viia. Sõita talveöös kuskile 
Põlva taha kooli modemeid installeerima ei ole palga eest tehtav asi}

Seesama \emph{case}, mille sa praegu lauale paned, võtab kokku kogu tolle aja. 
See arutelu üldse lauale ei  jõudnudki kunagi, et mis see koolide Internetti 
panek kõik maksab. Sellepärast, et väga paljudes kohtades raha  üldse polnudki. 
Arutati ainult ühte asja: \enquote{on vaja teha}, Marek\index[ppl]{Tiits, 
Marek} otsib, kust raha saab. Aga ega Marek ei kantinud seda raha kuskile 
kellelgi autodeks ega suvilateks. Marek tegi  europrojekti ja tuli. Põmm, kümme 
Suni. Põmm, kakskümmend Zyxelit. Ja lihtsalt tehti. Ja kutid tegid kogu aeg, 
neid ei olnud kümneid, neid oli umbes kaks või kolm, kes selle kõik ära tegid. 
Ja keegi nagu väga ei pahandanud. Aeg-ajalt käis mingisugune  Antsla kooli mees 
küsimas, et no kuidas läheb, no oktoobris tuleb või? Ja tuli ka. No vahel 
tuli novembris. Aga vahet ei olnud, keegi ei arutanud seda, et kas tehakse. 
Prooviti vaadata seda, et kui teenus laieneb,  kvaliteet ei kukuks. 

\question{Mis seda kõike edasi vedas siis?}

Minu arust selline iga agraar-ühiskonna tung harida maad, kus midagi ei kasva, 
mõistad? Ja tegelikult selle agraar-ühiskonna tung võtta kasutusele ressurss, 
või teha sellega midagi, mõnes mõttes kandus üle. Lihtsalt peab tegema, sest  on 
 mingi kool, kus veel ei ole. Mõnes mõttes oli ka lihtne, prioriteedi määras 
see, milline kool kõige rohkem ise nagu huvi tundis. Alguses ei tuntud üldse 
palju huvi. Nad ei saanud aru üldse, millest jutt.

\question{Selles mõttes väga õige lähenemine, et kõige suuremad hädalised, kes 
kõige rohkem selle Internetiga midagi teha oskasid, said selle ka esimesena 
kätte}

Ma ei tea, võib-olla mõnesse kohta jõudiski alles 2000. aastal Internet, aga 
vahet ei olnud, sest tegelikult selleks ajaks oli tegelikult see klõps juba 
ära käinud. Nii naljakas, kui see ka pole, üheksakümmend protsenti tööst oli 
veel tegemata, aga see kümme protsenti internetiühendusega koole oli kaalu juba 
nii alla vajutanud, et  ülejäänud osas oli lihtsalt aja küsimus, kunas nendeni 
juhe viiakse. Ja et seda kõike on vaja, oli Anne Villemsi\index[ppl]{Villems, 
Anne} ja tema pundi sügav veendumus. 

\question{Mis sa praegu teed?}

Võiks öelda, et see, meedia transformatsioon, mille on ära teinud umbes kümme 
aastat tagasi eraõiguslik meedia, püüan sellega ühele poole saada 
avalik-õiguslikus meedias.

\question{Kindlasti vääriline töö, kus väljakutseid, nii-öelda, jagub}

Kui avalik-õiguslik ringhääling\index{Eesti Rahvusringhääling} ise ei ole 
tunnetanud, et ta peaks selle  sellise
võrgu- või internetikasutaja keskse hüppe tegema, või  kaua aega ei tunnetanud, 
siis oli ka teistel institutsioonidel  raske seda tunnetust ju tema eest ära 
tunnetada. Selle tulemusena  tekkisid mitmed fundamentaalsed küsimused. Et 
kuidas te ütlete, et teil on sellise asja jaoks raha tarvis? Aga kus te siis 
olite, kui kõik ülejäänud teised organisatsioonid sellega tegelesid? A kust ma 
tean, kus nad olid, eks! Riigi jaoks on olnud see kõige keerulisem  üldse 
mõista, et kui avalik-õiguslik ringhääling niisuguse hüppe ette võtab, see on 
ikkagi ookeani ületamine ja parvega seda lihtsalt ei tee ära.

\question{Ja kui terve riik on läinud sinna teisele poole ookeani, siis on vähe 
sant, kui ERR teisele poole maha jääb}

Osaliselt on see mõistetav selle kaudu, et mõtle niimoodi. Sellel ajal, kui 
suurem osa audio-visuaalsest meediast toimus kinoringvaate vormis (oli 
filmilint, seda ilmutati ja siis kuskil mingi projektori abil näidati), oli  
televisioonil juba \emph{live}  signaali halduse kontseptsioon. Mis 
\emph{run}'is  iga päev aastal 1959 või 1960.  Kuuskümmend aastat tagasi oli 
olemas \emph{live} signaali halduse kontseptsioon, mis töötas ja mis oli 
piisavalt lollikindlaks  aetud, et sa said sellega Eestis eetris olla.  
Televisiooni tehnoloogia on läbi ajaloo arenenud oma kinniste protokollidega, 
oma signaalihaldusmudelitega. Ja Internetist oli ta väga kaua aega signaali 
loogikaga poolest  maas. See teleasi maksab muidugi ulmeliselt palju, aga 
ta oli kogu aeg nagu terviklik kinnine maailm, mis arenes mingit teist 
evolutsioonipuu haru mööda. Eelmisel või üle-eelmisel aastal jõudis Euroopa 
Ringhäälingute Liit koos teiste ülemaailmsete \emph{broadcasting} 
organisatsioonidega  oma standarditega nii kaugele, et on IP-põhine 
signaalihaldusstandard, mis ei ole veel valmis. Aga millest mingid tükid 
töötavad.  Aga see oli aasta 2017, kui nad sellega lagedale tulid koos 
korraliku \emph{roadmap}'iga. Vaata, kui kaua aega on olnud tegelikult 
normaalselt töötav Internet.

Praegu me saame öelda, et tegelikult ei ole mõttekas mitte-IP-põhist 
tehnoloogiat ehitada. Aga  terve tööstusharu läks teist rada pidi  kaugele 
edasi.

\question{Ehk sa ERRi vaatepunktist mitte ainult ei ületa parvega ookeani, vaid sul seal parvel on 
känguru ja hobune, keda sa üritad kuidagimoodi panna järglasi saama}

Seal juures on sul veel tugevad kogemused hundiga, kes puhub puust ja õlgedest 
maja ära. Järelikult on su parv igaks juhuks tehtud betoonist. 

Sul on nii äraspidised kogemused, et mingisugune puust paadi kontseptsioon 
tundub  algatuseks lihtsalt  ohtlik. Ja kui nüüd see panna nagu päris 
keelde, siis televisioonisignaali haldusloogika seisneb selles, et ehitame 
asja niimoodi, et see signaal ei saa katkeda. Palju maksab? Palju vaja! Teeme 
nii, et ei katke! IP-põhine paketihaldusloogika ütleb, et lükkame paketid läbi, 
parandame. Mitu paranduspaketti vaja on? Ta on paranduspõhine. Ja need on 
fundamentaalselt erinevad mudelid. Aga,  \emph{in the long run}, on parandamine 
odavam kui kohe hästi tegemine.
