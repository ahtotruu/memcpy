\index[ppl]{Lillenberg, Jaanus}
\question{Kuidas sina said arvutite juurde ja arvutid sinu juurde?}

See sai alguse aastal 1983, kui Tartu 
Ülikoolis\index{Tartu Ülikool} tehti Nõukogude Liidu ja Jaapani koostöö 
tulemusena personaalarvutite klass.

\question{Mis arvutid need olid?}

Need olid Yamaha MSXid\index{Yamaha MSX}, mõned vihasemad Sinclairid ja 
ka Apple II. Äge oli see, et nendes arvutites jooksis 
tegelikult Microsofti tava-kasutajatele mõeldud operatsioonisüsteem.

\question{Kas see oli Microsofti oma?}

MSX nagu Microsoft \emph{Extended} vist.\sidenote{Lühendi päritolu kohta liigub mitmeid variante, ka asja juures olnud inimesed ei mäleta enam täpselt.} Igatahes nägi see äge välja. Arvutiklass paiknes kooli peauksest kümne sammu kaugusel f
keldris, mille aken avanes täpselt kooliukse ette. Ühele üheteistaastasele, kes läks sellest igal hommikul ja õhtul mööda, 
oli see vastupandamatu. Selles mõttes valikuvõimalusi tegelikult 
ei jäetud. 

\question{Sa lihtsalt pidid sealt uksest sisse minema?}

Kõndisin ühel päeval otse aknast sisse, sest 
aken oli tänavaga samal tasapinnal. Küsisin, kas võib tulla vaatama, ja ära mind otseselt ei aetud. Kolmandal päeval 
andis keegi mulle MSX BASICu\index{BASIC!MSX BASIC} 
manuaali koopia. Ma küll ei saanud 
inglise keelest aru, aga mängude tegemine tundus huvitav. 
Arvutiklassis toimetasin kolm-neli aastat ja olin vahepeal ka abiõpetaja. Kirjutasin ise tekstiredaktoreid ja mänge ning loomulikult häkkisin 
lõputus koguses olemasolevaid mänge. Kirjutasin ka oma elu esimese viiruse, mis hävitas flopiketta. 

\question{Mis koolis sa käisid?}

Tartu 10. Keskkoolis\index{Tartu 10. Keskkool}, praegu on see
Tartu Mart Reiniku Kool\index{Mart Reiniku Gümnaasium}.
 Arvutiklass paiknes Vanemuise tänaval teatri vastas 
oleva õppehoone\index{Tartu Ülikool!Vanemuise tänava õppehoone} keldrikorrusel. 
Seal oli isegi kaks arvutiklassi. Teises olid 
Agatid\index{Agat}, mis olid venelaste pihta pandud 
Commodore'i või Apple II koopiad\sidenote{Agat kasutas küll sama 
6502 protsessorit, mis Commodore 64 ja Apple II, ning oli suuresti viimasest 
inspireeritud, kuid erines disaini poolest mõlemast ja otsese koopiaga tegu 
ei olnud.}. Kusjuures mul läks rohkem kui aasta, enne kui sain aru, et see oli tegelikult 
Apple II koopia. Klassis olid ka Apple 
II\index{Apple II} arvutid ja kuigi protsessori tasandil olid need sarnased, 
oli sisu väga erinev. 

\question{See nõukogude variant oli üsna industriaalse väljanägemisega.}

Jah. Kas sa tead näiteks, et kui inglise keeles on klaviatuuril vasakult paremale lugedes 
QWERTY, siis vene klaviatuuril tuleb sama moodi lugedes kokku \enquote{pidev \emph{lag}}? 

\question{Tol ajal ei teadnud veel keegi \emph{lag}'ist midagi.}

Kui sellest ajast kümmekond aastat edasi hüpata, siis olid Tartu Ülikoolis juba arvuti- ja 
terminaliklassid. Tol ajal olid arvutid nii võimsad, et neil oli 
hunnik terminale, mis moodustasid terminaliklassi. Siis oli juba ka
väga palju võrgutegevust. Arvutiklassi kõrval oli 
IBMi koopia või litsentsi alusel tehtud ES\index{ES 
EVM}\sidenote{ES EVM (\begin{russian}ЕС ЭВМ, единая система электронных 
вычислительных машин\end{russian}) oli sari IBM 
System/360\index{System/360} ja System/370\index{System/370} 
kloone. Nende riistvara põhines küll IBMi omal, kuid oli väheste eranditega 
siiski Nõukogude Liidus välja töötatud. Tarkvara seevastu oli IBMi tarkvara 
lokaliseeritud ja väheste muutustega koopia. Neid masinaid nimetati eesti 
keeles hellitavalt Jessukesteks.}, Nõukogude arvuti vene klaviatuuriga ja \enquote{pidev \emph{lag}} nii klaviatuuril kui arvutis oli väga ilmne kontseptsioon.

\question{Kõik üheteistaastased, kes arvutiklassist mööda kõndisid, ometi ei roninud 
aknast sisse. Sul pidi järelikult olema tehnika- või elektroonikahuvi.}

Ei olnud, ma käisin hoopis ratsutamistrennis. Aga mõni 
asi on kohe visuaalselt uus ja lahe ning vastandub 
kõigele muule ümbritsevale. Kujuta ette, et lähed mööda näiteks
lendamistrennist, kus inimest õpetatakse lendama. Sa ei hakka ju arutama, et ma pidin minema malet mängima või telekat vaatama. 
Lendamine on universaalselt väga \emph{cool} asi, kõigist 
teistest asjadest kümme korda kõvem.

\question{Ja siis ei oskagi pärast hästi seletada, miks sulle 
lendamistrenn meeldis ja miks sa ei läinud malet mängima.}

Lihtsalt kaldusid teelt kõrvale. Muide, ma ratsutasin neli aastat, see ei seganud.

Võtsin ühe klassivenna ka arvutiklassi kaasa. Mäletan, kuidas me arutasime omavahel, kuidas mänge tehakse. Kuidas assembler või 
masinkood näeb nii suvaline välja ja äkki on 
võimalik \emph{random} kombinatsioone katsetades saada 
lahedaid mänge. Mõtlesime küll, et see vist ikka ei ole tõsi, 
aga oleks äge, kui nii saaks! Katsetad kümmet tuhandet 
kombinatsiooni, kõikvõimalikke koodivariante ja vaatad, milline läheb käima 
ja milline mitte. Õnneks nädal hiljem olime juba \emph{Hello 
World}\sidenote{Siiamaani peetakse oluliseks, et uut programmeerimiskeelt katsetades luuakse esmalt programm, mis väljastab kuhugi teksti \enquote{Hello World}. Traditsioon pärineb kuulsast The C Programming Language raamatust\index{The C Programming Language}.} kirjutanud ja asjad läksid natuke selgemaks. MSX 
BASICust\index{BASIC!MSX BASIC} kasvas muide välja Visual 
Basic\index{Visual Basic}, nii et Visual Basicu õppimine oli meie jaoks 
\emph{what else is new}.

\question{Kes seda arvutiklassi vedas? Pidi ju olema keegi, kes sind aknast sisse 
lasi ja kohe välja ei visanud.}

Mind visati sealt mitu korda välja, aga nad olid oma väljaviskamises 
tunduvalt vähem veenvad kui mina sisseronimises. Ma ei osanud väljaviskamise peale
kuidagi solvuda või seda pahaks panna. 
Sain ju aru, et see klass ei ole minu jaoks tehtud. Näiteks ronisin 
mitu korda sisse õpetajate täiendkoolitusele, kus tegelikult ei õpetatud 
arvuti kasutamist, vaid seda, et maailm muutub ja et arvutiõpe on 
hea käegakatsutav asi seda muutust kirjeldama. Selle visiooni taga oli üks 
väga vinge inimene, Anne Villems\index[ppl]{Villems, Anne}.

Anne Villems on tohutult kirglik, tema kirg on maailma 
paremaks teha. Õpetame neid inimesi, kes õpetavad teisi! Näitame neile ja nemad
näitavad väikestele inimestele, milline maailm võiks olla! 
Arvan, et tema täienduskoolitustele jõudnud inimesed 
olid mõnes mõttes juba paremad. Nad suutsid endale 
sõnastada, et peaksid sinna minema, sest äkki maailm muutub sellest paremaks. Need, 
kes koolituse läbi tegid ja seal omavahel suhtlesid, 
olid üks suur rest kive selles vundamendis, mille peale meie IT-riik on 
ehitatud.

\question{Õpetajad õpetasid omakorda õpilasi, kellest said abiõpetajad, ja nii see teadmine levis.}

Bingo! Koolitusel oli näiteks
üks Tartu Kunstikooli\index{Tartu Kunstikool} õpetaja, kes hiljem tõi oma lapsed 
arvutiklassi tundi pidama. Kunstikooli õpilastel oli üks
joonistusprogramm -- 64 värvi, maa ja ilm. Ja nad tegid arvutiga päriselt mingeid asju, olgugi et printerit kahjuks ei olnud. 
Igatahes said nad tunnetuse, kuidas kontseptuaalselt täiesti uuel viisil kunsti teha. 

Isegi mina, kes ma mõtlesin primitiivselt, kuidas saaks mänge teha ja 
mängida, jõudsin lõpuks kuhugi välja. Aga nemad võtsid 
graafilise \emph{editor}'i ja tegid sellega võimsaid
asju\ldots Samasugune trikk nagu iPhone'i tulek: me ei teadnud algul, kui kõva asi see oli, aga kindlasti 
sajal ägedal moel. Omal ajal oli sama lugu personaalarvutite ja MSXidega.

\question{Iga uudsus läheb ju lõpuks üle, kas sinu jaoks arvutite puhul ei 
läinud üle?}

Ei läinud, see liikus baastasandilt järgmisele tasandile. Toon ühe näite kaks aastat hilisemast ajast -- Tõravere observatooriumi\index{Tõravere Observatoorium} 
astrofüüsikud, kes olid maailmaga hoopis teistsuguses kontaktis kui 
koolipoisid. Üks selline oli minu alumine naaber Enn 
Kasak\index[ppl]{Kasak, Enn}. Ühest küljest olid kontaktid 
teadusmaailmaga, aga teisest küljest maailm sulas ja oli 
võimalik bisnist teha. Nad tõid endale Amiga 
500\index{Amiga!Amiga 500}\sidenote{Tuntud ka kui A500, oli Amiga 500 
koduarvuti 1987. aastal Commodore'i poolt turule toodud professionaalse Amiga 
2000 vaste. Tegu oli populaarseima Amiga mudeliga, eriti Euroopas.}, mis olid 
järgmine põlvkond Commodore 64st. Sellel oli kümme 
korda võimsam protsessor ja hoopis teisest klassist graafika. 
Kui MSXi Z80 protsessor võimaldas kolmehäälset muusikat teha, siis Amiga 
suutis pakkuda kuutteist kanalit. Tänapäeva mõistes oli 1986. aastal võimalik täielik MIDI-lahendus kodus püsti panna. Oi, kuidas 
ma seal nende Amigadega muusikat tegin! Täiesti häbitult ja ööde kaupa.

\question{Kas sul muusikahuvi oli enne olemas või tekkis koos Amigadega?}

Igal inimesel on mingisugune arvamus, kas talle meeldib muusika või mitte. Osaliselt on see seotud sellega, 
kas pead viisi või ei pea. Kui mind ei võetud esimeses klassis lastekoori, siis sain aru, et mulle meeldib muusika, sest ma olin väga kurb. 

Kui ma Yamahadega tegelesin, siis see ei olnud ainult mängimine. 
Meil oli täiesti mitteametlik arvutiring: 
kutid vajusid iga päev pärast kooli kohale ja enne ära ei läinud, kui välja 
visati. Klassis tegutsesid tegelikult 
üliõpilased, näiteks Ain Sakk\index[ppl]{Sakk, Ain}, Alar Pandis\index[ppl]{Pandis, 
Alar} ja mõned teised kutid, kes jätkasid pärast ülikooli lõppu vist ka pedagoogidena. Nad olid lastesõbralikud ja toetasid meid. Meil oli võimalik seal käia sellepärast, et
arvuteid oli klassis viisteist tükki, aga 
täiendkoolitustel enamasti seitse kuni kümme inimest, nii et 
alati olid mõned arvutid vabad. Asi toimis põhimõttel \enquote{kes ees, see mees}. Kui arvuti said, siis enam seda ära ei 
andnud. Koolituse ajal muud võis teha, aga mängida mitte, nii et ootasime, hambad ristis, mingi \emph{manual} 
kõrval, mille järgi proovisime asju teha. 

MSX BASICuga\index{BASIC!MSX 
BASIC} sai samuti muusikat teha: noote ritta seada, 
rütmi kiiremaks ja aeglasemaks sättida, oktaavi muuta ning vist ka näiteks 
kolm erinevat meloodiat kokku panna. Ühetoonilist muusikat sai 
kindlasti teha -- kuulasin midagi ja proovisin 
järele teha. Amiga oli selle kõrval hoopis teine tera.
Erinevus oli sama suur, nagu panna endale papist 
tiivad külge ja mängida lennukit või minna päris lennuki peale. 
Ühel juhul paned teksti-\emph{editor}'is nooditähti paika, mängid selle ette ja kuulad. Teisel juhul on täisgraafiline muusika-\emph{editor} koos nootide ja digiklaveriga, mida saab arvutiklahvide peal mängida ja salvestada nii nagu tänapäeval. Pluss sadu pille, mille seast valida, mis 
kõlasid küll digipiiksudena, aga mis olid nii ära tuunitud, et viiul ja klaver kostsid kõrvale ikkagi erinevalt.

\question{Selleks et suuta kõrva järgi muusikat järele teha, peab kõrva olema. Kas sul oli muusikaline kuulmine olemas?}

Midagi oli jah. Ega noodid ju kõik õiged pruukinud olla, aga rõõm tegemisest oli suur! Iga kord, kui midagi 
natukenegi välja tuli, viskas see puid alla juurde ja leek läks suurema hooga põlema.

Tartus oli selline võimas 
organisatsioon nagu Tartu 
tähetorn\index{Tartu tähetorn}, aga infotehnoloogilise ajaloo prismas oli see ainult väike ripats Eesti 
Biokeskuse\index{Eesti Biokeskus}\sidenote{Eesti Biokeskus moodustati 1986. 
aastal Tartu Ülikooli\index{Tartu Ülikool} ja KBFI\index{KBFI} ühis\-asutusena.} 
küljes, mis oli tähetorni kõrval väikene kuut, aga kus toimusid 
ülisuured asjad. Tähetorni katusele oli hea panna \enquote{satipann}: sealt paistis kaugele, puid ümber ei olnud ja
signaal oli alati hea. Eesti kahest
esimesest internetiühendusest üks oli Tallinnas KBFIs\index{KBFI} ja teine, Tartu oma, paikneski 
tähetornis, õigemini biokeskuses, mille ruumid olid tähetorni 
lähedal. Biokeskuses tegutses Richard Villems\index[ppl]{Villems, 
Richard}\sidenote{Eesti Biokeskuse juht selle asutamisest 
alates.}, kes koos Lippmaadega üldse selle interneti-maailma Eestile avas.

Igatahes Amigad jõudsid tähetorni ja ühendati internetti, sest kõik, kes tee peal olid, istutasid ennast ka selle traadi peale, mis enne biokeskust katuselt 
alla tuli. 

\question{Kui veel ajas tagasi minna, siis sul pidi tublisti distsipliini olema -- koolituse ajal 
taganurgas istudes tuli ju vagusi olla.}

Ma mõtlesin välja sellise asja nagu võtmeluba: mulle anti klassi võti. Kuna nädalavahetustel koolitusi ei toimunud ja valvur ärkas kell seitse üles, 
siis selleks hetkeks sai ukse taha mindud. Ukse avas väga unine valvur, kes alguses ei uskunud, et mul on mingi võtmeluba, ja ajas mind minema. 
Aga kui olin juba kell seitse hommikul kohale läinud, siis ega ma sealt ära ei 
läinud. Palusin süüdimatult helistada pühapäeva hommikul mingitele inimestele, et need võtmeloa olemasolu kinnitaksid. Üksikud uued valvurid ei lasknud sisse, aga paari-kolme kuuga 
olid nad kõik välja õpetatud.

\question{Sest jama ei tekkinud, keegi ei läbustanud ja midagi ei 
varastatud.}

Läbustamiseks polnud aega. Ainukene jama oli vaba arvuti saamine: inimesed ootasid arvutiruumi ukse taga koolituse lõppu, et äkki järgmisel koolitusel on auk ja pääseb sisse. 

\question{Miks see luba just sulle anti? Kas paistsid kuidagi silma? 
Olid eriti tubli, korralik, pealetükkiv?}

Kõik see kokku. Samas tegin ma tänapäeva 
mõistes vabatahtlikku tööd. Tahtsin nii väga olla arvutite juures, et olin nõus tegema koolituste ajal abiõpetaja tööd, 
oma vabast ajast ja ilma rahata. 
Seal õpetati ülilihtsaid oskusi, mille laps omandab paari-kolme 
päevaga, nagu arvuti käimapanek ja mis tähendab \emph{press any key}. Mul ei olnud probleemi näidata tädidele, kuhu tuleb 
vajutada. Tädidel oli ka hea meel, et lapsed oskavad seda teha. Ja Anne Villems\index[ppl]{Villems, Anne} ei visanud ka mind välja eriti. 

\question{\enquote{Eriti}...}

Ma ei tea, kui palju oli sealpool seda, et nad ei 
jaksa enam võidelda ja ei ole mõtet välja visata. Meid oli vist kolm, kellel oli võtmeluba. 

\question{Seda on ikkagi vähe.}

Kõik ei jaksanud kogu aeg käia ega mahtunud ka. Eks visamad lõpuks jäid. 

\question{Kas sa käisid seal kuni keskkoolini?}

Jah. Keskkooli läksin teise kooli, 
Treffnerisse\index{Hugo Treffneri Gümnaasium}. Seal olid küll
oma arvutiklassid, aga siis oli juba oluline tähetorn\index{Tartu tähetorn}. Seal olid Amigad, seal lindistasime esimesed lood, mu naaber töötas seal, käisin astronoomiaringis
õppisin C-d\index{C} kirjutama. Seda õpetas mulle Kaur Virunurm\index[ppl]{Virunurm, Kaur}, 
ainuke tüüp, kes suutis, sest see, mida me seal tegime, on maailma kõige halvem 
õppimismeetod. Kujuta ette, et sinu kõrval on inimene, kes tahab mingit asja 
õudselt osata, aga ta ei oska mitte midagi ega viitsi \emph{manual}'i 
lugeda. Põhimõtteliselt sind muudetakse elavaks \emph{manual}'iks ja iga kahe-kolme minuti tagant küsib õpilane, et \emph{are we there yet}.

Tartus oli teine keskus veel, füüsikahoone\index{Tartu 
Ülikool!Füüsikahoone} Tähe tänava alguses. Seal toimetas Taavi 
Talvik\index[ppl]{Talvik, Taavi}, kes andis mulle ühe C 
\emph{manual}'i, mis oli vist fotoaparaadiga üles pildistatud.

\question{Kas see võis olla Richie kuulus sinise C-ga raamat\index{The C 
Programming Language}\phantomsection\label{sisu:richie}?}

Jah, aga see oli ainult must ja valge, sinist ei olnud seal midagi. Lehitsesin kümme-viisteist aastat hiljem neid 
üksikuid fotokoopiaid ja vaatasin, et päris hea kraam. 
Ega ma tollal väga palju asju C-s\index{C} ei kirjutanud, aga
hiljem küll. Igatahes Kaur\index[ppl]{Virunurm, Kaur} jaksas minu tohutut huvi taluda. 

Tol ajal olid ilmunud juba esimesed XTd ja tolle 
assembler\index{Assembler} oli hoopiski teisest klassist kui Z80 assembler -- kaheksa-, mitte neljakohaliste koodidega! 

Ühel hetkel lõi aga murdeiga sisse ja arvutid ei võtnud enam sada, vaid seitsekümmend protsenti ajast. 

\question{Tavaliselt tekib inimestel keskkooliajal mingisugune kultuuriline 
kontekst -- muusikat sa mainisid, aga raamatud? Veel midagi?}

Väga hea, et sa selle välja tõid. Me peame aru saama, millisele lavale 
idud kasvama läksid. Tartus oli akadeemiline keskkond ja see tähendas 
enamasti kõrget lugemust ning paremat kirjandusega kursis olekut.

Kasakul\index[ppl]{Kasak, Enn} oli tolle aja kohta täitsa hea raamatukogu, 
samuti mu tädil. Autoritest oli Asimov
kindlasti kõva sõna. 

Mu vanaema oli tõlkija, ta tõlkis kuuekümnendatel näiteks teose \enquote{I, 
robot} eesti keelde. Ta vist tõlkis kellegagi koos terve Asimovi 
kogumiku. Teine väga tugev liin oli sõrmused ja nende 
isandad. 
Võibolla mõtlen natuke üle, aga \enquote{Sõrmuste isand} on tegelikult lugu suurest ja palju tugevamast kurjusest, mille 
vastu ei saa. Mõtle, mis aastad need olid, 1987--1989! See lootus! Need 
raamatud sobisid hästi sellesse aega.

\question{\enquote{Kääbik} oli jah eesti keeles olemas, aga mina sain teada, et see on osa 
suuremast loost, alles üheksakümnendate lõpus. Kas sul olid ingliskeelsed 
raamatud?}

Jah. Need olid erilised, keskmise vene papitrüki kõrval läikivad ja ilusad. Kuidas tunda ära 
inimesi, kes olid tollal tegevad? Neil kõikidel on kodus nähtaval kohal kogu Tolkieni looming.

Ulme oli teine liin, näiteks Asimov ja Bradbury. Osa teoseid avaldati \enquote{Mirabilia} sarjas\sidenote[][-2.3cm]{\enquote{Mirabilia} oli 
kirjastuse Eesti Raamat aastatel 1973--2012 ilmunud raamatusari, mis 
keskendus peamiselt ulme- ja kriminaalromaanidele. Omal ajal oli tegu 
suurepärase võimalusega tutvuda üldjuhul väga hästi 
tõlgitud ulmekirjanduse klassikaga: Simaki, Lemi, Strugatskite, 
Asimovi, Bradbury ja paljude teiste romaanide ja lühilugude kogumikega. Paljuski 
kujundas just see sari terve põlvkonna ulmehuviliste maitse ja lääne klassikute 
hulgas ilmus ka Eesti, Soome ja Läti autorite loomingut.}. 

\question{Aga Strugatskid?}

Loomulikult nemad ka ja Stanisław Lem\sidenote{Stanis{\l}aw Herman Lem (1921--2006) oli Poola ulmekirjanik, kelle 
teosed olid ühekorraga nii filosoofilised kui ka satiirilised ja 
humoorikad.} ja teised. Oli selline ulmekirjanduse kogumik nagu \enquote{Lilled 
Algernonile}\sidenote{Ilmus aastal 1976 sarjas \enquote{Ajast aega}.}, see oli ka suhteliselt kohustuslik kirjandus. Kui 
tütarlastel oli võibolla kotis Herman Hesse \enquote{Stepihunt}, siis poisid raamatut 
kaasas ei kandnud, aga kuskil oli neil natuke äranäritud nurkadega \enquote{Lilled 
Algernonile}. USA ulmekirjanduse sissevool lõi toimuvale tõepoolest 
kultuurilise tagapõhja.

Muusikaga oli teistmoodi. Suured inimesed kuulasid suurte inimeste 
muusikat, noored noorte muusikat. Tähetornis\index{Tartu tähetorn} 
kuulati palju muusikat maki pealt. Seal olid koos naljakad 
kooslused: esiteks tähetorn ise, siis 
füüsikatudengid, kes pühendusid astronoomiasuunale (näiteks Kaur 
Virunurm\index[ppl]{Virunurm, Kaur}), ja teisalt tähetorni 
direktori lapsed, kes käisid Miina Härma 
Gümnaasiumis\index{Miina Härma Gümnaasium} ja vedasid sinna 
omi koolivendi ja -õdesid. 
Tartus oli tol ajal kaks kooli, kes defineerisid, mis on äge: 
Treffner\index{Hugo Treffneri Gümnaasium} ja Miina Härma ning mõlemad 
arvasid, et on parem kui teine. Tartu värk. Tähetornis olidki 
ka mõned Treffneri tüübid. Sedasi tekkis segu keskkoolist, 
ülikoolist ja internetist, millest ei saanudki tulla mitte midagi peale plahvatuse. Seal oli tüüpe, kes on tänapäeval Eestis kõik
väga asjalikud.

\question{Keskkoolinoorena astrofüüsikutega sammu pidada ja originaalkeeles 
Tolkieni lugeda ei ole lihtne. Sa pidid ikka nutikas 
inimene olema.}

Mul oli sõnaraamat kõrval, kuni ühel hetkel polnud seda
enam vaja. Treffneris\index{Hugo Treffneri Gümnaasium} 
oli gümnaasiumis bioloogia-keemia õppesuund, kus õpetati ladina keelt, 
tavalist inglise keelt, aga ka teaduslikku inglise keelt, mille 
kõrval tavaline inglise keel oli \emph{walk in the park}. Seda ainet andis 
muide bioloogiaõpetaja\sidenote{Õpetaja Tago Sarapuu\index[ppl]{Sarapuu, Tago} 
õpetas ka Meelis Roosile\index[ppl]{Roos, Meelis} bioloogiat.}, kes oli tugeva akadeemilise taustaga ja 
tegi hiljem pikalt akadeemilist karjääri. Sa 
mainisid sinise kaanega C-õpikut. Kujuta ette, et sul on näiteks geneetika 
kohta samasugune ning sa võtad ja närid ennast sellest lihtsalt läbi. Paberit lendab 
kahele poole, aga sa tõlgid selle kõik ära. See aitas hiljem väga hästi kaasa.

Meil oli saksa keel ka, ma sain saksa keele lõpueksamil kooli parima hinde. 
Aga miks? Sellepärast, et Kaur Virunurmel\index[ppl]{Virunurm, Kaur} oli samal 
ajal ülikoolis saksa keele eksam. Ta tõmbas sõnaraamatu arvutisse, tegi sellest baasi ja siis oli 
võimalik skoorida: õige vastus andis punkti, vale vastus miinuspunkti. Mina 
valmistusin saksa keele eksamiks niimoodi, et viimasel õhtu enne eksamit mängisin punktide peale saksa keele sõnade tõlkimist, ja 
see aitas mind väga hästi. See oli üks esimesi kordasid, kus 
võin kindlalt väita, et infotehnoloogiline tööriist parandas mu sooritust 
hüppeliselt. Lõpuklassis oli mul küll saksa keel vist ühel veerandil ka kaks, aga 
see oli ealine iseärasus, hinded ja teadmised ei ole alati 
omavahel lineaarses seoses.

Kui nüüd muusika juurde tagasi tulla, siis Miina Härma \index{Miina Härma Gümnaasium} kutid tõid tähetorni
The Smithsi, The Cure'i ja 
ka vene muusikat. Ilmusid välja kitarrid ja 
midagi ka lindistati, ilmselt kassettidele. 

Samas oli tähetornis nõukogudehõnguline teaduskultuur, mille juurde 
käis näiteks konjaki ja kohvi joomine. Keskkooli- ja 
üliõpilased ei saanud seda küll endale lubada, aga tubades oli see hõng üleval. Ma ei 
teagi, mis asjaoludel seal joodi, sest läbusid 
ei toimunud, aga see kõik tekitas erilise atmosfääri.

\question{Seal tehti ju teadust ja mitte halba.}

Just. Ma käisin isegi astronoomiaringis ja mind saadeti 
oma tööga kuskile rahvusvahelisele
õpilaskonverentsile esinema. Kõike sai teha.

\question{Kuhu sa pärast keskkooli õppima läksid?}

Tahtsin minna ülikooli ajakirjandust õppima. Ülikooli mitteametlik \emph{statement} oli see, 
et kui tahtsid tulla ajakirjandust õppima, siis pidi olema ette näidata portfoolio ehk
pidid olema midagi avaldanud. Ajakirjanduse või üldse meedia õpetamine on 
suhteliselt kallim tegevus kui näiteks keeleõpe ja nad tahtsid olla veendunud, et üliõpilane tõesti tahab ajakirjandust õppida, mitte ei astu 
juhuslikult sisse. Mulle tundus, et kultuuriajakirjanik on äge olla. Ilmselt selles vanuses 
arvab iga mees, et kultuuriajakirjanik on äge olla, sest on olemas oma arvamus maailmast, mille masstiražeerimine tundub 
veidral kombel teiste aitamisena.

Ühesõnaga, tegin ettevalmistusi: käisin 
kontsertidel, kirjutasin intervjuusid ja arvustusi. 
Aga samal ajal käisin ka näiteringis. Lõpuklassis olid veebruaris-märtsis 
Viljandis lavaka sisseastumiskatsete 
eelvoorud. Mõtlesin, et äge oleks minna Viljandisse trallima ja 
seiklema, aga sain eelvoorust edasi ja hiljem lavakasse sisse.

Ma ei pabistanud üldse ja eks see aitas. Teadsin, et mul on
ajakirjandusega plaanid ja head soovituskirjad 
toonastelt tuttavatelt Postimehe\index{Postimees} ajakirjanikelt. Õppisin 
lavakas ligi aasta, ent ühel hetkel sain aru, et see ei ole ikkagi see, mida ma teha tahan. 
Sellele otsusele jõudmist mõjutas kõik eelnevalt kirjeldatu 
väga tugevalt. See \enquote{lendamistrenn} paistis kogu aeg aknast. 

Kooli kõrvalt sattusin sellisesse ägedasse kohta nagu riigikogu 
kantselei\index{Riigikogu kantselei}.

\question{Kas sel ajal olid seal juba võrgud ja BBSid?}

Riigikogu kantseleis oli täiesti adekvaatne kraam juba aastal 1992. 
Ühtlasi üks äge tekstipõhise kasutajaliidesega võrgumäng, 
põhimõtteliselt tolle aja \enquote{Fortnite}, mille nimi oli 
\enquote{MUME}\sidenote{Üks populaarsemaid MUD-tüüpi mänge, mille nimi tuleneb fraasist 
\emph{Multi-Users in Middle-Earth} ja mis põhines 1991. aastal loodud ja siiani 
aktiivselt arendataval DikuMUDi\index{Muda!DikuMUD} mootoril. Vt ka märkust \ref{sidenote!muda} 
lk \pageref{sidenote!muda}.}, tegu oli Tolkieni ainetel loodud 
Mudaga\index{Muda}. Tuletan meelde, et need ringkonnad olid kõik 
tugevalt Tolkieni usku.

\question{Kas seal oli server, kuhu mängijad külge läksid?}

Jah. Telneti pordi kaudu tõmmati sind külge, kõik istusid oma 
\emph{socket}'is\sidenote{St omasid iseseisvat ühendust serveriga.}, aga nägid, mida teised teevad. Ja kuna see oli 
tekstipõhine, siis võrguühenduse kiirus ei olnud probleem.

\question{Kuidas adekvaatne kraam riigikogu kantseleisse\index{Riigikogu kantselei} 
sai? 1992. aastal ei olnud Eesti Vabariik veel kuigi 
heal järjel ja oli muudki, mida korrastada.}

Väga hea küsimus. Tarvi 
Martens\index[ppl]{Martens, Tarvi} kindlasti teab seda. Samuti
Toomas Mölder\index[ppl]{Mölder, Toomas}, kes oli nlibi, 
tollase rahvusraamatukogu\index{Rahvusraamatukogu} IT-juht tänapäeva mõistes. Ja eks KBFI\index{KBFI} rahvas aitas ka.

\question{Kas peale \enquote{MUME'i} mängimise tegite seal kasulikke asju ka?}

Ma käisin tol ajal lavakas ja teadsin küll, et nad teevad 
vingeid asju, aga tavaliselt siis, kui mina sinna saabusin, 
lõppes töö ära, sest tuli \emph{orc}'ideks kehastuda ja minna 
\emph{whiteskin}'e tapma. \enquote{MUME'i} tekitatav adrenaliinitase ei jäänud alla 
tänapäevaste arvutimängude omale. Näiteks olid seal haldjas ja sulle tuli 
teade: \enquote{\emph{An orc enters the room}.} Selle peale lükkab ka täna teatud 
seltskonnal vererõhu kakskümmend protsenti ülespoole. MUDe oli veel, aga 
\enquote{MUME} oli üks esimesi selliseid mänge, mis kestis aastaid. Esimesed 
eestlased, kes seal mängisid, tegid oma tegelased aastal 1991 või 1992 ja mängisid kolm-neli aastat järjest. Pronto\index[ppl]{Pronto} ehk Tanel Raja mängis muide väga kõvasti \enquote{MUME'i}. 

\question{Mäletan, et 1993. aastal sisenesid mingid inimesed Liivi tänaval 
VAXi klassi\index{Tartu Ülikool!Liivi õppehoone!Vase klass}\sidenote{\phantomsection\label{sidenote!vaks}Sõna \enquote{vask} mitmesuguste 
variatsioonidega kutsutud ja ilmselt klassi toitnud arvuti tüübinime 
VAX\index{VAX} järgi 
nime saanud klass asus matemaatikateaduskonna Liivi tänava 
õppehoone esimesel korrusel ja koosnes vask.ut.ee\index{vask.ut.ee} 
külge ühendatud terminalidest.} ja kui mina ükskord ülikooli lõpetasin, siis nad 
olid ikka veel seal.}

Jah, \enquote{pidev \emph{lag}} oli muide sealsamas VAXi klassi kõrval olevas 
ES-klassis, mis oli alati tühi, kuna need arvutid olid jamad\sidenote{Ilmselt 
peab Jaanus silmas Raua\index{raud.ut.ee} klassi, mis käis päris IBMi 
riistvara, mitte ESi peal. Raul Tölp\index[ppl]{Tölp, Raul} meenutab: \enquote{Sattusin kas 
1996. või 1997. aastal Liivi 2 hoonesse, kus mul paluti IBMi esindajana teha raud.ut.ee serverile masina viimane \emph{power off}. Räägiti, et masin küttis 
vesijahutusega tervet maja.}}. Liivi tänava 
VAXi klass on omaette peatükk, milleni kohe jõuame.\phantomsection\label{sisu:jaanus:vask}
%\hyphenation{esimese esi-mese}

Kui nüüd õpingute juurde tagasi tulla, siis lahkusin lavakast esi{\-}mese kursuse viimasel veerandil. Üheksateistaastane laseb end
välisest tugevalt mõjutada. Tollal erines
näitlejaamet sada protsenti sellest, mis see täna on. 
Oli nädalaid, kui jõin vähemalt pool pudelit viina päevas. 
Organism oli tugev, vedas ilusti välja, aga nägin kutte, kes olid 
seda kümme aastat teinud. Ühel hetkel küsisin endalt, kas ma jaksaksin 
ja tahaksin niimoodi elada. \emph{Hell no}! See ja \enquote{lendamistrenn} akna taga aitasid äratuleku otsust teha. 

Siis läksin Tartusse\index{Tartu Ülikool} eesti keelt, täpsemalt 
arvutuslingvistikat õppima.

\question{Kes seda õpetas? See oli tollal mujal maailmaski suhteliselt uus ala.}

Nüüd jään vastuse võlgu. Seal oli üks lahe lühike vanamees, kes 
oli tõeline guru. Hästi viisakas, vaikne ja rahulik sell, nii palju kui 
mina temaga suhtlesin. Aga tema juurde jõudsin alles kolmandal aastal pärast 
spetsialiseerumist. Enne olime lihtsalt ühes toredas teaduskonnas, kus 
põhiliselt õppisid tüdrukud.

Selle taustal oli mul ikkagi tunne, et peaksin ka mingit tööd tegema. Samas olin
ainult \enquote{lendamistrennis} käinud. Mängu tuli seesama Vase klass.

Mängisime seal \emph{StackMUD}i\index{Muda}. Stacken.kth.se\index{stacken.kth.se} oli Rootsi 
Kuningliku Tehnikaülikooli\index{Rootsi Kuninglik Tehnikaülikool} 
VAX\index{VAX}\sidenote{Virtual Address Extension -- arvutisari, mille töötas välja DEC
seitsmekümnendate keskel. Siiani üks tuntumaid omalaadseid arhitektuure,
oli see PDP-11\index{PDP-11} edasiarendus, peamiselt mälu virtuaalse
adresseerimise suunas.}, mille peal 
jooksis BSD\index{BSD}, mille peal pandi käima Muda. 
Originaalne DikuMUD on muidu tehtud Taanis.

Mängimise tegi huvitavaks see, et mängijad ei olnud matemaatika üliõpilased, 
nagu oleks võinud arvata, vaid eesti keele ja usuteaduse 
üliõpilased. Näiteks praegune kirjanik ja usuteaduskonna õppejõud Meelis 
Friedenthal\index[ppl]{Friedenthal, Meelis} oli väga originaalne. 
Oli teisigi tüüpe, kes käisid tõesti palju mängimas, mina 
sealhulgas. Suhtlus selle seltskonnaga ei piirdunud ainult 
mängimisega, me ka ehitasime seda maailma. Ma olin üks põhilisi ehitajaid.

\question{Kuidas see käis? Kas kirjutasid koodi või skripti?}

See oli väga lihtne: sain koodi koopia ja mul oli andmebaasi struktuur, kus 
täitsin väljad ära. Võtsin andmebaasi \emph{dump}'i 
teksti\emph{editor}'is lahti ja tegin näiteks ridadest sada 
kuni tuhat koopia ning kirjutasin sinna asjad teistmoodi. \emph{Editor} oli loomulikult vi\sidenote{Unixi spetsifikatsiooni osaks 
saanud, 1976. aastal kirjutatud tekstiredaktor, mis on siiani teatud 
ringkondades (ka käesolev tekst sünnib osalt vi abil) väga populaarne. Siin 
kontekstis on oluline, et erinevalt tänapäevastest tekstiredaktoritest ei 
olnud vi ainult tekstipõhine, vaid ka suhtles kasutajaga 
ähmaste käsu- ja klahvikombinatsioonide abil. Näiteks on 
legendaarsed algajate kasutajate tulutud katsed redaktorist väljuda, kuna 
selleks kasutatavad klahvikombinatsioonid \texttt{ZZ}, \texttt{:q!} ja veel kümmekond 
samalaadset ei ole just intuitiivsed.}. Kirjutamine käis tsoonide kaupa: ühes tsoonis oli 
sada ruumi ja iga ruumi kohta kirjutasin ingliskeelse kirjelduse. Mitmesuguseid asju sai 
ruumis olla vist kuni 255, kolle sai ka olla teatud
kogus. Täitsid kõigi ruumide kohta statistika ära, lisasin kirjeldused ja
tegevused ning postitasin. Kutid kompileerisid selle ära ja nii see tuli. 

Me saime Rootsi mängutegijatega üsna hästi läbi ning rääkisime 
vahel ka olmest ja inimlikest teemadest. Näiteks et meil on ainult üks klass ja 
seegi on kogu aeg pooltäis, ja kui mõni ahv FTPga tont teab mida tõmbab, ei 
saa üldse mängida. Nemad ütlesid, 
et neil on üks arvuti üle, kuna said uuema VAXi\index{VAX}, mille peal on 
BSD\index{BSD}. Ma küsisin, kas me vana arvutit kuidagi endale ei 
saaks, ja öeldi, et saate küll. 

\question{See masin oli ju Rootsis.} \phantomsection\label{sisu!jaanus_liivi_tn}

Jah. Anne Villems soovitas mul rääkida Otto Telleriga\index[ppl]{Teller, Otto}, kes oli vist 
arvutiteaduse õppetooli juht\sidenote{Tõenäoliselt toimetas Otto Teller siiski Tartu Ülikooli 
arvutuskeskuses\index{Tartu Ülikool!Arvutuskeskus}, mis oli 
eraldiseisev üksus.}, ja ütlesin, et ilmselt te mind ei usu, aga Rootsis on üks arvuti. Sellega tuleb kaasa 16 terminali, 
me saaksime teha terve klassi ja võin ise seda 
administreerida. (Kui muda sees mängid, siis oled kaelani porine, st 
süsteemide administreerimise oskus tekkis iseenesest.) 
Aga ma olen lihtsalt üliõpilane ega tööta siin, palun aidake. Lõpuks Otto Teller vastas, et see on küll kõik
väga imelik, aga olgu, ja ajas asja korda.

Siis kirjutasin igale poole kirju ja sain vastuseid. Ma ei tea, mida Otto 
Teller tegi, aga ta hüppas pea ees tundmatusse mingi kahekümneaastase 
kuti kätega vehkimise peale. 1993. sügisel sõitsime Rootsi klassi 
üle vaatama ja talvel oli korraga üks furgoon 
Laia tänava ukse taga. Tekkis Laia tänava arvutiklass\index{Tartu Ülikool!Laia tänava 
arvutiklass} ja mina sain selle administraatoriks. See oli minu 
esimene töökoht.

Tollal ei olnud administraator ainult tehniline töötaja, vaid ka 
administratiivne tegelane, kohalik jumal. Oleks mul võimuiha olnud, 
võinuksin seda väga hästi realiseerida, aga ma andsin hunnikule tüüpidele 
võtmed ja palusin, et nad serveriruumi ei läheks. Kuskilt 
veeti eraldi kaablid, Zyxeli\sidenote{1988. aastal 
Taiwanil asutatud Zyxel Communications Corporation tootis 
ülipopulaarseid ja hinnatud modemeid.}\index{Zyxel} modemid said üle püsiliini 
internetiühenduse ja seal me müttasime.

Olgem ausad, tänapäeva mõistes oli see klass totaalne õnnetus. Nii 
madala käideldavusega asja pole ma hiljem näinud. Arvuti oli väga 
vana, läks tihti katki ja ma ei tundnud nii professionaalset raudvara hästi. Mul oli abiks Viljo 
Soo\index[ppl]{Soo, Viljo}, kes oli tänapäeva mõistes \emph{sysadmin} ja kes 
aitas masinat palju kordi käima panna. Läks paar kuud ja 
saime klassi tööle. Klassi nimi oli Cure\index{cure.ut.ee}, 
The Cure'i järgi. 

Kunagi oli selline mäng nagu \enquote{NetHack}\index{NetHack}\phantomsection\label{sisu:nethack} ja sellest oli 
üks naljakas kloon tehtud. Otsustasin selle eesti keelde 
tõlkida. See käis põhimõtteliselt samamoodi: võtsin C koodi lahti ja 
hakkasin esimesest reast lugema.

\question{\enquote{NetHacki} lähtekood on niisamagi hea lugemine, see on üks kahest, 
mida ma olen oma elus lugemise eesmärgil välja trükkinud. Teine on Perl.}

Ühesõnaga ma tõlkisin kõik eesti keelde esimesest reast viimaseni, kaasa arvatud \emph{library}'d ja kõik muu, mis kaasas oli. Põhilise ekraani{\-}teadete osani jõudsin 
kell neli hommikul. Teadupärast tekib väga suure väsimuse korral ühel hetkel veider pooleufooriline meeleolu. Mul saabus see hetk keset tõlget ja mäng kukkus
naljakas välja.\sidenote{Mängus tembutanud \enquote{mõõkhambulisi varblasi} meenutatakse siiani hea sõnaga.} Seda mängiti klassis väga palju, seda enam, et 
võrguühendus alati ei töötanud, aga \enquote{NetHack} oli kohalik. Senikaua, 
kuni keegi Viljo Sood\index[ppl]{Soo, Viljo} otsis, et ta 
modemitele restardi teeks, mängiti \enquote{NetHacki}. Ma ise ei pidanud seda suureks saavutuseks, lihtsalt tegin valmis. Kahjuks läks kood koos Cure'i 
masinaga kaduma.

\question{Nutikal inimesel oli tollal tüüpiliselt kaks suunda, kuhu kiskus: 
kas akadeemilisse maailma teadust tegema või äri suunas. Kas sind ei tõmmanud 
kumbki?}

Mind äri ei tõmmanud, sest olen pärit äärmiselt vaesest perekonnast. Raha ei 
olnud midagi erilist, mul lihtsalt ei olnud seda kunagi. Kuude kaupa elasin saiast ja piimast. Ja kui raha ei ole, siis ei teki sellega ka lähedast 
suhet. Mis puutub akadeemilise maailma, siis olin sel ajal
alles esimesel kursusel.

Cure'i klassi tegemisest mõni kuu hiljem toimus üks tüvikursus. Pildile ilmus taas Anne Villems\index[ppl]{Villems, 
Anne}, kes korraldas 1994. aasta alguses Eesti esimesed \emph{webmaster}'ite kursused. 
Liivi tänaval olid kuulutused üleval.

Mina olin siis oma arust juba kõva käpp. 
Tol ajal kasutati Gopherit\index{Gopher}\sidenote{Gopher oli varajane hüperteksti{\-}protokoll, WWW 
protokollistiku eellane. Erinevalt suhteliselt lõdvalt struktureeritud veebist 
surus Gopher sisu küllalt rangesse hierarhiasse ja oli navigeeritav 
menüüsüsteemi abil.}, HTML 1.0 standardi eelkäijat, millest oli lihtne aru saada: oli klient, server ja \emph{markup language}, mille 
põhimõttest sai kümne minutiga aru.

Kursusel selgus, et arusaamisega läheb natuke 
rohkem kui kümme minutit. Kursusel käinud seltskond oli 
kirju, sinna sattus igasuguseid karvaseid ja sulelisi erinevatest 
teaduskondadest. Teiste hulgas oli seal näiteks Anto Veldre\index[ppl]{Veldre, Anto}, aga 
ma ei mäleta, kas õpetaja või õpilasena.

\question{Mis seal ikka nii väga vahet on.}

Tollal ei olnud jah vahet. Üks oli asja läbi lugenud ja rääkis teistele 
edasi. Aga Anne Villems\index[ppl]{Villems, Anne} oli kursuse väga hästi ette valmistanud. Päris kohe selle järel otseselt midagi suurt küll veel ei juhtunud, aga osalejate nimekiri jäi alles. 

Kui EENet\index{EENet} tegi endale veebilehte, olid nad kuulnud, et on olemas
\emph{webmaster}'id, kes oskavad veebi teha, tänu millele tekib organisatsioonis 
avaliku infohalduse funktsioon. (Nüüd oskan ma seda niimoodi nimetada, aga 
tollal panime lihtsalt asju internetti, näiteks 
võtsime Eesti kaardi ja sellele vajutades juhtus midagi.) EENet otsustas teha endale täiskohaga \emph{webmaster}'i ametikoha. Võimalik, et seni oli seda tööd teinud mõni ülimalt nutikas 
sekretär või tollal seal töötanud Marek 
Tiits\index[ppl]{Tiits, Marek}, aga 
asi lõppes sellega, et pool aastat pärast kursust kutsus EENet mind
\emph{webmaster}'iks. Palk oli kolm korda suurem kui arvutiklassis, nii et raske oli ei öelda.

Sealt edasi läks elu väga ägedaks. Saime tuttavaks Tarvi 
Martensi\index[ppl]{Martens, Tarvi} ja Toomas 
Mölderiga\index[ppl]{Mölder, Toomas}, tegime EENetile korraliku veebi ning käisime Eestit esindamas. Tollal oli veeb küll korralik, 
aga hiljem tehti veel ägedamaks, kui liitus näiteks Pille 
Pruulmann-Vengerfeldt\index[ppl]{Pruulmann-Vengerfeldt, Pille}, kes on praegu
Rootsis meediaprofessor\sidenote{Intervjuu toimumise ajal, 2019. aasta novembris, oli 
Pille Pruulmann-Vengerfeldt Malmö ülikoolis meedia ja kommunikatsiooni professor.} ja 
ERRi\index{Eesti Rahvusringhääling} nõukogu 
liige. Seal puutusin Marek Tiitsu\index[ppl]{Tiits, Marek} kaudu esimest 
korda kokku europrojektidega. Tollal küll veel ei olnud euro, vaid eküü\sidenote{Selle valuuta tähiseks oli ECU: 
\emph{European Currency Unit}.}. 

\question{Kas Marek oli see võlur, kes valdas unikaalset teadmist, 
kuidas fondidest raha saab?}

Just. Mina tulin lagedale veidrate ja üsna ebareaalsete
ideedega ja tema kasutas väikest osa neist hullustest, millel oli mingi \emph{point}, projektides ära.

\question{Tol ajal liikus EENeti ja IBSi\index{IBS|see{Institute of Baltic 
Studies}}\index{IBS} kaudu tohutult palju põnevaid ja kasulikke projekte.}

Seal jooksis igasuguseid naljakaid teenuseid, aga see polnud kõik, millega tegeleti. 
Näiteks suutis Marek hankida mulle tööarvutiks Silicon 
Graphicsi\index{Silicon Graphics}\sidenote{1990. aastal asutatud ja 
2009. aastal pankrotistunud Silicon Graphics oli peamiselt 3D-graafikale 
keskendunud riist- ja tarkvaratootja. Mitmed varased arvutiabi kasutanud 
filmid, näiteks 1993. aastal linastunud \enquote{Jurassic Park}, kasutasid just Silicon 
Graphicsi tööriistu. Ettevõttele tegi lõpu odavate laiatarbe x86-arvutite 
võimsuse kiire kasv.} masina. Silicon Graphics oli väga kõva asi, tänapäeval 
võibolla võrreldav Mac Proga. Ulmeline aparaat ja milline disain! 
Korpused olid värvilised, kõvaketas 
käis lahti kangiga. See oli nagu automaailma 
Bugatti või Porsche 911, täiesti \emph{over the 
edge}. Mõni ime, et ma selle töö hea meelega vastu võtsin, olles tulnud 
totaalselt vananenud VAXi klassi administraatori kohalt.

Kõige selle taga oli Richard Villemsi\index[ppl]{Villems, Richard} pikk, aga heledat ja positiivset 
värvi vari. Tema seda kõike püsti hoidis.

\question{Hoidis püsti, aga vist ka natuke nagu joonte sees, et inimesed päris 
hullustega ei tegeleks?}

Jaa. Richard Villems on muide Anne Villemsi\index[ppl]{Villems, Anne} abikaasa. 
Tänu Richardi suurele 
mõjule toimusid EENetis\index{EENet} väga kõvade projektide arutelud, 
mis ei käinud tegelikult üldse selle organisatsiooni põhikirjajärgse tegevuse alla. 
Sealsamas Liivi tänaval oli ka ülikool, arvutiteaduse 
instituut\index{Tartu Ülikool!Matemaatikateaduskond!Arvutiteaduse instituut}, samuti füüsikamaja. Nii et püssirohtu jagus ja 
tekkis igasuguseid initsiatiive.

\question{Kas selles maailmas BBSid ka kuidagi figureerisid?}

BBSid olid kogu aeg taustal ja mõnes olid mul isegi kasutajad. Aga BBSi tehnoloogiline \emph{carrier} oli modemiga üle
telefoniliini ühendumine serverisse. Mina sain varakult väga kiire 
interneti juurde, 64 kilobitti sekundis on tuntavalt kiire. 

Edasi tulid väga kiiresti IRCd\sidenote[][-7mm]{\emph{Internet Relay Chat} 
(IRC) on kliendi-serveri arhitektuuril põhinev tekstipõhise kommunikatsiooni protokoll. 
Peamiselt disainitud suhtluseks suuremates gruppides, kuid 
võimaldab ka üks ühele suhtlust.}, mis võtsid BBSide funktsiooni üle. 

EENetis\index{EENet} toimus palju lahedaid asju, mõnes mõttes oli see 
ebaproportsionaalselt nähtav organisatsioon. Näiteks aitasid
nad korraldada koolidesse internetti.

\question{Lisaks sellele, et keegi kuskil poliitilisi otsuseid teeb, peab keegi 
suutma ja viitsima neid otsuseid ka ellu viia. Sõita talveööl 
Põlva kanti kooli modemeid installeerima ei ole palga eest tehtav asi.}

Seesama \emph{case} võtab kokku kogu tolle aja mentaliteedi. 
Seda üldse ei arutatudki, kui palju koolide internetti 
ühendamine maksab, kuna paljudes kohtades raha polnudki. 
Arutati ainult ühte asja: tuleb teha ja Marek\index[ppl]{Tiits, 
Marek} otsib, kust raha saab. Marek ei kantinud raha autodeks ega suvilateks, Marek tegi europrojekti ja tuli. Põmm, kümme 
Suni. Põmm, kakskümmend Zyxelit. Paar-kolm kutti viisid asja ellu ja 
keegi ei pahandanud. Aeg-ajalt käis mõni Antsla kooli mees 
küsimas, kuidas läheb ja kas oktoobris tuleb. Ja tuligi, kuigi vahel 
novembris. Keegi ei arutanud, kas teha. Prooviti vaid vaadata, et teenuse laienedes kvaliteet ei kukuks. 

\question{Mis seda kõike edasi vedas?}

Küllap iga agraarühiskonna tung harida maad, kus midagi ei kasva. Mõnes mõttes oli see lihtne: prioriteedi määras 
see, milline kool kõige rohkem ise huvi tundis. Alguses ei olnudki huvi suur, sest ei saadud aru, millest üldse jutt käis.

\question{See on väga õige lähenemine, et kõige suuremad hädalised, kes 
kõige rohkem oskasid internetiga midagi teha, said selle ka esimesena 
kätte.}

Ma ei tea, võibolla mõnesse kohta jõudiski internet alles 2000. aastal, aga 
vahet ei olnud, sest selleks ajaks oli klõps juba 
ära käinud. Nii naljakas kui see ka pole, aga üheksakümmend protsenti tööst oli 
veel tegemata, kui kümme protsenti internetiühendusega koole oli kaalu juba 
nii alla vajutanud, et ülejäänute puhul oli üksnes aja küsimus, millal juhe nendeni 
viiakse. Aga et seda kõike oli vaja, oli Anne Villemsi\index[ppl]{Villems, 
Anne} ja tema pundi sügav veendumus. 

\question{Mida sa praegu teed?}

Püüan avalik-õiguslikus meedias saada ühele poole transformatsiooniga, mille eraõiguslik meedia on kümmekond aastat tagasi ära teinud.

\question{Kindlasti vääriline töö, kus väljakutseid jagub.}

Kui avalik-õiguslik ringhääling\index{Eesti Rahvusringhääling} ise kaua aega ei 
tunnetanud, et peaks internetikasutajakeskse hüppe tegema, 
siis oli ka teistel institutsioonidel raske seda nende eest ära 
tunnetada. Selle tagajärjel tekkisid mitmed fundamentaalsed küsimused: 
kuidas te ütlete, et teil on sellise asja jaoks raha tarvis? Aga kus te siis 
olite, kui teised organisatsioonid sellega tegelesid? Riigi jaoks on olnud keeruline 
mõista, et kui avalik-õiguslik ringhääling niisuguse hüppe ette võtab, on see
ikkagi ookeani ületamine ja parvega seda ära ei tee.

\question{Ja kui terve riik on läinud teisele poole ookeani, siis on pisut
sandisti, kui ERR teisele poole maha jääb.}

Sellal kui suurem osa audiovisuaalsest meediast toimus kinoringvaate vormis (oli 
filmilint, mis ilmutati ja mida projektori abil näidati), oli 
televisioonil kuuskümmend aastat tagasi juba \emph{live}-signaali halduse kontseptsioon, mis töötas ja oli 
piisavalt lollikindel, et sellega Eestis eetris olla. 
Televisiooni tehnoloogia on arenenud oma kinniste protokollide ja 
signaalihaldusmudelitega ning oli internetist kaua aega signaali 
loogika poolest maas. Teleasi maksab muidugi ulmeliselt palju, aga 
see on olnud kogu aeg terviklik kinnine maailm, mis arenes teist 
evolutsioonipuu haru mööda. Mõni aasta tagasi jõudsid Euroopa 
Ringhäälingute Liit ja teised üleilmsed ringhäälinguorganisatsioonid oma standarditega nii kaugele, et on olemas IP-põhine 
signaalihaldusstandard, mis ei ole veel valmis, aga millest mõned tükid 
töötavad. Aga nad tulid sellega lagedale aastal 2017. Mõtle, kui kaua aega on olnud 
normaalselt töötav internet.

Praegu saame öelda, et tegelikult ei ole mõttekas mitte-IP-põhist 
tehnoloogiat ehitada, aga tollal läks terve tööstusharu teist rada pidi kaugele 
edasi.

\question{Seega ERRi vaatepunktist mitte ainult ei ületata parvega ookeani, vaid parvel on 
känguru ja hobune, keda üritatakse panna kuidagi järglasi saama.}

Sealjuures on veel tugevad kogemused hundiga, kes puhub puust ja õlgedest 
maja ära. Järelikult on parv tehtud igaks juhuks betoonist. 

Meil on nii äraspidised kogemused, et puust paadi kontseptsioon 
tundub algatuseks lihtsalt ohtlik. Televisioonisignaali haldusloogika seisneb selles, et ehitame 
signaali nii, et see ei saaks katkeda. Kui palju see maksab? Nii palju, kui vaja! Teeme 
nii, et ei katke! IP-põhine paketihaldusloogika ütleb, et lükkame paketid läbi ja, kui vaja, parandame. Need on 
fundamentaalselt erinevad mudelid, aga pikas plaanis on parandamine 
odavam kui kohe hästi tegemine.
