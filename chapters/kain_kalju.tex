\index[ppl]{Kalju, Kain}
\question{Kuidas sina arvutite juurde jõudsid?}
Tere. Tervist, tervist. Kuidas sina üldse arvutite juurde jõudsid?
See oli umbes aastal üheksakümmend, üheksakümmend üks, kui tekkisid esimesed arvutid mu sõpradele. Ühele sõbrale tekkis selline imelik asi nagu Texas Instruments TI üheksakümmend üheksa, see oli lootus. Ei, see ei ole kalkulaator, see on Komodore. Ja Apple kahe sarnane riistapuu selles mõttes, et butis ta oli kuueteist bitise protsessoriga butis otsa Peisikusse. Et ta oli ühesõnaga selle ajastu pill, ehk siis seitsekümmend üheksa, kaheksakümmend kolm aasta pil.
See oli siis arvestades oli.
Aga see oli, kus kuskil üheksakümmend ma täpselt ei mäleta, võib-olla oli kaheksakümmend üheksa, kui ma arvan, et me olime kuskil selline kaksteist kuni neliteist vanad, see oli telekaga ühendatud ja siis seal olid mingisugused primitiivsed mängud stiilis pressingu Anders ja muud sarnased. Ja siis loomulikult ka siis Peisik, kogu programmi kood tuli kasseti lindilt, nii nagu tollel ajal kombeks, mingeid floppisid polnud olemas. Ja see oli minu esimene siis nii-öelda kokkupuude sellise arvutiga, millel oli klaar, suur, kuhu sai sisestada programmi koodi, kus me siis katsetasime ka esimest korda ise programme teha, Peisikus toksides neid ajakirjadest mõeldes ka ise välja. Mis linnas, aga mina olen Keilast pärit.
Jah, selles mõttes, et mul ei ole nagu kunagi olnud mingisugust sellist spetsiaalset ligipääsu kuhugi vaid teadusasutustele, koolidele ja nii edasi, ehk et minu ligipääs on selles mõttes suhteliselt nagu piiratud võib-olla mõnede teiste
Sama siin ja näiteks siin Asko Seebaga, kellega sai räägitud ka tema on Viljandi lähedalt. Et ega temal veel niipalju kui tal koolis oli tal, mida siis arvutit ligipääs. Aga kas sul seejuures üldse mingit nagu reaalainete huvi ka, nagu taustaks oli või?
Koolis ma käisin reaalkallakuga klassis, et meil oli väga-väga vahva nii-öelda lend gümnaasiumis, meil praktiliselt kõik poisid olid mingisuguse arvuti huviga ja nii palju, kui ma neid nende nii-öelda elukäiku jälginud olen, on praktiliselt kõik on selles arvutimaailmas miskit.
Pidi kui seegi oli, kas teil koolis oli siis niisugune korraliku õppe või?
Selles mõttes ongi väga huvitav, et gümnaasiumi esimestes klassides mõtlen siis tollel ajal me just olime läinud kaheteistkümne klassisüsteemil, ülem oli, oli üks, kümme, üksteist klass, siis meil olid kooli arvutiklassis jäid jukud loomulikult absoluutselt meid see ei huvitanud, seal oli Pascal selles mõttes, et meil oli juba ligipääs PC-dele tollel hetkel selles, kas Juku, Juku nagu ei julgenud
Mesilase põlved ja see oli nii äge.
Ja aga see tuli nagu selles mõttes nagu hilisemas faasis, et pärast seda, kui oli juba pissi ligipääs olemas ja kui mulle endale oli ka kodus juba pisi et et jah, ma arvan seda, et minu nii öelda kõige suurema arvutihuvi läkski sellest hetkest lahti, kui vanemad otsustasid mulle pissi osta, aga noh, natuke seda peab natuke lugu tagasi kerima selles mõttes, et seesama sõber, kellel oli siis see Texas Instruments imepill sai aastaid hiljem kaks, kaheksasaja kuue monokroomekraaniga, tal isa käis Ameerikas ja siis tõi sealt see naljakas oli veel see, mis näitab seda ajastut, nad elasid esimesel korrusel kortermajas ja siis piis siiani raudkapis, mis, kes kinni selles mõttes, et oli nii suur hirm, et keegi murrab sisse, varastab selle ära.
Ehk siis rohkem kui korter tol ajal väga võimalik, käsuker ehk raha, mis sinna alla pidi minema. No mõni imete nagu kaprundi.
Ja siis mu vanematele, kes see kohutavalt pinnale, et ma üldse ei viibi kodus, olen kogu aeg sõbra juures külas hilisööni välja ja, ja siis millalgi üheksakümnendatel vahetult enne Eesti krooni tulekut, et see oli siis see aeg, kui rubla devalveerub hästi kiiresti. Mu isa käest olen küsinud, kuidas see täpselt oli, ta meenutas, et tollel hetkel tema sai palka juba Ameerika dollarites millegipärast ja siis mingisugusest kooperatiivist või, või mis iganes tollel hetkel. Äriühingud olid, siis sai ostetud üks, kaks, kaheksa, kuus dollarite eest ja hinnaklassile umbes tuhat dollarit. See oli siis VGA ekraaniga ja, ja nii edasi ja selles mõttes täis mad, täiesti uus, et nad väga äge, aga kui kaks, kaheksa, kuus, ilmselt oli tolleks ajaks juba nii-öelda auteid, et natukene, sest olid juba kolm, kaheksa kuu ajastu.
Aga ikkagi, võrreldes nende xD-tega, mille abil Tartu Ülikoolis programmeerimist õpetati, oli see ikkagi nagu väga kõva sõna ja absoluutselt amise digital arvutilises.
Nagu noor poiss ikka tõenäoliselt mängisin, mind huvitasid kõikvõimalikud tarkvarad siis üks huvitav seik on veel, et me käisime sama sõbraga üheksakümne kolmandal aastal Ameerikas, sellist umbes aasta pärast seda, kui mul oli arvuti. Kusjuures see Ameerika, see minek oli ka väga kummaline selles mõttes, et me mäletan seda, et meil oli kolm kolmesaja kolmene punt oli, kes me elasime üksteise lähedal ja siis kõikidele kodus arvutit tollel ajal juba selles mõttes kas siis vanemate tööarvutit või siis nii-öelda isiklikud. Ja siis me sõitsime rongiga Tallinnast Keilasse ja millegipärast ma rongis hakkasime rääkima, et kuule, jube lahe oleks minna Ameerikasse. Ühel sõbral on tädi Ameerikas, et ta võtaks hea meelega vastu, aga kuidas me sinna saaksime siis minu isa töötas tollel hetkel Muuga sadamas. Ja siis kuidagi sai räägitud, aga ka põhimõtteliselt saaks ka laevaga minna. Noh, täiesti jah, mina ei tea, noored poisid, me olime kuskil kuusteist, viisteist, kuusteist aastat vanad ja ma kodus rääkisin selles sellest ja kuidagi hakkas see pall veerema niimoodi, et üks hetk, kui olime USA saatkonnas viisat taotlemas, järgmisel hetkel isal oli juba kokku lepitud, et me saame minna kaasreisijateks Ameerika suure kaabuga suure laevaga ja me sõitsime üle Atlandi ookeani laevaga Muuga sadamas New Orleansi, kus, kus siis laevakompanii pani meid lennuki peale ja sealt edasi lendasime JFK lennuväljal New Yorki, kus siis sõbra tädi meid vastu võttis. Kusjuures me saime veel laeva peal palka sellepärast et see oli laevafirmale palju odavam vormistada meid töötajateks, sest muidu oleks olnud tal suur kindlustus. Nii et selles mõttes täiesti täiesti kreisiarst ostsin esimest korda kaaludes võitsid kaks nädalat umbes neliteist, viisteist päeva laevasõit üle üle ookeani jaoks. Muidugi tegid siis kasulik, ei midagi kasulikku, tegelikult me veel me olime, hängisime nii-öelda ohvitserid nii-öelda siis piirkonnas. Noh, meile küll näidati, kuidas kõik laev töötab, aga me ei teinud selles mõttes midagi kasulikku tööd, et ma oleks meedia koristanud, teki või midagi sellist, Eime lihtsalt hängisime. Võib-olla heal juhul saime mingisugust sellist nii öelda ülevaatlik õpet, umbes nagu sa muuseumis käid, et näed, siin on see asi siin mootoriruumis on sellised nupud ja loomulikult keegi meil midagi teha ja lasknud välja arvatud see, et võib-olla avaookeanil me saime rooli keerata ja natukene nii öelda laeva juhtida millega taas tagasi me tulime lennukiga juba. Aga selles mõttes, miks ma seda üldse märgina on see, et kui Ameerika pinnal astusime Saimes laevast palka, meil oli päris palju raha, meil oli stiilis mingisugune tuhat viissada dollarit, mis oli tolle aja kohta nii-öelda üüratu summa. Ja siis mina isiklikult kulutasin selle raha loomulikult ära arvutipoes. Nii et ma tõin endale Ameerikast, ma arvan, et elu ühe kõige tähtsama riistapuu, milleks sellid, modem ja vot pärast seda läheks elu lahti. See oli mingisugune kahe tuhande neljasaja poodine modem, ma täpselt ei mäleta tüüpi enam. Ja, ja, ja siis ma teen veel soomlastel kuusteist helikaardi, mis oli täiesti
Tipp-tipp tulenenud ikka ähvardada ja siis tuli too aeg täpselt, kus on pastor, oli välja tulnud.
Just õmmigi stiilis ja paar kuud hiljem nii-öelda, et Tallinna kui täiesti täiesti tipp, et sest, sest üks asi, mis siis, mida on pärast hiljem avastasin, mis Peebeeessides levisid, olid need nii-öelda helimoodulid, et mul oli need hästi, palju ma mingil hetkel kogusin neid, ma arvan, et tollel ajastul paljud teised need kuulajale need on siis sellised helifailid, mida tollel ajal Amiiga arvutites kokku pandi, koosnesid Sämblitest põhimõtteliselt siis nagu mingisugune kaheksa treki, kuhu sa siis miksid, sämpleid niimoodi kokku, et sellest tekib mingisugune miini ful musiga vihikusse? Ei Midi on nagu rohkem selline klaverijuhtimisega ta oli, siis ma isegi ei oska täpselt öelda põhimõtteliselt, et see põhimõtteliselt nagu sul oleks tänapäeval mingisuguses video redaktoris helineva erinevad helid, rätid, kus sa siis need helid, Räki kassatsioon, limiteeritult helid, Räkid ja siis sa üritad selle nende puu, nende neid kasutades siis sellist sellist mõttekat muusikat kokku panna.
Ma usun, et ka kus need liiklus PPS?
Need liikusid Spebessides ja loomulikult see sai üritatud ise ka neid teha, aga mulle erilist muusikalist tausta ei ole, nii et selles mõttes midagi, midagi, midagi sealt selles mõttes välja.
Tulin Ameerikast tagasi, panid kohe püssi püsti.
Tulin Ameerikast tagasi, siis hakkasin avastama endale enda jaoks siis seda Peebeeessi maailmad ja.
Ma.
Nii-öelda nii-öelda vanu vanu asju üle vaadates selgus, et mul üks lemmikPeebeeesselli tarkooner, mis siis oli Priit Kasesalu veetud ja S S esmalt loomulikult sa üritad nii-öelda alla laadida kõike, mida sa selles mõttes kõik on ju puhas kuld, selles mõttes kõik tarkvara, mida sul pole veel kunagi olnud ja nii edasi. Siis huvitav oli veel see, et tollel ajal oli Ale eksisteeris selline asi nagu Kadaka turg ja Kadaka turul müüdi piraattarkvara. Ma arvan, et ma sain ka väga palju sealt, piraat, ärka Peeveeessides, kusjuures minu mäletamist mööda tegelikult otseselt piraattarkvara nii-öelda väljas ei olnud, et seal oli rohkem sellist nii-öelda häkkimise stiilis tarkvara, aga mitte nii otseselt.
Sa vindumised vist keegi ei taandu.
Jah, jah, just, et see sellist sellist asja nagu need olid nende taha nurkadesse ära peidetud neid nagu otse nii-öelda selles Peebeeessilistide see on. Aga seda mäletan küll, et mul oli kodus telefoniliin ja see tele ja minu meelest ei olnud minutitasu, tollel hetkel võisin minutitasu, oli nii odav, igaljuhul mul oli kodus liin praktiliselt ööpäevaringselt kinni kogu aeg selline võimalik helistada, sest arvuti helistas kogu aeg laadis midagi alla ja, ja nii edasi.
Muutsid räppi, kui ta käis, et kus sa teada said, et mis numbrite peale helistele ja kustkohast keegi vastu võtab.
Väga võimalik, et see tuli stiilis, eks ajakirjast ma ei, ma ei suuda seda enam meenutada, aga kui sa oled ühte PPS juba sisse pääsenud, siis see kogu see maailm juba on üks, üks üks teema, mis PPS levitas, on teiste Peebeeessid siis aadressidega failide nimekirjad, et mul on mingil hetkel. Priit Kasesalu pani kogu oma Peebeeessi viimase videoversiooni veebi üles ja ma laadisin selle alla, ma avastasin selle enda ketta pealt, vaatasin seda just hiljuti läbi, et see oli nagu päris huvitav, mis seal siis, missugused õidus. Kõikvõimalikke häkkimisvahendid, C-programmide näiteid, mingisuguseid raamatuid stiilis terrorist, säng, Pukk ja muud sarnased. Et igasugune, selline selline kraam, mis sellistele noortele inimestele põnevust pakkus.
Kui me korra veel sinuga selle arvutihuvi alguse juurde lähme, kas sa pigem olid seda tüüpi mees, kes nagu nagu mängis arvutiga või võrgutas arvutiga või programmeerida arvutiga, kas teil on ka erinevaid tüüpe või programmeeritud selleks, et mängida, või kuidas see oli?
Tema jäägu, mõtlen, ütlesin ka selle üle nii-öelda tagantjärgi, et kuidas see siis täpselt tellija mul tundub, et mul on olnud sellised nii-öelda nagu kolm kolm sellist ajajärku, et kuis oli see nii-öelda koduse kaks, kaheksa, kuue ja Peebeeesside ajajärk, siis see oli pigem selline, et et see lihtsalt üritad endale nii öelda sisse krahmata, kõike, mida sa näed, selles mõttes seal leidus ka arvutimänge, ma ei mäleta, et ma oleksin väga nii-öelda kohutad mängima, siis kui me sõbra juures mulle endale veel arvutit ei olnud, siis me mängisime loomulikult, et kõik need ajad täis, et me ei tegelenud programmeerimisega, vaid pigem ikkagi mängimisega. Aga hiljem jäi see mängimine pigem nagunii-öelda rohkem nii-öelda taha taustale ja ikkagi üritasid nii-öelda aru saada, kuidas arvuti töötab. Näiteks üks teema, mis mind kohutavalt paelus, olid viirused. Et mul on alati kõige viimane viirusetõrje tarkvara, siis ma usun, et mul olid selleks hetkeks juba ka mitu kõvaketast, ehk et mul oli võimalus nii-öelda katsetada, mida viirused teevad ka neid nii-öelda viiruse kollektsioone levitati Peeveeessides. Ja siis sai sellegi uuritud, et kui kuidas teine asi nii-öelda põhimõtteliselt nii-öelda töötav. Ja siis järgmine ojasse tuli siis, kui ma avastasin Linuxi enda jaoks. Ja see oli ka siis sama, kui tuli internet. Ehk et, et sealtsamast gümnaasiumi kõrvalt kaheteistkümnest klassist ma sattusin tööle riigi elekterside inspektsiooni, mis on täna. Mis ta nüüd ongi, Tehnilise Järelevalve Amet? Et ma sattusin nii-öelda siis selleks noh, jälle selliseks patsiga või arvutipoisiks mul patsi pole kunagi olnud, aga siis selliseks nagu nagu ikka selline arvutipoiss, kes kellele antakse mingisugused arvutid, palun seadista nüüd siin tee seda, seda, seda, kuidas sa sinna sattusid. Ühe külje sattuma. Tutvused või mul sõber, seesama sõber, töötas Pennus ja kuidagi tema kaudu tuli kontakte, et kuuled, et otsitakse sellist arvutitüüpi, kes oskab arvutitega midagi teha ja ma läksin kohale ja ja kuidagi võeti tööle poole kohaga.
Et siis teil klassist ikka mitmed töötasid siis keskkooliga?
Ja, ja meil selles mõttes mitmed töötasid jah, et, et üks üks klassivend näiteks töötas keele linnavalitsuse juures, kes oli selline kõva programmeerija juba tollel ajal, kes kinkis mulle esimese programmeerimisraamatus, Sii Programm, längits. Plain Kööninghami ja Dennis Ritchie.
See on seesama väljaanne, mis minul oli ja me vaatasime seda raamatut ja siin ei ole seda, kust on trükitud seal mingi piraatväljaanne. Just, see on mingisugune müstiline, see levis Nõukogude ajal juba minu meelest. Et kust selle, selle raamatu nagu ajaloost või kes see selline nagu välja andis ja korralikult tehtud nagu köidetud ja puha ja see ei ole näinud. Aga kujundus on täpselt see klassikaline sinine.
Just justkui praegu minna Amad seest vaatama, siis täpselt selline raamat on põhimõtteliselt müügil. Väga põnev, et see oli mu esimene programmeerimisraamat, aga see, see, see leidis nii-öelda kasutamist ikkagi aastaid hiljem, siis kui ma juba netid tegin ja mul oli nii-öelda praktiline vajadus programmeerida otsingusüsteemi mis oleks suurema jõudlusega.
Et tahaks ikkagi aru saada, et kuidas teil juhtus, see, et teil oli selline klass, kus nagu mitmed juba keskkooli ajal nagu töötasid ja programmeerimine käis, arvutid olid kodus ja kuidas see võib-olla sellist noh, mis see nagu ajas seda tagada.
Aga võib-olla see oligi see aeg, kus need arvutid nii-öelda ilmusidki rohkem nii-öelda kodudesse ja igale poole ja, ja ja oli tohutu puudus sellisest nii öelda oskusteabest ja vanemad inimesed võib-olla julenud neid veel kasutada ja, ja noored julgeksid nendega igasuguseid asju.
Sest igas keskkonnaklassis ei olnud see asi niimoodi, et kolm-neli-viis poissi töötasid nagu arvutispetsialistidel. Miks teie juures oli?
Ma ei oska seda tagantjärgi öelda küll aga mäletan sellist huvitavat seika, et meil oli üks nii öelda eksam oli põhimõtteliste arvutieksam ja see seisnes meil, see seisnenud meie puhul programmeerimises, see seisnes see meie puhul see, et, et me sisustasime arvutiklassi ehk et kool sai kuskilt mingisuguse Tiigrihüppe või ma ei tea mis iganes programmi kaudu peaaegu klasside ja arvuteid, ma arvan, et mingi kümme, kaksteist arvutit ja siis B-klassi poiste siis nii-öelda ülesanne oli võrgutada see klass füüsiliselt siis etherneti kaabliga installeerida need arvutid installeerida võrguserver, milleks oli siis Linuxi server, seda raske minu peale, kuna ma olin tollel hetkel kõige suurem linuxi käpp võrreldes siis teiste poistega ja, ja meie nii-öelda siis kaheteistkümnenda klassi arvutieksam seisnes selles, et nad põhimõtteliselt nii-öelda seadistasime koolile esimese PC klassi.
Seal oli siis mingisugune, see oli üheksakümmend viiskümmend viis ja see linux jõudnud selleks ajaks ju kuigi vana ei olnud selle otsa komistades.
Linuxi otsa ma komistasin niimoodi, et kui ma juba riigi elektrilisi diskussioonis töötasin, mis on ka selles mõttes huvitav, et kui ma sinna läksin, siis seal veel Internetti olnud, aga see tekkis sinna üsna pea, selles mõttes ma arvan, et mingisugune kuu-paar hiljem. Ja see oli siis üheksakümne neljanda lõpp mest See rektises institutsioon asus aadressil Ädala neli tee, mis on ka siis selline legendaarne Internetihoone ehk et meie allkorrusel oli valitsusside, kus toimetas Taavi Talvik ja siis Taavi andis meile siis riigi elektersideinspektsioonile nii-öelda juhtme otsa kätte, milleks oli siis tolleaegne kämne megabitine koaksiaalkaabel ja palun, siin on Internet ja siis sisse saja, see koaksiaalkaabel sai siis nii-öelda veetud kõikidesse ruumidesse. Ja, ja siis.
Et Hubiti huvid ju asjade täht, mis täht topoloogia
Just tol ta siis ma avastasin enda jaoks internetikoolis koolis loomulikult poiste rääkisin, kulges effida, need on nüüd mingi mingi vana Jawa, et seal mingi aeglane toimub üle modevik, et siin on üks palju, uurib uuem ja huvitavam asi, mis on internet, kusjuures siis valitsus sidet edasi olid kanaleid üsna kiired, ehk et tollel ajal oli juba. Ma mäletan, et, et Tartu Ülikooli FTP serverist sai kahe megabitise kiirusega faile alla laadida, see oli nagu meeletu viirus. Välislink oli loomulikult kuskil kuuskümmend neli või sada kakskümmend kaheksa, kilobitti, aga, aga.
Just tõmmata oli siis nii väga.
Vot seda ma isegi, seda ma täpselt ei mäleta, aga ju seal midagi oli, sest mul on nagu väga selgelt meeles nii-öelda kaadri uut ees olnud FTP server ja, ja, ja, ja nii edasi, et et aga see valitsusside ja see Ethernetikaabel, see oli nagu huvitav, et nagu tollel ajal mõnedki teised saates külas käinud on rääkinud, kuulasin, saate sinu saated läbi. Et arvuti turvalisus ei olnud nagu tollel hetkel eriti nagu teemaks. Aga mina noore poisina, kes ma siis olin just linuxi taastamas ja siis fit filonet oli selles mõttes ka tohutu kullaauk, et kida need taas loomulikult kõik oma nii-öelda need Eho kanalid. Aga siis internet avas meelingvistid ja kusagilt meelingvistist. Ma lugesin, et Anto Veldre teeb neljakümne kolmandast keskkoolis mingisuguseid selliseid interTakson kursuseid ja tollel ajal ilmus ka siis ajakiri ekse, kus Anto artikleid kirjutas. No mäleta, kumb, kummale eelnes, täpselt, aga igal juhul mäletan seda, et üks hetk, ma olin, olin ma seal neljakümne kolmandast keskkoolis, ütlesin, et siin ma olen. Nüüd ma tahan teadmisi saada, seal sa oled kaabel, velled tol ajal sellised legendaarsed koolipoisid nagu Indrek Mandre ja, ja Eno Ivanov vist ja nii edasi. Ja mäletan seda, et tagasi tulin ma sealt juba läkweri distributsiooni installeerimis floppidega, mida oli stiilis kuus tükki. Et installeerimise protsess käis ikkagi niimoodi, et esimene flopi, teine flopi kolmas-neljas ja nii edasi lõpuks sai installeeritud. Ja, ja siis.
Lisaks kõigele muule siis Anto peale jääb ka siis Eestis mingi Linux'i pisiku süstimine või.
Ma usun küll, jah, ma arvan, et temal, temal on väga suur roll selles osas linnukese Eestis käima läks igal juhul mina selle pisiku sealt sain. Kuna ma olin tollel hetkel juba siis mõnda aega elekter siis inspektsioonis töötanud ja ka palka saanud, mul oli päris korralik nii-öelda siis taskuraha, kui nii võtta, siis ma ostsin endale või ehitasin siis endale uue arvuti, kaks kaheksa, kuus Fida netis kuskil maa müüdud, mida netis käis ka suur nii-öelda riistvara hangeldamine ja ehitasin endale neli, kaheksa kuus arvuti, mis oli ja see oli ja, ja kusjuures mitte lihtsalt neli kaheksa sügavaid teeks nelisada, see oli siis nii-öelda absoluutne tipp. Ja.
Lihtsalt nooremale kuulajale teeks oli see, kus tal oli matemaatikaprotsessori otse otsese CPU peal mis tähendas, et ta oskas nagu arvutada ja mitte vähe.
Ehk et see oli kõige kõvem neli, kaheksa kuus, mis üldse kunagi tehti. Ja siis ma mäletan veel, et mul mingil hetkel oli masinas sellisesse Fidoneti aeg. See oli juba siis see aeg, kui mul oli juba vist nõud registreeritud. Et võib-olla tagasi hüpata sealtsamast tarkooneri Peeveeessist, Casa Peeveeessist, sealt ma sain esimese Fidoneti pointi, kus ma pääsesin ligi, siis nii-öelda Fida netiuudistekanalitele ja siis mingil hetkel tundus, et aga palju ägedam oleks, nõud sai kirjutatud taromaamersile, et kas oleks võimalik, Tarmo loomulikult lahke inimesena. Sest Tarmo oli Eesti regiooni siis nii-öelda nagu mänedžer, tema andis neid nii-öelda aadresse, väljavõte tema käes.
No igaks juhuks muidugi ei räägi oma loomulikust tagasihoidlikkusest selge.
Et siis sai Tarmo ei kirjuta, kas oleks võimalik registreerida nõud number kuuskümmend kuus ja Tarmo vastaselt tehtud ja, ja sealt sealt mulle sealt edasi oli mul nõud, mis mõnda aega eksisteeris kodus. Aga siis mingil hetkel sai seadistatud elektersideinspektsioonis linuxi server, sest meil on praktiline vajadus serveerida printerid ja serveerida faksi ehk et meil oli vaja tekitada siis hea faile ka. Loomulikult ehk siis sai nurka tekitatud siis Linuxi server, kes siis seeris faile üle samba teenuse ja võttis vastu fakse. Ja siis minul õnnestus ka enda Fidoniti nõud sellesse samasse serverisse sokutada.
Juba käis siis.
Et kui muidu filaneti tarkvarale MS tossi peal, siis oli ka alternatiiv juuniksitele if meili ta sar Ihmeli nimelise programmi näol.
Ja korra räägime sellest ametist. Miks, miks see üldse Internetti vaja oli, kas see oli puhas jõu, lugu, huvi või teinud midagi kasulikku ka selleni?
Ma arvan, et see oli vajalik iimelide ja muu selline jah, see vajalik täielikult see ja praktiline vajadus oli olemas, sellepärast et elektriinspektsioon tegi tihedat koostööd iiduga, kes siis juhib kõiki neid sageduste ja, ja protokollide kõike muud. Sellised rahvusvaheline jäära Soelikumi küsis just ja nendega neile hästi tihe kirjavahetus ja ma arvan, et see iimeile oli nagu selle jaoks praktiliselt ma ei tea, ma ei suuda meenutada, kuidas siin meilivahetus enne siis nii-öelda seda kaabliga interneti käis, aga siis pärast seda seesama linuxi masin oli ka loomulikult viimeliserver. Sellel sellest hetkest tekkis ka meil oma domeen nimega rei punkt-EE või äkki domeen oli juba varem olemas, igal juhul, tähendab pärast seda, kui linuxi serverali hakkas rei ee domeeni siis kirjuzasama linuxi purk vastu võtma ja ka mina sain endale isikliku esimese sellise ülilühikese iimile aadressi, mis oli tollel ajal ülikõva kained reibumbee.
Talle omase avastasid ennast suhteliselt niimoodi õrnas eas linuxi ruuduna nagu päris riigiasutuses.
Just, ja, ja kusjuures miks, miks ma vahel seda kümne megabitist Etherneti mainisin, oli see, et seal kõik, see liiklus oli ju näha ja tollele ja, ja kui ma külastasin, siis Anto Veldre arvutiklassi neljakümne kolmandas, siis üks asi, mis mulle, sealt nagu elu lõpuni meelde, et me kõik need noored tüübid, kes seal siis siil edu eeee nimelises Ko masinad aga istusid, oli tohutu kõvad häkkerid demonstreerisid, mida nad siin teevad, näitasid, et kuidas nad on, kuidas nad suudavad, eks Bloittida, mingisugust Tartu Ülikooli salevad masinaid, mingeid professoreid seal jälgida ja nii edasi, see avaldas mulle nii kohutavalt mulje, et mind hakkas ka see, lisaks siis sellele varasemale viisor, viiruste teema huvi huvile hakkas huvi tekkima siis nii-öelda arvuti turvalisuse teemale.
No see on ju põhimõtteliselt sama teema, et, et kui sind enne huvides see, et mida saab arvutiga nagu viiruse abil panna tegema niimoodi, et mida ta ei olnud ette nähtud tegema, siis häkkimise värk tundub nagu, nagu loogiline olevat.
Et siis sõnaga ma hakkasin uurima seda nii-öelda arvuti turvalisuse teemat ja ma arvan, et see on see esimest korda, kui ma nagu avalikult, et sellest räägib agamas Niifisin loomulikult ka meie nii-öelda võrku ja Snyifisin, mida siis valitsusside, insenerid seal oli seal vahel ka midagi, ei seal vahel midagi olnud, sellise sellesama sellesama kaabli otsas oli kaks ametit, oli riigi elektersideinspektsioon kõik oma töötajatega ja valitsusside ehk et kui valitsusside, siis insenerid käisid oma ruuterid või keskjaamu üle del neti konfimas, siis loomulikult see liiklus Leisis lahtise tekstina võrgus, et seda nagu päris huvitav, huvitav jälgida, mida nad siis seal vead. Et loomulikult ma seda kunagi kuskil pahatahtlikult ära ei kasutanud, see oli lihtsalt puhas, selline noore noore mehe huvi
Oskab suurepäraselt toda aega, sellepärast et kui selline asi nagu praegu sünniks, siis ma arvan, et umbes poole tunni jooksul oled noh, keegi teeks mingisuguseid, ma usun tähele. Aga tol ajal kuidagi tundus noh, et, et sa ju ei tee seda või et kui sul ei näinudki või tekitanud kius janu, lähed ütle, et kuule, et parandada ära või tee midagi sellega. Ju põnev.
Siis siis ma mäletan seda, et, et see mulje jäi niivõrd meelde, et kogu see võrguvärk on niivõrd ebaturvaline ja nii kui Soomest keegi härrasmees tekkis, kiur selli esimese versiooni, siis ma hakkasin seda praktiliselt kohe kasutama, kui ma sellest teada sain. Väga hästi just, et, et see pole, see pole nagu hea mõte ja siis veel Keila Gümnaasiumi juurde tagasi tulles siis pärast seda, kui ma kooli olin juba ära lõpetanud, ma jäin nii-öelda edasi siis administreerima seda serverit, mis sinna maha jäi. Kusjuures huvitav lugu on veel selle serveriga see, et nagu tollel ajal ikka, et kõikidel juuniksi masinatel peavad olema ilusad nimed siis kodus rääkisi.
Kunst, kuidas neid nimetatakse?
Kodus rääkisin sellest teemast ja isa pakkus välja, et aga krate oleks jube hea nimi. Ja praegu vaatasin nime serverist järgi, et siiamaani on keelegümnaasiumis oleva serveri nimi Hankrait. Domeen on siis vastavalt kratgeile edu.
Aga seda ei hakka seda nime siis takkajärgi muutma, ilmselt loodetavasti see riistvara ei ole päris seesama.
Ja ja riistvara kindlasti jälle seesama, aga.
Võib-olla virtukas isegi on võib-olla nõudmiganud selle.
Me ei oska öelda, ma.
Sest seda koolimaja füüsiliselt enam alles ei ole, et keelas, on nüüd uus koolimaja, kus mu enda lapsed käes, ma olen siiamaani Keilas ja, ja aga aadress ja kõik on, kõik, on olemas.
Et nad selle pressini asjade nimetamine ongi oluline, et see, kes võib kesta kolmkümmend ja nelikümmend, viiskümmend aastat takkajärgi veel stest
Et selles mõttes see oli huvitav aeg ja siis Fidoneti ajast veel üks huvitav Impact minu meelest, mis mul nagu hiljem on väga kasulik.
Kaks on osutunud.
On see, et modemid töötasid aate käsustikuga siin aate käsustikali, selles mõttes universaalne asi, et seda kasutati ka hiljem erinevates muudes rakendustes. Et loomulikult Veebeeessides sissehelistamine toimus lihtsa terminaliga ehk et sa pidid nagu häkker selles mõttes täpselt teada sisestamise ajal teede, tee, telefoninumber ja nii edasi, võib-olla veel sealistama protokolli enne seda, kui siis, siis helista. Loomulikult, tolleaegsed inimesed teavad täpselt, missuguse protokolliga vilistab sinu saate intro.
Ja nii edasi, et et selles mõttes
See see oli nagu, nagu ka võib-olla selline, mis nagu edaspidiselt nii-öelda mõnes mõttes mõnes mõttes nagu kaasa aitas.
Kas Peebeeessil oli kliendi suht ka, eks ole, ju? Ei olnud.
Kui seda ei sallinud, helistab terminaliga kliendis, oht oli ainult Fidanetil selles mõttes, et filonetil oli siis soft nimega frontoon, kes helistas ja oli siis soft, kes siis pakkis kokku pidaneti Ehod ja siis saatis siis selle paki edasi. Aga Peebeeessil, kui sellisel ei olnud geeni, saab, siis oleks vist eales lihtsalt telliti ja külge ja siis hakkasid seal edasi tegutsema.
Ja see läheb kokku minu mälestustega, küll aga oleks olnud loogiline, et sinna oleks ka keegi mingisuguse rondi teinud või, või mitte. Mind kässimised, niisugused asjad.
Jah, selles mõttes, et kui, kui vaadata, mis Ameerika maal toimus, kus siis olid need nii-öelda need Online Service praederid nagu Aaell ja kompuse järv ja nii edasi, siis neil oli tarkvara nende mäletan seda, et kui ma usas modem ja olin ostnud siis loomulikult noorte poistena, meil tuli proovida seda ja meil oli kujutada ette meile julgus.
Kruvikeerajaga lahti keerata üks.
Üks üks selline suur soliidne arvuti, sellist kompuuter kaks tuhat või mis iganes siis tolleaegne üks selline hästi kõva valge brände oli pissi bränd üks selline arvuti, kuna siis sõbra tädimees oli arhitekti ja tal oli selline väike arhitekt Arhitektibüroo ja meil oli julgus lahti keerata omavoliliselt kruvikeerajaga üks üks selline suur Tauer ja siis proovida siis see modem oli sisemise kaardiga, see ei olnud nii-öelda juht väga ja proovida seda modemit ja modemiga oli kaasas kas kompu servi või mingi muu sarnase teenuse CD plaat ja siis sai helistatud nii-öelda ameerika Peebeeessi plaat ja sisendit, CD plaat, juba äkiline flopi tegelikult äkki mäletan valesti?
Pidi olema floppides CD-d oli ikkagi päris palju.
Trükitud CD-d olid olemas juba tegelikult see, see on ju Niil või tead nii? Video üheksakümmend kolm arenenud trükitud see teed olid olemas juba kirjutatavad CD-d tulid hiljem, need olid kuskil üheksakümmend viis, üheksakümmend kuus. Aga ma arvan, et need trükitud seedeedellid juba olemas.
Kas sa kas nendes Peebeeessidesse, kus sa kolasid midagi muud peale tarkvara ka nagu silma jäänud raamatuid, mainisid moode mainisid?
Raamatud mind eriti nagu tollel hetkel ei köitnud, et Peebeeessidest mina ikkagi laadisin pea peaasjalikult tarkvara ja siis neid muusikamoode, aga kogu siis see infovoog, see tuli siis Gidonetist ehk et Fidanet, ta oli minu jaoks täiesti puhas kullaauk, selles mõttes, et nagu varem mainisin, et mul ei ole olnud ligipääsu kuhugi sellistesse teadusasutus või, või ülikoolidesse ehk et mul ei ole olnud piltlikult öeldes mentorite, ehk et meil oli kamp poisse, kes siis nagu aaaa vahel vahetasid infot, ehk et meil ei olnud nagu sellist vanemat, kes teab, kuidas asjad käivad, vaid meil oli kõik katse-eksituse meetodil piltlikult öeldes
Isegi hästi, et nagu kuidagi paha peale läinud selle kambaga et noh, tihtipeale noored poisid, eks ole, noh juhendajat ei ole, eks ole, sest tean, mida hakkab tegema.
Niisiis, me olime piisavalt mõistlikud, et seda nagu ei juhtunud, et ma arvan, et sellest ajast saadik on mul selline nii-öelda iseõppimise nii-öelda oskus, et võib-olla see sai ka nii-öelda saatuslikuks, miks ma Tehnikaülikoolis ei suutnud väga kaua õppida, kui ainult ühe aasta nagu tollel ajal võib-olla paljudel teistelgi kombeks.
Siis läksid.
Et jälle peale gümnaasiumi ma läksin kohe Tehnikaülikooli informaatikas, aga kuna ma juba tollel hetkel töötasin, siis igasuguseid huvipakkuvaid projekte oli nii palju kõrvale, et et siin on väga palju, noh, ja siis see ülikoolis hakata, mina eeldasin seda, et oh, et nüüd ma saan hakata programmeerimist ja igasugust muud sellist huvitavat asja õppima, aga siis tuli välja, et ei, et sa pead kõigepealt läbima füüsikat ja matemaatikat. Matemaatikast mul oli juba nagu natukene juba kopp ees, kuna meie meil oli selline väga püüdlik kui matemaatikaõpetaja gümnaasiumi ajal. Ehk et me tegime väga põhjaliku matemaatika, mingit sissesaamise probleemi Tehnikaülikoolist üldse absoluutselt ei olnud, et, et ma, et ma aitan eksamist, lihtsalt lendasid läbi.
Oleks olnud kasutu, aga oli lihtne ja kasutu. Just.
Ja ja siis sisse ülikool nii-öelda järgmisel aastal pooleli jäi.
Siis kuidas, mis eri vanuse ja läksid siis informaatika omandiks ja kuidas sul kaitseväega?
Ja siis tuli Kaitsevägi selles mõttes kus ülikoolis ei ole, siis varem või hiljem leitakse sind üles. Aga see kaitseväkke ma läksin üheksakümne seitsmenda aasta suvel, ehk et ma olin siis juba aasta otsa neti teinud. Ehk et ja, ja kuidas ma sinna sattusin, ma siis elekter siniviitsinktsioonist töötades hakkas natuke nagu selles mõttes ära tüütama, noh nagu ikka, et selles mõttes tahan edasi areneda ja, ja hakkasin otsima, et tahaks kuhugi nagu huvitavasse kohta tööle minna ja põhimõtteliselt mul mingil hetkel oli nii-öelda soov kindla peale töötada arvutifirmas, sest noh, kuhu sa ikka lähed selles mõttes arvutifirmasse, olen kindel see, et sain arvutitele väga lähedal, seal on palju aru.
Nagu selgitab, kõlab, kõlab loogiliselt.
Siis ma vanu nii-öelda arhiive Plapäcope läbi kammides jäi silma, et no mingil hetkel kandideerisin isegi Helme sesse, aga, aga siin sinna ma ei saanud õnneks tagantjärgi mõtlen ja siis mu järgmine nii-öelda valikalise, et okei, et a-Molinnas siis keskkooli ja ülikoolivahelisel ajal suvel ma töötasin poolteist kuud, et Tõnu saa mulli IT-firmas nimega eramees ja oma nii-öelda istusin samale kohale, kus oli just lahkunud Bronto, Tanel Raja ja siis Tõnu ütles mulle, et kuuled-näed, Pronto müüsin neid kraavis ultrasondi kaart, et kuule, hakka nüüd sina sellega tegelema. Aga ma olin noor koolipoiss, maid jäänud kaubandusest, nii et Es ja nii edasi. Ja siis ma ma ei usu, et minust erilist nii-öelda kasu kui sellise noh, välja arvatud, kui nii-öelda patsiga poisist selles saama oli ettevõttes eriti ära.
See asjade müümine kõlab nagu väga lugu läbiva mustriga, see on päris mitmed on noh, kuni Tarmo Tallinna välja ja tegelikult on mingist hetkest juba tegelikult müügimehe ja üldse mitte halvasti, nii et et see ei ole üldse nagu helistab.
No ära, et mina, selles mõttes ma küll selles mõttes
Ütleb reegleid, eks ole. Ja siis kuidas see Lätisse sattusid leti juurde?
Aga ma, ma räägiks veel sellest eramehest natukene seal seal oli üks asi, mis mul on veel eredalt meeles. Tõnu PB -s oli siis kontoris, see kontor asus Eesti Talleksi majas Mustamäe tee üks vist, kui ma ei eksi, on aadress. Ja PPS oli põhimõtteliselt nii-öelda laiali laotatud arvutijupid, et aknalaual ehk et seal oli sisse Uuesser kurjermale emaplaat toitev lank ja nii edasi loomulikult ei olnud kastis nii-öelda kas koos või lihtsalt ju hunnik juppe ja juhtmed, mis oli aknalauale lahti laotatud ja see oli siis Tõnu, Tõnu PPS või nõud.
Palju, kui ma aru saan, siis see mood on meil tagasi, et pannakse jupid, juuksed värvilist seina peale, veetakse kaabliga, eks ole, need Tõnu oli oma ajast väga palju ees.
Et, et see oli see, see on huvitav huvitav seik, mis mul sellest erame ajastust meeles on, aga siis pärast seda ma siis kandideerisin, Espak ta tasse, sest mulle tundus, et ISP, et see tegelikult veel huvitavam asi, sellepärast et nad tegelevad ju internetiga.
Kuigi tol ajal juba telefoni omavahel eraldi
Ta oli, ta oli siis tollel ajal eraldi, et kui õieti mäletan, siis Espak Data omanikuks oli siis Telekom, mitte nii-öelda Eesti Telefon, ehk et ta oli siis nii-öelda nagu Telekomi oma pigem ja siis ehk et oli täiesti selles mõttes eraldiseisev ettevõte Eesti telefonist. Ja siis
Huvitaval kombel kellelgi oli tulnud selline idee, et, et meil on kuidagi vaja edendada nii-öelda veebi sellist virtuaalhostimist ja siis keegi oli välja mõeldud netipunkt, tee nimelise domeeni ja selle domeeni alt siis üritati müüa sellist traditsioonilist veebi hostingut, tollel ajal ta veel traditsiooniline anda, aga ütleme, et siis nii-öelda nagu tänapäeva mõistes. Ja siis Espak Data palkas mind kui nii öelda veebasteri, kes siis oleks hoolitsenud või kes siis pidi hoolitsema siis selle nii-öelda veebi hostingu serveri ja teenuse eest. Ja siis muuseas, neil oli selline ideed, et, aga et kuidas me seda nii-öelda veebi hostingu äri ikka muud moodi edendama, kui meil on vaja mingit kataloogi, sest inimesed peavad leidma üles need veebilehed, mida siis kliendid sinna panevad?
Tollal meediamaa, mitte meil, jah, meedia maalib olemuseni.
Meediamaa startis umbes samal ajal. Enne seda oli olemas nii-öelda Eesti veebisaitide nimekiri, mis oli end Liivi ehk siis Rahvusraamatukogu domeenis, kus siis Toomas Mölder tegutses ja Toomas Mölder kolis, ma arvan, et siis sellesama nimekirja siis meedia maasse ja sealt tuli VVV punkt tee. Kuna meedia mas üks tegelane oli Tarvi Martens, siis neil õnnestus kuidagi Eeneetilt välja meelitada domeen nimega W W punkt ee ma arvan, et mitte kellelegi teisele, kui tarvil ei oleks sellist domeeni elu sees välja antud.
Seda ma kujutan küll ka, et eks see selle kataloogi tegid siis käsitsi alguses.
Ja ka alguses alguses sai seda kataloogi käsitsi tehtud, et ta oligi selline väga nii-öelda algeline puine aga asi hakkas lendama siis, kui kui ma kutsusin kataloogi puhul endale appi Jaanus Vainu, kellega ma olin kokku saanud riigi elektrist, inspektsioonist, kes on ka selles mõttes omamoodi huvitav tegelane, et esiteks elekter, siis sessioonis tema mõtles välja või siis nii-öelda planeeris kogu meie kehvem kahe sada kaheksa sageduse plaani näiteks ehk et kõik Eesti raadiojaamade sagedusnumbrid on tema tehtud, sellepärast et nõukogude ajal oli meil teistsugune FM sagedusala. Sellepärast Nõukogude Liit ei tahtnud kattuda läänesagedustega, sest see, kuidas siis saab nii et sa saad poest osta raadio, millega saab välismaa raadiojaama kuulata, et see ei sobinud kuidagi. Ja siis Eesti Vabariigi alguses koliti siis selle läänesagedustel üle ja siis Jaanus oli üks nendest, kes siis käis mööda Eestit ja siis see mõõtmas ja tegi siis sagedus plaani, tal oli väga detailselt korraldroos joonistatud kõik need nii-öelda sagedusringide Eesti kaardi peale. Et, et siis planeerida need sagedused niimoodi, et üle Eesti saatjatel oleksid sellised sagedused, millel on võimalikult vähe häireid, siis naaberriikidega, pluss siis nendel omavahel korraldrool. Ta tegi seda korraldroos ja kogu seda teadust kogu seda teadvust korraldroovasse.
Millega millega, mis käepärast parasjagu hullid?
Just ja siis Jaan Jaanus on selline nii-öelda tohutu pedant, aga tohutu sellise nii-öelda töövõimega kataloogi seeria tema, tema enda isiklik huvi on bluugress, muusika. Mäletan seda, et et tema oli minu minu teada esimene esimene inimene, keda mina tean, kes välismaalt e-poest asju tellis ehk et tema telliksiidi nous plaate endale. Mina väga imestasin, et kuidas selline asi üldse võimalik on. Et ta tellib kuskilt, ma ei tea, Jumal teab kust ja tulebki pakiga kohale. CD muusikaga.
Mõned on isegi, et juba üheksakümnendate lõpus oli selline asi ikkagi eksootika, siis kui Amazon tekkis ja selliste asjade see tundus kuidagi ka väga-väga veider asi oli isegi veel parem. Et ja siis mingi hetk, mis hetkel nagu kas täielik rooleri seal või kuidas, nagu automentiseerisite seda.
Ja siis ma arvan, et see, meie, see tandem Jaanusega töötas selles mõttes ülihästi, et mina olin siis nii-öelda see programmeerija ja, ja siis noh, tarkvara arendaja Jaanus oli siis katalogiseeria ehk et kui Jaanus siis selle projektiga liitus, siis võiks öelda, et see projekt hakkas täielikult lendama. Et ma arvan, et meil läks võib-olla paar kuud aega, kui me olime meediamaast igatpidi kõikide näitajate poolest mööda läinud. Et et me olime tollel ajal võib-olla isegi natukene liiga ebaviisakad, noored noormehe, et selles mõttes, et, et.
Näiteks reklaamisime netit spämmides, nüüd tekib see nagu ühe korra sellise nii öelda masspostituse saates kõikvõimalikele iimelidele, teate et nüüd on selline huvitav teenus olemas nagu netipunkt, e-tulge, külastage midagi, mida iganes sarnast. Kusjuures huvitav on see, et kui ma vaatasin enda päkk cape. Ma nimetasin enda kroolerit nuhiks, noh, seda otsingurobotid, kes mööda lehti Ringo ja huvitaval kombel Malin sele nuhi programmeerimist alustanud juba mitu kuud varem ehk nagu, nagu nagu miski oleks suunanud mind sellele teele, et seda võib vaja. Ja kusjuures otsingumootoreid ma olin ka natuke varem teinud, ehk et selle selle pärast seda nii-öelda erameest. Kui ma läksin ülikooli, siis su ära meest, ma sain uusi kontakte üks üks kontaktidest. Siis kutsus mind tegema ühte nii öelda.
Siis ärikataloogi sarnast teenust, mille pealkiri oli partan, et see asus estada serverisse, oli mingi Žanni server Akadeemia tee kakskümmend üks siin teisel korrusel samas majas, kus me hetkel viibime ja siis selles sani serveris, ma ei tea, mis asjaoludel, aga ma millegipärast sain seal teha FTP servereid otsingut ehk et ma panin seal püsti otsinguteenuse nimega Fileriks mis töötas umbes kolm-neli kuud, mille ainukeseks sisuks oli see, et ta võimaldas väga hõlpsasti üles leida siis igasugustest kohalikest FTP Miroritest ja tollel ajal siis Marek Tiits IV essist hostist, sellist asja nagu Youcous oli, et ühesõnaga minu siis nii-öelda otsingumootor võimaldas hõlpsasti failinimede järgi programmi nimede järgi üles leida tarkvara siis tolleaegsele Windows üheksakümne viiele, vanadele windasitele ja nii edasi ja nii edasi ja nii edasi. Ehk et ehk et tolle tollest mingisuguses pooleaastasest projektist nii-öelda kõrvalprojektina, ma tegin faili otsingut, see ikkagi tähendusse.
See just failide indekseerimine ei ole ju enam naljaasi, see tahtsime ikka mingit programmeerimisoskust saada, et kust sa selle ülesse korjavad?
Programmeerinud tollel hetkel ma oskasin programmeerida Perli ja siis kõike seda, mis oli juuniks siis sellis saada on. Et see tuligi selle sellest ajaperioodist, kui ma uurisin, mis on nii-öelda juuniksil teol kõhus
Edu siis.
Ta lihtsalt näppisid ja said. Ma mõtlen just seda, et, et seal ei ole ju sulle ainult sellest, et mis käsuga juhtub, printimiskäsuga saab faili kirjutada vaid, et see on ikkagi algoritmi ja, ja see, et ma ei külasta sama kohta nagu mitu korda, kuidas moptimiseerikside.
Jah, mis puudutab nii-öelda veebi krooli, mis siis jah, selle peale tuli juba nii-öelda mõelda, et.
Puhtalt nagu konteksti pärast kaasuse kroolev nagu läbi käis, et kogu see Eesti, vepsa
Novembri väga, ma arvan, et see oli mingi stiilis ööpäev või midagi sellist, siis selline väga väike veel oli tollel ajal, et et ma usun, et ma täpselt pole vaadanud, aga ma usun, et selle kataloogi suurusele stiilis võib-olla paar tuhat tallinki ja mitte rohkem tollel ajal. Ja, ja keskmine koduleht oli ka selline kolm kuni viis lehekülge, et see, see olnud eriline, nii-öelda eriline teema huvitavamaks läks pärast, siis kui see juba miljonitesse läheks, et siis mingil hetkel oli ikka selline krooler, mis töötas paralleelselt nii-öelda paljudest reedides ja nii edasi. Aga noh, see oli keik, senine loomulik evolutsioon, sealt sealt edasi. Aga jah, siis see ja siis ma mäletan tegelikult, et miks ma arvan, et miks mind Espaktaatesse tööle võeti ühe sellise kõrval nii-öelda projektina ma olin teinud HTML-i tutvustuse vara, et mul oli vist koolis olnud vaja seda nii-öelda kellelegi õpetada. Ehk siis gümnaasiumis siis tollel perioodil, kui ma siis keela seda serverit administreerida ja siis mulle tundus, et nagu, kuidas ma seda ikka õpetaja mingeid eestikeelset materjali pole ja siis ma tegin ühe esimese nii-öelda eestikeelse sellise HTML-i tutvustuse, mis võttis läbi kõiki üksikuid nii-öelda elemendid, millegipärast tuleb see muru kusjuures kus otsused, kusjuures ma olen üsna kusjuures seesama HTML tutvustus on sellel samal aadressil täna ka üleval ja ma olen üsna kindel, et see on üks kõige vanemaid veebilehti, mis seal ei leidu täna Eesti veebiruumis, mis on originaalkujul originaal aadressil. Ma olen selles üsna kindel, et see
Sest see on üheksakümmend kuus.
Ma arvan, et saavoodee on varem võib-olla jah, vaatame pärast. Aga see selleks.
Ja siis, ja siis ennem seda siis ühe huvitava veel projektina olin teinud veebi pokkeri, ehk siis sellise nii-öelda veebipõhise mängu, et selles mõttes, et ei saa öelda, et ma pole, et mul pole kunagi huvi olnud ka mänge teha, aga ma olen ka rohkem nii-öelda oma elus programmeerinud nii-öelda veebiasja, kui selliseid nii öelda test hoopis või, või siis nagu masinas töötavaid rakendusi. Ja, ja selle ja nende pea nende teadmiste baasil ma sinna Espaki mind siis tööle võeti. Tõenäoliselt ma siis näitasin ka seda, et vaadake, ma olen teinud sellise veebi pokkerimängu mulle teinud HTML-i tutvustuse ja, või võib-olla ma rääkisin ka seda, et ma olen siis seda kroolerid teinud igal juhul mind võeti sinna tööle ja ma sain jätkata piltlikult öeldes sellesama koha pealt, kus ma juba olin.
Kes teile seda toote poolt tegi või seal ei olnudki niisugust mõistet nagu tootejuht ei olnudki viga.
Piltlikult öeldes pandi mind nii-öelda istuma, et palun istu siia ja tee ja ka tegelikult tegelikult te siis tegelikult see oli ikkagi läbimõeldud selles mõttes. Espak Data tegi koostööd ühe reklaamiagentuuriga, reklaamiagentuurilt rentis ruume Kullo majas Mustamäe teel, need tegelikult minu töökoht oli selles füüsiline töökoht asus selles reklaamiagentuuris Kullo majas. Ja, ja ma töötasin siis nii-öelda püsiühend meil ühesõnaga minul oli arvuti, millel oli püsiühenduse üheksateist koma kaks kilobitti sekundis ja, ja sealt ma töötasin ja siis see ühesõnaga ma saan aru, et Est portaali teinud nii-öelda noh noore mehena nagu ikka, et sind ei huvita, kuidas rahad liiguvad ja nii edasi, et sind huvitab ainult see tehniline pool. Et, et idee siis seisnes selles, et reklaamiagentuur, siis aitab potentsiaalsetel Espak Data klientidel teha kodulehti, aitab teha neile reklaami ja nii edasi, et see, see oli umbes midagi, midagi sarnast, kokkulepet. Noortekoor ma ei tea, kas nimi kellelegi midagi ütlema, aga autokoolides, aga, aga selles mõttes oli huvitav, et üks kolleeg, kes noortekooris töötas, oli kunagise okka
OK jutuka.
Ka nii-öelda üks asutajatest
Mitte küll Kaupo Kalda, aga siis iga siis see teine.
Tüüp, mis kelle nimi on ikka veel ei ole hullu, tyyp.
Tiit Sermann kusjuures oligi nagu naljakas, et tema alias oli Ott ma tegelikult tean, päris nimi oli Tiit väga salakaval, ehk et niiviisi lihtsalt selline huvitav asi, et ehk kuidagi tundub, et kogu see maailm nagu tollel ajal enne nii pisikene, et kui see natukene selles maailmas ringi käisid, siis puutusid paratamatult kuidagi kõikide nende inimestega kokku, kes tollel ajal toimetasid ka.
Tegelikult nii pisike ei olnud, sest kui hakata nüüd neid mälestusi nende läbi käima, siis need nende inimeste hulk, kes siin figureerib, ma ikkagi noh, üllatavalt suur. Mina arvasin ka, et siin võib-olla jääb mingisuguse paarikümne inimese juurde, aga hakkab tasapisi juurde tulema. Räägi korra sellest, kuidas te veel meile tegite. Veebimeile votti.
Ja oi, see, see oli tegelikult ka päris huvitav. Selles mõttes. Ma käisin kaitseväes ära kaitseväes tagasi tulles, Eesti Telefon oli Espak Data ära söönud, selles mõttes. Espak Data lakkas olemast. Mingil hetkel ma töötasin siis Lasnamäel Koorti viisteist, kus Espak Data enneli vana Eesti Telefoni maja, aga siis õite pea koliti meid sealt siis nii-öelda päris Eesti Telefoni muudesse ruumidesse ära. Ja siis ma olin Eesti Telefoni sellises allüksuses, mille pealkiri oli teleteenuste arendus, ehk et Eesti Telefon nagu teleteenuseid tegev ettevõte ja siis sellist põhimõtteliselt nagu siis selle Eesti Telefoni arendusüksus, kelle eesmärk oligi nii-öelda välja töötada uusi teenuseid. Ja siis oma netitegemisega me sinna sattusime ja siis
Molin kontorikontoriruumi jagasin ühe ühe teise noormehega, kes arendas sissehelistamisteenust. Ja siis huvitaval kombel meil vedeles kapi peale üks sissehelistamiskeskus. Ja siis ma küsisin, et kas ma võin seda uurida ja nii edasi ja siis see oli üks sääskendi sissehelistamiskeskus, selline pisikene see seal väga palju nii öelda liin ei olnud.
See oli mingisugune tükk restoranis söömas, käisid mingeid ohte seal mudeli sisse, kui ta oli nagu.
Ei, ta oligi nii-öelda delikkeid sissehelistamiskeskuse, et sa põhimõtteliselt installeerisid ta Räki panid tal juhtmed külge ja ta hakkaski osutama teenust, kuuluma numbrid kuulama, Internet tuli tal teisest otsast sisse ja nii edasi. Aga miks ma seda räägin, on see, et selle käigus ma siis sain teada, kuidas sissehelistamiskeskus töötab. Avastasin selle sissehelistamiskeskuse Audendibennast vastu sellist nii-öelda autentimisserverit nagu raadius ja sealt edasi uurisin, et mis asi see raadius on, sain teada, et see on diktsionäri põhine protokoll üldsegi mitte keeruline ja mu programmeeris siin siis raadiuse serveri, kes siis suutis sissehelistamiskeskust juhtida ja avastasin, et see sissehelistamiskeskusse, firmer võimaldav igasuguseid huvitavaid asju, mis tundusid olevat nagu seni kasuta näiteks see, et sa võid kohe raadios serverist öelda sissehelistamiskeskusele, kui gaase kasutaja võib onlainis olla. Ja sellest teadmisest näiteks sündis selline toode nagu Atlas Surf, mida Eesti Telefon nii-öelda Priibeid internentina mäes. Et see ühesõnaga toode sündis puhtalt sellest, et mina häkki siin siis seda nii-öelda seda väikest sissehelistamiskeskust, mis oli tegelikult üldse mõeldud modemit, mis on üldse mõeldud mobiilidega sisse helistamiseks, ta toetas sellist huvitavat protokolli nagu vee kolmkümmend viis, paljud pole sellest ilmselt mitte kunagi kuulnud, aga see oli mingisugune selline nii-öelda vaipa, ent protokoll, mis töötas üle GSM-i, ehk et kui sul oli selline GSM telefon, mida sai siis arvutiga ühendada, siis ta võimaldas sisse helistada selle vee kolmkümmend viis protokolliga ja sa said veidi suurema kiiruse kui nii öelda tavalist modemit vilistades üle-üle-üle siis mobiili. Et näiteks mäletan sellist keissi, võib korraks hüppad natuke tulevikku, oli aasta kaks tuhat, kõik mäletavad y kaks ka kohutavalt hirmus, et kõik aru, et pooled arvutit lähevad katki sellepärast et kell ja lakkab töötamast ja nii edasi ja ka Eesti telefonis kardeti, seda siis Legaci süsteem oli tohutu palju ja kõik insenerid, kes olid mingisuguste süsteemidega seotud, pidid jääma nii-öelda valvesse, aga ma ei tea, kuidas minul õnnestus sellest ära nihverdada niimoodi, et tollel hetkel ma olin Soomes suusatamas, sõpradega lumelauaga mäest alla laskmas ja see stiilis paar päeva enne siis seda aastavahetust tuleb mulle siis klienditeenindusest kõne, et kuuled, et nüüd sisse helistada enam ei saa, et mingi jama on. See, läksin siis autoni munele läptop, kaasasime sellesse juba aasta kaks tuhat, onju? Läptop kaasas olin Soome, Soome Vabariigis, panin telefoni läptopi järgi, helistasin siis nii-öelda siis sellesse meie enda nii öelda privaatkeskuse sisse seesama vee kolmkümmend viis protokolliga ja hakkasin siis vaatama, et mille pärast siis kliendid siis selliste sa siis sotti kliendid siis tuli välja seda, et keegi oli veel viimase hetke mingisuguse turvaPätsi nii-öelda peale laadinud y kaks ka nii-öelda hirmus ja see muutis natukene seda nii-öelda teadet, mis raadiuse serverile saadeti ja siis raadiuses järvel läks selle peale katki, kuna talle tuli tundmatu tykk tundmatu sisuga, et insenerikulud, eks ole, just just aga siis jah, siis selle sealt selles surfist nii-öelda edasi. Meil turu tekkis selline olukord, kus Eesti Telefoni kontsessioonileping, mis ta oli juba lõppenud või lõppemas, ma arvan, et see oli just juba lõppemas või lõppenud ja turule tuli tele kaks Rootsist ja Tele kaks tema idee oli korrata Eestis täpselt sama, mida ta tegi Rootsis ehk ehk ehk et ta toota soovis nii-öelda siis sellelt suurelt telk olnud palju raha välja imeda. Ja me teadsime seda, et noh, kuna Eesti Telefon siis üüris ruume, liine ja nii edasi, ehk et meil oli teada, et paneb oma sissehelistamiskeskused püsti ja siis Eesti Telefoni juhtkond oli selles paanikas, ma ise külastasin mingisugust sellist laiendatud juhatuse koosolekut, kus sellest arutati ja ma mäletan, et ma tulin sellest üsna mornilt nagu tagasi, mulle tundus, et vanad härrad nagu vanad kolleegid ei suuda nagu midagi otseselt nagu ära teha, et vea sellise noore mehena oleks tahtnud vaherahu, Frank, et läksime ja siis ma ei mäleta, mis asjaoludel ma olin kodus, aga ma pidasin siis telefonikõne Priit Pirsaga, kes oli tollel hetkel siis selle valdkonna juht Eesti telefonis. Ja selle telefonikõne käigus me otsustasime, et, et me teeme Eesti Telefoni osutatavale Atlas Starter teenusele alternatiivse teenuse, sellepärast et Atlas Starter absoluutselt ei sobi Tele kahega konkureerimiseks. Meil on vaja sellist teenust, kus on siis kasutajate nii-öelda registreerimise protseduuri kõik selline automaatne noh, seal sööbinud nii-öelda kasuta saab ise regada ja nii edasi, nii edasi, kuna siis nii-öelda kuutasu niikuinii pärast seda tele kahe jampsi enam ei ole. Siis ainuainukesed, mis maksavad, on kõneminuti hinnad, ehk et Tele kaks lootis siis raha teenida sellest, et ta termineerib kõnet ja Eesti Telefon on sunnitud talle nii-öelda vahendama siis seda nii-öelda kliendi käest küsitud kõneminuti hinda. Ja siis selle telefonikõne käigus me leppisime kokku, kes mida teeb, kuidas teeb ja, ja ma olen üsna kindel, et selle kõne käigus me leppisime kokku, et selleks saab nimeks saab hot, sest ka muuseas juba arvutist vaatasin ja millised huvitavad domeenid on meil avad ütles kusjuures veel tollel ajal oli veel, sest see aeg, kui EENet ta ei nõustunud andma ühele ettevõttele mitut domeeni. Mina ei tea, kuidas, aga minu üks tänane kolleeg kui Kuido Kõiv temal õnnestus kuidagi netist saada hote ee domeen meile, ma ei tea, kuidas siis sarnane Insaid nagu nagu Darwin oli VW pump teele igal juhul me saime seal ühe või paari telefonikõnega väga lühikese ajaga kokku lepitud, kes mida teeb. Ja kujutad sa ette, kahe nad kahe nädala pärast me olime laivis, see tähendab, et meil toimus teenuse lanss ja meil hakkas kasutajaid registreeruma tuhat tükki päevas. Tempoga.
Tolle aja kohta oli meeletu-meeletu praegugi nagu täitsa okei kiirus. Et.
Et sealt sealt sai siis nii-öelda hote alguse, et minu minu teha jäi siis seesama raadiuse pool ehk siis Hothoti puhul Audentis kasutajad Oti siis oma ainesisu tegelikult seisnes selles, et meie huvi oli see, et inimesed helistaksid meele sisse siis tollel ajal hakkas ka juba olema kombeks, et anname ka kasutaja vii meili aga kuna varasemalt küsi tiimile eest raha, siis meile tundus lihtsalt nii, samas neil imele jagada ei tahaks. Ja siis sai tehtud sinna siis selline Kriuks, et sa saad küll nii-öelda veebipõhiselt konto luua. Kusjuures imelik chicken, Emmdeek probleem, et kuidas sul on konto loomiseks Internetti vaja, et et aga, aga tundus nagu see takistuseks, noh, see on nüüd ju siis sa saad internetisait kuskilt mujalt nüüd neid vanu nii-öelda internetiühendusi kasutades igal juhul tähendab jah, registreerimine, kes veebipõhiselt ja see, see nii-öelda II meilikonto ja ka kodulehekonto jaganud ennem tööle, kui Sa olid selle registreeritud kontoga vähemalt ühe telefonikõne teinud sissehelistamiskeskuses ehk et siis seda loogikat võimaldas siis minu kastam raadius, kes siis kõikidel nendel kasutajatel nii-öelda siis järge pidas. Ja.
Kas ühel hetkel oli seal veebi meil ka, eks ole?
Ja veel ei, Veeveeler oli, ma arvan, et suhteliselt algusest kohe juba sellesama esimese
Kui olemasoleva võtt enne, kui me lõpetame, et see, seda ma tahan
Aga see on minu programmeeritud sõeli internetist leitud vabavara, mida me saime kasutada, siis ma arvan, et me isegi seda nii-öelda ei rebrändinud nii öelda enda värvidesse, vaid see oli lihtsalt meie lehelt lingitud. Me isa hostisime teda.
Siis see seletab seda, et miks, kui me üritasime Hansapangas mõned aastad hiljem teha beebi meile, miks see meil nurja läks. Aga see on kohutavalt keeruline, taas hansanetis, üritasin nullist teha, millegipärast meil ei tulnud sellist mõtet pähe, et et internetist see asi, see asi.
Ja mulle ka ei tulnud selline mõte pähe, et seda nii-öelda ise teha. Küll aga mäletan seda, et hiljem kui keegi mäletab, olid sellised huvi, oli selline protokoll nagu vapp ehk siis mobiiliprotokollis, mobiilivarianti internetist ja vot selle vapp meili ma küll tegin täiesti nullist, ise sellelesamale, Otile.
Õnneks ei olnud väga pika elueaga, sest vapi olnud väga pika eluajaga jah, et seda oleks keegi päriselt kasutatud, ma tegin internetipanga isegi üle vapi, aga see Leret kuhugi.
Ja mina mäletan ka seda lehel legendaarset nii-öelda siis väidet Ando veendunult, kes oli tollel ajal siis seente üks arendusjuhte, kes, kes kommenteeris minu peo vapp meili sellega, et noh, sa võid ju sinna suahiili keelekava on, aga noh, ilmselt pole sellest väga palju kasu. Aga kogu see vapsai minul isiklikult alguse sellest, et, et ma olin saanud endale vapivõimelised telefoni. Et mäletan seda, et, et ma arvan, ma arvan, et see oli üks ainukese telefone, mida ma olen iialgi tööandjalt saanud sellises noh, ja seitse, üks üks null telefon sellisesse klapiga telefon, millel oli suur ekraan
Ja suurepärane põhjus, miks oli hädasti vaja Hansapangale see vapipõhine internetipank teha, sellepärast selle panga, selle internetipanga testimiseks ju pidi tööandja väljastama niisuguse suurepärase telefoni.
Väga, aga mul oli nagu Vaisveeerr selles mõttes, et ma sain kõigepealt telefoni, siis mul tuli idee taga, et mulle telefon nüüd on, aga mida ma sellega teen ja ja et jube äge oleks enda postkasti sinna sisse vaadata sellisel mugaval moel ja siis ma tegin siis selle meeli siin sellega.
Juba kuidas öeldakse, et algab uus sajand, algab uus ajastu ja sellest räägime võibolla mõni teinekord. Lõpetuseks küsiks veel korra lühidalt, et mis sa praegu teed?
Praegu ma olen boldis serveri infrapeal ehk et minu üks kauaaegseid kolleege ja sealtsamast Eesti telefonist Tarmo kople on üks nendest inseneridest kellelegi me siis alustasime Bolti kogu seda serveri majandust praktiliselt juba algusest peale, ehk et kui meie alustasime Tarmoga serverite poole majandamis boldis, siis meil oli stiilis tuhandeid, et kliente ja tuhandeid sõitusid kuus ja nüüd see on asendunud siis miljonite klientide ja miljonite sõitudega.
Et siis päris päris pikk tee on läbi käidud, sellest kuuest läkwyeri flopist saiaid saadud Anto Veldre käest.
Jah, Linuxi teadmine on ka täna minu igapäevane leib, kui nii võtta.
Et siis loodetavasti antava teeb rõõmu see teadmine, et kuu või kuhugi kuus klopiti inimeste, Aet. Aitäh sulle, aitäh.
