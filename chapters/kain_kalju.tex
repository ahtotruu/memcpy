\index[ppl]{Kalju, Kain}
\question{Kuidas sina arvutite juurde jõudsid?}
See oli umbes aastal 1990-1991, kui mu sõpradele tekkisid esimesed arvutid, 
olime kaksteist kuni neliteist aastat vanad. Ühele sõbrale tekkis selline 
imelik asi nagu Texas Instruments TI-99\sidenote{Täpsemalt Texas Instruments 
TI-99/4\index{Arvutid!Texas Instruments TI-99/4}. Ärilistel ja arhitektuursetel 
põhjustel lühikese elueaga koduarvutite perekond. Oli koos samal 1979. aastal 
turule tulnud Atari 8-bitiste arvutitega üks esimesi omataolisi, millel oli 
audio- ja videoülesanneteks omaette protsessorid.}, see oli Commodore ja Apple 
II sarnane riistapuu selles mõttes, et ta oli 16-bitise protsessoriga ja 
\emph{boot}is otse BASICusse\index{Keeled!BASIC}. 

See arvuti oli telekaga ühendatud ja seal olid mingisugused primitiivsed mängud 
Space Invaders\index{Mängud!Space Invaders} ja muud sarnased. Ja siis 
loomulikult ka BASIC. Kogu programmi kood tuli kassetilindilt, nii nagu tollel 
ajal kombeks, mingeid flopisid polnud olemas. See oli minu esimene kokkupuude 
sellise arvutiga, millel oli klaviatuur, kuhu sai sisestada programmi koodi, 
kus me siis katsetasime ka esimest korda ise programme teha BASICUS toksides 
neid ajakirjadest ja mõeldes ka ise välja. 

\question{Mis linnas see oli?}

Ma olen Keilast\index{Keila} pärit. Mul ei ole nagu kunagi olnud mingisugust 
sellist spetsiaalset ligipääsu kuhugi  teadusasutustele, koolidele ja nii 
edasi. Minu ligipääs arvutitele oli selles mõttes suhteliselt  piiratud 
võrreldes mõnede teistega.

\question{Kas sul seejuures mingit reaalainete huvi ka oli taustal?}

Koolis ma käisin reaalkallakuga klassis. Meil oli väga vahva lend 
gümnaasiumis\index{Koolid!Keila Gümnaasium}, meil praktiliselt kõik poisid olid 
mingisuguse arvutihuviga ja nii palju kui ma nende elukäiku jälginud olen, on 
praktiliselt kõik  arvutimaailmas miskit pidi tegevad.

\question{Aga kust see tuli? Teil oli koolis nii korralik tase?}

Selles mõttes ongi väga huvitav, et gümnaasiumi esimestes klassides (me just 
olime läinud kaheteistkümne klassi süsteemile), meil olid kooli arvutiklassis  
Jukud\index{Arvutid!Juku}. Need loomulikult absoluutselt meid ei huvitanud, 
seal oli Pascal\index{Keeled!Pascal}, meil oli juba ligipääs PC-dele tollel 
hetkel. 

\question{Juku oli ju igavesti äge aparaat omas ajas?}

Jah, aga nad tulid  selles mõttes  hilisemas faasis, pärast seda, kui meil oli 
juba PC ligipääs olemas ja kui mul endal oli ka kodus juba PC. Minu  kõige 
suurema arvutihuvi läkski sellest hetkest lahti, kui vanemad otsustasid mulle 
PC osta. Seda lugu peab natuke tagasi kerima selles mõttes, et seesama sõber, 
kellel oli see Texas Instrumentsi imepill, sai aasta hiljem 
286\index{Arvutid!286}  monokroomekraaniga. Tal isa käis Ameerikas ja  tõi 
sealt. Naljakas oli veel see, mis näitab seda ajastut, nad elasid esimesel 
korrusel kortermajas ja PC oli raudkapis, mis käis kinni. Oli nii suur hirm, et 
keegi murrab sisse ja varastab ära.

\question{See arvuti ju maksis rohkem kui korter tollel ajal. Mõni ime, et PC 
kappi pandi!}

Mu vanematele, käis see kohutavalt pinnale, et ma üldse ei viibi kodus, olen 
kogu aeg sõbra juures külas, hilisööni välja. Millalgi üheksakümnendatel, 
vahetult enne Eesti krooni tulekut oli aeg, kui rubla devalveerus hästi 
kiiresti. Ma isa käest olen küsinud, kuidas see täpselt oli, ja ta meenutas, et 
tollel hetkel tema sai millegipärast palka juba Ameerika dollarites  ja siis 
mingisugusest kooperatiivist või mis iganes tollel hetkel äriühingud olid, sai 
ostetud üks 286 dollarite eest.  Hinnaklass oli umbes tuhat dollarit. See oli 
siis VGA ekraaniga ja nii edasi. Täis mats, täiesti uus, väga äge, kuigi 286 
ilmselt oli tolleks ajaks juba \emph{outdated}  natukene,  oli juba 386-te 
ajastu.

\question{Ikkagi, võrreldes nende XT-dega, mille abil Tartu Ülikoolis 
programmeerimist õpetati, oli see ikkagi väga kõva sõna. Mis sa tegid tolle 
arvutiga?}

Nagu noor poiss ikka, tõenäoliselt mängisin, mind huvitasid kõikvõimalikud 
tarkvarad. 

Üks huvitav seik on veel, et me käisime sama sõbraga 1993. aastal Ameerikas. 
See oli umbes aasta pärast seda kui ma omale arvuti sain. See Ameerikasse 
minek oli väga kummaline. Ma mäletan seda, et meil oli  kolmene punt, kes me 
elasime üksteise lähedal ja  kõikidel oli juba kodus arvuti. Kas vanemate 
tööarvutid või siis isiklikud. Me sõitsime rongiga Tallinnast Keilasse ja 
millegipärast rongis hakkasime rääkima, et kuule, jube lahe oleks minna 
Ameerikasse. Ühel sõbral on tädi Ameerikas, et ta võtaks hea meelega vastu, aga 
kuidas me sinna saaksime. Minu isa töötas tollel hetkel Muuga 
sadamas\index{Muuga sadam}. Kuidagi sai räägitud, et põhimõtteliselt saaks ka 
laevaga minna. Mina ei tea, kust see tuli, noored poisid, me olime kuskil 
viisteist, kuusteist aastat vanad. Kodus rääkisin sellest ja kuidagi hakkas see 
pall veerema niimoodi, et üks hetk me olime USA saatkonnas viisat taotlemas, 
järgmisel hetkel isal oli juba kokku lepitud, et me saame minna kaasreisijateks 
Ameerika suurele kaubalaevale ja me sõitsime üle Atlandi ookeani laevaga Muuga 
sadamast New Orleansi. Seal pani laevakompanii meid lennuki peale ja sealt 
edasi lendasime JFK lennuväljale New Yorki, kus siis sõbra tädi meid vastu 
võttis. 

Kusjuures me saime laeva peal palka, sest laevafirmale oli palju odavam 
vormistada meid töötajateks. Muidu oleks olnud vaja tasuda suuri 
kindlustusmakseid. Selles mõttes täiesti kreisi.

\question{Sellist asja ma kuulen esimest korda! Kaua te sõitsite sinna?}

Kaks nädalat, umbes neliteist-viisteist päeva võttis see laevasõit üle
ookeani.

\question{Kas te midagi kasulikku ka seal laeva peal tegite või sõitsite 
lihtsalt kaasa?}

Midagi kasulikku me tegelikult ei teinud. Hängisime ohvitseride nii-öelda 
piirkonnas. Meile küll näidati, kuidas laev töötab, aga me ei teinud selles 
mõttes mingit kasulikku tööd, et me oleks  koristanud tekki või midagi sellist. 
Ei, me lihtsalt hängisime. Võib-olla heal juhul saime mingisugust sellist  
ülevaatlikku õpet, umbes nagu sa muuseumis käid, et näed, siin on see asi, siin 
mootoriruumis on sellised nupud. Loomulikult keegi meil midagi teha ei lasknud 
välja arvatud see, et võib-olla ava-ookeanil me saime rooli keerata ja natukene 
nii öelda laeva juhtida.

\question{Millega  te tagasi tulite?}

Tagasi me tulime lennukiga juba. Aga miks ma sellest üldse räägin on see, et 
kui Ameerika pinnale astusime, oli meil päris palju raha, saime ju laevast 
palka. Meil oli stiilis tuhat viissada dollarit, mis oli tolle aja kohta üüratu 
summa. Ja siis mina isiklikult kulutasin selle raha loomulikult ära arvutipoes. 
Ma tõin endale Ameerikast, ma arvan, et elu ühe kõige tähtsama riistapuu, 
milleks oli modem. 

Ja vot pärast seda läheks elu lahti. 

See oli mingisugune 2400 boodine modem, ma täpselt ei mäleta tüüpi enam. Lisaks 
tõin veel Sound Blaster 16\index{Sound Blaster} helikaardi, mis oli täiesti 
tipp tollel hetkel\sidenote{Sound Blaster oli Singapuri firma Creative 
Technology (tuntud USAs kui Creative Labs) helikaartide perekond. Need kaardid 
olid PC-maailmas \emph{de facto} standardiks, kuni Windows 95 vastavad liidesed 
standardiseeris ja PC audio kommoditiseerus}. See oli just välja tulnud, 
stiilis paar kuud varem. 

Üks asi, mille ma  hiljem avastasin, mis BBSides levisid, olid  helimoodulid,  
mul oli neid hästi palju, ma mingil hetkel kogusin neid. Ma arvan, et tollel 
ajastul paljud tegid seda. Need moodulid on  sellised helifailid, mida tollel 
ajal Amiga arvutites kokku pandi, koosnesid sämplitest. Põhimõtteliselt sul oli 
 mingisugune kaheksa \emph{track}i, kuhu sa siis miksid sämpleid niimoodi 
kokku, et sellest tekkis mingisugune \emph{meaningful} muusika. 

\question{Need liikusid siis BBSides?}

Jah. Loomulikult  sai üritatud ise ka neid teha, aga mul erilist muusikalist 
tausta ei ole, nii et sellest midagi välja ei tulnud.

\question{Tulid Ameerikast tagasi ja panid kohe BBSi püsti?}

Ei. Kui ma tulin Ameerikast tagasi, siis ma hakkasin avastama enda jaoks  BBSi 
maailma. Vanu asju üle vaadates selgus, et mu üks lemmik-BBS oli Dark 
Corner\index{BBS!Dark Corner}, mis oli Priit Kasesalu\index[ppl]{Kasesalu, 
Priit} veetud. Esmalt loomulikult sa üritad  alla laadida kõike, mida saad. 
Kõik on ju puhas kuld, kõik tarkvara, mida sul pole veel kunagi olnud ja nii 
edasi. Siis huvitav oli veel see, et tollel ajal eksisteeris selline asi nagu 
Kadaka Turg\index{Kadaka Turg}\sidenote{Aastal 1991 avatud ja 2002. aastal 
kaubanduskeskusega asendatud Mustamäel asunud turg oli küllalt metsik 
müügikeskkond, kust oli võimalik hankida kõike alates karvamütsidest ja 
Nõukogude aurahadest kõikvõimaliku piraatkaubani. Sisuliselt oli tegemist 
endise Nõukogude Liidu territooriumil toiminud varimajanduse väljundiga 
Eestisse. Turg oli turistide seas hinnas, parematel aegadel käisid sinna 
Tallinna Sadamast eribussid}, seal müüdi piraattarkvara. Ma arvan, et ma sain 
ka väga palju sealt tarkvara. BBSides, kusjuures minu mäletamist mööda 
tegelikult otseselt piraattarkvara väljas ei olnud. Seal oli rohkem sellist 
häkkimise stiilis tarkvara, aga mitte nii otseselt.

\question{Windowsi sealt vist keegi ei laadinud endale}

Jah, just, selliseid asju otse faililistides ei olnud, need olid taha 
nurkadesse ära peidetud. Aga seda ma mäletan küll, et mul oli kodus 
telefoniliin ja minu meelest ei olnud minutitasu tollel hetkel või see  
minutitasu oli nii odav. Igal juhul, mul oli kodus liin praktiliselt 
ööpäevaringselt kinni kogu aeg, sinna ei olnud võimalik helistada, sest minu 
arvuti helistas kogu aeg, laadis midagi alla.

\question{Kuidas \emph{bootstrap} toimus, kuidas sa teada said, mis numbri 
peale helistada?}

Väga võimalik, et see tuli stiilis .EXE ajakirjast\index{.EXE}, ma ei suuda 
seda enam meenutada. Aga kui sa oled ühte BBSi juba sisse pääsenud, siis kogu 
see maailm juba avaneb. Üks teema, mida BBS levitas, oli teiste BBSide 
aadressidega failid. Mingil hetkel Priit Kasesalu\index[ppl]{Kasesalu, Priit} 
pani kogu oma BBSi viimase versiooni veebi üles. Ma laadisin selle alla ja 
avastasin selle  ketta pealt, vaatasin just hiljuti läbi, oli  päris huvitav. 

\question{Mis seal siis leidus?}

Kõikvõimalikke häkkimisvahendeid, C-programmide näiteid, mingisuguseid raamatuid 
stiilis \emph{Terrorist Handbook}\sidenote{Ilmselt peab Kain silmas William 
Powelli raamatut \emph{The Anarchist Cookbook}. Vietnami sõja vastaste 
protestide laineharjal 1971. aastal USAs ilmunud (ja mitmel pool keelatud 
olnud) raamat sisaldas kõikvõimalikku vastandkultuuriga seotud sisu 
\emph{Thermite}-i ja LSD  valmistusõpetustest õpetusteni telefonisüsteemide 
murdmiseks. Raamat levis tekstifailina laialt ülikoolide serverite ja FidoNeti 
kaudu ning teda täiendati aja jooksul pidevalt: eriti kuulsad on anonüümse 
autori \enquote{\emph{The Jolly Roger}} täiendused} ja muud sarnased. Igasugune 
selline kraam, mis  noortele inimestele põnevust pakkus.

\question{Tulles korra veel sinu arvutihuvi alguse juure. Kas sa olid pigem 
seda tüüpi mees, kes mängis arvutiga, võrgutas arvutit või programmeeris 
arvutiga?}

Ma olen mõelnud selle üle, et kuidas see siis täpselt oli. Mulle tundub, et mul 
on olnud  mitu  ajajärku. Koduse 286 ja BBSide ajajärk oli pigem selline, et sa 
lihtsalt üritad endale sisse krahmata kõike, mida sa näed. Seal leidus ka 
arvutimänge, aga ma ei mäleta, et ma oleksin väga  kohutavalt mänginud. Siis, 
kui mul endal veel arvutit ei olnud,  sõbra juures me mängisime loomulikult 
kõik need ajad täis. Me ei tegelenud programmeerimisega, vaid pigem ikkagi 
mängimisega. Aga hiljem jäi see mängimine pigem taha taustale ja ikkagi 
üritasid aru saada, kuidas arvuti töötab. Näiteks üks teema, mis mind 
kohutavalt paelus, olid viirused. Mul oli alati kõige viimane viirusetõrje 
tarkvara. Ma usun, et mul oli selleks hetkeks juba ka mitu kõvaketast, ehk mul 
oli võimalus katsetada, mida viirused teevad. Ka  viiruse nii-öelda 
kollektsioone levitati BBSides. Ja siis sai uuritud, et kuidas selline asi 
põhimõtteliselt töötab. 

Ja siis järgmine ajastu tuli siis, kui ma avastasin enda jaoks 
Linuxi\index{OS!Linux}, samal ajal tuli ka Internet. Sealtsamast gümnaasiumi 
kõrvalt kaheteistkümnendas klassis ma sattusin tööle Riigi Elektriside 
Inspektsiooni\index{Riigi Elektriside Inspektsioon|see{Tehnilise Järelevalve 
Amet}}, mis on täna Tehnilise Järelevalve Amet\index{Tehnilise Järelevalve 
Amet}. Sattusin selliseks, noh,  patsiga või arvutipoisiks, mul patsi pole 
kunagi olnud. Olin selline arvutipoiss nagu ikka, kellele antakse mingisugused 
arvutid, et palun seadista nüüd need ära, tee seda ja teist.

\question{Kuidas sa sinna sattusid niimoodi kooli kõrvalt?}

Seesama sõber, töötas Pennus\index{Pennu} ja kuidagi tema kaudu tuli kontakt, 
et otsitakse sellist arvutitüüpi, kes oskab arvutitega midagi teha. Ma läksin 
kohale  ja kuidagi võeti tööle poole kohaga.

\question{Teil klassist ikka mitmed töötasid siis keskkooli ajal?}

Jah, meil mitmed töötasid. Üks klassivend näiteks töötas Keila Linnavalitsuse 
juures. Ta oli selline kõva programmeerija juba tollel ajal, kes kinkis mulle 
mu esimese programmeerimisraamatu C Programming Language\index{The C 
Programming Language}, Brian Kernighan and Dennis Ritchie.

\question{See on seesama salapärane väljaanne\sidenote{Kainil oli raamat 
jutuajamisel kaasas, selles puudus igasugune märge väljaandja ning trükkimise 
aja ning koha osas. Raamat levis (mälu järgi) juba nõukogude ajal, oli 
korralikult köidetud ja kopeeris isegi värvilist kaanekujundust täpselt}, mis 
minul oli}

Just, kui praegu minna Amazoni vaatama, siis täpselt selline raamat on müügil. 
See oli mu esimene programmeerimisraamat, aga see leidis kasutamist ikkagi 
aastaid hiljem, kui ma juba netit\index{neti.ee} tegin ja mul oli praktiline 
vajadus programmeerida otsingusüsteemi mis oleks suurema jõudlusega.

\question{Tahaks ikkagi aru saada, et kuidas teil juhtus selline klass olema, 
kus mitmed juba keskkooli ajal töötasid, kodudes olid arvutid ja inimesed 
programmeerisid}

Aga võib-olla see oligi see aeg, kus need arvutid ilmusidki rohkem koju ja 
kontorisse ja oli tohutu puudus sellisest nii-öelda oskusteabest. Vanemad 
inimesed võib-olla julgenud arvuteid veel kasutada ja noored julgesid nendega 
igasuguseid asju teha.

\question{Aga igas keskkooliklassis ei olnud see asi niimoodi, et neli-viis 
poissi töötasid arvutispetsialistidena, miks teil oli?}

Ma ei oska seda tagantjärgi öelda. Küll aga mäletan sellist huvitavat seika, et 
meil üks  eksam oli põhimõtteliselt arvutieksam ja see ei seisnenud meie puhul 
programmeerimises. Meie puhul tähendas see eksam, et  me sisustasime 
arvutiklassi.  Kool sai Tiigrihüppe või  mis iganes programmi kaudu peaaegu 
klassitäie arvuteid ja siis R-klassi \sidenote{R nagu Reaal} poiste ülesanne oli võrgutada see klass 
füüsiliselt Etherneti kaabliga, installeerida need arvutid, installeerida 
võrguserver, milleks oli  Linuxi server. Serveri \emph{task} jäi minu peale, 
kuna ma olin tollel hetkel kõige suurem Linuxi\index{OS!Linux} käpp võrreldes 
siis teiste poistega.  Meie kaheteistkümnenda klassi arvutieksam seisnes 
selles, et me põhimõtteliselt seadistasime koolile esimese PC klassi. See oli 
1995. aastal.

\question{Linux ei olnud selleks ajaks ju kuigi vana, kuidas sa selle otsa 
komistasid?}

Linuxi otsa ma komistasin siis, kui ma juba Riigi Elektriside 
Inspektsioonis\index{Riigi Elektriside Inspektsioon} töötasin. Kui ma sinna 
läksin, siis seal veel Internetti ei olnud, aga see tekkis sinna üsna pea, ma 
arvan, et mingisugune kuu-paar hiljem. See oli siis 1994. aasta lõpp.  
Elektriside Inspektsioon asus aadressil Ädala 4d, mis on ka siis selline 
legendaarne internetihoone.  Meie allkorrusel oli 
Valitsusside\index{Valitsusside}, kus toimetas Taavi Talvik\index[ppl]{Talvik, 
Taavi}. Ja Taavi andis Riigi Elektriside Inspektsioonile juhtmeotsa kätte, 
milleks oli  tolleaegne kümnemegabitine koaksiaalkaabel ja, palun, siin on 
Internet. See koaksiaalkaabel sai siis veetud kõikidesse ruumidesse. Ei mingeid 
hube ega täht-topoloogiat.

Siis ma avastasin enda jaoks Interneti. Koolis loomulikult poistele rääkisin, et 
see FidoNet on nüüd mingi  vana jama, aeglane, toimib üle modemi, et siin on 
üks palju uuem ja huvitavam asi. Kusjuures  Valitsussidest edasi olid kanalid 
üsna kiired. Mäletan, et Tartu Ülikooli FTP-serverist sai kahemegabitise 
kiirusega faile alla laadida, see oli  meeletu kiirus. Välislink oli 
loomulikult kuskil 64 või 128 kilobitti. 

\question{Mis sealt Tartu Ülikoolist siis tõmmata oli nii väga?}

Vot seda ma täpselt ei mäleta, aga ju seal midagi oli, sest mul on väga selgelt 
meeles kadri.ut.ee\index{Masinad!kadri.ut.ee}  FTP-server. 

Aga see Valitsusside\index{Valitsusside} ja see ethernetikaabel, see oli nagu 
huvitav. Tollel ajal, nagu teisedki on rääkinud, arvutiturvalisus ei olnud 
eriti teemaks. 

FidoNet oli selles mõttes tohutu kulla-auk, et ta avas loomulikult kõik oma  
\emph{echo} kanalid. Aga Internet avas meililistid ja kusagilt meililistist ma 
lugesin, et Anto Veldre\index[ppl]{Veldre, Anto} teeb 43. 
Keskkoolis\index{Koolid!Tallinna 43. Keskkool} mingisuguseid selliseid 
\emph{introduction} kursuseid. Tollel ajal ilmus ka ajakiri .EXE\index{.EXE}, 
kus Anto artikleid kirjutas. Ma ei mäleta, kumb kummale täpselt eelnes, aga 
igal juhul mäletan seda, et üks hetk olin ma seal 43. Keskkoolis, et 
\enquote{siin ma olen ma tahan teadmisi saada}. Seal olid koha peal veel tol 
ajal sellised legendaarsed koolipoisid nagu Indrek Mandre\index[ppl]{Mandre, 
Indrek} ja Heno Ivanov\index[ppl]{Ivanov, Heno} vist. Tagasi tulin ma sealt 
juba Slackware\index{Slackware} distributsiooni installeerimisflopidega, mida 
oli stiilis kuus tükki. Installeerimise protsess käis ikkagi niimoodi, et 
esimene flopi, teine flopi, kolmas-neljas ja nii edasi lõpuks sai 
installeeritud. 

\question{Aga siis jääb lisaks kõigele muule Anto peale ka Linuxi pisiku 
levitamine Eestis?}

Ma usun küll, jah. Ma arvan, et temal on väga suur roll selles osas, Linux 
Eestis käima läks. Igal juhul mina selle pisiku sealt sain. Kuna ma olin tollel 
hetkel juba mõnda aega Elektriside Inspektsioonis\index{Riigi Elektriside 
Inspektsioon} töötanud ja ka palka saanud, oli mul päris korralik nii-öelda  
taskuraha. Ja ma ehitasin  endale uue arvuti, 286 sai FidoNetis  kuskil maha 
müüdud (FidoNetis  käis ka suur riistvaraga hangeldamine) ja ehitasin endale 
486 arvuti. Kusjuures see ei olnud mitte lihtsalt 486 vaid 486DX4 
100 Mhz\sidenote{Inteli nomenklatuuris olid \enquote{DX} tähistusega protsessorid 
need, millel oli kiibil eraldi matemaatika koprotsessor, see andis olulise 
jõudlusvõidu}, see oli siis absoluutne tipp. 

See oli kõige kõvem 486, mis üldse kunagi tehti. See oli  juba siis see aeg, 
kui mul oli juba \emph{node} registreeritud. Sealtsamast Dark Corner 
BBSist\index{BBS!Dark Corner} sain ma esimese FidoNeti \emph{point}i, kus ma 
pääsesin ligi FidoNeti uudistekanalitele. Mingil hetkel tundus, et aga palju 
ägedam oleks \emph{node}. Sai kirjutatud Tarmo Mamersile\index[ppl]{Mamers, 
Tarmo} (sest ta oli Eesti regiooni \emph{manager}, tema neid aadresse jagas), 
et kas oleks võimalik registreerida \emph{node} number kuuskümmend kuus ja 
Tarmo vastas, et \enquote{tehtud}. Sealt edasi oli mul \emph{node}, mis mõnda 
aega eksisteeris mul kodus. 

Aga siis mingil hetkel sai seadistatud Elektriside Inspektsioonis 
Linuxi\index{OS!Linux} server, sest meil on praktiline vajadus serveerida 
printerit, faksi ja faile. Ehk siis sai nurka tekitatud Linuxi server, kes siis 
šeeris faile üle Samba teenuse ja võttis vastu fakse. Mul õnnestus ka enda 
FidoNeti \emph{node} sellesse samasse serverisse sokutada. Kui muidu FidoNeti 
tarkvara oli MS-DOSi peal, siis  oli ka alternatiiv Unixitele 
Ifmaili\index{Ifmail} nimelise programmi näol.

\question{Räägime korra sellest riigiametist. Miks seal üldse Internetti vaja 
oli? Kas see oli puhas sinu huvi või nad tegid midagi kasulikku ka sellega?}

Jah, selleks oli praktiline vajadus  olemas, sellepärast et Elektriside 
Inspektsioon\index{Riigi Elektriside Inspektsioon} tegi tihedat koostööd 
ITUga\sidenote{\emph{International Telecommunications Union (ITU)}}, kes siis 
juhib kõiki neid sageduste jaotust ja protokolle ja  kõike muud sellist. 
Nendega oli inspektsioonil tihe kirjavahetus ja ma arvan, et e-maili teel. Ma 
ei suuda meenutada, kuidas see meilivahetus enne kaabliga Interneti käis, aga  
pärast  seesama Linuxi masin oli ka loomulikult mailiserver. Sellest hetkest 
tekkis ka meil oma domeen nimega rei.ee. Või äkki domeen oli juba varem olemas, 
igal juhul pärast Linuxi server tuli, hakkas ta rei.ee domeeni  kirju vastu 
võtma ja ka mina sain endale isikliku esimese ülilühikese e-maili aadressi, mis 
oli tollel ajal ülikõva, kain@rei.ee\sidenote{Lühikesed meili ja muud aadressid 
olid staatusesümboliks, need näitasid kuulumist kas serveri-administraatorite 
kõrgesse kasti või neile väga lähedasse ringkonda}.

\question{Ehk sa avastasid ennast suhteliselt õrnas eas Linuxi ruuduna 
riigiasutuses?}

Just. Miks ma seda kümne megabitist ethernetikaablit mainisin oli see, et seal 
kõik liiklus oli ju näha. Ja kui ma külastasin siis Anto 
Veldre\index[ppl]{Veldre, Anto} arvutiklassi 43. 
Keskkoolis\index{Koolid!Tallinna 43. Keskkool},  jäi mulle sealt üks asi elu 
lõpuni meelde. Kuidas kõik need noored tüübid, kes seal siis 
siil.edu.ee\index{Masinad!siil.edu.ee} nimelise SCO\index{OS!SCO UNIX} masina 
taga istusid, oli tohutu kõvad häkkerid. Nad demonstreerisid, mida nad siin 
teevad, näitasid, et kuidas nad  suudavad \emph{exploit}ida mingisuguseid Tartu 
Ülikoolis olevaid masinaid, mingeid professoreid seal jälgida ja nii edasi. See 
avaldas mulle nii kohutavalt mulje, et mind hakkas lisaks sellele varasemale 
viiruste teemale huvitama ka arvutiturvalisus.

Ma arvan, et see on esimest korda, kui ma avalikult sellest räägin aga ma 
\emph{sniff}isin loomulikult ka meie  võrku ja \emph{sniff}isin, mida siis 
Valitsusside\index{Valitsusside} insenerid seal tegid. Ega seal vahel midagi 
olnud,  sellesama  kaabli otsas oli kaks ametit: oli Riigi Elektriside 
Inspektsioon\index{Riigi Elektriside Inspektsioon} kõigi oma töötajatega ja 
\emph{Valitsusside}. Et kui Valitsusside insenerid käisid oma ruutereid või 
keskjaamu üle telneti konfimas, siis loomulikult see liiklus levis lahtise 
tekstina võrgus. Seda oli päris huvitav jälgida, mida nad siis seal teevad.  
Loomulikult ma seda kunagi pahatahtlikult ära ei kasutanud, see oli lihtsalt 
selline puhas  noore mehe huvi.

\question{Eks see seik iseloomustab suurepäraselt toonast aega. Ma usun, et kui 
praegu keegi lahtise traadi peal lahtist kanalit kasutaks, korraldataks umbes 
poole tunni jooksul mingi jama}

Jah, ma arvan ka. See mulje jäi niivõrd meelde, et kogu see võrguvärk on 
niivõrd ebaturvaline, et nii kui Soomest keegi härrasmees\sidenote{Tatu 
Ylönen\index[ppl]{Ylönen, Tatu}, Helsingi tehnoloogiaülikooli teadlane} tegi 
\emph{secure shell}i esimese versiooni, siis ma hakkasin seda praktiliselt kohe 
kasutama, kui ma sellest teada sain. 

Veel Keila Gümnaasiumi\index{Koolid!Keila Gümnaasium} juurde tagasi tulles. 
Pärast seda, kui ma kooli olin juba ära lõpetanud,  jäin ma edasi 
administreerima serverit, mis sinna maha jäi. Nagu tollel ajal ikka,  pidid 
kõikidel Unixi masinatel  olema ilusad nimed. Kodus rääkisin sellest teemast ja 
isa pakkus välja, et aga \enquote{kratt} oleks jube hea nimi. Ja praegu 
vaatasin nimeserverist järgi, et siiamaani on Keila Gümnaasiumis oleva serveri 
nimi kratt.keila.edu.ee\index{Masinad!kratt.keila.edu.ee}.

\question{Hakka seda nime siis takkajärgi muutma. Loodetavasti riistvara ei ole 
päris seesama?}

Riistvara kindlasti ei ole seesama, sest seda koolimaja füüsiliselt enam alles 
ei ole. Keilas on nüüd uus koolimaja, kus mu enda lapsed käivad, sest ma elan 
siiamaani Keilas. Aga aadress on olemas.

\question{Seepärast ongi asjade nimetamine oluline, et need nimed võivad pikalt 
kesta}

Just. FidoNeti ajast veel üks huvitav \emph{impact} minu meelest, mis mul  
hiljem on väga kasulikuks osutunud oli see, et modemid töötasid
AT-käsustikuga\index{AT-käsustik}\sidenote{Hayes käsustik, tuntud ka kui
AT-käsustik, on käsukeel, mille Dennis Hayes\index[ppl]{Hayes, Dennis} lõi 1981. 
aastal omanimelise ettevõtte 300-boodise Smartmodem modemi juhtimise tarbeks}. 
Too käsustik oli selles mõttes universaalne asi, et seda kasutati hiljem 
erinevates muudes rakendustes. Loomulikult BBSidesse sissehelistamine toimus 
lihtsa terminaliga ehk et sa pidid nagu häkker käsustikku teadma. Enne 
helistamist pidi sisestama  ATDT, telefoninumber ja nii edasi, võib-olla veel 
seadistama protokolli. Loomulikult, tolleaegsed inimesed teavad täpselt, 
missuguse protokolliga vilistab  memcpy intro. See oli ka võib-olla selline 
asi, mis edaspidiselt  mõnes mõttes kaasa aitas.

\question{BBSil oli kliendisoft ka?}

Ei olnud. Helistasid terminaliga, kliendisoft oli ainult FidoNetil. Oli soft 
nimega FrontDoor, mis helistas, ja oli soft, mis pakkis kokku FidoNeti 
\emph{echo}d ja  saatis  selle paki edasi. Aga BBSil kui sellisel ei olnud 
kliendisofti. Läksid lihtsalt \emph{telnet}iga külge ja hakkasid seal edasi 
tegutsema.

\question{See läheb minu mäletamisega kokku küll. Oleks ju olnud loogiline, et 
keegi oleks mingisuguse tarkvara teinud BBSide ette, \emph{cache} jaoks 
näiteks?}

Jah, kui vaadata, mis Ameerikamaal toimus, kus siis olid need nii-öelda 
\emph{Online Service Provider}id  nagu AOL\index{AOL} ja 
CompuServe\index{CompuServe}\sidenote{Interneti-eelsel ajal domineerisid USA 
turul agressiivsete turunduskampaaniatega (ühel hetkel oli pool \emph{kõigist} 
toodetud CDdest AOLi logoga) teenusepakkujad, kes pakkusid kummalist segu 
BBS-laadsetest ja Interneti-teenustest. Neist suurimad olid CompuServe, Prodigy 
ja America Online} ja nii edasi, siis neil oli tarkvara. Ma mäletan seda, et 
kui ma USAs modemi  olin ostnud, siis loomulikult noorte poistena meil tuli 
seda proovida. Ja, kujuta ette, meil oli julgus kruvikeerajaga lahti keerata 
üks selline suur soliidne arvuti, see oli vist Computer 2000 või mis iganes 
tolleaegne  selline hästi kõva valge PC bränd oli. Sõbra tädimees oli 
arhitekt, tal oli selline väike arhitektibüroo, ning meil oli julgus  
omavoliliselt kruvikeerajaga  lahti keerata üks nende suur \emph{tower} ja 
sinna sisse proovida seda sisemist modemit. Modemiga oli kaasas kas 
CompuServe'i või mingi muu sarnase teenuse CD plaat või flopi, äkki mäletan 
valesti. Ja siis sai helistatud Ameerika BBSi.  

\question{Kui sa nendes BBSides kolasid, kas sulle midagi muud peale tarkvara 
ka silma jäi? Raamatuid ja MODe sa mainisid?}

Raamatud mind eriti  tollel hetkel ei köitnud, BBSidest mina ikkagi laadisin 
peaasjalikult tarkvara ja siis muusika MODe. Aga kogu infovoog 
tuli FidoNetist. FidoNet oli minu jaoks täiesti puhas kulla-auk. Nagu varem 
mainisin,  mul ei ole olnud ligipääsu sellistesse teadusasutustesse või 
ülikoolidesse,  mul ei ole olnud piltlikult öeldes mentorit. Meil oli kamp 
poisse, kes omavahel  infot vahetasid. Meil ei olnud nagu sellist vanemat, kes 
teab, kuidas asjad käivad,  kõik käis katse-eksituse meetodil.

\question{Isegi hästi, et te kuidagi paha peale ei läinud selle kambaga. Noored 
poisid, tont teab, mida hakkavad tegema}

Ju siis me olime piisavalt mõistlikud. Ma arvan, et sellest ajast saadik on mul 
selline ise õppimise  oskus. Võib-olla see sai ka saatuslikuks, miks ma 
Tehnikaülikoolis ei suutnud väga kaua õppida,  ainult ühe aasta nagu tollel 
ajal võib-olla paljudel teistelgi kombeks.

Peale gümnaasiumi ma läksin kohe Tehnikaülikooli informaatikasse\index{Tallinna 
Tehnikaülikool!Informaatika}, aga kuna ma juba tollel hetkel töötasin, siis 
igasuguseid huvipakkuvaid projekte oli  palju kõrval. Mina eeldasin seda, et 
nüüd ma saan hakata programmeerimist ja igasugust muud sellist huvitavat asja 
õppima, aga siis tuli välja, et ei,  sa pead kõigepealt läbima füüsikad ja 
matemaatikad. Matemaatikast mul oli juba nagu natukene \enquote{kopp ees}, kuna 
meie meil oli selline väga püüdlik matemaatikaõpetaja gümnaasiumi ajal. Me 
tegelesime väga põhjaliku matemaatikaga, mingit sisse saamise probleemi 
Tehnikaülikoolis  absoluutselt ei olnud,  matemaatika eksamist lihtsalt 
lendasid läbi.

Ja nii see ülikool järgmisel aastal pooleli jäi.

\question{Kuidas sul kaitseväega on?}

Siis tuligi Kaitsevägi\index{Kaitsevägi}. Kui ülikoolis ei ole, siis varem või 
hiljem leitakse sind üles. Aga  Kaitseväkke ma läksin 1997. aasta suvel, ehk et 
ma olin siis juba aasta otsa Netit teinud. 

Ahjaa, et kuidas ma sinna sattusin. Töö Elektriside Inspektsioonis\index{Riigi 
Elektriside Inspektsioon}  hakkas natuke nagu ära tüütama. Nagu ikka,  tahad 
edasi areneda. Hakkasin otsima, et tahaks kuhugi  huvitavasse kohta tööle 
minna.  Mul mingil hetkel oli soov kindla peale töötada arvutifirmas, sest kuhu 
sa ikka lähed. Arvutifirmasse, seal olen kindel, et saan arvutitele väga 
lähedale.

Vanu \emph{backup}e läbi kammides jäi silma, mingil hetkel kandideerisin isegi 
Helmesesse\index{Helmes}, aga sinna ma ei saanud. Õnneks, tagantjärgi ma 
mõtlen. Keskkooli ja ülikooli vahelisel ajal suvel ma töötasin poolteist kuud 
Tõnu Samueli\index[ppl]{Samuel, Tõnu} IT-firmas nimega Eramees\index{Eramees} 
ja ma istusin samale kohale, kust oli just lahkunud Pronto\index[ppl]{Pronto}.  
 Tõnu ütles mulle, et Pronto müüs siin  neid Gravis 
Ultrasound\sidenote{Üheksakümnendatel väga populaarsed helikaardid, mis 
esimesena omataoliste hulgas suutsid toimetada päris instrumentide 
sämplingutega} kaarte, et kuule, hakka nüüd sina sellega tegelema. Aga ma olin 
noor koolipoiss, ma ei  ei teadnud kaubandusest mitte essugi. Ma ma ei usu, et 
minust seal ettevõttes erilist kasu oli muidu kui nii-öelda patsiga poisist. 

\question{Ära ütle, päris mitmed inimesed kuni Tarmo Talini\index[ppl]{Tali, 
Tarmo} välja on mingil hetkel tegelenud müügitööga ja seejuures üldse mitte 
halvasti}

Eramehes üks asi, mis mul on veel eredalt meeles on, et Tõnu BBS oli siis 
kontoris. Kontor asus Eesti Talleksi majas, Mustamäe tee 1 vist, kui ma ei 
eksi. Ja BBS oli põhimõtteliselt  laiali laotatud arvutijupid  aknalaual. Seal 
oli siis USR Courier\index{US Robotics Courier} modem\sidenote{US Roboticsi 
ülemise otsa Courier tooteliin oli oma töökindluse ja suurte kiiruste tõttu 
BBSide ja varaste Internetipakkujate lemmik, ka Eestis},  emaplaat, toiteplokk  
ja nii edasi, lihtsalt hunnik juppe ja juhtmed, mis oli aknalauale laiali 
laotatud. Ja see oli siis Tõnu BBS või \emph{node}.

Pärast Erameest ma kandideerisin Estpak Datasse\index{Estpak Data}, sest mulle 
tundus, et ISP, et see on tegelikult veel huvitavam asi, sellepärast et nad 
tegelevad ju Internetiga.

\question{Kas Estpak oli tol ajal juba Eesti Telefoni oma või oli veel eraldi?}

Ta oli tollel ajal eraldi. Kui õieti mäletan, siis Estpak Data omanikuks oli 
siis Eesti Telekom\sidenote{Eesti Telekom pika nimega Riigiettevõte Eesti 
Telekommunikatsioonid oli Teede- ja Sideministeeriumi haldusalas töötav 
\emph{holding}-ettevõte, mis valdas Eesti Telefoni, Eesti Mobiiltelefoni, Eesti 
Kaugotsingu, EsData, Estpak Data ja TeleMedia aktsiaid. Hiljem viidi ettevõte 
börsile ja sealtkaudu sai tema ainuomanikuks Telia}, mitte  Eesti Telefon, ta 
oli täiesti eraldiseisev ettevõte Eesti Telefonist. Huvitaval kombel kellelgi 
oli tulnud selline idee, et meil on kuidagi vaja edendada veebi  
virtuaalhostimist. Keegi oli välja mõeldud neti.ee\index{neti.ee} nimelise 
domeeni ja selle domeeni alt siis üritati müüa sellist traditsioonilist 
veebihostingut. Tollel ajal ta veel traditsiooniline ei olnud, aga ütleme, et 
siis tänapäeva mõistes. Ja  Estpak Data palkas mind kui nii öelda webmasterit, 
kes pidi hoolitsema veebi hostinguserveri ja teenuse eest. Ja siis muu seas oli 
neil selline idee, et kuidas me seda veebi hostingu äri ikka muud moodi 
edendame, kui meil on vaja mingit kataloogi. Inimesed peavad ju leidma üles 
need veebilehed, mida  kliendid sinna panevad.

\question{Kas tol ajal Meediamaa oli juba olemas?}

Meediamaa\index{Meediamaa} startis umbes samal ajal. Enne seda oli olemas  
Eesti veebisaitide nimekiri, mis oli nlibi ehk siis 
Rahvusraamatukogu\index{Rahvusraamatukogu} domeenis, kus Toomas 
Mölder\index[ppl]{Mölder, Toomas} tegutses. Ja Toomas Mölder kolis, ma arvan, 
sellesama nimekirja Meediamaasse ja sealt www.ee\index{www.ee}'sse. Kuna 
Meediamaa üks tegelane oli Tarvi Martens\index[ppl]{Martens, Tarvi}, siis neil 
õnnestus kuidagi EENetilt\index{EENet} välja meelitada domeen nimega 
www.ee\sidenote{Alates oma asutamisest 1993. aastal kuni 2013. aastani oli 
EENet .ee domeeni registrar ja sellisena rakendas mitmeid suhteliselt rangeid 
reegleid. Näiteks oli domeeni registreerimine küll tasuta, kuid ühel 
organisatsioonil tohtis olla vaid üks domeen}. Ma arvan, et mitte kellelegi 
teisele kui Tarvile ei oleks sellist domeeni elu sees välja antud.

\question{Seda ma kujutan ette küll. Kas sa seda kataloogi tegid siis käsitsi 
alguses?}

Jah, alguses alguses sai seda kataloogi käsitsi tehtud. Ta oligi selline väga 
algeline ja puine. Aga asi hakkas lendama siis, kui kui ma kutsusin kataloogi 
puhul endale appi Jaanus Vainu\index[ppl]{Vainu, Jaanus}, kellega ma olin kokku 
saanud Riigi Elektriside Inspektsioonis\index{Riigi Elektriside Inspektsioon}. 
Jaanus on ka omamoodi huvitav tegelane. Elektriside Inspektsioonis tema mõtles 
välja kogu meie FM 108 sageduse plaani, ehk kõik Eesti raadiojaamade 
sagedusnumbrid on tema tehtud. Nõukogude ajal oli meil teistsugune FM 
sagedusala --- kuidas saab nii, et sa saad poest osta raadio, millega saab 
välismaa raadiojaama kuulata. See ei sobinud kuidagi, Eesti Vabariigi alguses 
koliti Lääne sagedustele üle. Jaanus oli üks nendest, kes käis mööda Eestit  
mõõtmas ja tegi sagedusplaani. Tal oli väga detailselt Corel Draw's\index{Corel 
Draw} joonistatud kõik need nii-öelda sagedusringid Eesti kaardi peale. Eesmärk 
oli  planeerida sagedused niimoodi, et üle Eesti saatjatel oleksid sagedused, 
millel on võimalikult vähe häireid naaberriikidega ja omavahel.  

\question{Kogu seda teadust tehti Corel Draw abil?}

Jah. Jaanus on selline tohutu pedant,  tohutu  töövõimega katalogiseerija. 
Tema enda isiklik huvi on \emph{bluegrass}. Mäletan seda, et tema oli esimene 
inimene, keda mina tean, kes välismaalt e-poest asju tellis.  Tema tellis 
CDNow'st\sidenote{CDNow oli 1994. aastal asutatud Interneti-põhine muusikamüüja, 
kes paraku esimest dot-com-mulli üle ei elanud ja sajandivahetusel uksed 
sulges}  plaate endale. Mina väga imestasin, et kuidas selline asi üldse 
võimalik on. Et ta tellib kuskilt, Jumal teab kust ja tulebki pakiga kohale CD 
muusikaga.

\question{Jaa, isegi üheksakümnendate lõpus oli Amazonist raamatute tellimine 
suhteliselt eksootiline tegevus. Aga mis hetkel ja kuidas te 
neti.ee\index{neti.ee} ära automatiseerisite?}

See meie tandem Jaanusega töötas selles mõttes ülihästi, et mina olin  
programmeerija ja arendasin tarkvara ja Jaanus oli  katalogiseerija.  Kui Jaanus 
selle projektiga liitus, siis võiks öelda, et projekt hakkas täielikult 
lendama. Ma arvan, et meil läks võib-olla paar kuud aega, kui me olime 
Meediamaast\index{Meediamaa} igatpidi kõikide näitajate poolest mööda läinud. 
Me olime tollel ajal võib-olla isegi natukene liiga ebaviisakad noored mehed. 
Näiteks me reklaamisime netit spämmides, tegime  ühe korra sellise 
masspostituse, saates kõikvõimalikele meiliaadressidele teate, et \enquote{nüüd 
on selline huvitav teenus olemas nagu neti.ee, tulge, külastage}. Midagi 
sarnast. Kusjuures huvitav on see, et kui ma vaatasin enda \emph{backupe}, siis 
ma nimetasin enda \emph{crawlerit} Nuhiks, seda otsingurobotit, kes mööda lehti 
ringi kolab. 

Ja huvitaval kombel ma olin selle Nuhi programmeerimist alustanud juba mitu kuud 
varem ehk nagu nagu miski oleks suunanud mind sellele teele, et seda võib vaja 
minna. Ja  otsingumootoreid ma olin ka natuke varem teinud. Kui ma pärast 
Erameest  ülikooli läksin, siis  üks sealt saadud kontaktidest kutsus mind 
tegema ühte ärikataloogi sarnast teenust,  mille pealkiri oli Bartanet. See 
asus EsData\index{EsData} serveris, oli mingi Suni server Akadeemia tee 21  
teisel korrusel, samas majas, kus me hetkel viibime. Ja selles Suni serveris, 
ma ei tea, mis asjaoludel, aga ma millegipärast sain seal teha FTP-serverite 
otsingut. Ma panin seal püsti otsinguteenuse nimega Filerix, mis töötas umbes 
kolm-neli kuud, mille ainukeseks sisuks oli see, et ta võimaldas väga hõlpsasti 
faile üles leida igasugustest kohalikest FTP \emph{mirror}itest. Tollel ajal  
Marek Tiits\index[ppl]{Tiits, Marek} IBSist\index{Institute of Baltic Studies} 
hostis sellist asja nagu TuCows\sidenote{TuCows (\emph{The Ultimate Collection 
Of Winsock Software}) keskendus oma algusaegadel tasuta tarkvarale. Kuna 
Interneti kiirus sõltus veel väga suurel määral geograafiast, opereeris 
ettevõte skeemi, kus  huvilised võisid jooksutada TuCows.com lehekülje 
lokaalseid peegleid. Ühte sellist Marek pidaski.}. Minu  otsingumootor 
võimaldas hõlpsasti failinimede järgi üles leida tarkvara tolleaegsele Windows 
95'le, vanadele Windowsidele ja nii edasi. Tollest pooleaastasest projektist nii-öelda kõrvalprojektina ma tegin failiotsingut.

\question{Suure hulga failide indekseerimine ei ole enam päris naljaasi ja 
eeldab programmeerimisoskust. Kust sa selle üles oled korjanud?}

Tollel hetkel ma oskasin programmeerida Perli\index{Keeled!Perl} ja siis kõike 
seda, mis  Unixi \emph{shell}is saada on. See tuligi  sellest ajaperioodist, 
kui ma uurisin, mis on nii-öelda Unixil  kõhus.

\question{Ise korjasidki üles selle algoritmika ja muu sellise?}

Jah, mis puudutab veebi \emph{crawl}imist,  siis jah, selle peale tuli juba 
mõelda.

\question{Puhtalt konteksti pärast, kaua su \emph{crawler}il aega läks, et 
kogu Eesti veeb üle käia?}

Ma arvan, et see oli mingi stiilis ööpäev või midagi sellist, sest veeb oli 
tollel ajal väga väike. Ma täpselt pole vaadanud, aga ma usun, et selle 
kataloogi suurus oli võib-olla paar tuhat linki ja mitte rohkem tollel ajal. Ja 
keskmine koduleht oli ka selline kolm kuni viis lehekülge, et see ei  olnud 
eriline teema. Huvitavamaks läks pärast, siis kui see linkide hulk juba 
miljonitesse läks, siis mingil hetkel oli ikka selline \emph{crawler}, mis 
töötas paralleelselt  paljudes \emph{thread}ides ja nii. Aga noh, see oli kõik 
selline loomulik evolutsioon. 

Aga jah, ma mäletan tegelikult, et miks ma arvan, et miks mind Estpak 
Datasse\index{Estpak Data} tööle võeti. Ühe sellise kõrvalprojektina ma olin 
teinud HTML-i tutvustuse. Ma arvan, et mul oli vist koolis olnud vaja seda 
kellelegi õpetada. Ehk siis gümnaasiumis tollel perioodil, kui ma siis Keila 
seda serverit administreerisin. Ja siis mulle tundus, et ma seda ikka õpetan, 
mingit eestikeelset materjali pole ja siis ma tegin ühe esimese  eestikeelse 
HTML-i tutvustuse, mis võttis läbi kõiki üksikuid elemendid. 

\question{Millegi pärast tuleb see maru tuttav ette, ma arvan, et ma olen sealt 
mingeid asju otsinud}

Kusjuures seesama HTML tutvustus on sellel samal aadressil täna ka üleval ja ma 
olen üsna kindel, et see on üks kõige vanemaid veebilehti, mis leidub täna 
Eesti veebiruumis, mis on originaalkujul originaalaadressil. 

\question{Mis aastast see on?}

1996. Ja siis veel ühe projektina ma olin teinud veebi pokkeri, sellise  
veebipõhise mängu. Selles mõttes,  ei saa öelda, et  mul pole kunagi huvi olnud 
ka mänge teha, aga ma olen  rohkem  oma elus programmeerinud nii-öelda 
veebi-asju, kui \emph{desktop}is või masinas töötavaid rakendusi. 

Nende teadmiste baasil mind sinna Estpaki  siis tööle võeti. Tõenäoliselt ma  
näitasingi seda, et vaadake, ma olen teinud sellise veebi pokkerimängu, ma olen 
teinud HTML-i tutvustuse ja võib-olla ma rääkisin ka seda, et ma olen selle 
\emph{crawler}i teinud. Igal juhul mind võeti sinna tööle ja ma sain jätkata 
sellesama koha pealt, kus ma juba olin.

\question{Kes teil seda toote poolt tegi, või polnud niisugust mõistet, nagu 
tootejuht?}

Ei olnudki. Piltlikult öeldes pandi mind  istuma, et palun istu siia ja tee. 
Tegelikult  see oli ikkagi läbi mõeldud mõnes mõttes. Estpak Data\index{Estpak 
Data} tegi koostööd ühe reklaamiagentuuriga, mis rentis ruume Kullo majas 
Mustamäe teel. Nii et tegelikult minu füüsiline töökoht  asus selles 
reklaamiagentuuris Kullo majas. Minul oli arvuti, millel oli püsiühendus 19.2 
kilobitti sekundis ja sealt ma töötasin. Noh, noore mehena nagu ikka, et sind 
ei huvita, kuidas rahad liiguvad ja nii edasi,  sind huvitab ainult see 
tehniline pool. Idee siis seisnes selles, et reklaamiagentuur aitab 
potentsiaalsetel Estpak Data klientidel teha kodulehti, aitab teha neile 
reklaami ja nii edasi, umbes selline kokkulepe oli. PRC Nord Decor\index{PRC Nord Decor} 
oli tolle agentuuri nimi, ma ei tea, kas see kellelegi midagi ütleb. Aga, aga 
selles mõttes oli huvitav, et üks kolleeg, kes Nord Decoris töötas, oli kunagise 
OK Jutuka\sidenote{OK Jutukas oli esimene tõeliselt massidesse läinud 
sotsiaalvõrgustiku laadne rakendus Eestis. Jututube - kohti, kus sai üle 
telneti kaaskodanikega suhelda - oli  veel, aga 1996. aastal käivitatud OK oli 
üks esimesi veebipõhiseid jutukaid ja tõenäoliselt omataolistest siinkandis 
suurim. Üheaegselt lobises omavahel kuni 300 inimest ja jutuka esimese 
aastapäeva pidu kajastas isegi toonane Päevaleht.} üks asutajatest. Mitte  
Kaupo Kalda\index[ppl]{Kalda, Kaupo}, aga Tiit Sermann\index[ppl]{Sermann, 
Tiit}. Kusjuures oligi nagu naljakas, et tema alias oli Ott \sidenote{OK Jutuka nimi tulenes siis asutajate nimedest: Ott ja Kaups}, aga tegelikult tema 
päris nimi oli Tiit. Lihtsalt selline huvitav asi. Kuidagi tundub, et kogu see 
maailm oli tollel ajal nii pisikene, et kui sa natukene selles maailmas ringi 
käisid, siis sa puutusid paratamatult kuidagi kõikide nende inimestega kokku, 
kes tollel ajal toimetasid.

\question{Räägi korra palun sellest, kuidas te Hoti tegite?}

Ja,see oli tegelikult ka päris huvitav. Kaitseväest tagasi tulles oli Eesti 
Telefon\index{Eesti Telefon}  Estpak Data\index{Estpak Data} ära söönud, Estpak 
Data lakkas olemast. Mingil hetkel ma töötasin siis Lasnamäel Koorti 15, kus 
Estpak Data enne oli, vana Eesti Telefoni maja, aga siis õite pea koliti meid 
sealt siis päris Eesti Telefoni muudesse ruumidesse ära. Ma olin Eesti Telefoni 
sellises allüksuses, mille pealkiri oli Teleteenuste Arendus.  Eesti Telefon 
oli teleteenuseid pakkuv ettevõte ja too üksus oli siis Eesti Telefoni 
arendusüksus, kelle eesmärk oligi välja töötada uusi teenuseid. Ja siis oma 
neti.ee tegemisega me sinna sattusime. 

Kontoriruumi jagasin ma ühe teise noormehega, kes arendas 
sissehelistamisteenust. Ja  huvitaval kombel meil vedeles kapi peal üks pisike 
Ascendi sissehelistamiskeskus, seal väga palju liine ei olnud. Ma küsisin, et 
kas ma võin seda uurida.

\question{See oli siis mingi tükk riistvara? Seal käisid tavalised modemid 
külge või oli ta juba valmis lahendus?}

Ei, ta oligi \emph{dedicated} sissehelistamiskeskus, et sa põhimõtteliselt 
installeerisid ta \emph{rack}i, panid  juhtmed külge ja ta hakkaski numbreid 
kuulama ja teenust osutama. Aga miks ma seda räägin, on see, et tolle keskuse 
uurimise käigus ma avastasin selle, et  sissehelistamiskeskus autendib ennast  
vastu sellist autentimisserverit nagu Radius. Sealt edasi uurisin, et mis asi 
see Radius on, sain teada, et see on \emph{dictionary}-põhine protokoll, 
üldsegi mitte keeruline ja ma programmeerisin siis Radiuse serveri, kes  suutis 
sissehelistamiskeskust juhtida. Avastasin, et selle sissehelistamiskeskusse 
\emph{firmware} võimaldab igasuguseid huvitavaid asju, mis tundusid olevat nagu 
seni kasuta. Näiteks see, et sa võid kohe Radiuse serverist öelda 
sissehelistamiskeskusele, kui kaua see kasutaja võib ühenduses olla. Ja sellest 
teadmisest näiteks sündis selline toode nagu Atlas Surf\index{Atlas Surf}, mida 
Eesti Telefon \emph{prepaid} Internetina \sidenote{sarnane kontseptsioon nagu mobiili kõnekaardid} müüs. Ühe sõnaga, see toode sündis 
puhtalt sellest, et mina häkkisin  seda väikest sissehelistamiskeskust, mis oli 
tegelikult üldse mõeldud  mobiilidega sisse helistamiseks. Ta toetas sellist 
huvitavat protokolli nagu V.35. Paljud pole sellest ilmselt mitte kunagi 
kuulnud, aga see oli selline \emph{wideband} protokoll, mis töötas üle GSMi. 
Kui sul oli selline GSM telefon, mida sai arvutiga ühendada, siis ta võimaldas 
sisse helistada selle V.35 protokolliga ja sa said veidi suurema kiiruse kui 
tavalist modemit vilistades üle  mobiili. 


Võib olla korraks hüppan natuke tulevikku. Oli aasta kaks tuhat, kõik mäletavad 
Y2K\sidenote{\enquote{Sinu lapselapsed neavad päeva, mil sa otsustasid oma 
koodi optimeerida}. Kuna pikka aega leiti aastaarvu hoidmiseks kahekohaline 
number piisav olevat, tehti sajandivahetuse paiku üüratus koguses tööd ja raha 
tagamaks, et aasta 2000 ei oleks arvutite arvates võrdne aastaga 1900. Vaata ka 
\enquote{Aasta 2038 probleem}.} probleem, kohutavalt hirmus, sest arvutid 
lähevad katki sest nende kell  lakkab töötamast ja nii. Ja ka Eesti Telefonis 
kardeti seda,  \emph{legacy} süsteeme oli tohutu palju. Kõik insenerid, kes 
olid mingisuguste süsteemidega seotud, pidid jääma  valvesse. Ma ei tea kuidas, 
minul õnnestus sellest ära nihverdada niimoodi, et tol hetkel ma olin Soomes 
suusatamas, sõpradega lumelauaga mäest alla laskmas. Stiilis paar päeva enne 
aastavahetust tuleb mulle  klienditeenindusest kõne, et kuuled, et nüüd sisse 
helistada enam ei saa, et mingi jama on. Läksin siis autoni, mul oli läptop 
kaasas. Olles Soome Vabariigis, panin telefoni läptopile järgi, helistasin 
sellesse meie enda privaatkeskusse sisse sellesama V.35  protokolliga ja 
hakkasin  vaatama, et mille pärast siis Hoti kliendid sisse helistada ei saa. 
Tuli välja, et keegi oli veel viimasel hetkel mingisuguse turvapaiga peale 
laadinud Y2K hirmus ja see muutis natukene seda teadet, mis Radiuse serverile 
saadeti ja siis Radiuse server läks selle peale katki, kuna talle tuli tundmatu 
sisuga \emph{dictionary}.

Surfist edasi tekkis selline olukord, kus Eesti Telefoni kontsessioonileping 
oli juba lõppenud või lõppemas, ja turule tuli Tele2\index{Tele2} Rootsist. 
Tele2  idee oli korrata Eestis täpselt sama, mida ta tegi Rootsis, ehk et ta 
soovis sellelt suurelt \emph{telco}lt palju raha välja imeda. Kuna Eesti 
Telefon üüris ruume, liine ja nii edasi, oli meil teada, et Tele2 paneb oma 
sissehelistamiskeskuseid püsti. Eesti Telefoni juhtkond oli sellest paanikas, 
ma ise ka külastasin mingisugust sellist laiendatud juhatuse koosolekut, kus 
sellest arutati. Ma mäletan, et ma tulin sealt üsna mornilt tagasi. Mulle 
tundus, et vanemad kolleegid ei suuda  midagi otsustada või ära teha. Mina 
sellise noore mehena oleks tahtnud kohe rauh-rauh, et läksime. Ma ei mäleta, 
mis asjaoludel ma olin kodus, aga ma pidasin siis telefonikõne Priit 
Pirsoga\index[ppl]{Pirso, Priit}, kes oli tollel hetkel selle valdkonna juht 
Eesti Telefonis. Ja selle telefonikõne käigus me otsustasime, et me teeme Eesti 
Telefoni osutatavale Atlas Starter teenusele alternatiivse teenuse, sellepärast 
et Atlas Starter absoluutselt ei sobi Tele2'ga konkureerimiseks. Meil on vaja 
sellist teenust, kus kasutajate  registreerimise protseduur ja selline on 
automaatne. \emph{Self-service}, kasutaja saab ennast ise registreerida ja nii 
edasi. Kuna  kuutasu niikuinii pärast seda Tele2 jampsi enam ei ole, siis 
ainukesed, mis maksavad, on kõneminuti hinnad. Tele2  lootis  raha teenida 
sellest, et ta termineerib kõnet ja Eesti Telefon on sunnitud talle 
nii vahendama kliendi käest küsitud kõneminuti hinda. 

Selle telefonikõne käigus me leppisime kokku, kes mida teeb, kuidas teeb ja ma 
olen ka üsna kindel, et selle kõne käigus me leppisime kokku, et toote saab 
nimeks saab Hot\index{hot.ee}. Sest ma muu seas juba arvutist vaatasin, et 
millised huvitavad domeenid on meil vabad. Kusjuures tollel ajal oli veel see 
aeg, kui EENet\index{EENet}  ei nõustunud andma ühele ettevõttele mitut 
domeeni. Mina ei tea, kuidas, aga minu üks tänane kolleeg, Guido 
Kõiv\index[ppl]{Kõiv, Guido}, temal õnnestus kuidagi EENetist saada 
hot.ee domeen meile, ma ei tea, kuidas. Sarnane \emph{inside}, nagu nagu 
Tarvil\index[ppl]{Martens, Tarvi} oli www.ee'le. Igal juhul me saime selle ühe 
või paari telefonikõnega, väga lühikese ajaga kokku lepitud, kes mida teeb. Ja 
kujutad sa ette, kahe nädala pärast me olime \emph{live}'is. See tähendab, et 
meil toimus teenuse \emph{launch} ja meil hakkas kasutajaid registreeruma 
tempoga tuhat tükki päevas.

Sealt sai siis hot.ee alguse. Minu teha jäi  seesama Radiuse pool. 
Hoti\index{hot.ee} omaaegne sisu tegelikult oli järgmine. Meie huvi oli see, et 
inimesed helistaksid meile sisse.  Tollel ajal hakkas ka juba olema kombeks, et 
anname ka kasutajale e-maili. Aga kuna varasemalt küsi e-maili eest raha, siis 
meile tundus, et lihtsalt nii samas neid e-maile jagada ei tahaks. Ja siis sai 
tehtud  selline kriuks, et sa saad küll veebipõhiselt konto luua (kusjuures 
imelik \emph{chicken-and-egg} probleem, et sul on Internetiühenduse konto 
loomiseks Internetti vaja, aga tundus, et see ei olnud takistuseks), aga see  
meilikonto ja ka kodulehekonto ei hakanud enne tööle kui sa olid selle 
registreeritud kontoga vähemalt ühe telefonikõne teinud 
sissehelistamiskeskusesse. Seda loogikat võimaldas siis minu \emph{custom} 
Radius, kes kasutajatel järge pidas. 

\question{Ühel hetkel oli hot.ee-s veebimeil ka, eks ole?}

Veebimailer oli, ma arvan, et suhteliselt algusest kohe juba sellesama esimese 
kujunduse osa juba. Aga see ei olnud minu programmeeritud, see oli Internetist 
leitud vabavara, mida me saime kasutada. Ma arvan, et me isegi seda nii-öelda 
ei \emph{re-brand}inud enda värvidesse, vaid see oli lihtsalt meie lehelt 
lingitud. Me ise hostisime teda.

\question{See seletab, miks meil mõned aastad hiljem veebimeileri tegemine 
Hansapangas\index{Hansapank} nurja läks, meil miskipärast ei tulnud pähe mõtet 
see lahendus Internetist lihtsalt alla laadida}

Mulle ei tulnud selline mõte pähe, et seda ise teha. Küll aga mäletan seda, et 
hiljem kui keegi mäletab, oli selline huvitav protokoll nagu WAP. Ehk siis 
mobiilivariant Internetist\sidenote{\emph{\enquote{WAP - Wireless Application 
Protocol}} oli sajandivahetuse paiku tekkinud ja põgusalt ka kasutusel olnud 
katse luua toona kasinate sidevõimalustega mobiiltelefonide jaoks lihtsam pinu 
internetiprotokolle 4. kuni 7. OSI kihini. Muu hulgas sisaldas standard ka 
erilist \emph{markup}-keelt toonase mobiiltelefoni mõnerealisele ekraanile 
sobivate kasutajaliideste loomiseks}. Vot selle WAP-meili ma küll tegin täiesti 
nullist sellelesamale Hotile.

\question{Mis õnneks ei olnud väga pika elueaga sest WAP ei olnud väga pika 
elueaga}

Mina mäletan ka seda legendaarset väidet Ando Meentalolt\index[ppl]{Meentalo, 
Ando}, kes oli tollel ajal EMT üks arendusjuhte, kes kommenteeris minu WAP 
meili nii, et \enquote{noh, sa võid ju sinna suahiili keele ka panna, aga 
ilmselt pole sellest väga palju kasu}. Aga kogu see WAP sai minul isiklikult 
alguse sellest, et ma olin saanud endale WAPi-võimelise telefoni. Ma arvan, et 
see oli üks ainukesi telefone, mida ma olen iialgi tööandjalt saanud. See oli 
siis Nokia 7110\sidenote{Tegu oli 1999. aastal uskumatult innovatiivse 
telefoniga: mitut tekstirida näitav ekraan, rullikuga kasutajaliides, T9 
ennustav tekstisisestus sõnumite puhul, vedruga uhkelt lahti hüppav klapp, WAP, 
ebamaiselt küütlev korpus jne. Oma isepärase kuju tõttu sai aparaat rahva seas 
hüüdnimeks \enquote{banaan}}, klapiga telefon, millel oli suur ekraan. 

\question{See telefon oli muide suurepärane põhjus Hansapangale WAP-i põhine 
internetipank teha. Sest selle testimiseks pidi ju pank ometigi väljastama ka 
sobiliku seadme}

Mul oli \emph{vice-versa} selles mõttes, et ma sain kõigepealt telefoni, siis 
mul tuli idee, et mul telefon nüüd on, aga mida ma sellega teen  ja et jube äge 
oleks enda postkasti sisse vaadata sellisel mugaval moel.  Ja siis ma tegin 
WAP-meili.

\question{Aga sellega algab juba uus sajand ja sellest me räägime võib olla 
mõni teine kord. Lõpetuseks küsin veel, et mis sa praegu teed?}

Praegu ma olen Bolt\index{Bolt}\sidenote{endise nimega Taxify ning asutatud kui mTakso} serveri infra peal.  Minu üks kauaaegseid 
kolleege sealt samast Eesti Telefonist Tarmo Kople\index[ppl]{Kople, Tarmo} on 
üks nendest inseneridest, kellega me alustasime Bolti kogu seda 
serverimajandust praktiliselt juba algusest peale. Kui meie alustasime Tarmoga 
serverite poole majandamist Boltis, siis meil oli stiilis tuhandeid kliente ja 
tuhandeid sõitusid kuus ja nüüd see on siis asendunud miljonite klientide ja 
miljonite sõitudega.
