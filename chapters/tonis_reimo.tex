\index[ppl]{Reimo, Tõnis}

Minu alguse lugu jäi nagu paljudel teistelgi ammu enne 
BBSinduse aega. Isa töötas tollal arvutitega 
ja oli vist seotud Rahvusraamatukogu\index{Rahvusraamatukogu} (tollal 
Kreutzwaldi-nimeline raamatukogu) arvutus- või arenduskeskusega. Seal pääsesingi arvutite ligi. 

\question{Mis vanuses see umbes oli?}

Võisin olla viiendas või kuuendas klassis. Tollal mindi aasta 
hiljem kooli, nii et siis tänapäeva mõistes kuuendas-seitsmendas klassis. 

Seal tekkis võimalus arvutitele üldse ligi pääseda ja loomulikult 
mängida, sest see oli tollal ainuke huvitav asi.

\question{Kas tõesti ainuke?}

Jah, progemisest oli asi veel kaugel.

Mänge ju ei olnud. Kas olid malmist põranda külge needitud mänguasjad või siis 
esimesed arvutimängud, mis jooksid pigem trükimasinal kui 
arvutil. 

\question{Mis arvutid need olid?}

Sain natuke näppida Jessukesi\index{ES EVM} ja 
Vene toodangut, aga tekkisid ka eimesed personaalarvutid, nagu 
Schneider\sidenote{Schneider oli kunagise arvutitootja Amstradi esindaja 
Saksamaal, Austrias ja \v{S}veitsis, kelle müügivõrku viimane kasutas ja kes 
Amstradi arvuteid ka mõnel määral oma turule kohaldas. 1988. aastast alates 
läksid ettevõtete teed lahku ja Schneider hakkas tootma oma PCsid.}, 
Lääne-Saksamaa importtoodang.

\question{Kas see oli XT analoog?}

Isegi XT-eelne, aga selle peal oli juba võimalik primitiivseid mänge 
mängida ja sealt see huvi vaikselt arenes. 

\question{Sina käisid mängimas, aga mida nende arvutitega päris tööks tehti?}

Milleks täiskasvanud neid rakendasid, sellest ma alguses aru ei saanud ja 
väga ei huvitanud ka. Minu jaoks olid täiskasvanud lihtsalt tüliks, sest nad 
takistasid arvutisse saamist. Arusaam, mida täiskasvanud arvutiga 
teevad, tuli alles aastaid hiljem, kui tegin esimesi katsetusi 
progemisel.

Laiem huvi süvenes siis, kui liitusin Jaak 
Loonde\index[ppl]{Loonde, Jaak} arvutiklubiga 
Ahhaa\index{Ahhaa}. See loodi vist 1985. või 1986. aastal, Jaak 
Loonde oli siis 3. keskkooli\index{Tallinna 3. Keskkool} legendaarne 
matemaatikaõpetaja. 

Ta vedas seda arvutiklubi alguses nõukaaegse toodangu peal --- seal oli üks
ESi-laadne masin, millel meie mõistes ekraani ei olnud, aga oli teletaip. 
Põhimõtteliselt oli see trükimasin, kuhu lõid käsud, mis trükiti 
paberile, ja käskude alla trükkis masin oma vastused. Loomulikult oli ka lindi pealt sisselugemise võimalus.

Sealt läksime kiiresti üle esimestele Yamaha 
MSXidele\index{Yamaha MSX}. Edasi tulid Jukud, 
Iskrad ja kogu see kloonindus.

\question{Kuidas sa arvutiklubisse sattusid? Kas sind ennast huvitas või mõni sõber 
kutsus?}

Ma ei mäleta, kuidas. Võib-olla teadsin Jaaku kooli kaudu, sest esimesed kolm aastat käisin ma 3. keskkoolis.

\question{Arvutiinimesed on tavaliselt \emph{sci-fi} sõbrad --- kas sul oli ka selline huvi?}

Ikka, Soome televisioonist sai vaadata \enquote{Battlestar Galacticat}.

\question{See oli kandiliste plekist robotitega originaalseeria, eks!}

Jaa, need olid Cylonid, tuli käis edasi-tagasi. Ja loomulikult töötasin läbi kogu nõukaaegse 
\emph{sci-fi} kirjanduse, nii palju kui kätte sain.

\question{Kas oskad mõnd konkreetset eredamat teost meenutada?}

Alustatud sai \enquote{Maailm ja Mõnda}\sidenote{\enquote{Maailm ja Mõnda} oli Eestis 
ilmunud raamatusari, mida algselt andis välja Eesti Riiklik Kirjastus, hiljem 
Eesti Raamat ja teised. Sari keskendus peamiselt reisikirjadele ja 
loodusraamatutele.} sarjast, keerukam ja elegantsem osa tuli kõik hiljem. 

Mäletan ka näiteks raamatut \enquote{Purpurpunaste pilvede 
maa}\sidenote{Seda raamatut meenutab ka Tarmo Mamers, vt lk
\pageref{sisu:purpur}.}. See oli nõukogude vaimustusest 
kantud \emph{sci-fi}, mis jutustas sellest, kuidas kommunistliku ühiskonna liikmed 
kangelaslikult kosmost vallutavad.

\question{Ega neid raamatuid ei olnud eesti keeles palju saada, seega teatud
seltskond luges paljuski samu teoseid ja tänu sellele saadi üksteisest paremini aru.} 

Jah, see oli üks asi, aga seltskond, kellega mina tollal kokku puutusin, oli 
ikkagi suhteliselt piiratud. Omavanustest olid need selliste inimeste 
lapsed, kes töötasid arvutitega. Nende vanemad olid kas 
KBFIs, Tallinna Tehnikaülikoolis või muudes asutustes, kus oli arvuteid. 
Tehniline intelligents. Päris juhuslikku rahvast seltskonnas eriti ei 
olnud.

Nii palju kui mina nägin, siis enamik, umbes 80\%, oli keskendunud arvutitega 
mängimisele. Edasi tulid praktilised probleemid, kuidas mänge 
kopeerida ja avada ning üldse üles otsida. Tegime 
opsüsteemiga tutvust ja lõpuks tekkis huvi, kuidas ise mänge teha.

\question{Kaua sa Ahhaa klubis käisid?}

Ahhaa klubis olid pigem üritused. Nii palju kui mina mäletan, oli 
kindlatel päevadel kusagil ligipääs arvutitele: Tööjõureservide õppekeskuses, tehnikaülikoolis ja muudes veidrates 
kohtades. Aja jooksul tuli neid kohti juurde, kui kellegi vanemad või sõbrad 
jälle sokutasid, ja nii me rändtirtsudena lendasime vabale 
arvutusressursile peale.

\question{Kas need täiskasvanud inimesed, kes loodetavasti tões ja väes tööd 
üritasid teha, ei pahandanud?}

Enamasti oli see tegevus ikkagi töökeskkonnast eraldatud. Ma väljaspool 
isa töökohta ei mäleta eriti, et oleksime otseselt tööruumides 
olnud, pigem kuskil õppeklassides. Tagantjärele mõeldes sai kindlasti isa kolleegidele kõvasti närvidele käidud, kuna me ju hõivasime nende arvutiterminale siis, kui nad tahtsid tööd teha. Aga 
see oli nagu põnev mäng, et nad peitsid mängud ära ja meil oli 
jälle põhjust vaeva näha nende ülesotsimiseks.

\question{Järeldan sellest, et ka täiskasvanutel olid masinates mängud!}

Olid-olid, ega me neid ise sinna pannud. No mõningatesse panime. Aga kui 
tollase Tallinna Linna Täitevkomitee (mis oli üks isa töökohtadest) 
keldrisse tekkis UNIX, siis sinna ikka ise midagi ei kopeerinud. Masin 
oli varustatud sellega, mis seal oli, ja siis tuli kähku orienteeruda, 
\verb|su| käsud ja ruudud selgeks saada.

\question{Huvitav, et kui UNIXid Eestisse jõudsid, olid need purgid pigem 
suletud ja kõik muud pigem sellised, kuhu sai ise asju sokutada.}

Jessukesse sai kindlasti sokutada, aga programmeerimine 
ja programmi sokutamine oli lihtsalt väga tülikas, sest see käis perfokaartide kaudu.

\question{Kuidas sul su ja root selgeks said? 
Kust see info tuli?}

Tõenäoliselt seeläbi, et 
vingusin oma mängu senikaua, kuni keegi midagi ette näitas. Vaata, kui noorelt
tänapäeval nutitelefon selgeks saadakse, enam-vähem samal ajal koos 
rääkimisega, asi siis \emph{super user}'i kasutamine selgeks saada 
on! Kusjuures oluliselt vanemana. Seda enam, et tollal oli infoturve 
suhteliselt olematu --- kõikide login oli eesnimi ja parool 
perekonnanimi.

\question{Kui tollal oleks infoturve olnud paremini paigas, kas siis oleks terve põlvkond inimesi jäänud
sisuliselt ilma arvutita?}

Ma arvan, et mitte, sest tegelikult saadi masinale ligi tänu vanematele ja eks me
naaksusime isa või ema kallal senikaua, kuni ta mängu käima pani. 
Vahepeal vaatasime ja panime tähele, mida tehti ja kuidas sai. Aga ilmselt oli tase erinev, sest mõned tundsid programmeerimise vastu 
põhjalikumalt huvi, mina alguses vähem. Ausalt öeldes ei ole
programmeerimine kunagi mu tugevaim külg olnud. Olen pigem müügi, 
turunduse, juhtimise, projektijuhtimise ja tootejuhtimise kallakuga.

\question{Teades, mis tooteid sa oled juhtinud, siis seda ju ei saa teha, saamata 
väga hästi aru, mis kapoti all toimub?}

Jah. See huvi on ka loomulikult alati olnud, kuidas asi töötab, aga selleks 
ei pea alati ise tegema.

\question{Kui said vanemaks, kas siis ühel hetkel see külakorda käimine 
muutus?}

Ühiskond läks edasi. Alguses tegutsesime isa loodud Eesti-Rootsi 
ühisfirma tiiva all, kus olid personaalarvutid: 286, 386, hiljem 486. Ja 
edasi lõime oma firma. Aga alguses sai selle ühisfirma ruumides ja tiiva 
all tegutsetud ja arvuteid maale toodud. Enamik
aega läks ikka mängimisele, aga kuskilt oli ju vaja saada riistvara, millega 
mängida. See oligi üks \emph{driver}. Ostad, mängid mõnda aega, müüd maha ja 
nii see äri käima läks. Hiljem võttis äri mängimise üle, siis ei olnud 
enam aega mängida. 

\question{Kas see oli pärast 
keskkooli?}

Jah, pärast tehnikumi. Ma lõpetasin Tallinna 
Polütehnikumi\index{Tallinna Polütehnikum} aastal 1990 ja vist 1991. aastal 
tegime oma HNSi\index{HNS}-nimelise firma, mis tulenes enne 
seda loodud BBSist, mis kandis nime Hackers Night System\index{Hackers 
Night System}. See omakorda ei tekkinud tühja koha peale, vaid tegelikult 
oli enne seda olemas Lembit Pirni\index[ppl]{Pirn, Lembit} Eesti BBS 
\#1\index{Eesti BBS \#1}. Lembit Pirn tegutses praegu Tornimäel asuvas 
madalas valges hoones, kus oli tollal transpordiinformaatika 
keskus või midagi sellist. Ja temal oli esimene modemiga töötav BBS püsti 
pandud, mida sai kõvasti külastatud. 

\question{Miks ta selle tegi ja miks seal oli vaja käia?}

Miks Lembit seda tegi, peab tema käest küsima. Seal oli esiteks 
mänge ja teiseks suhtluskeskkond, hakkas tekkima 
nii-öelda \emph{bulletin board}. BBS läks käima tarkvaravahetuse pealt, mis tollal 
oli täiesti tavaline, aga tänapäeva mõistes räigelt illegaalne tegevus. 
Nõukogude ja üleminekuajal oli see mõiste muidugi võõras. 

Kuna ta vist oli ka Fidoneti liige, siis sealt sai infot teiste BBSide 
kohta maailmas. Sai hakatud Soomes BBSe külastama, näiteks Jouni Salo 
BBSi\index{Jouni Salo BBS}, ja Ron Dwightiga\index[ppl]{Dwight, Ron} suhtlema, kes oli Fidoneti Euroopa tsooni pidaja. Eks nemad aitasid-juhendasid edasi 
ja sealt tekkis mõte oma BBS püsti panna. Tänu sellele, et saime 
isa firma ruumides tegutseda, oli meil unikaalne võimalus teha otse välismaale 
kaugekõnesid. See, mis on tänapäeval suhteliselt elementaarne, et helistad kuhugi
otse, ei olnud tollal isegi Eesti piires elementaarne. Kõik kõned 
tehti läbi keskjaama. Keskjaam oli kindel number, kuhu helistada ja 
kus võttis keegi naisterahvas vastu, kellele teatasid, kuhu sa tahad 
helistada, ehk lugesid oma numbri ette. Kui liin vabanes, siis ta helistas 
tagasi ja teatas, et nüüd on siis kõne. Aga kuna tegu oli Eesti-Rootsi 
ühisfirmaga, siis oli seal unikaalne võimalus helistada otse automaatvalimisega maailmas igale poole. Ja see võimaldas ka modemiga enam-vähem 
üle kogu maailma helistada.

\question{Kust modemid tulid?}

Esimene modem oli ühe arvutiga kaasas ja lasi 2400 bitti sekundis läbi. Hiljem meil õnnestus Ron Dwighti\index[ppl]{Dwight, Ron} kaudu 
saada esimene US Robotics\index{US Robotics}, mis oli vist 9600 ehk oluline edasihüpe. Kuna tollal liikus BBSides palju 
erinevat tarkvara (ma nüüd ei ütleks, et legaalset), siis nendes ringisurfamine ja sobramine andis ühelt poolt vahendid ning teisalt teadmise ja oskuse, 
kuidas erinevad tarkvarad töötasid ja mida nendega teha sai. 
Sisuliselt kõike sai ise õpitud, katse-eksitusmeetodil. Isegi nõukaaja lõpus, kui Nõukogude Liidu piires sai veel täiesti vabalt lennata, siis Vladivostokist, Moskvast ja Leningradist oli teisi Fidoneti 
kasutajaid, kes lendasid külla viietolliste flopidega.

\question{Inimesed tulid Vladivostokist viietolliste flopidega Fidoneti?!}

Jah, kuna modemiga imemine võttis rohkem aega ja raha, kui panna
sadu viietolliseid flopisid kohvrisse ja lennata Vladivostokist 
Tallinnasse ning neid kopeerida paar ööd-päeva.

\question{Nii et Vene suunal oli ka Fido side olemas?}

Jaa, nad helistasid samamoodi Eesti saitidele ja sealt liikus info nende 
suunas. 

\question{Sa mainisid, et Eesti BBS \#1\index{Eesti BBS \#1} ümber tekkis kogukond. Kes need inimesed olid?}

Ma seda eriti ei mäleta, me liikusime sealt kiiresti 
edasi. See oli suhteliselt staatiline, vaikne ja rahulik pärast 
loomist, nii et erilist kommuuni vist ei tekkinud. Kommuun tekkis siis, kui BBSi-pidajaid tuli juurde ja ühel
hetkel sai kokku kutsutud esimene süsoppide nõupidamine, mis toimus Viru 
hotelli 20. või 22. korrusel asunud väikeses äärepealses toas. Seal olime mina, 
Lõvi\index[ppl]{Lõvi}, Tarmo Ausing\index[ppl]{Ausing, Tarmo}, Tarmo 
Mamers\index[ppl]{Mamers, Tarmo} ja Virko Püss\index[ppl]{Püss, Virko}. 
Tegelikult on Mamersil isegi mingid memuaarid kirjas, mis olid mõnda aega 
internetiski üleval. 

\question{Kas too kokkusaamine oli rohkem sotsiaalne üritus või lahendasite 
probleeme ka?}

See oli mõlemat. Natukene oli juttu sellest, kuidas Fidonetti 
korraldada ja organiseerida ning kuidas meililiiklust teha. Kuna meil olid 
tollal väliskõned nii-öelda tasuta käes, siis me saime olla Eesti 
esimene \emph{node}, kelle kaudu meil liikus Eestist välja. 

\question{Ja see meie oli Hackers Night System?\index{Hackers Night System}}

Jah. Teised saatsid oma kirjad meile ja meie saatsime need öösel üle 
modemi järgmisele Euroopa \emph{node}'ile, kes need siis omakorda laiali jagas.

\question{BBSi püstipanekuks on ju tarkvara ka vaja.}

BBSist sai paketina
alla tõmmata sellist vahvat platvormi nagu Maximus BBS. 
Tõmbasid \emph{boot}'imisel BAT-failiga üles, mis jäi modemist ootama sissetulevat kõnet, ja kogu lugu. Selle all olid failikataloogid, 
meilisuhtlus ja kasutajate haldus.

\question{Kas BBSi sissehelistamiseks ei olnud eraldi softi vaja?}

Selleks kasutati tavalist terminalisofti, mis oli tollal enamikul 
opsüsteemidel olemas. Ma praegu muidugi oletan, aga tol ajal oli ju suur 
osa \emph{mainframe}'idel ja nendega suhtlemine käis üle telneti.

\question{Mäletan, et ka Norton Commanderi sees oli
helistamisvõimalus.}

Võib-olla. Seda mina ei taibanud kasutada. Kasutati seda, mis kellelgi käepärast oli. Terminalisoftiga
sai juhtida kasutatava modemi režiimi, terminali režiime ja tirida 
alla tarkvara. See oli meie tegevuste põhiskoop lisaks mängimisele 
ja kättesaadud tarkvara uurimisele.

\question{Kogukond tuli seega BBSi adminmide hulgast, aga tuumiku 
ümber pidi ju olema ka kasutajaid. Oskad sa hinnata, kui palju neid 
võis Eestis olla?}

Alguses, 1990. aasta paiku võis olla paarkümmend 
kasutajat \emph{node}'i kohta ja \emph{node}'e oli viie kuni kümne 
ringis. Hiljem, interneti tuleku eelsel ajal, läks asi päris 
massiliseks, aga selleks ajaks olime meie juba sellest kogukonnast eemaldunud, kuna äri võttis kogu aja ära. 

Üle Eesti oli vast paarsada inimest, aga võis ka rohkem olla. Tollal ei olnud ju igaühel oma arvutit ja modemit, kümme inimest kasutas samu vahendeid. 

\question{Klassitäis kaake helistas sisse ja mine võta kinni, palju neid 
seal oli.}

\enquote{Kaake} väga palju ei olnud, enamasti olid meievanused tegelased, kuigi oli ka natuke vanemaid. 
Päris akadeemilist seltskonda, 
teadlasi ja uurijaid, ma ei mäleta. Neil olid ilmselt oma vahendid ja 
võimalused. Ka Usenet oli kusagil olemas.

\question{Kas panite oma HNSi püsti ja hakkasite Nõukogude Liidu avarustest juppe korjama ning Eestis maha müüma?}

Pigem vastupidi: hakkasime tooma Saksamaalt arvutijuppe, neid siin kokku 
panema ja Eestis maha müüma. Võib-olla tänu sellele, et isa kõrval 
tekkis mul varakult impordi-ekspordikogemus, kuidas välismaalt midagi osta ja üle tollipiiride Eestisse tuua. See oli 20 aastat enne e-poode suhteliselt 
haruldane teadmine. Esiteks, kuidas leida üldse kontakt, kellele helistada ja 
hinnapakkumist saada. Sul ju ei ole aimugi, kes müüb.

\question{Kust sina teada said?}

Tänu sellele, et isa oli aktiivne tegelane. Kuna ta inglise keelt 
eriti ei osanud, siis pidin jooksvalt ka tema asju ajama ning tema 
eksimustest ja õnnestumistest sai õpitud. 

\question{Eks me kõik seisame hiiglaste õlgadel. BBSinduses alguses 
ärilist aspekti ju ei olnud?}

Ei, puhas fänlus. Paljuski seisis see kahel jalal. Esiteks suhtlemine 
ehk vestlustoad, inimesed suhtlesid erinevatel teemadel ja tekkis 
\emph{community}. Teiseks softi vahetamine.

\question{Tollal hakkasid esimesed jutukad ka tekkima.}

Jah, arvan, et jutukad saidki paljuski alguse BBSide 
tubadest.

\question{Mis olid esimesed jutukad?}

Ma ei käinud väga palju jutukates, aga üks oli OK jutukas\index{OK jutukas}
ja teine Cafe\index{Cafe jutukas}. Need hakkasid tekkima siis, kui 
ma enam väga sotsiaalse tegevusega ei tegelenud, kogu aeg ja energia 
läks oma firma arendamisele.

\question{Kuidas äri tol ajal toimus? Üheksakümnendate keskel oli veel
suhteliselt palju kogukondlikku toimetamist, kasvõi .EXE tegemine.}

Jah. Ma arvan, et äri seisis sellesama kogukonna õlgadel, mis sai alguse Fidonetist. Lõpuks pidid ju kõik kusagil tööd tegema. Pigem liikusid
suhted ja asjad ärisse edasi. Needsamad nimed, kes ajakirja .EXE\index{.EXE} tegid, olid samuti Fidonetist tuttavad. 

\question{Kas sina keskendusid ainult ärile?}

1990ndate alguses või keskel läks enam-vähem selline elu käima, et tööpäevad olid üheteisttunnised. 

\question{Selle kõrvalt vist väga palju enam programmeerimiseks aega ei jäänud?}

Ei. Me tegelesime eelkõige riistvara vahendamisega ja võrkudega, 
suhteliselt primitiivse tegevusega. HNS\index{HNS}, hilisem Zebra 
Infosüsteemid\index{Zebra Infosüsteemid}, tarkvaraarenduseni ei jõudnudki.

\question{Isegi tavalise võrgu ehitamine oli ju koaksiaalkaablil.}

Koaksiaalkaabli vedamine oli mul väga selge. Mitmete tänaste suurfirmade 
laudade all sai roomatud ja ladudest, seintelt ja lagedelt tolmu pühitud.

\question{Ühispanga kontoris olid kontoriarvutid seljaga 
kliendi poole ja kaabel jooksis ühe masina juurest teise juurde. Järjekorras 
seistes oleks võinud lihtsasti termika maha keerata ja terve kontor oleks 
seisma jäänud.}

Meie vedasime Seesamile\index{Seesam Kindlustus} IBM Token Ringi 
nimelist võrku. Token Ring oli esimene tähtkujuline topoloogia, millega me kokku puutusime. Sellel 
olid rusikasuurused stepslid, täielik müstika.
 
\question{Kust teil tuli mõte sellise müstikaga tegeleda?} 

Äris võetakse ikka riske ja öeldakse, et loomulikult saame selle asjaga hakkama. Mis see siis ära 
ei ole! Hakkad tegema ja selgub, et kõik töötabki nii, nagu ette nähtud.
Võrkude ja asjade ülespanek ei olnud progemisega võrreldes midagi
erilist --- konfigureerid süsteemi ära ja jalutad koju.

\question{Tol ajal ei olnud vist koolist niisugust teadmist üldse võimalik saada?}

Absoluutselt. Mina lõpetasin polütehnikumis\index{Tallinna Polütehnikum} 
raadioside ja levi eriala. See põhines valdavalt lampelektroonikal, kuigi 
meile õpetati ka pooljuhte ja muud säärast, aga loogilisi skeeme ja
ahelaid saime vist ainult pool aastat ja sedagi 
teoreetiliselt. Enamus meie elektroonikaõppest polütehnikumis oli 
analoogelektroonika: lambid, elektromehaanika, sammvalijatel 
põhinevad telefonijaamad. Tänapäeva mõistes oli see paras küberpunk. 

\question{Sinu jutust kumab välja soov asjadest aru 
saada, mõista, mis on karbi sees.}

Absoluutselt. Polütehnikumi minek oli perekonnatraditsiooni järgimine. Ma olen soodumuselt pigem humanitaar, tehnikumi minekuni olid mul neljad-viied kõik humanitaarained: 
keeled, kirjandus. Reaalained olid kahed-kolmed.

Ilmselt siis tasub õppida seda, mida ei tea, ülejäänu tuleb niigi. 

See on andnud selle plussi, et oskad 
asju, millest aru ei saa, üldistada või teisendada mustadeks 
kastideks, millel on mingid defineeritud sisendid ja väljundid, mille põhjal 
saad omakorda teha järeldusi musta kasti sisu kohta. 
Selline paradigma oli koolituses pidevalt olemas ja andis vajaliku vundamendi.

\question{Seda, mis kasti sees on, ei jõua sulle keegi õpetada, sest 
homme on seal teine asi. Aga lähenemine on kasulik.}

Tollal olid muutused palju aeglasemad. XT 
ja 286 kestsid aastaid, enne kui tulid 386 ja 486. Nüüd on 
muutuste tempo radikaliseerunud.

\question{Huvitaval kombel tundub alati, et just praegu on elu palju 
kiirem kui vanasti.}

Võib-olla vanemaks saades elutempo ise muutub.

\question{Mis hetkel hakkas BBSi kogukond selgelt alla jääma sellele, et igaühel 
oli laen-liising ja pere?}

BBSi kogukond ei olnud ju ainult \emph{bulletin board}'id, 
vaid toimusid ühisüritused: BBWinterid\index{BBWinter} ja 
BBSummerid\index{BBSummer}. Need said alguse sellestsamast 
süsoppide esimesest kokkutulekust ja hiljem hakati tegema laiemaid üritusi, kuhu olid kutsutud ka BBSide kasutajad. Tegelikult jätkus BBSindus ka pärast
interneti tulekut. Alles äsja arutati, kellel 
BBS veel töötab. Ma üldse ei imestaks, kui Eestis oleks mõni töötav BBS 
kusagil virtukas tiksumas. 

Minu jaoks vajus see laiali seetõttu, et ma ei 
jaksanud sinna enam ise panustada, aga elu ja tegevus toimus seal veel 
aastaid hiljemgi. Laiali hakkas see vajuma siis, kui 
internetipõhised keskkonnad hakkasid kasutajaid üle võtma. 
Suhtluskeskkond sulas igapäevase töö ja muude tegemistega seotud keskkonnaga 
üheks, modemiga kuhugi helistamine tundus pisut arusaamatu.

\question{Nii et kui BBSis vahetati faile ja aeti juttu, siis internetis sai tööd 
ka teha?}

Absoluutselt, internetis olid Cafe ja OK jutukad olemas ja enne 
Facebooki ja Rate'i olid ju ka olemas Geocities ja igasugused muud keskkonnad. Need ei ole midagi uut. Need on vahetanud kesta, vormi, 
platvormi, värvi ja natuke funktsionaalsust, aga inimesed on juba 30-40 
aastat liikunud sellesama tegevusega ühest keskkonnast teise.

\question{Kes BBSummereid ja BBWintereid korraldas ning kes seal käisid?}

Ma mäletan ühte korraldajat, kelle kasutajanimi oli vist Kristrap ja pärisnimi Piret 
Part\index[ppl]{Part, Piret}.

\question{Kas see oli puhas kogukonna värk, sponsorit taga ei olnud?}

Hiljem kui üritused läksid suuremaks, siis me HNSiga firma poolt loomulikult 
sponsoreerisime. Aga alguses olid üritused piisavalt väikesed: selleks ei ole vaja sponsorit, et kümme
inimest kokku tuleksid, asju arutaksid ja sealjuures mõned 
õlled teeksid. Hiljem arvutifirmad toetasid küll, sest samad inimesed 
töötasid enamjaolt arvutifirmades, olles seal olulised tegelased, kui mitte 
omanikud.

\question{Seega läks Eesti Fido kogukond sujuvalt üle Eesti IT-tööstuseks?}

Jah, absoluutselt. Paljud tuttavad on sealtsamast keskkonnast pärit, näiteks Tõnu 
Samuel\index[ppl]{Samuel, Tõnu}. Lihtsam on üles otsida tollased Fidoneti nimekirjad ja jututubade 
\emph{printout}'id, mida vist Mamers\index[ppl]{Mamers, Tarmo} peab, ja vaadata, 
kes need nimed on. Küllap sa leiad nad praeguste 
IT-meeste gruppidest üles.

\question{Kes nendel üritustel peale süsoppide veel käisid?}

Ma ei oska sulle öelda, aga eks need, kes ei 
tahtnud, viitsinud või ei saanud ise BBSi pidada. Kelle jaoks BBSi külastamine oli 
lihtsalt üks asi, mida arvutiga teha.

\question{Kuna progemiseks olid barjäärid kõrged ja kogu aeg mängida ei 
jaksatud, siis tehti midagi muud ka.}

Mõned tulidki läbi progemishuvi. Minu puhul tekkis see läbi mänguhuvi. See oli võimalus suhtlemiseks ja tarkvara ning tehniliste teadmiste vahetamiseks.

Tollal oli suhtlemine tunduvalt raskem, kõigil ei olnud 
isegi kodus telefoni. Tänapäeval on kõik käeulatuses: võtad
taskust telefoni ja paned Google'isse otsingu. Tollal 
pidid otsima telefoni ja mõtlema, kellele helistada, et küsida, kas ta teab 
kedagi, kes teab midagi. Fidonet andis ju ka selle võimaluse, et said 
avalikku \emph{board}'i panna küsimuse, kas keegi teab, kuidas 
ühte või teist probleemi lahendada.

\question{Kas vastuseid tuli ka?}

Kindlasti, kus nad pääsesid! Eks sõltus ka küsimusest. Seal oli asjalikke 
jutte, tehnilisi vestlusringe ja ka läbukohti, kus niisama jaurati.

\question{Aeg paneb tõepoolest asjad teise konteksti. Kui sul isegi kodus telefoni ei 
ole, siis võimalus rahvusvaheliselt inimestega suhelda on hoopis
teise väärtusega kui siis, kui sul on taskus mobiiltelefon ja internetti 
ühendatud arvuti alati käepärast.}

No just. Mõtle sellele, et tollal Google'it ju ei olnud.

Google'iks oli üle laua kolleeg, naaber või sõber ja sul endal pidi olema 
piisavalt suur ja lai kontaktibaas ja varamu, et teaksid kedagi, kes 
teaks kedagi, kes teab kedagi, kes oskab sulle öelda midagi.

\question{Kas oled nõus Pron to ütlemisega, et paljuski seesama 
kontaktibaas võimaldas väga loomulikult kogukonnast ärisse üle minna, sest ka 
äris sõltus palju kontaktidest?}

Seda võib nimetada ka \emph{street reputation}'iks --- sul oli mingil määral
\emph{credibility} olemas, sind juba teati ja tunti selles 
keskkonnas. Kui tänapäeval räägitakse \emph{Estonian Mafia}'st, siis tollal 
oli Fidoneti seltskond.

Loomulikult tulid paljud meie kliendid läbi Fidoneti või vähemalt teadsid 
meid sealtkaudu.

Paljud klientidest töötasid hiljem mõnes pangas või firmas, mis
laienesid ja tahtsid saada arvuteid. Neil oli palgatud itimees, 
kes pidi selle probleemi lahendama, ja ega tal ka ju ei olnud Google'it või 
e-poodi, kust parimat pakkumist küsima minna. Tal olid inimesed, keda 
ta teadis ja usaldas ning kelle käest seda pakkumist minna küsima.

\question{Ei olnud nii, et lähed poodi ja ütled: \enquote{Palun mulle 16 arvutit.}}

Alguses oli ka arvutiäri selline, et kuna valuutat 
eriti kellelgi ei ringelnud, siis laoseisud olid olematud. 
Võeti ettemaks, mis maksti välismaale, ja oodati, 
kuni arvuti kohale laekus, siis pandi see kokku ja tarniti kliendile. Kui hästi 
läks, sai alla kuu ajaga kätte. Arvutikategooria mõttes ei ole
hinnad eriti muutunud: hea arvuti oli üle 1000 ja tavaline 1000.

See, et andsid mingile matsile raha ära ja tema ütles, et ära 
muretse, kuu aja pärast saad kätte, eeldas üksjagu usaldust. Valuutat 
ei olnud ja alguses tehti tehingud enamasti rublades, aga rubla ei olnud 
konverteeritav millekski muuks kui rublaks. Lisaks oli sel ajal paljudel 
asjadel kaks hinda: ülekande rublahind ja sularaha rublahind. Odavam oli 
sularahas, sellepärast et pangast ei saanud alati sularaha kätte. Panka toodi sularaha kindlatel hetkedel ja selle saamiseks pidid teadma kas 
õigeid inimesi või õiget aega.

\question{See seletab legende, kuidas arvutifirmas ja pangas 
hoiti sularahapakke kuskil kapis ja vetsus.}

Kui taheti midagi ülekandega osta, siis oli meie enda valik, kas 
müüsime valuuta või rublade eest. Kui ülekandega, 
siis oli hind kallim puhtalt seetõttu, et pärast oli raha pangast tunduvalt keerulisem kätte saada.

\question{Kas sellest ajast saadik oledki jäänud arvutitega tegelevaid 
inimesi juhtima?}

Mida aeg edasi, seda enam on mind 
huvitanud, kuidas sisulised asjad töötavad, ja vähem 
inimeste juhtimine. Inimesed on keerulised, arvutid on 
lihtsad.

\question{Kuidas sa infoturbeni jõudsid?}

Esimest korda jõudsime selleni läbi väga praktiliste sammude --- me nimelt häkkisime 
täitevkomitee arvuteid, et sinna ligi ja mängima saada. 
Avastasime, et tollal oli Õnnepalees olemas üks modemite peal 
töötav teenus, mille kaudu sai abielusid, sünde ja surmasid registreerida. Muu hulgas sai sellesama 
modemi otsas eraldada kortereid --- siis ei saanud ju kortereid osta, 
vaid neid eraldas riik.

\question{Ja teie avastasite koha, kuhu sai sisse helistada ja 
kortereid eraldada?}

Me avastasime, et sinna sai sisse helistada, aga teenus oli parooliga kaitstud 
ja parool oli jällegi eesnimi ja perekonnanimi. Alustasime infoturbes nii-öelda 
tagumisest otsast.

Ma ise sattusin Privadori\index{Privador} 10---15 aastat hiljem, 
aastal 1990, kui Tarvi Martens oli tollasest Küberneetikast\index{Küber} teinud 
investorite kaasabil \emph{spin-off}'i, 
mida tänapäeval kutsutakse \emph{startup}'iks. Privadori eesmärk oli 
digitaalselt signeeritud dokumentide pikaajalise tõestusväärtuse loomise 
süsteem nimega TruSign. Seda hakati tegema kaks aastat enne ID-kaardi 
projekti algust, enne kui keegi polnud ühtegi signeeritud dokumenti veel näinudki.

\question{Väga huvitav! Krüptot enam paljakäsi ei tee, selleks on haridust 
vaja. Seega pidid juba üheksakümnendate lõpus kokku saama inimesed, kes 
olid iseõppijad, ja teistsugused inimesed, kes ei olnud 
iseõppijad. Kuidas see käis, kas hõõrumist ei tekkinud?}

Enamik tollastest kolleegidest olid vanad tuttavad Fido ajast. Mina liitusin Privadoriga 
turunduse ja müügi funktsioonis. Alles hiljem hakkasin tegelema
tootejuhtimisega ja sain nii-öelda juhatajaks. 

\question{Fido seltskond ei olnud ju väga suur.}

Ei olnud jah. Kogu see seltskond, kellel üldse oli üheksakümnendatel 
ligipääs arvutitele, oli tegelikult üsna väike. Siis oli veel see aeg, kui arvutid olid 
peamiselt firmades, mitte kodudes. Alles üheksakümnendate lõpus hakkas 
tekkima trend, et firmad olid enam-vähem arvutitega varustatud, osteti
rohkem rakendustarkvara ja eraisikud soetasid koju arvuti.

Tegelikult olid parematel sovhoosidel ja kolhoosidel omad 
arvutuskeskused olemas juba nõukogude ajal. Arvuti kui selline ei 
tekkinud Eestis üheksakümnendatel, vaid see oli ikka ammu enne minu sündi 
olemas.
