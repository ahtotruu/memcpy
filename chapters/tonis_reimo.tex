\index[ppl]{Reimo, Tõnis}

                 
\ldots see minu alguse lugu, nagu ma arvan paljudel, oli ammu enne 
BBS-indust. Ta on ikkagi otseselt  seotud sellega, et isa töötas tollal arvutite 
teemal ja vist oli Rahvusraamatukogu\index{Rahvusraamatukogu} (tollal  
Kreutzwaldi nimelise raamatukogu) mingi arvutus- või arenduskeskusega, mingi 
sellise imeasjaga, seotud. Sealt pääsesin arvutite ligi. 

\question{Mis vanuses see umbes oli?}

Ma arvan, et see võis olla mingi viies või kuues klass. Tollal mindi aasta 
hiljem kooli, nii et siis tänapäevase mõistes kuues-seitsmes klass. 

Sealt tekkis selline võimalus arvutitele üldse ligi pääseda ja loomulikult 
mängima saada, sest see oli tollal  ainuke asi, mis huvitas.

\question{Päriselt või?}

Ütleks niimoodi, et progemisest oli asi veel kaugel.

Mänge ju ei olnud. Kas olid malmist põranda külge needitud mänguasjad või siis 
need esimesed arvutimängud, mis  umbes nagu jooksid pigem trükimasinal kui 
arvutil. 

\question{Mis arvutid need olid?}

Eks sain natukene näpitud mingeid Jessukesi\index{ES EVM} ja sellist 
Vene toodangut, aga tekkisid mingid esmased personaalarvutid nagu 
Schneider\sidenote{Schneider oli kunagise arvutitootja Amstradi esindaja 
Saksamaal, Austrias ja \v{S}veitsis, kelle müügivõrku viimane kasutas ning kes 
Amstradi arvuteid ka mõnel määral oma turule kohaldas. 1988. aastast alates 
läksid ettevõtete teed lahku ja Schneider hakkas tootma oma PC arvuteid.}, 
Lääne-Saksamaa importtoodang.

\question{See oli siis mingi XT analoog?}

Isegi XT eelne. Aga tema peal oli juba võimalik  mingeid primitiivseid mänge 
mängida ja sealt vaikselt  see huvi arenes. 

\question{Mida nende arvutitega päris tööks tehti? Sina käisid mängimas, 
aga\ldots}

Milleks täiskasvanud neid rakendasid, sellest  ma alguses aru ei saanud ja 
väga ei huvitanud ka. Minu jaoks olid täiskasvanud lihtsalt tüliks, sest nad 
takistasid arvutisse saamist. See arusaamine, mida täiskasvanud arvutiga 
teevad,  tuli alles  aastaid hiljem, kui tulid ka esimesed katsetused 
progeda.

Sealt kasvas ka selline mingi laiem huvi välja, et sai liitutud  Jaak 
Loonde\index[ppl]{Loonde, Jaak} veetud arvutiklubiga 
Ahhaa\index{Ahhaa}. Ahhaa, loodi minu arust kas 1985 või 1986, Jaak 
Loonde oli siis 3. Keskkooli\index{Tallinna 3. Keskkool} legendaarne 
matemaatikaõpetaja. 

Ta vedas seda arvutiklubi alguses mingi nõukaaegse toodangu peal, mingi 
ES-i laadne masin, millel meie mõistes ekraani ei olnud, aga oli teletaip. 
Tekst trükiti paberile ja arvuti oma tulemuse trükkis ka paberile. 
Põhimõtteliselt oli trükimasin, kuhu lõid käsud, käsud trükiti sinnasamasse 
paberile, sinna sinu käskude alla trükkis masin oma vastused. Loomulikult olid ka
 mingid lindi pealt sisse lugemise võimalused.

Sealt läksime juba kiiresti üle esimestele Yamaha 
MSX-idele\index{Yamaha MSX} mis tulid. Sealt edasi juba Jukud ja 
Iskrad ja kogu see kloonindus.

\question{Kuidas sa sinna arvutiklubisse sattusid? Kas huvitas või mõni sõber 
kutsus?}

Ma isegi ei mäleta, kuidas. Võib-olla see asi, et ma esimesed kolm aastat oma 
elust käisin ise 3. Keskkoolis, seetõttu oli nagu mingi seos olemas. Võib-olla 
kooli kaudu teadsin Jaaku. 
                 
\question{Konteksti mõttes, aruvti-inimesed on tavaliselt \emph{sci-fi} sõbrad 
ja muud sellist, kas sul sihukest asja ka oli?}

Ikka, Soome televisioonist sai Battlestar Galactica-t vaadata.

\question{Originaalseeria, eks, kandiliste plekist robotitega!}

Jaa, Cylonid, tuli käis edasi-tagasi. Ja loomulikult kogu nõukaaegne 
\emph{sci-fi} kirjandus oli läbi töötatud, niipalju kui kätte ulatus.
      
\question{Mingit konkreetset eredamat asja oskad meenutada?}           

No ikka alustatud sai Maailm ja Mõnda\sidenote{Maailm ja Mõnda oli Eestis 
ilmunud raamatusari, mida algselt andis välja Eesti Riiklik Kirjastus, hiljem 
Eesti Raamat ja teised. Sari keskendus  peamiselt reisikirjadele ja 
loodusraamatutele.} sarjast, keerukam ja elegantsem osa kõik tuli kõik hiljem. 

Praegu ma panen puusalt, igasugused \enquote{Purpurpunaste pilvede 
maad}\sidenote{Seda raamatut meenutab ka Tarmo Mamers leheküljel 
\pageref{sisu:purpur}.} ja kõik sellised asjad. See oli nõukogude vaimustuses 
kantud \emph{sci-fi}, jutustas kuidas kommunistliku ühiskonna liikmed 
kangelaslikult kosmost vallutavad.

\question{Ega neid raamatuid ei olnud Eesti keeles palju saada, seega mingi 
seltskond luges väga paljus samu raamatuid ja saadi üksteisest paremini aru.}   

Jah, see oli üks asi, aga seltskond, kellega mina tollal kokku puutusin oli 
ikkagi suhteliselt piiratud. Omavanustest olid need ikkagi selliste inimeste 
lapsed, kes töötasid arvutitega. See tähendas seda, et nende vanemad olid kas 
KBFI-s, Tallinna Tehnikaülikoolis või mingites asutustes, kus oli arvuteid. 
Tehniline intelligents. Päris juhuslikku rahvast väga palju seltskonnas ei 
olnud.

Nii palju, kui mina nägin, siis enamus, nii 80\%, oli keskendunud arvutitega 
mängimisele. Sealt edasi tulid sellised praktilised probleemid, et kuidas mänge 
kopeerida ja kuidas mänge avada ja kuidas need üles otsida. Sealt tegid 
opsüsteemiga tutvust ja lõpuks siis tekkisid nagu esimesed huvid, et 
\enquote{aga kuidas neid ise teha?}
                 
\question{Kaua sa seal Ahhaa klubis käisid?}

Ahhaa klubil olid rohkem nagu üritused. Nii palju kui mina mäletan, et mingitel 
kindlatel päevadel oli sul kusagil ligipääs arvutitele. See \enquote{kusagil} 
oli mingi Tööjõureservide Õppekeskus, Tehnikaülikool, mingid sellised veidrad 
kohad.
Ja eks aja jooksul tuli neid kohti juurde, kellelgi jälle vanemad või sõbrad 
jälle sokutasid ja nii me rändtirtsudena lendasime peale vabale 
arvutusressursile.

\question{Kas need täiskasvanud inimesed, kes loodetavasti tões ja väes tööd 
üritasid teha, ei pahandanud?}
                 
Enamasti oli see tegevus ikkagi  töökeskkonnast nagu eraldatud. Ma väljaspool 
isa töökohta ei mäleta väga palju, et me oleksime otseselt tööruumides 
olnud. Mingit õppeklassid ja mingid sellised kohad. Loomulikult, hiljem 
vaadates, siis kindlasti sai kõvasti närvidele käidud isa kolleegidele. Sel 
moel, et nende arvutiterminale hõivatud siis, kui nad tahtsid tööd teha. Aga 
see oli nagu selline põnev mäng, et nad peitsid mängud ära ja siis meil oli 
jälle põhjust vaeva näha nende üles otsimiseks.

\question{Millest ma järeldan, et täiskasvanutel olid ka mängud kuskil seal 
masinates!}

Olid olid, ega ma ise neid sinna ei pannud. No mõningatesse panime. Aga  kui 
tollase Tallinna Linna Täitevkomitee (mis oli tollal üks isa töökohtadest) 
keldrisse tekkis UNIXi masin, siis sinna ikka ise midagi  ei kopeerinud. Masin 
oli  varustatud sellega, mis seal oli ja siis tuli seal kähku orienteeruda, 
\verb|su| käsud selgeks saada ja ruudud.

\question{Huvitav on see, et kui nad Eestisse jõudsid, olid UNIXi purgid pigem 
suletud ja kõik muud pigem sellised, kuhu sai ise asju sokutada.}

No ega sa Jessukesse ka kindlasti said sokutada, aga lihtsalt  programmeerimine 
ja programmi sokutamine olid väga tülikad, sest see käis perfokaartide kaudu.
                 
\question{Sa rääkisid, et su ja root said selgeks, kuidas nad selgeks said, 
kust see info tuli?}

Ma praegu loomulikult ei mäleta, tõenäoliselt ta pidi tekkima seeläbi, et 
vingusid oma mängu senikaua, kuni keegi sulle midagi ette näitas. Vaata, kui 
vanalt tänapäeval nutitelefon selgeks saadakse, enam-vähem samal ajal koos 
rääkimisega, et asi siis nüüd selle \emph{super user}-i kasutamine selgeks saada 
on. Oluliselt vanemana, kusjuures. Seda enam, et tollal  infoturbe teema oli 
suhteliselt olematu,  kõikide login oli eesnimi ja kõikide parool oli tema 
perekonnanimi.

\question{See võis konduktiivne küll olla ringi pusimisele. Ehk, kui tol hetkel 
oleks infoturve olnud paremini paigas, sisi terve põlvkond inimesi oleks 
sisuliselt ilma arvutita jäänud?}

Ma arvan, et mitte, sest tegelikult saadi masina ligi läbi vanemate ja eks sa 
ikka naaksud oma isa ja ema kallal senikaua, kuni ta selle mängu käima paneb. 
Eks siis vahepeal vaatad, jälgid, paned tähele, mida tehti, kuidas sai. Aga ma 
arvan, see tase oli erinev, sest mõned tundsid programmeerimise vastu nagu 
võib-olla põhjalikumalt huvi, mina  alguses vähem. Ausalt öeldes 
programmeerimine ei ole läbi elu mu tugevam külg olnud. Ma olen sellise müügi, 
turunduse, juhtimise, projektijuhtimise, tootejuhtimise kallakuga olnud.

\question{Teades, mis tooteid sa oled juhtinud, seda ju ei saa teha saamata 
väga hästi aru, mis kapoti all toimub?}

Jah. See huvi on ka loomulikult alati olnud, et kuidas asi töötab. Aga selleks 
ei pea alati ise tegema.
                 
\question{Kui sa said tonks vanemaks, siis mingil hetkel see külakorda käimine 
pidi ju muutuma?}

Ühiskond läks edasi, eks ole. Alguses tegutsesime  isa loodud Eesti-Rootsi 
ühisfirma tiiva all, kus olid personaalarvutid. 286, 386, hiljem 486. Ja sealt 
edasi siis lõime ka oma firma. Aga alguses sai selle ühisfirma ruumides ja tiiva 
all tegutsetud ja tegeletud arvutite maale toomisega. Sama eesmärk, et enamus 
aega läks mängimisele, aga oli ju vaja kusagilt saada seda riistvara, millega 
mängida. See oligi üks \emph{driver}. Ostad, mängid mõnda aega, müüd maha ja 
nii see äri käima läks. Hiljem võttis äri mängimise üle, sest et siis ei olnud 
enam aega mängimisega tegeleda, kogu aeg läks äri peale. 

\question{Mis võib olla nii positiivne, kui negatiivne. See oli pärast 
keskkooli?}

Jah, see on kusagil pärast tehnikumi. Ma lõpetasin Tallinna 
Polütehnikumi\index{Tallinna Polütehnikum} aastal 1990, 1991. aastal vist 
tegime siis oma HNS-i\index{HNS} nimelise firma mis omakorda tulenes juba enne 
seda loodud BBS-ist, mis kandis \emph{Hackers Night System}-i\index{Hackers 
Night System} nime. Mis omakorda ei tekkinud tühja koha peale, vaid tegelikult 
oli enne seda olemas Lembit Pirni\index[ppl]{Pirn, Lembit} Eesti BBS 
\#1\index{Eesti BBS \#1}. Lembit Pirn tegutses täna Tornimäel asuvas 
sellises madalas valges hoones. Seal oli tollal mingi transpordi informaatika 
keskus või midagi sellist. Ja temal oli esimene modemiga töötav BBS püsti 
pandud, seda sai kõvasti  külastatud. 

\question{Miks ta selle tegi ja miks seal oli vaja käia?}

Miks Lembit seda tegi, peab tema käest küsima. Seal oli väljas ühelt poolt 
mänge, teiselt poolt  oli mingi suhtluskeskkond, hakkas tekkima juba nagu 
selline \emph{bulletin board}. Ta läks käima tarkvaravahetuse pealt, mis tollal 
oli täiesti tavaline, tänapäevases mõistes räigelt illegaalne tegevus. Aga noh, 
nõukogude ajal ja üleminekuajal see mõiste oli võõras. 

Kuna ta vist oli ka Fidoneti liige, siis sealt sai infot ka teiste BBS-ide 
kohta maailmas. Sai hakatud Soomes BBS-e külastama, näiteks Jouni Salo 
BBS\index{Jouni Salo BBS} ja Ron Dwightiga\index[ppl]{Dwight, Ron} suhtlema, kes  oli 
 Fidoneti Euroopa \emph{tsooni} pidaja. Eks nemad aitasid-juhendasid edasi 
ja sealt tekkis loomulikult mõte oma BBS püsti panna. Tänu sellele, et me saime 
isa firma ruumides tegutseda, oli meil unikaalne võimalus teha otse välismaale 
kaugekõnesid. See, mis on tänapäeval suhteliselt elementaarne, et sa kusagile 
otse helistad, ei olnud tollal isegi Eesti piires elementaarne. Kõik kõned 
tehti läbi keskjaama. Keskjaam oli  selline kindel number, kuhu sa helistasid, 
kus võttis keegi naisterahvas vastu,  kellele  teatasid, kuhu sa tahad 
helistada, lugesid oma numbri ette. Kui liin vabanes, siis ta  helistas sulle 
tagasi ja teatas, et nüüd on siis kõne. Aga kuna tegemist oli Eesti-Rootsi 
ühisfirmaga, siis oli seal unikaalne võimalus  helistada otse automaatvalimisega 
 maailmas igale poole. Ja see võimaldas ka modemiga enam-vähem 
üle kogu maailma helistada.

\question{Kust need modemid tulid?}

Esimene modem oli  mingi arvutiga kaasas, mingi 2400 bitti sekundis läbi laskev 
asi. Hiljem meil õnnestus Ron Dwighti\index[ppl]{Dwight, Ron} ja nende kaudu 
saada esimene US Robotics\index{US Robotics} mis vist oli kas 9600 või midagi 
sellist, ehk oluline edasihüpe. Kuna tollal nendes BBS-ides liikus palju 
erinevat tarkvara (ma nüüd ei ütleks, et  legaalset), siis selles ringi 
surfamine, sobramine, andis ühelt poolt vahendid ja teisalt  teadmise ja oskuse 
kuidas  erinevad tarkvarad töötasid, mida nendega teha sai ja nii edasi. 
Sisuliselt kõik on  ise õpitud,  puhtalt  katsetades, eksitusmeetodil. Isegi 
ma mäletan nõukaaja lõpus, kui veel täiesti vabalt lennata sai Nõukogude Liidu 
piires, siis Vladivostokist, Moskvast ja Leningradist oli teisi Fidoneti 
kasutajaid, kes lendasid külla viietolliste flopidega.
                 
\question{Inimesed tulid Vladivostokist viietolliste flopidega Fidoneti?!}

Jah,  kuna modemiga imeda võttis  rohkem aega ja raha, kui lihtsalt võtta  
sadu viietolliseid flopisid kohvriga kaasa ja  lennata Vladivostokist 
Tallinnasse ja lihtsalt kopeerida neid paar ööd-päeva läbi.

\question{Ahaa, nii et Vene suunal oli ka Fido side olemas?}

Jaa, nad samamoodi helistasid peale Eesti saitidele ja sealt liikus info nende 
suunas. 

\question{Sa korra mainisid, et Eesti BBS \#1\index{Eesti BBS \#1} ümber 
toimus mingi suhtlemine ja tekkis kogukond. Kes need inimesed olid?}
                 
Vot ma seda väga palju ei mäleta. Ma mäletan, et sealt me liikusime kiiresti 
edasi. Ta vist oli suhteliselt staatiline, vaikne ja rahulik pärast  
loomist, et seal vist väga sellist kommuuni ei tekkinud või siis vähemalt ma ei 
mäleta sellest.  Kommuun tekkis siis, kui BBS-i pidajaid tuli juurde ja mingil 
hetkel sai kokku kutsutud  esimene süsoppide nõupidamine, mis toimus Viru 
hotelli kas  20. või 22. korrusel asunud väikeses äärepealses toas. Seal olid 
Lõvi\index[ppl]{Lõvi}, mina Tarmo Ausing\index[ppl]{Ausing, Tarmo}, Tarmo 
Mamers\index[ppl]{Mamers, Tarmo}, Virko Püss\index[ppl]{Püss, Virko} vist. 
Tegelikult on Mamersil vist isegi mingid memuaarid kirjas, nad olid mõnda aega 
veel isegi internetis üleval. 

\question{Kas too kokkusaamine oli rohkem  sotsiaalne üritus või oli seal 
mingisugust probleemi ka lahendada?}

Ta oli mõlemat. Natukene oli minu arust juttu  sellest, et kuidas Fidonetti  
korraldada, organiseerida, kuidas  meililiiklust vist teha. Kuna meil olid 
tollal väliskõned nii-öelda tasuta käes, siis meie saime olla Eesti  see 
esimene \emph{node}, kelle kaudu meil  liikus Eestist välja. 

\question{Ja see meie on Hackers Night System?\index{Hackers Night System}}

Jah. Teised saatsid oma kirjad meile, meie saatsime need öösel üle 
modemi järgmisele Euroopa \emph{node}-le, kes nad siis  omakorda laiali jagas.
                 
\question{BBS-i püstipanekuks on ju mingit tarkvara ka vaja?}

See oli  suhteliselt sellise paketina saadaval, samamoodi BBS-ist tõmbasid 
alla. Mingi Maximus BBS või midagi sellist. Selline vahva platvorm. 
Tõmbasid \emph{boot}-imisel BAT failiga üles, see jäi modemist ootama sisse 
tulevat kõnet ja kogu lugu. Seal all olid siis põhimõtteliselt failikataloogid 
ja meilisuhtlus ja kasutajate haldus.

\question{BBS-i sisse helistamiseks ei olnud mingit eraldi softi vaja?}

Tavaline terminali soft oli, see  oli vist tollal isegi, ma pakun, enamusele 
opsüsteemidele olemas. Ma praegu muidugi oletan, aga  tollal oli ju suur 
osa ju \emph{mainframe}-del ja nendega suhtlemine käis üle telneti.

\question{Ma miskipärast mäletan Norton Commanderi sees mingit 
helistamisvõimalust}

Võib olla. Seda mina ei taibanud kasutada. Mis kellelgi käepärast oli. Sellega 
sai siis juhtida nii kasutatava modemi režiimi, terminali režiime, sealt sai 
alla tirida tarkvara. See oli meie tegevuste põhiskoop lisaks siis mängimisele 
ja selle tarkvara uurimisele, mis me saime.
                 
\question{Ehk, kogukond tuli põhimõtteliselt BBS-i adminide hulgast. Aga tuumiku 
ümber pidi ju olema ju ka kasutajaid. Oskad sa suurusjärgus hinnata, palju neid 
võis Eesti peale olla?}

Ma arvan, et alguses, üheksakümnenda aasta paiku, võis olla mingi paarkümmend 
kasutajat per \emph{node} ja neid \emph{node}-sid oli ka nii viie kuni kümne 
ringis. Hiljem, interneti tuleku eelsel ajal, läks asi juba päris suureks ja 
massiliseks aga selleks ajaks olime meie juba sellest kogukonnast eemaldunud. 
Peamiselt seetõttu, et äri võttis kogu aja üle. 

\question{Arvutame siis. 20 inimeste per \emph{node}, 5-10 \emph{node} 
teeb\ldots} 

Vast paarsada inimest üle Eesti, neid võiks ka  rohkem olla. Tollal oli ka 
see asi, et igalühel ei olnud oma arvutit ja oma modemit vaid see oli selline 
\emph{round-robin}, et 10 inimest sama arvuti ja modemiga. 

\question{Nojah, klassitäis kaake helistab sisse, mine võta kinni, palju neid 
seal on}                 

Ei, vaata, neid \enquote{klassitäis kaake}  väga palju ei olnud, sest et 
ikkagi enamasti olid meievanused tegelased, oli ka natuke vanemaid. 
Päris sellist akadeemilist seltskonda, 
teadlasi ja uurijaid, ma ei mäleta. Neil olid ilmselt omad vahendid ja 
võimalused. Ka Usenet oli kusagil olemas.

\question{Panite oma HNS-i püsti ja hakkasite Nõukogude Liidu avarustest juppe 
ja neid Eestis maha müüma?}

Pigem vastupidi. Me hakkasime tooma Saksamaalt arvutijuppe, neid siin kokku 
panema ja siis Eestis maha müüma. Võib-olla tänu sellele, et isa kõrvalt 
tekkis varakult selline impordi-ekspordi kogemus, mis tollal oli üllatavalt 
haruldane kompetents. Et kuidas väljamaalt midagi osta ja üle tollipiiride 
Eestisse tuua. See oli (20 aastat enne e-poode) siiski  suhteliselt 
haruldane teadmine. Esiteks, kuidas leida üldse see kontakt, kellele helistada, 
Kuidas talt hinnapakkumist saada. Sul ju ei ole aimugi, kes müüb. Sa ei tea, 
kellele helistada, et talt üldse hinnapakkumist küsida. 

\question{Aga kust sina teada said?}

Tänu sellele, et isa oli selline aktiivne tegelane. Kuna tema  inglise keelt 
väga ei osanud, siis ma pidin jooksvalt ka tema asju ajama ja siis tema 
eksimustest ja õnnestumistest, eks siis sai õpitud. 

\question{Eks me kõik seisame hiiglaste õlgadel. BBS-induses alguses mingit 
ärilist aspekti ju ei olnud?}

Ei. Puhas selline fänlus. Paljuski seisis ta kahel jalal. Esiteks suhtlemine, 
ehk  vestlustoad, inimesed omavahel suhtlesid erinevatel teemadel, selline 
\emph{community} värk. Ja teine oli siis softi vahetamine.

\question{Tollel ajal hakkasid ju esimesed jutukad ka tekkima?}                 

Jah, ma arvan, et  jutukad saidki paljuski alguses nendest samadest BBS-ide 
tubadest.

\question{Mis need esimesed olid? Ma isegi ei mäleta nimesid, ei ole kunagi 
neis käinud?}

Ma jutukates ei ole väga palju olnud. Kas OK jutukas\index{OK jutukas} ei olnud 
üks ja Cafe\index{Cafe jutukas} vist oli teine. Need hakkasid tekkima siis, kui 
ma nagu enam väga selle sotsiaalse tegevusega ei tegelenud, kogu aeg ja energia 
läks oma firma arendamisele.
                 
\question{Kuidas see äri tol ajal toimus? Üheksakümnendate keskel oli 
suhteliselt palju veel kogukondlikku toimetamist, kasvõi .EXE tegemine.}

Jah. Ma arvan, et too äri seisis sellesama kogukonna õlgadel, mis Fidonetist  
alguse sai. Eks lõpuks pidid ju kõik kusagil tööd tegema. Ja pigem lihtsalt 
needsamad suhted ja asjad liikusid ärisse edasi. Paljuski needsamad nimed, kes 
ka seda ajakirja .EXE\index{.EXE} tegid olid Fidonetist tuttavad. 

\question{Sina keskendusid siis täitsa ärile?}

90.-te algusest või keskelt nagu läks enam-vähem selline elu käima, üheteist-tunnised 
tööpäevad ja. 

\question{Sealt kõrvalt vist väga palju enam programmeerimiseks aega ei jäänud?}

Ei. Me tegelesime eelkõige riistvara vahendamisega. Riistvara, võrgud, selline 
suhteliselt primi tegevus. HNS\index{HNS},  hilisem Zebra 
Infosüsteemid\index{Zebra Infosüsteemid}, nagu tarkvaraarenduseni ei jõudnudki.

\question{Isegi tavalise võrgu ehitamine oli ju koaksiaalkaablil?}

Koaksiaalkaabli vedamine oli mul väga selge. Mitmete tänaste suurfirmade 
laudade alt sai roomatud ja tolmu pühitud ladudest ja seintelt ja lagedelt.

\question{Jaa. Mäletan Ühispanga kontorit, kus kontoriarvutid olid seljaga 
kliendi poole ja kaabel jooksis masina juurest masina juurde. Järjekorras 
seistes oleks võinud lihtsasti termika maha keerata ja terve kontor oleks 
seisma jäänud.}

Siis, kui me saime kunagi Seesamile\index{Seesam Kindlustus} IBM Token Ring 
nimelist võrku vedada. Token Ring oli 
 esimene selline tähtkujuline topoloogia, millega me kokku puutusime. Sellel  
olid sellised rusikasuurused stepslid, siis see oli täielik müstika.
 
\question{Kust teil tuli mõte sihukese müstikaga tegelda, see hakkab ju 
tasapisi üle piiri kasvama, mille endale lihtsalt katsetades selgeks teeb?} 

Selles mõttes, et ega nagu äris ikka. Sa võtad mingeid riske ja ütled, et 
\enquote{loomulikult me saame selle asjaga hakkama}. Ja mis see siis nüüd ära 
ei ole. Hakkad tegema ja  selgub, et  kõik töötabki niimoodi, nagu ette nähtud.
Ja ega  nende võrkude ja asjade üles panek ei olnud  progemisega võrreldes nagu 
mingi eriline tegevused. Lihtsalt konfigureerid süsteemi ära ja jalutad koju.

\question{Tol ajal koolist niisugust teadmist vist üldse saada ei olnud 
võimalik}

Absoluutselt. Mina lõpetasin polütehnikumis\index{Tallinna Polütehnikum} 
raadioside ja levi eriala. See valdavalt põhines lamp-elektroonikal, kuigi 
meile õpetati ka pooljuhte ja nii edasi, aga loogilisi skeeme, loogilisi 
ahelaid, kõike seda me saime vist pool aastat. Ja sedagi  
teoreetiliselt. Enamus  meie  elektroonika õppimisest polütehnikumi ajal oli 
väga selline analoog-elektroonika: lambid, elektromehaanika, samm-valijatel 
põhinevad telefonijaamad. Noh, mis oli selline paras küberpunk tänapäeva 
mõistes juba. 

\question{Kui ma kuulan su juttu, siis sealt kumab välja soov asjadest aru 
saada, mõista, mis on karbi sees?}

Absoluutselt. Et ma läksin polütehnikumi, oli pigem nagu selline perekonna 
traditsiooni järgimine. Ma olen soodumuselt pigem nagu humanitaar olnud, kuna 
sinnamaani, ehk tehnikumi minekuni, olid mul neljad-viied kõik humanitaarained: 
keeled, kirjandus. Kõik reaalained olid kahed-kolmed.

Ilmselt siis tasub õppida seda, mida sa ei tea, ülejäänud tuleb lihtsalt niigi. 

Aga noh, see on andnud  selle tugeva külje või plussi või aluse, et sa oskad 
asju, millest sa aru ei saa, üldistada või teisendada sellisteks mustadeks 
kastideks, millel on mingid defineeritud sisendid ja väljundid, mille põhjal 
saad sa omakorda teha mingeid järeldusi selle musta kasti sisu kohta. 
Selline paradigma oli seal koolituses pidevalt olemas, ma arvan, et see on 
siiski selle põhja ja vundamendi andnud.
                 
\question{Nojah, mis seal kasti sees on, ei jõua sulle keegi õpetada, sest 
homme on seal teine asi. Aga lähenemine on kasulik.}

Vaata, tollal muutused olid palju aeglasemad. XT-d olid ikkagi  aastaid 
ja 286-t oli ka ikka aastaid, enne kui 386 ja 486 tulid. Lihtsalt nüüd on see 
muutuste tempo  radikaliseerunud.
                 
\question{Huvitaval kombel see alati tundub nii, et just praegu on palju 
kiirem, kui vanasti.}

Võib-olla jah, on see, et vanemaks saades elu tempo ise muutub.

\question{Mis hetkel see kogukond hakkas selgelt alla jääma sellele, et igaühel 
oli laen ja liising ja pere?}

BBS-i kogukond tegelikult oli ju mitte ainult need \emph{bulletin board}-id 
vaid toimusid ühisüritused. BBWinterid\index{BBWinter}, 
BBSummerid\index{BBSummer}. Iseenesest, need said alguse just sellestsamas 
süsoppide esimesest kokkutulekust. Hiljem hakati juba laiemaid üritusi  
tegema, kuhu olid kutsutud ka BBS-ide kasutajad. Ja seda tegelikult jätkus  
Interneti tulekust veel edasigi, et siin vist äsja veel arutati, et kellel veel 
BBS töötab. Ma üldse ei imestaks, kui veel leitaks Eestist mõni töötav BBS 
kusagil virtukas tiksumas. 

Ma arvan, et minu jaoks ta loomulikult vajus laiali rohkem sellega, et ise ei 
jaksanud sinna enam panustada. Aga seal elu ja tegevus toimus veel 
aastaid-aastaid hiljemgi. Ta hakkas  laiali vajuma sellest, kui 
internetipõhised keskkonnad hakkasid  kasutajaid lihtsalt üle võtma. 
Suhtluskeskkond sulas igapäevase töö  ja muude tegemistega seotud keskkonnaga 
üheks, modemiga kuhugi helistamine tundus nagu pisut arusaamatu.

\question{Et kui BBS-is vahetati faile ja aeti juttu, siis Internetis sai tööd 
ka teha?}

Absoluutselt, Internetis olid need Cafe ja OK jutukad  olemas ja ega enne 
Facebook-e ja Rate-sid olid ju ka olemas Geocities ja mis iganes need keskkonnad 
olid. Ega need ei ole mingid uued asjad. Nad on vahetanud kesta ja vormi ja 
platvormi ja värvi ja natuke funktsionaalsust, aga inimesed on juba 30-40 
aastat liikunud sellesama tegevusega  ühest keskkonnast teise.
                 
\question{Kes BBSummereid ja BBWintereid korraldas ja kes seal käisid?}

Ma mäletan ühte korraldajat, kelle kasutajanimi oli vist Kristrap aga Piret 
Part\index[ppl]{Part, Piret} oli vist pärisnimi.

\question{See oli puhas kogukonna värk, mingit sponsorit taga ei olnud?}

Hiljem, kui üritused suuremaks läksid, siis me HNS-iga firma poolt loomulikult 
sponsoreerisime. Aga alguses need üritused olid piisavalt väikesed. Et 10 
inimest omavahel kokku tuleksid, natuke asja arutaksid ja selle juures mõned 
õlled teeks, ei vaja sponsorit. Hiljem  arvutifirmad toetasid, samad inimesed 
töötasid ju paljuski arvutifirmades, olid olulised tegelased kui mitte 
omanikud, ja panustasid.

\question{Ehk, Eesti Fido kogukond läks sujuvalt üle Eesti IT-tööstuseks?}

Jah, absoluutselt. Ega paljud tuttavad nimed on ju sealtsamast pärit. Tõnu 
Samuel\index[ppl]{Samuel, Tõnu} näiteks on ju samamoodi sealt keskkonnast pärit. Ma arvan, et lihtsam on 
võtta ja üles otsida tollased nimekirjad Fidonetist, mingi jututubade 
\emph{printout}-id, mida vist Mamers\index[ppl]{Mamers, Tarmo} peab ja vaadata, 
kes need nimed on, kes seal eksisteerivad. Ma arvan, et sa leiad nad praeguste 
it-meeste gruppidest üles.

\question{Kes neil üritustel käisid? Süsopid, ma saan aru, aga lisaks?}
                 
Ega ma ei oska sulle öelda. Süsopid loomulikult, aga eks oli neid, kes ei 
tahtnud, ei viitsinud või ei saanud ise BBS-i pidada. Aga kellele see oli 
lihtsalt üks selline asi, mida arvutiga teha. Külastada BBS-e.

\question{Nojah, kuna progemiseks olid barjäärid kõrged ja kogu aeg mängida ei 
jaksa, siis tehti midagi muud ka.}

Eks mõned tulidki läbi progemise huvi. Minule endale tuli see läbi mänguhuvi, 
eks ole. See oli võimalus suhtlemiseks ja tarkvara vahetamiseks,  ka tehniliste 
teadmiste vahetamiseks.

Tollal kusjuures ju suhtlemine oli oluliselt raskem, kõigil ei olnud 
isegi mitte kodus telefoni. Tänapäeval tundub  kõik nagu käeulatuses, sa 
pead lihtsalt taskust telefoni võtma, Google'isse otsingu panema. Tollal sa 
pidid otsima telefoni, mõtlema, kellele helistada, et küsida, kas ta teab 
kedagi, kes teab midagi. Fidonet andis ju ka  selle võimaluse, et sa said 
avalikku \emph{board}-i panna mingi küsimuse, et kas keegi teab, kuidas 
lahendada ühte või teist probleemi.

\question{Tuli vastuseid ka?}

Kindlasti. Kus nad pääsesid. Eks see sõltus  küsimusest. Eks seal oli asjalike 
jutte, olid oma läbu kohad, kus niisama jaurati, tehnilised vestlusringid.
                 
\question{Jah, aeg paneb asjad teise konteksti. Kui sul isegi kodus telefoni ei 
ole, siis võimalus rahvusvaheliselt inimestega suhelda on märkimisväärselt 
teise väärtusega, kui siis, kui sul on taskus mobiiltelefon ja Internetti 
ühendatud arvuti alati käepärast.}

No just. Mõtle sellele, et Google'it ei olnud tollal.

Google oli sul üle laua su kolleeg, naaber ja sõber ja sul endal pidi olema 
selline kontaktibaas ja varamu piisavalt suur ja lai, et teaksid kedagi, kes 
teaks kedagi, kes teab kedagi, kes oskab sulle öelda midagi.
                 
\question{Kas sa oled nõus Pronto ütlemisega, et väga paljus seesama 
kontaktibaas võimaldas väga loomulikult kogukonnast ärisse üle minna sest ka 
äris sõltus palju kontaktidest?}

Seda võiks ka nimetada \emph{street reputation}-iks, eks ju. Sul oli 
\emph{credibility} mingil määral olemas. Sind juba teati ja tunti selles 
keskkonnas. Kui tänapäeval räägitakse \emph{Estonian Mafia}-st, siis tollal see 
oli Fidoneti seltskond.

Loomulikult, eks paljud meie kliendid tulid läbi Fidoeneti või vähemalt teadsid 
meid sealt kaudu.

Paljud klientidest töötasid hiljem, eks ole, mingis pangas või firmas, firmad 
laienesid, tahtsid saada arvuteid. Kusagil oli neil palgatud mingi itimees, 
kes pidi selle probleemi lahendama ja ega tal ka ju ei olnud Google'it või 
e-poodi, kust  parimat pakkumist küsima minna. Tal oli endal ka inimesed, keda 
ta teadis ja usaldas, kelle käest siis seda pakkumist minna küsima.
                 
\question{Ei olnud nii, et lähed poodi: \enquote{Palun mulle 16 arvutit}.}

Ega see arvutiäri alguses oli ka selline, et kuna valuutat 
ju väga palju kellelgi ei ringelnud, siis laoseisud olid olematud. 
Põhimõtteliselt võeti ettemaks, ettemaks maksti välismaale, selle eest oodati, 
kuni arvuti kohale laekus, siis pandi see kokku ja tarniti kliendile. Kui hästi 
läks, sai alla kuu aja kätte. Ka see eeldas tegelikult ju usaldust, et sa 
annad kellegile 1000 raha. Arvutikategooria mõttes 
hinnad ei ole eriti muutunud: hea arvuti oli üle 1000 ja tavaline 1000.

See, et sa annad mingile matsile selle raha ära, ta ütleb, et \enquote{ära 
muretse, kuu aja pärast saad kätte}, see eeldab üksjagu usaldust. Ega valuutat 
ei olnud, enamasti tehingud alguses tehti rublades aga rubla ei olnud 
konverteeritav millekski muuks kui rublaks. Ja oli veel see aeg, kui paljudel 
asjadel oli kaks hinda: ülekande rubla hind ja sularaha rubla hind. Odavam oli 
sularahas, sellepärast et sa ei saanud alati pangast sularaha kätte. Olid 
mingid kindlad hetked, millal panka toodi sularaha ja siis sa pidid teadma, kas 
õigeid inimesi õiget aega, et saada sularaha.

\question{See muidugi seletab kõiki neid legende, kuidas arvutifirmas ja pangas 
hoiti sularahapakke kuskil kapis ja vetsus ja kus iganes.}

Kui taheti midagi ülekandega osta, siis see oli nüüd meie enda valik, et kas me 
müüsime midagi valuuta või rublade eest. Kui  ülekande eest otsustasime müüa, 
siis oli hind kallim lihtsalt puhtalt seetõttu, et pärast selle raha kätte 
saamine pangast oli nagu oluliselt keerulisem.

\question{Ja sellest ajast sa oledki jäänud niimoodi arvutitega tegelevaid 
inimesi juhtima?}

Nojah, mida aeg edasi, seda enam on mind 
huvitanud rohkem sisulised asjad, kuidas asjad töötavad. Ja vähem huvitanud 
inimeste juhtimine. Ütleme niimoodi, et inimesed on keerulised, arvutid on 
lihtsad.

\question{Kuidas sa infoturbe ja selle maailma juurde jõudsid?}

Esimest korda me jõudsime sinna juurde läbi väga praktiliste sammude. Me nimelt häkkisime 
Täitevkomitee arvuteid. Selleks, et sinna ligi  ja mängima saada, me 
avastasime, et tollal Õnnepaleeks kutsutud majas on olemas üks modemite peal 
töötav teenus, mille 
 kaudu sai abielusid, sünde ja surmasid registreerida. Ja muu hulgas sellesama 
modemi otsas sai eraldada kortereid. Nimelt kortereid ei saanud tollal osta, 
vaid neid eraldati sulle riigi poolt.

\question{Ja teie avastasite koha, kuhu sai sisse helistada ja eraldada 
kortereid?}

Me avastasime, et sinna sai sisse helistada, aga teenus oli parooliga kaitstud 
ja parool oli jällegi eesnimi ja perekonnanimi. Me alustasime nii-öelda 
tagumisest otsast sellel infoturbel.

Mina ise sattusin Privadori\index{Privador}, siis juba  10-15 aastat hiljem 
aastal 1990, kui Tarvi Martens oli tollasest Küberneetikast\index{Küber} teinud 
investorite kaasabil sellise asja, mida tol ajal nimetati \emph{spin-off} ja 
mida tänapäeval kutsutakse \emph{startup}. Privadori  eesmärgiks oli  
digitaalselt signeeritud dokumentide pikaajalise tõestusväärtuse loomise 
süsteem nimega TruSign. Seda muidugi hakati tegema kaks aastat enne ID-kaardi 
projekti algust, enne kui ühtegi signeeritud dokumenti keegi näinudki polnud.

\question{Väga huvitav. Krüptot enam paljakäsi ei tee, selleks on haridust 
vaja. Ehk, juba üheksakümnendate lõpus pidid omavahel kokku saama inimesed, kes 
olid iseõppijad ja minigid teistsugused inimesed, kes kindlasti ei olnud 
iseõppijad. Kuidas see käis, seal mingisugust hõõrumist ei tekkinud?}
                 
Enamus tollastest kolleegidest olid vanad tuttavad Fido ajast. Mina Privadori 
liitusin  küll turunduse ja müügi funktsioonis. Alles hiljem hakkasin 
tootejuhtimise ja nii-öelda juhatajana seal tööle. Aga algne funktsioon oli mul 
seal pigem müük. 

\question{See Fido seltskond ei olnud väga suur ju}

Ei olnud. Tegelikult kogu see seltskond, kellel üldse oli üheksakümnendatel 
ligipääs arvutitele, ei olnud väga suur. See oli veel see aeg, kui arvutid olid 
peamiselt firmades aga mitte kodudes. Üheksakümnendate lõpus alles hakkas 
tekkima see trend, kus firmad olid enam-vähem arvutitega varustatud, hakati 
ostma rohkem rakendustarkvara (või noh, ütleme turu trendid liikusid rohkem 
rakendustarkvara poole) ja eraisikud hakkasid endale koju arvutit ostma.

Ega tegelikult  sovhoosidest ja kolhoosidest parematel olid omad 
arvutuskeskused olemas juba nõukogude ajal. Arvuti kui selline Eestis ei 
tekkinud päris üheksakümnendatest, vaid see ikka oli ammu enne minu sündi 
olemas. 
