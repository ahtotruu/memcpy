\index[ppl]{Reimo, Tõnis}

\question{Kuidas sina arvutite juurde said?}
                 
Ma arvan, et see minu alguse lugu, nagu ma arvan, paljudel tal on ammu enne seda Peebeeessindast, et et ta on ikkagi otseselt ikkagi seotud tänu sellele, et isa töötas tollal arvutite teemal, vist oli Rahvusraamatukogu. Tollal oli Kreutzwaldi nimelise raamatukogu mingi arvutuskeskuse arenduskeskus mingi sellise imeasjaga seotud ja, ja sealt, eks ole, pääsesin arvutite ligi ja, ja tollal tekst esimesed sellised.
Arenenumad kirja kirjamasinad. Mis, mis vanuses see umbes oli mingisugune viie aastane ei olnud, nii et ma arvan, et see võis olla mingi viies, kuues klass, et.
Et noh, tollal mindi aasta hiljem kooli, nii et siis tänapäevase mõistega mingi kuues, seitsmes klass. Ja ja sealt tekkis selline võimalus ligi pääseda neile üldse ja loomulikult mänge mängima saada, sest see oli nagu ainuke tollal, nagu uitas päriselt ka.
Ütleks niimoodi, et progemisestele veel asi kaugel okei, ega mänge ja ei olnud. Jah, eks see oli nagu, kas olid. Malmist põranda külge needitud.
Või, või siis olid need esimesed arvutimängud, mis olid umbes nagu jooksid.
Pigem trükimasinal kui arvutil, mis arvutit, no tol ajal olid seal
Eks sain natukene näpitud mingid jõsukesi ja sellised Vene toodangut, aga aga tekkisid mingid esmased personaalarvutid nagu snaiper ja, ja.
Selline Saksamaa Lääne-Saksamaal selline importtoodang.
                 
Kõneleja 2:
See oli siis mingisugune ykski analoog. Kas ta oli mingi XD ana?
                 
Egon Elstein:
Näitusele eelnes eelne okei. Aga seal oli juba võimalik nagu mingeid mingeid primitiivseid mänge mängida ja sealt nagu vaikselt, nagu see huvi arenes, aga mida nende arvutitega seal päriselt nagu iga päev tehti, seda käisid mängimas seal ja ei noh, seal neid, millega nagu täiskasvanud, milleks täiskasvanud neid rakendasid seda, nagu ma alguses nagu aru ei saanud ja väga ei huvitanud ka.
Minu jaoks olid nad lihtsalt nagu tüliks, sest nad takistasid arvutisse saamist. Tegime igasuguseid asju, just, et see arusaamine, mida täiskasvanud arvutiga teevad, see tuli alles nagu aastaid hiljem.
Siiski nagu tulidki esimesed katsetused progeda, aga, aga.
Noh, sealt kasvas ka selline mingi laiem huvi välja, et säilitatud sellise Jaak laande veetud arvutiklubiga. Mis selle nimi oli jaaklaande, aga selle klubi nimi oli ahaa, ahaa. Ahaa, loodi minast mingi 85 86 Jaak Londoni siis kolmanda keskkooli, selline legendaarne matemaatikaõpetaja
Tema vedas sellist arvutiklubi alguses mingi ka nõukaaegse toodangu peal, see oli mingi Jesso laadne masin, millel meie mõistes ekraan ei olnud, aga siis ta siis oli teletaip, oli põhimõtteliselt ehk siis tekst trükiti paberile. Arvuti, tule oma tulemuse trükkis ka paberile, et seal oli, põhimõtteliselt oli selline printer või noh, selline nutikas trükimasinaid sinna lõid käsud, käsud, käsud trükiti paberile sinnasamasse paberisse, sinna sinu käskude alla trükkis masin oma vastused. Ja noh, seal loomulikult oli, eks ole, mingid lindi lindi pealt sisse lugemisvõimalused ja selliseid asju.
                 
Kõneleja 2:
Mälu oli ka, eks? Selles mõttes?
                 
Egon Elstein:
Päris ei ole lindi peale et seda see natuke näppida, aga noh, see oli rohkem selline. Sealt läks nagu läks juba kiiresti üle.
Nendele esimestele mis olid
Et sealt edasi tulid juba Jukud ja viskad ja, ja kogu see, ärme ärme loendus.
                 
Kõneleja 2:
Ärme veel edasi, lähme, kas, kas see oli, nagu see sinu arvutiklubisse sattusid, kes oli sellepärast, et see tundus nagu huvitav. Et mis peale mängude veel teha saab või oli see keegi sõber, kutsus või üles?
                 
Egon Elstein:
Ma isegi ei mäleta, kuidas see sinna võib-olla see asi, et ma esimesed kolm aastat oma elust käisin ise kolmandas käsku lihtsalt, et, et seetõttu oli nagu mingi mingi seos olemas teadsid seda inimestel võib-olla selle kooli kaudu, siis teadsin, et ma, kuidas ma sinna sattusin, ma ausalt öeldes enam ei mäleta. Aga.
                 
Kõneleja 2:
Lihtsalt niimoodi konteksti mõttes, et, et kui sa nagu arvutitega tegelast, kes siin käis juurde vaata arvuti inimesed, klassikaliselt mingisugused Saifay sõbrad ja nihukest asja, et kes siukseid asju
                 
Egon Elstein:
Ei no ikka, eks ole, Soome televisioonist sai Star galaktikat vaadata.
Kriminaalseeriat, eks ole ja ikka loomulikult kandilised plekist nimelt saelanit, seal tuligi edasi-tagasi. Ja loomulikult kogu nõukaaegne Saifay kirjandus oli läbi töötatud, niipalju kui kätte hullu kätte ulatus.
                 
Kõneleja 2:
Mingit konkreetset Niukest, Haylaiti on ka meelde jäänud. Oskad öelda?
                 
Egon Elstein:
No ikka need, mis olid seal maailma mõnda sarjastaks, nendest sai alustatud, et seal keerukam ja elegantsem osa, see kõik tuli hiljem, et tavaliselt et.
Praegu ma panen puusalt igasugused purpurpunaste pilvede maad ja kõik sellised asjad siis.
No see oli pigem sellise nõukanostalgia vaimustuses kantud selline Saifaigusega selline. Jutustas kuidas kommunistliku ühiskonna kangelaslikult, aga inimesed kosmost vallutavad.
                 
Kõneleja 2:
Tulid ju ei olnud eesti keeles nagu palju saada, mis tähendab siis seda, et nagu sul mingi seltskond luges väga paljus nagu samu raamatuid ja loomulikult mis vist tähendab siis seda, et ilmselt staadioni üksteisest paremini aru?
                 
Egon Elstein:
Ja see oli üks asi, aga noh, enamus seltskonnast, kellega mina tollane kokku puutusin ta oli ikkagi suhteliselt piiratud, sest et need olid enamasti nagu noh, omavanustest olid nad ikkagi selliste inimeste lapsed, kes töötasid arvutitega. See tähendas seda, et nende vanemad olid kas KBFIs Tallinna Tehnikaülikoolis või mingites nagu asutustes, kus olid siis arvuteid ookeani. Et ta oli nagu siiski sellise noh, ma ei tea, tehnilise intelligentsist nagu et seda päris juhuslik kogu rahvast väga palju seal ei olnud, et sellega mingi sellist
Kuidagi need, kes olid nagu varem kokku puutunud, niikuinii, oli kirjanduslik huvi, millalgi tärkas kirjanduslik huvi, aga noh, arvutitega kokku puutunud, et see ei olnud minu arust mitte niivõrd nagu Saifayst nagu lahti pääsenud, aga aga kuivõrd lihtsalt arvutitest minu arust nii palju, kui mina nägin, siis enamus 80 prossa, kellel oli keskendunud arvutitega mängimisele
Aga mis me kõik hakkame kuskilt pihta, sealt edasi sellised praktilised probleemid, et kuidas nagu mänge kopeerida ja kuidas mändi avada ja kuidas need üles otsida ja sealt tegid apsesteemiga tutvuste ja lõpuks siis tekkisid nagu esimesed huvitatud, aga kuidas neid ise teha?
                 
Kõneleja 2:
Aga kuidas sa käisid seal ahhaa klubis, eks ole. Kas kaua-kaua sa käisid seal?
                 
Egon Elstein:
Ahaa, klubil oli rohkem nagu üritused, et, et tal ei olnud nagu sellist. Nii palju kui mina mäletan, et tal olid nagu sellistel mingitel kindlatel päevadel oli sul ligipääs kusagil arvutitele, kusagil oli, oli mingi tööjõureservide õppekeskus siis oli tehnikaülikaal mingid sellised veidrad.
Ja eks me aja jooksul tuli neid kohti juurde, otsisime juurde ja siis kellelgi jälle vanemad või sõbradele sokutasid ja siis me niimoodi käisid. Rändtirtsude lendasime peale vabale arvutusressursile, aga need täiskasvanud inimesed, kes seal
                 
Kõneleja 2:
Üritasin tões ja väes, loodetavasti siis töötajana arvutitega sneri pahandada.
                 
Egon Elstein:
Enamasti see oli ikkagi nagu sellises töökeskkonnas nagu eraldatud, et ma ei mäleta väga palju noh, väljaspool Ta isa töökeskkonda, et me oleksime otseselt nagu mingis tööruumides hullult olnud, et sa klubi värk oli, noh selles mingit õppeklassid mingid sellised kohad loomulikult hiljem vaadates siis kindlasti sai kõvasti närvidele käidud isa kolleegidele.
Ja selle sellel moel, et kas nende nii-öelda arvutiterminale hõivatud siis kui nad tahtsid tööd teha või või siis aga siis see oli nagu selline põnev mängijad, peitsid mängud ära ja siis meil oli jälle põhjust vaeva näha nende ülesotsimiseks, millest ma järeldan siis seda, et ega täiskasvanud trollid ka mängud kuskilt ikkagi sinu masinast sallid õlidega ei käi, ega ma ise neid sinna ei pannud, sest no mõningates panime, kui tollase Tallinna linna täitevkomitee keldrisse tekkis selline juuniksi masin mis oli tollal siis üks lisatöökohtadest.
Siis sinna ise midagi hullult ei kopeerinud, et see oli siis juba varustatud sellega, mis seal oli ja siis tuli huvitav seal seal nii-öelda orienteeruda kähku SU käsud selgeks.
Ruudud ja.
                 
Kõneleja 2:
Huvitav on siis see, et, et alguses, kui juuliks purgid Eestisse jõudsid, siis olid need pigem nagu suletud. Ja kõik muu oli pigem nagu kus sa igasuguseid asju sokutada ja ise teha.
                 
Egon Elstein:
No ega sa Yessakesse ka, noh, sa kindlasti said sokutada teda, eks ole, sest et, aga lihtsalt sinna programmeerimine ja programmil sokutab meil väga tülikas, sest et seal oli rohkem Se perfokaartide kaudu see aga
                 
Kõneleja 2:
Kustkaudu niisugune informatsioon liikus, et et ma saan selle arvutiga teha midagi või?
                 
Egon Elstein:
Tavaliselt seal oli ikkagi niimoodi, et sa läksid kas kellelgi sõbra vanemate töökohta või siis sinna haa raames või siis oma vanemate töökohta. Sellisel.
Pärast kooli ja siis toimetasid seal.
                 
Kõneleja 2:
Jaa, aga ma mõtlen just seda, et, et sa räägid suu ja ruut ja asjad said selgeks, kust need selgeks said.
                 
Egon Elstein:
Ei, no eks ma praegu loomulikult ei mäleta, et oleks, tõenäoliselt ta pidi tekkima seeläbi, et vingusid oma mängu senikaua, kuni keegi sulle midagi ette näitas ja ega noh, ma usun, et omal ju vaata, kui vanalt nagu nutitelefon sellega saadakse, et noh, enam-vähem samal ajal koos rääkimisega, et noh, ega asi siis nüüd selle super luuseri kasutamine selgeks saada oluliselt vanemale, kusjuures nii võtta siis tõepoolest ja seda enam, et tollal see infoturbeteema oli suhteliselt olematu, et kõikide laginal eesnimi ja kõikide parool oli tema perekonnaseenega
                 
Kõneleja 2:
Olla konstruktiivne küll, vaata siis ma mõtlen nüüd niimoodi, et siis oleks võinud ju olla ka teisiti, kui tol hetkel oleks olnud nagu infoturve oleks paremini paigas siis ei siis terve põlvkond inimesi oleks saanud ilma arvutita sisuliselt.
                 
Egon Elstein:
Ei noh, ma arvan, mitte, sest tegelikult ega nad ju said sinna vähem, minu väide on see saidi läbi vanemate ja eks sa ikka nahksed oma isa ema kallal, senikaua, kuni ta siis on selle mängu käima paneb, et leks siis vahepeal vaatad, jälgid, paneb tähele, mida tehti, kuidas? Kuidas sai?
Aga ma arvan, see tase on erinev, sest et mõned nagu tundsid nagu põhjalikumalt, võib-olla programmeerimise vastu huvi, mina nagu alguses vähem, et ausalt öeldes see programmeerimine ei ole nagu läbi elu selline tugevam külg olnud. Ma rohkem ma olen sellise müügi, turunduse, juhtimise, projektijuhtimise, tootejuhtimise kallakuga läbi elu.
                 
Kõneleja 2:
Aga teades, mis tooteid sa oled juhtinud, seda ju ei saa teha ilma aru saamata väga hästi, mis seal kapoti all käib.
                 
Egon Elstein:
Ja ei, see on teine eel, noh see huvi on ka loomulikult alati olnud, et kuidas, kuidas asi töötab. Okei, et, aga noh, selleks ei pea alati seda ise tegema.
                 
Kõneleja 2:
Okei, aga kui sa nüüd said tonks paremaks, siis mingil hetkel see niukene külakorda käimine nagu
                 
Egon Elstein:
Ja et
Jällegi, eks ole, ühiskond läks edasi. Alguses.
Tegutsesin jällegi isa loodud Eesti-Rootsi ühisfirma tiival kus olid arvutid, personaalarvutit kaks, 83, kaheksa kuued, neli kaheksa kuued hiljem ja sealt edasi siis lõime ka oma firma, aga noh, alguses sai nii-öelda selle ühisema ruumides tegutsetud ja, ja selle nagu tiival tegeletud siis nii-öelda arvutite maaletoomisega ja jällegi sama eesmärk, et ja enamus aega läks mängimisele, aga vaja kusagilt nagu saate seda riistvara, millega mängida, et siis nagu see oligi üks draiver etet. Et sa ostad, mängib mõnda aega meid maha. Ja nii see nagu äri käima läks, et noh, hiljem võttis nagu äri nagu selle mängimise üle, sest et siis ei olnud enam aega nagu mängimisega tegeleda. Kogu aeg nagu äridele võib olla nii positiivne kui negatiivne. Aga see on pärast keskkooli, eks ole, ja see on kusagil niimoodi pärast tehnika. Kuulge, ma lõpetasin Tallinna polütehnikumi üheksakümnendatel 90 90 vist 91 vist tegime siis oma firma Haaennessi nimelise mis omakorda tulenes
Juba enne seda, siis loodud sellest pressist ja mis kandis säkkers nats süsteemi nime ja seda häkkerit näiteks islami tekkinud tühja koha peale, vaid tegelikult. Enne seda oli olemas Lembit pirni, Eesti BBS üks. Lembit Pirn tegutses.
Täna tornimäel asuvas sellises madalas valges hoones pärast laagrist üle vaadata, see oli tollal oli mingid transfer, transpordi, informaatikakeskuse, midagi sellist okei. Ja tema, temal oli siis esimene muudenigadetav BBS püsti pandud sedasi kõvasti nagu külastatud. Miks, miks ta seda teinud, miks, mis seal külas?
Miks seal külas oli vaja käinud, miks ta seda tegi, seda peab Lembitu käest küsima. Aga.
Eks seal oli väljas ühelt poolt mänge teiselt poolt, eks ole, seal oli mingi suhtluskeskkond, seal hakkas tekkima juba nagu selline põletin board. Et ta läks nagu käima jällegi tarkvaravahetuse pealt, noh mis tollal oli täiesti tavaline, tänapäevases mõistes nagu räigelt illegaalne tegevus. Aga noh, nõukogude ajal see mõiste oli võõras, see on nõukogude aegne.
                 
Kõneleja 3:
Okei, et nõukaaja lõpus üheksakümnendad noh.
Sõnast selline ülemineku aeg, et.
                 
Egon Elstein:
Siis infot, nagu teiste Pebesside kohta maailmas hakati, sai hakatud Soomes külastama, mis tuli Jouni Salo BBS ja Randvaid suhtlema ja rand vait oli siis Fidaneti euroopamist.
Nõudi pidaja minu arust Euroopa tsooni pida mäletama. Okei, ja eks nemad aitasid-juhendasid edasi ja sealt tekkis loomulikult järgmine samm mõte endal oma Pebest püsti panna. Et jällegi, et tänu nagu sellele, et me saime nii-öelda isa firma ruumides tegutseda, oli meil unikaalne võimalus teha.
Otse välismaale Kaukenesid aa et see, mis on tänapäeval suhteliselt elementaarne, et sa kusagile otsa helistad, ei olnud tollal isegi Eesti piires nagu elementaarne, nüüd kõik kõned tehti läbi keskjaama, keskjaama oli siis sellele kindel number, kuhu sa helistasid, kus võttis siis kõigi naisterahvas vastu kellele siis teatasid, kuhu sa tahad helistada. Lugesid oma numbri ette, siis kuliin vabanes, siis ta helistas. Noh, võiks siis helistas sulle tagasi ja teatas, et nüüd on siis kõne. Aga kuna tegemist oli Eesti-Rootsi ühisfirmaga, siis oli seal unikaalse võimalusena võimalik automaatvalimisega helistada ots maailmas siis igale poole. Äge. Ja see võimaldas siis ka mootoriga enam-vähem üle kogu magnalist kus need mudelid tulid. Vanem oli veel modem, esimene mingi kaasas.
Selles firmas olemas, see oli mingi 2402 2400 bitti sekundis läbilaskev modem, hiljem meil õnnestus siis Randvati ja nende kaudu saada esimene juues robotiks mis vist oli kas 9600 või, või midagi sellist ja ehk siis oluline edasihüpe. Ja noh, jällegi, et kuna tollal nendes BBS-ides liikus palju erinevat tarkvara noh, ma nüüd ei ütleksite Style legaalset aga see oli segamini nii seda kui teist siin, mis selles.
Ringi.
Surfamine sõbramine andis nagu ühelt poolt selle
Need vahendid ja ka selle teadmise oskuse kuidas, nagu erinevad tarkvarad töötasid, mida nendega teha sai ja nii edasi, et noh, sisuliselt kõik nagu ise õpitud. Ja puhtalt niimoodi katsetades eksitusmeetodil, et et seal isegi mäletan nõukaaja lõpus, kui veel
Lennata sai Nõukogude liidu piires vabalt siis Vladivostokist ja, ja Moskvast Leningradist.
Oli teisi Fidaneti kasutajaid lendasid külla selliste imitolliste floppida.
                 
Kõneleja 2:
Inimesed tulid Vladivostokist viie tolliste floppidega, Fidoneti.
                 
Egon Elstein:
Jah ei noh, selles mõttes, et kuna Modeniga imeda võttis nagu rohkem aega ja raha, kui lihtsalt võtta nagu sadu viierrelised klapilseid kaasa Coffesse lennata paistakis Tallinnas.
Okupeerida neid siis nagu paar ööd-päeva läbi.
                 
Kõneleja 2:
Ahaa, nii et need vene suunas oli ka tegelikult oli ju Fido Fidaside olemas.
                 
Egon Elstein:
Sest et nemad samamoodi helistasid peale meile eesti saitidele ja sealt liikus info nende suunas. Aga korra mainisid.
                 
Kõneleja 2:
VVS peres ühe ümber oli mingisugune kogukond, ka toimus mingisugune suhtlemine, kes need inimesed olid? Kas arvutiinimesed või?
                 
Egon Elstein:
Vot ma seda väga palju ei mäleta, et ma mäletan, sealt liikusime kiiresti edasi, et see, ta vist oli suhteliselt niimoodi staatiline, vaikne ja rahulik pärast seda loomist, et, et seal vist nagu väga nagu.
Nüüd sellist kommuuni nagu ei tekkinud või, või siis vähemalt ma ei mäleta, sellest tekkis ka moon rohkem nagu.
Siis, kui neid BBS-i pidajaid tuli juurde ja mingil hetkel sai kokku kutsutud siis esimene Süsoppide, selline nõupidamine, mis toimus Viru hotelli ma ei tea. 20. korrusel 22. korrusel olevas väikeses sellises.
Hotelli äärepealses toas. Seal oli lõvi, oli seal.
Mina Tarmo Ausing Tarmo mammers, Virko püssist. Ja tegelikult on selle kohtadest mammersel isegi mingi memuaarid kirjas sellest samast esimesest koosolekust. Nad olid mõnda aega veel isegi internetis üleval, et ma ei tea, kas seal jah, eks siis saab otsida jah, et, et aga, aga mammersi, kas saad sa selle ajaloo kohta nagu rohkem küsida?
                 
Kõneleja 2:
Kas kas see oli rohkem niukene sotsiaalne üritus või oli seal mingisugust probleemi ka lahendada?
                 
Egon Elstein:
Ta oli mõlemat, et seal oli natukene minu arust jutt oli nüüd sellest, et kuidas nagu Fidaneti nagu korraldada, organiseerida, kuidas seal mingit meililiiklust vist teha. Kuna meil oli tollal see väliskõnet nii-öelda tasuta käes, siis meie saime olla siis Eesti nii-öelda see esimene nõud kelle kaudu meili liiklus liikus Eestist välja, see meie on selles kontekstis ajakirjas ainult meestele. Et teised saatsid oma kirjad meile, meie saatsime öösel siis need kirjad üle Modeni.
Järgmisele Euroopa nõudile, kes siis need omakorda laiali jagas.
                 
Kõneleja 2:
Kas seal BBS-i püstipanekuks on ju mingit tarkvara ka? Aga see on niukene.
                 
Egon Elstein:
Kas see oli, see oli nagu suhteliselt sellise paketina saada, et samamoodi Peebeeessidest tõmbasid alla, ma ei mäleta. Aga mingi Maximus BBS või midagi sellist ja, ja see oli selline vahva platvorm.
Puttimisel bat failiga üles see siis ei moodumis, oota ma sissetulevat kõnet ja, ja kogu lugu siis noh, seal all oli siis põhimõtteliselt faili kataloogid ja meilisuhtlus kasutajate halduse.
                 
Kõneleja 2:
BBS sissehelistamiseks ei olnud mingit eraldi softi vaja.
                 
Egon Elstein:
Detavaline terminini sahtliterminališaht oli vist tollal isegi, ma pakun. Enamusele opsüsteemis olemas.
Ma praegu muidugi oletan, aga kuna tollal oli ka suur osa ju Meinfreimidele, et siis nendega suhtlemine käis üle pelleti ja, ja mäletada
                 
Kõneleja 2:
Pärast Nortum komandöri sees mingisugust helistamisvõimalust võib-olla.
                 
Egon Elstein:
Seda mina ei taibanud kasutada.
Mis kellelgi käepärast oli? Just, aga mingi terminali klient oli see, mida peamiselt kasutab. Sellega sai siis juhtida nii seda kasutatava modemi režiimi terminali režiime, sealt alla tirida. Tarkvara see oli nagu see ütleme, meie tegevuste põhiskop. Lisaks siis mängimisele selle tarkvara uurimisele, mis me siis saime?
                 
Kõneleja 2:
Okei ühesõnaga kookon tuli põhimõtteliselt Peeveeessi nii-öelda admindide hulgast, kes need selle nende adminni tuumiku ümber sikhopile tuumiku ümber pidi olema ju veel mingisugune ports mingisuguseid lihtsalt kasutajaid, kes aeg-ajalt käisid vaatamas mingeid uudiseid jazzil oskate umbes hinnata, palju neid võis olla Eesti peale noh, suurusjärgus.
                 
Egon Elstein:
No ma arvan, et alguses võis olla mingi paarkümmend kasutajat pernõud ja noh, neid nõude oli ka nii mis 50 ringis aasta paiku kuskil 90 90 hiljem läks juba päris suureks massiliseks asjaks, ütleme need selle internetituleku eelsel ajal, aga noh, selleks ajaks Meie juba nagu sellest kogukonnast eemaldunud peamiselt seetõttu, et ärivõttis nagu
Kogu kogu aja üle. Valid sihukesed ajad.
                 
Kõneleja 2:
Vaata ka paljudes kokku, meil oli 20 30 inimest pernõud.
                 
Egon Elstein:
No räägime umbes viiest kuni 10-st nõudis, et siin paarsada inimest üle Eesti ja neid võiks ka nagu rohkem olla, et suurusjärk, eks ole. Tollal oli ka see asi, et igalühel ei olnud oma arvutit ja omamoodi vaid see oli selline rand, Robinet, et 10 sama samamoodi mingisuguseid.
                 
Kõneleja 2:
Kaak helistab sisse, eks ole, palju neid seal siis on?
                 
Egon Elstein:
Ja ei vaata neid klassideiskaake nagu väga palju ei olnud, sest et et ikkagi enamasti.
Jah, oli meievanused tegelased, aga olid ka sellised natuke vanemad tegelased, kes sellega tegelesid, et päris selline akadeemiline seltskond
Ma ei mäleta seal väga sellist teadlasi, purjed.
Ilmselt oli omad.
Omad vahendid ja omad võimalused, ma usun, et ka Youznettali olemas kusagil.
                 
Kõneleja 2:
Just kas Youznendil olid mingisugused selged grupikaagus eestlasi nagu palju küljes käis?
                 
Egon Elstein:
Ma ei usu, et ei ole, sest.
Noh, selle internetikasutusest nagu internetieelsed Eeellase kasutusest väga palju ei tea, et ma arvan, et see oli peamiselt selline akadeemilised võrgud ja kasutajad
                 
Kõneleja 2:
Okei. Nonii, panite oma HMS-i püsti, hakkasite tooma kuskilt Nõukogude Liidu avarustest juppe ja Eestis maha müüma?
                 
Egon Elstein:
Või pigem vastupidi, et me hakkasime tooma Saksamaalt arvutijuppe neid siin kokku panema ja siis Eestis maha müün. Okei, et noh, võib-olla tänu sellele, et,
Jällegi isa kõrvalt tekkis selline varakult sellise impordi-ekspordikogemus mis tollal oli üllatavalt selline haruldane, kompetentsed, kuidas väljamaalt midagi osta ja üle tollipiiride Eestisse tuua. Et see oli siiski nagu suhteliselt haruldane teadmineri. Esiteks, kuidas leida üldsuse kontakt, kellele helistada kuidas talt enne hinnapakkumiste enne internetti n e-poode umbes 20 aastat on e-pood. Et sul ju ei ole aimugi, kellel, kes müüb. Sa ei tea, kellele helistada, sinna pakkumist küsida. Aga kust sa teada said? Tänu sellele, et isa oli selline aktiivne tegelane ja sealtkaudu siis tema kõrvalt, noh, kuna tema nagu inglise keelt väga palju ei osanud, siis ma pidin jooksvalt ka tema asja ajama ja, ja siis tema eksimustest ja õnnestumistest, eks siis sai õpitud. Eks me kõik seisame hiiglaste õlgadel ja ja sealt nagu oli suhteliselt lihtne nagu edasi liikuda.
                 
Kõneleja 2:
Aga Peebeeessinduses alguses mingisugust nihukest ärilist värki olnud?
                 
Egon Elstein:
Puhas puhas, selline fändlus, puhas selline. Paljuski seal oli ta siis nagu kahel jalal, eks ole, see suhtlemine, ehk siis oli nii-öelda vestlustoad inimesed omavahel suhtlesid erinevatel teemadel, selline community värk ja teine oli siis softi.
                 
Kõneleja 2:
Need olid tol ajal hakkasid mingisuguseid esimesest jutukad ka tekkima.
                 
Egon Elstein:
Jah, ma arvan, et see jutuke saidki paljuski alguses nendest samadest Peebeeesside
                 
Kõneleja 2:
Kolisid lihtsalt kuskile interneedusest. Mis need esimesed olid? Ma isegi ei mäleta neid nimesid, sest ma ei ole kunagi seal käinud tema jutuks.
                 
Egon Elstein:
Ütles väga palju ei ole olnud, ma mäletan.
Kas okka jutukas ei olnud, üks kasseeris teine ja et ma need hakkasid tekkima siis, kui jällegi ma nagu enam väga selle
Sotsiaalse tegevusega nii palju enam ei tegelenud, et siis oli rohkem, läks kogu aeg energia oma firmale.
                 
Kõneleja 2:
No kuidas kuidas tol hetkel tol ajal oli, sest kui ma mäletan, siis kuskil juba 90.-te keskel oli veel täitsa okei see, et mingisuguse pangatöötaja kontributeeris kuskile mingisugusesse, Opensors mingisugusesse, tarkvarasse ja mingeid, eks need tehti ju täitsa kogukondlikud 90.-te alguses ja nii et see oli hästi palju selli, kogukondlik tegevus.
                 
Egon Elstein:
Aga noh, eks ta on, ma arvata seisis sedasama kogukonna õlgadel, mis filonekest nagu alguse sai, et eks lõpuks pidid ju kõik kusagil tööd tegema. Ja pigem lihtsalt needsamad suhted ja, ja asjad liikusid edasi.
Jah, et paljuski needsamad nimed, kes ka seda ajakirja punkte, eks ju, olid ka filanetest tuttavad. Sina ei olnud sellega seotud.
                 
Kõneleja 2:
Ja siis sul läks juba nagu äri 90.-te keskel.
                 
Egon Elstein:
90.-te algusest 90 keskelt nagu oleks enam-vähem selline üheteisttunnised tööpäevad ja seal vist väga palju programmeerimiseks enam aega ei meie tegelesime eelkõige riistora vahendamisega, et riistvaravõrgud selline suhteliselt primi tegevus, et.
HMS hilisem Zebra Infosüsteemid, nagu tarkvaraarenduse, nii ei jõudnudki.
                 
Kõneleja 2:
Aga ikkagi selle isegi nagu tavalise võrgu ehitamine, see tol ajal oli ju kaks koaksiaalkaablil.
                 
Egon Elstein:
Ja ok seal kaabli vedamine oli mul väga selge siin mitmete mitmete tänaste suurfirmade laudade alt raamatud ja ja tolmu pühitud ladudest ja seintelt.
                 
Kõneleja 2:
Pangakontoreid, kus oli koaksiaalvõrgu peal kaabel, kus oli terminaator oli lihtsalt nii-öelda kliendi poole oleksin järjekorras seistes selle terminaator sealt otsast maha keerata, terve võrk oleks seisma jäänud ja keegi oleks teadnud.
                 
Egon Elstein:
Käisime saime kunagi seesamile IBM tõuken Ringvist nimelist võrku vedada, mis oli siis?
Ja esimene selline tähtkujuline topoloogia, millega me kokku puutusime, millel olid sellised rusikasuurused stepslite siis see oli täielik müstika kuidas kunstil.
                 
Kõneleja 2:
Mõte pähe tuli seda müstikat käima ajada, et see hakkab ju natuke nagu üle selle piiri kasvama, milles lihtsalt katsetades nagu selgeks teed endale.
                 
Egon Elstein:
Ei noh, selles mõttes, et ega nagu äris ikka, et sa võtad mingeid riske, ütled, et loomulikult me saame selle asjaga hakkama. Ja mis see siis ära ei ole, etetena hakkad tegema, siis selgub, et noh kõik töötabki niimoodi, nagu nagu ette nähtud. Nagu ette nähtud arvata.
Mis oli ja ega sealt selles mõttes sega nende võrkude nende asjade üles paneks, ei olnud nagu progemisega võrreldes mingi eriline tegevused seal. Lihtsalt konfigureerida süsteemi ära ja jalutad koju, et.
                 
Kõneleja 2:
Tol ajal vist üks koolist niisugust teadmist ei olnud üldse võimalik.
                 
Egon Elstein:
Sai absoluutselt, sest et noh, mina lõpetasin tehnikumi siis, kui enamus mine, lõpetasin tehnikum, kes raadioside ja levieriala ja ütleme niimoodi, et see valdavalt põhines elektroonikal, kuigi meile õpetati ka pooljuhte ja nii edasi, aga ütleme loogiline, siis skeeme, loogilisi ahelaid, kõike seda me saime vist pool aastat. Ja sedagi niimoodi teoreetiliselt.
Et enamus nagu meie sellest elektroonika õppimisest polütehnikumi ajal oli väga selline analoog elektroonikalambid elektromehaanika telefonijaamad, mis põhinesid Samm valijatel.
Noh, mis oli selline paras küberpunk tänapäeva mõistes juba? Jah, aga see oli siis, kui ma nüüd kuulasime
                 
Kõneleja 2:
Siis sealt tuleb välja see ikkagi soov asjadest aru saada ja vaadata, mis on nagu masinas karbi sees. Absoluutselt. See on ju seesama asi, et seal ei ole väga palju vahet, kas ta on elektrooniline.
                 
Egon Elstein:
Noh võib-olla see, et, et ma läksin polütehnikumi, oli pigem nagu selline perekonna traditsiooni järgimine, et, et noh, ma ütleme soodumuselt pigem nagu humanitaarolnud, kuna sinnamaani ehk siis tehnikumi minek oli, mul olid neljad-viied, kõik humanitaarained, keeled, kirjandus ja kahed kolmed, kõik rea läinud.
Nii et eks siis ilmselt siis tasub õppida seda, mis nagu mida, sa ei tea? Jah, et ülejäänud tuleb liig, lihtsalt. Kõlab loogiliselt.
Aga noh, selles mõttes see on andnud nagu selle tugeva külje või plussi või aluse jätad, sa oskad nagu?
Asju, millest sa aru ei saa üldistada või teisendada sellisteks mustadeks kastideks, millel on mingid defineeritud sisendid, väljundid ja mille põhjal saad sa mingeid järeldusi teha selle musta kasti sisu kohta, et selline paradigma. Ma olin seal koolituses nagu pidevalt olemas, et et ma arvan, et see on siiski selle põhjavundamendi andnud.
                 
Kõneleja 2:
Nojah, et ega see just tõepoolest, et ega seal kasti sees on, seda jõua sulle keegi õpetada, sest homme kastides teistsugune asi, ka see lähenemine.
                 
Egon Elstein:
Vaata tollal nagu need muutused olid palju aeglasemad, et X-teed oli ikkagi nagu aastaid, eks ole.
Nojah.
Kaks kaheksa kuud oli ka ikka aastaid, enne kui kolm, kaheksa, kuus tuli, neli, kaheksa kuskil, et noh, et ka noh lihtsalt nüüd on nagu see muutuste tempo nagu palju, nagu radikal serval.
                 
Kõneleja 2:
Huvitaval kombel salati tundub nii, et, et just praegu on nagu palju kiirem kui vanasti.
                 
Egon Elstein:
No võib-olla jah, on see, et see vananedes nagu vanemaks saades elutempo ise muutub
                 
Kõneleja 2:
Mis mis hetkel see see kogukon, nagu kuidas ma ütlen mitte mitte oleks laiali vajunud, aga mis hetkel see nii-öelda koos toimetamine ja kogukond hakkas selgelt alla jääma sellele, et igalühel oli vaja nagu laen ja liising ja pere. Mida sa küsisid? Et mis hetkel see võis umbes olla.
                 
Egon Elstein:
Ja vot seal BBS-i kogukond tegelikult oli, nagu mitte ainult nagu need pulatan, pardid, eks ole, vaid toimusid ühisüritused Bebe vintereid perre, summereid.
See iseenesest, need said alguse just selles samas Süsoppide esimesest kokkutulekust, sealt hiljem hakati siis juba nüüd üritusi laiemalt tegema, kuhu oli kutsutud siis külastajat samamoodi. Ja BBS kasutajad saavad et ja seda tegelikult jätkus veel internetti, sellise tulekust veel edasi, et siin vist äsja veel arutati, et kellel veel Pebess töötab
Et ma üldse ei imesta, kui veel leitakse Eestist mõni töötav BBS kusagil mingi pordi peal kuumirtukast tiksumas. Eks mitte.
Ma arvan, et noh, minu jaoks ta loomulikult vajus laiali rohkem nagu selle kätte, et ise ei jaksanud sind enam panustada, aga see elu ja tegevus toimu seal aastaid-aastaid pärast sedagi.
Ma arvan, et hakkas nagu laiali vajuma selles, kui internetipõhised keskkonnad hakkasid nagu kasutajaid lihtsalt üle võtma ja see muutus nagu selliseks selle igapäevase niigi tööga ja muude tegemistega seotud keskkonnaga ühtseks keskkonnaks, et kus modemiga kuhugi helistamine tundus nagu pisut arusaamatu, lisasamm, eks.
                 
Kõneleja 2:
Kui BBS-is enamasti Paet faile aeti niisama juttu, siis seal internetis Sky tööd kaadris.
                 
Egon Elstein:
Absoluutselt, sest et internetis olid need cafe jutukad ja okka jutukad olemas ja, ja ega enne Facebook reitisid olid ju ka olemas Geositiivse, mis iganes need keskkonnad olid, et ega need ei ole nagu mingid uued asjad. Kõik asjad on maailmas olemasolul lihtsalt nad on vahetanud nagu kesta ja vormide platvormi ja värvi ja natuke funktsionaalsust, aga noh, inimesed on lihtsalt liikunud sedasama tegevus, aga.
Ma ei tea nüüd juba 30 40 aastat nagu ühes keskkonnas täis.
                 
Kõneleja 2:
Kes neid beibe, summeri, debintereid, korraldusi, kes käisid?
                 
Egon Elstein:
Ma mäletan ühte korraldajat, kes oli vist kast
Kasutajanimi tal oli vist Kristrap või aga Piret, partristeri oli nimi.
Okei, et tema oli üks kes aitas hiljem neid üritusi korraldada. Enne korralduskomitee.
                 
Kõneleja 2:
See puhas niisugune, seal ei olnud mingisugust sponsorit taga, lihtsalt mingisugune.
                 
Egon Elstein:
Hiljem, kui üritused suuremaks läksid, siis me Haaennessiga firma poolt loomulikult.
Sponsoreerisid alguses nad üritasid piisavalt väikesed, et noh, et 10 inimest omavahel kokku tuleksid natuke asja arutaksid ja selle juures mõned õletakse, ei vaja sponsorit, selle saab odavamalt hakkama. Et hiljem siis arvutifirmad olid ju põhimõtteliselt samad, inimesed töötasid arvutifirmadel seal paljuski ka seal olulised tegelased, mitte omanikud, panustasid sinna suhteliselt hõlpsalt.
                 
Kõneleja 2:
Siis see Fido Fido kogukondi sujuvalt läks Eesti IT-kindlasti.
                 
Egon Elstein:
Jah absoluutselt, sest, et ega palju tuttavad nimed on ju sealtsamast pärit. Tõnu saamall näiteks on ju samamoodi sealt keskkonnast pärit.
Et ma arvan, et lihtsam on nagu võtta olles said seda tollased, need nimekirjad Fidonetist mingit jututubade print, audit, mida vist mammers peab ja vaadata, kes need nimed, kes seal eksisteerivad ja ma arvan, et sa leiad nad itimeeste ja muude asjade hulgas nagu üles.
                 
Kõneleja 2:
Aga kes seal käisid, siis ma saan aru, eks ole, aga, aga lisaks lihtsalt Minnidele.
                 
Egon Elstein:
Ja ta mõistusele öelda, ega mingeid loomulikult, aga eks oli neid, kes ei tahtnud, ei viitsinud ise või ei saanud, eks ole ise seda Peebeeesse pidada. Aga kellel see oli lihtsalt üks selline koht, mida või asi, mida arvutiga teha külastada pedes.
                 
Kõneleja 2:
Kuna progeda oli ikkagi barjäärid, olid päris kõrge, et sa ikka mängid, eks ole, siis.
                 
Egon Elstein:
Ja et ega noh eks mõned tulidki läbi progemise huvi minule endale, loomulikult tuli seeläbi mängu huvi, eks ole. See oli võimalus info suhtlemise, tarkvara vahetamiseks ka tehniliste teadmiste vahetamiseks, niiet.
Tollal kusjuures ju see suhtlemine oli oluliselt raskem selles mõttes, et kui kõigil ei olnud isegi mitte kodus telefoni siis siis BBS oli jah, et siis tänapäeval tundub nagu kõik on nagu käeulatuses, et sa pead lihtsalt taskust telefoni võtma, Google'isse otsingu panema. Tollal sa pidid otsima telefoni mõtlema, kellele helistada, et küsida, kas ta teab kedagi, kes teab midagi noh, et, et selles mõttes filonet andis ka nagu sellistele kommimonititena ka selle võimaluse, sa said sinna avalikku boardi panna mingi küsimus, et kas keegi teab, kuidas lahendada ühte või teist probleemi ja tulid vastusega.
Kindlasti kindlasti pääsesid mõnel ilmselt ja mõnel inimestele, eks see sõltub nagu küsimused. Eks seal oli noh, see oli asjalike jutusele jumal, läbu kohad, kus niisama jaurati selle tehnilised vestlusringid.
                 
Kõneleja 2:
Siis jah, see paneb asjad muidugi teise konteksti, et, et loomulikult, et kui sul kudusid, mitte telefoni, ei ole sul võimalus rahvusvaheliselt lausa suhelda inimestega. See on nõu märkimisväärselt teistsuguse hinnaga kui siis, kui sul on mobiiltelefon, võib-olla kaks internetti ühendatud arvuti, eks ole, alati käepärast.
                 
Egon Elstein:
No just, aga mõtle nagu sellele, et Google'it õlut olema.
                 
Kõneleja 2:
Tol ajal ei olnud väga paljusid
                 
Egon Elstein:
Asjaolu et, et guugele oli sul üle laua Su kolleege, naaber ja sõber, sul endal pidi olema selline kontakti baasia varamu piisavalt suur ja lai. Et peaksid, kes peaksid kedagi, kes teab kedagi, kes oskab sulle öelda. Oskad talle öelda midagi.
                 
Kõneleja 2:
Bronto tõi välja sellise mõtte. Kas sa oled nõus? Bronto ütles, et kuna väga paljus nagu sa ise räägid, et väga paljuski see võimekus üldse vahetada informatsiooni või astmelisi kogukonda sõltus sellest et sul oli just nimelt see kontaktibaas siis seetõttu üleminek sellest nagu kogukonnast, milliseid äriliseks tegevuseks müüa mingeid asju, see oli kuidagi nagu loomil loomulik, sest sul niikuinii see kontakti baasil olemas.
                 
Egon Elstein:
Seda võikski nimetada Street repotion, eks ju, et, et seal oli see kredibility nagu mingil määral olemas, et sind juba teati ja tunti selles keskkonnas. Okei. Et noh, kui tänapäeval räägitakse Estonian Mafia, siis noh, tollal seal ei pidanud seltskond pidanud Fido maffia südama eksmeest sellest siin sellest räägimegi, et jah, ei loomulikult, eks.
Paljud ka meie kliendid tulid läbi selle sigareti või vähemalt teadsid meid sealt ka.
                 
Kõneleja 2:
Nojah, sest ega arvutit iga üksi ostudeks osteti siis, kui oli juba teada.
                 
Egon Elstein:
Ühe paljud neist töötasid, eks ole, mingis pangas või firmas hiljem ja, ja silmad laienesid, tahtsid saada arvuteid, kusagil oli neil ju palgatud mingi itimees, pidin selle probleemi lahendama ja, ja noh ega tal ka ju ei olnud Google'it, kust, või e-poodi, kust nagu minna küsima parimat pakkumist, tal oli endal ka need inimesed, keda ta usaldas. Kelle käest siis seda pakkumist minna küsima?
                 
Kõneleja 2:
Nojah jah, ja veel ei ole nii, et lähed poodi, ostad palun mulle nagu kuu, 16 arvutit.
                 
Egon Elstein:
Tõnis raputab pead. Nii me ütleme niimoodi, et ega sa arutelu alguses oli ka see, et ega kuna valuutat ju väga palju kellelgi ringelnud, siis laoseisud olid ju olematud, põhimõtteliselt võeti ettemaks, ettemaks maksti välismaale, selle eest oodati siis, kuni see arvuti kohale laekus, siis pandi see kokku ja siis tarniti kliendile, et kui hästi läks, sai alla kuu aja kätte. Et ka see endas tegelikult ju usaldust, et sa annad kellegile nagu noh, ta maksis 1000 raha ütleme selliseid arvutiklassi, arvutikategooria mõttes, hinnad ei ole eriti muutunud hea arvutile üle 1000, eks ole, ja tavaline nii-öelda 1000, mille.
Noh, see, et sa annad mingile Matzile selle raha ära, ta ütleb, et ära muretse, kuu aja pärast tuleb arvuti, saad kätte selle, et see ikka nagu üksjagu, nagu sellist usaldust ei ole, sest et ega sa valuutat ei olnud, et sul oli ju tegelikult enamasti tehingut algusest tehti rublades.
Rubla ei olnud konverteeritud millekski muuks.
                 
Kõneleja 2:
Kui rohelises Ruba võiks ka kuu aja jooksul päris nagu oluliselt muutuda
                 
Egon Elstein:
Ja noh, oli veel see aeg, eks ole, kui oli, paljudel asjadel oli kaks hinda, ülekande, rubla hind ja sularaha, rubla hind, kumb odavam oli? Odavam oli sularahas sellepärast et Sa ei saanud alati pangast sularaha kätte. Ahaa, see oli mingid kindlad hetked, millal panka toodi sularaha ja siis sa pidid teadma, kas õigeid inimesi õiget aega, et saada seda sularaha.
                 
Kõneleja 2:
See muidugi seletab kõiki neid legende, kuidas arvutifirmades ja pankades hoiti sihukesi sularahapakke kuskil kapis ja vetsus ja kus kus iganes veel.
                 
Egon Elstein:
Ja see oligi niimoodi, et kui taheti midagi ülekandega osta, siis noh, jaa, jaa. Noh, see oli nüüd meie enda valik, et kas meil siis müüsime midagi valuuta eest või rublades. Ja nagu ülekande eest otsustasime meiega, siis oli see hind kallim. Lihtsalt puhtalt seetõttu, et pärast selle raha kättesaamine pangast
                 
Kõneleja 2:
Ta oli nagu oluliselt keerulisem ja sellest ajast siis oledki jäänud niimoodi arvutite arvuteid arvutitega tegelevaid inimesi juhtima.
                 
Egon Elstein:
Nojah, mul on nagu see taust, selline etet, ütleks mind nagu vedas. Mida aeg edasi, seda enam on mind huvitanud nagu rohkem sisulised asjad, kuidas asjad töötavad. Ja vähem huvitanud inimeste juhtimine, et mõistan et ütleme niimoodi, et inimesed on keerulised. Arvutid on lihtsalt sõnu. Kuidas sa selle infoturbe ja selle maailma juurde jõudsid? Esimest korda me jõudsime läbi väga praktiliste sammude. Me nimelt häkkisin täitevkomitee arvuteid.
Selleks, et sinna ligi saada ja mängima saada. Me avastasime, et, et on olemas. Tollal siis õnnepaleeks kutsutud majas
Täitevkomiteel üks modemite peal töötav teenus. Ohoo, kus sai abielus registreerida sünde ja surmasid ja, ja muuhulgas sedasama modemi otsas sai.
Eraldada kortereid vaata kus lõhu nimelt kortereid ei sealt alla lasta, vaid neid eraldati sulle riigi poolt. Aastat koha, kus sai sisse helistada ja eraldada kortereid. Jah, me avastasime, et sinna sisse helistada, aga see oli parooliga kaitstud ja proual oli jällegi eesnimi. Nii.
Et me alustasime nii-öelda sellest tagumisest otsast, sellele infoturbe
Ega ja mina ise sattusin nagu privadori, siis juba ütleme 10 15 aastat hiljem, eks aastal 90. Kui Tarvi Martens oli.
Tollasest küberneetikast teinud investorite kaasabil sellise, tal on nimetati neid spin-offid, eks mida tänapäeval kutsutakse startup ideks. Nädalal tehti pinnas primaldab, mille eesmärgiks oli siis digitaalselt signeeritud dokumentide pikaajalise tõestusväärtuse loomise süsteem nimega truulsaid.
Kuigi hakata tegema kaks aastat enne ID-kaardi projekti algust, enne kui ühtegi signeeritud dokumenti keegi näinudki
                 
Kõneleja 2:
Et selle koha peal saavad huvitaval kombel kokku võtta krüptot, sihukesi asju enam nagu paljakäsi ei tee, seal on vaja mingisugust nagu teadust. Et siis ühest küljest, mis tähendas, et paratamatult 90.-te lõpuks juba pidid hakkama otsapidi kokku saama need inimesed, nagu siis räägidki, üsna kestad olid iseõppijad, mingid teistsugused inimesed, kes kindlasti olnud iseõppijad. Nende, kuidas, kas seal mingisugust, kuidas see käis ja kes on mingisugust hõõrumist joonud?
                 
Egon Elstein:
Või ütleme ka enamus dollastest kolleegidest olid paljud vanad tuttavad Fida ajast, nii et, et noh, mina privadori liitusime turunduse ja müügifunktsioonis. Hiljem siis alles hakkasin tootejuhtimise ja, ja, ja nii-öelda juhatajana seal tööle. Aga algne funktsioon oli mul seal pigem nagu müümine, aga paljud need, kes tollal kolleegid olid, olid ka pidanetistud sama, sama nagu seltskond, sest see seltskond ei olnud väga suur. Sest noh, tegelikult see aru üldse siis kogu see seltskond, kellel üldse oli ligipääs üheksakümnendatel arvutitele, ei olnud väga suur. Et see oli veel see aeg, kui arvutid olid peamiselt firmades aga mitte kodudes, 90.-te lõpus alles hakkas tekkima see trend, kus firmad olid enam-vähem arvutitega varustatud. Hakati astma rohkem rakendustarkvara või noh, ütleme turu trendid liikusid rohkem rakendustarkvara poole ja eraisikud hakkasin endale koju ostma, aga no kool oli ju juba toimetanud, sellel lool oli juba nõukaajast nagu ma mõtlengi, et see oli igavene kümnendates saite, tuli igal aastal mingisugune paarkümmend Niukest inimest välja absoluutselt jah, ega, ega tegelikult ju sovhoosidest kolhoosides parematel olid omad ju kestused arvutid olemas juba nõukogude ajal. Et see arvuti kui selline
Eestis ei tekkinud päris 90.-test, vaid see ikka oli. Ammu-ammu enne minu sündi oleme mõelnud meie kõigi sündi.
                 
Kõneleja 2:
Aitäh, see on kasuline, väga huvitav jutt.
Jah, aitäh sulle, palun.
