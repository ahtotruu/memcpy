%!TEX TS-program = arara
% arara: myindex

\index[ppl]{Aaslaid, Andrus}
\textbf{\enquote{Kuidas sa arvutite juurde jõudsid}}

Tihti on nii, et meil on elu muutvad otsused aga me ei mäleta, kuidas me neid 
tegime. Aga neid seda juhust ma mäletan täpselt. Mul oli juba toona 
raadio-hobi. Olin selline põhikooli juntsu ja mulle meeldis hirmsasti mööda 
lühilainet ringi kammida. Meil oli kodus selline Melodija 101 stereo, Riia 
raadiotehase\sidenote{A. S. Popovi nimeline Riia Raadiotehas, alates 1951 Rigas 
Radio Rupnica} toodang. Sellega ma siis seiklesin suviti, kui midagi targemat 
teha ei olnud, mööda eetrit. Mul oli tegelikult kaks raadiot: esimene juba 
mainitud Melodija, sellele lisaks veel detektorvastuvõtja, mille mu poolvend 
oli mulle ehitanud. Selle viimasega ma istusin pööningul, kus oli muu seas mitu 
aastakäiku ajakirja \begin{russian}Техника - 
молодёжи\end{russian}\sidenote{Aastast 1933 ilmuv algselt Nõukogude ja nüüd 
Vene populaarteaduslik ajakiri.}. Kuna perekond tegeles põllumajandusega ja oli 
üks konkreetne põllumajandusnipp. Nimelt raamatukogudest toodi vanu ajakirju, 
need rebiti lehtedeks, need lehed keerati sisuliselt ümber sellise õõnsa 
põhjaga pudeli sellisteks väikesteks pottideks ja  sinna sisse istutati taimed. 
Paber lagunes mulla sees, ära taim pääses põllul vabaks. Ka neid ajakirju oli 
seal tohutu hunnik ja nende ajakirjade juures oli mitu aastakäiku 
\begin{russian}Техника - молодёжи\end{russian}'t. Lappasin siis neid ajakirju, 
detektoriklapid peas. 

Igatahes ükskord astusin ma tuppa, lülitasin Melodija sisse ja sealt öeldi, et 
Tallinna 43. Keskkool\index{Koolid!Tallinna 43. Keskkool} on otsustanud hakata 
eksperimentaalseks Tehnikaülikooli\index{Tallinna Tehnikaülikool}\sidenote{Tol 
ajal oli ta veel Tallinna Polütehniline Instituut, TPI} ettevalmistuskooliks ja 
nad võtavad kümnendasse klassi vastu õpilasi, kes tahaksid edasi õppima minna 
TPIsse. Kuulasin uudise ära, lülitasin raadio välja, läksin vanemate juurde ja 
teatasin, et ma lähen Tallinnasse kooli. Ma olin siis 14.


\textbf{\enquote{Aga kust sa siis pärit oled, et niimoodi Tallinna kooli pidi 
minema?}}


Pärit olen ma tegelikult kahesaja meetri kauguselt sealt, kust ma täna elan, 
ehk siis Tallinnast. Aga mul perekond otsustas evakueeruda 
Muhusse\index{Muhumaa}, kui ma olin kahe- või kolmeaastane, siis mind 
deporteeriti sinna. Nii et oma põrsapõlve olen kõik Muhus  veetnud ja siis ühel 
hetkel sealt siis tagasi putku tulnud tehnoloogia juurde. 

\textbf{\enquote{No mõni ime, et te Mastiga\index[ppl]{Kaal, 
Madis}\index[ppl]{Mast|see{Kaal, Madis}} hästi läbi saate!}}

Me oleme Mastiga ühe kooli poisid tegelikult selles mõttes, et Mast oli samas 
koolis keskkoolis, kui mina olin seal põhikoolis. Me oleme isegi sama 
raadiosõlme väisanud mõnda aega. Aga tollel ajal noh, nagu ikka, eriti veel 
maakohtades, et ega nooremad ja vanemad väga läbi ei käinud. Aga Mast oli hea 
poiss, ei peksnud nooremaid ega midagi. 

\textbf{\enquote{Mis seal lühilaine pealt kostis, mida sa kuulasid? Muusikat?}}

Ei muusikat kuulati raadio Luksemburgist. Sealt lühilaine pealt tulid erinevad 
hääled. Tuli morset, tuli mingeid huvitavaid kahinaid ja sahinad ja siis keegi 
luges numbreid ja. Ega tegelikult lühilaine on siiamaani päris hea tervise 
juures, et seal on  eeter siiamaani on  maast laeni sodi täis, et ega ta olemus 
ei ole, väga palju muutunud. Võib-olla mingeid propagandasaateid on vähemaks 
jäänud ja Hiina raadiojaamu on vaikselt kinni pandud seoses sellega, kuidas 
Internet peale tuleb. Aga, üldiselt on see lühilaine samasugune, nagu ta oli 
nelikümmend aastat tagasi, ma arvan.

\textbf{\enquote{Kas seal sinu ajakirjade hulgas juba arvutiajakirju ka oli?}}

Esimest arvutit ma nägin tegelikult just tänu sellelesamale poolvennale, kes 
mulle selle detektori ehitas. Ühel hetkel ta ütles, et Guido 
Tammissaar\index[ppl]{Tammissaar, Guido}, tema oli Eesti Energia 
Arvutuskeskuse\index{Eesti Energia Arvutuskeskus} üks tegelastest. Ja ühel 
hetkel tuli sinna maale ja ütles, et tule kaasa paariks päevaks, näed, mis asi 
see arvuti on. Et sind see tehnika, asi huvitab. Ja lubati mind siis maalt 
linna paariks päevaks, Estonia puiestee arvutuskeskuses olid veel põhiliselt 
ESM-id tollel ajal\sidenote{Kas \url{https://en.wikipedia.org/wiki/BESM}? }. Ja 
üks mingisugune CP/Mi\sidenote{CP/M oli 1974. 
aastal Inteli 8080/85 protsessorisarja tarvis turule toodud 
operatsioonisüsteem, mille 1980ndatel asendas mitmes mõttes sarnane MS-DOS} 
masin, mis tagantjärgi tundub jube kosmiline selles mõttes, et ta tundus 
mingisugune sotsmaa disain, eks ta, mingi Bulgaarlane oli. Olen mõelnud, et 
peaks üles otsima, et  masin see selline võis olla, aga ma siiamaani täis sada 
protsenti kindel ei ole. 

Selle peale ma niisama natuke klõbistasin. 
SM-4\index{Arvutid!SM-4}\sidenote{SM-4 oli PDP-11/40\index{PDP-11} ühilduv 
Nõukogude päritolu ja terves Idablokis toodetud arvutisüsteem} peal ma 
kirjutasin oma esimese BASICu\index{Keeled!BASIC} programmi sellel samal 
päeval. See oli derivaat mingist asjast, mida mulle näidati, et näed, umbes nii 
käib. Ja edasi ma olin \emph{hooked}. Sellest ühest päevast piisas, et sõltlane 
tekitada. 

\textbf{\enquote{See oli siis enne seda, kui sa otsustasid, et nüüd sa oled 
neliteist ja lähed Tallinnasse kooli?}}
Ma ei oskagi öelda, ma ei ole sada protsenti kindel, kumb oli enne, kumb oli 
pärast. Et et kas, kas see huvi, et tulla Tallinnasse, mängis rolli. Ega nad ju 
arvutikallakut tegelikult ei propageerinud.  Suurema rõhuga oli  elektroonika, 
tarkvara osa, seda nad väga ei reklaaminud. Minust pidi ikka elektroonik saama 
tegelikult, mis minust nüüd ka vahepeal sai, aga tollel hetkel ikkagi arvutid 
tundusid nagu see päris asi. 

\textbf{\enquote{Kas selles 43. keskkoolis valmistati päriselt ette ka 
ülikooliks? Oli sellest kasu?}}

See oli selline kahe teraga mõõk selles suhtes, et valmistati ette ja 
valmistati väga hästi. Sellepärast et see keskkooliprogramm oli pandud kokku, 
tolle aja inseneri õpetajate poolt, kes teadsid suhteliselt hästi, mida oleks 
vaja õpetada selleks, et põhi alla tuleks. Mis tähendas seda, et me saime 
läbisegi  tavalisi keskkooliaineid ja siis ühel hetkel tuli härra 
Tiidemann\index[ppl]{Tiidemann, Tiit} ja hakkas meile rääkima võllide 
epüüridest\sidenote{Epüür (prantsuse sõnast épure) on teatava suuruse asukohast 
olenevate väärtuste graafiline esitus}. Sisuliselt me tegime käsitsi võllidele 
rakendavate jõudude arvutusi, et kust kohast läheb võll katki, kui ta on 
sellise jämedusega siit, tollase jämedusega sealt. Siis vahelduseks loeti meile 
mingisugust teise kursuse elektrotehnikat. Sinna vahele me saime mingisugust 
inseneripsühholoogiat, mida Toomsalu\index[ppl]{Toomsalu, Arvo} luges, mis ei 
olnud vist üldse TPI õppekavas sees. Ta oli nagu kõikidele eksperiment. 
ETEK\index{ETEK}  oli tema lühikene nimi, Ants Reili\index[ppl]{Reili, Ants} ja 
Peeter Krosberg\index[ppl]{Krosberg, Peeter} teda tegid. Ta oli selles mõttes 
väga äge üritus, et ta oli ikkagi täiesti \emph{green-field}. Eriti kuna me 
olime esimene lend. 

Meil olid muidugi seal veel omaette sellised lahedad asjad nagu see näiteks, et 
enne meid oli keskkool tühjaks löödud ja me olime kolm aastat keskkooli kõige 
vanem klass. Mis tähendas seda, et me olime sisuliselt nagu jumalad koolis. 
Tänu sellele mitmed probleemid jäid olemata, mida muidugi tavalistes 
keskkoolides tol ajal veel eksisteeris. Keegi kedagi väga ei toginud ega 
nüginud ja samal ajal kuidagi see areng toimus, nii et, et mingisugune väärikus 
tekkis kõigile. 

Kahe teraga mõõk oli ta sellepärast, et ta andis nii kõva põhja, et väga paljud 
läksid otse tööle. Me saime ju kõik, kes keskkooli lõpetasid, see on 
automaatselt TPIsse sisse. Meil ei olnud vaja sisseastumiseksamit teha. Mis 
tähendas, et kogu see vist kaheksateist õpilast marssis otse TPIsse. Nendest 
kooli lõpetas nominaalajaga vist  kas üks inimene või kaks või kolm, ma päris 
ei mäletagi. Ikkagi käputäis. Hästi paljud läksid otse tööle. Kuna aeg oli ka 
selline, et et kuidagi see, mida TPIs tollel ajal arvutiteadusena õpetati 
ikkagi päris elule veel järgi ei jõudnud. See pidi olema aasta 91-92. Siis, kui 
see kambriumi plahvatus siin Eestis toimus.

Ühest küljest  mina istusin arvuti taga ööd ja päevad ja kirjutasin 
mingisugusele suurele autopargile ihuüksi  mingisugust tarkvara, mis pidi 
üleval hoidma tervet autoparki. Ja samal ajal siis üritasin kuidagi nügida 
ennast läbi SuperCalci\index{SuperCalc}\sidenote{Varajane tabelarvutussüsteem, 
algselt loodud CP\textbackslash M operatsioonisüsteemile} arvestusest TPIs kus 
aegajalt tuli nagu õppejõule näidata, et ära nii tee, nii päris ei käi see asi. 
Mitte, et nad oleks rumalad olnud, aga nemad õpetasid seda, mida nad olid kogu 
aeg õpetanud. Aga nüüd tekkis ühel hetkel selline seis, kus reaalne elu liikus 
palju kiiremini kui õppekava.

\textbf{\enquote{Aga kuidas sa selle programmeerimise juurde ikka jõudsid? Sa 
pidid ikka kuskil harjutada saama seda?}}

See oligi  tänu sellele 43. keskkoolile, noh, tänane siis 
Tehnikagümnaasium\index{Koolid!Tehnikagümnaasium|see{Tallinna 43. Keskkool}}, 
et, see eksperiment kestis ja mõnes mõttes kestab tänaseni. Seal oli 
põhimõtteliselt  esimest korda selline päris arvuti-inimese elu. Kuna 
IT-spetsialiste  liiga palju ei olnud, siis juhtus selline hämar lugu, et meile 
Eero Tohvriga\index[ppl]{Tohver, Eero} ulatati arvutiklassi võtmed ja hakati 
koolist palka maksma kümnendas klassis. See natukene vist oli tegelikult seotud 
sellise koolipoolse kerge kaastundega. Sellist otseselt tööstuskooli peale 
kaheksandat klassi tulemise traditsiooni juba mõni kümmend aastat ei olnud 
vahepeal olnud ja kõigile tundus see, et laps tuleb üksi Tallinnasse kangesti 
hirmus. Ja kuidagi ma arvan, et see oli pigem selline koolipoolne stipendium. 
Aga jah, põhimõtteliselt meile maksti kahe peale täis õppejõu palk välja, mis 
oli põhimõtteliselt ma kahtlustan, et mitte palju väiksem kui need õpetajad 
ise seal tollel hetkel said. Nii hästi ei ole ma kunagi oma elus ei varem ega 
hiljem elanud nagu keskkooli ajal. 

\textbf{\enquote{Mis te siis tegite selle raha eest?}}

Käisime restoranis söömas ja mida ikka lapsed rahaga teevad. Aga kool sai 
selle, et nad rohkem ei pidanud selle arvutiklassiga tegelema.  Kõik, mis seal 
oli, neid 
Iskraid\index{Arvutid!Iskra}\sidenote{\begin{russian}Искра\end{russian} oli 
mitmel pool Nõukogude Liidus eri modifikatsioonides toodetud arvutisari, mis
 omakorda jagunes erinevaid Lääne süsteeme kopeerivateks mudeliperekondadeks.} oli 
siis vist kolm või neli (alguses oli kolm, siis tuli üks uuemat tõugu juurde), 
seda me siis seal niimoodi püsti hoidsime, et tunnid seal toimusid. Meie asi 
oli hoolitseda, et masinad töötaksid ja seal midagi saaks õpetada. Mingil 
hetkel, kui me juba ise natuke vanemad olime, hakkas sinna juurde tekkima 
mingisugune kamp nooremaid huvilisi, kes seal ka siis hängisid. Ta oli selline 
täitsa tüüpiline arvuti ökosüsteem. Kunagi suvel käisime, remontisime sellesama 
arvutiklassi ära: värvisime ja panime uued põrandakatted. Ühesõnaga käitusime, 
ma loodan,  heaperemehelikult temaga. 

\textbf{\enquote{Ei ole kuulnud, et kellelgi oleks heaperemeheliku käitumisega 
probleeme olnud sarnastes situatsioonides}}

Tead, aga ajad olid sellised, inimeste usaldus oli suur. See arvuti oli nagu 
selline ühtepidi eriti müstiline, teistmoodi teda nagu kardeti vanema 
generatsiooni poolt. Kujutad ette, et oli (ma siiamaani ei tea nende inimeste 
nimesid naljakal kombel, ega ma vist pole ka kunagi teadnud), aga kunagi oli 
Eestis selline turismibüroo nagu Sarved ja Sõrad\index{Sarved ja Sõrad}. Minu 
meelest just sedapidi, mitte Sõrad ja Sarved. Asus Rävala puiesteel seal, kus 
täna on NO-teater. Ja mina läksin selle aknalt seal Sakala 
tänavat\sidenote{Külgneb Rävala puiesteega} mööda ja nägin, et inimestel on 
arvuti, see oli aastal 1991 või midagi. Igal juhul ma veel ei töötanud 
Skriiningus\index{Skriining}. Aga arvutit tahtsin hirmsasti kasutada. Keskkool 
oli läbi, sinna enam sisse ei lastud ka no sõltlane käis mööda linna, eks ole. 
Järsku näed, akna taga arvuti. Ja marsid sama hooga sinna sisse täiesti 
tundmatusse firmasse, täiesti tundmatu värske keskkoolilõpetaja, et 
\enquote{teil on siin arvuti, ma tahaksin seda kasutada}. Ja ilma mingisuguse 
tänapäeval heaks kiidetud taustauuringu või millegita ütleb firma omanik sulle 
oma kirjutuslaua tagant \enquote{Jah, ta on meil siin küll, me tahaksime teda ka 
kasutada, loomulikult}. Ja ilma mingi töövestluse ja ilma mingisuguse sellise 
pikema jutuajamiseta antakse sulle kontorivõtmed, öeldakse, et \enquote{tee 
nii, et meie saaks seda arvutit kasutada, tee ta korda}. Ja sa avastad ennast  
arvuti tagant. Ilma et keegi oleks isegi su dokumenti vaadanud, et kas sa oled 
varas, või sa ei ole varastega või kas sa tahad terve firma ära varastada või 
ainult arvuti. See usaldus, mis tollel ajal valitses inimeste vastu, kes 
oskasid arvuti sisse lülitada ja sellega midagi teha, see oli \emph{enormous}. 
See oli selline, mida tänapäeval ei ole võimalik ette kujutada. See oli 
selline, et need värsked keskkoolilõpetajad, kes seal tulid, nendel 
põhimõtteliselt oli võimalik küsida ükskõik millise firma ükskõik millise  
arvuti võtmed, sest see kõik töötas. 

Noh, see lõppes muidugi sellega, et lõpuks tuli Imre Perli\index[ppl]{Perli, 
Imre}\sidenote{Imre Perli oli pehmelt öeldes raju elulooga Eesti 
arvutispetsialist, kes sai kuulsaks \enquote{Perli andmebaasi} koostajana. 
Kasutades ära ligipääsu mitmetele andmebaasidele, lõi ta üheksakümnendate keskel 
\enquote{superandmebaasi}, mis sisaldas isikustatud andmeid autode, (toona üsna 
haruldaste) mobiiltelefonide, aadresside jms. kohta. Andmebaas levis althõlma 
laialt. Imre Perli hukkus segastel asjaoludel 15. aprillil 2000 
politseioperatsiooni käigus} ja kopeeris ära kellelegi andmebaasid, eks iga 
aeg saab lõpuks otsa. 

\textbf{\enquote{Kuidas sul ikkagi see programmeerima õppimise protsess käis?}}

See on sihuke  kuidagi viimasel sajandil tekkinud paradigma, et 
programmeerimine on midagi, mida peab õppima ja see on midagi, millega tuleb 
nagu spetsiaalselt vaeva näha. Programmeerimine juhtub. Vajadusest. 
Programmeerimine juhtub tahtmisest. Keegi ei ole mulle mitte kunagi õpetanud 
ridagi Cd, keegi ei ole mulle kunagi õpetanud ridagi Assemblerit. 

\textbf{\enquote{Kuskilt saju ometi said teada, kuidas \texttt{malloc} käib?}}

Aga see on see tahtmine teha. No mina hakkasin õppima 
Pascalit\index{Keeled!Pascal}, sellepärast et see oli ainukene raamat, mis 
mulle kätte sattus. Seesama õudne Jürgensoni Pascali pruunide kaantega 
raamat\index{\enquote{Programmeerimine Pascal-keeles}}\sidenote{R. Jürgenson 
\enquote{Programmeerimine Pascal-keeles}, 1985. Legendaarne raamat, mis 
miskipärast huviliste hulgas laialt levis}, mis on ikka tagantjärgi vaadates 
päris õudne algus programmeerimisele. Aga sellega sai alustatud. Ja siis, kui 
Turbo Pascal hakkas ära tüütama, sellepärast et see tegelikult oli ka niisugune 
keel, milles midagi normaalset teha oli väga keeruline. Siis ühel hetkel ma 
leidsin, et ikka Assembler\index{Keeled!Assembler} on see päris asi. Kuna tol 
ajal oli popp kirjutada igasuguseid demosid ja häkkida kõiki tarkvarasid, mis 
kätte sattus, siis siis \ldots No küsi, kuidas õpetad inimesele nagu x86 
Assemblerit? No võtad raamatu ühte kätte ja AT86e teise kätte ja hakkad tegema.

\textbf{\enquote{Aga kust sa said selle raamatu? Neid ju ei liikunud?}}

Liiklus küll selle koha pealt tuleb anda tõenäoliselt varem või hiljem 
presidendi medal Tarmo Mamersile\index[ppl]{Mamers, 
Tarmo}\index[ppl]{MomraT|see{Mamers, Tarmo}}, kes tollel ajal 
TTÜ-s\index{Tallinna Tehnikaülikool}? Ma ei teagi, kes ta seal oli. No ta oli 
seal üks paljudest nendest, kes seda arvutiasja püsti hoidis esimesena. Aga 
Tarmo kaudu põhimõtteliselt kõik see asi liikus. Minu minu varane mentor oli 
raudselt Tarmo ja no oli seda tegelikult veel pikka aega ka peale seda, kui ma 
juba tegelikult tööl käisin. Kõik, see materjal käis käest kätte. Hiljem tuli 
juba FidoNet\index{FidoNet}. Kui ma oma esimese esimese FidoNeti \emph{point}i 
püsti panin, siis oli juba kõik palju lihtsam, sest siis oli nagu aken maailma 
olemas. \emph{Point}i püstipanemine käis ka loomulikult läbi TPI. Tarmo 
istus natuke eraldi, et Tarmo oli teises teises ruumis samal korrusel. Aga siis 
Aare Tali\index[ppl]{Tali, Aare}\label{sisu!aare_tali} ja Tõnu Raimla\index[ppl]{Raimla, Tõnu} olid 
seal, kus käis elu nii-öelda.  Tarmo juures oli selline natuke rahulikum 
õhkkond. Seal käis igatahes kogu aeg \emph{action}. Ja siis mul oli nagu ühel 
hetkel kinnisidee, et ma tahan endale teha nüüd FidoNeti \emph{point}i, et 
ikkagi lõpuks olla maailma osa. Siis ma töötasin juba 
Skriiningus\index{Skriining}. Läksin siis Aare juurde, et \enquote{noh, Aare, 
sa oled siin \emph{sysop} ja värk} ja Aare talle omase abivalmidusega ütles, 
jah, masin on seal. Mille peale leidsin ma ennast BBS-i masina tagant ja pidin 
endale sinna valmistama FidoNeti \emph{node}. Ma kardan, et Tõnu või keegi 
lõpuks halastas mu peale ja näitas, kuidas seda päriselt teha. 

Aga siis edasi oli materjal juba palju kättesaadavam, siis said juba kõiki 
igasuguseid dokumente risti-rästi laadida alla. 

\textbf{\enquote{Mis sa sinna TPIsse õppima läksid?}}

Ma läksin LIsse. Ma arvan, et selle tolle aja nimi oli informaatika, äkki?. 
Kuna ma suhteliselt ruttu sain aru, et ma ei ole võimeline hommikul loengutes 
käima, siis mina ja üks väikene punt teisi, kes olid otsustanud, et nemad 
peavad õhtuõppes käima, läksime dekanaati ja nõudsime, et tarvis on õhtust 
vahetust sellele üritusele. Sest  õhtust vahetust tol hetkel konkreetsel alal 
ei olnud. Läksime kateedrisse, kateeder ütles, et jaaa, väga tore mõte. Aga 
meie kogemus ütleb, et kui te juba sihukese jutuga tulete, siis mitte keegi 
teist ei kavatse seal õhtuses ka käia. Mis tähendab, et me ei hakka teie jaoks  
eraldi rühma püsti panema. Käite ehitajatega esimese aasta koos koolis. Ja kui 
teisel aastal veel siin olete, siis vaatame seda asja. No kas nüüd osalt selle 
pärast või sellepärast, et dekanaadil oli õigus, nii või teisiti kukkusime 
sealt kõik robinal kolmanda kuu lõpuks välja ja läksime tööle igale poole. Nii 
et TPI on mul siiamaani lõpetamata. 

\textbf{\enquote{Sa mainisid, et sa kirjutasid mingit autobaasi softi. Kuidas 
sa seda tegema sattusid?}}

Siis ma juba töötasin. Me kõik läksime ju ka suhteliselt kiiresti ikkagi päris 
tööle. Tol ajal mingid startupi kultuuri ja ettevõtluse ehitamist veel ei 
eksisteerinud. Me ka lõpetasime sellisel ajal, kui need esimesi 
arvutikooperatiive oli siin väike käputäis. Minu esimene ametlik töökoht pidi 
olema tegelikult Noorsooteatri\index{Noorsooteater} valgustaja. Kuna mulle juba 
tollel ajal meeldis audioga tegeleda, siis ma tahtsin sinna helimeheks minna, 
aga helimees oli juba värskelt tööle võetud ja valgustaja koht oli vaba. Aga 
siis minu meelest päev või kaks enne seda, kui ma pidin lepingu alla kirjutama, 
tuli Tarmo Mamers\index[ppl]{Mamers, Tarmo} küsis, et kas ma ikka päris tööd ei 
taha teha, et Skriining\index{Skriining} otsib programmeerijat. 

Siis ma sattusin Skriiningusse Kalle Lotamõisa\index[ppl]{Lotamõis, Kalle} 
tööle. Minu esimene siis ülesanne oligi see, et autopark on sellel aadressil. 
Neil oli mingi eriti eksootilise asja peal jooksev andmebaasisüsteem, see ei 
olnud isegi \emph{mainframe}, see oli mingi mini. Ja  see oli vaja siis 
moodsale vahendile ümber kirjutada. Moodne vahend tähendas tol ajal siis Novell 
Netware'i\index{Novell} ja värskelt oli Paul Leis\index[ppl]{Leis, Paul} toonud 
Eestisse asja nimega Dataflex\index{Keeled!Dataflex}. Mis oli selline päris 
päris korralik objektorienteeritud kõrgkeel tol ajal. Ma hakkasin selle 
Dataflexiga siis ühest otsast õppima, kuidas Dataflexis programmeeritakse ja 
teisest otsast õppima, kuidas autopark töötab. 

\textbf{\enquote{Ahhaa, läksid kohe äriprotsessi ka sisse!}}

Äriprotsessid olid seal paljuski ees olemas selles mõttes, et töötav tarkvara 
oli olemas. Pigem oli seal äriprotsesside seisukohast hea lastetuba, et ära 
kunagi eelda midagi. Näiteks mina oma IT-inimese mõistusega tegin seal oma 
arust mõned asjad paremaks ja siis selgus, et päris nii ei saanud hea, nagu 
mina olin mõelnud. Sest raamatupidaja vaatasid mind nagu idiooti ja küsisid, et 
\enquote{sa ikka saad aru, palju ma neid numbreid pean siia päevast sisestama 
ja mitu korda ma seda enterit, mille sa siia vahele toppisid, peal vajutame 
lihtsalt niisama. Need arvud on neljakohalised. Ma sisestan neli numbrit ära, 
ta läheb ise järgmisele väljale, mitte ma ei pea vajutama. Ja eriti ma ei pea 
vajutama tabi, mis on teises klaviatuuri otsas. Saad aru, ma ühe käega kasutan 
pabereid teise käega vajutan klaviatuuri. Kuidas ma sinna tabi juurde sinu 
meelest saan, kui mul on teises käes paber?} 

Nad olid väga innovatiivsed seal tegelikult selles mõttes, et nad olid 
kasutanud sedasama andmetöötlust selleks ajaks juba aastat kuus-seitse. See oli 
 meditsiinitehnika autobaas, Termak\index{Termak}, siiamaani elu ja tervise 
juures. 


\textbf{\enquote{Nad siis juba Nõukogude ajal alustasid arvuti-asjandusega?}}

Nad olid juba sügaval nõukogude ajal end täiesti ära automatiseerinud. Selleks 
ajaks, kui mina aastal 92 sinna jõudsin, oli nendel juba esimene IT-süsteemi 
jõudnud kätte moraalselt nii ära vananeda, et see tuli PCde peale ümber 
kirjutanud. Neil oli aastal 1992 juba \emph{legacy}. Nad olid nii palju ajast 
ees.


\textbf{\enquote{Kuidas Skriining jõudis selleni, et neil on programmeerijat 
vaja? Lihtsalt kasti võis ka ju edukalt müüa?}}

Kalle\index[ppl]{Lotamõis, Kalle} hammustaski läbi selle, et kuna nad olid kogu 
aeg seal meditsiinitehnika ümber sebinud ja meditsiinisüsteemi neid arvuteid 
proovinud müüa, siis nad avastasid ühel hetkel, et seal on arendusvõimalused 
ka. Siis tegelikult Skriiningust\index{Skriining} saigi sealsamas 
üheksakümnendate alguses  arendusfirma. See arvutimüük käis ka, aga mina tema 
tollal noore inimesena väga ei süüvinud sellesse, kust raha tuleb. Aga mulle 
tundub, et see suht palju sellest tuli arendusest puhtalt.


\textbf{\enquote{Kas sa Tehnikaülikoolis ka veel ringi hängisid?}}

Ma hängisin seal pikalt aga ma kunagi õppinud seal. Ta oli ikkagi niisugune elu 
elu epitsenter, kuna seal töötasid kõik olulised inimesed. 
Mast\index[ppl]{Kaal, Madis} ülemisel korrusel Tõnu\index[ppl]{Raimla, Tõnu} ja 
Aare\index[ppl]{Tali, Aare} ja Tarmo\index[ppl]{Mamers, Tarmo} alumisel 
korrusel. Hiljem oli seal siis epitsenter siis, kui sinna läksid veel tööle 
Martin Rinne\index[ppl]{Rinne, Martin}, ja Merle Alliksoo\index[ppl]{Alliksoo, 
Merle} ja kõik teised, kes hiljem Microlinkis\index{Microlink} lõpetasid. Ta 
oli selline  sotsiaalse elu keskus. 

\textbf{\enquote{Mulle see variant, et sa ei õpi aga hängid, tundub palju 
mõnusam kui see, et sa õpid aga ei hängi}}

Nojah, eks ma ise ikka soovitan inimestel reeglina, et  proovige oma kool kohe 
ära lõpetada, et pärast osutub see palju raskemaks. Nüüd mina ja mu ja sõbrad, 
kõik on sisuliselt neljakümnendates hakanud oma haridusega lõpuks tegelema. On 
tekkinud natuke rohkem vaba aega uuesti ja ka mingisugune moraalne vajadus, et 
kuidas sa oled kõige väiksemate pagunitega mees ruumis.

\textbf{\enquote{Tol ajal, kui tagasi mõelda, ülikool palju praktiliselt 
kasulikku ei andnud. Tänapäeval on teistmoodi}}

No nii nagu kõik ütlevad, et  ta oli selline ta ei olnud mitte tempel selle 
kohta, et sa tuled sealt välja targemana, vaid on tõestus selle kohta, et sa 
oled võimeline, järjepidevalt mitu aastat asjaga tegelema. Pigem ikka 
vastupidavuse ja hoolsuse proov kui koolitus.

\textbf{\enquote{Räägi palun BBSidest. Kuidas sa selle node ikkagi püsti said, 
selle jaoks oli vaja ju ennast kuskil registreerida?}}

BBS, kes ei tea, oli varane  arvutivõrk, mille mõte oli selles, et sa helistad 
kuhugi oma modemiga ja seal teises otsas on modem, kes sulle vastab. Nad saavad 
omavahel andmeühenduse püsti ja siis sa saad sellest teises arvutis, mille 
küljes modem oli, saad ringi sobrada. Kusjuures tõepoolest selles mõttes 
sobrad, et ega tollel ajal arvuti turvalisus oli selline noh kokkulepete 
küsimus. Ma arvan, et suvaline suvaline üks BBSi omanik oleks võinud teise 
BBSi omaniku BBSi lasta neljaks tükiks kaks korda tunnis ilma mingite 
probleemideta, aga seda lihtsalt ei tehtud. Sellepärast, et see oli nagu 
saarlase ukselukk. Et kui sa oled ta paika pannud väljapoole ukse ette, siis kõik 
teavad, et sind ei ole kodus ja nii on. Et ei ole vaja katsetada, et kas uks 
on lahti või kinni, kodus kedagi ei ole. Ja BBSidega oli  turva umbes sama. 
Nüüd BBSi teine ja tegelikult palju kasulikum omadusi oli see, et kui sul juba 
oli modem ja juba oli arvuti, siis sa said ennast FidoNeti\index{FidoNet} 
\emph{node}ks registreerida. Et BBS iseenesest ei eeldanud midagi sellest, lihtsalt 
modemi ja vastava tarkvara olemasolu. Kuskil mingeid hämaraid teid pidi levisid 
need telefoninumbrid, et kus see telefon on, kuhu helistada, seal kohapeal sa 
said ennast ära registreerida.

Nüüd FidoNet oli ikkagi juba esimene selline ülemaailmne arvutivõrk selles 
mõttes, et modemeid helistasid üksteisele automaatselt. Ja ta oli kaunikesti 
hästi toimiv,  tolle aja kohta elektronpostiteenus, mille üks  eriline omadus 
oli veel see, et ta kuidagi liikus väljaspool KGB huviala. Eks küll 
kahtlustati, et teda kuulatakse pealt ja aeg-ajalt mingid imelikud modellid 
üritasid sinu modemiga poole jutu pealt rääkida ka, aga üldiselt teda väga ei 
monitooritud vist. Mingisuguseid probleeme ma ei tea, et kellelgi oleks 
kaheksakümnendatel olnud nende modem-modemiga sidepidamisega ei Eesti ega 
välismaaga. Mis on selles mõttes eriti huvitav, et et kui läksid 
kaugekõneliinid nii palju lahti, et ta oli juba võimalik kuidagi automaatvalida 
kuhugile, siis ju me helistasime igale poole välja. Ja see 
FidoNeti\index{FidoNet} \emph{mail} oli tegelikult  esimene võiks öelda vaba 
demokraatlik sidekanal tegelikult väljapoole juba kaheksakümnendatel. See oli 
kaheksakümmend üheksa, kaheksakümmend kaheksa, umbes niimoodi ta Eestisse tuli.

Mina olin siis keskkoolis, esimese \emph{node} panin püsti hiljem, jah, aga 
noh, see süsteem ise töötas varem. Ma olingi vist Aare \emph{point}. 
\emph{Point}i numbrit ma enam ei mäleta, mis mul oli. \emph{Node} number mul 
lõpuks endal oli kolmkümmend viis, aga, aga kas \emph{pointi} number, mis oli 
kaksteist-kaksteist. Mis \emph{node} all, ka ei mäleta. Mina panin selle jah 
püsti minu meelest vist üheksakümmend üks. Aga siis oli ka veel see aeg, kus 
ikkagi veitsa oli see asi nagu hämar selles mõttes, et me ju veel päris 
vabariik ei olnud. Me olime selline üleminekuvabariik. Ja selline 
registreeritud postiaadress andis sulle  võimaluse mingite siis foorumites juba 
kaasa rääkida. Eestis endas oli kümmekond gruppi, kus käis jutt erinevatel 
teemadel. Ja selles mõttes ta oli ikkagi päris selline elu, nagu me täna oleme 
harjunud, kuigi natuke teistsuguste tehniliste vahenditega. Ta oli aeglasem ja 
ta ei olnud reaalajas selles suhtes, et see post saabus sulle paar korda 
päevas. Ta ikkagi ei olnud selline, et kirjutan oma kirja ja see läheb kohe 
kõigile laiali. Aga ta täitis kõik need ülesanded, millega me täna tegeleme 
ära, nii et tegelikult kaheksakümnendate lõpus, üheksakümnendate alguses see 
ökosüsteem, mida me täna oleme harjunud nägema, oli tegelikult täiesti olemas. 
Ja väike käputäis inimesi Eestis omasid seda privileegi, et seda kasutada. 

\textbf{\enquote{Kas see väike käputäis olid pigem entusiastid, akadeemilise 
seltskonna inimesed või kes?}}

Need olid ikka sada protsenti entusiastid. Ma arvan, et akadeemilised inimesed 
sellel hetkel, kes läks ärisse, üritas sellest raha teha ja panid püsti 
esimesed arvutifirmad kes oli lihtsalt tegevuses  ellujäämisega, kes õpetas 
seda, mida ta kogu aeg õpetanud oli. See ökosüsteem minu meelest koosnes sada 
protsenti entusiastidest.

\textbf{\enquote{Kas eksisteeris mingi spetsialiseerumine ka, et siit ma saan 
tarkvara ja seal on huvitavaid jutte, seal raamatuid?}}

BBSidel väike spetsialiseerumine oli. Aga ma arvan, et mitte eriti suur. Eks 
kõik enam-vähem proovisid korjata, mida nad vähegi endale seal kõhu alla said. 
See oli see aeg, kus juba esimesed need sellised suuremad kõvakettad tekkisid. 
Mis tähendas seda, et tegelikult lühikest aega valitses olukord, kus tarkvara 
oli vähem kui ruumi. Ruumi mõiste oli ka muidugi tollel ajal huvitav. Kõige 
rohkem ruumi võtsid Sierra\index{Sierra Entertainment}\sidenote{1979. aastal 
asutatud Sierra Entertainment (varasemalt On-Line Systems ja Sierra On-Line) 
oli vastutav paljude toonaste hitt-mängude eest. Eriti populaarsed olid nende 
seiklusmängude sarjad \emph{King's Quest}, \emph{Space Quest} ja \emph{Leisure 
Suit Larry}} mängud, mis olid flopiketaste peal. Neist suuremad, Space 
Questid\index{Mängud!Space Quest} ja muud, tulid mingisuguse viie-kuue flopi 
kaupa. Mäletan, et me istusime Eeroga\index[ppl]{Tohver, Eero} ja arutasime, et 
kui oleks võimalik panna kokku oma unelmate masinat siis kui suur kõvaketas 
peaks tal olema. Jõudsime sinna, et kui oleks umbes kaheksakümmend megabaiti, 
siis ilmselt jätkuks eluajaks, sinna saaks kõik mängud peale panna, kõik 
tööasjad ka ja jääks veel umbes pool jääks üle.

\textbf{\enquote{Sierra oli omaette fenomen, tema asju mängiti ikka palju. Kas 
neid müüs ka keegi?}}

No küsime siis laiemalt, kas Eestis üldse keegi tarkvara müüs tollel ajal. 
Äritarkvara, nagu Novell, oli võimalik osta. Eks teoreetiliselt oli kusagilt 
Windowsi või DESQview'd\index{DESQview}\sidenote{DESQview oli kaheksakümnendate 
lõpus ja üheksakümnendate algul populaarne tekstipõhine mitmetegumiline 
keskkond. Ta käis DOSi peal ja võimaldas korraga mitut programmi eri akendes 
käimas hoida}  kindlasti võimalik osta. Aga peale Novelli serveri ja DataFlexi 
litsentside, ma ei mäleta, et me oleks üheksakümnendatel näinud mingisugust 
legaalset tarkvara kellelgi. 


\textbf{\enquote{Tuleme tagasi selle BBSinduse juurde. Kas selle sisu hulk, 
mida enda kõhu alla õnnestus kokku kuhjata, oli ka mingit pidi staatuse 
sümboliks?}}

Ma ei oska öelda. Ma oskan ainult enda BBSide kohta nagu rääkida. Mina 
sisuliselt korjasin kokku kõik, mida ma kätte sain ja pakendasin ringi. See 
oli selline kultuuriküsimus, et sa skaneerisid selle tarkvara viiruste vastu 
kõige värskema viiruste skanneriga, mis sul parasjagu käeulatuses oli. See käis 
automaatselt muidugi. Siis sa lisasid sinna mingi väikese faili sinna arhiivi, 
mis sisaldas mingit sinu \emph{header}it. See oli siis niisugune väike 
failijupp, kus oli graafiliselt või siis tollel ajal pseudograafilised, sinu 
logo sisse punnitatud. Ja siis sa panid ta välja ja panid oma faili listi siis 
mingisuguse sellise nupukese, mis asi ta on. 

See oli nagu \emph{basic housekeeping}. Et kui see sinu fail läks mingisse 
järgmisse BBSi siis järgmine BBS viskas sinul logo välja ja pani enda oma 
endale  asemele. Et nagu \emph{tag}iti ära nagu \emph{graffiti}ga, et see on 
minu käest tulnud asi. Ja minul vähemalt on küll tunne selline, et välja läks 
kõik, mida sa ise olid endale mingil põhjusel hankinud. Et see ei olnud nüüd 
nii, et sa läksid ja tõmbasid öösel mingisuguse HNSi\index{BBS!HNS} tühjaks ja 
panid enda lehekülje peale välja. Aga mingid asjad, mida sina olid kätte 
saanud, sa panid üles. Neid duplikaate ei olnud väga palju üllataval kombel.

\textbf{\enquote{Ma tahtsingi küsida, et nii oleks pidanud üks hetk ju kõigil 
kõik asjad olemas olema, seda siis ei tekkinud?}}

Seda ei tekkinud, sest kuna need BBSid olid väga stabiilset üleval, siis 
tõmbasid ära mingeid asju, mida sina pidasid enda jaoks vajalikuks ja panid nad 
siis ka omakorda enda juurde üles. Aga sellist mõttetut \emph{leach}imist  väga 
palju ei olnud. Püüet iga hinna eest oma faili andmebaas kõige suuremaks saada, 
ma ei mäleta, et seda oleks nagu eraldi eesmärgina keegi järginud. 

\textbf{\enquote{Too mõni näide, mis laadi asjad sulle toona huvi pakkusid?}}

No mina olin juba tollel ajal vihane \emph{nerd}, minu jaoks igasugu 
programmeerimismaterjalid ja igasugused käsiraamatud ja igasugused 
tööriistakesed ja  programmeerimisvahendid ja need olid minu spetsialiteet. 
Kahjuks mul ei ole seda vana faililisti alles, sest kui ma Skriiningust ära 
läksin, siis suhteliselt lühikese aja peale lendas see vana SCSI ketas õhku, 
mille peal see BBS jooksis ja sellest ei olnud \emph{backup}i ja sinna ta jäi. 
Läks kogu FidoNeti \emph{node} koos failibaasiga hingusele.

Ma ise teda järgmisse kohta kaasa ei võtnud, sest ma läksin Skriiningust panka 
ja seal olid kõvad mehed nagu Mast\index[ppl]{Mast} ja  Marx\index[ppl]{Marx|see{Kliimask, Margus}}\index[ppl]{Kliimask, Margus}  ees, kes olid oma ökosüsteemi püsti pannud, siis ühele BBSile seal rohkem ruumi ei olnud. 

\textbf{\enquote{Mis panka sa läksid?}}

Mina läksin sellesse panka, mille lõpupidu nüüd siin kohe nädala-paari pärast 
kätte jõuab\sidenote{Intervjuu Andrusega toimus 2019. aasta novembri algul} 
novembris, mis lõpetas siis Danske\index{Danske Pank}\index{Danske 
Pank|see{Forekspank}} nime all aga alustas Forekspangana\index{Forekspank|see{Eesti Forekspank}}. See 
oli jälle omaette selline innovatiivne pangandustoode. 

\textbf{\enquote{See oli väga äge pank omal ajal. Miks sa sinna läksid? 
Skriiningus said programmi kirjutada ja BBSi pidada ju?}}

Nagu ma paljudesse kohtadesse oleme läinud, läksin sellepärast, et kutsuti. Ja 
teiseks, kuna parasjagu Eestis jooksis teleseriaal \emph{Capital City}, mis 
näitas panganduse elu väga glamuurse \emph{highroller}ina, siis mulle tundus, 
et mina tahan ka nii elada. Tuleb tunnistada, et üheksakümnendate panganduses  
väga ei pidanud pettuma, elu oli täitsa täitsa lill. Ütleme nii, et nii nagu 
selles eesti teleseriaalis Pank päris elu ikkagi meie majas vähemalt ei käinud. 
Päris hulle pidusid sai peetud, aga seda, et keegi oleks kuskile kokaiinise  
ninaga ringi käinud, seda mina ei tea. Meie meie kandis  kokaiin oli täiesti ma 
ei tea, kas tundmatu või seda tehti salaja või midagi, aga igal juhul ma neid 
narkootikumidega pidusid ei tea. Aga pidutsetud sai hästi.

\textbf{\enquote{Aga kas ma õigesti mäletan, et tollal te panka tõmbasite ikka 
püsiühenduse\sidenote{Enamik varasest internetiühendusest Eestis käis kuhugi 
sisse helistades. See tähendas, et pidev side puudus ja muul ajal sõltus side 
kvaliteet suuresti analoogtehnoloogial põhinevatest telefonikeskjaamadest. 
Püsiühenduseks kutsuti seda, kui asutusest füüsiline kaabel Interneti külge 
jooksis ja selle olemasolu oli IT-inimeste unelmates kesksel kohal} sisse?}}

Püsiühenduse tõmbasime me sisse väga konkreetsel päeval. Ühesõnaga, meil 
modemitega nii-öelda pool-püsiühendus oli juba pikemat aega olemas.  
Forekspank asus Rävala puiesteel ja Rävala puiesteel asus seal suhteliselt 
lähedal ka juhtumisi KBFI\index{KBFI}\sidenote{Keemilise ja Bioloogilise 
Füüsika Instituut\index{Keemilise ja Bioloogilise Füüsika Instituut|see{KBFI}}
 (KBFI). 1979. aastal Endel Lippmaa\index[ppl]{Lippmaa, Endel} 
poolt loodud teadusasutus. Tuntud ka kui \enquote{Lippmaa Instituut}. Just 
Lippmaade perekonna aktiivse ning laiahaardelise tegutsemise tõttu mängis 
instituut paljudes toonastes olulistes protsessides (sealhulgas kohaliku 
Interneti arengus) olulist rolli.}. Baumaniga\index[ppl]{Bauman, Andres} 
oli läbi räägitud, et kuidas seda Internetti saab ja meil oli suhteliselt 
rivitu ligipääs. Aga mingil hetkel tundus, et see asi võiks ikka päris 
permanentne olla. Võtsime Mastiga\index[ppl]{Mast} kaablirulli ja 
hakkasime siis üle Rävala puiestee katuste KBFI poole liikuma. Mis oli selle 
juures tähelepanuväärne oli see, et see juhtus päeval, mil esimest korda paavst 
Eestit väisas\sidenote{Paavst Johannes Paulus II külastas Tallinna 10. septembril 1993.}. 
Kõik katused olid snaipreid täis, oli mingi tohutu 
\emph{lockdown} et keegi paavsti siin käigu pealt ära ei tapaks.  
Seletasime kõigile, et meil on vaja kaablit vedada,  paneme Interneti 
püsiühendust. Ja see oli selline maagiline valem, mis  võimaldas ligipääsu 
kõikidele kesklinna katustele, ilma et keegi oleks küsinud sult midagi 
rohkemat. Me otseselt snaiperitega samale katusele ei sattunud, aga üldiselt 
jah, midagi ei küsitud ka. Natukene oli seal vist vaja mingit häkkimist ka, et 
mingist koodlukust pidime vist ikkagi läbi minema kogemata. Aga see oli tollel 
ajal mehaaniline ja seda tehti nuppude kulumise järgi, et see ei olnud kõige 
suurem takistus. 

\textbf{Millest ma siis järeldan, et toona oli maailm teistsugune. Internet ei 
olnud veel kommertsiaalne vaid pigem kogukondlik nähtus?}

Selle eest vististi ikka keegi maksis ka kellelegi lõpuks midagi, aga palju, 
seda ma jällegi ei mäleta. Eks ta ju paljuski käis inimsuhete baasil ikkagi. Et 
kuna me Andres Baumanni\index[ppl]{Bauman, Andres} tundsime siis kuidas see 
sealt tegelikult  liikus ega mina ka ei tea. Mast\index[ppl]{Mast} seda asja 
ajas. Millegi pärast ma arvan, et  me maksime KBFIle mingit mingit raha ka 
selle eest. Jällegi. Aasta oli siis minu teada üheksakümmend viis ja 
üheksakümne viiendast aastast alates tegelikult oli meil 
Forekspangas\index{Forekspank} elu, nagu me seda täna tegelikult näeme. 
Suhteliselt samal ajal tuli Mosaic'i\index{Mosaic}\sidenote{NCSA Mosaic oli üks 
esimesi internetibrausereid ja mängis WWW populariseerimisel olulist rolli. 
Sama meeskond lõi hiljem Netscape\index{Netscape} Navigatori, mis omakorda on 
Firefoxi eelkäijaks} brauser, suhteliselt samal ajal hakkas veeb arenema, 
suhteliselt samal ajal tekkisid meile kõigile e-maili aadressid (mis olid 
natuke küll varem juba KBFI kaudu olnud korraks, aga siis tekkisid nad päris 
meie oma foreks.ee domeeni külge). Kõik see ökosüsteem, miinus Facebook, olid 
üheksakümne viiendal aastal tegelikult meil käes. Sealt edasi,  tegelikult me 
elasime täpselt sellist elu nagu inimesed infotehnoloogiliselt täna ette 
kujutavad. 

See elu oli natuke tillukesem selles suhtes, et me täitsa tõsimeeli arutasime, 
et see ühendus, mis meil KBFIsse on, on ikkagi nii aeglane, et äkki peaks kogu 
veebi kohalikku serverisse ära kopeerima. Siis me isegi vahepeal arutasime, et 
kuna see mahub ühele DVD-le tõenäoliselt ka ära, ehk siis äkki peaks tegemiseks 
äri, et hakkaks müüma Internetiga DVD-d, et siis oleks saanud kohalikest 
\emph{cache}dest tõmmata. Kogu see WWW oli tollel hetkel selline, tõsimeeli sai 
arutatud, et paneks ta ühele DVD-le ära. 

\textbf{Ka teistest lugudest käib läbi, et toonane maailm käis väga suuresti 
inimsuhete peal. Aga ometi inimesed ei hakka arvutitega tegelema, kuna neile 
meeldib inimestega tegelda. Siiski tunduvad Eesti arvuti-inimesed olema küllalt 
suhte-altid ja neis osavad. Kuidas see nii on?}

Ma arvan, et see on sellepärast, et kuna sul on sellised huvid, siis sa oled 
terve keskkooli ja pool ülikooliaega olnud sotsiopaat ja kellegagi väga ei ole 
sul rääkida millestki olnud. Ja nüüd sa ühel hetkel leiad omasugused, 
omasuguste huvidega. Täitsa puhas \emph{nerd}i ja nohiku käitumine, eks ole, et 
kui sa paned  nohikud  kõik ühte tuppa kinni, siis nüüd ühel hetkel leiavad 
üksteist ja siis on kõigil järsku lõbus sest kõik naeravad samade naljade üle 
lõpuks ometi. Ja peod täpselt samamoodi, eks. Ega kõige karmimat peod, kus ma 
olen  osalenud on ikkagi olnud inimestel, kelle igapäevatöö on kaunikesti 
\emph{boring}. Tahtmata anda hinnangut mingisugustele inimgruppidele, aga   
raamatupidajad ja  andmesisestajad, kui nad ikka käima lähevad, siis see on 
ikka ikkagi täiesti teine tase. Et siis lõbusad inimesed on keskmiselt lõbusad 
kogu aeg. Aga kui sellised nohkarid lõpuks lõbusaks lähevad, siis juhtub asju.

See ökosüsteem toimis tänu sellele, et inimestel oli hea meel üksteist leida. 
Ta oli tollal tõesti väike ka, ta oli ikkagi alla saja inimese kindlasti,  
võib-olla isegi alla viiekümne inimese alguses. Need olid siis sellise uue 
laine  arvutitegelased, kust hästi palju meie tänast  start-up ettevõtlust 
tegelikult ju välja on kasvanud. See kamp juba tollel ajal oli väga tihedalt 
koos ja väga õnnelik üksteise leidmise üle. Ja tänu sellele ju hakkasid siis 
toimuma legendaarsed BBSummeri\index{BBSummer} nimelised üritused. 

\textbf{Räägime lõpetuseks sellest ka, et mis sa praegu teed?}

Ma ei tea, see on võib-olla masendav tõdemusega, ega ta pole mind väga palju 
sellest nagu kaugemale ega kuhugi mujale viinud. Ikka väga laias laastus 
tegelen täna täpselt sama asjaga, millega ma tegelesin kakskümmend viis aastat 
tagasi. Olen pendeldanud elektroonika ja tarkvara vahel siin edasi-tagasi ja 
olnud mitme firma CTO ja  asutanud firmasid ja neid kihva keeranud ja töötanud 
teiste juures ja töötanud endale. Ja kui keegi küsib, et millega sa tegeled, 
siis ma tavaliselt ütlen, et ma annan masinatele hinge. 


\textbf{See on ilus ütlemine ja läheb kokku küsimusega, mis mul enne jäi 
küsimata. Tavaliselt inimesed tegelevad kas riist- või tarkvaraga aga sinul 
tundub olevat üks jalg ühes ja teine teises?}

Vaadates oma elu, siis ma muidugi tahaks, et tarkvara oleks mu tõmmanud endasse 
sellepärast, et see on nii palju  lihtsam ala, mõnes mõttes. Vigu on palju 
lihtsam parandada ja asju ära visata peaaegu üldse ei tule, mis katki lähevad. 
Kettaruum ei maksa täna eriti palju erinevalt elektroonika valmistamisest ja
utiliseerimisest.

Mul on kuidagi juhtunud niimoodi, et mul on see reaalne maailm, et kui ma panen 
tule vilkuma ja näen, kuidas mu tehtu  manifesteerub päris päris asjades, et 
siis mul kuidagi läheb tuju paremaks. Kuna mul see elektroonika disaini puhul 
tundub, et tuleb ka välja, samal ajal ma ikkagi taustalt oled nagu 
programmeerija, siis ma olen nagu sattunud sinna sidemeheks. Ma suudan tõlkida 
riistvara tarkvara jaoks ja vastupidi. Selle kõige konkreetne töönimetus on 
\emph{embedded engineering}. Mis on, tundub täna, vaadates, mis meil koolidest 
saabub, siis täiesti väljasurev kunst. Neid tegelasi, kes suudavad nii 
riistvara valmistada kui sellele tarkvara peale kirjutada, neid üritatakse 
nimetada mehhatroonikuteks või kelleks iganes, aga aga fakt on see, et nende 
juurdekasv on järsult pidurdunud ja see varem või hiljem hakkab meil 
probleemiks muutuma. Tõsi küll, töömeetodid ka muutuvad. Me kasutame täna
võibolla töövahendid, mis iseenesest annavad näiteks tarkvaratiimile parema 
ettekujutuse riistvarast kui see vanasti oli. Kirjeldused ja mingisugused 
\emph{markup language}d, millega  seda tehakse, on paremad. See, mis ma enne ka 
ütlesin oma tööd kirjeldades, et ma annan masinatele hinge, et see on 
tegelikult see osa, et kui sa lülitad oma pesumasina sisse, siis mida ta sinu 
heaks teha oskab või ei oska. Kui hästi see  raua ja tarkvara vaheline kooslus 
on välja mõeldud, sellest tuleb ka see kasutajakogemus. 

\textbf{Sa ütlesid enne, et sa oled ka CTOna toimetanud. See tähendab ju, et 
kolmas element juurde, sa pead selle kõik suutma ka äriks tõlkida?}

No vot seda CTO ametit on kaht sorti. Tavaliselt väikestes firmades tähendab 
CTO olek seda, et koosolekule on vaja kedagi kaasa võtta ja kuidas sa võtad 
kaasa ja ütled, et on mul programmeerija, eks. Sa pead talle lihtsalt andma 
visiitkaardi, millega ta näeb välja presentaabel. Tihti väikese firma CTO 
tähendabki lihtsalt seda, et sa teedki kõike, millel on tehnikane maitse 
küljes. Suurema firma CTO tähendab seda, et sa oledki see\ldots. Täna on 
startupi maailmas \emph{customer fit} ja \emph{market fit} hästi kõva teema. 
Kui vanasti väga ei tegeletud sellega, siis nüüd, kus on tohutu kuhi 
investorite raha põlema pandud, ilma et sellest isegi sooja oleks saadud, et 
siis kõik on hakanud rääkima sellest, et su toodetut kellelegi nagu päriselt
tarvis peaks ka olema. See tundub olevat mingi uuem asi, viimase paari aasta 
paradigma. Juba mingi kaks-kolm aastat tagasi hakkas Silicon Valley poole pealt 
pihta see kultuur, et laste kätte ei taheta raha enam hästi anda. Ehk siis 
nende kaheksateistaastaste ime-ettevõtjate, kes suudavad  väga suure kuhja raha 
korraga põlema panna, nii et sooja ei saa, aeg sai seal mõned aastad 
tagasi läbi. Nüüd on siis  selgunud uus innovatiivne lähenemine, et toodet peab 
kellelegi tarvis ka olema, teine selline paradigma muutus. Mis tähendab muidugi 
seda, et erakordselt raske on olnud hakata raha saama projektidele, sest kõik 
on hirmus järsku pirtsakas muutunud ja nõudnud, et kust raha tagasi tuleb. 

\textbf{See jutt läheb ju kokku sinu kunagise ettevõtte uksest sisse minekuga: 
seal sa pidid ju ka kohe hakkama kasulik olema ja ei tohtinud asju tuksi 
keerata}

Selle kasumlikkus, see on tegelikult õudselt valus teema. Riistvaraga on see 
asi  selgem selles mõttes, et riistvara ei eskaleeru kui keegi teda ei osta. Sa ei 
saa valmistada sedasama \emph{recorderit}, millega me siin praegu salvestame, 
miljon tükki, kui keegi seda ei osta, sellepärast et sa lähed lihtsalt 
pankrotti. Tarkvara tiražeerimine ei maksa midagi. Ja täpselt samamoodi võib ju 
juhtuda, et tarkvara, millest mitte kellelegi mitte pennigi raha ei teki, on 
tegelikult väga kasulik. Ehk siis kasulikkus ja ärimudel ei tähenda veel mitte 
midagi omavahel. Ja mis  dotkommi mullidega tavaliselt kipub juhtuma ja 
igasuguste tarkusemullidega, on see, et piir selle vahel, kus asi ei teeni 
raha, sellepärast et ta on väga hea mõte, mida veel ei ole õpitud raha panna 
teenima ja nende asjade vahel, mis ongi täiesti mõttetud, on väga raske tõmmata. 
Seetõttu on väga palju tegelasi, kes suudavad maha müüa  täiesti kasutu idee 
öeldes, et see ongi enne monetariseerimist faas ja see peagi midagi tootma. 
Unustades ära selle, et see on ka ühtlasi täielik kräpp, eks ole. Siin  
viimasel ajal on tekkinud paar niisugust suuremat skandaali, üks neist on 
muidugi see õnnetu Theranose \emph{case}, kus sa suudad nii veenvalt endale 
valetada. Et sul ongi ehitatud üles terve ökosüsteem väga kasulikkudest 
asjadest, mille ainus viga on see, et see fundamentaalne eeldus, millele ta 
rajatud oli, oli täiesti vale. 

\textbf{Tundub, et selle kahekümne viie aastaga maailm väga teistsuguseks 
saanud ei ole aga siiski natuke toimib teisti?}

Üks asi on oluliselt erinev. Tollel ajal tarkvara valmistati kahel põhjusel. 
Üks oli see, et teda oli tarvis, mis tähendas, et  oli tugev kliendipoolne 
tõmme. Ja teine oli see, et ma tahtsin, et midagi sellist eksisteeriks 
maailmas, mis tähendab, et ma lihtsalt võtsin kätte ja kirjutasin ta kas enda 
või teiste rõõmuks. Ja lasin ta lihtsalt maailma. Hästi palju mingeid väikesi 
utiliite, mis midagi kasulikku tegid, olid ju tegelikult kirjutatud kellelgi 
enda jaoks ära, pakendatud ja saadetud laiali. Eestis seda, et tarkvaraga 
õnnestuks mingit raha teha, et mina kirjutan mingisuguse vidinaga ja keegi 
maksab selle eest, seda kontseptsiooni polnud olemas. \emph{Corporate} maailmas 
küll seal igasuguseid  raamatupidamissüsteeme osteti-müüdi juba tol ajal väga 
edukalt ja see kõik töötas. Mujal maailmas tegeleti mingisuguste utiliitide 
pealt raha teenimisega ka väikesel viisil. Aga Eestis üldse mitte. Nüüd 
tänapäeval eks ole, see tarkvara tootmine on läinud niimoodi, et mul tuleb mingi 
ilgelt hea idee. Ja ma tahan sellest nüüd teha raha tootmise masina, mis 
tähendab, et sa teed nagu teistpidi. Et see ei ole mitte nii-öelda 
vajaduspõhine vaid selline unistus-põhine. Et mina tahaksin, nagu me siin 
aeg-ajalt  Ivar Zaransiga\index[ppl]{Zarans, Ivar} naerame, et tänapäeva 
maailmas  inimesed  otsivad probleeme neid vajavatele lahendustele. Et kui 
vanasti otsiti probleemidele lahendust, siis nüüd otsitakse vastupidi ja see on 
 kõige suurem paradigma muutus selle kahekümne viie aasta jooksul.
