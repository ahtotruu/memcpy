%!TEX TS-program = arara
% arara: myindex

\index[ppl]{Aaslaid, Andrus}
\textbf{\enquote{Kuidas sa arvutite juurde jõudsid}}

Tihti on nii, et meil on elu muutvad otsused aga me ei mäleta, kuidas me neid tegime. Aga neid seda juhust ma mäletan täpselt. Mul oli juba toona raadio-hobi. Olin selline põhikooli juntsu ja mulle meeldis hirmsasti mööda lühilainet ringi kammida. Meil oli kodus selline Melodija 101 stereo, Riia raadiotehase\sidenote{A. S. Popovi nimelime Riia Raadiotehas, alates 1951 Rigas Radio Rupnica} toodang. Sellega ma siis seiklesin suviti, kui midagi targemat teha ei olnud, mööda eetrit. Mul oli tegelikult kaks raadiot: esimene juba mainitud Melodija, sellele lisaks veel detektorvastuvõtja, mille mu poolvend oli mulle ehitanud. Selle viimasega ma istusin pööningul, kus oli muu seas mitu aastakäiku ajakirja \begin{russian}Техника - молодёжи\end{russian}\sidenote{Aastast 1993 ilmuv algselt Nõukogude ja nüüd Vene populaarteaduslik ajakiri. Perekond tegeles põllumajandusega ja oli toonud istikute tarbeks raamatukogust vanu ajakirju}. Lappasin siis neid ajakirju, detektoriklapid peas. 

Igatahes ükskord astusin ma tuppa, lülitasin Melodija sisse ja sealt öeldi, et Tallinna 43. Keskkool\index{Koolid!Tallinna 43. Keskkool} on otsustanud hakata eksperimentaalseks Tehnikaülikooli\index{Tallinna Tehnikaülikool}\sidenote{Tol ajal oli ta veel Tallinna Polütehniline Instituut, TPI} ettevalmistuskooliks ja nad võtavad kümnendasse klassi vastu õpilasi, kes tahaksid edasi õppima minna TPIsse. Kuulasin uudise ära, lülitasin raadio välja, läksin vanemate juurde ja teatasin, et ma lähen Tallinnasse kooli. Ma olin siis 14.
