%!TEX TS-program = arara
% arara: myindex

\index[ppl]{Aaslaid, Andrus}
\question{Kuidas sa arvutite juurde jõudsid?}

Tihti on nii, et me ei mäleta, kuidas me oma elu muutvad otsused  
tegime. Aga seda juhust ma mäletan täpselt. Mul oli juba toona 
raadiohobi. Olin põhikooli juntsu ja mulle meeldis hirmsasti mööda 
lühilainet ringi kammida. Meil oli kodus Melodija 101 stereo, Riia 
raadiotehase\sidenote{A. S. Popovi nimeline Riia Raadiotehas, alates 1951 Rigas 
Radio Rupnica.} toodang. Sellega ma siis seiklesin suviti, kui midagi targemat 
teha ei olnud, mööda eetrit. Tegelikult oli mul kaks raadiot: lisaks Melodijale 
detektorvastuvõtja, mille mu poolvend 
oli mulle ehitanud. Sellega ma istusin pööningul. Vanemad tegelesid 
põllumajandusega ja neil oli 
üks konkreetne põllumajandusnipp: raamatukogudest toodi vanu ajakirju, 
need rebiti lehtedeks, keerati ümber õõnsa 
põhjaga pudeli väikesteks pottideks, mille sisse istutati taimed. 
Paber lagunes mulla sees ära, taim pääses põllul vabaks. Neid ajakirju oli 
pööningul tohutu hunnik, muu hulgas mitu aastakäiku 
\begin{russian}Техника - молодёжи\end{russian}'t\sidenote{Aastast 1933 ilmuv 
algselt Nõukogude ja nüüd Vene populaarteaduslik ajakiri.}. Lappasin siis 
pööningul neid ajakirju, detektoriklapid peas. 

Igatahes ükskord astusin ma tuppa, lülitasin Melodija sisse ja sealt öeldi, et 
Tallinna 43. Keskkool\index{Tallinna 43. Keskkool}\sidenote{Praegune 
tehnikagümnaasium.\index{Tehnikagümnaasium|see{Tallinna 43. Keskkool}}} 
on otsustanud hakata 
eksperimentaalseks tehnikaülikooli\index{Tallinna 
Tehnikaülikool}\sidenote{Tallinna Polütehniline Instituut, praegune Tallinna 
Tehnikaülikool.} ettevalmistuskooliks ja 
nad võtavad kümnendasse klassi vastu õpilasi, kes tahaksid TPI-sse edasi õppima 
minna. Kuulasin uudise ära, lülitasin raadio välja, läksin vanemate juurde ja 
teatasin, et lähen Tallinnasse kooli. Ma olin siis 14.


\question{Kust sa pärit oled, et tahtsid Tallinna kooli 
minna?}

Pärit olen ma tegelikult kahesaja meetri kauguselt sealt, kus ma täna elan, 
ehk siis Tallinnast. Aga kuna mu perekond otsustas evakueeruda 
Muhusse, kui ma olin kahe- või kolmeaastane, siis mind 
deporteeriti sinna. Nii et oma põrsapõlve veetsin Muhus ja siis ühel 
hetkel panin sealt tagasi tehnoloogia juurde putku. 

\question{Mõni ime, et te Mastiga\index[ppl]{Kaal, 
Madis}\index[ppl]{Mast}\sidenote{Vt lk \pageref{cptr:mast}.} 
hästi läbi saate!}

Me oleme Mastiga ühe kooli poisid, Mast oli keskkoolis, kui mina olin 
põhikoolis. Me oleme mõnda aega isegi sama 
raadiosõlme väisanud. Aga ega tollel ajal nooremad ja vanemad väga läbi käinud, 
eriti veel 
maakohtades. Mast oli hea 
poiss, ei peksnud nooremaid ega midagi. 

\question{Mis sealt lühilaine pealt kostis? Muusikat?}

Ei, muusikat kuulati Radio Luxembourgist. Lühilaine pealt tuli erinevaid 
hääli: morset, huvitavaid kahinaid ja sahinaid, keegi 
luges numbreid. Lühilaine on tegelikult siiamaani päris hea tervise 
juures, eeter on maast laeni sodi täis ja olemus 
ei ole väga palju muutunud. Võib-olla propagandasaateid on vähemaks 
jäänud ja Hiina raadiojaamu vaikselt kinni pandud  
interneti pealetulekuga. Üldiselt on lühilaine ilmselt ikka samasugune nagu 
nelikümmend aastat tagasi.

\question{Kas nende ajakirjade hulgas oli arvutiajakirju ka?}

Esimest arvutit nägin tänu poolvennale. Ta tundis Guido 
Tammissaart\index[ppl]{Tammissaar, Guido} Eesti Energia 
arvutuskeskusest\index{Eesti Energia!Arvutuskeskus}. Ühel 
päeval tuli poolvend maale ja ütles: \enquote{Tule kaasa paariks päevaks, näed, 
mis asi 
see arvuti on. Sind see tehnikaasi huvitab.} Ja lubatigi mind paariks päevaks 
maalt 
linna. Estonia puiestee arvutuskeskuses olid tollal veel põhiliselt 
SMid \index{SM EVM}\sidenote{\begin{russian}Система Малых ЭВМ (СМ 
ЭВМ)\end{russian} oli mitut tüüpi Nõukogude Liidus toodetud, enamasti lääne 
analoogidel põhinevate arvutite üldnimetus.}. Ja 
üks CP/M\sidenote{CP/M oli 1974. 
aastal Inteli 8080/85 protsessorisarja tarvis turule toodud 
operatsioonisüsteem, mille 1980ndatel asendas mitmes mõttes sarnane MS-DOS.} 
masin, mis tagantjärele tundub oma sotsmaa disaini poolest täiesti kosmiline. 
Küllap Bulgaarias toodetud. Olen mõelnud, et 
peaks üles otsima, mis masin see selline võis olla. 

Sellel CP/M masinal ma klõbistasin niisama, aga 
SM-4\index{SM EVM!SM-4}\sidenote{SM-4 oli PDP-11/40\index{PDP-11} 
ühilduv 
Nõukogude päritolu ja terves idablokis toodetud arvutisüsteem.} peal 
kirjutasin selsamal prevail oma esimese BASICu\index{BASIC} programmi. 
See oli derivaat mingist asjast, mida mulle näidati, et näed, umbes nii 
käib. Ja edasi ma olin \emph{hooked}. Sellest ühest päevast piisas, et sõltlane 
tekitada. 

\question{See oli enne seda, kui otsustasid, et nüüd oled 
neliteist ja lähed Tallinnasse kooli?}

Ma ei oskagi öelda, ma ei ole sada protsenti kindel, kumb oli enne, kumb 
pärast, ja kas huvi tulla Tallinnasse mängis rolli. Ega nad ju 
arvutikallakut tegelikult ei propageerinud, suurem rõhk oli elektroonikal. 
Tarkvara osa nad väga ei reklaaminud. Minust pidi tegelikult elektroonik saama 
ja see minust ka sai, aga tollal tundusid ikkagi arvutid 
see päris asi. 

\question{Kas 43. keskkoolis valmistati päriselt ka ette 
ülikooliks? Oli sellest kasu?}

See oli selline kahe teraga mõõk -- valmistati ette ja 
väga hästi. Keskkooliprogrammi olid kokku pannud 
tollaste inseneride õpetajad, kes teadsid suhteliselt hästi, mida tuleks 
õpetada, et põhi alla saada. Saime 
läbisegi tavalisi keskkooliaineid ja siis ühel hetkel tuli härra 
Tiidemann\index[ppl]{Tiidemann, Tiit} meile rääkima võllide 
epüüridest\sidenote{Epüür (pr épure) on teatava suuruse asukohast 
olenevate väärtuste graafiline esitus.}. Sisuliselt tegime käsitsi võllidele 
rakendavate jõudude arvutusi, näiteks kust läheb võll katki, kui see on siit
sellise ja sealt säärase jämedusega. Vahelduseks loeti meile 
teise kursuse elektrotehnikat ja 
inseneripsühholoogiat, mida andis Toomsalu\index[ppl]{Toomsalu, Arvo} ja mis ei 
olnud vist üldse TPI õppekavas. Meie õppekava lühinimetusega ETEK\index{ETEK}, 
mille koostasid Ants Reili\index[ppl]{Reili, Ants} ja 
Peeter Grossberg\index[ppl]{Grossberg, Peeter}, oli kõikide jaoks äge 
eksperiment ja täielik \emph{greenfield}, eriti kuna 
olime esimene lend\sidenote{Rohkem ETEKi tausta saab lugeda Anto jutust leheküljelt \pageref{sisu:43kool}.}. 

Lahe oli ka see, et enne meid oli keskkool tühjaks löödud ja me olime kolm 
aastat keskkooli kõige 
vanem klass. Olime koolis nagu jumalad ja 
tänu sellele jäid olemata mitmed probleemid, mida tavalistes 
keskkoolides tol ajal veel eksisteeris. Keegi kedagi ei toginud ega 
nüginud ja samal ajal tekkis kõigil mingisugune väärikus. 

Kahe teraga mõõk oli see aga sellepärast, et nii kõva põhja pealt läksid paljud 
otse tööle. Me saime ju keskkooli lõpetades kõik  
automaatselt TPIsse sisse, sisseastumiseksamit ei olnud vaja teha. Nii et kõik 
meie vist kaheksateist õpilast marssis otse TPIsse. Nendest 
nominaalajaga lõpetas kooli vist paar inimest. Paljud läksid tööle, kuna aeg 
oli 
selline, et see, mida TPIs tollal arvutiteadusena õpetati, ei jõudnud 
päris elule veel järele. See pidi olema aasta 1991 või 1992, kui 
see \enquote{kambriumiplahvatus} siin Eestis toimus.

Mina istusin ööd-päevad arvuti taga ja kirjutasin 
ihuüksi tarkvara, mis pidi 
üleval hoidma tervet suurt autoparki. Samal ajal üritasin ennast kuidagi nügida 
läbi SuperCalci\index{SuperCalc}\sidenote{Varajane tabelarvutussüsteem, 
algselt loodud CP\textbackslash M operatsioonisüsteemile.} arvestusest TPIs, 
kus 
aeg-ajalt tuli õppejõule näidata, et \enquote{ära nii tee, nii see asi päris ei 
käi}. 
Mitte et nad oleksid rumalad olnud, nad õpetasid seda, mida olid kogu 
aeg õpetanud. Nüüd aga tekkis selline seis, kus reaalne elu liikus edasi palju 
kiiremini kui õppekava.

\question{Kuidas sa ikkagi programmeerimise juurde jõudsid? Sa 
pidid seda ju saama kuskil harjutada?}

Tänu 43. keskkoolile see eksperiment kestis ja kestab mõnes mõttes tänaseni. 
Seal oli 
põhimõtteliselt esimest korda selline päris arvutiinimese elu. Kuna 
IT-spetsialiste liiga palju ei olnud, siis juhtus selline hämar lugu, et meile 
Eero Tohvriga\index[ppl]{Tohver, Eero} ulatati kümnendas klassis arvutiklassi 
võtmed ja hakati 
koolist palka maksma. Tegelikult oli see vist seotud 
kerge koolipoolse kaastundega. Peale 
kaheksandat klassi tööstuskooli tulemise traditsiooni ei olnud enam juba 
mõnikümmend aastat ja kõigile tundus see kangesti hirmus, et laps tuleb üksi 
Tallinnasse. Ma arvan, et see oli pigem koolipoolne stipendium. 
Kahe peale maksti meile täisõppejõu palka, mis 
ei olnud ilmselt palju väiksem kui õpetajad 
ise said. Nii hästi kui keskkooli ajal ei ole ma kunagi ei varem ega 
hiljem elanud. 

\question{Mida te selle rahaga tegite?}

Käisime restoranis söömas ja mida ikka lapsed rahaga teevad. Aga kool sai 
selle, et nad ei pidanud rohkem arvutiklassiga tegelema. Klassis oli kolm-neli  
Iskrat\index{Iskra}\sidenote{\begin{russian}Искра\end{russian} oli 
mitmel pool Nõukogude Liidus eri modifikatsioonides toodetud arvutiseeria, mis
 omakorda jagunes erinevaid lääne süsteeme kopeerivateks mudeliperekondadeks.}, 
mida me püsti hoidsime. Meie asi 
oli hoolitseda, et masinad töötaksid ja nendel saaks midagi õpetada. Ühel
hetkel, kui olime ise juba natuke vanemad, tekkis arvutiklassi 
kamp nooremaid huvilisi, kes seal pidevalt hängisid. Arenes
tüüpiline arvutiklassi ökosüsteem. Ühel suvel ka remontisime 
klassi: värvisime ja panime uued põrandakatted. Ühesõnaga käitusime 
loodetavasti heaperemehelikult. 

\question{Tollal ilmselt ei olnud sarnastes situatsioonides heaperemeheliku 
käitumisega eriti 
probleeme?}

Aeg oli selline, inimeste usaldus oli suur. Arvuti oli müstiline ja teistmoodi 
asi, vanem 
generatsioon justkui kartis seda. Kunagi asus Rävala puiesteel, seal, kus 
praegu on Sakala 3 teatrimaja, turismibüroo Sarved ja Sõrad\index{Sarved ja 
Sõrad} (ma ei tea siiamaani, kellele see kuulus). Juhtusin nende akna alt mööda 
minema ja nägin, et neil on 
seal arvuti. See oli vist aastal 1991, igatahes ma veel ei töötanud 
Skriiningus\index{Skriining}. Keskkool 
oli läbi, sinna mind enam sisse ei lastud arvutit kasutama. Eks sõltlane käis 
mööda linna ja järsku nägi arvutit. Tundmatu värske keskkoolilõpetaja 
marssis tundmatusse firmasse hooga sisse, et 
\enquote{teil on siin arvuti, ma tahaksin seda kasutada}. Ja ilma mingisuguse 
tänapäeval heaks kiidetud taustauuringuta ja töövestluseta ütles firma omanik 
oma kirjutuslaua tagant: \enquote{Jah, loomulikult, me tahaksime seda ise ka 
kasutada.} Ja ilma 
pikema jututa anti mulle kontorivõtmed ja öeldi: \enquote{Tee 
see korda, et meie saaksime ka arvutit kasutada}. Ja avastasingi end
arvuti tagant, ilma et keegi oleks isegi dokumenti vaadanud või mõelnud, kas 
tegu on
vargaga, kes tahab terve firma ära varastada või 
ainult arvuti. Usaldus, mis tollal valitses inimeste vastu, kes 
oskasid arvuti sisse lülitada ja sellega midagi teha, oli 
\emph{enormous}.\label{sisu:andrus_usaldus} 
Tänapäeval ei ole võimalik seda ette kujutada. Värskel keskkoolilõpetajal oli 
põhimõtteliselt võimalik küsida ükskõik millise firma ükskõik millise  
arvuti \enquote{võtmed}. 

Noh, see lõppes muidugi sellega, et lõpuks tuli Imre Perli\index[ppl]{Perli, 
Imre}\sidenote{Imre Perli oli pehmelt öeldes raju elulooga Eesti 
arvutispetsialist, kes sai kuulsaks \enquote{Perli andmebaasi} koostajana. 
Kasutades ära ligipääsu mitmele andmebaasile, lõi ta üheksakümnendate keskel 
althõlma levinud \enquote{superandmebaasi}, mis sisaldas isikustatud andmeid autode, (toona üsna 
haruldaste) mobiiltelefonide, aadresside jms kohta.
 Perli hukkus segastel asjaoludel 15. aprillil 2000 
politseioperatsiooni käigus.} ja kopeeris kellelegi andmebaasid. Eks iga 
aeg saab lõpuks otsa. 

\question{Kuidas see programmeerima õppimise protsess ikkagi käis?}

See on eelmisel sajandil tekkinud paradigma, et 
programmeerimine on midagi, mida peab õppima ja millega tuleb 
spetsiaalselt vaeva näha. Programmeerimine juhtub. Vajadusest ja tahtmisest. 
Keegi ei ole mulle mitte kunagi õpetanud 
ridagi C-d ega Assemblerit. 

\question{Ometi said ju kuskilt teada, kuidas \texttt{malloc} käib.}

See sündis tahtmisest teha. Mina hakkasin  
Pascalit\index{Pascal} õppima seepärast, et 
mulle sattus kätte Jürgensoni pruunide kaantega Pascali  
raamat\sidenote{Rein Jürgensoni 
\enquote{Programmeerimine Pascal-keeles} (1985), mis 
huviliste hulgas laialt levis.}, mis on tagantjärele mõeldes 
päris õudne algus programmeerimisele. Kui 
Turbo Pascal hakkas ära tüütama (selles 
keeles midagi normaalset teha oli väga keeruline), siis ühel hetkel 
leidsin, et Assembler\index{Assembler} on see päris asi. Kuna tol 
ajal oli popp kirjutada igasuguseid demosid ja häkkida kõiki tarkvarasid, mis 
kätte sattus, siis \ldots Kuidas õppida x86 
Assemblerit? Võtad raamatu ühte kätte ja AT86 teise kätte ning hakkad tegema.

\question{Kust sa selle raamatu said? Neid ju ei liikunud.}

Liikus küll. Selle eest tuleb tõenäoliselt varem või hiljem anda
presidendi auraha Tarmo Mamersile\index[ppl]{Mamers, 
Tarmo}\index[ppl]{Mamers, Tarmo}, kes oli tollal 
TTÜs\index{Tallinna Tehnikaülikool} üks arvutiasjanduse püstihoidjatest. 
Tarmo kaudu materjalid liikusidki, käest kätte. Tema oli raudselt minu varane 
mentor ja veel pikka aega ka siis, kui ma 
juba tööl käisin. Hiljem tuli 
Fidonet. Kui ma oma esimese Fidoneti \emph{point}'i 
püsti panin, siis oli kõik juba palju lihtsam, sest aken maailma oli
olemas. \emph{Point}'i püstipanemine käis ka loomulikult läbi TPI. Seal käis 
põhiline elu ja\emph{action}   
Aare Tali\index[ppl]{Tali, Aare}\phantomsection\label{sisu!aare_tali} ja Tõnu 
Raimla\index[ppl]{Raimla, Tõnu} toas. Tarmo juures teises ruumis oli natuke 
rahulikum 
õhkkond. 

Ühel  
hetkel (töötasin siis Skriiningus\index{Skriining}) tekkis mul kinnisidee teha 
endale Fidoneti \emph{point}, et 
lõpuks olla osa maailmast. Läksin Aare juurde: \enquote{Noh, Aare, 
sa oled siin \emph{sysop} ja värk} ning Aare ütles talle omase abivalmidusega: 
\enquote{Jah, masin on seal.} Leidsingi ennast seepeale BBSi masina tagant ja 
asusin
valmistama Fidoneti \emph{node}'i. Ilmselt Tõnu või keegi 
lõpuks halastas mu peale ja näitas, kuidas seda päriselt teha. 

Edaspidi oli materjal palju kättesaadavam, sai 
igasuguseid dokumente risti-rästi alla laadida. 

\question{Mida sa TPIsse õppima läksid?}

Ma läksin LIsse. Tollal nimetati seda vist informaatikaks. 
Kuna sain suhteliselt ruttu aru, et ma ei ole võimeline hommikul loengutes 
käima, siis läksime pundi inimestega, kes olid ka otsustanud, et nemad 
peavad õhtuõppes käima, dekanaati ja nõudsime õhtust vahetust. Kateedris öeldi, 
et jaa, väga tore mõte, aga 
meie kogemus näitab, et kui te juba sihukese jutuga tulete, siis vaevalt keegi 
teist seal õhtuses ka käima hakkab. Me ei hakka teie jaoks  
eraldi rühma püsti panema, käite ehitajatega esimese aasta koos koolis. Ja kui 
teisel aastal veel siin olete, siis vaatame seda asja. Kas nüüd osalt selle 
pärast või et dekanaadil oli õigus, nii või teisiti kukkusime 
sealt kõik robinal kolmanda kuu lõpuks välja ja läksime tööle. Nii 
et TPI on mul siiamaani lõpetamata. 

\question{Sa mainisid, et kirjutasid autobaasi softi. Kuidas 
sa seda tegema sattusid?}

Tol ajal \emph{start-up}-kultuuri ja ettevõtluse ehitamist veel ei 
eksisteerinud. Me lõpetasime kooli ajal, kui esimesi 
arvutikooperatiive oli väike käputäis. Minu esimene ametlik töökoht pidi 
olema tegelikult Noorsooteatri valgustaja. Kuna mulle juba 
tollel ajal meeldis audioga tegeleda, siis tahtsin sinna helimeheks minna, 
aga helimees oli värskelt tööle võetud ja valgustaja koht oli vaba. Paar päeva 
enne seda, kui pidin lepingu alla kirjutama, 
küsis Tarmo Mamers\index[ppl]{Mamers, Tarmo}, kas ma ei tahaks ikkagi päris 
tööd teha, kuna Skriining\index{Skriining} otsis programmeerijat. 

Nii sattusingi Skriiningusse Kalle Lotamõisa\index[ppl]{Lotamõis, Kalle} juurde 
tööle. Seal öeldi mulle esimese ülesandena, et \enquote{autopark on sellel 
aadressil}. 
Neil oli mingi eriti eksootilise asja peal jooksev andmebaasisüsteem, 
isegi mitte \emph{mainframe}, vaid mingi mini. Ja see tuli 
moodsale vahendile ümber kirjutada. Moodne vahend tähendas tol ajal Novell 
Netware'i\index{Novell} ja Paul Leis\index[ppl]{Leis, Paul} oli värskelt toonud 
Eestisse sellise asja nagu DataFlex\index{DataFlex}. Tegu oli päris 
 korraliku objektorienteeritud kõrgkeelega. Hakkasin ühest otsast õppima, 
kuidas DataFlexis programmeeritakse, ja 
teisest otsast, kuidas autopark töötab. 

\question{Ahaa, läksid kohe äriprotsessi ka sisse!}

Äriprotsessid olid seal paljuski olemas, st töötav tarkvara 
oli olemas. Pigem oli seal äriprotsesside seisukohast hea lastetuba, et ära 
kunagi eelda midagi. Näiteks mina oma IT-inimese mõistusega tegin oma 
arust mõned asjad paremaks ja siis selgus, et päris nii ei sobinud, nagu 
mina olin mõelnud. Raamatupidaja vaatas mind nagu idiooti ja küsis: 
\enquote{Kas sa ikka saad aru, kui palju ma pean numbreid siia päevas sisestama 
ja seda \emph{enter}'it, mille sa siia vahele toppisid, vajutama? Need arvud 
on neljakohalised. Ma sisestan neli numbrit ära ja 
need lähevad ise järgmisele väljale, mitte ma ei pea vajutama. Ma ei saa
vajutada \emph{tab}'i, mis on teises klaviatuuri otsas. Saad aru? Mul on ühes 
käes
paberid ja teise käega vajutan klaviatuuri. Kuidas ma sinna \emph{tab}'i juurde 
sinu 
meelest saan, kui mul on teine käsi kinni?} 

Nad olid väga innovatiivsed tegelikult selles mõttes, et nad olid 
sedasama andmetöötlust selleks ajaks juba aastat kuus-seitse kasutanud. See oli 
 meditsiinitehnika autobaas, Termak\index{Termak}, siiamaani elu ja tervise 
juures. 


\question{Kas nad olid juba nõukogude ajal arvutiasjandusega alustanud?}

Nad olid jah juba sügaval nõukogude ajal end täiesti ära automatiseerinud. 
Selleks 
ajaks, kui mina aastal 1992 sinna jõudsin, oli nende esimene IT-süsteem 
jõudnud moraalselt nii ära vananeda, et see tuli PCde peale ümber 
kirjutada. Neil oli siis juba \emph{legacy}, nad olid nii palju ajast 
ees.

\question{Kuidas Skriining jõudis selleni, et neil on programmeerijat 
vaja? Lihtsalt kasti sai ju ka edukalt müüa?}

Kalle\index[ppl]{Lotamõis, Kalle} hammustaski selle läbi, et kuna nad olid kogu 
aeg meditsiinitehnika ümber sebinud ja proovinud meditsiinisüsteemi arvuteid 
müüa, oli seal ka arendusvõimalusi. Nii saigi Skriiningust\index{Skriining} 
üheksakümnendate alguses arendusfirma. Arvutimüük käis ka, aga mina  
noore inimesena ei süüvinud sellesse, kust raha tuleb. Ilmselt päris palju tuli 
arendusest.


\question{Kas sa tehnikaülikoolis ka veel ringi hängisid?}

Ma hängisin seal pikalt, kuigi ma ei õppinud seal. Seal oli 
elu epitsenter, kuna seal töötasid kõik olulised inimesed: 
Mast\index[ppl]{Kaal, Madis} ülemisel korrusel, Tõnu\index[ppl]{Raimla, Tõnu}, 
Aare\index[ppl]{Tali, Aare} ja Tarmo\index[ppl]{Mamers, Tarmo} alumisel 
korrusel. Lisaks veel 
Martin Rinne\index[ppl]{Rinne, Martin}, Merle Alliksoo\index[ppl]{Alliksoo, 
Merle} ja kõik teised, kes hiljem Microlinkis\index{Microlink} lõpetasid. Tegu
oli sotsiaalse elu keskusega. 

\question{Mulle tundub see variant, et sa ei õpi, aga hängid, palju 
mõnusam, kui et õpid, aga ei hängi.}

Eks ma ise ikka soovitan teistele kool kohe 
ära lõpetada, sest pärast osutub see palju raskemaks. Mina ja mu sõbrad oleme 
hakanud 
neljakümnendates oma haridusega lõpuks tegelema. On 
tekkinud natuke rohkem vaba aega ja ka moraalne vajadus ---
kuidas sa oled kõige väiksemate pagunitega mees ruumis \ldots

\question{Tol ajal ülikool kuigi palju praktiliselt 
kasulikku ei andnud. Tänapäeval on teistmoodi.}

Paljud ütlevad, et diplom ei olnud mitte tempel selle 
kohta, et tuled koolist välja targemana, vaid tõestus, et 
oled võimeline järjepidevalt, mitu aastat asjaga tegelema. See on pigem 
vastupidavuse ja hoolsuse proov kui koolitus.

\question{Räägi palun BBSidest. Kuidas sa selle \emph{node}'i ikkagi püsti 
said? 
Selleks tuli ju ennast kuskil registreerida?}

BBS oli varane arvutivõrk, mille mõte oli selles, et helistad 
kuhugi oma modemiga ja teises otsas on modem, kes vastab. Modemid saavad 
omavahel andmeühenduse ja siis saab teises arvutis, mille 
küljes teine modem on, ringi sobrada. Kusjuures tollal tõepoolest
sobrati, arvutiturvalisus oli pigem kokkuleppe 
küsimus. Üks suvaline BBSi omanik oleks võinud teise 
omaniku BBSi ilma mingi 
probleemita kaks korda tunnis neljaks tükiks lasta, aga seda lihtsalt ei 
tehtud. See oli nagu 
saarlase ukselukk: kui oled luku ukse ette paika pannud, siis kõik 
teavad, et sind ei ole kodus ja nii on. Ei ole vaja katsuda, kas uks 
on lahti või kinni, kedagi ei ole kodus. BBSidega turvalisusega oli sama lugu. 

BBSi teine ja palju kasulikum omadus oli see, et kui 
oli olemas modem ja arvuti, siis sai ennast Fidoneti 
\emph{node}'iks registreerida. BBS iseenesest ei eeldanud midagi sellist, vaja 
oli vaid
modemi ja vastava tarkvara olemasolu. Mingeid hämaraid teid pidi levisid 
telefoninumbrid, kuhu helistada ja end kohapeal ära registreerida.

Fidonet oli esimene üleilmne arvutivõrk selles 
mõttes, et modemid helistasid üksteisele automaatselt. See oli ka kaunikesti 
hästi toimiv elektronpostiteenus, mille üks eriline omadus 
oli veel see, et see liikus väljaspool KGB huviala. Eks küll 
kahtlustati, et seda kuulatakse pealt ja aeg-ajalt mingid imelikud modemid 
üritasid sinu modemiga poole jutu pealt rääkida, aga üldiselt seda vist väga ei 
jälgitud. Ma vähemalt ei tea, et kellelgi oleks 
kaheksakümnendatel olnud modem-modemiga sidepidamisega probleeme, ei Eestis ega 
välismaaga. Mis on selles mõttes eriti huvitav, et kui kaugekõneliinid läksid 
nii palju lahti, et oli võimalik kuhugi automaatvalida, siis me ju helistasime 
igale poole välja.  
Fidoneti \emph{mail}, mis tuli Eestisse umbes aastal 1988 või 1989, oli esimene 
vaba ja 
demokraatlik sidekanal väljapoole.

Mina olin siis keskkoolis, esimese \emph{node}'i panin püsti umbes 1991. 
aastal. Ma olingi vist Aare \emph{point}. 
Omaenda \emph{point}'i numbrit ma enam ei mäleta, võibolla oli 
kaksteist-kakstest. \emph{Node}'i number oli
kolmkümmend viis. Eesti oli sel ajal ülemineku vabariik. 
Registreeritud postiaadress andis võimaluse foorumites 
kaasa rääkida. Eestis oli kümmekond gruppi, kus käis jutt erinevatel 
teemadel. Mõnes mõttes oli elu selline, nagu oleme täna 
harjunud, kuigi natuke teistsuguste tehniliste vahenditega. Post oli aeglasem 
ja 
saabus paar korda 
päevas, mitte reaalajas. Ei olnud nii, et kirjutan kirja ja see läheb kohe 
kõigile laiali. Sanas täitis see kõik need ülesanded, millega täna tegeleme, 
ära. Nii et kaheksakümnendate lõpus, üheksakümnendate alguses oli see 
\enquote{ökosüsteem}, millega täna oleme harjunud, täiesti olemas ning 
väike käputäis inimesi Eestis omasid selle kasutamise privileegi. 

\question{Kas see väike käputäis olid pigem entusiastid, akadeemiline 
seltskond või kes?}

Fidoneti ökosüsteem koosnes sada 
protsenti entusiastidest. Akadeemilised inimesed läksid ärisse, panid püsti 
esimesed arvutifirmad ja üritasid raha teha. 

\question{Kas eksisteeris ka mõningane spetsialiseerumine, et siit saab 
tarkvara ja seal on huvitavaid jutte-raamatuid?}

BBSidel väike spetsialiseerumine oli, aga mitte eriti suur. Eks 
enam-vähem kõik proovisid endale kõhu alla korjata, mida vähegi said. 
See oli aeg, kus tekkisid esimesed suuremad kõvakettad. 
Lühikest aega valitses olukord, kus tarkvara 
oli vähem kui ruumi. Ruumi mõiste oli ka muidugi tollal huvitav. Kõige 
rohkem ruumi võtsid Sierra\sidenote{1979. aastal 
asutatud Sierra Entertainment (varem On-Line Systems ja Sierra On-Line) 
disainis paljud toonased hittmängud. Eriti populaarsed olid 
seiklusmängude sarjad \emph{King's Quest}, \emph{Space Quest} ja \emph{Leisure 
Suit Larry.}\index{Larry (mängusari)}} mängud, mis olid flopiketaste peal. Neist suuremad, Space 
Questid\index{Space Quest} ja muud, tulid viie-kuue flopi 
kaupa. Mäletan, kuidas arutasime Eeroga\index[ppl]{Tohver, Eero}, et 
kui oleks võimalik panna kokku oma unelmate masin, siis kui suur kõvaketas sel 
peaks olema. Jõudsime järeldusele, et kui oleks umbes kaheksakümmend megabaiti, 
siis ilmselt jätkuks eluajaks, sinna saaks kõik mängud ja
tööasjad peale panna ning umbes pool jääks veel üle.

\question{Sierra oli omaette fenomen, seda mängiti palju. Kas keegi
seda müüs ka?}

Küsime laiemalt, kas Eestis üldse keegi tol ajal tarkvara müüs. 
Äritarkvara, nagu Novell, oli võimalik osta. Teoreetiliselt oli 
Windowsi või DESQview'd\index{DESQview}\sidenote{DESQview oli kaheksakümnendate 
lõpus ja üheksakümnendate algul populaarne tekstipõhine mitmetegumiline 
keskkond, mis toimis DOSi peal ja võimaldas korraga mitut programmi eri akendes 
käimas hoida.} kindlasti kuskilt võimalik osta. Aga peale Novelli serveri ja 
DataFlexi 
litsentside ei mäleta ma, et oleks üheksakümnendatel kellelgi 
legaalset tarkvara näinud. 

\question{Tuleme tagasi BBSinduse juurde. Kas selle sisu hulk, 
mida enda kõhu alla õnnestus kokku kuhjata, oli ka staatuse 
sümbol?}

Ma ei oska öelda, oskan ainult enda BBSide kohta rääkida. Mina 
korjasin kokku kõik, mida kätte sain, ja pakendasin ringi. See 
oli selline kultuuriküsimus, et tarkvara skaneeriti viiruste vastu 
kõige värskema skanneriga, mis parasjagu käeulatuses oli, ja see käis 
muidugi automaatselt. Siis lisasid arhiivi väikese faili, 
mis sisaldas sinu \emph{header}'it -- väikest 
failijuppi, kus oli graafiliselt (või tollal pseudograafiliselt) sinu 
logo sisse punnitatud. Ja siis panid selle välja ja oma faililisti nupukese, 
millega tegu. 

See oli nagu \emph{basic housekeeping}. Kui sinu fail läks 
järgmisse BBSi, siis see viskas sinu logo välja ja pani enda oma 
asemele, \emph{tag}'iti ära nagu grafitiga, et see on 
minu käest tulnud asi. Vähemalt mul oli küll tunne, et välja läks 
kõik, mida olid ise endale mingil põhjusel hankinud. Mitte küll nii, et 
tõmbasid öösel HNSi\index{HNS} tühjaks ja 
panid enda lehekülje peale välja, küll aga mõned asjad, mille olid kätte 
saanud. Duplikaate ei olnud väga palju üllataval kombel.

\question{Tahtsingi küsida, et sedasi oleks pidanud ühel hetkel ju kõigil 
kõik olemas olema, aga seda siis ei tekkinud?}

Seda ei tekkinud. Kuna BBSid olid väga stabiilselt üleval, siis enda jaoks 
vajalikud asjad tõmmati
ära ja pandi omakorda enda juurde üles. Mõttetut \emph{leach}'imist ja püüet 
iga hinna eest oma failiandmebaas kõige suuremaks saada 
väga ei olnud. 

\question{Too mõni näide, mis laadi asjad sulle toona huvi pakkusid.}

Olin siis juba vihane \emph{nerd}, minu spetsialiteet oli 
programmeerimismaterjalid ja -vahendid, käsiraamatud ja
tööriistakesed. 
Kahjuks mul ei ole seda vana faililisti alles, sest kui ma Skriiningust ära 
läksin, lendas see vana SCSI-ketas, 
mille peal BBS jooksis, õhku. \emph{Backup}'i sellest ei olnud ja 
kogu Fidoneti \emph{node} koos failibaasiga läks hingusele.

Ma ise seda järgmisse kohta kaasa ei võtnud, sest läksin Skriiningust panka, 
kus 
olid ees sellised kõvad mehed nagu Mast\index[ppl]{Mast} ja 
Marx\index[ppl]{Marx|see{Kliimask, Margus}}\index[ppl]{Kliimask, Margus}, kes 
olid oma ökosüsteemi püsti pannud. Ühele BBSile seal rohkem ruumi ei olnud. 

\question{Mis panka sa läksid?}

Mina läksin sellesse panka, mille lõpupidu kohe 
kätte jõuab\sidenote{Intervjuu Andrusega toimus 2019. aasta novembri algul} -- 
praegune Danske\index{Danske Pank}\index{Danske 
Pank|see{Forekspank}}, toona Forekspank\index{Forekspank|see{Eesti 
Forekspank}}. 

\question{Miks sa sinna läksid? 
Skriiningus said ju programmi kirjutada ja BBSi pidada.}

Nagu ma paljudesse kohtadesse olen läinud -- sellepärast, et kutsuti. Ja 
kuna parasjagu jooksis Eestis teleseriaal \emph{Capital City}, mis 
näitas panganduselu väga glamuurse \emph{highroller}'ina, siis mulle tundus, 
et mina tahan ka nii elada. Tuleb tunnistada, et üheksakümnendate panganduses  
ei pidanud väga pettuma, elu oli täitsa lill. Päris nii nagu 
teleseriaalis \enquote{Pank} elu meie majas küll ei käinud. 
Päris hulle pidusid sai peetud, aga et keegi oleks kokaiinise  
ninaga ringi käinud, seda mina ei tea. Meie kandis oli kokaiin täiesti 
tundmatu või ehk tehti seda salaja, mina küll
narkootikumidega pidusid ei näinud.

\question{Kas mäletan õigesti, et tollal tõmbasite panka 
püsiühenduse\sidenote{Enamik varasest internetiühendusest Eestis toimis kuhugi 
sisse helistades. See tähendas, et pidev side puudus ja side 
kvaliteet sõltus suuresti analoogtehnoloogial põhinevatest 
telefonikeskjaamadest. 
Püsiühenduseks kutsuti seda, kui asutusest jooksis füüsiline kaabel interneti 
külge 
ja kaabli olemasolu oli IT-inimeste unelmates kesksel kohal.} sisse?}

Püsiühenduse tõmbasime sisse väga konkreetsel päeval. 
Modemitega oli n-ö poolpüsiühendus juba pikemat aega olemas.  
Forekspank asus Rävala puiesteel, nagu 
juhtumisi ka KBFI\index{KBFI}\sidenote{Keemilise ja Bioloogilise 
Füüsika Instituut\index{Keemilise ja Bioloogilise Füüsika Instituut|see{KBFI}}
 (KBFI). 1979. aastal Endel Lippmaa\index[ppl]{Lippmaa, Endel} 
loodud teadusasutus, tuntud ka kui \enquote{Lippmaa Instituut}. Just 
Lippmaade perekonna aktiivse ja laiahaardelise tegutsemise tõttu mängis 
instituut rolli paljudes toonastes olulistes protsessides (sh kohaliku 
interneti arengus).}. Baumaniga\index[ppl]{Bauman, Andres} 
oli läbi räägitud, kuidas internetti saab, ja meil oli suhteliselt 
rivitu ligipääs. Samas tundus ühel hetkel, et see võiks ikka päriselt 
permanentne olla. Võtsime Mastiga\index[ppl]{Mast} kaablirulli ja 
hakkasime üle Rävala puiestee katuste KBFI poole liikuma. Tähelepanuväärne oli, 
et see juhtus päeval, mil Eestit väisas esimest korda paavst 
\sidenote{Paavst Johannes Paulus II külastas Tallinna 10. septembril 1993.}. 
Kõik katused olid snaipreid täis, kehtestati tohutu 
\emph{lockdown}, et keegi paavsti käigu pealt ära ei tapaks.  
Seletasime kõigile, et meil on vaja kaablit vedada ja paneme interneti 
püsiühendust. See oli maagiline valem, mis võimaldas ligipääsu 
kõikidele kesklinna katustele, ilma et keegi oleks midagi küsinud. Me küll 
otseselt snaiperitega samale katusele ei sattunud. Natukene tuli häkkida ka, 
et ühest koodlukust läbi minna, aga see ei olnud suur takistus. 

\question{Toona oli maailm järelikult teistsugune. Internet ei 
olnud veel kommertsiaalne, vaid pigem kogukondlik nähtus.}

Selle eest vististi keegi maksis ka kellelegi midagi, aga kui palju, 
seda jällegi ei mäleta. Eks see käis paljuski inimsuhete baasil. 
Kuna me tundsime Andres Baumani\index[ppl]{Bauman, Andres}, siis kuidas raha
seal tegelikult liikus, seda ma ei tea. Mast\index[ppl]{Mast} ajas seda asja. 
Millegipärast ma arvan, et maksime KBFI-le midagi. 
Tegelikult oli meil alates
üheksakümne viiendast aastast
Forekspangas\index{Forekspank} infotehnoloogiliselt selline elu nagu 
tänapäeval. 
Suhteliselt samal ajal tuli Mosaici\index{Mosaic}\sidenote{NCSA Mosaic oli üks 
esimesi internetibrausereid ja mängis WWW populariseerimisel olulist rolli. 
Sama meeskond lõi hiljem Netscape\index{Netscape} Navigatori, mis oli  
Firefoxi eelkäija.} brauser, hakkas veeb arenema ja 
tekkisid meile kõigile e-posti aadressid (need olid 
küll juba pisut varem KBFI kaudu korraks olnud, aga siis tekkisid need 
meie oma foreks.ee domeeni külge). Kogu see ökosüsteem, miinus Facebook, oli 
meil siis juba olemas. 

Tollal me ka täitsa tõsimeeli arutasime, 
et KBFI ühendus on ikkagi nii aeglane, et ehk peaks kogu 
veebi kohalikku serverisse kopeerima. Ja 
kuna see mahuks tõenäoliselt ühele DVD-le ära, siis ehk peaks tegema 
äri ja hakkama müüma internetiga DVDd. 

\question{Ka teistest intervjuudest käib läbi, et toonane maailm põhines 
suuresti 
inimsuhetel. Ometigi ei hakka inimesed arvutitega tegelema, kuna neile 
meeldib tegeleda inimestega. Samas tunduvad Eesti arvutiinimesed küllaltki 
suhtealtid ja -osavad. Miks see nii on?}

Kui inimesel on arvutihuvi, siis on ta
terve keskkooli ja pool ülikooliaega olnud sotsiopaat ning tal ei ole eriti 
olnud kellegagi millestki rääkida. Ja ühel hetkel leiab ta üles omasugused, 
samasuguste huvidega. Puhas \emph{nerd}'i ja nohiku käitumine, eks ole! 
Kui panna nohikud kõik ühte tuppa kinni, siis nad leiavad 
üksteist ja kõigil on järsku lõbus, sest kõik lõpuks ometi naeravad samade 
naljade üle. Pidudega on sama lugu. Kõige karmimad peod, kus ma 
olen osalenud, on ikkagi olnud inimestel, kelle igapäevatöö on kaunikesti 
\emph{boring}. Ma ei taha anda hinnangut teatud inimgruppidele, aga kui näiteks
 raamatupidajad ja andmesisestajad käima lähevad, siis see on 
ikka täiesti teine tase. Keskmised lõbusad inimesed on lõbusad 
kogu aeg. Aga kui nohkarid lõpuks lõbusaks muutuvad, siis juhtub asju.

Nii et see ökosüsteem toimis tänu sellele, et inimestel oli hea meel üksteist 
leida. 
Algul oli neid alla saja, 
võib-olla isegi alla viiekümne inimese. Tegu oli uue 
laine arvutitegelastega, kelle seast suur osa meie tänasest 
\emph{startup}-ettevõtlusest 
ongi välja kasvanud. Tänu tihedale suhtlusele hakkasid 
toimuma ka legendaarsed BBSummeri\index{BBSummer}-nimelised üritused. 

\question{Räägi lõpetuseks, mida sa praegu teed.}

See on võib-olla masendav tõdemus, aga elu pole mind sellelt kursilt
kaugemale ega kuhugi mujale viinud. Laias laastus 
tegelen täna täpselt sama asjaga, millega kakskümmend viis aastat 
tagasi. Olen pendeldanud elektroonika ja tarkvara vahel, 
olnud mitme firma CTO, asutanud firmasid ja neid kihva keeranud, töötanud 
teiste juures ja endale. Ja kui keegi küsib, millega ma tegelen, 
siis tavaliselt ütlen, et annan masinatele hinge. 

\question{See on ilus ütlemine ja läheb kokku küsimusega, mis jäi enne 
küsimata. Tavaliselt inimesed tegelevad kas riist- või tarkvaraga, aga sinul 
tundub olevat üks jalg ühes ja teine teises?}

Mõeldes oma elu peale, siis ma muidugi tahaksin, et tarkvara oleks mu tõmmanud 
endasse. See on mõnes mõttes nii palju lihtsam ala. Vigu on palju 
lihtsam parandada ja katkiseid asju ei tule peaaegu üldse ära visata. 
Kettaruum ei maksa täna eriti palju erinevalt elektroonika valmistamisest ja
utiliseerimisest.

Mul on kuidagi juhtunud niimoodi, et kui panen 
tule vilkuma ja näen, kuidas minu tehtu manifesteerub päris asjades, 
siis mul läheb tuju paremaks. Mul tuleb elektroonika disain välja ka. Kuna ma 
olen ikkagi ka
programmeerija, siis olen sattunud sinna omamoodi sidemeheks. Ma suudan tõlkida 
riistvara tarkvara jaoks ja vastupidi. Selle konkreetne töönimetus on 
\emph{embedded engineering}. Vaadates, mis meil täna koolidest 
saabub, siis on see täiesti väljasurev kunst. Neid tegelasi, kes suudavad nii 
riistvara valmistada kui ka sellele tarkvara peale kirjutada, 
nimetatakse mehhatroonikuteks või kelleks iganes, aga fakt on see, et nende 
juurdekasv on järsult pidurdunud ja varem või hiljem hakkab see 
probleemiks muutuma. Tõsi küll, ka töömeetodid muutuvad. Me kasutame täna
töövahendeid, mis annavad näiteks tarkvaratiimile parema 
ettekujutuse riistvarast kui vanasti. Kirjeldused ja 
\emph{markup language}'id, millega seda tehakse, on paremad. Masinale hinge 
andmine tähendab seda, et kui sa näiteks lülitad oma pesumasina sisse, siis on 
oluline, mida see oskab või ei oska sinu 
heaks teha. Hea kasutajakogemus tuleb sellest, kui hästi raua ja tarkvara 
kooslus 
on välja mõeldud. 

\question{Sa ütlesid enne, et sa oled ka CTOna toimetanud. Järelikult tuleb
kolmas element juurde -- sa pead suutma selle kõik ka äriks tõlkida.}

CTO ametit on kaht sorti. Tavaliselt väikestes firmades tähendab 
CTO olek seda, et koosolekule on vaja kedagi kaasa võtta ja kuidas sa 
ütled, et ta on mul programmeerija, eks. Sa pead talle andma 
visiitkaardi, millega ta näeb presentaabel välja. Väikefirma CTO 
teeb kõike, millel on tehnika maitse 
küljes. Suurema firma CTO tähendab, et ta ongi CTO. Tänases 
\emph{start-up}-maailmas on \emph{customer fit} ja \emph{market fit} kõva 
teema. 
Vanasti sellega väga ei tegeletud, aga nüüd, kus on tohutu kuhi 
investorite raha põlema pandud, ilma et sellest oleks isegi sooja saadud, on 
hakatud rääkima sellest, et toodetut peaks kellelegi päriselt ka
tarvis olema. See paistab olevat uus asi, viimase paari aasta 
paradigma. Kaks-kolm aastat tagasi hakkas Silicon Valleys 
pihta see kultuur, et laste kätte ei taheta raha enam hästi anda. Ehk 
nende kaheksateistaastaste imeettevõtjate aeg, kes suudavad väga suure kuhja 
raha 
korraga põlema panna, nii et sooja ei saa, on läbi saanud. Nüüd on selgunud 
innovatiivne lähenemine, et toodet peab 
kellelegi tarvis olema. See tähendab, et projektidele on erakordselt raske raha 
saada, sest kõik 
on järsku pirtsakas muutunud ja nõudnud, kust raha tagasi tuleb. 

\question{See läheb ju kokku sinu kunagise ettevõtte uksest sisse minekuga: 
seal pidid ka kohe kasulik olema ja ei tohtinud asju tuksi 
keerata.}

Kasumlikkus on tegelikult õudselt valus teema. Riistvaraga on  
asi selles mõttes selgem, et riistvara ei skaleeru, kui keegi seda ei osta. Sa 
ei 
saa valmistada sedasama \emph{recorder}'it, millega me siin praegu salvestame, 
miljon tükki, kui keegi ei osta. Sa lähed 
pankrotti. Tarkvara tiražeerimine ei maksa aga midagi. Ja täpselt samamoodi 
võib  
juhtuda, et tarkvara, millest mitte kellelegi mitte pennigi ei teki, on 
tegelikult väga kasulik. Seega kasulikkus ja ärimudel ei tähenda veel mitte 
midagi. Dotcomi- ja igasugu tarkusemullidega kipub tavaliselt juhtuma, et väga 
raske on tõmmata piiri selle vahel, kus asi ei teeni 
raha sellepärast, et väga head mõtet ei ole veel õpitud rahaks  
tegema, ja nende asjade vahel, mis ongi täiesti mõttetud. 
Seetõttu on väga palju tegelasi, kes suudavad maha müüa täiesti kasutu idee, 
öeldes, et tegu ongi monetariseerimiseelse faasiga ja see ei peagi midagi 
tootma. 
Unustades ära, et tegu on ühtlasi täieliku kräpiga. 
Viimasel ajal on tekkinud paar niisugust suuremat skandaali, näiteks
õnnetu Theranose \emph{case}, kus suudetakse endale nii veenvalt 
valetada. Terve ökosüsteem on üles ehitatud väga kasulikest 
asjadest, mille ainus viga on see, et fundamentaalne eeldus, millele süsteem 
rajati, oli täiesti vale. 

\question{Nii et selle kahekümne viie aastaga ei ole maailm väga muutunud,
aga toimib siiski natuke teisti?}

Üks asi on oluliselt erinev. Tollal valmistati tarkvara kahel põhjusel. 
Esiteks oli seda tarvis, mis tähendas tugevat kliendipoolset 
tõmmet. Teiseks taheti, et midagi sellist eksisteeriks 
maailmas, mis tähendab, võeti lihtsalt kätte ja kirjutati tarkvara kas enda 
või teiste rõõmuks ning lasti lihtsalt maailma. Hästi palju väikesi ja 
kasulikke 
utiliite olid ju tegelikult kirjutatud kellelgi 
enda jaoks, siis pakendatud ja laiali saadetud. Eestis seda kontseptsiooni 
polnud, et teha tarkvaraga 
raha: kirjutada mõni vidin ja küsida selle eest tasu. \emph{Corporate} maailmas 
tollal 
küll juba osteti-müüdi igasuguseid raamatupidamissüsteeme väga 
edukalt ja see kõik töötas. Mujal maailmas tegeleti utiliitide 
pealt raha teenimisega ka väikest viisi. Eestis üldse mitte. Tänapäeval on 
tarkvara tootmine läinud niimoodi, et kellelgi tuleb mõni väga
hea idee ja ta tahab sellest teha raha tootmise masina. Asi on vastupidine: 
mitte 
vajadus-, vaid unistuspõhine. Nagu me 
aeg-ajalt Ivar Zaransiga\index[ppl]{Zarans, Ivar} naerame, et kui vanasti 
otsiti probleemidele lahendust, siis tänapäeva 
maailmas otsitakse probleeme neid vajavatele lahendustele. See on viimase 
kahekümne viie aasta jooksul kõige suurem paradigma muutus.