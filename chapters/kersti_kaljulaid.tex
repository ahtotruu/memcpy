\index[ppl]{Kaljulaid, Kersti}

\question{Alustaks kohe sealt, kust asjad ikka algavad ja kust me oleme kõiki 
neid oma jutuajamise alustanud. Ehk  sellega, et kuidas jõudsite teie arvutite 
juurde ja kuida arvutid jõudsid teie juurde?}

No see tegelikult juhtus üsna ammu, see juhtus Nõukogude Liidus ja see juhtus 
Õpilaste Teaduslikus Ühingus\index{Õpilaste Teaduslik Ühing}. Ma arvan, et 
päris paljud nendest inimestest, kes hiljem  on töötanud Eesti e-riigi või ka 
meie startup kogukonnas ja on sellised minuvanused,  teavad, mida tähendavad 
sõnad nagu Küber\index{Küber} või Küberi arvutuskeskus. Tartu 
Ülikoolil\index{Tartu Ülikool} olid samuti olemas arvutuskeskused. Nii põhjas 
kui lõunas mingil põhjusel otsustasid täiskasvanud, et nad lasevad lapsi sinna 
mängima. 

\question{Aga kes täiskasvanutest?}

Õpilaste Teadusliku Ühingu eestvedajad. Näiteks Peeter 
Lorents\index[ppl]{Lorents, Peeter}, kes juhtis matemaatikasektsiooni. Aga neid 
oli seal teisi ka. Kes Tartus eest vedas, ma ei tea, aga ma tean, et ka Tartu 
koolinoored, näiteks Unineti\index{Uninet} Taavi Talvikul\index[ppl]{Talvik, 
Taavi} oli ligipääs Tartu Ülikooli arvutipargile. Toimus selline instinktiivne 
õpe, mis viis meid näiteks Õpilaste Teaduslikus Ühingus selliste aruteludeni, 
et kuidas kirjutada sellist asja, mida kekserveril oleks mõnusam analüüsida. 
Tollel ajal oli nii, et üks asi arvutas ja ümberringi olid terminalid, kus me 
siis oma koodi kirjutasime. Tolleaegsed masinad olid mitteselektiivsed, 
kindlasti ta ei otsinud, kes meist efektiivsema rea on kirjutanud, et seda siis 
töödelda. Aga meile nagu tundus, et äkki oleks võimalik üksteisega kunagi 
niimoodi võistelda, et kes kirjutab sellise asja, mida sellel keskserveril 
oleks mõnusam analüüsida. Ma mäletan laadilisi debatte, ükskord isegi vist 
Õpilaste Teadusliku Ühingu suvelaagri\index{Õpilaste Teaduslik 
Ühing!Suvelaager} öine matemaatika debatt. 

Mina ei kuulunud, rõhutan, matemaatikasektsiooni, aga vahest mind kuidagi võeti 
kampa. Ega ma täpselt ei tea, miks. Ma olin Õpilaste Teadusliku Ühingu 
teaduslik peasekretär üheteistkümnendas klassis, aga ma ise olen tegelikult 
ornitoloogiasektsioonist. Mulle meeldis sihuke Linné-aegne 
bioloogia\sidenote{Carl Linnaeus, pärast aadliseisusse tõstmist 1761. aastal 
Carl von Linné (1707-1778), oli Rootsi teadlane, kes formaliseeris organismide 
nimetamise süsteemi ja keda tuntakse moodsa taksonoomia isana.}, mis on ju 
koolilapsele kättesaadav. Linnud, loomad, taimed, kõik see viis mind lihtsalt 
laia maailma. Selles mõttes laia maailma, et  küll kuuendikul planeedist, aga 
sai reisida, eks ole, kui sa olümpiaadidel käisid ja niisugused asjad. Aga, 
jah, Õpilaste Teaduslikus Ühingus,  tekkis mul kokkupuude selle Küberi pundi ja 
Küberi arvutusvõimsusega.

\question{Ehk, isegi ornitolooge viidi arvuti juurde?}

Ei, ornitolooge ei viidud, mul lihtsalt olid matemaatikasektsioonis sõbrad. Aga 
Lorents\index[ppl]{Lorents, Peeter}, Engelbrecht (TODO) ja teised kuidagi nagu 
nad ei teinud üldse vahet, selles mõttes, et me võisime olla 
neli-viisteist-kuusteist, aga me saime osaleda täiskasvanute, ütleme siis, 
akadeemilistes mängudes.

\question{14--15 aastane noor inimene, eriti, kui tal on ornitoloogiahuvi, siis 
on tal miljon muud asja teha. Mis tõmbas just arvuti juurde?}

See oli ikkagi põnev maailm ja, teine asi, et minu ornitoloogia-alane ainukene 
publitseeritud teadustöö tegeles vainurästa pesitsuskommetega. Aga see ei 
käinud nii, et ma käisin metsas ja vaatasin, kus vainurästas elab. Selle olid 
teised inimesed ära teinud. Eestimaa ornitoloogiaühingus (TODO: otsi nimi ja 
pane indeks) (või kuidas iganes vene ajal seda kogukonda ka ei nimetatud) oli 
kogunenud meeletus koguses pesitsuskaarte ja see oli kõik täiesti 
süstematiseerimata materjal. Minu akadeemiline tegevus Õpilaste Teaduslikus 
Ühingus, seisneski selles, et ma statistiliste meetoditega otsisin erinevaid 
korrelatsioone. Sealt tuli välja näiteks selliseid asju, et linnas vainurästas 
pesitseb kõrgemal kui näiteks siis, kui ta on looduslikus biotoobis. Hea lihtne 
näide mitte-ornitoloogile. Minu töö oli tegelikult statistiline ja ma arvan, et 
 sealtkaudu ma kuidagi sattusin arvutite juurde. Ma arvutasin, jah. 
Vainurästast polnud vaja selle töö jaoks isegi mitte metsas ära tunda. Mitte, 
et ma ei tunneks, aga vaja ei olnud, eks.

\question{Milles too akadeemiline mängimine seisnes, mis tüüpi ülesandeid te 
arvutiga lahendasite?}

Mõni täiskasvanu oli alati ka juures  või tulime me ise selle peale, lihtsad 
programmeerimisharjutused ilmselt, mida täna teevad paljud lapsed 
algkoolideski. Mina kaugemale sellega väga ei jõudnudki, sest, ma rõhutan, ma 
ei olnud matemaatika sektsioonis.

\question{Aga mis tüüpi inimesed seal matemaatikasektsioonis olid? Sellised 
tüüpilised nohikud?}

Ei, seal oli erinevaid. Ma ei tea, Tarvi Martens\index[ppl]{Martens, Tarvi}, 
Tarmo Uustalu\index[ppl]{Uustalu, Tarmo} on täitsa eri kategooriate inimesed, 
eks ole. Oli  erinevaid matemaatikahuvilisi noori ja neid oli eri Eesti  
nurkadest ka. Minu arust ei ole olemas mingit stereotüüpi, mida alati 
otsitakse. Gruppide vahelised erisused, teadupärast, on väiksemad kui 
grupisisesed.


\question{Kas Õpilaste Teaduslikul Ühingul oli ka võrgustiku loomise 
funktsioon? Inimesed said üksteist tundma?}

Kindlasti.  Olid erinevad sektsioonid matemaatika ja loodusteaduste ja 
geograafia ja ka ajaloosektsioon, sealhulgas NSV Liidu ajaloo sektsioon, mis 
pandi ükskord kinni, mille üles mõned punasemad noored olid vähe nördinud. Ja 
paljud inimesed, kes olid mujal sektsioonides, näiteks Teet 
Jagomägi\index[ppl]{Jagomägi, Teet}, tänaseks selgelt IT-ettevõtja, juhtis 
geograafiasektsiooni.Päris palju seda praeguseks umbes 50-aastast, pluss-miinus 
kolm-neli aastat, kogukonda on sealt ühel vnoi teisel viisil läbi käinud. Kõik 
tundsid kõiki, nagu ikka tollel ajal see asi käis.

\question{Mis siis sai, kui Teadusliku Ühing ära lõppes?}

Ülikool tuli. Tartu Ülikool minu jaoks ja  üksiti ka nagu hüvastijätmine 
Linné-aegse bioloogiaga, sest et minu juhendaja Raivo Mänd\index[ppl]{Mänd, 
Raivo} ütles, et tuleb õppida uusi asju. Neid, mis  tulevikus leiva lauale 
toovad. Ja sellepärast minu eriala on geneetika, täpsemalt plasmiidi geneetika 
või bakterigeneetika. Ja ega  bioloogias on ju keemia, füüsika ja matemaatika 
kõik koos, et kindlasti tugevnes selle stuudiumi jooksul minu arusaam 
matemaatikast kui  kõike kirjeldavast ja kõiges toeks olevast teadusarust. 
Matemaatika on minu jaoks nagu keel. Ma ei ole selles ülearu osav, aga vajaduse 
piiresse on toimetatud.

\question{Tartu Ülikooli Arvutuskeskuse\index{Tartu Ülikool!Arvutuskeskus} 
kohta on olnud selliseid ebamääraseid jutte sellest, kuidas seal käis koos üsna 
kirju seltskond teoloogidest jumal teab kelleni. Kas selle kohaga ka 
kokkupuudet oli?}

Arvutuskeskuse seltskonna kirjusus tulenes ka muuseas sellest, et matemaatika 
ja füüsika olid erialad, millele konkurss Tartu Ülikoolis puudus. Seal oli 
alati kohti rohkem kui rahvast ja tihtipeale pugesid nendesse teaduskondadesse 
peitu ka inimesed, kes iga hinna eest tahtsid näiteks vältida Nõukogude 
sõjaväge. Minu aastal ülikooli astunud füüsikutest vist üks lõpetas füüsikuna. 
Aga näiteks astus sinna sisse minu mälu järgi näiteks Anzori 
Barkalaja\index[ppl]{Barkalaja, Anzori}, kindlasti teise eriala inimene. See 
seltskond oligi kirju, aga tegelikult ju arvutiteadus ongi suuresti 
interdistsiplinaarne, see ei ole mingisugune spetsiifiline asi. Ma küll olen 
ise kodus märganud, et see on kuidagi nagu pärilik. Et  minu esimene abikaasa 
ja minu vanem poeg mõlemad elavad selles arvutimaailmas, elavad ja hingavad 
bittee ja baite, ja kuigi seda on ju võimatu näidata, et kuidas pärilikkus saab 
millegi nii tehnogeensega koos käia, aga mulle tundub, et mingisugune aju tüüp 
siiski on. Minu vanem tütar samamoodi. Nad evivad seda arvuti-inimese aju. 
Mulle tundub, et tollel ajal hakkas selguma, et inimestel on mingid 
arvuti-inimeste ajud, nad nagu tõmbusid arvutuskeskusesse kuidagi kokku, said 
üksteisest aru ja vähehaaval hakkasid kaotama sidet humanitaarset poolega 
ühiskonnast.

\question{Mõtlesin, et see lause lõpeb sellega, et \enquote{hakkasid kaotama 
sidet reaalsusega}, sest ka see juhtus seal majas kergesti\ldots}

Ei, ma arvan seda mitte. Ma arvan, et see, mis juhtus, oli see, et osasid 
inimesi, see maailm ei kõneta ja teisi kuidagi intuitiivselt nagu kõnetab. Ma 
olengi kogu aeg tundnud, et ma seal sees ei ole, aga võib-olla suudan kahe 
maailma vahel natuke tõlkida. See tunnetus on mul olnud päris varasest 
noorusest peale. Ma olen telekomisektoris töötanud, ega  telefonijaam on ju 
arvuti: neid tuli konfigureerida, programmeerida, kõik need asjad. Ma olen 
palju koos töötanud selliste inimestega, kes peavad oma tööks töötamist 
arvutitega ise seda tegemata, küll aga püüdes luua neile töötingimusi mõnes 
ettevõttes, eks ole, üheksakümnendatel. Ma olen nagu olnud piirpinnal kõndija.

\question{Vot see ongi väga-väga põnev, sest seltskonna seest tunduvad mõned 
asjad ilmselgeid ja mõned asjad ebaselged ning külje pealt vaadates võivad 
asjad  paremini paista. Aga kuidas siis juhtus nii, et just oli  
bakterigeneetika ja siis ühtäkki on telekom?}

See oli imelihtne. Kui ma lõpetasin ülikooli, siis täpselt samal päeval tuli 
Eesti kroon\sidenote{20. juunil 1992.} ja, ütleme, ülikoolide pakutud tulutase 
(ma jäin pärast lõpetamist ülikooli tööle) oli selline, et sellega ei olnud 
võimalik lasteaia tasusidki katta, see oli nii väike. Ja ma sain aru, et selles 
uues Eestis läheb kas väga kaua aega, kuni see valdkond hakkab ära tasuma või 
siis tuleb minna Eestist ära. Hästi paljud minu kursusekaaslased läksid Eestist 
ära ja neil on ka kõigil olnud väga edukas karjäär. Paljud on täna tulnud 
tagasi, paljud on professorid Tartu Ülikoolis ja EPAs. Sellepärast et nemad 
läksid ja ehitasid oma karjääri mujal üles ja kui nüüd Euroopa Liit asus 
laienema, asus meie teadustaristut üles ehitama, siis neil oli super võimalus 
30.-te keskpaigas naasta Eestisse ja kujundada täitsa oma näo järgi neid 
laboreid ja uurimiskeskusi. Super, tõsine võitjate põlvkond selles mõttes ka. 
Aga mina ei tahtnud Eestist ära minna, sest olid väiksed lapsed, ma ikkagi 
tahtsin, et nad oleksid eestlased ja  läksin ära teadusest. Läksin  puhtalt 
raha pärast erasektorisse. Pisike ettevõte, tegeles Siemens'i 
telekommunikatsioonijaamade paigaldamisega. Algne põhjus, miks mind tööle 
võeti, oli tõlkima materjale eesti keelde. Ega siis korralik inglise keele 
oskus ei olnud tollel ajal levinud. Aga siis selgus, et tõenäoliselt kõlban ma 
päris hästi ka müüma ja, noh, väikestel ettevõtetel on üks juht, eksju, ja siis 
sa oled nii müügidirektor kui lihtsalt direktor. 

\question{Mis aastal see oli?}
1994 sügisel läksin ma sinna. 

\question{Lihtsalt tausta mõttes. 1994 ikkagi oli suhteliselt hull aeg veel, 
kes tol ajal üldse Siemensi jaama endale paigaldas? Isegi analoogtelefon oli 
mõnes kohas haruldane?}

Metsikult pandi. See oli just see aeg, kus saadi aru, et büroohoonetes, 
ülikoolides, raamatukogudes, et kõigil on vaja, et telefon oleks toas. 
Vastupidi, mul oli mulje, et Siemens ühelt poolt, Ericsson teiselt poolt ja 
võib-olla väiksemate tegijatena  väiksemates ettevõtetes Panasonic ja teised, 
et me müüsime terve selle linna täis alates Rahvusraamatukogust lõpetades 
kõikide pankadega. Vana Reti kvartal, sellest sai büroone ja\ldots mu arust oli 
turgu nagu kõvasti! Tuli Eesti esimene riigihangete seadus, ma mäletan, et 
selle järgi tuli teha pakkumine. Eee tundus tohutu põnev, aga ka pisut 
hirmutav, sest varem ei olnud  nagu niimoodi käinud, et teed pakkumise ja siis 
need loetakse kõik ette. Mul on meeles, et istusime vist Tallinna Vangla  
telefonikeskjaama hankel ja selline tunne oli, et ega ei teagi, kas välja enam 
saab. Mitte, et me oleks tahtnud midagi valesti teha või teinud midagi valesti, 
aga lihtsalt maailm nagu muutus ja süsteemi tekkis nagu selgroogu struktuuri 
juurde.

\question{Te jõudsite pärast ka Telekomi, aga mina tollest ajast mäletan, et 
Eesti Telekom oli mingisugune õudne monstrum!}

See polnudki siis Eesti Telekom, see oli Eesti Telefon\index{Eesti Telefon} ja 
mina töötasin sellises peenes kohas nagu Eesti Telefoni äriklienditalitus. Ja 
tulin täitsa erasektorist, läksin sinna lihtsalt sellepärast, et seal tundus, 
et on rohkem karjäärivõimalusi. Väikses ettevõttes ma olin tipus, tundsin, et 
tahaks võib-olla edasi liikuda, suuremat struktuuri vaadata. Eesti Telefonis 
oli tollel ajal veel päris keeruline. Ma mäletan, et ükskord oli selline 
olukord, et mulle öeldi, et see kuu ei saa rohkem müüa, sest meil sai 
sisseostuplaan täis. Siis ma olin jube kuri ja tegin üheselt selgeks, et ma ei 
taha mitte kunagi enam sellist väidet kuulda. Et kui me müüme, siis me müüme ja 
kui te ei taha, siis ma lähen, teen midagi muud. Aga müüsime küll.

\question{Eesti Telefon oli tol ajal niisugustele minusugustele nohikutele 
sihuke üsna õudne ettevõtte, küll ta ei suutnud traati pakkuda, ta ei müünud 
isegi internetti mingisugusel hetkel ja oli väide, et telefoniliini peal ei 
peagi internet töötama, et too on helistamiseks}

Ma ei tea tolle äriklienditalituse loojate kaalutlusi, aga ma arvan, et asja 
mõte oligi tuua sinna struktuuri üks üksus, mis hakkaks seestpoolt õõnestama 
sellepärast et, see rahvas pidi müüma täitsa tavalistele eraettevõtetele ja 
müüs neid samu jaamu, mida Siemens ja Ericsson eraldi, me müüsime ka Siemensit 
ja Ericssoni. See pidigi seda organisatsioonikultuuri nagu vähehaaval muutma 
hakkama. Ma täpselt ei mäleta, kuidas Valdo Kalm\index[ppl]{Kalm, Valdo} selle 
talitusega seotud oli,  aga igal juhul kuidagi ta oli ja me ju teame, et ta 
selle ettevõtte muutumisele kindlasti aitas hästi palju kaasa. 

Aga algus oligi minu jaoks keeruline. Mõni inimene küsis, et miks sind kunagi 
oma laua taga ei ole. Minu arust müügijuht ei peagi olema oma laua taga. Ta 
oligi väga teistmoodi, see kultuurimuutus seal kuskil tekkiski, jah.

\question{Too oli üldse kultuuri mõttes huvitav aeg, kui räägime sidest, et 
seal kuidagi äri, akadeemiline maailm ja sihuke häkkerite ning nohikute maailm 
kombineerusid. Kas see paistis Telegoni poole pealt ka välja?}

Muidugi paistis, sest need inimesed olid ju needsamad. Võtame või Taavi 
Talvik\index[ppl]{Talvik, Taavi}, minu esimene abikaasa. Tuli ta sealt Tartu 
poole pealt, Ülikooli Arvutuskeskusest\index{Tartu Ülikool!Arvutuskeskus} läbi 
Valitsusside\index{Valitsusside}. Siis nad tegid oma ettevõte, Andes 
Baumaniga\index[ppl]{Bauman, Andres} Unineti. Uninetist sai hiljem, müüdi ära, 
Elisa. Selles mõttes nad see oligi nagu üks maailm, me näinudki seda nagu 
eraldi. Mis seal vahet on lõppude lõpuks, kas sa helistad või, või  saadad muid 
andmeühikuid. Digitaalne tehnoloogia tollel ajal just tuli. Tekkisid need 
probleemid, et kuidas tagada läbilaskvus, ühenduste laiused, kõik see maailm 
hakkas vaikselt arenema ja kasvama. Et ta ei olnudki eraldi, minu meelest ta 
pole Eestis kunagi eraldi olnud. Kandja pool ei olnud kindlasti eraldi ja sisu 
poolest\ldots Tollel ajal sisu poole ettevõtteid ju ei olnud veel. Mingi sisu 
oli, esimene internetipank oli 1994 vist? 1997 oli juba ju  e-maksuamet.

Sisuteenused hakkasid ka ikka üsna kiiresti tulema, aga siis hakkas kohe see 
võistlus alati, et toru on, aga sisu tahab laiemat. Toru laiemaks, sisu tahab 
veel laiemat. Eks sealt hakkas see tulema. Aga jah, ma olin selgelt toru poole 
peale, ma ei olnud kunagi sisu poole peal.


\question{Vahetaks nüüd korra käiku. Rääkisime, et stereotüüpe ei ole. Aga 
ometigi on Eesti Vabariigis olnud laialt kuaibel niisugune mõiste, nagu 
\enquote{patsiga poiss}.  Mis inimene see on? Mis teda iseloomustab?}

No neid on väga erinevaid. Eesti Telekomi aegadest ma ei mäleta peaaegu kedagi 
peale Valdo Kalmu\index[ppl]{Kalm, Valdo}, kes seal minuga veel koos töötasid, 
palun vabandust kolleegide ees! Aga ma mäletan, et näiteks Uku 
Kuut\index[ppl]{Kuut, Uku} istus meil, oli süsadmin, eks ole, patsiga poiss. Ja 
kui sa tema tuppa läkside, sest midagi oli paigast ära siis muusika alati käis. 
Tundus küll nagu teine maailm nagu võrreldes paljude teistega. Aga kindlasti 
oli ka pöetud habemega rahvast.  

Need olidki üsna erinevad seltskonnad. Kui sa müüd telefonijaama ja  tegeled 
pankadega, siis oli selgelt näha, et pankade tehnikajuhid vastutasid juba 
tollel hetkel suhteliselt suure struktuuri püsti hoidmise ja edasi arendamise 
eest, olid hästi makstud ja erinenud millegi poolest pankade, ma ei, tea 
raamatupidajatest. Neid ei kutsutud siis veel CTO-deks, aga seda nad sisuliselt 
olid ja ei erinenud millegi poolest muude valdkondade eest vastutajatest. 

Ja siis olid need, kes olid kuidagipidi (ega seda ülikoolides ei õpetatud ka) 
ise arvuteid pidi ringi nuhkinud ja saavutanud oskuse hoida asjad töös. 
Sellised iseõppijad vennad. Nende hulgas jah võib olla oli seda stereotüüpi, et 
suhtleb parema meelega masinaga. Aga masin ei olnud tollel ajal nii huvitav 
suhtluspartner, ma arvan, et tollel ajal võimalust internetti päris ära kaduda 
ei olnud. Ma arvan, et see on üle võimendunud kombo, sellest, mida me täna 
teame, kuidas sa võid kogu oma ööpäeva ära sisustada seal kastis. Mina ei tea, 
minu arust oli neid ikkagi igasuguseid.

\question{Äkki võib-olla siis see, et enne oli juttu et mingisuguse peakujuga 
inimesi tõmbas Tartu Ülikooli arvutuskeskusse,   võib-olla see tõmme on see 
ühine nimetaja?}

No kindlasti, jah. Kui ma mõtlen, et need, kes siin Tallinnas 
Küberis\index{Küber} koos käisid, on kõik selles sektoris tänini  leitavad, kes 
oli püsivam ja järjepidevam, kui mina. Kindlasti midagi ju on, mis meid tõmbab 
matemaatika juurde ja midagi, mis tõmbab meid keelte juurde. Need on erinevad 
asjad. Võib-olla Arvid Tavast\index[ppl]{Avast, Tarvi}, praegune 
Keeleinstituudi\index{Eesti Keele Instituut} direktor on selline mõlemal poolel 
kõndija. Ühtpidi on ta olnud IT-tegija, teistpidi on teda sügavalt huvitanud 
keeled ja need kaks asja tänases maailmas saavad  kokku. Aga enamus inimesi 
kipub olema, jah, paremal või vasakul. Ma ei tea, miks.

\enquote{Sellel teemal veel küsides. Minu käest just küsiti, ja ma ei osanud 
vastata, mul lihtsalt ei olnud vastust. Ja ma tean, et Teie olete kindlasti 
selle peale palju mõelnud, et äkki Teil on mingisugune. Miks just patsiga 
poiss? Kui ma vaatan  inimesi, kellega ma räägin sellest ka selles sarjas, siis 
naisetrahvaid on vähe. Tol ajal oligi neid vähe selles seltskonnas. Miks?}

See on selle pärast, et naised toimetavad valdavalt sektorites, mis on 
sissetöötatud. On alalhoidlikumad. Paratamatult ka karjääri mõttes on nad  
alalhoidlikumad. Isegi näiteks. Mul oli ka mingil hetkel valik, mul oli päris 
palju ideid, et mida võiks teha. Et kas minna edasi, tekitada mingi oma butiik, 
hakata seda arendama. Ja ma ei teinud seda sellel lihtsal põhjusel, et mul oli 
vaja üleval pidada kaks alaealist last. Viia neid iga päev lasteaeda, tuua 
sealt ära, ma olin tollel hetkel üksikema. Ja ma jätsin selle tegemata 
teadlikult, sest mul oli tarvis rohkem kindlustunnet ja struktureeritud elu. Ma 
ei saanud endale lubada, et ma võib-olla järgmine viis kuud palka ei saa. Mul 
ei olnud seda võimalust lihtsalt teha. Ja ma arvan, et kuna see oli tollel 
hetkel nagu selline uus valdkond, siis naised lihtsalt ei võtnud neid riske 
üles. Ma mõtlen, üksikemad on kohe väljas, nagu mina, eks ole. Ja võib-olla see 
seiklusjanu oli võib-olla esialgu nagu väiksem ja ega keegi ei näinud ju 
sellega ka teadlikult vaeva. Tollel ajal, me räägime sügavast 90.-test! Niteks 
ehe näide. Mina müüsin siin Siemensi Hicom 300-t\sidenote{Siemensi paindlik 
telefonijaamade sari.}, Soomes oli üks Tiina, tema müüs ka. Ja siis me läksime 
Siemensi suurte telefonikeskjaamade müügimeeste kokkutulekule. Seal olid ainult 
mehed peale meie ja kõik käisid ja küsisid sellise uskumatu ilmega, et 
\enquote{kas tõesti nüüd neid suuri telefonijaamu?} Me ei saanud nagu aru, miks 
me ei võiks seda teha. Aga see ei olnud, see ei olnud tol ajal mingi naiste 
maailm. 

\enquote{Vot, see on nüüd huvitav perspektiiv. Et  Eestis sihuke avantürism, 
eks ole, on nagu arusaadav, aga mujal maailmas oli maailm selles osas ikkagi 
nagu päris teistsugune!}

Ei no kui me endale ette kujutame, ma ei tea, Kesk-Euroopa, Saksamaa. Saksamaal 
ikkagi minu põlvkonnas on veel päris palju koduperenaisi, just Lääne-Saksas. 
Idas ei ole. Kui sa praegu vaatad, siis statistiliselt Ida-Saksa naiste 
pensionid on kõrgemad,  lahutuste arvud on ka palju kõrgemad, sest nad saavad 
seda endale lihtsalt lubada, Käänes ei saa. Kui me ette kujutame endale, et 
üheksakümnendad oli tohutu võrdõiguslikkus ärikultuuris, siis ma kaldun arvama, 
et see pole sinna päriselt jõudnudki. Ja selle nimel võideldaksegi, et seda 
seal ei olnud ja ikka veel ei ole kohal. Meil ei maksa endale illusioone luua, 
et seda tööd ei pea enam tegema. Selles mõttes on Taavi Kotka\index[ppl]{Kotka, 
Taavi} Unicorn Squand, Rakett ja teised sellised ettevõtmised tüdrukute 
toomiseks tehnoloogia ligi ühiskonnale tohutu väärtusega. Sest tegelikult ei 
ole mitte ühtegi põhjust, miks tütarlaps ei saa toimetada tehnoloogiarikastel 
aladel, eks ole.

\question{Aga tütarlapsel võib olla nagu kuidagi vähem loomulik see, et ta 
lihtsalt magab kontorilaua all, sest tal tuli tol hetkel uni peale?}

Ära hakka seda \enquote{loomulik või vähem loomulik}!  Ei ole niimoodi! Seda 
võib-olla paljud arvavad kuidagi, eks ju. Mis mõttes!? Mis seal vahet on? Sul 
on 20-aastane vaba inimene, lapsi ega peret ei ole, on ta mees või naine. Kui 
ta seal kontorilaua all magab, teksad on nagunii jalas, ega sa soengu ja 
seelikuga sinna ei lähe ju. Mõtle nüüd, see on ju täitsa absurdne väide 
tegelikult! Mina arvan, et see ongi see stereotüüp, see ei ole pahatahtlik ja 
see on alateadlik tihtipeale. Aga see on täiesti olemas, nagu ka selles sinu 
väites!

\question{Tõsi, sest punkareid oli igasuguseid!}

Jah, ja samamoodi võib-olla insenere ja kõike muud.

\question{Lihtsalt konkreetselt tuleb üks habemega punkar silma ette, kes 
vedeles niimoodi hommikul laua all. Aga tõepoolest, et see on minu stereotüüp, 
see on minu kujutluspilt et see on see habemega punkar, miks ta ei võiks olla 
teistsugune punkar!}

Mul näiteks tütar rääkis ülikooliajast ühe loo. Üks ettevõte otsis tööjõudu, 
päris paljud käisid ennast pakkumas ja üks poiss võeti. Ta oli ülikoolis nagu 
silmnähtavalt teistest laisem ja kehvema õppeedukusega ja sel poisil tekkis 
endal küsimus oma tööandjale. Ta mõne aja pärast, kui ta oli kohanenud, noh, 
võeti õlut, eks ole, küsis, et \enquote{kuule, meilt kandideerisid veel need ja 
need inimesed, miks nemad ei saanud?}. Mille peale talle öeldi, et 
\enquote{vaata ümberringi, näed sa siin mõnda naist?}. Selline tõrjuv kultuur, 
eks ole. See ei ole nüüd side, ega telekomiettevõte näide, ega ka IT-ettevõtte 
näide, aga lihtsalt insenerikultuuri näide sulle. Eestist! Ja need inimesed, 
kellest ma täna räägin, on praegu 32, 33 ei ole 50!

\question{Mugavam on palgata omasugust, kas see kasulikum on, on iseküsimus.}

Ei ole sellepärast, et statistiliselt tuleb naiste pähe 50 protsenti headest 
ideedest ja meeste pähe 50 protsenti ja kui sa nüüd kasutad ära vähem.

\question{Pigem on isegi nagu teistpidi\ldots}

Ma ei lähe ka kunagi sinna. Et ma näiteks üldse ei arva, et me peame ka ütlema, 
et naised ongi kuidagi teistmoodi näiteks kas juhid või insenerid. Ma ei taha 
sinna kunagi minna! Ma väidan, et me oleme jumala võrdsed nagu oma 
ajupotentsiaali mõttes. Aga millega sa ikka puutud kokku, et minu käest on nagu 
ka küsitud, et \enquote{kuidas siis nüüd nii et naine juhib siin elektrijaama?} 
No siis ma ikka väga otsekoheselt olengi nagu küsinud, et kuule, räägi nüüd, 
mida sa selle tilliga teed jaama juhtides? On sul mingi idee? Kus see vahe veel 
on? Ei ole!

\question{Kuna kõik see meie IT-värk tuleb sellest seltskonnast ja see 
seltskond oli faktiliselt ühele poole kallutatud, siis kas see on meid kuidagi 
digiühiskonnana tagasi hoidnud või aidanud või mingit mõju avaldanud? Kas võib 
midagi välja tuua?}

Ma arvan, et tegelikult ei. Sest et niipea kui sa tood mängu riigi ja ma olen 
hästi palju mõelnud selle meie digiriigi arengu peale. Miks me oleme nii 
teistmoodi, kuigi mujal on palju tugevama IT-sektoriga erasektor, kui meil 
Eestis on. Aga, vaata, ühiskonda muudab ikkagi riik. Midagi ei ole teha. 
Erasektoris võidakse teha geniaalseid asju, aga \emph{mainstreami}-mise 
maailmameistrid on kõik riigid. Ehk et see on see koht, kus riik, kui ta midagi 
tegema hakkab, siis ta peab kõik kaasa võtma. Vanad, noored, mehed, naised! Ja 
vot selle muudatuse on Eesti ära teinud.  Priit Alamäe\index[ppl]{Alamäe, 
Priit} vist tõi termini \emph{digitally transformed nation} kasutusele. Siin on 
nüüd see koht, kus me läksime teist teed, kui teised maailma riigid. Et kui 
riik hakkab huvi tundma mingisuguse sektori võimaluste kasutamiseks 
riigiteenuste osutamiseks, siis see hakkab reaalselt päriselt ühiskonda muutma 
ja kujundama ja see juhtus meil. Ja sellepärast on meil ühest hetkest kaasas 
mehed, naised, lapsed, vanurid,  ja tekivad ka positiivsed kõrvalefektid. Et, 
vaata, kui nüüd corona tuli kallale, siis meil ei olnud ju häda, et 
seitsmekümneaastased inimesed ei saa pangas käidud, ei saa, ei saa 
telekomilepinguid uuendatud, sest nad olid 50, kui ID-kaart tuli. Et need 
positiivsed kõrvalefektid kogu ühiskonnale on hästi suured olnud. 

\question{Aga miks see meil juhtus niimoodi?}

Sellepärast, et meil ei olnud midagi. Ajalugu on ju selline, et esimene 
e-teenus oli maksuamet, on ju. Kuidas sa kujutad ette, et Eesti inimesed oleks 
nõustunud seisma tundide pikkustes sabades, et riigile oma maksud ära viia?

\question{Praegu enam ei kujuta.}

Aga siis ka ei kujutanud. Oli tõenäone, et maksulaekumised ei ole ülearu head, 
kui sa loodad nagu sellisele asjale. Ja õnneks ikkagi ajastu oli selline, et 
e-pangandus ju oli, sai teha e-maksuameti. Keegi ei taha ju maksuametniku näha! 
 Ja eestlased viisid selle unistuse ellu, nagu see vana hea Ameerika vanasõna.

\question{Vot mul tekkis kohe küsimus. Ma ei ole kuskil näinud sellist usaldust 
sellise keskmise bürokraadi ja \emph{hardcore} inseneri vahel. Ühest küljest 
insener usaldab, et see bürokraat ei keera asja kihva. Ja teisest küljest 
bürokraat ikkagi usaldab, et kui see tehnik hakkab rääkima XML sõnumite 
vahetamisest, et ta päris udu ei aja ja tulemuse ära tarnib. Kust meil see 
tuleb? Sest see IT-kogukond isekeskis küll teadis, tundis, usaldas üksteist. 
Aga kuidas laiem ühiskond sinna juurde tuli?}

No, ega ta ei tulnudki. Ma mäletan, et kui ma töötasin Laari\index[ppl]{Laar, 
Mart} juures, peaministri juures, 1999 me sinna läksime, siis tema tellimus oli 
 suhteliselt mittespetsiifiline. Ta võttis oma nõunikud kokku ja ütles, et nüüd 
on nihuke asi, et me kõik saame aru, et Eestis palgad kasvavad, varsti me ei 
ole enam nagu rikka mehe kuluefektiivsuse lahendus. Et kuhu edasi, vaadake 
ringi. Sealt tulid need asjad, et kui minu juurde tuli Andres 
Metspalu\index[ppl]{Metspalu, Andres}, ütles, et oleks vaja teha 
Geenivaramu\index{Geenivaramu}, siis sai kõik rattad käima lükatud. Eiki 
Nestor\index[ppl]{Nestor, Eiki} tuli ka appi, tegime geeniseaduse, Geenivaramu 
seaduse, sai nagu käima lükatud. Teine valdkond oligi see, mida siis Linnar 
Viik\index[ppl]{Viik, Linnar} välja pakkus. Minu arvates tollel ajal ikka juba 
hakkas tekkima ka sisuteenust, Kaarel Tarand\index[ppl]{Tarand, Kaarel} nägi 
väga hästi ka seda pilti, kuhu see sisu pool, kommunikatsioonis ja nii edasi, 
minemas on. Ja sealt siis tuli ID-kaardi idee, mis vist tekkis ka sellest, et 
ega siis ju pangad ei olnud pikaajaliselt nõus võtma vastutust, et riik oma 
e-teenuseid nende platvormidelt jooksutab. Mistõttu tuli teha ID-kaart. Aga see 
sündis ühise veenmistöö tulemusel. 

Jube raske oli, ma mäletan, Rahandusministeeriumi ära veenda, kes tahtis kohe 
teada, kus on nagu ROI\sidenote{\emph{Return On Investment} --- Investeeringu 
tasuvuse näitaja.}, eks ole. Mis täna tundub naljakas küsimus: \enquote{Mis 
mõttes, vaadake, milline e-riik meil on!} Aga kust siis Linnar\index[ppl]{Viik, 
Linnar} tollel ajal oleks need arvud võtnud? See põhjendus oli ju absurdne, 
millega tegelikult ID-kaarti valitsusele müüdi: e-valitsus, et valitsuses olid 
arvutid laua peal, oli toonud meile nii palju tasuta artikleid 
välisajakirjanduses, et see süsteem oli ennast kolme kuuga tasa teeninud 
võrreldes sellega, kui me oleks lihtsalt ostnud \emph{Estonia --- Positively 
Transforming}\sidenote{Tegu oli 2002. aastal käivitatud suure ja mitmesugust 
meediatähelepanu pälvinud \emph{Brand Estonia} kontseptsiooni \enquote{Welcome 
to Estonia} põhilise tunnuslausega. } lehepinda, eks ju. Selle argumendiga 
tehti ID-kaart! Ehk et see arvamus, et kõik tulid  Linnar 
Viigi\index[ppl]{Viik, Linnar} ja teistega kohe kaasa, see on jumala vale. Need 
olid teised teised argumendid: kulu ei olnud nii suur, ja  tõepoolest see 
Vabariigi Valitsuse ruum oli väga palju välismaist positiivset tähelepanu 
saanud. Teenimatut, seejuures, sest needsamad inimesed, kes tulid, Saksast ja 
Soomest, nende erafirmades oli ju täiesti tavaline, et et sul oli intranet. 
Eestis ka juba ammu selleks ajaks. Esimene Eesti intranet hakkas tööle, ma 
arvan, see oli Postimehe\index{Postimees} toimetuses 1991 või 1992, midagi 
sellist. Aga vat riigid ei teinud, ja see  siis oligi müügiargument. ID-kaardi 
müügiargument oli sirgelt see, et me võime saada palju välismaist tähelepanu.

\question{Mis ilmselt oli ära tuntav asi: korra oleme juba saanud, et küllap 
siis nüüd ka!}

Just. Ja geniaalsus oli selles asjas see, et, vaata, Saksamaal sa pead 
siiamaani omale taotlema, et sul oleks digi-ID.  Aga meie pistsime selle 
lihtsalt kõikidele kaardi peale. Kasutavad või kasuta, aga ega ta liiga ka ei 
tee. Et meil on ainult ühte tüüpi ID-kaart. Kiibiga. Ja see oli geniaalselt 
õige lahendus.

\question{See on see asi, mida mitte keegi minu teada väga ei ole järgi teinud.}

Nüüd vist ikka juba tuleb, aga 20 aastat, püha Jumal\ldots

\question{Mul on ikka olnud see küsimus, et kui palju oli arusaamist, et 
tegemist on geniaalse lükkega, palju oli sellist pikka visiooni ja kui palju 
oli lihtsalt praktilist kaalutlust, et \enquote{paneme lihtsalt käima}?}

Oli palju, sest ma mäletan seda arutelu. Et äkki ei teeks. Äkki teeks. Teeks, 
teeme kõigile. Maksab nii palju. Kui me teeme osadele, kas maksab vähem? Ei 
maksa tegelikult vähem, tegelikult  rohkem, kui sul on erinevaid süsteeme. Ja 
oli tõesti üks hulk inimesi, Infotehnoloogiafirmade 
Liit\index{Infotehnoloogiafirmade Liit} oli seal kindlasti pundis, Linnar 
Viik\index[ppl]{Viik, Linnar} oli seal kindlasti pundis, ilmselt oli neid  
veel, kes ütlesid, et anname kõigile. Et võtab aega, kuni need teenused peale 
lähevad, aga anname kõigile. Ja analoog oli see, kuidas me olime mingil hetkel 
otsustanud, et mitte keegi ei saa sularahas palka siin riigis, et kõik said 
omale pangaarved. Ja pidid tegema ja raha hakkas minema panka. Kuigi siis ju ka 
esimesed kuud, võib-olla isegi esimesed aastad, kui oli palgapäev, siis ATM-i 
juures oli saba, inimesed võtsid kogu raha välja. Ehk et ta ei hakanud 
tegelikult seda funktsionaalsust tarvitama kohe. Seda analoogi, ma mäletan, me 
hästi palju oma  aruteludes tõime ka, et selle pärast peaks ID-kaardi tegema 
ära universaalsena. Aga see oli Mart Laari\index[ppl]{Laar, Mart} suur nagu 
selline lükkamine ja lükata tuli tihtipeale ka just nimelt vastu neid, kes 
hästi hoolega raha lugesid. Reformierakonda, tegelikult.

\question{Ja see lükkamine käis tal põhimõtteliselt sellesama arusaama pealt, 
et see on strateegiliselt oluline asi?}

Jah, tema uskus seda. Linnar\index[ppl]{Viik, Linnar} oli suutnud ta seda 
uskuma panna. Ja mina pidin selles protsessis olema kaasas sellepärast, et ma 
olin majandusnõunik. Minule ütles rahandusminister, et unustage ära, sest te ei 
suuda mulle ROI-d näidata. Siis me mõtlesimegi välja selle, et  \enquote{aga 
see asi tasus ennast ju kolme kuuga ära!}. Mina tulin selle nurga pealt nagu 
selle asjaga kaasa,  geeniseadus oli nagu rohkem minu laps seal peaministri 
büroos.

\question{Järelikult siis legend sellest, kuidas olla kord 
Linnar\index[ppl]{Viik, Linnar} ja Mart\index[ppl]{Laar, Mart} istunud laua 
taga ja Mart oli öelnud, et \enquote{Linnar, mis me teeme?} ja Linnar oli 
öelnud, et \enquote{teeme interneti}, see vastab siis tõele!}

Vastab tõele küll, aga Mardil ei olnud nii kitsas vaade. Ta tahtis lihtsalt 
teada, et öelge kõik midagi, mis me nagu võiksime teha.

\question{Huvitav, selline kombinatsioon praktikast ja visioonist. Tolle aja 
Eesti vabariik oli ikkagi oluliselt teistsugune, kui ta on praegu, muresid oli 
miljon!}

Ja oligi. Ma jään selle juurde ka, et me ei saa seda au kõike endale võtta. 
1999 saime ju Euroopa Liidu teadus-arendusprogrammi liikmeks. Me saime 
tegelikult sealt rahasid peale. Siis hakkas üldse Euroopa liit, meid ette 
valmistama liitumiseks, meil tulid institutsioonide ehitamiseks rahad peale, 
meile tulid igasuguste asjade ehitamiseks rahad peale. Ehk et mina väidan, et 
digiriiki ehitada, kui keegi teine maksab su koolide, teede ja muude asjade 
remondi kinni, on oluliselt lihtsam. Euroopa Liidu rolli ei tohi alahinnata. 
Raha hakkas siin riigis liikuma palju rohkem ja tänu sellele oli ka võimalik 
mingi kõõl sellest digiteenuste arengusse nagu suunata.

\question{Marek Tiitsu\index[ppl]{Tiits, Marek} igasuguseid 
IBS-id\index{Institute of Baltic Studies} ja asjad olid ju kõik mingisuguste 
väliste fondide rahadega.}

Ja kui teised nägid, et mis me teeme, siis üsna kiiresti tekkis laboriefekt. Et 
aitame, toetame, me saame ise ka nagu näpud vahel, et mis nad seal teevad. Aga 
kindlasti ei oleks meist saanud sellist e-riiki, kui meil ei oleks avanenud  
võimalust saada suures koguses välisabi. Vähehaaval oma toonasest SKT-st (me 
elasime tegelikult ju tollel ajal Maailmapanga ikkagi nagu vaeste riikide 
kriteeriumite järgi) me päris ei oleks seda asja teinud. Kuigi, mööngem, et 
täna ei saa sa ilmselt ühte korralikku e-maksuametit selle raha eest, mis me 10 
esimest aastat oma e-riiki tegime.

\question{Kuidas see muutus käis nii-öelda operatiivtasandil?  Minu käest 
näiteks Prantsuse ametnikud on küsinud, et, arusaadav, tegite e-riigi, aga 
kuidas te selle ametnikele ära seletasite?}

Jaa, seda küsis minu käest, mäletan, Prantsuse peaminister Eduard Philipp, 
Macroni esimene peaminister, oli otsustanud, et nüüd tuleb Prantsusmaal ka teha 
digipööre. Küsis, et kuidas nüüd siis selle asjaga on, et mis nendest 
töökohtadest avalikus sektoris sai. Siis ma ütlesin talle, et vaata, Eduard, 
nüüd on nii, et meil läks ka maksuametis 60 protsenti töökohti kaotsi, aga meil 
ei olnudki üldse nii korraliku maksuametit nagu teil on. Aga ära selle pärast 
muretse, te ju teete tööturu liberaliseerimise reformid ka ära, eks ju! Mida 
nad ongi teinud, eks ole, ehkki muidugi mitte määrani, mida meie peame igal 
juhul normiks. Vastab tõele jah, et Eestis kadus ka töökohti, aga, see on nüüd 
teisest valdkonnast, aga kui Hoiu- ja Hansapank liitusid\sidenote{See toimus 
jaanuaris 1998.}, said Eesti ettevõtted omale finantsjuhid. Miks? Sellepärast, 
et pangandusest sisuliselt pooled spetsialistid, kes suutsid nagu arvutada, 
jäid lihtsalt üle ja ettevõtetel tegelikult oligi vaja neid spetsialiste kasvõi 
selleks, et nendesamade pankadega läbi rääkida. Aga polnud kuskilt võtta ja kui 
need kaks panka ühinesid, siis korraks tekkis sellest valdkonnas tööjõu ülejääk 
ja, hopsti! Ja täpselt samamoodi vajab ju erasektor selleks, et e-riiki 
ehitada, kogu aeg inimesi, kes teavad, kuidas riigis protsessid käivad ja minu 
arust on nad kogu aeg ise ära söönud sellesama tööjõu, mille nad on avalikus 
sektoris hävitanud.

\question{Hüve hüveks, aga miks ametnikud vastu ei hakanud  töötama}

Aga sellepärast ei hakanudki, et ega siis eestvedajad, need, kes võatavad, 
juhtrolli, ei jää ju kunagi ilma tööta. Kindlasti oli kuskil ka neid, kes 
kannatasid ja kelle töökohad kadusidki, kes olid õnnetud. Ega eestvedajatel ei 
ole mingit muret selle asja pärast, sa õpid ise selles protsessis nii palju. Ja 
väga paljud hüppasid teise paati, läksid koos erasektoriga, osteti üle 
tegelikult sellepärast, et  pakkuda riigile seda teenust tagasi.

\question{Sest meil ei olnud tol ajal see aparaat veel kivistunud, sa ei saanud 
olla olnud 15 aastat ametis, sest vabariiki polnud nii kaua olnud}

Ja igal pool noored ju tegelikult hästi palju muutusid meie riigi näoks. Ma 
mäletan, et ükskord üks Leedu erastamisagentuuri juht ütles mulle (Laariga 
olime Leedus visiidil), et te eestlased teete hästi julgeid asju, sellepärast, 
et te olete kõik nii noored, te ei taju üldse, mis hirmsad riskid selle kõigega 
kaasnevad. Selles kindlasti oli nagu mingi oma iva. Panganduses ka. Kunagi üks 
ungarlane tuli siia, Hoiu-Hansa ühinemine oli just olnud. Näitasin talle seda 
ülemist korrust, mis oli täis selliseid oma nappi kolmekümmet eluaastat 
prilliraamide taha varjata püüdvaid tüüpe. Ja ta küsis mu käest, et 
\enquote{Ütle, Kersti, mis te vanemate inimestega tegite?}. Aga selle hind on 
see, et meie põlvkond pidaski üleval oma vanemad ja kasvatas oma lapsed ja 
tõenäoliselt jääb sinna meil mingi keskmise eluea osas mingi negatiivne hüppe 
sisse.

\question{Panga seltskond oli tol ajal jah selline nooruslik. Ja kui me vaatame 
Laari, siis ta tänapäeva mõistes ajalooõpetaja hariduse pealt ta istus maha ja 
lihtsalt tegi maksureformi!}

Mardil oli meeletu usaldus oma nõunike vastu. Ma ei mäleta seda, 1992-1994 
perioodi\sidenote{Mart Laar\index[ppl]{Laar, Mart} oli peaminister 1992-1994 ja 
1999-2002.}, mina siis seal ei töötanud. Aga ta lasi Ardo 
Hanssonil\index[ppl]{Hansson, Ardo} ilmselgelt otsustada ja möllata nii, nagu 
ta lasi hiljem Linnar Viigil\index[ppl]{Viik, Linnar}, Kaarel 
Tarandil\index[ppl]{Tarand, Kaarel} või Simmu Tiigil\index[ppl]{Tiik, Simmu} 
või meil teistel möllata. Me saime ikka jumala vabad käed ja ta oli meie selja 
taga. Ja sa ütlesid, et \enquote{kuule, me nüüd tahaks sellise asja ära 
teha}\ldots Vaata, tollel ajal oli igal ministeeriumil oma pank. Täna on meil 
ainult üks EAS-i ja KredExi, et mu arust on  seegi risuks jalus, aga hea küll. 
Tollel ajal oli Siseministeeriumis umbes kaks sellist fondi, mis andsid raha 
välja. Ja siis ma ütlesin Mardile, et see on tegelikult jube ebaefektiivne ja 
peale selle üleüldse milleks need pangad välja on mõeldud võib olla see raha 
seal võib-olla ei kulu kõige efektiivsemalt. Mart ütles kohe, \enquote{Tee ära, 
koristage ära need asjad}. Pärast hakkas eurorahasid liikuma, siis oli seda 
EAS-i nagu ka vaja. Minu magistritöö teema tegelikult oli riigi poolt asutatud 
sihtasutuste juhtimine ja see oli just nimelt seotud selle koondamise ja selle 
tegevusega. Mart lasi teha ja ütles, et andke tuld. Aga ministrid pidid ikka 
ise ära veenma, seda ta ei hakanud sinu eest tegema. Tehke, aga siis edasi oli 
juba sinu asi. Käisid ja veensid. Padarit ei veennud ära, jäigi Maaelu 
Arendamise Sihtasutus eraldi\sidenote{Ivari Padar\index[ppl]{Padar, Ivari} oli 
Mart Laari teises valitsuses 1999-2002 põllumajandusminister. Maaelu Edendamise 
Sihtasutus (MES) on tänaseni Maaeluministeeriumi valitsemisalla kuuluv 
sihtasutus.}.

\question{Slles mõttes väga huvitav, et see on täpselt selline kombinatsioon 
oma strateegilisest vaatest, aga sa lased ka toimetama suhteliselt 
mitte-poliitilise seltskonna, lihtsalt praktilised inimesed, kes saavad aru, 
mida on vaja teha.}

Meil oli eile Latitude-l\sidenote{Konverents Latitude59 toimus Kultuurikatlas 
19-20 mail 2022.} arutelu. Mu  vestluspartneriks oli prantslane ja ta küsis 
kogu aeg seda küsimust, et mida peab riik tegema selleks, et \ldots mida me 
Euroopas\ldots, et ka meil oleks need toredad digifirmad nagu ameeriklastel. Ja 
ma lõpuks ütlesin talle, et kuule, me oleme siin nüüd pool tundi rääkinud, 
kuidas täpselt Eesti riik ei sekku sellesse, millised majanduslikud valikud 
erasektoris tehakse, millised sektorid peavad arenema, me ole 
dirižiste\sidenote{Dirižism, prantsuskeelsest sõnast \emph{diriger}, suunama, 
on majanduslik doktriin, milles riik mängib tugevat suunavat rolli 
kapitalistliku majanduse suhtes.} meie hulgas ei leidu. Tulemus: 10 
\emph{unicorn}-i ühe miljoni kohta. Et ütleb see sulle midagi või ei ütle? 
Umbes selline peabki minu arvates olema see poliitika tegemise roll, et sa nagu 
võimaldad. Samamoodi meie maksusüsteem. Et ettevõtete tulumaksuvabastus, mida 
see tähendab? Igas investeeringus, rõhutan igas, mitte valdkondlikult valitud 
mingisuguses eelisarendatavas valdkonnas, on riik ju 20 protsendiga sees. Võtab 
riski nagu ettevõtja, võib-olla ei hakka sealt kunagi dividende tulema, mida 
saaks maksustada.

Kihvt riik on meil tegelikult!

\question{Jumal hoidku! Iga kord, kui neid jutte räägin või transkribeerin, 
lähen mõttes tagasi sellesse 90.-te aastate aega ja iga kord tunnen ennast 
hästi, et kui tore riik meil on!}

Oled sa vahest mõelnud, et aga kui me oleksime ennast valima valinud juhtima 
inimesed, kellel kogu riigi ja oma rahakott läheksid segamini. Meil, muide, ka 
praegu on poliitikas selliseid inimesi päris palju, kes kogu aeg jäävad vahele 
sellega, et nad on oma võimupositsiooni kasutanud enda või oma partei 
mingisuguseks hüvanguks. Väga vabalt oleks võinud niimoodi minna. Ja siis me 
oleks täna omadega Ukrainas. Täpselt seal.

\question{Väga paha oleks. Meil õnnestus valida mingi ime läbi mingisuguseid 
mõistlikud inimesed\ldots}

Tegelikult ei õnnestunud.  Lennarti\index[ppl]{Meri, Lennart}\sidenote{Lennart 
Meri, Eesti president aastatel 1992-2001.}  kõige kontroversiaalsem tegu 
põhiseaduse kontekstis oli see, et ta lasi Laaril moodustada valitsuse, kui 
Savisaar oli valimised võitnud. Aga kuna kolmikliit\sidenote{Kolmikliit oli 
kolme Eesti partei --- Eesti Reformierakonna, Isamaaliidu ja Mõõdukate --- 
valimisliit.} oli eel-moodustatud, siis ta läks sellele teele. Ja miks ta pidi 
minema, sest see kolmikliit oli ka nii habrastel alustel, et poleks üldse 
moodustunud, kui Edgar Savisaar\index[ppl]{Savisaar, Edgar} oleks saanud 
võimaluse mõni nendest potentsiaalsetest partneritest ära rääkida. Et nii 
habras oli see. Tegelikult oli see täpselt nii habras, et me oleksime võinud 
minna märksa konservatiivsemat majandusarengut, märksa lõdvemat eelarve 
poliitikat ja võib-olla ka märksa sellist oligarhsemat majandusmudelit pidi. 
Mõeldes, mida tollel ajal või ka täna Keskerakond endast kujutab, on ju nii?

\question{Ja tolles keskkonnas ilmselt siis see IT-inimeste kogukond ilmselt ei 
oleks saanud ka oma ideid kuidagi realiseerida. Sest kui palju on Tarvi Martens 
ID-kaardiga, Küberi seltskond oma küberturbega ja teised saanud minna riigi 
juurde ja  öelda, et \enquote{kuulge, see mõistlik asi, teeme!} ja neid on 
kuulatud!}

Jah, see ei oleks pidanud ka päris nii halvasti minema, kui ta läks, eks ole, 
Ukrainaga, et oligarhid võtsid majanduse päriselt vangi. Aga vaata Sloveeniat, 
näiteks. See oli riik, mis liitus Euroopa Liiduga Euroopa Liidu keskmise 
tulutasemega umbes 70. Ja täna ma küll ei näe, et nad kuidagi meist nagu 
paremad oleksid, pigem meie statistika on parem. Aga nendel läks niimoodi, et 
kuidagi majandus läks rohkem ettevõtete juhtide kätte, meil olid üksikud 
sellised ettevõtted. Nad ei loonud  väga palju uusi sidemeid uute turgudega, 
nad jooksutasid seda asja nii, nagu teda ikka oli jooksutatud ja majandus 
restruktureerus palju aeglasemalt. Meil ikkagi see see Mart Laari tohutu valus 
1992-1994 periood, kus vana asi istuti katki, oli kohutavalt valus suurele 
osale tolle hetke töötajaskonnast. Loomulikult. Ja kõrvalefekte me ei hallanud 
ja paljudel inimestel on siiamaani rusikas taskus. Põhjusega, sest me ei osanud 
neid kõrvalefekte hallata. Meil ei olnud ka selle jaoks raha ja me kõik 
uskusime, et tõusulaine tõstab kõiki paate.

Ja selle pärast, muide, mina tundsin 2016. aastal, et nüüd on aeg teadvustada 
endale seda, et tõusulaine kõiki paate ei tõsta, et nõrgematele tuleb padi alla 
panna, et see on nagu heaoluühiskond, eks. Nii et kui ma jõudsin 20 aastat 
hiljem ringiga tagasi riigi tegemiste juurde, olles vahepeal igal pool mujal 
olnud, mul tekkiski võimalus asuda seda viga parandama ja ma loodan, et me 
oleme õigel teel.

\question{Kindlasti. Aga lähme lõpetuseks tagasi sinna päris päris asjade 
alguse juurde. Miks on niimoodi, et Eestis on selline IT-inimeste kogukond, mis 
siiamaani toimib koos, aga näiteks lätlastel ei ole?}

Minu jaoks kannab ka meie IT ja,  tänapäeval isegi mitte ainult IT, sest kõik 
startupid ei ole ju mitte IT-sektori ettevõted, seda tol ajal tekkinud 
kultuuri, et võtame vastutuse selle riigi eest nagu enda kätte. Kui ma võrdlen 
seda ka meie vana majanduse ettevõtetega ja selle paradigmaga. Kui sa ikkagi 
olid peaministri nõunik, siis tulid sinu juurde tähtsad vana majanduse 
ettevõtjad, ütlesid, et me maksame nii palju makse, mis te meie jaoks teete? 
Siis IT-kogukond on kuidagi alati  olnud selline, et \enquote{riik ei saa seda 
teha, riik on selle jaoks liiga jäik, paindumatu, \emph{fine}, me teeme ise!}. 
Teeme Jõhvi koodikooli või mille iganes! See suhtumine on alles jäänud ja ma 
olen hästi rahul sellega, et see on alles jäänud. 

Aga miks ta niimoodi on, sellel kindlasti on mingisugune juurikas ka selles, et 
lasti üheksakümnendatel hästi palju teha, tekkis selline positiivne tagasiside. 
Meie Pavlovi refleks on see, et saab küll. Ja muidugi ikkagi  paljud 
IT-ettevõtjad ütlevad, et alati tahaks tõmbuda looteasendisse, kui sa oled 
natuke aega ametnikega rääkinud, eks ole. Aga seda vastu nina saamist ikkagi ei 
ole nii palju olnud, et oleks nagu lõplikult alla antud. Inimestele on ikkagi 
see usk, et tegelikult me saame need asjad tehtud ja meil on ka poliitikutel 
see lootus, et nende ametnikud ei ole ainult nagu tennisesein. Pall tuleb ja 
läheb kohe tagasi, vaid et kuskilt peab see pall ka läbi minema. Ja isegi, kui 
läbi läheb üks ajast, on see päris hea tulemus.


\question{Sest on kogemus olemas, et on ju töötanud, on ju saanud on ju 
toiminud!}

Just. Ja sellepärast on ka meie startup-kogukond hästi sotsiaalselt 
vastutustundlik minu meelest. Ma arvan, et  kui võtta vahelt ära parlament (ja 
mingis mõttes ongi võetud, meie parlament täna tegelikult ei ole mõttekoda, 
mida ta võiks olla) ja  lasta neil ilma keskse organiseerimisega, difuusselt, 
seda riiki ajada, siis väga palju hullem nagu ei saaks.

\question{Siinkohal on võib-olla mõistlik lõpetada tõdemusega, et päris hea 
riik on saanud?}

On. Aga ärme ära riku! Mida kõrgemal tulu tasemel sa oled, seda suuremad on 
sinu riskid midagi teha ja muuta. Me peame  kogu aeg suutma endiselt kogu aeg 
nagu uueneda, minna edasi. Ja ma tegelikult täna näen küll, et me oleme 
proovinud tekitada lubavat seadusruumi uutele tehnoloogiatele, aga tegelikult 
ei tule. Ja, õudselt ebapopulaarne, aga meil on vaja, et meie parlamendi palgad 
oleksid väga palju paremad selleks, et parlament töötaks sellise  mõttekojana, 
mis viib ka seda tehnoloogilist poolt edasi. Ta peab hakkama tõmbama ligi ka 
sedasama startup kogukonda, akadeemilist kogukonda uuesti, mida ta täna ei 
kõneta. Sest midagi ei ole teha, kahjuks on nii, mida maksad, seda saad.