\index[ppl]{Kaljulaid, Kersti}

\question{Alustaks kohe sealt, kust asjad ikka algavad ja kust me oleme kõiki 
neid jutuajamisi alustanud. Kuidas Teie jõudsite arvutite 
juurde?}

See juhtus üsna ammu, Nõukogude Liidu päevil 
Õpilaste Teaduslikus Ühingus\index{Õpilaste Teaduslik Ühing}. Ma arvan, et 
päris paljud minuvanused inimesed, kes on hiljem töötanud Eesti e-riigi või 
\emph{start-up}-kogukonnas, teavad, mida tähendab 
Küber\index{Küber} või Küberi arvutuskeskus. Tartu 
Ülikoolil\index{Tartu Ülikool} olid samuti olemas arvutuskeskused. Nii põhjas 
kui ka lõunas otsustasid täiskasvanud millegipärast, et lasevad lapsi sinna 
mängima. 

\question{Millised täiskasvanud?}

Õpilaste Teadusliku Ühingu eestvedajad. Näiteks Peeter 
Lorents\index[ppl]{Lorents, Peeter}, kes juhtis matemaatikasektsiooni. Ma ei tea, kes Tartus seda eest vedas, küll aga seda, et ka Tartu 
koolinoortel, näiteks Unineti\index{Uninet} Taavi Talvikul\index[ppl]{Talvik, 
Taavi} oli ligipääs Tartu Ülikooli arvutipargile. Toimus instinktiivne 
õpe, mis viis meid Õpilaste Teaduslikus Ühingus aruteludeni, 
kuidas kirjutada sellist asja, mida keskserveril oleks mõnusam analüüsida. 

Tollal oli nii, et üks asi arvutas ja ümberringi olid terminalid, kus me 
oma koodi kirjutasime. Tolleaegsed masinad olid mitteselektiivsed -- 
need ei otsinud, kes meist efektiivsema rea on kirjutanud, et seda siis 
töödelda. Aga meile tundus, et äkki oleks võimalik üksteisega kunagi 
võistelda, kes kirjutab sellise asja, mida keskserveril 
oleks mõnusam analüüsida. Mäletan laadilisi debatte, ükskord toimus
Õpilaste Teadusliku Ühingu suvelaagris\index{Õpilaste Teaduslik 
Ühing!Suvelaager} vist isegi öine matemaatikadebatt. 

Rõhutan, et mina ei kuulunud matemaatikasektsiooni, aga mind võeti kuidagi 
kampa. Olin üheteistkümnendas klassis ühingu 
teaduslik peasekretär, aga ise tegelikult 
ornitoloogiasektsioonist. Mulle meeldis Linné-aegne 
bioloogia,\sidenote{Carl Linnaeus, pärast aadliseisusse tõstmist 1761. aastal 
Carl von Linné (1707--1778), oli Rootsi teadlane, kes formaliseeris organismide 
nimetamise süsteemi ja keda tuntakse moodsa taksonoomia isana.} mis oli 
koolilapsele kättesaadav. Linnud, loomad, taimed -- kõik see viis mind 
laia maailma, küll kuuendikul planeedist, aga olümpiaadidel käies
sai reisida. 

Igatahes Õpilaste Teaduslikus Ühingus tekkis mul kokkupuude Küberi pundi ja 
arvutusvõimsusega.

\question{Nii et isegi ornitolooge viidi arvuti juurde?}

Ei, ornitolooge ei viidud, mul lihtsalt olid matemaatikasektsioonis sõbrad. Aga 
Lorents\index[ppl]{Lorents, Peeter}, Engelbrecht\index[ppl]{Engelbrecht, Jüri} ja teised ei teinud selles mõttes vahet, et me võisime olla 
neliteist-viisteist-kuusteist, aga saime osaleda täiskasvanute 
akadeemilistes mängudes.

\question{Neljateistaastasel noorel, eriti ornitoloogiahuvilisel, 
on miljon muud asja teha. Mis tõmbas just arvuti poole?}

Esiteks oli see põnev maailm. Teiseks, minu ainuke ornitoloogiaalane 
publitseeritud teadustöö, mis tegeles vainurästa pesitsuskommetega, viis mind statistilise tööni ja ilmselt seeläbi arvutini. See ei 
käinud nii, et vaatasin metsas, kus vainurästas elab -- selle olid 
teised inimesed ära teinud. Eestimaa Ornitoloogiaühingus\index{Eesti Ornitoloogiaühing} (või kuidas iganes Vene ajal seda kogukonda ka ei nimetatud\sidenote{Eesti Looduseuurijate Seltsi ornitoloogiasektsioon.}) oli 
kogunenud meeletus koguses pesitsuskaarte, kõik täiesti 
süstematiseerimata materjal. Minu akadeemiline tegevus Õpilaste Teaduslikus 
Ühingus seisneski selles, et otsisin statistiliste meetoditega erinevaid 
korrelatsioone. Näiteks tuli sealt välja, et linnas pesitseb vainurästas 
kõrgemal kui kuskil looduslikus biotoobis. Vainurästast ennast polnud vaja selle töö jaoks isegi mitte metsas ära tunda. Mitte 
et ma ei tunneks, aga arvutamiseks seda vaja ei olnud.

\question{Milles akadeemilised mängud seisnesid? Mis tüüpi ülesandeid te 
arvutiga lahendasite?}

Lihtsaid programmeerimisharjutusi, mida täna teevad paljud lapsed algkoolideski. Mõni täiskasvanu oli meil alati juures või siis tulime ka ise selle peale. Mina kaugemale väga ei jõudnudki, sest ma 
ei olnud matemaatikasektsioonis.

\question{Millised inimesed matemaatikasektsioonis olid? Kas
tüüpilised nohikud?}

Ei, seal oli erinevaid. Näiteks Tarvi Martens\index[ppl]{Martens, Tarvi} ja 
Tarmo Uustalu\index[ppl]{Uustalu, Tarmo} on täitsa erinevate kategooriate inimesed. Oli mitmesuguseid matemaatikahuvilisi noori erinevatest Eesti 
nurkadest. Minu arust ei ole olemas sellist stereotüüpi, mida alati 
otsitakse. Gruppidevahelised erisused on teadupärast väiksemad kui 
grupisisesed.


\question{Kas Õpilaste Teaduslikul Ühingul oli ka võrgustiku loomise 
funktsioon?}

Kindlasti. Olid erinevad sektsioonid: matemaatika, loodusteaduste, 
geograafia ja ajaloo, sealhulgas NSV Liidu ajaloo sektsioon, mis 
pandi ükskord kinni, mille üle mõned punasemad noored olid väheke nördinud. Näiteks Teet 
Jagomägi\index[ppl]{Jagomägi, Teet}, kes on tänaseks selgelt IT-ettevõtja, juhtis 
geograafiasektsiooni. Päris palju praegusest umbes 50aastaste kogukonnast on sealt ühel või teisel viisil läbi käinud. Kõik 
tundsid kõiki nagu ikka tollal.

\question{Mis siis sai, kui teaduslik ühing lõppes?}

Läksin Tartu Ülikooli. Üksiti oli see ka hüvastijätmine 
Linné-aegse bioloogiaga, sest minu juhendaja Raivo Mänd\index[ppl]{Mänd, 
Raivo} ütles, et tuleb õppida uusi asju, mis toovad tulevikus leiva lauale. Sellepärast ongi minu eriala geneetika, täpsemalt plasmiidigeneetika 
või bakterigeneetika. Bioloogias on ju keemia, füüsika ja matemaatika 
kõik koos, nii et stuudiumi jooksul tugevnes kindlasti minu arusaam 
matemaatikast kui kõike kirjeldavast ja kõiges toeks olevast teadusharust. 
Matemaatika on minu jaoks nagu keel. Ma ei ole selles ülearu osav, aga vajaduse 
piires olen suutnud toimetada.

\question{Tartu Ülikooli Arvutuskeskuse\index{Tartu Ülikool!Arvutuskeskus} 
kohta on räägitud, et seal käis koos üsna 
kirju seltskond teoloogidest jumal teab kelleni. Kas Teil oli selle kohaga ka 
kokkupuudet?}

Arvutuskeskuse seltskonna kirjusus tulenes muu seas sellest, et matemaatika 
ja füüsika olid erialad, millele konkurss Tartu Ülikoolis puudus. Seal oli 
alati kohti rohkem kui rahvast ja tihtipeale pugesid nendesse teaduskondadesse 
peitu ka inimesed, kes iga hinna eest tahtsid näiteks vältida Nõukogude 
sõjaväge. Minu aastal ülikooli astunud füüsikutest vist üks lõpetas füüsikuna. 
Küll aga astus sinna sisse näiteks Anzori 
Barkalaja\index[ppl]{Barkalaja, Anzori}, kindlasti teise eriala inimene. 

Seltskond oli kirju, aga arvutiteadus ongi suuresti 
interdistsiplinaarne, mitte spetsiifiline. Olen 
ise kodus märganud, et see on kuidagi pärilik -- minu esimene abikaasa 
ja vanem poeg elavad mõlemad arvutimaailmas, elavad ja hingavad 
bitte ja baite. Kuigi seda on ju võimatu näidata, kuidas pärilikkus saab 
millegi nii tehnogeensega koos käia, tundub mulle, et mingisugune ajutüüp 
peab selleks siiski olema. Ka vanemal tütrel on arvutiinimese aju. 
Tol ajal hakkaski selguma, et osal inimestel on 
arvutiinimeste ajud -- nad tõmbusid arvutuskeskusesse kokku, said 
üksteisest aru ja hakkasid vähehaaval kaotama sidet humanitaarsema poolega 
ühiskonnast.

\question{Mõtlesin, et lause lõpeb sellega, et \enquote{hakkasid kaotama 
sidet reaalsusega}, sest ka seda juhtus seal majas kergesti \ldots}

Ei, seda mitte. Arvan, et osa 
inimesi see maailm ei kõneta ja teisi intuitiivselt kõnetab. 
Olen kogu aeg tundnud, et ma ise arvutimaailmas sees ei ole, aga võibolla suudan kahe 
maailma vahel natuke tõlkida. See tunnetus on olnud päris varasest 
noorusest peale. Olen telekomisektoris töötanud, telefonijaam on ju nagu
arvuti: seda tuli samamoodi konfigureerida ja programmeerida. Töötasin üheksakümnendatel
palju koos inimestega, kes pidasid oma tööks arvutitega töötamist; ma ise seda ei teinud, küll aga püüdsin luua neile töötingimusi mõnes 
ettevõttes. Ma olen olnud nagu piirpinnal kõndija.

\question{See on väga põnev, sest seestpoolt vaadates tunduvad mõned 
asjad ilmselged ja mõned ebaselged ning kõrvalt vaadates võivad 
asjad paremini paista. Kuidas Te 
bakterigeneetikast ühtäkki telekomi sattusite?}

See oli imelihtne. Kui ma lõpetasin ülikooli, siis täpselt samal päeval tuli 
Eesti kroon\sidenote{20. juunil 1992.} ja ülikoolide pakutud tulutase 
(jäin pärast lõpetamist ülikooli tööle) oli nii väike, et sellega ei olnud 
võimalik lasteaiatasusidki katta. Sain aru, et 
uues Eestis läheb kas väga kaua aega, kuni see valdkond hakkab ära tasuma, või 
siis tuleb minna Eestist ära. Paljud kursusekaaslased lahkusidki Eestist 
ja neil kõigil on olnud väga edukas karjäär. Paljud on tulnud ka
tagasi ning on nüüd professorid Tartu Ülikoolis ja Eesti Maaülikoolis. Nad
ehitasid oma karjääri mujal üles ning kui Euroopa Liit asus 
laienema ja ka meie teadustaristut üles ehitama, siis neil oli super võimalus 
oma kolmekümnendate keskpaigas Eestisse naasta ja oma näo järgi 
laboreid ja uurimiskeskusi kujundada. Selles mõttes tõsine võitjate põlvkond.
 
Mina ei tahtnud Eestist ära minna, sest meil olid väiksed lapsed ja ma 
soovisin, et nad oleksid eestlased. Läksin puhtalt 
raha pärast teadusest erasektorisse. Töötasin pisikeses ettevõttes, mis paigaldas Siemensi 
telekommunikatsioonijaamu. Tööle võeti mind selleks, et tõlgiksin materjale eesti keelde, sest korralik inglise keele 
oskus ei olnud tollal nii levinud. Siis aga selgus, et tõenäoliselt kõlban 
päris hästi ka müüma. Ja kuna väikestel ettevõtetel on üks juht, siis 
oledki nii müügidirektor kui ka lihtsalt direktor. 

\question{Mis aastal see oli?}

Läksin sinna 1994. aasta sügisel. 

\question{1994 oli veel suhteliselt hull aeg. 
Kes tol ajal üldse Siemensi jaama endale paigaldas? Isegi analoogtelefon oli 
mõnes kohas haruldane.}

Metsikult pandi. See oli just see aeg, kus saadi aru, et büroohoonetes, 
ülikoolides, raamatukogudes ja igal pool mujal on tuppa telefoni vaja. 
Mul endal on vastupidine mulje, et Siemens, Ericsson ja 
väiksematest tegijatest näiteks Panasonic 
müüsid terve linna täis alates rahvusraamatukogust ja lõpetades 
pankadega. Turgu oli kõvasti! 

Siis tuli Eesti esimene riigihangete seadus ja
pidi hakkama selle järgi pakkumisi koostama. Tundus tohutu põnev, aga ka pisut 
hirmutav, sest varem ei olnud niimoodi käinud, et teed pakkumise ja siis 
loetakse kõik ette. Mul on meeles, kuidas istusime vist Tallinna Vanglas sealse  
telefonikeskjaama hankel ja selline tunne oli, et ei teagi, kas siit välja enam 
saab. Mitte et oleksime midagi valesti teinud, 
aga maailm muutus, süsteemi tekkis selgroogu ja struktuuri 
juurde.

\question{Te jõudsite pärast ka Eesti Telekomi, aga mina mäletan tollest ajast, et 
see oli mingisugune õudne monstrum!}

See polnud siis veel Eesti Telekom, vaid Eesti Telefon\index{Eesti Telefon} ja 
mina töötasin sellises peenes kohas nagu Eesti Telefoni äriklienditalitus. Läksin erasektorist sinna, sest seal tundus 
olevat rohkem karjäärivõimalusi. Olin väikses ettevõttes tipus ja tundsin, et 
tahaksin edasi liikuda, suuremat struktuuri vaadata. 

Eesti Telefonis oli tollal päris keeruline. Ükskord öeldi mulle, et sel kuul ei saa rohkem müüa, sest meil sai 
sisseostuplaan täis. Siis ma olin jube kuri ja tegin üheselt selgeks, et ma ei 
taha mitte kunagi enam sellist väidet kuulda -- kui müüme, siis müüme, ja 
kui te ei taha, siis ma lähen midagi muud tegema. Aga müüsime küll.

\question{Eesti Telefon oli minusuguste nohikute jaoks 
üsna õudne ettevõtte: küll ei suutnud traati pakkuda ja ei müünud mingil hetkel
isegi internetti, väites, et telefoniliini peal ei 
peagi internet töötama, sest liin on helistamiseks.}

Ma ei tea äriklienditalituse loojate kaalutlusi, aga arvan, et asja 
mõte oligi tuua sinna struktuuri üks üksus, mis hakkaks senist organisatsioonikultuuri seestpoolt õõnestama ja muutma. 
Tolle talituse rahvas pidi müüma täitsa tavalistele eraettevõtetele ja 
 neidsamu jaamu, mida Siemens ja Ericsson ka eraldi müüsid. Ma täpselt ei mäleta, kuidas Valdo Kalm\index[ppl]{Kalm, Valdo} selle 
talitusega seotud oli, aga igal juhul oli ja aitas
ettevõtte muutumisele kindlasti väga palju kaasa. 

Algus oli minu jaoks keeruline. Mõni inimene küsis, miks mind kunagi 
oma laua taga ei ole. Minu arust müügijuht ei peagi olema oma laua taga. Aga tasapisi see kultuurimuutus tekkis.

\question{Kultuuri mõttes oligi huvitav aeg, kui sidevaldkonnas kombineerusid
äri, akadeemiline kogukond ja häkkerite-nohikute maailm. Kas see paistis Eesti Telefoni poolelt ka välja?}

Muidugi paistis, sest inimesed olid samad. Võtame või Taavi 
Talviku\index[ppl]{Talvik, Taavi}, minu esimese abikaasa, kes tuli Tartu 
Ülikooli arvutuskeskusest\index{Tartu Ülikool!Arvutuskeskus} läbi 
Valitsusside\index{Valitsusside}. Seejärel tegid nad Andes 
Baumaniga\index[ppl]{Bauman, Andres} oma ettevõtte Uninet, mis hiljem müüdi ära ja millest sai 
Elisa. 

Selles mõttes oligi üks maailm. Mis seal lõppude lõpuks vahet on, kas helistad või saadad muid 
andmeühikuid? Digitaalne tehnoloogia tol ajal just tuli. Tekkisid 
probleemid, kuidas tagada läbilaskvus ja ühenduste laius -- kogu see maailm 
hakkas vaikselt arenema ja kasvama. Minu meelest 
pole see Eestis kunagi eraldi olnud. Kandja pool ei olnud kindlasti eraldi ja sisu 
poole ettevõtteid tollal ju eriti ei olnudki veel. Esimene internetipank tekkis vist aastal 1994? Aastal 1997 oli juba e-maksuamet.

Sisuteenused hakkasid ka üsna kiiresti tulema, aga siis algas kohe ka see 
võistlus, et toru on küll olemas, aga sisu tahab laiemat, ja kui toru saab laiemaks, siis tahab sisu 
veel laiemat. Ma ise olin selgelt toru, mitte sisu poolel.

\question{Rääkisime, et stereotüüpe ei ole. 
Ometigi on Eesti Vabariigis olnud laialt käibel niisugune mõiste nagu 
\enquote{patsiga poiss}. Mis inimene see on? Mis teda iseloomustab?}

Neid on väga erinevaid. Eesti Telekomi aegadest ma ei mäleta peaaegu kedagi 
peale Valdo Kalmu\index[ppl]{Kalm, Valdo}, kes seal minuga veel koos töötasid -- palun vabandust endiste kolleegide ees! Aga näiteks mäletan Uku 
Kuuti\index[ppl]{Kuut, Uku}, kes oli meil süsadmin, patsiga poiss. Ja 
kui tema tuppa läksid, sest midagi oli paigast ära, siis tal muusika alati käis. 
Tundus küll nagu teine maailm võrreldes paljude teistega, aga kindlasti 
oli ka pöetud habemega rahvast.

Need olidki üsna erinevad seltskonnad. Kui müüsid telefonijaama ja tegelesid 
pankadega, siis oli selgelt näha, et pankade tehnikajuhid vastutasid juba 
tollal suhteliselt suure struktuuri püstihoidmise ja edasiarendamise 
eest, olid hästi makstud ja ei erinenud millegi poolest pankade
raamatupidajatest. Neid ei kutsutud siis veel CTOdeks, aga seda nad sisuliselt 
olid ja ei erinenud muude valdkondade eest vastutajatest. 

Ja siis oli iseõppinud vendi, kes olid kuidagipidi (ega seda ülikoolides ei õpetatud) 
ise arvuteid pidi ringi nuhkinud ja saavutanud oskuse hoida asjad töös. 
Nende hulgas oli jah võibolla seda stereotüüpi, et nad
suhtlevad parema meelega masinaga. Samas ei olnud masin tol ajal nii huvitav 
suhtluspartner ja ei olnud võimalust internetti päris ära kaduda. Ma arvan, et see on üle võimendatud, 
kuidas seal kastis saab kogu ööpäeva ära sisustada. 

Ühesõnaga, inimesi oli igasuguseid.

\question{Meil on olnud teistega juttu sellest, et teatava peakujuga 
inimesi tõmbas Tartu Ülikooli arvutuskeskusse -- võibolla see tõmme on 
ühine nimetaja?}

Kindlasti. Need, kes Tallinnas 
Küberis\index{Küber} koos käisid, on kõik selles sektoris tänini leitavad, nad on olnud
püsivamad ja järjepidevamad kui mina. Midagi ju on, mis tõmbab meid
matemaatika või keelte juurde. Need on erinevad 
asjad. Võibolla praegune 
keeleinstituudi\index{Eesti Keele Instituut} direktor Arvi Tavast\index[ppl]{Tavast, Arvi} on mõlemal poolel 
kõndija: ühtpidi IT-tegija ja teistpidi on teda sügavalt huvitanud 
keeled ning need kaks asja saavad tänases maailmas kokku. Enamik inimesi 
kipub siiski olema paremal või vasakul. Ma ei tea, miks.

\question{Kui vaadata kasvõi inimesi, kellega ma selle raamatu raames rääkinud olen, siis 
naisi on vähe. Tol ajal oligi neid selles valdkonnas vähe. Miks?}\phantomsection\label{sisu:tydrukud}

Naised on alalhoidlikumad ja toimetavad valdavalt sektorites, mis on 
sisse töötatud. Paratamatult on nad ka karjääri mõttes 
alalhoidlikumad. Minulgi oli ühel hetkel päris 
palju ideid ja valikuid, mida võiks teha, näiteks kas tekitada oma butiik ja 
hakata seda arendama. Ma ei teinud seda sel lihtsal põhjusel, et pidin ülal pidama kaht alaealist last -- olin tol hetkel üksikema. See oli teadlik valik, sest mul oli tarvis rohkem kindlustunnet ja struktureeritud elu. Ma 
ei saanud endale lubada, et olen võibolla järgmised viis kuud ilma palgata. 

Kuna see oli tollal uus valdkond, siis naised lihtsalt ei võtnud neid riske. Võibolla 
seiklusjanu oli ka esialgu väiksem ja ega keegi ei näinud ju 
sellega ka teadlikult vaeva. Me räägime sügavatest 1990ndatest! Toon ühe
eheda näite. Mina müüsin siin Siemensi Hicom 300,\sidenote{Siemensi paindlik 
telefonijaamade sari.} Soomes tegeles müügiga üks Tiina. Kord läksime 
Siemensi suurte telefonikeskjaamade müügimeeste kokkutulekule, kus olid peale meie ainult 
mehed, kes küsisid uskumatul ilmel: 
\enquote{Kas te tõesti müüte neid suuri telefonijaamu?} Me ei saanud aru, miks 
me ei võiks seda teha. Tol ajal ei olnud see naiste maailm. 

\question{Kui Eestis oli selline avantürism arusaadav, siis muu maailm oli selles mõttes ikkagi teistsugune?!}

Võtame Kesk-Euroopa, näiteks Saksamaa. Seal on
minu põlvkonnas veel päris palju koduperenaisi, just Lääne-Saksamaal. 
Idas ei ole. Statistiliselt on Ida-Saksamaa naiste 
pensionid kõrgemad, samuti lahutuste arv, sest nad saavad 
seda endale lubada erinevalt läänest. Kui me kujutame ette, et 
üheksakümnendatel oli ärikultuuris tohutu võrdõiguslikkus, siis ma kaldun arvama, 
et see pole sinna päriselt jõudnudki ja selle nimel võideldakse. Meil ei maksa luua endale illusioone, 
et seda tööd ei pea enam tegema. Selles mõttes on Taavi Kotka\index[ppl]{Kotka, 
Taavi} Unicorn Squad, Rakett ja teised sarnased ettevõtmised tüdrukute 
toomiseks tehnoloogia ligi ühiskonnale tohutu väärtusega. Ei 
ole mitte ühtegi põhjust, miks tütarlaps ei võiks toimetada tehnoloogiarikastel 
aladel.

\question{Tütarlapse jaoks võib ehk olla vähem loomulik see, et ta 
magab kontorilaua all, sest uni tuli peale?}

Ära hakka seda \enquote{loomulik või vähem loomulik}! Ei ole niimoodi, kuigi paljud võivad sedasi arvata! Mis seal vahet on? Sul 
on 20-aastane vaba inimene, lapsi ega peret ei ole -- ükskõik, kas ta on mees või naine. Kui 
ta seal kontorilaua all magab, siis teksad on nagunii jalas, soengu ja 
seelikuga sinna ju ei lähe. See ongi see alateadlik stereotüüp, isegi kui see ei ole pahatahtlik. Aga see on täiesti olemas, nagu ka sinu 
väites!

\question{Tõsi, ka punkareid oli igasuguseid.}

Jah, samamoodi võib olla insenere ja keda iganes.

\question{Mul lihtsalt tuleb silme ette üks konkreetne habemega punkar, kes 
vedeles niimoodi hommikul laua all. Aga tõepoolest, see on minu stereotüüp ja 
kujutluspilt, et see on habemega punkar -- miks ta ei võiks olla 
teistsugune!}

Mu tütar rääkis ülikooliajast loo, kuidas üks ettevõte otsis tööjõudu. 
Päris paljud käisid ennast pakkumas ja võeti üks poiss, kes oli ülikoolis 
silmnähtavalt teistest laisem ja kehvema õppeedukusega. Mõne aja pärast, kui ta oli kohanenud, küsis poiss
tööandjalt: \enquote{Kuule, meilt kandideerisid veel need ja 
need inimesed, miks nemad ei saanud?}. Mille peale talle öeldi: 
\enquote{Vaata ümberringi, näed sa siin mõnda naist?} Selline tõrjuv kultuur, 
eks ole. See ei ole nüüd side-, telekomi- ega ka IT-ettevõtte, vaid lihtsalt insenerikultuuri näide Eestist. Ja need inimesed, 
kellest ma räägin, on praegu 32--33, mitte 50-aastased.

\question{Mugavam on palgata omasugust. Iseküsimus, kas ka kasulikum.}

Ei ole, sellepärast et statistiliselt tuleb naiste pähe 50 protsenti headest 
ideedest ja meeste pähe 50 protsenti.

\question{Pigem on isegi teistpidi \ldots}

Ma ei lähe sinna kunagi. Me ei peaks ütlema, 
et naised on kuidagi teistmoodi juhid või insenerid. Väidan, et oleme 
ajupotentsiaali mõttes võrdsed. Paraku on ka minu käest 
küsitud: \enquote{Kuidas siis nüüd nii, et naine juhib elektrijaama?} 
Olen siis väga otsekoheselt vastu küsinud, et kuule, räägi nüüd, 
mida sa selle tilliga teed jaama juhtides? 

\question{Kuna meie IT-värk tuleb sellest seltskonnast, mis oli
faktiliselt ühele poole kallutatud, siis kas see on meid kuidagi 
digiühiskonnana tagasi hoidnud või edasi aidanud või üldse mingit mõju avaldanud?}

Tegelikult ei ole. Olen meie digiriigi arengu peale palju mõelnud. Miks me oleme nii 
teistmoodi, kuigi mujal on palju tugevama IT-sektoriga erasektor kui meil 
Eestis? Ühiskonda muudab ikkagi riik, midagi ei ole teha. 
Erasektoris võidakse teha geniaalseid asju, aga \emph{mainstream}'imise 
maailmameistrid on kõik riigid. See on see koht, kus riik peab midagi 
tegema hakates kaasa võtma kõik: vanad, noored, mehed, naised! Ja 
selle muudatuse on Eesti ära teinud.

Priit Alamäe\index[ppl]{Alamäe, 
Priit} vist tõi kasutusele termini \emph{digitally transformed nation}. Siin on 
nüüd see koht, kus me läksime teist teed kui teised maailma riigid. Kui 
riik hakkab tundma huvi mõne sektori võimaluste kasutamise vastu 
riigiteenuste osutamiseks, siis see hakkab päriselt ühiskonda muutma 
ja kujundama. Meil juhtus see, et ühest hetkest olid kaasatud 
mehed, naised, lapsed, vanurid, ja tekkisid ka positiivsed kõrvalefektid. Näiteks koroonapandeemia tekkides meil ju ei olnud häda, et 
70aastased ei saa pangas käidud või 
telekomilepinguid uuendatud, sest nad olid 50aastased, kui ID-kaart tuli. Sellised
positiivsed kõrvalefektid on olnud kogu ühiskonna jaoks hästi suured.

\question{Miks meil juhtus niimoodi?}

Sellepärast, et meil ei olnud midagi. Ajalooliselt oli ju esimene 
\mbox{e-teenus} \mbox{e-maksuamet}. Kas sa kujutad ette, et Eesti inimesed oleksid 
nõustunud seisma tundidepikkustes sabades, et riigile oma maksud ära viia?

\question{Praegu enam ei kujuta.}

Aga siis ka ei kujutanud. Oli tõenäone, et maksulaekumised ei ole ülearu head, 
kui loodad sellisele asjale. Õnneks oli selline aeg, et 
e-pangandus ju oli ja sai teha e-maksuameti. Keegi ei taha ju maksuametnikku näha! 

\question{Ma ei ole kuskil mujal näinud sellist usaldust 
keskmise bürokraadi ja \emph{hardcore} inseneri vahel. Ühest küljest 
insener usaldab, et bürokraat ei keera asja kihva. Teisest küljest 
bürokraat usaldab, et kui tehnik hakkab rääkima XML-sõnumite 
vahetamisest, siis ta päris udu ei aja ja tarnib tulemuse. Kust meil see usaldus
tuli? IT-kogukond isekeskis küll teadis, tundis ja usaldas üksteist, 
aga kuidas laiem ühiskond juurde tuli?}

Ega ei tulnudki. Kui ma läksin 1999. aastal peaminister Laari\index[ppl]{Laar, 
Mart} juurde tööle, siis tema tellimus oli suhteliselt mittespetsiifiline. Ta võttis oma nõunikud kokku ja ütles: \enquote{Nüüd 
on nii, et me kõik saame aru, et Eestis palgad kasvavad ja varsti me ei 
ole enam rikka mehe kuluefektiivsuse lahendus. Vaadake ringi, kuhu edasi minna.} 
Sealt algas see, et kui minu juurde tuli Andres 
Metspalu\index[ppl]{Metspalu, Andres} jutuga, et oleks vaja teha 
Geenivaramu\index{Geenivaramu}, siis sai kõik rattad käima lükatud. Eiki 
Nestor\index[ppl]{Nestor, Eiki} tuli ka appi ja tegime inimgeeniuuringute 
seaduse. 

Teise valdkonna -- ID-kaardi -- pakkus välja Linnar 
Viik\index[ppl]{Viik, Linnar}. Tol ajal 
hakkas juba tekkima ka sisuteenust ning Kaarel Tarand\index[ppl]{Tarand, Kaarel} nägi 
väga hästi seda pilti, kuhu see sisu pool kommunikatsioonis ja mujal
minemas on. ID-kaardi idee vist tekkiski sellest, et 
pangad ei olnud ju pikaajaliselt nõus võtma vastutust, et riik jooksutab oma 
e-teenuseid nende platvormidelt. Seetõttu tuligi teha ID-kaart, aga see 
sündis ühise veenmistöö tulemusel. 

Jube raske oli veenda rahandusministeeriumit, kes tahtis kohe 
teada, kus on ROI.\sidenote{\emph{Return on investment} -- investeeringu 
tasuvuse näitaja.} Täna tundub see naljakas küsimus: \enquote{Mis 
mõttes, vaadake, milline e-riik meil on!} Aga kust Linnar\index[ppl]{Viik, 
Linnar} tollal need arvud oleks võtnud? Põhjendus, 
millega tegelikult ID-kaarti valitsusele müüdi, oli ju absurdne: e-valitsus. See, et valitsuses olid 
arvutid laua peal, oli toonud meile nii palju tasuta artikleid 
välisajakirjanduses, et see süsteem oli ennast kolme kuuga tasa teeninud 
võrreldes sellega, kui oleksime lihtsalt ostnud \emph{Estonia -- Positively 
Transforming}\sidenote[][-1cm]{2002. aastal käivitatud suur ja mitmesugust 
meediatähelepanu pälvinud Brand Estonia kontseptsioon \enquote{Welcome 
to Estonia} põhiline tunnuslause.} lehepinda. Selle argumendiga 
tehti ID-kaart! 

Nii et see arvamus, et kõik tulid Linnar 
Viigi\index[ppl]{Viik, Linnar} ja teistega kohe kaasa, on vale. Töötasid teised argumendid: kulu ei olnud nii suur ja 
Vabariigi Valitsuse ruum oli tõepoolest väga palju välismaist positiivset tähelepanu 
saanud. Seejuures teenimatut, sest nendesamade Saksa ja Soome inimeste erafirmades oli intranet ju täiesti tavaline. 
Eestis samuti -- Eesti esimene intranet hakkas tööle Postimehe\index{Postimees} toimetuses aastal 1991 või 1992. Aga riikide tasandil seda ei tehtud ja see siis oligi sirgelt ID-kaardi müügiargument: 
võime saada palju välismaist tähelepanu.

\question{Ilmselt oli maitse suus, et korra oleme juba saanud, küllap 
nüüd ka!}

Just. Ja asja geniaalsus seisnes selles, et näiteks Saksamaal pead 
siiamaani omale digi-IDd taotlema, aga meie pistsime selle 
lihtsalt kõikidele kaardi peale. Kasutavad või ei kasuta, aga ega see liiga ka ei 
tee. Tegime ainult ühte tüüpi kiibiga ID-kaardi ja see oli geniaalselt 
õige lahendus.

\question{Minu teada ei ole keegi teine seda ka eriti järele teinud.}

Nüüd vist ikka juba on, aga mitukümmend aastat hiljem \ldots

\question{Kui palju oli arusaamist, et 
tegemist on geniaalse lükkega, kui palju pikka visiooni ja kui palju 
praktilist kaalutlust, et \enquote{paneme lihtsalt käima}?}

Ma mäletan seda arutelu. Äkki ei teeks, äkki teeks. Teeks 
kõigile. Maksab nii palju. Kui teeme osadele, kas maksab vähem? Ei 
maksa vähem, vaid tegelikult rohkem, kui on erinevad süsteemid. 

Oli tõesti hulk inimesi, sealhulgas Infotehnoloogiafirmade 
Liit\index{Infotehnoloogiafirmade Liit} ja Linnar 
Viik\index[ppl]{Viik, Linnar} ning ilmselt teisigi, kes ütlesid, et võtab küll aega, kuni teenused peale 
lähevad, aga anname kõigile. Analoogselt olime ühel hetkel 
otsustanud, et mitte keegi ei hakka enam siin riigis sularahas palka saama, vaid kõik said 
omale pangaarved -- kõik pidid tegema ja raha hakkas minema panka. Kuigi ka siis oli algul palgapäeval ATMi 
juures saba, sest inimesed võtsid kogu raha välja. Funktsionaalsus ei läinud kohe käima. 

Seda analoogi tõime 
palju oma aruteludes ka põhjendusena, miks peaks ID-kaardi tegema 
universaalsena. Mart Laar\index[ppl]{Laar, Mart} lükkas seda hoogsalt tagant ja tihtipeale tuli lükata just nimelt neid, kes 
hästi hoolega raha lugesid -- Reformierakonda.

\question{Kas see lükkamine käis tal põhimõtteliselt sellesama arusaama pealt, 
et see on strateegiliselt oluline asi?}

Jah, tema uskus seda, Linnar\index[ppl]{Viik, Linnar} oli suutnud ta seda 
uskuma panna. Mina pidin selles protsessis olema kaasas sellepärast, et 
olin majandusnõunik (muidu oli peaministri büroos geeniseadus nagu rohkem minu laps). Minule ütleski rahandusminister, et unustage ära, sest te ei 
suuda mulle ROId näidata. Siis me mõtlesimegi välja selle, et \enquote{aga 
see asi tasus ennast ju kolme kuuga ära!}.

\question{Järelikult vastab tõele legend sellest, kuidas 
Linnar\index[ppl]{Viik, Linnar} ja Mart\index[ppl]{Laar, Mart} olla istunud laua 
taga ja Mart olla öelnud: \enquote{Linnar, mis me teeme?} ja Linnar olla 
vastanud: \enquote{Teeme interneti.}}

Vastab tõele küll, aga Mardil ei olnud nii kitsas vaade. Ta tahtis lihtsalt 
teada, et öelge kõik midagi, mida me võiksime teha.

\question{Huvitav kombinatsioon praktikast ja visioonist. Tollane 
Eesti Vabariik oli ikka oluliselt teistsugune kui praegu, muresid oli 
miljon!}

Oligi. Jään ka selle juurde, et me ei saa kogu seda au endale võtta. 
1999. aastal saime ju Euroopa Liidu teadus-arendusprogrammi liikmeks. Siis hakkas Euroopa Liit valmistama meid ette 
liitumiseks ja institutsioonide ning igasuguste muude asjade ehitamiseks tulid rahad peale. Väidan, et Euroopa Liidu rolli ei tohi alahinnata -- 
digiriiki on oluliselt lihtsam ehitada, kui keegi teine maksab koolide, teede ja muude asjade 
remondi kinni. Raha hakkas siin riigis liikuma palju rohkem ja tänu sellele oli ka võimalik 
teatud kõõl sellest digiteenuste arengusse suunata.

\question{Marek Tiitsu\index[ppl]{Tiits, Marek} 
IBS\index{Institute of Baltic Studies} ja igasugused muud asjad olid ju kõik 
välisfondide rahaga tehtud.}

Ja kui teised nägid, mida me teeme, siis tekkis üsna kiiresti laboriefekt: 
aitame, toetame ja saame ise ka näpud vahele sellele, mida nad seal teevad. Meist ei oleks
kindlasti saanud sellist e-riiki, kui meil ei oleks avanenud 
võimalust saada suures koguses välisabi. Vähehaaval oma toonasest SKTst (elasime tollal tegelikult ju ikkagi Maailmapanga vaeste riikide 
kriteeriumite järgi) me seda ei oleks teinud. Kuigi, mööngem, et 
täna ei saa sa ilmselt isegi ühte korralikku e-maksuametit selle raha eest, millega me 
esimesed kümme aastat oma e-riiki ehitasime.

\question{Kuidas see muutus operatiivtasandil käis? Minu käest on
näiteks Prantsuse ametnikud küsinud, et arusaadav, tegite e-riigi, aga 
kuidas te selle ametnikele ära seletasite?}

Jaa, seda küsis ka minu käest president 
Macroni esimene peaminister Édouard Philippe, kes oli otsustanud, et nüüd tuleb Prantsusmaal ka teha 
digipööre. Ta uuris, mis avaliku sektori
töökohtadest sai. Ütlesin talle, et vaata, Édouard, 
nüüd on nii, et meil läks maksuametis 60 protsenti töökohti kaotsi, aga meil 
ei olnudki üldse nii korralikku maksuametit nagu teil. Ära selle pärast 
muretse, te ju teete tööturu liberaliseerimise reformid ka ära, eks! Mida 
nad ongi teinud, ehkki muidugi mitte määrani, mida meie peame igal 
juhul normiks. 

Vastab tõele, et Eestis kadus ka töökohti, aga (see on nüüd 
teisest valdkonnast) kui Hoiu- ja Hansapank liitusid,\sidenote{See toimus 
jaanuaris 1998.} said Eesti ettevõtted endale finantsjuhid. Miks? Sest panganduses jäid üle sisulise poole spetsialistid, kes suutsid arvutada, 
ja ettevõtetel oli neid vaja kasvõi 
selleks, et nendesamade pankadega läbi rääkida. Neid polnud aga kuskilt võtta ja kui nüüd
kaks panka ühinesid, siis korraks tekkis selles valdkonnas tööjõu ülejääk 
ja hopsti! Täpselt samamoodi vajab erasektor e-riigi 
ehitamiseks kogu aeg inimesi, kes teavad, kuidas riigis protsessid käivad, ja minu 
arust on nad kogu aeg ise ära söönud sellesama tööjõu, mille nad on avalikus 
sektoris hävitanud.

\question{Hüve hüveks, aga miks ametnikud vastu ei hakanud töötama?}

Sellepärast ei hakanudki, et eestvedajad, kes võtavad 
juhtrolli, ei jää ju kunagi ilma tööta. Kindlasti oli kuskil ka neid, kes 
kannatasid ja kelle töökohad kadusidki. Eestvedajad ei 
muretse selle asja pärast, sa ise õpid selles protsessis nii palju. Ja 
väga paljud hüppasid teise paati koos erasektoriga -- nad osteti üle, et pakkuda riigile seda teenust tagasi.

\question{Meil ei olnud tol ajal see aparaat veel kivistunud, sa ei saanud 
olla olnud 15 aastat ametis, sest vabariiki polnud nii kaua eksisteerinud.}

Igal pool muutusid ju tegelikult noored meie riigi näoks. Ükskord kui olime Laariga Leedus visiidil, ütles üks Leedu erastamisagentuuri juht mulle, et te eestlased teete hästi julgeid asju, sellepärast 
et te olete kõik nii noored ja ei taju üldse, mis hirmsad riskid selle kõigega 
kaasnevad. Selles oli kindlasti oma iva. Sama lugu oli panganduses. Pärast Hoiu-Hansa ühinemist tuli siia üks 
ungarlane ja ma näitasin talle panga
ülemist korrust, mis oli täis oma nappi kolmekümmet eluaastat 
prilliraamide taha varjata püüdvaid tüüpe. Ta küsis mu käest: 
\enquote{Ütle, Kersti, mis te vanemate inimestega tegite?} Aga selle hind on 
see, et meie põlvkond pidaski üleval oma vanemaid ja kasvatas oma lapsi ning
tõenäoliselt jääb meil keskmise eluea osas mingi negatiivne hüpe 
sisse.

\question{Panga seltskond oli tol ajal jah nooruslik. Ja kui vaatame 
Laari, siis ta tänapäeva mõistes ajalooõpetaja hariduse pealt istus maha ja 
tegi maksureformi!}

Mardil oli meeletu usaldus oma nõunike vastu. Ma ei mäleta 1992.--1994. aasta 
perioodi,\sidenote{Mart Laar\index[ppl]{Laar, Mart} oli peaminister aastatel 1992--1994 ja 
1999--2002.} mina siis seal ei töötanud, aga ta lasi Ardo 
Hanssonil\index[ppl]{Hansson, Ardo} ilmselgelt otsustada ja möllata, nii nagu 
ka hiljem Linnar Viigil\index[ppl]{Viik, Linnar}, Kaarel 
Tarandil\index[ppl]{Tarand, Kaarel}, Simmu Tiigil\index[ppl]{Tiik, Simmu} 
või meil teistel. Saime vabad käed ja ta oli meie seljataga. Ütlesime lihtsalt: \enquote{Kuule, me tahaksime nüüd sellise asja ära 
teha.} 

Tol ajal oli igal ministeeriumil oma pank. Täna on meil 
ainult üks EAS ja KredEx -- mu arust on seegi risuks jalus, aga hea küll. 
Ja siseministeeriumis oli umbes kaks fondi, mis andsid raha 
välja. Ütlesin Mardile, et see on jube ebaefektiivne ja 
milleks need pangad üldse välja on mõeldud, kuna see raha 
seal ei kulu just kõige efektiivsemalt. Mart ütles kohe: \enquote{Tee ära, 
koristage need asjad ära.} Pärast hakkas eurorahasid liikuma ja siis oli 
EASi ka vaja. (Muide, minu magistritöö teema oli riigi asutatud 
sihtasutuste juhtimine ja see oli just nimelt seotud koondamise ja muu säärasega.) Mart ütles niisiis, et andke tuld, aga ministreid tuli ikka 
ise veenda, seda ei hakanud ta meie eest ära tegema. Käisime ja veensime. Padarit ei veennud ära ja Maaelu 
Arendamise Sihtasutus jäigi eraldi.\sidenote[][-1.7cm]{Ivari Padar\index[ppl]{Padar, Ivari} oli 
Mart Laari teises valitsuses aastatel 1999--2002 põllumajandusminister. Maaelu Edendamise 
Sihtasutus (MES) kuulub tänaseni maaeluministeeriumi valitsemisalasse.}

\question{See on täpselt selline kombinatsioon, et sul on 
oma strateegiline vaade, aga toimetama lased suhteliselt 
apoliitilise seltskonna, praktilised inimesed, kes saavad aru, 
mida on vaja teha.}

Meil oli eile Latitude'il\sidenote[][]{Konverents Latitude59 toimus Kultuurikatlas 
19.-20. mail 2022.} arutelu. Minu vestluspartneriks oli prantslane, kes küsis 
kogu aeg, mida peab riik tegema selleks, et ka Euroopas oleksid toredad digifirmad nagu ameeriklastel. Lõpuks
ütlesin talle: \enquote{Me oleme siin nüüd pool tundi rääkinud täpselt sellest, 
kuidas Eesti riik ei sekku sellesse, millised majanduslikud valikud 
erasektoris tehakse ja millised sektorid peavad arenema. 
Dirižiste\sidenote{Diri{\v z}ism, prantsuskeelsest sõnast \emph{diriger} (suunama) 
on majanduslik doktriin, milles riik mängib tugevat suunavat rolli 
kapitalistliku majanduse suhtes.} meie hulgas ei leidu. Tulemus: kümme
\emph{unicorn}'i ühe miljoni kohta. Kas see ütleb sulle midagi või ei ütle?} 

Umbes selline peabki minu arvates olema poliitikategemise roll: sa 
võimaldad asju teha. Samamoodi meie maksusüsteem -- mida tähendab ettevõtete tulumaksuvabastus? Igas investeeringus (rõhutan \emph{igas}, mitte valdkondlikult valitud 
eelisarendatavas valdkonnas) on riik ju 20 protsendiga sees ning võtab 
riski nagu ettevõtjagi. Võibolla ei hakka sealt kunagi dividende tulema, mida 
saaks maksustada.

Meil on tegelikult kihvt riik!

\question{Ka mina olen selle raamatu koostamise ajal korduvalt 
läinud mõttes tagasi 1990ndatesse ja iga kord tunnen ennast 
hästi, kui tore riik meil on!}

Oled sa vahel mõelnud, mis oleks saanud, kui oleksime valinud ennast juhtima 
inimesed, kellel riigi ja oma rahakott lähevad segamini? Meil, muide, on ka 
praegu poliitikas selliseid inimesi päris palju, kes jäävad kogu aeg vahele 
sellega, et nad on oma võimupositsiooni kasutanud enda või partei 
hüvanguks. Väga vabalt oleks võinud niimoodi minna. Ja siis 
oleksime täna omadega Ukrainas.

\question{Meil õnnestus valida mingi ime läbi 
mõistlikud inimesed \ldots}

Tegelikult ei õnnestunud. Lennarti\index[ppl]{Meri, Lennart}\sidenote{Lennart 
Meri, Eesti president aastatel 1992--2001.} kõige vastuolulisem tegu 
põhiseaduse kontekstis oli see, et ta lasi Laaril moodustada valitsuse, kui 
Savisaar oli valimised võitnud. Aga kuna kolmikliit\sidenote{Kolme Eesti partei -- Eesti Reformierakonna, Isamaaliidu ja Mõõdukate --
valimisliit.} oli eelmoodustatud, siis ta läks sellele teele. Kusjuures kolmikliit seisis ka nii habrastel alustel, et poleks üldse 
moodustunud, kui Edgar Savisaar\index[ppl]{Savisaar, Edgar} oleks saanud 
võimaluse mõne nendest potentsiaalsetest partneritest ära rääkida. Oleksime võinud 
minna märksa konservatiivsemat majandusarengut, märksa lõdvemat eelarvepoliitikat ja võibolla ka märksa oligarhsemat majandusmudelit pidi, kui 
mõelda, mida Keskerakond tollal või ka täna endast kujutab.

\question{Ja tolles keskkonnas ei oleks ilmselt ka IT-kogukond saanud oma ideid realiseerida. Kui palju on Tarvi Martens 
ID-kaardiga, Küberi seltskond küberturbega ja teised saanud minna riigi 
juurde ja öelda: \enquote{Kuulge, see on mõistlik asi, teeme!} ja neid on 
kuulatud!\nopagebreak[100000]}

Jah, ka Ukrainaga ei oleks pidanud nii halvast minema, nagu läks, sest
oligarhid võtsid majanduse päriselt vangi. Samas näiteks Sloveenia puhul, kes ühines Euroopa Liiduga ELi keskmise 
tulutasemega umbes 70, ma täna küll ei näe, et nad oleksid meist kuidagi 
paremad -- pigem on meie statistika parem. Nendel läks
majandus rohkem ettevõtete juhtide kätte, samas kui meil oli üksikuid 
selliseid ettevõtteid. Nad ei loonud väga palju uusi sidemeid uute turgudega, 
vaid jooksutasid majandust nii, nagu seda ikka oli jooksutatud, ja majandus 
restruktureerus palju aeglasemalt. 

Meil see Mart Laari  
1992.--1994. aasta periood, kui vana asi istuti katki, oli kohutavalt valus suurele 
osale töötajaskonnast. Paljudel inimestel on siiamaani rusikas taskus ja põhjusega, sest me ei osanud 
neid kõrvalefekte hallata. Meil ei olnud selleks ka raha ja 
uskusime, et tõusulaine tõstab kõiki paate.

Muide, seepärast ma tundsingi 2016. aastal, et nüüd on aeg teadvustada 
endale, et tõusulaine kõiki paate ei tõsta ja nõrgematele tuleb padi alla 
panna, kui me tahame heaoluühiskonda. Nii et kui jõudsin 20 aastat 
hiljem ringiga tagasi riigi tegemiste juurde, olles vahepeal igal pool mujal 
olnud, tekkiski võimalus asuda seda viga parandama ja ma loodan, et 
oleme õigel teel.

\question{Lõpetuseks lähme tagasi päris asjade 
alguse juurde. Miks on Eestis IT-kogukond, mis 
siiamaani toimib koos, aga näiteks lätlastel ei ole?}

Minu jaoks kannab ka meie IT -- tänapäeval isegi mitte ainult IT, sest kõik 
\emph{start-up}'id ei ole ju IT-sektori ettevõtted -- tol ajal tekkinud 
kultuuri, et võtame vastutuse riigi eest enda kätte. Kui võrrelda 
seda meie vana majanduse ettevõtetega ja paradigmaga, siis tähtsad vana majanduse 
ettevõtjad tulid peaministri nõuniku juurde ja ütlesid: \enquote{Me maksame nii palju makse, mida te meie jaoks teete?} 
IT-kogukond on aga alati olnud sellise suhtumisega, et riik ei saa seda 
teha, riik on selle jaoks liiga jäik ja paindumatu, \emph{fine}, me teeme ise! 
Teeme Jõhvi koodikooli või mille iganes! See suhtumine on alles jäänud ja ma 
olen hästi rahul. 

Mingisugune juurikas on kindlasti ka selles, et 
üheksakümnendatel lasti hästi palju teha ja tekkis positiivne tagasiside. 
Meie Pavlovi refleks on see, et saab küll. Paljud 
IT-ettevõtjad ütlevad küll, et alati kui nad on natuke aega ametnikega rääkinud, siis tahaks looteasendisse tõmbuda, aga seda vastu nina saamist ei ole ikkagi 
nii palju olnud, et lõplikult alla anda. Inimestel on 
usk, et saame tegelikult asjad tehtud, ja ka poliitikutel 
on lootus, et nende ametnikud ei ole ainult nagu tennisesein -- pall tuleb ja 
läheb kohe tagasi, vaid kuskilt peab see ka läbi minema. Isegi kui 
läbi läheb üks sajast, on see päris hea tulemus.

\question{Sest meil on kogemus, et on ju saanud ja
toiminud!}

Just. Sellepärast on ka meie \emph{start-up}-kogukond sotsiaalselt väga
vastutustundlik. Kui võtta vahelt ära parlament (mõnes
mõttes ongi võetud, sest meie parlament ei ole täna tegelikult mõttekoda, 
mida ta võiks olla) ja lasta neil ilma keskse organiseerimiseta, difuusselt 
seda riiki ajada, siis väga palju hullem see ei saaks.

\question{Siinkohal ongi ehk mõistlik lõpetada tõdemusega, et päris hea 
riik on saanud!}

On. Aga ärme riku seda ära! Mida kõrgemal tulutasemel oled, seda suuremad on 
riskid midagi teha ja muuta. Peame suutma kogu aeg 
uueneda ja edasi minna. Näen täna, et oleme 
proovinud tekitada lubavat seadusruumi uutele tehnoloogiatele, aga tegelikult 
hästi ei õnnestu. Ja see kõlab nüüd õudselt ebapopulaarselt, aga meie parlamendi palgad 
peaksid olema palju paremad selleks, et parlament töötaks mõttekojana, 
mis viiks ka tehnoloogilist poolt edasi. Ta peab uuesti hakkama tõmbama ligi ka 
akadeemilist ja \emph{start-up}-kogukonda, mida ta täna ei 
kõneta. Midagi ei ole teha -- kahjuks on nii, et mida maksad, seda saad.