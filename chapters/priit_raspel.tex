\index[ppl]{Raspel, Priit}
                 

\ldots minul on selline Jaapani ajaarvamine millega ma asju paika panen. Mul on 
mingid sündmused, mida ma tean, ja mille suhtes ma teisi määratlen. Näiteks 
1993. aastal, seda ma tean, käisin ma maailmameistrivõistlustel. 

\question{Mille maailmameistrivõistlustel?}

4GL\sidenote{\emph{Fourth-generation programming language (4GL)}.} 
programmeerimise maailmameistrivõistlustel. Kolmekesi käisime. Mina, Veiko 
Herne\index[ppl]{Herne, Veiko} ja Tiiu Lumberg\index[ppl]{Lumberg, Tiiu},  
tulime viiendaks, saime veel eripreemia  kõige  elegantsema lahenduse eest. 
Mõtlesime välja Amazoni, mida tollel hetkel veel ei olnud olemas. 

Ja vot nüüd ma tean, et samal ajal ma hakkasin ära tulema 
Innovatsioonipangast\index{Innovatsioonipank}, see paneb selle sündmuse 
paika, eks ole. 

See jaapani \emph{native} ajaarvamine käib ju suurte sündmuste vahel. Et selle 
keisri võimule tulemusest kolmas aasta. Ja kui siis suur maavärin tuli, siis 
see keiser unustati ära. Teati, et see maavärin oli selle keisri võimaluse 
tulemusest nii mitu aastat aga edaspidi arvutati aega maavärisemine järgi. 
Sedasi on jube hea, sest sa ei pane neid asju muidu enam liini peale.  

\question{Kuidas sina arvutite juurde said?}

Mina olen juhuslik inimene, täiesti konkreetselt juhuslik. Keskkoolis olin 
täiesti veendunud, et see on viimane roppus, mida ma õppima lähen. 

\question{Mis aastal see oli?}

Keskkooli ma lõpetasin 1979. Ma käisin Kusti-koolis, see on siis esimene 
keskkool\index{Tallinna 1. Keskkool}, Gustav Adolfi gümnaasium. Tollel 
ajal Tallinna esimene keskkool kus arvutid olid fakultatiivselt sees ja kus oli 
matemaatika-füüsika eriklass. 

\question{Seitsmekümnendate lõpus juba?!}

Jaa! Meile õpetajate programmeerimist Fortranis\index{Fortran} ja pagan 
teab milles. Arvutis käisime Teaduste Akadeemias\index{Teaduste Akadeemia}, 
arvuti oli Lenini puiesteel (praegu Rävala puiestee), seal nurga peal, kus 
raamatukogu ja muu on. Arvuti oli Minsk-32\index{Minsk!Minsk-32}, mis 
koliti just välja. Asemele kolis vist ES-1022\index{ES EVM!ES-1022}. Ma 
mäletan, kuidas seda Minsk-32-te välja koliti,  terve alumine klaasfuajee oli 
tükke täis. Mul ühe klassivenna onu oli seal programmeerija, see ütles, et 
arvuti visatakse prügimäele, me käisime seepeale sealt plaate välja konkimas. 
Sellepärast, et seal oli P13 transistor peal, mis oli kõige defkam transistor 
ja igasuguste asjade tegemisel kasulik.

\question{Sul siis juba mingi elektroonikahuvi oli olemas?}

Jaa, ma esimese raadio panin kokku vist kuueaastaselt. Isa oli ju raadiotehnik. 
Ja kui juppe vedeles kodus nagu jube ja raamaturiiulis oli raamat \enquote{Noor 
raadioamatöör}\sidenote{\enquote{Noor raadioamatöör}, 
Viktor Borissov, tõlkinud Arnold Isotamm, Tallinn: Eesti Riiklik Kirjastus, 
1953.}, kus oli kirjas, kuidas detektorvastuvõtjat teha. Eks ma siis 
nihverdasin isa sahtlist mõned dioodid, pooli tegin ja panin kõrvaklapi külge 
ja sain mingi Majaki\sidenote{1964. aastal käivitatud üleliiduline raadiojaam, 
tegutseb siiani.} kätte. Muidugi Majak oli nii võimsa signaaliga, et see tuleks 
pliidiraua pealt ka. Kui pliit oleks pilti näidanud, oleks pilti ka tulnud. 

Aga õppimisega oli selline värk, et meile jube kehvasti õpetati. Õpetajaid ei 
saanud ja iga igal poolaastal luges erinev õpetaja mingisugust täiesti erinevat 
asja, mida ta parasjagu ise just oskas. Metoodika puudus, aga me saime käia 
seal Teaduste Akadeemia arvutis, seal olid need perfokaardi 
\enquote{Kivipurustajad}, mis lõid auke läbi me saime ühte kappi panna oma 
pakid ja nii palju ma ikka tegin, et see oli päris huvitav asi. 

Ma olin täitsa kindel, et ma lähen elektroonikat õppima. Ma tegin poistele 
raadiosaatjaid ja. Isa, küsis, et \enquote{poiss, kas see on sinu töö, 
peilingaator sõidab akna all?}. Muidugi oli minu töö. \enquote{Näita, mis sa 
tegid?  Kurat, sul on sihukesed lõpptransid, me peame võimsuse tagasi tõmbama}. 
Me elasime Tõnismäel, segajad olid sealsamas\sidenote{Nõukogude Liidus oli kena 
komme välismaiseid raadiojaamu sihipäraselt segada. Üks selleotsarbelistest 
raadiojaamadest (need allusid Sideministeeriumile), nr. 602, asus Tõnismäe 
ligidal Luha tänaval.}. Vennad püüdsid signaali kinni ja tulid otsima, kus on. 
Lõpuks rehkendasime siis 300 meetri peale, et kust lähemale ei tohi minna ja  
kui lähed, siis tuleb ruttu ära kaduda. 

Nii et ma tinutasin igasugu asju kokku. Aga juhtus selline jama, et mul tuli 
üks üsna raske haigus. See haigus oli selline, mis viskas mul korra aastas 
teadvuse ära. Tõbi tuli kallale, eks ole, ja see oli sihuke raske tõbi, et ravi 
oli ka keeruline. Pool aastat oli üpris kangete tablettidega ravi, täpselt pidi 
jälgima režiimi ja kui välja ei tulnud, siis pidi pool aastat pausi pidama. No 
ja see tuli mul välja kuskil niimoodi, et veel ilusti keskkooli lõpus oli ta 
olemas, kolm aastat võttis see selle haiguse ära võtmine aega. Keskkooli lõpus 
oli ta mul ilusasti olemas ja sellele oli vastunäidustatud  elektroonika õppima 
minek, mida ma tahtsin teha, eks ole, sest ma olin lapsest peale neid aparaate 
ehitanud. 

Ma olin noor inimene ja mõtlesin, et ei lähe kuskile, olin nii vihane. Ei taha, 
mul pole vaja, ma lähen tööle elektroonikuks! Kolb püsib käes, skeemist ma saan 
aru, isegi ise oskan skeemi koostada, montaažplaate oskan teha, telekat oskan 
käsikaudu parandada (selles mõttes, et ilma igasuguste mõõteriistadeta, leian 
vea üles ja parandan ära), raadioga saan hakkama. Aga isa rääkis augu pähe. 

\question{Aga ema oli sul\ldots}

Ema oli mul Eesti Raadio\index{Eesti Rahvusringhääling!Eesti Raadio} 
majandusjuhataja. Tema viis mind sealse tehnokeskuse meestega kokku. Poes 
polnud ju suurt midagi. Kõige kohta, mida ma poest ei saanud, koostasin listi, 
läksin tehnokeskusse ja sealt leiti mulle. Ühe haaval küsisin, sealt kuskilt 
karbist need mulle leiti. Ja teinekord, kui nad midagi maha kandsid, siis nad 
andsid need tükid ka mulle. Mul on praegu veel üks karbitäis neid asju olemas.

\question{Sa kokkuvõttes leidsid, et \emph{midagi} sa pead ikka õppima minema?}

Jah, isa arvas, et võiksin ikkagi õppima minna. Et ole nüüd kaval, et esimene 
aasta on üldained. Mine õpi midagi sellist, kus eksamid on samad. Ja siis ma 
läksin sinna IT-sse, mis oli siis see  Majandusliku informatsiooni 
mehhaniseeritud töötlemise organiseerimine, Eesti kõige pikema nimega eriala 
üldse. Kusjuures see on selles suhtes vahva eriala nimi, et sõjavägi ei saanud 
aru, keda me koolitame. Terve selle eriala jooksul ei läinud mitte ükski poiss 
ohvitserina pärast kooli sõjaväkke. Sest nad ei saanud aru, et seal 
puhtaverelisi programmeerijaid koolitatakse.

\question{Nii et sa pole vene kroonus käinud?}

Ei ole. 

No eks  ma siis läksin seda keerulise nimega asja õppima ja oli suhteliselt 
hea, mul oli seal kuus kooliõde-venda ka  ja niisugune seltsielu läks käima 
kohe ja grupp oli lahe. Muidugi tolle aja kohta peab teadma seda, et IT grupp 
oli 25 inimest, kellest poisse oli ainult kuus. Sest meil oli 
majandusteaduskonna grupp ja majandusteaduskonda tulid tüdrukud, meil olid väga 
ilusad targad tüdrukud, kes programmeerisid ka päris kõvasti. 

Poolteist kuud läks ja ma olin müüdud mees.

\question{Mille peale see sul juhtus?}

Juhtus selle peale, et ma nägin selle maailma ilu. Ma nägin neid võimalusi, 
kuidas kõik, mis on kõrvade vahel olemas, see on võimalik ka päriselt. Ega see 
mul nii lihtsalt ei läinud, sellepärast et Gustav Adolfi gümnaasiumis õpetati 
neid asju ikkagi väga hüplikult. Mulle eluaeg meeldinud süsteemne lähenemine, 
aga seal räägiti mulle midagi, mida keegi ei pannud minu jaoks süsteemi ja 
tollel ajal ju ei olnud kohta ka, kust lugeda. Internetti ju põlnd. Ja 
raamatuid ka ju ei väga võimalik saada. 

Üks sõber, Lembit Sammel\index[ppl]{Sammel, Lembit} vedas mu Leo 
Võhandu\index[ppl]{Võhandu, Leo} jutu peale kolmanda ühika alla. Kolmanda ühika 
all olid Nairi-2\index{Nairi!Nairi-2}, AP keel\index{AP keel} ja 
konsul elektriline kirjutusmasin. Hakkasime AP-s kirjutama, tegime biorütme, 
silusime neid, tegin oma esimese mängu. Tiku ära võtmise mäng, et kes võtab 
viimase tiku, kus masin mängis vastu ja sain sellega päris ilusti hakkama. Vaat 
sedasama mängu kirjutamise ajal see asi haaraski mind, sest, ma ütlen,  ma ei 
saanud sellest asjast nagu lõpuni aru.  Nagu tollel ajal oli, istusid laua 
taga, paber ees ja kirjutasid blokk-skeemi. Ja siis ma kirjutasin, seda 
blokkskeemi, arvan, et oli oktoober, õhtul, üsna vihmane ilm oli, mul laualamp 
põles.  Kuidagi jäävad sihukesed asjad eluks ajaks meelde. Paber laua peale, 
mina pusin, ei tule ja ühel hetkel käis siukile täiesti kuuldav nips ja ma sain 
aru, mis ma tegema pean. Ma kirjutasin üsna kiiresti  algoritmi valmis ja 
pärast seda mul ei ole algoritmide kirjutamisega mitte mingisugust probleemi 
olnud. 

\question{Sa sattusid üsna varsti ju tööle ka?}
Teisel kursusel, sest esimene kursus ma kammisin ikkagi ühika vahet. 

Meil oli kuidagi niimoodi, et mul esmaspäev oli üle nädala vaba, hommikul 
läksin nii vara kohale, kui ühika alune klass lahti tehti ja kui mind jõuga 
välja visati tulin ära. Lembit Sammel, hüüdnimega Sass, tema täpselt samamoodi. 
Panime nagu paaris härjad. Ei olnud päeva, kus me mõnda masinat nässu ei 
keeranud, sest kui sa pidevalt piinasid neid, siis nad põlesid läbi. Seal 
klassis ma sain tuttavaks ka Lindre Reinuga\index[ppl]{Lindre, Rein}, kes oli 
seal inseneriks,  kellega me hiljem koos tegime veel ühe vahva asja. 

Teisel kursusel läksin EPT-sse tööle, EPT on siis Eesti 
Põllumajandustehnika\index{Eesti Põllumajandustehnika}. Keskkontor oli Salve 
tänaval, kus olid Minsk-32\index{Minsk!Minsk-32} ja 
ES-1022\index{ES EVM!ES-1022}. ES-i numbriga ma võin eksida, aga 
Minsk-32 oli küll. Suured saalid olid täis, mürisesid. 

Minu jaoks oli see hea asi, et ma hakkasin operaatoritega hästi läbi saama ja 
ma ei pidanud enam TPI arvutuskeskuses päev otsa perfokaardiga jamama, et see 
saada järgmine päev kätte kuskilt kapist, kuhu ta jäeti. Ma läksin tüdrukute 
juurde, ütlesin, et \enquote{kuulge,  laske mu pakk läbi}. 

\question{Kas sind võeti sinna mingit konkreetset asja programmeerima?}
Ei, lihtsalt otsiti inimest, kes oleks noor ja avatud ja keda nad ise saaksid 
oma käe järgi välja õpetada. Oleg Kase\index[ppl]{Kase, Oleg} oli selle tiimi 
juht, seal tegutses Toomus Tõnu\index[ppl]{Toomus, Tõnu}, kes läks kahjuks 
Estoniaga minema. Väga geniaalne, \emph{väga} geniaalne programmeerija. Nad 
hakkasid mind õpetama ja ma hakkasin tegema lihtsamaid asju. Esialgu selliseid 
alltöid, mingid funktsioonid ja värgid ja mis neil vaja oli. Aga üsna pea 
jõudsin selleni, et tahaks ise midagi teha. Selle peale öeldi, et vali siis 
ise, mis  sa teha tahad. Ja kuna seal oli just parasjagu igasuguste 
kasutajaliideste tegemine ja laoarvestuse ja kõige viimine suurde 
SM-i\index{SM-4}, sellest samast suurest ES-ist ära toomine, see 
alguses oli ka seal, see oli suhteliselt värske koht, siis ma tegin vormi 
generaatori, kus joonistasid ekraani, sidusid andmebaasiga ära ja siis sidusid 
selle MUMPS\sidenote[][-2cm]{\enquote{Massachusetts General Hospital Utility 
Multi-Programming System - MUMPS} on transaktsiooniline võti-väärtus andmebaas, 
millega on integreeritud ka programmeerimiskeel. Selle süsteemi puhul oli 
fookus jõudlusel (tema kaudu käib tänini rohkem kui poolte USA patsientide 
terviseinfo) ja mitte loetavusel: Kõiki käske võis lühendada ja reavahetus ei 
olnud oluline. Tulemuseks oli sageli, ütleme, raskesti loetav kood. \texttt{S 
\%='\%M!'\%D,\%Y=\%Y-141,\%H=\%H+(\%Y*365)+(\%Y\\4)-(\%Y>59)+\%,\%Y=\$S(\%:-1,1:
\%H+4\#7)} on valiidne rida MUMPS-i koodi\ldots} süsteemi puuga ära ja\ldots

\question{Ma olen lugusid kuulnud. Kui sa oled tänapäeval õpetatud 
programmeerija, siis MUMPS on mingist hoopis teisest maailmast!}

Ja ta on täitsa olemas, ma leidsin ta üles ükspäev, täitsa kogemata jäi mulle 
internetis jalgu miskipärast. 

Vaata, seal ei ole indekseid, sa pead ise indeksid tegema üle inverteeritud 
immituste. Kuna seal on puu, siis sa pead puu võtme, \emph{path}-i, kirja 
panema ja registreerima et nüüd on sul selle kohta võti olemas. Nüüd kui sul on 
see võti, võid \emph{path}-i järgi otse peale minna näiteks. 

MUMPS-is on näiteks niimoodi, et kui sa tahad mingit seadet kasutusele võtta, 
siis sa pead teadma selle seadme numbrit. Näiteks ma mäletan, et printer oli 
vist 80. Kirjutad \verb|U:80|, see tähendab, et nüüd kõik ülejäänud jama, mida 
sa väljundisse paned, läheb printeri peale. Kui sa tahtsid kuvarit saada, siis 
igal kuvaril oli oma number. Kuna need olid füüsilised masinad, siis sa tegid 
andmebaasi kõigepealt loendi nendest kuvaritest mis sul on, nimetasid kuvarid 
ära. Nime järgi lugesid kuvarid sisse, ütled \verb|U:1| näiteks, siis sattusid 
esimese kuvari, konsooli, peale. 

Ja siis oli veel võimalik anda \emph{wait} aegu, aga ma ei mäleta, mis sümbol 
seal vahel oli. Et \enquote{oota nii kaua siis mine edasi}. Aga üldiselt ütleme 
niimoodi, et programmeerimiskeele loogika on ikkagi enam-vähem. Kuna tal on 
hierarhiline või puu-kujuline andmebaas all, siis see pani muidugi omaette 
pitseri, sest \emph{pre-} ja \emph{post-order} ja igasugused siuksed asjad 
pidid jube hästi käpas olema. 

\question{Sellel EPT-l oli igal pool kontoreid, eks?}

Igal pool jah. Tähendab, ma tean, et oli Sauel, Paides, Tartus, ma rohkem 
lihtsalt ei mäleta praegu, aga tal ilmselt oli kuskil veel mõni. Igal pool olid 
tugevad inimesed eesotsas. Tiimid olid väikesed, neli inimest. Muidugi suitsu 
tehti, kohvi joodi, sest kuvarid olid eraldi ruumis, arvuti oli teises ruumis, 
igalühel oli tuhatoos ühel pool kuvarit, teisel pool kohvitass. Kohv täiesti 
must ja suhkruta, sellepärast, et piim läheb ju hapuks, seda ei saa kuskil 
kapis hoida, suhkru saab otsa, mis sa tast ikka ostad?  Seal ma õppisingi musta 
kohvi jooma, ilma suhkruta, mida ma siiamaani joon hea meelega. Ja seal ma 
õppisin suitsu tegema, mis mind ikkagi tükati on päris mitmeid aastaid niimoodi 
saatnud. Aga mul on see, et mul ei ole sõltuvust. Ma võin jätta suitsetamise 
maha niimoodi, et ma panen pakki laua serva peale, tikutopsi peale ja seal ta 
seisab, mind see ei häiri. 

\question{Kuidas nende kontorite vahel side käis?}

Side käis kahte moodi. Kas ümbrikuga või läbi teletaibi kanali. Igas EPT 
kontoris oli teletaibi aparaat. Teletaip oli teksti edastamiseks, kirjutusmasin 
oli tal peal, eks ole. Aga tema küljes oli ka perfolindi lugeja. Ja nüüd meie  
masinast lasti perfolint välja, söödeti teletaipi, teiselt poolt,  näiteks 
Paidest, lasti lint välja ja söödeti sealsesse masinasse. Aga ega see nii 
lihtsalt ei käinud, seal oli protokoll ka. Sellepärast, et side ei olnud püsiv 
ja kippus kukkuma. Kui side kukus, siis teletaip tegi \enquote{piiks}. Selle 
peale tõstis inimene telefoni, helistas teisele osapoolele ja ütles 
\enquote{Kuule, tõstan kümme kirjet tagasi}. Teisel pool tõstis operaator õla 
üles, tõstis perfolindi tagasi, vastuvõtja tõmbas lindile poole pastakaga 
joone. Seda võis mitu korda juhtuda. Ja kui lint oli lõpuni jõudnud, siis 
vastuvõtja lõikas lindid märgitud kohast katki, võrdles, kus asi kokku langeb, 
liimis otsad kokku (spetsiaalne rakis oli, millega augud läbi torgati, et need 
puhtad oleksid) ja söötis lindi masinasse sisse. 

\question{Ühesõnaga elektrooniline seade muutis andmed kõigepealt 
pabermeediasse, siis elektroonilisse meediasse, siis uuesti pabermeediasse ja 
lõpuks tagasi elektroonilise meediasse. Ja veaparandus oli manuaalne!}

Jah. Neli koopiat linte tekkis. Üks siit poolt mis välja lasti ja sealtpoolt 
vastu võeti ja teine, mis sealtpoolt sisse lasti ja siitpoolt vastu võeti. 
Töötas, töötas suurepäraselt. Aga siis õpetas Leo Võhandu\index[ppl]{Võhandu, 
Leo} mulle mingit andmeedastust ja protokolle ja värke. Läksin selle Rolandi 
juurde, kes oli meie keskuses peainsener. \enquote{Roland, kuule, kas sa mulle 
sellise seadme saad teha, mis paneb selle arvuti ja selle teletaibi kokku ja 
teisele poole samasuguse  vastu võtmiseks?}. Ma ei teadnud, et see modem on, 
sihukest asja ei olnud olemas. \enquote{Uh, ma just Radio ajakirjast (selline 
tolleaegne igasuguste selliste tehniliste nikerdajate ajakiri), nägin, seal oli 
üks skeemi, oot, ma vaatan}. Ja tegigi valmise. Montaažplaat oli 50 korda 50 
sentimeetrit, sellepärast et see oli selle SM-i sisemine plaat, nagu riiul läks 
talle sisse. Aga skeem oli nurga peal 10 korda 10 sentimeetrit. Pani selle 
käima,  mina otsisin vahepeal mööda opsüsteemi, et mis võimalused on. Leidsin 
ühe sihukse, ütleme, struktureerimata ala, et eraldad lihtsalt mälu ja 
struktureerid ära. Ehitasin sinna peale kataloogisüsteemi ja kirjutasin need 
programmid ringi, mis neid linte väljastasid niimoodi, et nad kirjutasid sinna 
kataloogisüsteemi, mitte ei saatnud. Ja ühtlasi kirjutas, kellele saata. 
Esialgu ei olnud kellelgi rohkem, kui  Paidesse saata. 

Aga igal juhul perfektsionistist inimene kirjutab ikkagi adressaadi ka juurde, 
mine tea kellele on veel vaja saata. Noh, ja siis oli üks programm, mis 
aeg-ajalt vaatas, et kas on linte tekkinud, helistas teise poole välja. Kui 
kukkus, siis tõstis 10 kirjet tagasi, teine teadis, et ta tõstis 10 kirjet 
tagasi ja hakkas uuesti saatma. Probleem oli selles, et sa ei teadnud, millise 
hetke peal side kukkus, ei olnud võimalik määrata, mis kirjed olid ära läinud. 
Saatis minema, teine võttis vastu võrdles, kus hakkasid samasugused kirjed 
tulema ja loksutas paika. Asjad läksid täitsa ilusti üle. Teinekord oli 
niimoodi, et kui side ei taastunud, siis võis seanss olla terve päeva katki. 
Süsteem proovis, vahet ei ole ju, eks ole.

\question{Suurusjärgus mis tempoga see andmeside toimus?}

See võis mingisugune, 100-300 bitti sekundis, kuskil sealkandis ma arvan. Ei 
või praegu öelda, aga ega ta kiirem küll olla ei saanud, eks ole. 

Side hakkas kiiremini minema, kui me elektrooniliselt läksime, sest paberlint 
läks muidugi aeglaselt, kiirust polnud ollagi.  Nüüd läks kõik hästi niikaua 
kuni Tartu tuli ütles, et \enquote{me tahame ka}. Ka nüüd läks kõik esialgu 
hästi. Ainult et juhtus niimoodi, et Paide helistas mulle peale, side kukkus ja 
siis helistas Tartu peale. Aga ma ei teadnud, et see Tartu on. Üritasin vastu 
võtma hakata, aga sealt ei tule midagi tuttavat, pole näinud elus sihukest 
asja. Mõtlesime välja sessioonivõtme. Lihtsalt mingi \emph{hash}. Enne seda, 
kui seanss tekitati, saatis \emph{hash}-i ette ja nüüd oli teada, eks ole, et 
kui ta uuesti tuleb, siis sellesama \emph{hashiga} ja nüüd võisin segamini ka 
vastu võtta. 

\question{Ega TCP põhimõtteliselt väga teistmoodi ei käi}

Selles mõttes on jah vahva, et tunned pidevalt asju ära. 

Ühesõnaga, ma töötasin seal, kuni TPI lõpuni. Nii kaua kui ma õppisin, nii kaua 
ma seal töötasin, kokku neli aastat ja täitsa huvitav oli.

\question{Sa lõpetasid siis kõige selle kõrvalt nominaalajaga? Töö koolis 
käimist ei seganud?}

Ei! Sellest oli palju kasu, ma olin teistest peajagu üle kogu aeg, sest ma olin 
saanud kõike seda, mis meile õpetati, elus katsetada. See EPT seltskond lubas 
mul väga vabalt toimetada ja ütles, et \enquote{kasuta kõike, mis saad, peaasi 
et hea on}. 

Muidugi mis me seal tegime, me tegime esimese onlain messi Eestis üldse 
arvutitega. Ise vedasime pool kilomeetrit kaablit posti otsas Saue mõisa. Tol 
ajal oli niimoodi, et hooaja lõpus müüdi kõik varuosad, see käis tavaliselt 
niimoodi, et  istusid kaubatundjad suurte paberitega ja tõmmati maha, mis 
müüdud sai. Aga meie vedasime Saue mõisa side ja panime viis kuvarit üles. Ja 
tegime selle kõik elektrooniliselt. Peainsenerid ja kolhoosiesimehed said ise 
arvutist valida ja selekteerida asju. Töötas suurepäraselt. Väga innovatiivne 
kamp oli.

\question{EPT kui innovatsioonikeskus ei ole küll kuskilt läbi jooksnud}

EPT-s oli üks tuntud mees. Tõnu Lume\index[ppl]{Lume, Tõnu}, mängis filmis 
Lurichit\sidenote{Tallinnfilmis 1984. aastal valminud film \enquote{Lurich}.}, 
oli EPT arvutuskeskuse juhataja asetäitja ja Jaak Raja\index[ppl]{Raja, Jaak} 
oli arvutuskeskuse juhataja. Ta oli karm mees, aga minusse ta suhtus jube 
hästi. 

\question{Miks sa EPT-sse pikemaks ei jäänud?}

Tuli kooli lõpp ja suunamine. Tänapäeval keegi ei teagi, mis suunamine on. 
Suunamine oli niimoodi, et pingejärjekorras saad valida, kuhu läheb. Kui sul 
hinded on, siis selle järgi saad valida. Aga  nimekirjas oli kaks kohta, mis 
olid spetsiaalselt mulle. Et kui mina ei neid ei valinud, siis keegi teine neid 
ka valida ei saanud. Üks oli EPT ja teine TTÜ. Ja saad sa aru, TTÜ oli sellel 
ajal põnevaks kohaks muutunud selles mõttes, et sinna hakkas juba väljamaalt 
tehnikalt tulema. Personaalarvutid ja. 

Ma ise käisin tõin esimese personaalarvuti rongiga Moskvast. Yamaha, 
kaheksabitine, Z80 prosega. Käisin koos  Sven 
Jürgensoniga\index[ppl]{Jürgenson, Sven}, ma ei tea, kus ta nüüd esindaja on, 
vahepeal oli USA suursaadik, praegu vist NATO juures esindaja, ta oli tudeng. 
Tõime kaks esimest rongiga,  kupees. 

\question{Mis ametikohta sulle pakuti?}

Selline koht nagu \enquote{juhtivinsener}. Ma lugesin ka, see oli lihtsalt 
teadusliku uurimise sektori ametikoht infotehnoloogia kateedri 
all\index{Tallinna Tehnikaülikool!Infotehnoloogia kateeder}. Pakuti ka head 
palka, kõigele lisaks. Ega EPT ka kehva ei pakkunud. Tollel ajal kui hea palk 
oli 120, siis mulle pakkuda 155. Eluaeg mäletan. Aga,  muidugi, mu elu läks 
kehvemaks. Sest ma elasin ikkagi jube priskelt, kui ma õppisin, sest ma sain 
stippi kõrgendiku, see oli lõpus 60 rubla, siis ma sain EPT-st 60 rubla poole 
koha eest. Ja siis ma tegin veel Teadusliku Uurimise Sektoris\index{Tallinna 
Tehnikaülikool!Teadusliku Uurimise Sektor} ka tööd, sealt ma sain 40 rubla 
kuus. Ma sain kakssada rubla kuus, eks ole! Lisaks sellele ma sain EPT-s veel 
60 rubla kvartalis preemiat, kui mitte rohkem, vahel 100 isegi. Aga nüüd 155 
rubla. Mul läks tükk aega, enne kõik liinid tööle sain ja Teadusliku Uurimise 
Sektor mulle lisa hakkas maksma. 

\question{Misasi see Teadusliku Uurimise Sektor oli?}

TPI-s kõik lepingulised tööd tehti sellise asja alt nagu Teadusliku Uurimise 
Sektor. Seal oli oma juht ja kui kellegagi tahtsid lepingut sõlmida, siis 
Teadusliku Uurimuse Sektor sõlmis lepingu, võttis oma obroki sealt vahelt ja 
sina said oma lepingulised tasud. 

\question{Mida sealt telliti?}

Varsti räägin ühest konkreetsest tööst, Lindre Reinuga\index[ppl]{Lindre, Rein} 
seotud. Aga tehti igasuguseid asju, mida iganes. Ma tean, et sealt alt tehti 
näiteks mingeid kriminalistika infosüsteeme, tean selle pärast, et meil kõik 
näited olid üksvahe kriminalistika pealt. Igasuguseid asju, need teemad käisid 
meil kõik enam-vähem loengutest näidetena läbi, mis seal tehti. 

No jah, mina valisin TTÜ ja läksin Raja Jaagule\index[ppl]{Raja, Jaak} 
lahkumisavaldust viima. Ta vaatab mulle otsa, ütleb \enquote{Priit, kuule. Jää 
poole kohaga ikka tööle. Sa ei pea üldse kogu aeg käima, käi vahel läbi ja 
vaata, ütle mis arvad ja too uuemaid sõnumeid}. Nii ma mingi kaks või kolm 
aastat töötasin veel täitsa niimoodi rahulikult seal. Käisin küll,  
südametunnistus ei luba, et kui raha antakse, eks ole, päris ilma ei saa. 
Võtsingi mingi pausi, pikema puhkuse, et natukene sellest kõigest lõõgastuda, 
täitsa huvitavaid asju sai veel tehtud. Kuni sinnamaani välja, et viimane, kus 
ma käisin, aastaid hiljem, siis ma enam ei töötanud seal, siis olid SM-i 
matused. Viin ja kartulisalat ja masin ise oli maha müüdud, aga see nii-öelda 
protsessori kast maeti maha. Kuskile sinna Sauele, ma ei mäleta enam. 

\question{Vot nüüd see küsimus läheb teemast mööda aga see on mind huvitanud, 
ma olen ikka tahtnud küsida ja nüüd ma küsin ära. Kuidas sa laulu sisse said? 
Sa oled ainus inimene, keda ma tean, kes päriselt laulus sees 
on.}\sidenote[][-4.5cm]{Ansambli Folkmill 1996. aasta albumi \enquote{Paksult 
rahul} siiani populaarses avaloos \enquote{Madis Mäekalle Valss} on salm 

Üks talv oli see 

jube libe oli tee

Madis mütakil istuli kukkus

Aga igav oli maas

seltsiks vaid kaevu kaas

Madis ohkas ja tudile tukkus

Siis ühmatas \textbf{Raspeli Priidu}

kes kunagi ei kiskund riidu

Sa aja end Madis nüüd püsti

ja tunne end pagana hästi
}

Vaata, Lauri Saatpalu\index[ppl]{Saatpalu, Lauri}\sidenote{Ansambli Folkmill 
laulja ja käilakuju.} on hea sõber ja tal on niisugune komme, et kui tal 
millestki muust enam laule pole kirjutada, siis ta hakkab sõpradest kirjutama. 

\question{Kust sa teda tead?}
Lauriga me käisime koos EÜE-s\index{Eesti Üliõpilaste 
Ehitusmalev}\sidenote{Tagantjärele vaadates nõukogude aega oma vaimsuse, 
suhtumise ja ärimudeliga hämmastavalt halvasti sobitunud tudengite jaoks 
organiseeritud suvise töö tegemise vorm. EÜE organisaatoritest, legendaarsetest 
trubaduuridest, sõpruskondadest ja kontaktidest on hiljem nii mitmeski 
valdkonnas suuri asju võrsunud.}, me oleme Lauriga koos mitmed laulud teinud. 
Üldiselt tal tulevad sõnad hästi, aga on juhtumeid, kus ei tule ja siis ma olen 
katalüsaatorina  töötanud. Ma olen ise ka maleva jaoks laulusõnu teinud. 

Lauriga me kohtusime\ldots see oli esimene Tiirimetsa suvi, aastat ei hakka 
praegu nimetama. Kuskilt ta tuli meile sinna, hakkasime kohe hästi läbi saama 
ja ta on vaimukas inimene. Aga kõik need Serbati Tom ja 
Mäekalle\sidenote{Tegeleased eespool viidatud laulust.}  on kõik reaalsed 
inimesed.

\question{Sa oled muusikamees ka või?}

Jah. Ma olen õppinud muusikat päris palju. Hakkasin õppima kuueaastaselt, 
õppisin neli aastat muusikakeskkoolis\index{Tallinna Muusikakeskkool} suisa, 
aga siis ma sain aru, et see on see koht, kus me olema ei peaks. Sellepärast, 
et seal mitte midagi muud ei õpetatud, aga mul olid muud asjad ka tähtsad. 
Lisaks sellele ma olen natuke rutiinitalumatu nagu infotehnoloog korralikult 
olema peab. Kogu aeg peab mingi \emph{action} käima, sama pala kaheksakümnendat 
korda mängida oli natuke piinav. Aga ma ei  tahtnud seda katki jätta.  Lisaks 
sellele oli seal üks õpetaja, kes mind terroriseeris. Solfi õps oli, ta on 
kõiki terroriseerinud, aga ma olin tal eriline lemmik. Ja see lõi mu lukku, ma 
ei saanud hakkama solfiga. Ja siis ma tulin ära. Ütlesin emale, et ma lähen 
õppima laste muusikakooli\index{Tallinna Lastemuusikakool}, ma tahan klarnetit 
õppida. Seal ma sattusin Aleksander Rjabovi\index[ppl]{Rjabov, Aleksander} 
juurde, kes on Eesti džässi suurkuju, eks ole. Kuldne mees, väga hea õpetaja. 
Porrason oli solfi õps, kes oli ka täiesti  kuldne inimene. Ja, saad aru, 
selgus, et mul on kõik oskused olemas ainult et nad olid lukus. Tegelikult mul 
on absoluutne kuulmine. Mitte küll  kõige kõrgemal, aga täiesti arvestataval 
tasemel. Ja noh, see on piin ka natukene elus, eks ole, kui natukene midagi 
valesti on, kohe kratsib. 

Kuna ma viiuli  õppimise lõpetasin muusikakoolis ära, kaks aastat polnud midagi 
teha, siis ma õppisin laulmist. Nii et poistekoor pluss eraldi ansamblitunnid, 
see andis pärast kõva laulmise kooli. Ise õppisin hiljem saksi ja kitarri 
juurde, klaverit natuke. Ma ei ole nii ammu mänginud, aga klarnet ja saks on 
nii käes, et need tuleb ainult kastist välja võtta. Lisaks sellele, mul on 
kapis üks siinkandi paremaid klarneteid. Võib-olla proffidele ei olegi nii 
head. Kahju iseenesest, et  ta minu käes on, aga kuna see on kingitud pill, 
siis seda ei saa ära anda ka, et võtaks kehvema. Muuseas, selle kinkis üks 
Eesti välishelilooja, kes on sünninimega Elmar Rossman\index[ppl]{Rossman, 
Elmar} ja \enquote{Kuldrannake} on tal kirjutatud Priit Ardna\index[ppl]{Ardna, 
Priit|see{Rossman, Elmar}} nime all. Nädalavahetusel käisime just Ugalas, seal 
on tema ooperi reklaam seina peal ajaloomuuseumis. Väärt inimesed on elust läbi 
käinud. 

\question{Jaa. Aga tuleme tagasi sinu ja sinu Tehnikaülikooli juurde}
 
Tehnikaülikoolis ma sattusin selles suhtes huvitavasse kohta, et ma sattusin 
Toomas Mikli\index[ppl]{Mikli, Toomas} juurde, kellega ma hakkasin väga hästi 
läbi saama. Ta oli väga raske tüüp selles mõttes, et temaga oli raske rääkida. 
Seda suutsid suhteliselt vähesed inimesed, sest ta jättis umbes kolm loogilist 
taset vahele ja alustas neljandalt ja sa pidid ise puuduvad kihid vahele 
ehitama ja ma suutsin seda. Nii me hakkasime Tomiga hästi läbi saama. Ja noh, 
tema oli ka see, kes mind andmebaasidest suutis  innustada tegelikult, tema oli 
mu diplomitöö juhendaja. Diplomitöö, muuseas, oli meil 300 lehekülge paks. 

Jälle, võib natuke uhkustada, et keegi selle peale ei mõelnud, et mul oli üks 
töö osa pühendatud konsultatiivinfole ehk \emph{help}-tekstidele. Aastal 1984. 
Mul oli töökoht, kus ma tegelesin metoodilise palgaarvestusega, kus ma muu 
hulgas kasutajat õpetada ka, mismoodi süsteem töötab ja teda juhendada. 

\question{See oli täitsa innovatiivne mõte tol ajal!}

Tollel ajal keegi sellest suurt ei rääkinud, midagi veel, ma ei mäleta, kust ma 
selle üles korjasin. Kuidagi Tomiga mingi vestluse käigus korjasin üles, 
mõtlesin, et peaks ära tegema, ja siis ta tuli. Mudel, töö. Muidugi, selle 
diplomitöö käigus, ma sain jälle ühe asjaga hakkama. TTÜ-s kasutati 
SETOR\sidenote{Varastel kaheksakümnendatel liikvele läinud TOTAL 
andmebaasisüsteemi kloon ES-ide jaoks.}, see on siis andmebaas, mida ülejäänud 
maailm tunneb nimega Total\sidenote{Ka TOTAL. 1968. aastal asutatud Cincom 
Systems Inc.-i andmebaasimootor, mis oli esimene omasuguste seas}. Arvutitel 
oli mälu vähe, 256 K, millest üle jäi 16 kilo puhvrisse, kui sul kuvarid taga 
olid ja nüüd ei mahtunud enam kompilaatorid ja linkurid mällu ära. Diplomitöö 
käigus ma kirjutasin skripti, mis vaatas su programmis järele, milliseid teeke 
sul vaja on ja linkis ainult need teegi osad sinna külge, mida tõepoolest vaja 
oli. Nii oli võimalik 16K-ga hakkama saada. Veel ma mõtlesin välja puhverdamise 
süsteemi, kuidas läbi puhvri erinevaid mooduleid omavahel siduda, sest korraga 
suur tükk ei mahtunud mällu, eks ole. 

Tom pani kokku grupi, kuhu kuulusin mina, Mart Roost\index[ppl]{Roost, Mart} 
(praegu väga tunnustatud õppejõud, dotsent vist on ta), Lea 
Elmik\index[ppl]{Elmik, Lea} ja Tiiu Lumberg\index[ppl]{Lumberg, Tiiu}. Ma ei 
teagi, kus Tiiu praegu töötab, aga temaga me hiljem käisime koos 
maailmameistrivõistlustel. Meid hakati kutsuma \enquote{Mikli noorteks 
ekstremistideks}.Me  kõik kirjutasime oma teaduslikku tööd, aga me ei teinud 
kunagi midagi nii, nagu teised teevad. 

Moskvas oli üks kaval juut Tjomov, kes istus mingis instituudis, 
Iskra-226\index{Iskra!Iskra-226} peale, mis oli laetava Basicuga arvuti, 
kirjutanud opsüsteemi \enquote{Skoropis}, kiirkiri. Esimene viitadega, keel, 
mis ma nägin. Noh, nagu tänapäeva viidad on, eks ole. Aga ta oli selles suhtes 
kihvt, et tal \emph{time sharing} oli sisse ehitatud ilusti. Programmi täitmine 
käis nii, et sa tõmbasid programmi stringi, panid viida peale ja ütlesid, et 
selle viida järgi hakkad nüüd täitma. Aga mälu oli jälle vähe, 64 K, aga meil 
oli sellesama Lindre Reinuga\index[ppl]{Lindre, Rein}, keda ma seal 
arvutisaalis tundma õppisin, et paneks Iskrale veel kaks kuvarit külge. Et mina 
kirjutan opsüsteemi ringi, tema teeb kaks videokaarti, paneme Videotoni kuvarid 
taha ja vaatame, kas paneme sellesama masina tööle.  Selleks ma tegin 
\emph{overlapping}-u ära selles mõttes, et kui ma tundsin ära, et programm on 
juba mälus, siis panin teise viida veel ja panin ta veel tööle. Programm visati 
välja alles siis, kui viitasid enam ei olnud, tollel ajal ei olnud midagi 
sihukest. 

\question{Mingi mälukaitse või turve või midagi ei olnud probleem?}

Ei. Kogu infoturve oli see, et sa ei saanud masinat käimagi, flopi oli välja 
võetud ja tuba käis lukku. Kuigi oli ka  kahe-megane ketas, nägi välja selline 
suur valge \emph{baraban}, mul on kaks tükki praegu sihukeses kapi otsas, üks 
Iskra ja teine SM-i oma. 

Mul on terve muuseum. Kapi otsas on mul lint, siis on mul olemas kolmesajane 
modem (Nightingale on nimi, laksutab nagu ööbik). Siis mul on olemas üks 
esimesi läpakaid, mis Eestisse tuli, Siim Kallase\index[ppl]{Kallas, Siim} oma, 
mis oli tal siis, kui ta panga president oli. Arvelaud, lükat, kaheksa tollised 
flopid, viietollised fopid, kolme tollised flopid, magnetoptilised kettad, 
ühesõnaga kõik, mis elust on külge jäänud. Kõige vanem eksemplar on pärit 
aastast 1936. Taskukalkulaator.

\question{Felix?\sidenote{Nõukogudemaal aastatel 1920-2970 toodetud 
mehaaniliste kalkulaatorite sari. Nende tootmise algatas Nõukogude 
julgeolekuteenistuse asutaja Feliks Edmundovitš Dzeržinski, mistõttu laienes 
tema hüüdnimi \enquote{Raudne Feliks} ka kalkulaatoritele.}}

Feliks on ka. Aga see on taskukalkulaator, mis on numbrilise näiduga, liidab ja 
lahutab, mehhaaniline, on umbes kolm millimeetrit paks ja umbes kuus korda 
kümme sentimeetrit suur. Sakslaste tehtud, pulkadega liigutad seal neid asju, 
see on kõige vanem asi. Minu vanaise kinkis selle minu isale kuuendaks 
sünnipäevaks ja minu isa kinkis selle mulle. 

\question{Kas sa Tehnikaülikoolis mingit teadust ka tegid?}

Jaa, ma hakkasin tegelema andmeedastusega, aga tulid segased ajad, raha sai 
otsa ja see jäi seisma. Tegelesin sünkronisatsioonimudelitega, millega ma olen 
elus tegelikult hiljem väga palju tegelenud ja praegu võiks kirjutada sihukese 
töö, mida keegi pole kunagi välja mõelnud. Aga muud huvid on tekkinud ja\ldots 

\question{Sünronisatsioonimudelid?}

Põhimõte on selles, et kui sul on kaks infosüsteemi, siis millist mudelit 
kasutada, et kõige odavamalt välja tulla ja mismoodi see automaatselt käima 
saada, et nad süngis oleks. Tollel ajal ma mõtlesin välja ühe termini, mida (ma 
avastasin rõõmuga) on  tänapäeval  kasutama hakatud, isegi reklaamides. See on 
\enquote{automaagiline}. Jaak Tepandi\index[ppl]{Tepandi, Jaak} kunagi küsis, 
kui ma ühel konverentsil seda kasutasin, et \enquote{Priit, kuule, mida sa 
mõtled sõnaga automaagiline}. See on asi, mis muutub automaatselt, aga ma ei 
tea täpselt, mismoodi. Ja nüüd ma olen kuulnud, et reklaamidest kasvatatakse 
seda. 

Aga mis seal oma rühmaga tegema hakkasime, miks rühm üldse tekkis 
Reinuga\index[ppl]{Lindre, Rein} oli see, et AutoVaz, Žiguli autotehas, tuli 
meie juurde ja ütles, et teeme lepingu, tehke meile süsteem väikejaamade jaoks. 
Ladu, remont, see värk, eks ole. Me ütlesime, et me teeme küll, aga omamoodi, 
et meil peab teadus sees olema. Me võime ju kirjutada küll teile ükskõik 
missuguse süsteemi, aga teadus peab sees olema. Mis me siis tegime, tegime 
neljakesi süsteemi, mille loogikat poleks tänagi häbi näidata. Ta oli küll must 
ekraan ja tead sihuke värk. Mart\index[ppl]{Roost, Mart} kirjutas nullist 
andmebaasi mootori. Tiiu\index[ppl]{Lumberg, Tiiu} kirjutas vormi generaatori 
nullist. Lea\index[ppl]{Elmik, Lea} kirjutas raporti generaatori nullist. Ja 
mina kirjutasin süsteemi arhitektuuri kirjelduse,  nullist, ja mõtlesin välja 
tollel hetkel XML-i. Suurem-väiksem märgi asemel olid kandilised sulud ja 
\emph{slash}-i asemel oli sõna \enquote{END} aga keel oli sama. Selle kohta on 
tõestus olemuse ühes TPI kogumikus, kus mul on selle kohta artikkel. 

Nii et mul oli väga lihtne tõsta asjad seal keeles ringi ja süsteem hakkaski 
teistmoodi menüüsid ehitama ja igasuguseid küsimusi küsima. Otsustuskohad ja 
kõik.

\question{Kas nad võtsid süsteemi kasutusele ka?}

Jah me kasutasime seda AutoVazi jaoks ja  kuskil veel. Ma olen sedasama ideed 
kasutanud hiljem mujalgi.

\question{Ühel hetkel sa sukeldusid pangandusse?}

No see selleks nüüd natuke pikemat teed pidi, seal vahepeal on veel see 
maailmameistrivõistlus. 

Ühesõnaga, ühel hetkel juhtus selline asi, et raha sai otsa ja palka sai 
TPI-st\index{Tallinna Tehnikaülikool} nii palju, et kui auto oli olemas, 
jaksasid autoga tööl käimiseks bensiini osta. Ja tekkis niisugune huvitav mees, 
kes praegu, ma ütleksin niimoodi, elab Euroopas kodutuna. Ta tahabki seda, see 
ei tähenda, et halvasti elaks, ta lihtsalt rändab ringi. See oli tema elu 
unistus olla vaba. Kirjutab mobiiliäppe pargi nurga peal. Kui kuskil on 
põllumajandusperiood, siis läheb põllumajandusse tööle ja aeg-ajalt paneb 
feissarisse, kus ta käinud on. Ma jälgin, mis ta seal teinekord teeb. Veiko 
Herne\index[ppl]{Herne, Veiko}. Välimuselt minu täielik vastand: sihuke 
pisikene, ümmarguste prillidega ja väga kõhetu. Ma ei mäleta, kuidas me temaga 
kokku saime, aga ta kutsus, et \enquote{teeks tarkvara natukene}. Ta ise on 
tegelikult geniaalne programmeerija. Aga sihuke orgunn-meister ka. Kutsus, et 
\enquote{ma just tegin siin OÜ Tarkvara. Ma annan kolmandiku osakuid sulle, see 
ei maksnud midagi.}.

\question{Kui on juba OÜ, siis peab ajahetk olema 1991 ja edasi?}

Seal ta kuskil oli jah.

Ja siis ta ühel päeval tuleb ja ütleb, et \enquote{kuule, hakkame tõsiselt 
tegema, et ma  leidsin Microsoft Magazin-i sabast ühe süsteemi. 
Gupta\index{Gupta} SQLBase on nimi\sidenote{Tegu oli esimese klient-server 
relatsioonilise andmebaasiga, mis jooksis PC platvormil, mitte miniarvutitel.}. 
Pakuvad, et tulge ja hakake esindajaks, et ma lähen korra nüüd Inglismaale}. Ma 
olin TPI-st selleks ajaks  otsad juba lahti võtnud. Üks asi, millega me raha 
teenisime oli see, et programmeerimise Robotroni nõelprintereid ümber eesti 
tähestiku peale. Toodi kivid, programmeerisime ringi. Sealt sai juba natukene 
algkapitali sisse. Ilge raha läks kogu selle Inglismaa-sõidu ja lansseerimise 
peale, 100 000 rubla. Aga kuskilt mingeid lepinguid me tegime, selle kokku 
kraapisime. Igal juhul Veiks tuli tagasi, ja olime ametlikult volitatud 
esindajad. Esimene klient-server süsteem, mis  ametlikult Eestisse toodi. Ei 
olnud veel ei Oracle-t, Cybase-i ega mitte kedagi. Tegime seminari. Ainult 
vilistada tuli, terve paganama Küberi amfiteater oli nii täis, et inimesed ei 
mahtunud ära. 

Tegime lepingu Põlva Piimaga\index{Põlva Piim} ja Võrus vist oli see 
eksperimentaalne õmblustootmiskoondis, tegime nendega süsteemide 
arenduslepinguid. Uurisime süsteemid välja, panime andmebaasid käima.  Põlva 
Piim oli väga suur projekt, seda me ei kannatanud enam kolmekesi ära, võtsime 
Andres Lombi\index[ppl]{Lomp, Andres} ja IE-tarkvara\index{IE-Tarkvara} appi 
omale programmeerijataks. 

Juhtus selline lugu, see oli otsustava tähtsusega maailmameistrivõistlustele 
jõudmisel, et  mul on õudselt hea nina igasuguste vigade peale. Nad kohe jäävad 
mulle näppu. Ja ma leidsin sealt SQLBase-st ühe jõle laheda vea, et kui sul 
olid \verb|IN| ja  \verb|NOT IN| tingimus järjest, siis täitusid suvalised 
tingimused. Ja kui sa panid sinna \verb|1=1 AND| vahele, siis hakkas tööle. 
Vennad ei uskunud seda minu juttu ja kaks venda tulid suisa kohale. Me 
korraldasime  ruttu seminari ja panime nad sinna esinema. Rääkisime nendega 
juttu ja ma näitasin neile oma asju, mis ma teinud olin, mismoodi seal nende 
süsteemi kasutanud oleme. Saime täitsa \enquote{kuuma liini} nendega, jube 
hästi läks. Ükspäev Veiko\index[ppl]{Herne, Veiko} ütleb, et \enquote{kuule, 
Gupta\index{Gupta} otsib, et kes läheks neid esindama 
maailmameistrivõistlustele 4GL programmeerimises} See oli juba 
objektorienteeritud keel, milles me kirjutasime. Alguses ei olnud, aga selleks 
hetkeks 1993. aasta juba oli. Et \enquote{mis sa arvad, lähme? Kolmele 
võistkond, võtame kellelegi veel ja lähme!}. Guptast öeldi, et \enquote{te 
olete nii kõvad vennad küll, minge. Aga ise peate kohale jõudma oma 
arvutitega}. No kuidas me need arvutid sinan Rootsi saame! 

Aga oli selline väga tark soome poiss nagu Pauli Visuri\index[ppl]{Visuri, 
Pauli} kes müüs Olivettisid, tema Muniveti esinduste kaudu sai nii kaugele, et 
üks Rootsi Olivetti esindaja tõi meile tuttuued masinad sinna messiboksi. Meie 
lihtsalt sõitsime lennukiga kohale. Tahtsime ööbida mingis tagasihoidlikus 
kohakeses aga Gupta ütles, et \enquote{ei, meie meeskond ööbib ainult Kung 
Carl-is}, korralik vana hõnguga hotell, eks ole, \enquote{meie maksame selle 
teil kinni}. 

Läksime sinna, esimesed Suprema masinad, 66 megahertsi. Esimesed 
multimeediamasinad, mida  ma nägin, pandi üles. Plug-n-Play Windows 3.11, 
esimest korda nägin. Hakkame installima. Ei tule, hiir ei lähe külge. Mõtlesin, 
et seal on Microsofti meeskond. Läksin, panin Microsofti meeskonnas käed puusa: 
\enquote{see teie opsüsteem on igavene pask! Plug and play aga hiired külge ei 
lähe!}. Kaks venda tuli, istus meie masina taha, ja hea oli vaadata kuidas 
inifailid\sidenote{.INI laiendiga failides hoiti Windowsi platvormil 
koventsiooni kohaselt programmide konfiguratsiooni.} lendavad näppude alt 
välja. Lasevad, lasevad, lasevad, üks masin läks käima. Ajasin nad minema, 
kopeerisin ini failid kõikidesse masinatesse ja oligi  korras.  Veiko hakkas 
proovima häältuvastust, mis just oli välja tulnud. Aga kuna halli helifoon oli 
väga kõva, siis ta karjus oma arvutile peale \enquote{õupen, õupen, õupen, 
klõus, klõus, klõus, ran!}. Järsku teiselt poolt seina kostis hääl 
\enquote{clear all!}. Küllap ta siis käis närvidele. 

\question{Mis ülesannet te lahendasite?}

Ülesanne oli tegelikult vahva. Ülesanne oli kirjeldatud stiilis, et 
\enquote{Kass ärkas, sirutas, hüppas ja sattus klaviatuurile. Arvuti tegi 
piiks, kass tegi näu ja selle  peale ärkas üles tema perenaine Celia. Celia, 
mõtles, et  täna on kolmapäev, mis ma olen tellinud, mis kaubad peaksid täna 
tulema ja mis mul veel tellida oleks vaja}. Siis räägiti Peterist, kes istub 
keskses laos ja paneb kaupu liini peale, veel on Larry, kes sõidab 
\emph{lorry}-ga ringi ja veab  kaupu laiali. Klassikaline niisugune logistika, 
veebikaubandus, eks ole. Tollel ajal siukest asja veel ei olnud. Meile anti 
ette  kaart,  GPS signaal ja nüüd me pidime programmeerime auto armatuurlaua 
koos GPS-i liigutamisega ja tsentrumi. Millega meie teistest erinesime: Ma ei 
viitsinud seda kuradi igavat ladu programmeerida ja ma tegin keskele 
logistikakeskuse, kus laod olid eraldi. Ladusid imiteerisime omaette failidest 
ja  Peter oli lihtsalt see, kes võttis tellimusi vastu ja jagas neid. 
Lõpetamisel istusime žüriiga ühes lauas, need see peale ütlesid, et 
\enquote{kurat, mingid postsotsialistlikud vennad tulevad meile kapitalismi 
õpetama!}. 

Võistlus kestis 24 tundi. Hakkas ühel päeval kell kolm, lõppes teisel päeval 
kell kolm, seejärel hakkasid järjest esitlused tulema. Meie saime esitlema 
alles kell kümme õhtul, nii et me olime 24 tundi üleval olnud. Ja muidugi noh, 
ütleme, võistlus oli ise niimoodi, et mina kirjutasin kogu koodi ja 
projekteerisin peas asjad. Veiko suhtleja, süstematiseeris mu küsimused ja tõi 
žürii käest vastused. Tiiu joonistas vorme ja tegeles kogu kasutajaliidesega. 

Alustasin 15:20 ja kell viis öösel läksid näpud krampi. Umbes pool tundi näpud 
ei liikunud, siis läks uuesti lahti. Mingi psühholoogiline tõrge oli. Selle 
ületamiseks läks pool tundi, tegime edasi, kaks tundi enne tähtaega saime 
valmis. Muidugi, igavene jama tekkis  hommikul kell kaheksa, kui  mingisugune 
mäki meeskond lõpetas. Mõtlesin, \enquote{ma olen ikka jube sant mees}. Lõpuks 
selgus, et nad katkestasid. Seal olid magamisruumid, kui sa tahtsid, kasepuust 
lavatsid. 

\question{Kes korraldas sihukest üritust?}

Korraldas mingi Rootslaste, ma ei mäleta, kes. Mingisugune softi liit. 

Ma teadsin sellest, mul üks sõber, Kalle Kulmann\index[ppl]{Kulmann, Kalle}, 
Tartu EPT juht, oli seal kunagi ülesande tegijana osalenud see. Temaga me saime 
kokku Marksistliku-leninliku kommunismi kandidaadimiinimumi täiendusloengus. Ma 
teadsin ta nime ja ei midagi muud. Seal loengus me istusime kõrvuti. Loengut 
luges selline vend nagu Otto Stein\index[ppl]{Stein, Otto}. Teda kutsuti Otto 
von Steiniks, ta oli saadetud Tartust Tallinnasse kommunistlike filosoofide 
kaadri tugevdamiseks. Mispeale mõlemad tugevnesid. Hull vanamees oli.

Ma mäletan siiamaani, et loeng oli \enquote{Kommunismi on kolm allikat, kolm 
komponenti}. Need on Inglise poliitökonoomia ja Saksa utopism ja \ldots, 
ühesõnaga seal on need kolm asja, eks ole\sidenote{Kommunismi kolm allikat 
toonase õppe järgi olid saksa klassikaline filosoofia (peamiselt Georg Wilhelm 
Friedrich Hegeli ja Ludwig Feuerbachi järgi), inglise poliitiline ökonoomia 
(Adam Smith, David Ricardo) ja prantsuse utopistlik sotsialism (Claude Henri de 
Saint-Simon ja Charles Fourier). Neid \enquote{arendasid edasi} 
marksismi-leninismi kolm komponenti: dialektiline ja ajalooline materialism, 
poliitiline ökonoomia ja teaduslik kommunism.}.  Läheb esimese juurde: 
\enquote{Öelge esimene, nii, õige}. Teise juurde: \enquote{Öelge teine}. 
Kolmas: \enquote{Öelge kolmas}. Ja nii edasi ühe inimese juurest teise juurde: 
\enquote{Teine, esimene, kolmas, teine}. Kui tiir minuni jõudis, tõusin püsti 
ja ütlesin \enquote{Teate, mina selles tsirkuses ei osale} ja jalutasin välja. 
Kui Stein püüdis asja leevenda ja ütles Kallele, et \enquote{no öelge siis 
teie}, ütles Kalle \enquote{mina ka mitte} ja tõusis samuti püsti. Läksime 
välja, istusime Tuljaku baari maha, ajasime juttu ja oleme siiamaani suured 
sõbrad.

\question{Aga kuidas ikkagi pank\ldots}

Nüüd oli niimoodi, et olin niisugune vabakutseline häkker. 91. aastal oli 
elurütm ikkagi selline, et päevarütm oli täiesti sassis võrreldes teistega ja 
tööpäevad on 72 tunnised pärast mida ma sõidan autoga Valka ja Põlvasse asju 
üle andma. Mul oli kaks last juba, Anna oli just sündinud, ja selgus, et 
selline töörütmi ei klapi enam. Ja seesama Tiiu\index[ppl]{Lumberg, Tiiu}, 
kellega me koos käisime maailmameistrivõistlustel ütles, et 
\enquote{Innovatsioonipank\index{Innovatsioonipank}\sidenote{18. 
septembril 1989. aastal ENSV Ministrite Nõukogu Presiidiumi otsusega number 21 
Eesti NSV riigieelarve \enquote{üle plaani laekunud tulude} arvel asutatud 
pank.} otsib IT juhti}. Sihuke pank, ta pankrotipesana oli veel mõned aastad 
tagasi olemas. Juhiks oli niisugune mees  nagu Peep 
Sillandi\index[ppl]{Sillandi, Peep} pärastine mikro- ja makroökonoomika 
õppejõud EBS-is, kes õpetas tudengeid softi peal mudeleid koostama ja tegema. 
Niisugused mängud ja värgid. Peep oli üks igavene lahe kuju, läksin temaga 
rääkima ja jutt klappis kohe ja. Peep ütleski, et \enquote{Homme siis tuled 
või? Näe, tool on siin}. Mõtlesin, et olgu peale. 

Ja nii ma olingi IT-juht. Hommikul tuli minna poole üheksaks tööle, harjuda 
sellega ära, et ei saa kella kolmeni öösel üleval olla (siiamaani ei ole ära 
harjunud). Üsna üsna tihti ka praegu on mu magamamineku aeg kell kaks, kell 
seitse üles. Viis tundi on minu jaoks \emph{enough}. 

\question{Räägi palun sellest, mida kujutas endast 91. aastal sihukese väikese 
panga IT-juhi töö. Mis sa tegid?}

Oh igasuguseid asju. Kui esimene päev uksest sisse tulin (ma olen muuseas seda 
pärast igas ettevõttes teinud), istusin maha ja mõtlesin, kust mind on võimalik 
vangi panna. 

Hakkasin otsima seda kohta ja leidsin. Tollel ajal käis keskpangaga 
informatsiooni vahetus mingi programmiga. Ma ei mäleta, see selle programmi 
nime, pean kuskilt otsima, see on päris huvitav fakt. Väidetavalt mingi 
armeenlane kirjutas, kes oli kõik juuksed peast ajanud, et kammimiseks aega ei 
läheks ja kuna suhkur on ajutoit, sõi ainult suhkrut, geniaalne vend. Tal oli 
nii käsuliidese kui dialoogiga käiv suhteliselt pisike programm, mis natuke 
krüptis ka. Küll väga vähe, tolle aja kohta kõvasti ilmselt, ja saatis maksed 
panka ära. Iga päev tehti pangas kaks faili, üks hommikul ja teine õhtul. 
Õhtusest failist hommikuni käibed, mis tuli keskpanka saata ja teine oli 
teistpidi. Igal pangal oli oma aeg, millal ta failid ära pidi saatma, samal 
ajal sai teistest pankadest tulnud asjad vastu. Ja see asi seisis vabalt. Ja ma 
sain kohe aru, et panka saab röövida niimoodi, et ma ei võta kellegi kontolt 
raha ära, vaid tekitan sellesse kanalisse raha juurde. Siis ei hakka keegi 
selle järele igatsema, natuke aega tuleb ainult nostro ja vostro kontode sisu 
varjata, et ei oleks näha, et seal mingisugused jamad on tekkinud, aga sellega 
saab hakkama. Esimene asjana tegime nii, et sellesse kohta sai faile panna  
ainult üks konkreetne programm, teine sai faile võtta ja kui keegi sellesse 
piirkonda sisse logis, katkestati kõik ära. No näiteks selline asi. Siis  
pangakontorite vahelised ühendused, jälle igavene jama, eks ole. Kuidas saada 
Mustamäele kontor püsti nii, et see meie süsteemiga kokku saaks. Istusime ja 
testisime mingeid seadmeid, ehitasime ja proovisime. Telefoni kanal ju ei 
püsinud. Ja nii edasi, päevalõpu teemad. 

Näiteks, triviaalne asi. Ühel päeval tuleb teller minu juurde,  ütleb, et 
\enquote{kuule, Priit, klient küsib oma käivet sellest ajast, seda ei ole}. 
Hakkama siis uurima, selgub, et süsteemis on niimoodi, et kaks aastat vanad 
käibed hävitatakse ilma küsimata ja pikemat aega panna ei saagi. 

\question{Mis too panga tuum siis oli, mis niimoodi tegi?}

See oli Midas Kapiti süsteem Kapiti, istus AS/400\index{AS/400} otsas. 
Aga kuna meil AS/400-t ei olnud, siis meil käis OS/2 Warp\index{OS/2!OS/2 
Warp}\sidenote{OS/2 oli IBM-i poolt arendatud personaalarvutite 
operatsioonisüsteem. OS/2 Warp oli selle kolmas versioon, mis tuli turule 1994. 
aastal.}, mille peal istus AS/400 emulaator ja mille sees käis panga tuum. 

\question{See oli siis ostetud kuskilt, eks?}

Jaa. Kapiti käest. Midaseks läks ta hiljem, alguses oli ta Kapiti. No mida 
Priit tegi, istus maha,  poisid hakkasid sortima \emph{backup}-e (neid me 
tegime hoolega) ja kirjutasime siukse softi,mis lappas \emph{backup}-idest 
kokku andmelao. Me ei teadnud, et see asi andmeladu on. Sellesama SQLBase 
peale, see oli pangal ka olemas. Kui ma panka läksin lasin osta, see oli hea 
klient-server lahendus, lihtsalt kättesaadav ja ei olnud väga kallis võrreldes 
teistega. 

Lisaks sellele ma olin panga nõukogu sekretär. Peep arvas, et ma oskan 
piisavalt loetavalt kirjutada küll ja pärast puhtaks lüüa ja saan asjast aru 
ka. Ütles veel, et \enquote{ega sa liige ei ole, aga arvamuse saad ikka sekka 
öelda}. Noh, eks ma olin niisugune hääleõiguseta arvamusliidesr seal nõukogus.

\question{Aga see on ju IT-juhile väga praktiline koht, sa saad info kätte!}

Just nimelt, jah, just see oligi Peebu mõte ja see oli jube hea mõte. Nõukogus 
olid väga vahvad liikmed, kellega ma tuttavaks sain. Näiteks Arvo 
Kallion\index[ppl]{Kallion, Arvo}, omaaegne me ei tea, mis mees ta oli, 
partei-boss, valitsuses keegi, aga väga tark mees. 

Innovatsioonipank oli taskupank. Genin\index[ppl]{Genin, Alex}\sidenote{Alex 
Genin, Innovatsioonipanga nõukogu esimees.} oli niisugune juut Ameerikast, kes 
elas sellest, et tegi erinevates riikides pankasid, ajas nad riigi süül 
pankrotti ja siis hakkas kahjutasu nõudma. 
Sotsiaalpank\index{Sotsiaalpank} läks pankrotti, sealt pankrotipesast 
ostis ta kõige vingema kontori ning sellel tulvil lasi panga põhja. Aga ma 
nägin seda. Ma peaaegu igal õhtul nägin pagana bilanssi ja ma olen 
majandusharidusega. Ma nägin, et see läheb. Selleks ajaks Peepu enam ei olnud 
sest Genin oli oma Miša (ma ei mäleta Mihhaili perekonnanime) panga etteotsa 
pannud. Tore poiss muidu aga tegi täpselt, mis Genin ütles. Läksin Miša juurde, 
panin avalduse lauale ja ütlesin, et \enquote{Miša, ma lähen nüüd ära.}  
\enquote{Aga miks sa lähed?} \enquote{See pank läheb varsti pankrotti.} 
\enquote{Sa eksid!} \enquote{Ei eksi, Miša, ma olen majandusharidusega ja ma 
näen iga õhtu bilanssi, see läheb varsti pankrotti.} 

Mul oli on uus koht olemas, Tööstuspank\index{Tööstuspank}. Üks 
laenujuht läks sinna ja kutsus mind arendustiimi juhiks, et neil on uut 
infosüsteemi vaja. Ja mina ei saanud aru, mis toimub. Kõik mis küsin, kõik 
saan. Oma kontor ehitati Koplisse. Magamisruumi, köögi ja kõikide asjadega, 
kõik mis küsisin, kõik sain. Tirisin sealt pangast Eeriku\index[ppl]{Matt, 
Erik}\sidenote{Erik Matt.} ja  Raivo\index[ppl]{Tali, Raivo}\sidenote{Raivo 
Tali} ka kaasa veel. Kuskilt tõin ära ka Ville Remmeri\index[ppl]{Remmer, 
Ville}.

Kuus kuus uurisime ja puurisime ja kui meil oli kõik valmis, siis öeldi mulle 
\enquote{Tead, me valetasime sulle. Me ei kutsunudki sind siia uut süsteemi 
tegema. Meil on siin üks IT-osakond, aga me ei usalda neid}. Seal IT-osakonnas 
oli igasugu tegelasi, Poldzadze oli juht, nägi välja nagu Kirgiisi bai. Ja 
panga juhtkonnal oli vaja inimest, kes teaks, mis panga IT-s toimub, sest nad 
hakkasid just Hoiupangaga kokku minema. 

Öeldi \enquote{Sa võtad nüüd IT juhtimise üle}. Ma sain omale ametinimetuse, 
mis on kõige uhkem ametinimetus, mis mul kunagi olnud on: Panga Esimehe 
Volitatud Erisindaja IT küsimustes. Mul oli õigus käskida, puua ja lasta. See 
IT oli selline sotsialistlik kamp, neil oli vaja ülemuse nime. Mul oli õigus 
teha ükskõik mida, peaasi, et pank püsti püsiks ja midagi ära ei läheks. Ja 
ühel päeval läks kuus pool miljonit krooni minema. Täpselt sedasama pidi, mida 
ma arvasin. 

Sellel päeval, kui ma võimu võitsin, mul ei olnud veel võimalik midagi teha. 
Tõnu Liik\index[ppl]{Liik, Tõnu}, kes oli Hoiupanga ja IT-juht tuli sinna, Ants 
Leitmäe\index[ppl]{Leitmäe, Ants} oli kaasas. Ants istus aknalaua peale, kuulas 
tuima näoga pealt, tema pidi oma tiimiga tehniliselt toetama hakkama, sest ma 
ei teadnud kedagi usaldada, eks ole. Kutsuti kõik kokku, Tõnu teatas, et 
\enquote{Uus juht on nüüd siin}. Ja kõik läks ilusti. Esimese asjana sai kõigil 
kasutaja õigused ära võetud, panin oma poisid masinate taha ja hakkasime uuesti 
õigusi õigetesse kohtadesse tagasi andma. Õhtul istun Irina juures (ei mäleta 
perekonnanime, pearaamatupidaja, kes toodi ka enne liitumist majja), ajasime 
juttu.  Irina võtab lahti selle tagasi tulnud hommikuse kontrollsummade faili 
ja teeb istest meetrise hüppe üles. Kuus pool miljonit oli teisiti pankade 
vahel jagunenud, kui oli hommikul välja läinud. See tähendab, et kuus pool 
milli oli läinud Rakvere Maapanka\index{Rakvere Maapank}. Egas midagi, 
joostes minema, kogu IT osakond puhtaks, arvutid tuleb lahti jätta. Oma poisid 
peale ja hakkame kontonumbri järgi \emph{search}-i tegema. Ja leidsime ühest 
masinast. See, et me ta avastasime oli tänu sellele, et ma kõigil hommikul 
õigused ära võtsin. Tal oli kontrollsumma muutus ka tehtud, aga ta ei saanud 
õhtul enam vajalikule kohale ligi. Irina helistas kohe Rakverre, blokkis summa 
ära, järgmine päev saime tagasi. Sest õhtul pangad olid kinni, pangaautomaati 
ei olnud ja pangad pandi kella neljast juba kinni. Nad järgmisel hommiku 
plaanisid hakata tegutsema.  

Vend võeti kinni, viidi raudus minema. Masina panime raha koti, pitseerisime 
kinni, panime seifi. Järgmine päev, manukate juuresolekul tegime lahti, võtsime 
ketta, tegime kolm koopiat. Üks läks TTÜ-sse analüüsimine, üks läks Hoiupanga 
tiimile ja üks tuli minu tiimile. Seal masinas oli peale kontonumbri veel üks 
klimp, parooliga zip-itud. \enquote{Mis \emph{password} on?} \enquote{Ei, ma ei 
tea, kui te mulle ütleksite, mul oleks endalgi huvitavaid asju seal sees}. 
Läksime kontorisse, installisime murdmisklastri. Läksime 
Raivoga\index[ppl]{Tali, Raivo} suitsu tegema, tuleme tagasi, 
Erik\index[ppl]{Matt, Erik} ütleb \enquote{Poisid, vabandust, te oleks ka 
kindlasti seda näha tahtnud. Ma lasin prooviks käima, sõnastiku alguses 
\emph{konjak} väikse tähega\ldots}. Raivo ütles selle peale \enquote{Jube madal 
profiil, ma oleks vähemalt Mercedes-Benz  pannud}. 

Seal failis oligi kogu värk sees. Süsteem oli Foxis, andmebaas oli DBF, siis 
oli tehtud üks \emph{browse}, mida klikkisid, jäeti meelde, lõpuks, kui F10 
vajutasid, siis kanti kõik ühe konto peale kokku ja tehti fail valmis. Tehti ka 
kontrollsumma fail juba valmis nii et see oleks õhtul lihtsalt õigesse kohta 
tõstetud. 

\question{Nii et sulle köögi ehitamine tasus kohe esimesel päeval ära}

Ma ütlen sulle, et siin ei ole mitte midagi oodata, mitte kunagi. Kohe juhtub.

Hoiupank ostis Eesti kindlustuse ära, läksin sinna IT-juhiks. Seal oli IT-s 
kaks ja pool meest, kellest ühte ma ei näinudki. Oli osakonna juhataja, kellel 
olid mingid probleemid, ja kui ta kuulis, et on uus IT-juht, siis ta ei 
tulnudki enam. Iseenesest geniaalne mees oli, oli mitmeid matemaatikaõpikuid 
kirjutanud, aga mingid probleemid olid. Võtsin oma poisid, Ville 
Remmer\index[ppl]{Remmer, Ville}, Erik Matt\index[ppl]{Matt, Erik} ja Raivo 
Tali\index[ppl]{Tali, Raivo}, kaasa ja kolistasime sinna. Leidsime mingi 
õuduste maa. Katastroof kuubis. Või okse kolme x-iga, ma ei tea, kuidas seda 
öelda. Näiteks elukindlustus käis niimoodi, et paberi peal korjati dokumendid 
kokku ja need läksid kümne fakiir-sisestaja kätte. Neil oli ekraani peal triip, 
kus olid postid vahel ja sinna triibu vahele nad sisestasid andmed mille pealt 
siis tehti reservi arvutusi ja kõike muud. Postivõrk käis, venelased kutsuvad 
pastlavõrguks. Postikott on ju ametlik dokument. Kõik raportid pandi posti, 
saadeti Tallinna, siin sisestati ära, trükiti raportid välja, pandi postikotti 
ja saadeti tagasi. Ma panin Ville kiiresti kirjutama lokaalset kasutajaliidest. 
Raivo ja Eerik hakkasid otsima võimalusi, kuidas side ära teha kõikide meie 
kontoritega. Ise kadjasin  esimesed nädalad mööda kõiki esindusi, mul igas 
nädalas oli üks päev, kus ma sõitsin teatud hulga esinduse läbi, et uurida, mis 
probleemid seal on. Ühe tehnikutest võtsin kaasa, et ta vaataks tehnilist 
poolt. 

See kõik juhtus augustis. Mulle lubati viis päeva puhkust kahe töökoha vahele, 
nii kiire oli asjaga. Uue süsteemi tõmbasime käima kuskil jaanuaris. Selleks 
ajaks olid meil kõik ühendused tehtud, kõik korras, uus server, selle jaoks 
soft töötas, inimesed koolitatud. Iga agent hakkas ise oma asju sisestama.

\question{Mis aastal see oli?}

1996 umbes. 

Nüüd tuli hakata muud süsteemi uuendama aga võrk oli kohutav, eks ole. 
Tellisime võrguehitustööd, ehitasime korraliku serveriruumi, panime talle 
tulekindlad materjalid seina, jahutused väljapoole ja kõik. Selle 
serveriruumiga on üks vahva jutt veel. Ta olid maja keskel kõige paksemate 
müüride vahel, lasin sealt WC ja duširuumid ja mis seal oli koos torudega välja 
lõhkuda. Ja siis ühel päeval, see oli vist märtsis, juhtus niisugune lugu. Kui 
me hakkasime üle viima oma  viimast serverit, kus raamatupidamisandmed olid, 
selgus, et \emph{backup} ei loe ja ketas läks nässu. \emph{Backup} oli meil 
korralikult tehtud aga majas käis remont ja tolm oli selle ära rikkunud. Ja 
lindiseadmed olid tol ajal nii lollid, et seda ei näidanud. Ja kes see tol ajal 
\emph{backup}-e kontrollis, kui ma kahes eksemplaris tegin, siis oli ju piisav, 
eks. Mõlemad olid tuksis. Seal peal olid raamatupidamisandmed ja me olime ju 
börsiettevõte. Egas midagi, Priit võttis  ketta kaenlasse ja sõitis Inglismaale 
järgmine päev, OnTracki, kes taastab kettaid. Väga kihvt firma, kõik maailma 
suured on nende kliendid kaasa arvatud CIA ja KGB aga vaevalt, et nad oma 
kettaid taastavad. Jõudsin kohale reedel ja teisipäeva hommikul tulin tagasi, 
see on täitsa omaette pikem jutt, mis seal toimus, aga igal juhul ma tulin kahe 
kassetiga tagasi, kus kõik andmed peal  olid. Kutid olid selle ajaga kõik 
valmis pannud, nii kui ma lennukist maha astusin võeti need lindid, taastati  
ära ja asi läks käima. 

Siis hakkas Hansapank Hoiupanka ära sööma. Tegelikult alguses oli ühinemine, 
pärast oli üle võtmine. Mul ei ole selle kohta paberit aga minu arvates seest 
vaadates käis asi niimoodi, et kõigepealt kuulutati välja ühinemine ja kui kõik 
oli teada, siis võeti üle lihtsalt. 

Tõnu Liik\index[ppl]{Liik, Tõnu} viis suurem osa IT-st 
Hanschmidti\sidenote{Toonane legendaarne Ühispanga juhatuse esimees.} juurde, 
tollel ajal siis Ühispanka\index{Ühispank}. Me tegime uut vinget 
infosüsteemi ka, aga see ei saanud valmis, sest kindlustus lõpetati ära. 

Elasin Mähel suvilas ja Tõnu Liik üks laupäeva hommik helistab mulle. Tõnu ma 
tundsin juba Tööstuspanga aegadest, me tegime koos  esimesi kaardiprojekte, 
Sulo Muldia\index[ppl]{Muldia, Sulo} ja kes meil seal kõik olid, omaaegsed 
sihukesed karismaatilised kujud erinevatest pankadest, Sulo oli 
Raepangast\index{Raepank}. Tõnu helistab mulle, et \enquote{Kus sa oled, 
ma tulen kohe sinu juurde}. Ma ei jõudnud hommikumantlitki ära võtta, jõin aias 
hommikust kohvi. Tõnu astub uksest, sisse: \enquote{Nüüd on sihuke värk, et 
ütle \enquote{jah} ja ma lähen kohe ära. Ega ma enne ei lähe ka. Tule, ma annan 
sulle uue kindlustuse, tee see valmis, mis tegemata jäi.}.  Pakkusin kohvi, 
jõime ära ja oligi. Läksingi SEB panka\index{SEB} kindlustuse 
arendusjuhiks. Tegime hea elukindlustuse, tegime hea mudeli ja see oli, ütleme, 
selles mõttes märgiline süsteem, et see on üks \emph{proof of concept}. Vaata, 
mul on see oma andmete modelleerimise teooria. Too süsteem on mudel, mis 
näitab, et teooria töötab. Sest need süsteemi osad, mis me sisi tegime, on 
aastast 1999 samasugused. 

\question{Aga see on kõige parem kvaliteedi näitaja, et asi kõigile muutustele 
vastu peab}

Tegelikult on veel üks vahvam näide, aga see on varasemast ja ma ei teinud seda 
teadlikult. Ma aastal 1986 kirjutasin ühe palgasüsteemi. Ma müüsin seda ka 
mõnele aga siis lõpetasin ära sest ta oli FoxPro ja must ekraan. Ja aastal 1994 
helistati mulle ja öeldi, et \enquote{Kuulge see teie palgasüsteem \ldots} 
\enquote{Mul ei ole ühtegi palgasüsteemi} \enquote{Ei mäletate, et aastal 1986 
te müüsite meile, et meil firma nimi muutus ja me ei oska seda ära vahetada. Me 
teisi  süsteemi proovisime, seal on vead sees, me oleme kõik muu suutnud järgi 
häälestada.} Sotsialismist tuli süsteem kapitalismi ja elas selle asja üle! 

\question{Mis sa praegu teed?}

Ütleme niimoodi, et SEB-s\index{SEB} ma olin 17 aastat. Käisin küll 
vahepeal ära,  oli üks kümnekuuline huvitav periood, tänapäeval nimetatakse 
neid startuppideks. Aga tol ajal me  lihtsalt arvasime, et oleks teha üks 
produkt, mis  õnnetuseks sattus IT-mulli lõhkemisega samale ajale ja me ei 
saanud enam riskirahasid peale. Me lõime sellist süsteemi, mis kirjelduste 
pealt teeb suvalisi dialooge. Ühesõnaga salvestab andmed andmebaasi. Sul on 
kirjeldused olemas ja sa võid sellest samast kirjeldust teha \emph{voice} või 
mingisuguse mobiilirakenduse või ükskõik mis.  Midagi sellist. Oma aja kohta  
oli see väga kõvasti ajast ees.

Sealt ma tulin tagasi SEB-sse. SEB-st läksin RIA-sse\index{Riigi Infosüsteemi 
Amet}\sidenote{Riigi Infosüsteemi Amet}. RIA-s juhtus niisugune lugu, et Katrin 
Reinhold\index[ppl]{Reinhold, Katrin}\sidenote{Priidu kolleeg Riigi 
Infosüsteemi Ametis.} tuli TEHIK-usse\index{Tervise ja Heaolu Infosüsteemide 
Keskus}\sidenote{Tervise ja Heaolu Infosüsteemide Keskus, Sotsiaalministeeriumi 
IT-maja.} direktoriks. Katrin hakkas mind aeg-ajalt kutsuma arvamust avaldama 
ja ühel korral, kui me siia alla kööki läksime, küsisin ta käest, et 
\enquote{Katrin, sa vist lubasid Taimarile\index[ppl]{Peterkop, 
Taimar}\sidenote{Riigi Infosüsteemi Ameti peadirektor ajal, kui Katrin ja Priit 
seal töötasid.}, et sa ei võta kedagi kaasa}. \enquote{Jah, ma lubasin.} 
\enquote{Aga kui ma ise küsin?} Ja ta vastas, nagu Teele, et \enquote{Ma 
mõtlesin, et sa ei küsigi!}. See oli elu kõigele lühem tööle värbamise vestlus. 

Kui ma tulin, siis oli meil sihuke asi nagu analüüsi osakond, aga nüüd on selle 
nimi Andmekorralduse ja andmeanalüüsi osakond. Selle all on kaks talitust. Üks 
on andmekorralduse talitus. Nemad teevad HL7\sidenote{Meditsiinis laialt 
kasutatav andmestandard.} standardi peale andmevorminguid, millega kogu 
tervishoiu infovahetus Eestis käib. Sis mul on olemas arhitekt, kellele mina 
käin alguses mingisuguse asja välja ja tema joonistab lõpuni, aga ta mõtleb ise 
edasi, väga-väga hea detailse mõtlemisega, täiesti müstiline, Andrus 
Tamboom\index[ppl]{Tamboom, Andrus}. Kahe analüütikuga ehitan üles  
analüüsikeskkonda ja palkan parasjagu omale Andmelao Talituse juhti, et kogu 
andmelaondus korralikult üles ehitada. 

\question{Kui a selle kõik kokku võtan, siis on mul täiesti ühene tunne, et sul 
läheb hästi?}

Ei, mul pole häda. Mul on huvitav, see on jube huvitav. Tähendab, ma tegelen 
selliste asjadega, millega ma varem pole otseselt tegelenud aga see muutub ka 
siukse kiirusega et seal teinekord on raske kannu kannul püsida. Kogu aeg 
lappad kohutavat kogust mingisuguseid materjale läbi, et kas on midagi uut. 

\question{Äge on kuulata, et nii on vist kogu aeg olnud?}

Just nimelt, ma ei ole kunagi töötanud sellise ameti peal, kus mulle ei ole 
meeldinud. See on kõige olulisem asja juures. 