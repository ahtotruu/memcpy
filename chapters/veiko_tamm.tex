\label{chptr:lucifer}
\index[ppl]{Tamm, Veiko}

Siia tuleb jutt arvutite toomisest Moskvast\label{sisu!veiko_moskvas}.

                 
Kõneleja 3:
Tere, see siin on memm, kopi. Täna oleme külas mehel, kelle jutt viib meid Tartusse ning kelle soov inimesteni teadmise heledat valgust viia. Piis kunagi Lutsifer BBS asutamiseni. Oleme külas, Veiko Tammel, head kuulamist.
                 
Kõneleja 2:
Nii tere hommikust. Hommik. Mis on? Sinu lugupeetud nimi? On? Väga-väga meeldib. Seekord ei ole kogunenud nõmme mändide alla. Seekord oleme kogunenud mujale. Aga jutt on ikka endine. Hakkame peale sealt, kust kasid pihta hakkavad. Kuidas arvutit sinu juurde said?
                 
Kõneleja 1:
Kombrjorteerit komplekt, komplitseeritud küsimus, nii et kui võtta ma ise olen ülikoolis keemia eriala lõpetanud. Aga ei, oma neljanda kursuse kursusetööd ja diplomitööd ei ole ma ühegi kolviga katseklaasidega Solverdanud või mind kutsuti tollal üsna põnevasse asja. Nimelt juhendajaks oli mul Mati Karelson, kes alustas arvuti ja kompuuterkeemiaga uke kvantkeemia, ühesõnaga ja sealt siis oli minu esimesed kokkupuuted muidugi ka ise oma näppudega tagasi on tulnud. Meie tööteljeks oli perfolintidega alguses nende viie augulised.
                 
Kõneleja 2:
Tol ajal ta üritas teha siis kvantkeemiat
                 
Kõneleja 1:
Ja siis oli tund arvutitega, see on kaheksakümnendatel juba hakkas siis asi pihta, 80 80.
                 
Kõneleja 2:
Okei ja sul nagu enne seda üldse mingisugust
                 
Kõneleja 1:
No meil matemaatika kursuse käigus väga lühidalt näidati selliseid arvuteid nagu nairigaks. Ja nende Peeaa peegeel siis sai trükkida endale lehelaua sel ajal seal näed siit. Aga no ütleme, need masinad, millega ma hiljem te olete ju alguses oli miinus 32 ja hiljem oli Jeltsukesed kutsutud ees ime siis teemal 1020 kapsaavutad mis olid siis seal ülikooli arvutuskeskuses. Ja seal reaavutamine käiski nii et ta perfoleerisid oma programmid sinna sisse viia, sinna täitma lahendus, mis, mis programm tegi selle, noor selle põhimõtteliselt minu tähendus oligi, seal oli mitmesuguseid kvantkeemia meetodeid. Aatom, orbitaalliit, kuidas need keemilised sidemed moodustuvad, kui ta nüüd elektronpillertruud toodab, seega ja kas see aitaks seletada neid keemia asju, seal lähedaks ampsu tulemusi sõjaldajatele tulema ja need arvutused oli ikka need ütleme, me võtsime lihtsalt kahe aatomid, molekulid oli nii, on nagu on aga keerukamalt näiteks metaan CH nelja molekuli välja arvutamine nõudis seal sellelt suurelt arvutilt umbes 10 korda rohkem tööd kui terve lihakombinaadi aastaaruande välja. Et see oli päris kõva kallis arvutiaeg, mis sinna alla läks.
                 
Kõneleja 2:
Kus ja kuidas see siis oli, et sulle juhendaja ütles, et nyyd nüüd ta hakkab BKK programmeerima saax programmeerima niukseid asju.
                 
Kõneleja 1:
Ei programmeerimisest jäi asi kaugele seal noh, tuleb usaldada küll natukene keeltega ka, aga põhimõtteliselt oli seal mõningaid asju Fortranile kirjutada need et midagi, ma nagu aru leina käega igal seda programmeeriti asju ajule selgeks saabki siis kakas klaasi tükiks ajaks ära. Ja taaskohtumine arvutitega oli nüüd võiks öelda juba üle 30 aasta tagasi, kui Tartu tähetornis ajas tol ajal suur auruti, pan Enn Kasak, kuld juulis ringi otsimisi, hankisin arvuteid ja, ja siis see oli 88 ja siis oli, algas suur kooperatiivide ajastu. Ja siis sai tehtud operatiivTaeda, mille liige ma kolin ja mille, nagu eesmärk oli Promentaariumise planetaarium sai isegi ostetud ja seal füüsika instituudi fuajeesse kuidagi üles aetud, aga ei ole ka mingit tööd tegema hakatud. Aga samal hetkel oli see teadlik ja teaduslik ja mille varjus sai siis nagu arvuteid osta-müüa. Sest arvutitega oli üldse seadsiinu, hiljem keelati igasugustel väikestel asjadel ära saavutamiseks. Ja kuna no kui sa oled näinud kunagi filmi brigaada ei ole Venezerjaara sellest segaste aegade maffia elus seal kus hävitati ka kõigega, siis juhtuski niiviisi, et ma sattusin lihtsalt mine ja too mingit arvutit kuskilt sealt käisin Peterburis Moskvas Saingijate otsekontaktid ja sidemed ja nii siis arvutiärisse sattumine nagu järsult puhkeski välja. Sest tol ajal ju kiirusalt kallis, minu esimene arvuti näiteks amigo 500 v maksis sama palju, kui ütleme, tutikas null kaheksa auto null kaheksa väga jahutab. No oli ikka mingi 40000 rubla, hirmul.
                 
Kõneleja 2:
Aga miks sa sinna Tähetorni juurde läksid, sigavinge arvutivärk nagu tõmbas.
                 
Kõneleja 1:
Sõber kutsus kuulegi, tule ka kaasa löövad ja põnev asi värk.
Ja siis ma mäletan jah, et kuidas siis tuli algusest peale on see kõik see selgeks teha, kuidas vaadata, et millised paljude loop mälu on testiprogramme kasutada. Alguses kurva trennina, aga järjest hakkas särsi nagu liikuma ja kui endale arvuti koju siis oli jube põnev, lõid nii, et maja alguses rändasime Amigate 500 siis 500 pluss isamiga 1000 Tomiga 2000. Ja siis sai sinna kõrvale siis esimene pyysiivsest tuli üks suurem tellimus arvuteid saadel, kuradi raske. Sellepärast et iga embargo otsesse pidamiseks põhimõtteliselt oligi, tõime arvuteid Moskvast ja Peterburist kogust noorukitest, sinna soid. Noored, see Moskva kala, mis oli, millega ma põhimõtteliselt embargo, arvuteid, viinerit, siin tulid läbi saatkondade. Orilise Singapuri saatkonnas toodi paljundusmadinakastis siis arutada, tee sinna saatkonda ja siis rumba üliõpilaste kaudu. Läks ja siis ütleme kohalike ärikate kaudu seal tantsisime jaostaja siukene torn vahel oli ja see oli päris võimas kursuselt arr naeras oli ikka niiviisi, et hinnad olid, liikusid väga põnevalt, ütleme et iga ots pani ikka omale kuskilt seal ikka ütleme julgelt 30 kuni 50 protsenti otsa, aga aga väga sageli ei viitsinud hakata liigutama, kui 100 protsenti kasumit ei andnud. See oli riskantne tulla tuliseks Kontheri kaablist, Jüria vein ikka väga palju tuttavaid, kes said koolieluga ja olen isegi kihutanud öised Peterburis punase tule alt läbi, sabad järel ja kõik, ja kui me no sa ikka sõidad sinna näiteks kolm miljonit rubla sularahas seljakottidega kaelas, siis, siis ta on riskantne.
Aga no vahe oli seal selles, et kui sa võtsid näiteks ütleme, ütleme, ülekande rubla ajal nimetati pensionaalik ütleme, arvuti hind, ülekande rublaga oli kaks miljonit, ok, kui sa sõita sularahas selle raha täitsa käte, ütleme, hõõruma leinaga. No aga kui sa maksid valuutas ühe miljoni ümberarvutatult väärtustes ja teine lisaväärtus, mis tuligi ühte ütleme Moskvaga tuulitades oli see, et Venemaal, Moskvas, süvaVenemaal hinnati saksa malka hinnati dollarit. Meil siin vastupidi, olid hinnas Rootsi kroonid, Soome marga särika tärje, sest seal käisin ja siis oli nii, et ostsin tsink kokku saksa Margad, ostsin kokku, dollarid, läksid Moskvasse, maksin jällegi üle ja kõik need saksa margaga, dollareid, mis õejaid Need vahetasid seal siis soome ja markeldakse Rudjorooni, eks. Ja tulid siia ja vaatasid ära ja üksinda selle ülejäänud nii-öelda valuutavahetuse eest muusika rahulikult iga mingisugune Lada 1000 pael, muidugi. See oli hirmus aega ja siis eriti kütitud, kuigi nõgaja lõpuniketisuse valuutaseadus. Et suurtes hulkades valuutaharidusega võisid saada seal seitse aastat ja suureks Ulav loeti paar seda, kui sul oli rohkem kui 100 dollarit. Nii et seal tervelt tuhandetega sai arvestatud, siis kes need on? Siis need arvuteid siis kui jumal kõik olid aedid taga, nii-öelda asja, riiklikud ettevõtted instituudi vaata näiteks, kui ma oma esimese PC arvuti ostsin siis see läks Tartu Ülikooli füüsika-keemiateaduskonna pea serveriks. Ja see oli kaks, kaheksa, kuus arvuti 20 megahertsiga, kaks kaks kuulpenserveriks. Vuimal 20 mega ote tavalisel masinate ultiku Actes väga, et see tal kaks megabaiti mälu ja 120 Megane kõvaketas puu muutub, aga mis, mis seal siis teise arvuti, sellise ma ostsin endale, muidugi käisid paljud tuttavad Genskijana valuta mida sa lollidesse endale otsi, mis sa teed seal arvutiga kula seal 120, mina arvan, tähendab täis saavad. Siis läheb, mul siiamaani meeldib, ütleme, Enn Kasak kui üks selline preemia väide, mida ma olen pidevalt kasutanud. Et kui auto teaduv Mul on samamoodi arvestanud nagu arvutiteadus siis me praegust Mersedes sõidaks valvurite kiivriga võtavad seal 10000 kilomeetriperve tilga bensiini ja makstakse pool senti.
                 
Kõneleja 2:
Jaa, tun tundub tõepärane, aga mis talle kaks 206 serveri peal jooksis novell.
                 
Kõneleja 1:
Vähem võimelised, mida jooksutatikaalid juuniksid? Juuliks v viis näiteks ja kõik sellised verised sinika käisid rahvas ikka välismaalt külas meie teadlastel ja siis kõik väga imestasid seda. Nii palju teatri sinisilmselt ja siis kõigile joonitsete seerianumber üks. Noh, ega peale Pilatuse kuskil Küba võta tarkvarahindade, kes andis mingi Tarboraja aususest.
                 
Kõneleja 2:
Ühesõnaga piraatluse abil sai tegelikult kuna ei ühte ega viit ei jaksanud osta, siis piraadid diviis ja nii oligi.
                 
Kõneleja 1:
Ja edusamm, et nagunii ühtegi teist laadi midagi vea, vähemalt mis vastase PC asüüli osta, mingi 89.
                 
Kõneleja 2:
Siis selle väga ruttu ja sa ketrasin Stamigatest läbi endast PC peale.
                 
Kõneleja 1:
Mul oli 90.-te alguses juba siis kokku, on meil siis oli mul kõige esimene kulu kolm, kaheksa kuud jälle nimetati, milleks seda vaja on, kes sellise asjaga, mis sa sellega teed. Tüüpiline, mis pidi ikkagi, et, et sul sinna kooperatiivi nagu sisse pidi ikkagi olema mingis arvutikooperatiivi meil väga kaua see asi kestnud sellepärast et väga kiiresti tuli peale see seadus, mis keelas mitteriiklikele ettevõtetele kooperatiivide arvutitega äritseps. Suu. Ja kuna mul olid siis olemas nende kooperatiivil on vähe, siis me tuul, paari tuttavaga, nagu siin kliente leidsime ja nii edasi, tekkis küsimus, et nood, kus kohta, mida me teeme, siis mida on veel vaja, tootmisruume ei ole vaja. Meil ei ole vaja mingeid laulusid, mingeid luurajaid, midagigi, raam.
Tore koht, kuhu läksime sinna veetud jutule ja poisid ütlesid ka, et kui lähed, räägin, mis on vaja, nüüd võtab nii, et ma ei peaks oma nagu raha looksutadad. Küsi kuskilt salaload, noh siis oli Miina selliselt, et sellega ta juba enam-vähem madin osta. Et muidu äri käis ju kogu aeg ettemaksetega, ta tuli ära, siis võisid tükk aega pööritada, siis ostsid selle masina kliendi kätte, see kuu aega oodates oli tavaline nähtus. Tulid jälle suuremad summad, siis suri. Meil olid käsi ühte asja sees, see oli ju see elektroonikatehas, kes tegi neid Panasonicu katsed viibe viie pealt maha viksitud neid vene videomakke v parteist. Laadisid furgooni neid täis linna läksid Peetrus ja luua kahest vana Tallinnat porgastri suitsusinki meie oma suitsuvana Kõmsega Novgorodi see siis videomakke täis ja ma ei viitsinud üksikult müüa, siis selle müüsime ulgilisate koopratiivsikutele maha ja neid siis oma kooperatiivipoodides partnereid sedasi. Et see asi vist rahakeerutust tehtud, seal kogu aja. Aga siis 100 mõeldud tuisulane käimaan, algkapital öeldakse. Et noh, tavaliselt teadlikule midagi küsitlejad, eks ole, siis tõmmatakse maha. Ja rääkisin v ära, et otsin autori kui uisud, kõrvad liikusid, et kas neile alaks kõikuda, mis ta meile pakuti ja ütlesin, et 11 protsenti meie näiteks poos sisse tropp.
Avakarva päike miljon peale.
Kusjuures meie firmal ka selline lõbuse, suur kommertspanga pank pankrotti minna. Minu üks väheseid, kes selle raha tagasi uhke oleks võinud põhimõtteliselt ka teha mingi pori külmadel ära kõndida ja pankrotti. Aga meele kasvasime kogu selle raha nii et, et selles mõttes kuskile nagu orgia võluv seisund jäänud.
                 
Kõneleja 2:
Tamigaside pissid, keerutasid, pidi ikkagi siis mingisugune huvi olema nende asjade vastu, mis sa tegid nendega kodus. Kui sul oli Amyga laualambiga ülemise
                 
Kõneleja 1:
No kui see tähendab, niisiis need esimesed emmessiks algselt olin, ma mäletan ise, kuidas see siis ma veel töötasin seal keemiainsenerina kuskil pärast tööd sõidad bussiga linna, lähed torni ja siis istud, mängid seal täpselt nii kaua neutraal Kerbussi pöörde kas või sõita. Et külaelu aega keerad, vaatate hommikused ringid üles, kergkabinetti luku, ker palama harudesse minna, mängud olid naiivsed, aga teate, oli nii põnev aeg kui endale arvutitugi košmaar, vaktilised, järjekord pidevalt ukse taga, kõik tulid tasuta mängumise. No seda ma mäletan. Mängisid sai muidugi igasugu asju uuritud ja muidugi siis, kui tuli internet afis. Tegelikult hakkas see maailmaga ringi käimine juba enne seda pimesi.
                 
Kõneleja 2:
Vaat seda vaat-vaat sinna ma tahaks jõuda sinna, ma tahaks jõuda, kus sul niisugune mõte tuli, et paneks Peeveeess endale püsti. Ja mis ja millal, see oli?
                 
Kõneleja 1:
No see oli kuskil Üheksakümnendad tell varad väga varastel
Internetti jõudis päris internet, tütre relvur, kahe satelliiditaldriku ka, üks oli seal KBFI peal Tallinnas ja teine oli Tartus Tähetornis. Ja sealt siis üle Rootsi, selle kuningliku tehnoloogiainstituudi katedraal käis siis sidemeid. Siis hakkas juba seeeffibjazimsus vaikselt juba ära vajuma, kuigi ta veel sellisel töötlusel edasi, eks ole. Mäletan, et internet jõudis Tartusse, tähendab siis ma elasin seal märtsikuus 92. Ja mina sain üle selle eebeezzee bioKerspireesti kuskile endale oma isikliku Acoundi juba selle.
                 
Kõneleja 2:
Siis oli juba suhted sees, seal.
                 
Kõneleja 1:
No ja enne seda oli muidugi see, et kui arvutiäri tegid siis oligi, no kes sellel üles tõmbasid, olid tol ajal veel meil kontrolli Rüütli tänavas. Kohe Treffneri kooli vastas. Muidugi ägedalt kremli realistid käisid seal kõik hoolega seda näppima, said arvutite taga istuda kõike ja üks põhimehi, kes seda asja suunatšev ülespanija majandustarkvara poole pealt. Reina rentsik praegugi kõva kinnisvaraärimees. Nii et tema oli nagu see Meie peamine shop ja mina olin siis koosis shop, kus siis tiksus hiljem seal konto, likvideerisime siis Bibi-sse Lutsifer piibers tegutses kodus edasi kuni peaaegu lõpuni, kui see Vibesi maailm ära hääbus. Siiamaani mäletan veel aadressis kaks kool on 491 punkt kuus kuus kuus.
                 
Kõneleja 2:
Ja millest selline nimi oli? Lutsifer, kus.
                 
Kõneleja 1:
Ta tõi valgust maa. Aa no muidugi, meil oli väga palju igasuguseid, selliseid jäeti värki, kuna ma loomult olen, anti kristlane olnud eluaeg igasugusi, no nüüd ma veel suurem alti islamist. No seda ma üldse taguda selle juures kristlased väiksel varakendadega vaimupimedust, mis siin on keskaega tagasi võrgimiste alla. Aitäh. Selle info oli mitmeid ütel loomamaterjalid püsti ja selgelt oli kunagi suur jama, kui kapo käis meiereid materjali uurima Chate ostlemas seitsme kui mingisugused nõndanimetatud satanistid kes pussitasid Tartus Andrunnerit, Nende uurimine siis oli ja kõik, kus saadud ja, ja siis ma ei tea, me kellelegi kaudu püüdis välja, et nad mõni meie viibi s siis oli väga palju neid materjale, kõigil saate Arabella või oma ja võta siis käid ja uurid ja vaadati ja ma mäletan üks üks mehikesel Ductiv nii põhjalikud täitsa ära.
Kusjuures küsis, kas ma olen satanist, mõtlesin või ei ole? Kumma ei usu kristlust, kuidas normaalsest mopeed, pilti, kummale.
                 
Kõneleja 2:
Miks, miks te selle BBS-i tegite nagu äriseni muud moodi ka teha?
                 
Kõneleja 1:
Ei, see oli lihtsalt tobi poisitud ja igasugused sidemed, materjalid üle maailma. Tol ajal.
Telefon hinnantsemantaatil ma võin tundide kaupa kirpude neid seal Ameerika või Iisraeli või kuskil Euroopa põlisküsimus. Kaugekõnede kolivad soki Jaapanisse, sealt igasugusi materjale tõmmatud sai ja siis sai seal vahetatud kogu skeene ringkondadega.
                 
Kõneleja 2:
Aga kuidas need numbrid said, kuhu sa helistad? Kujutan ette 90.-te alguse Tartut, ma ise olin ka seal. Et kust ma võisin saada jaapani telefoninumbri.
                 
Kõneleja 1:
Aga sibi S-i teel olid kõigil suurel distsingul olemas, kus oli laulda beebi, kes ei toeta nende kuhugi, suid küll siis sealt juba hakkas tulema ja kõigepealt CDC tõmbatsin, uuendati pidevalt seda aadresse, raamatu telefoniraamat, seal on kõik riikide ideesid sees loetletud ja ega siis käisid infot, lugemus. Nii et infot on sellele. Nüüd on kõiki Ito tõisspukis, aga siis seda ju ei olnud, siis oligi Bibi esindaja kaudu käis, et see see info vahetamine, meili saatmine, kõikene. No hiljem tuli siis, kui internet tuli, siis olid juba teised ajad, aga ka graafikat olime me 92. aastal kaugele selleks, et seal tegutseda, tuli endale jooniksi aluliselt selgest ja seal käsuribal töötada ja kõike iko, sest siis ei olnud ju veel Liinupsidki olemas midagi sellel algusajal ütlesime, põhimine oli Santa Cruz, operation b5 juuniks, millega tegutsesime.
                 
Kõneleja 2:
Nii et see, see BBS kestil juuniks jal
                 
Kõneleja 1:
Ei Pebecilon oma tarkvara, see on märgi hiljem, kui internetis hakkasime käima juba siis igatsetud kliendiga, lehvik kuskil terminali otsa näiteks siis kaasa ja üldse, ega siis kellelgi kodus internet võimalik saadagi. Siis pidid teadma, kus istusid modemit sisse helistamisel modemit, millega slaididel kuskile internetti arvutisse sisse logida, kaugelt näiteks henna toomemäele Tähetorni ja sealt siis juba läksid edasi liikus internetiavarustes nagu.
                 
Kõneleja 2:
Ja kõik, kes teil neid käsureal
                 
Kõneleja 1:
Siis hiljem hakkasid tulema seal kohta ruumid ja muud sellised algelised otsingusüsteemid olid seal juba infopangad, kus oli raamatuid ja viaalastel kõigil aladel siis tekkisid internetti Bibi assid. Seal oli näiteks printa, oli Euroopa üks suurimaid ja Iska Iowas kiudel Corporation Association. Oli siis SKP-s ja maailma kõige suurem interneti teel puhul natuke onu, ütleme viibeeessegi teadetetahvlil kaval, lihtsalt sinna ligipääsuga interneti kaudu.
                 
Kõneleja 2:
Kas sa selle eest teiste Süssoppidega suhtlesid?
                 
Kõneleja 1:
Ja ikka meil olid ju need üritused, igasugused Bibi summerit ja Bibi winderid, kus sai käidud kus kõik need niisugused sopid kui ka sees kasutad, oli niisugune päris tihe seltskond, kes seal käis omavahel niisama suhted, no nagu praegu Facebook'i tasemel siis see seltskond koos ja see on nii suur, et ka nüüd ei oleks võimalik hallata. Katsu sa teha näiteks Eesti Facebooki liikmete kokku. Tähendab seda, võib-olla kokkutulekule jääb tulemata võib-olla viis protsenti inimesi.
                 
Kõneleja 2:
Kui sa selle BBS-i püsti panid, siis annab mingisugused suurusjärku, et palju seal neid liini küljes oli palju kasutajaid küljes käis.
                 
Kõneleja 1:
Ja sellega oligi see, et kui kasuta, tuli telefonil inimesi stabiilse rippuma. Kõneaeg jooksis kogu aeg. Need materjal selles mõttes, et kui sa näiteks sikutasid mingeid tarkvara kuuskete Ameerikast siis rippusid kogu aeg kaugekõnega, treenib soolane kopikas, kiiksud, koba ja eraldi vii need üüriliinid tulid alles Stenni ajastu pur, tulid need 64 bitise teada, 128 kilobitti kiiruste Kaljal. Algselt kõige esimene modem, mille ma sain, oli 2400 poodi.
                 
Kõneleja 2:
Aga see oli isegi juba kiires 1200..
                 
Kõneleja 1:
Ja olid 30 600. 2400 oli Finlandia üks piibeeessori 2004-l päeval ülejäänud kõik olid aeglasemalt peale need, no ega see mudel maksis ka kaks korda rohkem kui sõiduauto Žiguli, Tavariend.
                 
Kõneleja 2:
Aga no ikkagi see hobi jaoks tundub nagu Žiguli nagu kallis investeerida.
                 
Kõneleja 1:
No võtame niiviisi mõnel rikkal, mehel on sobi, jooks korjab selle vanemaid autosid seal kokku igasuguseid, mis maksavad ka, eks ole, seal ütleme meie praeguste paarsada ja 200000 uunikumi teevad oma automuuseumidel mummud joobi. Kas te sellega ka midagi muud ei teeninud? Piletit ei küsi, kõik on.
                 
Kõneleja 2:
Ma uurigi mis sul, mis sind selle hobi juures nagu siis paelus, mis sind hoidis selle asja juures?
See, et sa said minna ja uurida, lugeda,
                 
Kõneleja 1:
Kui, kui vaadata praegust maski kohustus on, seal ju reageerisid, et oleks võimalik teadus enne surma üle kalda internetti. Läheks, hülgas taldiks, elasin selles virtuaalmaailma. Selles mõttes ma olen ja selle küberajastu ka niisiis.
                 
Kõneleja 2:
Kas sa tol ajal juba Gibsonit lugesite jäika? No siis on selge, kust see tuleb.
                 
Kõneleja 1:
Kõik see küberpunk ja see värk, see oli noh sisuliselt kõigile neile, kes olid tolle arvuti friigid. Meil oli see mastaabi siis muidugi virtuaalmaa, ütleme kuuma erid sisse ei jõudnudki, olid mudad. Mina sattusin sellesse virtuaalmängumaailma siis, kui tuli selline asi nagu Everkastuks. Ja Aivar, kes siis seal sai järk-järgult läbi mindud. Ja see
Mängus sees oli onlain kammutamis luges, kui palju saab leida mitu päeva, mitut tundi hädasid ei too Meelis kokku. Ja siis, kui ma vaatasin, kuna ma seal mängimist alustasin. Ja kui palju tunde tuli kokku, siis aeg oli selline. Selline nelja aasta poole aasta jooksul, iga jumala päev oleks pidanud mängima neli pool tundi täis või noh, mõnikord päifor kuskilt ära ei mängi. No polnud sagedased. Aga polnud ka väga olulisel kohal, kui ikka kaks tundi jutti näiteks suuri raile beeta teha.
                 
Kõneleja 2:
Gibson ja, ja muu selline küberkungi kraam. Kasse levis BBS-ides faili füüsilised raamatut ka.
                 
Kõneleja 1:
Noori füüsid raamatut, suuremad fännid siin Jack'i tõlkisid, tegelikult isegi avaldati.
Jääks see oli koht, kus tooli sained inglise keeles suured raamatuarhiivid ei olnud ju midagi eriti just ulme ilu kirjeldati, ütleme siin teadusRamo tobud ostsid rohkem teaduskirjanduse rahasid, millest ilukirjandus oli ikka väga vähe, aga no siis hakkasid nad liikuma juba digitaalsel kujul. Digitaalsed arhiivid, sealse, sama töötamatult mõtlen, ma olen läbi ja siis, kui oli hästi suur kõik kättetoolide sõja ülesse rabakäise laiereid.
                 
Kõneleja 2:
Elu ja lugeja sai, sai valgemaks.
Kes rahvas õliks käis, seal palju käpiliselt olid mingisugused.
Elu mingid mingi aimdus oli, oli, kes?
                 
Kõneleja 1:
Noori õpilastest ja tudengitest kuni paremate inimestena allkirjastasin välja, kas meil on seal väga palju, katsus põnevat tarkvara, põnevat arvutialast kirjandust ja materjale, mida nad kuskil Gailiti liitunud. Ja, ja siis, kuna selles printaalsemadel juhtisin ka üht tuba oli moderaatoriks, nagu siis saime nendega tuttavaks nende Atherograudiga, seal selle AIS-i piraadigrupi liikmena sai oldud selja säritusmaterjaliga täitsa kurioosseid olukordasid, kui noh tollel ajal ei olnud ju mingit väli FTP, siis tehti seda nii, et kuskil, kus olid virmalised, PD FTP serverit püsti. Siis kui ta tegid mingi asi algatsul punktiga seninähtamatu. No siis mingisuguse nina muina kataloogi, tavaline inimene, midagi teadnud süsteemsete adron, fina sünteetilised punktiga algavad kataloogi, kui seda hakati avastama, siis olid seal igasugused asjad, need ütleme seal mingid kontroll, sümbolid, fantor, sümbol, mis tegi näiteks piiksumist ja kirjavahetuse sellesamale entry vajutamine, et sa sisestad asja see kontro sümboli kui võimalik kirja panna. Ja kui seal sümbol ees, siis pidid teadma, et sa sinna ette panid. Selle control sümbolümi psüühist oli kontor relvani, mis käskis lugeda järgnevaid asju, ülistate tekstistringe.
                 
Kõneleja 2:
Muresite kellelegi FTP serverisse lihtsalt oma pira.
                 
Kõneleja 1:
Ja niiviisi tsükloni täis seal kõik katoloogil Kaupolific võimekas mängudes, kuni ütleme korrani.
                 
Kõneleja 2:
Kas see rahvusvaheliselt peagrupid ei olnud ju väga niuksed nagu noh, ütleme sinna ligisaamine oli ikkagi nagu seotud raskustega päris nagu.
                 
Kõneleja 1:
Ei suuda, osi oli suurte raskustega, praeguste torandiaaess sõbrannadega uist sõitsin, siis oli ma ütlen, et selle Britta Bibi S-i kaudu tulid, kutsusid ja seal siis oma tegevusega jäid kuidagi silma. Seal tuli ja igasugu põnevaid asju juurde ja vaadata ja see tänu sellele, et ega sa ei oska näiteks midagi soovitada, kui sa ei ole ise seal rääkid. Sest see on muidugi ka see, et nendele tõsistama gruppidele seal lootis, et kui sa kuskil mingit piratarkvara müüd, raha teenid nagu Vena lõpuseid, piraatkogumikud igal pool müüdi mustal turgudel ja laatadel seal igasugused viimse ja nii edasi. Seda loeti märgis soita kohe kangi kohta. Pott ja, ja aga siis oli jah, et üks kõige võtku tuli välja, see oli alguses kohutava suure saladus. On nüüd on näha, Microsoft on olemas täiesti teisipidi pöörama. Tahab võtke uusi versioone, uurige. Vaadake, nad on lõpuks arusaadav, et see sellele, et sa midagi püüad, ära kinni hoida pileti, mis ei takista. Aga kui sa saad endale miljonise, kulukas, uut toorest, tark, markatseta, kaku juuksed lähevad halliks tänu sellele, et kõik hunnikusse lendab seatud tagasitee, kes küll sõimad, parandage sära, vaat see on valesti. Sa ei jõua endale nii palju töötajaid otsida, kes selle kõik selle pugemise töö ära teevad, kui see vabatahtlik jõuk toimub, aga sisulise jaani keelatud tikutuli, see Windows 95 kool nimede dollar, Chicago ja oli siis Ahteni muidugi olemas, on siis poisid igavese pulli, nad panid Microsofti peamisse FTP serverisse, tegid seal punktidega taru, imbusid, järgmine päev, mis ei Kaago ülesse sinna isiksuse üle maailma kulutulena Vi Microsoftist sapirao vindus tõmmata viis t, oi kuidas siis Microsoft. Kõik, kes seal käinud, mina käisin, lihtsalt vaatame sirbi irvitama, tulge kohale ja vaadake, mis seal seisab. Hinnad olid ära logeenuterdaksele, et näed, selline värk olivad käsitöö ja siis sealtkaudu oma üks, üks aadresse, mille kaudu ma käisin, eks.
Seal olid mõned kohad, oli, mis olid häda abil, eks ole, kui näiteks vihmashop tal oli vaja, noh siis mina põhimõtteliselt lippusi, teade, teater, karistage vodka, säcountaraasand, otsik, piraat, kes vaatab viirus meie rasse seda Windows, kuigi ma ei tõmmata tantra ammu olemas. Aga see oli väga palju ja siis seal.
Et neil on jube kallis, neil pooled ohvrid, kohustuseta. Statistika programm, Statistika päkke Furzov, seal sai siis arutama ja siis on seal uut, neil juba bord, lisamoodulite kõik, et seda oskust saada. Küsin selle aisi, pimedad kuulge tungijaid.
Või kuidas siis meie olnud teadlased olid rõõmsad, kui soomlased tulid vaatama ja Eestis on täispakett kuute?
                 
Kõneleja 2:
See toimus nagu nihe, et ühel hetkel peeti nagu auasjaks. Ühesõnaga hakkas muutuma see ikkagi makstud, lõpuks.
                 
Kõneleja 1:
Oma rühmaks tuli ja kõiki asju hakata ostma, hinnad muutusid ka nagu Noomoriseerus. No ma ütlen, vene aja lõpul olid need hinnad ju no kujuta ette, kahekümneVegase kõvaetas maksab tsa 45000 rubla.
                 
Kõneleja 2:
Hoomatult kuupalk oli mingisugune.
                 
Kõneleja 1:
Kaks kaheksa no mäletan meie eesee kõige esimene neli, kaheksa, kuus, mis läks sellele punasele räti SKP spets konstrueerimisbüroole, selle vaesime välja, see toodi mingisuguse, ma ei tea, mis kuradi värgiga. Sisuliselt üldjuhul nii, et need Ukrainergikas hakkasid eralduma ja valgemad. See tuli Minskis peidetud traatorides nei kaheksa kuuendal ju totaalsem komme ei lubanud neid sotsmaadesse viia. Kolm, 206 kõige laiemad persoonid olid ainult need, mida ta juba võis. Ametnikud. Ja selle masina hind oli neli, pool miljonit rubla.
                 
Kõneleja 2:
Oh jumal hoomamatu number, isegi nagu rublades isegi eurodes.
                 
Kõneleja 1:
Aga jah Unteetilises omalgi piirnev majanduspiirkonna Rajoppele. Sõidame sinna, vastik ilm oli, siis oli see reegel, et ainult juht tootis läbi sõita. Delfi minema jalgsi, putka ei viitsinud minna. Kahekesi mikrobussiga tuleb siis sõdurpoiss ja kulbid, mis, miks kohakesi olete? Mõtlesin, saadan kaupa. Üksnes viia.
Väidab ei huvitanud, nende selle jaoks oli ka Ženni hoomamatu. Aga seal oli jah, meil niivõrd head sidemed vene poistega tahavad Venemaaga ärihaidest, sa pead teadma, kuidas ja mismoodi usalduse nii suured teinekord läksid. Nii täpselt ei teagi, palju ilmutab kaasaegsena. Lavdelludu jälle näiteks ütleme poolteist miljonit üle. Täitsime rahul temalt seifi. Selles mõttes, et edasi-tagasi kõige riskantsem. Kui sul mingeid saba peale maha võtta, tulistada autojuhid, hiljem helistada, kuulemaid tuli selline väga. Mul on väga hea hinnaga. Huvitab, huvitab okei, ma võtan selle raha siis endale. Ja samamoodi sinna kahte masinat tooma. Mul õnnestus kolm tükki saada. Tahad Leedus ja ikka võluda karbert Hermi rutada, haarased.
                 
Kõneleja 2:
Ja kui ma sind kuulen, siis see ei ole ju mitte nagu kümned ja sajad ja konteinerid, vaid see on nagu kaks kolm masinat.
                 
Kõneleja 1:
Ja vähem ütelda suuremaid tehinguid näiteks kunagi.
Saime kontakti kuramiirlik, kontaktid arvutite juurde mõeldud distel meisterlikult, tegid raha üle siis oli meil niisuta sõbralikud suhted EVEA Panga mitmed tegelinskid tega, kes ajas imelikke, tõsi, talviseid välja veidike. Tõime interigal suurel pika reisi karuspulsitäie arvuteid Poolas näiteks. Ja siis sain kõik viidud Peterburi seal plaaditud sõjaväe transport kopterite peale. No kõik, kõik on kuulipildujate värkide all relvastatud eriväelased, nelki, 10 no ka see tehing oli ka seal, ma ei tea, kui palju miljoneid kokku läks.
Parktuli siitkaudu saime müüdud. Ja igaüks sai oma, sellest tulevad jälle need poisid tuttuda mõmmi konjakit, kaasotsa toodud koera minema. Korrusse.
Sularahas kohvriga küpsetatud närvigaas.
See see oli, niisugune väga on, ütleme selline kauboikapitalismi aeg-ajalt üsna.
Kas õnneks või kahjuks ei kestnud väga kaua ja kuidas, kellele kellelegi, see on võrreldav protsessid kuuliga kuklas, teised lõpetasid praeguste tippmiljardäride hulgas keskuda, mida jõudis kinnitada.
                 
Kõneleja 2:
Õli on, tuleme tagasi Lutsiferi juurde, et ta oli ikkagi üks hetk, kui Eestis nagu tipp tippkoht kõige populaarsem koht peaaegu et kus, kus käidi, noh, vähemalt nii on räägitud, nii, mul endal ka meeles. Kristjani populaarseks tegi.
                 
Kõneleja 1:
No see infohulk, mis seal oli sellepärast, et ütleme siis, kui see Bibi esindusel õitses mõtleme internetile, kus sai, näitab materjalid, sikutada oli ligipääsu väga vähestele. Enamik solite tõsiseid töötegijaid, kes tegid tõsist tööd allitaksele ja oma kolleegidega seal vahetasid email'e ja saad sõita, eks ole. Nemad ei kaevanud või mingi mingisuguseid suuri raamatute, andaadiorite, igasugusi ladusid pidi noodid toppinud. Tänu sellele kuskilt ei jõudnudki mujale. Ja äikesekeele Lulike Bibi, kellelgi ei olnud jälle sellist finantsvõimekust, et sikutada kogu seda materjali lihtsalt BBC kaudu ühest teise, nüüd selle. Sest see oli väga kallis. Kaubivalmede mõttes.
                 
Kõneleja 2:
Ja sinu juures said kokku niisugune huvi, arusaam nende asjade väärtusest ja finantsvõimekus ja ja, ja seetõttu Sinuse staž või see hunnik, kus sul need kõik asjad olid, oli populaarne, sinna oli, oli popp külge tulla.
                 
Kõneleja 1:
Lauset igaveses reisis midagi põnevat. Ärme võtame kõike, alates igasul maagiasse Ocutiimis, Niifist kuni üle arvutikirjanduse, ulme ja Salfectioni, kokaraamatute ja värki, pioneeridest, retseptikogumik, keni välja. Kõike.
                 
Kõneleja 2:
Räägime korra neist onlain mängudesse korra mainisid, et sa hakkasid, mis, mis aastal sul see esimene mäng tuli?
                 
Kõneleja 1:
Toad, mina sattusin sinna, ma ei teagi seda väga eriti. Sest ütleme enne seda suur mängur, mõningaid üksikuid mängijad sealt sai toksitud, aga põhimõtteliselt oli interrattalise ma olingi kolase. Ja huvitaval kombel joad mudad.
Et seda tantsida, teatagondit. Tiit kao ja ma ise mängisin sinna, vedas mind ka üks päev kõva, noh nüüd nüüd on ta iga puga. Jaanus Lillepärg nüüd ERR-is ja tema oli Mul esimene TM, see oli vist 94, roomasin mängus. Ahaa. Ja noh, praegu siiamaani saab seda mängitud. Käime siin vaikselt ja toksima korra nädalas täringuid ringi ja.
                 
Kõneleja 2:
Väga põnev, sest mina olin ka 90 93, mina sattusin Tartusse ja minu jalg küll ei puutunud nagu dientii, nagu maailma peale.
                 
Kõneleja 1:
No see oli vähevõimelistele, tõi Tartusse tõi sellised mehed nagu Arlisid Arusbergiauukja, kes siin no niivõrd sööda siis üheks kõige esimeseks maaletoojaks Henryl on, jättes video ajaduta Vormsi Enn.
Läksid poodi ja ostsid siis selle teised išinised, tantšernandaja kontsed, kõik tantšer, materjali, raamatute, säält pukid Jaagupi kuskil arutluse panna seda asja, aga siis, kui süvenesid asjasse, siis oli aga kõva ja see oli see kunstiinstituudi punt, kes seal mängis. Mees Mihkel näiteks väga kõva tantsija vastane kõik ja siis, kui sealt lõpetanud tõstis omaette üks ports Tartusse kolis, siis tuli nendega koos kose tindi ja siis sai mängitud ja siis oli mul seal vahepeal 90-lt sel keskpaik ja nii edasi. Või ikkagi hullumeelne aeg, kus ise mängisid näiteks kahes mängus ja tegid ise kolme, nelja mängu, need noh, otseselt terve nädala iga päev oli mingisugune seltskond.
                 
Kõneleja 2:
Teised tegid mängu, siis söed süsteemselt tantsimisest ei näe seda fantaasiat saada, see ei ole niisama.
                 
Kõneleja 1:
No võiks öelda, et jagub tagasi mängimas käia, praegu siiamaani käivad ja painavad senior siin üks seltskond filmiinimesed Aablima ingliskeelset mängu hakkaks tegema seen ja osa nii jah, ja kes seal on kauplik? Põnevil sellest, aga no vot, küsimus ongi Teeemmide vähesus, kes viitsiks oskaks teha? Sest ma olen mõned korrad sattunud mängu mängima, kus käib asi niiviisi? Näete nädal aega midagi, jõudu nüüd kari lendav lõvisid AK lööma. Süttigi veeretavad täringut, eks sellel kolm tundi tärin, klõbisuke kakleb nii, nüüd on lubidul, lähete veel kuu aega midagi juhtu, sisulist mängu nagu ei olegi. Teine kui sa võtad kätte ja laseb rahval mängida ja tõmbadele konkse ja igasugu asju üles vaatan ühes mängus oli, kus kõik mängisid nii hästi oma osa ja loomulikult kõik eri asju vaatenurgale, ka teadete ülejäänud rahvas, kuule kus tegelikult selles muudmist, mida pidime, liikusime.
Kogu asi käis omavahel, kus keegi ei tea, kõik katsutasi, et seal on mingisugune kuligollel vaidlanejale ja siis kuidas seal morjobitsolanti ja siis keegi, kes veel kellelgi nad omavahel igavesi lahingud lõi, visiimis kirju ja nii edasi. Ja kui pärast siis sessori kokku tõmbasime siis nagu avalikud asjad. Nii et kõik said ikka teise korra, kellele laadad?
Mina arvasin seda ja kõik ja siis see viimane asi oli need kõik rahul, naersin, et Vaksid, püksid märjad oli see, kui sissimis kerkis mängisusest printsikest aadlipoissi. Kössis.
Kuule, see ainelise tantsijanna arvutimaailma
D värke ja ja siis läks siis neelad, siis ei olnud aega enam isegi sedasamast Taevolitajaga midagi, sest ma ütlen, noh, aeg oli veetud real. Parimatel aegadel Everkest ühes ma jõudsin, nii et ma olin maailma seitsmes Warrior.
Et tuli meeldisin üldse betoon või kõrgemale tõusnud Reinserina, tema oli muidugi veel hulk, kuid nagu ma ütlesin, minul tuli keskmine mänguaeg neli pool tundi ööpäeva kohta, ütlesid samal kuus pool.
                 
Kõneleja 2:
Oh jumal, see tahab ju, see on ju selline investeering.
                 
Kõneleja 1:
Lauligi sellel ajal ja.
                 
Kõneleja 2:
Kui ma siin niimoodi kuulan, siis sealt tuleb ju niukene.
Tulevad inimesed kuidagi väga-väga palju välja, sa kuidagi tundnud inimestega nagu hakkama saavadki neist aru saada.
                 
Kõneleja 1:
Noh eks siin ole ka, laulsin praegustki tilde juhtinud.
                 
Kõneleja 2:
Et selles mõttes ka, et ega sul seal Venemaal nende inimestega nagu jutu peale saada ei ole ka nagu lihtne, eks ole, see tahab nagu pealehakkamist ja, ja inimestega suhtlemise oskust, eks ole.
                 
Kõneleja 1:
Seal see on hoopis teine, kui te seda ei mõista ja praeguse aja ütleme nooltel ja kõigi rongi see küll jätame poliitika kui sellisele pole, paljudel äri ajamis, ongi senised Palf lääne inimesed pole üldse ei saa aru. Et kui tuleb mõõta piinla selle peale näiteks laoaed, selle tehingu väärtus on nüüd siis meil 20 miljonit niisi teemesi niiviisi, et viis miljonit ei saa aru, kuidas, mis, miks. No me anname ühe arve, kus kolm miljonit Šveitsi palkavad.
                 
Kõneleja 2:
Sa saad ju sellest nagu eestlase hingest ka aru. Kui sa ikkagi sa saad aru, mida inimesed nagu vajavad ja mis neid huvitab, kasvõi selle Lutsiferi püsti panekuga.
Aga kus sul see tuleb, see on sul lihtsalt sündinud sellega või oled sa mõelnud?
                 
Kõneleja 1:
Aga võib-olla son oskus just nimelt nii palju analüüsinud, nagu ma ütlen, et kuule juutimis asjaga ma olen tegelenud palju, ütleme, sai käidud kõik ülikooliaegsed, muu rahaliselt, selle võttis ja rüüpas Eues oma elu, siis mul kõik suved-talved olid kassal veel suusamatkadel, suvel mägimatkadel alpilaagrites. Et ma olen hästi palju neid matkagruppe juhtinud.
                 
Kõneleja 2:
Juhtimise soon juhtimisel.
                 
Kõneleja 1:
No seal ongi see, kuidas asja teha, ainult mingisuguse arvutifirmasid, erinevad on saanud juhitud töö, kõike. Et paljusid asju ma olen intsatiivsetel, ma mäletan, kui ilmuse sattus kätte raamatu, kuidas võita sõpru, inimesi, karnishi oma, siis paljusid asju kitsa, kuidagi instinktiivselt.
Sa oled inimene, teeks siis selline olukord, ta ise tahab. Ta ise tahab niiviisi mõelda. Vot see on meie poliitikute üks suur puudus ka, et kõik tahavad kedagi juhtida, sunnida öelda, sa oled väga loll, sa mõtled valesti. Anna parem võimalus talle, et, et mina tähendab minu arvamus ümber pöörata selle arvamuse peale, mida ma tahan, ta tegelikult teeksime.
Hoopis rohkem tuleks vaid saavutus ja võetakse palju inimesi. Nad tahavad keegi midagi ära, otsustaks keegi midagi ära teeks. Nad ei ole huvitatud sellesse, nad peavad oma poja peaga mõtlema. Mis veel hullem sinna mõtlemisega tagajärjel tehtud tegude eest vastutama. Probleem inimeste jõle hea näidata, valitsus on loll minister on küll, euroliit on loll, punu trump on jumal, taevas on ka loits, lõpuks ainult mitte mina.
                 
Kõneleja 2:
Ja siis sellest järeldame, et sina oled ka loll.
                 
Kõneleja 1:
Õlu pulk, see võtab päris kaua aega, enne kui võiks hakata vara kreeklase kombel ütlema, et ma tean, et ma midagi
Teadus on juppe võimsalt edasi. Kõik need kvandid ja mustad augud ja võtab seal pargitaoline. Mis edasi, kuidas edasi? Kust see kõik tuli, suud Paukaks, paugu tegi enne suurt pauku kui ütleme niiviisi, et enne midagi nüüd korraga tuli maailm, sest see ongi nagu manu. Selles mõttes jumala mõiste, kui me hakkasime mõtlema, mingit valja habemega taati, kes karjasekepp käes pilve peal jalgu kõigutab või loodus, seaduste, loodusteaduste, looduse enda kompleksides.
                 
Kõneleja 2:
Teil on väga sarnane küll.
                 
Kõneleja 1:
Kui sa oled juhuslikult lugenud Ivan tihke kosmoselendude päeviku Joom muidugi mäletad seda, kui ta tormis äikseilmaga maja ukse taga koputas, sisse tahetud lasta. Ja siis, kui lõpuks lastis hulluteadlane eestlastel oma ülakorrusele neid plaadikaste, kus lootine noor neiu
Aga äkki ma olen ise ka plaadimängijat kellegi tolmunud pööningul? Ei tea, vaata uue efektid ja kõike parapsühholoogia minu arust ei olnud väga ilusti illustreerivad seda ära. Aga ma ei tea seda, mis seal seda kontrolli. Sind huvitavad siuksed asjad, aga noh ütleme peaaegu kõik räägivad, kes tõestab jumala olemasolu, kes ei, ta töötab, selle mitte olemas. Mina olen võrrelda seda sellega seoses, kui meil varbaküüne üks rakk hakkaks tõestama inimese olemasolu mitte olemas. Kudetati.
Selles mõttes on meie professor Viktor Palm eksida orgaanilise keemia professor, kui ta luges, kõigasid teadusliku maailmavaatele see tõlke sest ta ütles tol ajal väga julgelt 81. aastal ikkagi et tema arust on näiteks teaduslik kommunism ja ateism täpselt samasugused pseudoteadlased, teadlased, teadvus, diis, teaduslik jumalad, nendel asjadel kui Uma või teadusega mitte midagi.
                 
Kõneleja 2:
Et juut 81. aastal öeldi auditooriumis
                 
Kõneleja 1:
Käsi kõrval Saprodeerimas sellisel juhul kusjuures noh, see ongi noh, miks nimetada mingit asja näiteks teaduslikuks seal ei ole teadusega mitte mingit pistmist.
                 
Kõneleja 2:
No oli vaja nimetada kuidagimoodi, et oleks, oleks uhkem.
                 
Kõneleja 1:
No eks ta praegu on ka, kui me vaatame, igasugused majandusteadused olid nii edasi, jube palju soolapuhumist, üks mees võtab, need meetodid tõestab ühe asja ära, ütleb must, teine tuleb. Ütleb oli, ei, keegi ei tõsta. Praegu kõik on valge. Pulma saigi, pallis, räägib üldse kokku sulatusplekkidest. Ma võtan siit, mis on täpselt see, kellele mida vaja.
                 
Kõneleja 2:
Meil hakkab tasapisi aeg otsa saama, sellepärast küsin selle kohta, mis seal praegu need praegused, need kirjatööd väga palju. Kus oli kirjutamise juurde jõudsid, üks asi on lugeda palju, teine asi on kirjutada palju.
                 
Kõneleja 1:
No miski aeg tagasi kahe tuhandete alguses töötas mu naine sellise ajakirja nagu arvutimaailm, peatõmmet.
Ja, ja noh, kuna ma tegelesin kõigi nende asjadega mind kirjutama siis oli HP parasjagu, on mingisugune suur järjekordne konverents turilise tore iga idee ajakirjanikke ka mööda maailma loogiliselt mitte nagu nüüd, kus elav keegi huvi ei tunne meie vastu ja võetakse autoajakirjanikke kes teeb kohe suisa kadedaks.
Ja siis nende üks kaasünniteks pidin minema selgunud tema pass ära. Äsja üle piiri.
Ja siis ta hulled, sul pass, korras, luulet, seda värki vedela lambiga. Läksin sinna, tegin ära ja pöördusin nii põhjaliku ülevaate, nii sisukat, nagu ma olin diaato masinatega ka juba aastaid aastaid kokku puutunud varem kui ta selles mõttes. Ma ei ole kunagi sellist teadust õppinud. Aga vaatad, kuidas praegu tulevad põlvkonnad on hädas, kasvõi tossi käsureaga. Kuule ikka hiirega Madistatavad värvilist ekraani ees ei ole siis kaks käppa püsti, tähendab. Aga ma olen sellega kõigi nendega üles kasvanud, nende arvutid, mis mulle endale läbi käinud, järjest arenenud kõik asjad. Ja, ja loomulikult, kui sa nendega tegeled, siis sul tekib huvi. Luule ei ole vahet, kas sa müüd kartuleid, arvutati kaalikad, eks ole. Ja tegin selle ära pärast tooli kohe, oi kuule, tead, siin on jälle asi ja tegelikult ongi nii, et väga palju nüüd ütleme nooremast põlvest, on ka neid, kes ei tunne seda raudtee olevat, rauaajakirjandus on see tegema, sa pead teadma, oskama küsida ja eks ma nende bassein nagu neile silma ka kukud, igale poole saatma. Teine asi on see, et mul ei ole nagu mainisite, Bratiimse läbidame, mingi kohutav, see näiteks, kui ma nüüd konverentsi saame kokku näiteks intervjuu presidentidega või siis, kui uutel liini oli veel see kõige suurem võõrad käed juurde, pita pihku ja aetakse juttu, küsita käest kõike. Ja kui paljud mehed näiteks põhimine, tehnikaohvitser, vitsa, president, mitmetel inte lülitutes rumalal käidud, talve eemaldunud, seal tuli kohe juttu rääkima. Seal tähendab, kas asjast aru? No eks see oli ka üks asi, miks siis nagu kutsuti, miks uite obiilsega mõtlen praegu see asi ära vaibunud? Omal ajal sai ikka. Ma olen vist neli korda USAs käinud Singapuris ja ma ei tea, mitu korda Koreas Hiinas igal pool. Euroopat räägigi seal vaevu pidevalt üks konverents ise järele, aga nüüd on kuidagi itifirmad nii maha vaibunud. Midagi.
                 
Kõneleja 2:
Ehk siis see on vist saabunud see ka selles vallas nii-öelda pudukaupmeeste ajastu.
                 
Kõneleja 1:
Tiiteriga, see Nad tulid turgus šveid sisse, kes saab elu, kes teeb, kellelt kirjutatakse kõik olid väga põnevil sellest asjast. Aga eks nüüd on kuskil turu stabiliseerumine on käes, kuskil seal ei kasva näiteks ütleme, kasvõi lauaarvutitel äravad ei kao, nagu paljud lennus oli, sellepärast et noh, mängurites tahavad ikka suure neli ka ekraanidega korralikult mängida. Räpukaraagil, juhul siin on küll läpakas päris kõval asjas sees ja kõik ikkagi nii võimelist seal Allablokid, võib-olla selgamise lauaarvutis, Paulikule ja neil on oma nišš, on olemas ja aga samal ajal niisugust huvi ei ole konkurentsi mõttes, nagu on Udo kõrvadele.
                 
Kõneleja 2:
Nüüd ma küsin, viimase küsimus. See oli, tõmbab joone alla selle küsimusega, mis asi, mis viimane mäng oli, mis sul tõstis nagu karvad püsti, et see on nagu äge asi. Viimane tõesti äge arvutimängud.
                 
Kõneleja 1:
Kui sa mõtled seda mingit Nivea, mis on tulemas, siis sellel on Mandjana Benton, Panther idee, praegust, selline mees nagu Brad McCoy, kes oli ka esimese Everkesti taga peamiseks ajuks. Ja selline, mis on grupitöö, mõtlen, ma olen ise koos oma paljude kaaslastega tragunissid olid kunagi ja nii edasi, nii-öelda meie kutsume seda mängult kuldajastuks, mida paljud moodsad mängurid sõimavad, meemängijad, neid voodite ja publik temisel, kiirepi paun on mainimata neid seljaajumängudeks. Ega seal muud pole vaja, nagu öeldakse, af pale õppi kiiresti punast nuppu, mängib mänge, selliseid, kus sa pead ikka tõesti strateegiat, kombinatoorikad gupide juhtimisel, suured ai, see oli mingi teine ajastu, jood, seda aega me katsume ja seda lubab Bradmaco. No ma olen, moodsad mängud on, mis on tulnud, ma olen reaalselt osasid saadetud, osasid olen ostnud, mõned on kõige jubedamad, kus on sul see, et no raha tõid endale elu osta, eks ole? No siin hiljuti mängisin seda foolaud 70 kuutel parasjagu huumoriga jälle kuskil levelid on kõrgel midagi hääl kõik viimseni tehtud minna käia, kogu aja korjata sodi, et seal rohi maha müüa tiimile kuuega mängitud ja kõik, aga kui meenutada tõsi diaat mängu viimasest ajast jätame Aivar, kes sinna kaugustesse siis üks hea nimi oli Folerörs ka niisugune ja tuumasõjajärgne maailm, mis oli päri, ma ütleksin, kõige parem Grastimise ja mängu siis mängijate vahelise ärisüsteem oli tulles mängu siis muidugi valan klassika loetelu sealhulgas Online Kersoniga Tolkieni vanni kuumalaska loendaksele, lisaks sajab fikse loomult tatari keeles, kõik oli väga kõva sõna ja see kindlasti mängudest väga kõva nimi tuli ju väga paljulubav, aga jälle on raha sai otsa ja teiselt poolt oli jälle see kahjuks suure fännibaasil KM minema, oli Secret World. See on nagu murran Horeid, tüüpisin kõik laugurarhitja põud ja need selline maailm, kus mingisugune must maagia või kaasa ja maagia maailma tungidessaa ühingud võitlevad selle vastu ja uitavad omavahel seal templirüütlid, illuminaadid, Dragon'i, Tiinas ja ja seal seal ma ei ole näinud nii mõttekaid ja põhjalikke keerulisi paste Ühegi mängu. Enamikel on seal, et noh, ülesannete värka on muutoniga nii labaseks, mine keldrisse tappa 10 rotti, nüüd suur kangelane, maailma vähestamine uuesti kehtestada 10 rottide sabaga ära.
Millal seal on kohti, kus sa pidid näiteks võtma midagi kiilkirjas saviteadvusele, hieroglüüf vaatama ja muidugi on selge, et keegi seda nii täpselt ei tea neid asju. Siis oli see, et sa mängu sisse oli kee ta kohe mängust välja minema. Muti lastele mängu sisse veetud Google'i brauser. Sa said sealt abi otsida ja näiteks seal oligi üks jälle, kus oli mingi vihjemgi nimega, leidsid laiba ja seal oli boss, kaart oli seal peal kellelegi saatnud tabel. Ja siis palkasid guugeldama seda nime siis tuleb välja, et see oli üks saksa kõvemaid krüptograafia alus, panjeed väga vähe teatakse. Otsid siis välja? Tal on olemas spetsiaalne algoritm. Jää, kes tahtsid, võisid seda algoritmi käsitsi kasutada. Aga sa võisid selle programmi alla tõmmata. On need reaalsed sisse kopeerida? Tahtsid raisk, politseist pildistasid ekraanilt ära, otseeerrorisid selle teksti sinna masinasse, sealt tuli siis tagasi mõtestatud tekst, kus valitses korjevikusta minema pidid.
                 
Kõneleja 2:
Hea meel näha, et sa jätkuvalt elavad huvitavalt elu.
                 
Kõneleja 1:
Noh, loodame nii, kaua sul see antud on? Loodame, et Kauks aitäh sulle.
