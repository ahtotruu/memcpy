\label{chptr:lucifer}
\index[ppl]{Tamm, Veiko}

\question{Kuidas arvutid 
sinu juurde said?}

See on komplitseeritud küsimus. Olen ise ülikoolis keemia eriala\index{Tartu 
Ülikool!Keemia Instituut} lõpetanud, aga ei oma neljanda kursuse tööd ega 
diplomitööd ei ole ma ühegi kolvi ega katseklaasiga solberdanud, sest mind 
kutsuti tollal üsna põnevasse asja. Nimelt oli mu juhendajaks Mati 
Karelson\index[ppl]{Karelson, Mati}, kes alustas arvuti ja kompuuterkeemiaga, 
ühesõnaga kvantkeemiaga. Sealt olid minu esimesed kokkupuuted arvutiga. Meie 
tööväljaks olid alguses viieaugulised perfolindid. 

\question{Kas ta üritas juba tol ajal teha nonde arvutitega kvantkeemiat?}

Jah, see asi hakkas pihta juba kaheksakümnendatel.

\question{Kas sul enne seda ei olnud üldse mingit kokkupuudet arvutitega?}

Matemaatikakursuse käigus näidati meile väga lühidalt sellist arvutit 
nagu Nairi-2\index{Nairi!Nairi-2} ning selle PA ja AP keelt.\sidenote{Vt ka lk 
\pageref{sisu:apkeel}.} Seal sai \enquote{Hello World!} trükkida ja 
\emph{that's it}. 

Hiljem töötasime 
Minsk-32\index{Minsk!Minsk-32} ja Jessukesteks\index{Jessuke|see{ES 
EVM}} kutsutud ES-1022 arvutitega\index{ES EVM!ES-1022}, mis olid ülikooli 
arvutuskeskuses. Arvutamine käis nii, et perforeerisid oma programmi sisse, 
viisid sinna ja arvuti lahendas. 

\question{Mida need programmid tegid?}

Oli mitmesuguseid kvantkeemia meetodeid. Aatomorbitaalid, kuidas 
keemilised sidemed moodustuvad, kuidas elektronpilved suhtuvad üksteisega ja 
kas see aitaks seletada keemiat ning lähendaks arvutuslikke tulemusi 
reaalsetele. Võtsime lihtsad kahe aatomiga 
molekulid ja nendega sai hakkama. Aga keerukamate, näiteks metaani 
ehk \ce{CH4} molekuli väljaarvutamine nõudis suurelt arvutilt umbes kümme 
korda rohkem tööd kui terve lihakombinaadi aastaaruande väljaarvutamine. See 
oli päris kõva ja kallis arvutiaeg, mis sinna alla läks.

\question{Kuidas see käis? Kas juhendaja ütles: \enquote{Veiko, hakka 
programmeerima} ja sa hakkasid programmeerima?}

Ei, programmeerimisest jäi asi kaugele. Natukene keeltega tuli küll
pusserdada ja ka mõningaid asju Fortranis\index{Fortran} 
kirjutada. Midagi ma aru sain, aga programmeerimist ma ei 
olegi selgeks saanud. 

Siis katkes see asi tükiks ajaks ära. Taaskohtumine arvutitega oli nüüdseks juba üle 31 aasta tagasi, kui Tartu Tähetornis\index{Tartu Tähetorn} ajas 
tol ajal suur arvutifänn Enn Kasak kokku arvutihuviliste ringi ja hankis sinna 
arvutid.

\question{Mis aastal see oli?}

See oli 1988. aasta suvel. Sel ajal algas suur kooperatiivide ajastu ja sai 
tehtud kooperatiiv Tähetark\index{Tähetark}, mille liige ma olin ja mille 
eesmärk oli hankida planetaarium. See sai isegi ostetud ja 
Füüsika Instituudi\index{Tartu Ülikool!Füüsikahoone} fuajeesse üles 
seatud, aga ega sellega mingit tööd tegema ei hakatud. 

Samas oli see teaduslik keha, mille varjus sai arvuteid 
osta-müüa. Arvutimüügiga oli üldsegi nii, et hiljem keelati igasugustel 
väikestel asjadel see ära. Ehk oled kunagi näinud vene seriaali 
\enquote{\begin{russian}Бригада\end{russian}} segaste aegade maffiaelust, kus äritseti 
kõigega? Juhtuski niiviisi, et mul kästi tuua Venemaalt 
arvuteid. Käisin Peterburis ja Moskvas, sain otsekontaktid ja sidemed ning 
nii see arvutiärisse sattumine järsult puhkeski. 

Arvutid olid tol 
ajal hirmsalt kallid. Näiteks minu esimene arvuti Amiga 
500\index{Amiga!Amiga 500} maksis sama palju kui tutikas 08\sidenote{Nii 
kutsuti autosid VAZ-2108, tuntud ka kui Lada Samara. Tegu oli oma aja ja 
Nõukogude Liidu kohta innovatiivse autoga -- Samara oli teine (esimene oli siiani 
populaarne Niva) ise arendatud AvtoVAZi mudel ja esimene, mis ei tuginenud 
Fiat 124 mehaanikale.}, kuskil 40 000 rubla.

\question{Miks sa Tähetorni läksid? Kas arvutivärk tõmbas?}

Sõber kutsus, et tule kaasa, põnev värk.

Mäletan, kuidas kogu värk tuli algusest peale endale selgeks 
teha: kuidas vaadata, kui palju opmälu on, ja testiprogramme kasutada. 
Alguses kuiva trennina, aga asi hakkas järjest liikuma. Ja kui endale 
sai arvuti koju ostetud, siis oli jube põnev. Alguses rändasin üle Amigate: 
500\index{Amiga!Amiga 500}, siis 500 Plus\index{Amiga!Amiga 500 Plus}, seejärel 
1000\index{Amiga!Amiga 1000} ja lõpuks Amiga 2000\index{Amiga!Amiga 2000}. Ja siis 
tekkis sinna kõrvale esimene PC, sest tuli üks suurem tellimus. Arvuteid saada oli 
väga raske, sest igal pool oli nende sisseveo embargo. 
Tõime neid Moskvast ja 
Peterburist.\label{sisu!veiko_moskvas}

\question{Kuidas arvutid Moskvasse ja Peterburi said?}

Näiteks Moskva kanal, mille kaudu ma embargo-arvuteid tõin ja 
äritsesin, tuli läbi saatkondade. Singapuri saatkonda toodi arvuti paljundusmasina 
kastis ja Lumumba\sidenote{Moskvas asutati 1960. aastal 
Vene Rahvaste Sõpruse Ülikool, mis 1961. aastal nimetati Kongo poliitiku ja 
vabadusvõitleja Patrice Émery Lumumba auks ümber Patrice Lumumba Ülikooliks. 
Kooli eesmärk oli toetada vastselt koloniaalsõltuvusest vabanenud riike, 
koolitades sealset tulevast teadustehnilist eliiti. Praktikas oli tegu Moskva 
ühe vähese ülikooliga, kus õppis suurel määral välisüliõpilasi.} üliõpilaste 
kaudu läks see kohalike ärikate kätte, kellelt sain siis mina osta. 

\question{Selline tarneahel!}

Jaa, tarneahel oli päris võimas, kusjuures hinnad liikusid tarneahelas väga 
põnevalt. Iga lüli pani julgelt 30--50 protsenti 
otsa, aga väga sageli ei viitsitud hakata liigutamagi, kui 100 protsenti 
kasumit ei olnud. 

\question{See oli ju riskantne äri?!}

Oli jah, tean väga palju tuttavaid, kes said kuuli ja nuga. Olen isegi 
kihutanud öises Peterburis punase tule alt läbi, saba järel. Kui 
ikka sõidad sinna, kolm miljonit rubla sularahas seljakottidega kaasas, siis 
on riskantne.

Arvuti hind ülekanderublas (Vene ajal nimetati seda \begin{russian}песналик\end{russian}) 
oli kaks miljonit. Kui tõid selle sularahas, said arvuti kätte 1,4 miljoniga. Aga kui maksid valuutas, 
võisid selle ühe miljoniga kätte saada ümberarvutatud väärtustes. Teine 
lisaväärtus Moskvaga tuuritades oli see, et Moskvas, 
süva-Venemaal, hinnati Saksa marka ja USA dollarit. Meil siin vastupidi 
olid hinnas Rootsi kroonid ja Soome margad, millega ärikad äri ajasid. Ostsid siit kokku Saksa margad ja dollarid, läksid 
Moskvasse, maksid nendega ära ja kõik need margad ja dollarid, mis üle 
jäid, vahetasid seal Soome markadeks ja Rootsi kroonideks. Tagasi tulles
vahetasid need ära ja ainüksi selle nii-öelda ülejäänud valuutavahetuse eest võtsid 
rahulikult iga raksu pealt sada tuhat vahelt. 

See oli hirmus aeg. Siis eriti enam ei kütitud, kuigi nõukaaja lõpuni kehtis 
ju valuutaseadus. Suurtes hulkades valuuta äritsemise eest võis saada 
seitse aastat. Suureks hulgaks loeti rohkem kui sada 
dollarit, aga seal sai tuhandetega arvestatud.

\question{Kes neid arvuteid ostis?}

Kõik ajasid taga: riiklikud ettevõtted, instituudid \ldots Kui ma oma 
esimese PC ostsin, siis see läks Tartu Ülikooli 
füüsika-keemiateaduskonna peaserveriks. See oli 286, 20 MHz -- võimas masin! Tavalisel masinal oli ju ainult 12 MHz. Opmälu oli sel kaks 
megabaiti ja kõvaketas oli 120megane. Teise sellise arvuti ostsin 
endale. 

Muidugi käisid paljud tuttavad, kes ka arvutitega tegelesid, vaatamas, 
et miks sa, loll, sellise endale ostsid! Mida sa teed selle arvutiga? Kuna 
sa arvad selle 120 mega täis saavat?! Mulle siiamaani 
meeldib Enn Kasaku\index[ppl]{Kasak, Enn} paralleelne näide, mida olen pidevalt kasutanud, et kui autoteadus oleks samamoodi arenenud nagu 
arvutiteadus, siis sõidaks Mercedes praegu valguse kiirusega, võtaks 10 000 
kilomeetri peale tilga bensiini ja maksaks pool senti.

\question{Tundub tõepärane. Mis tolle 286 serveri peal jooksis? Novell?}

Ma ei mäleta, mis sinna pandi. Põhiliselt jooksutati 
ikka Unixeid, näiteks V5\index{Unix!System V} ja muud säärast. 
Meie teadlastel käis välismaa rahvast külas ja kõik väga imestasid, 
et nii palju tehakse Unixitega. Aga kõigi Eesti Unixite seerianumber oli üks, 
tarkvara hindu arvestades peale piraatluse muud võimalust ei olnud. Kes siis 
andis mingi tarkvara jaoks sellist raha!

\question{Ühesõnaga, kuna ühte ega viite ei jaksanud osta, siis piraaditi viis ja 
oligi edusamm!}

Nagunii ei olnud ei ühte ega teist, aga siis saadi vähemalt midagi teha!

\question{Mis aastal see PC lugu oli?}

Vist 1989. aastal.

\question{Järelikult sa ketrasid ennast Amigatest väga ruttu läbi PC peale?}

Jah, kogu aeg vahetasin. 

Üheksakümnendate alguses, kui Eesti hakkas iseseisvuma, oli mul kõige esimene 
kodu-386. Jälle imestati, milleks seda vaja on, kes sellise asjaga tegeleb ja mis 
ma sellega teen. See oli tüüpiline.

\question{Sul pidi suur huvi olema, et arvuteid nii sageli 
vahetasid ja ka kooperatiivi jäid?} 

Kooperatiivindus meil väga kaua ei kestnud, sest kiiresti 
tuli peale seadus, mis keelas mitteriiklikel ettevõtetel ja 
kooperatiividel arvutitega äritsemise. Ja kuna mul olid sidemed olemas, polnud 
kooperatiivi enam vaja. Leidsime paari tuttavaga kliente ja 
tekkis küsimus, mida meil üldse vaja on. Tootmisruume 
ei olnud vaja ega ka ladusid või tooraineid. Mida 
vaja on? Raha!

Mõtlesime, et Tartu Kommertspank\index{Tartu Kommertspank}\sidenote{1988. 
aastal tegevusloa saanud Tartu Kommertspank oli esimene aktsiaseltsina tegutsev 
ning ka välisvaluutatehingute litsentsi saanud kommertspank NSV Liidus. Panga 
tegevus lõppes pankrotiga 1994. aastal. See pank oli mõnes mõttes oma aja 
tõeline sümbol, põledes heledalt ja kiiresti. Ka Hansapank\index{Hansapank} 
alustas tegevust Tartu Kommertspanga filiaalina.} on tore koht! Läksime 
Veetõusme\index[ppl]{Veetõusme, Ants}\sidenote{Ants Veetõusme, kes oli kuni 1990. 
aastani Tartu Kommertspanga juhatuse esimees.} jutule. 

Poisid ütlesid, et küsi kuskil sada tuhat. 
Räägi, mida vaja on, ja et selleks võiks olla oma raha 
loksutada. Hinnad olid sellised, et saja tuhandega sai juba 
enam-vähem masina osta! Äri käis ju kogu aeg ettemaksetega. Raha tuli ära 
ja sa võisid seda tükk aega pööritada, siis ostsid masina ja andsid kliendile 
kätte. See, et kuu aega tuli oodata, oli tavaline nähtus. Kui tulid suuremad 
summad, siis meil oli käsi üsna hästi sees Novgorodi elektroonikatehases, mis 
tegi Panasonicu MV-25 pealt maha viksitud Vene videomakke VM-12. Sinna minnes oli kaasas kastitäis Vana Tallinnat, 
paar kasti suitsusinki ja meie oma suitsukana, et häid suhteid luua. Kõmm Novgorodi ja furgoon
videomakke täis! Neid me ei viitsinud üksikult müüa, vaid müüsime hulgi 
koperativštšikutele maha, kes oma kooperatiivipoodides parseldasid edasi. 

Sellist rahakeerutust sai tehtud kogu aeg, aga siis mõtlesime, et võiks olla 
käibevahendeid. Algkapitali, nagu öeldakse. Tavaliselt tead, et kui midagi 
küsid, siis niikuinii tõmmatakse maha. Rääkisin 
Veetõusmele\index[ppl]{Veetõusme, Ants} ära, et vot meil on selline arvutiäri. Ta oli 
väga huvitatud, kõrvad liikusid, et kas neile ka saaks. Ikka saab! \enquote{Aga 
mis te meile pakute?} Ütlesin, et 11\% -- meie teenime näiteks miljoni, teie 
saate 110 000. \enquote{Täitsa hea mõte! Ja palju te meie käest tahate?} 
Mõtlesin, et küsin rohkem, niikuinii kaubeldakse alla. Ütlesin miljon. 
\enquote{Ahah. Avage arve ja pange miljon peale!}

Muide, kui kommertspanga pankrot pihta 
hakkas, siis oli meie firma üks väheseid, kes pangale raha tagasi maksis. Oleksime 
võinud teha ka varifirmad, raha ära kantida ja külma 
teha, aga meie tasusime kogu raha ja kuskile võlgu ei jäänud.

\question{Kui sa Amigasid ja PCsid keerutasid, siis sul pidi ka olema huvi 
nendega midagi teha. Mida sa tegid?}

See oli omaette nuhtlus! Kui Tähetorni\index{Tartu Tähetorn} 
tulid esimesed MSX\index{Yamaha MSX} arvutid, siis ma töötasin veel
keemiainsenerina. Pärast tööd sõitsin bussiga allalinna, läksin Tähetorni ja 
istusin ja mängisin seal täpselt nii kaua, et oli aeg bussi peale minna ja 
tagasi tööle sõita. Vaatasin hommikused ringid üle, keerasin kabineti lukku ja heitsin
magama. Mängud olid naiivsed, aga nii põnevad! Ja kui endale 
arvuti sain, siis see oli košmaar! Järjekord oli pidevalt ukse taga, kõik tulid 
tasuta mängima. 

\question{Nii et sa mängisid?}
Jah. Sai muidugi ka igasuguseid asju uuritud ja kui tuli internet, siis \ldots 
Tegelikult hakkas selle maailmaga toimetamine juba enne, BBSi ajal.

\question{Millal ja kust sul tuli mõte panna enda 
BBS püsti?}

See oli väga varastel üheksakümnendatel. Päris internet jõudis Eestisse 
kahe satelliiditaldrikuga, üks oli Tallinnas KBFI peal ja teine Tartus 
Tähetornis. Ja sealt siis üle Rootsi Kuningliku 
Tehnoloogiainstituudi\index{Rootsi Kuninglik Tehnikaülikool}, KTH, käis meil 
side. Siis hakkas BBSindus vaikselt juba ära vajuma, kuigi töötas veel 
edasi. Mäletan, et internet jõudis Tartusse, kus ma tollal elasin, 
1992. aasta märtsikuus. Sain üle EBC ehk biokeskuse\index{Eesti Biokeskus} endale 
isikliku \emph{account}'i juba aprillis, kuu aega hiljem.

Tol ajal oli meie kontor Rüütli tänavas, Treffneri kooli\index{Hugo 
Treffneri Gümnaasium} vastas. Ägedad trefneristid käisid seal muidugi 
hoolega arvuteid näppimas. Üks põhimehi, kes seda asja suunas ja üles pani ning tarkvara poole pealt
majandas, oli Einar Entsik\index[ppl]{Entsik, Einar}, 
praegu kõva kinnisvaraärimees. Tema oli meie peamine \emph{sysop} ja mina 
olin \emph{co-sysop}. Hiljem, kui kontori likvideerisime, siis Lucifer 
BBS\index{Lucifer BBS} tegutses mul kodus edasi, kuni peaaegu lõpuni, kui 
BBSi maailm hääbus. Siiamaani mäletan veel oma aadressi: 2:491.666.

\question{Millest selline nimi -- Lucifer?}

Ta tõi valgust maailma! 

Meil BBSis oli muidugi väga palju igasugust maagiat ja värki. Olen oma 
loomult olnud eluaeg antikristlane ja nüüd olen veel suurem 
antiislamist. Seda ma ei suuda üldse taluda, selle kõrval on kristlased 
väikesed voonakesed. Ja ma ei talu seda vaimupimedust, mis siin on, keskaega tagasi pürgimist! 

Mäletan, et oli suur jama, kui KAPO käis meie materjale uurimas, 
kui nõndanimetatud satanistid pussitasid Tartus Hando 
Runnelit\index[ppl]{Runnel, Hando}. Uuriti, kust materjalid olid saadud, ja kellegi 
kaudu tuli välja, et meie BBSis levis neid väga palju, näiteks LaVey \enquote{Saatana 
Piibel}\sidenote{LaVey, Anton Szandor. The Satanic Bible. New York: Avon Books, 
1969.}. Üks mehike uuris nii põhjalikult, 
et pööras täitsa ära ja hakkas ise ka satanistiks!

Kui sa küsid, kas ma olen satanist, siis ei ole. Kui ma ei 
usu Kristust, kuidas ma saan siis tema peegelpilti kummardada?

\question{Miks te BBSi tegite? Äri sai ju muudmoodi ka teha.}

See oli lihtsalt hobi, poisid tahtsid teha. See aitas luua sidemeid ja tõmmata igasuguseid materjale üle 
maailma. Tol ajal olid ju kaugekõned tasulised, aga BBSis võis tundide kaupa rippuda, helistada 
Ameerikasse, Iisraeli või Jaapanisse. Euroopa polnud üldse küsimus!

\question{Kust sa numbrid said, kuhu helistada? Kust võis näiteks üheksakümnendate alguse Tartus 
saada Jaapani telefoninumbri, mille taga oli BBS?}

Kõigil BBSidel olid suured \emph{listing}'ud, kus olid loetletud 
olulised BBSid maailmas. See oli nagu aadressi- või telefoniraamat! 

\question{Ja sa käisid seal infot lugemas?}

Jah, lugesin uudiseid ja infot. Nüüd istuvad kõik Facebookis, aga siis seda ju 
ei olnud. Siis olidki BBSid, mille kaudu vahetati infot ja saadeti meili. Hiljem, kui internet tuli, olid juba teised ajad. Et seal 
tegutseda, tuli endale UNIXi alused selgeks teha ja käsureal töötada. Siis ei 
olnud ju veel Linuxitki olemas. Põhiline oli Santa Cruz 
Operation\sidenote{Siit tuleb lühend SCO.} V5 UNIX\index{Unix!System V}, millega 
me kõik siin tegutsesime.

\question{Kas BBS käis teil UNIXi all?}

Ei, BBSil oli oma tarkvara ja UNIX tuli hiljem, kui hakkasime internetis 
käima. Ega siis kellelgi kodus 
internetti polnud, ei olnud võimalik saadagi. Pidid teadma, kus istusid sissehelistamise modemid, millega said ennast kaugelt internetiarvutisse 
sisse logida. Näiteks Toomemäel Tähetornis\index{Tartu Tähetorn} oli ja sealt siis 
liikusid siuh-säuh edasi internetiavarustesse.

\question{Ja kõik käis käsureal!}

Jah. Hiljem hakkasid tulema Gopheri-ruumid ja muud algelised 
otsingusüsteemid ning infopangad, kus oli erialaseid raamatuid. Siis tekkisid 
interneti-BBSid. Printa oli näiteks Euroopa suurimaid ja 
Iska\sidenote{Iowa Student Corporation Association.} BBS\index{Iska BBS} 
maailma kõige suurem interneti-BBS sellisel kujul -- nagu BBS 
teadetetahvliga, lihtsalt ligipääs oli interneti kaudu.

\question{Kas sa teiste Eesti sysop'idega ka suhtlesid?}

Jah, ikka, meil olid igasugused üritused: BBSummerid\index{BBSummer} ja 
BBWinterid\index{BBWinter}. Seal käisid nii sysop'id kui ka kasutajad. See oli 
päris tihe seltskond, kes seal käis ja omavahel ka niisama suhtles, 
nagu praegu toimub Facebookis. Samas ei olnud see seltskond nii suur, et 
ei oleks võimalik hallata. Katsu sa teha näiteks Eesti Facebooki liikmete 
kokkutulekut, võibolla jääb tulemata viis protsenti inimesi!

\question{Kui palju teil Luciferis\index{Lucifer BBS} liine oli ja kui palju kasutajaid küljes käis?}

Üksainuke telefoniliin. Kui kasutaja tuli liini külge, 
siis ta pidi seal rippuma, kõneaeg jooksis kogu aeg. Näiteks kui 
sikutasid Ameerikast mingit tarkvara, siis rippusid kogu aeg 
kaugekõnega liini peal -- päris soolane kopikas tiksus! Eraldi üüriliinid tulid 
alles ISDNi ajastul, kui tekkisid 64- ja 128kilobitised asjad. Kõige 
esimene modem, mille ma sain, oli 2400 boodi.

\question{See oli isegi kiire, sest 1200boodised olid ka levinud.}

Isegi 600boodised! Finlandia BBS\index{Finlandia BBS} oli 2400 peal, kõik ülejäänud 
olid aeglasematel. Tavaline modem maksis ka kaks korda rohkem kui sõiduauto 
Žiguli.

\question{Hobi jaoks tundub Žiguli kallis investeering!}

Mõni rikas mees korjab hobi korras vanemaid autosid, 
uunikume, mis maksavad ka praeguses rahas sada ja kakssada tuhat, 
teeb oma automuuseumi. See on samuti hobi, millega ju ei 
teeni ja piletit ka ei küsi!

\question{Mis sind just arvutihobi juures paelus?}

Seesama kübermaailm. 

Kui vaadata näiteks \enquote{Ghost in the Shelli}\sidenote{Masamune Shirow samanimelisel 
mangal põhinev frantsiis, millesse kuulub nii animeid kui ka 2017. aastal 
Hollywoodis linastunud film. Frantsiisi tegevus leiab aset postküberpungi 
maailmas ja selle peategelane major Motoko Kusanagi on küborg, kelle 
mehaanilises kehas (\emph{shell}) toimib inimese teadvus (\emph{ghost}). Tegu 
on kunagistes küberpungi ringkondades kultusliku teosega, mille mõju on 
võrreldav Willigam Gibsoni loomingu omaga.}, siis lahe oleks 
oma teadvus enne surma internetti üle kanda. Läheks küll sinna 
virtuaalmaailma tondiks!

\question{Kas sa Gibsonit ka lugesid?}

Jah, ikka.

Küberpunk ja muu selline värk oli \emph{must be} kõigile tollastele
arvutifriikidele. Kuhu ma eriti 
sisse ei jõudnudki, oli Muda\index{Muda}. Mina sattusin virtuaalmängumaailma 
siis, kui tuli \enquote{EverQuest I}\index{EverQuest}, kus sai 
järk-järgult läbi gildide mindud. Mängisin seda neli ja pool aastat jutti. 
Mängul oli \emph{online counter}, mis luges kokku, kui palju mänginud oled. Kui hiljem arvutama hakkasin, siis selle nelja ja poole aasta jooksul oleksin pidanud iga jumala 
päev mängima neli ja pool tundi. Aeg-ajalt sai 
kakskümmend tundi jutti suuri \emph{raid}'e peetud.

\question{Kas Gibson ja muu küberpungi kraam levis BBSides või olid füüsilised 
raamatud ka?}

Olid ka füüsilised raamatud. Suuremad fännid, nagu Jack\index[ppl]{Lippmaa, 
Jaak}\sidenote{Ilmselt peab Veiko silmas Jaak Lippmaad, mitte Jaak Loondet, 
keda sama nime all tunti.}, tõlkisid neid eesti keelde ja isegi 
avaldati. Samuti olid suured ingliskeelsed raamatuarhiivid. Ulmet ei olnud ju eriti saada. Teadusraamatukogud ostsid rohkem 
teaduskirjandust, ilukirjandust oli väga vähe -- see hakkas liikuma 
juba digitaalsel kujul. Tõmbasin digitaalsetest arhiividest raamatuid ja kõik, mis kätte tuli, läks
BBSis üles. Rahvas käis sealt
sikutamas ja luges hoolega. 

\question{Ja lugeja vaim sai valgemaks! Mis rahvas sul BBSis käis? Tudengid?}

Oli õpilasi ja tudengeid kuni arvutiinimesteni välja. Meil 
oli seal väga palju põnevat tarkvara ja arvutialast kirjandust, mida mujal eriti ei liikunud. 

Kuna Printas juhtisin ka üht tuba ehk olin moderaator, siis saime nende 
\emph{underground}'iga tuttavaks. Olin ka IC piraadigrupi\sidenote{Ei tea, keda Veiko silmas peab. Suuremate gruppide 
nimekirjast sarnase nimega seltskonda leida ei õnnestunud.} liige, 
seal liikus vahvat materjali. 

Oli täitsa kurioosseid olukordi. Tollal 
ei olnud ju päris pira-FTPsid. Tehti nii, et kui asi algas 
punktiga, siis oli see nähtamatu. Ja kuskil firmas oli FTP-server püsti, 
siis tehti kuhugi kataloogi, kus olid süsteemsed asjad ja 
kuhu tavaline inimene ei läinud, punktiga algavaid katalooge. Kui seda 
hakati avastama, tulid kasutusele muud asjad, näiteks 
kontrollsümbolid, mis tegid piiksu või reavahetuse. 
Ka Enteri vajutamine oli võimalik 
kontrollsümbolina kirja panna. Ja kui sümbol oli ees, siis pidid 
teadma, et sinna ette tuleb panna just selle kontrollsümboli, mis oli vist CTRL +L ja mis 
käskis lugeda järgnevaid asju kui lihtsalt tekstistringe.

\question{Ehk te munesite kellelegi FTP-serverisse oma piraatvara?}

Jah, niiviisi oli terve maailm täis! Kataloogide kaupa oli kõikvõimalikke asju.

\question{Suurde rahvusvahelisse piraatgruppi oli ju 
raske saada. Ilmselt päris avasüli vastu ei võetud?}

Suurte raskustega! Praegusel torrentiajastul ei kujuta seda ettegi \ldots

\question{Kuidas see sul õnnestus?}

Printa BBSi kaudu, kutsuti. Igasuguseid põnevaid asju sai uurida ja vaadata. Kui ei ole ise näppida saanud, siis ei oska ju midagi soovitada.

Vene aja lõpus müüdi mustal turul ja laatadel piraatkogumikke. Tõsisemates gruppides peeti aga
piraattarkvaraga raha teenimist 
väga halvaks märgiks, selle eest sai kohe kangiga vastu 
pead. See oli patt. 

Kui tuli välja Windows 95, oli see alguses kohutavalt suur saladus. Nüüd on 
Microsoft ennast täiesti teistpidi pööranud: tahate, võtke uusi versioone, 
uurige, tutvuge, vaadake! Nad on lõpuks aru saanud, et see ei takista, kui püüad midagi 
kinni hoida. Parem on miljon kuju, kes katsetavad uut 
toorest tarkvara ja kakuvad oma juukseid, mis lähevad 
halliks, kuna kõik hunnikusse lendab. Sealt tuleb tagasiside! Kõik sõimavad: 
\enquote{Parandage see ära, see on valesti!} Sa ei jõua endale nii palju 
töötajaid otsida, kes kogu debugimistöö nii põhjalikult ära teevad 
kui see vabatahtlik jõuk. 

Siis aga oli see nii keelatud, et kui tuli 
Windows 95 koodnimega Chicago, siis 
poisid tegid igavese pulli: tegid Microsofti peamises FTP-serveris 
punktidega kataloogid ja panid Chicago sinna üles. Ja 
kui see üle maailma kulutulena levis -- \enquote{Microsoftist saab pira-Windows 
95 tõmmata!} --, oi kuidas Microsoft siis marutas! Nad logisid ära kõik, kes seal käinud 
olid (ma käisin lihtsalt vaatamas ja irvitamas), muu hulgas ühe aadressi, mille kaudu 
mina käisin. Mul oli neid üle maailma palju, sest üks või teine modem oli kinni, aga
kuskil ikka leidus auk. 

Tuligi teade, et \enquote{karistage, võtke 
\emph{account} ära, see on igavene vastik piraat, käis piilus meie juures 
Windowsi!}. Kuigi ma ei tõmmanud seda, mul oli see ammu olemas! Sysadmin ütles mulle:
\enquote{Sa vaatasid Windowsi, kas sa saad sellistele asjadele ligi? Kuule, Soome kolleegid näitasid SPSS\sidenote[][-5mm]{Algselt IBMi 
välja töötatud tarkvarapakett, mille nimi on lühend väljendist 
\emph{Statistical Package for the Social Sciences}.} 5.0, aga see on jube kallis, neil 
on ainult pooled moodulid ostetud. Kas seda oleks kuskilt saada?} Küsisin ICE 
põhimeestelt ja need vastasid: \enquote{Miks sa viit tahad? 
Poolteist kuud tagasi tuli kuus välja!} Oi kuidas siis meie onud-teadlased 
rõõmustasid, kui soomlased tulid vaatama, et Eestis on täispakett SPSS 
6.0, mida Soomes ei ole mitte kellelgi! 

\question{Ühel hetkel toimus ikkagi nihe ja tarkvara eest hakati
maksma.}

Oma riik tuli, kõiki asju sai hakata ostma ja ka hinnad normaliseerusid. Vene 
aja lõpul tuli näiteks kahekümnemegase kõvaketta eest
maksta 45 000 rubla! Mäletan, et Eesti esimene 486 läks Punase 
RETi\sidenote{Asutatud 1935. aastal OÜ Raadio-Elektrotehnika Tehase nime all. 
Tehas tegutses 1993. aastani ja tootis raadioid ning mitmesugust 
audiotehnikat.} spetsiaalsele konstrueerimisbüroole. See toodi peidetud transpordiga Minskisse (tol ajal Ukraina, Valgevene ja teised hakkasid ka 
eralduma), sest 486d olid totaalse embargo all, kuna 
CoCom\sidenote{\emph{Coordinating Committee for Multilateral Export Controls.} oli mitteformaalne multilateraalne organisatsioon, mille abil USA ja 
tema liitlased üritasid koordineerida erinevaid kommunistlikke riikide suhtes 
strateegilistele kaupadele rakendatud piiranguid.} ei lubanud neid sotsmaadesse 
viia. Ametlikult võis tuua ainult 386 kõige lahjamaid versioone ja masina hind oli neli ja pool miljonit rubla.

\question{Hoomamatu number toona rublades, isegi täna eurodes!}

No just. Narva jõe peale tehti majanduspiir, kus kehtis reegel, et ainult juht tohtis läbi sõita, teised pidid minema jalgsi läbi putka. Sõitsime sinna, vastik 
ilm oli, me olime kahekesi mikrobussiga ega viitsinud jalgsi minna. Sõdurpoiss astus ligi, lõi kulpi ja küsis: \enquote{Miks te kahekesi 
olete?} Ütlesin, et saadan kaupa, seda ei tohi üksi viia. \enquote{Mis kaupa? 
Tehke lahti!} Meil oli kolm suurt matkaseljakotti, millest paistsid nurgelised 
klotsid. \enquote{Mida te veate?} \enquote{Raha!} \enquote{Mida?} Ta tegi
koti lahti, seal olid sajaste klotsid -- sada rahatähte pakis ja kümme sellist pakki oli üks \enquote{tellis}. \enquote{Palju 
siin on?} \enquote{Ligi viis miljonit \ldots} \enquote{Vabandust!} Rohkem teda
ei huvitanud, tema jaoks oli see ka hoomamatu. 

Meil olid vene poistega niivõrd head sidemed. Venemaaga äri ajades pead 
teadma, kuidas ja mismoodi on. Meil oli usaldus nii suur, et teinekord läksime rahahunnikuga kohale, 
valisime ühe ja teise arvuti ning
näiteks poolteist miljonit jäi üle, mille jätsime rahulikult sinna 
seifi. Edasi-tagasi sõidutamine oli kõige riskantsem: saba võis
peale lennata ja rajalt maha võtta, autot tulistada \ldots Hiljem 
helistati: \enquote{Kuule, tuli üks väga huvitav asi väga hea hinnaga. 
Huvitab?} \enquote{Huvitab!} \enquote{Okei, tõstan selle raha siis endale.} 
Teinekord läksime kahte masinat tooma ja tüüp ütles: \enquote{Tead, 
mul õnnestus kolm tükki saada. Tahad? Võta kaasa, järgmine kord tood raha 
ära!} 

\question{Kui ma sind kuulan, siis see ei olnud mitte kümned, sajad või 
konteinerid, vaid kaks-kolm masinat?}

Jah. Suuremaid tehinguid oli vähe. Meil olid kontaktid arvutite juurde, 
teistel meestel raha juurde. 
EVEA Pangas\index{EVEA Pank} olid meil sõbralikud suhted mitme tegelinskiga, kes 
ajasid meile tõsiseid ärisid välja. Näiteks viisime Ikaruse 
bussi täie arvuteid Poolast Peterburi, kus need laaditi
sõjaväe transpordikopteritele, kuulipildujad ja relvastatud eriväelased ümber. Kopterid sõitsid põrr-põrr, kogu 
Kuibõševi linnavalitsuse arvutipark tuli siitkaudu. Igaüks sai oma jao! 
Poisid tulid, ajasid juttu, väga head konjakid kaasas. Kui sai joodud, 
hakkasid ära minema. \enquote{Oot, kohvri unustasid maha!} \enquote{Aa, jajah.} Miljon 
sularahas kohvriga näpu otsas kaasas \ldots See oli täielik
kauboikapitalismi aeg. 

\question{Õnneks või kahjuks see aeg ei kestnud väga kaua.}

Jah, kuidas kellelegi. Mõned lõpetasid kuuliga kuklas, teised 
praeguste tippmiljardäride hulgas. 

\question{Tuleme tagasi Luciferi\index{Lucifer BBS} juurde. Kuuldavasti oli see Eestis 
ühel hetkel kõige populaarsem koht, kus käidi. Mis selle nii populaarseks tegi?}

Seal saada oleva info hulk. Kui BBSindus veel õitses, siis internetile, kust 
sai materjale sikutada, oli ligipääs väga vähestel. Enamik internetikasutajaid olid tõsised töötegijaid, kes vahetasid seal 
kolleegidega meile ja \emph{that's it}. Nemad ju ei kaevanud 
suurtes raamatu- ja faililadudes ringi. Nad ei 
toppinud oma nina igale poole ja tänu sellele kuskile mujale ei jõudnudki. Ja 
kellel olid BBSid, ei olnud jälle sellist finantsvõimekust, et sikutada kogu 
seda materjali ühest BBSist teise, see oli väga kallis.

\question{Nii et sinu juures said kokku huvi ja arusaam nende asjade väärtusest ning 
finantsvõimekus?}

Jah.

\question{Ja seetõttu oli sinu \emph{stash} populaarne, sinna oli popp külge 
tulla?}

Igaüks leidis sealt midagi põnevat! Seal oli kõike alates maagiast, 
okultismist ja satanimist ning arvutikirjandusest, ulmest ja \emph{science 
fiction}'ist kuni kokaraamatute ja retseptikogumikeni välja.

\question{Mis aastal 
sul esimene online-mäng tuli?}

Mina sattusin sinna umbes 2000. aastal. Enne ma suur mängur ei olnud, 
üksikuid mänge toksisin, aga põhiliselt kolasin internetis ringi. Ja huvitaval kombel jäi Muda\index{Muda}, see 
mittegraafiline \emph{dungeon}, kõrvale. Muidugi sai kõvasti mängitud 
\enquote{Dungeons \& Dragonsit}\sidenote{\enquote{Dungeons \& Dragons}, tihti lühendatud ka 
DnD või D\&D, on 1974. aastal ilmavalgust näinud rolli-lauamäng. 
See oli esimene omataoline, võimaldades suhteliselt vaba vooluga, kuid siiski 
kindla struktuuriga mängukarakterite ja lugude arendust. Pikad mängukampaaniad 
võivad kesta aastaid.}. Sinna vedas mind Jaanus 
Lillenberg\index[ppl]{Lillenberg, Jaanus}, kes on nüüd ERRis. Tema oli mul 
esimene DM\sidenote{\emph{Dungeon Master} on \enquote{Dungeons \& Dragonsi} kohtunik 
ja jutustaja, kes täidab ka loo mittemängijatest tegelaste rolle. DM 
kontrollib ja organiseerib kogu mängu, temast sõltub mängukogemuse kvaliteet.} ja mängu sattusin 
vist 1994. aastal. Mängime seda siiamaani.

\question{Väga põnev, sest mina olin ka sel ajal Tartus, aga minu jalg tolle 
maailma ukse vahele küll ei sattunud.}

Tartusse tõid selle mängu Arlis 
Narusberg\index[ppl]{Narusberg, Arlis} ja Uuk\sidenote{Ei ole selge, keda Veiko 
silmas peab.}. DnD üks esimesi maaletoojaid oli vana 
hea tuttav Vormsi Enn\index[ppl]{Vormsi Enn|see{Mikker, 
Enn}}\index[ppl]{Mikker, Enn}, kes sai neil segastel lõpuaastatel Soome sõita. 
Tema käest telliti, et too mõni arvutimäng. Tema tuttavad, vanemad 
inimesed, ei teadnud ka täpselt, mida vaja -- läksid poodi ja ostsid talle teise 
\emph{edition}'i \enquote{Dungeons \& Dragonsit}, \emph{dungeon master}'i raamatud ja 
\emph{players handbook}'id. 

Algul oli pettumus, aga kui süvenesid, oli see väga kõva. Seda mängis 
kunstiinstituudi\index{Eesti NSV Riiklik Kunstiinstituut} punt ja näiteks Meelis Mikker\index[ppl]{Mikker, Meelis} oli väga kõva DM. 
Kui sealt lõpetanutest üks ports Tartusse kolis, tuli nendega koos ka DnD ja 
seda sai ikka mängitud. Vahepeal üheksakümnendate keskpaigas oli mul 
hullumeelne aeg, kus mängisin kahes mängus ja tegin ise kolme-nelja 
mängu. Nii et nädal otsa oli iga päev mingisugune seltskond mängimas.

\question{Kui sa \enquote{tegid mängu}, kas see tähendab, et olid DM?}

Jah.

\question{See eeldab ju fantaasiat, mitte ei ole niisama!}

Selleks internetimaailm ja ulmekirjandus ongi, et fantaasiat arendada!

\question{Ja fantaasiat sul on?}

Võiks öelda, et jagub. Siiamaani käivad ja painavad. Üks filmiinimeste seltskond tahab, et hakkaksin tegema ingliskeelset mängu. Kõik 
on põnevil, aga vähe on DMe, kes viitsiks ja oskaks teha. Olen 
mõned korrad sattunud mängima sellisesse mängu, kus käib asi niiviisi: 
\enquote{Lähete nädal aega, midagi ei juhtu. Nüüd tuleb kari lendavaid lõvisid, 
hakake lööma!} Kõik veeretavad täringut, kolm tundi täring klõbiseb, kõik 
kaklevad, lõvid on surnud. \enquote{Nüüd lähete veel kuu aega, midagi ei juhtu.} 
Sisulist mängu ei olegi. 

Rahval tuleb lasta mängida, tõmmata neile konkse ja muid asju 
üles. Ühes mängus mängisid kõik nii hästi oma osa. Loomulikult 
aeti asju nurga taga, et ülejäänud rahvas ei kuuleks, aga tegelikult ei liigutud sellest 
moodulist, mida pidi me liikusime, ühtegi sammu, kogu asi käis 
omavahel. Keegi ei teadnud täpselt, mis toimub, ja kõik kahtlustasid, et see või teine on kuri 
koll. Mäletan, kuidas Mario Pizzolanti\index[ppl]{Pizzolanti, 
Mario}\sidenote{Tartus tuntud kuju, kes pidas legendaarset baari 
Zavood\index{Zavood}.} kellegagi igavesi lahinguid lõi. Kui pärast 
sessiooni kokku tõmbasime ja asjad avalikuks tulid, said nad teise keretäie veel: 
\enquote{Mina arvasin seda!} \enquote{Aga mina arvasin nii!} Ja vahel kõik 
naersid nii, et püksid märjad. 

Siis tuli \enquote{EverQuest}\index{EverQuest}, seesama \emph{dungeon}, aga 
arvutimaailmas -- enam ei pidanud ise olema DM, vaid arvuti tegi selle sinu eest ära. 
Enam ei olnud aega DnDd teha, aeg läks kõik sinna. Parimatel 
aegadel olin \enquote{EverQuest I-s} maailma seitsmes \emph{warrior}. Eestist tuli 
neid veelgi, üks sõber on \emph{ranger}'ina veelgi kõrgemale tõusnud. Minul oli 
keskmine mänguaeg ööpäeva kohta neli ja pool tundi, temal kuus ja pool. 

\question{Tundub, et sa saad inimestega hästi hakkama ja mõistad neid.}

Jah. Praegugi juhin gilde. 

\question{Ega Venemaalgi tõsiste inimestega jutu peale 
saada ei ole lihtne. See eeldab pealehakkamist ja suhtlusoskust.}

Nagu öeldakse, peab tundma \begin{russian}русская душа\end{russian}'d, vene 
hinge. Jätame poliitika 
kõrvale, aga äriajamises ongi praegustel noortel ja lääne inimestel see probleem, et 
nad ei saa sellest aru. Nad ütlevad hinna ja selle peale öeldakse: \enquote{Ahah, 
selle tehingu väärtus on meil 20 miljonit, aga teeme nii, et on 15 
miljonit!} Kuidas? Mis? Miks? \enquote{No me anname ühe Šveitsi 
panga arve, kuhu kolm miljonit panna.}

\question{Sina saad ju ka eestlase hingest aru, mida 
ta vajab ja mis teda huvitab. Seda näitab kasvõi Luciferi püstipanek.}

Jah, kindlasti.

\question{Kust see oskus sul tuleb?}

Võibolla on see oskus, ma ei oska öelda, ei ole analüüsinud. 
Nii-öelda juutimisega olen ma palju tegelenud. Kui muu rahvas rüüpas EÜEs 
oma elu, siis mina veetsin kõik suved ja talved alpilaagrites mägi- ja suusamatkadel. Olen palju matkagruppe juhtinud.

Olen ka erinevaid arvutifirmasid 
juhtinud. Kunagi sattus kätte Carnegie raamat \enquote{Kuidas võita 
sõpru ja mõjutada inimesi}\sidenote{Carnegie, Dale. How to Win 
Friends and Influence People. Simon \& Schuster, 1936.} ja olen sealt paljusid asju 
instinktiivselt teinud. \enquote{Ma tahan sind midagi 
tegema panna}, isegi kui see on kasulik, tekitab trotsi, et kes 
sa selline oled? Kui tahad, et inimene midagi teeks, kujunda selline 
olukord, et ta ise tahab niiviisi mõelda. See on ka meie poliitikute 
suur puudus, et kõik tahavad kedagi juhtida ja sundida. Selle asemel et öelda, et teine inimene on 
loll ja mõtleb valesti, anna talle parem võimalus mõelda, et sa ise oled kuidagi rumal. 
Lase tal arvamus ümber pöörata selle arvamuse peale, mida sa tahad, et ta 
tegelikult teeks ja mõtleks. Sedasi saavutad hoopis rohkem. 

Neid inimesi on palju, kes tahavad, et keegi midagi ära otsustaks ja 
teeks. Nad ei ole huvitatud oma peaga 
mõtlema ja, mis veel hullem, selle mõtlemise tagajärjel tehtud tegude eest 
vastutama. Jõle hea on näidata, et valitsus on loll, minister on loll, euroliit 
on loll, onu Trump on loll, jumal taevas on ka loll. Ainult mitte mina!

\question{Sellest järeldame, et sina oled ka loll?}

Loomulikult! See võtab päris kaua aega, enne kui võiks hakata vana kreeklase 
kombel ütlema, et ma tean, et ma midagi ei tea. 

Võtame kasvõi jumala ning alguse ja lõpu teemad. Teadus on jube 
võimsalt edasi arenenud -- kvandid, mustad augud ja mis kõik veel. Aga mis edasi, kuidas 
edasi? Kust see suur pauk tuli? Kes tegi? Ja mis enne seda oli? 
Ütleme niiviisi, et kui enne ei olnud midagi ja nüüd korraga tuli maailm, siis 
see ongi maailma loomine. Selles mõttes võib jumalat mõtestada (kui me ei mõtle
siin mingit halli habemega taati, kes, karjasekepp käes, pilve peal 
jalgu kõigutab) loodusseaduste ja -teaduste, looduse enda 
kompleksina. 

Oled sa juhuslikult lugenud \enquote{Ijon Tichy kosmoselendude 
päevikuid}\sidenote{Lem, Stanisław. Loomingu Raamatukogu nr 22. Ijon Tichy 
kosmoserändude päevikud. Ajalehtede-Ajakirjade Kirjastus, 1962.}? Mäletad, 
kuidas ta äikselise ilmaga maja ukse taga koputas, aga sisse ei tahetud lasta ja kui 
lõpuks lasti, siis hullunud teadlane näitas talle ülakorrusel 
plaadikaste, kus loeti, et \enquote{see on noor neiu ja see on keegi teine}. 
Aga äkki oleme ise ka plaadimängijad kellegi tolmunud pööningul? Ei tea! Võtame kasvõi
déjà-vu efektid ja parapsühholoogia \ldots Minu arust Ijon Tichy illustreeris selle väga ilusti 
ära, aga ma ei tea, me ei saa seda kontrollida! 

\question{Nii et sind huvitavad sellised asjad?}

Aga loomulikult! Kõik räägivad jumala olemasolu või
mitteolemasolu tõestamisest. Mina olen võrrelnud seda sellega, kui varbaküüne 
üks rakk hakkaks tõestama inimese olemasolu või mitteolemasolu. Kuidas ta seda 
teeb? Peremees võib käärid kätte võtta ja küüned lühemaks lõigata \ldots

Meie orgaanilise keemia professor Viktor Palm\index[ppl]{Palm, Viktor}, kes
luges teadusliku maailmavaate aluseid, ütles 1981. aastal tolle aja kohta väga julgelt, et tema arust on näiteks teaduslik kommunism ja teaduslik 
ateism täpselt samasugused pseudoteadused nagu teaduslik teism või teaduslik 
jumalaõpetus. Nendel asjadel pole teadusega midagi pistmist!

\question{Kas 1981. aastal öeldi auditooriumi ees selline asi välja?}

Jah. Ta sai selle eest ka vastu pead, sest usinad tegelased käisid, käsi kõrva ääres, raporteerimas. Aga miks nimetada mingit asja 
teaduslikuks, kui sellel ei ole teadusega mitte mingit pistmist?

\question{Eks kuidagi oli vaja nimetada, et uhkem oleks!}

Eks praegugi on olemas igasugused majandusteadused ja muud säärased, kus on
jube palju soolapuhumist! Üks mees võtab meetodid, tõestab ühe 
asja ära ja ütleb: \enquote{Must.}. Siis ütleb teine: \enquote{Ei-ei-ei!} ja tõestab 
ära, et kõik on valge. Kolmas räägib pallist ja neljas räägib üldse kokkusulanud spektrist. Mine võta siis kinni, mis on! Täpselt see, kellele mida vaja.

\question{Mida sa praegu teed? Tean, et kirjatööd, aga kuidas sa selle juurde jõudsid? 
Üks asi on palju lugeda, teine asi palju kirjutada.}

2000ndate alguses töötas mu naine ajakirja 
Arvutimaailm\index{Arvutimaailm} peatoimetajana. Oli selline tore aeg, kui 
ka IT-ajakirjanikke mööda maailma lohistati, mitte nagu nüüd, kus keegi meie 
vastu huvi ei tunne ja veetakse ainult autoajakirjanikke -- see teeb kohe suisa 
kadedaks. HP-l oli järjekordne suur konverents tulemas ja 
Arvutimaailma kaastööline, kes pidi sinna minema, ei saanud üle piiri, kuna tema pass oli aegunud. Abikaasa küsis: \enquote{Kas sul on pass korras? Sa tunned 
seda värki, teed ära?} Mis seal ikka! Läksin sinna, tegin ära. HP ütles, et pole varem nii
põhjalikku ja sisukat ülevaadet näinud. Ma olin HP masinatega 
juba aastaid kokku puutunud. 

Ma ei ole kunagi arvutiteadust õppinud, aga vaata, kuidas praegu nooremad 
põlvkonnad on hädas, kasvõi DOSi käsureaga. Kui hiirega lohistatavat 
värvilist ekraani ees ei ole, on kaks käppa püsti. Ma olen selle kõigega 
üles kasvanud ja koos arvutitega, mis mul endal läbi on käinud, järjest 
arenenud. Ja kui nendega tegeled, siis tekib huvi. Vastasel 
juhul ei ole vahet, kas müüd kartuleid, arvuteid või kaalikaid. 

Ühesõnaga, tegin HP loo ära ja pärast tuldi uusi lugusid küsima. Suur osa nooremast 
põlvest ei tunne raudvara, aga raudvara-ajakirjandusega on see lugu, 
et pead teadma ja oskama küsida, ning eks ma sellepärast jäingi 
silma ja mind kukuti igale poole saatma. Teiseks ei ole mul, nagu sa mainisid, inimestega läbisaamisega probleeme. Mind ei kohuta, 
kui saame konverentsil kokku näiteks Inteli viitsepresidentidega või ajame juttu
Otelliniga\sidenote{Paul S. Otellini oli Inteli CEO 2005--2013.}. Kui olin juba mitmel üritusel käinud, siis paljud mehed, näiteks 
mõni tehnoloogiajuht või viitsepresident, tundsid mind juba eemalt ära ja tulid kohe 
juttu rääkima. Ütlesid: \enquote{Sa oled ainuke, kes asjast aru saab!}

Eks see oligi üks põhjus, miks mind kirjutama kutsuti ja orbiidil hoiti. Praegu 
on see teema vaibunud. Olen vist neli korda USAs ja Singapuris käinud, mitu korda Koreas, Hiinas ja mujal. Rääkimata Euroopast, seal oli
vahepeal üks konverents teise järel. Aga nüüd on IT-firmad nii vaikseks jäänud. 

\question{Ehk on ka selles vallas saabunud nii-öelda pudukaupmeeste ajastu?}

Eks varem tuldi turule, söödi ennast sisse, nendest tuli kirjutada ja kõik olid väga põnevil. Aga nüüd 
on turg stabiliseerunud ega kasva enam. Võtame kasvõi 
lauaarvutid. Ära need ei kao, nagu paljud ennustasid, sest mängurid 
tahavad ikka suure 4K-ekraani taga korralikult mängida. Läpakal on küll 
päris kõvad asjas sees, aga see ei ole ikkagi nii võimas -- see on alla 
\emph{clock}'itud võrreldes lauaarvuti \emph{power}'iga. Neil on 
oma nišš olemas, aga samal ajal konkurentsi mõttes niisugust huvi ei ole 
nagu autofirmadel.

\question{Mis oli viimane äge mäng, mis tõstis heas mõttes karvad 
püsti?}

Kui mõtled seda nime, mis on tulemas, siis selleks on 
\enquote{Pantheon}\sidenote{\enquote{Pantheon: Rise of the Fallen} on MMORPG, mille ilmumist on 
mitu korda edasi lükatud. Novembris 2022 on Pantheon eel-alfa staatuses.}. \enquote{Pantheoni} teeb praegu selline mees nagu Brad McQuaid, kes oli ka 
esimese \enquote{EverQuesti}\index{EverQuest} peamine aju\sidenote{\enquote{EverQuesti} kaasautor ja \enquote{Pantheoni} tootjafirma loovdirektor Brad McQuaid suri 2019. aasta lõpus, ajasime Veikoga juttu 2019. aasta algul.}. Kutsume oma paljude 
kaaslastega, kes kunagi olid dragonistid, seda mängurite kuldajastuks, 
mida paljud moodsad mängurid sõimavad. Meie ei mängi neid kiireid 
piu-pau mänge -- ma nimetan neid seljaajumängudeks. Ega seal muud pole vaja: 
võta ahv, õpeta kiiresti punast nuppu vajutama ja ta mängib paremini. Ootan ikka 
sellist mängu, kus on tõelist strateegiat, kombinatoorikat, gruppide juhtimist ja 
suured AId. Selliseid mänge tehti teisel ajastul, seda aega me igatseme 
ja seda Brad McQuaid ka lubab.

Mõned moodsad mängud, mis on tulnud -- osa on saadetud, osas olen ostnud, 
mõni on \emph{free-to-play} --, on kõige jubedamad, kus võid raha eest 
endale elu osta. Hiljuti mängisin \enquote{Fallout 76} -- paras huumor. 
Level'id on kõrgel, \emph{quest}'id kõik viimseni tehtud, midagi teha ei 
ole. Kuu aega mängid ja kõik. 

Üks tõsiselt hea mäng on \enquote{Fallen Earth} -- 
tuumasõja ja kataklüsmide järgne maailm, millel on
kõige parem graafiline ja mängijatevahelise äri süsteem. Muidugi 
vana klassika LOTRO, \enquote{Lord of the Rings Online}, Tolkieni fännidele, kelle hulka 
ma ka ennast loen. Ja kindlasti \enquote{Secret World} -- väga kõva ja paljulubav, 
aga nagu ikka, sai raha otsa ja kahjuks aetakse taga suurt fännibaasi. See on \emph{modern horror}'i tüüpi, Lovecrafti 
ja Poe stiilis: kaasaja must maagia üritab maailma 
tungida ning salaühingud võitlevad selle vastu ja ka omavahel. Templirüütlid, 
illuminaadid, Hiina dragon'id \ldots Aga ma ei ole näinud nii mõttekaid, 
põhjalikke ja keerulisi \emph{quest}'e ehk ülesandeid üheski mängus. Enamikus on 
\emph{quest}'id muutunud labaseks: mine 
keldrisse, tapa kümme rotti. Nüüd, suur kangelane ja maailma päästja, mine uuesti 
keldrisse, tapa kümme rotti ja too sabad ka ära! Või vii pakikene kõrvalkülla onu 
Juliusele. Andke andeks, kas saab veel lollim olla?!

\enquote{EverQuestis} oli kohti, kus pidid näiteks võtma midagi kiilkirjas või 
teadma hieroglüüfe, ja selge see, et keegi neid asju nii täpselt ei tea. 
Selleks oli mängu sisse ehitatud Google'i brauser, kust sai abi otsida ilma mängust väljumata. Näiteks ühes kohas oli vihje 
nimega. Leidsid laiba, mille juures oli saatmata postkaart sellele nimele, ja 
otsides tuli välja, et see oli üks Saksa kõvemaid krüptograafia 
alusepanijaid, keda väga vähe teatakse. Otsingust selgus, et tal on olemas 
spetsiaalne algoritm. Soovi korral võisid algoritmi käsitsi kasutada, aga 
võisid ka programmi alla tõmmata ja read sinna sisse kopeerida. Ja kui olid laisk, siis pildistasid ekraanilt ära, OCRisid tekstiks, lasid teksti 
programmi ja tagasi tuli juba mõtestatud tekst, kuhu mängus edasi minna.