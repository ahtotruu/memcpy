\label{chptr:lucifer}
\index[ppl]{Tamm, Veiko}

\question{Hakkame peale sealt, kust asjad ikka pihta hakkavad. Kuidas arvutit 
sinu juurde said?}
                 
See on komplitseeritud küsimus. Ma ise olen ülikoolis keemia eriala\index{Tartu 
Ülikool!Keemia Instituut} lõpetanud. Aga ei oma neljanda kursuse kursusetööd ja 
diplomitööd ei ole ma ühegi kolvi ega katseklaasiga solberdanud, sest  mind 
kutsuti tollal üsna põnevasse asja. Nimelt juhendajaks oli mul Mati 
Karelson\index[ppl]{Karelson, Mati}, kes alustas arvuti ja kompuuterkeemiaga,  
ühesõnaga kvantkeemiaga.  Ja sealt siis olid minu esimesed kokkupuuted. Meie 
tööväljaks olid alguses perfolindid alguses, viieaugulised.

\question{Juba tol ajal ta üritas teha kvantkeemiat? Nonde arvutitega?}   

Jah, see asi hakkas pihta juba kaheksakümnendatel.
                 
\question{Ja sul enne seda üldse mingit kokkupuudet arvutitega ei olnud?}

No meil matemaatika kursuse käigus väga lühidalt näidati selliseid arvuteid 
nagu Nairi-2\index{Nairi!Nairi-2}, ja nende PA ja AP keel\sidenote{Vt. lk. 
\pageref{sisu:apkeel}.}. Sai endale \enquote{Hello World!} trükkida ja 
\emph{that's it}. 

Aga no, ütleme, need masinad, millega me hiljem töötasime\ldots Alguses oli 
Minsk-32\index{Minsk!Minsk-32} ja hiljem  Jessukesteks\index{Jessuke|see{ES 
EVM}} kutsutud ES-1022 arvutid\index{ES EVM!ES-1022}, mis olid Ülikooli 
Arvutuskeskuses. Arvutamine käiski nii et  perforeerisid oma programmi sisse, 
viisid sinna ja tema lahendas. 

\question{Mida need programmid tegid?}

Oli mitmesuguseid kvantkeemia meetodeid. Aatomorbitaalid, kuidas need 
keemilised sidemed moodustuvad, kuidas elektronpilved suhtuvad üksteisega ja 
kas see aitaks seletada neid keemia asju, kas  lähedaks arvutuslikke tulemusi 
reaalsetele. Ja need arvutused oli ikka nii, et võtsime lihtsad kahe aatomiga 
molekulid ja nendega oli nii,  nagu on. Aga keerukamate, näiteks seal metaan 
\ce{CH4},  molekuli välja arvutamine nõudis  sellelt suurelt arvutilt umbes 10 
korda rohkem tööd kui terve lihakombinaadi aastaaruande välja arvutamine. See 
oli päris kõva ja kallis arvutiaeg, mis sinna alla läks.
           
\question{Kuidas see käis? Juhendaja ütles \enquote{Veiko, hakka 
programmeerima} ja sa hakkasid programmeerima?}      

Ei, programmeerimisest jäi asi ikka kaugele seal. Tuli natukene keeltega 
pusserdada küll,  põhimõtteliselt tuli mõningaid asju Fortranis\index{Fortran} 
kirjutada ka. Et midagi ma nagu aru sain, aga seda programmeerimise asja ma ei 
ole selgeks saanudki. 

Siis katkes see asi tükiks ajaks ära. Taaskohtumine arvutitega oli, nüüd võiks 
öelda juba üle 31 aasta tagasi, kui Tartu Tähetornis\index{Tartu Tähetorn} ajas 
tol ajal suur arvuti-fänn Enn Kasak kokku arvutihuviliste ringi ja hankis sinna 
arvuteid.

\question{Mis aastal see oli?}

See oli 1988. aasta suvel. Sel ajal algas ju suur kooperatiivide ajastu. Sai 
tehtud kooperatiiv Tähetark\index{Tähetark}, mille liige ma olin ja mille 
eesmärk oli hankida planetaarium. See planetaarium sai isegi ostetud ja  
Füüsika Instituudi\index{Tartu Ülikool!Füüsikahoone} fuajeesse kuidagi üles 
säetud, aga ega temaga mingit tööd tegema ei hakatud. 

Aga samal oli see selline teadlik ja teaduslik keha,  mille varjus sai arvuteid 
osta-müüa. Arvutite müügiga oli üldse see asi, et hiljem keelati igasugustel 
väikestel asjadel see ära. Kui sa oled kunagi näinud vene seriaali 
\begin{russian}Бригада\end{russian}, segaste aegade maffia-elust, kus äritseti 
kõigega. Juhtuski niiviisi, et ma sattusin  olukorda, et mine ja too mingeid 
arvuteid. Käisin Peterburis, Moskvas, sain kätte otsekontaktid ja sidemed ja 
nii see arvutiärisse sattumine nagu järsult puhkeski. Sest arvutid olid tol 
ajal ikkagi hirmsalt kallid asjad. Minu esimene arvuti näiteks, Amiga 
500\index{Amiga!Amiga 500},  maksis sama palju, kui tutikas 08\sidenote{Nii 
kutsuti autosid VAZ-2108, tuntud ka kui Lada Samara. Tegu oli oma aja ja 
Nõukogude Liidu kohta innovatiivse autoga, Samara oli teine (esimene oli siiani 
populaarne Niva) ise arendatud AvtoVAZ-i mudel ja esimene, mis ei tuginenud 
Fiat 124 mehaanikale.}. No ikka mingi 40 000 rubla.
                 
\question{Miks sa sinna Tähetorni juurde läksid, arvutivärk ikka nagu tõmbas 
või?}

Sõber kutsus, et tule kaasa, põnev värk selline.

Ja siis ma mäletan jah, kuidas tuli algusest peale endale kõik see värk selgeks 
teha. Kuidas vaadata, et palju tal op mälu on ja testiprogramme kasutada. 
Alguses kuiva trennina, aga järjest hakkas see asi nagu liikuma. Ja kui endale 
sai arvuti koju ostetud,  oli jube põnev. Alguses ma rändasin  üle Amigate.  
500\index{Amiga!Amiga 500}, siis 500 Plus\index{Amiga!Amiga 500 Plus}, siis 
Amiga 1000\index{Amiga!Amiga 1000}, Amiga 2000\index{Amiga!Amiga 2000}. Ja siis 
sai sinna kõrvale esimene PC, sest tuli üks suurem tellimus. Arvuteid saada oli 
kuradi raske, sellepärast et igal pool olid embargod nende sisse vedamiseks ja  
põhimõtteliselt tõime arvuteid Moskvast ja 
Peterburist.\label{sisu!veiko_moskvas}

\question{Kuidas need arvutid Moskvasse ja Peterburi said?}

No näiteks see Moskva kanal, mis mul oli ja millega ma embargo-arvuteid tõin ja 
äritsesin, tuli läbi saatkondade. Singapuri saatkonnas toodi paljundusmadina 
kastis arvuti saatkonda ja siis Lumumba\sidenote[][-3cm]{Moskvas asutati 1960. aastal 
Vene Rahvaste Sõpruse Ülikool (\begin{russian}Российский университет дружбы 
народов\end{russian}), mis 1961. aastal nimetati Kongo poliitiku ja 
vabadusvõitleja Patrice Émery Lumumba auks ümber Patrice Lumumba Ülikooliks. 
Kooli eesmärk oli toetada vastselt koloniaalsõltuvusest vabanenud riike 
koolitades sealset tulevast teadus-tehnilist eliiti. Praktikas oli tegu Moskva 
ühe vähese ülikooliga, kus õppis suurel määral välisüliõpilasi.} üliõpilaste 
kaudu läks see kohalike ärikate kätte ja sealt sain siis mina osta. 

\question{Sihuke tarneahel!}

Jaa, tarneahel oli päris võimas, kusjuures hinnad tarneahelas liikusid väga 
põnevalt. Iga ots pani omale ikka julgelt kuskilt seal 30 kuni 50 protsenti 
otsa, aga väga sageli ei viitsinud hakata liigutamagi, kui 100 protsenti 
kasumit ei olnud. 

\question{See oli ju riskantne äri?}

Oli jah, tean ikka väga palju tuttavaid, kes said kuuli ja nuga. Olen isegi 
kihutanud öises Peterburis punase tule alt läbi, sabad järel ja kõik. Kui sa 
ikka sõidad sinna kolm miljonit rubla sularahas seljakottidega kaasas, siis ta 
on riskantne.

Võtame ülekande rubla, vene ajal nimetati \begin{russian}песналик\end{russian}. 
Arvuti hind ülekande rublas oli kaks miljonit. Kui sa tõid sularahas selle 
raha, said arvuti kätte, ütleme, 1.4 miljoniga. Aga kui sa maksid valuutas, 
võisid üldse  ühe miljoniga kätte saada ümberarvutatult väärtustes. Ja teine 
lisaväärtus, mis tuligi  Moskvaga tuuritades oli see, et Venemaal, Moskvas, 
süva-venemaal, hinnati saksa marka ja  dollarit. Meil siin jälle, vastupidi, 
olid hinnas Rootsi kroonid, Soome margad, kellega ärikad äri ajasid. Ja siis 
oli nii, et ostsid siit kokku saksa margad, ostsid kokku dollarid, läksid 
Moskvasse, maksid nendega ära. Ja kõik need saksa margad ja dollarid, mis üle 
jäid,  vahetasid seal Soome markadeks ja Rootsi kroonideks. Ja tulid siia ja 
vahetasid ära ja üksinda selle ülejäänud nii-öelda valuutavahetuse eest võtsid 
ka rahulikult iga raksu pealt seal mingisugune sada tuhat vahelt. 

See oli hirmus aeg. Siis eriti enam ei kütitud, kuigi nõuka aja lõpuni kehtis 
ju  valuutaseadus. Et suurtes hulkades valuuta äritsemise eest võisid saada  
seitse aastat. Suureks hulgaks loeti juba seda, kui sul oli rohkem kui 100 
dollarit. Aga seal sai tuhandetega arvestatud.

\question{Kes neid arvuteid ostis?}

Oh jumal, kõik ajasid taga. Riiklikud ettevõtted, instituudid\ldots  Kui ma oma 
esimese PC arvuti ostsin, siis see läks Tartu Ülikooli 
Füüsika-Keemiateaduskonna pea serveriks. Ja see oli 286, 20 MhZ. Võimas masin, 
kahekümne megahertsine!. Tavalisel masinal oli ju ainult 12 MhZ. Tal oli kaks 
megabaiti op mälu ja 120-megane kõvaketas. Ja teise sellise arvuti ma ostsin 
endale. Muidugi käisid paljud tuttavad, kes ka arvutitega tegelesid, vaatamas, 
et \enquote{Mida sa, loll, tast endale ostsid, mis sa teed selle arvutiga, kuna 
sa selle 120 mega arvad täis saavat?}. Vot sellised ajad olid. Mulle siiamaani 
meeldib, Enn Kasaku\index[ppl]{Kasak, Enn}  üks selline paralleelne näide, mida 
ma olen pidevalt kasutanud. Et kui auto-teadus oleks samamoodi arenenud, nagu 
arvutiteadus, siis sõidaks Mercedes praegu valguse kiirusega, võtaks  10 000 
kilomeetri peale tilga bensiini ja maksaks pool senti.

\question{Tundub tõepärane. Mis tolle 286 serveri peal jooksis? Novell?}

Tead, ma isegi ei mäleta, mis nad sinna panid. Põhiliselt, mida jooksutati, 
olid ikka Unixid. V5\index{Unix!System V} näiteks ja kõik sellised asjad. Siin 
ikka käis rahvast meie teadlastel välismaalt külas ja kõik väga imestasid seda, 
et nii palju tehakse Unixitega. Aga kõigi Eesti Unixite seerianumber oli üks. 
Ega peale piraatluse muud võimalust ei olnud. Tarkvara hindade juures, kes 
andis mingi tarkvara jaoks sellist raha!

\question{Ühesõnaga, kuna ei ühte ega viit ei jaksanud osta, piraaditi viis ja 
oligi edusamm!}

Et nagunii ei ühte ega teist ei olnud aga saadi vähemalt midagi teha!

\question{Mis aastal see PC-lugu oli?}

Mingi 1989? 
                 
\question{Sa ketrasid ennast Amigatest siis ikka väga ruttu läbi PC peale?}

Jah, kogu aeg vahetasin. 

Üheksakümnendate alguses, kui juba iseseisvus hakkas, oli mul kõige esimene 
kodu-386. Jälle imestati, milleks seda vaja on, kes sellise asjaga tegeleb, mis 
sa sellega teed. See oli tüüpiline.

\question{Sul pidi ikka siis mingi huvi olema, et sa neid arvuteid nii sageli 
vahetasid ja kooperatiivi ka sisse jäid?} 

Kooperatiivis meil väga kaua see asi ei kestnud,  sellepärast et väga kiiresti 
tuli peale see seadus, mis keelas mitteriiklikele ettevõtetele ja 
kooperatiividele arvutitega äritsemise. Ja kuna mul olid sidemed olemas, polnud 
kooperatiivi enam vaja. Me paari tuttavaga kliente leidsime ja nii edasi, 
tekkis küsimus, et kus kohta ja mida me teeme. Mida meil vaja on? Tootmisruume 
ei ole vaja! Meil ei ole vaja mingeid ladusid, mingeid tooraineid, midagi. Mis 
meil vaja on? Raha!

Mõtlesime, et Tartu Kommertspank\index{Tartu Kommertspank}\sidenote{1988. 
aastal tegevusloa saanud Tartu Kommertspank oli esimene aktsiaseltsina tegutsev 
ning ka välisvaluutatehingute litsentsi saanud kommertspank NSV Liidus. Panga 
tegevus lõppes pankrotiga 1994. aastal. See pank oli mingis mõttes oma aja 
tõeline sümbol põledes heledalt ja kiiresti. Ka Hansapank\index{Hansapank} 
alustas tegevust Tartu Kommertspanga filiaalina!}! Tore koht! Läksime 
Veetõusme\index[ppl]{Veetõusme, Ants}\sidenote{Ants Veetõusme, kes kuni 1990. 
aastani oli Tartu Kommertspanga juhatuse esimees.} jutule. Poisid ütlesid ka, 
et kui lähed, räägi, mis vaja on, et võiks olla nagu oma raha ka raha 
loksutada, küsi kuskil sada tuhat. Hinnad olid sellised, et sellega sai juba 
enam-vähem masina osta! Muidu äri käis ju kogu aeg ettemaksetega. Raha tuli ära 
ja sa võisid teda tükk aega pööritada, siis ostsid  masina ja andsid kliendile 
kätte. See, et kuu aega tuli oodata, oli tavaline nähtus. Tulid jälle suuremad 
summad, siis meil oli käsi üsna hästi sees Novgorodi elektroonikatehases, mis 
tegi  Panasonicu MV-25 pealt maha viksitud vene videomakke VM-12. Laadisid 
furgooni neid täis! Kui sinna läksid, et oleks hea suhe, kast vana Tallinnat 
paar kasti suitsusinki, meie oma suitsukana. Sellega, kõmm, Novgorodi, auto 
videomakke täis ja neid me ei viitsinud üksikult müüa, müüsime hulgi 
koperativšikutele maha, need siis oma kooperatiivipoodides parseldasid edasi. 

Sellist rahakeerutust sai tehtud kogu aja. Aga sai, jah, mõeldud, et võiks olla 
käibevahendeid. Algkapitali, nagu öeldakse. Tavaliselt tead, et kui  midagi 
küsid, siis nii kui nii tõmmatakse maha. Rääkisin 
Veetõusmele\index[ppl]{Veetõusme, Ants} ära, et vot selline arvuti-äri. Ta oli 
väga huvitatud, kõrvad liikusid, et kas neile ka saaks. Ikka saab! \enquote{Aga 
mis te meile pakute?} Noh, ütlesin, et 11\%. Meie näiteks teenime miljoni, teie 
saate 110 000. \enquote{Täitsa hea mõte! Ja palju te meie käest tahate?}. 
Mõtlesin, et küsin rohkem, niikuinii kaubeldakse alla. Ütlesin miljon. 
\enquote{Ahah. Avage arve, pange miljon peale!}.

Kusjuures meie firma oli selline, et kui see kommertspanga  pankrot pihta 
hakkas, siis meie olime üks väheseid, kes selle raha tagasi maksis. Oleks 
võinud põhimõtteliselt ka teha mingid varifirmad ja asjad ära kantida ja külma 
teha. Aga meie tasusime kogu selle raha ja kuskile võlgu ei jäänud.

\question{Kui sa neid amigasid ja PC-side keerutasid, sul pidi mingi huvi 
olema, mis sa tegid nendega?}

Oh, jumal! See oli ka omaette nuhtlus! Kui Tähetornis\index{Tartu Tähetorn} 
need esimesed MSX\index{Yamaha MSX} arvutid tulid, ma veel töötasin 
keemiainsenerina. Pärast tööd sõidad bussiga alla linna, lähed Tähetorni ja 
siis istud ja mängid seal täpselt nii kaua, et on aeg bussi peale minna ja 
tagasi tööle sõita. Vaatad hommikused ringid üle, keerad kabineti lukku, keerad 
magama. Mängud olid naiivsed, aga tead, ta  oli nii põnev aeg! Ja kui endale 
arvuti tuli, see oli košmaar! Järjekord oli pidevalt ukse taga, kõik tulid 
tasuta mängima. 

\question{Sa siis mängisid?}
Jah. Sai muidugi igasugu asju uuritud ja kui tuli internet, siis\ldots 
Tegelikult hakkas see maailmaga ringi käimine juba enne seda, BBS-i ajal.
                 
\question{Vot sinna ma tahtsin jõuda! Kust sul tuli mõte, et paneks endale 
BBS-i püsti? Ja millal see oli?}

See oli kuskil väga varastel üheksakümnendatel. Päris internet jõudis Eestisse 
kahe satelliiditaldrikuga, üks oli seal KBFI peal Tallinnas ja teine oli Tartus 
Tähetornis. Ja sealt siis üle Rootsi Kuningliku 
Tehnoloogiainstituudi\index{Rootsi Kuninglik Tehnikaülikool}, KTH,  käis meil 
side. Siis hakkas BBS-indus vaikselt juba ära vajuma, kuigi ta  töötas veel 
edasi, eks ole. Mäletan, et internet jõudis Tartusse,  ma elasin tollal  seal, 
märtsikuus 1992. Sain üle  EBC, Biokeskuse\index{Eesti Biokeskus}, endale oma 
isikliku \emph{account}-i juba aprillis, kuu aega hiljem.
                 
Tol ajal oli meie kontor Rüütli tänavas, kohe Treffneri kooli\index{Hugo 
Treffneri Gümnaasium} vastas. Muidugi ägedad trefneristid  käisid seal kõik 
hoolega arvuteid näppimas. Üks põhimehi, kes seda asja suunas ja üles pani ja 
majandas tarkvara poole pealt oli Einar Entsik\index[ppl]{Entsik, Einar}, 
praegu kõva kinnisvaraärimees. Tema oli nagu meie peamine \emph{sysop} ja mina 
olin siis \emph{co-sysop}. Hiljem, kui me kontori likvideerisime, siis Lucifer 
BBS\index{Lucifer BBS} tegutses mul kodus edasi, kuni peaaegu lõpuni, kui see 
BBS-i maailm ära hääbus. Siiamaani mäletan veel oma aadressi: 2:491.666.
           
\question{Millest selline nimi, Lucifer?}      

Ta tõi valgust maailma! 

Muidugi, meil oli väga palju igasugust sellist maagiat ja värki, kuna ma 
loomult olen anti-kristlane olnud eluaeg, nüüd olen ma veel suurem 
anti-islamist. Seda ma ei suuda üldse taluda,  selle kõrval kristlased on 
väikesed voonakesed. Vaimupimedust, mis siin on, keskaega tagasi pürgimist! 

Mäletan selgelt, et oli suur jama, kui KAPO käis meie neid materjale uurimas, 
kui mingisugused nõndanimetatud satanistid pussitasid Tartus Hando 
Runnelit\index[ppl]{Runnel, Hando}. Neid uuriti, et kust saadud ja kellegi 
kaudu tuli  välja, et meie BBS-is oli väga palju neid materjale, Lavey Saatana 
Piibel\sidenote{La Vey, Anton Szandor. The satanic bible. New York: Avon Books, 
1969.}. Käidi, uuriti ja vaadati. Ma mäletan üks mehike tutkis nii põhjalikult, 
et pööras täitsa ära, hakkas ise ka satanistiks!

Kusjuures kui sa küsid, kas ma olen satanist, ma ütlen, et ma ei ole. Kui ei 
usu kristlust, kuidas ma saan siis tema peegelpilti kummardada?

\question{Miks te BBS-i tegite? Äri sai ju muud moodi ka teha?}                 

See oli lihtsalt hobi, poisid tahtsid teha. Igasugused sidemed, materjalid üle 
maailma\ldots Tol ajal kaugekõned olid ju kõik tasulised aga no selle äri 
juures telefoni hinnad! See polnud tähtis, ma võisin tundide kaupa rippuda 
Ameerika või Iisraeli või kuskil\ldots Euroopa polnud üldse küsimus! Sai 
helistatud Jaapanisse, sealt igasugusi materjale tõmmatud, sai sealse skeenega 
suheldud ja.

\question{Kust sa numbrid said, kuhu helistada? Üheksakümnendate alguse Tartut 
meenutades, kust ma võisin saada Jaapani telefoninumbri, kus taga BBS vastas?}

BBS-idel olid ju kõigil suured \emph{listing}-ud olemas, kus oli maailma 
olulised BBS-id loetletud. See on nagu aadressiraamat,  telefoniraamat, seal on 
kõik riikide BBS-id sees! 

\question{Ja sa käisid seal infot lugemas?}

Jah, loed uudiseid, infot\ldots Nüüd  kõik istuvad Facebookis, aga siis seda ju 
ei olnud. Siis olidki BBS-id, mille kaudu käis info vahetamine, meili saatmine 
ja kõik. No hiljem, kui internet tuli, olid juba teised ajad. Et seal 
tegutseda, tuli endale UNIX-i alused selgeks teha ja käsureal töötada. Siis ei 
olnud ju veel Linux-itki olemas ega midagi. Põhiline oli just Santa Cruz 
Operation\sidenote{Siit tuleb lühend SCO.} V5 UNIX\index{Unix!System V}, millega 
me kõik siin tegutsesime.

\question{Ohoh, too BBS käis teil UNIXi all?}

Ei BBS-il on oma tarkvara, UNIX tuli hiljem, kui hakkasime juba internetis 
käima. Kliendiga, lehvik kuskile terminali otsa\ldots Ega siis kellelgi kodus 
interneti polnud, ei olnud võimalik saadagi. Pidid teadma, kus istusid  sisse 
helistamise modemid, millega sa said ennast kaugelt kuskile interneti arvutisse 
sisse logida. Näiteks Toomemäele Tähetorni\index{Tartu Tähetorn} ja sealt siis 
juba edasi liikusid, siuh-säuh, internetiavarustes.
                 
\question{Ja kõik käis käsureal!}
                 
Jah. Hiljem hakkasid tulema Gopher ruumid ja muud sellised algelised 
otsingusüsteemid. Infopangad, kus oli erialaseid raamatuid. Siis tekkisid 
interneti BBS-id. Printa oli näiteks Euroopa üks suurimaid ja 
Iska\sidenote{Iowa Student Corporation Association.} BBS\index{Iska BBS} oli 
maailma kõige suurem interneti BBS veel sellisel kujul. Nagu BBS, 
teadetetahvliga, lihtsalt sinna ligipääs oli Interneti kaudu.

\question{Kas sa teiste Eesti sysopidega ka suhtlesid?}

Ja, ikka, meil olid ju igasugused üritused, BBSummerid\index{BBSummer} ja 
BBWinterid\index{BBWinter}. Seal käisid nii sysopid kui ka  kasutajad, oli 
niisugune päris tihe seltskond, kes seal käis ja omavahel niisama suhtles, no 
nagu praegu Facebookis käivad suhted.  See seltskond ei olnud nii suur, et ka  
ei oleks võimalik hallata. Katsu sa teha näiteks Eesti Facebooki liikmete 
kokkutulekut, võib-olla  jääb tulemata  viis protsenti inimesi!
        
\question{Kui palju teil Luciferis\index{Lucifer BBS} liine oli ja, anna palun 
suurusjärku, kui palju kasutajaid küljes käis?}         

Üksainuke telefoni liin. Sellega oligi see, et kui kasutaja tuli  liini külge, 
siis ta pidi seal rippuma. Kõneaeg jooksis kogu aeg. Et kui sa näiteks 
sikutasid mingit tarkvara kuskilt Ameerikast, siis sa rippusid kogu aeg 
kaugekõnega liini peal, päris soolane kopikas tiksus! Eraldi üüriliinid tulid 
alles ISDN-i ajastul, kui tulid 64 ja 128 kilobitised asjad. Algselt, kõige 
esimene modem, mille ma sain, oli 2400 boodi.
                 
\question{See oli isegi juba kiire, sest 1200-sed olid ka levinud}

Isegi 600-sed! Finlandia  BBS\index{Finlandia BBS} oli 2400,  ülejäänud kõik 
olid aeglasemate peal. Ega see modem maksis ka kaks korda rohkem kui sõiduauto 
Žiguli, niisugune tavaline.

\question{Hobi jaoks tundub Žiguli nagu kallis investeerida?}

No võtame niiviisi, mõni rikas mees korjab hobi jaoks vanemaid autosid, 
uunikume, mis maksavad ka meie praeguses rahas seal sada ja kakassada tuhat. 
Teevad oma automuuseumi. See on samuti hobi, ega te sellega ka midagi muud ei 
teeni, piletit ka ei küsi!
       
\question{Ma ikka ei jäta. Mis sind just selle hobi juures paelus, mis hoidis 
sind arvutite juures?}          
Seesamane kübermaailm. 

Kui vaadata praegu Ghost in the Shell-i\sidenote{Masamune Shirow samanimelisel 
mangal põhinev frantsiis, millesse kuulub nii animesid kui ka 2017. aastal 
Hollywoodis linale tulnud film. Frantsiisi tegevus leiab aset post-küberpunk 
maailmas ja selle peategelane, Major Motoko Kusanagi, on küborg, kelle 
mehaanilises kehas (\emph{shell}) toimib inimese teadvus (\emph{ghost}). Tegu 
on kunagistes küberpunk ringkondades kultusliku teosega, mille mõju on 
võrreldav Willigam Gibsoni loominug omaga.} või midagi sellist, et oleks 
võimalik oma teadvus enne surma Internetti üle kanda. Läheks küll sinna 
virtuaalmaailma tondiks!  

\question{Kas sa tol ajal Gibsonit ka juba lugesid?}
Ja, ikka.

\question{Siis on selge!}
                 
Kõik see küberpunk ja see värk oli sisuliselt \emph{must be} kõigile, kes olid 
toll ajal arvuti-friigid. Siis muidugi virtuaalmaailmades\ldots Kuhu ma eriti 
sisse ei jõudnudki, olid mudad\index{Muda}. Mina sattusin virtuaalmängumaailma 
siis, kui tuli selline asi nagu EverQuest I\index{EverQuest}. Seal sai 
järk-järgult läbi gildide mindud, mängisin seda mängu neli pool aastat jutti. 
Mängus sees oli \emph{online counter}, mis luges, kui palju sa mänginud oled, 
mitu päeva, mitu tundi ja nii edasi. Summeeris kokku. Ja kui ma pärast sealt 
vaatasin, siis selle nelja poole aasta jooksul, ma oleksin pidanud iga jumala 
päev mängima neli ja pool tundi. Aga mõnikord oled välismaal kuskilt ära, ei 
mängi. No, polnud sagedased, aga polnud ka väga haruldased juhtumid, kui ikka 
kakskümmend tundi jutti näiteks suuri \emph{raid}-e peetud.
                 
\question{Kas Gibson ja muu küberpungi kraam levis BBS-ides või olid füüsilised 
raamatud ka?}

Olid füüsilised raamatud, suuremad fännid siin, Jack\index[ppl]{Lippmaa, 
Jaak}\sidenote{Ilmselt peab Veiko silmas Jaak Lippmaad ja mitte Jaak Loondet, 
keda sama nime all tunti.} ja nii edasi, tõlkisid neid Eesti keelde,  isegi 
avaldati. Ja eks olid ingliskeelsed suured raamatuarhiivid. Ei olnud ju midagi 
eriti saada, just ulmet ja ilukirjandust. Teadusraamatukogud  ostsid rohkem 
teaduskirjandust,  ilukirjandust oli ikka väga vähe, ja need hakkasid liikuma 
juba digitaalsel kujul. Digitaalsetest arhiividest sai raamatuid tõmmatud, mul 
endalgi BBS-is kõik, mis kätte tuli, läks sinna üles ja rahvas käis ja sai neid 
sikutada ja hoolega lugeda. 

\question{Ja lugeja vaim sai valgemaks! Mis rahvas sul seal BBS-is käis? Mingi 
aimdus sul ju oli, tudengid või\ldots?}

Olid õpilastest ja tudengitest kuni paremate arvuti-inimesteni välja. Sest meil 
oli seal väga palju igasugust põnevat tarkvara, põnevat arvutialast kirjandust 
ja materjale, mida kuskil ka eriti ei liikunud. 

Ja kuna Printas juhtisin ka üht tuba, olin moderaatoriks,  siis saime nende 
\emph{underground}-iga  tuttavaks, IC piraadigrupi\sidenote{Siin peab Veiko ise 
seletama, mis tolle \emph{warez} grupi nimi täpselt oli. Suuremate gruppide 
nimekirjast sarnase nimega seltskonda leida ei õnnestunud.} liige sai oldud, 
seal liikus vahvat materjali. Oli täitsa kurioosseid olukordasid. Tollel ajal 
ei olnud ju mingeid päris pira FTP-sid. Tehti seda nii, et kui mingi asi algas 
punktiga, siis see oli nähtamatu. Ja kuskil, kus oli firmal FTP server püsti, 
siis mingisugusesse huina-muina kataloogi, kus on mingid süsteemsed asjad ja 
kuhu tavaline inimene ei lähe, tehti punktiga algavaid katalooge. Kui seda 
hakati avastama, tulid kasutusele igasugused muud asjad, näiteks mingid 
kontrollsümbolid. Sümbol, mis tegi näiteks piiksu või mis tegi reavahetuse. 
Seesama \emph{enter}-i vajutamine, et sa sisestad asja, oli ka võimalik 
kontrollsümbolina kirja panna. Ja kui seal see sümbol oli ees, siis sa pidid 
teadma, et sa sinna ette panid just selle kontrollsümboli, vist oli CTRL+L, mis 
käskis lugeda järgnevaid asju kui lihtsalt tekstistringe.

\question{Ehk te munesite kellelegi FTP serverisse  oma piravara?}
                 
Jah, niiviisi oli terve maailm täis! Katoloogide kaupa oli kõikvõimalikke asju.
                 
\question{Suurde rahvusvahelisse piragruppi ligi saamine oli ju seotud, ütleme, 
raskustega. Päris avasüli ei võetud vastu?}

Suurte raskustega! Praeguses torrenti-ajastus\ldots

\question{Aga kuidas see sul õnnestus?}

Selle Printa BBS-i kaudu tulid, kutsusid.

Igasugu põnevaid asju sai uurida ja vaadata. Vaata, ega sa ei oska ju midagi 
soovitada, kui sa ei ole ise seda näppida saanud.

Vene aja lõpus müüdi igal pool mustal turul ja laatadel piraat-kogumikke. 
Igasugused Gamez ja nii edasi. Aga et sa piraat-tarkvaraga raha teenid, loeti 
tõsisemates gruppides väga halvaks märgiks, selle eest said kohe kangiga vasta 
pead. See oli patt. 

Kui tuli välja Windows 95, oli see alguses kohutavalt suur saladus. No nüüd on 
Microsoft ennast täiesti teisipidi pööranud. Tahad, võtke uusi versioone, 
uurige, tutvuge, vaadake! Nad on lõpuks aru saanud, et see, et sa midagi püüad 
kinni hoida, see ei takista. Aga kui sa tahad endale miljonit kuju, kes uut 
toorest tarkvara katsetab ja kakub oma juukseid, mis lähevad tänu sellele 
halliks, et kõik hunnikusse lendab? Sealt tuleb tagasiside! Kõik sõimavad 
\enquote{Parandage see ära, see on valesti!} Sa ei jõua endale nii palju 
töötajaid otsida, kes  kõik selle debugimise töö nii põhjalikult ära teevaed, 
kui see vabatahtlik jõuk. Aga siis oli see, jah, nii keelatud, et kui tuli 
Windows 95, tollal koodnimega \enquote{Chicago}, siis ta oli muidugi olemas, ja 
poisid tegid igavese pulli, nad panid selle Microsofti peamisse FTP serverisse. 
Tegid seal punktidega kataloogid, ja järgmine päev panid sinna Chicago üles. Ja 
kui see üle  maailma kulutulena levis, \enquote{Microsoftist saab pira Windowsi 
tõmmata, 95t!}, oi kuidas siis Microsoft marutas!. Ja kõik, kes seal käinud 
olid (mina käisin lihtsalt vaatamas ja irvitamas,  et \enquote{vaadake, mis 
seal seisab}), nad olid ära loginud. Muu hulgas ka ühe  aadressi, mille kaudu 
mina käisin. Mul oli neid palju, üle maailma, sest üks modem oli kinni, teine 
kinni, eks ikka leidus mingi auk. Tuligi teade, et \enquote{karistage, võtke 
\emph{account} ära, see on igavene vastik piraat, käis piilus meie juures 
Windowsi!}. Kuigi ma ei tõmmanud teda, ammu olemas! Sysadmin tuli minu juurde, 
et \enquote{Windowsi vaatasid, sellistele asjadele saate ligi? Kuule, aga mul 
on üks küsimus. Meil Soome kolleegid näitasid SPSS\sidenote{Algselt IBM-i poolt 
välja töötatud tarkvarapakett, mille nimi on lühend väljendist 
\emph{Statistical Package for the Social Sciences}.}  5.0-i, jube kallis, neil 
on ainult pooled moodulid ostetud. Kas seda oleks kuskilt saada?} Küsisin ICE 
põhimeeste käest. Sealt tuleb vastu, et \enquote{Äh, miks sa viit tahad? 
Poolteist kuud tagasi tuli kuus välja!} Ja oi kuidas siis meie onud-teadlased 
olid rõõmsad, et kui Soomlased tulid vaatama, et Eestis on täis pakett SPSS 
6.0-i, mida Soomes ei ole mitte kellelgi! 

\question{Ühel hetkel toimus ikkagi nihe, mingi hetk hakati tarkvara eest ju 
maksma?}
Oma riik tuli, kõik asju sai hakata ostma ja hinnad ka normaliseerusid. Vene 
aja lõpul olid need hinnad ju\ldots No kujuta ette, kahekümnemegase kõvaketta 
eest maksad sa 45 000 rubla! Mäletan, Eesti kõige esimene 486 läks Punase 
RET-i\sidenote[][-4mm]{Asutatud 1935. aastal OÜ Raadio-Elektrotehnika Tehas nime all. 
Tehas tegutses 1993. aastani ja tootis raadioid ning mitmesugust 
audiotehnikat.} spets konstrueerimisbüroole. See toodi mingisuguse, ma ei tea, 
mis kuradi värgiga (tol ajal Ukraina, Valgevened ja teised hakkasid ka 
eralduma)  peidetud transpordiga Minskisse. 486-d olid ju totaalse embargo all, 
CoCom\sidenote{\emph{Coordinating Committee for Multilateral Export Controls 
(CoCom)} oli mitteformaalne multilateraalne organisatsioon, mille abil USA ja 
tema liitlased üritasid koordineerida erinevaid kommunistlikke riikide suhtes 
strateegilistele kaupadele rakendatud piiranguid.} ei lubanud neid sotsmaadesse 
viia. Ainult 386 kõige lahjemad versioonid olid need, mida juba võis ametlikult 
tuua. Ja selle masina hind oli neli pool miljonit rubla.

\question{Hoomamatu number toona rublades, isegi täna Eurodes!}

Tehti seesamanegi piir, majanduspiir, Narva jõe peale. Sõidame sinna, vastik 
ilm oli. Vahepeal oli see reegel, et ainult juht tohtis läbi sõita, teised 
pidid minema jalgsi läbi putka. Ei viitsinud minna. Olime kahekesi, 
mikrobussiga. Tuleb siis sõdurpoiss, lööb kulpi, \enquote{miks te kahekesi 
olete?} Ütlesin, et, saadan kaupa, seda ei tohi üksi viia. \enquote{Mis kaupa? 
Tehke lahti!} Kolm suurt matkajate seljakotti, näha, et mingid nurgelised 
klotsid on sees. \enquote{Mis te veate?} \enquote{Raha!} \enquote{Mida?} Teeb 
koti lahti, seal on sajaste klotsid, sada rahatähte oli pakk, mis panderolliga 
ümber tõmmati. Kümme sellist pakki oli üks \enquote{tellis}. \enquote{Palju 
siin on?} \enquote{Kuskil ligi viis miljonit\ldots} \enquote{Vabandust!} Edasi 
ei huvitanud, tema jaoks oli see ka hoomamatu. 

Aga meil olid niivõrd head sidemed vene poistega. Venemaaga äri ajades sa pead 
teadma, kuidas ja mismoodi on. Meil oli usaldus nii suur, et teinekord läksid,  
täpselt ei teagi, palju sul on. Võtad kaasa, valid selle ja teise arvuti ja 
jääb näiteks ütleme poolteist miljonit üle. Jätsime rahulikult tema juurde 
seifi. See edasi-tagasi sõidutamine oli kõige riskantsem, kui sul võis saba 
peale lennata, sind maha võtta, ära tulistada auto ja kõik, eks. Hiljem 
helistab \enquote{Kuule, meil tuli selline väga huvitav asi väga hea hinnaga. 
Huvitab?} \enquote{Huvitab!} \enquote{Okei, ma tõstan selle raha siis endale} 
Ja samamoodi, teinekord lähed sinna kahte masinat tooma ja ütleb \enquote{Tead, 
mul õnnestus kolm tükki saada. Tahad? No võta kaasa, järgmine kord tood raha 
ära!} 

\question{Kui ma sind kuulan, siis see ei ole mitte kümned ja sajad ja 
konteinerid, vaid kaks-kolm masinat?}

Jah. Suuremaid tehinguid oli vähe. Meil olid kontaktid arvutite juurde, mõnedel 
teistel meestel olid kontaktid raha üle. Olid meil näiteks niisugused 
sõbralikud suhted EVEA Panga\index{EVEA Pank} mitmete tegelinskitega, kes 
ajasid meile ikka tõsiseid ärisid välja. Tõime suure-pika reisi-Ikarus 
bussitäie arvuteid Poolast, näiteks. Sai need kõik viidud Peterburi, seal 
laaditud sõjaväe transport kopterite peale. Kõik on kuulipildujate ja värkide 
all, relvastatud eriväelased ümber. No aga see tehing oli ka seal, ma ei tea, 
kui palju miljoneid seal kokku läks. Kopterid sõitsid põrr-põrr-põrr, kogu 
Kuibõševi linnavalitsuse arvutipark tuli siitkaudu. Igaüks sai oma sellest! 
Tulevad poisid, istuvad, ajavad juttu, väga head konjakid kaasas, saab joodud, 
hakkavad ära minema. \enquote{Oot, kohvri unustasid maha} Ah, jah, miljon 
sularahas kohvriga näpu otsas kaasas\ldots See oli väga, ütleme, selline 
kauboikapitalismi aeg. 

\question{Kas õnneks või kahjuks see aeg ei kestnud väga kaua}

Jah, kuidas kellelegi. Mõned lõpetasid kuuliga kuklas, teised lõpetasid 
praeguste tippmiljardäride hulgas. Kes kuda mida jõudis kinni võtta.

\question{Tuleme tagasi Luciferi\index{Lucifer BBS} juurde. Ta oli Eestis 
ikkagi üks hetk kõige populaarsem koht, kus käidi. Vähemalt nii on räägitud ja 
endalgi on meeles. Mis ta nii populaarseks tegi?}

Info hulk, mis seal oli. Sest kui BBS-indus veel õitses, siis internetile, kust 
sai neid materjale sikutada, oli ligipääsu väga vähestel. Enamik interneti 
kasutajaid olid tõsised töötegijaid, kes tegid tõsist tööd, eksole. Oma 
kolleegidega seal vahetasid emaile ja \emph{that's it}. Nemad ju ei kaevanud 
ringi mingisuguseid suuri raamatute ja igasugu failide ladusid pidi. Nad ei 
toppinud oma nina igale poole ja tänu sellele kuskile mujale ei jõudnudki. Ja 
kellel olid BBS-id, ei olnud jälle sellist finantsvõimekust, et sikutada kogu 
seda materjali lihtsalt BBS-i kaudu ühest teise, see oli väga kallis.
                 
\question{Ja sinu juures said kokku huvi ja aru saam nende asjade väärtusest ja 
finantsvõimekus?}

Jah.

\question{Ja seetõttu sinu \emph{stash} oli populaarne, sinna oli popp külge 
tulla?}

No sealt igaüks leidis midagi põnevat! Seal oli kõike, alates igasugu maagiast 
ja okultismist ja satanimist üle arvutikirjanduse, ulme ja \emph{science 
fictioni} kokaraamatute ja retseptikogumikeni välja. Kõike.
                 
\question{Räägime korra nendest online-mängudest, mida sa mainisid. Mis aastal 
sul esimene mäng tuli?}
                 
Mina sattusin sinna kuskil kahetuhandendal. Enne ma suur mängur ei olnud, 
mõningaid üksikuid mänge  sai toksitud, aga põhimõtteliselt oli internet see 
maailm, kus ma ringi kolasin. Ja huvitaval kombel muda\index{Muda}, see 
mitte-graafiline \emph{dungeon}, jäi kõrvale. Muidugi sai kõvasti mängitud 
\emph{Dungeons \& Dragons}-it\sidenote{Dungeons \& Dragons, tihti lühendatud ka 
DnD või D\&D, on 1974. aastal esmakordselt ilmavalgust näinud rolli-lauamäng. 
Mäng oli esimene omataoline võimaldades suhteliselt vaba vooluga kuid siiski 
kindla struktuuriga mängu-karakterite ja lugude arendust. Pikad mängukampaaniad 
võivad kesta aastaid.}. Sinna vedas mind Jaanus 
Lillenberg\index[ppl]{Lillenberg, Jaanus} ta on nüüd ERR-is. Tema oli mul 
esimene DM\sidenote{\emph{Dungeon Master} on Dungeons \& Dragons mängu kohtunik 
ja jutustaja, kes täidab ka loo mitte-mängijatest tegelaste rolle. DM 
kontrollib ja organiseerib kogu mängu, temast sõltub mängukogemuse kvaliteet.}, 
see oli vist 1994 kui ma sinna mängu sattusin. Siiamaani saab seda mängitud. 
Käime siin vaikselt ja toksima korra nädalas täringuid ringi.

\question{Väga põnev, sest mina olin ka sel ajal Tartus aga minu jalg tolle 
maailma peale küll ei sattunud?}

No seda oli vähe. Põhimõtteliselt  tõid ta Tartusse sellised mehed nagu Arlis 
Narusberg\index[ppl]{Narusberg, Arlis} ja Uuk\sidenote{Ei ole selge, keda Veiko 
silmas peab.}. Võiks öelda, et DnD üheks kõige esimeseks maaletoojaks on, vana 
hea tuttav Vormsi Enn\index[ppl]{Vormsi Enn|see{Mikker, 
Enn}}\index[ppl]{Mikker, Enn}. Ta sai neil segastel lõpu-aastatel Soome sõita. 
Tema käest telliti, et too ikka mõni mäng, arvutimäng. Tema tuttavad, vanemad 
inimesed, ega nemad ka täpselt ei teadnud, läksid poodi ja ostsid talle teise 
\emph{edition}-i Dungeons \& Dragonsit, \emph{dungeon master}-i raamatud, 
\emph{players handbook}-id ja kõik. Algul oli pettumus, polnudki nagu kuskile 
arvutisse panna seda asja, aga kui süvenesid, oli väga kõva. See oli 
Kunstiinstituudi\index{Eesti NSV Riiklik Kunstiinstituut} punt, kes seda 
mängis. Meelis Mikker\index[ppl]{Mikker, Meelis} näiteks oli väga kõva DM. Ja 
kui sealt lõpetanutest üks ports Tartusse kolis,  tuli nendega koos ka DnD ja 
seda sai ikka  mängitud. Vahepeal üheksakümnendate keskpaik oligi mul selline 
hullumeelne aeg, kus ise mängisid näiteks kahes mängus ja tegid ise kolme-nelja 
mängu. Nii et terve nädal otsa iga päev oli mingisugune seltskond.

\question{Kui sa \enquote{tegid mängu}, siis sa olid DM?}

Jah.
                 
\question{See tahab ju fantaasiat saada, ei ole niisama!}

Aga selleks interneti maailm ja ulmekirjandus ongi, et fantaasiat arendada!

\question{Ja fantaasiat sul on?}

Võiks öelda, et jagub. Siiamaani käivad ja painavad. Üks seltskond, 
filmi-inimesed, tahavad, et ma ingliskeelset mängu hakkaks tegema. On kõik 
põnevil, aga küsimus ongi DM-ide vähesus, kes viitsiks ja oskaks teha. Ma olen 
ka mõned korrad sattunud sellisesse mängu mängima, kus käib asi niiviisi, et 
\enquote{Lähete nädal aega, midagi ei juhtu. Nüüd tuleb kari lendavaid lõvisid, 
hakake lööma!}. Kõik veeretavad täringut, kolm tundi täring klõbiseb, kõik 
kaklevad, lõvid on surnud, \enquote{nüüd lähete veel kuu aega, midagi juhtu}.  
Sisulist mängu nagu ei olegi. 

Kui sa võtad kätte ja lased rahval mängida, tõmbad neile konkse ja igasugu asju 
üles\ldots Ühes mängus oli, kus kõik mängisid nii hästi oma osa. Loomulikult 
kõiki aeti asju nurga taga, et ülejäänud rahvas ei kuule. Tegelikult sellest 
moodulist, mida pidi me  liikusime, ei liigutud ühtegi sammu, kogu asi käis 
omavahel. Keegi ei tea, kõik kahtlustasid, et see või teine on mingisugune kuri 
koll. Mäletan, kuidas Mario Pizzolanti\index[ppl]{Pizzolanti, 
Mario}\sidenote{Tartus tuntud kuju, pidas legendaarset baari 
Zavood\index{Zavood}.}  kellegagi  igavesi lahingud lõid. Ja kui pärast 
sessiooni kokku tõmbasime ja asjad avalikuks tulid, said teise keretäie veel: 
\enquote{Mina arvasin seda!} \enquote{Aga mina arvasin nii!} Ja vahel kõik 
naersid nii, et püksid märjad. 

Siis tuligi seesamunegi EverQuest\index{EverQuest}, seesama \emph{dungeon}, aga 
arvutimaailmas. Et sa ei pea ise olema DM vaid arvuti teeb selle sinu eest ära. 
Ei olnud enam aega DnD-d teha ega midagi, aeg läks kõik sinna. Parimatel 
aegadel ma olin  EverQuest I-s maailma seitsmes \emph{warrior}. Eestist tuli 
neid veelgi, üks sõber on \emph{ranger}-ina veelgi kõrgemale tõusnud. Minul oli 
keskmine mänguaeg ööpäeva kohta neli ja pool tundi, temal oli kuus ja pool. 

\question{Oh jumal, see on ju investeering!}

Jah, ta sel ajal oligi.
                 
\question{Kui ma sind kuulan, siis kerkivad inimesed kuidagi esile. Sa tundud 
nendega hakkama saavat, neist aru saavat?}

Jah. Praegugi juhin gilde. 

\question{Selles mõttes ka, et ega Venemaal nende tõsiste inimestega jutu peale 
saada ei ole lihtne. See tahab pealehakkamist ja suhtlemise oskust?}

Nagu öeldakse, sa pead teadma \begin{russian}русская душа\end{russian}-d, vene 
hinge. 
See on hoopis teine, kui sa seda ei mõista\ldots Jätame poliitika kui sellise 
kõrvale. Aga  praeguse aja noortel ja lääne inimestel äri ajamisel ongi see, et 
nad ei saa aru. Et kui tema ütleb hinna ja selle peale öeldakse \enquote{Ahah, 
et selle tehingu väärtus on meil 20 miljonit? Aga teeme nii, et on 15 
miljonit!} Ei saa aru! Kuidas? Mis? Miks? \enquote{No me anname ühe Šveitsi 
panga arve, kuhu kolm miljonit panna.}
                 
\question{Sa saad ju sellest  eestlase hingest ka aru, sa saad aru, mida 
inimesed vajavad ja mis neid huvitab, kasvõi Luciferi püsti panekuga}

Jah, kindlasti.

\question{Kust see sul tuleb, oled lihtsalt sündinud sellega? Oled sa mõelnud?}

Võib-olla on see oskus. Ma ei oska öelda, ei ole nii palju analüüsinud. Aga 
nii-öelda juutimise asjaga ma olen tegelenud palju. Kui muu rahvas rüüpas EÜE-s 
oma elu, siis mul kõik suved ja talved olid talvel suusamatkadel ja suvel 
mägimatkadel alpilaagrites. Ma olen palju igasugu matkagruppe juhtinud.

Seal ongi see, et kuidas seda asja teha. Arvutifirmasid, erinevad, on saanud 
juhitud ja\ldots Ma mäletan, kui  sattus kätte raamat \enquote{Kuidas võita 
sõpru ja mõjutada inimesi}, Carnegie oma\sidenote{Carnegie, Dale. How to Win 
Friends and Influence People. Simon \& Schuster, 1936.}, siis paljusid asju 
sealt ma olen kuidagi instinktiivselt teinud. \enquote{Ma tahan sind midagi 
tegema panna}, isegi kui see on kasulik, tekitab trotsi. Vastumeelsust. Et kes 
sa selline oled? Kui sa tahad, et inimene midagi teeks, kujunda selline 
olukord, et inimene ise tahab niiviisi mõelda. Vot see on meie poliitikute üks 
suur puudus ka, et kõik tahavad kedagi juhtida, sundida. Öelda, et sa oled väga 
loll, sa mõtled valesti. Anna parem talle võimalus, et mina olen kuidagi rumal. 
Lase tal minu arvamust ümber pöörata selle arvamuse peale, mida ma tahan, et ta 
tegelikult teeks ja mõtleks. Ja hoopis rohkem tuleks  saavutusi. 

Nagu öeldakse, palju on inimesi, kes tahavad, et  keegi midagi ära otsustaks, 
keegi midagi ära teeks. Nad ei ole huvitatud sellest, et nad peavad oma peaga 
mõtlema ja, mis veel hullem, selle mõtlemise tagajärjel tehtud tegude eest 
vastutama. Jõle hea on näidata, et valitsus on loll, minister on loll, euroliit 
on loll, onu trump on loll, jumal taevas on ka loll. Ainult mitte mina!
                 
\question{Sellest järeldame, et sina oled ka loll?}

Loomulikult! See võtab päris kaua aega, enne kui võiks hakata vana kreeklase 
kombel ütlema, et ma tean, et ma midagi ei tea. 

Kasvõi kõik needsamad jumala teemad, alguse ja lõpu teemad. Teadus on jube 
võimsalt edasi. Kõik need kvandid ja mustad augud ja. Aga mis edasi, kuidas 
edasi? Kust see kõik tuli? Suur pauk? Kes paugu tegi? Mis enne suurt pauku oli? 
Ütleme niiviisi, et kui enne ei olnud midagi ja  nüüd korraga tuli maailm, siis 
see ongi nagu maailma loomine. Selles mõttes jumala mõiste, kui me ei hakka 
siin  mõtlema mingit halli habemega taati, kes karjasekepp käes pilve peal 
jalgu kõigutab, võib võtta loodus seaduste, loodusteaduste, looduse enda 
kompleksina. 

Kui sa oled juhuslikult lugenud Ijon Tichy kosmoselendude 
päevikuid\sidenote{Lem, Stanisław. Loomingu Raamatukogu 1962 Nr. 22. Ijon Tichy 
kosmoserändude päevikud. Ajalehtede-Ajakirjade Kirjastus, 1962.}? Mäletad seda, 
kui ta  äikselise ilmaga maja ukse taga koputas, sisse ei tahetud lasta ja kui 
lõpuks lasti, siis hullunud teadlane näitas talle oma ülakorrusel neid 
plaadikaste, kus loeti, et \enquote{see on noor neiu ja see on keegi teine}. 
Aga äkki me olen ise ka plaadimängijad kellegi tolmunud pööningul? Ei tea! Need 
deja vu efektid ja  parapsühholoogia. Minu arust Ijon Tichy väga ilusti 
illustreeris selle ära. Aga ma ei tea , me ei saa seda kontrollida! 

\question{Sind huvitavad sihukesed asjad?}

Aga loomulikult! Kõik räägivad, et kes tõestab jumala olemasolu, kes tõestab 
selle mitte-olemasolu. Mina olen võrrelnud seda sellega, kui meil varbaküüne 
üks rakk hakkaks tõestama inimese olemasolu või mitte-olemasolu. Kuidas ta seda 
teeb? Peremees võib käärid kätte võtta ja küüned lühemaks lõigata\ldots

Meie orgaanilise keemia professor Viktor Palm\index[ppl]{Palm, Viktor} , kui ta 
luges  teadusliku maailmavaatele aluseid, siis ta ütles tol ajal väga julgelt,  
1981. aastal ikkagi, et tema arust on näiteks teaduslik kommunism ja teaduslik 
ateism täpselt samasugused pseudoteadused nagu teaduslik teism või teaduslik 
jumala-õpetus. Nendel asjadel pole teadusega midagi pistmist!
                 
\question{81. aastal öeldi auditooriumi ees selline asi välja?}

Jah. Nad olla ikka selle eest vasu pead ka saanud, sest usinad tegelased käisid 
ikka, käsi kõrva ääres, raporteerimas. Aga miks nimetada mingit asja 
teaduslikuks, kui seal ei ole teadusega mitte mingit pistmist?
                 
\question{No oli ju vaja kuidagi nimetada, et uhkem oleks!}

No eks  praegu on ka, kui me vaatame, igasugused majandusteadused ja nii edasi. 
Jube palju on seal soolapuhumist! Üks mees võtab need meetodid, tõestab ühe 
asja ära, ütleb \enquote{must}, teine ütleb \enquote{ei, ei, ei} ja tõestab 
ära, et kõik on valge. Kolmas räägib pallist ja neljas räägib üldse kokku 
sulanud spektrist. No võta siis kinni, mis on! Täpselt see, kellele mida vaja.
                 
\question{Meil hakkab tasapisi aeg otsa saama, sellepärast küsin selle kohta, 
mis sa praegu teed. Sa teed palju kirjatööd, kuidas sa selle juurde jõudsid? 
Üks asi on palju lugeda, teine asi on  palju kirjutada.}

No miski aeg tagasi, kahe tuhandete alguses, töötas mu naine sellise ajakirja 
nagu Arvutimaailm\index{Arvutimaailm} peatoimetajana. Oli selline tore aeg, kui 
ka IT-ajakirjanikke  mööda maailma lohistati mitte nagu nüüd, kus  keegi meie 
vastu huvi ei tunne ja veetakse ainult autoajakirjanikke, see teeb kohe suisa 
kadedaks. HP-l oli parasjagu mingisugune järjekordne suur konverents tulemas ja 
Arvutimaailma kaastööline, kes pidi sinna minema, tema pass oli ära aegunud, ta 
ei saanud üle piiri. Ja abikaasa küsis, et kas sul on pass korras, sa tunned 
seda värki, teed ära? No mis seal ikka! Läksin sinna, tegin ära. HP ütles, et 
nii põhjaliku ülevaate, nii sisukat, nad pole näinud. Ma olin HP masinatega ka 
juba aastaid aastaid kokku puutunud. 

Ma  ei ole kunagi arvutiteadust õppinud. Aga vaata, kuidas praegu nooremad 
põlvkonnad on hädas, kasvõi DOS-i käsureaga. Kui ikka hiirega lohistatavat 
värvilist ekraani ees ei ole, on kaks käppa püsti. Aga ma olen selle kõigega 
üles kasvanud, nende arvutitega, mis mul endale läbi on käinud, järjest 
arenenud. Ja loomulikult, kui sa nendega tegeled, siis sul tekib huvi. Vastasel 
juhul ei ole vahet, kas sa müüd kartuleid, arvuteid või kaalikad, eks ole. 

Tegin selle HP loo ära, pärast tuldi veel, \enquote{oi kuule, tead, siin on 
jälle üks asi teha} ja tegelikult ongi nii, et väga palju nüüd ütleme nooremast 
põlvest, on neid, kes ei tunne seda raudvara. Raudvara-ajakirjandusega on see, 
et sa pead tegema, sa pead teadma, oskama küsida ja eks ma selle pärast jäin 
neile nagu silma ka. Kukuti igale poole saatma. Teine asi on see, et mul ei 
ole, nagu mainisid, inimestega läbi saamine probleem. Mind ei  kohuta  näiteks, 
kui me konverentsil saame kokku näiteks Inteli viitsepresidentidega. Või kui 
Otellini\sidenote{Paul S. Otellini oli Inteli CEO 2005--2013.}  oli veel see 
kõige suurem füürer, lähed juurde, pistad viis pihku ja ajad  juttu, küsid ta 
käest kõike. Ja kuna mitmetel üritustel sai käidud, siis paljud mehed, näiteks 
põhimine tehnikaohvitser, viitsepresident, juba eemalt tundsid ära,  tuli kohe 
juttu rääkima. Ütlesid, et \enquote{sa oled ainuke, kes asjast aru saab!}

No eks see oli ka üks asi, miks kirjutama kutsuti, miks hoiti orbiidis. Praegu 
on see asi ära vaibunud. Ma olen vist neli korda USAs käinud Singapuris ja ma 
ei tea, mitu korda Koreas, Hiinas, igal pool. Euroopast ei räägigi, seal 
vahepeal oli pidevalt üks konverents teise järel. Aga nüüd on  IT-firmad 
kuidagi nii maha vaibunud. 
                 
\question{Ehk vist saabunud  ka selles vallas nii-öelda pudukaupmeeste ajastu?}

Tihti oli ka see, et nad tulid turgu, sõid ennast sisse, kes saab vedu, kes 
teeb, kellest kirjutatakse, kõik olid väga põnevil sellest asjast. Aga eks nüüd 
on turu stabiliseerumine  käes, turg enam ei kasva. Ütleme, kasvõi 
lauaarvutitega. Ära nad ei kao, nagu paljud ennustasid, sellepärast, mängurid 
tahavad ikka suure 4K ekraani taga korralikult mängida. Siin on küll läpakas 
päris kõvad asjas sees ja kõik, aga ikkagi nii võimas ta ei ole, ta on alla 
\emph{clockitud} võrreldes sellega, mis lauaarvutis \emph{power}-it on. Neil on 
oma nišš olemas ja aga samal ajal niisugust huvi ei ole konkurentsi mõttes, 
nagu on autofirmadel.
                 
\question{Viimane küsimus. Mis oli viimane mäng, mis tõstis heas mõttes karvad 
püsti, et \enquote{see on äge asi!}?}

Kui sa mõtled seda mingit nime, mis on tulemas, siis selleks on 
Pantheon\sidenote{Pantheon: Rise of the Fallen on MMORPG, mille ilmumist on 
mitu korda edasi lükatud ning 2019. aasta lõpus suri juba varem mainitud 
EverQuesti kaasautor ning Pantheoni tootjafirma loovdirektor Brad McQuaid. 
Juunis 2022 on Pantheon eel-alfa staatuses, ajasime Veikoga juttu 2019. aasta 
algul.}. Pantheoni teeb praegu selline mees nagu Brad McQuaid, kes oli ka 
esimese EverQuesti\index{EverQuest} taga peamiseks ajuks. Oma paljude 
kaaslastega, kes dragonistid olid kunagi, kutsume seda mängurite kuldajastuks, 
mida paljud moodsad mängurid sõimavad. Aga meie ei mängi jälle neid kiireid 
piu-pau mänge, ma nimetan neid selja-aju-mängudeks. Ega seal muud pole vaja, 
võta ahv, õpeta kiiresti punast nuppu vajutama, mängib paremini. Ootan ikka 
sellist mängu, kus on tõesti strateegiat, kombinatoorikat, gruppide juhtimist, 
suured AI-d. Selliseid mänge tehti mingil teisel ajastul, seda aega me igatseme 
ja seda lubab Brad McQuaid. 

Mõned moodsad mängud, mis on tulnud, osasid on saadetud, osasid olen ostnud, 
mõned on \emph{free-to-play}, need on kõige jubedamad, kus raha eest võid 
endale elu osta. Hiljuti mängisin seda Fallout 76-te, parasjagu huumoriga, 
jälle  levelid on kõrgel, \emph{quest}-id kõik viimseni tehtud, midagi teha ei 
ole. Käi ringi, kogu sodi, et see sodi maha müüa. Kuu aega mängitud ja kõik. 
Aga kui meenutada viimast tõsiselt head mängu viimasest ajast, jätame 
EverQuest-id sinna kaugustesse, siis üks hea nimi oli Fallen Earth. Ka 
niisugune hea  tuumasõja ja kataklüsmide järgne maailm, millel oli, ma 
ütleksin, kõige parem graafiline ja mängijate vahelise äri süsteem. Muidugi 
vana klassika LOTRO, Lord of the Rings Online, Tolkieni fännidele, kelle hulka 
ma ka ennast loen. Ja, kindlasti Secret World. Väga kõva ja paljulubav nimi, 
aga jälle, nagu on, raha sai otsa ja teiselt poolt oli kahjuks suure fännibaasi 
taga ajamine. See on nagu \emph{modern horror}-i tüüpi,  kõik need Lovecraft'id 
ja Poe'd. Selline maailm, kus mingisugune must kaasaja maagia üritab  maailma 
tungida ja salaühingud võitlevad selle vastu ja ka omavahel. Templirüütlid, 
illuminaadid, dragon'id Hiinas\ldots  Aga ma ei ole näinud nii mõttekaid ja 
põhjalikke ja keerulisi \emph{quest}-e ühelgi mängul. Enamikel on 
\emph{quest}-ide, või noh ülesannete, värk muutunud nii labaseks: mine 
keldrisse, tapa 10 rotti. Nüüd, suur kangelane, maailma päästja, mine uuesti 
keldrisse, tapa 10 rotti ja too sabad ka ära! Või vii pakikene kõrvalkülla onu 
Juliusele. Andke andeks, kas veel lollim saab olla, aga nii ta on.

EverQuestis oli kohti, kus sa pidid näiteks võtma midagi kiilkirjas, sa pidid 
teadma  hieroglüüfe ja muidugi on selge, et keegi neid asju nii täpselt ei tea. 
Selleks oli mängu sisse ehitatud Google'i brauser, sa ei pidanud mängust välja 
minema, said sealt abi otsida. Näiteks oli üks koht, kus oli mingi vihje 
nimega. Leidsid laiba, mille juures oli saatmata postkaart sellele nimele. Ja 
kui hakkasid otsima tuli välja, et see oli üks Saksa kõvemaid krüptograafia 
alusepanijaid, keda väga vähe teatakse. Otsid siis välja: tal on olemas 
spetsiaalne algoritm. Kes tahtsid, võisid seda algoritmi käsitsi kasutada. Aga 
sa võisid  programmi alla tõmmata ja read sinna sisse kopeerida. Ja kui laisk 
olid, pildistasid ekraanilt ära, OCR-isid tekstiks, lasid teksti sinna 
programmi ja tagasi tuli juba mõtestatud tekst, kuhu sa minema pidid.