%!TEX TS-program = arara
% arara: myindex

\index[ppl]{Roos, Meelis}
\textbf{\enquote{Kuidas sina arvutite juurde said?}}

Kõige esimene mälestus millestki arvutitega seoses on koolieelsest ajast, kui
mõnikord läksime emaga lasteaiast koju mööda Liivi tänavat. Paremat kätt mäe
otsas oli üks neljakordne maja. Ema ütles, et see on arvutuskeskus, ja see
kõlas aukartustäratavalt ja põnevalt. Ema töö juures nägin Nairit kah paaril
korral, aga olin siis liiga noor, et see mulle kuidagi muidu korda oleks
läinud kui lihtsalt mingi masin. Ema kabinetis oli üksvahe ajutiselt teletaip,
mis polnud vist kuhugi ühendatud, aga mis suutis ka perfolinti väljastada.
Päris arvutite juurde sattusin isa töö juures kaheksakümnendate lõpus. Füüsikud
ostsid omale mõned arvutid elektromeetria laborisse, nende poolt konstrueeritud
aparaatide (elektromeetrite\sidenote{elektromeetrid on aparaadid üliväikeste
voolude ja laengute mõõtmiseks kõrgeoomilistes (väga suure takistusega)
ahelates}) skeemide arvutusteks ja signaalide uuringuteks reaalajas ning
hilisemaks analüüsiks. Elektromeetria labori juhataja Olev
Saksa\index[ppl]{Saks, Olev} poolt selleks otstarbeks konstrueeritud nn.
dünaamiline kondensaator ja selle baasil ehitatud elektromeeter olid
1960-1970-ndatel aastatel N-Liidu selle ala absoluutne tipptase (mis omaduste
poolest ei jäänud alla maailmas tunnustatud Jaapani Takeda Rikeni firma
dünaamilistele elektromeetritele).

Arvutid, mida 1980-ndatel aastatel elektromeetria laboris kasutati, olid
CAMAC\sidenote{\emph{Computer-Aided Measurement And Control (CAMAC)}
(elektroonikastandard andmete kogumiseks ja seadmete kontrolliks; kasutusel
(osakeste) füüsikas aga ka tööstuses}) kontrolleriga vene
DVK-d\index{Arvutid!DVK}\sidenote{\begin{russian}ДВК, Диалоговый вычислительный
комплекс\end{russian}. Nõukogude personaalarvuti, ühilduv DECi
PDP-11\index{PDP-11} perekonnaga. Varasemad mudelid on tuntud ka kui
Elektronika MS-0501\index{Arvutid!Elektronika} ja Elektronika MS-0502}.
Vahendid ja võimalused (tookordses defitsiidiühiskonnas) arvutite ostuks sai
elektromeetria labor koostöölepingutest Venemaa teadusinstituutidega, eriti N.
Liidu Kosmoseinstituudiga, kelle heaks labor 1980-ndatel põhiliselt töötaski.
Nende tellimusel konstrueeriti ka uus (kosmose jaoks ikkagi!) dünaamiline
kondensaator, ulmeliselt väike (käekella suurune), mis oli gabariitidelt ja
massilt eelmisest mudelist ehk 5-6 korda väiksem ning võimaldas ka ligi 5 korda
suuremat mõõtmiskiirust, võimaldades töötada ka avakosmoses (vaakumis). Kui
leping lõppes, ei jäänud kosmoseinstituudi töötajatel teha muud kui joonistada
skeemid ja joonised ümber ja panna konstrueerijatena oma töötajate nimed alla.

\textbf{\enquote{Kus see kõik sündis?}}

Sel ajal, kui mina seal käisin, oli Tartu Ülikooli\index{Tartu Ülikool}
Füüsikaosakond\index{Tartu Ülikool!Füüsikaosakond}\sidenote{Täpsemalt oli tegu
Füüsika-Keemiateaduskonna Füüsikaosakonnaga} ja ka selle elektromeetria labor
Tartus\index{Tartu} Tähe tn.4, nn. füüsikamajas. Isa töötas seal elektromeetria laboris
elektroonik-konstruktorina. Nad tegelesid elektromeetrite välja töötamisega ja
nende rakendusvõimaluste uurimisega. Mitmetele uuendustele-leiutistele saadi ka
N-Liidu autoritunnistused. Mingilt varasemalt üleliiduliselt näituselt sai
peakonstruktor Olev Saks ühe elektromeetri eest ka NL riikliku preemia.
Mäletan, et sel ajal, kui mina isa juures käisin, konstrueeriti seal ükskord ka
seadet, mis möödalendavate osakeste laenguid mõõtis. Näiteks visati pastaka
kuul, millel oli mingi laeng, mõõteseadmest läbi ja mõõdeti see laeng liikumise
pealt ära. Neil oli seal elektromeetria sektoris lahe töögrupp, noored
ülikoolist tulnud mehed tegid kogenuma juhtimisel lahedaid asju. Nende katsete
juures oli vaja suhteliselt kiiresti muutuvaid signaale registreerida ja
andmeid töödelda, selleks käis arvuti külge spetsiaalne lisaplokk. Ploki sees
oli analoog-digitaaalmuundur (võibolla vastupidine ka aga igatahes niipidi neid
kasutati). Füüsikud õppisid programmeerima, et suuta oma eksperimendi andmeid
reaalajas kätte saada ja seejärel töödelda.

\textbf{\enquote{Aga mis arvuti see selline oli, mis suutis andmeid niimoodi
reaalajas kätte saada?}}

DVK-2M. Vene LSI-11\sidenote{DECi PDP-11 perekonna liige, tuntud ka kui
PDP-11/03. Masinat tutvustati 1975. aastal ja ta oli oma sarjas esimene, mille
CPU oli integreeritud. Mitte küll ühele, vaid neljale Western Digitali poolt
toodetud \emph{Large Scale Integraton (LSI)} kiibile). Meelise sõnul:
\enquote{PDP-11 oli legendaarne DECi masin iidsel ajal enne meie aega}. PDP-11l 
 kirjutati muu hulgas suur osa  UNIXist, ta oli Bell Labsi kuulsa \enquote{\emph{The Unix Room}i} süda}
analoogid. Peaaegu täpne kloon aga natuke kohapeal ka täiendatud. Programmide
poolest ühilduv aga mitte identne. DVK peal jooksis näiteks DECi
originaal-opsüsteem RT-11\index{RT-11}. RT-11SJ oli igapäevane opsüsteem, see
oli \emph{single job} ja RT-11FB'l oli \emph{foreground} ja \emph{background},
millega sai taustal jooksutada mingisugust teist tegevust.

\textbf{\enquote{Kui vana sa olid, su isa töö juurde need arvutid hangiti?}}

Põhikooli teises pooles. Ega mul ei olnud põhjalikku teadmist, mida selle
arvutiga teha saab. Kui ma tegin isale tekstisisestustööd, näiteks sugupuu
andmete sisestamiseks, siis sain ma pärast seda kuni õhtuni mängida. Lemmikmäng
oli Wall\index{Mängud!Wall}, seina pommitamine tennisepalliga. Isa pani mind
arvuti taga kohe tööle, et mu huvist miskit miskit kasu oleks, mis ma niisama
aega raiskan. Programmeerima õpetati ka, eks nad ise ka õppisid. Isa rühmas
programmeeriti BASICus\index{Keeled!BASIC}, FORTRANis\index{Keeled!FORTRAN} ja
CASICus\index{Keeled!CASIC}. See viimane oli CAMACi kontrollerite
programmeerimiseks mõeldud BASICu ja Pascali\index{Keeled!Pascal} vahepealne
keel\sidenote{Ilmselt peetakse silmas keelt formaalse nimega \emph{ANSI
Standard Real-Time BASIC}, mille spetsifitseerib IEEE standard
\enquote{726-1982 - IEEE Standard Real-Time BASIC for CAMAC}}. Selles viimases
mina ei sattunud programmeerima, küll aga BASICus. Minu parim programm oli
programm, mis ajas inimesega eesti keeles juttu. Programm ütles ühe lause,
kasutaja ütles lause ja programm valis juhuslikult vastuse sisseprogrammeeritud
lausete hulgast. Ta suutis mõnikord teemas ka püsida. Näiteks kui programm
ütles \enquote{Osta elevant ära}, siis järgmised kaks lauset olid, et
\enquote{Kõik ütlevad nii, aga osta elevant ära}. Enne ta ei läinud järgmist
lauset valima kui ta oli kaks vastust saanud. Seda mängu teiste töötajate
lapsed mängisid ja neil oli lõbus. See oli lahe emotsioon, et ma tegin midagi,
mis teistele lahe oli.

\textbf{\enquote{Huvitav, et sa kohe hakkasid mängu tegema ja seejuures kohe
midagi AI-sarnast}}

See tundus kõige lahedam asi mida teha. Pealegi oli arvuti nimes dialoog, aga
korralikku dialoogi nagu eriti ei toimunud, ja ma siis tegin nii, et toimuks.

\textbf{\enquote{Need füüsikud pidid ju kähku õppima, sest reaalajas
riistvarast andmeid lugeda on ju keeruline?}}

Neid oli seal rühmas vähemasti kolm-neli meest, kes programmeerimist õppisid. Pundis
oli teistest veidi noorem Lauri Kärner nende põhiline arvuti-mees ja tema jagas
seda paremini kui teised ja arvutas DVK-ga ka signaalitöötluse jaoks Fourier'
pööret välja jms. Tema juures oli see CAMAC kontroller, millest enne juttu oli.
Arvuteid oli selle labori peale kaks-kolm tükki, aga üks oli see põhiline, mis
oli eksperimendi külge ühendatud. Mina kasutasin arvutit, mis oli niisama
masinakirjutaja toas ja mida kasutati programmide sisestamiseks ja muidu
andmetöötluse jaoks. Näiteks isa tegi selle abil sugupuu üles joonistamist,
neid puid sai rullpaberile\sidenote{Toonaste printerite puhul oli tavaline, et
paberi jooksis printerisse perforeeritud servadega rullist, nii sai paberit
kiiremini liigutada} maatriksprinteriga välja trükkida. Kui hiljem tulid kooli
mingid suguvõsauurijad tudengid ja andsid igaühele paberi, et joonistage kodus
oma sugupuu üles, siis mina palusin isal lihtsalt ühe koopia välja trükkida.

\textbf{\enquote{Aga miks sa lasid ennast sellesse suhteliselt igavasse
andmesisestaja rolli suruda? Lihtsalt, et saaks mängida?}}

Algul selleks, et saaks mängida. Siis selgus, et varsti mõistsin ma seda
tekstiredaktorit K52 paremini kui isa (makro kasutamiseks polnudki vaja seda
iga kord uuesti defineerida). Aga kui selgus, et ise programmeerida saab ka ja
see on täitsa lahe, siis ma keskendusin mängimise asemel rohkem sellele. Ma ei
jätnud mängimist päris maha, mängisin ikka ka vahel.

Mind köitis programmeerimise juures, et programm võis vähendada käsitööd.
Näiteks ESC koodidega Robotroni printerile õigeid asju saates\sidenote{\emph{Epson
Standard Code for Printers, ESC/P\index{ESC/P}} on Epsoni poolt
maatriksprinterite jaoks välja töötatud (ja termoprinteritel siiani kasutusel
olev) keel, mis võimaldab juhtida rastrivõimekuseta printerit. Keel sai oma
nime sellest, et tema käsud algasid sümboliga ESC (ASCII 27). Näiteks ESC E
lülitas sisse ja ESC F välja rasvase trüki} trükkisin oma õpikusilte, kus oli
rasvases ja suuremas või väiksemas kirjas kõik vajalik erinevatel ridadel
kirjas. Ema tuttav tahtis oma firma Tensiid logo visiitkaartidele, see logo tuli
siis teisendada Epsoni printeri keelseteks graafika ESC-jadadeks. Ma joonistasin selle
\emph{bitmap}ina üles aga siis leidsime, et ei tasu vaeva ja seda logo ma ei
teinud. See oli näiteks koht, kus ma leidsin, et programmist võiks oluliselt
kasu olla. Ja üheksandas klassis oli seik, kus ma jäin füüsika tunnis
programmeerimisega vahele -- kirjutasin oma vihikusse mingit BASIC-programmi ja
õpetaja läks mööda ja ütles midagi stiilis, et siin tunnis tegeleme füüsikaga,
mitte programmeerimisega. Ja keskkoolis tegin programmi, mis otsis lähendusi
kaheteistkümnendale juurele kahest, nii et saaks isaga süntesaatori ehitamisel
sagedusjagaja võimalikult täpse teha -- oli esimene kasulik programm, mida ma mäletan.

\textbf{\enquote{Kuidas õppimine käis?}}

Ma sain mingisuguseid venekeelseid raamatuid. Osalt raamatukogust isa tõi,
osalt oli ehk mõni raamat tal töö juures olemas. Need olid enamasti kusagilt
laenatud. Näiteks mul oli segadus ASCII koodi ja \emph{Escape} koodidega, mida
sai printerile ja terminalile saata. Siis ma mäletan, et küsisin isalt nõu, et
mis neil vahet on et kas see on seesama asi. Ja siis oli erinevaid raamatuid.
Näiteks oli üks raamat BASICu kohta, kust on mul siiamaani meeles mingisuguse
käsu kirjeldus, mis minu meelest ei sobinud niisugusesse raamatusse:
\begin{russian}\enquote{эта команда работает хорошо}\end{russian}. See käsk
töötab hästi. Minu meelest oli see lati liiga madalale laskmine. Minu meelest
peaks kõik hästi töötama, asjad tuleks nii teha.
Mäletan, et leidsin füüsikamaja raamatukogust riiulist eestikeelse raamatu
"Programmeerimine keeles C". Uurisin, mis see on. Ema vaatas, et vist mingi vana asi.
Ei, see on uus raamat, selgitas kõrvalt raamatukoguhoidja. Aga C oli tol hetkel
võõras asi ja raamat jäi laenutamata.

\textbf{\enquote{Sul oli ju siis päris korralik vene keele oskus?}}

Jah, ma olin üheksandas klassis umbes kui ma programmeerimist õppisin ja
kannatas venekeelset raamatut lugeda küll. Meil oli põhikoolis selline vene
keele õpetaja, kellega pidi õppima, mul tõenäoliselt oli üsna normaalne vene
keele oskus selle vanuse kohta. Ma käisin Tartu 12.
Keskkoolis\index{Koolid!Tartu 12. Keskkool}. Meil oli üks ukrainlanna, Zinaida
Tovkatš vene keele õpetajaks. Tema kohta meie kirusime, et ta on väga range ja
isegi haige ei ole kunagi. Muudkui peab õppima ja muidu ei pääse.

\textbf{\enquote{Kas keegi sind õpetas ka või käis ainult raamatu järgi see
asi?}}

Isa õpetas mulle neid asju, mida tema teadis. Näiteks õpetas ta mulle
plokkskeeme, sest ta ise õppis nende abil. See kestis kuni keskkooli ajani
välja, et kui mina tegin programmi ja see ei töötanud, siis oli kaks viisi
silumiseks. Üks oli see, et ma trükin ta rullpaberil välja ja loen õhtul kodus.
Teine võimalus on see, et ma joonistan selle asja plokkskeemiks ja lähen näitan
isale. Sealt pealt tema oskas vigu leida küll. Ja plokkskeemiks joonistamisel
leidsin ma tihti vead ise ka üles. Ja isegi kui ma Pascal-keeles kirjutasin,
mida isa ei osanud, ma sain temalt ikkagi plokkskeemide tasemel abi. Sest isal
oli hea loogiline mõtlemine ja ta seletas mulle minu vead ära küll.

Minu ülesanne oli kodus keskkütte katla alla tuli teha. Selle süütamiseks oli
füüsikaosakonnast toodud vanapaberit, mille hulgas oli teinekord mingeid arvuti
väljatrükke, mida ma lugesin. Panin need kõrvale samal ajal kui ajalehed ja
muud läksid katla süütamiseks. Näiteks ma leidsin Minsk
32\index{Arvutid!Minsk!Minsk-32}-e\sidenote{Minsk-32 loodi kuuekümnendatel, nagu
nimigi ütleb, Minskis. Tegu oli mitmest mudelist koosneva Minsk suurarvutite
sarja kõige võimekama esindajaga. Oli laialdasel kasutusel, kuni asendati
seitsmekümnendatel IBM 360 kloonidega} mingisugused 32-bitised krahhi- või
muidu mälutõmmised. Ma olin üllatunud, et minul on 16-bitised PCd (see oli tol
hetkel hiljem vist kui ma juba PC taga olin) aga nendel oli juba siis 32-bitine
arvuti. Ja seal olid FORTRAN-programmid, mida ma huviga lugesin. Isa kõrvalt
ütles, et ah, need ei ole suurt midagi väärt, et see mees, kelle programmid
need on, ei oska veel eriti programmeerida, tema programmide pealt pole eriti
mõtet eeskuju võtta. Aga põnev oli neid lugeda sellegi poolest. FORTRANit
õppisin keldris katla kütmise juures!

\textbf{\enquote{Miski pani sind tulehakatust lugema, mis see oli?}}

Seal olid uued põnevad asjad!

\textbf{\enquote{Kas sa peale tulehakatuse midagi muud ka lugesid? Või oli
näiteks muusika huvi?}}

Ulme huvi natuke oli. Mul õnnestus saada venekeelsed Asumi\sidenote{\label{sidenote!asum}Isaac
Asimovi poolt kirjutatud sari. Ilmus esmakordselt triloogiana 1951. aastal,
tunnustati 1966. aastal Hugo auhinnaga \enquote{\emph{Best All-Time Series}}.
Alates 1981. aastast lisandus triloogiale veel köiteid} seeria raamatud, neid
oli rohkem kui kaks esimest\sidenote{Eesti keeles ilmusid kaks esimest Asumi
raamatut \enquote{Asum} ning \enquote{Asum ja impeerium} vastavalt 1985. ja
1989. aastal Linda Ariva tõlkes}. Asumid mulle meeldisid ja ühe isa sõbra käest
laenasime venekeelsed ülejäänud Asumi raamatud. Mul õnnestus vene keeles
raamatut lugeda, ma olin selle üle sügavalt üllatunud. Isa luges neid algul
ise, hiljem mina. Nii et ulme huvi oli küll, aga see ei olnud väga sügav. Seda
oli valdavalt nii palju kui kodus sattus Mirabilia sarja ulmekaid olema. Need
said kõik läbi loetud. See ei olnud esialgu eriti seotud arvutitega, arvutid
olid asi, mis tuli reaalsest maailmast. Näiteks sõitsin bussiga koju ja ükskord
Pärmivabriku peatusest mööda sõites parajasti ema seletas mulle arvutiviiruste
kohta, mida ta oli kuskilt Horisondist või mõnest niisugusest kohast lugenud.
Väga põnev oli. Parajasti sõitsime Pärmivabriku peatusest mööda, kui ma esimest
korda arvutiviirustest kuulsin. Seda ma mäletan.

\textbf{\enquote{Kas sa olümpiaadidel ka käisid?}}

Jaa, käisin. Matemaatikaolümpiaadil käisin neljandast klassist saadik. Oli
naljakas korrelatsioon: lastest, kellega ma olin koos käinud ülikooli töötajate
lasteaias, neist nii mõndagi sai seal olümpiaadidel kohatud. Järgmine laine
olümpiaadidega oli keskkooli minnes.

Miks ma vanast koolist ära läksin? Vanas koolis oli nii, et keskkoolis pidi
tulema kaks klassi. Reaalkallakuga ja humanitaarkallakuga. Ja
humanitaarkallakuga pidi see \enquote{A} ja eliitklass tulema, kuhu paremad
õpilased lähevad ja ülejäänud võinuksid minna sinna reaalkallakuga klassi. Ma
leidsin, et see on lati alla laskmine, et ma tahaksin ikka paremat. Mind
kutsuti Nõkku\index{Koolid!Nõo Keskkool}. Hilisem ülemus
Cyberneticast\index{Cybernetica}, toonane Nõo kooli direktor Uuno
Puus\index[ppl]{Puus, Uuno} saatis laiali kõikidele olümpiaadikutele Nõo kooli
kutseid. Sain ka. Kaalusin. Oli kaugel. Raske. Siis selgus, et esimene keskkool
Tartus\index{Koolid!Tartu 1. Keskkool} on ka täitsa kõva tasemega. Helistasin
kooli ja küsisin, et kas teil arvutiklass on. Direktor võttis vastu ja
reklaamis, et neil on väga hea arvutiklass. Selle peale ma otsustasin sinna
minna. Viisin 1990. aasta kevadel paberid Esimesse Keskkooli, kui sügisel
kohale läksin, oli see juba Hugo Treffneri Gümnaasium\index{Koolid!Hugo
Treffneri Gümnaasium|see{Tartu 1. Keskkool}}. Olid tõesti väga head arvutid
(Yamaha MSX-II), lisaks põhilisele arvutiklassile oli seal ka
Juku\index{Arvutid!Juku}-klass.

\textbf{\enquote{Sul oli siis selge arusaam, et sa just sinna kooli tahad
minna?}}

Jah, ma läksin nimelt sinna. Selle kohta tegi ajaloo õpetaja meil kunagi
pisikese kiire küsitluse üheksanda klassi kevadel. Et paljud teist siia jäävad
ja paljud lähevad kuhugi mujale. Ja siis ta küsis kolme tema nina all oleva
tegelase käest. Esimeses pingis sattusin mina istuma ja minu tagant kahe
tüdruku käest, kes olid ka kätt tõstnud, et lähevad mujale, küsiti, mis nad
teevad. Need oli täpselt need kolm, kes läksid Esimesse Keskkooli. Nii et kõigilt,
kellelt ta küsis, sai vastuseks, et läheme ära esimesse keskkooli. Tüdrukud läksid
teise paralleeli, bioloogia-keemia harusse. See tundus olevat umbes see vanus,
kus mõned hakkasid ise mõtlema oma tulevikule ning seda planeerima ja mõned
lasid asjadel isevoolu teed minna. Mina olin nende hulgas, kes leidis, et ma
tahan ise oma tulevikku kujundada.

\textbf{\enquote{See oli see aeg, kui ühiskonnas hakkas juba muutus tulema, eks
ole}}

Natuke oli juba varem selles mõttes, et kooperatiivid\sidenote{Nõukogude Liidu
lõpuaastatel lubatud spetsiifiline ettevõtlusvorm, neid kasutati esimesel
võimalusel massiliselt väike-ettevõtluse alustamiseks} olid juba varem olemas
ja asjadest tohtis rääkida. Selle sama üheksanda klassi jooksul ma jõudsin kaks
korda kirjutada ühele õpetajale referaate, millest võib olla aasta varem oleks
vanematel pahandus tulnud. Aga siis juba tohtis. Selle õpetaja kohta oli teada,
et ta on üks paras punane. Aga sain nende referaatide eest isegi kiita, mis oli
üllatav. Ma mõtlesin, et tuleb kuidagi oma seisukohti kaitsta, sain hoopis
kiita.

\textbf{\enquote{Kas sind keskkooli ajal tööle ei tõmmatud kuhugi?}}

Ainult natukene. Tiražeerisin isa töö juures elektromeetrite trükkplaate.
Joonistasin ahjulakiga ja risti ära lõigatud otsaga süstlaga rajad, söövitasin
plaadi ära, tinatasin ära ja jootsin sinna peale kõik elemendid vastavalt
skeemile.

\textbf{\enquote{Aga see tahab ju käelist oskust ja elektroonikahuvi, kust sul
see?}}

Seitsmeaastaselt oli mulle vist isa töö juures jootekolb esimest korda kätte
sattunud, kui ma suvalisi tükke kokku jootsin. Eks ma oskasin kolbi hoida ja
elektroonikahuvi mul oli. Aga elektroonikat ma ei osanud, analoogelektroonikat
ei ole ma kunagi ära õppinud. Üldisi põhimõtteid tean aga ise midagi teha ei
ole osanud.

Digielektroonika oli seal kõrval. Kui keskkool hakkas läbi saama ja oli vaja
ülikooli minna, siis mina olin neljandast klassist peale kindel olnud, et ma
lähen füüsikat ja nimelt elektroonikat õppima. Aga siis tulid arvutid, kah
põnev elektroonika värk, neid sai matemaatikateaduskonnas ka õppida. Mul oli
kuhugi ilma eksamiteta sisse saamised, äkki matemaatikasse ja füüsikasse
olümpiaadi tulemuste pärast või midagi. Otsustasin matemaatika kasuks, sest
füüsikaosakonnas ma olin kogu aeg kohal ja mulle ei meeldinud see. Tundus, et
kui midagi ära tahta teha, siis peab ainult endale lootma. Oli nihukesi
saarekesi, kes tegelesid oma kitsa erialaga, aga laiemat kandepinda ma ei
märganud. Oli töögruppe, kes olid vingel tasemel ja tegelesid oma asjaga. Võib
olla, et ma ei sattunud õigete inimestega kokku, aga tundus, et pigem on
füüsika nihukene seisev konnatiik. Igaüks on seal kinni, kus on, ja nii on.

Ega seal oli huvitavaid ja põnevaid asju ka. Näiteks olid füüsikapäevad, kus mu
isa käis kuulamas Undo Uus\index[ppl]{Uus, Undo}i, kes rääkis materialismi
ümber lükkamisest filosoofiliselt. Isa tuli koju, jutustas. Mina panin kõrva
taha. Selliseid asju oli sealt ikka päris mitmeid. Füüsikalist maailmapilti
tuli vanemate kõrvalt üksjagu, see oli mul olemas.

\textbf{\enquote{Kuidas sa siis ikkagi matemaatikat sattusid õppima? Lihtsalt
seepärast, et sai eksamiteta sisse?}}

Füüsikasse ma oleks vist ka saanud ilma eksamiteta, need ei oleks probleem ka
olnud, ma arvan. Olin lihtsalt laisk, laisad me olime kõik. Keskkoolis
klassijuhatajal tuli kaheteistkümnendas klassis üritada meile ikka auku pähe
rääkida, et poisid, olge tublid ja võtke tehke need eksamid ikka ära, siis saab
medalile pretendeerida, muidu ei saa. Aga medaleid oleks ju vaja. Siis me
tegime vist kolm medalit klassi peale või midagi. Mina sain hõbeda. Ma täpselt
ei mäletanudki, kunagi hiljem kooli koduleheküljelt lugesin. Seda ma mäletasin,
et medal oli, aga mis medal, seda ei mäletanud. Polnud oluline, see tuli
iseenesest.

\textbf{\enquote{Ühesõnaga, matemaatikasse sa läksid seepärast, et füüsika
tundus natuke seisev vesi olevat?}}

Jah. Ja ma olin kuu aega enne paberite sisse andmist kindel, et matemaatikasse
ma küll ei lähe. Me käisime koolist tiimiga Moskva lahtisel
olümpiaadil matemaatikas. Seal olid mingid doktorandid, kes meiega tegelesid.
Ühtlasi toimus seal ka \begin{russian}Международная конференция старшикласников
"Наука, природа, человек"\end{russian}\sidenote{\enquote{Rahvusvaheline
vanemate klasside konverents \enquote{Teadus, loodus, inimene}}} kus
keskkooliõpilased said ise tehtud asju esitada. Keegi oli teinud kiiret
vektorgraafikat, et voldime siin kuubikut kiiremini kui AutoCAD, või mis
iganes. Ägedaid asju oli tehtud. Seal oli mingit Hollandi rahvast ka, oli
rahvusvaheline küll. Seal need doktorandid, kes meiega tegelesid, olid
nihukesed parajad uhuud. Näiteks tuleb tegelane hommikul tahvli ette, triiksärk
on lükatud alukate sisse, alukad ulatuvad kümme sentimeetrit pikkade pükste
pealt välja ja tuleb niimoodi tahvli ette. Ma leidsin, et vot matemaatikuks
mina küll ei lähe. Aga siis ma mõtlesin ikkagi ümber. Matemaatikuks ma ei
tahtnudki, ma läksin neid arvuteid õppima matemaatikateaduskonna\index{Tartu
Ülikool!Matemaatikateaduskond} poolt. Mitte elektroonika poolt aga
programmeerimise poolt.

\textbf{\enquote{Kuidas sulle ülikooli üleminek tundus? Sa ütlesid, et olla
laisk olnud. Minu mälu järgi pidi ülikoolis kohe hakkama tööd tegema?}}

Jaa. Keskkoolis ma sain endale lubada laisk olemist isegi seal eliitkoolis, no
vähemalt mingil tasemel. Ja ma sain keskkoolis arvutimängude mängimise isu täis
mängida. Ostsin omale üheksanda klassi lõpus ZX Spectrum-i\index{Arvutid!ZX
Spectrum}\sidenote{ZX Spectrum oli Sinclair Research'i poolt 1982. aastal
Ühendkuningriigi turule lastud 8-bitine personaalarvuti, mõeldud peamiselt
koduseks kasutamiseks. Selle kloone liikus Nõukogude Liidus hulganisti, skeemid
olid koguni hobiajakirjades avaldatud} Leningradi turu klooni 1500 rubla eest,
kui rubla juba kukkus. Siis oli suur rahanumber, aga ma sain oma isu täis
mängida. Joystick\sidenote{Eesti keeles \enquote{juhtkang}. Eelmise sajandi
algul Ameerika Ühendriikides patenteeritud, Teises Maailmasõjas Saksa vägede
poolt laialt kasutatud ja kuuekümnendate lõpus arvutimängude külge jõudnud
kaheteljeline juhtimisvahend. 21. sajandil kaotas ta mängude juhtimisel
kiiresti populaarsust hiirtele ja on praegu peamiselt kasutusel lennunduses}
sai peeneks mängitud, plastmassi paikasin alumiiniumiga. Tuttav treial tegi
talle uue varre, pärast kippusid kontaktid läbi põhja tulema. Aga Spectrum oli
nii hea arvuti, sellest sai aru igat pidi! Sai programmeerida BASICus ja Z80
Assembleris\index{Keeled!Assembler}. Sellest arvutist võis lõpuni aru saada.
Elektroonikast peaaegu ka, välja arvatud videopildi genereerimise osa.
Originaalis kasutati ULA kivi, vene variandis realiseeriti see
laus-elektroonikana\sidenote{Originaalne ZX Spectrum sisaldas kahte suurt
40 jalaga mikroskeemi - Z80 protsessor ja üks eelprogrammeeritud loogikamassiiv
(ULA - Uncommitted Logic Array). N-liidus tehtud Sinclairi koopiad kasutasid
viimase asemel poolt trükkplaaditäit lihtloogikaelemente.}, sest seda kivi ei
olnud kloonina võtta. Nii et ma sain sõbra Sinclairi diagnoosimisega hakkama.
Näiteks, et sul on ROMi see ja see jalg lahti ja ei anna kontakti, seetõttu on
tähtedel vertikaalsed kriipsud läbi, nagu dollarimärgid. Tähtede tabel oli
ROMis ja kui seal bitt oli maas, siis joonistati selle biti koha peale alati
täpp ja tekkis püstkriips. Järelikult pidi sellel ROMi kivil selle biti jalg
mitte kontaktis olema.

\textbf{\enquote{See tähendab, seda, et sa pidid neid asju põhjalikumalt
uurima?}}

Skeeme ma ikka kuskilt raamatutest ja mujalt nägin. Keskkooli lõpus, kui
Venemaal käisin, ostsin metroost raamatu \begin{russian}Введение в схемотехники
IBM PC / AT\end{russian}\sidenote{Eesti keeles \enquote{Sissejuhatus IBM PC/AT
skeemitehnikasse}. Ilmselt peab Meelis silmas kodanike \begin{russian} Г. Н.
Левкин\end{russian} ja \begin{russian}В. Е. Левкин\end{russian} 1991. aastal
ilmutatud raamatut}. Venelased olid 286 skeemid välja ajanud arvuti järgi ja
üles joonistanud. Neil oli seal viga, minu mälu järgi. Mingi reset signaali
puhul oli aktiivne null ja aktiivne üks kusagil segamini, niisugust asja
trükitud raamatus avastada oli igatahes lõbus. See Venemaal käik oli seesama
kord, kui me olümpiaadil ja konverentsil käisime. Konverentsi osast ei teadnud
me enne midagi, kui me sinna kohale sattusime. Meil ei olnud mingeid
ettekandeid, kuulasime niisama, mis räägitakse. Ja vaatasime, mihukesed on
kenamad tüdrukud. Üks vene Maša oli kõige kenam.

Olümpiaadil me eriti hiilgavaid tulemusi keegi ei saanud. Mina sain meie
pundist kõige parema tulemuse, sest ma ei joonud eelmisel õhtul alkoholi. Seda
oli seal saada ja siis järgmisel hommikul pohmakaga inimesed ei esinenud oma
võimete tasemel. Nii tuligi välja, et mina olin meie omadest parim, kuigi
vähemasti üks kaasas olnud meestest oli parema peaga. Minu jaoks oli õppetund,
mida rõõmsalt teistele edasi jagada: et näe, olümpiaadi tulemus sõltus selgelt
sellest, kes ja mida eelmisel õhtul jõi.

\textbf{\enquote{Räägi palun ülikoolist, me sattusime seal 1993. aastal kokku.
Kuidas sulle see matemaatika tundus, mida me kohe esimese semestri alguses
saama hakkasime?}}

See oli üks suur kukkumine. Ma näiteks mõtlesin ülikooli tulles, et ma tean,
mis on reaalarv. Siis tuli matemaatilise analüüsi esimene loeng, kus hakati
neid defineerima. Kõike hakati algusest peale defineerima, kõik muu ehitati
ainult nende definitsioonide otsa. See kõik tahtis palju harjumist ja palju
tööd aga mina ei olnud harjunud tööd tegema.

Ma mõtlesin, et ma oskan programmeerida, kui ma ülikooli tulin. Aga Rein
Pranki\index[ppl]{Prank, Rein} matemaatilise loogika õppeprogrammid näitasid,
et on veel palju asju, millest ma aru ei saa. Seal joonistati näiteks ekraanile
tõestuspuu ja ma mõtlesin, et \enquote{Vau, puud ma niimoodi joonistada ei
oska}. Me õppisime seda küll hiljem umbes kolmandal kursusel Varmo
Vene\index[ppl]{Vene, Varmo} Funktsionaalses Programmeerimises, kus me mingi
\emph{minimax}i\sidenote{\emph{Minimax} on algselt nullsummamängude analüüsiks
formuleeritud otsustusalgoritm, kuid mida on hiljem oluliselt laiendatud ning
mis leiab laiemalt kasutust tehisintellekti puhul, statistikas, filosoofias ja
mujal. Algoritm minimeerib võimalikku kahju halvimal, maksimaalse kahjuga,
juhul andes optimaalse mängustrateegia eeldades, et ka oponent mängib
optimaalselt} ülesandetüübi näiteülesandeks puu paigutust tegime. Esimese
kursuse järel oleks seda ehk rekursiooniga ka kuidagi teha saanud, aga see oli
jah näide sellest, et kõik ei ole ikka triviaalne. Ei saa igale asjale jõuga
peale minna.

\textbf{\enquote{matemaatiline analüüs, eriti matemaatiline analüüs II, võttis
meil kursuse peal palju rahvast hõredamaks, see tahtis harjumist saada}}

Algebra tahtis ka. Kogu see matemaatiline lähenemine, et me ehitame asju üles
mingite definitsioonide ja aksioomide otsa. Kogu see asi tahtis kõvasti tööd.
Lisaks kukkusin ma esimesel kursusel haiglasse. Eksamisessiooni ajal ei jõudnud
ma mõnesid eksameid tehtudki, tegin neid alles järgmise semestri sees. Käisin
dekaanilt küsimas sessi pikendust, sest vanemad õpetasid, et nii tuleb teha.
Siis dekaan ütles, et meie ajal enam niisugust asja pole, lihtsalt tehke need
eksamid ära, kuidas saate.

\textbf{\enquote{Mis hetkel oli võimalik minna arvutiteadust õppima?}}

Mingid põhimoodulid oli vaja ära teha ja siis vist esimese aasta järel sai
spetsialiseeruda. Kuna ma need moodulid sain kokku, siis kaldusin üldisest
õppekavast kõrvale sellega, et läksin võtsin koos aasta vanematega põnevaid
arvutiteaduse aineid. Käisin aasta vanema rahvaga koos lahedaid asju kuulamas.
Ja siis järgmine aasta tuli võtta need ained ka, mis õppekavast tegemata olid.
Minu oma kursus oli need ära teinud, mina tegin neid siis koos aasta
noorematega. Mingeid tõenäosusteooriaid ja mingisuguseid matemaatikaaineid.

Juhtus ka seda, et ma kodutöö programme teiste pealt maha kirjutasin. Meil oli
Algebra ja Analüüsi Numbrilised Meetodid, kus me arvutusmeetoditega
numbriliselt tegelesime. Ma sain algoritmidest aru, nad ei pakkunud mulle
algoritmi tasemel pinget ja ma ei viitsinud neid teha. Piisas, kui ma olin aru
saanud, mis seal tehakse. Leidus üks lahke kaastudeng Jane, kelle programme ma
esitamiseks kasutasin. Muutsin vist natuke treppimist ja muutujate nimesid.
Mäletan, ma kirjutasin ühele kommentaaridesse üles \enquote{Viimati
modifitseerinud Meelis Roos}\sidenote{Enne, kui vabavaralised tsentraliseeritud
ja hajutatud koodirepositooriumid laialt levima hakkasid, hoiti koodi enamasti
lihtsalt kettal. Seetõttu oli levinud praktikaks faili päisesse lisada
kommentaar faili autori, viimase muutmise kuupäeva ja muu tarvilikuga}. Eks see
praktikumi juhendaja teadis, et neid programme üksteise pealt üksjagu maha
võetakse. Seepärast lasi ta endale ette seletada, mida see programm täpselt
teeb, sellega polnud probleemi ja nii sain kõik asjad ilusti tehtud. Kirjutasin
programme tüdrukute pealt maha, sest ma ei viitsinud programmeerida.

\textbf{\enquote{Kas see ülikooli arvutuskeskus seal Liivi tänaval ei neelanud
sind kuidagi endasse, nagu ta nii mõnedki neelas?}}\index{Tartu
Ülikool!Liivi õppehoone}

Neelas ka mind aga natuke teistel viisidel. Mina ei kadunud ära
Muda\index{Mängud!Muda}\sidenote{\label{sidenote!muda}Originaalis \enquote{Multi User Dungeon
(MUD)}. Paljude osapooltega reaalajaline tekstipõhine seiklusmäng. Täpsemalt
siiski mängude alaliik, sest leidus mitmeid eri rõhuasetusega eri koodibaase
kasutavaid versioone, mida jooksutati mitmetes eri serverites. Kuna Muda pakkus
toona ainulaadset koos mängimise ja suhtlemise viisi, tekkis paljudel kiiresti
sõltuvus ja liigne Mudas veedetud aeg oli sagedane ülikoolist välja langemise
põhjus.} mängima. Muda oli küll tore: kui ma oma telneti klienti kirjutasin,
sai seda Muda serveri vastu testida näiteks. Selleks oli Muda tore.

\textbf{\enquote{Miks sa kirjutasid oma telneti kliendi?}}

Võrguprogrammeerimise harjutamiseks. Tahtsin osata igasuguseid sokliühendusi
teha. Ma kirjutasin oma netcati laadset asja, mis ei teinud mingisugust telneti
\emph{handshake}'i ja ei osanud \verb|echo off|i ja selliseid keerulisemaid
asju, vaid lihtsalt sokli kuhugi ühendas. Sellise asja kirjutasin endale
igasuguste asjade torkimiseks. Seal olid mingid mured stiilis et kui pikkade
pakettidega asju saata ja vastu võtta võis. TCP võis andmed ju suvalise koha
pealt ära hakkida. Ei saanud eeldada, et kui teiselt poolt rida sisse
kirjutatakse, et sa selle täpselt rea suuruste tükkidena kätte saad. See oli
põnev.

Aga mind neelas see arvutuskeskus natuke teistmoodi. Teisel korrusel Ülo
Kaasiku\index[ppl]{Kaasik, Ülo} kabineti kõrval oli magistrantide arvutiklass,
kus olid värvilised Sun'id. See oli ette nähtud magistrantidele, aga kellelgi
ei olnud eriti probleeme, kui mina ka sinna imbusin. Aegajalt seal ei olnud
kohti ja tuli ette, et ma kellelegi oma koha pidin loovutama, aga enamasti ei
pidanud. Aasta vanema Raul Tölbiga\index[ppl]{Tölp, Raul} istusime seal koos ja
seal sai õpitud ära Unix.

Kuidas ma üldse sinna Unixit kasutama sattusin, oli omakorda lõbus. Seda ma
võin lausa rääkida, kust on pärit minu kasutajanimi \enquote{mroos}. Minu
esimene online konto oli masinas vask.ut.ee\index{Masinad!vask.ut.ee}. See oli
VAX\index{Arvutid!VAX} tüüpi arvuti
VMS\sidenote{VAX arvutite \enquote{kohalik} operatsioonisüsteem} opsüsteemiga.
Selline umbes kuupmeetrine kast pluss kettad seal kõrval\sidenote{Huvitav, et Meelis meenutab just \emph{serverit} samas kui Asko (lk. \pageref{sisu:vase_klass}) ja Jaanus (lk ) meenutavad \emph{terminale}. Kes millega kokku puutus\ldots}. Teine VAX oli
rubiin.physic.ut.ee\index{Masinad!rubiin.physic.ut.ee} füüsikamajas. See oli
MicroVAX, ainult sahtlitumba suurune masin. Vot need olid VMSid. Esimesel
kursusel, selle asemel, et sessi ajal õppida, olin mina raamatukogust võtnud
omale VAX/VMSi raamatu ja õppisin VMSi. Seal oli huvitavaid asju! Näiteks olid
struktuursed failid. Sa võisid tekitada tühja faili, millel on ette antud
kirjestruktuur. Opsüsteemi tasemel oli \emph{Record Management System}, millega
mingis keeles kirjeldati struktuur ära ja tekitati selle kirjelduse järgi fail.
Fail võis olla ka tühi, aga tal oli struktuur olemas.

Kogu õiguste süsteem selles operatsioonisüsteemis oli keeruline. Windows
NT\index{OS!Windows NT} on selle sisemiselt pärinud või umbes niimoodi. Nii
keerukas ei ole minu meelest kui VMSis aga kui ma nägin Windows \emph{syscalli}
\verb|CreateProcess| koos portsu argumentidega, siis tuli tuttav ette, sest
VMSi SYS\$CREATEPROCESS oli umbes samasuguste argumentidega. SYS\$ käis
syscallide funktsioonide nimede ette lihtsalt.

Sealt ma käisin näiteks Lynxiga veebis surfamas. Tõmbasin FTP-ga mingeid faile,
mida kuskilt kolmandat teed mööda kuidagi flopi peale sain. Käisin Internetis
ka igasugu asju lugemas. Ma eriti ei programmeerinud VMSis. Kui vaja oli
kursaõele Pascalis programmeerimist õpetada, aga ainult VAXu klass\index{Tartu Ülikool!Liivi Õppehoone!Vase klass}\sidenote{Meelis peab silmas sedasama klassi, millest juttu leheküljel \pageref{sidenote!vaks}, vt. märkus number \ref{sidenote!vaks}} vaba oli,
siis ma näitasin talle Pascalis programmeerimist VAXu peal. Ta oli väga
üllatunud, et seda arvutit saab ka programmeerida. Aga sai.

Seal oli lahe programm nimega SWIM, mis lasi ühe terminali peale multipleksida
mitu akent, sai lausa akende suurusi muuta. Sellega ma kasutasin kolme
rakendust korraga. Aga SWIM kippus ajama terminali hanguma, kõditas vist mingit
VMSi terminali draiveri bugi või mida iganes. Siis tuli leida administraator,
keda tihti majas ei olnud, või siis keegi sõber tudeng logis üle võrgu
rubiini\index{Masinad!rubiin.physic.ut.ee} ja talk-is Ville
Hallikuga\index[ppl]{Hallik, Ville}, kes oli sealne VMSi admin. Villel oli
juurdepääs vaske olema ja ta sai tulla ja terminali päästa - hangunud terminali
tagant ei saanud keegi enam midagi kasutada. Tappis SWIMi ja mingid asjad ära
seal, nii et terminal sai jälle vabaks. Nii et SWIM oli tülikas. Keegi rääkis,
et arvutiteaduse instituudi Sun'ides on Unixis programm nimega screen, millega
sedasama teha saab. Ja siis tekkis mõte seda kasutada. Ma olin Unixit seni juba
korra kasutanud. Math.ut.ee-s\index{Masinad!math.ut.ee}, kui tekkis online
võrk, tuli 386BSD\index{OS!386BSD}. Ja see uuendati 93. aasta lõpus mingile
uuele tundmatule opsüsteemile. Sinna osteti 486-arvuti asemele, suure
kahe-gigase\sidenote{Meelis peab silmas kahte gigabaiti. Konteksti mõttes on
oluline märkida, et tol ajal piisas keskmist sorti arvutifirma failiserveri
kõvakettaks ühest gigabaidist üsna pikaks ajaks. Aastal 2020 täidab keskmine
koduinternetiühendus selle mahu umbes minutiga} SCSI vindiga. Selle SCSI kaardi
jaoks 386BSD enam ei sobinud ja pandi asemele mingi uus tundmatu asi nimega
Linux\index{OS!Linux}. Versioon 0.99pl3 või midagi, kui õigesti mäletan.

\textbf{\enquote{Kust selline asi sattus Tartu linna?}}

No aga kust 386BSD sai? Internet oli ju olemas. Kasutajad koliti 386BSDst
Linuxisse siuhti üle ja mul oli mingis Linuxis kasutaja. Jaanuaris umbes
uuendati see Linux ära kerneli versioonile 1.0.2. Ma olin natukene nuusutanud
Linuxit. Kui ma tahtsin seal Liivi tänaval Unixi screeni, siis math.ut.ee
ühendus oli päris aeglane\sidenote{math.ut.ee asus füüsiliselt
matemaatikateaduskonna hoones Vanemuise tänaval. Seega peetakse järgnevas
silmas internetiühendust kahe, linnulennult 550 meetrise vahega, hoone vahel
Tartu linnas}. 9600-ne ühendus jagatud paljude kasutajate ja meilide ja muude
vahel. Siis ma küsisin omale cs3-e (hilisem
romulus.cs.ut.ee\index{Masinad!romulus.cs.ut.ee}) konto ja põhjendasin seda, et
tahaksin näppida mõnda mitte-Linux Unixit. Seal oli Solaris\index{OS!Solaris}.
Ja see tundus Toomas Soomele\index[ppl]{Soome, Toomas} piisavalt hea põhjendus.
Toomas Soome kasutajanimi oli \enquote{tsoome}, ma mõtlesin, et ahaa, et eks
Unixis käib see niimoodi. Küsisin siis omale tema süsteemi sama skeemi järgi
kasutajanimeks \enquote{mroos}. Antigi. Seda ma olen sellest ajast edaspidi
kasutanud igal pool. Isegi kui mul on kodus testarvuti, seal olen ma ka seal
harjumusest mroos. Et tsoome mulle kasutajanime teeks, tuli öelda, et ma tahan
Solarist kasutada ja kasutajanimi peaks ka samas formaadis olema, et
võimalikult vähe küsimusi oleks.

Mul möödunud aastal \sidenote{Intervjuu Meelisega toimus 2020. aasta kevadel}
oli väga sürr kogemus, kui kevadel 2019 võttis minuga ühendust Toomas Soome, kellel
oli siiamaani magistrikraad kaitsmata. Ta tahtis, et ma juhendaksin tema
magistritööd. Ma mõtlesin, et muna õpetab kana, et mida mina siin teen. Aga tal
oli korralik tehniline töö olemas ja mina teadsin, mismoodi üks magistritöö
peab enam-vähem välja nägema. Sellest teadmisest oli kasu, see töö sai tal
vormistatud magistritööks ja ta kaitses selle edukalt. Aga algul lihtsalt oli
väga sürr reaktsioon. Arvutiteaduse Instituudis\index{Tartu
Ülikool!Matemaatikateaduskond!Arvutiteaduse Instituut} oli terve hulk rahvast,
kes kaitsesid oma magistrikraadi hiljem.

\textbf{\enquote{Kas sind teadust ei tõmmanud tegema?}}

Ei, vot teadust tegema ei ole mind kunagi eriti tõmmanud ja keegi ei suutnud
mulle ka auku pähe rääkida sel teemal. Väga ei proovitud ka. Meelitati
erinevate viisidega, mingeid materjale ette söötes. Materjalid olid nii
teadusega kui mitte-teadusega seotud. Näiteks Jaanus Pöial\index[ppl]{Pöial,
Jaanus} jagas mulle omal algatusel kunagi \emph{Java Language Specification}i,
et näe üks uus moodne asi. Selliseid asju ülikoolist ikka sattus.

Ma mäletan, ma olin rebane ning ei olnud veel spetsialiseerunud Arvutiteaduse
Instituuti informaatika erialale. Aga mul oli vaja kusagil välja trükkida
viietollise flopi pealt mingit tekstifaili, raamatukogust mingisuguse kataloogi
otsingu tulemus mingite raamatute otsimiseks. Äkitselt tekkis vajadus
laupäevasel päeval trükkida. Ma lihtsalt vajusin kohale Liivi tänavale ja
käisin mööda uksi koputamas. Oli vist laupäev ka või muidu õhtune aeg ja seal
ei olnud palju rahvast. Sattusin Mati Tombaku\index[ppl]{Tombak, Mati} ukse
taha, kes lahkelt lasi trükkida. Ja sellest tekkis nihukene tänutunne kogu
selle instituudi vastu, et siin on lahked inimesed. See oli minu esimene
isiklikul tasemel kontakt instituudi inimestega.

\textbf{\enquote{Millal sa tööle läksid?}} 	

Minu esimene ametlik töökoht oli Tartu Ülikooli Täppisteaduste
Koolis\index{Tartu Ülikool!Täppisteaduste Kool} metoodik. See oli tegelikult
postmasteri töö. Aga postmasteri nimelist ametinimetust ei olnud, oli metoodik.
Korraldati programmeerimise kursust e-mailitsi koolides. Mina olin see, kes
pidas arvet selle üle, kellel olid mis ülesanded lahendatud, ja saatis neile
järgmisi. Arvutiõpetajad, kellele vastused saadeti ja kes neid parandasid,
saatsid minule seisu ja mina siis selle järgi saatsin järgmisi ülesandeid. Mina
olen laisk inimene. Esimesel tööpäeval võtsin nägin pool päeva vaeva ja
kirjutasin skripti. Panin kuhugi tekstifaili valmis nimed. Programm võttis
sealt järjest nimesid ja saatis neile ülesande ja pidas arvestust, et kellele
on juba saadetud, et kellelegi topelt ei saaks. Ja kui ma selle skripti käima
panin, siis rubiin.physic.ut.ee\index{Masinad!rubiin.physic.ut.ee}, tollane
füüsikamaja Unixi server, kõristas umbes pool tundi. Pärastpoole ma õppisin
\verb|nice| käsu\sidenote{Võimaldab Unixi keskkonnas kontrollida, kas kogu
programm kasutab ära kogu saadaoleva arvutusressursi või jätab midagi ka
teistele arvutikasutajatele} ka ära. Aga see tähendas, et kogu minu edasine töö
pärast selle skripti kirjutamist oli copy-paste meili seest sinna sisendfaili
ja skript tööle lükata. Automatiseerisin oma töö lihtsalt ära.

\textbf{\enquote{Aga kuidas sa sinna sattusid?}}

Ma arvan, et Indrek Jentson\index[ppl]{Jentson, Indrek} Täppisteaduste koolist
kutsus mind. Indrek oli matemaatikateaduskonnas vanem tegelane ja
olümpiaadidega tegelenud. Ma läksin Täppisteaduste Kooli ukse taha, tuli Viire
Sepp\index[ppl]{Sepp, Viire} vastu, kes juhataja oli, ütlesin, et tere, tulin
töölepingut tegema. \enquote{Mis töölepingut?}, küsis tema. Ma siis seletasin,
et Indrek Jentson saatis mind siia postmasteri töölepingut tegema. Kuskil 95.
või 96. aasta algul, täpselt ei mäleta.

\textbf{\enquote{See oli üsna vara ju? Tuleb häbiga tunnistada, ma läksin 93.
aastal tööle juba}}

Te olite Veljo Haguga\index[ppl]{Hagu, Veljo} Korelis\index{Korel IN}, eks? Ma
käisin Veljo töö juures vahel. Seal olid mingid mängud.
Dune'i\index{Mängud!Dune} mängis Veljo näiteks õhtul näiteks millalgi kui ma
sinna sattusin, vaatasin, kuidas see käib. Mängimisega ei olnud mul erilist
suhet. Ma sain keskkooli ajal oma mängimise isu täis mängida Sinclairi peal ja
lülituda juba programmeerimisele sellega, et ma tean, et see on palju põnevam
asi. Ma kirjutasin näiteks oma \emph{binary editor}i, millega mängudest
järgmiste levelite paroole välja nuuskida ja muid nihukesi asju. See oli juba
keskkoolis, et sai igasugustel arvutiturva teemadel nuusitud ja huvi tuntud.

Arvutiturva teema on mul keskkoolist saadik sees tõesti. Meil olid keskkoolis
väga põnevad võidujooksud arvutiõpetajaga. Väga harivad. Näiteks oli õpetaja
arvuti klaviatuur parooli all. Aegajalt tehti sellega meilivahetust, nii et
masinal klaviatuur oli lukus aga muidu masin töötas edasi. IBM
PS/2\index{Arvutid!IBM PS/2}\sidenote{PS/2 oli IBMi kolmas personaalarvutite
põlvkond, mida tutvustati 1987. aastal. Paljud tolle masina innovatsioonid nagu
näiteks VGA video muutusid \emph{de facto} standardiks pikkadeks
aastateks}tedel oli mingi selline klaviatuuriluku võimalus. Küll ma üritasin
leida meetodeid sellest mööda hiilimaks. Kui ma sain mingeid skeeme kuskilt
näha, siis mul tekkis idee, kuidas i8042 klaviatuurikontrolleri kaudu teha
masinale sobivat \emph{warm booti}, et sealt mööda hiilida, aga
klaviatuurikontroller oli lukus edasi. Kirusin, et IBMi omad on kavalad olnud.
See oli algul.

Lõpuks selle arvuti parool saadi teada lihtsal viisil. Vaadati üle selle
arvutiõpetaja õla, kes aeglasemalt tippis. Kui see oli teada saadud, ega me
sellega midagi ei teinud, see ei olnud eesmärk. Aga minul oli edasi põnevam
see, kui keskkoolis viimasel aastal oli 386d kohale jõudnud ja nende C: ketas,
kõvaketas, pandi kirjutuskaitse alla nii, et mingi spetsiaalne draiver laaditi
\verb|config.sys|-ist, mis tegi virtuaalse D: ketta ja keeras kogu C:
\emph{read-only}'ks. Ja ma avastasin selle niimoodi, et mul oli mingi enda
softi katsetamiseks see asi autoexec.bati või \verb|config.sys|i panna või
sealt midagi välja kommenteerida, et minu asi ära mahuks või täpselt ei mäleta,
miks. Igatahes oli mul vaja sinna sekkuda. Kui ma sekkutud sain, siis ma pärast
alati taastasin endise olukorra.

\textbf{\enquote{Ka tol ajal mingit võrgu häkkimist ei toimunud?}}

Anto Veldre\index[ppl]{Veldre, Anto} rääkis jah\sidenote{Meelis peab ilmselt
silmas varem eetrisse läinud memcpy episoodi Anto Veldrega}, kuidas tema poisid
ülikooli adminidel ruutusid käest ära võtsid\sidenote{Unixi-laadsetel
süsteemidel on root (mis eesti kõnekeeles mugandub tihtipeale sõnaks
\enquote{ruut}) ees süsteemi täielike õigustega peakasutaja. Seega tähendab
termin \enquote{ruutu võtma} arvutisüsteemi üle täieliku kontrolli saavutamist,
tihti algset peakasutajat virtuaalse ukse taha jättes}. Tema jagas oma poistele
modemeid ja terminale, mis tulid kuskilt humanitaarabina. Meil oli üks modem
õpetaja arvuti küljes. Ühel poisil oli oma modem korra koolis kaasas, mida ta
näitas, aga me ei osanud nendega midagi teha ja kohalikku võrku meil ei olnud.
LAN\sidenote{\emph{LAN - Local Area Network}, kohtvõrk} tekkis meile alles 12.
klassi kevadel, kui ma enam väga ei tegelenud sellega. OK, ma häkkisin
LANtasticu\sidenote{LANtastic oli \emph{peer-to-peer} LANi operatsioonisüsteem,
mida arendas Artisoft ja mis jäi hiljem Novelli ja Microsofti toodete varju}
lahti \emph{social engineering}u meetodil. Sügisel pärast minu ära minekut oli
kellelgi vaja saada LANtasticule juurdepääsu. Servermasinas oli nihuke koht
nagu \emph{network control directory}. Seal olid andmebaasid binaarsena. Ja vot
minu programm oskas käia ja binaarselt andmebaasi modifitseerida ja tekitada
ühe administraatori juurde või panna kellelegi õigusi juurde või midagi. Ehk
siis tuli meelitada noorem arvutiõpetaja flopi pealt ühte programmi käivitama
seal masinas, viisakalt tänada ja puha. Tema poolt oli ka kõik OK.

Aga varem oli see C:-ketta kirjutuskaitse. Algul me käisime Nortoni \emph{Disk
Editor}iga kuskil seal \verb|config.sys| algust ära sodimas, et seda ei
loetaks. Järgmisel tarkvara versioonil oli see koht paremini kaitstud ja siis
oli vaja ikka flopi pealt bootida. Aga BIOS oli parooli all. A: ja C: vs C: ja
A:. Noh, siis järelikult muugime BIOSi paroolid lahti. Need on obfuskeeritud
kujul kirjutatud kuhugi CMOS-mälusse ja masina ROM oli välja loetav. Ma võtsin
ja disassembleerisin selle Sourcereri-nimelise disassembleriga ja matemaatika
tunni ajal kirjutasin omale matemaatika vihikusse kõrvallehe peale programmi,
mis seda obfuskeeritud asja lahti võtab. Järgmine tund oli ajaloo tund. Läksin
ajaloo tunnist ära arvutiklassi, realiseerisin selle programmi ära ja muukisin
BIOSi paroolid lahti. Mul tuli suur pahandus, sest see oli ajaloo tund, kust
väga paljud olid puudunud, õpetaja oli väga kuri ja keeras käkki. Mul oli
pärast vaja see tund järgi teha ja õnnestus ikkagi. Põhjendasime ikka, kui väga
hea programmi me tegime, spetsifitseerimata, mis see oli. Et väga hea idee oli
ja tuli lihtsalt minna arvutiklassi ja kohe ära teha. Parool oli obfuskeeritud
jadašifrina või baithaaval võibolla isegi, et otsast proovides järjest
tähthaaval sai selle ära arvata. Ma kunagi arvutiõpetajalt küsisin, et miks
teil nii imelik parool on. Siis ta lahendas selle turvaprobleemi niimoodi, et
delegeeris osa vastutust arvutiklassi haldamises ja võttis appi arvutiklassi
haldama. Väga hea pedagoogiline meetod, töötas. Ei häkitud enam, ei olnud enam
huvi edasi jagada paroole, mida ma kätte saan.

\textbf{\enquote{Aga kust sul see krüpto huvi?}}

Seda läks sealsamas kandis ka vaja. Näiteks meie õpetaja ässitas Norton
Diskreet'i\sidenote{Diskreet oli tarkvarapaketi Norton Utilities 6.0 osa ning
sisaldas paljuski kurikuulsat (Kevin Mitnicku\index[ppl]{Mitnick, Kevin}
andmetel kasutati väidetud 56 biti asemel 30-bitist võtit, ka teised uurijad on
osundanud mitmetele olulistele nõrkustele) DESi implementatsioonis}
DESi\sidenote{\emph{DES - Data Encryption Standard} on sümmeetriline algoritm
andmete krüpteerimiseks. Algoritm on oma väikese võtmeruumi tõttu tänapäeval
kasutamiseks sobimatu (murti avalikult jaanuaris 1999), kuid oli siiski alates
1977. aastast USA föderaalse andmetöötlusstandardi (FIPS) osa.} kallale. DESist
ma ei saanud jagu, ma ei saanud DESist arugi tol hetkel. Aga tema suunas. Ta
oli üldse sedasorti kaval mees, et kui ta näiteks kuulis kunagi, kui meil
pinginaaber Veljo Haguga\index[ppl]{Hagu, Veljo} oli plaan kirjutada viirus,
siis ta suutis meid sellest eemal hoida. Me olime mingeid olemasolevaid viirusi
disassembleerinud ja vaadanud, kuidas need käivad. Õpetaja sattus pealt kuulma,
kui me rääkisime viiruse tegemisest ja ütles, et kui teha, siis teha kohe
selline \enquote{stealth}-viirus. Me olime väga nõus, aga seda me ei viitsinud
teha, ja nii jäi viirus tegemata.

Ta leidis meile muidu ka rakendust. Keskkoolis üldine taustaülesanne oli midagi
arvutada. Minu arvutusülesanne oli arvutada arvu $e$ kahe tuhande komakohaga 30
sekundiga 10 MHz 286 peal. Üks klassivend arvutas $\pi$-d tuhande komakohaga 60
sekundiga, sest see koondus aeglasemalt. Ja kust tulid ajapiirangud? Õpetaja
oli vaadanud, kui kiiresti temal vastus tuleb selle arvuti peal. Ma sain
35-sekundilise programmiga juba viie kätte, sest vastus oli õigem kui õpetajal.
Kuna need erinesid, siis ta võttis targa raamatu ja siis selgus, et minul oli
õige. Mul oli selleks hetkeks 21 sekundiline programm, mis käigu pealt
suurendas mingi hetk arvutüübi pikkust. Algul tegi lühema tüübiga ja hiljem
pikemaga, et kiiremini saaks. Aga see oli veel bugine ja ei töötanud õigesti.
Ma kontrollisin oma enda programmi vastu. Ma olin minut aega töötava
programmiga algul tulemuse välja arvutanud tulemuse faili kirjutanud. Siis oli
mul ka näiteks variant programmist, mis küsis, et kui mitme sekundiga oli vaja
arvutada ja siis ütles \emph{hard-coded} vastuse. Aga see ei sobinud õpetajale.
Aga 35-sekundiline juba sobis, kui vastus oli õigem tema oma. Minu
21-sekundiline ei läinud tööle, aga õpetaja seepeale võttis ja kirjutas ise
asja haljas assembleris\index{Keeled!Assembler} ja sai kolme sekundiga. Muidu
me kirjutasime Pascalis\index{Keeled!Pascal}.

Teine asi, mida me tegime, millega oli keskkooli ajal hulga nuputamist, oli
interferentsi simuleerimine arvuti ekraanil. On kaks punktlaineallikat
ringlainetega ja tuleb arvutada, kuidas lained liituvad, et tekiks
interferentspilt. Seal ma nägin ka vaeva, arvutasin ruutjuurt
assembleris\index{Keeled!Assembler} Newtoni meetodil. Ma arvutasin iga ekraani
punkti kohta pimesi selle faasi välja nii, et ühtegi punkti samal ajal näha ei
olnud, aga ma sättisin pikslite väärtus siis, kui palett oli seatud üleni mustaks.
Arvutasin kõik väärtused ära assembleris optimeeritud arvutusvalemiga ja õpetaja
õpetas Newtoni meetodit sinna juurde. Oli abiks. Assembleris sai Newtoni meetodit
tehtud! Oli väga hariv.

Ja lõpuks ma siis ketrasin VGA paletti. Tehnilise dokumentatsiooni failid
liikusid. Seal oli kirjas, kuidas VGA paletti muuta ja ma seadsin siis paleti
niimoodi, et need värvid, mis mul on, liikusid sujuvalt heleduse järgi. Ja siis
tulemus oli see, nagu oleks liikunud lained ekraanil. Ja see oli minu meelest
tippsaavutus, see oli väga ilus sujuv liikumine selle kümne megahertsi juures,
punkte üle arvutada poleks kuidagi jõudnud. Pärast viis õpetaja mind ühe teise
õpetaja tehtud programmi vaatama. See teine õpetaja ütles, et tema ideest see
alguse saigi, et interferentsi simuleerida. Tema tegi Juku peal \verb|circle|
käsuga valgeid rõngaid üksteise ümber viiemillimeetrise vahega. Need läksid
mida edasi seda aeglasemaks ja minu reaalajalise sujuva pildi vastu ei olnud
see midagi. Mul oli tükk tegu, et mitte naerma hakata. Aga kiitsin siis takka.
Meie õpetaja suutis anda sellise ülesande, mille peale mul kulus ikka kaua ja
sain palju targemaks. Õpetaja oli Tarmo Ainsaar\index[ppl]{Ainsaar, Tarmo}.
Seesama, kes suunas meid viiruse kirjutamiselt ära ja kes lahendas selle BIOSi
paroolide haldusteema probleemi meiega nii, et probleemi ei tulnud. Väga hea
õpetaja. Ta suutis meid suunata tegema õigeid asju nii, et me seejuures õpime
ja paha peale ei lähe.

\textbf{\enquote{Kuidas sa Cyberisse sattusid?}}

Ma töötasin HClubis\index{HClub} (sattusin HClubi tööle seoses sellega, et ma
installisin sinna Linuxi serveri, \emph{gateway} veebi ja meili jaoks) ja
mõtlesin, et mida võiks magistriks teha. Seal tegeldi hajusate andmebaasidega.
Me saatsime SQL-käsk-haaval andmebaaside \emph{diff}e üle võrgu mitmes suunas.
See oli põnev, me saime selle tehniliselt lahendatud. Algul käis see mul üle
UUCP, hiljem üle PPP ja POP3 ja SMTP. Mina ehitasin Internetti sinna alla, oli
ka põnev. Ja neid \emph{diff}e siis saatsime ja tekkis küsimus andmebaaside
konsistentsusest: mis tingimustel jääb ja mis tingimustel ei jää andmestik
konsistentseks. Et kas me saame mingi \emph{eventually consistent} mudeli sealt
või mitte. Ma mõtlesin, et ma hakkan sel teemal magistrit tegema.

Aga HClubis Interneti teemal, mis mind huvitas tol hetkel, ei olnud mul eriti
kuhugi areneda. Seal ei olnud kellegi teise käest sedasorti asju õppida. Kui
siis, ise õppida ja ehitada. Asju, mida oleks võinud pisi ISP-na veel teha ---
ehitada sinna ISDN sissehelistamiskeskus, kui oleks leitud raha ja et see
rentaabel on. Nihukesi asju oleks ehk saanud.

Samal ajal ma käisin mõnes koolis abiks Linuxit installimas ja käisin laenamas
RedHati installplaati 1997. a. suvel Elmer Joandi\index[ppl]{Joandi, Elmer}
käest Tartu lähedalt maalt. Tal oli see plaadina kohe olemas ja ei pidanud
flopidega mässama. Elmer ütles, et Tarvil\index[ppl]{Martens, Tarvi} olla plaan
Tartusse meiesuguste jaoks pesa teha. Ja siis mina käisin umbes juunikuus
Tallinnas Cyberneticas\index{Cybernetica} Helger Lipmaa \index[ppl]{Lipmaa,
Helger} juures, et tuleks magistrit tegema hoopis krüpto teemal. Ma mõtlesin,
et näiteks pordiks OpenSSLi\sidenote{Teek arvutite omavaheliseks turvaliseks
suhtluseks krüptograafia abil. Tuntuim ja levinuim omataoline} Windowsile, sest
mul oli Windowsi all krüptot vaja olnud aga seda polnud kuskilt võtta. Sellest
konkreetsest ideest arvati kehvasti, et keegi vist juba on portinud ka midagi.
Aga Tarvi kutsus niisama progema, mitte-krüptot. Oleks peaaegu
Küberisse tulemata jäänud. Helger kutsus mu ikka turvaasju tegema. Ja siis
kutsuti mind 1997. aasta sügisel Arula motelli Küberi\sidenote{Kõnekeeles on
\enquote{Küber}, \enquote{Küberneetika AS}, \enquote{Küberneetika Instituut} ja
\enquote{AS Cybernetica} sisuliselt sünonüümid. Aastal 1960 asutati Eesti
Teaduste Akadeemia Küberneetika Instituut. 1997. aastal reorganiseeriti see
Küberneetika AS-iks ja hiljem nimetati ümber Cybernetica AS-iks. Aga sisu jäi
suuresti samaks} väljasõiduistungile ehk \enquote{kvartalnajale}\sidenote{Nii
kutsutakse Cybernetica töötajate regulaarseid (algselt kvartaalseid, sellest
nimi) ja legendaarsed meeskonnaüritusi.}. Seal oli kutsutud kogu tulevane
Küberi Tartu Andmeturbelabor\index{Cybernetica!Andmeturbelabor}. Viljar
Tulit\index[ppl]{Tulit, Viljar} seal oma habemese diktsiooniga ütles, et seda
sa pead ikka ise suutma ära otsustada millegi järgi, kas sa tahad siia tulla
või ei taha, kui ma ütlesin, et segane on veel, kas ma tulen või ei tule.

Aga läksin ma Küberisse tarkade inimeste juurde. Seal olid Arne
Ansper\index[ppl]{Ansper, Arne} ja Viljar Tulit, kes oli kogenud süsadmin
(kogenum kui mina). Kui mina tegin näiteks tükk aega FTP otsingumootorit
Nuuskur\index{Nuuskur} koos teiste tudengitega, siis Arne oli selle stiilis
nädala otsa õhtutega ära teinud. Arnel oli Vosa\index{Vosa} nimeline FTP
otsingumootor Eesti FTP serverite kohta. Vosa nagu \enquote{Vanaisa Oli Sulle
Archie\sidenote{Archie oli üks esimesi Interneti otsingumootoreid, mis
võimaldas otsingut üle FTP arhiivide}}. Tal oli ainult veebiliides, meil oli
muid liideseid ka. Meil oli telneti liides ja archie prospero
protokolli\sidenote{Archie kataloogides navigeerimiseks loodud protokoll, mida
võib pidada tänapäevase www protokolli eellaseks. Prosperot kasutades võis
terve Internet välja näha nagu üks suur ühine kataloogipuu} liides, millega
vana archie klient töötaks, ja meililiides ka. Meil oli võimas vinge süsteem
tehtud kamba peale. Kõike ei teinud mina, teised tegid ka. Ma olin lihtsalt üks
vedajaid ning lõpuks see, kes tegi kõige rohkem tükke. Ja sellega selgus, et
Arne on tark. Seal oli veel asju, millega see selgus. Näiteks tal oli Fido ja
Interneti vaheline gateway. Ma olin selle kaudu
Fido\index{FidoNet}\sidenote{FidoNet oli ülemaailmne arvutivõrk, mida kasutati
BBSide omavaheliseks suhtluseks} lugeja. Ma pole päris Fidonetti kunagi
näinudki. Minu jaoks Fido oli lihtsalt järjekordne
NNTP\sidenote{\emph{Network News Transfer Protocol (NNTP). Useneti}
uudiste vahetamiseks kasutatud protokoll} server stiilis
keeks.ioc.ee\index{Masinad!keeks.ioc.ee}. Sinna tuli kasutajanime ja parooliga
läheneda ja sai tavalise \emph{newsreader}iga lugeda ja kirjutada. Minu jaoks
oli Fido teenus üle Interneti, mida vahendas Arne tehtud süsteem.

\textbf{\enquote{Mis sa praegu teed?}}

Praegu ma olen Küberis\index{Cybernetica} turvainsener. Praktikas ka
tarneinsener, kes pakendab lahendusi ja ehitab nende jaoks automatiseeritult
mingeid keskkondi.  Õpetan ülikoolis\sidenote{Tartu Ülikool}, olen ülikoolis
hajussüsteemide külalislektor, õpetan operatsioonisüsteeme baaskursusena,
andmeturvet baaskursusena ja magistrantidele õpetan turvalist programmeerimist.
See viimane tegeleb sellega, kuidas teha nii, et koodis poleks auke. Mõni auk
ikka kuskil leidub aga eks neid ole aja jooksul endale piisavalt vastu tulnud.
Andmeturbe kursus sai tehtud siis, kui ma olin alustav doktorant Helger
Lipmaa\index[ppl]{Lipmaa, Helger} juhendamisel. Helger ütles, et kuule, et sa
võiks ülikooli sellise andmeturbe kursuse teha. Mõeldud tehtud, tegingi.
Kellegagi eriti nõu ei pidanud. Küberi turvaraamatu\sidenote{Hanson, V., Lipmaa
H., Buldas A., Ansper A., Tulit V., Martens T. \enquote{Infosüsteemide turve 1.
osa}, 1997. Hanson, V., Lipmaa H., Buldas A., Ansper A., Tulit V., Martens T.
\enquote{Infosüsteemide turve 2. osa}, 1998.} võtsin kondikava jaoks aluseks,
vist valdavalt esimese köite.

\textbf{\enquote{See tundub olevat nii sinu moodi. Võtad, teed ja saab väga
hea?}}

Parim kiitus, mis ma andmeturbe ainele kuulnud olen, oli siis, kui hakati
küberkaitse magistrikava tegema. Tallinnas oli sel teemal koosolek. Ja oli
häda, et kui me tahame neile kõike seda õpetada, mida tahaks, ei mahu see meil
ainetesse ära. Selle peale oli vist Enn Tõugu\index[ppl]{Tõugu, Enn}, kes
ütles, et \enquote{Kuidas, Meelis jõuab andmeturbe kursuses neist kõigist
asjadest rääkida, mahutame ikka magistrikavasse ka ära}. Mis sest, et
põhjalikumalt, aga küll me mahutame. See oli hea kompliment kursusele, et
Meelis räägib sellest kõigest.

