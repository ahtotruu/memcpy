%!TEX TS-program = arara
% arara: myindex

\index[ppl]{Roos, Meelis}
\textbf{\enquote{Kuidas sina arvutite juurde said?}}

Mina sattusin arvutite juurde isa töö juures kaheksakümnendate lõpus. Füüsikud ostsid omale mõned arvutid elektromeetria laborisse, eksperimendi juhtimiseks. Arvutid olid CAMAC\sidenote{\emph{Computer-Aided Measurement And Control (CAMAC)}. Elektroonikastandard andmete kogumiseks ja seadmete kontrolliks. Kasutusel (osakeste) füüsikas aga ka tööstuses} kontrolleriga vene DVKd\index{Arvutid!DVK}\sidenote{\begin{russian}ДВК, Диалоговый вычислительный комплекс\end{russian}. Nõukogude personaalarvuti, ühilduv DECi PDP-11\index{PDP-11} perekonnaga. Varasemad mudelid on tuntud ka kui  Elektronika MS-0501\index{Arvutid!Elektronika} ja Elektronika MS-0502}.

\textbf{\enquote{Kus see kõik sündis?}}

See juhtus Tartus\index{Tartu}, Tartu Ülikooli\index{Tartu Ülikool} juures. Isa oli Tartu Ülikooli füüsika institituudis\index{Tartu Ülikool!Füüsika instituut} füüsik. Nad tegelesid elektroonika mõõteseadmete välja töötamisega ja näiteks said mingisuguse auhinna elektromeetri eest, mis eriti väikesi laenguid registreeris. Näiteks visati pastaka kuul, millel oli mingi laeng kusagilt läbi ja mõõdeti selle laeng mööda minnes ära. Neil oli lahe töögrupp, elektromeetria sektor, mida vedas üks mees, kes sellesse ilmselt uskus. Noored ülikoolist tulnud mehed tegid koos lahedaid asju, minu arusaamise järgi. 

Et katseid juhtida ja mingeid andmeid töödelda oli spetsiaalne lisablokk, mis käis arvuti külge. Seal sees oli analoog-digitaaal muundur (võibolla vastupidi ka aga igatahes nii pidi neid kasutati). Nad õppisid programmeerida, et suuta oma eksperimendi andmeid reaalajas kätte saada. 

\textbf{\enquote{Aga mis arvuti see selline oli, mis suutis andmeid niimoodi reaalajas kätte saada?}}

DVK-2M. Vene LSI-11\sidenote{DECi PDP-11 perekonna liige, tuntud ka kui PDP-11/03. Masinat tutvustati 1975. aastal ja ta oli oma sarjas esimene, mille CPU oli integreeritud. Mitte küll ühele, vaid neljale Western Digitali poolt toodetud \emph{Large Scale Integraton (LSI)} kiibile). Meelise sõnul: \enquote{PDP-11 oli legendaarne DECi masin iidsel ajal enne meie aega}} analoogid. Peaaegu täpne kloon aga natuke kohapeal ka täiendatud. Programmide poolt ühilduv aga mitte identne. DVK peal jooksis näiteks DECi originaal opsüsteem RT-11\index{RT-11}. RT-11SJ oli igapäevane opsüsteem, see oli \emph{single job} ja RT-11FB sellel oli \emph{foreground} ja \emph{background}, millega sai taustal jooksutada mingisugust teist tegevust. 

\textbf{\enquote{Kui vana sa olid, kui su isa need arvutid omale hankis?}}

Põhikooli teises pooles. Ega mul ei olnud põhjalikku teadmist, mida selle arvutiga teha saab. Minu jaoks sai arvutiga teha kahte asja. Kõigepealt, kui ma tegin isale teksisisestustööd, näiteks sugupuu andmete sisestamiseks, siis sain ma pärast seda kuni õhtuni mängida. Lemmik mäng oli \emph{wall}\index{Mängud!Wall}, seina pommitamine mingi reketiga. Mind pandi kohe tööle, et miskit kasu oleks. Mis ma niisama aega raiskan. Programmeerima õpetati ka, eks nad ise ka õppisid. Basicus\index{Keeled!Basic} ja Fortranis\index{Keeled!Fortran} ja CASICus\index{Keeled!CASIC}. See viimane oli CAMACi kontrollerite programmeerimiseks mõeldud Basicu ja Pascali\index{Keeled!Pascal} vaheline keel\sidenote{Ilmselt peetakse silmas keelt formaalse nimega \emph{ANSI Standard Real-Time BASIC}, mille spetsifitseerib IEEE standard \enquote{726-1982 - IEEE Standard Real-Time BASIC for CAMAC}}. Selles mina ei sattunud programmeerima. Aga ma õppisin Basicus programmeerima. Minu parim programm oli programm, mis ajas inimesega eesti keeles juttu. Ütleb sulle ühe lause, sina ütled lause, tema ütleb lause, ütled lause ja tema valib juhuslikult ühe lause. Aga ta suutis mõnikord teemas püsida. Näiteks ütleb \enquote{Osta elevant ära} ja siis järgmised kaks lauset olid, et \enquote{Kõik ütlevad nii aga osta elevant ära}. Enne ta ei läinud järgmist lauset valima kui ta oli kaks vastust saanud. Ja teiste töötajate lapsed mängisid seda ja neil oli lõbus. See oli lahe emotsioon, et ma tegin midagi, mis teistele lahe oli. 

\textbf{\enquote{Huvitav, et sa kohe hakkasid mängu tegema ja seejures kohe midagi AI-sarnast}}

See tundus kõige lahedam asi mida teha! 

\textbf{\enquote{Need füüsikud pidid ju kähku õppima, sest reaalajas riistvarast andmeid lugeda on ju keeruline}}

Neid oli vähemasti kolm meest, kes õppisid programmeerimist ja neil oli üks natuke noorem pundis, kes oli nende põhiline arvuti-mees ja kes seda vist paremini jagas, kui teised ja kelle juures CAMAC kontroller oli. Arvuteid oli vist vähemasti kolm tükki selle labori peale aga üks oli see põhiline eksperimendi juhtimise oma. See, mida mina kasutasin oli niisama masinakirjutaja toas. Seda sai kasutada siis näiteks programmide sisestamiseks ja muidu andmetöötluse jaoks. Näiteks isa tegi sugupuu üles joonistamist arvutisse. Oli puukujuline puu, puu läks vasakult paremale ja siis sai rull-paberile välja trükkida ja siis oli pärast mitmeid rulle. Kui koolis tulid mingid tudengid ja andsid igaühele paberi, kuhu oli natuke templeiti ette tehtud isa ema ja lapse kohta, et joonistage oma sugupuu üles, siis mina palusin isal ühe koopia välja trükkida. 

\textbf{\enquote{Aga miks sa lasid ennast sellesse suhteliselt igava andme-sisestaja rolli suruda? Lihtalt, et saaks mängida?}}

Algul selleks, et sai mängida aga kui selgus, et ise programmeerida saab ka ja see on täitsa lahe, siis ma pigem mängimise asemel keskendusin sellele rohkem. Ma ikka mängisin ka vahel, ma ei jätnud mängimist päris maha. 

\textbf{\enquote{Mis selle programmeerimise juures lahe oli? Mis sind köitis?}}

Ma tegin ka mingeid asju käsitsi. Näiteks ESC koodidega printerile õigeid asju saates\sidenote{\emph{Epson Standard Code for Printers, ESC/P\index{ESC/P}} on Epsoni poolt maatriksprinterite jaoks välja töötatud (ja termoprinteritel siiani kasutusel olev) keel, mis võimaldab juhtida rastrivõimekuseta printerit. Keel sai oma nime sellest, et tema käsud algasid sümboliga ESC (ASCII 27). Näiteks ESC E lülitas sisse ja ESC F välja rasvase trüki} trükkisin oma õpiku silte, kus oli boldis ja suuremas ja väiksemas kirjas kõik erinevatel ridadel asjad kirjas. Ja siis üks ema tuttav tahtis oma firma logo visiitkaartidele. See firma logo tuli siis teisendada EPSONi printeri graafika ESC-jadadeks. Ma joonistasin selle \emph{bitmap}ina üles aga siis leidsime, et ei tasu vaeva ja seda logo ma ei teinud. Aga programselt oleks selliseid asju lihtsam teha. See oli näiteks koht, kus ma leidsin, et programmist võiks oluliselt kasu olla. 

Oli firma nimega Tensiid, mille logol oli mingi kolmeharjuline neljast aatomist koosnev molekul visualiseeritud. Keskel üks lömmis ja kolm tükki külgede pealt sees. Nad tegelesid keemiliste mingisuguste ühendite sünteesiga. Ülikooli keemiahoonest välja kasvanud firma minu arusaamist mööda. Muu hulgas käisid aegajalt reisidel. Sellest kasvas välja Tensi Reisi, keemia asemel tegeletigi reiside korraldamisega. 

Aga mina üritasin alguses Tensiidi logo arvutis joonistada ja mõtlesin, et programm võiks seda minu eest teha. 

\textbf{\enquote{Kuidas õppimine käis?}}

Ma sain mingisuguseid venekeelseid raamatuid. Osalt raamatukogust isa tõi, osalt oli ehk mõni raamat tal töö juures olemas.  Need olid enamasti kusagilt laenatud ajutiselt. Näiteks mul oli segadus ASCII koodi ja \emph{escape} koodidega. \emph{Escape} koode tuli terminalile saata ja printerile sai saata ja. Siis ma mäletan, et küsisin isalt nõu, et mis neil vahet on et kas see on seesama asi. Ja siis oli raamatuid erinevaid. Näiteks oli üks raamat Basicu kohta. Mingisuguse käsu kohta on mul siia maani meeles kirjeldus, mis minu meelest ei sobinud niisugusse raamatusse \begin{russian}\enquote{эта команда работает хорошо}\end{russian}. See käsk töötab hästi. Minu meelest oli see lati liiga madalale laskmine. Minu meelest peaks kõik hästi töötama. Asjad tuleks nii teha. 

\textbf{\enquote{Sul oli ju siis päris korralik vene keele oskus?}}

Jah, ma olin üheksandas klassis umbes kui ma programmeerimist õppisin ja kannatas vene keelset raamatut lugeda küll. Meil oli põhikoolis selline vene keele õpetaja, kellega pidi õppima niiet mul tõenäoliselt oli üsna normaalne vene keele oskus selle vanuse kohta. Ma käisin Tartu 12. Keskkoolis\index{Koolid!Tartu 12. Keskkool}. Meil oli üks ukrainlanna, vana Zinaida Tovkatš vene keele õpetajaks kelle kohta meie kirusime et ta on väga range ja isegi haige ei ole kunagi, et muudkui peab õppima ja muidu ei pääse. 

\textbf{\enquote{Kas keegi sind õpetas ka või käis ainult raamatu järgi see asi?}}

Isa õpetas mulle neid asju, mida tema teadis ja õpetas blokk skeeme ja see keskis kuni keskkooli ajani välja, et kui mina tegin programmi ja see ei töötanud, siis oli kask viisi debugimiseks. Üks oli see, et ma trükin ta rullpaberil välja ja loen õhtul kodus. Teine võimalus on see, et ma joonistan selle asja blokkskeemiks ja lähen näitan isale. Sealt pealt tema oskas vigu leida küll. Ja blokkskeemiks joonistamisel leidsin ma tihti vead ise ka üles. Et blokkskeemid oli asi, mida isa mulle õpetas sest tema õppis programmeerimist nendega. 

Ja isegi kui ma Pascal keeles kirjutasin, mida isa ei osanud, ma sain ikkagi isalt abi blokkskeemide tasemel. Sest isal oli hea loogiline mõtlemina ja ta seletas mulle minu vead ära küll. Mis veel lõbus oli, ma süütasin kodus katelt. Minu ülesanne oli keskkütte katla alla tuli teha. Ja süütamiseks oli toodud vanapaberit füüsika osakonnast. Ja seal oli teinekord mingeid arvuti väljatrükke, mida ma lugesin. Panin paberi kõrvale ja ajalehed ja muud läksid katla süütamiseks. Näiteks ma leidsin Minsk 32\index{Arvutid!Minsk-32}-e\sidenote{Minsk-32 loodi kuuekümnendatel, nagu nimigi ütleb, Minskis. Tegu oli mitmest mudelist koosneva Minsk suurarvutite sarja kõige võimekama esindajaga. Oli laialdasel kasutusel, kuni asendati seitsmekümnendatel IBM 360 kloonidega} mingisugused crash dumpid või mälu dumpid kolmekümne kahe bitised. Ma olin üllatunud, et vau, minul on 16-bitised PCd (see oli tol hetkel hiljem vist kui ma juba PC taga olin) aga nende oli juba siis 32-bitine arvuti. Ja seal olid Fortran programmid, mida ma huviga lugesin. Isa kõrvalt ütleb, et ah, need ei ole suurt midagi väärt eriti et see mees, kelle programmid need on ei oska veel eriti programmeerida et tema Fortran programmide pealt pole eriti mõtet eeskuju võtta. Aga põnev oli neid lugeda sellegi poolest. Fortranit õppisin keldris katla kütmise juures!

\textbf{\enquote{Miski pani sind tulehakatust lugema, mis see oli?}}

Seal olid uued põnevad asjad!

\textbf{\enquote{Kas selle asja juurde mingi kirjanduse või muusika huvi ka käis?}}

Ulme huvi natuke oli. Mul õnnestus saada venekeelsed Asumi seeria raamatud, mida oli rohkem kui kaks esimest. Asumid mulle meeldisid ja ühe isa sõbra käest laenasime venekeelsed ülejäänud Asumi raamatud. Ja mul õnnestus vene keeles raamatut lugeda ja ma olin selle üle sügavalt üllatunud. Isa algul luges neid ise ja mina lugesin ka vist midagi neist. Niiet ulme huvi oli küll aga see ei olnud niimoodi väga sügavavalt. Seda oli valdavalt nii palju, kui kodus sattus Mirabili ja mida iganes seal ulmekaid olema. Need said kõik läbi loetud aga see ei olnud esialgu kuidagi eriti seotud arvutitega. Arvutid olid asi, mis tuli reaalsest maailmast. Näiteks sõitsin bussiga koju ja ükskord Pärmivabriku peatuse juures mööda sõites parajasti ema seletas mulle arvuti viiruste kohta mida ta oli lugenud kuskilt Horisondist või mõnest niisugusest kohast. Ja väga põnev oli. Parajasti sõitsime Pärmivabriku peatusest mööda, kui ma esimest korda arvutiviirustest kuulsin. Seda ma mäletan. 

\textbf{\enquote{Kas sa olümpiaadidel ka käisid?}}

Jaa, käisin. Neljandast klassist saadik käisin matemaatikaolümpiaadil. Seal nägi natukene tuttavaid. Oli naljakas korrelatsioon: need lapsed kellega ma olin koos käinud ülikooli töötajate lasteaias, neist nii mõnigi oli seal olümpiaadidel. Järgmine laine sellega oli minnes keskkooli. Miks ma läksin vanast koolist ära? Vanas koolis oli nii, et keskkoolis pidi tulema kaks klassi. Reaalkallakuga ja humanitaarkallakuga. Ja humanitaarkallakuga pidi see \enquote{A} ja eliitklass tulema kuhu paremad õpilased lähevad ja ülejäänud võiks minna sinna reaalkallakuga klassi. Ma leidsin, et see on lati alla laskmine, et ma tahaksin ikka paremat. Mind kutsuti Nõkku\index{Koolid!Nõo Keskkool}. Hilisem ülemus Cyberneticast\index{Cybernetica} toonane Nõo kooli direktor Uuno Puus\index[ppl]{Puus, Uuno} saatis laiali kõikidele olümpiaadikutele Nõo kooli kutseid. Sain ka. Kaalusin. Oli kaugel. Raske. Siis selgus, et esimene keskkool Tartus\index{Koolid!Tartu 1. Keskkool} on ka täitsa kõva. Helistasin kooli ja küsisin, et kas teil arvutiklass on. Direktori võttis vastu ja reklaamis, et neil on väga hea arvutiklass. Selle peale ma otsustasin, et ma lähen esimesse keskkooli. Viisin paberid esimesse keskkooli, kui sisse astusin 1990 oli juba Hugo Trefneri Gümnaasium\index{Koolid!Hugo Trefneri Gümnaasium|see{Tartu 1. Keskkool}}. Olid väga head arvutid. Oli arvutiklass ja Juku\index{Arvutid!Juku} klass. 

\textbf{\enquote{Sul oli siis selge arusaam, et sa just sinna kooli tahad minna?}}

Jah, ma läksin nimelt sinna. Selle kohta tegi ajaloo õpetaja meil kunagi pisikese kiire küsitluse üheksanda klassi kevadel. Et paljud teist siia jäävad ja paljud lähevad kuhugi mujale. Ja mina olin see, kes leidis, et ma tahan ise oma tulevikku kujundada, et mulle sobib see asi paremini. Ja siis ta küsis kolme tema nina all oleva tegelase käest. Esimeses pingis sattusin mina istuma ja minu tagant kahe tüdruku käest, kes olid ka kätt tõstnud, et lähevad mujale küsiti, mis nad teevad. Need oli täpselt need kolm, kes läksid esimesse Keskkooli. Niiet kõik, mis ta küsis, sai vastuseks, et lähme ära esimesse keskkooli. Nemad läksid teise paralleeli, sinna bioloogia-keemia harusse. Aga see tundus olevat sinnakanti, et umbes see vanus oli koht, kus mõned hakkasid ise mõtlema oma tulevikule ja planeerima ja mõned lasid isevoolu teed minna. Et mõned olid need, kes planeerisid. 

\textbf{\enquote{See oli see aeg, kui ühiskonnas hakkas juba muutus tulema, eks ole}}

Natuke oli juba varem selles mõttes, et koperatiivid olid varem ja asjadest tohtis rääkida varem. Selle sama üheksanda klassi jooksul ma jõudsin kaks korda kirjutada mingisuguseid referaate millest võib olla aasta varem oleks vanematel pahandus tulnud. Aga siis juba tohtis. Sellesama õpetaja kohta oli teada, et ta on üks paras punane. Aga temale ma need referaadid kirjutasin ja sain kiita, mis oli üllatav. Ma olin üllatunud, ma mõtlesin, et tuleb kaitsta kuidagi oma seisukohti. 

\textbf{\enquote{Kas sind keskkooli ajal tööle ei tõmmatud kuhugi?}}

Ainult natukene käisin. Tiražeerisin isa töö juures elektromeetrite trükkplaate. Joonistasin ahjulakiga ja risti ära lõigatud otsaga süstlaga rajad, söövitasin plaadi ära, tinatasin ära ja jootsin sinna peale kõik elemendid vastavalt skeemile. 

\textbf{\enquote{Aga see tahab ju käelist oskust ja elektroonikahuvi, kust sul see?}}

Seitsmeaastaselt oli mulle vist isa töö juures kolb kätte sattunud esimest korda kui ma suvalisi tükke kokku jootsin. Eks ma oskasin kolbi hoida ja elektroonika huvi mul oli. Aga elektroonikat ma ei osanud, ei ole kunagi ära õppinud analoogelektroonikat. Üldisi põhimõtteid tean aga ise midagi teha ei ole osanud. 

Digielektroonika oli seal kõrval. Kui keskkool hakkas läbi saama ja ülikooli oli vaja minna, siis mina olin neljandast klassist peale kindel olnud, et ma lähen füüsikat ja nimelt elektroonikat õppima. Aga siis mingid arvutid tulnud, kah põnev elektroonika värk. Aga arvuteid sai matemaatika poolt ka õppida. Mul oli kuhugi ilma eksamiteta sisse saamisesd, äkki matemaatikasse ja füüsikasse olümpiaadi tulemuste pärast või midagi ja ma otsustasin matemaatika kasuks sest füüsika osakonnas ma olin kogu aeg kohal ja mulle ei meeldinud see. Tundus nihukene, et kui midagi ära tahta teha siis peab ise muudkui tegema. Oli nihukesi saarekesi, kes tegelesid oma kitsa erialaga aga laiemat kandepinda ma ei märganud seal. Oli töögruppe, kes olid vingel tasemel ja tegelesid oma asjaga. Aga võibolla ma ei sattunud õigete inimestega kokku aga tundus, et pigem nihukene nagu oleks seisev konnatiik et igaüks on seal kinni, kus on ja nii on. 

No seal oli huvitavaid ja põnevaid asju ka. Näiteks olid füüsika päevad, kus rääkis Undo Uus\index[ppl]{Uus, Undo}, keda mu isa käis kuulamas, rääkis materialismi ümber lükkamisest filosoofiliselt. Isa tuli koju, jutustas. Panin kõrva taha. Selliseid asju oli sealt ikka päris mitmeid. Sellist füüsikalist maailmapilti tuli vanemate kõrvalt üksjagu, see oli mul olemas. 

\textbf{\enquote{Kuidas sa siis ikkagi matemaatikat sattusid õppima? Lihtsalt seepärast, et sai eksamiteta sisse?}}

Füüsikasse ma oleks vist ka saanud ilma eksamiteta. Eksamid ei oleks probleem ka olnud, ma arvan. Lihtsalt laisk. Laisad me olime kõik. Keskkoolis klassijuhatajal tuli üheksandas klassis üritada meile ikka auku pähe rääkida, et poisid olge tublid ja võtke tehke need eksamid ikka ära, siis saab medalile pretendeerida, muidu ei saa. Vaja oleks ju medaleid ka. Siis me tegime vist kolm medalit klassi peale või midagi. Mina sain hõbeda. Ma täpselt ei mäletanudki. Kunagi hiljem kooli koduleheküljelt lugesin, et ma hõbemedali sain. Seda ma mäletasin, et medal oli aga mis medal, seda ei mäletanud. Polnud oluline, see tuli iseenesest. 

\textbf{\enquote{Ühesõnaga, matemaatikasse sa läksid seepärast, et füüsika tundus natuke seisev vesi olevat?}}

Jah. Ja ma olin kuu aega enne paberite sisse andmist kindel, et matemaatikasse ma küll ei lähe. Me käisime Moskva\index{Moskva} lahtisel olümpiaadil matemaatikas koolist tiimiga. Seal olid mingid doktorandid, kes tegelesid meiega. Seal ühtlasi toimus \begin{russian}Международная конференция старшикласников "Наука, природа, человек"\end{russian} kus keskkooli õpilased said ise asju esitada, mis nad olid teinud. Keegi oli teinud kiiret vektorgraafikat, et voldime siin kuubikut kiiremini, kui AutoCAD või mis iganes wirefreimis. Ja ägedaid asju oli tehtud. Seal oli mingit Hollandi rahvast, oli rahvusvaheline küll. Seal need doktorandid, kes meiega tegelesid, olid nihukesed parajad uhuud. Näiteks tuleb tegelane hommikul tahvli ette triiksärk on lükatud alukate sisse, alukad ulatuvad kümme sentimeetrit pikkade pükste peale välja ja tuleb niimoodi tahvli ette. Ma leidsin, et vot matemaatikuks mina küll ei lähe. Aga siis ma mõtlesin ikkagi ümber. Matemaatikuks ma ei tahtnudki, ma läksin neid arvuteid õppima Matemaatikateaduskonna\index{Tartu Ülikool!Matemaatikateaduskond} poolt. Mitte eleketroonika poolt aga programmeerimise poolt. 

\textbf{\enquote{Kuidas sulle ülikooli üleminek tundus? Sa ütlesid, et olla laisk olnud aga minu mälu järgi pidi ülikoolis kohe hakkama tööd tegema?}}

Jaa. Keskkoolis ma sain endale lubada laisk olemist isegi seal eliitkoolis, no mingil tasemel vähemasti. Ja ma sain keskkoolis arvutimängude mängimise isu täis mängida. Ostsin omale üheksanda klassi lõpus ZX Spectrum-i\index{Arvutid!ZX Spectrum}\sidenote{ZX Spectrum oli Sinclair Research'i poolt 1982. aastal Ühendkuningriigi turule lastud 8-bitine personaalarvuti, mõeldud peamislt koduseks kasutamiseks. Selle kloone liikus Nõukogude Liidus hulganisti, skeemid olid kogunisti hobiajakirjades avaldatud} Leningradi turu klooni 1500 rubla eest kui rubla juba kukkus. Siis oli suur rahanumber aga ma sain mängida täis oma mängimise isu. Joystick sai peeneks mängitud ja plastmassi paigatud alumiiniumiga. Tuttav treial tegi sinna uue varre, pärast kippusid kontaktid läbi põhja tulema. Aga Spectrum oli nii hea arvuti, sellest sai aru igat pidi! Basicus sai programmeerida, Assembleris\index{Keeled!Assembler} sai programmeerida Z80 peal. Sellest sai täiesti aru saada. Ja elektroonikast võib peaaegu üleni aru saada, välja arvatud videopildi kildi genereerimise osa, see ULA kivi või selle realiseerimine niisama lause-elektroonikana nagu selles vene variandis oli kui seda kivi ei olnud kloonina võtta. Niiet ma sain sõbra Sinclairi diagnoosimisega hakkama, et sul on ROMi see ja see jalg lahti ja ei anna kontakti. Et sellest tulevad tähtedel vertikaalsed kriipsud läbi nagu dollarimärgid. Sest ROMis oli see tähtede tabel ja kui seal bitt oli maas, siis on vertikaal. Seal oli kaks ROMi kivi et sellel ROMi kivil see jalg peab järelikult mitte kontaktis olema. 

\textbf{\enquote{See tähendab, seda, et sa pidid neid asju põhjalikumalt uurima?}}

Skeeme ma ikka kuskilt raamatutest ja niimoodi nägin. Keskkooli lõpus, kui Venemaal käisin, ostsin metroost omale raamatu \begin{russian}Введение в схемотехники IBM PC / AT\end{russian}. Venelased olid 286 skeemid välja ajanud arvuti järgi ja üles joonistanud. Neil oli seal viga minu meelest. Mingi reset signaal, selles oli aktiivne null versus aktiivne üks kusagil vist segamini, mul on nihuke mälestus. See oli lõbus igatahes avastada niisugust asja trükitud raamatus. See oli seesama kord, kui me olümpiaadil käisime ja konverentsil, millest me enne ei teadnud midagi, kui me sinna kohale sattusime. Meil ei olnud mingeid ettekandeid, me kuulasime niisama, mis räägitakse. Ja vaatasime, mihukesed on kenamad tüdrukud. Üks vene Maša oli kõige kenam. 

Olümpiaadil me eriti hiilgavaid tulemusi keegi ei saanud. Mina sain meie pundist kõige parema tulemuse, sest ma ei joonud eelmisel õhtul alkoholi. Aga seda oli seal saada ja seda käis ringi ja siis järgmine hommik oli pohmakas ja siis inimesed ei esinenud oma võimete tasemel. Ja mina olin meie omadest parim kuigi seal oli meil vähemasti üks nendest meestest, kes veel oli kaasas, oli parema peaga, kui mina. Minu jaoks oli õppetund, mida rõõmsalt teistele edasi jagada, et näe, olümpiaadi tulemus sõltus selgelt sellest. 

\textbf{\enquote{Räägi palun ülikoolist. Me sattusime seal 1993. aastal kokku, kuidas sulle see matemaatika tundus, mida me kohe esimese semestri alguses saama hakkasime?}}

See oli üks suur kukkumine. Ma mõtlesin ülikooli tulles, et ma tean, mis on reaalarv näiteks. Siis tuleb Matemaatilise Analüüsi esimene loeng, kus hakatakse neid defineerima. Ja kõike pidi algusest hakkama defineerima, ainult nende definitsioonide otsa ehitati kogu seda kõike. See tahtis palju harjumist ja palju tööd, mina ei olnud harjunud tööd tegema. Mina mõtlesin, et ma oskan programmeerida, kui ma ülikooli tulin. Aga asi, mis mulle näitas, et on veel palju, oli Rein Pranki\index[ppl]{Prank, Rein} mat. loogika õppeprogrammid, kus tehti tõestuspuu layouti ja ma mõtlesin, et vau, puu layouti niimoodi teha ma ei oska. Me õppisime seda küll alles hiljem umbes kolmandal kursusel Varmo Vene\index[ppl]{Vene, Varmo} funktsionaalses programmeerimises, kus me mingi minimaxi ülesande tüübi näite ülesandeks tegime puu layouti. No seda oleks saanud rekursiooniga esimese kursuse järel ka ehk kuidagi tehtud saanud. Aga see oli jah näide sellest, et ei ole kõik ikka triviaalne jõuga peale ja teeme ära. 

\textbf{\enquote{Matemaatiline analüüs, eriti Matemaatiline Analüüs II, võttis meil kursuse peal palju rahvast hõredamaks, see tahtis harjumist saada}}

Ja algebra tahtis ka. Kogu see matemaatiline lähenemine, et me ehitame asju üles mingite definitsioonide ja aksioomide millegi otsa. Kogu see asi tahtis kõvasti tööd. Ja ma kukkusin esimesel kursusel haiglasse. Et sessi ajal ma ei jõudnud mõnesid eksameid tehtudki, tegin neid alles järgmise semestri sees. Käisin dekaanilt küsimas sessi pikendust, sest vanemad õpetasid, et nii tuleb teha. Siis dekaan ütles, et meie ajal enam niisugust asja pole, lihtsalt tehke need eksamid ära, kuidas saate. 

\textbf{\enquote{Mis hetkel oli võimalik minna arvutiteadust õppima?}} 

Selleks oli kas esimese aasta järel spetsialiseerumine. Mingid põhimoodulid oli vaja ära teha ja siis vist sai. Kuna ma sain need vist kokku, siis mina kaldusin üldisest õppekavast kõrvale sellega, et mina läksin võtsin koos aasta vanematega põnevaid arvuti aineid. Käisin aasta vanema rahvaga koos kuulamas asju, mis olid lahedad. Peast ei mäleta, aga igasugu aineid, mis meil seal oli. Ja siis järgmine aasta tuli mul võtta siis need asjad ka, mis õppekavast puudu olid. Minu oma kursus oli need ära teinud, mina tegin neid siis koos aasta noorematega. Mingeid tõenäosusteooriaid ja mingisuguseid matemaatika aineid. 

Juhtus ka seda, et ma kirjutasin maha kodutöö programme teiste pealt. Meil oli mingisugused algebra ja analüüsi numbrilised meetodid, kus me arvutusmeetoditega numbriliselt tegelesime. Ma sain algoritmidest aru, nad ei pakkunud mulle algoritmi tasemel pinget ja ma ei viitsinud neid teha. Kui ma olin aru saanud, mis seal tehakse, siis sellest piisas. Siis oli üks lahke kaastudeng Jane, kelle programme ma kasutasin selleks, et neid esitada. Muutsin vist natuke treppimist ja muutujad nimesid mõnes. Mäletan, ma kirjutasin ühele kommentaaridesse üles \enquote{Viimati modifitseerinud Meelis Roos}. Eks see praksi juhendaja, et neid üksteise pealt üksjagu maha võetakse ja kuna ta lasi endale ette seletada, mida see programm täpselt teeb ja algoritm, siis sellega polnud probleemi, ma sain kõik asjad ilusti tehtud. Kirjutasin programme tüdrukute pealt maha, sest ma ei viitsinud programmeerida. 

\textbf{\enquote{Kas see ülikooli arvutuskeskus seal Liivi tänaval ei neelanud sind kuidagi endasse, nagu ta nii mõnedki neelas?}}\index{Tartu Ülikool!Matemaatikateaduskond!Liivi tänava õppehoone} 

Neelas ka mind aga natuke teistel viisidel. Mina ei kadunud ära Muda\index{Mängud!Muda} mängima. Muda oli küll tore, aga siis kui ma kirjutasin oma telneti klienti, siis sai seda Muda serveri vastu testida näiteks. Selleks oli Muda tore. 

\textbf{\enquote{Miks sa kirjutasid oma telneti kliendi?}} 

Võrguprogrammeerimise harjutamiseks. Tahtsin osata sokliühendusi igasuguseid teha. Ma kirjutasin oma netcati laadset mingit asja, mis telneti handshake'i ei teinud mingisugust ja ei osanud echo offi ja selliseid advanced featuure vaid lihtsalt sokli kuhugi ühendas. Sellise asja kirjutasin endale, et torkida igasuguseid asju. Seal olid mingid mured stiilis kui pikkade pakettidega asju saata ja vastu võtta ja TCP võis selle suvalisel koha pealt ära hakkida. Ei saanud eeldada, et kui teiselt poolt rida sisse kirjutakse, et sa täpselt rea suuruste tükkidena kätte saad. See oli põnev.

Aga mind neelas see arvutuskeskus natuke teistmoodi. Teisel korrusel Ülo Kaasiku\index[ppl]{Kaasik, Ülo} kabineti kõrval oli magistrandide arvutiklass, kus olid värvilised Sunid. See oli ette nähtud magistrandidele aga kellelgi ei olnud eriti probleeme, kui mina ka sinna imbusin. Aegajalt seal ei olnud kohti ja tuli ette, et ma kellelegi kohta pidin loovutama mõnikord aga enamasti töötas. Aasta vanema Raul Tölbiga\index[ppl]{Tölp, Raul} istusime seal koos ja seal sai õpitud ära Unix. 

Ja lõbus oli omakorda see, kuidas ma üldse sinna Unixit kasutama sattusin. Seda ma võin lausa rääkida, kust on pärit minu kasutajanimi mroos. Minu esimene online konto oli masinas vask.ut.ee\index{Masinad!vask.ut.ee}. See oli VAX\index{Arvutid!VAX}\sidenote{Arvutisari, mille töötas DEC välja seitsmekümnendate keskel. Siiani üks kõige tuntumaid omalaadseid arhitektuure, oli ta PDP-11 edasiarendus, peamiselt mälu virtuaalse adresseerimise suunas. \emph{VAX - Virtual Address Extension}} tüüpi arvuti VMS\sidenote{VAX arvutite \enquote{kohalik} operatsioonisüsteem} opsüsteemiga. Nihuke umbes kuupmeetrine kast pluss kettad kõrval. Teine VAX oli rubiin.physic.ut.ee\index{Masinad!rubiin.ut.ee} füüsika majas. See oli MicroVAXm sahtlitumba suurune masin ainult. Vot need olid VMSid. Esimesel kursusel, selle asemel, et sessi ajal õppida, mina olin raamatukogust võtnud omale VAX VMSi raamatu ja õppisin VMSi. Seal oli huvitavaid asju! Näiteks olid struktuursed failid. Sa võisid tekitada tühja faili, millel on ette antud kirje struktuur. Opsüsteemi tasemel oli record management system, RMS, millega mingis keeles kirjeldati struktuur ära ja tekitati selle kirjelduse järgi fail. Fail võis olla ka tühi aga tal oli struktuur olemas. 

Õigusi oli seal jõle palju ja keeruliselt. Kogu see õiguste süsteem op süsteemis oli keeruline tokenite süsteem. Windows NT\index{Windows NT} on selle sisemiselt pärinud või umbes niimoodi. Nii keerukas ei ole minu meelest kui VMSis aga kui ma nägin Windows syscalli create process koos portsu argumentidega, siis tuli tuttav ette sest VMSi sys\$createprocess oli umbes samasuguse listi argumentidega. sys\$ käis syscallide funktsioonide nimede ette lihtsalt. 

Sealt ma käisin näiteks Lynxiga veebis surfamas, tõmbasin mingeid faile FTPga, mida ma sain kuskilt kolmandat teed mööda kuidagi flopi peale. Käisin internetis veel midagi lugemas. Ma eriti ei programmeerinud VMSis. Kui vaja oli kursaõele Pascalis programmeerimist õpetada aga ainult VAXu klass vaba oli, siis ma näitasin talle Pascalis programmeerida VAXu peal. Ta oli väga üllatunud, et saab seda arvutit ka programmeerida. Aga sai. Aga seal oli lahe programm nimega swim, mis lasi ühe terminali peale multipleksida mitu akent. Akende suurusi lausa sai muuta. Ja see oli lahe ja sellega ma kasutasin kolme rakendust korraga. Aga swim kippus ajama terminali hanguma, kõditas vist mingit VMSi terminali draiveri bugi või mida iganes ja siis tuli leida administraator, keda tihti majas ei olnud või siis keegi sõber tudeng logis kuhugi üle võrgu rubiini\index{Masinad!rubiin.physic.ut.ee} ja talk-is Ville Hallikuga\index[ppl]{Hallik, Ville}, kes oli sealne VMSi admin ja kellel oli juurdepääs vaske olema ja kes sai tulla ja terminali päästa sest selle terminali tagant ei saanud keegi enam midagi kasutada, terminal oli hangunud. Tappis swimi ja mingid asjad ära seal niiet terminal sai jälle vabaks. Ja swim oli tülikas. Ja siis keegi rääkis, et arvutiteaduse instituudi SUNides on Unixis programm nimega screen, millega seda sama teha saab. Ja siis tekkis mõte, et kasutaks seda. Ma olin Unixit seni juba korra kasutanud. Math.ut.ee-s\index{Masinad!math.ut.ee}, kui tekkis online võrk, tuli 386BSD. Ja see upgreiditi 93. aasta lõpus mingile uuele tundmatule opsüsteemile. Sinna osteti 486 arvuti asemele, suure kahekigase scsi vindiga ja selle scsi kaardi jaoks ei sobinud enam 386BSD vaid pandi asemele mingi uus tundmatu asi nimega Linux. Versioon 0.99PL midagi. 

\textbf{\enquote{Kust selline asi sattus Tartu linna?}} 

No aga kust 386BSD sai? Internet oli ju olemas. Ja kasutajad koliti 386BSDst Linuxisse siuhti üle ja mul oli mingis Linuxis kasutaja. Jaanuaris umbes apgreiditi see Linux ära versioonile 1.0.2, kerneli versioon. Mul oli siis Linuxis kasutaja. Ma olin natukene nuusutanud Linuxit. Kui ma tahtsin seal Liivi tänaval Unixi screeni, siis math.ut.ee ühendus oli aeglane, lagis päris kõvasti. 9600ne ühendus jagatud paljude kasutajate ja meilide ja muude vahel. Siis ma küsisin sinna cs1, cs2, cs3 olid masinad ühise loginiga. Küsisin omale cs3-e (hilisem romulus.cs.ut.ee) konto ja põhjendasin seda, et tahaksin näppida mõnda mitte-Linux Unixit. Seal oli Solaris. Ja see tundus Toomas Soomele\index[ppl]{Soome, Toomas} piisavalt hea põhjendus. Toomas Soome kasutajanimi oli tsoome, ma mõtlesin, et ahaa, et eks Unixis käib see niimoodi. Küsisin siis omale tema süsteemi sama skeemi järgi kasutajanimeks mroos. Antigi. Seda ma olen sellest ajast edaspidi kasutanud igal pool. Isegi kui mul on kodus testarvuti  siis seal olen ma ka harjumusest mroos. Et tsoome mulle kasutajanime teeks tuli öelda, et ma tahan Solarist kasutada ja kasutajanimi peaks ka samas formaadis olema, et võimalikult vähe küsimusi oleks. 

Mul möödunud aastal oli väga sürr kogemus, kui kevadel võttis minuga ühendust Toomas Soome, kellel oli siiamaani magister tegemata. Ta tahtis, et ma juhendaksin tema magistritööd. Ma mõtlesin, et muna õpetab kana, et mida mina siin teen. Aga tal oli korralik tehniline töö olemas ja mina teadsin, mismoodi üks magistritöö peab enam-vähem välja nägema. Sellest teadmisest oli kasu, niiet see töö sai tal vormistatud magistritööks ja ta tegi selle edukalt ära. Aga algul lihtsalt oli väga sürr reaktsioon. Arvutiteaduste Instituudis\index{Tartu Ülikool!Matemaatikateaduskond!Arvutiteaduste Instituut} oli terve hulk rahvast, kes tegid hiljem magistrit. 

\textbf{\enquote{Kas sind teadust ei tõmmanud tegema?}} 

Ei, vot teadust tegema ei ole mind kunagi eriti tõmmanud ja keegi ei suutnud mulle ka auku pähe rääkida sel teemal. Väga ei proovitud ka. Meelitati erinevate viisidega, mingeid materjale ette söötes. Materjalid olid nii teadusega kui mitte-teadusega seotud. Näiteks Jaanus Pöial\index[ppl]{Pöial, Jaanus} jagas mulle omal algatusel kunagi Java Language Specificationi, et näe üks uus moodne asi. Et selliseid asju ülikoolist ikka sattus. 

Ma mäletan, ma olin rebane. Ma ei olnud veel spetsialiseerunud Arvutiteaduse Instituuti informaatika erialale. Aga mul oli vaja kusagil välja trükkida viietollise flopi pealt mingit tekstifaili. Ma lihtsalt vajusin kohale Liivi tänavale ja käisin mööda uksi koputamas. Äkki oli mingi laupäev ka või midagi või muidu õhtune aeg et seal ei olnud palju rahvast. Sattusin Mati Tombaku\index[ppl]{Tombak, Mati} ukse taha, kes lahkelt lasi trükkida. See oli raamatukogust mingisuguse kataloogi otsingu tulemus välja trükitud mingi raamatute otsimiseks. Äkki oli laupäevasel päeval vaja välja trükkida. Ja siis Mati Tombak oli see, kes lasi mul trükkida. Ja sellest tekkis nihukene tänutunne kogu selle ATI vastu, et siin on lahked inimesed. See oli minul nihukene esimene sedasorti kontakt. 

\textbf{\enquote{Millal sa tööle läksid?}} 	

Minu esimene ametlik töökoht oli Tartu Ülikooli Täpisteaduste Koolis\index{Tartu Ülikool!Täpisteaduste Kool} metoodik. See oli postmasteri töö tegelikult. Aga postmasteri nimelist ametinimetust ei olnud, oli metoodik. Korraldati programmeerimis kursust e-mailitsi koolides. Ja mina olin see, kes pidas arvet selle üle, kellel olid mis ülesanded lahendatud ja saata neile järgmisi. Arvutiõpetajad, kellele vastused saadeti ja kes parandasid saatsid minule seisu ja mina siis selle järgi saatsin edasi. Mina olen laisk inimene. Mina esimesel tööpäeval võtsin nägin pool päeva vaeva ja kirjutasin skripti. Panin kuhugi tekstifaili valmis nimed. Programm võttis sealt järjest nimesid ja saatis neile ära ja pidas arvestust, et kellele on juba saadetud, et kellelegi topelt ei saaks. Ja kui ma selle skripti rubiin.physic.ut.ee tollane Füüsikamaja Unixi server kõristas umbes pool tundi. Pärastpoole ma õppisin nice käsu ka ära. Aga see tähendas, et kogu minu edasine töö pärast selle skripti kirjutamist oli copy-paste meili seest sinna sisendfaili ja skript tööle lükata. Automatiseerisin oma töö lihtsalt ära. 

\textbf{\enquote{Aga kuidas sa sinna sattusid?}}

Ma arvan, et Indrek Jentson\index[ppl]{Jentson, Indrek} Täpisteaduste koolist kutsus mind, kes mat teaduskonnas oli vanem tegelane ja olümpiaadidega tegelenud. Tema kutsus mind. Ma läksin Täpisteaduste Kooli ukse taha, tuli Viire Sepp\index[ppl]{Sepp, Viire} vastu, kes juhataja oli, ütlesin, et tere, tulin töö lepingut tegema. Mis töölepingut? Ma siis seletasin, et Indrek Jentson saatis postmateri töölepingut tegema mind siia. Kuskil 95 või 96 algul, täpselt ei mäleta. 

\textbf{\enquote{See oli üsna vara ju? Tuleb häbiga tunnistada, ma läksin 93. aastal tööle juba}}

Te olite Veljo Haguga\index[ppl]{Hagu, Veljo} Korelis\index{Korel IN}, eks? Ma käisin Veljo töö juures vahel. Seal olid mingid mängud. Dune'i\index{Mängud!Dune} mängis Veljo näiteks õhtul näiteks millalgi kui ma sinna sattusin, vaatasin, kuidas see käib. Mängimisega ei olnud mul erilist suhet. Ma sain keskkooli ajal oma mängimise isu täis mängida Sinclairi peal ja lülituda juba programmeerimisele juba sellega, et ma tean, et see on palju põnevam asi. Ma kirjutasin näiteks oma binary editori näiteks, millega mängudest järgmiste levelite paroole välja nuuskida ja muid nihukesi asju. See oli juba keskkoolis, et sai igasugustel arvutiturva teemadel nuusitud ja huvi tuntud. 

Arvutiturva teema on mul keskkoolist saadik sees tõesti. Meil olid keskkoolis väga põnevad võidujooksud arvutiõpetajaga. Väga harivad. Näiteks oli õpetaja arvuti klaviatuur parooli all. Aegajalt tehti sellega meilivahetust niiet masinal klaviatuur oli lukus aga muidu masin töötas edasi IBM PS/2\index{Arvutid!IBM PS/2}\sidenote{PS/2 oli IBMi kolmas personaalarvutite põlvkond, mida tutvustati 1987. aastal. Paljud tolle masina innovatsioonid nagu näiteks VGA video muutusid \emph{de facto} standardiks pikkadeks aastateks}tedel oli mingi selline keyboard locki feature. Küll ma üritasin leida meetodeid sellest mööda hiilimaks. Kui ma sain mingeid skeeme kuskilt näha, siis mul tekkis idee, kuidas i8042 klaviatuurikontrolleri kaudu teha masinale sobivat warm booti, et sealt mööda hiilida aga klaviatuuri kontroller oli lukus edasi. Kirusin, et IBMi omad on kavalad olnud. See oli algul. 

Lõpuks selle arvuti parool saadi teada lihtsal viisil. Vaadatai üle selle arvutiõpetaja õla, kes aeglasemalt tippis. Kui see oli teada saadud, ega me sellega midagi ei teinud, see ei olnud eesmärk. Aga minul oli edasi põnevam see, kui keskkoolis viimasel aastal oli 386d kohale jõudnud ja nende C ketas, kõvaketas pandi kirjutuskaitse alla nii, et mingi spetsiaalne draiver laaditi config.sys-ist, mis tegi virtuaalse D draivi ja keeras kogu C ready-only-ks. Ja ma avastasin selle niimoodi, et mul oli mingi enda softi katsetamiseks see asi autoexec.bati või config.sysi panna või sealt midagi välja kommenterida, et minu asi ära mahuks või täpselt ei mäleta mis. Igatahes oli mul vaja sinna sekkuda. Kui ma sekkutud sain, siis ma pärast taastasin endise olukorra alati. 

\textbf{\enquote{Ka tol ajal mingit võrgu häkkimist ei toimunud?}}

Anto Veldre\index[ppl]{Veldre, Anto} rääkis jah, kuidas tema poisid ülikooli adminidel ruutusid käest ära võtsid. Tema jagas oma poistele modemeid ja terminale mis tulid kuskilt humanitaarabina. Meil oli üks modem. Õpetaja arvuti modemit ei puutud ja ühel poisil oli oma modem korra koolis kaasas, mida ta näitas aga vot me ei osanud nendega midagi teha ja LANi meil ei olnud. LAN tekkis meile alles 12. klassi kevadel, kui ma enam väga ei tegelenud sellega. OK, ma häkksin selle Lantasticu lahti social engineeringu meetodil. Sügisel pärast minu ära minekut oli kellelgi vaja saada Lantasticule juurdepääsu ja oli server masinas oli nihuke koht nagu network control directory. Seal olid andmebaasid binaarsena. Ja vot minu programm oskas käia ja binaarselt andmebaasi modifitseerida ja tekitada ühe administraatori juurde või panna kellelegi õigusi juurde või midagi. Ehk siis tuli meelitada noorem arvutiõpetaja flopi pealt ühte programmi käivitama seal masinas, viisakalt tänada ja puha. Tema poolt oli ka kõik OK. 

Aga varem oli see C-ketta kirjutuskaitse. Algul me käisime Nortoni Disc Editoriga kuskil seal config.sys algust ära sodimas, et seda ei loetaks. Aga siis oli paremini kaitstud järgmisel softil ja ei saanud sellega ka ligi ja siis oli vaja ikka flopi pealt bootida. Aga BIOS oli parooli all. A ja C, C ja A. Noh, siis järelikult muugime BIOSi paroolid lahti. See on obfuskeeritud kujul kirjutatud kuhugi CMOS mälusse. Selle sai sealt obfuskeeritud kujul välja lugeda. Ja masina ROM oli välja loetav. Ma võtsin ja disassembleerisin selle sourcereri nimelise disassembleriga ja matemaatika tunni ajal kirjutasin omale matemaatika vihikusse kõrval lehe peale programmi, mis seda obfuskeeritud asja lahti võtab. Järgmine tund oli ajaloo tund. Läksin ajaloo tunnist ära arvutiklassi ja realiseerisin selle ära ja muukisin BIOSi paroolid lahti. Mul tuli suur pahandus, sest see oli ajaloo tund, kust väga paljud olid puudunud ja õpetaja oli väga kuri ja keeras käkki. Mul oli pärast vaja järgi teha ja õnnestus ikkagi. Põhjendasime ikka kui väga hea programmi me tegime spetsifitseerimata, mis see oli. Et väga hea idee oli ja tuli lihtsalt minna arvutiklassi ja kohe ära teha. Parool oli obfuskeeritud striimsšifrina või bait haaval võibolla isegi, et otsast proovides järjest täht haaval sai selle ära arvata. Ma kunagi arvutiõpetajalt küsisin, et miks teil nii imelik parool on. Ja siis ta lahendas selle turvaprobleemi niimoodi, et delegeeris osa vastutust arvutiklassi haldamises ja võttis nigu appi arvutiklassi haldama, natuke. Väga hea pedagoogiline meetod, töötas. Ei häkitud enam, ei olnud enam huvi edasi jagada paroole, mida ma kätte saan. 


\textbf{\enquote{Aga kust sul see krüpto huvi?}}

Seda läks sealsamas kandis ka vaja. Näiteks meie õpetaja ässitas Norton Diskreet'i\sidenote{Diskreet oli tarkvarapaketi Norton Utilities 6.0 osa ning sisaldas paljuski kurikuulsat (Kevin Mitnicku\index[ppl]{Mitnick, Kevin} andmetel kasutati väidetud 56 biti asemel 30 bitist võtit, ka teised uurijad on osundanud mitmetele olulistele nõrkustele) DESi implementatsiooni} DESi\sidenote{\emph{DES - Data Encryption Standard} on sümmeetriline algoritm andmete krüpteerimiseks. Algoritm on oma väikese võtmeruumi tõttu tänapäeval kasutamiseks sobimatu (murti avalikult jaanuaris 1999), kuid oli siiski alates 1977. aastast USA föderaalse andmetöötlusstandardi (FIPS) osa.} kallale. DESist ma ei saanud jagu, ma ei saanud DESist arugi tol hetkel. Aga tema suunas. Ta oli üldse sedasorti kaval mees, et kui ta näiteks kunagi kui meil Veljo Haguga\index[ppl]{Hagu, Veljo} oli plaan kirjutada viirus. Me olime mingeid olemasolevaid viirusi disassembleerinud ja vaadanud, kuidas need käivad. 302 oli kõige lühem vist. Veljo oli mu pinginaaber. Õpetaja sattus pealt kuulma, kui me rääkisime viiruse tegemisest ja ütles, et kui teha, siis teha kohe selline stealth-viirus. Me olime väga nõus aga seda me ei viitsinud teha ja jäi tegemata viirus. 

Ta leidis meile muidu ka rakendust. Keskkoolis üldine taustaülesanne oli midagi arvutada. Minu arvutusülesanne oli arvutada arvu $e$ kahe tuhande komakohaga 30 sekundi 10Mhz 286 peal. Üks klassivend arvutas $\pi$-d tuhande komakohaga 60 sekundiga sest see koondus aeglasemalt. Ja kust tulid ajapiirangud? Õpetaja oli vaadanud, kui kiiresti temal vastus tuleb selle arvuti peal. Ma sain 35 sekundilise programmiga juba viie kätte, sest vastus oli õigem, kui õpetajal. Kuna need erinesid, siis ta võttis targa raamatu ja siis selgus, et minul oli õige. Mul oli selleks hetkeks 21 sekundiline proramm, mis käigu pealt suurendas mingi hetk arvutüübi pikkust. Algul tegi lühema tüübiga ja hiljem pikemaga, et kiiremini saaks. Aga see oli veel bugine. See veel ei töötanud õigesti. Ma kontrollisin oma enda programmi vastu. Ma olin minut aega töötava programmiga algul arvutanud tulemuse välja ja faili kirjutanud. Siis oli mul ka näiteks variant programmist, mis küsis, et kui mitme sekundiga oli vaja arvutada ja siis ütles hard-coded vastuse. Aga see ei sobinud õpetajale. Aga 35 sekundiline juba sobis, kui vastus oli õigem tema oma.  Minu 21-sekundiline ei läinud tööle aga õpetaja seepeale võttis ja kirjutas ise asja haljas assembleris ja sai kolme sekundiga. Muidu me kirjutasime Pascalis. 

Teine asi, mida me tegime, millega oli keskkooli ajal hulga nuputamist, oli interferentsi simuleerimine arvuti ekraanil. Kaks punktlaineallikat ringlainetega, kuidas lained liituvad, et tuleb interferentspilt. Seal ma nägin ka vaeva, arvutasin ruutjuurt assembleris Newtoni meetodil. Ma arvutasin iga ekraani punkti kohta pimesi selle faasi välja nii et ühtegi punkti näha ei olnud aga ma sättisin pixelite väärtused nii et palett oli seatud üleni mustaks. Arvutasin kõik väärutsed ära assembleris optimeeritud arvutusvalemiga ja õpetaja õpetas Newtoni meetodit sinna juurde. Oli abiks. Assembleris sai Newtoni meetodit! Oli väga hariv. 

Ja lõpuks ma siis ketrasin VGA paletti. Tehnilise dokumentatsiooni failid liikusid, seal oli kirjas, kuidas VGA paletti muuta ja ma seadsin siis paletti niimoodi, et need värvid, mis mul on liikusid sujuvalt heleduse järgi. Ja siis tulemus oli see, nagu oleks liikunud lained ekraanil. Ja see oli minu meelest tippsaavutus, see oli väga ilus sujuv liikumine selle kümne megahertsi juures, punkte üle arvutada poleks kuidagi jõudnud. Ja siis näidati mulle ühe teise õpetaja tehtud programmi. Tema ütles, et minu ideest see alguse saigi, et interferentsi simuleerida. Tema tegi Juku peal circle käsuga valgeid rõngaid üksteise ümber viie millimeetrise vahega. Need läksid mida edasi seda aeglasemaks ja minu reaalajalise sujuva pildi vastu ei olnud see midagi ja mul oli tükk tegu, et mitte naerma hakata. Aga kiitsin siis takka. Õpetaja suutis anda sellise ülesande, mille peale mul kulus ikka kaua ja sain palju targemaks. Õpetaja oli Tarmo Ainsaar\index[ppl]{Ainsaar, Tarmo}. Seesama, kes suunas meid viiruse kirjutamiselt ära ja kes lahendas selle BIOSi paroolide haldusteema probleemi meiega nii et probleemi ei tulnud. Väga hea õpetaja. Ta suutis meid suunata tegema õigeid asju nii, et me seejuures õpime ja paha peale ei lähe. 

\textbf{\enquote{Kuidas sa Cyberisse sattusid?}}

Ma töötasin HClubis ja mõtlesin, et mida võiks magistriks teha. Seal tegeldi hajusate andmebaasidega. Me saatsime SQL käsk haaval andmebaaside diffe üle võrgu mitmes suunas. See oli põnev, me saime selle tehniliselt lahendatud. Algul käis see mul üle UUCP, hiljem üle PPP ja POP3 ja SMTP. Mina ehitasin internetti sinna alla, oli ka põnev. Ja neid diffe siis saatsime ja tekkis küsimus andmebaaside konsistentsusest, mis tingimustel jääb ja mis tingimustel ei jää konsistentseks. Et kas me saame mingi eventually consistent mudeli sealt või mitte. Ma mõtlesin, et ma hakkan sel teemal magistrit tegema. Aga siis kõrval oli tulnud ... Ma sattusin HClubi tööle seoses sellega, et ma installisin sinna Linuxi serveri, gateway. Ja selle peal käis veeb ja meil ja kõik. 

Jah. Ma kusjuures mõtlesin, et ma võtan sul nööbist kinni ja küsin, miks sa sealt ära läksid. Kas ikka tasub. Aga ma vist ei sattunud sind tol hetkel tabama ja siis ma ei käinud sinu käest küsimas. Või küsisin kunagi tagantjärgi, mul on nagu mälestus olemas. Jaa, küsisin, sain vastuse ka. 

Ja siis HClubis interneti teemal, mis mind huvitas tol hetkel, ei olnud mul eriti kuhugi areneda. Seal ei olnud kellegi teise käest sedasorti asju õppida. Kui siis, ise õppida ja ehitada.  Asju, mida oleks võinud pisi ISP-na veel ehitada sinna ISDN sissehelistamiskeskusi kui oleks leitud raha ja et see rentaabel on. Nihukesi asju oleks ehk saanud aga siis samal ajal ma käisin mõnes koolis abiks Linuxit installimas ja käisin laenamas RedHati install plaati suvel Elmer Joandi\index[ppl]{Joandi, Elmer} käest Tartu lähedalt maalt. Tal oli see plaadina kohe olemas ja ei pidanud flopidega mässama. Ja Elmer ütles, et muide, Tarvil\index[ppl]{Martens, Tarvi} olla plaan Tartusse meiesuguste jaoks pesa teha. Ja siis mina käisin juunikuus umbes Tallinnas Cyberneticas\index{Cybernetica} Helger Lippmaa \index[ppl]{Lipmaa, Helger} juures, et tuleks magistrit tegema hoopis krüpto teemal. Ma mõtlesin, et näiteks pordiks Open SSLi Windowsile, sest mul oli Windowsi all krüptot vaja olnud aga ei olnud. Sellest konkreetsest ideest arvati kehvasti, et näe keegi vist juba on portinud ka midagi. Aga tule meile niisama progema, mitte-krüptot. Ütles Tarvi kõrvalt. Oleks peaaegu Küberisse tulemata jäänud. Aga Helger kutsus mu ikka turva asju tegema. Ja siis kutsuti mind Küberi väljasõiduistungile ehk \enquote{kvartalnajale}. 1997. aasta sügisel Arula motelli. Seal oli kutsutud kogu tulevane Küberi Tartu Andmeturbe Labor. Ja Viljar Tulit\index[ppl]{Tulit, Viljar} oma habemesse diktsiooniga ütles, et seda sa pead ikka ise suutma ära otsustada millegi järgi, kas sa tahad siia või ei taha, kui ma ütlesin, et segane on veel, kas ma tulen või ei tule. Ja siis räägiti ka tehnilistest teemadest ja mina läksin Küberisse tarkade inimeste juurde. Seal olid Arne Ansper\index[ppl]{Ansper, Arne} ja Viljar Tulit, kes oli kogenud süsadmin (kogenum, kui mina). Kui mina tegin näiteks tükk aega FTP otsingumootorit Nuuskur koos teiste tudengitega, siis Arne oli selle stiilis nädala otsa õhtutega ära teinud või niimoodi. Arnel oli ka Vosa nimeline FTP otsingumootor Eesti FTP serverite kohta. Vosa nagu \enquote{Vanaisa Oli Sulle Archie\sidenote{Archie oli üks esimesi interneti otsingumootoreid, mis võimaldas otsingut üle FTP arhiivide}}. Tal oli ainult veebiliides, meil oli muid liideseid ka. Meil oli telneti liides ja archie prospero protokolli\sidenote{Archie kataloogides navigeerimiseks loodud protokoll, mida võib pidada tänapäevase www protokolli eellaseks. Prosperot kasutades võis terve internet välja näha, nagu üks suur ühine kataloogipuu} liides, millega vana archie klient töötaks ja meililiides ja. Meil oli võimas vinge süsteem tehtud kamba peale. Kõike ei teinud mina, teised tegid ka. Ma olin lihtsalt üks vedajaid lõpuks, kes tegi kõige rohkem tükke. Ja sellega selgus, et Arne on tark. Seal oli veel asju, millega see selgus. Näiteks tal oli Fido ja Interneti vaheline gateway. Ma olin selle kaudo Fido lugejda. Ma pole päris Fidonetti kunagi näinudki. Minu jaoks Fido oli just another NNTP server stiilis keeks.ioc.ee. Sinna tuli kasutajanime ja parooliga läheneda ja sai tavalise newsreaderiga lugeda ja kirjutada. Minu jaoks oli Fido teenus üle interneti, mida vahendas Arne tehtud süsteem. 

\textbf{\enquote{Mis sa praegu teed?}}

Praegu ma olen Küberis turvainsener ja praktikas ka tarneinsener, kes pakendab asju ja ehitab mingeid keskkondi automatiseeritult nende otsa. Õpetan ülikoolis, olen ülikoolis hajussüsteemide külalislektor, õpetan operatsioonisüsteeme baaskursusena, andmeturvet baaskursusena ja magistrandidele õpetan turvalist programmeerimist. Kuidas teha nii, et auke poleks koodis. Mõni ikka kuskilt leidub aga eks seda ole aja jooksul endale piisavalt vastu tulnud. Andmeturbe kursus sai tehtud siis, kui ma olin magistrand Helger Lippmaa juhendamisel. Helger ütles, et kuule, et sa võiks teha sellise andmeturbe kursuse ülikooli. Mõeldud tehtud. Tegingi. Kellegagi eriti nõu ei pidanud. Küberi turvaraamatu võtsin vihjete jaoks aluseks. Infosüsteemide Turve Esimene köide, oli vist esimene valdavalt, võibolla esimene ja teine. 


\textbf{\enquote{See tundub olevat nii sinu moodi, et võtad, teed ja saab väga hea}}

Parim kiitus, mis ma aineturbe ainele kuulnud olen oli kunagi kui hakati küberkaitse magistrikava tegema. Oli Tallinnas sel teemal siis koosolek. Ja oli häda, et kui me tahame neile õpetada seda, seda, ja kõiki asju, et see ei mahu meil ainetesse ära. Ja selle peale oli vist Enn Tõugu\index[ppl]{Tõugu, Enn}, kes ütles, et kuida, et Meelis jõuab andmeturbe kursuses neist kõigist asjadest rääkida, et mahutame ikka magistrikursusesse ka ära. Mis sest, et põhjalikumalt aga küll me mahutame. Et see oli hea kompliment kursusele, et Meelis räägib neist kõigist. 


