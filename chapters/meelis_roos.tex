\label{sisu:mroos}
\index[ppl]{Roos, Meelis}
\textbf{Kuidas sina arvutite juurde said?}

Kõige esimene mälestus arvutitest on koolieelsest ajast, kui
läksime emaga lasteaiast koju mööda Liivi tänavat. Paremat kätt mäe
otsas oli üks neljakordne maja. Ema ütles, et see on arvutuskeskus, ja see
kõlas aukartustäratavalt ja põnevalt. Ema töö juures nägin Nairit kah paaril
korral, aga olin siis liiga noor, et see mulle kuidagi muidu korda oleks
läinud kui lihtsalt mingi masin. Ema kabinetis oli üksvahe ajutiselt teletaip,
mis polnud vist kuhugi ühendatud, aga mis suutis ka perfolinti väljastada.

Päris arvutite juurde sattusin isa töö juures kaheksakümnendate lõpus. Füüsikud
ostsid mõned arvutid elektromeetria laborisse, nende konstrueeritud
aparaatide (elektromeetrite\sidenote{Aparaadid üliväikeste
voolude ja laengute mõõtmiseks kõrgeoomilistes (väga suure takistusega)
ahelates.}) skeemide arvutusteks ja signaalide uuringuteks reaalajas ning
hilisemaks analüüsiks. Elektromeetria labori juhataja Olev
Saksa\index[ppl]{Saks, Olev} poolt selleks otstarbeks konstrueeritud nn
dünaamiline kondensaator ja selle baasil ehitatud elektromeeter olid
1960--1970ndatel N. Liidu selle ala absoluutne tipptase, mis omaduste
poolest ei jäänud alla maailmas tunnustatud Jaapani firma Takeda Riken
dünaamilistele elektromeetritele.

Arvutid, mida 1980ndatel elektromeetria laboris kasutati, olid
CAMACi\sidenote{Computer-Aided Measurement And Control ---
elektroonikastandard andmete kogumiseks ja seadmete kontrolliks; kasutusel
(osakeste) füüsikas, aga ka tööstuses.} kontrolleriga vene
DVKd\index{DVK}\sidenote{\begin{russian}Диалоговый вычислительный
комплекс\end{russian} -- Nõukogude personaalarvuti, mis ühildus DECi
PDP-11\index{PDP-11} perekonnaga. Varasemad mudelid on tuntud ka kui
Elektronika MS-0501\index{Elektronika} ja Elektronika MS-0502.}.
Vahendid ja võimalused tookordses defitsiidiühiskonnas arvutite ostuks sai
elektromeetria labor koostöölepingutest Venemaa teadusinstituutidega, eriti N.
Liidu Kosmoseinstituudiga, kelle heaks labor 1980ndatel põhiliselt töötaski.
Nende tellimusel konstrueeriti ka uus (kosmose jaoks ikkagi!) dünaamiline
ulmeliselt väike (käekellasuurune) kondensaator, mis oli gabariitidelt ja
massilt eelmisest mudelist viis-kuus korda väiksem ning võimaldas ka ligi viis korda
suuremat mõõtmiskiirust, et töötada avakosmoses (vaakumis). Kui
leping lõppes, ei jäänud kosmoseinstituudi töötajatel teha muud, kui joonistada
skeemid ja joonised ümber ning panna konstrueerijatena oma töötajate nimed alla.

\textbf{Kus see kõik sündis?}

Kui mina seal käisin, oli see Tartu Ülikooli\index{Tartu Ülikool}
füüsikaosakonnas\index{Tartu Ülikool!Füüsikaosakond}\sidenote{Täpsemalt oli tegu
füüsika-keemiateaduskonna füüsikaosakonnaga.} ja ka selle elektromeetria laboris
Tähe tn 4 ehk nn füüsikamajas\index{Tartu Ülikool!Füüsikahoone}. Isa töötas elektromeetria laboris
elektroonik-konstruktorina. Nad tegelesid elektromeetrite väljatöötamisega ja
nende rakendusvõimaluste uurimisega. Mitmetele uuendustele-leiutistele saadi ka
N. Liidu autoritunnistused. Ühelt varasemalt üleliiduliselt näituselt sai
peakonstruktor Olev Saks ühe elektromeetri eest ka NLi riikliku preemia.

Sel ajal, kui mina isa juures käisin, konstrueeriti seal ükskord
seadet, mis mõõtis möödalendavate osakeste laenguid. Näiteks visati pastaka
kuul, millel oli mingi laeng, mõõteseadmest läbi ja mõõdeti see laeng liikumise
pealt ära. Neil oli elektromeetria sektoris lahe töögrupp, kus noored
ülikoolist tulnud mehed tegid kogenuma juhtimisel lahedaid asju. Nende katsete
juures oli vaja suhteliselt kiiresti muutuvaid signaale registreerida ja
andmeid töödelda, selleks käis arvuti külge spetsiaalne lisaplokk. Ploki sees
oli analoog-digitaalmuundur (võib olla ka vastupidi, aga igatahes niipidi neid
kasutati). Füüsikud õppisid programmeerima, et suuta oma eksperimendi andmeid
reaalajas kätte saada ja seejärel töödelda.

\textbf{Mis arvuti see oli, mis suutis andmeid 
reaalajas kätte saada?}

DVK-2M, Vene LSI-11\sidenote{DECi PDP-11 perekonna liige, tuntud ka kui
PDP-11/03. Masinat tutvustati 1975. aastal ja see oli oma sarjas esimene, mille
CPU oli integreeritud -- mitte küll ühele, vaid neljale Western Digitali 
toodetud Large Scale Integraton (LSI) kiibile. Meelise sõnul oli
PDP-11 legendaarne DECi masin iidsel ajal enne meie aega. PDP-11-l kirjutati muu hulgas suur osa UNIXist, see oli Bell Labsi kuulsa The Unix Roomi süda.}
analoog. Peaaegu täpne kloon, aga natuke ka kohapeal täiendatud. Programmide
poolest ühilduv, aga mitte identne. DVK peal jooksis näiteks DECi
originaalopsüsteem RT-11\index{RT-11}. RT-11SJ oli igapäevane opsüsteem, see
oli \emph{single job}, ning RT-11FB-l oli \emph{foreground} ja \emph{background},
millega sai taustal jooksutada mõnda teist tegevust.

\textbf{Kui vana sa olid, kui su isa töö juurde need arvutid hangiti?}

Põhikooli teises pooles. Ega mul ei olnud põhjalikku teadmist, mida
arvutiga teha saab. Kui tegin isale tekstisisestustööd, näiteks sugupuu
andmete sisestamiseks, siis sain pärast seda kuni õhtuni mängida. Lemmikmäng
oli \enquote{Wall}\index{Wall}, seina pommitamine tennisepalliga. Isa pani mind
arvuti taga kohe tööle, et mu huvist miskit kasu oleks ja ma niisama
aega ei raiskaks. 

Programmeerima õpetati mind ka, eks nad ise õppisid samuti. Isa rühmas
programmeeriti BASICus\index{BASIC}, Fortranis\index{Fortran} ja
CASICus\index{CASIC}. See viimane oli CAMACi kontrollerite
programmeerimiseks mõeldud BASICu ja Pascali\index{Pascal} vahepealne
keel\sidenote{Ilmselt peetakse silmas keelt formaalse nimega ANSI
Standard Real-Time BASIC, mille spetsifitseerib IEEE standard
\enquote{726-1982 -- IEEE Standard Real-Time BASIC for CAMAC}.}. Selles 
mina ei sattunud programmeerima, küll aga BASICus. Minu parim
programm ajas inimesega eesti keeles juttu. Programm ütles ühe lause,
kasutaja järgmise ja programm valis juhuslikult vastuse sisseprogrammeeritud
lausete hulgast ning suutis mõnikord ka teemas püsida. Näiteks kui programm
ütles: \enquote{Osta elevant ära,} siis järgmine lause oli:
\enquote{Kõik ütlevad nii, aga osta elevant ära.} Enne ei läinud järgmist
lauset valima, kui oli kaks vastust saanud. Seda mängu mängisid ka teiste töötajate
lapsed ja neil oli lõbus. See oli lahe emotsioon, et ma tegin midagi,
mis oli teiste jaoks lahe.

\textbf{Huvitav, et sa hakkasid kohe mängu tegema ja seejuures kohe
midagi AI sarnast.}

See tundus kõige lahedam asi, mida teha. Pealegi oli arvuti nimes dialoog, aga
korralikku dialoogi eriti ei toimunud ja ma siis tegin nii, et toimuks.

\textbf{Füüsikud pidid ju kähku õppima, sest reaalajas
riistvarast andmeid lugeda on keeruline.}

Rühmas oli vähemasti kolm-neli meest, kes programmeerimist õppisid. Pundis
oli teistest veidi noorem Lauri Kärner, nende põhiline arvutimees, ja tema jagas
seda paremini kui teised -- arvutas DVKga signaalitöötluse jaoks Fourier'
pööret välja ja muud säärast. Tema juures oligi see CAMACi kontroller.
Arvuteid oli labori peale kaks-kolm tükki, aga üks oli põhiline, mis
oli eksperimendi külge ühendatud. Mina kasutasin arvutit, mis oli 
masinakirjutaja toas ja mida kasutati programmide sisestamiseks ja muidu
andmetöötluseks. Näiteks isa joonistas selle abil sugupuu, mida
sai rullpaberile\sidenote{Toonaste printerite puhul oli tavaline, et
paber jooksis printerisse perforeeritud servadega rullist, nii sai paberit
kiiremini liigutada.} maatriksprinteriga välja trükkida. Kui hiljem tulid kooli
suguvõsauurijatest tudengid ja andsid igaühele paberi, et joonistage kodus
oma sugupuu üles, siis mina palusin isal lihtsalt ühe koopia välja trükkida.

\textbf{Miks sa lasid ennast sellesse suhteliselt igavasse
andmesisestaja rolli suruda? Lihtsalt et saaks mängida?}

Algul selleks, et saaks mängida. Siis selgus, et varsti mõistsin ma 
tekstiredaktorit K52 paremini kui isa (makro kasutamiseks polnudki vaja seda
iga kord uuesti defineerida). Aga kui selgus, et ise saab ka programmeerida ja
see on täitsa lahe, siis keskendusin mängimise asemel rohkem sellele. Ma ei
jätnud mängimist päris maha, mängisin ikka ka vahel.

Mind köitis programmeerimise juures, et programm võis vähendada käsitööd.
Näiteks ESC-koodidega Robotroni printerile õigeid asju saates\sidenote{Epson
Standard Code for Printers -- Epsoni poolt
maatriksprinterite jaoks välja töötatud (ja termoprinteritel siiani kasutusel
olev) keel, mis võimaldab juhtida rastrivõimekuseta printerit. Keel sai oma
nime sellest, et selle käsud algasid sümboliga ESC (ASCII 27). Näiteks ESC E
lülitas sisse ja ESC F välja rasvase trüki.} trükkisin oma õpikusilte, kus oli
rasvases ja suuremas või väiksemas kirjas kõik vajalik erinevatel ridadel
kirjas. Ema tuttav tahtis oma firma Tensiid logo visiitkaartidele, see logo tuli
siis teisendada Epsoni printeri keelseteks graafika ESC-jadadeks. Ma joonistasin selle
\emph{bitmap}'ina üles, aga siis leidsime, et ei tasu vaeva ära, ja seda logo ma ei
teinud. See oli näiteks koht, kus ma leidsin, et programmist võiks oluliselt
kasu olla. 

Üheksandas klassis jäin füüsikatunnis
programmeerimisega vahele: kirjutasin oma vihikusse mingit BASIC-programmi ja
õpetaja läks mööda ja ütles, et siin tunnis tegeleme füüsikaga,
mitte programmeerimisega. Keskkoolis tegin programmi, mis otsis lähendusi
kaheteistkümnendale juurele kahest, nii et saaks isaga süntesaatori ehitamisel
võimalikult täpse sagedusjagaja teha. See oli esimene kasulik programm, mida ma mäletan.

\textbf{Kuidas õppimine käis?}

Sain mingisuguseid venekeelseid raamatuid: osa tõi isa raamatukogust ja
mõni raamat oli tal töö juures olemas. Need olid enamasti kusagilt
laenatud. Näiteks mul oli segadus ASCII koodi ja \emph{Escape}'i koodidega, mida
sai printerile ja terminalile saata. Küsisin isalt nõu,
mis neil vahet on või kas need on üks ja sama asi. 

Üks raamat oli BASICu kohta, kust on mul siiamaani meeles ühe
käsu kirjeldus, mis minu meelest ei sobinud niisugusesse raamatusse:
\begin{russian}\enquote{эта команда работает хорошо}\end{russian} ehk \enquote{see käsk
töötab hästi}. Minu meelest oli see lati liiga madalale laskmine -- asjad tuleks teha nii, et kõik töötaks hästi.

Füüsikamaja raamatukogu riiulist leidsin eestikeelse raamatu
\enquote{Programmeerimine C keeles}“. Uurisin, mis see on. Ema vaatas, et vist mingi vana asi.
\enquote{Ei, see on uus raamat,} selgitas kõrvalt raamatukoguhoidja. Aga C oli tol hetkel
võõras asi ja raamat jäi laenutamata.

\textbf{Sul oli siis päris korralik vene keele oskus?}

Jah, ma olin umbes üheksandas klassis, kui ma programmeerimist õppisin, ja
kannatas venekeelset raamatut lugeda küll. Meil oli põhikoolis selline vene
keele õpetaja, kellega pidi õppima, ja mul oli tõenäoliselt üsna normaalne vene
keele oskus selle vanuse kohta. Ma käisin Tartu 12.
Keskkoolis\index{Tartu 12. Keskkool}. Vene keelt õpetas ukrainlanna Zinaida
Tovkatš. Kirusime, et ta on väga range ja
isegi haige ei ole kunagi. Muudkui peab õppima ja muidu ei pääse.

\textbf{Kas keegi sind õpetas ka programmeerima või käis see ainult raamatu järgi?}

Isa õpetas mulle neid asju, mida tema teadis. Näiteks 
plokkskeeme, sest ta ise õppis nende abil. See kestis kuni keskkooliajani
välja, et kui mina tegin programmi ja see ei töötanud, siis oli kaks viisi
silumiseks. Üks oli see, et trükkisin selle rullpaberil välja ja lugesin õhtul kodus.
Teine võimalus oli see, et joonistasin selle asja plokkskeemiks ja näitasin
isale. Sealt pealt tema oskas vigu leida küll. Plokkskeemiks joonistamisel
leidsin ma tihti vead ise ka üles. Ja isegi kui ma Pascal-keeles kirjutasin,
mida isa ei osanud, sain temalt ikkagi plokkskeemide tasemel abi, sest isal
oli hea loogiline mõtlemine ja ta seletas mulle mu vead ära.

Minu ülesanne oli kodus keskkütte katla alla tuli teha. Selle süütamiseks oli
füüsikaosakonnast toodud vanapaberit, mille hulgas oli teinekord mingeid arvuti
väljatrükke, mida ma lugesin. Panin need kõrvale, samal ajal kui ajalehed ja
muu läksid katla süütamiseks. Näiteks leidsin Minsk
32\index{Minsk!Minsk-32}-e\sidenote{Minsk-32 loodi kuuekümnendatel, nagu
nimigi ütleb, Minskis. Tegu oli mitmest mudelist koosneva Minski suurarvutite
sarja kõige võimekama esindajaga, mis oli laialdaselt kasutusel, kuni asendati
1970ndatel IBM 360 kloonidega.} 32bitised krahhi- või
muidu mälutõmmised. Ma olin üllatunud, et minul on 16bitine PC (see oli ilmselt
hiljem, kui ma juba PC taga olin), aga nendel oli juba siis 32bitine
arvuti. Ja seal olid FORTRANi programmid, mida ma huviga lugesin. Isa kõrvalt
ütles, et ah, need ei ole suurt midagi väärt ja et see mees, kelle programmid
need on, ei oska veel eriti programmeerida, tema programmide pealt pole eriti
mõtet eeskuju võtta. Aga põnev oli neid lugeda sellegipoolest. Nii et FORTRANit
õppisin keldris katlakütmise juures!

\textbf{Miski pani sind tulehakatust lugema, mis see oli?}

Seal olid uued põnevad asjad!

\textbf{Kas sa peale tulehakatuse midagi muud ka lugesid? Või oli sul
näiteks muusikahuvi?}

Ulmehuvi natuke oli. Mul õnnestus saada venekeelsed \enquote{Asumi}\sidenote{\label{sidenote!asum}Isaac
Asimovi kirjutatud sari. Ilmus esmakordselt triloogiana 1951. aastal ja seda
tunnustati 1966. aastal Hugo auhinnaga \enquote{Best All-Time Series}.
Alates 1981. aastast lisandus triloogiale veel köiteid.} seeria raamatud, neid
oli rohkem kui kaks esimest\sidenote{Eesti keeles ilmusid kaks esimest \enquote{Asumi}
raamatut \enquote{Asum} ning \enquote{Asum ja impeerium} vastavalt 1985. ja
1989. aastal Linda Ariva tõlkes.}. \enquote{Asumid} mulle meeldisid ja ühe isa sõbra käest
laenasime ülejäänud venekeelsed osad ka. Mul õnnestus vene keeles
raamatut lugeda ja olin selle üle sügavalt üllatunud. Isa luges neid algul
ise, hiljem mina. 

Nii et ulmehuvi oli küll, aga see ei olnud väga sügav. Seda
oli valdavalt nii palju, kui kodus sattus \enquote{Mirabilia} sarja ulmekaid olema. Need
said kõik läbi loetud. See ei olnud esialgu eriti seotud arvutitega -- see teema tuli reaalsest maailmast. Näiteks ükskord
bussiga Pärmivabriku peatusest mööda sõites seletas ema mulle arvutiviiruste
kohta, mida ta oli \enquote{Horisondist} või mujalt lugenud.
Väga põnev oli. 

\textbf{Kas sa olümpiaadidel ka käisid?}

Jaa. Matemaatikaolümpiaadil käisin neljandast klassist saadik. Oli
naljakas korrelatsioon: lastest, kellega ma olin koos käinud ülikooli töötajate
lasteaias, sai nii mõndagi olümpiaadidel kohatud. Järgmine
olümpiaadilaine oli keskkooli minnes.

Miks ma vanast koolist ära läksin? Vanas koolis oli nii, et keskkoolis pidi
tulema kaks klassi: reaal- ja humanitaarkallakuga. 
Humanitaarkallakust pidi saama \enquote{A} ehk eliitklass, kuhu läksid paremad
õpilased, ja ülejäänud pidid minema reaalkallakuga klassi. Ma
leidsin, et see on lati alla laskmine ja et ma tahaksin ikka paremat. 

Mind kutsuti Nõkku\index{Nõo Keskkool}. Hilisem ülemus
Cyberneticast\index{Cybernetica}, toonane Nõo kooli direktor Uuno
Puus\index[ppl]{Puus, Uuno} saatis kõikidele olümpiaadikutele Nõo kooli
kutsed. Sain ka. Kaalusin. Oli kaugel. Raske. Siis selgus, et Tartu 1. Keskkool\index{Tartu 1. Keskkool|see{Hugo
Treffneri Gümnaasium}} on ka täitsa kõva tasemega. Helistasin
kooli, et küsida, kas neil arvutiklass on. Direktor võttis vastu ja
reklaamis, et neil on väga hea arvutiklass. Selle peale otsustasin sinna
minna. Viisin 1990. aasta kevadel paberid 1. keskkooli ja kui sügisel
kohale läksin, oli see juba Hugo Treffneri Gümnaasium\index{Hugo
Treffneri Gümnaasium}. Olid tõesti väga head arvutid
--- Yamaha MSX-II, lisaks põhilisele arvutiklassile oli ka
Juku\index{Juku}-klass.

\textbf{Sul oli siis selge arusaam, et tahad just sinna kooli
minna?}

Jah, ma läksin nimelt sinna. Ajalooõpetaja tegi meile 
üheksanda klassi kevadel pisikese kiirküsitluse, et paljud teist siia jäävad
ja paljud lähevad kuhugi mujale. Ta küsis kolme tema nina all oleva
tegelase käest. Esimeses pingis sattusin mina istuma ja minu taga tõstsid ka kaks
tüdrukut kätt, et lähevad mujale. Nii et kõigilt,
kellelt ta küsis, sai vastuseks, et läheme ära 1. keskkooli. Tüdrukud läksid
teise paralleeli, bioloogia-keemia harusse. See tundus olevat umbes see vanus,
kus mõned hakkasid ise mõtlema oma tulevikule ning seda planeerima ja mõned
lasid asjadel isevoolu teed minna. Mina olin nende hulgas, kes leidis, et ma
tahan ise oma tulevikku kujundada.

\textbf{See oli see aeg, kui ühiskonnas hakkas juba muutus tulema, eks
ole?}

Natuke oli juba varem, selles mõttes, et kooperatiivid\sidenote{Nõukogude Liidu
lõpuaastatel lubatud spetsiifiline ettevõtlusvorm, mida kasutati esimesel
võimalusel massiliselt väikeettevõtluse alustamiseks.} olid juba varem olemas
ja asjadest tohtis rääkida. Sellesama üheksanda klassi jooksul jõudsin ma kaks
korda kirjutada ühele õpetajale referaate, millest võib olla aasta varem oleks
vanematel pahandus tulnud. Aga siis juba tohtis. Selle õpetaja kohta oli teada,
et ta on üks paras punane. Ma sain nende referaatide eest isegi kiita, mis oli
üllatav. Mõtlesin, et tuleb kuidagi oma seisukohti kaitsta, ent sain hoopis
kiita.

\textbf{Kas sind keskkooli ajal kuhugi tööle ei tõmmatud?}

Ainult natukene. Tiražeerisin isa töö juures elektromeetrite trükkplaate.
Joonistasin ahjulakiga ja risti ära lõigatud otsaga süstlaga rajad, söövitasin ja tinatasin
plaadi ära ja jootsin sinna peale kõik elemendid vastavalt
skeemile.

\textbf{See tahab ju käelist oskust ja elektroonikahuvi, kust sul
see on pärit?}

Seitsmeaastaselt oli mulle vist isa töö juures jootekolb esimest korda kätte
sattunud, kui ma suvalisi tükke kokku jootsin. Eks ma oskasin kolbi hoida ja
elektroonikahuvi mul oli. Aga elektroonikat ma ei osanud, analoogelektroonikat
ei ole ma kunagi ära õppinud. Üldisi põhimõtteid tean, aga ise midagi teha ei
ole osanud.

Digielektroonika oli seal kõrval. Kui keskkool hakkas läbi saama ja oli vaja
ülikooli minna, siis mina olin neljandast klassist peale kindel olnud, et
lähen füüsikat ja nimelt elektroonikat õppima. Aga siis tulid arvutid, kah
põnev elektroonikavärk, ja neid sai matemaatikateaduskonnas ka õppida. Mul oli tänu olümpiaaditulemustele
matemaatikasse ja füüsikasse ilma eksamiteta sissesaamine. Otsustasin matemaatika kasuks, sest
füüsikaosakonnas olin kogu aeg kohal ja mulle ei meeldinud see. Tundus, et
kui midagi ära tahta teha, siis peab ainult endale lootma. Oli 
saarekesi, kes tegelesid oma kitsa erialaga, aga laiemat kandepinda ma ei
märganud. Oli töögruppe, kes olid vingel tasemel ja tegelesid oma asjaga. Võibolla ma ei sattunud õigete inimestega kokku, aga tundus, et pigem on
füüsika selline seisev konnatiik. Igaüks on seal kinni, kus on, ja nii on.

Huvitavaid ja põnevaid asju oli seal ka. Näiteks füüsikapäevad, kus
isa käis kuulamas Undo Uusi\index[ppl]{Uus, Undo}, kes rääkis materialismi
ümberlükkamisest filosoofiliselt. Isa tuli koju, jutustas mulle ja mina panin kõrva
taha. Selliseid asju oli päris mitmeid. Füüsikalist maailmapilti
tuli vanemate kõrvalt üksjagu, see oli mul olemas.

\textbf{Kuidas sa ikkagi matemaatikat sattusid õppima? Lihtsalt
seepärast, et sai eksamiteta sisse?}

Füüsikasse oleksin vist saanud ka ilma eksamiteta ja needki ei oleks probleem
olnud. Olin lihtsalt laisk nagu kõik. Keskkooli kaheteistkümnendas klassis üritas
klassijuhataja meile auku pähe
rääkida, et poisid, olge tublid ja tehke need eksamid ikka ära, siis saab
medalile pretendeerida, muidu ei saa. Aga medaleid oleks ju vaja. Siis me
tegime vist kolm medalit klassi peale, mina sain hõbeda. Ma täpselt
ei mäletanudki, kunagi hiljem kooli koduleheküljelt lugesin. Seda ma mäletasin,
et medal oli, aga milline, seda mitte. See polnud oluline, tuli
iseenesest.

\textbf{Ühesõnaga matemaatikasse läksid seepärast, et füüsika
tundus natuke seisev vesi olevat?}

Jah. Ja ma olin kuu aega enne paberite sisseandmist kindel, et matemaatikasse
ma küll ei lähe. Käisime kooli tiimiga Moskva lahtisel
matemaatikaolümpiaadil.
Ühtlasi toimus seal \begin{russian}Международная конференция старшикласников
\enquote{Наука, природа, человек}\end{russian}\sidenote{Rahvusvaheline
vanemate klasside konverents \enquote{Teadus, loodus, inimene}.}, kus
keskkooliõpilased said isetehtud asju esitada. Näiteks keegi oli teinud kiiret
vektorgraafikat, et voldime siin kuubikut kiiremini kui AutoCAD. Ägedaid asju oli tehtud. Seal oli Hollandi rahvast ka, nii et oli küll
rahvusvaheline. Need doktorandid, kes meiega seal tegelesid, olid
parajad uhuud. Näiteks üks tegelane tuli hommikul tahvli ette, triiksärk
lükatud alukate sisse ja alukad ulatusid kümme sentimeetrit pikkade pükste
pealt välja. Ma leidsin, et vot matemaatikuks
mina küll ei lähe, aga siis mõtlesin ümber. Matemaatikuks ma ei
tahtnudki, ma läksin arvuteid õppima matemaatikateaduskonna\index{Tartu
Ülikool!Matemaatikateaduskond} poolt. Mitte elektroonika, aga
programmeerimise poolt.

\textbf{Kuidas sulle ülikooli üleminek tundus? Sa ütlesid, et olid
laisk. Minu mälu järgi pidi ülikoolis kohe hakkama tööd tegema.}

Jaa. Keskkoolis sain endale lubada laisk olemist isegi eliitkoolis,
vähemalt mingil tasemel. Ja ma sain keskkoolis arvutimängude mängimise isu täis
mängida. Ostsin omale üheksanda klassi lõpus ZX Spectrumi\index{ZX
Spectrum}\sidenote[][-1.6cm]{Sinclair Researchi poolt 1982. aastal
Ühendkuningriigi turule lastud 8bitine personaalarvuti, mis oli mõeldud peamiselt
koduseks kasutamiseks. Selle kloone liikus Nõukogude Liidus hulganisti, skeemid
olid koguni hobiajakirjades avaldatud.}, Leningradi turu klooni 1500 rubla eest,
kui rubla juba kukkus. Siis oli see suur rahanumber, aga ma sain oma isu täis
mängida. Joystick\sidenote{Juhtkang -- eelmise sajandi
algul Ameerika Ühendriikides patenteeritud, teises maailmasõjas Saksa vägede
poolt laialt kasutatud ja 1960ndate lõpus arvutimängude külge jõudnud
kaheteljeline juhtimisvahend. 21. sajandil kaotas see mängude juhtimisel
kiiresti populaarsust hiirtele ja on praegu peamiselt kasutusel lennunduses.}
sai peeneks mängitud, plasti paikasin alumiiniumiga. Tuttav treial tegi
sellele uue varre, pärast kippusid kontaktid läbi põhja tulema. 

Spectrum oli
nii hea arvuti, sai programmeerida BASICus ja Z80
Assembleris\index{Assembler}. Sellest arvutist sai lõpuni aru,
elektroonikast peaaegu ka, välja arvatud videopildi genereerimise osa.
Originaalis kasutati ULA kivi, Vene variandis realiseeriti see
lauselektroonikana\sidenote{Originaalne ZX Spectrum sisaldas kahte suurt
40 jalaga mikroskeemi: Z80 protsessorit ja üht eelprogrammeeritud loogikamassiivi
(ULA -- Uncommitted Logic Array). Nõukogude Liidus tehtud Sinclairi koopiad kasutasid
viimase asemel poolt trükkplaaditäit lihtloogikaelemente.}, sest seda kivi ei
olnud kloonina võtta. Nii et ma sain sõbra Sinclairi diagnoosimisega hakkama.
Näiteks kui ROMi jalg oli lahti ja ei andnud kontakti, siis olid
tähtedel vertikaalsed kriipsud läbi nagu dollarimärgid. Tähtede tabel oli
ROMis ja kui seal bitt oli maas, siis joonistati selle biti koha peale alati
täpp ja tekkis püstkriips. Järelikult pidi sellel ROMi kivil konkreetse biti jalg
mittekontaktis olema.

\textbf{Kas see tähendab, et sa uurisid neid asju põhjalikumalt?}

Skeeme ma ikka kuskil raamatutes ja mujal nägin. Keskkooli lõpus, kui
Venemaal käisin, ostsin metroost raamatu \enquote{\begin{russian}Введение в схемотехники
IBM PC/AT\end{russian}}\sidenote{\enquote{Sissejuhatus IBM PC/AT
skeemitehnikasse}. Ilmselt peab Meelis silmas Gennadi ja Vadim Ljovkini 1991. aastal
ilmutatud raamatut.}. Venelased olid 286 skeemid arvuti järgi välja ajanud ja
üles joonistanud. Neil oli seal minu mälu järgi viga: mingi \emph{reset}-signaali
puhul oli aktiivne null ja aktiivne üks kusagil segamini, niisugust asja
trükitud raamatus avastada oli igatahes lõbus. 

Venemaal käik oli seesama
kord, kui me olümpiaadil ja konverentsil osalesime. Konverentsi osast ei teadnud
me midagi, enne kui sinna kohale sattusime. Meil ei olnud 
ettekandeid, kuulasime niisama, mis räägitakse, ja vaatasime, millised on
kenamad tüdrukud. Üks Vene Maša oli kõige kenam.

Olümpiaadil me eriti hiilgavaid tulemusi keegi ei saanud. Mina sain meie
pundist kõige parema tulemuse, sest ma ei joonud eelmisel õhtul alkoholi. Seda
oli seal saada ja siis järgmisel hommikul pohmakaga inimesed ei esinenud oma
võimete tasemel. Nii tuligi välja, et mina olin meie omadest parim, kuigi
vähemasti üks kaasas olnud meestest oli parema peaga. Minu jaoks oli õppetund,
mida rõõmsalt teistele edasi jagada, et näe, olümpiaadi tulemus sõltus selgelt
sellest, kes ja mida eelmisel õhtul jõi.

\textbf{Ülikoolis sattusime sinuga 1993. aastal kokku.
Kuidas sulle see matemaatika tundus, mida me kohe esimese semestri alguses
hakkasime saama?}

See oli üks suur kukkumine. Ma näiteks mõtlesin ülikooli tulles, et ma tean,
mis on reaalarv. Siis tuli matemaatilise analüüsi esimene loeng, kus hakati
kõike algusest peale defineerima ja kõik muu ehitati
ainult nende definitsioonide otsa. See tahtis palju harjumist ja 
tööd, aga mina ei olnud harjunud tööd tegema.

Ma mõtlesin ülikooli minnes, et oskan programmeerida, aga Rein
Pranki\index[ppl]{Prank, Rein} matemaatilise loogika õppeprogrammid näitasid,
et on veel palju asju, millest ma aru ei saa. Seal joonistati näiteks ekraanile
tõestuspuu ja ma mõtlesin, et vau, puud ma niimoodi joonistada ei
oska. Õppisime seda küll hiljem, umbes kolmandal kursusel Varmo
Vene\index[ppl]{Vene, Varmo} funktsionaalses programmeerimises, kus me tegime
Minimaxi\sidenote{Minimax oli algselt nullsummamängude analüüsiks
formuleeritud otsustusalgoritm, mida hiljem oluliselt laiendati ja
mis leiab laiemalt kasutust tehisintellekti puhul, statistikas, filosoofias ja
mujal. Algoritm minimeerib võimalikku kahju halvimal ehk maksimaalse kahjuga
juhul, andes optimaalse mängustrateegia ja eeldades, et ka oponent mängib
optimaalselt.} ülesandetüübi näiteülesandeks puu paigutust. Esimese
kursuse järel oleks seda ehk rekursiooniga ka kuidagi teha saanud, aga see oli
näide sellest, et kõik ikka ei ole triviaalne. Igale asjale ei saa jõuga
peale minna.

\textbf{Matemaatiline analüüs, eriti matemaatiline analüüs II, võttis
meil kursuse peal palju rahvast hõredamaks, see tahtis harjumist saada.}

Algebra tahtis ka -- kogu see matemaatiline lähenemine, et ehitame asju üles
definitsioonide ja aksioomide otsa, tahtis kõvasti tööd.
Lisaks kukkusin ma esimesel kursusel haiglasse. Eksamisessiooni ajal ei jõudnud
ma mõnesid eksameid tehtudki, neid tegin alles järgmisel semestril. Käisin
dekaanilt sessi pikendust küsimas, sest vanemad õpetasid, et nii tuleb teha.
Siis dekaan ütles, et meie ajal enam niisugust asja pole, lihtsalt tehke need
eksamid ära, kuidas saate.

\textbf{Mis hetkel oli võimalik minna arvutiteadust õppima?}

Mingid põhimoodulid oli vaja ära teha ja vist esimese aasta järel sai
spetsialiseeruda. Kuna ma sain moodulid kokku, siis kaldusin üldisest
õppekavast kõrvale -- võtsin koos aasta vanematega põnevaid
arvutiteaduse aineid. Ja siis järgmisel aastal tuli võtta ka neid ained õppekavast, mis olid tegemata: tõenäosusteooriaid ja mingisuguseid matemaatikaaineid.
Minu oma kursus oli need ära teinud, nii et mina tegin neid koos aasta
noorematega.

Juhtus ka seda, et kirjutasin kodutöö programme teiste pealt maha. Meil oli
algebra ja analüüsi numbrilised meetodid, kus tegelesime arvutusmeetoditega
numbriliselt. Ma sain algoritmidest aru, need ei pakkunud mulle
algoritmi tasemel pinget ja ma ei viitsinud neid teha. Piisas, kui olin aru
saanud. Leidus üks lahke kaastudeng Jane, kelle programme ma
esitamiseks kasutasin. Muutsin vist natuke treppimist ja muutujate nimesid.
Mäletan, et kirjutasin ühele kommentaaridesse \enquote{Viimati
modifitseerinud Meelis Roos}.\sidenote{Enne kui vabavaralised tsentraliseeritud
ja hajutatud koodirepositooriumid hakkasid laialt levima, hoiti koodi enamasti
lihtsalt kettal. Seetõttu oli levinud praktika lisada failipäisesse
kommentaar faili autori, viimase muutmise kuupäeva ja muu tarvilikuga.} Eks
praktikumi juhendaja teadis, et neid programme üksteise pealt üksjagu maha
võetakse. Seepärast lasi ta endale seletada, mida see programm täpselt
teeb, aga sellega polnud probleemi ja sain kõik asjad ilusti tehtud. Ühesõnaga, kirjutasin
programme tüdrukute pealt maha, sest ma ei viitsinud programmeerida.

\textbf{Kas ülikooli arvutuskeskus Liivi tänaval ei neelanud
sind kuidagi endasse nagu mõnda inimest?}\index{Tartu
Ülikool!Liivi õppehoone}

Neelas ka mind, aga natuke teistel viisidel. Mina ei kadunud ära
Muda\index{Muda}\sidenote{\label{sidenote!muda}\enquote{Multi User Dungeon
(MUD)} -- paljude osapooltega reaalajaline tekstipõhine seiklusmäng. Täpsemalt
siiski mängude alaliik, sest leidus mitmeid eri rõhuasetusega ja erinevaid koodibaase
kasutavaid versioone, mida jooksutati mitmes serveris. Kuna Muda pakkus
toona ainulaadset koos mängimise ja suhtlemise viisi, tekkis paljudel kiiresti
sõltuvus ja liigne Mudas veedetud aeg oli sagedane ülikoolist välja langemise
põhjus.} mängima. Muda oli küll tore: kui ma oma telneti klienti kirjutasin,
sai seda Muda serveri vastu testida näiteks. Selleks oli Muda tore.

\textbf{Miks sa kirjutasid oma telneti kliendi?}

Võrguprogrammeerimise harjutamiseks -- tahtsin osata igasuguseid sokliühendusi
teha. Ma kirjutasin oma netcatilaadset asja, mis ei teinud mingisugust telneti
\emph{handshake}'i ega osanud \verb|echo off|'i ja keerulisemaid
asju, vaid lihtsalt ühendas sokli kuhugi. Kirjutasin selle endale
igasuguste asjade torkimiseks. Näiteks oli mingi mure, kui pikkade
pakettidega võis asju saata ja vastu võtta. TCP võis andmed ju suvalise koha
pealt ära hakkida. Ei saanud eeldada, et kui teiselt poolt rida sisse
kirjutatakse, siis saab selle täpselt reasuuruste tükkidena kätte. See oli
põnev.

Mind neelas see arvutuskeskus jah teistmoodi. Teisel korrusel Ülo
Kaasiku\index[ppl]{Kaasik, Ülo} kabineti kõrval oli magistrantide arvutiklass,
kus olid värvilised Sunid. Kellelgi
ei olnud eriti probleemiks, kui mina ka sinna imbusin. Aeg-ajalt ei olnud seal
kohti ja pidin siis oma koha loovutama, aga enamasti seda ei
juhtunud. Istusime seal koos aasta vanema Raul Tölbiga\index[ppl]{Tölp, Raul} ja
õppisime Unixi ära.

Kuidas ma üldse Unixit kasutama sattusin, oli omakorda lõbus. 
Võin lausa rääkida, kust on pärit minu kasutajanimi \enquote{mroos}. Minu
esimene online-konto oli masinas vask.ut.ee\index{vask.ut.ee}. See oli
VAX\index{VAX} tüüpi arvuti
VMS\sidenote{VAX arvutite \enquote{kohalik} operatsioonisüsteem.} opsüsteemiga.
Selline umbes kuupmeetrine kast pluss kettad kõrval\sidenote{Huvitav, et Meelis meenutab just serverit, samas kui Asko (lk \pageref{sisu:vase_klass}) ja Jaanus (lk \pageref{sisu:jaanus:vask}) meenutavad terminale. Kes millega kokku puutus\ldots}. Teine VAX oli
rubiin.physic.ut.ee\index{rubiin.physic.ut.ee} füüsikamajas. See oli
MicroVAX, ainult sahtlitumba suurune masin. Vot need olid VMSid. Esimesel
kursusel, selle asemel et sessi ajal õppida, olin mina raamatukogust võtnud
VAX/VMSi raamatu ja õppisin VMSi. Seal oli huvitavaid asju! Näiteks 
struktuursed failid. Võisid tekitada tühja faili, millel on ette antud
kirjestruktuur. Opsüsteemi tasemel oli \emph{Record Management System}, millega
mingis keeles kirjeldati struktuur ära ja tekitati selle kirjelduse järgi fail.
Fail võis olla ka tühi, aga sel oli struktuur olemas.

Kogu õiguste süsteem selles operatsioonisüsteemis oli keeruline. Windows
NT\index{Windows NT} on selle sisemiselt pärinud. Nii
keerukas ei ole minu meelest kui VMSis, aga kui ma nägin Windows \emph{syscall}'i
\verb|CreateProcess| koos portsu argumentidega, siis tuli tuttav ette, sest
VMSi SYS\$CREATEPROCESS oli umbes samasuguste argumentidega. SYS\$ käis lihtsaltsyscall'ide funktsioonide nimede ette.

Seal ma käisin näiteks Lynxiga veebis surfamas. Tõmbasin FTPga flopi peale faile,
mida sain kuskilt kolmandat teed mööda. Käisin internetis
ka igasuguseid asju lugemas. VMSis ma eriti ei programmeerinud. Kui oli vaja
kursaõele Pascalis programmeerimist õpetada, aga ainult VAXu klass\index{Tartu Ülikool!Liivi Õppehoone!Vase klass}\sidenote{Vt lk \pageref{sidenote!vaks}, märkus \ref{sidenote!vaks}.} oli vaba,
siis ma näitasin talle seda VAXu peal. Ta oli väga
üllatunud, et seda arvutit saab ka programmeerida. Aga sai.

Seal oli lahe programm nimega SWIM, mis lasi ühe terminali peale multipleksida
mitu akent ja lausa akende suurust muuta. Sellega ma kasutasin kolme
rakendust korraga. Aga SWIM kippus ajama terminali hanguma, kõditas vist mingit
VMSi terminali draiveri bugi. Siis tuli leida administraator,
keda tihti majas ei olnud, või logis keegi sõber-tudeng üle võrgu
rubiini\index{rubiin.physic.ut.ee} ja \emph{talk}'is Ville
Hallikuga\index[ppl]{Hallik, Ville}, kes oli sealne VMSi admin. Villel oli
juurdepääs vaske olema ja ta sai tulla ja terminali päästa -- hangunud terminali
tagant ei saanud keegi enam midagi kasutada. Ta tappis SWIMi ja mingid asjad ära, nii et terminal sai jälle vabaks. 

SWIM oli tülikas. Keegi rääkis,
et arvutiteaduse instituudi Sunides on Unixis programm nimega screen, millega saab
sedasama teha, ja siis tekkis mõte seda kasutada. Ma olin Unixit juba
korra kasutanud. Math.ut.ee-s\index{math.ut.ee}, kui tekkis \emph{online}-võrk, tuli 386BSD\index{386BSD}, mis uuendati 1993. aasta lõpus ühele
uuele tundmatule opsüsteemile. Asemele osteti 486 arvuti suure
kahegigase\sidenote{Tol ajal piisas keskmist sorti arvutifirma failiserveri
kõvakettaks ühest gigabaidist üsna pikaks ajaks. Aastal 2022 täidab keskmine
koduinternetiühendus selle mahu umbes minutiga.} SCSI vindiga. Selle SCSI kaardi
jaoks 386BSD enam ei sobinud ja pandi asemele uus tundmatu asi nimega
Linux\index{Linux}, vist versioon 0.99pl3.

\textbf{Kust selline asi sattus Tartu linna?}

No aga kust 386BSD sattus? Internet oli ju olemas. Kasutajad koliti 386BSDst
Linuxisse siuhti üle ja mul oli ühes Linuxis kasutaja. Millalgi
uuendati see Linux ära kerneli versioonile 1.0.2. Ma olin natukene Linuxit nuusutanud. Kui tahtsin Liivi tänaval Unixi \emph{screen}'i, siis math.ut.ee
ühendus oli päris aeglane\sidenote{math.ut.ee asus füüsiliselt
matemaatikateaduskonna hoones Vanemuise tänaval. Seega peetakse 
silmas internetiühendust kahe linnulennult 550meetrise vahega hoone vahel
Tartus.}. 9600ne ühendus oli jagatud paljude kasutajate, meilide ja muu
vahel. Siis küsisin endale cs3 (hilisem
romulus.cs.ut.ee\index{romulus.cs.ut.ee}) konto põhjendusel, et
tahaksin näppida mõnda mitte-Linux/Unixit. Seal oli Solaris\index{Solaris}.

Toomas Soomele\index[ppl]{Soome, Toomas} tundus see piisavalt hea põhjendus.
Tema kasutajanimi oli \enquote{tsoome} ja ma mõtlesin, et ahaa,
Unixis käib see niimoodi. Küsisin omale sama skeemi järgi
kasutajanimeks \enquote{mroos}. Antigi ja ma olen seda edaspidi
igal pool kasutanud. Isegi kui mul on kodus testarvuti, olen seal ka
harjumusest mroos. Et tsoome mulle kasutajanime teeks, tuli öelda, et tahan
Solarist kasutada ja kasutajanimi peaks samas formaadis olema, et
võimalikult vähe küsimusi tekiks.

Mul oli möödunud aastal\sidenote{Intervjuu Meelisega toimus 2020. aasta kevadel.}
väga sürr kogemus, kui minuga võttis ühendust seesama Toomas Soome, kellel
oli senimaani magistrikraad kaitsmata. Ta tahtis, et juhendaksin tema
magistritööd. Ma mõtlesin, et muna õpetab kana, aga tal
oli korralik tehniline töö olemas ja mina teadsin, mismoodi üks magistritöö
peab enam-vähem välja nägema. Sellest teadmisest oli kasu -- töö sai tal
vormistatud magistritööks ja ta kaitses selle edukalt. Arvutiteaduse instituudis\index{Tartu
Ülikool!Matemaatikateaduskond!Arvutiteaduse instituut} oli terve hulk rahvast,
kes kaitsesid magistrikraadi hiljem.

\textbf{Kas sind teadust ei tõmmanud tegema?}

Ei, teaduse tegemine ei ole mind kunagi eriti tõmmanud ja keegi ei suutnud
mulle ka auku pähe rääkida. Väga ei proovitud ka, niisama meelitati
erineval viisil mingeid materjale ette söötes. Materjalid olid nii
teadusega kui ka mitteteadusega seotud. Näiteks Jaanus Pöial\index[ppl]{Pöial,
Jaanus} jagas mulle kunagi raamatut \enquote{The Java Language Specification},
et näe, üks uus moodne asi. Selliseid asju ülikoolis ikka juhtus.

Mäletan, kui olin rebane ega olnud veel spetsialiseerunud arvutiteaduse
instituudi informaatika erialale. Aga mul tekkis äkitselt laupäeval vajadus välja trükkida
viietolliselt flopilt üht tekstifaili, raamatukogus tehtud kataloogiotsingu tulemust mingite raamatute otsimiseks. Vajusin lihtsalt Liivi tänavale kohale ja
käisin mööda uksi koputamas. Oli vist õhtune aeg ja seal
ei olnud palju rahvast. Sattusin Mati Tombaku\index[ppl]{Tombak, Mati} ukse
taha, kes lasi lahkelt trükkida. Sellest tekkis tänutunne kogu
instituudi vastu, et siin on lahked inimesed. See oli minu esimene
isiklikul tasemel kontakt instituudi inimestega.

\textbf{Millal sa tööle läksid?} 	

Minu esimene ametlik töökoht oli metoodik Tartu Ülikooli Täppisteaduste
Koolis\index{Tartu Ülikool!Täppisteaduste Kool}. See oli tegelikult
postmasteri töö, aga sellenimelist ametinimetust ei olnud, vaid oli metoodik.
Korraldati programmeerimise kursust e-mailitsi koolides. Mina
pidasin arvet selle üle, kellel mis ülesanded olid lahendatud, ja saatsin neile
järgmisi. Arvutiõpetajad, kellele vastused parandamiseks saadeti,
saatsid minule seisu ja mina selle järgi saatsin järgmisi ülesandeid. 

Ma olen laisk inimene. Esimesel tööpäeval nägin pool päeva vaeva ja
kirjutasin skripti. Panin kuhugi tekstifaili valmis nimed. Programm võttis
sealt järjest nimesid, saatis neile ülesande ja pidas arvestust, kellele
on juba saadetud, et keegi topelt ei saaks. Ja kui skripti käima
panin, siis rubiin.physic.ut.ee\index{rubiin.physic.ut.ee}, tollane
füüsikamaja Unixi server, kõristas umbes pool tundi. Pärastpoole õppisin
\verb|nice| käsu\sidenote{Võimaldab Unixi keskkonnas kontrollida, kas kogu
programm kasutab ära kogu saadaoleva arvutusressursi või jätab midagi ka
teistele arvutikasutajatele.} ka ära. Aga see tähendas, et kogu minu edasine töö
pärast selle skripti kirjutamist oli \emph{copy-paste} meili seest sisendfaili
ja skript tööle lükata. Automatiseerisin oma töö lihtsalt ära.

\textbf{Kuidas sa sinna sattusid?}

Arvan, et Indrek Jentson\index[ppl]{Jentson, Indrek} täppisteaduste koolist
kutsus mind 1995. või 1996. aastal. Indrek oli matemaatikateaduskonnas vanem tegelane ja
olümpiaadidega tegelenud. Läksin täppisteaduste kooli ukse taha, vastu tuli juhataja Viire
Sepp\index[ppl]{Sepp, Viire} ja ma ütlesin, et tere, tulin
töölepingut tegema. \enquote{Mis töölepingut?} küsis tema. Ma siis seletasin,
et Indrek Jentson saatis mind postmasteri töölepingut tegema.

\textbf{Siis ju üsna vara! Tuleb häbiga tunnistada, et ma läksin juba 1993.
aastal tööle.}

Te olite Veljo Haguga\index[ppl]{Hagu, Veljo} Korelis\index{Korel IN}, eks? Ma
käisin vahel Veljo töö juures, kuna seal olid mingid mängud. Veljo mängis õhtuti näiteks
\enquote{Dune'i}\index{Dune} ja ma vaatasin, kuidas see käib. Mängimisega ei olnud mul erilist
suhet. Sain keskkooli ajal Sinclairi peal oma mängimisisu täis mängida ja
lülitusin juba programmeerimisele, teades, et see on palju põnevam. Ma kirjutasin näiteks oma \emph{binary editor}'i, millega mängudest
järgmiste \emph{level}'ite paroole välja nuuskida.

Arvutiturva teema on mul keskkoolist saadik sees. Meil olid keskkoolis
väga põnevad ja harivad võidujooksud arvutiõpetajaga. Näiteks oli õpetaja
arvutiklaviatuur parooli all. Aeg-ajalt tehti sellega meilivahetust, nii et
klaviatuur oli lukus, aga muidu masin töötas edasi. IBM
PS/2-l\index{IBM PS/2}\sidenote{PS/2 oli IBMi kolmas personaalarvutite
põlvkond, mida tutvustati 1987. aastal. Paljud tolle masina innovatsioonid, nagu
näiteks VGA video, muutusid \emph{de facto} standardiks pikkadeks
aastateks.} oli selline klaviatuuriluku võimalus. Küll ma üritasin
leida meetodeid sellest mööda hiilimaks! Kui nägin kuskil mingeid skeeme, siis mul tekkis idee, kuidas i8042 klaviatuurikontrolleri kaudu teha
masinale sobivat \emph{warm boot}'i, et sealt mööda hiilida, aga
klaviatuurikontroller oli ikka lukus. Kirusin, et IBMi omad on kavalad.

Lõpuks saadi arvuti parool teada lihtsal viisil: vaadati üle selle
arvutiõpetaja õla, kes aeglasemalt tippis. Ega me muidugi
selle parooliga midagi teinud, see ei olnud eesmärk. 

Edasi oli põnevam keskkooli viimasel aastal, kui jõudsid kohale 386d. Nende C: ketas ehk
kõvaketas pandi kirjutuskaitse alla nii, et
\verb|config.sys|-ist laaditi spetsiaalne draiver, mis tegi virtuaalse D: ketta ja keeras kogu C:
\emph{read-only}'ks. Avastasin selle niimoodi, et mul oli ühe enda
softi katsetamiseks vaja see autoexec.bat-i või \verb|config.sys|-i panna või
sealt midagi välja kommenteerida, et minu asi ära mahuks. Igatahes oli mul vaja sinna sekkuda. Kui sain sekkutud, siis pärast
alati taastasin endise olukorra.

\textbf{Kas tol ajal võrgu häkkimist ei toimunud?}

Anto Veldre\index[ppl]{Veldre, Anto} rääkis\sidenote{Meelis peab ilmselt
silmas eetris olnud memcpy osa Anto Veldrega.} jah, kuidas tema poisid
ülikooli adminnidel ruutusid\sidenote{Unixi-laadsetel
süsteemidel on root'i (mis eesti kõnekeeles mugandub tihtipeale sõnaks
\enquote{ruut}) ees süsteemi täielike õigustega peakasutaja. Seega tähendab
termin \enquote{ruutu võtma} arvutisüsteemi üle täieliku kontrolli saavutamist,
tihti algset peakasutajat virtuaalse ukse taha jättes.} käest ära võtsid. Tema jagas oma poistele
modemeid ja terminale, mis tulid kuskilt humanitaarabina. Meil oli üks modem
õpetaja arvuti küljes. Ühel poisil oli oma modem kord koolis kaasas, aga me ei osanud sellega midagi teha ja kohalikku võrku meil ei olnud.
LAN\sidenote{\emph{Local Area Network} -- kohtvõrk.} tekkis meile alles kaheteiskümnenda
klassi kevadel, kui ma enam väga ei tegelenud sellega. 

Ma häkkisin LANtasticu\sidenote{LANi \emph{peer-to-peer} operatsioonisüsteem,
mida arendas Artisoft ning mis jäi hiljem Novelli ja Microsofti toodete varju.}
lahti \emph{social engineering}'u meetodil. Sügisel pärast minu äraminekut oli
kellelgi vaja saada LANtasticule juurdepääsu. Servermasinas oli selline koht
nagu \emph{network control directory}, kus olid andmebaasid binaarsena. Ja vot
minu programm oskas käia ja binaarselt andmebaasi modifitseerida -- tekitada
ühe administraatori lisaks või panna kellelegi õigusi juurde. Selleks
tuli meelitada noorem arvutiõpetaja flopi pealt masinas ühte programmi käivitama ja teda viisakalt tänada. Tema poolt oli ka kõik OK.

C:-ketta kirjutuskaitse puhul käisime Nortoni Disk
Editoriga \verb|config.sys|'i algust ära sodimas, et seda ei
loetaks. Järgmisel tarkvaraversioonil oli see koht paremini kaitstud ja siis
oli vaja ikka flopi pealt boot'ida. Aga BIOS oli parooli all, A: ja C: vs. C: ja
A:. Järelikult muukisime BIOSi paroolid lahti. Need olid obfuskeeritud
kujul kirjutatud kuhugi CMOS-mälusse ja masina ROM oli väljaloetav. Ma disassembleerisin selle Sourcereri-nimelise disassembleriga ja matemaatikatunni ajal kirjutasin vihikusse kõrvallehele programmi,
mis obfuskeeritud asja lahti võtaks. Järgmine tund oli ajalugu. Läksin
ajalootunni asemel arvutiklassi, realiseerisin programmi ära ja muukisin
BIOSi paroolid lahti. 

Mul tuli suur pahandus, sest see oli ajalootund, kust
paljud olid puudunud, õpetaja oli väga kuri ja keeras käkki. Mul oli
pärast vaja tund järele teha. Puudumist põhjendasime sellega, kui 
hea programmi me tegime, spetsifitseerimata, mis see oli. Lihtsalt nii hea idee oli, et
tuli minna arvutiklassi ja kohe ära teha. Parool oli obfuskeeritud
jadašifrina või isegi baithaaval, nii et järjest tähthaaval otsast proovides
sai selle ära arvata. Ma kunagi arvutiõpetajalt küsisin, miks
tal nii imelik parool on. Siis ta lahendas turvaprobleemi niimoodi, et
delegeeris osa vastutusest ja võttis õpilased appi arvutiklassi
haldama. Väga hea pedagoogiline meetod, töötas. Rohkem ei häkitud, sest enam ei olnud 
huvi kättesaadud paroole edasi jagada.

\textbf{Kust sul krüptohuvi tuli?}

Seda läks sealsamas kandis vaja. Näiteks meie õpetaja ässitas Norton
Diskreeti\sidenote{Tarkvarapaketi Norton Utilities 6.0 osa, mis
sisaldas paljuski kurikuulsat DESi implementatsiooni: Kevin Mitnicku
andmetel kasutati väidetud 56 biti asemel 30bitist võtit, ka teised uurijad on
osutanud mitmele olulisele nõrkusele.}
DESi\sidenote{\emph{Data Encryption Standard} -- sümmeetriline algoritm
andmete krüpteerimiseks. Algoritm on oma väikese võtmeruumi tõttu tänapäeval
kasutamiseks sobimatu (murti avalikult jaanuaris 1999), kuid oli siiski alates
1977. aastast USA föderaalse andmetöötlusstandardi (FIPS) osa.} kallale. DESist
ma jagu ei saanud jagu, ma ei saanud sellest tol hetkel arugi. Aga tema suunas. 

Ta oli üldse sedasorti kaval mees, et kui ta näiteks kuulis, et meil oli
pinginaaber Veljo Haguga\index[ppl]{Hagu, Veljo} plaan kirjutada viirus,
siis ta suutis meid sellest eemal hoida. Me olime olemasolevaid viirusi
disassembleerinud ja vaadanud, kuidas need käivad. Õpetaja sattus pealt kuulma,
kui rääkisime viiruse tegemisest, ja ütles, et kui teha, siis kohe
selline \enquote{stealth}-viirus. Me olime väga nõus, aga seda me ei viitsinud
teha, ja nii jäi viirus kirjutamata.

Ta leidis meile muidu ka rakendust. Keskkoolis oli üldine taustülesanne midagi
arvutada. Minu ülesanne oli arvutada arvu $e$ kahe tuhande komakohaga 30
sekundiga 10 MHz 286 peal. Üks klassivend arvutas $\pi$-d tuhande komakohaga 60
sekundiga, sest see koondus aeglasemalt. Ja kust tulid ajapiirangud? Õpetaja
oli vaadanud, kui kiiresti temal vastus tuleb selle arvuti peal. Ma sain
35sekundilise programmiga juba viie kätte, sest vastus oli õigem kui õpetajal.
Kuna need erinesid, siis võttis ta targa raamatu ja selgus, et minul oli
õige. Mul oli selleks hetkeks 21sekundiline programm, mis käigu pealt
suurendas ühel hetkel arvutüübi pikkust. Algul tegi lühema tüübiga ja hiljem
pikemaga, et kiiremini saaks. Aga see oli veel bugine ega töötanud õigesti.
Kontrollisin seda enda programmi vastu. Ma olin minut aega töötava
programmiga algul tulemuse välja arvutanud ja faili kirjutanud. Mul oli
ka näiteks variant programmist, mis küsis, kui mitme sekundiga oli vaja
arvutada, ja siis ütles \emph{hard-coded} vastuse. Aga see ei sobinud õpetajale.
35sekundiline juba sobis, kui vastus oli õigem kui tema oma. Minu
21sekundiline ei läinud tööle, aga õpetaja seepeale kirjutas ise
haljas assembleris\index{Assembler} ja sai kolme sekundiga. Muidu
me kirjutasime Pascalis\index{Pascal}.

Teine asi, mida me tegime ja millega oli keskkooli ajal hulga nuputamist, oli
interferentsi simuleerimine arvutiekraanil. On kaks punktlaineallikat
ringlainetega ja tuleb arvutada, kuidas lained liituvad, et tekiks
interferentspilt. Seal ma nägin ka vaeva, arvutasin ruutjuurt
assembleris\index{Assembler} Newtoni meetodil. Ma arvutasin iga ekraani
punkti kohta pimesi selle faasi välja nii, et ühtegi punkti samal ajal näha ei
olnud, aga sättisin pikslite väärtuse siis, kui palett oli seatud üleni mustaks.
Arvutasin kõik väärtused assembleris optimeeritud arvutusvalemiga ja õpetaja
õpetas Newtoni meetodit sinna juurde. Oli abiks ja väga hariv.

Lõpuks ma ketrasin VGA paletti. Tehnilise dokumentatsiooni failid
liikusid ringi ja seal oli kirjas, kuidas VGA paletti muuta. Seadsin paleti
nii, et need värvid, mis mul olid, liikusid sujuvalt heleduse järgi. 
Selel tulemusena liikusid ekraanil justkui lained. See oli minu meelest
tippsaavutus -- väga ilus sujuv liikumine kümne megahertsi juures,
punkte üle arvutada poleks kuidagi jõudnud. 

Pärast viis õpetaja mind ühe teise
õpetaja tehtud programmi vaatama. See õpetaja ütles, et tema ideest interferentsi simuleerimine alguse saigi. Tema tegi Juku peal \verb|circle|
käsuga valgeid rõngaid üksteise ümber viiemillimeetrise vahega. Mida edasi need läksid,
seda aeglasemaks muutusid ja minu reaalajalise sujuva pildi vastu ei olnud
see midagi. Mul oli tükk tegu, et mitte naerma hakata, aga kiitsin takka.

Meie õpetaja Tarmo Ainsaar\index[ppl]{Ainsaar, Tarmo} suutis anda sellise ülesande, mille peale mul kulus kaua aega ja
sain palju targemaks. Tema suunaski meid viiruse kirjutamiselt ära ja lahendas BIOSi
paroolide haldusteema probleemi meiega nii, et probleemi ei tulnud. Väga hea
õpetaja! Ta suutis meid panna tegema õigeid asju nii, et me seejuures õppisime
ja ei läinud paha peale.

\textbf{Kuidas sa Cyberisse sattusid?}

Ma töötasin HClubis\index{HClub} (sattusin sinna tööle seoses sellega, et
installisin neile Linuxi serveri ja \emph{gateway} veebi ja meili jaoks) ja
mõtlesin, mida võiks magistriks teha. Seal tegeldi hajusate andmebaasidega.
Me saatsime SQL-käskhaaval andmebaaside \emph{diff}'e üle võrgu mitmes suunas.
See oli põnev ja saime selle tehniliselt lahendatud. Algul käis see mul üle
UUCP, hiljem üle PPP, POP3 ja SMTP. Mina ehitasin internetti sinna alla, mis oli
ka põnev. \emph{Diff}'e saates tekkis küsimus andmebaaside
konsistentsusest: mis tingimustel jääb andmestik konsistentseks ja mis tingimustel ei jää ning kas saame sealt mingi \emph{eventually consistent} mudeli 
või mitte. Mõtlesin hakatagi sel teemal magistrit tegema.

Aga HClubis ei olnud mul interneti teemal, mis mind tol hetkel huvitas, eriti
kuhugi areneda. Seal ei olnud kellegi teise käest sedasorti asju õppida, ainult
ise õppida ja ehitada. Asju, mida oleks võinud pisi-ISPna veel teha ---
näiteks ehitada ISDNi sissehelistamiskeskus ---, kui oleks leitud raha ja et see on
rentaabel.

Samal ajal käisin mõnes koolis abiks Linuxit installimas ja laenasin 1997. aasta suvel
RedHati installiplaati Elmer Joandi\index[ppl]{Joandi, Elmer}
käest. Tal oli see plaadina olemas ja ei pidanud
flopidega mässama. Elmer ütles, et Tarvil\index[ppl]{Martens, Tarvi} olla plaan
Tartusse meiesuguste jaoks pesa teha. Umbes juunikuus käisin ma
Tallinnas Cyberneticas\index{Cybernetica} Helger Lipmaa \index[ppl]{Lipmaa,
Helger} juures rääkimas, et tuleksin magistrit tegema hoopis krüpto teemal. Mõtlesin näiteks,
et pordiks OpenSSLi\sidenote{Teek arvutite omavaheliseks turvaliseks
suhtluseks krüptograafia abil. Tuntuim ja levinuim omataoline.} Windowsile, sest
mul oli Windowsi all krüptot vaja läinud, aga seda polnud kuskilt võtta. Selle
konkreetse idee kohta arvati kehvasti, sest keegi vist juba oli midagi portinud.

Tarvi kutsus mind niisama progema, mitte krüptoga tegelema. Oleksin peaaegu jätnud
Küberisse tulemata, aga siis Helger kutsus mu ikka turvaasju tegema. 1997. aasta sügisel
kutsuti mind Arula motelli Küberi\sidenote{Kõnekeeles on
\enquote{Küber}, \enquote{Küberneetika AS}, \enquote{Küberneetika Instituut} ja
\enquote{AS Cybernetica} sisuliselt sünonüümid. Aastal 1960 asutati Eesti
Teaduste Akadeemia Küberneetika Instituut, mis 1997. aastal reorganiseeriti
Küberneetika ASiks ja hiljem nimetati ümber Cybernetica ASiks, kuid sisu jäi
suuresti samaks.} väljasõiduistungile ehk \enquote{kvartalnajale}\sidenote{Nii
kutsutakse Cybernetica töötajate regulaarseid (algselt kvartaalseid) ja legendaarseid meeskonnaüritusi.}. Seal oli koos kogu tulevane
Küberi Tartu Andmeturbelabor\index{Cybernetica!Andmeturbelabor}. Viljar
Tulit\index[ppl]{Tulit, Viljar} ütles seal oma habemese diktsiooniga, et seda
sa pead ikka ise suutma otsustada, kas sa tahad siia tulla
või mitte, kui ma ütlesin, et veel on segane, kas tulen või ei tule.

Ühesõnaga, ma läksingi Küberisse tarkade inimeste juurde. Seal oli ka Arne
Ansper\index[ppl]{Ansper, Arne}, kes oli kogenud süsadmin
(kogenum kui mina). Kui mina tegin koos teiste tudengitega näiteks tükk aega FTP otsingumootorit
Nuuskur\index{Nuuskur}, siis Arne oli selle
nädala ajaga ära teinud. 

Arnel oli Vosa\index{Vosa}-nimeline FTP
otsingumootor Eesti FTP-serverite kohta. Vosa nagu \enquote{Vanaisa Oli Sulle
Archie\sidenote{Archie oli üks esimesi interneti otsingumootoreid, mis
võimaldas otsingut üle FTP arhiivide.}}. Sellel oli ainult veebiliides, meil oli
muid liideseid ka: telneti liides ja Archie Prospero
protokolli\sidenote{Archie kataloogides navigeerimiseks loodud protokoll, mida
võib pidada tänapäevase WWW protokolli eellaseks. Prosperot kasutades võis
terve internet välja näha nagu üks suur ühine kataloogipuu.} liides, millega
vana Archie klient töötaks, samuti meililiides. Meil oli kamba peale tehtud võimas ja vinge süsteem. Ma olin lihtsalt üks
vedajaid ning lõpuks see, kes tegi kõige rohkem tükke. 

Siis selguski, et
Arne on tark. Lisaks oli tal Fido ja
interneti vaheline \emph{gateway}. Ma olin selle kaudu
Fido lugeja. Ma pole päris Fidonetti kunagi
näinudki. Minu jaoks oli Fido järjekordne
NNTP\sidenote{Network News Transfer Protocol -- Useneti
uudiste vahetamiseks kasutatud protokoll.} server stiilis
keeks.ioc.ee\index{keeks.ioc.ee}. Sinna tuli kasutajanime ja parooliga
läheneda ja lugeda-kirjutada sai tavalise \emph{newsreader}'iga. Minu jaoks
oli Fido teenus üle interneti, mida vahendas Arne tehtud süsteem.

\textbf{Mida sa praegu teed?}

Praegu olen Küberis\index{Cybernetica} turvainsener, praktikas ka
tarneinsener, kes pakendab lahendusi ja ehitab nende jaoks automatiseeritud
keskkondi. Lisaks õpetan ülikoolis\sidenote{Tartu Ülikool.}, olen 
hajussüsteemide külalislektor. Õpetan baaskursustena operatsioonisüsteeme ja
andmeturvet ning magistrantidele turvalist programmeerimist ehk
kuidas teha nii, et koodis poleks auke. Mõni auk
ikka kuskil leidub, aga eks neid ole aja jooksul endale piisavalt vastu tulnud.

Andmeturbe kursus sai tehtud siis, kui olin alustav doktorant Helger
Lipmaa\index[ppl]{Lipmaa, Helger} juhendamisel. Helger ütles, et kuule, sa 
võiksid ülikooli sellise andmeturbe kursuse teha. Mõeldud, tehtud.
Kellegagi eriti nõu ei pidanud. Küberi turvaraamatu\sidenote{Hanson, V., Lipmaa
H., Buldas A., Ansper A., Tulit V., Martens T. \enquote{Infosüsteemide turve 1.
osa}, 1997;
\enquote{Infosüsteemide turve 2. osa}, 1998.} võtsin kondikava jaoks aluseks,
vist valdavalt esimese köite.

\textbf{See tundub olevat nii sinu moodi. Võtad, teed ja saab väga
hea!}

Parim kiitus, mida ma andmeturbe aine kohta kuulnud olen, oli siis, kui hakati
küberkaitse magistrikava tegema. Tallinnas oli sel teemal koosolek. Häda oli
selles, et kui tahame neile kõike seda õpetada, mida sooviks, ei mahu see meil
ainetesse ära. Selle peale ütles vist Enn Tõugu\index[ppl]{Tõugu, Enn}:
\enquote{Kuidas nii? Meelis jõuab andmeturbe kursusel kõigist neist
asjadest rääkida, mahutame ikka magistrikavasse ka ära. Mis sest, et
põhjalikumalt, aga küll me mahutame.}

