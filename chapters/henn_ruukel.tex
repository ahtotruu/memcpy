\index[ppl]{Ruukel, Henn}

                 
\question{Nagu ikka, hakkame pihta päris algusest. Kuidas, lillekene, arvutid 
sinu juurde jõudsid? }

Ma mäletan, kuidagi kohe seda, et jõudsid kooli arvutiklassi kaudu.

\question{Mis kool see oli?}

Tallinna 10. keskkool\index{Koolid!Tallinna 10. keskkool}, tänane Nõmme 
Gümnaasium\index{Koolid!Nõmme Gümnaasium|see{Tallinna 10. keskkool}}. Ja kui ma 
õieti mäletan, oli see niimoodi, et  pärast mingit suvevaheaega, ma pean nüüd 
hoolega mõtlema, mis klass võis olla. Aga pärast suvevaheaega tulime kooli ja 
meie mata klassi oli ehitatud arvutiklass. Klassis olid sellised lauad, mis 
nägid välja nagu koolilauad aga sai lauaplaadi üles tõsta ja sealt seest tuli 
välja omakorda mingi  pööratava metallkonstruktsiooniga  
Elektronika\index{Arvutid!Elektronika} arvutid. Ja nad olid omavahel võrgus 
niimoodi, et  õpetaja laua peal oli siis flopidraiviga 
Iskra\index{Arvutid!Iskra}. Nendes Elektronikates jooksis ainult Basic. 
Lükkasid käima, Basic\index{Keeled!BASIC} kohe jooksis, said koodi 
kirjutada ja olid Basic-u \emph{command}-id, millega sai oma programmi \emph{store}-ida oma 
sinna Iskra flopikettale. Ehk sa viisid õpetaja kätte oma 
flopi ketta, ta pani selle masinasse, tegi midagi seal Iskras ilmselt ka. Ja 
siis said salvestada oma proge maha või pärast järgmine päev laadida. Ja nii 
kui see juhtus ära, siis muidugi me hakkasime manguma kohe mata õpetajat kes 
oli Ramil Izamentinov\index[ppl]{Izamentinov, Ramil}, kes on selles mõttes äge 
mees, et \emph{never} oma elus ma pole enam näinud inimest (ta oli astmahaige), 
kes nii ägedalt naeraks oma naljade üle. Ta andis meile matat, eksole, ja kui 
ta tegi mingi nalja, siis et kõik saaksid aru, et see oli nali, tegi ta sihukest 
spetsiifilist kähinat. 

No, \emph{anyway}, ta lasi meil pärast tunde olla seal klassis ja me 
hakkasime kirjutama igast suvalisi progesid. 

\question{Mis klassis see oli?}

See pidi olema enne keskat kõvasti. Ma arvan, et see võis olla mingi seitsmes 
klass või? Järelikult, kui seitsmes klass, siis 1986? Aga see oli kindlasti 
enne keskat. Ja minu arust see kooli arvutiklassi aeg, kui ma seal käisin, 
kestis suurusjärk ühe õppeaasta. Põhiliselt sai kirjutatud mingeid progesid. 
Mänge seal üldse ei olnud, selles mõttes äge \emph{envrionment}, et  
ei olnud lihtsalt ühtegi arvutimängu, mida mängida. Nii, et kõik mängud, mis me 
tahtsime, et saaks mängida, tuli ise Basic-us teha, 
 kirjutada flopi peale ja sealt pärast laadida. Mingeid 
lihtsaid mänge me tegime, trips-traps-trullid ja asjad. Ma mäletan kindlasti, 
et äge \emph{challenge} oli\ldots Elektronikal  oli graafiline liides, nii et 
sa said ekraanile joonistada. Näiteks, ma mäletan, kindlasti üks programm oli 
selline, et sa andsid ruutvõrrandi parameetrid, ta arvutab lahendid välja pluss joonistab 
graafiku. Selle järgi kusjuures saaks õppeaasta \emph{detect}-ida. Selliseid asju 
tegime. 

\question{Oot, aga miks te hakkasite manguma, et nondele arvutitele ligi saaks?}

Seda ma ei mäleta. 

Aga igatahes see kuidagi tundus kohe sihuke asi, et tahaks ja oleks äge, et kas 
võib käia ja mis õhtuti. Ikkagi algul oli vist nii, et õpetaja oli ise kohal, 
aga vist  juba lõpuks oli meil mata klassi võti. Aga see on kõik nii ammu. Paar huvilist 
oli veel minu klassist, kellega me käisime seal, ja eraldi \emph{arrangement} 
tuli õppealajuhatajaga teha selleks, et saada algklasside pikapäeva rühma söögi 
peale. Et ei peaks koolist ära käima, kui tunnid lõpevad, et me saaksime 
väikeste lastega koos sööklas süüa ja siis minna arvutiklassi õhtuni. 

Eks ta oli ikka nii põnev asi, see oli ikka hoopis teine aeg ju, kodus polnud 
videomakkigi arvutitest rääkimata või et kellegi töö juures oleks arvuti, 
sellist asja polnud. 

Kui nüüd sealt edasi liikuda, kuidas järgmine faas tuli, oli see, et ma kuulsin 
ühelt oma kaugelt sugulaselt mingil pere sünnipäeval, et tema käib TPI-s 
arvutiringis\index{Arvutiklubi!TPI arvutiring}. Ja see on see kuulus 
arvutiring, mida pidasid Julius\index[ppl]{Raimla, Tõnu}\index[ppl]{Julius|see{Railma, Tõnu}}\sidenote{Henn peab ilmselt silmas Tõnu Raimlat toonase hüüdnimega \enquote{Julius}} ja Aare Tali\index[ppl]{Tali, Aare}. 

Nad pidasid TPI-s nihukest \emph{need to know 
basis} arvutiring, et sa pidid teadma, et sa lähed teisipäeval Raadiotehnika 
kateedri\index{Tallinna Tehnikaülikool!Raadiotehnika Kateeder} taha otsa ja 
ootad ühe ukse taga. Seal oli sinusuguseid jõnglasi nagu murdu. Ja mingil 
kindlal kellaajal Julius tegi ukse lahti. Klassis oli  
Yamaha-d\index{Arvutid!Yamaha MSX}, MicroBee-d\index{Arvutid!MicroBee} ja 
Robotron 1715-d\index{Arvutid!Robotron!Robotron 1715}. Kolmes reas. Keskel 
MicroBeed,  ääres olid Yamahad ja akna ääres Robotronid. Yamahade peale käis 
tegelikult see \emph{run},  kõik tahtsid Yamahadesse saada, sest Yamahades olid 
mängud ja kihvtid mängud, onju. MicroBeed eriti ei huvitanud kedagi ja taga 
Robotronid oli ka nagu põnevad. Mina põhiliselt olin Robotronide peal. Seal sai 
progeda seal sai, ma mäletan, mingeid kooli referaate teha. Yamahadesse mina 
kunagi löögile ei saanud. 

Yamahad olid 3.5 tolliste diskettidega juba. 

See oli nagu järgmine level. See pidi olema 1988. aasta.

\question{Ja oligi nii, et uks tehti lahti ja kes istuma sai, sai istuma?} 

Jaa. Kui midagi ära lõhkusid, siis oli selleks päevaks \emph{ban} aga järgmine 
päev võisid tagasi tulla. Ma mäletan seda selle pärast, et ma ükskord istusin 
mingi Robotroni taha, kus oli mingi probleem klaviatuuriga ja ma võtsin teise 
masina klaveri. Sellel on mingi imelik pistik, millega klaviatuur masina külge 
käis ja ma panin selle kuidagi niimoodi kehvasti sisse, et rikkusin pistiku 
pin-id  kõveraks. Ja siis tuli minna taha ruumi  
ja  öelda, et näed,  niisugune jama. Ma ei mäleta, kumb neist tuli, kas Julius 
või Tali, vaatas, ütles, et \enquote{okei, tänaseks kõik,  tule järgmine 
teisipäev tagasi}. Nad nagu hoidsid korda.  Ürituse  
 lõpp sõltus sellest, millal nemad enam ei viitsinud olla ja tahtsid 
koju minna. Sest tegelikult nad olid mingid töötajad seal kateedris ja ma ei 
tea, mida nad seal tagaruumis ise tegid, ilmselt progesid. Aga, ühesõnaga, meie 
olime kõik jõnglased seal, \emph{average} oli mingi 13. Ma praegu mõtlen välja, 
et see pidi olema 1988.  aasta, sest 89 läks mu ema tööle 
Diagnostikakeskusse\index{Diagnostikakeskus}, millest ma pärast poole räägin ja 
sinna ma juba liikusin edasi. 

Ja kuidas nad siis korda hoidsid, oli toitekas. Jõnglased, arvutiruum, kõigil jube 
põnev, kes mida teeb , kes progeb, kes mängib mänge, kes häkib mänge, et teha 
sinna mingisugused oma tegelased sisse. Saad teisipäeva õhtul umbes õhtul kell 
kuus uksest sisse. Ametlik lõpp oli ka, aga tegelikult keegi ei pidanud 
sellest kinni. Selles mõttes see arvutiring lihtsalt oli, mind ei pandud mitte 
kunagi kuskile kirja, keegi ei õpetanud mitte midagi, ise teed. Ühel hetkel 
viskas kuttidel alati üle ja siis nad tegid tagant ruumi ukse lahti ja ütlesid 
\enquote{viie mintsa pärast toitekas}. See oli kõik, mis nad ütlesid. Viie 
mintsa pärast toitekas. See tähendas seda, et sul on viis mintsa aega mis 
iganes sul pooleli on ära salvestada, sest viie mintsa pärast emb kumb neist  
tuli,  ei hakanud nende jõnglastega midagi vaidlema, jalutasid kilbi juurde, 
tõmbasid, laks, peakilbi välja. Pimedaks. Ja see oli signaal, et nüüd tuleb koju 
minna. 

\question{Võru 1. Keskkoolis oli sama skeem ja kedagi ei huvitanud, mis selle 
arvutiga juhutub. Minu meelest ka ei juhtunud suurt midagi.}

Täiesti \emph{ruthless}. Väge ei juhtunud aga eks ilmselt oli oht, et kui 
kirjutamine jäi pooleli, siis ilmselt mingi \emph{disc corruption} võis 
juhtuda. 

See oli tegelikult äge periood. Sealt ma mäletan paari tüüpi uduselt, aga seal 
\emph{collaboration}-it või suhtlust oli nagu vähe. Oli justkui mingi 
generatsioon Yamahade juures, kes tundsid teineteist ennem ja kes hoidsid 
kokku, aga ülejäänud  igaüks  nokitses omasoodu. 

Aga see kõik oli ikkagi alles algus. Aga niisugune \emph{breaktrhough}, kus 
mina jõudsin paradiisi ja õndsusesse  toimus  tänu Mihhail Gorbatšovile. Sest 
Mihhail Gorbatšov, kes oli Nõukogude Liidu  kompartei esimees, otsustas 
perestroika käigus aastal 1986 või midagi sellist, et meditsiiniga üle Nõukogude 
Liidu on üldiselt probleem. Ja et kõige suurem probleem Nõukogude meditsiinis 
on see, et pannakse valesid diagnoose või diagnoosi jaoks info või analüüsid on 
ebatäpsed, valed või liiga aeglased. Seega tuleb investeerida infrastruktuuri, 
millega meditsiinipersonal saaks kiiremini teha ära olulised mõõtmised ja 
analüüsid. On see mingi südame EKG, on see vereanalüüs, on see siis mingi 
magnetresonantstomograafia. Nihukesed asjad, mis on täna põhimõtteliselt  
külahaiglas olemas. Selliseid asju polnud ja vereanalüüsi tegemine võttis mega 
kaua aega, \emph{result}-id tulid tagasi ebatäpsed, kõik sihuke kraam.  Ja see 
oli globaalselt Nõukogude Liidu suur probleem. Gorbatšov otsustas, et ta teeb 
sellise liigutuse, et kuna tal ei ole piisavalt valuutafondi\sidenote{Nõukogude 
elu valitsesid, laias laastus, kaks terminit: fond ja limiit. Fond ütles, kui 
palju midagi kellegi jaoks olemas oli ja limiit seda, kui palju midagi kas 
toota või tarbida tohtis. Ehk, \enquote{pole piisavalt valuutafondi} tähendab 
tänases mõistes \enquote{ei olnud eraldatud piisavalt valuutat}.}, et teha neid 
võimalusi igasse jumala haiglasse, siis igasse Nõukogude Liidu osariiki, 
näiteks Eestisse, pidi pealinna moodustatama selline asi nagu 
diagnostikakeskus. Kuhu  investeeritakse valuutafondi, ostetakse välismaalt 
\emph{top notch} tehnika sisse ja terve vastava osariigi analüüsid või uuringud 
tehakse ühes kohas. A la, kui sul on mingi kopsuhaigus, siis sind ravib, jah, 
sinu arst näiteks Mustamäe haiglas, aga Mustamäe haiglasse ei ole ressurssi 
osta mingeid kopsuanalüsaatoreid, vaid need ostetakse ühte kohta,  tehakse 
eraldi \emph{facility}, eraldi asutus.

No vot, ja Eestisse pidi ka tulema sihuke. Mis seal oli oluline oli see, et 
 tol ajal kehtis Nõukogude Liidule välismajandusembargo, ehk Nõukogude 
Liitu ei saanud sisse tuua lääneriikide moodsat tehnoloogiat, \emph{except} 
meditsiinitehnoloogiat, see oli okei. Mille pärast meil ei olnud PC-sid siin 
oligi see, et  embargo oli peal. Isegi kui mingid välkarid oleks tahtnud müüa, 
isegi kui siin oleks kellelgi valuutat  olnud tol ajal, ei saanud  
selliseid asju osta. 

Hakati siis Tallinnasse moodustama seda 
diagnostikakeskust\index{Diagnostikakeskus}. Ago Kivilo\index[ppl]{Kivilo, Ago} 
nimeline mees seda vedas, tänaseks on ta siit ilmast kadunud. Keskus pidi tulema 
sinna, kus praegu on vana Tallinna Panga maja Tallinna linnaosavalitsuse 
kõrval. Muidugi läks eht eestlaslikult kemplemiseks selle ümber, et kes 
saab keskuse juhiks, millise haigla kõrvale see üldse teha, et asukoht ei ole 
ikka hea, haiglatest kaugel ja nii edasi. Et äkki teha Mustamäe haigla kõrvale. 
Kivilo ütles, ma teen selle ise, ma teen eraldi asutusena. Mustamäe haigla 
ilmselt tahtis keskust umbes enda allasutuseks. Ma neid detaile ei tea, sest ma 
olin umbes 14 tol ajal, aga natuke tean selle pärast, et mina olin see mees, 
kes aitas Agu Kivilol teha kõiki powerpointe, mis tal oli vaja teha. Tol ajal 
see oli \emph{skill}. 

\question{Kas PowerPoint oli olemas tol ajal?} 

Kohe jõuan sinna. Miks meil oli PowerPoint olemas, miks oli niisugune maagiline 
kohta olemas nagu Diagnostikakeskus, oli siis see, et Eesti riigis selliseid 
asju nagu laserprinterid, PC-d, värvilised kuvarid, arvutivõrgud polnud olemas, polnud 
nähtud. Okei arvutivõrke oli, aga need olid pigem 
CP/M-ide arvutivõrgud ja sellised, Tehnikaülikoolis või niimoodi. Aga kuna 
Diagnostikakeskus oli meditsiiniasutus ja  nad ostsid Soomest 
meditsiinitehnikat, siis oli Soomes olemas  mingisugune OY, minu arust see oli 
Pekka OY. See oli reaalselt mingi omaniku eesnimi,  võib-olla oli mingi sõna 
veel. Ja tema, ma saan aru, oli põhimõtteliselt see vahendusfirma, kes vahendas 
Soomest igasugu kopsuanalüsaatoreid, vereanalüsaatoreid ja sihukest kama. Ma kujutan 
ette, et juriidiline skeem pidi ka suht keeruline olema, kuna see kõik liikus 
mingite Moskva valuutafondide kaudu. \emph{Anyway}, aga kuna see kaup liikus, 
siis selle kauba raames tuua Eestisse \emph{top notch} PC-sid polnud mingi probleem. 
Sest \emph{on paper} paisits see kõik nagu meditsiinitehnika. 

\question{Ta vist on võrreldes muu tehnikaga ka odav?}

Esimene magnetresonantstomograaf tuli sealt kaudu ju. Millega omakorda 
\emph{back} või \emph{side story} on see, et Gorbatšov eraldas valuuta, valuuta 
oli reaalselt olemas aga seda ei suudetud ära kulutada, sest kulus lõputu aeg, 
\emph{calendar time}, selle peale, et vaieldi, kuhu  keskuss ehitada. Aga 
selle tomograafi jaoks peab maja vundament juba spetsiifiline olema, et ei oleks 
värinaid ja vibratsioone. Sinna kulus aeg ja Nõukogude Liit hakkas juba lagunema 
aga valuuta oli ikka veel kontol. Ja oli nagu reaalne probleem, kuidas see 
raha kuhugi ära kulutada, enne kui ta kaduma läheb,  
Kivilo\index[ppl]{Kivilo, Ago} tegeles sellega. See tomograaf näiteks saadigi 
Eestisse kätte, see on tänaseni olemas. Kokkuvõttes ehitati keskus  Magdaleena 
haigla kõrvale. Kivilo sai selle tomogoraafi niimoodi kätte, et kas   
tarnijafirma või mingi vahefirma (neid detaile ma ei tea) sai sisuliselt 
\emph{prepaymenti} ja kunagi aastaid hiljem reaalselt tarnis selle seadme. 

\emph{Anyway}. Selle kõige käigus tuli ka arvuteid. Kuidas mina sinna 
sattusin, oli see, et mu ema läks sinna vereanalüsaatori peale tööle, ta on med 
taustaga inimene, apteeker. Ja küsis Kivilolt, et kas tema poeg võib siin 
arvutis käia, umbes pärast kooli. Kivilo ütles, et las poiss käib. Seal ma sain 
kokku selle mehega, kelle nimi on Mart Palmas\index[ppl]{Palmas, Mart} ja kes 
põhimõtteliselt on Eesti IT-sse toonud  mind ja Madis Kaalu\index[ppl]{Kaal, 
Madis}. Madis Kaalu ta püüdis kuskilt Tipi-kooli pealt kinni, kes oli sealt 
välja kukkumas, ütles, et \enquote{kuule, tule nüüd, ma panen su arvutite 
peale}. Nii et Palmas hakkas õpetama mind  progema, ennem olin oma käe 
peal harjutanud. Ja siis ma käisin seal niimoodi, pärast tunde. 

\question{Mida Mart Diagnostikakeskuses tegi?}

Diagnostikakeskuse arvutite hooldamiseks oli loodud väikeettevõte 
Skriining\index{Skriining}, mis on siiamaani olemas, Kalle 
Lotamõis\index[ppl]{Lotamõis, Kalle} asutas. Mart töötas seal programmeerijana, 
ma arvan. Aga noh, põhimõtteliselt, mida ta tegi oli, et hoidis neid arvuteid 
korras või kogu Skriining hoidis neid arvuteid korras. Ja mina sain seal 
hängida nagu \emph{for free}, ma ei olnud kuidagi juriidiliselt seotud. 
\emph{Except}, minu esimene üldse töö, välja arvatud TPL-id\index{TPL - Töö ja 
Puhke Laager. Nõukogude koolilastele pakutud võimalus suviti organiseeritult 
tööd teha ja elu nautida.} ja rohimised, oligi üks suvi seal, Mu esimene 
\emph{task} oli vedada maja peale laiali koaksiaalvõrk, mille otsa me panime käima 
Novell Network-i. Tuli paigaldada  arvutitesse võrgukaardid, häälestada, 
õiged IRQ-d, mingi soft tuli peale panna, kõik sihukesed asjad. See oli mu 
esimene suvetöö, ma mäletan.

\question{Kes sulle sellise ülesande andis ja miks oli tal alust arvata, et sa 
seda teha oskad?}

Eks ma olin esiteks sellel kevade läbi hänginud juba vähemalt. See oli 
tegelikult Kalle\index[ppl]{Lotamõis, Kalle}, Skriiningu juhi, otsus. 
\enquote{Kuule, et kas oleks mingit suvetööd?} \enquote{No tule, pane neid 
arvuteid siin kokku ja võta kastist välja, pane üles ja kui kellelgi on 
mingi mure, siis aita}. Me olime kõik selles Suur-Ameerika 18 majas. 

Ma tahtsin rääkida lihtsalt sellest, et see kontekst, kus me viibisime\ldots Ma 
rääkisin varem, et meil olid PowerPoint-id ja  laserprinterid. Aga, kujutad sa 
ette, et (ma täpselt ei tea muidugi, aga ma arvan) terves Tipi-koolis oli heal 
juhul suurusjärk 10 8086-t, ehk siis XT-d, ja ilmselt mustvalge 
\emph{display}-ga. Meil oli reaalselt üks ruum, mis oli triiki täis 
avamata arvuteid, mis olid juba kõik 80286-d, kõigil 40 mega IDE vinti ja VGA 
graafika. Lihtalt \emph{unpacked}. 

Nii et noh, kui näiteks Mamers\index[ppl]{Mamers, Tarmo} meile tööle tuli, 
siis ta \emph{unpack}-is omale kohe kaks tükki! Ühe peal jooksis BBS, teise 
peal tegid tööd, mis iganes see töö siis oli. Ega paljuski, kui ta panna 
tagantjärele konteksti, mida tänapäeval tööks nimetatakse, siis seal tööd 
tegelikult ei olnud. Selles mõttes, et meditsiinipersonali arvuteid oli 
suurusjärk kümme, neid hooldas suurusjärk kümme inimest, kes põhiliselt 
lihtsalt kaifisid seda, mis tehnika keskele nad sattusid, on ju. Ega me tegime 
muidugi kõik asjad ära, mis  teha oli vaja. Aga see oli see, kus ma 
arvan, mina sattusin reaalselt sellisesse niivõrd viljastavasse keskkonda. Mul 
olidki seal Mamers\index[ppl]{Mamers, Tarmo}, Palmas\index[ppl]{Palmas, Mart}, 
Hannu Krosing\index[ppl]{Krosing, Hannu} astus korra nädalas läbi. See oli ka 
koht, kus ma sain aru, et minust ei saa kunagi progejat. Sest ma mäletan nii 
hästi, kuidas ma nädal aega pusin millegi kallal ja siis tuleb Hannu, ma näitan 
talle selle nädalaga kirjutatud koodi ja ta võtab paberilehe ja sisuliselt 
kirjutab kahe reaga sellesama koodi. Turbo C-s.\index{Keeled!Turbo C}

\question{Leidsid ka kellega ennast võrrelda!}

Ma sain aru, milline on delta, onju. Võib-olla see ei olnud ainus põhjus, aga 
see oli üks koht, kus ma sain aru, et talendi või \emph{skill}-i vahe on ikka 
hüpersuur.

\question{Hannuga võrreldes on kellega iganes talendivahe väga suur!} 

Tõsi, tõsi.

Tol ajal põhiline projekt, mille kallal ma töötasin ja millel oli ka üks 
\emph{user}, oli selline programm, mille nimi oli clabel. Sellele ma ostsin 
ametlikult 25 krooniga juba Madis Kaalult\index[ppl]{Kaal, Madis} graafilise 
liidese \emph{library}, millega sai juba menüüd ja pop-uppe ja selliseid asju 
teha. Nimi oli SLACK, \emph{Slim} \ldots midagi sellist. 25, toona juba, 
krooni maksin selle eest, et mul oleks ametlik \emph{lifetime} litsents. 
Kirjutasin siis sihukese programmi nagu clabel. Programm tegi  sellist 
asja, et sul oli muusika (meil olid tol ajal kõigil kopeeritud muusika 
kassetid\sidenote{Vaata ka märkus lk \pageref{sisu!kassetid},}), sul oli kassette palju ja sa tahtsid omale normaalset andmebaasi, et 
mis kasseti peal sul mis lood on, mis bändid sul on. Pluss, sa tahtsid need 
välja trükkida sellisena, et sa saad nad \emph{fold}-ida ilusti ümber kasseti 
ja panna kasseti karbi kaane alla. Ta trükis kohe sellise paberi välja, et üleval serva 
peal oli näha, mis on A- ja mis on B-poolel ja suure külje peale trükkis 
A-poole kõik laulud ja B-poole omad. Pluss oli siis graafiline liides, kus sa 
said brausida ja edida ja printida. 

\question{Minu esimene arvutiga teenitud raha tuli täpselt samasuguse softi 
abil.}

No vot. Mul oli üks \emph{user}, kelleks oli Toivo Annus\index[ppl]{Annus, 
Toivo}. Väga kasulik \emph{user}, sest ta reaalselt kasutas seda ja tegi  mulle 
kogu aeg bugireporte.

\question{Kuidas sa Toivoga kokku said?}

Fido kaudu. Ma ilmselt \emph{upload}-esin selle softi kuskile BBS-i ja Fidos 
promosin, et mul on niisugune asi. Toivo hakkas kasutama ja hakkas mul külas 
käima, rääkima, mis ei tööta, mis töötab. Ma arvan, et tema oli 16, mina olin 
umbes 14. 

\question{Järelikult juba Diagnostikakeskuses sukeldusid sa Fido maailma?}

Ma ei mäleta, millal Fido tekkis, ma isegi ei mäleta, kumb oli enne, kas minu
 Diagnostikakeskusse minek või Fido tulek. Ilmselt 
oli nii, et ma algul läksin ja Fidot polnud ja siis millalgi tekkis. Täpselt 
sündmuste ahelat enam ei mäleta. Aga Fido ja BBS-id oli meil seal mingi hetk 
täitsa nagu \emph{bread and butter}. Mardil\index[ppl]{Palmas, Mart} oli oma 
\emph{node}, muidugi Mamersil\index[ppl]{Mamers, Tarmo} Mambox. 
BBSummer\index{BBSummer} tuli ka juba, esimene toimus kas äkki esimesel või  
järgmisel suvel. Diagnostikakeskusel lisaks 
kõigele muule oli ka humanitaarabina Rootsist saadud täiesti töökorras Volvo 
põhjale ehitatud kiirabiauto. Kuna diagnostikakeskusel polnud sellega 
midagi teha, sõitis sellega ringi Mart. Aga sellel autol olid kõik  
operatiivauto load olemas, vilkurid peal, täismäng, taga oli kanderaam, 
põhimõtteliselt pane inimene sisse. Sellega, ma mäletan, käisime mina, 
Mamers\index[ppl]{Mamers, Tarmo}, äkki ka Kaido Kärner\index[ppl]{Kärner, 
Kaido}, toomas Saku Õlletehasest BBSummeri jaoks õlut. Sellega oli hea vedada, 
hea suur. Esimene summer toimus Väänas Tugamanni Veskis, mina veel läksin 
lõpuks sinna mopeediga kohale. See oli, vaata, juba niisugune ülemineku aeg, 
kus poes polnud midagi saada. Ma ei mäleta, kelle tutvuste kaudu kuidagi saadi 
Sakuga kokkuleppele, et sealt sai otse õlut osta. 

\question{Kas see oli see kord, kui, nagu Mast\index[ppl]{Kaal, Madis} rääkis, 
õlut sai villitud Fanta kankudesse ja õllel oli apelsini mekk man?}

Täitsa võimalik. Mina olin selles vanuses, et  mina õlut ei  joonud, 14 või mis 
iganes, ja ei oska kommenteerida. Aga mäletan, et \emph{event} oli kihvt. Ja ma 
mäletan, et see esimene BBSummer oli ka koht, kus ma esimest korda kuulsin 
asjast nimega Internet ja asjast nimega emaili aadress. Tartu Füüsika 
Instituudis\index{Tartu Ülikool!Füüsika Instituut} oli ka mingi kamp, lausa 
mingi perekond itikaid, aga ma ei mäleta, mis nende nimed olid\sidenote{Tarmo Mamers\index[ppl]{Mamers, Tamo} arvab, et tõenäoliselt oli tegu perekond Pruulmannidega ehk muudestki juttudest läbi käiva Jaan Pruulmanni\index[ppl]{Pruulmann, Jaan} ja tema abikaasaga.}. Igatahes keegi 
nendest pidas ettekande ja ta oli tolleks ajaks juba käima pannud Interneti ja 
Fidoneti vahelise \emph{gateway} meili jaoks. Nii et põhimõtteliselt kõigil 
Fido inimestel oli sel ajal, ja sisuliselt \emph{effectively} kohe ka mul,   
emaili aadress. Ma ei mäleta, mis see tagumine ots oli, domeenid ja asjad, aga 
see ülejäänud ots moodustus sellest, mis \emph{node} küljes olid ja kes sa 
olid. Nii et teoreetiliselt sai saata mulle emaili ja mina sain saata välja. 
Aga ma ei mäleta, et ma seda praktikas kasutanud oleks. Mul oli 
too emaili aadress isegi kuskile märkmikusse üles kirjutatud, aga sellega 
polnud kellelegi saata. 

\question{Esimese faksiomaniku probleem. Aga järelikult siis BBSummeril oli  
hariduslik või sisuline sisu ka?} 

Jah. Oli ikka väga palju loenguid, või \emph{knowledge-sharing}-ut. Minu, 
14-aastase, arust oli väga äge. Mäletan Martiini\index[ppl]{Martiini|see{Rinne, 
Martin}}, Martin Rinne\index[ppl]{Rinne, Martin}, demos seda, mida saab teha ja 
mida nad teevad Amigaga. Ta oli Eesti Televisiooni\index{Eesti Rahvusringhääling!Eesti Televisioon} 
juures, tegi kõiki neid 
saatetiitreid ja asju ja demos seda poolt. Mulle täiesti uus maailm, ma polnud 
seda osa üldse näinud. Siis ma mäletan seda interneti-teemalist loengut, aga 
eks seal oli väga palju vaba suhtlemist ja jutuajamist ka. See oli see koht, 
kus ma esimest korda nägin paljusid inimesi, kellega ma olin umbes aasta juba 
suhelnud. 

\question{Suhelnud siis sõnumitega Fidos?}

Jah, olid jututoad, teemade kaupa, põhimõtteliselt nagu tänapäeval foorumid. 
Ta ei olnud kaugeltki \emph{real time} seepärast, et sa sisuliselt tõmbasid 
alla värsked sõnumid, lugesid need läbi, kirjutasid nendele valmis 
\emph{reply}-d ja siis  helistasid sisse ja \emph{upload}-isid oma valmis 
kirjutatud asjad. 

\question{Järelikult pidi sul olema mingi kliendisoft?}

Jah, nendega oli lihtne. Kõige keerulisem asi oli see, et pidi olema modem ja 
telefoniliin. Need olid nagu pigem tol ajal asjad, mida oli raske hankida. 
Niipea, kui me saime selle Novelli võrgu püsti, siis meil majas sees liikusid 
sõnumid sealtkaudu. Selleks, et mina lugeda ja kirjutada saaksin, ei pidanud ma 
oma masinas modemit omama, sõnumid läksid otse Mambox-i. Täpseid samme, mis 
seal vaheetapid olid, ma ei mäleta. Pigem ma mäletan, et kui ma sealt liikusin 
Salva Kindlustusse, kus me Toivoga moodustasime Salva Kindlustuse IT-osakonna, 
siis oli meil lihtsalt ühes masinas \emph{point}. 

\question{Miks te seda kõike tegite? Diagnostikakeskuse jaoks ei olnud seda ju 
vaja, kas teil lihtsalt oli aega ja tahtmist mängida või keegi visiooniga 
inimene andis teile ülesande võrk ehitada?}

Novelli võrk oli väga praktiline asi tol ajal, tänapäeval on sama praktiline 
asi teha igasse kontorisse internetiühendus. Ta andis ettevõttele väärtuse 
mõttes kaks asja: võrguketta (mina salvestan maha ja sina saad teisest 
arvutist kohe kätte) ja võrguprinteri. See on ju \emph{pre-Windows} aeg. Sa 
said omale võrguketta külge, said faile jagada ja  programmidele ligi, 
said printida. Laserprinterid olid kallid ja see oli hästi suur efekt, et sa 
said maja peale ühe laseri osta ja ükskõik mis arvutist sinna trükkida, see oli 
\emph{magic} tol ajal. Kõiki asju tuleb ju konteksti panna ja ma pean ka nagu 
mõtlema, kuidas esile tuua seda, mis oli tol ajal tolle aja kohta eriline. 

\question{Diagnostikakeskuses olid siis kuni keskkooli lõpuni?}

Tekkis tegelikult see asi, et Diagnostikakeskus\index{Diagnostikakeskus} oli 
ikkagi meditsiiniettevõte, Skriining\index{Skriining} oli selle küljes 
tütarfirma. Algul need väikeettevõtted pidigi moodustama mingisuguse 
riikliku ettevõtte juurde. Nii et, ma ei tea, mis aastal see täpselt oli, aga 
millalgi sai Skriining ennast Diagnostikakeskuse küljest lahti aktsiaseltsiks. 
Kõigepealt hakkasin  Skriiningus suvetööl käima, tegin  võrguhaldust. Üsna pea 
õnnestus mul ennast sebida  kooli kõrvalt palgale. Nii et kogu keskkooli aja 
kindlasti ma käisin sellises režiimis, et  pärast kooli kohe linna, 
Skriiningusse. Siis meil oli juba rohkem kliente, mitte ainult 
Diagnostikakeskus, vaid erinevaid meditsiiniettevõtteid. Põhiäri oli 
 arvutivõrk, on see siis mingi haigla või midagi muud. Näiteks 
Haigekassa\index{Haigekassa} arvutivõrgu vedamine, mäletan, oli üks  minu 
töid. Ta asus tol ajal seal, kus praegu on Prantsuse Lütseumi 
algkool\sidenote{Hariduse 8, Tallinn.}. Ma mäletan seda sellepärast hästi, et, 
esiteks, tänapäeval keegi ei laseks sedasi vedada, kaabel veeti lihtsalt pinna 
peale. Tõmbasin kaablit ja lõin klambreid seina peale, 
tänapäeval tunduks robustne. Teiseks, töövahenditeks oli haamer, klambrid, 
kaabel ja midagi, mis meenutab kaugelt vaadates trelli. Aga tänapäeval keegi 
ilmselt ei kavatseks sellega ühtegi auku puurida. Mina pidin sellega kõik augud 
puurima. Puure trelli otsas oli ka üks, mis meenutas puuri, ja sellega tuli 
suvalisest materjalist läbi minna. See toimus sellisel nühkimis-meetodil. 

\question{Kas puuri diameeter oli suurem, kui kaabli diameeter?}

Jah, vähemalt see oli hea. Põhimõtteliselt ainuke lootustandvad kohad, kust 
õnnestus läbi minna, kuna puuri pikkus ei olnud ka väga pikk, olid ikkagi 
uksepiidad. Koaksiaalkaabel on ju niisugune, et see peab läbi kogu maja 
põhimõtteliselt moodustama pideva \emph{loop}-i. Ta saab kuskil alata ja 
kuskil lõppeda, aga ta ei saa katkeda ja teda ei saa olla mitu. Nii et sa 
pidid nagu mõtlema, et nendesse tubadesse kõigisse on arvutivõrku vaja, et 
kuidas see ahel nagu teha. Pidid terve toa läbi jalutama ja uuesti kaabliga välja 
minema. Ja, loomulikult, kui ta kuskilt katki läks, oli kogu võrk maas, 
seepärast, et ta ei ole nagu \emph{twisted pair}-i võrk, kus mingist ruuterist 
või \emph{switch}-ist läheb kaabel seadmini ja kogu moos. 

Ühesõnaga, Skriiningus ma käisin palgatööl. Ühel hetkel kolisime Suur Ameerikast 
ära, Skriining sai nii-öelda oma ruumid ja me olime pikalt, minu arust ikkagi pea 
kogu mu keska aja, Estonia puiesteel. Seal, kus 
praegu on Mati Mobiiliäri\sidenote{Estonia puiestee 5, Tallinn.}, 
põhimõtteliselt kohe Estonia teatri vastas. See oli äge aeg. 

Aaslaid\sidenote{Henn peab silmas Andrus Aaslaidu\index[ppl]{Aaslaid, Andrus}} elas kontoris. Tal oli mingi korter ka kuskil, aga ma ei tea mis iganes 
põhjusel ta ei viitsinud seal käia. Mina ka, tulin koolist ja tiksusin viimase 
bussini seal kas  mingeid oma programme kirjutades või siis mingeid töö 
\emph{task}-e tehes. Töö \emph{task}-id olidki siis võrgu laiali vedamised,  
võrkude hooldused, serverite paigaldamised. Need muidugi käisid maha vahepeal, 
tuli joosta kuskile teise linna otsa, asjad uuesti käima ajada. 

Kõva \emph{bread and butter} Skriininugs oli tol ajal see, et arvuteid pandi  
komponentidest kokku. Klient ütles, et \enquote{andke mulle üks arvuti}. Lepiti 
mingis enam-vähem konfis kokku, aga ma ei ole kindel, et see teine osapool, kes 
tellis, teadis, mis need parameetrid on. Otsiti linna pealt komponendid 
mingitelt partner arvutifirmadelt kokku, umbes kust mäluplaat, kust kõvaketas, 
kust korpus. Mina keerasin selle kõik kokku, panin sinna ööseks mingid 
testid peale jooksma, hommikuks, kui kõik toimis, läks arvuti karpi, Skriiningu 
kleepekas peale ja  kliendi juurde. Siis panid ta üles, näitasid inimesele (kelle 
jaoks see enamasti oli elu esimene arvuti), kust sisse 
lülitatakse,  turbo nupp\sidenote{Vanematel PC tüüpi arvutitel oli küljes nupp 
sildiga \enquote{turbo}. Vastupidiselt ootusele, sellele vajutamine vähendas, 
ja mitte ei suurendanud arvuti töökiirust. Asi oli selles, et 8088 
protsessoriga arvuti jaoks loodud tarkvara (eriti mängud) olid mõnikord 
sõltuvad masina 4.77 Mhz taktsagedusest ja seetõttu uuematel kiirematel 
masinatel korralikult ei käinud. Võimaldamaks tagurpidi ühilduvust, lisati 
riistvaraline võimalus arvutit aeglasemaks teha.}, et seda ära vajuta, seda 
pole vaja, kogu aeg las olla sees. Ja mida sa siis arvutiga teha saad, kuidas 
võrku saad logida, enamasti oli mingi Novelli võrk. Kõige tavalisemalt  oli 
kasutajal vaja saada ligi kas mingile raamatupidamisprogrammile, mida ka 
Skriining ise kirjutas, või siis olid Skriiningu sihukesed enda \emph{custom} 
mõnes mõttes nagu haiglate infosüsteemid. Oli see siis mingi kaardiregistrite 
süsteem või midagi sellist. Tegid mingi koolituse ja\ldots 

Sedasi läks keskkooli osa. 

\question{Kas see kõik õppimist ei hakanud segama?} 

Ei hakanud. Õppimine ei seganud seda pigem. Ega ma nüüd mingi viieline ei 
olnud, tol ajal oli mu elu ikkagi, tagantjärele mõeldes, väga IT poole kaldu. 
Mind see kooli asi  absoluutselt ei huvitanud. Mitte, et ma ei saanud 
aru aga ma tegin \emph{bare minimum}-i, et saaks kähku arvuti taha. Tundus, et 
kõik põnev asi toimub seal. 

Ja mitte ainult arvuti taha, vaid, kuna me olime nii tsentraalselt keskkohas, 
siis Skriiningu kontor oli nagu läbikäiguhoov.  Kogu aeg keegi astus uksest 
sisse, oli see siis Tanel Raja\index[ppl]{Raja, Tanel} või Hannu 
Krosing\index[ppl]{Krosing, Hannu} või keegi teine. Ajas juttu, said jälle 
midagi teada, mida tema oli kuskil näinud või kuulnud. Mast\index[ppl]{Kaal, 
Madis} oli ju Forekspangas kohe seal üle hoovi, tema käis külas. Väga 
sotsiaalne oli see elu tol ajal. Tänapäeval see on kõik kuidagi onlainis, tol 
ajal oli see kõik ka IT-meeste vahel üsna \emph{offline} mingis mõttes. 

\question{Aga BBS-id?}

BBS-id olid olemas, aga minu arust sel ajal Fido vunk hakkas juba hääbuma. 
Sellel oli ilmselt mitu draiverit. Üks oli see, et kapitalism jõudis kohale, 
tööd oli vaja teha. Ega ei saanud päevad läbi lihtsalt istuda ja häkkida 
mingeid arvuteid, et umbes kui palju ma suudan mälu siin efektiivsemaks panna 
või kui ilusti ma oskan oma faile pakkide ketta peale niimoodi, et nad 
võimalikult kiiresti \emph{access}-itavad oleksid. Selliste asjadega ühel 
hetkel enam ei olnud aega tegeleda, vaid tuli teha reaalset tööd. Mis iganes, 
telliti siis sult mingeid progemist, mingit arvuti kokkupanemist. Ma arvan, et  
elu läks  tõsisemaks ja selle tõttu läks  suhtlemist nagu vähemaks. 

\question{Aga füüsiliselt ikka üksteisel külas käisite?}

Käisime, aga ka see hakkas tegelikult ühel hetkel hääbuma. Kui ma nüüd 
tagantjärele mõtlen siis Skriiningus\index{Skriining} see veel nii oli. Kui me 
Toivoga\index[ppl]{Annus, Toivo} juba Salva Kindlustus IT-d tegime, siis ma ei 
mäleta, et seal siukest nagu hängimist või kohvitamist oleks olnud, et keegi 
oleks külla tulnud ja oleks olnud aega nagu väga pläkutada. 

\question{Korraks BBS-ide ja Fido juurde tagasi minnes, oskad sa mõnda näidet 
tuua, mis kohtades, mis teemalistes tubades sa juttu rääkisid?}

Ma nüüd väga kaudselt mäletan, aga minu arust oli olemas mingid siuksed ruumid 
nagu umbes EW.NALJAD, kus lihtsalt keegi jagas mingeid anekdoote, sihuke väga 
\emph{social}. EW.JUTUTUBA  oli sihukene \emph{general topic}. Ikkagi väga kõva 
diskussioon käis igasugustes tehnilistes 
\emph{channel}-ites. Ma kahjuks enam ei mäleta, mille kaupa need olid. 
Kusjuures eal ei olnud tegelikult ainult ju eesti kanalid, sa said 
\emph{subscribe}-da tegelikult ju globaalsetes kanalitesse ka. Näiteks mäletan 
kindlasti, et  ma mingil hetkel lugesin  Novelli halduse ja adminimise kanaleid 
ja need olid  globaalsed,  tänapäeva foorumi laadsed asjad. Need olid 
teemapõhised, aga kus ma kõige aktiivsemalt\ldots 

Päris algul see Eesti Fido ringkond oli suurusjärk 40-50 inimest, täitsa 
\emph{manageable size}. Muidugi ta läkski ühel hetkel  läbuks kätte, kui 
maht kasvas ja lisandus väga erineva taustaga inimesi. Näiteks Salvas, kui meil 
oli \emph{point} niikuinii püsti ja Novellis \emph{access}-itav, istusid seal sees, ma 
ei tea, sekretärid ja kes iganes, kellel oli aega. Varieeruvus läks väga 
suureks, inimeste taust ja huvid läksid väga erinevaks. 

\question{Ei olnud enam nii elitaarne klubi?}

Elitaarne\ldots Fido ei olnud minu arust kunagi kuidagi suletud klubi,  ma ei ole 
kunagi tajunud seda, et see on mingisugune nagu salajane või mingisuguse 
erilise müstilise ligipääsuga, vaid pigem lihtsalt neid inimesi oli algul vähe. 

Ivar Zarans\index[ppl]{Zarans, Ivar} sai ju Ehinaga\index[ppl]{Zarans, Ehin} 
Fido kaudu tuttavaks, sellest ajast koos elanud, ühised lapsed suureks 
kasvatanud. Zarans oli, ma võib-olla teen talle nüüd liiga, aga tuntud 
poissmees Tallinnas. 

Ühesõnaga Fido hakkas ühel hetkel minu elus küll kuidagi kõrvale jääma või 
hääbuma ja mul on sihuke tunne, et sinna tekkis nagu rohkem \emph{traffic}-ut, 
rohkem inimesi. Tegelikult ka  üritused hakkasid ära jääma. BBSummerid, 
mis algul toimusid regulaarselt, olid BBWinterid, onju. Ja kuna  seltskonna 
taust  läks väga varieeruvaks, siis nähtavasti enam ei oldud nagu nii ühiste 
huvidega võib olla. 

\question{Tulles tagasi sinu isiku juurde. Tolleks hetkeks olid sa juba 
leidnud, et kuna maailmas on olemas Hannu Krosing, siis sina enam 
programmeerida ei taha?}

Jah, ma ei mäleta isegi, kuidas see nii selgelt läks. Aga kuidagi ma tööalaselt 
läksin pigem nagu mujale. Kõigepealt võrkude adminimine ja haldus, see oli 
kuidagi mulle väga jõukohane ja ma sain aru, kus ma väärtust lisan. Samas 
progemises\ldots. Tol ajal ka progemises võib-olla nõudlust, vähemalt ses osas, 
kuhu mina nägin, et ma saaks nagu pakkumist esitada, oli vähe. See  buum, kus 
oli vaja vett kõrbe viia, oli ikkagi see, et kuidas arvutid võrku saada, 
arvutid tööle saada, kuidas üldse arvuteid kokku panna. Paljud inimesed, kes 
oskasid arvuteid progeda, ikkagi oma leiva teenisid sellist tüüpi asjadega. 
Arvuteid kokku panime ja konfisime me Skriiningus  minu arust küll kõik. 
Ega see arvuti kokkupanek ei olnud ka nii lihtne, nagu ta tänapäeval on. No 
nüüd just poistele tegin, või aitasin oma poistel, mänguarvutid  kokku panna. 
Neil olid läpparid, ütlesid, et  on lahjad, tahavad \emph{desktop}-e ja 
siis ma huvi pärast tellisin komponendid ja tegin nendega koos. See on 
tänapäeval super lihtne, põhimõtteliselt sul ei ole pistikut võimalik valesti 
panna, ta läheb ainult ühte kohta kogu süsteemis. Tol ajal sa pidid ikka väga 
täpselt teadma, mis \emph{jumper}-id panna, kuidas, kuhu panna pistik ja kui sa 
tegid seda valesti siis masin kärssas lihtsalt läbi. Ja ei olnud olemas 
Google'it. Sul oli kaasas dokument või dokumendilaadne asi, kust nagu leiutada, 
et mis \emph{jumper}-itega see kõvaketas töötada võiks, mida sa ilmselt nägid 
esimest ja viimast korda, sest  ka  juppide tarned Eestisse käisid suhteliselt 
naljakal viisil. 

\question{Kes tõi kohvris kuskilt Singapurist või jumal teab kust.}

Sellega seoses on mul muideks üks hea lugu. Mul olid ikka juba autojuhiload, ma 
pidin järgult üle kaheksateistkümne olema. Ja oli niimoodi, et oli üks habemega 
mees, kes pidas siin Mustamäel arvutifirmat, mille nimi mul pole enam meeles, 
aga ta läks Estoniaga põhja, seda ma mäletan. Sihuke suur mees nagu karu oli. 
Ja siis oli Kalle Lotamõis oma Skriininguga\index{Skriining}. Emb-kumb neist, 
info liikus tollal faksidega, said Hiinast faksi, et on soodsalt pakkuda mälu 
SIMM-e, mälumooduleid. Hind oli röögatult hea. Nad kahe peale, oli mingi 
miinimum \emph{order},  ja ilmselt kuskilt veel  laenates ajasid raha kokku, 
tegid  ülekande ära. Pappkast jõudis kohale, jõudis tolli. Ma mäletan seda 
sellepärast hästi, et meie Palmasega käisime kastil tollilaos järel. Ja kurb 
oli siis pilt, kui selle pappkasti tegid lahti ja seal sees oli ainult 
vahtplast. Seda raha muidugi enam mitte kunagi ei nähtud. Sinna pandi suur raha 
alla, Skriining ikka lakkus neid haavu, peab Kalle käest küsima, kui kaua 
täpselt, aga kindlasti rohkem kui kuu ja pigem aasta. Ja samas, kui seal oleks 
olnud \emph{legit} asi, siis  oleks olnud muidugi vinge marginaal kohe. Väga 
sihuke \emph{cowboy times}. 

\question{Kuidagi mulle jäi kõrva, kuidas sa ütlesid, et sulle tundus, et sa 
lisad niimoodi rohkem väärtust. Kas sa tõesti mõtlesid juba tol ajal sellest, 
kuidas sa saad kasulik olla?} 

Võib-olla lihtsalt see, et mis sul nagu välja kukub. Mulle tol ajal juba 
tundus, et mul kukub hästi välja tehnoloogia ja inimeste vahel  liimiks 
olemine. Et ma lähen selle inimese juurde, kes kunagi pole ühtegi arvutit 
näinud, ma pakin talle selle lahti, panen tööle ja näitan, kuidas käib. Ma 
olin kõrvust tõstetud sellest, et ma sain talle kasulik olla. Tema sai oma tööd 
hakata nüüd tegema hoopis teistmoodi kui varem. 

\question{Ahjaa, sul ei olnud mitte abstraktne klient  vaid konkreetne 
inimene, kellel läks nägu särama!}

Jah. Sealt hakkas tulema ka  esimene niisugune 24/7 kogemus. Servereid, okei, 
öösiti enamasti õnneks küll ei kasutatud, aga need olid tegelikult ju 
ärikriitilised ja kui nad läksid maha, siis tuli väga kiiresti kohale jõuda ja 
nad väga kiiresti tööle saada. Pluss siis selle väga vihase kliendiga tegeleda. 
Talle seletada, mis juhtus. Nende jaoks olid need mingid maagilised kastid 
nurgas. Kuidas sa siis inimarusaadavas keeles seletad, mis juhtus ja miks sa 
arvad, et seda uuesti ei juhtu.

\question{Ja miks see sinu süü ei ole!} 

Või siis, et miks ma arvan, et tõenäosus, et see kohe nüüd uuesti juhtuks, on 
väiksem. Eks seal oli tihti mingeid raua probleeme, voolukõikumisi, miljon 
asja, ma täpselt isegi ei mäleta, mis seal oli. 

\question{Kõik käis ju tol ajal tati ja teibiga kokku, kellel oli serveriruum?}

Ei, ei, sellist asja polnud olemas. See oli liiga moodne sõna selle aja kohta. 
Need muidugi tekkisid millalgi aga tol ajal ei olnud. Valiti, et kus oleks 
niisugune puhas ruum, et otseselt vett ei tilguks, et  veeavarii tõenäosus 
oleks väiksem. Tihti mingi raamatupidaja kabinet või mingi sihuke, kus ta kõige 
loogilisem panna oli. 

\question{Ja ühel hetkel sa läksid Salvasse?}

Jah, Skriiningust ma sattusin juba Salvasse\index{Salva Kindlustus}. See oli 
see hetk, kus Mast\index[ppl]{Kaal, Madis} oli juba 
Foreksis\index{Pangad!Forekspank}. Ma veel mäletan hästi seda, et 
Toivo\index[ppl]{Annus, Toivo} kutsus mind Salvasse ja Madisega oli ka juttu, 
et äkki minna Forekspanka. Ma ei mäleta isegi, mis põhjusel see oli nii, et sai 
Toivo kasuks otsustatud. Salvas oli lihtne. Toivo ise käis ülikoolis, temal 
polnud aega sellega \emph{full time} tegeleda ja ta otsis kedagi, kes oleks 
päeval kontoris kohal. Midagi arvutivõrgust oli juba olemas, aga  
ettevõte kasvas  kiiresti, nii et  oli vaja esiteks tööjaamasid ja võrku 
hooldada ja teiseks inimesi \emph{support}-ida. 

Ajad läksid kogu aeg kiiresti moodsamaks. Meil oli mitte enam 
Novell 3.11, vaid 3.12. Laserprinterid olid mitmel korrusel. Ma 
mäletan, minu projekt oli vedada maja peal laiali kaablikanalid nii, et kaablid 
polnud enam lihtsalt naelaga seina peale löödud, vaid käisid ilusasti 
plastikkanali sees. Suur ja äge asi oli internetipanga eellane telefonipank. 
See oli siis niimoodi, et minu arvutis oli modem ja telefoniliin. Ja kuna me 
olime  Novelli võrgus, siis raamatupidaja sai oma arvutis maksed ette 
valmistada, sisetelefoniga mulle helistada, et nüüd on kõik valmis ja mina 
tegin  sessiooni oma modemiga Hansapanka\index{Pangad!Hansapank}. Ülekanded 
läksid üle ja samal ajal tõmbasime ära panga väljavõtte. 

\question{Hansapangal Telehansa oli siis vist tõesti juba olemas.}

Ma ei mäleta, mis selle toote täpne nimi oli, aga see käis modemi teel ja mingi 
\emph{fat client} tuli omale installida, millega sai tõmmata panga väljavõtteid 
ja teha ülekandeid. Arvuti ekraanil oli ikkagi ülekandevorm, mille täitsid ära. 
Ja \emph{roles and rights} oli juba olemas. Näiteks mina sain teha  
pangasessioone, aga ei saanud teha ülekandeid. 

\question{Aga tõenäoliselt see fail ei olnud kuidagimoodi krüptitud, nii et 
mõningase vaevaga oleks saanud raha kanda ka kusagile mujale?}

Seda ma kusjuures peast isegi ei mäleta, kui kõva see \emph{security} seal oli. 
Mul on nagu see \emph{awareness} olemas, et seal ikkagi esimesed jäljed 
\emph{roles and rights}-ist olid juba olemas, selle peale juba mõeldi. 

Aga nagu töö või protsessi mõttes, see oli \emph{huge step forwards}. Vanasti 
ju raamatupidajad tiksusid panga vahet kogu aeg. Nüüd ei pidanudki 
põhimõtteliselt pangas käima. Ainult  sularaha oli vaja ära viia. Aga see, et 
palju meil kontol laekumisi on, palju kontojääk on, sellised asjad 
\emph{magically} olid tema arvutis olemas põhimõtteliselt iga kell kui ta  
tahtis. 

See oli ka see hetk, kus Salva tegi selle otsuse, et enam ei maksta sularahas 
palka, mis oli tol ajal ju tavaline, vaid Salva Kindlustus avas kõigile 
töötajatele Hansapangas kontod ja deebetkaardid. Konto avamine ja deebetkaart 
olid nii kallid asjad, et ilmselt kui oleks jäetud  töötajate teha, siis enamus 
poleks selle projektiga kaasa tulnud. Palju lihtsam oleks olnud palk sulas iga 
kuu välja võtta, kõik olid sellega harjunud. Aga siis hakkas palk reaalselt panka tulema, 
ja pidi leidma ATM-i, kust palk korraga või jupikaupa välja välja võtta. 

\question{Miks see kõik Salvale kasulik oli?}

Et saaks sulahraga mässamisest lahti, et kõik oleks \emph{clean}-im. Ega 
kellelegi, ilmselt eriti mitte raamatupidajale, ei meeldinud sellega tegeleda. 

\question{Mille peal Salva peamine äriprotsess jooksis?}

Hea küsimus. Udune mälestus on, et minu arust meil oli mingi naljakas 
Inglismaalt ostetud programm, nime ei mäleta. Küll aga ma mäletan seda, et me 
käisime  Tõnu Laagiga\index[ppl]{Laak, Tõnu} kahekesi mingis Lõuna-Soome 
kindlustusseltsis vaatamas nende infosüsteemi plaaniga see osta. Ost 
jäi kokkuvõttes katki, aga seda käiku ma mäletan. Ajastu näitena oli see, 
et Toivo\index[ppl]{Annus, Toivo} ei saanud sinna kaasa tulla sellepärast, et 
temal ei olnud välispassi aga minul oli.

\question{Miks sul välispass oli?}

Miks mul välispass oli, seda ma ei mäleta. Olin kuskil äkki spordi pärast 
võistlemas käinud? See pidi olema vist ikkagi veel 
vene passiga. Või sai see olla Eesti pass ja mul oli viisa aga Toivol polnud? 
Mingite paberite tõttu Toivo ei saanud tulla igatahes. Aga Soomes, oli neil 
ikka sihukene \emph{proper} infosüsteemi moodi asi. 

\question{Ärme siit nüüd kiiresti üle lähme. Kas sa tegid sporti ka?}

Pigem ma sel ajal just enam ei teinud. Ma olin kooli ajal teinud 
orienteerumist ja kui ma tulin IT-sse, siis jäi see katki. Nii et sisuliselt ma 
vahetasin  spordi IT vastu natuke Nõukogude Liidu lagunemise tõttu, natuke, 
sest arvutid tulid. Orienteerumine oli meil väga tugevalt finantseeritud Saue 
sovhoosi\index{Saue sovhoos}\sidenote{Täpsemalt V. I. Lenini nimeline 
köögiviljakasvatuse näidissovhoos.} poolt ja kui sovhoos ära kadus \ldots Minu 
jaoks pigem oli see, et IT tuli varem sisse aga tegelikult treening-grupp vajus 
ka laiali.

\question{Aga sa teed ju praegu ka?}

Jah, pärast sõjaväge hakkasin uuesti tegema. Sõjaväkke ma sattusin aastaid 
hiljem, kui ma olin juba väga vana.  Ma algul ikkagi olin ülikoolis, tol ajal 
ülikool vabastas sõjaväest või pikendas. Aga oli vaja  nii palju tööl käia, 
et  ei jõudnud ülikoolis käia ja  lõpuks ma kukkusin  ülikoolist välja. Siis 
jõudsin veel olla mõnda aega nii, et ei saadetud sõjaväkke aga lõpuks ikkagi 
läks asi tõsiseks  ja tuli ära käia. 

\question{Mis sa ülikoolis õppisid?}

Infosüsteemid, ma arvan, oli TPI-s\index{Tallinna Tehnikaülikool!Informaatika}. 
Käisin koolis Salva kõrvalt. Oligi nii, et keska lõppedes mul oli koolist ja 
õppimisest töö kõrvalt nii suur tüdimus peal, et ma olin endale lubanud, et ma 
kõigepealt teen aasta aega nüüd ainult rahus tööd. Ma jätsin aastase nagu 
vaheaasta ja siis läksin TPI-sse. Aga see oli tegelikult viga, sest siis ma 
olin juba niivõrd \emph{vested} sellesse töö-\emph{mode}-i, et koolis 
käimisest ei tulnud eriti midagi välja. See oli nagu kõige madalama 
prioriteediga asi ja pigem käis kummi venitamine, kuni lõpuks mingi kahe või 
kolme aastaga eksmatt tuli. 

Pärast sõjaväge ma tegin muidugi EBS-is baka ära. EBS tegi omale IT-juhtimise 
eriala, see oli muidugi juba kahetuhandendatel juba. Ma olin esimeses lennus 
koos Alek Kozlovi\index[ppl]{Kozlov, Alek} ja  kambaga. Tegime omale täitsa  
\emph{proper} viieaastase baka veel, kusjuures, nüüd on kolmene baka. See oli 
ka tegelikult väga kihvt, aga see on juba teine ajalugu. 

\question{Mis sa pärast Salvat tegid?}

Salvast ma sattusin sinna kohta, kus ma nüüd saan uhkustada, et  töötasin seal 
koos Eesti Vabariigi presidendiga. Jälle ühe sugulase tõttu. See sugulane 
rääkis mulle, et \enquote{üks firma, kes tegeleb sidesüsteemidega, otsib 
inimest, et kas tahaksid rääkida}. Ja miks ka mitte. Ma olin Salvas ma ei 
mäleta, mitu aastat olnud juba. Sain kokku mehega, kelle nimi oli Rene 
Maksimovski\index[ppl]{Maksimovski, Georgi-Rene}. Kes on siis tänane presidendi 
abikaasa ja kes oli  sellise ettevõtte omanik, kes pani Eestis üles Siemensi 
telefoni keskjaamasid. Sihtgrupp oli põhimõtteliselt suured ettevõtted: pangad, 
riigiasutused. See oli nagu \emph{next phase} selles mõttes, et kui 
personaalarvutid olid raamatupidamisosakondades olemas, siis aastal umbes 95 
oli see aeg, kus oli sidet vaja. Oli lihtsalt vaja võimalust helistada nii 
majas sees kui välismaale nii, et ei krõbiseks ja saaks kaugekõnesid teha. See, 
kuidas me tol ajal Siemensi telefoni keskjaamasid paigaldasime ja hooldasime 
oli samamoodi sihukene vee viimine kõrbesse. Täna polegi lauatelefone nagu 
õieti kellelegi tarvis, mobiiltelefonid teevad asja ära. Aga see oli veel enne 
mobiiltelefonide aegne aeg ja lihtsalt see, et sul oleks igal inimesel laual 
telefon, millega saab helistada maja piires või majast välja ja majast väljast 
saab talle otse laua peale\ldots 

\question{Jagada mingit väikest hulka telefoniliine!}

See oli vinge etapp, aga ta oli IT-st kaugemal, telekomi maailm. Haaberstist\index{Haaberst} 
edasi ma sattusingi tegelikult juba Uninetti\index{Uninet}. Seal ma tegin kaasa selle aja, 
kui Uninet tuli telefonivõrgu turule, tolle telefoni keskjaama paigaldus siia 
Eestisse ja sealt omakorda Elisasse\index{Elisa}, Elisast Skype'i\index{Skype}. 

\question{Kui ma su juttu kuulan, jääb mulje, et sul on olnud alati selline 
kaabli tõmbamise töö. Aga nii kaua, kui mina sind tean, on su tegevuse 
subjektiks eestvedaja või juhina pigem inimene kui kaabel. Mis hetkel sul see 
vahetus toimus ja kas sa üldse näed siin vahet?}

Selline kaabli vedamine oli see jah, kust mu karjäär nagu alguse sai, 
kaablitööd ma ei oska tegelikult üldse hästi. Tegelikult periood kuni Haaberstini oli 
sügavalt ikkagi arvutite ja arvutivõrkudega  seotud ja kogu see Haabersti ja 
Unineti periood oli pigem ikkagi keskjaamade progemine. Ja, noh, kui keskjaamad 
olid valmis progetud ja töötasid, siis nende laiendamine või siis mingid 
veasituatsioonid. Ka hästi niisugune 24/7. Kui Jõhvi piirkonna politsei või 
misiganes side maha läheb ja seal piirkonnas 112 ei tööta, siis on päris suur 
probleem. 

\question{Aga inimesed?}

See algas Haaberstis\index{Haaberst}. Seal  organisatsiooni kasvades ühel hetkel oli vaja 
formaliseerida ja, ma isegi ei mäleta, kuidas, aga keegi ilmselt tegi mulle 
siis selles osas ettepaneku. Et ma võiksin hakata teisi insenere juhtima. Ma 
kasvasin sisuliselt tiimist välja tiimi juhiks. Meil oli tehnikaosakond, sain 
tehnikaosakonna juhiks. Põhimõtteliselt on kindlasti murdepunkt  see, kus sa 
võtad vastutuse mitte ainult enda töö eest, vaid sa võtad vastutuse teiste 
inimeste töö eest. Sealt edasi on  karjäär edasi paraku peaaegu kogu aeg 
olnud kas siis tootejuhtimine, projektijuhtimine või inimeste juhtimine või 
nende segu. 

\question{Miks \enquote{paraku}?} 

Alati oleks ilus minna tagasi spetsialisti liistude juurde, kus ma saaks 
vastutada ainult oma töö eest. 

\question{Oleks või?}

Kõvasti lihtsam oleks. 

\question{Aga miks sa pole läinud?}

Pole julgenud vist, ma pean mõtlema. 

Okei, üks põhjus on kindlasti see, et ma arvan, et ma olen oma kompetentsi 
kõvasti kaotanud. Jah, see on võib-olla kõige õigem vastus. Nüüd, juhtides 
tehnilisi tiime on see siis Skype'is või Elisas või siin Fleep'is\index{Fleep}, sa näed 
seda, kui andekaid tehnilisi talente on tegelikult olemas. Ja siis sa näed, et  
nende tehnilistes oskustes ei ole sa nagu üldse konkurentsivõimeline. See on 
üks.

\question{See on väga tuttav tunne, jah.} 

Ja teine kindlasti on nagu illusioon sellest, et  inimestevaheline suhtlus ja 
koordinatsioon on asi, milles ma olen mingid \emph{skill}-id omandanud ja ma 
parem siis nagu sõidan nende peal. Ma arvan, et see vist on see vastus. 

\question{Sul tuli see pärast esimest otsust nagu loomulikult?}

Jah. Kuigi tegelikult Uninetti  ma läksin ka ikkagi seda suurt telefoni 
keskjaama paigaldama ja haldama. Haaberstis olid  ettevõtete keskjaamad, 
väikesed. Uninetiga me panime üles selle, mis on nii-öelda telekomi keskjaam. 
See, mille külge  väiksed klientide keskjaamad käivad käivad ja mis siis 
ühendub luba SS7-ga ülejäänud telefonivõrku. Meil olid ühendused nii Soome 
Finneti  kui Eesti kõigi operaatoritega. See oli ka väga põnev aeg. Aga seal ma 
paraku jälle kasvasin  ikkagi tehnikatiimi või siis selle võrguoperaatorüksuse 
juhiks. Ja pärast Elisas juhtisin mobiilivõrku, kus oli ka raadiopool sees. 
Ikkagi sisuliselt juba inimeste ja tehnoloogia koordineerimine. Enamasti selles 
rollis on ikkagi ka mingi tehnoloogia strateegia pool sees. 

\question{See kõik kuidagi seletab väga hästi seda, miks ma sind hästi 
praktilise inimesena tean.}

Ei tea jah, minu arust mul sihukest teadlikku karjääri planeerimist pole kunagi 
olnud. Pigem on see, et oled sattunud mingitesse väga huvitavatesse tiimidesse 
või kollektiividesse, need kõik arendavad sind kuhugi suunas ja annavad  
mingeid kogemusi. Ja alati avavad uksi kuskile järgmises suunas. Ma ei ole ise 
kunagi  väga teadlikult oma karjääri arendanud või mingeid samme ette mõtelnud, 
et kuhu ma tahaksin jõuda. Pigem olen alati otsinud selliseid meeskondi, kus ma 
sisse minnes tajun, et ma olen kõige rumalam inimene ruumis. 

\question{Vot, ma just tahtsin küsida, milline on äge tiim aga just vastasid 
sellele.}

Muud teemad ka. Et oleks võimalikult mitmekesine, et oleks teineteist toetav, 
igasugused niisugused asjad on mulle ka tähtsad. Aga ennekõike see, et ma 
tunnen, et mul on niisugune arendav valdkond või ala. Võib olla kõige parem 
näide oli Salvast Haaberstisse minek. Ma arvan, et Novelli adminimises ja 
arvutivõrkudes ma tundsin ennast juba suhteliselt kindlalt. Aga see, mis asjad 
on telefoni keskjaam, ISDN, mis on kahe megabitised ühendused? Õrna aimugi 
polnud! Veelgi enam, mind saadeti Rahvusraamatukokku, kus oli just keskjaam 
üles pandud, et mine tee koolitus. Ma polnud seda telefoni mitte kunagi elus 
näinud, mille koolitust ma tegema pidin. Hüppa vette, uju ja õpi selle käigus, 
 kuidas ujumine käib. 

Väga kihvtid inimesed on olnud ja see on minu arust kõige tänuväärsem asi. Ja 
samas on veel kihvt, kuidas nagu mingid inimesed käivad ringiga. See, kuidas ma 
jõuan lõpuks Masti\index[ppl]{Kaal, Madis} või Toivoga\index[ppl]{Annus, Toivo} 
Skype'is\index{Skype} kokku tagasi. Mõnes mõttes on see maailm väga suur, aga  teistpidi 
jälle nagu kummaliselt mingid inimesed tulevad su juurde  ringiga tagasi. Või 
ka mingeid hooned, näiteks see hoone, kus me praegu oleme\sidenote{Ajasime juttu Tehnopoli hoones 
Akadeemia teel.} või siis see Suur-Ameerika 18 hoone, kuhu ma sattusin  
Haaberstis uuesti ja töötasin kõrvaltoas sellest, kus ma esimest korda  
286 taga istusin. Vahepeal käid suure tiiru ära kuskil mingites teistes ettevõtetes ja 
teistes kantides ja aastaid läheb mööda ja siis järsku leiad ennast kõrvaltoas 
sellest, kus sa kunagi oled töötanud. 

\question{Ja siinsamas majas oli ju Skype!}

Täpselt selles tiivas, mina tulin tööle siiasamma tiiba, esimesele korrusele. 
Siin vastas üle koridori oli Paananen\index[ppl]{Paananen, Tiit} oma 
\emph{certification} tiimiga. Kuigi, jah, nad vist alustasid kuskilt mingist 
teisest ruumist, ka siin maja sees koliti. Aga mul oli väga nagu \emph{coming 
back home} mingis mõttes, kui see osa siin valmis sai ja meile siia ruumi 
pakuti. 