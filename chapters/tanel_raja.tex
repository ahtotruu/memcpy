\index[ppl]{Raja, Tanel}
\label{sisu:pronto}
                 
\question{Kuidas sihuke asi juhtus, et sa Pronto oled?}

Jäi lihtsalt külge, ma isegi täpselt ei mäleta, mis asjaoludel. Lihtsalt tol 
ajal pidi olema igaühel oma nimi ja siis minu nimi oli lõpuks see.

\question{Aga kuidas sa arvutite juurde sattusid?}

Arvutite juurde sattusin ma enne, kui must Pronto sai.

Ma ei tea, kas sihukest raamatut oled lugenud nagu Professor Lillepooli 
Kroonika\sidenote{\enquote{Professor Lillepooli kroonika} on eesti kirjaniku 
Herta Laipaiga ulmelugu. Raamat ilmus kirjastuse Eesti Raamat väljaandes 1982. 
aastal. Raamatu peategelased kohtuvad muu hulgas arvutiga nimega Kunigunde.}. 
Seal toimus selline kohutavalt põnev tegevus, kus tüübid tegid oma Musta Kassi 
Ordu ja selle käigus nad käisid vist TPI-s ja tegid mingi kohutava skeemi 
valmis. See on  tagasiulatuvalt vaatas nagu naeruväärne, aga see tundus 
kohutavalt  põnev, kui ma  olin väike poiss. Sealt tekkis nagu huvi. 

Edasi oli nagu kaks liini. Mu onu on tegelikult IT valdkonnas olnud tegev juba 
tükk maad varem, kui mina. Mis tähendas seda, et ta juba siin sügavas 
nõuka-ajas oli tugevalt arvutite kallal. See oli nagu üks liin, aga voolas 
Tartus. Ma käin seal, kõik oli kohutavalt põnev ja puha, kui ma neid asju 
vaatasin. See kõik oli nii põnev, et ma nagu täitsa nagu\ldots

\question{Mis selle asja põnevaks tegi?}

Põnevad olid need asjad selles mõttes, et seal on nagu nupud ja asjad toimuvad. 
See on niisugune väikese poisi, ütleme niimoodi, et rohkem kui pisut märg 
unenägu, kus ta saab kontrollida mingit masinat täielikult. Et jõud käib 
sellest masinast üle või, õigemini, võiks potentsiaalselt käia, sest alguses 
loomulikult mitte. 

Ja teine, Tallinna poole pealt, oli see klass, mis Jaak 
Loondel\index[ppl]{Loonde, Jaak} oli Luise tänaval\index{Tallinna 
Oktoobrirajooni Õppetootmiskombinaat} ja kus olid siis Yamaha 
MSX-id\index{Arvutid!Yamaha MSX}\sidenote{Koht, mis mujal jookseb läbi kui Tallinna Oktoobrirajooni Õppetootmiskombinaat, mida on vahel nimetatud Roopa tänaval asunuks ning mille kohta vahel öeldakse lihtsalt \enquote{Luise tänava klass}.}. Sama tüüpi aparaat, mis mul siin külje all 
seisab. Mul on muide üks niisugune masin, kusagilt Soomest ostsin, mis on 
veetud Venemaale, sealt Soome ja siis Soomest tagasi Eestisse.

\question{Mis klass see Jaagul oli? Mõne kooli juures?}  

Ma isegi ei mäleta, see oli siuke  nagu huvialamaja, huvigrupid olid seal. Kui 
ma mõtlen natuke, siis mul tuleb see nimi ka meelde, mis selle maja nimi oli, 
see oli üsna tükk aega tagasi, puusalt päris ei tule.
                 
Seal oli nagu võimalus käia klassis istumas ja nikerdamas. Noored nagad tahtsid 
loomulikult hullupööra mängida aga oli niisugune kurja juur nagu Jaak Loonde, 
kes ütles, et ei-ei, tuleb hirmsal kombel ikka programmeerida. Nii oligi 
tasakaal nende kahe asja vahel üsna hästi paigas, sellepärast et niipea kui 
Jack kõrvale vaatas, olid poisid kohe mingid asjad käima tõmmanud. Sealt tulid 
need kuulsad lauset nagu \enquote{Markus tervitab} ja niisugused asjad. Kui 
keegi lasi võrku mängu, siis said kõik seda laadida endale. 

\question{Kust need mängud tulid? Poest neid ju ei saanud osta?}

Klassis oli õpetaja arvuti ja siis õpilase töökohad. Nende vahel oli võrk, mis 
oli naljakal kombel üles ehitatud MIDI kaabli otsa. MIDI on küll rohkem mõeldud 
 muusikariistade juhtimiseks ja need Yamahad olid tegelikult algselt 
muusika-arvutid, ja olid läbi selle muusika-süsteemi kohapeal võrku pandud.

Mängud liikusid kassettide ja ketastega. Aeg-ajalt käis keegi välismaal. Ma 
mäletan, see on küll väheke teine episood, kus TTÜ-s tuli keegi välismaalt 
tagasi,  mingisugune teadustööga seotud ettevõtmine. Tuli tagasi ja pani laua 
peale kolm kolmetollist flopit. Kõik olid seal kõrval ja ootasid, et 
\enquote{nüüd hakkame vaatama, mis siin on}. See oli kohutavalt harras hetk.
                 
\question{Sest ega ei osta ega alla laadida kuskilt ju midagi ei olnud, kõik 
käis käest kätte}

Ta oli selles mõttes keeruline, et see oli ikkagi üsna sügav nõuka aeg, 
kaheksakümnendate keskpaik. Kusagilt servast hakkasid vabaduse kiired juba 
terendama, aga need ei olnud kuskilt otsast veel materialiseerunud. Põhiline 
värk oli see, et inimestel lasti enamasti teha mingeid asju mille eest varem 
oleks türmi pistetud.

\question{Seal Luise tänaval sa käisid keskkooli ajal?}
                 
See oli enne keskkooli, ma arvan, et ma oli mingisugune 13-14 aastat vana, 
selleks hetkeks. Täpselt niimoodi, et kael juba kandis, aga mitte väga. 
Täiskasvanuks  veel ei peetud, selline ebamäärane aeg, tiinekate alguse ots. Ei 
tea veel, mis sust saab.

\question{Mingil ajal hakkasid tulema BBS-id ja Fido\ldots}

Need olid  tükk aega hiljem. Alguseks oli tegelikult see asi, et sedavõrd, 
kuidas inimesed kasvasid, hakkasid ka täiskasvanud  meiesuguseid nolke 
tõsisemalt võtma. Siin korra põgusalt rääkisime enne, et kui nõuka-aeg lõppes 
ja Eesti aeg algas, siis esimene palk, mis inimestele kätte anti, oli 300 
krooni. Mis praeguses ajas on 20 eurot. See oli kuupalk ja sellest elas ära 
ikka kenasti. Mitte küll niimoodi, et oleks midagi hullupööra huvitavat sellest 
ostnud, aga elas ära. See võtab tegelikult üsna hästi kokku, kuidas nõuka-aja 
elatustase oli: konverteeritav valuutas see sissetulek umbes 15-20 eurot 
praeguses mõistes oligi. Ja kuna asjad, mis meil siin ringi liikusid, olid 
valmistatud kusagil oma Normas või Salvos või Kommunaaris või kus iganes,  siis 
 oligi niimoodi,  et hinnad olid väikesed, aga sisetulek oli ka väike, asjad 
olid tasakaalus. 

\question{Aga 20 euro eest arvutit ei osta}

20 euro eest arvutit tõesti ei osta, vastavad vahendid olid olemas asutustel. 
Teine asi oli see, et Nõukogude Liitu oli keelatud eksportida kõvemat 
arvutustehnikat. See, et meile näiteks tekkisid siia need MSX-id, oli osaliselt 
tingitud sellest, et tegemist oli suhteliselt alumise otsa masinatega,  rohkem 
niisuguste mänguasjade kui  päris tööriistadega. Näiteks ma mäletan, et kui 
Soomest veeti Eestisse üks 386, kui nad just just äsja välja tulnud,  siis tuli 
sellest kohutav rahvusvaheline skandaal. Ühendriigid võtsid soomlastel kõri 
pihku ja ütlesid, et \enquote{mis mõttes te veate Nõukogude Liitu niisugust 
tehnoloogiat, millega on võimalik rakette arvutada ja mida iganes teha}. Meie 
mõistes oli see masin tegelikult tol hetkel väga kõva sõna. Praegu ta muidugi 
on naeruväärne, suvaline kell on ka võimsam.

\question{Ehk, et arvutile ligi saada, sa pidid mingisugusele asutusele külje 
alla pugema}

Jah. Asutustel olid arvutid, millega nad üritasid oma asjatoimetusi läbi viia.
Arvuteid oli mitmesuguseid, oli klassikalisi nõuka-aegseid näiteks Ural-idel 
põhinevaid mingeid süsteeme, kus olid terminalid ja sihukesed suured kastid.
Aga ajapikku tekkisid ka muud, välismaise päritoluga, masinad. Peaasjalikult 
286-d ja niisugused asjad. Koos nendega tekkisid mingid võrgud. Seega oli vaja 
inimesi, kes seda kõike haldaksid. Aga inimestest oli selles mõttes põud, et 
keegi ei teadnud, mis nende kastidega peale hakata. Ja kui seal kõrval 
nikerdamas käia, siis päris kähku muutusid sa kasulikuks. Noore poisina oli sul 
aega, ei olnud mingeid perekondlikke kohustusi ega midagi, hästi palju huvi oli 
ja saidki seal eksperimenteerida. Läksid õhtul pärast kooli sinna, nemad läksid 
töölt ära ja said seal istuda, kuni mingi üheksa-kümneni, sõltub, nii kaua kui 
lubati. Ja kokkulepe oli see, et sa siis üritad kuidagimoodi kasulik olla ja 
eks lõpuks sai sellest taga istumiskohast töökoht, kui kool läbi sai.

\question{Mis asutuses sina käisid?}
        
Minul oli alguses linnavalitsus\index{Tallinna Linnavalitsus} ja hiljem 
Riigikantselei\index{Riigikantselei}. Ehk siis põhimõtteliselt kogu selle 
huvitava perioodi, kui Eesti Vabariik välja kuulutati, ma töötasin juba 
Riigikogu majas, mitte nagu Stenbockis, seda siis veel ei olnud. Seda nimetati 
vist peaministri kantseleiks tegelikult. Seal olid ka 
Uralid\index{Arvutid!Ural}\sidenote{\begin{russian}Урал\end{russian} oli 
Nõukogude Liidus, Pensas, aastatel 1956 kuni 1964 toodetud arvutite sari.}, 
üüratud kapid, mille sees olid viiemegabaidised trummel,kettad, mis tuli 
hommikul käima lükata.

\question{Sul see akadeemiline või ülikooli asi jäi kuidagi vahele?}

Mul on ülikool, on selline pooles vinnas kogu aeg, lõpetamata kõrgharidus. Ma 
üritan kusjuures seniajani seda lõpetada ja loodetavasti mingi paari aasta 
jooksul ka seda teen.

Tol ajal oli valida, kas sa tegeled arvutitega või õpid. Suuresti oli valik 
väga selge, mis tuli ette võtta. Seda, mida ma tahtsin õppida, ma õppisin 
arvuti taga oluliselt rohkem.
                 
\question{Millal Fido Eestis jalad alla võttis?}

Ma nüüd päris täpset aastaarvu ei oska öelda, aga üks grupp, kes selle ümber  
tegutses oli Tõnis Reimo\index[ppl]{Reimo, Tõnis}, Tarmo 
Aausing\index[ppl]{Ausing, Tarmo} ja Virko Püss\index[ppl]{Püss, Virko}. Ja 
kes seal  veel nagu jõlkusid olime mina ja Miko Raud\index[ppl]{Raud, Miko}.  

\question{Kus see \enquote{seal} oli?}

Erinevates kohtades, mina mäletan näiteks Narva maanteel. Eks grupp kogunes ja 
tehti asju, nemad teavad nüansse paremini kui mina. Vahepeal tekkisid seal 
võimalused  peaasjalikult soomlastega asju ajada. Avastad enda jaoks BBS-id, 
eks ole, siis avastad, et tarkvara peab kusagilt tulema. See oli nüüd see aeg, 
kus lisaks floppidele hakkas juba tekkima võimalus ka modemiga 
\emph{download}-eda asju ja  tulid 9600-sed modemid.

Ehk, asjad hakkasid jumet võtma. Tol ajal olid tarkvarapaketid sihukesed paari 
megabaidised, maksimaalselt. Ehk need olid tõmmatavad umbes päevaga.

\question{Ehk siis, helistati Soome BBS-idesse sisse?}
                 
Jah. 

\question{Kuidas see käis, sest Jaan Tallinn\index[ppl]{Tallinn, Jaan} on 
rääkinud läbi inim-operaatori arvuti külge helistamisest?}

Oi, leiti võimalusi, igasuguseid imeasju tehti. Näiteks selgus, et lifti 
telefoniühendusest oli võimalik välismaale helistada, keegi polnud taibanud 
seda sealt välja lülitada. Ehk, liftist said sa helistada ja öelda, et 
\enquote{Appi, appi, olen siia kinni jäänud} aga sellesama lihtsama 
liftiühendusega sai helistada ka Soome. Ja keegi ei olnud päris kindel, kes 
selle nõuka ajal kinni peab maksma ja  seetõttu jäigi see kuidagi ripakile. 
Loomulikult olid ligipääsud erinevatele keskjaamadele ja raha, mis tekkis, 
tekkis kusagil seal süsteemides ning kadus kuhugi süsteemi niimoodi, et 
tegelikult kasutati seda kõike ühte või teistpidi kurjasti ära. Ütleme 
niimoodi, et see oli üks viis Nõukogude süsteemi õõnestada.

Sellega tekkisid kontaktid. BBS-is teinekord, kui sa logid sinna peale, siis 
\emph{sysop} istus teinekord seal kõrval, ja vaatas, et \enquote{Ohoo,  Eestist 
on mingid tüübid, kes need sihukesed on?} ja ütles kohe, et ta räägiks paar 
sõna juttu. See oli üsna tavaline, et BBS-i operaator rääkis külalistega.

BBS, ütleme niimoodi, ei olnud väga erinev tegelikult tänapäeva 
sotsiaalmeediast. Selles mõttes, et BBS pandi ikkagi kahel põhjusel püsti. Üks 
neist oligi eelkõige see, et endale mingisuguseid kontakte luua või 
\emph{networking}-ut teha, olla nii öelda elu pulsil. Teine oli tavaliselt see, 
et propageerida mingit asja,  näiteks oli BBS mingi firma juures, võis 
mingisugune demo grupp oli enda oma püsti pannud. See BBS, kust me esimese 
kontakti saime, selle nimi oli Poison Door\index{BBS!Poison Door}. Mingist 
laulust pärit, aga ma ei suuda kohe meenutada.

\question{See võis mingi demo grupi juures ka olla, soomlaste 
\emph{demoscene}\sidenote{Demo on arvutikunsti teos, mis kujutab endast 
terviklikku, sageli väga väikest, arvutiprogrammi, mis mängib maha 
audiovisuaalset vaatemängu. Demo eesmärgiks on demonstreerida (nagu nimigi 
ütleb) autorite programmeerimise, visuaalkunsti ja arvutimuusika oskusi. Demode 
ümber tekkis kogukond, \emph{demoscene}, mis sai kokku demopidudeks kutsutud 
festivalidel, üks kuulsamaid selliseid on siiamaani regulaarselt Helsingis 
toimuv Assembly.} oli väga kõva tol ajal}
                 
Mul endal oli kontakt sihukese \emph{Future Crew}\index{Future 
Crew}\sidenote{Future Crew oli Soome demogrupp, mis peamiselt tegutses 
1987-1994. Nende tehtud oli tõenäoliselt kõigi aegade kõige mõjukam demo nimega 
Second Reality (avaldati Assembly demopeol 1993. aastal). Second Reality tegi 
tänapäeva mõistes olematu riistvara peal reaalajas asju, mis tundusid täiesti 
võimatud, nägi üliäge välja ja sisaldas muusikat, mis siiani kananahka tekitab. 
1999. aastal hääletasid Slashdoti lugejad selle demo kõigi aegade kümne kõige 
vingema häki hulka.} tüüpidega. Ma mäletan, ma \emph{download}-esin niisugust 
toredat mängu nagu Wing Commander\index{Mängud!Wing Commander} nende BBS-ist. 
Seal oli niimoodi, et omadele meestele anti ikkagi nagu asju, mis tegelikult 
päris ametlikult väljas ei olnud. Tagatoad olid praktiliselt kõikidel BBS-idel, 
kus hoiti mingisugust nodi, mida siis kasutati vahetuskaubana. Tarkvara oli sel 
hetkel omamoodi kõva valuuta. Me panime püsti kahepoolse ühenduse, mina 
\emph{upload}-esin mingit muud asja, mille ma olin kusagilt mujalt saanud ja 
siis  sealtpoolt tõmbasin vastu Wing Commanderit ja samal ajal sai rääkida ka. 
See oli selline tarkvara, mis võimaldas kahepoolset sidet ja samal ajal ka 
\emph{chat}-ida, see viimane ei võtnud väga palju ühenduse mahtu.

\question{Nojah, sest iga klahvivajutus oli üks sümbol, fondi või värvide 
informatsioon kaasa ei liikunud}                 

Absoluutselt, see oli tavaline tekst. Kogu selle mängu allatõmbamine võttis 
aega ikka tunde. Sel hetkel ma olin ise omale ühenduse sebinud, Riigikantseleil 
selline võimalus nõuka-aja lõpus. 

Nojah, tutvumine toimus, info liikus. Ja oli aja küsimus, millal lõpuks siingi  
oma BBS püsti pandi ja omale Fido kontakt saadi. Ma just hiljuti küsisin selle 
kohta ja paistab, et Fido on nüüdseks lõplikult hinge heitnud.

\question{Üsna kaua võttis aega!}

Võttis aega aga võib-olla on mõttekas ta uuesti üles tõmmata, aga ma ei ole 
päris kindel, kuivõrd võimalik või mõttekas see on. Ta eksisteerib endiselt ja 
tänapäeval on retroasjad moes, nii et võib-olla see märkab kunagi ellu.
                 
\question{Mis oli Eesti üks esimesi suuri BBS-e, kus rahvas hulgakaupa sees 
käis?}

Esimene tõsiseltvõetav, just nimelt Fido mõistes, BBS oligi Hackers Night 
System\index{BBS!HNS}\index{BBS!Hackers Night System|see{HNS}}. Mis oli, nagu 
nimigi ütles, häkkerite öösüsteem. Ehk siis päeval olid telefoniliinid muuks 
otstarbeks, öösel käis nende peal BBS-idesse helistamine. 

\question{Miks tal ingliskeelne nimi oli?}

Sellepärast, et väga rahvusvaheline ja puha ja et oleks ägedam.

\question{Kes HNS-i opereeris?}
Seesama kamp oligi, Reimo\index[ppl]{Reimo, Tõnis}, Ausing 
\index[ppl]{Ausing, Tarmo} ja Virk\index[ppl]{Püss, Virko}.
 
\question{BBS-i jaoks oli ju mingit riistvara, modemeid ja asju vaja?}          
      

Oli jah. See oli kõik omamoodi sebimine, mis seal juures käis. Ütleme niimoodi, 
et tol ajal olid need vahendid suuresti riigi rahakott. Selle küljes siis 
istuti ja kui oldi juba kasulikud, siis sai alati ka juhtida neid ressursse 
niimoodi, et nad liiguksid õiges suunas. 

\question{See oli aasta\ldots}

Kaheksakümnendate lõpp. Ehk siis mitte isegi 89, vaid ma arvan juba veel varem.
                 
Ma ausalt öeldes täpselt ei mäleta, need olid ajad, eks ole, mis ongi  hägused 
selles mõttes, et vanus oli täpselt selline, et keegi ei olnud veel  täiskasvanu, 
keegi ei olnud enam päris laps ka. Aeg omab teist tähendust ja pärast on seda 
raske mõõtkava peale panna.

\question{Tol ajal BBS-il oli ikkagi üks modem ja üks liin?}
                 
Oo jaa. Tegelikult muid ka, mingisuguseid paralleelse side katseid oli. Näiteks 
PirnBox\index{BBS!PirnBox} oli üks ja oli veel muidki, aga sellised 
klassikalisi Fido  mõistes BBS-idest, ma arvan, oli HNS esimene ja sealt läks 
asi nagu krõbinal laiali\sidenote{Pronto ise pidas BBS-i New Age 
System\index{BBS!New Age System} Fidoneti aadressiga 2:490/12.}.
                 
\question{Anna mingisugust aimu, et kui krõbinal. Palju neid BBS-e tipphetkel 
oli?}
                 
Tipphetkel, ma arvan, oli 20-30. Süsteem nägi ette \emph{point}-e. 
\emph{Point}-id olid sihukesed pool-BBS-id, ütleme niimoodi. BBS ei ole nagu 
mingisugune näitaja selles mõttes, et olid \emph{point}-id ja olid \emph{full 
node}-d. \emph{Node}-l olid mingid kohustused, mingeid meile tõmmata, hoida ja 
jagada, eks ole. \emph{Point}-il oli lihtsalt see, et ta võis tõmmata. Paljud 
BBS-id otsustasid \emph{point}-iks  olemise kasuks puhtalt sellepärast, et nad 
ei saanud ühte või teistpidi ennast kogu aeg käimas hoida.
                 
Ehk, \emph{node}-del olid \emph{point}-id keda nad siis varustasid 
informatsiooniga. \emph{Node} käimas hoidmine eeldas ühte või teistpidi 
sihukest võimekust olla  mingitel hetkedel üleval.
                 
\question{See siis tähendab, et Eestis oli tol hetkel ikkagi mingisugune 
paarkümmend inimest, kellel oli võimekus sebida liin, riistvara ja ka 
opereerida süsteemi.}

See oli jah, nagu tipphetk. Sest mingil hetkel asi läks ladusamaks, vahepeal 
juhtus ka niisugune asi, et  nõuka-aeg sai otsa ja tuli Eesti Vabariik. Mingil 
hetkel hakkas see asi selles mõttes käest ära minema, et raha hakkas omama 
tähendust. Sa ei saanud, enam kusagil ettevõtte küljes istuda ja lihtsalt 
mingeid asju teha. BBS oma telefonikõnedega tekitas ikkagi kulusid. Ja siis 
hakkasid need peod vaikselt kinni minema, inimesed vahetasid kohti, uutes 
kohtades enam eri vaadatud lahkelt selle asja peale\ldots

\question{Kuidas sellest ürgsupist Eesti arvutifirmad tekkisid? BBS-id 
seltskond läks sujuvalt üle teenuseid pakkuma?}

Osaliselt küll, need inimesed ühte või teistpidi olid kuidagi nende 
arvutifirmadega seotud. Aga need võib-olla ei olnud päris tihti päris needsamad 
inimesed. Teatavasti sogases vees on kõige parem kala püüda, seal on kõige 
suuremad purikad. Sogased ajad, inimesed leidsid erinevaid viise, kuidas endale 
raha teha. Näiteks Peterburist veeti igasugust IT-tehnikat Tallinnasse. Hüpati 
autosse, sõideti sinna, Peterburis olid mingisugused punktid, kus sai asju 
osta. Asjad toodi Eestisse, niimoodi see elu vaikselt edenes.
                 
\question{Sina olid sel ajal veel Riigikantseleis\index{Riigikantselei}}        
         

Jah. Aga oli näha, kuidas hakkasid tekkima esimesed firmad, mõned olid edukad, 
mõned vähem edukamad. Mingil hetkel ma läksin ise nagu ka Riigikantseleist 
minema, sest ka seal toimusid muudatused struktuurides, inimesed tulid, 
inimesed läksid.

\question{Mis su peamine asi oli tol ajal, mis sa arvutiga tegid? Kirjutasid 
koodi või mis?}

Nüüd on võib-olla nagu naljakas öelda, aga see oli nagu eluviis, ütleme 
niimoodi. Ega ta nagu väga ei erine praegusest eluviisist, ainult selle vahega, 
et kujuta ette, kui sa peaksid näiteks Facebooki ligipääsu saamiseks kulmu 
lihaste päeval ringi roomama. See ei olnud kõikidele kättesaadav. 
                 
Sel ajal oli kogu sellel asjal sihuke küllalt kõrge lävi selles mõttes, ta 
eeldas ikkagi mingisugust teatavat ülevaadet tehnikast, võimalustest. Veidike 
meenutab selles mõttes algusaja Internetti, sest praegult on Internet iseennast 
sõlme tõmmanud. Aga varem oli niimoodi, et sa sa pidid aadresse täpselt teadma, 
kuhu minna, sest polnud otsinguid ega asju. Siis hakkasid tekkima esimesed 
otsingumootoreid, WebCrawler, AltaVista ja siis lõpuks Google, eks ole. Need 
oleksid tõmbasid selle läve  väikeseks. 

\question{Mis tolle BBS-iga siis teha sai?}
                 
Sai faile jagada, oli kirjavahetus. Fidonet oli suuresti tänapäeva mõistes 
Interneti meilisüsteemi kui sellise sarnane. Aga uudisgrupid olid ka, aga 
\emph{usenet}-i grupid samamoodi eksisteerivad Internetis. Uudisgrupid on ühte 
või teistpidi asendunud paljuski Facebooki ja muude kohtadega, kus kus käib 
info vahetamine, Reddit on näiteks niisugune.

\question{Aga see struktuur on nagu laiali vajunud, \emph{usenet}-i grupid olid 
ju kuidagi hierarhias ka}                 

Jah, olid hierarhiad, omad etiketid ja asjad, mida võhikul oli väga raske 
aduda. Tihtipeale inimesed tundsid ennast teineteist üsna lähedalt ja siis 
teinekord, kui jutuajamist jälgida, siis sa olid täielik \emph{outsider}, mitte 
midagi ei saanud aru, millest jutt käib. Kõikidel oli mingi oma oma nagu taust.

\question{Kust need inimesed üksteist tundma said, kas sealtsamast gruppidest?}

See hakkab hakkab tavaliselt pihta ühega kahest. Kas keegi \emph{introduce}-b 
sind, viib sind asja juurde ja aitab sind ree peale või siis kiibitsed mingi 
aja ja siis mingil hetkel hakkad aru saama, mis toimub. Ja kui ei hakka, siis 
ei hakka, see ei olnud lihtne.

\question{Kui tihedalt Eesti Fido seltskond omavahel läbi käis?}

Ütleme nii, et ta pidi paratamatult läbi käima, sellepärast et esialgu, kui 
need Fidoneti asjad tekkisid, seal tekkisid ka grupid ja gruppidesse pidi sisu 
tekkima. Ja kuna esialgu oli neid inimesi vähe, siis paratamatult seal ei olnud 
mingisugust meeletut kommunikatsiooni. Inimesi oli seal mingisugune mingi 
paar-kolmkümmend tükki ja ei ole väga keeruline nendega  mingil hetkel ka 
isiklikult tuttavaks saada.

\question{Kes need inimesed olid?}

Enamasti olid samasugused inimesed nagu mina. IT valdkond, väga suur huvi. 
Ütleme nii, et sarnased huvid olid tugevad ja oli, millest rääkida. Olid ka 
mingisugused teemad, mis siis olid parasjagu \emph{zeitgeist} põhimõtteliselt. 
Näiteks ma mäletan, et kui ma mängisin ma niisugust asja nagu King's Quest 
IV\index{Mängud!King's Quest}, siis ei olnud niisugust võimalust, et lähed 
veebi ja otsid omale \emph{walkthrough}. Inimesed ikkagi oma jõuga üritasid 
sealt läbi närida ja siis aeg-ajalt vahetati kogemusi. Sealtkandist pärineb 
isegi see raamat \enquote{Selles mängus ei hüpata}\sidenote{Juhan Habichti sama 
nimega novellikogumik ilmus 1993. aastal kirjastuse Katherine väljaandel}. 
Habichti raamat oli see minu meelest. See mäng muide 
Larry\index{Mängud!Larry}\sidenote{\emph{Leisure Suit Larry} oli Al Lowe poolt 
loodud seiklusmängude sari, mis ilmus aastatel 1987 kuni 2009. Mängud olid 
tuntud oma, ütleme, omapärase huumori ja, lastele mittesobiliku sisu poolest. 
Mängija: katsub riiulil seisvat kopratopist. Larry: \enquote{\emph{I've always 
liked the feeling of a good beaver}}}.

Tine klassikaline näide sellest, millest räägiti, oli võimalused ja nende 
vahetamine. Kellelgi oli üks asi, kellelegi teine, pandi seljad kokku. Kuna 
inimesi oli vähe ja inimesed tundsid teineteist, siis ei olnud niisugust väga 
suurt kana kattumist ka.

\question{Kas trollimist või midagi selle sarnast ka toimus?}

Kui sul ei ole auditooriumi, siis inimesed jäävadki tavalisteks inimesteks. Kui 
sa annad normaalsele inimesele anonüümsuse ja publiku, siis saab tast igavene 
tõpranahk.
                 
\question{Isegi, kui sul oli \emph{handle}, sa tegelikult ei olnud ju anonüümne}

\emph{Handle} oli lihtsalt teie nimi, kõik tegelikult teadsid, kes on kes. 
Isegi kui sa olid anonüümne, siis sa ei olnud selles mõttes anonüümne, et ei 
kasutatud selliseid laest võetud nimesid, mis on  äraviskamiseks. Kui sa oled 
ikkagi feimi tekitanud sellele nimele, siis sa ei tahtnud käia sellega 
uisapäisa ringi. 

\question{Sa oled siiamaani Pronto ja see tähendab siiamaani midagi, eks. 
Lõpuks, kui sa hakkasid väga sigatsema, siis sind visati ikka välja ka?}

Jah, sinu juhe tõmmati põhimõtteliselt seinast,  sa olid kohemaid \emph{persona 
non grata}. Ja arvestades sellega, et see oli seotud sinu enda huvidega ja sinu 
enda mineviku, oleviku ja tulevikuga siis sa ei saanud seda omale lubada.

\question{Kõik olid huvitatud viisakalt käituma}

Kõik olid seal paadis selles mõttes võrdsed. Et kui keegi hakkas paati 
kõigutama, siis ta kõigepealt kõigutas seda paati enda alla. Ja kui ta jätkas 
seda tegevust, siis ta lihtsalt eemaldati paadist ja pidi ise vaatama, kuidas 
ta seal veekogus hakkama saab.

\question{Kas nii juhtus ka?}

Otseselt ei juhtunud. Või kui juhtus, siis vähemasti juba hilisemal ajal. 
Alguses oli ikkagi selline tihe seltskond ja kuigi nagu kõik ei saanud omavahel 
ideaalselt läbi, mõistsid kõik, et nad on selles paadis  koos. Ja seetõttu 
lihtsalt välditi teemasid, mis oleks võinud põhjustada palju tüli.

\question{Inimesed said aru, et mingitest asjadest me ei räägi}

Põhimõtteliselt trollimine ongi see, et sa räägid asjadest, mis teisele 
inimesele peavalu valmistab, eks ole.
                 
\question{Tuleme sinu juurde tagasi. Kui sa 
Riigikantseleist\index{Riigikantselei} ära tulid, mis sa siis tegid?}

Siis oli mul selline koht nagu Marvin-Ekspert, seal sain oma käe valgeks 
ostmiste ja määmistega. Mul olid sellised teemad nagu Gravis Ultrasound ja 
IOMega.

See oli selles mõttes huvitav aeg, et Gravis Ultrasound maksis väikse 
varanduse, aga samas oli see tükk maad parem kui järgmine. Ja see oligi mu 
katsed müügi koha pealt, ma arvan, et ma müüsin neid umbes sama palju, kõik 
ülejäänud asju kokku müüdi, kuigi ta oli tükk maad kallim. Mõnes mõttes nagu 
Apple, kallimat asja on alati lihtsam müüa. Kalliduse taga on tavaliselt 
väärtus, ta ei ole kallis niisama.

\question{Mis aastal see oli, suurusjärk?}

1994, pakun huupi. Või varem isegi, 92. aasta kanti.

\question{Sel ajal hakkas tasapisi Microlink tekkima?}                 

Microlink tekkis tegelikult üsna alguses. Nad olidki üks nendest firmadest, kes 
hakkas sellega pihta, et kotis toodi Peterburist asju. Esialgu müüdi need 
arvutid firmadele loomulikult, sest seal oli ikkagi raha. Firmadel olid 
sihukesed hägused positsiooni, sest omanikusuhted olid hägusad, selge 
omandiomandisuhe polnud veel paigas, ütleme niimoodi.

\question{Sihukest kapitalisti oli vähe}

Kapitalisti oli vähe, olid veel nõuka-aja jäägid, keegi oli kusagile käpa peale 
pannud. Ausalt öeldes see oli niisugune aeg et paljusid asju ma ei teadnud ja 
paljusid asju ma teadsin, aga ei tahtnud teada. Asjad, mis juhtusid sel ajal, 
nagu öeldakse, et mis juhtus Vegases, las jäävad Vegasesse. Sel ajal tehti 
igasuguseid asju ja ma arvan, et kuigi need võivad praegu kõlada küsitava 
eetilise ja moraalse taustaga, siis ma arvan, et otsused, mis tehti olid 
tegelikult siiski õiged ja vajalikud.

\question{Tol ajal ju kujuneski, et mis on õige ja mis ei ole}

Jah. Loomulikult sel ajal tehti igasugust erastamist ja ärastamist, ka see oli 
hädavajalik puhtalt sellepärast, et miks need otsused tehti, eristabki meid 
tänapäeva  Moldovast. Seal tehti teistsuguseid otsuseid ja ma leian, et isegi 
need, kes meil siin omal ajal ärastasid, tegid seda teataval määral nagu 
\enquote{eesmärk pühendab abinõu} kaalutlustel.

\question{Mina ei jääks endaga kuidagimoodi rahule, kui ma ei küsiks sellise 
asja nagu .EXE\index{Ajakiri!.EXE} kohta sinu käest. Räägi sellest palun 
natuke.}                 

Ajakiri oli osaliselt Microlinki püüd või isegi ambitsioon ennast nähtavale 
tuua.  Eestist oli tegelikult kaks arvutiajakirja tollal, oli 
Arvutimaailm\index{Ajakiri!Arvutimaailm} ja .EXE. Arvutustehnika \& 
Andmetöötlus\index{Ajakiri!Arvutustehnika \& Andmetöötlus} ei olnud ju tegelikult ajakiri klassikalises mõistes oli rohkem 
nagu vihik.

Nõuka-aja lõpus ja eesti  aja alguses olid niisugused vihiku formaadis asjad, 
eriala väljaanded. Ta ei olnud otseselt mõelnud laiatarbevahendina.

.EXE tekkis umbes samal ajal, kui Arvutimaailm ja oli Microlinki sihuke katse 
püüd tekitada omale laiatarbe väljund.

\question{Mina küll ei mäleta, et seal sees palju laiatarbe-asja oleks olnud, 
seal oli ikka stiilipuhas kõva koore küberpungijutt?}                 

Ta oli selles mõttes \enquote{laiatarbe}, et oli aeg, kus inimestel ei olnud 
raha, et omale arvutit soetada. Pidi olema ikkagi väga palju tahtmist ja tänu 
sellele oli ajakirja sisu selline, mis eeldas väga palju tahtmist. 

Ehk siis asjal oli ajastu hõng juures. Mida inimesed arvutiga parasjagu tegid, 
see  sealt ka läbi kumas. 

\question{Kuidas sa sinna juurde sattusid?}

Täpselt nüansse nagu ei mäleta selles mõttes, see oli ikkagi kõva 25 aastat 
tagasi. Tol ajal oli sihuke asi, et naljaviluks kirjutasime mängude 
mingisuguseid asju, sisu tootmine käis taas kord vabatahtlikul alusel. 
Gruppidesse postitati mingisuguseid asju ja dokke ja häkiti ja nii. 

\question{Ehk, sa kirjutasid enne, kui .EXE juurde jõudsid?}

Mingil hetkel tuli see nagu jutuks ja kuna ma olin nende mängudega juba silma 
paistnud ja enam-vähem oli kirjaoskus ka olemas, siis nii ma sinna sattusin. 

Ma isegi ei mäleta täpselt, kelle vahendusel, aga sattusin. See oli päris 
naljakas aeg selles mõttes, et ajakirja koostamine oli sihuke omamoodi 
häkkimine. Tavalist käis see, et kolleegium (seltskond, kes sisu kokku pani) 
kogunes kokku. Kolleegiumile lükati ette kaks kasti õlut ja enne toast välja ei 
lastud, kui kui ajakiri on kokku pandud. Igaüks võttis omale mingid kohustused 
ja kadus siis nendega tegelema.

\question{Kaua .EXE üldse ilmus?}

Ta ilmus poolteist aastat või niimoodi.

\question{Nii vähe?}

Ma võin vaadata, korjasin vahepeal numbrid kokku\sidenote{Aadressil 
\url{punktexe.ee} on kõik ilmunud numbrid täies mahus ilmarahva rõõmuks 
vaatamiseks väljas.}.

\question{(vaatab ilusaid kaanepilte) Kes neid kaanepilte joonistas?}

Kaspar Loit\index[ppl]{Loit, Kaspar} alias BKnows.

Esimene number ilmus Aprillis 1993 ja viimane ilmus, keegi ta täpselt millal, 
aga 1995. Nii et kahe aasta jooksul. Aga see ilmumine läks vahepeal niisuguseks 
eklektiliseks.

\question{Arvestades, milline mõju tal oli, palju teda loeti ja kuidas fännati, 
siis väga mõjukas asi selle lühiduse kohta}

Jah, numbreid on kokku vist kaheksa, mitte kümneid. Igaüks oli nagu omaette 
šedööver, ütleme siis nii.

\question{Kas see andiski talle selle mõju, et iga number oli šedööver?}

See oli südamega tehtud eriala inimestelt eriala inimestele. .EXE oli tehtud, 
selleks, et (selline peenike sõna on) skenet juurutada, mitte et selle pealt 
mingit üüratut kasumit teenida. 

\question{Millist skenet? Lihtsalt arvuti-inimeste oma?}

Jah. See oli suuresti ka võib-olla selle asja nagu algus ja ots. Sellepärast, 
et ta oli pisut raskevõitu võib olla inimesele tänavalt. Tol ajal oli 
arvutiajakirjandus  teistsugune, kui praegu. Arvesta sellega, et praegu on 
igaühel arvuti ja minnakse ja loetakse, mis seal toimub. Kuidas oma mobiiliga 
ühte või teist või kolmandat teha. Aga tol ajal oli arvuti ikkagi suur asi, 
seda polnud kaugeltki mitte kõigil. Ma arvan, et praeguses mõistes mingi üks 
või kaks protsenti inimestest tabas tegelikult reaalselt arvutit ja oskas teda 
kasutada igapäevaelus.
                 
\question{Aga inimesi, kes tahtsid kasutada, oli rohkem. Ja nii nad siis 
istusid kodus ja lugesid hardalt, kuidas Pronto seikleb Day of the 
Tentacle's\index{Mängud!Day of the Tentacle}\sidenote{Legendaarse mängu Day of 
the Tentacle (Ilmus 1993. aastal LucasArtsi väljalaskel ja uuendatud graafikaga 
2016. aastal) legendaarne \emph{walk-through} ilmus .EXE teises numbris 
novembris 1993. Autoriteks BKnows\index[ppl]{BKnows} ja 
Pronto\index[ppl]{Pronto}.}}

Eks see kõik pisitasa hakkaskinagu tuult tiibadesse võtma, sel ajal toimus 
jõhker inflatsioon, ehk siis räägitud kahekümnest eurost said päris kiiresti 
sajad. Ja arvutid muutusid jõukohaseks ka teistele. Siis hakkasid tekkima 
esimesed ringid, muidugi läbi humanitaarabi, Rootsist veeti siia pruugitud 
tehnikat siia, mille inimesed omavahel laiali laotasid.

\question{Ja kõik see aeg sina muudkui müüsid Gravist}

Gravist ja Iomega Bernouilli draive\sidenote{Oma aja kohta suure mahutavusega 
eemaldatava kettaga 1992. aastal turule tulnud salvestussüsteem Bernouilli Box 
oli Iomega esimene laialt kasutust leidnud toode.} ja sihukesi asju. QIC-80 
teipe ja nii.

\question{Huvitav on see, et mitmed inimesed on mingis faasis tegelnud just 
nimelt arvutustehnika müügiga}

Kusagilt tuleb oma raha saada. Tuli see aeg, kui varad said nii-öelda laiali 
jagatud ja sa pidid põhjendama oma tegevust. Et miks sul on mingisugused 
BBS-id. Ja ainuke võimalus seda asja edasi edendada, oligi seda teha müügi 
egiidi all.

\question{Koodi kirjutamine elatise teenimise vahendina?}
                 
Tol ajal ei olnud väga mingisugust koodi, mida kirjutada. Loomulikult oli 
mingeid vahendid, väikseid asju, aga valdavalt käis koodi kirjutamine 
mingisuguste andmebaaside ümber, mingid FoxBase-d ja DBase-d, kus siis tehti 
ettevõtete  raamatupidamist,  inventuuri ja sihukesi asju. 

\question{Iga ettevõte pusis endale ise tolle rakenduse kokku...}

Või siis osteti mingite firmade käest, aga süsteem koosnes tavaliselt 
mingisugusest andmebaasi lahendusest. Teinekord olid mingid muud asjad ka, nagu 
näiteks  meditsiiniga seotud lahendused kus olid juba mingid infosüsteemid aga 
need olid väga spetsiifilised ja need enamasti arendati, niivõrd kuivõrd, 
väikesena.

\question{Huvitav paralleel. Eestlane üldiselt ei ole suurem asi müügi-inimene 
aga IT-asja on meil õnnestunud rahvusvaheliselt päris hästi müüa. Äkki 
seetõttu, et kriitilisel hulgal inimestel on oma elust müügi kogemus?}

Kindlasti on. Tol ajal oli see paratamatu, sellepärast et kui sa tahtsid omale 
saada ligipääsu, sa pidi juba ennast müüma. See on üks asi, mis on muutnud  
vana kooli IT-vennad nagu teistsuguseks on see, et sa pidid suutma müüa 
paratamatult. Kui sa ei suutnud, siis polnud sul IT valdkonda asja. Sest kõige 
tähtsam kaup, mis oli, pidi  iseennast müüma.

\question{Sest muud sul ei olnud}  

Muud sul ei olnud, ei olnud isegi kogemusi, sest kogemused tulevad töö käigus 
aga sul seda ei olnud. Sa pidid suutma endast teha väga vajaliku tegelase.

\question{Jah, tõesti, kui sul enesemüügi oskus olemas on, siis võib igasugu 
asju juhtuda}

IT valdkonnas müügis ongi see, et kui sa tähelepanelikult vaatad, siis nende 
läbimurrete taga on ühed ja samad inimesed pahatihti. Ja  need on just nimelt 
vana-vana-kooli kaader ja enamasti on nad oma läbimurde, müügi, teinud mitte 
tänu  avalikkusele, vaid vaatamata sellele. Sest avalikkus teatavasti tunneb 
kohemaid muret, kui keegi teenib paremini või tunneb ennast kuidagi paremini. 
Inimese hari läheb kohemaid kadedusest punaseks.
               
\question{Tihti öeldakse, et meil on vedanud, sest õiged inimesed on sattunud 
õigetesse kohtadesse. Sinu jutust tuleb välja, et tollest seltskonnast tulidki 
inimesed, kes võiksid sattuda õigetele kohtadele}
  
Jah, ongi täpselt. Kui sa hakkad ringi vaatama, siis need inimesed on siiani 
alles ja nad ei ole kuhugi ära kadunud, nad olid tegelikult suhteliselt noored. 
Osad neist on üle viiekümne, osa alla , aga sealkandis nad nüüd parasjagu on. 
Ja just nimelt, kui sa hakkad tähelepanelikult vaatama, siis suuremate 
läbimurret taga on üks või teine nendest nendest inimestest.
                 
\question{Seepärast me neid jutte räägimegi. Oskad sa kuidagi osundada seda 
hetke, mil see maailm oma süütuse kaotas? Kui romantilisest õllekasti abil 
toimetamisest sai raha teenimine?}

Ma ei oska seda nagu niimoodi paika panna, sest tegelikult on see suuresti 
ikkagi kuvand, mis on loodud väljastpoolt. Ehk siis, kui on mingisugune grupp, 
siis paratamatult tekivad  autsaiderid, kes tunnevad, ma ütleks, pahatahtlikku 
kadedust. 

\question{Ja nimetavad inimesi häkkeriteks}

Ei noh, \enquote{häkker} hakkas omandama lihtsalt nagu teistsugust tähendust.

\question{Viidates ühele .EXE jutule, misasi on 
küberpunk\sidenote[][-2cm]{Allkirjastamata, kui BKnowsi\index[ppl]{BKnows} piltidega 
lugu pealkirjaga \enquote{Kes sa selline oled, küberpunk?} ilmus .EXE kolmandas 
numbris 1994. aasta aprillis. Seejuures tuleb tunnustada artikli asjakohasust. Nii ilmumise (eba) regulaarsuse ja lühiduse kui kultusliku staatuse osas .EXE omaga sarnane kuid suurema levikuga ajakiri MONDO 2000 (aastatel 1984 kuni 1998 ilmus Ameerikamaal 17 numbrit) avaldas oma samateemalise satiirilise artikli \enquote{\emph{R.U. A CYBERPUNK?}} oma 10. väljaandes 1993. aastal.}?}

Kõik asjad, mis on punk, nagu aurupunk, küberpunk, diiselpunk, on lihtsalt 
žanr. Žanr, mis läbib mitut asja, valdavalt seda, et kuidas teadvus siduda 
tehnikaga. Mõnes mõttes on meie ühiskond praegult nii-öelda küberpungi jaoks 
esimesel tasemel, sest see, kui inimesed istuvad ninapidi telefonis, on 
lihtsalt liidestamise küsimus. Inimesed on ennast tegelikult arvutiga juba väga 
intiimselt liidestanud.

\question{Aga see algas ju, nagu sa rääkisid, juba kaheksakümnendate lõpus, kui 
kogu sinu elu oli arvutis. Lihtsalt liides oli kandilisem.}

Liides oli kandilisem ja vähestel inimestel, seetõttu see polnud elu, vaid see 
oli mu alter ego. See ongi üks põhjus, mille pärast valiti omale sihukesed 
tunnused, nagu mul on Pronto, et teha vahet sellel, mis toimub arvutis ja mis 
toimub niisama. Põhimõtteliselt inimene lõi omale identiteedi. 

\question{Just nimelt mitte ei valinud aga lõi!}

Ja sellega siis osaliselt sa elasid tulevikus, aga ka muu elu jäi alles. Pere 
ja  sõbrad ja see õlu, mida joodi, see kõik jäi teise ellu.

\question{BBS-i rahvas käis ju koos ka?}

Käis. Aga hakkas pihta sellest, et kõigepealt olid \emph{sysop}-ide saunad ja 
asjad, kust Fidonet aga välja kasvas. Ja siis hiljem sai sellest niisugune asi 
nagu BBSummerid\index{BBSummer}.
                 
\question{Palju neid toimus?}
                 
Ma mäletan, et nad nõuka-aja lõpus hakkasid pihta ja neid oli üsna üksjagu. Üks 
BBSummeritest lükati edasi sellepärast, et sel ajal sõitsid tankid Eestisse 
sisse. Ja see jättis meelde, millal üritus toimus. Oli vist teine või kolmas 
BBSummer.
                 
\question{Olen näinud BBSummeri pilte, kus Micolinkid, Skyped, Uninetid ja 
teiste hilisemate suurte asjade alustajad kõik peal on. Kas tol ajal, asja sees 
olles, ei olnud niisugust tunnet, et \enquote{oi, me oleme ägedad}?}

Muidugi oli! Mismõttes? Absoluutselt oli sihuke tunne, me olime hullult ägedad! 
See oli ka üks põhjus, miks me sellega tegelesime, sest see oli õudselt äge.

\question{Tulles meie jutu alguse juurde tagasi, kas selle ägeduse tuum oli 
jätkuvalt see, et sai masina mõne näpuliigutusega oma tahtele allutada?}

Kindlasti. Teine asi oli see, see läks ka kaugemale. Ei piirdunud enam õuega 
see elu, vaid koos Fidonetiga tekkis ka ülejäänud maailma sinna otsa. See ei 
olnud väga erinev tänapäeva Reddititest, Facebookidest, Twitteritest, kus sa 
saad suhelda mitte ainult oma paari lähima tuttava, vaid kogu ülejäänud 
maailmaga. See tegelikult avas selles mõttes kogu ülejäänud maailma. Andis 
näiteks võimaluse keeli omandada,  suhelda  erinevates keeltes, soome keeles, 
inglise keeles, see veel omakorda aitas nagu edasi.
                 
\question{Maailma avardumise tunne oli juures}

Maailma avardumine oli kindlasti. See oli veidike sihuke tunne nagu 
kosmonaudil, kui ta  atmosfäärist väljub. Eriti veel, kui see pind, kust üles 
tõusti, oli tükk maad sügavamal kui enamuse maailma jaoks, sest meil tuli see 
otse nõuka ajast šahh tulevikku\sidenote{Tänapäeval on seda maailma avanemise 
tunnet isegi raske ette kujutada. Minu avanemise tunne oli väga konkreetne ja 
seotud Lennart Meriga\index[ppl]{Meri, Lennart}. 1993. aastal edukate 
koolilõpetajate vastuvõtul Roosiaias õnnestus Lennarti kätt suruda. Mäletan 
seda tunnet, et selle sama käega, mida ma surusin, tõmbas ta vaid mõned aastad 
varem Valges Majas president Bushi antiiksele gloobusele risti märkimaks 
suurepärast kalakohta Kamtšatkal. Euroopa serval asuva pisikese riigi serval 
asuvast pisikesest linnast tulnud poisi jaoks oli see elu muutev hetk.}.

\question{Ühel hetkel olid Nõukogude pioneer ja mitte väga palju hiljem 
vestlesid mingite Kalifornia kuttidega keskjaamadest}

Jah, absoluutselt. Olid ka võimalused ja kogemused. Näiteks mõnes mõttes 
positiivne nähtus oli see, et Eestis puudusid \emph{legacy} süsteemid, meil 
puudus IT valdkonnas  mineviku taak, meil lihtsalt oli see asi nagu poolik. 
Mineviku taaga puudumine tegelikult võimaldas Eestil kihutada päris kiiresti 
päris kaugele võrreldes ülejäänud maailmaga, kes pidid oma asju käimas hoidma. 
Me oleme nüüd jõudnud sinnamaani, kus meil on oma taak tekkinud ja me peame 
sellega tegelema.

\question{Lõpuks ikka saab inerts otsa aga ta on, jah,  meid päris kaugele 
vedanud}

See sai üsna hästi ära kasutatud just nimelt sellepärast, et ütleme niimoodi, 
et õigel hetkel sattusid õiged inimesed pumba juurde ja seda saagi tõmmati 
käima ikka nii kaua kui jõuti, enne kui ärimehed jaole jõudsid. 

Loomulikult on ka see, et kuna oli hulk inimesi, kes tegid midagi, mis oli 
müstiline, keeruline, käsitamatu ja ilmselt ka veidike elitaarne, siis hakkasid 
peatselt tekkima inimesed, kes hakkasid kaikaid kodaratesse pilduma. Inimesed, 
kes tahtsid ka nagu löögile pääseda, kas tundsid ennast halvasti, et neid ei 
võetud jutule puhtalt sellepärast, et nad ei saanud aru paadi mittekõigutamise 
mentaliteedist. See mõnes mõttes nagu oligi tegelikult ajastu lõpp, kui igaühel 
oli ligipääs, lävi läks palju madalamaks ja ka lühemate pükstega mehed said 
paati astuda.

Tekkisid  inimesed, keda keegi teadnud, kelle, kes olid anonüümsed, kellel olid 
ambitsioonid, aga puudusid võimekused ja soov panustada ja nii edasi. 

BBSummerid hakkasaid ka kasvama, hakkasid kihistuma. BB-ürituste lõpp oligi 
võib olla see, et hakkasid toimuma BB-üritused BB-ürituste sees, ehk siis 
heideti nahka või niimoodi. 
       
\question{Mis sa praegu teed, kuhu see tee sind on toonud?}
                 
Praegu olen juba viimased kümme aastat tegelenud veebipoodidega. Minu eriala on 
veebiarendused ja veel täpsem eriala veebipoed. Ehk, kui asi puudutab 
e-kaubandust, siis ma olen selle kuidagi nagu enda valdkonnaks valinud. 

Olen rohkem nagu programmeerimisel peale nüüd hoopistükkis. Sest tänapäeval on 
praktiliselt kõik ühte või teistmoodi seotud tarkvara arendusega. Tol ajal ei 
olnud Internetilehte firmadel, tänapäeval on. Tol ajal ei pakutud teenuseid 
läbi Interneti, nüüd pakutakse. Ja on tekkinud vajadus inimeste järgi, kellel 
on tehniline kapatsiteet ja võimekus. Üks võimalus on värvata need inimesed 
endale või siis teine asi on palgata omale firma, mis sellega tegeleb.