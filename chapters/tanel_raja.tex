\index[ppl]{Raja, Tanel}
\label{sisu:pronto}

\question{Miks sind Prontoks kutsutakse?}

Jäi lihtsalt külge, ma täpselt ei mäleta, mis asjaoludel. Tol 
ajal pidi olema igaühel oma hüüdnimi ja minu nimi oli lõpuks see.

\question{Kuidas sa arvutite juurde sattusid?}

Arvutite juurde sattusin ma enne, kui minust sai Pronto.

Kas sa oled lugenud sellist raamatut nagu „Professor Lillepooli 
kroonika"\sidenote{Herta Laipaiga ulmelugu, mis ilmus kirjastuse Eesti Raamat väljaandes 1982. 
aastal. Raamatu peategelased kohtuvad muu hulgas arvutiga nimega Kunigunde.}? 
Seal toimus kohutavalt põnev tegevus: tüübid tegid oma Musta Kassi 
ordu ja selle käigus käisid vist TPIs ning tegid mingi skeemi 
valmis. See on tagantjärele mõeldes naeruväärne, aga väikse poisina tundus 
tohutult põnev. Sealt tekkiski arvutihuvi. 

Edasi oli kaks liini. Mu onu oli IT valdkonnas tegev 
tükk maad varem kui mina, ta oli juba sügaval 
nõukaajal arvutite kallal. Käisin Tartus tema juures ja see kõik oli kohutavalt põnev.

\question{Mis selle põnevaks tegi?}

Põnevad olid igasugused nupud ja see, et asjad toimusid. 
See oli väikese poisi jaoks unistus, et mingit masinat saab täielikult kontrollida ja et jõud võiks sellest üle käia.

Teine liin oli Tallinnas, Jaak 
Loonde\index[ppl]{Loonde, Jaak} Luise tänava\index{Tallinna 
Oktoobrirajooni Õppetootmiskombinaat} klass, kus olid Yamaha 
MSXid\index{Yamaha MSX}\sidenote{Seal asus Tallinna Oktoobrirajooni Õppetootmiskombinaat, mida on vahel paigutatud ka Roopa tänavale ja mille kohta öeldakse ka lihtsalt Luise tänava klass.}. Sama tüüpi aparaat, mis mul siin külje all 
seisab.

\question{Mis klass see Jaagul oli? Kas see asus mõne kooli juures?}

See oli pigem huvialamaja, ma täpselt ei mäleta. Igatahes sai seal klassis käia arvuti taga istumas ja nikerdamas. Noored nagad tahtsid 
loomulikult hullupööra mängida, aga kurjajuur Jaak Loonde
ütles, et ei-ei, tuleb hirmsal kombel ikka programmeerida. Nii oligi 
tasakaal nende kahe asja vahel üsna hästi paigas, sest niipea kui 
Jack kõrvale vaatas, olid poisid kohe mingid asjad käima tõmmanud. Kui 
keegi lasi võrku mängu, siis said kõik seda endale laadida. 

\question{Kust mängud tulid? Poest neid osta ju ei saanud.}

Klassis oli õpetaja arvuti ja õpilaste töökohad. Nende vahel oli võrk, mis 
oli naljakal kombel üles ehitatud MIDI kaabli otsa. MIDI on küll rohkem mõeldud muusikariistade juhtimiseks – Yamahad olid tegelikult algselt 
muusikaarvutid ja olid läbi muusikasüsteemi kohapeal võrku pandud.

Mängud liikusid kassettidel ja ketastel. Aeg-ajalt käis keegi välismaal. Mäletan, kuidas TTÜs tuli keegi välismaalt 
mingi teadustööga seotud ettevõtmiselt tagasi ja pani lauale kolm kolmetollist flopit. Kõik seisid kõrval ja ootasid, mis 
nende peal on. See oli kohutavalt harras hetk.

\question{Tol ajal ju osta ega alla laadida kuskilt midagi ei olnud, kõik 
käis käest kätte.}

See oli keeruline jah, sest oli sügav nõukaaeg, 
kaheksakümnendate keskpaik. Servast hakkasid vabaduse kiired juba 
terendama, aga need ei olnud kuskilt otsast veel materialiseerunud. Põhiline 
värk oli see, et inimestel lasti teha asju, mille eest varem 
oleks türmi pistetud.

\question{Kas Luise tänaval käisid keskkooli ajal?}

See oli enne keskkooli, olin umbes 13-14aastane – kael juba kandis, aga mitte väga. 
Täiskasvanuks veel ei peetud, selline ebamäärane aeg, kui veel ei
tea, mis sust saab.

\question{Kas ühel hetkel hakkasid tulema BBSid ja Fidonet?}

Need olid tükk aega hiljem. Me kasvasime suuremaks ja sisi hakkasid ka täiskasvanud meiesuguseid nolke 
tõsisemalt võtma. Enne rääkisime põgusalt, et kui nõukaaeg lõppes 
ja Eesti aeg algas, siis esimene palk oli 300 
krooni, mis praegusel ajal on 20 eurot. See oli kuupalk ja sellest elas kenasti ära. Mitte küll nii, et oleks midagi hullupööra huvitavat selle eest 
saanud osta, aga elas ära. See võtab tegelikult nõukaaja 
elatustaseme üsna hästi kokku: kuna asjad, mis meil siin ringi liikusid, olid 
valmistatud oma Normas, Salvos või Kommunaaris, siis olid nii hinnad kui ka sissetulek väiksed. Asjad 
olid tasakaalus. 

\question{Arvutit 20 euro eest ei osta.}

Arvutit tõesti ei osta, aga need vahendid olid olemas asutustel. 
Teine asi oli see, et Nõukogude Liitu oli keelatud eksportida kõvemat 
arvutustehnikat. Et meile tekkisid siia näiteks MSXid, oli osaliselt 
tingitud sellest, et tegu oli suhteliselt alumise otsa masinatega, rohkem 
mänguasjade kui päris tööriistadega. Näiteks kui 
Soomest veeti Eestisse üks 386, kui need oli just äsja välja tulnud, siis sündis
sellest kohutav rahvusvaheline skandaal. Ühendriigid võtsid soomlastel kõri 
pihku ja ütlesid, et mis mõttes te veate Nõukogude Liitu niisugust 
tehnoloogiat, millega on võimalik rakette arvutada ja mida iganes teha! Meie 
mõistes oli see masin tol hetkel väga kõva sõna. Praegu on see muidugi 
naeruväärne, suvaline kell on ka võimsam.

\question{Ehk et arvutile ligi saada, pidi mõnele asutusele külje 
alla pugema.}

Jah. Asutustel olid arvutid, millega nad üritasid oma asjatoimetusi läbi viia.
Arvuteid oli mitmesuguseid, näiteks klassikalisi nõukaaegseid Uralitel 
põhinevaid süsteeme, kus olid terminalid ja suured kastid.
Ajapikku tekkisid ka muud, välismaise päritoluga masinad, peaasjalikult 
286d ning koos nendega võrgud. Seega oli vaja 
inimesi, kes seda kõike haldaksid. Aga inimestest oli põud, kuna 
keegi ei teadnud, mida nende kastidega peale hakata. Ja kui seal kõrval 
nikerdamas käisid, siis muutusid päris kähku kasulikuks. Noore poisina oli 
aega, ei mingeid perekondlikke kohustusi ega muud säärast, huvi oli suur ning 
saidki seal eksperimenteerida. Läksid õhtul pärast kooli sinna, nemad läksid 
töölt ära ja said seal istuda kuni üheksa-kümneni. Ja kokkulepe oli see, et üritad kuidagimoodi kasulik olla, ning 
lõpuks sai arvuti taga istumiskohast töökoht, kui kool läbi sai.

\question{Mis asutuses sina käisid?}

Minul oli alguses linnavalitsus\index{Tallinna Linnavalitsus} ja hiljem 
riigikantselei\index{Riigikantselei} – kogu selle 
huvitava perioodi, kui Eesti Vabariik välja kuulutati, töötasin ma
riigikogu majas. Seda nimetati 
vist peaministri kantseleiks, Stenbocki maja siis veel ei olnud. Seal olid ka 
Uralid\index{Ural}\sidenote{Nõukogude Liidus Pensas aastatel 1956–1964 toodetud arvutite sari.}, 
üüratud kapid, mille sees olid viiemegabaidised trummelkettad, mis tuli 
hommikul käima lükata.

\question{Kas sul akadeemiline haridus jäi pooleli?}

Mul on jah lõpetamata kõrgharidus. 
Üritan seda seniajani lõpetada ja loodetavasti paari aasta 
jooksul seda ka teen.

Tol ajal oli valida, kas tegeled arvutitega või õpid. Suuresti oli mu valik 
väga selge: ma õppisin arvuti taga oluliselt rohkem.

\question{Millal Fidonet Eestis jalad alla võttis?}

Ma täpset aastaarvu ei oska öelda, aga sellega tegutsesid 
Tõnis Reimo\index[ppl]{Reimo, Tõnis}, Tarmo 
Ausing\index[ppl]{Ausing, Tarmo} ja Virko Püss\index[ppl]{Püss, Virko}. Lisaks
jõlkusime seal mina ja Miko Raud\index[ppl]{Raud, Miko}.

\question{Kus te tegutsesite?}

Erinevates kohtades, näiteks Narva maanteel. Eks grupi tuumik 
teab nüansse paremini kui mina. Vahepeal tekkis seal 
võimalusi peaasjalikult soomlastega asju ajada, kui avastati enda jaoks BBSid ja aduti, et tarkvara peab ka kusagilt tulema. Sel ajal
hakkas lisaks flopidele tekkima võimalus modemiga asju alla laadida ja tulid 9600sed modemid.

Asjad hakkasid jumet võtma. Tol ajal olid tarkvarapaketid maksimaalselt paari 
mega baidised ja olid tõmmatavad umbes päevaga.

\question{Seega helistati Soome BBSidesse sisse?}

Jah. 

\question{Kuidas see käis? Jaan Tallinn\index[ppl]{Tallinn, Jaan} on 
rääkinud läbi inimoperaatori arvuti külge helistamisest.}

Igasuguseid imeasju tehti. Näiteks selgus, et lifti 
telefoniühendusest oli võimalik välismaale helistada, sest keegi polnud taibanud 
seda sealt välja lülitada. See tähendab, et liftist sai helistada ja öelda, et 
appi-appi, olen siia kinni jäänud, aga sellesama ühendusega sai helistada ka Soome. Keegi ei olnud nõukaajal kindel, kes 
selle kinni peab maksma, ja seetõttu jäigi see kuidagi ripakile. 
Loomulikult olid ligipääsud erinevatele keskjaamadele ja raha 
tekkis kusagil süsteemides ning kadus kuhugi, nii et 
tegelikult kasutati seda ühte- või teistpidi kurjasti ära. See oli üks viis Nõukogude süsteemi õõnestada.

Sellega seoses tekkisid kontaktid. Näiteks BBSi sisse logides vaatas
\emph{sysop}, et ohoo, Eestist 
mingid tüübid, ja tahtis paar 
sõna juttu ajada. See oli üsna tavaline, et BBSi operaator rääkis külalistega.

BBS ei olnud väga erinev tänapäeva 
sotsiaalmeediast. BBS pandi püsti kahel põhjusel: esiteks, et kontakte luua ja
\emph{networking}'ut teha, olla nii-öelda elu pulsil. Teiseks millegi
propageerimiseks, näiteks oli BBS mõne firma juures või oli 
mingisuguse demo grupp enda oma püsti pannud. See BBS, kust me esimese 
kontakti saime, oli Poison Door\index{Poison Door}.

\question{BBS võis ka mingi demo grupi juures olla, soomlaste 
\emph{demoscene}\sidenote{Demo on arvutikunsti teos, mis kujutab endast 
terviklikku, sageli väga väikest arvutiprogrammi, mis esitab 
audiovisuaalset vaatemängu. Demo eesmärk on demonstreerida (nagu nimigi 
ütleb) autorite programmeerimise, visuaalkunsti ja arvutimuusika oskusi. Demode 
ümber tekkis kogukond, \emph{demoscene}, mis sai kokku demopidudeks kutsutud 
festivalidel. Üks kuulsamaid on siiamaani regulaarselt Helsingis 
toimuv Assembly.} oli tol ajal väga kõva.}

Mul endal oli kontakt \emph{Future Crew}\index{Future 
Crew}\sidenote{Soome demogrupp, mis peamiselt tegutses aastatel
1987–1994. Nende tehtud oli tõenäoliselt kõigi aegade mõjukaim demo 
„Second Reality“ (avaldati Assembly demopeol 1993. aastal). See tegi 
tänapäeva mõistes olematu riistvara peal reaalajas asju, mis tundusid täiesti 
võimatud, nägi üliäge välja ja sisaldas muusikat, mis siiani kananahka tekitab. 
1999. aastal hääletasid Slashdoti lugejad selle demo kõigi aegade kümne 
vingeima häki hulka.} tüüpidega. Ma laadisin nende BBSist alla niisuguse 
toreda mängu nagu „Wing Commander“\index{Wing Commander}. 
Omadele anti asju, mis tegelikult ei olnud
päris ametlikult väljas. Peaaegu kõikidel BBSidel olid tagatoad, 
kus hoiti nodi, mida kasutati vahetuskaubana. Tarkvara oli sel 
ajal kõva valuuta. Me panime püsti kahepoolse ühenduse: mina 
laadisin üles mingi muu asja, mille olin kusagilt saanud, ja 
sealtpoolt tõmbasin vastu „Wing Commanderit“ ning samal ajal sai rääkida ka. 
See tarkvara võimaldas kahepoolset sidet ja samas ka 
\emph{chat}'ida, mis ei võtnud väga palju ühenduse mahtu.

\question{Iga klahvivajutus oli üks sümbol, fondi või värvide 
informatsioon kaasa ei liikunud.}

Just, see oli tavaline tekst. Kogu mängu allatõmbamine võttis 
aega tunde. Selleks ajaks olin endale ise ühenduse sebinud, Riigikantseleil oli
selline võimalus nõukaaja lõpus. 

Ühesõnaga, tutvuti ja info liikus. Ja oli aja küsimus, millal lõpuks siingi 
oma BBS püsti pandi ja Fidoneti kontakt saadi. Ma just hiljuti uurisin selle 
kohta ja paistab, et Fido on nüüdseks lõplikult hinge heitnud.

\question{Üsna kaua võttis aega!}

Võttis küll, aga võibolla on mõttekas see uuesti üles tõmmata. See eksisteerib endiselt ja 
tänapäeval on retroasjad moes, nii et ehk ärkab see kunagi uuesti ellu.

\question{Mis oli Eesti üks esimesi suuri BBSe, kus rahvas hulgakaupa sees 
käis?}

Esimene tõsiseltvõetav BBS, just nimelt Fidoneti mõistes, oli Hackers Night 
System\index{HNS}\index{Hackers Night System|see{HNS}}. Nagu 
nimigi ütleb, oli tegu häkkerite öösüsteemiga. Päeval olid telefoniliinid muuks 
otstarbeks, öösel käis nende peal BBSidesse helistamine. 

\question{Miks sel ingliskeelne nimi oli?}

Et oleks rahvusvaheline ja äge.

\question{Kes HNSi käigus hoidis?}
Seesama kamp: Reimo\index[ppl]{Reimo, Tõnis}, Ausing 
\index[ppl]{Ausing, Tarmo} ja Virk\index[ppl]{Püss, Virko}.
 
\question{BBSi jaoks oli ju mingit riistvara ja modemeid vaja?}

Oli jah, sinna juurde käis paras sebimine. Tol ajal olid vahendid suuresti riigi rahakotis. Selle küljes siis 
istuti ja kui oldi juba kasulikud, siis sai alati ka neid ressursse juhtida õiges suunas. 

\question{Mis aastal see oli?}

Kaheksakümnendate lõpus, mitte 1989, vaid varem. Ma täpselt ei mäleta, 
vanus oli selline, et keegi ei olnud veel täiskasvanu, aga ka mitte enam laps. Aeg omas siis teist tähendust ja nüüd hiljem on
raske mõõtkava peale panna.

\question{Kas tol ajal oli BBSil üks modem ja üks liin?}

Ojaa. Tegelikult oli muid ka, paralleelse side katseid, näiteks 
PirnBox\index{PirnBox}. Fidoneti mõistes klassikalistest BBSidest oli HNS esimene ja sealt läks 
asi krõbinal laiali.\sidenote{Pronto ise pidas BBSi New Age 
System\index{New Age System} Fidoneti aadressiga 2:490/12.}

\question{Kui palju neid BBSe tipphetkel 
oli?}

Tipphetkel oli 20–30. Süsteem nägi ette \emph{point}'e, mis olid nii-öelda pool-BBSid. Täpsemalt olid \emph{point}'id ja \emph{full 
node}'id. \emph{Node}'il olid kohustused: meile tõmmata, hoida ja 
jagada. \emph{Point}'iga sai lihtsalt tõmmata. Paljud 
BBSid otsustasid \emph{point}'iks olemise kasuks puhtalt sellepärast, et need 
ei saanud ennast kogu aeg käimas hoida. \emph{Node}'idel olid \emph{point}'id, keda nad varustasid 
informatsiooniga, ja \emph{node}'i käimas hoidmine eeldas ühte- või teistpidi 
võimekust olla teatud hetkedel üleval.

\question{Seega oli Eestis tol ajal 
paarkümmend inimest, kellel oli võimekus sebida liin ja riistvara ning ka 
süsteemi käigus hoida.}

Tipphetkel küll jah. Vahepeal sai nõukaaeg otsa ja tuli Eesti Vabariik ning ühel 
hetkel hakkas asi selles mõttes käest ära minema, et raha hakkas omama 
tähendust. Enam ei saanud lihtsalt kusagil ettevõtte küljes istuda ja oma
asju teha. BBS koos telefonikõnedega tekitas kulusid ja peod
hakkasid vaikselt kinni minema. Inimesed vahetasid töökohti ja uutes 
kohtades ei vaadatud selle peale enam lahke pilguga.

\question{Kuidas sellest ürgsupist Eesti arvutifirmad tekkisid? Kas BBSide 
seltskond läks sujuvalt üle teenuste pakkumisele?}

Osaliselt küll. Need inimesed olid ühte- või teistpidi 
arvutifirmadega seotud, aga tihtipeale ei olnud need päris samad 
inimesed. Teatavasti on sogases vees kõige parem kala püüda, seal on kõige 
suuremad purikad. Sogasel ajal leiti erinevaid viise, kuidas endale 
raha teha. Näiteks Peterburist veeti autoga Tallinnasse igasugust IT-tehnikat. Peterburis olid punktid, kust sai asju 
osta ja Eestisse tuua. Nii see elu vaikselt edenes.

\question{Kas sina olid sel ajal veel Riigikantseleis\index{Riigikantselei}?}

Jah, aga oli näha, kuidas hakkasid tekkima esimesed firmad, mõned edukad, 
mõned vähem edukad. Ühel hetkel läksin Riigikantseleist 
minema, sest ka seal toimusid struktuurimuudatused.

\question{Mida sa tol ajal peamiselt arvutiga tegid? Kas kirjutasid 
koodi?}

Nüüd tundub see ehk naljakas, aga siis oli see nagu eluviis. Ega see väga ei erinenudki praegusest eluviisist, vahe on ainult selles, 
et nüüd ei pea näiteks Facebookile ligipääsu saamiseks kulmulihastel ringi roomama. Tollal ei olnud see kõikidele kättesaadav. 

Arvuti kasutamisel oli siis küllaltki kõrge lävi, mis eeldas teatud ülevaadet tehnikast ja võimalustest. Praegu on internet ise ennast 
sõlme tõmmanud, aga varem pidi täpselt teadma aadresse, 
kuhu minna, sest polnud otsinguid. Siis alles hakkasid tekkima esimesed 
otsingumootorid: WebCrawler, AltaVista ja lõpuks Google. Need tõmbasid läve madalaks. 

\question{Mida BBSiga teha sai?}

Sai faile jagada ja kirju vahetada. Fidonet oli tänapäeva mõistes suuresti
interneti meilisüsteemi sarnane. Olid ka uudisegrupid ja \emph{usenet}'i grupid, mis on asendunud näiteks Facebooki ja Redditiga, kus käib 
info vahetamine.

\question{\emph{Usenet}'i grupid olid tollal hierarhilised, aga praeguseks on see struktuur laiali vajunud.}

Jah, olid hierarhiad ja etiketid, mida võhikul oli väga raske 
aduda. Tihtipeale inimesed tundsid küll üksteist üsna lähedalt, aga 
teinekord mõnd jutuajamist jälgides tekkis täieliku \emph{outsider}'i tunne, kui ei saanud aru, millest jutt käib. Kõikidel oli oma taust.

\question{Kus inimesed tuttavaks said? Kas nendes gruppides?}

Oli kaks varianti. Keegi tutvustas ja aitas ree peale või siis kiibitsesid mõnda 
aega ja ühel hetkel hakkasid aru saama, mis toimub. Kui üldse hakkasid, see ei olnud lihtne.

\question{Kui tihedalt Eesti Fidoneti seltskond omavahel läbi käis?}

Seltskond pidi paratamatult läbi käima, sest Fidoneti tekkides moodustusid ka grupid, kus tuli sisu 
tekitada. Ja kuna esialgu oli inimesi vähe, siis paratamatult ei olnud ka
kommunikatsioon meeletult tihe. Fidonetiga tegeles 
paar-kolmkümmend inimest ja isiklikult tuttavaks saamine ei olnud keeruline.

\question{Kes need inimesed olid?}

Enamasti samasugused IT valdkonna inimesed nagu mina, kellel olid
sarnased huvid – meil oli, millest rääkida. Olid ka 
teemad, mis siis olid parajasti \emph{zeitgeist}. 
Näiteks „King's Quest 
IV“\index{King's Quest} mängides ei olnud võimalust minna 
veebi ja otsida \emph{walkthrough}'d. Inimesed üritasid omal jõul 
läbi närida ja aeg-ajalt vahetati kogemusi. Muide, sellest ajast pärineb 
ka Habichti raamat \enquote{Selles mängus ei hüpata}\sidenote{Juhan Habichti novellikogumik, mis ilmus 1993. aastal kirjastuse Katherine väljaandel.}. 
See mäng oli 
„Larry“\index{Larry}\sidenote{„Leisure Suit Larry“ oli Al Lowe'i
loodud seiklusmängude sari, mis ilmus aastatel 1987–2009 ning oli 
tuntud omapärase huumori ja alaealistele sobimatu sisu poolest. 
Näiteks katsus mängija riiulil seisvat kopratopist ja Larry teatas: \enquote{\emph{I've always 
liked the feeling of a good beaver}}.}.

Samuti räägiti võimalustest ja nende 
vahetamisest. Ühel oli üks asi, teisel teine ja pandi seljad kokku. Kuna 
inimesi oli vähe ja üksteist teati, siis ei olnud ka väga 
suurt kanakitkumist.

\question{Kas trollimist või muud säärast ka toimus?}

Kui auditooriumi ei ole, siis inimesed jäävad tavalisteks inimesteks. Kui 
annad normaalsele inimesele anonüümsuse ja publiku, siis saab tast igavene 
tõpranahk.

\question{Isegi kui sul oli \emph{handle}, siis sa ju tegelikult ei olnud anonüümne.}

\emph{Handle} oli lihtsalt nimi, tegelikult kõik teadsid, kes on kes. 
Isegi kui olid anonüümne, siis ei 
kasutatud laest võetud nimesid. Kui olid oma nimele
feimi tekitanud, siis sa ju ei tahtnud sellega 
uisapäisa ringi käia.

\question{Sa oled siiamaani Pronto ja teistele tähendab see senini midagi. 
Kui keegi hakkas sigatsema, kas ta visati siis välja?}

Jah, juhe tõmmati seinast ja olid kohemaid \emph{persona 
non grata}. Kuna see oli seotud sinu enda huvidega ning
mineviku, oleviku ja tulevikuga, siis ei saanud seda omale lubada.

\question{Seega käitusid kõik viisakalt?}

Kõik olid seal paadis võrdsed. Kui keegi hakkas paati 
kõigutama, siis ta kõigepealt kõigutas seda enda all ja kui ta seda jätkas, siis ta lihtsalt eemaldati paadist ja pidi ise vaatama, kuidas 
veekogus hakkama saab.

\question{Kas seda juhtus ka?}

Otseselt mitte või kui juhtus, siis juba hilisemal ajal. 
Alguses oli ikkagi tihe seltskond ja kuigi kõik ei saanud omavahel 
ideaalselt läbi, mõisteti, et selles paadis ollakse koos. Seetõttu tüli põhjustada võivaid teemasid
lihtsalt välditi.

\question{Seega saadi aru, et teatud asjadest ei tasu rääkida.}

Jah. Trollimine ju ongi rääkimine asjadest, mis teisele 
inimesele peavalu valmistavad.

\question{Tuleme sinu juurde tagasi. Kui sa 
Riigikantseleist\index{Riigikantselei} ära tulid, mida sa siis tegid?}

Töötasin sellises kohas nagu Marvin-Ekspert, sain seal ostmise ja müümisega käe valgeks. Tegelesin selliste toodetega nagu Gravis Ultrasound ja 
IOMega\sidenote{Gravis Ultrasound oli toona PC-maailmas tipp helikaartide tootja ja IOMega tegeles väga innovatiivsete andmesalvetuslahendustega.}.

See oli selles mõttes huvitav aeg, et Gravis Ultrasound maksis väikse 
varanduse, aga samas oli see tükk maad parem kui mõni teine toode. Müüsin neid umbes sama palju kui kõiki 
ülejäänud asju kokku müüdi, kuigi see oli kallis. Mõnes mõttes oli sellega sama lugu nagu 
Apple'iga: kallimat asja on alati lihtsam müüa, sest kalliduse taga on tavaliselt 
väärtus, toode ei ole kallis niisama.

\question{Mis aastal see oli?}

Ilmselt 1994. või 1992. aasta kanti.

\question{Kas sel ajal hakkas tasapisi Microlink tekkima?}

Microlink tekkis tegelikult üsna aegade alguses. See oli üks nendest firmadest, kes 
alustas sellest, et hakati kotiga Peterburist asju tooma. Esialgu müüdi 
arvuteid firmadele, sest nendel oli raha, kuigi 
omandisuhted polnud veel päris paigas.

\question{Kapitalismi oli veel vähe?}

Kapitalismi oli jah vähe, olid veel nõukaaja jäägid --- keegi oli kuskil käpa peale 
pannud. Oli niisugune aeg, kui ma paljusid asju ei teadnud ja 
paljusid teadsin, aga ei tahtnud teada. Asju, mille kohta võib öelda, et mis juhtus Vegases, las jääb Vegasesse. Sel ajal tehti 
igasuguseid asju, mis praegu võivad näida küsitava 
eetilise ja moraalse taustaga, kuid siis olid 
tegelikult õiged ja vajalikud.

\question{Tol ajal ju kujuneski välja, mis on õige ja mis mitte.}

Jah. Loomulikult tehti sel ajal igasugust erastamist ja ärastamist, aga ka see oli 
hädavajalik puhtalt sellepärast, et tookord tehtud otsused eristavadki meid 
tänapäeva Moldovast, kus omal ajal tehti teistsuguseid otsuseid. Isegi 
need, kes meil siin ärastasid, tegid seda teataval määral 
\enquote{eesmärk pühendab abinõu} kaalutlustel.

\question{Räägi pisut ka ajakirjast .EXE\index{.EXE}.}

See ajakiri oli osaliselt Microlinki püüd ennast nähtavaks 
teha. Eestis oli tollal kaks arvutiajakirja: 
Arvutimaailm\index{Arvutimaailm} ja .EXE. Arvutustehnika \& 
Andmetöötlus\index{Arvutustehnika \& Andmetöötlus} ei olnud klassikalises mõistes ajakiri, vaid rohkem 
vihik. Nõukaaja lõpus ja Eesti aja alguses anti välja vihikuformaadis 
erialaväljaandeid, mis ei olnud mõeldud laiaks tarbeks.

.EXE tekkis umbes samal ajal kui Arvutimaailm, Microlink 
püüdis tekitada endale laiatarbeväljundi.

\question{Selles ei olnud palju laiatarbeasju, 
vaid stiilipuhas \emph{hard core} küberpungijutt!}

.EXE oli selles mõttes \enquote{laiatarbeväljund}, et sel ajal ei olnud inimestel 
raha arvuti soetamiseks. Pidi olema ikkagi väga suur tahtmine ja vastavalt sellele kujunes ka ajakirja sisu. 

Sel ajakirjal oli ajastu hõng juures. Mida inimesed arvutiga parasjagu tegid, 
see sealt ka läbi kumas. 

\question{Kuidas sa selle juurde sattusid? Kas kirjutasid juba enne .EXEt?}

Tol ajal kirjutati näiteks naljaviluks mängudest, sisu toodeti vabatahtlikult. 
Gruppidesse postitati dokke, häkiti ja nii edasi. Kuna ma olin mängudest kirjutamisega silma 
paistnud ja ka kirjaoskus enam-vähem olemas, siis nii ma ajakirja sattusin. 

See oli päris 
naljakas aeg – ajakirja koostamine oli omamoodi 
häkkimine. Tavalist kogunes kolleegium (seltskond, kes sisu kokku pani) 
kokku, lükati ette kaks kasti õlut ja enne toast välja ei 
lastud, kui ajakiri oli kokku pandud. Igaüks võttis endale mingid kohustused 
ja kadus nendega tegelema.

\question{Kaua .EXE üldse ilmus?}

See ilmus umbes poolteist aastat.

\question{Nii vähe?}

Jah, ma ühel hetkel korjasin kõik numbrid kokku\sidenote{Aadressil 
\url{punktexe.ee} on kõik ilmunud numbrid täies mahus olemas.}. Esimene number ilmus aprillis 1993 ja viimane 
1995. aastal. Nii et kaks aastat, vahepeal läks ilmumine eklektiliseks.

\question{Kes neid ilusaid kaanepilte joonistas?}

Kaspar Loit\index[ppl]{Loit, Kaspar} alias BKnows.

\question{Arvestades, milline mõju ajakirjal oli, palju seda loeti ja kuidas fännati, 
siis oli ilmumise lühidusest hoolimata tegu väga mõjuka asjaga.}

Jah, numbreid oli kokku vist kaheksa. Igaüks oli omaette šedööver, kuna see oli südamega tehtud, eriala inimestelt eriala inimestele. .EXEt anti välja selleks, et skenet juurutada, mitte et selle pealt 
üüratut kasumit teenida. 

\question{Millist skenet? Arvutiinimeste oma?}

Jah. Inimesele tänavalt
oli see ajakiri ehk pisut raskevõitu. Tol ajal oli 
arvutiajakirjandus teistsugune kui praegu, mil
igaühel on arvuti ja loetakse, kuidas oma mobiiliga 
ühte, teist või kolmandat teha. Arvuti oli siis suur asi, 
seda polnud kaugeltki mitte kõigil. Praeguses mõistes üks-kaks protsenti inimestest tabas tegelikult reaalselt arvutit ja oskas seda 
igapäevaelus kasutada.

\question{Seevastu inimesi, kes tahtsid kasutada, oli rohkem. Ja nii nad lugesidki hardalt, kuidas Pronto seikleb „Day of the 
Tentacle'is“\index{Day of the Tentacle}\sidenote{Legendaarne mäng, mis ilmus 1993. aastal LucasArtsi väljalaskel ja uuendatud graafikaga 
2016. aastal ning mille \emph{walkthrough} avaldati .EXE teises numbris 
novembris 1993, autoriteks BKnows\index[ppl]{BKnows} ja 
Pronto\index[ppl]{Pronto}.}}.

Eks see kõik hakkaski pisitasa tuult tiibadesse võtma. Sel ajal toimus 
jõhker inflatsioon ehk räägitud kahekümnest eurost said päris kiiresti 
sajad eurod. Arvutid muutusid jõukohaseks ka teistele ja Rootsist veeti siia humanitaarabi korras pruugitud tehnikat.

\question{Kas kogu selle ajal jooksul müüsid sina muudkui Gravist?}

Gravist ja Iomega Bernouilli draive\sidenote{1992. aastal turule tulnud, oma aja kohta suure mahutavuse ja 
eemaldatava kettaga salvestussüsteem Bernouilli Box 
oli Iomega esimene laialt kasutust leidnud toode.}, QIC-80 
teipe ja muud säärast.

\question{Huvitav, et mitmed inimesed on teatud faasis tegelnud just 
arvutustehnika müügiga.}

Kuskilt tuleb raha teenida. Kätte jõudis aeg, kui varad said laiali 
jagatud ja sa pidid oma tegevust põhjendama, näiteks miks sul on 
BBS. Ainuke võimalus seda asja edasi edendada oligi müügi 
egiidi all.

\question{Kas koodikirjutamisega ei saanud elatist teenida?}

Tol ajal ei olnud eriti mingeid koode, mida kirjutada. Väikseid asju loomulikult oli, aga valdavalt käis koodikirjutamine 
andmebaaside ümber, näiteks olid FoxBase ja DBase, kus tehti 
ettevõtete raamatupidamist ja inventuuri. 

\question{Kas iga ettevõte pusis endale ise tolle rakenduse kokku?}

Kas ise või osteti firmadelt, aga süsteem koosnes tavaliselt 
mõnest andmebaasilahendusest. Oli ka muid asju, 
näiteks meditsiiniga seotud lahendusi, millel olid juba infosüsteemid, aga 
need olid väga spetsiifilised ja neid arendati enamasti väikses mahus.

\question{Eestlane üldiselt ei ole suurem asi müügiinimene, 
aga IT-asja on meil õnnestunud rahvusvaheliselt päris hästi müüa. Kas ehk
seetõttu, et kriitilisel hulgal inimestel on olnud müügikogemus?}

Kindlasti. Tol ajal oli see paratamatu, sest kui tahtsid 
saada ligipääsu, pidi juba siis ennast müüma. See on üks asi, mis on muutnud 
vana kooli IT-vennad teistsuguseks – sa pidid paratamatult suutma müüa. Kui ei suutnud, siis polnud sul IT valdkonda asja. Kõige 
tähtsam kaup olid sa ise.

\question{Sest muud sul ei olnud?}

Muud ei olnud, isegi mitte kogemusi, sest kogemused tulevad töö käigus. Sa pidid suutma endast teha väga vajaliku tegelase.

\question{Nii et kui enesemüügi oskus on olemas, siis võib igasuguseid 
asju juhtuda.}

Kui tähelepanelikult vaadata, siis IT valdkonna müügis ongi 
läbimurrete taga tihtipeale ühed ja samad inimesed ning just 
vana kooli kaader, kes enamasti on oma läbimurde ehk müügi saavutanud mitte 
tänu avalikkusele, vaid vaatamata sellele. Teatavasti tunneb avalikkus 
kohemaid muret, kui keegi teenib paremini või tunneb ennast kuidagi paremini. 
Hari läheb kohe kadedusest punaseks.

\question{Tihti öeldakse, et meil on vedanud, sest õiged inimesed on sattunud 
õigetesse kohtadesse. Sinu jutust tuleb välja, et tollest seltskonnast tulidki 
inimesed, kes sattusid õigetele kohtadele.}

Täpselt nii. Need inimesed on siiani 
alles, osa neist üle viiekümne, osa alla selle, aga üks või teine on suuremate 
läbimurrete taga.

\question{Oskad sa öelda, mis 
hetkel kaotas see maailm oma süütuse? Kui romantilisest õllekasti abil 
toimetamisest sai raha teenimine.}

Ma ei oska seda niimoodi paika panna, sest tegelikult on see 
ikkagi suuresti väljaspool loodud kuvand. Kui on mingisugune grupp, 
siis paratamatult tekivad autsaiderid, kes tunnevad pahatahtlikku 
kadedust. 

\question{Ja nimetavad inimesi häkkeriteks?}

\enquote{Häkker} hakkas omandama lihtsalt teistsugust tähendust.

\question{Viidates ühele .EXE loole, mis on 
küberpunk\sidenote[][-2cm]{Allkirjastamata, kuid BKnowsi\index[ppl]{BKnows} piltidega 
lugu \enquote{Kes sa selline oled, küberpunk?} ilmus .EXE kolmandas 
numbris 1994. aasta aprillis. Seejuures tuleb tunnustada artikli asjakohasust: nii ilmumise (eba)regulaarsuse ja lühiduse kui ka kultusliku staatuse poolest .EXEga sarnane, kuid suurema levikuga ajakiri MONDO 2000 (aastatel 1984–1998 ilmus USAs 17 numbrit) avaldas oma samateemalise satiirilise artikli \enquote{R.U. A CYBERPUNK?} oma 10. väljaandes 1993. aastal.}?}

Kõik asjad, mis on punk, nagu aurupunk, küberpunk või diiselpunk, on lihtsalt 
žanr, mis läbib mitut asja; valdavalt seda, kuidas siduda teadvus 
tehnikaga. Mõnes mõttes on meie ühiskond praegu nii-öelda küberpungi jaoks 
esimesel tasemel, sest see, kui inimesed istuvad ninapidi telefonis, on 
lihtsalt liidestamise küsimus. Inimesed on ennast tegelikult arvutiga juba väga 
intiimselt liidestanud.

\question{Nagu sa mainisid, siis algas see juba kaheksakümnendate lõpus, kui 
kogu sinu elu oli arvutis. Lihtsalt liides oli kandilisem.}

Liides oli kandilisem ja olemas vähestel inimestel; seetõttu polnud see elu, vaid 
mu \emph{alter ego}. See ongi üks põhjus, millepärast valiti omale sellised
tunnused, nagu mul on Pronto – et teha vahet sellel, mis toimub arvutis ja mis 
niisama. Põhimõtteliselt loodi endale identiteet. 

\question{Just nimelt loodi, mitte ei valitud!}

Ja sellega elati osaliselt tulevikus, aga ka muu elu jäi alles. Pere, sõbrad ja see õlu, mida joodi, jäi kõik teise ellu.

\question{BBSi rahvas käis ju koos ka.}

Käis küll. Kõigepealt olid \emph{sysop}'ide saunad ja muud üritused, kust kasvas välja Fidonet. Hiljem tekkisid
BBSummerid\index{BBSummer}.

\question{Kui palju neid toimus?}

Need said alguse nõukaaja lõpus ja neid toimus üksjagu. Üks 
BBSummeritest, vist teine või kolmas, lükati edasi sellepärast, et tankid sõitsid Eestisse 
sisse.

\question{Olen näinud BBSummeri pilte, mille peal on kõik Microlinki, Skype'i, Unineti ja 
teiste hilisemate suurte asjade alustajad. Kas tol ajal, asja sees 
olles, ei olnud niisugust tunnet, et oi, küll me oleme ägedad?}

Muidugi oli! Me olimegi hullult ägedad! See oli ka üks põhjus, miks me sellega tegelesime.

\question{Tulles meie jutu alguse juurde tagasi, kas selle ägeduse tuum oli 
jätkuvalt see, et sai masina mõne näpuliigutusega oma tahtele allutada?}

Kindlasti. Teiseks ei piirdunud elu enam oma õuega, vaid koos Fidonetiga tekkis ka ülejäänud maailm sinna otsa. See ei 
olnud väga erinev tänapäeva Redditist, Facebookist või Twitterist, kus ei
saa suhelda mitte ainult paari lähema tuttavaga, vaid kogu ülejäänud 
maailmaga. See andis 
näiteks võimaluse keeli omandada ja suhelda erinevates keeltes, mis omakorda aitas edasi.

\question{Nii et see tekitas maailma avardumise tunde?}

Maailm avardus kindlasti. See oli mõneti samasugune tunne nagu 
kosmonaudil, kui ta atmosfäärist väljub. Eriti kui see pind, millelt üles 
tõusti, oli tükk maad madalamal kui enamiku maailma jaoks --- me tegime
otse nõukaajast sammu tulevikku.

\question{Ühel hetkel olid Nõukogude pioneer ja pisut hiljem 
vestlesid California kuttidega keskjaamadest.}

Jah, absoluutselt. Tekkisid võimalused ja kogemused. Näiteks mõnes mõttes 
positiivne nähtus oli see, et Eestis puudusid \emph{legacy} süsteemid, meil 
polnud IT valdkonnas mineviku taaka, vaid asi oli lihtsalt poolik. 
Mineviku taaga puudumine võimaldas Eestil kihutada päris kiiresti 
päris kaugele võrreldes ülejäänud maailmaga, kes pidi oma asju käimas hoidma. 
Me oleme nüüd jõudnud sinnamaani, kus meil on oma taak tekkinud ja peame 
sellega tegelema.

\question{Lõpuks ikka saab inerts otsa, aga seni on see meid päris kaugele 
vedanud.}

Seda sai üsna hästi ära kasutatud just nimelt sellepärast, et õigel hetkel sattusid õiged inimesed pumba juurde ja saagi tõmmati 
käima nii kaua, kui jõuti, enne kui ärimehed jaole jõudsid. 

Kuna oli hulk inimesi, kes tegid midagi, mis oli 
müstiline, keeruline, käsitamatu ja ilmselt ka veidi elitaarne, siis loomulikult hakkasid 
tekkima needki, kes hakkasid kaikaid kodaratesse pilduma. Inimesed, 
kes tahtsid ka löögile pääseda ja tundsid ennast halvasti, et neid ei 
võetud jutule puhtalt sellepärast, et nad ei saanud aru paadi mittekõigutamise 
mentaliteedist. See oligi mõnes mõttes ajastu lõpp, kui igaühel 
tekkis ligipääs, lävi läks palju madalamaks ja ka lühemate pükstega mehed said 
paati astuda.

Tekkisid inimesed, keda keegi ei teadnud, kes olid anonüümsed ja kellel olid 
ambitsioonid, aga puudusid võimekus ja soov panustada. 

BBSummerid hakkasid samuti kasvama ja kihistuma. Ürituste lõppu tähistas see, kui hakkasid toimuma BB-üritused BB-ürituste sees. 

\question{Mida sa praegu teed? Kuhu see tee sind on toonud?}

Praegu olen juba viimased kümme aastat tegelenud veebipoodidega. Minu eriala on 
veebiarendused, täpsemalt veebipoed ehk e-kaubandus. 

Olen nüüd rohkem programmeerimise peal, sest tänapäeval on 
peaaegu kõik ühte- või teistmoodi seotud tarkvaraarendusega. Tol ajal ei 
olnud firmadel internetilehte nagu praegu. Tol ajal ei pakutud teenuseid 
interneti kaudu, aga nüüd pakutakse. Ja seega on tekkinud vajadus tehnilise võimekusega inimeste järele. Üks võimalus on värvata nad 
endale või siis palgata firma, kes sellega tegeleb.