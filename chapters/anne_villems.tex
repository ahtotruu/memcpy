\index[ppl]{Villems, Anne}

\question{Kuidas sina arvutite juurde said?}

Mina käisin Mari Ülikoolis\index{Mari Ülikool|see{Tartu 10. Algkool}}. Mari 
Ülikooli nime rahva seas kandis tolleaegne Tartu 10. Algkool\index{Tartu 10. 
Algkool} mis on kohe Vanemuise\index{Tartu Ülikool!Vanemuise tänava õppehoone} 
kõrval ja vana Vanemuise vastas. Ja nimi tuli tal tema karismaatilisest 
matemaatikaõpetajast Marvetist\sidenote{Vladimir Marvet\index[ppl]{Marvet, 
Vladimir} (1903-1994).}, kes oli seal õppealajuhatajaks.

Ja kui me selle kooli aasta 1960 kevadel ära lõpetasime oma pinginaabriga, siis 
meie vanemad olid väga huvitatud meid panemast viiendasse kooli\index{Tartu 5. 
Keskkool}. Mis tol ajal oli siis Rostovtsevi ülikoolis, mis oli siis esimene 
ülikool, mis naisterahvaid vastu võttis. Aga seal asusid parajasti 5. keskkooli 
õpperuumid. Meie ei tahtnud 5. kooli minna, ma ei tea, miks. Meie otsustame! No 
ikkagi 14 aastat isiklikku vanust, loomulikult tuleb ise otsustada, kuhu kooli 
sa lähed. Ja meie otsustasime, et meie lähme hoopiski ikka vanasse endisesse 
Treffneri Gümnaasiumisse\index{Hugo Treffneri Gümnaasium} mis tol ajal kandis 
nimetust Anton Hansen Tammsaare nimeline keskkool. Minu meelest tol ajal ei 
olnud üldse seda probleemi, et pidi mingeid katseid tegema ja midagi. Aga ega 
meil katsetega ka oleks üsna hästi läinud, me mõlemad õppisime üsna 
korralikult. Kui me juba aasta otsa olime Treffneris ära olnud, siis avati seal 
matemaatika klass. Minu jaoks oli see kohe otse sirge tee, sellepärast et 
matemaatika oligi mu lemmikaine. Kõiges on süüdi muidugi Mari, tähendab õpetaja 
Marvet\index[ppl]{Marvet, Vladimir}. Sellepärast et nii kihvte matemaatikatunde 
nagu Marvet tegi meil viiendast seitsmenda klassini ma tean pärast ainult siis, 
kui Olaf Prinits\index[ppl]{Prinits, Olaf} tuli keskkoolis meid õpetama. Aga 
see oli juba mata klassis. Nii et kui me üheksandasse klassi läksime, sest 
tolleaegne algkool lõppes seitsmenda klassiga, läksin mina kõigepealt mata 
klassi ja kahe kuu pärast tuli mu pinginaaber ka järgi, sellepärast et siis 
tuli mingisugune praktiline õpetus keskkoolis ja  ta vaatas, et see, mida nende 
klassile pakuti, ei pakkunud nagu mingit erilist pinget. Tuli ära ka meile, 
kuigi tema oli humantitaarsete huvidega. 

Ja siis oli mata klass! Viis tundi matemaatikat nädalas, minu suureks rõõmuks, 
ja muude ainete seas ka sellised toredad ained nagu elektrotehnika, ligikaudne 
arvutamine ja programmeerimine. 

\question{Mille peale programmeerimist tol ajal õpetati?}

Selleks ajaks oli Eestis olemas parajasti üks elektronarvuti. Mis kandis 
nimetust Uural\index{Ural}. Talle ei olnud veel numbrit \enquote{1} külge 
pandud ja asus Tartu Ülikooli Arvutuskeskuses\index{Tartu 
Ülikool!Arvutuskeskus}, mis oli 59. aastal moodustatud. Masin ise asus peahoone 
kõrval majas. Ja see on Treffnerist ju üldse mitte väga kaugel, üks kvartal. 
Õpetasid meid seal alguses Ülo Kaasik\index[ppl]{Kaasik, Ülo} ja pärast Mati 
Krull\index[ppl]{Krull, Mati}. Mati Krull oli sinna tööle läinud sellepärast, 
et tema oli see esimene kursus ülikoolis, kellele Ülo Kaasik\index[ppl]{Kaasik, 
Ülo} programmeerimist õpetas. 

\question{Nii et  Ülo Kaasik kuidagimoodi ilma arvutita hakkas programmeerimist 
õpetama?}

Ei, arvutiga. Sellepärast, et meie olime üheksandas klassis 62. aastal. 
Seesamune arvuti oli juba kaks aastat kohal. 

\question{Kellelgi pidi seal keskkoolis olema parasjagu visiooni, et aru saada, 
et elektrotehnika, ligikaudne arvutamine ja arvutid on miskipärast olulised. 
Kes see inimene oli?}

Ma arvan, et see visioon võib-olla, et ei olnud isegi keskkoolil. Kuigi meil 
oli ka väga karismaatiline direktor omal ajal, Allan Liim\index[ppl]{Liim, 
Allan}, aga tema oli ajaloolane. Ma arvan, et see initsiatiiv tuli tegelikult 
Olaf Prinitsa\index[ppl]{Prinits, Olaf} ja Ülo Kaasiku\index[ppl]{Kaasik, Ülo} 
poolt. See ei oleks olnud Nõukogude Liidus erakordne. Kui nad selle arvuti 
Nõkku\index{Nõo Keskkool} panid ja Nõus oli esimene kooli arvutuskeskus, siis 
see oli kogu liidus esmakordne juhus. Aga matemaatikaklassid olid tol ajal juba 
olemas ja nad olid Moskvas olemas. Nii et nemad võisid öelda, et nemad jälgivad 
Moskva malli ja sellega keerulisi probleeme enam ei tulnud. a usun, et 
Treffneri kool sai neil valitud võib-olla sellepärast, et see oli südalinnas, 
ei olnud vaja kuskile kaugele minna. Sest oli selge, et vähemasti esimestel 
aastatel hakkavad õpetama ülikooli õppejõud. Meile õpeta Olaf Prinits 
matemaatikat. Ma ütlen, mul ongi olnud sihukesed täiesti fantastilised 
matemaatikaõpetajad algkoolis õpetaja Marvet ja keskkoolis Olaf 
Prinits\index[ppl]{Prinits, Olaf}. 

\question{Kustkohast see matemaatika huvi arvuti huviks üle läheb?,Matemaatika 
on abstraktne ja kaunis kunst aga arvutid tolmavad ja flogistonid\sidenote{Nagu 
kõik teavad, töötavad arvutid flogistonide abil. Ja me teame seda sellest, et 
kui arvuti katki läheb, siis lahkuvad flogistonid temast sinise suitsuna ning 
arvuti enam käima ei lähe.} lendavad aegajalt välja ja\ldots?} 

Tolm tuleb pärast. Tolmu tol ajal ei olnud. 

Nad muidugi omavahel saavad seotud olema oluliselt kõrgemal tasemel. See 
tähendab seda, et sa pead ikka natuke matemaatikat enne oskama, kui sa aru 
saad, kuidas matemaatika informaatikale või programmeerimisele kasulik on. Ja 
enne pead natukene programmeerida oskama, kui hakkad aru saama, et mõnikord on 
matemaatikat ka vaja. Näiteks alustades sellest, eks ole, et algoritmiliselt 
mittelahenduvaid ülesandeid eristada algoritmiliselt lahenduvatest ja need neid 
lahenduvaid jagada ka sellistesse klassidesse, mille lahendid ei tule mitte 50 
või 50 000 aasta pärast vaid millel lahendid tulevad arvutist kui mitte homme 
hommikuks, siis vähemasti ülehomme lõunaks.

\question{Seda enam tekib küsimus, et miks matemaatikahuvilel  inimesel ei ole 
tingimata arvutihuvi. Miks sul oli?}

Alguses oli muidugi lihtsalt jube põnev. Kogu Eestis ainult üks arvuti ja meile 
õpetatakse! Tuled vilguvad ja perfolindilt (mis on filmilint, mitte 
telegraafilint) loetakse andmeid,  no lihtsalt jube põnev oli! Aga seda 
muidugi, eks ole, et kuskil mõnikümmend või natuke enam aastaid edasi igalühel, 
meil on siin taskus laua peal ja meie $\epsilon$-ümbruses\sidenote{Olgu{\ } $(X 
; \rho)$ meetriline ruum, $p \in X$ ja $\epsilon > 0$. 
Punkti $p{\ } \epsilon$-ümbruseks nimetatakse hulka $\{x \in X | 
\rho(x,p)<\epsilon\}$. Ehk, lihtsalt öeldes on $\epsilon$-ümbrus meie otseses 
ümbrus.} tuhandeid kordi võimsamad arvutid kui see, mis meil seal terves suures 
saalis ikka mitukümmend ruutmeetrit enda alla võttis. Seda ma muidugi tol ajal, 
62 või 63, ette ei näinud. 64, ma lõpetasin keskkooli. 

Ja õppima ma läksin muidugi puhtalt matemaatikat. Sellepärast et, vabandage 
mind väga, ega kuskilgi, võib-olla TPI-s, ei õpetatud mingisuguseid niisuguseid 
tehnikaaineid, kus ka arvutid otsapidi sisse tulid. Võib-olla õpetati aga ma 
nagu ei tahtnud. Tehnika pool mind väga ei tõmmanud. 

Aga programmeerimine iseenesest on niisugune maagiline tegevus, eks ole. Enne 
seda arvuti midagi ei oska, siis sa kirjutad talle siukse kihvti programmi ja 
siis ta järsku oskab midagi. Näeb välja juba peaaegu, et nagu nagu saaks 
millestki aru. 

\question{Huvitav on see, et Meelis Roosi\index[ppl]{Roos, Meelis} esimene 
programm oli ka vestlemise programm. Mis on just täpselt see asi, et sul jääb 
mulje, et arvuti saab millestki aru. See on läbiv joon.}

Jah. Ma usun üheksandas või kümnendas klassis me tegime oma elu esimese 
programmi. Minu  programm ei pakkunud mulle nii palju pinget, aga elu lõpuni 
ilmselt on meelde jäänud ühe mu koolikaaslase oma (ja ma ei mäleta, kes selle 
programmi tegi, pean klassikokkutulekul küsima), kus ülesanne seisnes selles, 
et tuli anda kuupäev ja siis  välja pidi trükkima, mis nädalapäev see on. Aga 
nüüd arvestage Ural-1-ga\index{Ural!Ural-1}. Tema ainuke väljundseade oli 
kitsas printer, kus sai trükkida ainult arve. Ja kui õppejõud siis proovis neid 
meie programme, siis ta kõigepealt andiski mingi mõistliku kuupäeva ette ja sai 
vastuse ja siis andis ette 30. veebruari. Mille peale programm hakkas siis 
printeri peal siukest ülevalt alla nullide joru trükkima. Mille peale õppejõud 
ütles, et \enquote{nojah, ilmselt trükib mingit jama}, katkestama ära. Ja 
õpilane väga vaikselt ja tagasihoidlikult \enquote{las ta natukene veel 
trükib}. Ja siis ta trükkiski välja ühe nullide joru ülevalt alla, siis ühe 
nullidega täidetud rea, siis ühe tühja rea, siis natuke nulle, siis natuke 
nulle äärtes, siis natuke nulle keskel ja siis veel kord selle nullide joru 
ülevalt alla ja ühe terve nullide rea ja veel kord sama. No ja kui ma selle 
paberi kätte saime, siis oli kõigile näha \enquote{LOLL}. 

See oli asi, mis mulle meie elu esimestest programmidest kõige paremini meelde 
jäi. 

\question{Kas te selliseid nuputamisülesandeid tegitegi või mis programme te 
kirjutasite?} 

See oli üks huvitavamaid. Igasuguseid asju tegime. Mingite jalgade keskmisi 
leidsime, ma kõikide ülesandeid ei teagi. Nad olid individuaalselt antud antud 
ülesanded. Sihukesed tavaliselt, mida ka praegused programm algajad 
programmeerijad teevad.

Selle jaoks on valem olemas, muuseas. Kuidas seda kuupäevast nädalapäeva teha. 

\question{Kas sel ajal mingisugust  kogukonda ka  oli, nagu Nõukogude Liidu 
peale arvutiasjade ümber või mingeid olümpiaade juba või midagi sellist?} 

Ma ei tea, kunaprogrammeerimist koolides õpetati ikka äärmiselt vähestes. Ma ei 
teagi, näiteks, kas Moskva matemaatika klassis arvuteid ka õpetati. Ja kui ma 
Ülo Kaasiku\index[ppl]{Kaasik, Ülo} käest kunagi ühe intervjuu käigus küsisin, 
et  kust ta metoodika võttis meile programmeerimise õpetamiseks, siis ta vaatas 
mulle suurte silmadega otsa ja ütles, et \enquote{aga metoodikat ei olnud mitte 
mingisugust!}. Vähe sellest, et ei olnud metoodikat, ei olnud ka kirjandust. 
Nii et ta ise hakkas  raamatuid kirjutama ja see metoodika oli tal, nii-öelda 
käigu pealt välja töötatud.

\question{Ühesõnaga, tegelikult kogu see eesti asi käib Ülo Kaasiku peal?} 

Jah, see asi käis ikka kindlasti. Tema ja tema õpilased. Sellepärast, et pärast 
seda hakkas neid matemaatikaklasse tekkima ikka hulgi. Ma mäletan, et Ove 
Karu\index[ppl]{Karu, Ove}, kes oli omaaegne Nõo direktor\index{Nõo Keskkool}, 
tema istus meie matemaatika lõpueksamitel. Et nii-öelda vaadata, kuidas me 
oleme matemaatikas arenenud. Ja ma mäletan isegi omaenese vastust. Kuna mul oli 
keeruline joonis teha, siis selle seletamisel ma läksin puntrasse. Astusin kaks 
sammu tahvlist eemale ja alustasin otsast pihta. No see on see tõestus, et 
sirge on risti tasapinnaga, kui ta on risti kahe sellega lõikuva sirgega. See 
oli (praegu ei või milleski kindel olla) keskkooli programm. Meil on 
Arvutiteaduse Instituudis\index{Tartu Ülikool!Arvutiteaduse Instituut} 
projektikoor, koorilaul tuleb meelde poolteist aastat enne laulupidu ja siis me 
võtame oma koori kokku ja valmistume nii hästi, kui suudame  ettelaulmiseks ja 
käima laulupidudel. Ja kui laulupeol küsiti meie käest, et kuidas meie koori 
iseloomustada, siis igaüks pakkus midagi välja. Ja mina pakkusin, et meie koori 
iga liige teab Pythagorase teoreemile ise erinevat tõestust. Neid on umbes 200, 
meie koori liikmeid on kuskilgi suurusjärk 40. Mille peale matemaatika 
õpetamisega tegelevad õppejõud meie koorist tuletasid mulle meelde, et 
Pythagorase teoreemi enam ei tõestata koolis. Nii et ma nüüd ei tea enam mitte 
midagi.

\question{Ühel hetkel sai keskkool otsa. Ja siis tuli ülikool, Tartu Ülikool ja 
matemaatika, eks ole?} 

Jah, Tartu Ülikool\index{Tartu Ülikool}. Vaadake aasta oli siis 1964. See 
tähendab väga sügav nõukogude aeg. Mis välistas minu jaoks arusaadavatel 
põhjustel absoluutselt kõik humanitaar-alad. 

Järgi jäi suhteliselt vähe. Järgi jäi meditsiin. Mu isa oli kirurg ja ju ta 
siis vaikselt ikka lootis, et äkki. Aga mul on selline loogiliselt mõtlev mälu, 
mul on väga hea igasuguseid tõestusi meelde jätta ja nii edasi. Aga kui ma pean 
pähe õppima 2000 kontide nimetust, ja sinna juurde veel lihaste ja veresoonte 
ja aju sagarate ja jumal teab veel millede ladinakeelsed nimed, siis ma ei 
arva, et ma ennast väga hästi tunneksin. No ja peale selle me juba 5. keskkooli 
ei tahtnud selle pärast minna, et vanematel olid seal liiga head suhted. 
Tahtsime ikka ise olla. Ka  ülikoolis tahtsin olla rohkem ise, kui keegi muu. 
Ja siis jäigi järgi matemaatika. Ja matemaatikat ma armastasin koledasti. 

Pange tähele, mul olid viis, kuus, seitse klass head õppejõud. Ma ei saa midagi 
paha öelda ka oma kaheksanda klassi õpetaja kohta, kes mind soovitas, mata 
klassi. Ja siis nii-öelda eesti matemaatika koolis õpetamise korüfee Olaf 
Prinits\index[ppl]{Prinits, Olaf}, kelle tunde ma mäletan siiamaani. Ma mäletan 
funktsionaalse seose selgitust siiamaani, ligi 50 aastat pärast seda, kui ma 
seda õppisin. 

\question{Kas Tartu Ülikoolis õpetati toona matemaatikutele ikka ka 
programmeerimist?}

Jah, meil oli kaks kursust programmeerimist. Üks oli masinkoodis 
programmeerimine, Ural-4\index{Ural!Ural-4} peal. Selle ma läksin ja tegin 
nii-öelda kohe septembrikuu jooksul ette ära. Mille peale õppejõud võttis mu 
vastusse ja ütles, et \enquote{lõhnab natuke Ural-1 järgi aga programmeerida te 
oskate} ja pani mulle arvestuse. Teine oli siis Ülo Kaasik\index[ppl]{Kaasik, 
Ülo} Algol 60\index{Algol!Algol 60} õpetus ja vot seda ta õpetas küll pliiatsi ja 
paberiga. Sellepärast ühtki translaatorit tol ajal Algoli jaoks ei olnud. 

\question{Miks just Algol?}

Vaadake, aasta siis oli 66, 67, mis keeled siis  üldse olemas olid? 

Jumalale tänu, et keegi ei hakanud meile Cobolit\index{Cobol} õpetama. See 
oleks mind küll programmeerimisest viie kilomeetri kaugusele peletanud! Olete 
te proovinud kunagi Cobolit lugeda? Ei? Ärge vaadake ka! Kui programmeerimine 
on selline kontsentreeritud väljendus, sa saad valemeid kirjutada valemitena ja 
siis sul on mõningad sellised kenad koodsõnad \verb|for| ja \verb|do| ja 
\verb|if-then-else| ja nii edasi. Siis pange nüüd sinna mingisugune filoloogide 
soust peale, kust \enquote{palk pluss preemia}  tuleb välja kirjutada kolme 
sõnana! Vot niisugune programmeerimiskeel. Noh\ldots Ja ma tänan jumalat, et ma 
mitte kunagi ei ole pidanud programmeerima Cobolis. Cobol oli vahepeal väga 
elujõuline. 

\question{Mis on huvitav selle jutu juures on see, et siin on ikkagi 
eksplitsiitne  programmeerimise õpetus. Keegi metoodiliselt õpetas programmi 
kirjutama ja see on üsna haruldane. Reeglina inimesed ütlevad, et 
programmeerimine kuidagi jäi külge. Ja nad ei oska öelda, kust või kes õpetas.}

Kui tahtmine on väga suur, siis kindlasti on võimalik programmeerimist 
iseseisvalt õppida. Aga  motivatsioon peab ikka seinakõrgune olema. 

\question{Mis toona arvutite perspektiiv oli, milleks neid kasutati? 
Programmeerime, programmeerime, aga kas see on teaduslik töövahend või aitab 
rahvamajandusele kaasa või veel midagi?} 

Rahvamajanduses kasutus oli juba Ural-1\index{Ural!Ural-1} ajal, kuigi mälu oli 
väga vähe. Ural-1 mälu, mida keegi täna ei usu, oli kaks kilo. Õnneks mitte 
baiti, sest baiti tol ajal ei tuntud, vaid sõna. Kaks kilo sõna. Võtke oma 
telefon ja vaadake siis nende mega või gigabaitide arvu, mis teil taskus on. 
Siis väga suur probleem oli see, kuidas neid andmeid, mille pealt midagi pidi 
arvutatama, kuidas neid  arvutisse ära mahutada. Välisseadmetega olid ka veel 
omad probleemid. Aga need pakiti niimoodi kokku. 

Ma mäletan, et kuidas ma ise masinkoodis programmeerimist õpetasin. Ja õpetasin 
selle pärast et ma miskipärast läksin viimase kursuse viimasel semestril 70. 
aastal mehele. Ja siis järsku langes ära minu Tallinnasse tööle minek, sest mu 
mees teatas surmkindlalt, et tema küll Tallinnasse ei kavatse minna. Siiamaani 
ei ole läinud. Ja siis  ma pidin ruttu Tartus endale töökoha otsima. Ja siis 
siis Ülo Kaasik\index[ppl]{Kaasik, Ülo},  oli siis see kes ütles, et 
\enquote{jah, ma usun küll, et ta võib tudengite ette saata, tal jalad ei 
värise!}. 

No ja siis mind saadetigi pärast lõpetamist. Lõpetasin juunis ära ja  
septembris läksin tudengitele programmeerimist õpetama. 

\question{Kas mingisugust teadustööd ei sirgunud sealt, oli puhas õppejõu töö?}

Teadustegevusega ma hakkasin tegelema hoopis palju hiljem. Sellepärast et kui 
meil praegu räägitakse,  misasi on õpetaja norm töökoormus koolis. Te teate 
peast? 24-28 tundi. Algajate õppejõudude, nii-öelda jalul seismise, 
auditooriumi ees seismise koormus (vähemasti siis, kui mina õpetasin), oli 24 
või 28 tundi nädalas. Sealt kõrvalt te väga midagi ei tee, eriti kui ta esimesi 
aastaid õpetate. Ja peale selle õpetajate kahes keeles. Ega meilsiis 
matemaatika poole peal vene keelt rääkivaid inimesi oli väga vähe. 

\question{Olidki kaks eraldi grupi? Eesti ja vene?} 

Jah. Ja majandusteaduskonnas ma õpetasin elu esimese loengu. Lugesin vene 
keeles majandusteaduskonna, ma ei tea, kas kaugõppijatele või. Vene keelt ma 
oskasin väga nirult. Esiteks ei olnud seda vaja. Ja teiseks olin küll lugenud 
väga palju matemaatikaalaseid õpikuid, aga ma võin neid tõestusi lugeda ka 
prantsuse keeles, mida ma ei oska, sellepärast, et seal nagu neid vahesõnu on 
suhteliselt vähe. Aga selleks, et õpetada, selleks on vaja palju sõnu. Mul 
õnneks jätkus taipu. Vaadake, meie venekeelsetes rühmades oli alati sees 
kakskeelseid inimesi. Ja mul jätkus taipu teha nendega tudengitega, keda ma 
õpetasin, nendega niisugused kokkulepped. Esiteks, kui ma ütlen midagi sellist, 
millest ei ole võimalik ka hea tahtmise juures aru saada, siis esimene rida 
ütleb mulle. No ja siis ma püüan ümber sõnastada niimoodi, et arvuga oleks 
võimalik saada. Ja kui mul sõna meelde ei tule, siis ma ütlen selle sõna eesti 
keeles ja need kakskeelsed ütlevad mulle vastava venekeelse termini. 

Minu vene keele mitteoskust iseloomustab võib-olla see, et ma näiteks ei 
teadnud, kuidas on vene keeles \enquote{lahutama}. Ma küll üritasin kasutada 
\begin{russian}отнять\end{russian}, mis ei ole ka väga vale, aga sõna 
\begin{russian}вычесть\end{russian} ma ei teadnud.

\question{Ja see on ju pedagoogilise metoodika mõttes õudselt hea kool!} 

Jah! Igal juhul kuskilgi umbes seitse või kaheksa aastat pärast seda ma 
sattusin Moskvas (ei olnud vist Moskvas, oli kuskilgi ääretul Nõukogude 
Kodumaal) konverentsile, kus ma siis midagi seltskonnas ütlesin ja mille peale 
minu käest küsiti, et \enquote{\begin{russian}А вы из Москвы?\end{russian}}. 
Mille peale ma siis ütlesin, et kui ma teist korda veel suu lahti teen, siis te 
saate kohe aru, et ma ei ole ei Moskvast ega ka Leningradist. 

\question{Ja selline õppejõu töö Tartu Ülikoolis läkski edasi, kuni saabusid 
aastat 80?} 

Vist veel edasi, siiamaani. 

\question{Nojah, aga aastast 80 algab see aeg, kui asjad lähevad lihtsamaks ja 
arvutid väiksemaks ja neid tuleb juurde?}

Arvutid läksid väiksemaks, jah. See arvutite saamine seal vahepeal, 
kaheksakümnendatel, oli ikka omaette tsirkus. 

\question{Jaan Tallinn\index[ppl]{Tallinn, Jaan} on rääkinud\sidenote{On 
rääkinud tõesti, aga mitte nende kaante vahel olevas jutus.}, et tema tõi oma 
esimese arvuti käsipagasis laevaga Rootsist?} 

Minul ei õnnestunud väga Rootsi saada. Käisin küll, aga siis ma nii rikas seal 
Rootsis küll ei olnud. Mul abikaasa oli post-docis Uppsalas ja mina sealt 
arvutit ei toonud. Kuigi ma tean küll, et kui oleks seal ostnud arvuti ja siin 
maha müünud, oleks selle eest väga palju muid asju saanud. Aga, ei, mul 
sihukest ärivaimu ei olnud. 

\question{Aga millal hakkas Tartu Ülikoolis\index{Tartu 
Ülikool!Matemaatikateaduskond} arvuti tavaliseks asjaks muutuma, kus oli 
selline igapäevane osa näiteks Matemaatikateaduskonna elust?} 

Võibolla üks selline murdepunkt oligi 1982. Juhtus selline asi, et Tallinna 
sõbrad organiseerisid näituse, välisnäituse, kuhu tulid vist 100 firmat sinna 
Tallinna näituseväljakule. Aga miks see minu nii-öelda tähelepanu köitis (seal 
oli neid näitusi enne ka olnud), oli see, et pidi tulema kohale kellegi sõber 
Suurbritanniast, kes oli hakanud Apple'i diileriks. Apple'i diileriks oli ta 
hakanud selle pärast, et ta pidi ehitama ühe seadme ülikõrgete rõhkude jaoks, 
aga seda seadet oli vaja juhtida. Juhtimiseks valis ta välja Apple'i ja kõige 
odavam Apple'i saamise viis oli see, ta hakkas Apple'i diileriks. Ja nüüd ta 
siis tuli siia näitusele. Ma ei tea, mida ta muud siian tõi, aga igal juhul 
Teaduste Akadeemia\index{Teaduste Akadeemia} instituudid ja pooled mulle head 
tuttavad inimesed hakkasid ette valmistama, et tema käest Apple'i arvuteid 
osta. 

No aga kuulge! Milleks näiteks kadunud professor Lippmaale\index[ppl]{Lippmaa, 
Endel} arvuti? Minul on arvutit vaja! Ma arvan, et mul on arvutit rohkem vaja, 
kui tal! Tal on mingisugused oma magnetresonantsid, mida ta ka peab juhtima. 
Aga meie õpetama tulevasi programmeerijaid välja ja meie kasutame 
\emph{input-output} kappi, kui keegi teab, mis asi see on. Kuhu tudeng paneb  
perforeerimiseks oma blanketile kirjutatud programmi ja siis kolme nelja päeva 
pärast saab sealt tagasi süntaksvigadega väljatrüki. Ja mul läks hammas 
koledasti verele: sõbrad saavad miskipärast arvuteid, aga tegelikult on neid 
hoopis mulle vaja. Ja siis ma läksin ja rääkisin oma sõpradega. No ja tuli 
välja, et sõbrad on head sõbrad. Näiteks selleks, et üldse aru saada, milles 
Apple koosneb (sest mul ei olnud teda vaja millegi juhtimiseks, mul vaja 
siukest alasti arvutit) ma istusin Tõraveres\index{Tõravere Observatoorium} 
sest seal käis Byte\sidenote{Byte oli 1975. kuni 1998. aastani välja antud USA 
arvutiajakiri, mis oli kaheksakümnendatel vägagi mõjukas.} ja sealt sai teada, 
mis on arvutil sees ja mis talle külge käib. Istusin Tõraveres pühapäevade 
kaupa ja panin oma konfiguratsiooni kokku ja sain aru, et mul ongi vaja 
\enquote{alasti} arvutit. Tallinna sõpradega Lippmaa 
instituudist\index{KBFI}\sidenote{Nii tunti KBFi-d} pidasin aru, mis on 
mõistlik summa, mida plaanikomiteest küsida. Leidsin, et 10000 kuldrubla (rubla 
ei toiminud välisturul, ainult kuldrubla) oleks piir, mida võiks küsida. See on 
nii pisike summa, et keegi võib-olla et äkki annabki selle. 

Saime selle eest kolm arvutit, sest mul oli vaja paljaid arvuteid, isegi 
monitore ei ostnud. Ühe igaks juhuks, et mine sa isahane tea. Võib-olla meie 
nõukogude televiisoritega äkki ei töötagi. 

\question{Mis arvutid need olid?}

Apple II-d\index{Apple II}. Plaanikomitees käisid minu eest Lippmaa 
Instituudi sõbrad Tõnu Karu\index[ppl]{Karu, Tõnu} ja Riivo 
Sinijärv\index[ppl]{Sinijärv, Riivo}. Mina ajasin kõik paberid korda, korjasin 
ülikoolist kõik allkirjad kuni rektorini peale ja nemad käisid kohal. Antigi 10 
000 me saimegi oma kolm arvutit.

Ja tegime nendest arvutiklassi. Kolmest arvutist! Ja panime käima 
programmeerimise õpetamise individuaalgraafikus praktikumidega. See oli väga 
tore aeg, sellepärast et inimesed olid harjunud \emph{input-output} kapiga, eks 
ole, et annad sisse, siis unustad kõik ära, mis sa sinna kirjutasid, siis saad 
kolme või nelja päeva pärast siis mingi paberirulli tagasi oma süntaksivigade. 
Siis sa parandad neid süntaksi vigu ja nii edasi. Selle pisikese elu esimese 
või teise või kolmanda programmi silumine võttis niimoodi ikka õige mitu 
nädalat aega. 

\question{Aga see, see tähendab siis seda, et tegelikult see nii-öelda 
tarkvaratehniline pool muutus radikaalselt!} 

Jah. Arvutid saime kätte kuskilgi jaanuaris, veebruari algusest panime  
programmeerimise algõpetuse käima. Kuigi meil Basicu\index{Basic} keel ei 
meeldinud ja Apple II\index{Apple II} on sündinud Basicuga, mõtlesime, et 
suudame need Basicu hädad vast ehk kompenseerida oma hea õpetamisega. Esimese 
programmi jaoks ta kõlbab. Ja kevadel paistis Liivi tänaval\index{Tartu 
Ülikool!Liivi õppehoone} paistis päike sinna klassi. Seisin nende kolme arvuti 
selja taga, et tudengeid aidata ja tegelikult ma ei näinud, mis nad sinna 
ekraani peale kirjutasid, vaid nende enda peegeldust. Ja see, kuidas tudeng 
sisestas oma programmi, pani käima, sai sealt ise oma süntaks vead kohe kätte. 
Noh, see miimika, eriti tütarlaste oma, see ikka ikka tasus vaatamist, et aru 
saada, et me oleme kuskilegi suunas õige sammu astunud! 

\question{Õpetamise metoodika  pidi ka ju muutuma?} 

Jah, koos keelega muutub see alati. Põhikisma, mida algõpetuses arutatakse, on 
see, et missugust programmeerimiskeelt õpetada esimesena. Sest sealt jäävad 
asjad külge. Ja Basicu\index{Basic} häda on muidugi see, et kui ma olin Basicus 
valmis kirjutanud (küll juba Yamaha arvutil, mida me kasutasime  laialdasemalt 
arvutite tutvustamiseks) valmis programmi, mille väljatrükk oli umbes minu enda 
pikkune siis ma vandusin, et see on minu viimane programm Basicus. Baicus 
nimelt ei ole ei funktsioone ega alamprogramme. Ja siiamaani ma olen seda 
vannet pidanud, ma ei ole Basicus rohkem programme kirjutanud. 

Kuna mulle õpetati kõigepealt masinkoodi, sest vabandage väga Urali arvuti 
peal, mingit  kõrge taseme keelt ei olnud ja assemblerit ka ei olnud. Ja nii 
mulle tundus, et \enquote{siis ma saan ju aru, mängin otse registritega ja ma 
saangi aru, mis masinas toimub ja mida see aritmeetiline plokk seal teeb ja nii 
edasi ja sealt edasi on juba kõik väga lihtne}. Üks muu Novosibirski nüüdseks 
juba kadunud sõber kunagi kui ma võitlesin selle eest ühes üleliidulises 
seminaris, et ikkagi masinkoodist tuleb alustada. Siis ta minu seisukoha 
ilmestamiseks rääkis anekdoodi. Anekdoot oli sihukene, et vot vaene mees läheb 
kirikuõpetaja juurde ja ütleb, et kole raske on elu. Naine ja lapsed ja ämm ja 
kõik peame elama oma onni ühes toas ja see on ikka väga raske. Kirikuõpetaja 
küsib, et aga kas sul kits on? Ja kits mul on! Ole hea, võta kits ka tuppa 
juurde. Milleks veel kits? Aga kirikuõpetaja käskis ja talumees võtab siis ka 
kitse tuppa. Ja kirikuõpetaja ütleb, et vot nüüd nädala pärast vii kits välja 
ja tule räägi minuga. Nädala pärast viib mees kitse välja ja tuleb, räägib, et 
vot nüüd on küll juba väga hästi, kitse ei ole ja oma naise ja ämma ja laste ja 
kõikidega ma saan nüüd juba palju paremini hakkama. Masinkoodist alustamine on 
nagu kitse toomine sinna tuppa, et kui lõpuks saab hakata programmi kirjutama 
\verb|for| ja \verb|if| ja \verb|else|-ga, on suur lõõgastus. Et enam ei pea 
tõstma midagi registrisse ja kontrollima ja andma suunamist ja nii edasi. 


\question{Siin meie jutu käigus, tundub, on toimunud mingisugune nihe. Kui me 
alustasime sellest, et meeldis programmeerimine, siis ühel hetkel andis 
positiivse emotsiooni näoilme muutus ekraani peegelduse peal. Mis hetkest 
programmeerimise õpetamine muutus nagu huvitavamaks kui programmeerimine? Kui 
üldse niisugune hetk on olnud}

Jah ma usun küll. 

Selleks, et midagi väga tähelepanuväärset programmeerimises ära teha on vaja 
head meeskonda. Ja ma arvan, et parim nii-öelda suur asi, mida ma 
programmeerimises olen teinud on, et me Ain Isotammega\index[ppl]{Isotamm, Ain} 
sisuliselt kahekesi tegime omal ajal suure süsteemi Villis, mis oli aruannete 
generaator ja millel olid ikka väga tähelepanuväärsed omadused. Ja minu elu 
kõige keerulisem programm, mis tegeles magnetlindi ja printeri juhtimisega 
aruannete välja trükkimise ajal, kui aruanded on pandud segamini magnetlindi 
peale (küll tekitamise järjekorras aga 15 aruannet korraga, jupid vaheldumisi) 
ja et  printerist tuleksid kõik aruanded õiges järjekorras. Mul oli kaks 
katkestuste allikat ja siis tasakaalu hoidmine printeri puhvrite ja magnetlindi 
puhvrite vahel. Lisaks teisendamise töö seal vahel, et andmetest teksti 
tekitada, see oli köömes. 

See on vast, jah, mu kõige keerulisem programm, mille tegemise käigus õnnestus, 
muu seas, (see on iga programmeerija unistus), avastada arvuti viga. Teile kogu 
aeg tundub, et arvuti teeb valesti, sellepärast, et teie teete kõik õigesti, 
aga arvuti ju eksib. Ja siis te lähete seda inseneridele rääkima. Ja kuna 
insener ei tea teie kontekst, siis te hakkakte inseneridele seletama seda asja 
algusest peale. Vot niimoodi, niimoodi, niimoodi, niimoodi. Kuskil poole peal 
te saate aru, millal te ise olete vea teinud, haarate kõik oma väljatrükid ja 
ütlete insenerile, kes sinnamaani ei ole veel mitte millestki aru saanud, aitäh 
posti mängimise eest ja lähete oma viga parandama. Aga selles programmis mul 
õnnestus leida arvuti viga. Sellepärast et sünkro impulss traat printerile oli 
halvasti joodetud. Kõik inimesed trükkisid sümbol-haaval. Või siis äärmisel 
juhul terve rea kaupa. Mina tahtsin nii, et tegin lehekülje valmis ja tahtsin 
tervet lehekülge kooraga trükkida. Trükib mulle pool lehekülge, veerand 
lehekülge ja edasi lihtsalt ei trüki. Lähen kontrollin printeri juht käsku, 
kõik on õige, kõik korras, aga ei trüki. Noh, ja siis insenerid avastasid, 
milles asi. See on ainus kord minu programmeerija karjääris, kui mul õnnestunud 
arvuti viga avastada. 

\question{Kusjuures mitte arvuti kui tüübi viga vaid et see konkreetne tükk on 
spetsiifilisel viisil katki!}

Jah, lihtsalt konkreetne printer. Aga see on ikkagi riistvara! Kkka inseneride 
pärusmaa, mitte  programmeerija oma.

\question{Ikkagi, kuidas õpetamine huvitamaks läks, kui programmeerimine?}

Õpetamise värk on huvitav olnud kogu aeg, sest ega ma siis ei oleks kaua selle 
koha peal\ldots Tallinnas oli palju arvutuskeskusi, Tartus väga nagu 
arvutuskeskusi ei olnud, neid oli Tartus väga vähestel asutustel. Aga ega mind 
väga ei tõmmanud ka see, see tolleaegne  automatiseeritud juhtimissüsteemide 
tegemine. See nõukogude tehnika töökindlus oli ikka niivõrd vilets. Isegi 
nõukogude esimesed ameeriklastelt üle kavaldatud personaalarvutid ei kippunud 
hästi töötama. 

Aga õpetamine oli ju palju toredam! Kuni siiamaani räägitakse meile, kuidas 
Eestis on ikka veel 8000 IT-spetsialisti puudu. 

\question{Need maagilised Schrödingeri 8000 spetsialisti\ldots}

Jah, ma usun küll, see on enam-vähem konstantne suurus, seisnud juba niimoodi 
väga palju aastaid. Põhiline ei ole võib-olla see number ise, 8000. Vaid see, 
et neid on puudu ja et neid on puudu,  näeme kogu aeg oma teise kursuse pealt. 
Nimelt inimesed omandavad esimese kursuse programmeerimise algõpetuse, teevad 
oma esimese projekti objektorienteeritud programmeerimises ja siis on ta 
kasulik firmale ja ülikool lükatakse teiseks plaaniks. Ja kes siis veab välja 
ja kes ei vea. Enamus ei vea ja ülikool jääb kõrvale. 

\question{Kustkohast tuli idee õpetada välja arvutiõpetajaid? Sest  tahab  ikka 
 ägedat visiooni saada, et keskkoolis või üldse koolis peaks  arvutiõpetust 
õpetama?} 

Sellel on kaks juurikat. Üks juurikas on muidugi see, et arvutid hakkasid 
jõudma ka koolidesse. Mitte ainult mata klassidesse vaid ka mujal. Sest 
tekkisid lihtsalt sellised arvutid, mis said jõuda. Ural-1-te ei jõudnud ükski 
kool osta, see on täiesti selge, füüsiliselt oleks ta võimlemissaali  mahtunud, 
aga seda ei saanud ka koolist arvuti alla ära võtta. Nii et need õpetajad, kes 
läksid koolis arvutit õpetama, neid oli vaja välja õpetada. Teine juurikas oli 
muidugi arvutiside. Sellepärast, et arvutiside hiilis koolides tagauksest. See 
ei olnud mingisugune niisugune riiklik programm kui meil pärast tuli, eks ole, 
Tiigrihüpe ja igasugused asjad. Esimesed, ma usun, ligi 100 kooli said arvutis 
sidega ühendatud (ise küsimus, mismoodi ühendatud) nii, et paljud ülemused ei 
teadnud üldse, et on. 

\question{Soh. Mille külge nad läksid siis?}

Osa nendest läks modemiga, telefoniside kaudu. 

\question{Tartusse kuskile?} 

Alustame algusest. Aasta siis oli. Oh, anna mul mälu! 80 alguses. Kõigepealt 
muidugi olid Fido-vennad. Ja ma arvan, et nende tegevusest ei teadnud ka mina 
suurt midagi, sest amatööride ridadesse ma ei kuulunud. Ja seda ma kuulsin 
pärast, et Fido vendadel oli modemside juba enne, kui arvutid jõudsid UUCP-ga 
võrku. 

Aga kui siis arvutivõrgust juba midagi kuulda oli (ja kaheksakümnendatel juba 
oli midagi kuulda) siis ka meie Soome sõbrad otsustasid Nõukogude Eesti järgi 
aidata. Muidugi Tallinn on Helsingile palju lähemal kui Tartu. Nii et Tallinna 
Tehnikaülikool\index{Tallinna Tehnikaülikool} sai Soomlaste käest modemi. Ja 
kuigi meil oli uuendatud telefoniside seoses 80. aasta olümpiaga, ei pidanud 
selle liinid modemi kiirusele vastu ja soomlased pidid kinkima neile natukene 
aeglasema modemi, mis võttis ka madalamaid kiiruseid ja nii saadi Tallinnast 
UUCP protokolliga side. Meie Tartus olime ka muidugi uljad ja muretsesime 
endale ka modemi ning üritasime Tartust Helsingisse helistada. Aga Tartus ei 
olnud isegi mitte 80. aasta olümpiamängude purjeregatti ja meie telefoniliinid 
ei pidanud ka aeglasi modemeid  vastu. Siis me võtsime kätte ja üritasime 
Tallinnasse helistada. Ja kuna me ei jõudnud neid kiireid moslemeid osta, siis 
me ostsime mingi sisemise odava modemi. Ja aitäh Mati Kilbile\index[ppl]{Kilp, 
Mati}, kes tol ajal oli Matemaatikateaduskonna\index{Tartu 
Ülikool!Matemaatikateaduskond} dekaan ja need mingid pisikesed valuutasummad 
leidis. Sest kust, sa hing, kaheksakümnendatel ikka midagi ostad, valuutapoest! 
Ja siis meil käiski niimoodi, et meie helistasime Tallinnasse, Tallinn UUCP-ga 
helistas Helsingisse. Alguses kaks korda nädalas, siis kolm korda nädalas, siis 
iga päev, siis ma ei tea, mitu korda päevas, kuna mahud läksid järjest 
suuremaks. Helsingist alates oli juba päris päris korralik internet olemas. 

\question{Aga mille külge need keskkoolid tulid?} 

Vahepeal toimus see, et meil pandi tänu Rootsi Kuninglikule Akadeemiale ja 
Sorose Fondile\index{Sorose Fond} üles satelliitside. Nad said nii palju raha 
kokku, et hankida otsa-seadmed. Kahes eksemplaris, sest Tartu-Tallinna vahet ei 
jõua ju keegi ära kakelda. Nii et Tartus Tähetorni\index{Tähetorn} otsas oli 
Tele-X satelliidi  vastuvõtujaam ja Tallinnas (kuna Lippmaa 
Instituut\index{Lippmaa Instituut|see{KBFI}} oli ikka endiselt veel Teaduste 
Akadeemia raamatukogu all sisuliselt, raamatukogu oli teisel korrusel, nemad 
esimesel) siis sinna katusele sai teine satelliitseade. 

Orienteeriti alguses vist valele satelliidile, meie seal kaasa ei mänginud, see 
oli puhtalt rootsipoolse otsa tegevus. Siis tekkis küsimus, et kes selle 
järgmise inseneri kinni maksab, kes tuleb ja õige peale seab. Aga siis tuli 
Rootsi kuninga visiit Tartusse\sidenote{Nende Majesteetide Rootsi Kuninga Carl 
XVI Gustafi ja Kuninganna Silvia riiklik visiit Eesti Vabariiki toimus 22.-24. 
aprillil 1992.}. Ja loomulikult teenusepakkuja (ma isegi ei mäleta, mis firma 
see oli) tahtis näidata, et nemad on kõikjal. Et Rootsi kuningas saab võtta 
telefonitoru Tartu Ülikooli rektori kabinetis ja ühenduda kohe oma koduga 
Rootsis. Ja siis nad saatsid selle mehe, kes side õige satelliidi peale pani. 
Nii et Rootsi kuninga visiidist alates oli meil  64 kilo side üle satelliidi. 

Sel hetkel oli muidugi Tallinn-Tartu side väga huvitav. Tallinnast Tartusse, 
nagu me teame, on 285 kilomeetrit, teistpidi on alati rohkem olnud. Aga kui 
nüüd vaadata, missuguse tee pidi läbima elektronkiri, et jõuda Tartust 
Tallinnasse, siis ta kõigepealt pidin minema Tartusse Tähetorni, siis Tähetorni 
satelliidi pealt Kuninglikku Rootsi Tehnikaülikooli Stockholmis, sealt, olles 
avastanud siis ee-domeeni (kas meil tol ajal oli .ee? see tuli natuke hiljem? 
Esimesed aadressid tulid, muuseas, .su lõpuga) läks kiri üle satelliidi 
Tallinna Tehnikaülikooli ja siis natuke veel Tallinnas. Nii et see kõik oli 
vist, ma kunagi arvutasin kokku, üle 70 000 kilomeetri\label{sisu!70k}. Kaugemal ei ole Tartu 
Tallinnast kunagi asetanud! Aga see tee läbiti õnneks valguse kiirusel. 

\question{Kust see visioon tuli, et sihukest asja üldse vaja on, milleks 
internet hea oli?}

No tolleks ajaks me olime selle UUCP sidega juba  imeasju teinud. Ma näiteks 
õpetasin tollel hetkel internetti tudengitele. Vaadake e-kirja teel oli 
võimalik igasuguseid asju saada! RFC-sid, mis on, eksole interneti 
alusdokumendid, neid oli võimalik tellida e-kirjaga. Ja tudengite 
arvestusülesanne oli mõni RFC kohale meelitada ja seega oli meil ketta peal 
peaaegu et kogu interneti dokumentatsioon ja üksikuid neist isegi üritasime 
välja trükkida. Näiteks RFC 822, mis oli elektronkirjade aluseks. Väga huvitav 
oli neid uurida. Peale selle me olime juba terve suure hulga ülemaailmsete 
listide liikmed. Informatsioon levis. Ja üleüldse oli internet tore asi. 

\question{Saite maigu suhu?}

Jah, sealt edasi said aru kõik aru, et meil on päris internetti ka vaja. Kuigi, 
pange tähele, veebi ei olnud veel sündinud. Meil oli Gopher. 

Veeb on CERN-is välja mõeldud, kuna neid hakkas jubedasti ära tüütama.  CERN-is 
toodetakse palju artikleid ja need artiklid viitavad üksteisele. Ja kuidas sa 
artiklist aru saad, eks ole, sa pead neid viidatavaid artikleid ka lugema ja 
see on üks igavene tüütus käia neid kuskiltki otsimas. Ja siis mõeldi välja 
World Wide Web, kus on artikkel ja kus on kohe ka  pildid seal sees (see nõudis 
juba graafilist brauserit) ja kus on viited niimoodi, et sa saad klõksida nende 
peale ja siis tuleb järgmine artikkel kohale. 

\question{Kaks küsimust. Kuidas see va veeb jõudis Tartusse ja kuidas tuli 
mõte, et võiks hakata inimestele õpetama, kuidas seda teha?} 

Need mõtted tulid peaaegu et korraga. Aga kuidas veeb jõudis Eestisse, minge 
küsige Marek Tiitsu\index[ppl]{Tiits, Marek} käest. Tema töötas tol ajal Tartu 
Ülikooli raamatukogus\index{Tartu Ülikool!Raamatukogu}, kus oli jälle kuskiltki 
päranduseks kingitusena saadud mingisugune arvuti. Ma ei mäleta, mis arvuti 
oli, aga sellele oli  võimalik veebiserver peale panna. Sellepärast et, 
vabandage mind väga, veebiserverid töötavad kõiki Unixis. 

Ja Eesti esimese veebiserveri pani püsti Marek Tiits\index[ppl]{Tiits, Marek} 
ja siis selles kursuses, mis vist kandis juba tol ajal pealkirja informaatika 
didaktika ja kus ma igasuguseid uusi asju, kaasa arvatud internetti, üritasin 
lugeda (Sest ega mul internetikursust ei olnud. Mul oli informaatika didaktika 
kursus, kus me üritasime ka RFC-sid ja kätte saada ja nii edasi ja nii edasi) 
ma kutsusin Marek Tiitsu, et ta siis räägiks meile natukene nii Gopher-ist kui 
ka veebist. 

Marek Tiits\index[ppl]{Tiits, Marek}, kes praegu kindlasti on üks parimaid 
lektoreid üldse oli tol ajal teise kursuse tudeng. Ja kui ma pärast küsisin oma 
tudengitelt, kellele ta  veebist rääkis et \enquote{saite kõik aru, mis ta 
rääkis?} siis tudengid vastasid \enquote{vot ei tea, kui enne ei oleks midagi 
teadnud, siis vist ei oleks aru saanud. Aga kuna me enne ka midagi teadsime, 
siis me saime teda üsna hästi jälgida.}. Nii et esimese veebiserveri au on 
Marek Tiitsusl. Aga kuskilgi, kunagi oli mingi Sun SPARCStation\index{Sun 
SPARCStation}  ka Eesti Biokeskuses\index{Eesti Biokeskus}. Ja seal peal ma 
võtsin mingil hetkel oma tudengid (üldse mulle meeldib tudengitega igasuguseid 
lollusi teha) ja  saatsin ülikooli peale (igaüks sai ise teaduskonna) et 
hankigu igasugust informatsiooni, mida õnnestub kätte saada. Teaduskonna 
allüksusi, loetavaid aineid, mida iganes. Ja siis me nõelusime nendest sellise 
toreda ülikooli veebi kokku. Ja siis meil jätkus nahaalsust kutsuda kohale 
rektor ja kaks prorektorit ja paluda neil istuda arvuti taha ja vaadata, kuidas 
Tartu Ülikool\index{Tartu Ülikool} veebis välja näeb. 

\question{Ja nägi välja küll!}

Nägi välja niisugune, ma usun, nagu Eesti metsad praegu välja näevad. Et 
noorendik ja lageraie ja siis mingisugune vana tükk ja nii edasi. Ta oli väga 
lapiline, kuna kes kust mida kätte sai, kellel olid tuttavad kus teaduskonnas 
ja nii edasi, kindlasti oli asi tükati väga halvasti kajastatud. Ja me  muidugi 
kujunduse peale, ikkagi mata teaduskond, väga palju auru ei raisanud. Aga igal 
juhul sai rektor teada, mis asi on veeb, kuigi meil vist tol hetkel oli 
näppudel üles lugeda, mitu veebiserverit Eestis üldse tol hetkel oli. Ja siis 
nad võtsid asja üle ja hakkasid päris ülikooli veebi tegema. Rektorid on meil 
alati olnud suhteliselt taibukad. Nii palju kui mina neid näinud olen. 

\question{Enne kui oma jutule joone alla tõmbame, on mul üks niisugune 
abstraktne küsimus. Ma ei teagi, kas sellele ongi head vastust, aga kui keegi 
kuskil teab, siis tõenäoselt sina tead kõige täpsemalt. Mis osa kogu sellest 
asjast, mis on meid toonud  70.-te juurest siiamaani välja -- kogu  IT-tööstus 
ja kõik see, et me  Liivi tänavalt oleme nüüd jõudnud siia Deltasse, kõik see  
areng ja kogu see värk mis professor Kaasiku\index[ppl]{Kaasik, Ülo} pusimisest 
 pihta hakkas -- on selline mäetipul kaugusse vaadates püsti pandud visioon ja 
kui palju sellest on lihtsalt, et \enquote{teeme järgmised kaks nädalat ägedaid 
asju}?} 

No osakaaludeks ma seda jagada ei jõua. Aga näiteks internetikoolitus 
õpetajatele oli küll see, et teeme kahe nädalaga kihvte asju. Kuna Eestis 
internet arenes pärast seda, kui see päris internet siia kohale jõudis,  ikka 
väga penikoormasaabastega, siis me otsustasime, et me hoiame ka õpetajaid 
kursis. Ja igasuguste koolituste käigus tegime  suures ringauditooriumis (tol 
ajal muuseas ülikooli ei küsinud auditooriumi eest tasu, raha meil ei oleks 
olnud) infopäevi. Rääkisime, mis asi on internet ja kuidas ta on arenenud ja 
nii edasi. 

Siis mingil hetkel tekkis Marek Tiitsul\index[ppl]{Tiits, Marek} üks 
europrojekt, mille käigus ta sai 100 modemit. 50 oli tal projekti jaoks vaja, 
aga 50 võisime koolidele jagada. Me ei hakanud neid niimodi loopima vaid me 
korraldasime süsadminide või postmaasterite kursused. Aga siis me tegime juba 
siukseid kombineeritud kursusi, et meil oli viis õpetajat rühmas: kellelegi me 
õpetasime hiire liigutamist, kellelegi me õpetasime vist  programmeerimist, 
kellelegi õpetasime, kuidas modemit paika panna ja kuidas sinna teenused peale 
tõmmata. Kellelegi me õpetasime, mis on veeb ja kuskilgi õpetasime veel mingeid 
süsadmine või jumal teab keda. 

Neid postmasterite kursuseid me saime neid teha kaks korda niimoodi, et kõik 
kohale tulnud koolid, kes edukalt kursuse (kahepäevane kursus, laupäev, 
pühapäev, õpetajad koolituvad entusiasmist, me mingit tasu ei maksnud. Mingil 
korral meil oli raha, me saime piletid kinni maksta, aga mitte alati) läbisid, 
said modemi kaasa. Meil puudus kontroll, mis nendest modemitest pärast sai. 95. 
aastal meil nagu enam ei mahtunud õpetajat kuskilegi ära, isegi mitte Vanemuise 
suurde auditooriumisse\index{Tartu Ülikool!Vanemuise tänava 
õppehoone!Ringauditoorium} ei kippunud ära mahtuma ja meie võhm hakkas otsa 
saama. Ma käisin PTUIS, Pedagoogika Teadusliku Uurimise Instituudis, kes 
organiseeris õpetajate koolitust, küsimas, et kas nad meie kursustele ei taha 
raha anda. Ja rääkisin, et telekommunikatsioon tuleb kohe ja siis siis vastab 
ülemus, kelle nime ma kahjuks ei mäleta, \enquote{Misasi? Tele? 
Kommunikatsioon? See asi ei tule eesti kooli mitte kunagi!} Ma panin suu kinni 
ja jätsin ütlemata, et  50 kooli on juba moodemiga ühendatud. Keerasin otsa 
ringi, tulin sealt välja, aga sain aru, et sealt august raha ei tule. Sorose 
Fond\index{Sorose Fond}, või siis Avatud Eesti Fond\index{Avatud Eesti Fond}, 
oli see, mis meile pärast igasugusteks üritusteks natuke raha andis. 

Ja siis me tegimegi ilma rahata. Keegi EENetist, Enok Sein\index[ppl]{Sein, 
Enok} või keegi, alguses oli Marek Tiits\index[ppl]{Tiits, Marek}, siis mu 
tudengid. Ja nagu ma ütlesin, oli aeg, kui Tartu Ülikool ei küsinud 
auditooriumide ja arvutiklasside kasutamise eest tasu, sest raha meil ei olnud. 
Kõik me tegime seda puhtast entusiasmist ja ka esimesed e-kursused tegime 
puhtast entusiasmist. Me lihtsalt ei jõudnud neid suuri kahepäevaseid kursusi 
enam teha. Ja siis me istusime Terje Tuisuga\index[ppl]{Tuisk, Terje} kahekesi 
kokku ja mõtlesime, et aga teeks nüüd seda õige teisiti. Modemid on ju olemas, 
e-kirju nad saavad, üks inimene on koolis, kes oskab modemi käima panna. 
Korjame tema ümber viis õpetajat, teeme neile koolituse. Esimesel koolitusel me 
unustasime piirarvu panemata. Andsime õpetajatele teada, et niisugune koolitus 
tuleb, registreeruge. Et just see inimene, kes modemiga hakkama saab 
registreerigu oma kool. Me mõtlesime, et kui viis kooli tuleb, on ikka jube 
hästi. Panime mingisuguse fiktiivse emaili aadressi, kuhu nad registreeruma 
peaksid. Ja siis meil oli nii kiirem, ei me ei käinud seda vaatamas ja kui me 
seda vaatamas käisime siis oli seal juba üle 20 kooli. Ja siis me mõtlesime, et 
aga mis vahet seal on? Mõningad asjad tuli ära muuta: kui me mõtlesime, et viis 
kooli, igast ühest viis inimest, siis nad võivad kõik meile oma elu esimese 
e-kirja saata ja me saame kõigile individuaalselt vastata. Kui neid koole 
pärast oli 50 ja osavõtjaid 400, siis me saime aru, et meie isiklikult igalühel 
individuaalselt kirju ei kirjuta. Aga  lasime neil omavahel suhelda, panime nad 
paari. Panime paari niimoodi, et nad ei oleks üksteise poole külla saanud väga 
lihtsalt sõita, 50 kilti pidi vähemasti vahet olema ja nad pidid sama aine 
õpetajad olema. See oli üks igavene paaritamine, ma mäletan.  

\question{Mulle nii meeldib, et see on täpselt programmeerija lähenemine 
ülesandele! Paneme paari ja tohi olla üksteisele lähedal ja peab olema sama 
aine!}

Lihtsalt loogiliselt tuleb mõelda, loogiline mõtlemine tuleb elus õige mitme 
koha peal kasuks! 

\question{Vat need on kuldsed sõnad, millega võikski ehk lõpetada. Aga mul on 
üks küsimus veel: millega professor Anne Villems täidab oma aega praegu?}

Esiteks, nagu te kuulsite, ma tulin just USA-st. Kahjuks minu need tuttavad, 
kellega ma vanasti sain tihedalt läbi  käia ja kes olid Moskva erinevates 
instituutides ja ülikoolides, need ei ole enam Moskvas. Need on Californias. Ja 
nüüd ma olen avastanud, et California külastamise kõige meeldivam aeg on 
jaanuari lõpp-veebruari algus, mis just sobib mulle, sest siis on ülikooli 
vaheaeg. Ja nüüd ma olen neli aastat vist järjest käinud oma sõpradel külas 
Californias. 

Üks parimaid sõbrannasid elab seal. Nüüd mul õnnestus seal veeta kaks nädalat 
ja siis ta mind ära saates ütles, et \enquote{nüüd sa oled aru saanud, et  kaks 
nädalat on õige aeg, mitte üks nädal. Nii et järgmine kord tule ka kaheks 
nädalaks!}. Nii et jaanuari lõpp-veebruari algus võite te mind alati leida 
Californiast Palo Altost, California kevadest, kus nii kohalikud kui ka 
sissetoodud taimed nagu näiteks eukalüptipuud õitsevad. 

\question{Mis on kõik hurmav lisaks sellele, et tegemist on Palo Altoga!}

Mis on eksole, Stanfordi kodulinn ja kus Palo Alto ja ookeani vahel on toredad 
mäed. 

Ja teispool mägesid on ka päris ookean. Mäe otsas on ka tore käia. Üks sõber 
viis mu San Jose juures mäe otsa, kust me oleksime pidanud nägema ühele poole 
San Franciscosse, seal on väga pikk vahemaa vahepeal. Ja teist poolt siis oleks 
näinud San Josed. Ja oli piimjas udu ja vihmapilved. Ja siis korraks tuli tuul 
ja lõuna poole nägime siis seda vaadet, mida pidi nägema. Nii et jah, väga tore 
on reisida! Aga ma siin töötan veel. Küll tunnitasu alusel ja loen oma 
armastatud andmebaaside kursust. Kolmes versioonis. 