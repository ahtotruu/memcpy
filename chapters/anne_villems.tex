\index[ppl]{Villems, Anne}

\question{Kuidas sina arvutite juurde said?}

Mina käisin Mari Ülikoolis\index{Mari Ülikool|see{Tartu 10. Algkool}}. Seda nime
nime rahva seas kandis tolleaegne Tartu 10. Algkool\index{Tartu 10. 
Algkool}, mis asub Vanemuise\index{Tartu Ülikool!Vanemuise tänava õppehoone} 
kõrval ja vana Vanemuise vastas. Nimi tulenes karismaatilisest 
matemaatikaõpetajast Marvetist\sidenote{Vladimir Marvet\index[ppl]{Marvet, 
Vladimir} (1903--1994).}, kes oli seal ka õppealajuhatajaks.

Kui me pinginaabriga selle kooli 1960. aasta kevadel lõpetasime, siis 
meie vanemad olid väga huvitatud panemast meid 5. kooli\index{Tartu 5. 
Keskkool}. See asus tol ajal Rostovtsevi ülikoolis, mis oli esimene 
ülikool, mis naisterahvaid vastu võttis. 

Meie ei tahtnud 5. kooli minna, ma ei tea, miks. Meie otsustame! 
Ikkagi 14 aastat isiklikku vanust ja loomulikult tuleb ise otsustada, kuhu kooli 
lähed. Nii otsustasimegi, et läheme hoopis 
Treffneri Gümnaasiumisse\index{Hugo Treffneri Gümnaasium}, mis siis kandis 
nimetust Anton Hansen Tammsaare nimeline Keskkool. 

Minu meelest tol ajal ei 
olnud üldse seda probleemi, et pidi katseid tegema, aga 
meil oleks ka katsetega hästi läinud, sest õppisime mõlemad üsna 
korralikult. Kui olime aasta Treffneris olnud, siis avati seal 
matemaatikaklass. Minu jaoks oli see kohe sirge otsetee, sest
matemaatika oli mu lemmikaine. Kõiges on süüdi muidugi Mari, tähendab õpetaja 
Marvet\index[ppl]{Marvet, Vladimir} -- nii kihvte matemaatikatunde, 
nagu Marvet meile viiendast seitsmenda klassini tegi, tean pärastisest ainult siis, 
kui Olaf Prinits\index[ppl]{Prinits, Olaf} tuli keskkoolis meid õpetama. Aga 
see oli juba matemaatikaklassis. Nii et kui läksime üheksandasse klassi, sest 
tolleaegne algkool lõppes seitsmenda klassiga, läksin mina kõigepealt matemaatikaklassi ja kahe kuu pärast tuli pinginaaber ka järele, sest 
keskkoolis algas mingi praktiline õpetus ja talle ei pakkunud see, mida nende 
klassile anti, erilist pinget. Tuligi ära meile, 
kuigi ta oli humanitaarsete huvidega. 

Matemaatikaklassis oli minu suureks rõõmuks viis tundi matemaatikat nädalas 
ja muude ainete seas ka sellised toredad ained nagu elektrotehnika, ligikaudne 
arvutamine ja programmeerimine. 

\question{Mille peal programmeerimist õpetati?}

Sel ajal oli Eestis olemas üks elektronarvuti, mis kandis 
nimetust Ural\index{Ural}. Sellele ei olnud veel numbrit \enquote{1} külge 
pandud ja see asus Tartu Ülikooli arvutuskeskuses\index{Tartu 
Ülikool!Arvutuskeskus}, mis moodustati 1959. aastal. Masin ise asus peahoone 
kõrval majas, Treffnerist üldse mitte kaugel, üks kvartal. 
Õpetasid meid seal alguses Ülo Kaasik\index[ppl]{Kaasik, Ülo} ja pärast Mati 
Krull\index[ppl]{Krull, Mati}, kes oli sinna tööle läinud sellepärast, 
et tema oli esimeselt kursuselt ülikoolis, kellele Ülo Kaasik\index[ppl]{Kaasik, 
Ülo} programmeerimist õpetas. 

\question{Nii et Ülo Kaasik hakkas ilma arvutita programmeerimist 
õpetama?}

Ei, arvutiga. Me olime üheksandas klassis 1962. aastal ja 
seesamune arvuti oli juba kaks aastat kohal. 

\question{Kellelgi pidi keskkoolis olema parasjagu visiooni ja arusaama, 
et elektrotehnika, ligikaudne arvutamine ja arvutid on olulised. 
Kes see inimene oli?}

Visioon võib-olla ei olnud isegi keskkoolil, kuigi meil 
oli ka väga karismaatiline direktor Allan Liim\index[ppl]{Liim, 
Allan}, aga tema oli ajaloolane. Arvan, et initsiatiiv tuli tegelikult 
Olaf Prinitsalt\index[ppl]{Prinits, Olaf} ja Ülo Kaasikult\index[ppl]{Kaasik, Ülo}. See ei oleks olnud Nõukogude Liidus erakordne. Kui nad arvuti 
Nõkku\index{Nõo Keskkool} panid ja Nõos oli esimene kooli arvutuskeskus, siis 
see oli kogu liidus esmakordne juhus. Aga matemaatikaklassid olid tol ajal juba 
olemas ja ka Moskvas. Nii et nad võisid öelda, et järgivad 
Moskva malli, ja probleeme ei tulnud. Treffneri kool sai neil valitud võib-olla sellepärast, et see oli südalinnas ja 
ei olnud vaja kaugele minna, sest oli selge, et vähemasti esimestel 
aastatel hakkavad õpetama ülikooli õppejõud. 

Meile õpetas Olaf Prinits 
matemaatikat. Mul ongi olnud täiesti fantastilised 
matemaatikaõpetajad: algkoolis õpetaja Marvet ja keskkoolis Olaf 
Prinits\index[ppl]{Prinits, Olaf}. 

\question{Kuidas matemaatikahuvi arvutihuviks üle läheb? Matemaatika 
on abstraktne ja kaunis kunst, aga arvutid tolmavad ja flogistonid\sidenote{Teadupärast
töötavad arvutid flogistonide abil ja 
kui arvuti katki läheb, siis lahkuvad flogistonid sellest sinise suitsuna ning 
arvuti enam käima ei lähe.} lendavad aeg-ajalt välja \ldots} 

Tolm tuleb pärast. Tol ajal tolmu ei olnud. 

Need on muidugi omavahel seotud oluliselt kõrgemal tasemel. Pead ikka natuke matemaatikat oskama, enne kui saad aru, kuidas matemaatika informaatikale või programmeerimisele kasulik on. Ja 
enne pead natuke programmeerida oskama, kui adud, et mõnikord on ka
matemaatikat vaja. Näiteks alustades sellest, et algoritmiliselt 
mittelahenduvaid ülesandeid eristada algoritmiliselt lahenduvatest ja neid 
lahenduvaid jagada ka sellistesse klassidesse, mille lahendid ei tule mitte 50 
või 50 000 aasta pärast, vaid arvutist kui mitte homme 
hommikuks, siis vähemasti ülehomme lõunaks.

\question{Seda enam tekib küsimus, miks matemaatikahuvilisel inimesel ei ole 
tingimata arvutihuvi. Miks sul oli?}

Alguses oli lihtsalt jube põnev. Kogu Eestis oli ainult üks arvuti ja meile 
õpetati! Tuled vilguvad ja perfolindilt (mis on filmilint, mitte 
telegraafilint) loetakse andmeid! 

Mõnikümmend aastat hiljem on meil igaühel arvuti taskus või laua peal ja meie $\epsilon$-ümbruses\sidenote{Olgu{\ } $(X 
; \rho)$ meetriline ruum, $p \in X$ ja $\epsilon > 0$. 
Punkti $p{\ } \epsilon$-ümbruseks nimetatakse hulka $\{x \in X | 
\rho(x,p)<\epsilon\}$. Teisisõnu on $\epsilon$-ümbrus meie otsene 
ümbrus.} tuhandeid kordi võimsamad arvutid kui see, mis meil tookord 1962. või 1963. aastal (1964 lõpetasin ma keskkooli) võttis suures 
saalis enda alla mitukümmend ruutmeetrit. 

Õppima läksin muidugi puhtalt matemaatikat, sest ega kuskil (võib-olla TPIs) ei õpetatud niisuguseid 
tehnikaaineid, kus ka arvutid otsapidi sisse tulid. Ehk isegi õpetati, aga ma 
ei tahtnud, sest tehnika pool mind väga ei tõmmanud. 

Programmeerimine on maagiline tegevus. Enne 
seda arvuti midagi ei oska, siis kirjutad kihvti programmi ja 
siis järsku oskab ning näeb peaaegu välja nii, nagu saaks 
millestki aru. 

\question{Meelis Roosi\index[ppl]{Roos, Meelis} esimene 
programm oli ka vestlusprogramm, mis jätab
mulje, et arvuti saab millestki aru. See on läbiv joon.}

Jah. Üheksandas või kümnendas klassis tegime elu esimese 
programmi. Minu programm ei pakkunud mulle küll nii palju pinget, aga elu lõpuni 
jääb meelde ühe koolikaaslase oma, kus ülesanne seisnes selles, 
et tuli anda kuupäev ja siis pidi välja trükkima, mis nädalapäev see on. Aga 
Ural-1\index{Ural!Ural-1} ainuke väljundseade oli 
kitsas printer, kus sai trükkida ainult arve. Kui õppejõud proovis 
meie programme, siis andiski kõigepealt mingi mõistliku kuupäeva ette ja sai 
vastuse ning seejärel andis ette 30. veebruari, mille peale programm hakkas 
printeril ülalt alla nullide joru trükkima. Seepeale ütles õppejõud: 
\enquote{Nojah, ilmselt trükib mingit jama,} ja katkestas ära. 
Õpilane sõnas väga vaikselt ja tagasihoidlikult: \enquote{Las natuke trükib veel.} Ja trükkiski välja ühe nullide joru ülalt alla, siis ühe 
nullidega täidetud rea, ühe tühja rea, natuke nulle, siis natuke 
nulle äärtes ja keskel, seejärel veel kord nullide joru 
ülalt alla ja ühe terve nullirea ning lõpuks veel kord sama. Kui me selle 
paberi kätte saime, siis oli kõigile näha \enquote{LOLL}. 

\question{Kas tegitegi selliseid nuputamisülesandeid või mida te veel 
kirjutasite?} 

See oli üks huvitavamaid, aga tegime igasuguseid asju, näiteks leidsime mingite jadade keskmisi. Ma kõikide ülesandeid ei teagi, need olid individuaalselt antud tavalised ülesanded, mida ka praegused algajad 
programmeerijad teevad.

Muuseas, selle jaoks, kuidas kuupäevast nädalapäeva teha, on olemas valem. 

\question{Kas sel ajal oli Nõukogude Liidu peale juba ka mõnd arvutiasjade kogukonda, olümpiaade või muud sellist?} 

Ma ei tea, kuna programmeerimist õpetati ikka äärmiselt vähestes koolides. Ma ei 
tea ka, kas näiteks Moskvas matemaatikaklassis arvutit õpetati. Kui ma kunagi
Ülo Kaasiku\index[ppl]{Kaasik, Ülo} käest ühe intervjuu käigus küsisin, 
kust ta võttis metoodika meile programmeerimise õpetamiseks, siis ta vaatas 
mulle suurte silmadega otsa ja ütles: \enquote{Metoodikat ei olnud mitte 
mingisugust!} Vähe sellest, et ei olnud metoodikat, ei olnud ka kirjandust. 
Nii et ta hakkas ise raamatuid kirjutama ja metoodika oli tal nii-öelda 
käigu pealt välja töötatud.

\question{Ühesõnaga, tegelikult käis kogu Eesti asi Ülo Kaasiku peal?} 

Jah, see asi arenes kindlasti tänu talle ja ta õpilastele. Pärast 
seda hakkas matemaatikaklasse tekkima hulgi. Mäletan, et omaaegne Nõo direktor\index{Nõo Keskkool} Ove 
Karu\index[ppl]{Karu, Ove} istus meie matemaatika lõpueksamitel, et vaadata, kuidas me 
oleme matemaatikas arenenud. Mäletan isegi omaenese vastust. Kuna mul oli 
keeruline joonis teha, siis selle seletamisel läksin puntrasse. Astusin kaks 
sammu tahvlist eemale ja alustasin otsast peale: \enquote{See on see tõestus, et 
sirge on risti tasapinnaga, kui see on risti kahe sellega lõikuva sirgega.} See 
oli keskkooliprogramm. 

Meil on arvutiteaduse instituudis\index{Tartu Ülikool!Arvutiteaduse instituut} 
projektikoor -- koorilaul tuleb meelde poolteist aastat enne laulupidu ja siis 
võtame oma koori kokku ja valmistume nii hästi, kui suudame, ettelaulmiseks ning 
käime laulupidudel. Ja kui laulupeol küsiti, kuidas me oma koori 
iseloomustame, siis pakkus igaüks midagi välja. Mina ütlesin, et iga meie koori 
liige teab Pythagorase teoreemile erinevat tõestust. Neid on umbes 200, 
kooriliikmeid ligikaudu 40. Seepeale tuletasid matemaatika 
õpetamisega tegelevad õppejõud meie koorist mulle meelde, et 
Pythagorase teoreemi enam ei tõestata koolis. Nii et ma nüüd ei tea enam mitte 
midagi.

\question{Ühel hetkel sai keskkool otsa ja tuli niisiis Tartu Ülikool ja 
matemaatika?} 

Jah, aasta oli siis 1964 ehk sügav nõukogude aeg, mis välistas minu jaoks arusaadavatel 
põhjustel absoluutselt kõik humanitaaralad. Järele jäi suhteliselt vähe, sealhulgas meditsiin. Isa oli kirurg ja ju ta 
vaikselt lootis, et äkki lähen seda õppima. Mul on loogiliselt mõtlev mälu ja 
igasugused tõestused jäävad kergesti meelde, aga kui pean 
pähe õppima 2000 kontide nimetust ning lihaste, veresoonte, 
ajusagarate ja jumal teab veel mille ladinakeelsed nimed, siis ma ei 
arva, et tunneksin ennast väga hästi. Peale selle ei tahtnud me juba 5. keskkooli 
minna selle pärast, et vanematel olid seal liiga head suhted. 
Tahtsime ikka ise olla. Ka ülikoolis tahtsin olla rohkem ise kui keegi muu. 
Nii jäigi järele matemaatika ja seda ma armastasin koledasti. 

Mul olid viiendast seitsmenda klassini head õpetajad ja ma ei saa midagi 
paha öelda ka oma kaheksanda klassi õpetaja kohta, kes soovitas mind matemaatikaklassi. Lisaks matemaatika õpetamise korüfee Olaf 
Prinits\index[ppl]{Prinits, Olaf}, kelle tunde (näiteks funktsionaalse seose selgitust) mäletan siiamaani, ligi 50 aastat hiljem. 

\question{Kas Tartu Ülikoolis õpetati toona matemaatikutele ikka ka 
programmeerimist?}

Jah, meil oli kaks kursust programmeerimist. Üks oli masinkoodis 
programmeerimine Ural-4\index{Ural!Ural-4} peal, mille tegin 
kohe septembrikuu jooksul ette ära. Õppejõud võttis mu 
vastuse ja ütles: \enquote{Lõhnab natuke Ural-1 järele, aga programmeerida te 
oskate} ning pani mulle arvestuse. Teine oli Ülo Kaasiku\index[ppl]{Kaasik, 
Ülo} Algol 60\index{Algol!Algol 60} õpetus ja vot seda õpetas ta küll pliiatsi ja 
paberiga, sest ühtki translaatorit tol ajal Algoli jaoks ei olnud. 

\question{Miks just Algol?}

Mis keeled aastatel 1966--1967 üldse olemas olid? 

Jumalale tänu, et keegi ei hakanud meile Cobolit\index{Cobol} õpetama. See 
oleks mind küll programmeerimisest viie kilomeetri kaugusele peletanud! Olete 
proovinud kunagi Cobolit lugeda? Ei ole? Ärge proovige ka! Programmeerimine 
on kontsentreeritud väljendus, kus saab valemeid kirjutada valemitena, ja 
siis on mõningad kenad koodsõnad \verb|for|, \verb|do|, 
\verb|if-then-else| ja nii edasi. Nüüd pange sinna mingisugune filoloogide 
soust peale, kus \enquote{palk pluss preemia} tuleb välja kirjutada kolme 
sõnana! Vot niisugune programmeerimiskeel ja see oli vahepeal väga elujõuline.

\question{See kõik kõlab ikkagi 
eksplitsiitse programmeerimisõpetusena. Keegi õpetas metoodiliselt programmi 
kirjutama ja see on üsna haruldane. Üldjuhul ütlevad inimesed, et 
programmeerimine jäi kuidagi külge. Nad ei oska öelda, kes ja kus õpetas.}

Kui tahtmine on väga suur, siis kindlasti on võimalik programmeerimist 
iseseisvalt õppida, aga motivatsioon peab ikka seinakõrgune olema. 

\question{Mis toona arvutite perspektiiv oli, milleks neid kasutati? 
Programmeeriti küll, aga kas see oli teaduslik töövahend, aitas 
rahvamajandusele kaasa või veel midagi?} 

Rahvamajanduses oli kasutus juba Ural-1\index{Ural!Ural-1} ajal, kuigi mälu oli 
väga vähe -- kaks kilo, mida täna keegi ei usu. Õnneks mitte 
baiti, sest baiti tol ajal ei tuntud, vaid sõnu -- kaks kilo sõnu. Võtke oma 
telefon ja vaadake nende mega- või gigabaitide arvu, mis teil taskus on. 
Siis oli väga suur probleem, kuidas neid andmeid, mille pealt midagi pidi 
arvutatama, arvutisse ära mahutada. Aga need pakiti kokku. Välisseadmetega olid ka veel 
oma probleemid.  

Mäletan, kuidas ma ise masinkoodis programmeerimist õpetasin. Nimelt läksin miskipärast viimase kursuse viimasel semestril 1970. 
aastal mehele ja siis järsku langes ära minu Tallinnasse tööleminek, sest 
mees teatas surmkindlalt, et tema küll Tallinnasse ei kavatse minna. Siiamaani 
ei ole läinud. Pidin ruttu Tartus töökoha otsima ja Ülo Kaasik\index[ppl]{Kaasik, Ülo} ütles: 
\enquote{Jah, ma usun küll, et tema võib tudengite ette saata, tal jalad ei 
värise!} Lõpetasin ülikooli juunis ja 
septembris läksin tudengitele programmeerimist õpetama. 

\question{Kas mõnd teadustööd ei sirgunud sealt, oli puhas õppejõu töö?}

Teadusega hakkasin tegelema palju hiljem. Algajate õppejõudude nii-öelda jalul 
ehk auditooriumi ees seismise koormus (vähemasti siis, kui mina õpetasin) oli 24 
või 28 tundi nädalas. Selle kõrvalt ei jõudnud midagi muud teha, eriti esimestel 
õpetamisaastatel. Pealegi tuli õpetada kahes keeles,
matemaatika poolel oli vene keelt rääkivaid inimesi väga vähe. 

\question{Kas oligi kaks eraldi gruppi? Eesti ja vene?} 

Jah. Õpetasin elu esimese loengu majandusteaduskonnas vene 
keeles, vist kaugõppijatele. Vene keelt 
oskasin väga nirult. Esiteks ei olnud seda vaja. Teiseks olin küll lugenud 
väga palju matemaatikaõpikuid, aga võin neid tõestusi lugeda ka 
prantsuse keeles, mida ma ei oska, sest vahesõnu on seal
suhteliselt vähe. Õpetamiseks aga on vaja palju sõnu. 

Õnneks jätkus mul taipu: meie venekeelsetes rühmades oli alati 
kakskeelseid tudengeid ja ma tegin kokkuleppe, et kui ütlen midagi sellist, 
millest ei ole võimalik ka hea tahtmise juures aru saada, siis esimene rida 
annab mulle teada ja püüan ümber sõnastada. Ja kui sõna meelde ei tule, ütlen eesti 
keeles ja kakskeelsed ütlevad mulle venekeelse vaste.

Minu vene keele mitteoskust iseloomustab näiteks see, et ma ei 
teadnud, kuidas on \enquote{lahutama}. Üritasin kasutada sõna
\begin{russian}отнять\end{russian}, mis ei ole ka väga vale, aga sõna 
\begin{russian}вычесть\end{russian} ma ei teadnud.

\question{See on ju pedagoogilise metoodika mõttes õudselt hea kool!} 

Jah! Igal juhul umbes seitse-kaheksa aastat pärast seda 
sattusin kuskil Nõukogude 
kodumaal konverentsile, kus ütlesin midagi seltskonnas vene keeles, mille peale 
küsiti: \enquote{\begin{russian}А вы из Москвы?\end{russian}}. 
Vastasin seepeale, et kui ma teist korda veel suu lahti teen, siis 
saate kohe aru, et ma ei ole ei Moskvast ega ka Leningradist. 

\question{Ja õppejõu töö Tartu Ülikoolis läkski edasi, kuni saabus 
aasta 1980?} 

Veel kauem, siiamaani. 

\question{Aastal 1980 algas see aeg, kui asjad läksid lihtsamaks ja 
arvutid väiksemaks ning neid tuli juurde.}

Arvutid läksid jah väiksemaks. Kaheksakümnendatel oli arvutite saamine omaette tsirkus. 

\question{Jaan Tallinn\index[ppl]{Tallinn, Jaan} on rääkinud\sidenote{Mitte küll nende kaante vahel olevas jutus.}, et tema tõi oma 
esimese arvuti käsipagasis laevaga Rootsist.} 

Minul see ei õnnestunud, sest ma ei olnud nii rikas. Abikaasa oli Uppsalas postdoktorantuuris ja ma käisin seal küll, aga
arvutit ei toonud. Kuigi ma tean, et kui oleksin sealt ostnud arvuti ja siin 
maha müünud, oleks selle eest väga palju muid asju saanud. Paraku mul 
sellist ärivaimu ei olnud. 

\question{Millal hakkas Tartu Ülikoolis\index{Tartu 
Ülikool!Matemaatikateaduskond} arvuti tavaliseks asjaks muutuma ja sai 
igapäevaseks osaks näiteks matemaatikateaduskonna elust?} 

Võib-olla üks murdepunkt oli aastal 1982. Tallinna 
sõbrad organiseerisid Tallinna näituseväljakul välisnäituse, kuhu tuli sadakond firmat. Kuigi seal oli näitusi varem ka olnud, köitis see minu tähelepanu, kuna
kohale pidi tulema kellegi sõber 
Suurbritanniast, kes oli hakanud Apple'i diileriks. Ta pidi ehitama ühe seadme ülikõrgete rõhkude jaoks, 
aga seadet oli vaja juhtida ja selleks valis ta välja Apple'i. 
Odavaim viis seda saada oli hakata Apple'i diileriks. Ja nüüd pidi ta siia
näitusele tulema. Ma ei tea, mida muud ta veel tõi, aga igal juhul valmistusid
Teaduste Akadeemia\index{Teaduste Akadeemia} instituudid ja pooled head 
tuttavad tema käest Apple'i arvuteid ostma. 

Mul läks hammas 
koledasti verele: sõbrad saavad miskipärast arvuteid, aga tegelikult on neid 
hoopis mulle vaja. Milleks näiteks kadunud professor Lippmaale\index[ppl]{Lippmaa, 
Endel} arvuti? Minul oli arvutit vaja! Rohkem
kui tal! Tal olid magnetresonantsid, mida ta pidi juhtima, 
aga meie õpetasime tulevasi programmeerijaid välja ja pidime kasutama 
\emph{input-output} kappi, kui keegi teab, mis see on. Tudeng paneb sinna
perforeerimiseks oma blanketile kirjutatud programmi ja saab kolme-nelja päeva 
pärast tagasi süntaksivigadega väljatrüki. 

Selleks, et üldse aru saada, millest 
Apple koosneb, istusin pühapäeviti Tõraveres\index{Tõravere Observatoorium}, 
sest seal käis Byte\sidenote{Aastatel 1975--1998 välja antud USA 
arvutiajakiri, mis oli kaheksakümnendatel vägagi mõjukas.}, kust sain teada, 
mis on arvuti sees ja mis selle külge käib. Panin oma konfiguratsiooni kokku ja sain aru, et mul ongi vaja nii-öelda
alasti arvutit. Pidasin aru Tallinna sõpradega Lippmaa 
instituudist\index{KBFI}\sidenote{Selle nime all tunti KBFid.}, mis on 
mõistlik summa, mida plaanikomiteest küsida. Leidsin, et 10 000 kuldrubla (välisturul rubla 
ei toiminud, ainult kuldrubla) oleks piir, mida võiks küsida. See on 
nii pisike summa, et keegi äkki annabki. 

Saime selle eest kolm arvutit, isegi 
monitore ei ostnud, ainult ühe igaks juhuks, sest mine sa isahane tea. Võib-olla meie 
nõukogude televiisoritega ei töötagi. 

\question{Mis arvutid need olid?}

Apple II\index{Apple II}. Plaanikomitees käisid minu eest Lippmaa 
instituudi sõbrad Tõnu Karu\index[ppl]{Karu, Tõnu} ja Riivo 
Sinijärv\index[ppl]{Sinijärv, Riivo}. Mina ajasin kõik paberid korda, korjasin 
ülikoolist allkirjad peale kuni rektorini välja ja nemad käisid kohal. Antigi 10 000 ning me saime oma kolm arvutit.

Tegime nendest arvutiklassi ja panime käima 
programmeerimisõppe individuaalgraafikus praktikumidega. See oli väga 
tore aeg, sest inimesed olid harjunud \emph{input-output} kapist saadud paberirulliga oma süntaksivigadega, mida tuli siis parandada. 
Selle pisikese, elu esimese või teise programmi silumine võttis mitu 
nädalat aega. 

\question{Järelikult tarkvaratehniline pool muutus radikaalselt!} 

Jah. Arvutid saime kätte jaanuaris ja veebruari alguses panime 
programmeerimise algõpetuse käima. Kuna meile BASICu\index{BASIC} keel ei 
meeldinud, aga Apple II\index{Apple II} on sündinud BASICuga, mõtlesime, et 
suudame ehk kompenseerida BASICu hädad oma hea õpetamisega. Esimese 
programmi jaoks kõlbas küll. 

Kevadel paistis sinna Liivi tänava\index{Tartu 
Ülikool!Liivi õppehoone} klassi päike. Seisin arvutiklassis tudengite 
selja taga, et neid aidata, ja tegelikult ma ei näinud, mida nad 
ekraani peale kirjutasid, vaid nende enda peegeldust. Ja see, kuidas tudeng 
sisestas oma programmi, pani käima, sai sealt ise oma süntaksivead kohe kätte ... 
See miimika, eriti tütarlaste oma, tasus vaatamist, mõistmaks, et oleme kuhugigi suunas õige sammu astunud! 

\question{Õpetamise metoodika pidi ka ju muutuma?} 

Jah, koos keelega muutub see alati. Põhikisma, mida algõpetuses arutatakse, on 
see, missugust programmeerimiskeelt õpetada esimesena, sest sealt jäävad 
asjad külge. BASICu\index{BASIC} häda on see, et kui olin selles
valmis kirjutanud (küll juba Yamaha arvutil, mida me kasutasime laialdasemalt 
arvutite tutvustamiseks) programmi, mille väljatrükk oli umbes minu enda 
pikkune, siis ma vandusin, et see on minu viimane programm BASICus. Nimelt ei ole selles keeles 
funktsioone ega alamprogramme. Olen seda vannet siiamaani pidanud. 

Mulle õpetati kõigepealt masinkoodi, sest Urali 
peal mingit kõrge taseme keelt ei olnud, samuti mitte assemblerit. Ja nii 
mulle tundus, et mängin otse registritega ja 
saangi aru, mis masinas toimub ja mida see aritmeetiline plokk seal teeb, ning edasi on kõik väga lihtne. 

Kunagi kui võitlesin ühel üleliidulisel 
seminaril selle eest, et alustada tuleb ikkagi masinkoodist, rääkis üks mu nüüdseks 
juba kadunud sõber Novosibirskist minu seisukoha 
ilmestamiseks anekdoodi. Vaene mees läheb 
kirikuõpetaja juurde ja ütleb, et elu on kole raske: peab naise, laste ja ämmaga
elama oma onni ühes toas. Kirikuõpetaja 
küsib: \enquote{Kas sul kits on?} \enquote{Jaa, kits mul on!} \enquote{Ole hea, võta kits ka tuppa.} Talumees imestab, milleks veel kits, aga kuna kirikuõpetaja käsib, siis võtabki 
kitse tuppa. Siis ütleb kirikuõpetaja: \enquote{Nädala pärast vii kits välja 
ja tule räägi minuga.} Nädala pärast viibki mees kitse välja ja kiidab kirikuõpetajale:
\enquote{Nüüd on küll väga hästi, kitse ei ole ja saan oma naise, ämma ja lastega palju paremini hakkama.} Masinkoodist alustamine on 
nagu kitse toomine tuppa, et kui lõpuks saab hakata programmi kirjutama 
\verb|for|, \verb|if| ja \verb|else| abil, siis tekib suur lõõgastus. Enam ei pea 
tõstma midagi registrisse, kontrollima, andma suunamist ja nii edasi. 

\question{Kui 
alustasime sellest, et sulle meeldis programmeerimine, siis ühel hetkel andis 
positiivse emotsiooni näoilme muutus ekraanipeegeldusel. Mis hetkest muutus
programmeerimise õpetamine huvitavamaks kui programmeerimine, kui 
üldse niisugune hetk on olnud?}

Jah, ma usun küll. Selleks et midagi väga tähelepanuväärset programmeerimises ära teha, on vaja 
head meeskonda. Parim suur asi, mida ma
programmeerimises olen teinud, on omal ajal koos Ain Isotammega\index[ppl]{Isotamm, Ain} kirjutatud süsteem Villis, mis oli aruannete 
generaator ja millel olid väga tähelepanuväärsed omadused. See oli ühtlasi elu 
keerulisim programm, mis tegeles magnetlindi ja printeri juhtimisega 
aruannete väljatrükkimise ajal, kui aruanded on pandud segamini magnetlindile (küll tekitamise järjekorras, aga 15 aruannet korraga, jupid vaheldumisi) 
ja printerist pidid tulema kõik aruanded õiges järjekorras. Mul oli kaks 
katkestuste allikat ja tasakaalu hoidmine printeri ja magnetlindi 
puhvrite vahel. Lisaks teisendamistöö seal vahel, et andmetest teksti 
tekitada, aga see oli köömes. 

Selle programmi tegemise käigus õnnestus mul 
muuseas avastada arvuti viga -- see on iga programmeerija unistus. Sulle kogu 
aeg tundub, et arvuti teeb valesti, sest teed ise kõik õigesti, 
aga arvuti eksib. Ja siis lähed seda insenerile rääkima. Kuna 
insener ei tea konteksti, siis hakkad asja seletama
algusest peale. Kuskil poole peal saad aru, millal oled ise vea teinud, haarad oma väljatrükid ja 
ütled insenerile, kes sinnamaani ei ole veel millestki aru saanud, aitäh 
posti mängimise eest ning lähed oma viga parandama. Aga selles programmis oli tõesti
arvuti viga, sest sünkro impulss traatprinterile oli 
halvasti joodetud. Kõik inimesed trükkisid sümbolhaaval, äärmisel 
juhul terve rea kaupa. Mina tahtsin terve lehekülje valmis teha ja
korraga trükkida. Printer trükkis mulle veerand lehekülge, pool 
lehekülge ja edasi lihtsalt ei trükkinud. Kontrollisin printeri juhtkäsku -- 
kõik on õige, aga ei trüki. Siis insenerid avastasid, 
milles asi. See on ainus kord minu programmeerijakarjääris, kui mul on õnnestunud 
arvuti viga avastada. 

\question{Kusjuures mitte arvuti kui tüübi viga, vaid et konkreetne tükk on 
spetsiifilisel viisil katki!}

Jah, konkreetne printer. Aga see on ikkagi riistvara! Inseneri, mitte programmeerija pärusmaa.

\question{Kuidas ikkagi õpetamine läks huvitamaks kui programmeerimine?}

Õpetamise värk on huvitav olnud kogu aeg, muidu ma ei oleks selle 
koha peal nii kaua\ldots{ }Tallinnas oli palju arvutuskeskusi, Tartus mitte eriti, aga ega
tolleaegne automatiseeritud juhtimissüsteemide 
tegemine mind väga ei tõmmanud ka. Nõukogude tehnika töökindlus oli niivõrd vilets. Isegi 
ameeriklastelt üle kavaldatud esimesed Nõukogude personaalarvutid ei kippunud 
hästi töötama. 

Õpetamine oli palju toredam! Meile räägitakse siiamaani, kuidas 
Eestis on ikka veel 8000 IT-spetsialisti puudu. 

\question{Need maagilised Schrödingeri 8000 spetsialisti \ldots \sidenote{Scrödingeri kass on kvantsuperpositsiooni illustreeriv mõtteline eksperiment, mille käigus kass on korraga nii elus kui surnud. IT-spetsialistid on meil korraga nii puudu (sest nii öeldakse) kui ka olemas (sest tööhõive sektoris kogu aeg kasvab).}}

Jah, see on enam-vähem konstantne suurus juba 
palju aastaid. Põhiline ei ole number 8000 ise, vaid
et neid ja neid on puudu -- näeme seda kogu aeg oma teise kursuse pealt. 
Nimelt omandavad inimesed esimese kursuse programmeerimise algõpetuse, teevad 
oma esimese projekti objektorienteeritud programmeerimises ja siis on nad 
firmale kasulikud ja ülikool lükatakse tagaplaanile. Kes veab välja 
ja kes mitte. Enamus ei vea ja jätavad ülikooli sinnapaika. 

\question{Kustkohast tuli idee õpetada välja arvutiõpetajaid? See eeldab 
ägedat visiooni, et keskkoolis või üldse koolis peaks arvutiõpetust 
õpetama.} 

Sellel oli kaks juurikat. Esiteks hakkasid arvutid 
jõudma ka koolidesse. Mitte ainult matemaatikaklassidesse, vaid ka mujale, sest 
tekkisid lihtsalt sellised arvutid, mis said jõuda. Selge see, et Ural-1 ei jaksanud ükski 
kool osta; füüsiliselt oleks see võimlemissaali mahtunud, 
aga seda ei saanud arvuti pärast ära kaotada. Nii et tekkis vajadus õpetada välja õpetajaid, kes 
läksid kooli arvutit õpetama. 

Teine juurikas oli arvutiside, mis hiilis koolidesse tagauksest. Riiklikud programmid, näiteks
Tiigrihüpe, tulid märksa hiljem. Esimesed ligi sada kooli said arvutisidega ühendatud nii, et paljud ülemused ei 
teadnudki üldse, et see olemas on. 

\question{Mille külge need läksid?}

Osa läks modemiga, telefoniside kaudu. 

\question{Tartusse kuskile?} 

Alustame algusest, 1980ndatest. Kõigepealt 
olid Fido vennad, kelle tegevusest ei teadnud ka mina 
suurt midagi, sest amatööride ridadesse ma ei kuulunud. Kuulsin 
alles hiljem, et Fido vendadel oli modemside juba enne, kui arvutid jõudsid UUCPga 
võrku. 

Kui siis arvutivõrgust midagi kuulda oli (ja kaheksakümnendatel juba
oli), siis otsustasid meie Soome sõbrad Nõukogude Eestit järele 
aidata. Tallinn on Helsingile palju lähemal kui Tartu, nii et Tallinna 
Tehnikaülikool\index{Tallinna Tehnikaülikool} sai soomlastelt modemi. Ja 
kuigi meil oli uuendatud telefoniside seoses 1980. aasta olümpiaga, ei pidanud 
selle liinid modemi kiirusele vastu ja soomlased pidid kinkima natuke 
aeglasema modemi, mis võttis ka madalamaid kiiruseid. Sedasi saadi Tallinnast 
UUCP-protokolliga side. 

Meie Tartus olime ka uljad: muretsesime 
endalegi modemi ja üritasime Tartust Helsingisse helistada. Aga Tartus ei 
olnud isegi mitte 1980. aasta olümpiamängude purjeregatti ja meie telefoniliinid 
ei pidanud aeglasele modemile vastu. Siis üritasime 
Tallinnasse helistada. 

Kuna me ei jõudnud kiireid modemeid osta, hankisime
mingi odava sisemise modemi. Aitäh Mati Kilbile\index[ppl]{Kilp, 
Mati}, kes oli tol ajal matemaatikateaduskonna\index{Tartu 
Ülikool!Matemaatikateaduskond} dekaan ja need pisikesed valuutasummad 
leidis. Sest kust sa hing kaheksakümnendatel ikka midagi ostad -- valuutapoest! 

Ühesõnaga, lõpuks käiski niimoodi, et meie helistasime Tallinnasse ja Tallinn helistas UUCPga 
Helsingisse. Alguses kaks, siis kolm korda nädalas, seejärel juba 
iga päev ja mitu korda päevas, kuna maht läks järjest 
suuremaks. Helsingist alates oli päris korralik internet olemas. 

\question{Mille külge keskkoolid tulid?} 

Vahepeal pandi meil tänu Rootsi Kuninglikule Akadeemiale ja 
Sorose Fondile\index{Sorose Fond} üles satelliitside. Nad said nii palju raha 
kokku, et hankida otsaseadmed -- kahes eksemplaris, sest Tartu-Tallinna vahet ei 
jõua ju keegi ära kakelda. Nii et Tartus tähetorni\index{Tartu tähetorn} otsas oli 
Tele-X satelliidi vastuvõtujaam ja teine satelliitseade oli Tallinnas Teaduste Akadeemia katusel (kuna Lippmaa 
instituut\index{Lippmaa Instituut|see{KBFI}} oli endiselt veel 
akadeemia raamatukogu all -- raamatukogu teisel ja nemad 
esimesel korrusel). 

Alguses orienteeriti vist valele satelliidile, meie seal kaasa ei mänginud, see 
oli Rootsi-poolse otsa tegevus. Tekkis küsimus, kes maksab kinni
järgmise inseneri, kes tuleb ja seab õige peale. Aga siis toimus
Rootsi kuninga visiit Tartusse\sidenote{Nende Majesteetide Rootsi kuninga Carl 
XVI Gustafi ja kuninganna Silvia riiklik visiit Eesti Vabariiki toimus 22.--24. 
aprillil 1992.} ja teenusepakkuja (ma ei mäleta, mis firma 
see oli) tahtis näidata, et nemad on kõikjal ja et Rootsi kuningas saab võtta 
telefonitoru Tartu Ülikooli rektori kabinetis ning ühenduda kohe oma koduga 
Rootsis. Seetõttu nad saatsidki mehe, kes pani side õige satelliidi peale. 
Nii et Rootsi kuninga visiidist alates oli meil 64kilone side üle satelliidi. 

Tallinna-Tartu side oli väga huvitav. Nagu me teame, on Tallinnast Tartusse 285 kilomeetrit, teistpidi on alati rohkem olnud. Aga kui 
nüüd vaadata, missuguse tee pidi läbima elektronkiri, et jõuda Tartust 
Tallinnasse, siis kõigepealt pidi see minema Tartusse tähetorni\index{Tartu tähetorn}, siis tähetorni 
satelliidi pealt Kuninglikku Rootsi Tehnikaülikooli Stockholmis ja sealt, olles 
avastanud ee-domeeni -- kas meil tol ajal oli .ee või tuli see natuke hiljem? 
Esimesed aadressid tulid muuseas .su lõpuga --, läks kiri üle satelliidi 
Tallinna Tehnikaülikooli ja siis kulges natuke veel Tallinnas. Arvutasin kunagi, et see tegi
kokku üle 70 000 kilomeetri\phantomsection\label{sisu!70k}. Kaugemal ei ole Tartu 
Tallinnast kunagi asetsenud! Õnneks läbiti see tee valguse kiirusel. 

\question{Kust tuli visioon, et internetti on üldse vaja?}

Olime tolleks ajaks UUCP-sidega juba imeasju teinud. 
Õpetasin internetti ka tudengitele, sest e-kirja teel oli 
võimalik igasuguseid asju saada, näiteks RFCsid, interneti 
alusdokumente. Tudengite 
arvestusülesanne oli mõni RFC kohale meelitada ja seega oli meil kettal 
peaaegu kogu interneti dokumentatsioon, üksikuid üritasime isegi 
välja trükkida, näiteks RFC 822, mis oli elektronkirjade aluseks. Neid oli väga huvitav uurida. Peale selle olime juba suure hulga ülemaailmsete 
listide liikmed. Informatsioon levis. Ja üleüldse oli internet tore asi. 

\question{Nii et saite maigu suhu?}

Jah, edasi said kõik aru, et meil on päris internetti ka vaja, kuigi 
veeb ei olnud veel sündinud. Meil oli Gopher.

Veeb mõeldi välja CERNis, kus
toodetakse palju artikleid ja need artiklid viitavad üksteisele. 
Artiklist arusaamiseks pead viidatavaid artikleid ka lugema ja 
igavene tüütus on käia neid kuskilt otsimas. Siis mõeldigi välja 
World Wide Web, kus on artikkel ja kohe ka pildid sees (see nõudis 
juba graafilist brauserit) ning viited niimoodi, et saad nendel klõpsida 
ja siis tuleb järgmine artikkel kohale. 

\question{Kuidas jõudis veeb Tartusse ja tuli 
mõte, et võiks hakata seda ka õpetama?} 

Need mõtted tulid peaaegu korraga. Kuidas veeb jõudis Eestisse, 
küsige Marek Tiitsu\index[ppl]{Tiits, Marek} käest. Tema töötas tol ajal Tartu 
Ülikooli raamatukogus\index{Tartu Ülikool!Raamatukogu}, kus oli kuskilt 
päranduseks või kingitusena saadud arvuti. Ma ei mäleta, milline, aga sellele oli võimalik veebiserver peale panna, sest, 
vabandage mind väga, veebiserverid töötavad kõiki Unixis. 

Eesti esimese veebiserveri pani püsti Marek Tiits\index[ppl]{Tiits, Marek} 
ja kursusel, mille nimetus oli vist juba tol ajal informaatika 
didaktika, üritasin ma lugeda igasuguseid uusi asju, kaasa arvatud internetti, sest ega mul internetikursust ei olnud. 
Kutsusin Marek Tiitsu tudengitele rääkima nii Gopherist kui 
ka veebist. 

Marek Tiits\index[ppl]{Tiits, Marek}, kes on praegu kindlasti üks paremaid 
lektoreid üldse, oli tollal teise kursuse tudeng. Kui ma pärast küsisin oma 
tudengitelt, kas kõik said aru, mida ta 
rääkis, siis tudengid vastasid: \enquote{Kui enne ei oleks midagi 
teadnud, siis vist ei oleks aru saanud, aga kuna teadsime enne ka midagi, 
siis saime teda üsna hästi jälgida.} Nii et esimese veebiserveri au on tõepoolest
Marek Tiitsul. 

Eesti Biokeskuses\index{Eesti Biokeskus} oli Sun SPARCStation\index{SPARC!SPARCStation}. 
Saatsin oma tudengid (mulle üldse meeldib tudengitega igasuguseid 
lollusi teha) hankima ülikooli pealt (igaüks sai ise teaduskonna) 
igasugust informatsiooni, mida õnnestub kätte saada: teaduskonna 
allüksusi, loetavaid aineid, mida iganes. Sellest nõelusime kokku
toreda ülikooliveebi. Seejärel jätkus meil nahaalsust kutsuda kohale 
rektor ja kaks prorektorit -- palusime neil istuda arvuti taha ja vaadata, kuidas 
Tartu Ülikool\index{Tartu Ülikool} veebis välja näeb. 

\question{Ja nägi välja küll!}

Nägi välja niisugune nagu Eesti metsad praegu: 
noorendik ja lageraie, siis mõni vana tükk ja nii edasi. See oli väga 
lapiline, tulenevalt sellest, kes kust mida kätte sai, ja tükati oli info kindlasti väga halvasti kajastatud. Kujunduse peale me muidugi 
väga palju auru ei raisanud (ikkagi matemaatikateaduskond!). Igal 
juhul sai rektor teada, mis asi on veeb, kuigi tol hetkel sai Eesti veebiserverid
näppudel üles lugeda. Nad võtsid asja üle ja hakkasid päris ülikooliveebi tegema. Rektorid on meil 
alati olnud suhteliselt taibukad, nii palju kui mina neid näinud olen. 

\question{Enne kui jutule joone alla tõmbame, on mul üks 
abstraktne küsimus. Ma ei tea, kas sellele ongi head vastust, aga kui keegi 
teab, siis tõenäoselt sina. Kui suur osa kõigest sellest, mis algas professor Kaasiku\index[ppl]{Kaasik, Ülo} pusimisest 1970ndatel ja on meid Liivi tänavalt
siia Deltasse toonud, on mäetipul kaugusse vaadates püsti pandud visioon ja 
kui palju sellest on \enquote{teeme järgmised kaks nädalat ägedaid 
asju}?} 

Osakaaludeks ma seda jagada ei jõua, aga näiteks internetikoolitus 
õpetajatele oli küll see \enquote{teeme kahe nädalaga kihvte asju}. Kuna Eestis 
arenes internet pärast siia jõudmist
penikoormasaabastega, siis otsustasime, et hoiame ka õpetajaid 
kursis. Tegime igasuguste koolituste käigus suures ringauditooriumis (tol 
ajal ei küsinud ülikool auditooriumi eest tasu, raha meil ei oleks 
olnud) infopäevi: rääkisime, mis on internet ja kuidas see on arenenud. 

Ühel hetkel tekkis Marek Tiitsul\index[ppl]{Tiits, Marek} 
europrojekt, mille käigus ta sai 100 modemit -- 50 oli tal projekti jaoks vaja, 
aga 50 võisime koolidele jagada. Me ei hakanud neid niisama loopima, vaid 
korraldasime süsadminnide või postmasterite kursused. Tegime 
kombineeritud kursusi, kus oli viis õpetajat rühmas: osadele
õpetasime hiire liigutamist, teistele programmeerimist, 
kolmandatele, kuidas modemit paika panna ja sinna teenuseid peale 
tõmmata, ja neljandatele, mis on veeb. Lisaks õpetasime 
süsadminne. 

Kursuseid saime teha kaks korda nii, et kõik 
kohaletulnud koolid, kes läbisid kahepäevase kursuse (nädalavahetusel, õpetajad koolitusid entusiasmist ja omal kulul), 
said kaasa modemi. Meil puudus kontroll, mis nendest pärast sai. 1995. 
aastal ei mahtunud õpetajad kuskile ära, isegi mitte Vanemuise 
suurde auditooriumisse\index{Tartu Ülikool!Vanemuise tänava 
õppehoone!Ringauditoorium}, ja meie võhm hakkas otsa 
saama. 

Käisin PTUIs, Pedagoogika Teadusliku Uurimise Instituudis, kes 
organiseeris õpetajate koolitusi, küsimas, kas nad meie kursustele ei tahaks 
raha anda. Rääkisin, et telekommunikatsioon tuleb kohe, ja 
ülemus, kelle nime ma kahjuks ei mäleta, vastas: \enquote{Misasi? Tele? 
Kommunikatsioon? See asi ei tule Eesti kooli mitte kunagi!} Panin suu kinni 
ja jätsin ütlemata, et 50 kooli on juba modemiga ühendatud. Sain aru, et sealt august raha ei tule, ja keerasin otsa 
ringi. (Hiljem andis meile natuke raha Sorose 
fond\index{Sorose Fond} ehk Avatud Eesti Fond\index{Avatud Eesti Fond}.)

Siis tegimegi ilma rahata. Keegi EENetist, Enok Sein\index[ppl]{Sein, 
Enok} vist, alguses ka Marek Tiits\index[ppl]{Tiits, Marek}, lisaks mu 
tudengid. Nagu ma ütlesin, siis Tartu Ülikool ei küsinud 
auditooriumide ja arvutiklasside kasutamise eest tasu, sest raha meil nagunii ei olnud. 
Kõik me tegime seda puhtast entusiasmist, samuti esimesed e-kursused, kui ei jõudnud enam suuri kahepäevaseid kursusi 
korraldada. Istusime Terje Tuisuga\index[ppl]{Tuisk, Terje} kahekesi 
koos ja mõtlesime, et teeks nüüd õige teisiti. Modemid on ju olemas, 
e-kirju nad saavad, üks inimene on koolis, kes oskab modemi käima panna. 
Korjame tema ümber viis õpetajat, teeme neile koolituse. 

Esimesel koolitusel unustasime piirarvu panemata. Andsime õpetajatele teada, et niisugune koolitus 
tuleb, ja just see inimene, kes modemiga hakkama saab, 
registreerigu oma kool. Arvasime, et kui viis kooli tuleb, on jube 
hästi. Panime fiktiivse meiliaadressi, kuhu nad pidid registreeruma. Meil oli nii kiire, et ei käinud seda vaatamas, ja ühel hetkel 
oli juba üle 20 kooli kirjas. Siis mõtlesime, et 
mis vahet seal on. Mõningad asjad tuli ära muuta: kui on viis 
kooli, igaühest viis inimest, siis nad võivad kõik meile oma elu esimese 
e-kirja saata ja vastame kõigile individuaalselt. Kui aga pärast oli kokku 50 kooli 
ja 400 osavõtjat, saime aru, et me ei jõua isiklikult igaühele 
kirjutada. Lasime neil omavahel suhelda -- panime nad 
paari niimoodi, et neil pidi vähemalt 50 kilomeetrit vahet olema ja nad pidid olema sama aine 
õpetajad.

\question{Täpselt programmeerija lähenemine 
ülesandele! Paneme paari, ei tohi olla üksteise lähedal ja peab olema sama 
aine!}

Tuleb loogiliselt mõelda, see tuleb elus õige mitmes 
kohas kasuks! 

\question{Nende kuldsete sõnadega võikski ehk lõpetada, aga mul on 
üks küsimus veel: millega professor Anne Villems praegu oma aega täidab?}

Esiteks tulin just USAst. Kahjuks minu tuttavad, 
kellega vanasti tihedalt läbi käisin ja kes olid Moskva erinevates 
instituutides ja ülikoolides, ei ole enam Moskvas, vaid Californias. Avastasin, et California külastamiseks on kõige meeldivam aeg 
jaanuari lõpp ja veebruari algus, mis sobib mulle, sest just siis on ülikoolis 
vaheaeg. Nüüd olengi vist neli aastat järjest käinud oma sõpradel Californias külas. 

Veetsin seal just kaks nädalat 
ja mind ära saates ütles sõbranna: \enquote{Nüüd sa oled aru saanud, et kaks 
nädalat on õige aeg, mitte üks nädal. Nii et tule järgmine kord ka kaheks 
nädalaks!} Seega talvel võib mind alati leida 
Californiast kevadisest Palo Altost, kus nii kohalikud kui ka 
sissetoodud taimed, nagu näiteks eukalüptipuud, õitsevad. 

\question{Mis on kõik hurmav lisaks sellele, et tegemist on Palo Altoga!}

Mis on Stanfordi kodulinn ja kus Palo Alto ja ookeani vahel on toredad 
mäed. Mägedes on tore käia. Üks sõber 
viis mu San Jose lähedal mäe otsa, kust oleksime pidanud nägema ühel pool 
San Franciscot ja teisel pool San Josed. Oli piimjas udu ja vihmapilved, ja siis korraks tuli tuul 
ning lõuna pool nägimegi seda vaadet, mida pidi nägema. Nii et jah, väga tore 
on reisida! Aga siin ma veel töötan, tunnitasu alusel, ja loen oma 
armastatud andmebaaside kursust. Kolmes versioonis.