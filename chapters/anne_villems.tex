\index[ppl]{Villems, Anne}
Kõneleja 4
Tere, see siin on memm kopi. Meie tänase külalise üle on mul kohutavalt hea meel. Ja mitte ainult seetõttu, et tema sisuline panus Eesti IT-maailma on sõna tõsises mõttes hoomamatu vaid seepärast, et tegemist on äärmiselt sooja ja toreda inimesega kellega oli lihtsalt niivõrd tore juttu rääkida. Külasena Anne Villems, head kuulamist. 

Kõneleja 3
Tohib sina öelda? Tohib küll. Mis on sinu auväärt nimi? 

Kõneleja 1
Minu nimi on Anne Villems. 

Kõneleja 3
Tere tulemast, oleme jälle kord kogunenud kaamerate ja mikrofonidega kaamerad, olles jälle maha unustanud Tartu linna suurepärasesse uude delta hoonesse. Ja on tõeline rõõm. Olen sinuga istuda ja juttu rääkida, sellepärast et sa vist oled see inimene, kes kõige rohkematest lugudest nagu läbi käinud tegelasena. Et alates sellest, kuidas korraldati Eesti esimene veeemmasterite koolitus, kui, kui lõpetades kõikide muude asjadega. Aga kõige selleni me võib-olla jõuame, kui hästi läheb. Kui hästi ei lähe, siis räägime muudest huvitavatest asjadest. Et aga nagu Mika pihta hakkame, hakkame algusest. Kuidas sina arvutite juurde said? 

Kõneleja 1
Mina käisin mari ülikoolis. 

Mari ülikooli nime rahva seas kandis tolleaegne Tartu 10. algkool mis on kohe Vanemuise kõrval vana Vanemuise vastas. Ja nimi tuli tal tema karismaatilisest matemaatikaõpetajast. Marvetist. 

Sellel, kes oli seal lõpare juhatajaks ja kui me selle kooli aastal 1960 kevadel ära lõpetasime oma pinginaabriga siis meie vanemad olid väga huvitatud meid panemast, viiendasse kooli. No mis tol ajal oli siis Rostov gümnaasiumis, vabandust Rostovi ülikoolis, mis oli siis esimene ülikool, mis naisterahvaid vastu võttis. Aga, aga seal asus parajasti viienda keskkooli õpperuumid. Aga meie ei tahtnud viiendasse kooli minna, ma ei tea, miks, aga meie otsustasime, meie otsustame. No ikkagi 14 aastat isiklikku vanust. Loomulikult tuleb ise otsustada, kuhu kooli sa lähed ja meie otsustasime, et meie lähme hoopiski ikka vanasse endisesse Treffneri gümnaasiumisse mis tol ajal kandis siis nimetust Hansen Tammsaare-nimelises nimeline keskkool. Sinna aga minu meelest tol ajal ei olnud nagu üldse seda probleemi, eks ole, pidi mingeid katseid tegema ja midagi. Aga ega meil katsetega ka ollakse üsna hästi läinud. Me mõlemad õppisime üsna korralikult. Ja, ja siis, kui me juba aasta otsa olime seal ära olnud, siis avati seal matemaatika klass, minu jaoks oli see kohe otsest sirge tee, sellepärast et matemaatika oligi mu lemmikaine. Ja koolis ja kõiges on süüdi muidugi Marisin, tähendab õpetaja Marvet, sellepärast et nii kihvt matemaatikatunde nagu tegin meil viiendast seitsmenda klassini. Ma tean pärast ainult siis, kui Olaf previds tuli meid õpetama keskkoolis, aga see oli juba mata klassis niimoodi ja, ja siis nii et kui me üheksandasse klassi läksime, sest noh, tolleaegne algkool lõppes seitsmenda klassiga. Kui me üheksandasse klassi läksime, siis mina läksin kõigepealt ja kahe kuu pärast tuli mu pinginaaber ka järgi, sellepärast et siis tuli töö, mis, mis õpetus. 

See ei olnud tööõpetus, see oli, see oli mingisugune noh, ühesõnaga praktiline õpetus keskkoolis. Ja siis ta vaatas, et, et see, mida neid nende klassile pakuti. Et see nagu mingit erilist pinget ei pakkunud, tuli ära ka meile, kuigi tema oli Homantitaarsete huvidega. 

Ja siis olime alles viis tundi matemaatikat minu suureks rõõmuks nädalas ja muude ainete seas ka sellised toredad ained nagu elektrotehnika ligikaudne arvutamine ja programmeerimine. 

Kõneleja 3
Aga mille peale seda programmeerimist õpetati tol ajal 

Kõneleja 1
Selleks ajaks oli Eestis olemas parajasti üks elektronarvuti üks üks ja mis oli kandis nimetust Uural. Talle ei olnud veel numbrit külge pandud number üks. Ja see oli siis Tartu ülikooli arvutuskeskuses, mis oli 59. aastal moodustatud masin ise asus peahoone kõrval Majas. Ja seal treffnerist jõmmid üldse väga kaugel, üks kvartal. Ja, ja õpetasid meid seal alguses Ülo Kaasik ja pärast Mati Krull, Matti Krull oli sinna tööle läinud, sellepärast et tema oli see esimene kursus ülikoolis, kellele Ülo Kaasik programmeerimist õpetas. 

Kõneleja 3
Nii et ühesõnaga Ülo Kaasik kuidagimoodi ilma arvutita 

Kõneleja 1
Ei? Arvutiga. 

Kõneleja 3
Juba Ülo Ülo Kaasik 

Kõneleja 1
Arvutiga sellepärast, et meie olime üheksandas klassis 62. aastal seesamune arvuti oli juba kaks ja pool aastat kohal. Kaks aastat. 

Kõneleja 3
Aga kellelgi pidi seal keskkoolis olema siis kui parasjagu nagu visiooni, et noh, et aru saada, et elektrotehnika ligikaudne arvutamine ja arvutid ja et see on nagu oluline millegipärast, kellele. 

Kõneleja 1
Ei, ma arvan, et see visioon võib-olla, et ei olnud isegi keskkoolil kuigi meil oli ka väga karismaatiline direktor omal ajal Allan Liim, aga tema oli ajaloolane. Ja, ja ma arvan, et see initsiatiiv tuli tegelikult Olav Priinitsa ja Ülo Kaasiku poolt et nemad organiseerisid selle, see ei oleks olnud nõukogude liidus erakordne. Et selle arvuti nõkku panid ja näos oli esimene kooli arvutuskeskus, siis see oli kogu liidus esmakordne juhus, aga matemaatikaklassid olid tol ajal juba olemas ja nad olid Moskvas olemas. Nii et nemad võisid öelda, et nemad jälgivad Moskva mallija, sellega keeruv keerulisi asju enam ei tulnud, igal juhul. Ma usun, et Treffneri kool Sonnil valitud võib-olla sellepärast, et Treffneri kool oli südalinnas, ei olnud vaja kuskilgi kuskile kaugele minna, sest oli selge, et vähemasti esimestel aastatel ülikooliga Beud hakkavad õpetama nii et meid õpeta soolast prionits, matemaatikat. Täiesti ma ütlesin, mul ongi olnud siuksed täiesti fantastilised matemaatikaõpetajad algkoolis õpetaja Marvet ja keskkoolis Olaf Ernits. 

Kõneleja 3
Aga kust kohast see matemaatiku või nagu arvuti võiks üle läheb, et matemaatika niuke abstraktne ja kaunis kunstidega arvutid, need on mingid tolmavad ja blogistonid lendavad. 

Kõneleja 2
Tolm tuleb pärast on tal. 

Ikkagi messi. 

Kõneleja 1
Kuidas tähendab nad muidugi omavahel saavad seotud olema oluliselt kõrgemal tasemel see tähendab seda, et sa pead ikka natuke matemaatikat enne oskama, enne kui sa aru saad, kuidas matemaatika-informaatika või noh, programmeerimisele on. Aga, aga enne pead natukene programmeerida oskama enne kui hakkad aru saama, et mõnikord on matemaatikat ka vaja. Näiteks alustades sellest, eks ole, algoritmiliselt The lahenduvaid ülesandeid eristada algoritmiliselt lahenduvatest ja need neid lahenduvaid jagada ka sellistesse klassidesse, mille lahendid ei tule mitte 50 aasta pärast või 50000 aasta pärast vaid millel lahendit tulevad arvutist noh, kui mitte homme hommikuks, siis vähemasti ülehomme. 

Kõneleja 3
Aga seda enam tekib küsimus, et matemaatikahuvilisi on inimesel ei ole tingimata nagu arvutihuvi. Miks sul oli? 

Kõneleja 1
Ma ei tea, kas kas see huvi tekkis, tähendab, alguses oli muidugi lihtsalt jube põnev noh, kogu Eestis ainult üks arvuti ja meile õpetatakse ja tuled vilguvad ja mingisuguse perfolindilt, mis on filmilint, filmilindilt loetakse, mitte see telegraafilint loetakse andmeid siis no lihtsalt jube põnev oli. Aga, aga seda muidugi, eks ole, et, et kuskil mõnikümmend või natuke enam aastaid edasi igalühel, meil on siin taskus laua peal ja, ja meie epsilon ümbruses. 

Tuhandeid kordi võimsamad arvutid kui see, mis meil seal terves suures saalis noh, ikka mitukümmend ruutmeetrit enda alla võttis. Seda muidugi tol ajal, eks ole, 60 62, kolm et ei näinud, 64, ma lõpetasin keskkooli. Ja õppima läksin muidugi ka puhtalt matemaatikat, sellepärast et vabandage mind väga, ega kuskilgi, võib-olla tippis, vabandust, tehnikaülikoolis mingisuguseid siukseid tehnikaaineid, kus ka arvutid otsapidi sisse tulid, võib-olla õpetati, aga ma nagu ei tahtnud. Tähendab, tehnika pool mind väga ei tõmmanud. Ja, aga programmeerimine iseenesest on niisugune maagiline tegevus, eks ole, et enne seda arvuti midagi ei oska, siis sa kirjutad talle siukse kihvti programmi ja siis ta järsku oskab midagi, näeb välja juba peaaegu et nagu nagu saaks millestki aru. 

Kõneleja 3
Ja huvitav on see, et Meelis Roosi esimene programm oli ka vestlemise programm, mis on just täpselt see asi, et sul jääb mulje, et arvuti saab millestki arvab, millestki ära jäänud. Just see on, nagu on, on läbiv joon. 

Kõneleja 1
Ja meile anti ka, meil oli, ma usun, üheksandas või kümnendas klassis me tegime oma elu esimese programmi. Minu, minu programm ei pakkunud mulle nii palju pinget, aga elu lõpuni ja ilmselt on meelde jäänud ühe mu koolikaaslasi ja ma ei mäleta, kes selle programmi tegime Open klassikokkutulekul küsima. Kus ülesanne seisnes selles, et talle tuli anda, et kuupäev ja siis ta pidi välja trükkima, mis nädalapäev see on. Aga nüüd arvestage eurole ühega, tema ainuke väljundseade oli kitsas printer, kus sai trükkida ainult arve. Ja siis siis nohtunud. Kui õppejõud siis proovis neid meie programme, siis ta kõigepealt andiski mingi mõistliku kuupäeva ette ja sai vastuse ja siis andis ette 30. veebruari mille peale programm hakkas siis printeri peal siukest nullide joru ülevalt alla trükkima. Mille peale õppejõud ütles, et nojah, ilmselt trükib mingit jama, eks ole, katkestama ära ja õpilane väga vaikselt ja tagasihoidlikult lasta natukene veeldatud. Ja siis siis ta trükiski välja ühe nullide joru ülevalt alla, siis ühe nullide, kas ka täidetud rea siis ühe tühja rea, siis natuke null, siis natuke nulle äärtes, siis natuke nulle keskel ja siis veel kord selle nullide joru ülevalt alla ja ühe terve nulli, tere ja, ja veel kord sama. No ja kui ma seal paberis kätte saime, siis oli kõigile näha looli. 

Nii et see, see oli asi, mis mulle meie elu esimestest programmidest kõige paremini meelde jäi. 

Kõneleja 3
Ja programmeerimine on inimese inimese tegevus. Aga mis te nagu lahe siukseid, nagu nuputamisülesandeid tegid tegitegi või mis programmiga? 

Kõneleja 1
Me ei, me tähendab, see oli üks huvitavamaid seal igasuguseid asju tegime mingite jalgade keskmise leidsime ja ja ma kõikide ülesandeid ei teagi seal, kus nad olid individuaalselt antud antud ülesanded ja niuksed niuksed tavaliselt midagi, praegused programm, algajad programmeerijad. 

Kõneleja 3
Kas kas see? 

Kõneleja 1
Selle jaoks on valem olemas, muuseas kuidas seda kuupäevast nädalapäeva teha? Kindlasti. 

Kõneleja 3
Kas sel ajal mingisugust nihukest nõu kogukonda ka nagu oli, nagu liidu peale selle arvutiasjade ümber või mingid olümpiaad juba sel sel teemal peeti või? 

Kõneleja 1
Ma ei tea, kuna kuna programmeerimist koolides õpetati ka äärmiselt vähestes, ma ei teagi, näiteks kas moskva matemaatika klassis arvuteid ka õpetati, ma ei tea seda. Ja kui ma Ülo Kaasiku käest kunagi ühe intervjuu käigus küsisin, et aga kust ta metoodika võttis meile programmeerimise õpetamiseks, siis ta vaatas mulle suurte silmadega otsa ja ütles, et aga metoodikat ei olnud mitte mingisugust. Vähe sellest, et ei olnud metoodikat, ei olnud ka kirjandust. Nii et ise hakkas siis raamatuid kirjutama ja ja, ja, ja see metoodika oli tal, ta ütles, et noh, kui nii-öelda käigu pealt välja 

Kõneleja 3
Ühesõnaga, tegelikult kogu see eesti asi käib Ülo Kaasiku peal. 

Kõneleja 1
Jah, see see asi käis ikka kindlasti tähendab tema ja tema õpilased sellepärast et lihtsalt edasi, jah, sellepärast et noh, pärast seda hakkas neid matemaatikaklasse tekkima ikka kuigi kaks aastat hiljem või kolm aastat hiljem. Sest ma mäletan, et OV karu, kes oli omaaegne näodirektor, tema istus meie matemaatik, ka lõpueksamitel. Et nii-öelda vaadata, kuidas, kuidas, kuidas me siis oleme matemaatikas arenenud. Ja siis siis siis ma mäletan isegi omaenese vastust. Kuna mul oli keeruline joonist teha, siis sellel seletamisel mul läksime puntrasse, siis ma astusin kaks sammu tahvlist eemale ja alustasin otsast pihta. No see on see tõestus, et sirge on ristitasapinnaga, kui ta on ristikahesel lõikuma sirgega, noh, see on keskkooliprogrammi oli keskkooli programm praegu, ei milleski kindel olla. Kui me oma projekti kooriga meil nimelt on arvutiteaduse instituudis projektikoor. Meil tuleb kooli laul meelde poolteist aastat enne laulu võidu ja siis me võtame oma koori kokku ja siis käime, valmistume nii hästi, kui suudame mõndagi ettelaulmiseks laulupidudel. Ja kui me laulupeol küsiti meie käest, et kuidas meie koori iseloomustada, siis igaüks pakkus midagi välja ja siis siis mina pakkusin. Meie koori iga liiget ja detaagorusse teoreemile ise erinevat tõestus, neid on umbes 200, meie kooriliikmeid oli kuskilgi suurusjärk 40 mille peale matemaatika õpetamisega tegelevad õppejõud meie koorist tuletasid mulle meelde, et aga Pythagorase teoreemi enam ei tõestata koolis. Nii et ma nüüd ei tea enam mitte midagi. Paraku. 

Kõneleja 3
Modempro huumoris, aga ühel hetkel sai keskkool otsa. Ja siis tuli ülikool, Tartu Ülikool, eks ole? 

Kõneleja 1
Jah, Tartu ülikool ja noh, vaadake aasta oli siis, eks ole, 1964. See tähendab väga sügav nõukogude aeg, mis välistas minu jaoks absoluutselt kõik humanitaaralad. 

Järgi jäi suhteliselt vähe järgi jäi meditsiin, mu isa oli kirurg ja noh, ju ta siis vaikselt ikka lootis, et äkki ma aga, aga kirurgias mul on selline loogiliselt mõtlev mälu mul on, mul on väga hea igasuguseid tõestusi meelde jätta ja nii edasi. Aga kui ma pean pähe õppima 2000 kontide nimetust, siis ja sinna juurde veel, eks ole, lihaste ja veresoonte ja jumal teab veel ajus agarate, millede ladinakeelsed nimed. Siis ma ei arva, et ma ennast väga hästi tunneksin ja peale selle me juba viiendasse keskkooli ei tahtnud selle pärast minna, et vanematel olid seal liiga head suhted. Tahtsime ikka ise olla. No siis koolis ma ka tahtsin ülikoolis olla rohkem ise kui keegi muu. Ja siis jäigi järgi matemaatika ja matemaatikat. Ma armastasin koledasti. No pange tähele, mul olid, eks ole, viis, kuus, seitse head õppejõud, ma ei saa midagi paha öelda ka oma kaheksanda klassiõpetaja kohta, kes mind soovitas, mata klassi ja noh, siis oli nii-öelda eesti matemaatika koolis õpetamise korüfeed olla. Priinits oli siis mu küll tundmumale siiamaani mäletan funktsionaalse seose selgitust siiamaani, noh, see on siis, eks ole. Ligi 50 aastat pärast seda, kui ma seda õppisin. 

Kõneleja 3
No aga siis oli hästi-hästi tehtud. 

Pidigi olema seal Tartu Ülikoolis toona. 

Matemaatikul ikkagi nagu õpetati lõpuks ka programmeerimist. 

Kõneleja 1
Jah, meil oli kaks kursust programmeerimist, üks oli masinkoodis, programmeerimine, uuror nelja peale selle ma läksin ja tegin nii-öelda kohe septembrikuu jooksul mitte ära mille peale õppejõud võttis mu vastusse ja ütles, et lõhnab natuke Uural õe järgi. Aga programmeerida te oskate ja pani mulle arvestuse. Ja teine oli siis Ülo Kaasik kuu algul 60 õpetus ja vot seda ta õpetas küll pliiatsi ja paberiga. Sellepärast ühtki translaatoreid tol ajal algollist ei olnud. 

Kõneleja 3
Aga miks ta Algurit just õpetas? 

Kõneleja 1
Vaadakem, eks ole, aasta siis oli 66, seitse, mis keeled siis seal üldse olemas olid? 

Jumalale tänu, et keegi jaganud meile Koboliit õpetama, see oleks mind küll programmeerimisest viie kilomeetri kaugusele peletanud. Olete te proovinud kunagi Koboliit lugeda või? Ei ole? Ärge vaadake ka see on niisugune, eks ole, kupp, programmeerimine on selline kontsentreeritud väljendus, eks ole, sa saad valemeid kirjutada valemitena ja, ja siis sul on mõningad sellised kenad koodsõnad, foor ja too ja if-then else ja nii edasi. Siis pange nüüd sinna mingisugune filoloogide soust peale, kust te teate isegi, et palk pluss preemia välja kirjutada kolmesõnana? Vot niisugune programmeerimiskeel, noh. Ja ma tänan jumalat, et ma mitte kunagi ei ole pidanud programmeerima koolis oli vahepeal väga elujõuline. 

Kõneleja 3
Ja siiamaani on inimesed ju tegelevad sellega testi aktiivsed siiamaani ja mis on huvitav, selle jutu juures on see, et et siin on ikkagi nagu ekspetsiitne, nagu programmeerimisõpetus käis kuskil nagu keegi metoodiliselt õpetas nagu programmi kirjutama, siis on üsna haruldane. Sellest tol ajal kindlasti jah, aga, aga nendest lugudest nagu räägitakse üldist harva reegli inimesed ütlevad, et programmeerimine kuidagi jäi külge. Ja nad ei oska öelda, kust või kes õpetas. Aga, aga. 

Kõneleja 1
Ja ei, kindlasti on võimalik programmeerimist mu kui tahtmine on väga suur, siis kindlasti on võimalik programmeerimist iseseisvalt õppida. Aga no motivatsioon peab ikka niimoodi seinakõrgune olema. 

Kõneleja 3
Ja kas tol ajal oli, mis see nagu arvutite niisugune perspektiiv oli, milleks neid kasutati? Programmeerime, programmeerime, aga kas see on nagu teaduslik töövahend või mingi rahvamajandust aitab kaasa? 

Kõneleja 1
Ja tähendab, rahvamajanduses kasutus oli juba uurali ajal kuna mälu oli väga väike. Ärgem ära unustagem, et Uural ühe mälu, mida keegi ei usu, oli kaks kilo, aga noh, õnneks mitte baitisest baiti tol ajal ei tuntud, vaid sõna niimoodi kaks kilo sõna. Võtke oma telefoni, vaadake siis nende mega või gigabaitide arvu. Mis teil taskus on. Siis väga suur probleem oli see, et kuidas neid andmeid, mille pealt midagi pidi arvutama, tatama kuidas neid sinna arvutisse ära mahutada, välisseadmetega olid ka veel omad probleemid. Aga kas need pakiti niimodi? Ja, ja, ja ma mäletan, et kui ma ise programmeerimist õpetasin masinad voodis programmeerimist. Sellepärast et ma miskipärast läksin Mäele viimase kursuse viimasel semestril 70. aastal. Ja siis järsku nagu langes ära minu Tallinnasse tööleminek, sest mu mees teatas surmkindlalt, et tema küll Tallinnas ei kavatse minna. Noh ja siiamaani ei ole läinud. Ja, ja siis siis ma pidin ruttu Tartus endale töökoha otsima. Ja siis siis Ülo Kaasik, kui oli siis see kes ütles, et jah, ma usun küll, et ta võib tudengite ette saada, ta, tal jalad ei päris. 

Siis mind saadetigi niimoodi, pärast lõpetamist lõpetasin juunis vara ja siis septembris läksin siis. Tudengitele jah, põhiliselt programmeerimist õpetama. 

Kõneleja 3
Aga kas mingisugust teadustööd ei sirgunud sealt või poosist, õppejõu töö? 

Kõneleja 1
Teadustegevusega ma hakkasin tegelema hoopis palju hiljem sellepärast et kui meil praegu räägid, Öeldakse, et mis asi on õpetaja koolis norm töökoormuste teatubjastid ja 24 28 tundi siis algajate õppejõudude, nii-öelda jalul, seismise, auditooriumi ees seismise koormus. Vähemasti see, kus, kui mina õpetasin, oli 24 või 28 tundi nädalas noh, sealt kõrvalt vedama. Ja eriti kui ta esimesi aastaid õpetajatega ja, ja peale selle õpetajate kahes keeles selle pärast, et ega meil siis matemaatika poole peal neid vene keelt rääkivaid inimesi oli väga vähe. 

Kõneleja 3
Ja olidki kaks eraldi grupil eestlane. 

Kõneleja 1
Jah ja majandusteaduskonnas ma õpetasin elu esimese loengu, ma lugesin vene keeles majandusteaduskonna, ma ei tea, kas kaugõppijatele või vene keelt, ma oskasin väga nirult. Sellepärast et esiteks ei olnud seda vaja. Ja teiseks ma olin küll lugenud väga palju matemaatikaalaseid õpikuid, aga noh, mis ma võin neid tõestusi lugeda prantsuse keeles, mida ma ei oska. Sellepärast et seal on nagu neid vahesõnu on suhteliselt vähe. Aga, aga selleks, et õpetada, selleks on vaja palju sõnu. Aga siis mul õnneks jätkus taipu. Vaadake, meie venekeelsetes rühmades oli alati sees kakskeelseid inimesi. Ja siis mul jätkus teha nendega tudengitega, keda ma õpetasin nendega niisugused kokkulepped. Esiteks, kui ma ütlen midagi sellist, millest ei ole võimalik hea tahtmise juures aru saada, siis esimene rida ütleb mulle. No ja siis ma püüan ümber sõnastada niimoodi, et arvuga oleks võimalik saada. Ja KuMul sõna meelde ei tule, siis ma ütlen selle sõna eesti keeles ja need kakskeelsed ütlevad mulle vastava venekeelse termini. Minu vene keele mitteoskust iseloomustab võib-olla see, et Ma näiteks ei teadnud, kuidas on. 

Siis ma küll üritasin kasutada sõna Ati jäätis või noh, mis ei ole ka väga vale, aga aga Võõtšest sõna ma ei teadnud. 

Kõneleja 3
Ja see on ju pedagoogilise metoodika, mõtlesin õudselt hea kooli. 

Kõneleja 1
Jah igal juhul kuskilgi, umbes seitse või kaheksa aastat pärast seda, ma sattusin Moskvas, mind ei olnud Moskvas kuskilgi ääretul Nõukogude Kodumaal. Sattusin kuskilgi konverentsile, kus ma siis midagi seltskonnas ütlesin ja mille peale minu käest küsiti, et võismask. Noh, mille peale ma siis ütlesin, et kui ma teist korda veel suu lahti teen, siis teist, siis te saate kohe aru, et ma ei ole ei Moskvast ega Leningradist. 

Kõneleja 3
Ja selline õppejõu töö siis Tartu Ülikoolis läkski edasi. Kuni saabusid aastat 20. 

Kõneleja 1
Ei vist vist veel edasi. 

Kõneleja 3
Mis maa aga, aga lihtsalt aastast 80 algab see kuu, kus asjad lähevad, arvutid lähevad väiksemaks. 

Kõneleja 1
Arvutused lähevad väiksemaks ja noh, see aeg, kui meil ei olnud ohja sõjaarvutite saamine seal vahepeal ka 80.-te oli oli, oli ikka omaette tsirkus. 

Kõneleja 3
Nojaa vot Jaan, Tallinn on rääkinud, et tema tõi oma esimesed arvutid käsipagasis laevaga Rootsist. 

Kõneleja 1
Minul ei õnnestunud, väga rootsi, ai, ma käisin küll, aga siis siis siis manirikas seal Rootsis küll ei olnud, mul abikaasa oli Bostokis Uppsalas ja mina sealt arvud, et ei toonud, kuigi ma tean küll, eks ole, et kui oleks seal ostnud arvuti siis siin maha müünud, siis oleks selle eest väga palju muid asju saanud, aga ei, mul siukest ärivaimu ei olnud. 

Kõneleja 3
Aga Tartu Ülikoolis hakkas kuskil millal hakkas Tartu Ülikoolis arvutine tavaliseks asjaks muutuma, kus ta nagu noh, oli nagu selline igav. 

Kõneleja 1
Mul on üks selline murdepunkt, oligi 82 82 juhtus selline asi, et Tallinna sõbrad organiseerisid näituse välisnäituse, kust tulid vist 100 firmat, et sinna Tallinna näituseväljakule. Aga, aga miks minu nii-öelda tähelepanu köitis, sest noh, seal oli neid näitusi enne ka olnud. Sealt pidi tulema kohale kellegi sõber Suurbritanniast, kes oli hakanud Apple'i diileriks Apple'i diileri, kes oli ta hakanud sellepärast, et ta pidi ehitama ühe ühe seadme ülikõrgete rõhkude jaoks. Aga seda seadet oli vaja juhtida ja siis juhtimiseks tavalis välja Apple'i ja kõige odavam Apple'i saamise viis oli see, ta hakkas Apple'i diileriks niimoodi ja nüüd ta siis tuli siia näitusele, ma ei tea, mida ta muud siiani tõi, aga igal juhul Teaduste Akadeemia instituudid ja, ja pooled mulle head tuttavad inimesed hakkasid ette valmistama, et tema käest Apple'i arvuteid osta. No kuulge, kellele on, milleks näiteks üks kadunud professor Lippmaale arvud? Minul on arvutit vaja, eks ole, mitte ma arvan, et mul on arvutit rohkem vaja, kui tal on mingisugused oma magnetresonantsid, mida ta ka peab juhtima ja, ja aga, aga, aga meie õpetama tulevasi programmeerijaid välja ja ja, ja meie kasutame input, output, kapi, kui keegi teab, mis asi see on. Kuhu tudeng paneb oma perforeerimiseks oma Blaketile kirjutatud programmi ja siis kolme nelja päeva pärast saab sealt tagasi, süntaks vigadega. Tema väljakäsitluses esimeste süntaksi vigadega ja aga aga mitme eaga, kui ta neid mitu tükki tegi. Ja mul läks hammas koledasti verele, sõbrad saavad miskipärast arvuteid, aga tegelikult on neid mulle hooaja. Ja siis ma läksin ja rääkisin oma sõpradega. No ja siis siis tuli välja, et sõbrad on head sõbrad. Näiteks selleks, et üldse aru saada, millestki pill koosneb, sest noh, mul ei olnud teda vaja millegi juhtimiseks, mul vaja siukest alasti arvutit. Ja siis ma istusin. Siis ma istusin Tõraveres. 

Sest seal käisid baidid, baitidest sai, sai siis teada, mis on arvutil sees ja mis talle külge käib. Ajakiri balletiga, mõistusin pühapäevade kaupa seal ja panin siis oma konfiguratsiooni kokku ja sain aru, et mul ongi vaja alati arvutit. Ja siis, kui ma oma konfiguratsiooni siis siis Tallinna sõpradega siis Lippmaa instituudist siis pidasin aru, mis on mõistlik summa, mida plaanikomiteest küsida. Ja siis ma leidsin, et noh 10000, kuldrubla rubla ei toiminud välisturul ainult ainult kuldrubla 10000 kuldrubla oleks siis piir, mida võiks, on nii pisike summa, et keegi võib-olla et äkki annabki selle. 

Arvuti kolm arvutit saime selle eest, sest mul oli vaja paljaid arvuteid, ma isegi monitore ei ostnud, igaks juhuks, et mine sa isahane tea. Võib-olla meie nõukogud televiisorite äkki ei töötagi? Oota, ütles ja et Apple kahed. Ja siis Plaanikomitees käis minu eest Lippmaa Instituudi sõbrad. 

Tõnu Karu ja vahepeal oli välisministriks mul nimedega raskusi. 

Kõneleja 3
Lippmaa instituut on meil siis küberneetika, istud. 

Kõneleja 1
Meil ei ole ka PÖFF-i 

Seda nimetati kogu aeg Lippmaa instituudis ja tol ajal oli ta veel seal Teaduste Akadeemia raamatukogu all. Estonia puiesteel. 

Ja siis Sinijärv Riivo Sinijärv oli või need, kes minu meelest plaanikomitees käisid paberitega, mina ajasin kõik paberid korda, korjasin ülikoolist kõik allkirjad kuni rektori nii peale. Ja siis nemad käisid nendega ja antigi 10000 me saimegi oma. Ja siis me tegime nendest arvutiklassi kolmest arvutist ja panime programmeerimise õpetamise individuaalgraafikus praktikum idega. See oli juba väga tore aeg, sellepärast et inimesed olid harjunud input, output kapiga, eks ole, et annad sisse, siis unustad kõik ära, mis sa sinna kirjutasid, siis saad kolme või nelja päeva pärast siis mingi paberirulli tagasi oma oma süntaksivigade sisse, paratme, parandad neid süntaks vigu ja, ja nii edasi. Noh, selle pisikese elu esimese või teise või kolmanda programmi silumine võttis niimoodi ikka õige mitu nädalat aega. 

Kõneleja 3
Aga see, see tähendab siis seda, et tegelikult see nii-öelda tarkvara tehniline pool muutus nagu radikaalselt 

Kõneleja 1
Kui ma nimelt aegajalt see oli kevadsemester, eks ole, Me arvutit saime kätte kuskilgi jaanuaris veebruari algusest, panime siis algul programmeerimise algõpetuse, kuigi meil Peisiku keeli ei meeldinud ja aeg kell kaks on sündinud Peisikuga siis siis mõtlesime, et noh, me suudame need Peisiku hädad vast ehk kompenseerida oma hea õpetamisega. Et alg algesimese programmi jaoks ta kõlbab. Ja siis siis kevadel paistab Liivi tänaval paistis päike sinna klassi. Ja mina siis seisin, eks ole, nende kolme arvuti selja taga, et tudengid aidata. Ja siis tegelikult ma ei näinud, mis nad sinna ekraani peale kirjutasid, vaid ma nägin nende enda peegeldust. Ja see, kuidas tudeng sisestas oma programmi pani käima, sai sealt ise oma süntaks, vead kohe kätte. Noh, see miimika, eriti tütarlaste oma, see oli ikka ikka tasus vaatamist, et aru saada seda, et me oleme kuskil ligi suunas õige õige sammu astunud. 

Kõneleja 3
Suur tänu. Kas see metoodika ka pidi sisu muutuma, kuidas seda asja õpetati? 

Kõneleja 1
Ja koos keelega muutub alati sellepärast, et küsimus on alati võsa põhi. Põhikisma, mida algõpetuses arutatakse, on see, et missugust programmeerimiskeelt õpetada esimesena sest noh, sealt jäävad asjad külge. Ja pisiku häda on muidugi see, et kui ma olin valmis kirjutanud nende arvutitega on need, olid küll macho arvutid juba aga me kasutasime neid ka laialdasemalt arvutite tutvustamiseks. Ja siis kui ma olin Peisikus kirjutanud valmisprogrammi, mille väljatrükk oli umbes minu enda pikkune siis ma vandusin, et see on minu viimane programm, Peisikus, mina rohkem Peisikuse kirjutab esikus nimelt ei funktsioon ei alamprogramme. Ja, ja siiamaani ma olen seda vannet pidanud, ma ei ole Peisikus rohkem programme kirjutanud. Ei ole. 

Nii et tükk aega me muidugi, kuna mulle õpetati kõigepealt masinkoodi, vabandage väga uurali arvuti peal, mingit mingit kõrge taseme keelt ei olnud ja assembleriga ei olnud. Ja mulle tundus see nii, noh siis ma saan ju aru, mängin otse registritega ja ma saangi aru, mis, mis masinas toimub ja mida see aritmeetiline plokk seal teeb ja nii edasi ja sealt edasi on juba kõik väga lihtne. Siis siis üks muu Novosibirski nüüdseks juba kadunud sõber kunagi kumas võitlesin selle eest nii-öelda ühes üleliidulises seminaris, et ikkagi masinkoodist tuleb alustada. Siis ta minu seisukoha ilmestamiseks rääkis anekdoodi. Toot oli sihukene, et vot vaene mees läheb kirikuõpetaja juurde ja ütleb, et kole raske on elu naine ja lapsed ja ämm ja ja, ja, ja kõik peame elama, eks ole, oma onni ühes toas ja see on see, see on ikka väga raske. Ja siis kirikuõpetaja küsib, et aga kas sul kits on ja kits mul on? Ole hea, võta kits ka tuppa juurde. Ja. 

Milleks veel kits, aga kirikuõpetaja käskis, talumees võtab siis ka kitsetuppa ja, ja ütleb, et vot nüüd nädala pärast nüüd piigits välja. Ja, ja siis, ja siis siis tule räägi minuga. Nädala pärast viib mees kitse välja ja tuleb, räägib, et vot nüüd on küll juba väga hästi kitse ei ole ja oma naise ja ämma ja lasteaia kõikidega ma saan nüüd juba palju paremini hakkama. Must ja, ja siis sellest masinkoodist alustamine on nagu siis kitsetoomine sinna tuppa, et ja kui lõpuks siis saab hakata programmi kirjutama fooriaiitsia Helsiga. Et noh, siis on suur lõõgastus, et noh, enam ei pea, eks ole, tõstma midagi registrisse ja kontrollima ja andma suunamist ja nii edasi. 

Kõneleja 3
Kui ma nüüd seda juttu kuulama, siis linn on toimunud üks mingisugune kuskil jutu sees mingisugune nihe toimunud. Alustasime sellest, et meelisprogrammeerimine siis nüüd ühel hetkel oli, andis selle nii-öelda positiivse emotsiooni, see näoilme muutus selle ekraani peegelduse peale. Mis hetkest see nagu programmeerimise õpetamine muutus nagu huvitavamaks või nagu põnevamaks kui programmeerimine. Kui üldse niisugune hetk. 

Kõneleja 1
Jah, ei, ma usun küll. 

Selleks, et midagi väga tähelepanu väärset programmeerimises ära teha. Selleks on vaja head meeskonda. Ja ma arvan, et parim siis nii-öelda suur asi, mida ma programmeerimises olen teinud. Me Ain Isotammega sisuliselt kahekesi tegime omal ajal suure süsteemi villis, mis oli aruannete generaator ja millel olid ikka väga tähelepanuväärsed omadused ja minu elu kõige keerulisem programm, mis tegeles siis magnetlindi ja printeri juhtimisega aruannete väljatrükkimise ajal, kui aruanded on pandud segamini magnetlindi peale küll tekitamise järjekorras aga, aga 15 aruannet korraga niimoodi Juppé vaheldumisi ja siis printerist tulevad kõik aruanded õiges järjekorras ja, ja mul oli kaks katkestuste allikad ja siis tasakaalu hoidmine, printeri, puhvrite ja magnetlindi puhvrite vahel teisendamise töö seal vahel, et et andmetest teksti tekitada, see, see oli köömes. See on vast jah, mu kõige-kõige keerulisem programm, mille käigus muuseas on iga programmeerija unistus avastada arvutit Diviga. Teile kogu aeg tundub, eks ole, et arvuti teeb valesti, sellepärast teie teete kõik õigesti, aga arvuti ju eksi. Ja siis te lähete seda inseneridele rääkima. Ja kui te hakkate punainseneri, tea teie seda epsilon ümbrust seal või kontekstis ja Stakate inseneridele seletama seda asja algusest piir peale. Vot niimoodi niimoodi niimoodi niimoodi kuskil poole peal te saate aru, millal te ise olete pea teinud, haarata kõik oma väljatrükid ja ütlete insenerile, kes sinnamaani ei ole veel mitte millestki aru saanud. Ööd ütlete talle aitäh posti mängimise eest ja siis lähete oma viga parandama. Aga, aga selles programmis mul õnnestus leida arvutiga. Sellepärast et Sünkro impulss traatprinterile oli halvasti joodetud. Ja see tähendab seda, et kõik inimesed trükkisid sümbol haaval. Või siis rida, eks ole, äärmisel äärmisel juhul terve rea. Mina tahtsin, tegin lehekülje valmis ja tahtsin tervet lehekülge. Trükib mulle pool lehekülge, veerand lehekülge ja edasi lihtsalt ei trüki. Lähen kontrollin seda, printeri juht käsku, kõik on õige, kõik vaga ei trüki. Noh, ja siis insenerid avastas, see on minu ainus kord minu programmeerija karjääris, kui mul õnnestunud arvuti viga avastada. 

Kõneleja 2
Kusjuures mitte arvuti kui tüübi viga suur all nagu üldiselt, nagu teeb see konkreetne tükk spetsiifilise viisil. Lihtsalt konkreetne printer, jah, aga noh, see on ikkagi riist ikka inseneride pärusmaa, mitte mitte mitte programmeerija oma. Kuidas, kuidas õpetamise värk huvitavamaks läks? 

Kõneleja 1
Õpetamise värk on huvitav olnud kogu aeg sest ega ma siis ei oleks kaua seal selle koha peal, noh, Tallinnas oli palju keskusi, Tartus väga nagu arvutuskeskusi ei olnud väga Tartus väga vähestel asutustel olid, aga ega mind väga ei tõmmanud ka see, see tolleaegne noh, ma ei tea, ASC tegemine automatiseeritud juhtimissüsteemide tegemine, see, see, see nõukogude tehnika töökindlus oli ikka niivõrd vilets. Et isegi Nõukogude esimesed üle Kahaldatuse ameeriklastelt üle kavaldatud personaalarvutid ei kippunud hästi töötama. 

Nii et, aga õpetamine oli ju palju toredam. Seal õpetas isa ja, ja peale selle noh, kuni siiamaani eks ole, räägitakse meile, kuidas Eestis on veel ikka 8000. IT-spetsialisti puudu. 

Kõneleja 3
Need maagilised žürii kiri 8000. 

Kõneleja 1
Jah, ma usun küll ja see on enam-vähem konstantne suurus, see on seisnud juba niimoodi väga-väga palju aastaid. Noh, põhiline ei ole võib-olla see number ise 8000, vaid see Neid on puudu ja see, et neid on puudu. Seda me näeme kogu aeg oma teise kursuse peal. Nimelt inimesed omandavad esimese kursuse programmeerimise algõpetuse ja siis teevad oma esimese projekti hoopis tähendab objektorienteeritud programmeerimisest. Ja siis on ta kasulik firmale. Ja ülikool lükatakse teiseks plaaniks, noh ja kes siis peab välja, kes ei vea välja, enamus ei pea, see tähendab, ülikool jääb kõrvale. 

Kõneleja 3
Selle kohta on mul kaks küsimust, enne kui me jõuame selle küsimuse juurde, et kuidas arvuti siia jõudis. Aga mul on see küsimus, et. 

See õpetada välja arvutiõpetajaid. Kustkohast see idee tuli, sest see tahab ka ikka nagu ägedat visiooni saada, et just see, et et keskkoolis peaks nagu või üldse koolis peaks nagu arvutiõpetust õpetama, kust. 

Kõneleja 1
Sellel on nagu siuksed, kaks juurikad, üks üürikas on muidugi see, eks ole, et arvutid hakkasid jõudma ka koolidesse mitte ainult, eks ole matta klassidesse vaid ka. Sest tekkisid lihtsalt sellised arvutid, mida Uural ühte ei jõudnud keegi ükski kool osta, see on ka Eesti selge ja tal ei olnud peale omatud kuskil. Füüsiliselt ta võimlemissaali oleks mahtunud, aga aga seda ei saanud ka koolist arvuti alla ära võtta. Nii et noh, need õpetajad, kes läksid koolis arvutit õpetama, neid oli vaja välja töötada, välja õpetada. Teine oli muidugi arvutiside, sellepärast arvutiside hiilis koolides tagauksest. See ei olnud mingisugune niisugune riiklik programm Komil pärast tuli, eks ole, Tiigrihüppe ja igasugused asjad siis esimesed, ma usun, ligi 100 kooli. 

Arvutis sidega ühendatud noh ise küsimus, mismoodi ühendatud, nii et paljud ülemused ei teadnud üldse, et on. 

Osa nendest läks modemiga. 

Telefoniside kaudu. 

Alustame algusest. Tähendab, et, et aasta siis oli. Oh, anna mul mälu. 

80 alguses. Kõigepealt muidugi olid Fidu vennad. 

Ja ma arvan, et nende tegevusest ei teadnud mina suurt midagi, sest sest amatööride ridadesse ma ei kuulunud. Ja, ja seda ma kuulsin pärast. Fido vendadel oli modemside juba enne kui arvutis jõudsid. 

UDP-ga. 

Aga aga kui, kui siis arvutivõrgus juba midagi kuulda oli ja kaheksakümnendatel juba oli midagi kuulda siis siis ka meie, Soome sõbrad otsustasid Eesti Nõukogude Eesti järgi aidata. Ja noh, muidugi Tallinn on Helsingile palju lähemal kui. 

Tallinna tehnikaülikool sai siis soomlaste käest modemi, aga ja kuigi meil oli uuendatud telefoniside seoses 80. aasta olümpia, aga siis ei pidanud selle modemi kiirus. Ja ma nüüd ei nimeta numbreid, ma neid enam ei mäleta, ma usun, et 19000 millegiga bitt biti mitmes hakanud oli see mõõduühik see vist ei pidanud vastu. Siis pidid soomlased kinkima neile natukene aeglasema modemi, mis võttis ka madalamaid kiirused said Tallinnast uudsetes juuniks, juuniks. Me käisime protokolliga, side, meie tartus olime ka muidugi muljed muretsesime endale ka muude selle modemi üritasime Tartust Helsingisse helistada, aga Tartus ei olnud 80. aasta olümpiamängude jah, isegi mitte purjeregatti. Ja, ja meie telefoniliinid ei pidanud kaeglasse moode meid vastu. Siis me võtsime kätte ja üritasime Tallinnasse helistada. Ja kuna me ei jõudnud neid kiireid moslemeid osta, siis me ostsime mingi sisemise odava. Ja aitäh Mati kilbile, kes tol ajal oli matemaatikateaduskonna. 

Dekaan ja need mingid pisikesed valuutasummad, sest noh, kust sai kaheksakümnendatel ikka midagi, ostad valuutapoest ja muretses meile modemi. Ja siis siis meil käiski niimoodi, et meie helistasime Tallinnasse, Tallinn uudsopeega helistas, noh, alguses kaks korda nädalas, siis kolm korda nädalas, siis iga päev siis ma ei tea, mitu korda päevas, kuna mahud läksid järjest suuremaks. Helsingisse. Helsingist alates oli juba päris päris korralik internet olemas. 

Kõneleja 3
Aga mille külge need keskpolitseist tulid? 

Kõneleja 1
Tähendab, vahepeal toimus see Meil pandi ju satelliitside ülesse ja see on siis tänu Rootsi kuninglikule akadeemia suurusse fondil ja nad said nii palju siis raha kokku, et need otsa otsa seadmed kahes eksemplaris, sest noh, Tartu-Tallinna vahet ei jõua ju keegi ära kakelda. Nii et Tartus Tähetorni otsas oli siis Teleiks satelliidi, see vastuvõtu ja, ja Tallinnas konna Lippmaa instituut oli ikka endiselt veel Teaduste Akadeemia raamatukogu all sisuliselt raamatukogule teisel korrusel, nemad esimesel siis sinna katusele sai siis teine satelliitseade. 

Orientiiriti alguses vist valele satelliidile, meie seal kaasa ei mänginud, see oli puhtalt rootsipoolse otsa tegevus. Siis tekkis küsimus, et kes selle järgmise inseneri kinni maksab, kes tuleb ja õige peale seab. Aga siis tuli Rootsi kuninga visiit Tartusse. Ja loomulikult. Ma isegi ei mäleta, mis firma see oli, tahtis ta näidata, et nemad on kõikjal. Et Rootsi kuningas saab võtta telefonitoru rektori kabinetis Tartu Ülikooli rektori kabinetis ja ühenduda kohe oma koduga. 

Rootsis ja siis nad saatsid selle meheks õige satelliidi peale pani, nii et Rootsi kuninga visiidist alates on siis meil oli meil siis 64 kiloside satelliit üle satelliidi. Sel hetkel oli muidugi Tallinn-Tartu side väga huvitav. Tallinnast Tartusse, nagu me teame, on 285 kilomeetrit. Kui nüüd teistpidi on alati rohkem olnud. Aga kui nüüd vaadata, missuguse tee pidi läbima elektronkiri selleks et jõudu Tartust Tallinnasse siis ta kõigepealt pidin minema, eks ole, noh, see Tartusse jõnks kuni sinna tähetorni siis tähetorni, sellest satelliidi pealt Kotefoossete tähendab kuningliku Rootsi tehnikaülikoolist Stockholmis sealt olles avastanud siis eelnumbriga, kas meil tol ajal oli, see tuli natuke hiljem? Meil esimesed aadressi tulid suuga muuseas. Ja siis siis sealt Kothoost üle satelliidi Tallinna tehnikaülikooli ja siis natuke veel Tallinnas. Nii et see kõik oli vist mu, ma kunagi vist arvutasin kokku üle 70000 kilomeetri kaugemale ei ole Tartu Tallinnast kunagi asetanud. Aga see läbiti õnneks valguse kiirusel. 

Kõneleja 3
Visioon tuli, et siukest asja üldse nagu vaja on, et milleks internet hea oli. 

Kõneleja 1
No tolleks ajaks me olime selle selle uudsepe öös sibisidega juba juba imeasju teinud. Ma näiteks õpetasin tollel hetkel internet tudengitele tudengitega, vaadake e-kirja teel oli võimalik igasuguseid asju saada. ECB Sid RF Tseesid rikast fokument, mis on, eksole internetti, dokumentide alusdokumendid, Internetidokumentatsiooni alusdokumendid, neid oli võimalik tellida e-kirjaga, siis me tudengitega tudengite arvestusülesanne oli mõni FC kohale meelitada. Ja siis meil oli ketta peal peaaegu et kogu internetidokumentatsioon ja üksikud neist isegi üritasime välja trükkida. Näiteks FC kaheksa, kaks, kaks või kaheksa kaheksa kaks numbrit ei püsi hästi meeles. Mis oli elektronkirjade aluseks. See dokument on väga huvitav oli neid uurida peale, kui me olime juba terve suure hulga listide ülemaailmsete listide liikmed. Informatsioon levis. Ja üleüldse 

Üleüldse oli internet tore asi. 

Kõneleja 2
Kui suur ja siis ütleks? 

Kõneleja 1
Siis sealt muidugi jah, kõik said aru, et meil on päris internetti ka vaja. Kuigi pange tähele, veebi ei olnud veel sündinud. 

Kõneleja 3
Oot, aga kuidas meil oli kohver? Ja, ja just lugesin artiklit, kuidas? Kohver oli ikka see õige intervjuu. 

Kõneleja 2
Beebivärk see on vale ja mingisugune korporatsioonide välja mõeldud. 

Kõneleja 1
Ja see on tserni selles keskuses välja mõeldud, kuna neid hakkas uuesti ära tüütama, see CERN-is toodetakse palju artikleid ja need artiklid on, eks ole, viitavad üksteisele. Ja siis siis kuidas sa saad aru artiklist, eks ole, siis sa pead neid viidatavaid artikleid ka lugema ja see on üks igavene tüütus, käia neid kuskiltki otsimas. Ja siis mõeldi välja World vaid veed, kus on artikkel ja kus on kohe ka, eks ole, need pildid seal sees, see tähendab, see nõudis juba graafilist brauserit ja kus on viited niimoodi, et sa saad lõksida nende peale ja siis tuleb järgmine artikkel kohale. 

Kõneleja 3
Kas või kuidas? 

Kaks eraldi küsimust, kuidas see Mobeeb jõudis Eestisse Tartusse, täpsemalt ja b kust tuli mõte, et võiks hakata inimestele õpetama, kuidas seda veebi teha? 

Kõneleja 1
Need mõtted tulid peaaegu et korraga, aga kuidas Peeb jõudis Eestisse, minge küsige Morrektiitsugerest. Marek Tiits jälle ja tema töötas tol ajal. 

Tartu Ülikooli raamatukogus, kus oli jälle kuskiltki päranduseks kingitusena saadud mingisugune arvud teed ja ma ei mäleta, mis arvutis oli, aga sellel oli, sellel oli võimalik veebiserver peale panna, sellepärast vabandage mind väga, veebiserverit töötavad juuniks, siis. 

Ja Eesti esimese veebiserveri panin püsti Marek Tiits ja siis selles kursuses, mis vist kandis juba tol ajal pealkirja informaatika didaktika ja kus ma igasuguseid uusi asju, kaasa arvatud internet, üritasin lugeda ja ma usun, et selle kursuse nimi stega mulle internetikursust ei olnud. Mul oli informaatika didaktika kursus, kus me siis üritasime ka nüüd Errefftzeesid seal tagasi saada ja kätte saada ja nii edasi ja nii edasi. Ma kutsusin Marrektiitsu, et ta siis räägiks meile natukene nii kohvrist kui ka veebist. 

Marek Tiits, kes praegu kindlasti on üks parimaid lektoreid üldse tol ajal oli teise kursuse tudeng. Ja kui ma pärast küsisin oma tudengitelt, kellele ta siis sellest veebist rääkis et saite kõik aru, mis ta rääkis, sest tudengid vastasid ja vot ei tea, kui enne ei oleks midagi teadnud, siis vist ei oleks aru saanud. Aga kuna me enne ka midagi teadsime, siis me saime teda üsna hästi jälgida. Nii et nii et esimese serveri au on, on Marek Did sul aga siis kuskilgi, kunagi oli mingi sanspark täis ka Eesti biokeskuses. Ja seal peal ma võtsin oma tudengid üldse tudengitega igasuguseid. 

Kõneleja 3
Tudengitele ka ütlen oma mälestuste järgi. 

Kõneleja 1
Ja mingil hetkel ma saatsin oma, ma ei mäleta, mis kursuse võib-olla needsamad informaatika didaktika tudengid ülikooli peale, et hankige, ülikooli teadus, kui nii igaüks iseteaduskonna edasi hankige igasugust informatsiooni, mis teil õnnestub kätte saada selle samusegi teaduskonna allüksusi, loetavaid aineid ja mida iganes. Ja siis me nõelusime nendest sellise toreda ülikooli veebi kokku ja siis meil jätkus nahaalsust kutsuda kohale rektor, jagaks prorektorit ja paluda neil istuda arvuti taha ja siis vaadata, kuidas Tartu ülikool veebis välja näeb. 

Nägi välja niisugune, saad aru, niisugune, ma usun, nii nagu Eesti metsad praegu välja näevad, et noorendik ja lageraie ja siis mingisugune vana tükk ja nii edasi, see tähendab seda, et ta oli väga lapiline kuna keskust, mida kätte sai, kellel olid tuttavad, kus teaduskonnas ja nii edasi kindlasti oli asi tükati väga halvasti kajastatud. Aga muidugi kujunduse peale, noh, ikkagi Maida teaduskonna õeldi kujunduse peale väga palju auru ei raisanud, aga igal juhul said rektorid teada, mis asi on veeb, kuigi meil vist tol hetkel oli näppudel üles lugeda, mitu veebiserverit meil Eestis üldse tol hetkel oli ja siis ja siis, siis nad võtsid asja üle. Siis nad hakkasid päris päris päris ülikooli veebil. 

Kõneleja 3
Kiituseks öelda. 

Kõneleja 1
Ei, tead, rektorid on meil alati olnud suhteliselt taibukad. Nii palju kui mina neid näinud olen. 

Kõneleja 3
Sellist ideaalis peaks olema nagu nuga ometi eeldusi. 

Kõneleja 1
Ja igal juhul jah. Ma usun, et otse rumalusega ei ole hiilanud ükski Tartu Ülikooli rektor, nii et minu silmis olnud olnud kõik suhteliselt nutikad inimesed olnud. 

Kõneleja 3
Enne kui tõmbame joone alla, siis mul on üks niisugune abstraktne küsimus, et ma ei teagi, kas sellel ongi hea vastus, aga. 

Sinu käest saab selle, ühesõnaga, kui keegi kuskil nagu teab, siis tõenäoselt sina tead kõige täpsemalt seda vastust. 

Mis osa sellest kogu sellest asjast, mis on meid toonud sellest kuskil 70.-te juurest siiamaani välja? Kogu see IT-tööstus ja kõik see, et me sealt Liivi tänavalt oleme nüüd jõudnud siia, deltas kõik see nagu areng ja kogu see värk mis professor Kaasiku nihukesest bussimisest nagu pihta hakkas. Kui palju sellest on Niukest nagu mäetipul kaugusse vaadates nagu püsti pandud visiooni ja kui palju sellest on lihtsalt, et nagu teeme ägedaid asju nagu järgmise kahe nädala jooksul. 

Kõneleja 1
Osakaalud, eks ma seda jagada ei jõua, aga näiteks internetikoolitusõpetajat, et selle, see oli küll see teeme kahe nädalaga neidsamuseid, kihvte asju, sellepärast et me tegime infopäevi, et kuna Eestis internet arenes seal pärast seda, kui, kui, kui see päris internet siia kohale jõudis ikka ikka väga penikoormasaabastega siis siis me otsustasime, et me hoiame ka õpetajaid kursis ja igasuguste koolituste käigus tegime siis suures ringauditooriumis tol ajal muuseas ülikooli küsinud auditoorium, et ei tasu raha meil ei oleks olnud. Ja siis me rääkisime, eks ole, kuidas nüüd, mis asi on internet ja kuidas ta on arenenud ja nii edasi, siis mingil hetkel 

Tekkis Marek Tiits, sul üks, üks europrojekt, mille käigus ta sai 100 moodemit. 50 oli tal projekti jaoks vaja, aga 50 oli, võisime koolidele jagada, me ei hakanud neid niimodi loopima vaid me korraldasime vot selle. Nende. 

Sest saad minna või postmaasturite kursused, aga siis me tegime juba siukseid kombineeritud kursusi, et meil oli viis õpetajat rühma kellelegi, Me õpetasime hiire liigutamist kellelegi. Me õpetasime vist midagi programmeerimist, kellelegi õpetasime, kuidas modemit paika panna ja kuidas sinna teenused peale tõmmata. Kellelegi me õpetasime, mis on, veeb ja ja siis kuskilgi õpetasime, veel mingeid siis saad minna või jumal teab keda. Ja siis nendele postmaasturite kursusele me saime neid teha kaks korda niimoodi, et kõik kohale tulnud koolid, kes tahtsid edukalt kursuse, kahepäevane kursus, laupäev, pühapäev õpetajat koolituvad entusiasmist. Mingit tasu ei maksnud mingil korral, meil oli raha, me saime piletid kinni maksta, aga mitte alati. Ja siis me saime modemi kaasa anda. Meil puudus kontroll, mis nendest muuseumitest pärast sai. Aga kui 95. aastal meil nagu enam ei mahtunud nüüd õpetajat kuskilgi ära, isegi Vanemuise suurde auditooriumisse ei kippunud ära mahtuma ja meie võhm hakkas otsa saama. Ma käisin tuis pedagoogika teadusliku uurimise instituut tudis, kes organiseeris õpetajate koolitust. Käisin küsimus, eks ole, et kas nad meie kursustele ei taha raha anda? Ja rääkisin, et, et telekommunikatsioon tuleb kohe ja ja siis siis vastab ülemus, kelle nime ma kahjuks ei mäleta. Ütles mulle, mis asi teil telle kommunikatsioon, see asi ei tule eesti kooli mitte kunagi. No ma panin suu kinni ja jätsin ütlemata teate, veel 50 kooli on juba moodemiga ühendatud. Ma keerasin otsa ringi, tulin sealt välja, aga sain aru, et sealt august raha ei tule. Sorose fond oli see, mis meile pärast igasugusteks üritusteks natuke raha andis. 

Avatud Eesti fond. Ja siis, ja siis, siis me tegimegi ilma rahata, see tähendab keegi Eeennetist oli Enok Sein või keegi algus, oli Marrektid, seal siis mu tudengid. Ja Tartu ülikool, nagu ma ütlesin, oli aeg, kui Tartu ülikooli küsinud auditooriumide ja arvutiklasside kasutamise eest tasu, sest nagu ma ütlesin, raha meil ei olnud. Kõik, me tegime seda puhtast entusiasmist ja, ja ja ka esimesed e-kursused tegime puhtast entusiasmist. Me lihtsalt ei jõudnud neid suuri kahepäevaseid kursusi enam teha. Ja siis me istusime. 

Terje Tuisuga kahekesi kokku ja mõtlesime, et aga teeks nüüd seda õige. 

Muuseumid on ju olemas, e-kirju nad saavad, üks inimene on koolis, kes oskab modemi käima panna. Teeme siis korjame tema ümber, viis õpetajat, teeme neile koolituse. Esimesel koolitusel me unustasime piirarvu panemata. Andsime õpetajatele teada, et niisugune koolitus tuleb registreeruge ja registreeruge, noh, just see inimene, eks ole, kes modemiga hakkama saab. Registreerige oma kool. Ja me mõtlesime, et nagu viis kooli tuleb, on ikka jube hästi. Kui me jaole saime, panime mingisugused Aktiivse emaili aadressi koonal registreeruma peas ja siis meil oli nii kiirem, ei käinud seda vaatamas vaatamas käisime siis oli seal juba 20 üle 20 kooli ja siis me mõtlesime, et oot, aga mis vahet seal on? Mõningad asjad tuli ära muuta, Ta on sellest, kui me mõtlesime, eks ole, viis kooligast viis inimest, siis nad võivad kõik meile oma elu esimese kirja saata ja me saame kõigile individuaalselt vastata. Kui neid kool ja pärast oli 50 ja osavõtjaid 400, siis me saime aru, et Meie isiklikult igalühel individuaalselt kirju ei kirjuta. Aga siis, siis, siis me lasime neil omavahel, panime nad paar. Panime paari niimoodi, et nad ei oleks üksteise poole külla saanud, väga lihtsalt sõita. Et 50 kilti pidi vähemasti vahet olema ja nad pidid sama aineõpetajad olema. See oli üks igavene paaritaminema, mäletan. 

Terjega sõime. 

Kõneleja 3
Mulle nii meeldib, et see on täpselt see nagu programm, programmeerija lähenemine ülesandele, paneme baari ja ei tohi olla lähedal ja kuidas. 

Kõneleja 2
Sama aine siis ma ise on see üsna loogiliselt tuleb mõelda, eks ole, programmeerimisega. 

Kõneleja 1
Peab ka loogiliselt mõtlema. Loogiline mõtlemine tuleb elus õige mitme koha peal kasuks. 

Kõneleja 3
Vot, need on vaat-vaat, need on kuldsed sõnad, millega võikski lõpetada. Aga mul on üks küsimus veel ja see küsimus on see, millega professor Villems Anevilemist täidab oma aega. Kaunistage moega praegu. 

Kõneleja 1
Aeg esiteks, nagu te kuulsite, ma tulin just usast meie kahjuks minu need tuttavad, kellega ma valas vanasti sain tihedalt läbi käima käia ja kes olid Moskva nii-öelda erinevates instituutides ja ülikoolides need ei ole enam Moskvas. Need on Californias. Ja nüüd ma olen siis avastanud, et California külastamise kõige meeldivam aeg on jaanuari lõpp-veebruari algus mis just sobib mulle, sest siis on ülikoolivaheaeg. Ja nüüd ma olen neli aastat vist järjest käinud oma sõpradel külas. 

Üks parimaid sõbrannasid elab seal. Ja ma veedan tema juures ja nüüd nüüd mul õnnestus seal veeta kaks nädalat ja siis ta mind ära. Saatejuht ütles, et nüüd sa oled aru saanud kaks nädalat on õige aeg, mitte üks nädal. Nii et järgmine kord tuleb kaheks nädalaks. Nii et jaanuari lõpp-veebruari algusvõtjate mind alati leida. Californiast Palo Aaltost. 

California kevadest, kus nii need kohalikud kui ka sissetoodud taimed nagu näiteks lüpsipuud, õitsevad. 

Kõneleja 3
Hurmav lisaks sellele, et tegemist on Palo Alto 

Kõneleja 1
Ei hakka, mis on eksole, Stanfordi kodulinn ja kus Palo Alto ja ookeani vahel on toredad mäed. 

Ja seal on siis ka päris ookean teispool mägesid. Mäe otsas on ka tore käia. Üks sõber viis Sanhose juures mäe otsa, kust me oleksime pidanud nägema ühele poole? San Franciscosse. Sagaruse asub Sanhose juures, nüüd seal on väga pikk vahemaa vahepeal ja teist poolt siis oleks näinud. Sanhozeetsed on ikka saanud seest natuke põhja pool ja ja, ja, ja siis oli piimjas udu ja vihmapilved. Ja siis korraks tuli tuul ja lõuna poole nägime siis seda vaadet, mida pidin nägema. Nii et jah, väga tore on reisida. Aga ma siin töötan veel küll tunnitasu alusel ja loen oma armastatud andmebaaside kursust. Kolmes versioonis. 