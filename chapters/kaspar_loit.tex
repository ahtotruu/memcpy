\index[ppl]{Loit, Kaspar}
\index[ppl]{BKnows}
\index[ppl]{BKnows|see{Loit, Kaspar}}

\question{Kes sa oled?}

Mina olen Kaspar. Ja kuna me peame ajas miljard aastat tagasi kerima, siis 
kunagi oli mu \emph{aka} BKnows. 

\question{Kuidas sa selle said?}

Seda ei mäleta enam keegi. Sellel on kaks komponenti: 
\enquote{B} ja \enquote{knows} ehk see B peaks midagi 
teadma. Pronto\index[ppl]{Pronto} kutsus mind alati Buttknowsiks.

\question{Kuidas sina said arvutite juurde või said arvutid sinu juurde?}

Mu tädi, kes on superlahe ja minust mõnevõrra vanem, 
töötas Tartus bioloogia instituudis. Talle oli jäänud mulje, et mind võivad 
sellised asjad huvitada. Kui tal ükskord Tartus kaheksa- või üheksa-aastasena
külas käisin, viis ta mind instituuti, muidugi peale tööd, kui kõik oli juba 
pime. 
Ühes kabinetis seisis laua peal masin, mille nimi oli Apple II 
Europlus\index{Apple II}\sidenote{Apple II Europlus oli Apple'i Euroopa 
turule kohandatud versioon. Muu hulgas erines sellel toiteblokk, aga ka video 
osa tuli 
ümber teha, sest Steve Wozniaki trikid NCTS-signaali genereerimisel keerukama 
PAL-süsteemi puhul enam ei toiminud.}. See oli \emph{freaking awesome}. 
Tädi oli ühe laborandi käest küsinud, kuidas miski käib. Arvutil oli paar 
superägedat mängu, mis tekstiekraanil jooksid. Üks oli vist \enquote{Train 
Robbery} ja sellest hetkest olin müüdud. Ma ei mäleta, kas 
olin arvutitega enne ka kokku puutunud, aga tõenäoliselt mitte. 

Mulle tohutult meeldisid kooliajal Nintendo väiksed Game \& 
Watch\index{Nintendo Game \& Watch} mängud. Need olid tänapäeva 
telefoni suurused ja neil oli LCD-ekraan, millel olid joonistatud 
tegelased ja mängimiseks nupud\sidenote{Vt ka lk
\pageref{sisu:gameandwatch}.}. Mõtlesin, et 
lahe oleks ise niisuguseid mänge teha, aga sain aru, et seal taga 
on tootmine ja see ei ole reaalne. Ja nüüd järsku saada tolle Apple peal aru, 
et 
selliseid asju on võimalik masina sisse programmeerida ilma, et peaksid 
elektroonikaskeemi tootma või tehast omama, oli tõeline
\emph{revelation}. See tõenäoliselt viiski mind hiljem
mängude tegemiseni. 

Päris arvutiga toimetamine hakkas ilmselt tänu Jaak 
Loondele\index[ppl]{Loonde, Jaak}, kes oli ülioluline tegelane, sest 
sel ajal oli arvutile ligipääs oluline. 
Olin kaardistanud endale kõik kohad Eesti Vabariigis, kus üldse 
arvutitele ligi pääses. Nõo oli liiga kaugel, aga Tallinnas neid kohti jagus. 

Jaak Loonde\index[ppl]{Loonde, Jaak} kaudu saingi vist esimest korda midagi 
programmeerida. 

\question{Kas tema andis võimaluse või õpetas ka?}

Ta ikka õpetas loomulikult. Ma sattusin 3. 
keskkooli\index{Tallinna 3. Keskkool}, kus oli suur arvutiklass. Tõenäoliselt 
olid need MSXid\index{Yamaha 
MSX}, BASIC\index{BASIC} ees ja võis hakata klõpsima. 
MSX oli pull masin, mis mõeldi välja selleks, et 
ühtlustada koduarvutite standardit ja BASICut. Selle 
initsiatiiviga tuli vist lagedale üks Jaapani Microsofti juht.

\question{See \emph{boot}'is otse BASICusse, eks?}

Jah. Esimesel ekraanil tuli kirjutada \verb|10| ja üks rida, siis \verb|20| ja 
teine 
rida ning seejärel \verb|list| ja tulemus oli näha. Kirjutad uuesti 
\verb|20|, kirjutad selle rea üle, siis \verb|run| ja ongi kõik, 
hakkaski kohe käima. 

Klassist sain teada paari tuttava kaudu. Üks koolikaaslane nägi sellist
masinat esimest korda ja kui öeldi, et hakake midagi tegema, siis ta 
kirjutas \verb|please draw me a circle|. NLP\sidenote{\emph{Natural Language Processing} -- NLP.} aga ei olnud veel nii 
kaugel 
ja sealt midagi ei tulnud.

Klassi ümber sagis palju rahvast, seal toimetas Karel Kannel\index[ppl]{Kannel, 
Karel}. 
Jaak Loondel\index[ppl]{Loonde, Jaak} oli üks Nõukogude masin 
MIR-2\index{MIR-2}\sidenote{\begin{russian}МИР\end{russian} oli varane 
Nõukogude miniarvutite sari, mille kolm generatsiooni (MIR, MIR-1 ja MIR-2) 
töötati välja aastatel 1965--1969. Lühend tulenes pikemast nimetusest 
\begin{russian}mашина для инженерных расчётов\end{russian} (inseneriarvutuste 
masin).}, mis nägi välja nagu 
pikk kõrge kapp. See tegi meeletut villakraasimismasinaga sarnast 
müra, mis tekkis jahutuse tõttu. Aeg-ajalt, kui inimestel 
viskas kopa ette, lülitati jahutus välja ja siis 
Jack\index[ppl]{Jack}\index[ppl]{Jack|see{Loonde, Jaak}} pistis
röökima, sest ilma jahutuseta oleks masin ära küpsenud. 

Ta oli selles mõttes geniaalne, et kust ta nii võimsa masina üldse kätte oli 
saanud? Sellel oli võimalik klaviatuurist käske 
sisestada, kusjuures klaviatuur oli elektronkirjutusmasin, mis toimis
nagu klaver ja printer ühekorraga. Nii et \enquote{kõva koopia} tuli ka samast 
masinast. Sellega oli ühendatud mustvalge telekas ja ka 
valguspliiats, millega sai ekraanil tabada mingisuguseid punkte ja 
masin tundis puudutuse ära, ilmselt luges kineskoobi kiirt ja selle järgi 
pani asukoha kokku. Lisaks suutis see algelist graafikat 
kuvada ehk ei olnud ainult tekstiekraan, vaid kuvas ekraanile 
punkti. See oli arvuti jaoks tohutu ülesanne: punkt ei 
püsinud hästi paigal, vaid ujus pisut. 

MIR-2 peal kasutati venekeelset programmeerimiskeelt, igasuguseid
lühendeid. Mu esimene programm oli üks graafiline neljatipuline täht. Täht
koosnes umbes kuueteistkümnest punktist, mis 
tõmbas arvuti täiesti kooma -- terve ekraan ujus, aga mul oli väga uhke tunne. 

Jack\index[ppl]{Loonde, Jaak} õpetas meile veel näiteks 
perfolindi lugemist. Masin sõi kahte moodi meediat. Üks oli õhukesest paberist 
perfolint, mis lasti vurinal masinast läbi. Teoorias teravamad vennad 
suutsid nõela või augurauaga perfolindile programmeerida, võimalik, et ma isegi 
kunagi. Ja teiseks olid seal magnetkaardid. 

\question{Perfokaarti ma tean, aga magnetkaardist ei ole kuulnud.}

Magnetkaart oli tänapäeva telefonist suurem pruun latakas, mis materjali 
poolest meenutas flopi sisu. See sisestati ühte
avasse, masin tõmbas selle 
surtsti läbi ja luges sealt midagi. Paras müstika. 

\question{Noore nagamanni jaoks oli ekraanile 
tähe joonistamine ilmselt tohutult huvitav.}

See oli müstiliselt äge! Kirjutasid midagi ja ekraanile 
tekkis pilt.

\question{Kas huvitav oli see, et sina andsid käsu ja masin tegi 
midagi?}

Täpselt! Kui see oleks nii lihtne olnud, et kirjutad \verb|please draw| \verb|me a circle|, 
siis ilmselt oleks huvi kiiresti kadunud, aga see oli ikkagi \emph{complicated} 
värk. 
Selles oli ülikõva alkeemiline element. Väikestele poistele 
meeldivad salakeeled, koodid, lipukirjad ja muu säärane. Arvuti
oli kõike seda ja veel enamgi.

MIRi võimalused olid loomulikult piiratud ja kaua ei viitsinud sellega
jännata. Pealegi oli seda ainult üks. Õnneks MSXi klass oli
suurem, aga sealgi olid piirangud, kuna see asus 
keskkoolis. Ilmselt sellepärast Jack\index[ppl]{Loonde, Jaak} sebiski 
Roopa tänavale ÕTKsse\index{Tallinna Oktoobrirajooni 
Õppetootmiskombinaat}\sidenote{Tallinna Oktoobrirajooni 
Õppetootmiskombinaat, vt lk
\pageref{content!OTK}.} terve arvutiklassi. ÕTKs oli üks
juhtarvuti, millel oli flopidraiv ja millest ülejäänud arvutid said 
asju alla laadida. Sai ka oma programmi kirjutada, aga kuna 
draive oli ainult üks, siis programmi salvestamiseks pidi selle 
juhtarvutisse saatma. 

Kuna Jackil\index[ppl]{Loonde, Jaak} ei olnud aega klassis väga käia,
möllas seal kogu aeg poistekari ja ta pani paar nutikamat venda 
seda vedama. Üks legendaarne tegelane oli 
Mukats\index[ppl]{Mukats|see{Edesi, Linnar}} ehk Linnar Edesi\index[ppl]{Edesi, 
Linnar}, kes täna toimetab vist kuskil Soomes ja kes oli minu esimene 
guru. Ta oli umbes minuvanune, aga juba omandanud 
arvutiasjanduse peenemad alged. Põhiline, mida ta oskas, oli kahest 
programmijupist ühe terviku kokkupanek ja selle paketeerimine nii, et 
tulemust sai programmina laadida.
Väga suur osa tarkvarast levis tavalistel magnetofonikassettidel ja 
šefimad mängud olid umbes 32 kilobaiti pikad. 
Kassetile mahtumiseks olid need tehtud pooleks: 16 + 16 
kilobaiti ja vahepeal tuli kassett ümber keerata. Aga kui oli juba nii kõva asi 
nagu 
flopidraiv, siis sai selle kasseti pealt kuusteist kilobaiti sisse lugeda, 
tõsta 
see mälus mujale, lugeda ülejäänud kuusteist kilobaiti sisse ja need esimese 
jupiga 
kokku panna. Tekkis tervik, mida sai kuidagimoodi käivitada. 
Täielik maagia. 

\question{Järelikult tekkis sul teadmistel põhinev eeskuju -- inimene, kellele 
sa vaatasid alt üles, sest ta teadis rohkem kui sina?}

Ojaa, selliseid tegelasi oli veel. Üks oli 
Kont (ma ei mäleta, mis ta eesnimi oli\sidenote{Tõenäoliselt peab Kaspar 
silmas juba mõnda aega tagasi manalateele läinud Mart Konti\index[ppl]{Kont, 
Mart}.}). Tal oli 
metallkohvrikene, mille sees oli kogu MSXi \emph{manual}'i koopia. Ta käis 
kohvriga uhkelt ringi, aeg-ajalt tegi 
selle lahti ja kirjutas midagi selle alusel.

Kuna draivile otsest juurdepääsu polnud ja sai põhiliselt 
mängida või oma BASICut\index{BASIC} kirjutada, siis tulid sellele 
tegevusele piirid ette. Mind huvitas graafikapool, 
üritasin pilte ekraanile manada. MSXil\index{Yamaha MSX} ei olnud 
graafikaekraani, seda emuleeriti tekstiekraaniga. See tähendab, et
iga pilt ja mäng, mis MSXil jooksis, oli 
tegelikult otse mälus tähe{\-}generaatori ümber programmeerimine. 

\question{Jukuga\index{Juku} oli sama lugu: kuskile mällu sai laadida
oma nii-öelda fondi. Iga tähe asemele pandi \emph{bitmap} ja nendest 
sai mida tahes kokku laduda.}

Just. MSXi ekraan oli otse adresseeritav. Kui 
teadsid, selle režiimi ekraan alguse aadressi, sai sinna järjest kirjutada. 
Iga bait 
oli üks rida ja võis olla kas läbipaistev, taustavärv või esivärv, ja teatud
režiimides sai rida-realt neid värve vahetada. 
Mängude puhul on niisugune mõiste nagu sprait. Need on graafikatükid, mis 
tausta ees liiguvad. MSXil 
emuleeriti neid ka sellesama tähegeneraatoriga, mida programmeeriti jooksvalt 
ringi. 
Ekraan kirjutati sümboleid täis, mida kogu aeg adresseeriti ja 
kirjutati ringi. Ekraan oli jagatud kolmeks osaks ja igas osas sai erinevat 
tähestikku väänata.

Mind võlus assembler\index{Assembler}, kus sai ühele mäluaadressile ühe 
baidi laadida. Veetsin suure osa oma ärkveloldud 
ajast mingisuguseid tegelasi millimeetripaberile joonistades, neid heksaks 
tõlkides ja kuskile mällu laadides. 

\question{Tänapäevane arvutigraafika ei 
ole ka lihtne, aga keerukus tundub olevat teises kohas -- peab 3D-geomeetriast 
aru saama ja spetsiifilisi APIsid tundma.}

Täna on sinu ja pildi vahel ikkagi \emph{layers of stuff}, aga MSXi puhul tuli 
otse ekraanile midagi
toorelt toppida. 

\question{Sa pidid ikkagi ise välja mõtlema, mida mällu toppida, ja 
hoolitsema, et see värskendatud saaks.}

Selleks et see töötaks, tuli teada igasuguseid trikke, aga kuna keelestik ja 
kõik muu oli nii lihtne, siis oli ka tulemus lihtne ja elegantne. Minu
programmeerimisaeg jäigi kaheksakümnendate keskele. Hiljem nokitsesin
natuke HTMLi ja CSSi, aga see võlu läks üle kohe, kui asjad läksid 
keerulisemaks. Õnneks tulid asemele 
graafikapaketid ja muu säärane. 

\question{Millal see oli? Kas keskkooliajal?}

Ma olin ju kaardistanud kõik kohad, kus sai arvuteid 
näppida. TPIs\index{Tallinna Tehnikaülikool} oli ka üks klass, kus olid 
MSXid\index{Yamaha MSX}, ja seal oli igal masinal juba draiv taga. 
Seal tegutsesid laborandid, näiteks vist Aare Tali\index[ppl]{Tali, 
Aare}\sidenote{Aare tegutses TTÜs küll, vt ka 
lk \pageref{sisu!aare_tali}.}. Igal juhul toimetasid seal 
üliõpilased ja neid loomulikult tüütas nagade jada ukse taga. Nad kehtestasid 
reeglid ja olid suured jumalused. Näiteks kui info levis ja järjest 
rohkem kutte tekkis ukse taha, oli vaja reglementeerida, kes saab 
ligi ja kes mitte. 
Nad võtsid kõige popima mängu, \enquote{King's Valley}\index{King's Valley}, ja trükkisid 
välja kogu selle 
\emph{source}-koodi. Nad lugesid koodi, tõmbasid punase pastakaga ringe ja 
programmeerisid mängu ringi nii, 
et said iseehitatud \emph{joystick}'idega kolle juhtida. Minu jaoks oli see 
nagu jumaluse tase! 
Ja reegel oli, et kui said 
jumaluste vastu ühest tasemest läbi, siis võisid ühe päeva klassis käia. Põnev 
oligi just see, et nad kirjutasid
minu jaoks siis superkeerulisena tundunud mängu ringi. Tagantjärele tundub see 
muidugi väga lihtne.

\question{Eks see oli \emph{gamification}\sidenote{Inimestele mulje 
jätmine, et nende kasulik tegevus on tegelikult mäng, milles on oluline 
\enquote{võita}. Vt ka Tom Sawyer ja tema plank.}, mis on praegugi populaarne, 
ja 
kindlustas jumalastaatust poistekamba silmis veelgi.}

Lõpuks sattusin Kullo\index{Kullo} 
arvutiklassi, kus olid natuke kõvemad MSXid\index{Yamaha 
MSX} graafikarežiimi ja muude asjadega, ehkki kui 
tahtsid midagi kiiresti liigutada, pidid ikkagi kasutama tekstiekraani. Toda 
klassi majandas selline legend nagu Räni Meister\index[ppl]{Meister, Räni}, 
tore punkar, kes oli tulnud kuskilt Võru 
gaasianalüsaatorite tehasest ja viitsis poistega jahmerdada. Tema 
hakkas tegelema Commodore'i 
Amigadega\index{Amiga}\sidenote{Amiga oli Commodore'i poolt 1985. aastal 
turule toodud personaalarvutite sari. Teistest põlvkonnakaaslastest eristas 
seda perekonda spetsiaalse graafika- ja heliriistvara lisamine ning 
väljatõrjuva mitmetegumilisuse realiseerinud AmigaOS.}, mis oli 
\emph{super advanced} raudvara. 

Kuidagi mahtus see kõik videotootmise tähe alla ja tänu 
sellele oli temagi loomulikult kaardistanud, kus niisugune asi veel toimub. 
Eesti Televisioon\index{Eesti Rahvusringhääling!Eesti Televisioon} oli selgelt 
selline koht, lisaks üks vene metalliärikate turundusharu. Ilmselt keegi 
vend oli piisavalt palju lobi teinud ja kuskil Kristiines keldris püsti 
pannud väikse reklaamistuudio, kus ta tootis värki, ja tal oli 
seal ka üks Amiga\index{Amiga}. 

\question{See pidi siis olema üheksakümnendate algus juba, eks?}

Jah. Kullos hakkasime ka siis juba mänge tegema. Seal toimetas näiteks Markus 
Klesman\index[ppl]{Klesman, Margus}. Raul 
Keller\index[ppl]{Keller, Raul}, kelle \emph{aka} oli 
Killer\index[ppl]{Killer|see{Keller, Raul}}, üritas MSXi mänge 
publitseerida, aga see tundus (vähemalt mulle ja toona) väga 
kahtlane ja naiivne tegevus.

Räni\index[ppl]{Meister, Räni}, kes nägi minus mingit potentsiaali, 
meelitas mu Eesti Televisiooni\index{Eesti Rahvusringhääling!Eesti 
Televisioon}. Ma ei olnud isegi veel keskkooli viimases klassis, 
kui juba töötasin \enquote{Aktuaalses kaameras}. Uudistetoimetuse kõrval oli 
väike kubrik, kus tegime \enquote{Aktuaalse kaamera} infonurki, mis olid diktori 
taga seina peal. Ja kuna Amigasse\index{Amiga} 
sai lasta videosignaali sisse ja sealt tuli videosignaal välja, sai 
sellega digimiksi teha. 

\question{See oli toona PC peal jõhkralt kallis riistvara!}

Oligi. Miks Amigad siin kandis selles vallas levisid, oli just see, et PC 
jaoks oli selliste võimalustega videokaart Hollywoodi hinnaga. Enamasti oli 
PCdel 
CGA\sidenote{1981. aastal turule tulnud \emph{Color Graphics Adapter} (CGA) oli 
IBMi esimene värviline graafikakaart ning kehtestas \emph{de facto} 
graafikastandardi. See võimaldas 320x200 ekraani{\-}lahutusega kuvada nelja ja 
640x200 lahutusega kahte värvi 16 värvi hulgast.} ja neli värvi, samas kui 
Amiga\index{Amiga} oli \emph{full video}. Põhimõtteliselt polnud 
vaja isegi arvutimonitori, sellele võis teleri järele panna. See oli 
kodutarbimisest arenenud. 

Seal tegime oma ilmakaarte ja lisasime videopilte ning põhilise osa ajast 
muidugi mängisime, sest Amigal olid šefid arvutimängud. 

\question{Mis tarkvaraga te seda kõike tegite? Ega te ju 
kogu kraami nullist kirjutanud?}

Olid olemas täitsa viisakad graafikapaketid, näiteks Deluxe Paint\index{Deluxe 
Paint}\sidenote{Deluxe Paint on rastergraafika redaktorite sari, mille lõi 
Electronic Artsi jaoks Dan Silva. Programm alustas elu majasisese 
graafikaprogrammina, kuid sai pärast avaldamist \emph{de facto} standardiks 
Amiga platvormil.} oli üks šefimaid graafikatarkvarasid, mis oli 
Photoshopist ja teab veel millest kümme aastat ees. 

Me olime \emph{in-the-know} Amiga-vennad ja vaatasime kõikide 
PCdega tegijate peale ülalt alla. Paraku oli Amiga bisness kehv ja 
läks lõpuks nurja, aga tehnika iseenesest oli äge. 

Meil tekkis punt tegelasi, kellel kas oli kodus Amiga või kes 
tegeles sellega kuskil mujal. Näiteks teles oli Martin Rinne\index[ppl]{Rinne, 
Martin}, kes 
täna teeb Directot\index{Directo},
siis oli meil veel Margus Kliimask\index[ppl]{Kliimask, Margus}, kes tegeles 
Eesti 
Videos\index{Eesti Video} Siilatsi asjadega, ja Mati 
Veermets\index[ppl]{Veermets, Mati}, kellest sai hiljem Tallinna linna disainer.

Mind võlus pigem see, et sai tekitada 
elava pildi -- ei pidanud olema kaameraid ja näitlejaid, vaid 
võis teha väikseid animatsioone otse arvutis.

\question{Kas see vedas isegi animatsiooni välja?}

Animatsiooni sai teha \emph{stop 
motion}'iga. Reaalajas animatsioon kippus jõnksutama, kuigi 
tegelikult tegime telepäid ka reaalajas, sest keegi ei viitsinud 
\emph{stop motion}'iga lasta. Aga enamik asju käis 
reaalajas. Deluxe Paintis\index{Deluxe Paint} olid sisse ehitatud igasugused 
nutikad asjad, nagu liikumise aeglustamine või kiirendamine. Näiteks kui andsid 
ette, et siin on kast ja nüüd see peab liikuma 
viiekümne kaadriga sinna, siis programm täitis need 
viiskümmend kaadrit automaatselt ära. Ja kui andsid käsu \emph{ease in}, siis 
tõmbas lõpus hoo maha. 

Olin alles telesse tööle läinud, kui tegime 
öölaulupeole\index{Öölaulupidu}\sidenote[][-1.6cm]{Esimene öölaulupidu peeti 1987. aasta juunis Tallinna vanalinna päevade ajal, aga 
toona meediakajastus puudus ja üritus toimus spontaanselt. 1988. aastal oli 
öölaulupidu juba ametlikult vanalinna päevade programmi lülitatud.} 
valgusklippe, see oli jällegi \emph{super advanced}.

\question{Võru poisina ei tea ma öölaulupidudest midagi, aga 
öötelevisioonil\sidenote{1990. aastal aset leidnud, omas ajas mitmes mõttes 
innovatiivne teleprojekt, mille käigus Eesti Televisioon\index{Eesti 
Rahvusringhääling!Eesti Televisioon} oli öö läbi katkematult otse-eetris.} oli 
väga äge graafika.}

Jajah, see oli ka meie tehtud. Tegelikult olid kõik sellised asjad teles meie 
rida, sest alternatiiv oli tiitrimasin, mis oli poolanaloogpult 
ja tootis eriti koledat jälge. Meil oli 
animatsioonidega ja värviline, sai teha mida iganes. Vahel tegime reklaamide 
jaoks
haltuurat ja ka igasuguseid lollusi.

\question{Kas see oli puhas iseõppimine või hakkas kusagilt 
informatsiooni ka juba tulema?}

See oli puhas iseõppimine. Vahendid olid suhteliselt 
piiratud ja midagi keerulist ei olnud. Hiljem tulid ka 
esimesed 3D-paketid, nendega oli rohkem pusserdamist. Tase oli nendega hoopis 
teine: punkt punkti haaval tuli pindu konstrueerida ja 
siis nendega kuidagi opereerida. Tänapäeval väänatakse kihtide kaupa mingeid 
\emph{bump mapping}'uid\sidenote{Arvutigraafika tehnika, mille abil 
kolmemõõtmelise objekti pinnale simuleeritakse kühme ja kortse. Lihtsalt 
öeldes tehakse oranžist kerast usutava väljanägemisega apelsin.} ja 
asju ning tulemus on käes -- täielik müstika. 

\question{Mida sa pärast Eesti Televisiooni\index{Eesti 
Rahvusringhääling!Eesti Televisioon} tegid?}

Videograafika oli küll väga põnev, aga siis hakkas tekkima 
\emph{business}. Sõbrad olid rohkem sattunud 
trükigraafika peale, näiteks kujundasid Eesti Ekspressi. Sain aru, et Amigaga 
teeme 
videot, aga \emph{business}'i tarvis peaks ennast 
PCde peale sebima. Sealsamas telemajas tekkisid 
potentsiaalsed kliendid ja pidin hakkama tootma trükikõlbulikku kujundust. 

Ma ei olnud kunagi näinud sellist programmi nagu Corel Draw\index{Corel Draw}, 
aga töö tuli ära teha. Istusin öö läbi üleval ja tegin programmi endale 
selgeks, kuigi see oli tohutult frustreeriv, sest see oli täiesti teine maailm. 
Tänapäeval on nii, et joonistad ja su joonistatud pilt on ekraanil. Siis 
tuli konstrueerida mustvalgelt \emph{vector 
mesh}, panna sinna värvid peale, vaadata \emph{preview}'d ja siis 
joonistus pilt aeglaselt ette. Alles seejärel sai uuesti 
pildi kallale minna.

\question{Kuidas sul Corel Draw õppimine välja tuli? See 
ei olnud kuigi töökindel ja tegi aeg-ajalt faile katki.}

Arvutitega üles kasvades tuli ju arvestada, et aeg-ajalt 
jooksid need kokku ja tegid rumalusi. Võibolla mängis natuke 
rolli ka poisikesepõlves ÕTKs\index{Tallinna Oktoobrirajooni 
Õppetootmiskombinaat} õpitud arvutist üleolek läbi ühe lihtsa fakti. 
MSXil\index{Yamaha MSX} oli paremas nurgas port, mille sisse käis kas 
kettaseade või mälukassett. See oli päris suur 
sahtel ja et kasseti sisselükkamise hetkel mitte midagi tuksi 
keerata, oli sahtli sees väike lüliti, mis tegi masinale \emph{reseti}. 
Õppisin kiirelt ära, et kui olen midagi tuksi keeranud, näiteks 
kirjutanud programmi, mis jäi tsüklisse, siis selle asemel et voolu välja 
võtta, panin kohe näpud auku ja masin oli surnud. Teadsin alati, et 
mingi valemiga saab temast jagu. See teadmine on olnud minuga siiani, et kui 
kuskilt seinast lõpuks ikka juhtme kätte saan, siis on masin surnud. Ma ei pea 
seda pelgama.

\question{Kuidas sa MicroLinki\index{MicroLink} ja 
.EXEni\index{.EXE} jõudsid?}

Kui olin kord prindiga alustanud, siis vahepeal peamiselt sellega tegelesingi. 
Olin 
Margusega\index[ppl]{Kliimask, Margus}\sidenote{Kaspar peab ilmselt silmas 
Margus Kliimaskit.} televisioonis varem suhelnud ja tema omakorda 
Lõviga\index[ppl]{Lõvi}. Lõvi oli kõige olulisem 
tegelane üldse. Kui Jaak Loondest\index[ppl]{Loonde, Jaak} algas kogu Eesti 
arvutiteadvus, siis 
Lõvist algas kogu arvuti-\emph{business}, kuigi ta ise pole vist 
äri kunagi ajanud. 
Rainer Nõlvak\index[ppl]{Nõlvak, Rainer} oli teinud Margusele ettepaneku 
toimetada üht ajakirja. Tema võttis mul varrukast kinni ja ütles, 
et nüüd on vaja ajakirja teha. Mina muidugi oleksin pigem 
mänginud arvutimänge -- teles tegime seda üheksakümmend 
protsenti ajast --, aga sain endale 486, mille peal jooksis \enquote{Ultima 
Underworld}\index{Ultima Underworld}\sidenote{1992. aastal Blue Sky 
Productionsi üllitatud \enquote{Ultima Underworld} oli mitmes mõttes teed rajav rollimäng
(kolmemõõtmeline keskkond, simuleeritud mittelineaarne mängu käik jne).}, ja 
see oli täitsa tore. 

Toimetamisvõhikuna arvasin, et ajakirja tegemist peaks alustama 
esikaanest, ja seda ma siis Corel Draws hiirega 
joonistasin. Tagantjärele mõeldes tundub, et 
kuude kaupa. Tõenäoliselt see nii ei olnud, aga sinna läks tohutu aur. 
Pronto\index[ppl]{Pronto} luges kokku, et numbreid väga palju ei olnud, kuid 
aega kulus 
suhteliselt palju. Ja kuna tegu polnud otseselt 
äriettevõtmisega, vaid .EXE oli pigem promo, siis keegi eriti ei 
survestanud ka tagant. Meil ei olnud kohustust, et see peab iga kuu ilmuma, 
kuna meil ei olnud 
tellijaid.

\question{Samal ajal kuskil Võrus istus üks nohik, kes kurvastas, et 
uut .EXEt ei ole veel tulnud!}

Me ei adunud siis, et avaldame kellelegi mõju.

\question{Mõju oli kindlasti olemas. Võin seda enda näitel kinnitada ja see, et 
.EXE on
Pronto panduna praegugi internetis olemas\sidenote{\url{punktexe.ee}}, on ka 
selge märk selle mõjukusest.}

See oli oluline igas mõttes. Olles selles asjas sees, ei olnud minu jaoks
küsimus, kas arvutid tulevad maailma muutma. Ma isegi ei mõelnud sellele, 
nendega oli lihtsalt hea asju teha. Kõrvalt vaadates ma isegi ei saanud aru, 
kuivõrd 
vähe tegelikult arvuteid toona kasutati, sest me istusime MicroLinki 
peakontoris ja seal käis kogu aeg mingisugune sebilung. Mul oli telemajas ja 
igal pool päris hea 
juurdepääs arvutitele. 
Mäletan, et .EXE esimeses 
numbris oli arhitekt Kalle Rõõmuse\index[ppl]{Rõõmus, Kalle} büroo 
väike tutvustus, kuna nad hakkasid kasutama arvuteid 
projekteerimisel. See oli täiesti epohhi loov. Toona ma ei 
saanud sellest isegi aru, kui imelik see üldse on, et keegi teeb midagi 
paberil. Mina laksisin artiklid paika, panin pildid külge ja mind 
väga ei huvitanudki, mis seal kirjas oli, kui ma just ise midagi ei kirjutanud. 
Aga tolles Kalle Rõõmuse büroo artiklis kirjutati, et üks 
arhitekt käis Kanadas stažeerimas. Kanadas tegeleti just sellega, et osteti 
personaalarvutid ja töö muutus efektiivsemaks võrreldes sellega, kui arhitektid 
ja konstruktorid päevad läbi kalkale joonestasid.
Tegelikult on huvitav vaadata, kuidas tänaseks on meil 
BIM-modelleerimine\sidenote{\emph{Building Information Modeling} -- protsess, 
mille käigus füüsilisi ruume käsitletakse digitaalsete vahenditega.}, 
ja kuulda, milliseid väljakutseid see esitab. Üks mu sõber töötab 
\emph{start-up}'is, mis tegeleb BIM-mudelite konfliktide analüüsiga. Nad 
üritavad aru saada, 
et näiteks ventilatsioonitoru ei tohi läbi akna minna, ja siis mõtled, et 
issand jumal, millega need inimesed on tegelenud? Miks nad pole
arvutit varem kasutusele võtnud? Kui palju on aega raisatud!

.EXE\index{.EXE}, olgugi et sellest jäi mulje kui \emph{super 
advanced} ja häkkerite värgist, üritas anda pilti sellest, mis tegelikult 
toimub -- et arvuti ei ole ainult raamatupidaja kalkulaator. 

\question{Kuidas sa joonistamise juurest kirjutamise juurde jõudsid?}

Oli vaja sisu toota ja ega toona ei olnud keegi sündinud
arvutiajakirjanik. Mulle meeldis arvutimänge mängida ja ka
kirjutamine on tore tegevus. 

\question{Kas kirjutamise soon lõi sul juba kooliajal välja?}

Ei, ma olen võimeline kirjutama okeilt. Joonistada meeldib mulle
rohkem, sest kirjutamine on raske -- laused peab läbi mõtlema 
ja need ei pruugi head tunduda. Formaat on liiga konkreetne.

\question{Mõni aeg tagasi lükkas Tõnis Kahu 
(keda tuleb ilmselt uskuda)\index[ppl]{Kahu, Tõnis} ümber mu arusaama 
sellest, mis on küberpunk, aga minu arusaam tuleneb konkreetsest 
.EXE artiklist, kus on sinu ja Pronto\index[ppl]{Pronto} nimed 
all\sidenote[][-4.7cm]{Kaspar Loit (1994). Kes sa selline oled, küberpunk? 
.EXE, 
(3), 60--63. Siin mu mälu veab alt: järgi kontrollides Pronto nime artikli 
juurest ei leia, küll aga 
sedastuse, et \enquote{õiged küberpungid lasevad selliste loetelude 
peale suht laias kaares}.}. Kuidas te selle artikli sisu 
produtseerisite?}

Tagantjärele on seda väga raske öelda, aga eks meil oli mingi ettekujutus. Ega 
küberpunk 
ole geneetiline organism, mis on välja arenenud ja mida pärast 
on lihtne klassifitseerida, et pool on hüljes ja teine pool gepard. Eeldan, et 
me toona juba teadsime Gibsoni teost 
\enquote{Neuromancer}\sidenote[][-5cm]{\enquote{Neuromancer} on William 
Gibsoni 1984. aastal ilmunud romaan, esimene tema Sprawli triloogiast. 
Romaani peetakse \v{z}anri üheks mõjukamaks ning on ainsana 
võitnud nii Nebula, Hugo kui ka Philip K. Dicki auhinna. Triloogia torkab 
tagantjärele silma oma hämmastava võimega tulevikku ette näha. Seejuures ei 
toimi tänapäeval kirjeldatud viisil mitte ainult tehnoloogia, vaid ka näiteks 
häkkimisprotsessi kirjeldus ja arvutikaitse on kirjeldatud vägagi tõetruult.}. 
Kui kõik räägivad 
\enquote{Hitchiker Guide'ist}\sidenote[][]{Douglas Adams, \enquote{The 
Hitchhiker's 
Guide to the Galaxy} (\enquote{Pöidlaküüdi reisijuht galaktikas}). 1978. aastal 
raadiokuuldemänguna alustanud komöödia, mis 
avaldati viieosalise raamatutriloogiana ja millele kuuenda osa lisas pärast 
autori surma avaldamata materjali põhjal Eoin Colfer. Sarja raamatud levisid 
tekstifailidena laialt BBSide ja interneti vahendusel, olles ka siinmail 
kergesti kättesaadavad.\phantomsection\label{sidenote!adams}}, siis see oli väga oluline teos, 
aga minul lasid
Gibsoni \enquote{Burning Chrome}\sidenote[][]{\enquote{Burning Chrome} on 
William 
Gibsoni 1982. aastal ilmunud novell, kus tutvustatakse Sprawli 
triloogia maailma ja mille sündmusi ning tegelasi mainitakse triloogias 
korduvalt.} ja \enquote{Neuromancer} ajud täiesti välja.

\question{Ma loen neid siiamaani pidevalt üle. Gibson kirjutas 
need trükimasinaga paberile ja aastal 2019 täpselt nii ongi, nagu ta kirjutas!}

Oled sa tema uuemaid raamatuid ka lugenud? Need lähevad veel hirmuäratavamalt 
tõepärasemaks ja ajahorisont tuleb üha lähemale.

\question{Siit minu küsimus, kas teil oli mõni allikas? Te ei 
mõelnud ju küberpungi mõistet (mida Gibson ei maini) ise välja? Olid teil 
välismaa BBSid, internet?}

Tõenäoliselt kõik nimetatu klikkis kuidagi kokku. Ma ei oska 
Pronto\index[ppl]{Pronto} eest rääkida, aga \enquote{Blade 
Runner}\sidenote{\enquote{Blade Runner} on 1982. aastal linastunud Ridley 
Scotti film, milles peaosa mängib Harrison Ford ja unustamatu improviseeritud 
lõpumonoloogi esitab Rutger Hauer. Film toetub Philip K. Dicki samuti 
klassikaks peetavale, 1968. 
aasta novellile \enquote{Do Androids Dream of 
Electric Sheep?}.} on eepiline nurgakivi ja Syd Mead\sidenote{Sydney Jay Mead 
oli USA tööstusdisainer ja kunstnik, kes tegi lisaks \enquote{Blade Runnerile} selliste 
filmide nagu \enquote{Aliens} (1986) ja \enquote{Tron} (1982) visuaalid.} oli futuroloog, kes 
joonistas ilusaid düstoopilisi pilte. See kujundas meil välja düstoopilise 
arusaama tehnilisest maailmast, kus kõik on kõige külge ühendatav. Samuti
\enquote{Battle Angel Alita}\sidenote{\enquote{Alita: Battle Angel} on Robert 
Rodriguezi 2019. aastal linastunud film, mis tugineb Jaapani mangakunstniku 
Yukito Kishiro 1990. aastate sarjale \enquote{Battle Angel Alita}.}, mis mõni 
aeg tagasi kinodessegi jõudis. Ilmselt vähesed Eestis 
teavad seda originaallugu, ma olin toona selle totaalne fänn. Käisin 
aeg-ajalt Helsingis Akateeminen Kirjakauppas\sidenote{
Akateeminen Kirjakauppa on Helsingi kesklinnas asuv raamatupood, 
mis kõikvõimalike servapealsete huvidega eestlasi pikki aastaid 
raamatutega varustas.} pidevalt vaatamas, kas uus osa on tulnud. Mul on kõik 
üheksa 
raamatut olemas. Meile imponeeris, kuidas metalltorud lähevad silmamuna sisse 
ja ajust on järel ainult kiibid ja natukene putru. See maailm kujundas meid, 
elasime selle asja sees. Ja ka mängumaailmas olid paar mängu, mis andsid oma 
panuse. 

Me muidugi üritasime ka mänge teha, veel enne, kui me 
Bluemooniga\index{Bluemoon} \sidenote{Vt ettevõtte Bluemoon 
Interactive'i kohta lk
\pageref{sisu!bluemoon}. } üritasime Amiga maailmas midagi korda saata koos Ott 
Aaloe\index[ppl]{Aaloe, Ott} ja Juhan Soometsaga\index[ppl]{Soomets, Juhan}. 
Tegime Rocketsi-nimelise\index{Rockets} mängu, mille 
Bluemoon pani hiljem PC peale. 
Selle mängu intro oli selgelt kantud küberpungi V8test, rakettidest ja muust 
säärasest. Näiteks päikseprillide õige kuju oli kindlasti väga tähtis. Andrus 
Aaslaid\index[ppl]{Aaslaid, Andrus} 
rääkis sinna kõrvale lugusid, kuidas plinkiva valgusega saab 
aju ümber programmeerida. See kõik absorbeerus ja tekitas 
omaette alternatiivse reaalsuse, mis oligi meie arusaam 
küberpungist. 

\question{Kui teie tuleviku ettekujutuses oli ajust järel natuke putru ja palju 
kiipe ning tulevik oli muidu ka düstoopiline, siis miks te sellele vaatamata 
pika 
sammuga tolle tuleviku suunas astusite?}

Seda tagasi hoida on tõenäoliselt mõttetu, ludiidid ka üritasid. 
Parem on olla kohal enne teisi, et panna juba õiged kiibid endale 
õigesse kohta ära ja vähendada pudru osakaalu. 

\question{Et üheksakümnendatel selliseid mõtteid mõelda, oli tarvis korralikku 
visiooni. Räägi Bluemoonist -- kuidas sa Ahti\index[ppl]{Heinla, 
Ahti}, Jaani\index[ppl]{Tallinn, Jaan} ja tolle pundiga kokku sattusid?}

Toona pidas igaüks ennast superhäkkeriks -- nii see, kellel oli kastis 
MSXi \emph{manual}, kui ka see, kes oskas faile kokku panna. Ma olin selles 
nii-öelda superhäkkerite seltskonnas üks väheseid, kes joonistas 
pilte. Oskan ka ilma arvutita päris hästi joonistada, aga arvutis 
tundus see lahedam, sest seal sai asju salvestada ja ka
\emph{undo} teha, mis lihtsalt joonistades on 
väga raske, võiks isegi öelda, et peaaegu võimatu. 

Igatahes see seltskond ei olnud suur ja kõik \emph{connect}'isid 
kuidagi omavahel. Teen korraks kiire kõrvalehüppe. Meil oli telemajas ka asja 
vastu huvi,  kuna meil olid Amigad ja mänge pidi ju kuskilt tõmbama. 
Üks viis mänge saada oli leida ligipääs mõnda 
BBSi. Sinna aga ei saanud lihtsalt niisama sisse astuda, seal istusid mingid 
vennad, kes jälgisid
tegevust. Eesti tundus eksklusiivne ja veider koht, sama hea kui 
eskimod. Ühel hetkel tekkis meil isegi mingi \emph{trading capacity}, kui 
juba oli midagi, mida vastu pakkuda, aga tavaliselt mängisime 
vaest sugulast ja isegi 
\emph{bluebox}'isime\sidenote[][-.7cm]{Instruktsioonid 
telefonikeskjaamadele konkreetse kõne kohta liikusid samas kanalis kui kõne 
ise. Seega, 
tõstes toru ja vilistades õigeid signaale, sai muuta kõne teekonda 
keskjaamade vahel ja, mis kõige olulisem, saada mööda kõnetasudest. Kõvemad 
spetsialistid, nagu Joe Engressia ehk Joybubbles, suutsid kaugekõneliini 
lähtestamiseks vajalikku 2600\phantomsection\label{sisu:2600} Hz signaali suuga vilistada. 
Natuke nõrgemad, 
nagu John Draper ehk Captain Crunch, vajasid tehnilisi vahendeid. 
Lihtsurelikud kasutasid elektroonilisi seadmeid. Neist esimene, mille 
ehitas 1960. aastal Robert Barclay, oli pakendatud sinisesse kesta, sealt ka 
nimetus \enquote{blueboxing}.} ennast sinna sisse. 

\question{Kas see tähendas, et tuli
toonasele telefonikeskjaamale kõrva vilistada midagi, mida too tingimata 
kuulda ei tahtnud?}

Margus\index[ppl]{Kliimask, Margus} 
oli põhiline \emph{bluebox}'i spetsialist, aga kas me ka reaalselt 
\emph{bluebox}'imiseni jõudsime, seda täpselt ei mäleta.

\question{.EXEs ilmus selle kohta manuaal\sidenote{Mark Tabas 
(1993). Blueboxing parimates peredes. .EXE, 1-2. Mark Tabas oli legendaarse 
häkkerirühmituse Legion of Doom 
asutajaliige.}.}

Jaa. See oli üsna lihtne, kuna 
keskjaamad olid suhteliselt rumalad. Modemid seevastu olid kiired, meil oli 
kolmesajane Hayes. See oli vist
Mast\index[ppl]{Kaal, Madis}, kes oli väidetavasti suuteline selle modemi 
\emph{handshake}'i\sidenote{Enne, kui kaks modemit omavahel sidet hakkavad 
pidama, viiakse läbi \emph{handshake} ehk tutvumine: milliseid protokolle oskab 
keegi rääkida, millised veaparandusmeetodid tunduvad sobilikud jne. Kuna ka see 
kokku leppimine käib teatud kindlate sagedustega helisid tekitades, ongi 
inimesel võimalik lihtsamaid asju ka järele vilistada.} ära vilistama. See oli 
piisavalt aeglane, et
suusõnaliselt oma soovi selgeks teha.

Igaühel meist oli oma fookus: kes tahtis rohkem 
\emph{network}'i häkkida, kes lihtsalt häkkida, kes 
programmeerida. Mind huvitasid mängud, liikuvad värvilised 
pildid ja 3D. See viiski mind kokku
Bluemooni\index{Bluemoon} pundiga, kellel oli kindel soov mängu 
teha. Minu jaoks oli see natuke nõme ülesanne, kuna 
mul oli miljoni värviga Amiga\index{Amiga} ja neil 
EGA (alguses vist isegi CGA), kus tuli nelja värviga 
midagi valmis nikerdada. Miks mitte, teeme ära. Sellest sündis 
\enquote{Kosmonaut}\index{Kosmonaut}. 

Seda, kuidas nad seda mängu turustasid ja toimetasid, oli lahe vaadata. Nad 
olid toona ja on siiamaani väga pühendunud ja keskendunud oma asja 
tegemisele. \enquote{Kosmonaudi} graafiline pool oli lihtne: nokkisin selle valmis ja 
nemad tegid 
\emph{editor}'i oma muusika (kitarrid ja trummid), millele ma tegin ikoonid 
juurde.

\question{Arvutiga joonistada oskajaid 
oli vähe. Kas sa hakkasid kõigepealt arvutiga joonistama 
või oli sul juba enne joonistamishuvi?}

Joonistasin enne ka. Kes see ikka kiidab, kui ise ennast ei kiida -- 
akadeemilist joonistamist valdan suhteliselt hästi, nii et arvutiga 
joonistamise oskusi 
ei olnud keeruline omandada. Toona ei joonistatud tahvelarvutiga,
vaid hiirega, mille muna jooksis aeg-ajalt pahna täis ja seda tuli
küünega puhastada. Nii et minu jaoks ei ole vahet, kas joonistada
pliiatsi, hiire või tahvliga.

\question{Kas arvuti oli seega sinu kunstitegemise laiendus?}

Jah, see oli lihtsalt teistsugune tehnika ja tunduvalt andeksandvam 
kui näiteks akvarell. Täna kasutavad kunstnikud 
Cintiqut, kus on ägedad tööriistad. Ma kasutan siiamaani aeg-ajalt oma 
sülearvuti \emph{touchpad}'i , 
kui mul on vaja midagi nikerdada, ja teised
vaatavad, et olen peast soe. See on mugav ja käe järgi tööriist, kui see 
omandada ja kursor piisavalt kiireks keerata.

\question{Kas ühel hetkel tekkis sul mõte hakata veebi tegema?}
 
See juhtus pigem tänu arusaamale, et ma ei ole piisavalt 
järjepidev ja programmeerimine tundus toona liiga kuiv. Mu sõbrad 
tegelesid sellega ja tulemus ei olnud seksikas. Veeb oli alguses ka super 
\emph{boring}, esimene brauser Mosaic oli \emph{ugly 
as hell}. Aga kui sain aru, et tabelitel saab ported maha keerata ja 
üksikute ühepiksliste tükkidega asju paika panna, 
olin müüdud mees. Sain pildi oma käe seest mitte ainult
ekraanile, vaid pauh kõigile nina ette. 
Töötasin ühes reklaamibüroos ja seal ma katsetasin. 
Mindworks\index{Mindworks} oli juba olemas ja selle asutajad vaatasid, et jagan 
natuke reklaamibisnessi ka ja et võiks seljad kokku panna. Aasta 
oli 1996 või 1997, olime umbes kahekümneviiesed, mitte enam
poisikesed, ja saime juba \emph{business}i teha.
Meil oli kliendiks näiteks Reval Hotel Group, kellele tegime ägedaid asju.

\question{Kas sind liigutav faktor oli jätkuvalt see, et said oma pildi 
inimeste silme ette?}

Tegelikult mul ei olnud vahet, kas inimeste silme ette, vaid luges
see, et oli mingisugune distsipliin, HTML. Teadsin, kuidas 
optimeerida GIFe, mul oli \emph{toolset} ja valdasin seda 
suhteliselt hästi. Tekitas rõõmu, et sai teha asju, mida 
teised võibolla ei osanud teha. Tähtis oli \emph{job satisfaction}. Kujutan 
ette, et muruniitmine on ka selles mõttes lahe, et näed, kuidas 
niidetud muru jääb taha maha.

\question{Mida sa praegu teed?}

Olen disainerirollist liigselt distantseerunud, aga samas ka
mitte. Ajapikku olen aru saanud, et pildi tegemine on mõnes mõttes 
käsitöö, veidi sarnane lõikelaudadega, mida minu lapsepõlves 
turul müüdi ja millele oli põletiga kirjutatud \begin{russian}ну 
погоди\end{russian}. Palju šefim on mõista
äriprotsesse ja inimeste mõttemalle ning 
disainida nende põhjal midagi. Programmeerimine on minu jaoks see, et keegi 
teeb selle valmis ja see muutub 
päriseks. Protsessi toetav asi masina sees toimetab täpselt nii, nagu 
sellele on öeldud. 

\question{Kas nüüd ei tule käe seest pilt, vaid pea seest mõte, 
kuidas miski võiks välja näha, ja programmeerijad teevad selle valmis?}

Just. Lapsepõlves võlus mind mõte, et tehas on 
tore asi, sest seal pannakse toormetest ja detailidest asi kokku. Ringiga 
tagasi Nintendo \emph{device}'i 
juurde jõudes võib öelda, et füüsilisel kujul seda toota on tüütu. Palju 
lihtsam oleks teha 
seda nii, et oleks jada bitte, mis kõik grupeeruvad ja moodustavad 
mustreid ning sellest tekivad peaaegu nagu võluväel asjad, mis on
tänaseks inimeste jaoks sama reaalsed tööriistad kui haamer ja höövel.
