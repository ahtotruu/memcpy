\index[ppl]{Loit, Kaspar}
\index[ppl]{B'Knows}
\index[ppl]{B'Knows|see{Loit, Kaspar}}

\question{Kes sa oled?}

Mina olen Kaspar. Ja kunagi, kuna see võtame selle teema, et me peame tagasi kerima mingisugune miljard aastat, siis toona see aka oli B'Knows. 

\question{Aga kust sa said sihukese aka?}

Seda ei mäleta enam keegi. Seal on nagu kaks komponenti. Üks on nagu \enquote{B} ja siis on nagu \enquote{knows}, ehk siis see B peaks  midagi nagu teadma. Pronto\index[ppl]{Pronto} alati kutsus mind Buttknows.

\question{Kuidas sina arvutite juurde said või arvutid said sinu juurde?}

Mul on selge mälupilt, et mu tädi, kes on superkuul ja minust mõnevõrra vanem, töötas Tartus
vist Bioloogia Instituudis. Ja  talle oli kuidagi jäänud mulje, et mind võivad huvitada sellised asjad. Ma arvan, et ma olin mingi, ma ei tea, kaheksa-üheksa-kümme, \emph{something like that}. Ja siis, kui ma tal ükskord Tartus külas käisin, ta viis mind instituuti. Muidugi peale tööd, kõik oli juba pime ja  seal oli mingi kabinet lahti ja seal seisis laua peal mingisugune masin, mille nimi oli Apple II Europlus\index{Arvutid!Apple II}\sidenote{Apple II Europlus oli Apple Euroopa turule kohandatud versioon. Muu hulgas erines toiteblokk aga ka video osa tuli ümber teha, sest Steve Wozniaki trikid NCTS signaali genereerimisel ei toiminud enam keerukama PAL süsteemi puhul}. See oli nagu legendaarne selline nagu fakinossem. Ja, ja, ja seal sai, ta oli seal mingi laborandi käest küsinud, et kuidas seal midagi käima, enne seda üles kirjutades sai mingi plaadiga mingisugune maine, sellel paar mängu, mis olid teksti ekraanil, jooksid ülisuper ägedad. Ja eks see oli vist mingi trein Roveri, ma mäletan, ma olen seda asja, see pilt silme ees. Ja sellest hetkest ma arvan, ma olin, möödus ka, et ma, ma ei oska nagu meenutada, kas, kas ma enne olin kokku puutunud juba seal? Tõenäoliselt mitte, see oli ikkagi liiga liiga vara ja, ja ma arvan, et selliseid, esimene. Ja, ja kuna mul on mulle tohutult meeldisid koolis Nintendo väikseid Keymengu otsmängud. Võib-olla mäletate, kuidas ei, ei mäleta. Ühesõnaga siuksed, ma ei tea, tänapäeva telefonisuurused umbes. Ja, ja see oli LCD ekraan, millesse oli nagu ette programmeeritud ette joonistatud mingisugused tegelased ja siis nuppudega näoga täna kaugele neid noh, Janno paha nii, ja, ja kogu selle aja eest. Aga Nintendo oli see mingi põhiline tootja neile, neil olid siuksed väga-väga lahedad mängud ja siis ma või isegi kunagi mõtlesin, et oh, kui lahe oleks, kui saaks ise niisuguseid teha, aga no ma sain aru, et seal taga on mingi tootmine ja see ei ole nagu reaalne on ju. Ja nüüd järsku saavad aru, et tegelikult selliseid asju on võimalik sinna noh, nagu masina sisse programmeerida, firmad seda minid elektroonikaskeemi tootma või mingisuguseid noh, muide see võib üldse mingit tehast olema, et see oli nagu see oli üks niisugune realism on ju see muidugi noh, tõenäoliselt viis mind kunagi ka sellele, et me vaikselt tegi mänge. Aga niisugune päris toimetamine hakkasin tööle. Kuskil oligi seal Jaak Laande nimi, Eesti Põlev jookseb ta kindlasti nagu ülioluline tegelane, sest et sellistel noh, siis oli selge see, et arvutile ligipääs oli see asi, mis oli nagu oluline ka valutaks. Ja, ja ma mäletan, et ma just siia sõites tuletasin meelde, et ma tegelikult olin kaardistanud omale kõik kohad, kus üldse nagu tõenäoliselt Eesti Vabariigis arvutite ligi pääses. Nõo oli liiga kaugel selgelt. Aga Tallinnas neid kohti, noh, need ikka olid. Aga Jaak Londoni nagu selles mõttes oli nagu lahe tegelane, et minu meelest ema kaudu ma esimest korda sain progeda. Kas te siis nagu andis selle võimaluse või ta õpetas ka või toika nagu õpetas loomulikult, sest et ega alguses oli lihtsalt mingi mingil põhjusel ma ei tea, kuidas ma sattusin sinna selline neljas keskkoolist või, või kolmas kurat, ja see oli see suur klass, seal olid mingisugused masinad, tõenäoliselt ka need olid emmessiksid. Ma arvan. Ja, ja see kari tegelasi paar tükki, ma tundsin nende kaudu vist kuidagi nendest jadad. Ma mäletan, et üks mu koolikaaslane nägi siukest masinat esimest korda ja meil öeldigi, et natuke midagi tegema selle teisikali esseevõistlusel lõksid ainet emmessiks selle sellepärast tuli üldse pull masinaid ta tegelikult vist mõeldi välja tagantjärgi tarkusega võin öelda selleks, et ühtlustada koduarvutite standardit ja peidikuid, milles nad jooks, mida nad jookseksid ja see oli isegi tegelikult algatatud initsiatiiv mingisuguse Jaapani Microsofti executive poolt, okei butis Peisikusse lihtsalt otsa. Saab võistlus katkestati pea niimoodi, et ma eile just ostsin välja omale ühe ühe emulaatorit tahab selle ekraani ette ja sa võid seal kirjutada kümme ja siis kirjutada reha, paks kerkivad teisegi, antud listist näitab, mis sul on, sa kirjutad uuesti kakskümmend, kirjutad selle rea üle ja, ja kirjutada on nii, et, et see tähendab, see on nagu su kohe käitma. Mõnus. Ja siis ma mäletan lihtsalt seda ka, kuidas mul võiks sõber, kes läks sinna, nägi seda asja esimest korda läinud, et nüüd siin saab midagi teada, kirjutas Liis deroomiiesse.
Ja noh, ütleme, et NLP Janar veel, nagu ta on, äkki on seal midagi tunud?
Aga ja siis seal üldse nagu tegelikult paljudes Karel Kannel oli seal kuidagi toimetas ja, ja kõik need väiksed tattnokad õitseksid Kaasani. Aga palju huvitavam oli tegelikult see, et Jaak Loonder oli ka üks masin, nimi oli Mir, kaks mingi nõukogudeaparaat, mingisugune nõukogudeaparaat, mis oli noh, siukse põhimõtteliselt oli ikka pikem kui viis meetrit oligi pikap poisikesest kõrgem Marje ja Jaagu leida võib-olla ninani.
Ja Tein meeletute häält, sest et oli nagu noh, põhiline osa ilmselt oli jahutuseni. Just seda tegigi villa kraažinis masinaid ilge müra ja siis aeg-ajalt, kui inimestel viskas nagu Kopli ette, siis nad lülitasid selle välja selle jahutuse ja siis see oli, tšekk tuli muidugi tohutult rääkis sest, et see kokku ja ilmselt on. Ja, ja see oli, selles mõttes oli ka nagu geniaalne vaim, kus ta seal üldse kätte saanud, see oli nagu nad lahendust, lahendust, masin, tal oli ikkagi võimalik klaviatuurist sisestada käske kus klaviatuur oli elektronkirjutusmasin, mis need põhimõtteliselt oli nagu klaveriprinter ühekorraga. Nii et nagu kaaned kapitaliga sellest samast masinast ja siis tal oli
Mustvalge telekas aga tal ei ole, kas valguspliiats ehk siis sa said nagu ekraanil tabada mingisuguseid punkte ja see masin tundis selle ära, ilmselt ta luges seda kuradi kineskoopkiirt ja selle järgi pani selle kokku. Ja, ja ta suutis ka mingisugust Rudimentaarselt mingit graafikat, Kuvalehti tal ei olnud nagu ainult teksti ekraan, vaid ta suutis ekraanile kuvada mingit punkti, see oli selle arvuti, eks niisugune tohutu ülesanne see punkti püsinud hästi paigaldas, ikkagi õrnalt ujus asjadega hakkama. Ja siis mu esimene programm, ma mäletan, oli, see oli muidugi see programmeerimiskeel oli vene keeles vene tähe, et kõik olid mingid lühendid super lõkseni ja, ja siis ma progesin mingisuguse graafilise neli tipulised tähele, sest ma arvan, et ta koosnes, ütleme siis Megist umbes kuueteistkümnest punktist tulla ja see ikka tõmbas selle arvutiga täiesti koomas ja kõik see ekraan ujus selle, aga väga uhked. Aga meil on veel lisaks veel, meile õpetas, oli näiteks polindi lugemist nad seda masimiseks õi kahte moodi meediat, üks oli paber, perfolint, siuke õhukene, mis lasti läbiteooriasse džäki, kotid, teravamad vennad suutsid torkida nõelaga augurauaga oleks võinud tegelikult seda perfolindi peale seda proge kirjutada, ma arvan.
Mul on selline tunne, et äkki oskasin.
Aga aga siis olid seal veel mingisugused vahvad asjad, mingisugused magnetkaardid, mis olid umbes sellise magnetkaart, jah, ta nagu perfokaarte ma küll tean, aga ma usun, et magnetkaart oli selline jälle niisugune umbes tänapäeva või tänavale telefonist suurem, noh niisugune mingisugune. Ma arvan, et mingi kaheksa senti korda mingi viisteist senti sihukene troon latakas, mis on põhjust meenutada oma materjalid, seda, mis floppy diski sees on tegelikult. Ja siis on põhjust, panid selle mingist latist sisse, tõmbas seda solisti läbi ta luges sealt midagi mingisuguses koguses hästi, läks ja luges keelde. Ja see oli nagu noh, siuke sel juhul müstika seda enam lugeda ei saadud, see. Aga see niisugune noor nagamannid, see peab olema ikkagi päriselt märksa tahtmine, et sellest ju ekraani peale tähejoonistamine, huvid oleks või see nagu mästidesta ja sellest tõesti sa, sa nagu kirjutasid midagi ja see pilt tekkis seda sinna ekraanile. Siis see huvitav oli just see mina andsin käsu, mis ta tegi midagi või? Noh, see olnud oleks olnud noh, täpselt, et kui see oleks nii lihtne, et plii, Stroomi eeskõneleja, siis see ilmselt oleks kaotanud huvi, aga see oli ikkagi kombigeiti värke seri. Selles oli mingi algkeemiline element, selline ülikõva ja seal oli nagu niisugune noh, mis maistele poistele meeldivad igasugused Salagi keeled ja igasugune koodid ja ma ei tea lipukirja Eestis asjad, et noh, selline nagu see oli kõike seda. Ja veel, mida see oli ikkagi super hästi kukuvad ja, ja, ja sisena jällegi, et sealt tekkis, aga noh, see noh, selge oli see, et see oli nagu meeletult piiratud on, et kaua sa seal ikka seda seal ja, ja lisaks sellele, et seda merre, et üks noh, õnneks emmessiks, las see suurem, aga seal oli vist jälle midagi mingisugused piirangud, kuna see oli keska olla onju ja, ja ilmselt sellepärast säpizzewiski. Ma arvan, et see oli kuidagi ega vist säkiga seotud semis roopa tänavale selle võttega, mida siin ka teistest lugudest läbi jooksnud minema. Ja seal oli siis ka terve klass.
Kus siis oli üks niisugune nagu juhtarvuti, millel oli mingisugune draiv? Ma eeldan, et see oli mingi flopidraiv. Ja, ja siis oli terve klassides arvutid, mis said sellest nagu peaarvutist omale alla laadida asju, isa füüsika, eks ole seal lihtsalt kirjutada oma neid programme, aga kuna kuna Trai oli ainult üks, siis kui sa tahtsid salvestada või midagi, et siis sa viid selle sinna, saab vaid siis tavaliselt see asi. Tegevus oli üsna nagu lihtne, et.
Kõik tšekid ja aeglasel väga nagu hängida on ja siis oli seal kogu aeg oli mingi poiste karjed olid talle lisaks bar nutikamalt vend olid siis nad pannud seda vedama. Üks legendaarne tegelane Emmucats ehk linnade edesiis tänasest kuskil Soomes toimetamine. Aga noh, tema oli nagu selgelt minu esimene nagu niisugune guru, keda ma nägin, et ta oli, ta oli, ma arvan, oli umbes minuvanune, aga ta oli omandanud juba täiesti kõik need peenemad alged, ehk siis põhiline, mida ta oskas, oli see, et oskas kahest programmijupist panna kui ühe terviku ja selle nagu paketeerideni eestlase laadideks. Point oli selles, et väga suur osa seda softi levis tavalist magnetofoni kassettidele. Ja vist oli see kuidagi, ma ei tea, kes oli võrguprotokollist kinni või sellest kas endist kinni või enam-vähem niisugune šehhivad mängud olid kolmkümmend kaks, kilobaiti umbes pikalt, noh see oli ka mässimine. Ja, aga selleks, et nad sinna ka sätivad ära mahuks, nad olid tehtud pooleks kuus kas kuusteistpidi vahepeal nagu keerama ja ühesõnaga see kogu see kasseti majandusele keeruline. Aga, aga kui seal oli juba nii kõva asi nagu flopidraiv, et siis sa said sellega seti pealt lugeda selle kuusteist kilo sisse siis tõsta kuskil mälus mujale ja siis lugeda teise kuusteist kilo, saanud kokku panna asju, tekkis see tervik, mida sai nagu Rootsi või kuidagimoodi, ühesõnaga see oli kõik täielik, seal juba seal juba supermaagia siis oli seal tegelikult tekkis sul siis niisugune teadmise põhinenud nagu eeskuju, see oli keegi inimene, kes, kes vaatasid alt ülesse sellepärast et ta teadis rohkem kui sina, oi, ealiseks tegelasi veel, seal oli mingisugused, ometi võiks ainult käivaid Kont, toimetuste eesnimi oli ühtlasi ka natukene tüsedam lendude, tal oli väikene kohvrikene metallkohver mille sees oli kogu emmessikse manual, see oli fotokopeeritud käes täkazzbeeber, siis ta käis sellega renni väga uhked saajat, et see lahti siis seda midagi selle alusel kirjutas see nagu jälle superlux. Eks tegelasi oli veel seal. Jaa. Ja ega mul seal ka nagu selles mõttes ongi, et kui on, sa istusid seal seal nagu otsest läks, sassi sellele draivi ei olnud, võib-olla seal seal mängisid, mingeid tüütas ära, on ju sisse kirjutasite sama teisikut. Ja sellele tuli mingid mingid piirid ette. Loomulikult miljonites graafika pool. Ma üritasin sinna midagi mingisuguseid pilte manada ja kuna emmessiksil oli tegelikult tal ei olnud graafika ekraani ja seda aimuleeriti tekstiekraaniga, ehk siis põhimõtteliselt iga pilt, iga mäng, mis emmessiksin toona jooksis, oli tegelikult otse mälus tahe generaatori ümber programmeerida. Juku ja juku tehti sama lugu, et seal said laadida mingisuguse oma nii-öelda fondi kuskile mällu. Ja see oli ikkagi tähed, olid mingid, panin siin teha asemele pit, näppisin ennast või mida iganes kokku võtta, et põhimõtteliselt sama sama laks, et seal oli noh, ekraan põhimõtteliselt emmessiksil oli otse aadresseeritav, et kui sa teadsid, et nad selle režiimi ega algas aadressil, seal mingisugune eksas mingi jõuate värvata veel on ja siis sa lähed sinna järjest panna iga, iga bait oli üks rida on ja tal oli, kas tal oli läbipaistev taustavärv või esivesivärve, siis veel mingites režiimides said need iga rida-realt neid värve vahetada, selline Välitel väest. Ehk siis põhimõtteliselt niisugune mõiste nagu ma ei tea, kas sa ilmselt oled kokku puutunud, et mängudes on siuksed tegelikult nagu Spraynyden ja need on need, mis liiguvad, eks ole, tausta ees. Et MSI CD-lt spaitega emuleeri sellesama tähe genega ehk siis kohalik tähe Gene Programmeeriti jooksvalt ringi, eks ekraan kirjutati täis ABCD, ehk ei ole ma nädal aega kõik märgid, mis talle pähe tulid, mis olid, on ja, ja siis need kogu aeg adresseeritud ja kirjutajate ring ja selle jälle iga ekraanist oli alati üks mingis kolmeks osaks. Igas osas sa said eri eri nagu tähestiku väänata ja, ja see oligi nagu see minu jaoks oli võlu, kui ma sain selgeks, et on olemas Assembler see Marissa, plaanida ühele mälu aadressile üks Payton ja ja siis ma mõttelist veetsin suure osa oma ärkveloleku ajast millimeetri paberile, joonistades mingisuguseid tegelasi neid tõlkides paineriks või eksakse siis laanide kuskile mällu. Aga kui ma nüüd refereerin tagasi, siis see ega, ega tänapäeval arvutigraafika ei ole ka lihtne. Aga see keerukus tundub olevat nagu teises kohas, et tänapäeval sa pead aru saama, sest kolm t/a, kolm tee geomeetrilist ja sa pead nendest spetsiifilistest kepihoopidest teadma ja nii edasi, et noh, seal on ikkagi leierzazdafan seal, vahel ka sellele, kas sa põhimõtteliselt võiksid ja panin nagu toorendiselt toppisid otse ekraanile selline, aga kui see sinna toored toppimine, et ega see ei olnud nii et sa ütlesid talle nüüd joonista sellesse kohta mingisugune asi, vaid seal pidin ikkagi tegema allikast algoritmilise tööd. Et noh, mis, kuidas ma seda värski maja ees ja nõia Graszek nagu hea, et ta kähku majakaks ja nii edasi. Noh, ütleme, et seal oli igasugused trikid, et, et ta nagu töötaks, aga, aga kuna see, see keelest ja kõik, see asi oli nii lihtne, siis oligi nagu super elegantne, super nagu lihtne ja ma arvan, et ega minu progemis aeg jäigi sinna kaheksaga.
Keskel pärast ma olen võib-olla natuke mingist mingisugust HTML ja võib-olla see SS-i nokkinud, aga tegelikult ega ma sellesse Võru oleks nagu üle kohe, kui tal läks nagu, nagu läksid nagu keerulisemaks. Aga õnneks tuli tasemele igasugused graafikapaketid ja, ja muudes, aga millal see oli juba keskkooli ajal? No seal jah, sealt ütleme olles enda jaoks kaardistanud ära kõik kohad, kus sai midagi arvutitega näppida, jõudsin ma läbi. Väga tähtis on kusjuures see, et tippis oli ka üks klass, kus teeme siis. Aga seal olid juba igal masinal oli Drive tava. See oli ka juba super, et vahest ja, ja seal oli ka muidugi seal guru Guru staatus oli nagu juba järgmisele levelile, seal olid mingid laborandid.
Ühe nimi neli koma ekselgiga. Ja ei tule enam meelde. Aga aare, tali tuleb kuidagi ette ja ma ei ole kindel, kas see on õige nimi õige näo ees. Aga igal juhul olid seal mingisugust juba üliõpilased on juba nagu kõvemad vennad või isegi oma mingisugune maine postkäädavat või mis iganes ja, ja, ja loomulikult see nabade ja jada seal ukse taga neid selgelt tüütas ja siis nad seal tegid omaette mingisuguseid reegleid, olid suured jumalused ja näiteks mingi hetk oli, kui neil juba olid täiesti noh, infoga leviks, eks ole, et järjest rohkem Kunder äkki see on ju sinna värava taha ja ja siis, kui oli vaja reglementeerida, et keskmise, seda saab ligi, siis nad võtsid ühe kõige popima mängu, mis seal parasjagu oli. Kinks, väli, trükise välja kogu selle soos kooli. See oli nagu täkkov seda perforeeritud paberit ilusti nadid. Ja siis nad otseselt meie naha tavaliselt punase pastakaga ringe ja progesid selle mänguringis selguvad täiest seal juba minu jaoks oli see juba nagu Jumal, see tase. Ja, ja nad progress niimoodi ringi, et nad said vaikselt iseehitatud Sostikudega juhtida neid selle mängukolle. Ja, ja siis põhimõtteliselt oli umbes. Reegel oli see, et kui sa said nende jumalate vastu ühe leveli läbi siis sa said selle ühe päeva käia. Ja teha seda jah, jällegi see iseenesest mängufaktor ja see kõik oli põnev, oli see, et nad tõesti nagu nad võtsid selle mängu, mis minu jaoks tundus nagu superkeeruline nagu ja, ja lihtsalt nad kirjutasid, tuleb aina ringiga, et kirjutasin, mitte lihtsalt ei, ei teinud seal mingile tegelasele mütsi pähe, on ju vaid nad lihtsalt nagu tegidki, kõik ringid, käitumine muutus ja tänapäeval muidugi tagasi vaadates tundub, et see tegelikult
Väga lihtne.
Aga noh, see on seesama see kelleltki, kes ja mis praegu nagu üles tõuseb ja mille peale kõik asjad ehitatakse ja ja pluss see töötas veel ka ju sellel tasemel. Et see on just nimelt tõust testis näed veel nagu nende petnokkadel rivisi, mis veel nagu kõrgemale apteegi veel ihaldusväärsem sinna sissesaamiseks absoluutselt. Aga igal juhul lõppkokkuvõttes lõpetasin kuskil Kullos, kus oli ka üks klass, kus olid vist juba natukene kõvemad emmessisid. Seal oli juba nagu mingi graafika reziim ja, ja igasugused muud asjad, ehkki, ehkki ütleme, et kui seal midagi kahtlast, mis kiiresti liigutaks või riike kasutama seda teistega. Ja, ja seal oli selline legend nagu ränimeister, kes seda seal nagu majandas kes on üks tore toona, ta oli selgelt sihuke tore punkar, kes oli tulnud kuskilt Volgaga gaasianalüsaatorite tehasest temast ja seal vits poistega jahmerdada, aga ta vaikselt seal hakkas tegelema Komoderamiigadega mis oli juba sihukene, Superad väest. Raud. Ja, ja, ja kuidagi tali seda kõike ei seksivideo tootmise ja, ja sellise asja kuidagi tähe all. Ja tänu sellele oli ka loomulikult siis on välja arvestanud, et kus, kus niisugune asi veel toimub, on ju, ja Eesti Televisioon oli selgelt üks ja siis oli mingisugust vene metalliärikat. Kuidagi niisugune turundusharu oli tekkinud Eestisse ilmselt keegi vend oli piisavalt palju lobi teinud ja tal ei ole muud teha. Siis ta oli kuskil Kristiines kuskil keldris oli püsti pannud väikse mingisuguse nii-öelda reklaamistuudio kus ta siis tootis märki ja tal oli seal ka, eks saviga. See pidi olema siis üheksakümnendate algus, eks ole, jah, kuskil sealkandis. Aga ühesõnaga kogu sa selle Kullos nikerdamine sama jaoks või ka mingeid mänge tegema ju võimalus seksida ja Marcus klass Mandel toimetas ja, ja, ja. Raul Keller, kelle alkoholi killer? Ja, ja seal me midagi seal noppisime, isegi ta üritas seal mingisuguseid emmessiksi mängimist nagu pubitsee midagi, aga, aga see tundus kuidagi ikka väga niisugune kahtlane ja, ja naiivne tegevus mulle vähemalt toona, aga siis juba räni kuidagi nähes minus potentsiaali meelitas mu Eesti Televisiooni ja siis olin põhimõtteliselt ma olen isegi veel kestva viimases klassis, ma töötasin juba Aktuaalses Kaameras ja uudistetoimetuse kõrval oli siuke väike Kubrick. Kus me siis tegime Aktuaalse Kaamera infonurki, mis selle diktori taga oli nagu seina peal ja kuna abiga oli selline tore masinat sinna see lasta videosignaali sisse, sealt ise sinna välja, et sa said ega teha nagu tiimixi vist juba toona, eksole, kus on see ju, see oli ju jõhkralt kallis riistvaraga piisiide peal, et enam Keilasse see oligi, et Abigail oli, miks, miks need Amingat siinkandis selles vallas levisid, oli just see, et noh, PC jaoks mingisuguse tasemega videokaart seal ikka mingi, kas see oli mingi hollivuudi või, või sellise tasemega asjani ja, ja need masinad olid ka neil oli, eks ole, mingisugune Seega ja, ja neli värvi on ja samas kui amiga oli nagu Full videos on, põhimõtteliselt võisid talle panna selle võrra isegi arvuti monitori see võistkonna teleka taha ja see toimis ja see oligi ilmselt see point, miks tal oli videosignaal, et oli nagu seks, sest nagu kodukodutarbimisest arenenud selliseks nagu ja, ja seal tegime oma ilmakaarte ja panime need videopilte sinna ja põhilise osa ajast muidugi mängisime arvutimänge, sest jälle Amigo super šefimängudeni. Millega te tegite selleks teiega teie nullist ja kirjutanud kogu seda ka. Öelge, kas seal oli olemas täitsa viisakad graafikapaketid, De Lux Pent on nagu šefimaid graafikas ohte, mis oli nagu igasugustest asjadest, igasugustest Photo soppidest ja kurat teab millest ikka. Kümme aastat enne tuli. Ja, ja me siis ainult noh, me olime nagu selliselt nagu India nõu abiga vennad ja me vaatasime ikka kõikide pisi muude mängude peale ikka ülevalt alla, sest et nad ikka ei teadnud, milles nad seal Sarkisideni paraku lihtsalt saviga pisesena oli, oli kehva ja ta lõpuks jooksis, läks nurja ja aga iseenesest see tehnika oli nagu superäge. Ja seal meil tekkis mingi väikese kohaga selline punt tegelastest, kellel oli, kas oli kodusaami ja/või kuidagi tegelesid siis noh, Martin Rinne, kes täna teeb, direktor on ju tema juba tulid tekkis sealt kuidagi sinna telesse ja siis Margus kliimas Marx samamoodi oli seal, tema tegeles Eesti videos Siilatsi kuidagi sellele, noh, kõigil oli nagu mingisugune Äkses ja, ja siis jällegi loomulikult seal ka võlus mind pigem see, et et sa midagi seal nikerdasid ja sa tekitasid mingisuguse elava pildi sees selle pidanud olema minikaamera ja näitlejad ja mingi asi, sa võisid teha sõna, mingeid väikseid, mingid animatsioone või animatsioonide vedas välja. Noh, selles mõttes, et sa said seda teha põhimõtteliselt stop mausseniga. Noh, ütleme, et seda niisugust animatsiooni reaalajas
Bussiga Nõgisema juba ikkagi reaalajas täie, mis aga ei, me tegelikult ikka tegime tele, tegime reaalajas ka, sest keegi viitsinud stock mossega vastane, aga põhimõtteliselt sai seda teha ka Stognoosiniga, aga minu meelest isegi enamus sellised asjad käisid ikkagi reaalajas, et tal olid juba sellest eluks peenraid sisse ehitatud igasugused nutikad asjad, näiteks nagu liikumise aeglustamine või kiirendamine andsid talle põhjust ette, et siin on sul mingisugune see kast see kastab liikuma mingisuguse viiekümne kaadriga siia, siis ta automaatselt täitis need viiskümmend kaadrit ära. Ja vajadusel, kui sa ütlesid, et siin, siis ta tõmbas lõpus hoo maha ja kõik oli väga-väga fain näiteks. Mativeermet sellist naist, kes, kes pärast sellist naist Tallinna linna mingisugune disainer või, ja, ja ma mäletan, kui ma olles läksin sinna teles ja selle järgi võib muidugi mingi aasta paika panna ja tegime öölaulupeole, tegime mingisuguseid valgus, Kippe, et, et see oli juba ikkagi jällegi superväest, mäletan, et mingisugune nagu asi oli nagu televisioon, Jajah, seal seal olid väga ägedad ja, ja see oli ka nagu kogu teletegelikult siukseid asju tegime, sest et alternatiiv oli tiitri masin, eks ole, mis oli mingi räme puit ja mis oli nagu ikka eriti.
Oleme jälginud.
Sihukene pool analoog ja siis meil oli niisugune super, et väest animatsiooni ja värviline ja seal sain teha mida iganes ja siis me tegime seal vaid vaest Tegime mingit haltuurat mingite reklaamide jaoks ja igast lollusi rist puhas ning iseõppimise värk või kuskilt hakkas tulema, mingit informatsiooni ka juba ei, see oli ikka puhas iseõppimise teema, et selles mõttes, et oli noh need, need vahendid olid suhteliselt piiratud ja ega seal midagi nagu väga keerulistel kunagi mingi hiljem tulid ka, mida sellise koldepaketid sellega sai seal, kus sa rõhutad, meil olid noh, jällegi see tasemete vahe, et on ikka hoopis teine, et seal sa pidid ikkagi mingi punkt ajal konstrueerima mingisuguseid kolm teeb hindu ja siis neid seal kuidagi opereerima, et et tänapäeval vaatad, kuidas väänatakse mingit pump, Mäppinguid ja mingisuguseid asju pleierite kaupa ja siis see kõik kuidagi tuleb välja, et see on nagu täiesti müstika. Mis, mis sa siis tegid, kui sa Eesti Televisioonis enam ei olnud, sest ühel hetkel sa enam jootmist? Jah, seal oli kuidagi tundus, et see videograafika oli nagu väga põnev, aga tundus, et kuna siis oli, hakkas tekkima niisugune nagu Business. Et siis niisugused sõbrad, kes kuidagi olin rohkem sattunud trükigraafika peale, kes seal kujundas Eesti Ekspressi ja kes seal tegi midagi, et see tundus nagu kuidagi nagu rohkem piinas. Ja siis ma kuidagi sattusin, sain aru, et ahah, et videote abi, aga, aga selle sellele Business juurde peaks nagu pisside peale ennast kuidagi sebima ja siis sealsamas telemajas kuidagi tekkisid mingid potensiaalsed kliendid ja, ja ma pidin hakkama tootma mingisugust kujundust, mis on nagu trükikõlbulik mujalt kunagi näinud sellist programmi nagu korraldro on, mul oli see vaja nagu ära teha, siis ma istusin mingi öö läbi, tegin endale selgeks sõiduautot, frustreeriv, sest et ta oli täiesti teine maailm. Ja, ja tänapäeval on ikka see, et saab joonistada, siis see pilt on nagu ekraanil, mida see joonistada, aga siis oli niimoodi, et seal midagi konstrueerisid mustvalgelt mingisuguse vektormessi. Sa panid sinna mingid värvid peale ja siis vaatasin Bregioodist teisele joonistasin aeglasem sa oled. Siis sa läksid nagu uuesti selle kallal, kuidas me selle kohe gruppi, mis oli siis, kui mina üheksakümnendate keskpaigast mäletan korral troon, siis see, see rajakas kipun töötama. Ta tegi mingisuguseid asju nagu tennispalli katki ja lihtsalt, kui Sa salvestad selle valesti ja selles mõttes sa arvestasid ju noh, olles kasvanud, arvestada joast, harrastasid nad aeg-ajalt jooksid kokku ja aeg-ajalt nad aga võib-olla jälle seal mängis ka natuke see, et selles poisikesepõlves selles mõttekas õpitud arvutist ikkagi üleolek läbi ühe lihtsa fakti, et emmessiksil oli paremas nurgas oli port, mille sisse käis kas siis kettaseade või mingi mälukaart vits, pisikene, pisikene, üpriski suur sahtel. Ja nüüd selleks, et mitte seal midagi asja tuksi keerata, siis selle kvartalisse sisselükkamise hetkel seal sees oleks väike lüliti, mis tegi masinaga sätti. Ja loomulikult selle lapiti kiirelt ära, et selle asemel, et Poola väljas selleks, et teha mingi kord, kui sa oled midagi tuksi keerab, näiteks olid kirjutanud programmi, mis loopima ainesse, panid kohe nagu näpud sinna auku oli masinalis surnud, onju ehk siis see kontrollmasinale oli sellest, et selle ühesõnaga, et vaatasin, teadsid, et mingi valemiga saad jao palju tegijaid ülemasinast jah, et see, see teadmine on olnud ma ka siiani, et ma alati tean, et kui ma kuskilt heinast ikka lõpuks juhtme kätte saame nüüd siis on ta surnud on ju, võib ühendada, pelgab.
See on hea teadmine. Selle koha peal ma nüüd pean andma järgi kihule ja ja lõpuks küsin, ma arvan, need küsimused, mida ma väga tahan küsida, me jõuame Maiko ringi ja punkte, eks see, kuidas sa sinna jõudsid. See oligi selles mõttes, et kui ma olin juba selle prindiga alustanud, on ja ja, ja siis ma vahepeal kuidagi sattusin mingisse
Niisugusesse maailma, kus, kus nagu oligi nagu print, oli niisugune asi, millega ma tegelesin ja, ja kuna ma olin vargusega varem suhelnud seal televisioonis ja tema omakorda suhtles selliste tegelastega nagu lõvi, kes on muidugi kõige olulisem tegelane üldse, kes, kelle juurest ilmselt algab kogu Eesti arvuti pisest, kui Jaak Loondest algab kogu Eesti arvutiteadust, siis ma arvan, et lõvist algab kogu arvutipises, Alugete ise poleks pisest kunagi käinud. Aga seal Rainer Nõlvak ja kõik, kõik see nagu plekist kokku ja siis ma saan aru, et Rainer oli Margusele teinud ettepanek toimetada siis mingisugust ajakirja, mis siis nagu alguses toimetaja seal ja ta oli, ta oli noh, nagunii-öelda asutaja, toimetaja mis iganes alguses ja tema siis vits mul varrukast kinni ütles, et davai, et need on vaja teha seda ajakirja, onju mina muidugi, pigem oleks mänginud arvutimänge nagu ma olen harjunud teleselline ikkagi üheksakümmend protsenti meie tegevusest oli arvutimängude mängimine. Aga noh, seal oli ka täiesti. Ma sain omale väga korrektse neli, kaheksa, kuue, ma arvan, seal ja seal jooksis Ultima andev roll ja just asjad, et see oli täitsa tore. Ja siis.
Ma muidugi tegelesin sellise klassikalise nagu noh, toimetus tegevusega, mitte tuttava inimesena mõtlesin, et millest peaks alustama, peaks alustama ikkagi ajakirja esikaanest välja. Ja siis ma sellest korralisse, seda esikaant, sellel hiirega joonist sinu ma joonistasin minu arust praegu tagantjärgi mõeldes muidugi tuleb mõista seda, kui aga noh
Siis tõenäoliselt see nii ei olnud. Aga.
Aga siin oleks jah mingi tohutu aur, et õnneks järgmiste numbritega, kuna siin Brontoloogas ka kokku, et neid nii väga palju ei olnud ja nendega läks palju aega. Ja kuna see on nagu otseselt ka nagunii, kui äriline ettevõtmine vaid oligi sihuke nagu promo, siis keegi nagu väga ei survestanud seda ajaaja poolt ka nii et meil ei olnud nagu kohustust, mille tellija heidetele meil iga kuu ilmuma või vähemalt alguses algusest Olysikenena kuskil Võrus istus üks kuradi nohik siis miks ei ole tulnud, eks, et miks ei ole loobunud näeme, ei adunud, et.
Meil on selline efekt.
MP imper kindlasti oli individuaalne oma näitel võin kindlasti öelda see, et see, see punkt exe praegu foto tehtud niimoodi internetis on, et see on ka kindlasti nagu märk sellest, et ta on ikka päris oluline asi, millest ma ka küsib teise selles mõttes, et oli oluline igas plaanis, sest et tegelikult ta tõesti noh, jällegi, et olles selles asjas sees, siis noh, minu jaoks ei olnud nagu küsimus, et kas arvutid tulevad, muud maailma me siin ei mõelnud sellele, et nendega oli lihtsalt hea asja teha ja nad tõenäoliselt olid inimesed ikka täiesti rumalad, kes seda ei teinud, on ju midagi. Ja noh, kõrvalt vaadates ma isegi ei saanud aru, kuivõrd vähe tegelikult arvuteid kasutati toona, sest me istusime, eks ole, MicroLinki peakontoris seal, kes kauem mingisugune, see vilunge enne ju seal ka noh, telemajas igal pool, noh, mul oli Äkses arvutite oli päris hea. Aga ma mäletan, kas see, eks see esimeses numbris või lihtsalt ei marsi. Esimeses oli arhitekt Kalle Rõõmuse büroo.
Niisugune väikene tutvustus läbi selle, et nad hakkasid kasutama arvuteid projekteerimisel ja see oli see midagi täiesti siukest epohhiloovat ja, ja ma isegi toona ei saanud sellest aru, et kuivõrd imelik see on üldse, et keegi teen nagu paberil midagi ja seepärast tundus kuidagi ära noh, naisena väed hakkavad aga pärast noh, jällegi selle sellega on võib-olla nagu isegi tagantjärgi seda artiklit nagu lugedes üks kord, sest malakas siis sinnapaika ja panin pildid külge ja, ja noh, mind see võib olla vägagi huvitav, mis on kirjutanud, veeretanud näiteks ma ise kirjutasin. Aga, aga tegelikult tõesti, et et kui üks arhitekt käis, eks ole, Staseerimas kuskil Kanadas ja seal tegeleti just sellega, et osteti personaalarvutid ja see nagu jällegi muutis selle töö efektiivsust sellest, kui mingisugused arhitektid, konstruktorid päevad läbi joonistasid kaika peale midagi ja siis järsku maid Bach valitsusele arvutisse ja, ja kõik on nagu hästi, onju. See oli ka nagu väga-väga põnev mõte ja, ja tegelikult on huvitav vaadata seda teed, mis täna toimub, on see, et meil on see nii-öelda see pimm modelleerimine ja, ja, ja, ja, ja siis sa kuuled, mis on need nagu väljakutsed. Et tõesti, et, et et mu üks sõber tõotab start, tapmiseks, tegeleb pimm-mudelite konfliktide analüüsi, et, et kuidagi üritada aru saada, et näiteks ventilatsioonitoru ei tohi läbi akna minna näiteks siis ma vaatan, siis mõtlen, et issand jumal, millega need inimesed on tegelenud, et see noh, miks nad seda arvutit pole varem kasutusele võetud.
Kui raisatud aega, onju?
Seda saab lihtsasti teha programmiga jah, olen teinud jah, aga just see, et, et see, eks see, eks see tõesti ta nagu ta tõesti üritas tuua olu jutust jäi mulje, et see on ikka super, on väheste mingi häkkerite Räkani tegelikult ma arvan, et ikkagi inimestele andis pildi, et mis, mis tegelikult toimub. Noh, nagu üldse, et see arvuti seal nurgas ei ole nagu raamatupidaja kalkulaator ainult, või noh, mingeid muid asju ka teha. Aga kuidas sa selle kirjutamiseni jõudsid joonistamise juures? No ma ütlen, et seal oli vaja ju kontenti toota ja ega keegi toona ei olnud arvutiajakirjanik Ain ja, ja kuna mulle meeldis arvuteid, arvutimänge mängida ja ja ma arvan, et noh, kirjutamine on iseenesest tore tegevus. Siis kuidagi nagu kas sul kooli ajal juba oli see nagu kirjutab ise kirjandisoon oli kuidagi olema? Ei, ma olen võimeline kirjutama okeilt ja iganenud jah, mulle joonistada meeldib võib-olla rohkem, sest kirjutamine on sellin, raske asi, et sa pead nagu lause peaks läbi mõtlema ja siis sulle tundub, et nendele headele. Et, et on nagu liiga.
Kreetne.
Ja siis selleks, et teema jätkuks, mul ikkagi veel üks oluline küsimus.
Nüüd mõni aasta tagasi.
Tõnis Kahu seletas mulle pikalt, kui nad, minu arusaam sellest, mis asi on küberpunk on täitsa vale. Nii pikalt ja põhjalikult, ilmselt tuleb ära, härra Kahu selle koha pealt just uskuda, mis ta teab, millest ta räägib. Nüüd aga minu arusaam sellest, mis asi on küberpunk ühest väga konkreetsest exe artiklist, mille kirjutasid sina ja proto. Ma olen laid, oli seal ka kindlasti öelda, minu meelest olid DVD ka nimed olid seal all. Aga see jutt sellest, mis asi on küberpunk ja kuidas ta teab, mis asi on meie kaheksa. Ja tal on V8 mootoriga auto näiteks muuhulgas Eesti pik nimega. Neid asju, mida kübervähk tegid, räägib ka, kuidas see oli, kuidas te niisuguse sisu sisu suutsite provotseerida.
Raske.
Ta ainult äri, aga eks meil oli, eks meil oli mingi ettekujutus sellest, see on ju jälle, ega, ega küberpunk ei ole mingisugune geneetiline, mingisugune organism, mis on välja arenenud ja siis on, pärast on hea klassifitseerida, et vot see on pool on hüljes ja pool on mingisugune või veel paari ähvardamine või mis iganes. Ei noh, huvitav jah, et meeleolu selgelt olime. Jällegi ma eeldan, et me toona juba teadsime, et mis on, ma eeldan, et oli olemas juba Gibsoni nekromancer, onju, ja see oli kaheksakümnendate keskpõhjust ja see oli kui kõik nagu räägivad sellest Itšalker Kailist, mis oli muidugi ja see oli väga oluline teosena. Aga noh, minu jaoks Gibsoni pööning, kroom ja Nekromant selline ligi lasi ka ju täiesti välja, et see oli noh see oli aru saanud, tõsi, ma siiamaani loen seda regulaarselt üle. Ja härra Gibson kirjutas need raamatud kaheksakümnendate keskpaigast trükimasinasse paberi peale ja ja täpselt nii ongi aasta kaks tuhat üheksateist oled sa näinud uuemaid raamatuid ka lugenud? Mõnda need, need lähevad veel hirmuäratavalt nagu tõepäraseks. Ja aja ajahorisont tuleb lähemale, siis ma ei oska nüüd järgmine küsimus, tee pidi olema mingisugune nagu väikesed, ei mõtelnud selle kedagi nagu mõistet nagu välja kahekesi ei mõista seda saiti, see oligi nagu Gibsoni, kui öeldi, et kui tol hetkel juba mingisugused rahvusvahelised mingid Peebeeessid Internetis, kus te väikesed nagu ma arvan, et see oli kõik kuidagi klikkis tõenäoliselt kokku, et tegelikult noh, jällegi, et ma arvan, et ma ei oska prantslasest rääkida, aga noh, pleier on eraon, eks ole, eepiline nagu nurgakivi on ja siit Meier Olin ja see fotoroloog, kes joonistas ilusaid pilte, tsiklil Distopiline noh see kõik kujundas meil välja mingisugused Pästopilisem pildi tehnilisest maailmast, mis kõik on nagu külge ühendatav. Ja, ja, ja ma arvan, et see pättele ei saa liita, mis täitsa juhuslikult praegu hinna jõudis, onju. Et ma arvan, et vähesed inimesed Eestis teavad seda originaallugu. Ja mina olin selle toonane, mainin nagu totaalfänn. Ma käisin aeg-ajalt Helsingis akadeemilises kirja kaupas, siis sellega ma vaatasin, et kas uus osa on tulnud üheksa raamatut mul on kõik olemas ja see on kõik, on kuidas mingid metalltorude või jäävad su silmamuna sisse ja ajust on selle järgi ainult mingisugused džiibid ja natukene pudru, onju, ja noh, selles mõttes, et see, see, see kuidagi koerad, me elasime selle asja sees, ma arvan ja ja ka mingisugune jälle mingis mängumaailmas tõenäoliselt olid paar mängu jälle, mis, mis kuidagi sinna kontriculteerisid, pluss on see, et me muidugi üritasime ka siis teha mänge enne veel, kui me pruumooniga tegime midagi, nagu mängu ideed see sees kuidagi paralleelselt selles amiga maailmas me üritasime ka midagi teha, vaid ta ja Juhan Soonets
Me tegime Rockexi-nimelise mängu, mis pärast ka ploomul keeras pissi peale palju ägedama, võib-olla võib-olla vähem kena ja, ja siis selle intro, ma mäletan, oli väga selgelt kantud kõigest sellest vee kaheksatest ja, ja rakettidest jama ja kõikides väga tähtis oli kindlasti see, et päikseprillid olid õige kujuga ja siiamaani ja, ja muidugi Andrus Aaslaid veel kõrvale rääkis lugusid, kuidas või, või kirjutas või, või teleri, kuidas siis mingi prinki valgusega saab su aju ümber programmeerida. Ja noh, see, see, see kõik nagu absopeerusel tekitas mingisuguse omaette alternatiivse reaalsuse ja ma arvan, et see on see meie, meie arusaam küberpungist, aga enne tõu kohta küsida siis mind hakkas huvitama see, et kui sa räägid, et te kujutate, siis kuidas tootmine maailmas peast on. Ajust on ainult niisugune kodu ja natuke Tšiped. Ometi rohkem silpe, vähem rohkem sitta raamatut. Kõlab nagu väga hea teise maadel. Aga ometi see te nagu astusite pika sammuga toe tuleviku poole, ilma mingit kõhklust loomata, et see on nagu õige suund sinna, sinna tuleb minna. Seda nagu tagasi hoida on nagu, tõenäoliselt on mõttetu, sest et noh, Loviidid ka üritasid midagi aine, aga, aga noh, parem on olla seal enne teisi. Et sa paned juba õiget Tzipi taha ära ja võtad selle pudru osakaalu väiksemaks. See tahab ikkagi siukseid mõtteid mõelda, üheksakümnendatel see tahab ikkagi nagu visiooni saada. Plumbum, kuidas sa sattusid selle ahti ja jaanitule? Jällegi, et ma kuidagi, kuna ma olin kogu selle super häkkerid, noh, kes igalt ostult toona võib, igaüks mõtles, et on super häkker, see, kellel oli, kattis see mees seksi Vano All või ka see, kes oskas neid faile kokku panna, nii et ma olin üks väheseid tegelasi, kes joonistas pilte. Ja jällegi ma ma tegelikult oskan, oskan ka ilma arvutita päris hästi joonistada, et see on lihtsalt seal tundus see kuidagi nagu lahedam, et seda sai salvestada seal see on tuul ja see on nagu põhiline, et kui sa lihtsalt joonistad, siis saanud uut teha on väga raske. Isegi võiks öelda, et võimatu peaaegu ja kuidagi jällegi, et see seltskond ei olnud nagunii suur ja nad kõik nagu kollektisid kuidagi, eks ole, näiteks seesama nagu siin ka teised on rääkinud, et see sihuke välismaailma riitsimine, et, et see kõik oli nagu nii see sisemaa ja välismaale, et see kuidagi ikka klikist kokku, et et sellega ma teen korra kõrvalehüpe, kiire, tuli lihtsalt meelde, kuidas näited, kuidas Peeveeesside ja värkidega suheldi, et et meil oli sealsamas telemajas oli täpselt samasugune ambitsioon, meile lihtsalt oli see, et olid amiga, ta on ju ja mänge ei olnud, siis pidi neid mänge kuskilt jälle piirama on ju ikka. Ja ma loodan, et tagantjärgi mägi kahekümnendat pidamisasjatundjad ei hakka, ei hakka need peale lendama, aga igal juhul üks viis oligi see, et sa pidid noh, täpselt jõudma kuskile mingisugusesse Peeveeessi, onju ja, ja, ja, ja kuidagi sinna sisse pääsema ja ega see oli olnud niimoodi, et tasku sisse, et seal nagu ennegi on kuulnud satavastist mingid vennad, kes monitorist tegevust jälle Eesti tundus eksklusiivne, niisugune veider koht on ja see on sama hea kui eskimo naine. Ja, ja, ja mingeid, meil isegi tekkis mingisugune treeningpassid, et meil juba oli midagi, mida nagu vastu pakkuda, aga tavaliselt me ikka mängisime sellist vaest sugulast ja, ja siis oli ka, et me isegi noh, nagu bluuboksisime ennast sinna sisse ja noh Otto, tundub, et räägime sellest lähemalt, see tähendab siis seda, et sa pidid kuidagi toonase Telekomi keskjaamale midagi kõrva vilistama, mida too kuulda ei tahtnud tingimata. Kusjuures nüüd, kui ma hakkan mõtlema, et igasugused Margus oli, meil põhineb loobuksid spetsialist, aga kas me nagu reaalselt loobuksime sinna ka jõudsime, seda ma nüüd hea küsimus, sest ma mäletan, et punkteksidele küll manuaal, sellega ja, ja aga põhimõtteliselt samad asjad lugeda, see on ju iseenesest üsna lihtne, kuna need vidinad toona olid suhteliselt rumalad, on ju näiteks simkalle võimeta, kellelegi ma annaksin või kellega sa rääkisid, et et kui kiired olid modemine, mina mäletan, seda tõi kolmesaja Heyes. Ma mäletan ka seda, et minu meelest ma ei mäleta, kas oli mast või keegi oli väidetavasti suuteline händseigi ära vilistama sellele kes ta oli nagu piisavalt aeglane. Et kui sa nagu raagu suutsin talle suusõnaliselt selgeks teha, et et see on see legendaarne käpp, Francise vile, mida Ameerika moel jagati, mis kaks tuhat kuussada üldse välja vilistas täpselt ja, ja just-just-just, sest seda saab kasu kui teha, kui vaja. Ja aga jah, et see kõik oli nagu kuidagi jällegi, et võib-olla oli see, et igaühel meist oma fookus on ju, et kõik, kes oli, tahtis rohkem seal mingit Networki, äkki ta on ja kes tahtis rohkem lihtsalt häkkida häkkida, kes tahtis progeda? Jällegi mind huvitas, mida ütlesid selgelt, mind hoidsid mängud, liikuvad pildid, värvilised pildid, kuidas neid ise teha, kolm tee kõik niisugune värk, et ma nagu, nagu olin. Pigem nagu otsis neid võimalusi, eks see viis meid ka tegelikult kokku siis lõpuks pruumoni pundiga, kellele oli lihtsalt ta oli kindel soov, et nad tahavad teha mängu, onju. Ja kuna minu jaoks oli see lihtsalt natukene niisugune nagu nõme ülesanne, kuna neile oli, mul oleks amiga, seal oli kuradi Maidan miljonit värvid on ju, neil oli mingisugune VGA ekraan ei, alguses ühtegi C ja sinna pidi mingi nelja värviga midagi valmis hingeldama väga palju, et noh, teeme ära. Ja, ja, ja seal kõige naljakam on see, et see kosmonaut, mis sellest sündis. Jällegi see, kuidas sa turustasid, mis ma nagu lihtsalt vaatasin ja imestasin. Ja järelaeg mul nõrgurist meeles seekord nagu telgeitid, need vennad olid lihtsalt toona ja on vist siiamaani, eks. Et nad noh, olid nagu tõesti nagu Pakosta teeme seda asja, aga see, see graafiline pool oli, oligi nagu super lihtne, nokkisin valmis. Siis nad tegid oma selle mingi mingi musa, ehitage sinna marakesin ka mingisugused ikoonid, mingit kitarre ja mängi trummid ja väga vaheli. See tähendas seda, ma mäletan küll, et inimesed, kes oskasid arvutiga joonistada, et neid oli nagu vähe ja kas sul tekkis. Ühesõnaga, et kas sul kõigepealt tuli arvutiga joonistamine ei siis joonistamine või oli sul enne ka joonistada? Ei, ma ikka Ene Jaaniste, selles ma noh, jällegi kes ikka ennast kiidab, kui mitte ise, eks ole, ma nagu. No akadeemilist joonistamist ma valdan nagu suhteliselt väga-väga hästi okeid, et selles mõttes mul ei olnud nagu keeruline omandada enamus inimesi, vaatasin toona joonistatud tarvetitega ega mini, vaid oli seesama kuradi munaga hiir, mille, eks ole, muna aeg-ajalt jooksis mingit kuradit pahna täis, siis jälle küünega puhastama. Ja, aga noh, ma ikkagi endale võtsime kiire, mis nagu enam-vähem jooksis, on ju, et selles mõttes minu jaoks ei ole vahet, kas see on pliiats või, või, või Nov tablett või, või mad Evelin ja et see on nagu noh see arvuti oli sinu, see nii-öelda kunsti tegemise ja selle selle ande nii-öelda laiendus. Jah, ta oli lihtsalt ütleme nagu mingi teistsugune tehnika ja tunduvalt andeks andma kui näiteks mingi akvarell või mis iganes. Et selles mõttes oli.
Täna sa vaatad, eks ole, et et iga kõik kunstnikud kasutavad mingit Maidamegidzindiku tabletti või, või noh, neid noh, neil on kõik super ägedat toolid on ja ma siiamaani aeg-ajalt, kui mul on vaja omanikega midagi, kus täna maanikerd on tegelikult temale parim tants päri ja kõik vaatavad, et ma olen peast soe. See on nagu noh, mugav ja käe järgi tuua tegelikult, et kui seda raamandade keerata, keerad kursori viinud kiireks. Ühel hetkel sul tekkis see koht, kus tekkis mõte, et võiks hakata Weeki tegema. Ja see oli ka Pauluse mainib, et siin jah, see oli ka nagu pigem läbi selle, et kuna ma olin aru saanud, et ma ei ole piisavalt järjepidev ja, ja see sealt ütleme, see pragemise osa oli nagu kõik tundus toona liiga kui välja. Et noh, mul olid sõbrad, kes sellega tegelesid ja, ja miks see üldse nagu see tulemus ei olnud seksikas. Aga teeb järsku noh, ta jälle algusest oli ka mingi superpoorid on ju, et seal ei olnud nagu noh, mis seal oli see osaühing või mis esimene oli see sealt pikemasse? Adria seljaga siis kui ma sain aru, et kui sa said juba tabelite keelata ported maha ja sinna mingite üksikute ühe mikstiste tükkidega seal neid hakata mingit leiavadki tegema ja siis ma olin müünud mees, siis ma nagu hääletades ka siis ma tahtsin ühes reklaamibüroos ja midagi sealt katsetasin, nokkisin. Ja, ja siis mainitakse juba olemas. Ja, ja selle asutajad tulid siis otseselt andmed maja veel natuke reklaamimise, sest ka et paneme midagi mingit seljad kokku ja hakkame, hakkame seal nagu vaatama ja eks seal jällegi, et lihtsalt see, et ma sain mitte lihtsalt enam selle pildi oma käe seest ekraanile, vaid ma ise sain selle pildi nagu pauh kõigile nina ette, eks ole, paljudel ekraanidel jah, ja, ja, ja, ja toona oli noh poisikesed, mis ma tegelikult ikka nimega poisikesed olime, siis oli juba aasta oli siis kakskümmend viis või? Üheksakümnendate keskpaik, teine pool. Jah, jah, et üheksakümmend kuus, üheksakümmend seitse oli juba, et see siis oli juba, noh, Siis sa juba Business teha, siis. Panid jälle, tegid lõikest neid piksleid ja, ja mingisugused, ma ei tea, ma mäletan, meil oli klient, oli Reval Hotel Group, et noh, siuksed mingid nagad tulid ja võtsid siis kliendid ja tegid neile mingeid ägedaid asju. Sõber. Ja, ja see jätkuvalt oli see, et et sa said omale selle pildi panna inimestele silma, et eks ole seal sees ju liigutav faktor, ma olen isegi võib-olla see, et tegelikult mul ei oleks isegi vahet, on kas sa nagu inimestel silma ette vaid just nimelt see, et sa tegid mingisuguse, sul oli mingisugune distsipliin siis selle seal mingisugune HTML, onju. Ja, ja, ja, ja sa teadsid, kuidas optimeerida sa oled, seal oli mingisugune mingisugune tuulised ja saab alles seda suhteliselt hästi ja see tekitas sulle nagu rõõmu, et sa said sellega teha mingisuguseid asju, mida võib-olla teised nagu aga saab teha. Ja, ja see Jobs saatis, läksin värk, et tegelikult noh, ega ma kujutan ette, et muru niitmine on ka selles mõttes lahe, sa näed, kuidas on nagu Oru ja taga maha niidetud selle instan suhteliselt instanud prätifykeissimat. Võib ka nii mõelda, et, aga noh, samas kui sa muidugi ei oska, siis ei tule mingit hetke kehvade isenditega täisajaga ja see ongi, et sa tead, et sa oled sinna mingil määral panustanud, on ühe seal on nagu mingisugune Technical, et on ju, et sa saad. Et, et iga mats ei tule, ei tee seda, et sa saad nagu öelda, et Aak maisse, mis sa praegu teed. Praegu.
Ma olen kuidagi lihtsalt distantseerunud sellest disaineri rollist aga samas mitte, et jälle ajab ikka oma arusaamad, tegelikult selle pildi tegemine on, mõnes mõttes võiks nagu niisugune käsitöölise töö, et tegelikult need lõikelauad, mida minu lapsepõlves turul müüdi, kus olid need selle põletiga oli tehtud Nuubavaliibiale, onju ta nagu veits sarnane, et palju on Tšehhimaad tegelikult võtta ja aru saada mingitest äriprotsessidest või mingisugusest inimeste mõttemallidest ja disainida neist midagi ja, ja, ja jällegi, et see progemine minu jaoks on see, et kui see õigesti sõnastada, siis mingisugused vennad teevad selle valmis ja see, see muutub nagu päris, eks ole, et seal seal jällegi sellise protsessi toetav mingisugune asi seal masina sees mis toimetab täpselt nii nagu sa oled talle nagu öelnud, et toimetan ja et et seal olid need nüüd mitte ei ole, tule sinu käe seest sinna, see pilt vaid tuleb sinu pea seest mingi mõte, kuidas see kupatus võiks käia, Swing programmeerib, valavad selle valmis ja siis käibki niimoodi just just et mul oli lapsepõlves oli kuidagi ma mäletan selgelt, et mul oli mõnus mõte, et tehas on tore asi, sest et ta võtab mingisugused toorme ja see pannakse kokku mingite detailide, siis pannakse sellest kokku mingi asi. Ja jällegi, et me jõuame sedasama selle Nintendo Diva esijuurde, et, et füüsilisel kujul seda toota. Jõle tüütu, palju lihtsam oleks teha seesama asi, nii et oleks bittide jada, mis kõik liigrupeeruvad moodustavaid mustreid ja sellest nagu peaaegu nagu võluväel, eks ole, tekivad mingisugused asjad, mis inimestele tegelikult on tänaseks sama reaalselt tööriistad kui haamer ja, ja höövel, eks.
Nii on.
Aitäh ja sain küsida palju huvitavaid küsimusi, palju targemaks. No ma loodan ka, et ma nagu liiga ei läinud rändama, et see on juba kõik, on, kõik on väga hästi.
Aitäh sulle, aitäh Sulle.
