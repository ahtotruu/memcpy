\index[ppl]{Loit, Kaspar}
\index[ppl]{B'Knows}
\index[ppl]{B'Knows|see{Loit, Kaspar}}

\question{Kes sa oled?}

Mina olen Kaspar. Ja kunagi, kuna me peame tagasi kerima mingisugune miljard aastat, siis mu aka oli B'Knows. 

\question{Aga kust sa said sihukese aka?}

Seda ei mäleta enam keegi. Seal on nagu kaks komponenti. Üks on nagu \enquote{B} ja siis on nagu \enquote{knows}, ehk siis see B peaks  midagi nagu teadma. Pronto\index[ppl]{Pronto} alati kutsus mind Buttknows.

\question{Kuidas sina arvutite juurde said või arvutid said sinu juurde?}

Mul on selge mälupilt, et mu tädi, kes on superkuul ja minust mõnevõrra vanem, töötas Tartus
vist Bioloogia Instituudis. Ja  talle oli kuidagi jäänud mulje, et mind võivad huvitada sellised asjad. Ma arvan, et ma olin mingi, ma ei tea, kaheksa-üheksa-kümme, \emph{something like that}. Kui ma tal ükskord Tartus külas käisin, viis ta mind instituuti. Muidugi peale tööd, kõik oli juba pime. Mingi kabinet oli lahti ja laua peal seisis masin, mille nimi oli Apple II Europlus\index{Arvutid!Apple II}\sidenote{Apple II Europlus oli Apple Euroopa turule kohandatud versioon. Muu hulgas erines toiteblokk aga ka video osa tuli ümber teha, sest Steve Wozniaki trikid NCTS signaali genereerimisel keerukama PAL süsteemi puhul enam ei toiminud.}. See oli \emph{freaking awesome}. Ta oli seal mingi laborandi käest küsinud, et kuidas  midagi käib. Sai laadida paar üli superägedat mängu, mis tekstiekraanil jooksid. Üks oli vist \emph{Train Robbery}\index{Mängud!Train Robbery}, ma mäletan, see pilt on täitsa silme ees. Ja sellest hetkest ma arvan, ma olin müüdud ka. Ma ei oska  meenutada, kas ma olin arvutitega enne ka kokku puutunud aga tõenäoliselt mitte, see oli ikkagi liiga  vara. 

Kuna mulle tohutult meeldisid koolis Nintendo väiksed Game \& Watch\index{Nintendo Game \& Watch} mängud, võib-olla mäletad? Tänapäeva telefoni suurused umbes, neil oli LCD ekraan, millesse oli ette joonistatud mingisugused tegelased ja siis nuppudega said mängu mängida seal ekraanil\sidenote{Vaata ka märkust \ref{sidenote!gameandwatch} leheküljel \pageref{sidenote!gameandwatch}.}. Ja  ma kuidagi mõtlesin, et \enquote{oh, kui lahe oleks, kui saaks ise niisuguseid teha}, aga no ma sain aru, et seal taga on mingi tootmine ja see ei ole nagu reaalne. Ja nüüd järsku saada aru, et selliseid asju on võimalik  masina sisse programmeerida ilma, et sa pead mingit elektroonikaskeemi tootma, et sul ei pea üldse mingit tehast olema. See oli üks niisugune \emph{revalation}, onju. See muidugi  tõenäoliselt viis mind kunagi ka mängude tegemiseni. 

Aga niisugune päris toimetamine hakkas ilmselt kuskil seoses Jaak Loondega\index[ppl]{Loonde, Jaak}. Ta oli kindlasti ülioluline tegelane, sest sel ajal oli arvutile ligipääs oluline asi. Ja ma mäletan, et ma tegelikult olin kaardistanud omale kõik kohad, kus üldse  tõenäoliselt Eesti Vabariigis arvutitele ligi pääses. Nõo oli liiga kaugel selgelt, aga Tallinnas neid kohti ikka oli. 

Aga Jaak Loonde\index[ppl]{Loonde, Jaak} oli selles mõttes lahe tegelane, et minu meelest vist tema kaudu ma esimest korda sain midagi progeda. 

\question{Kas ta siis andis selle võimaluse või ikka õpetas ka?}

Ta ikka õpetas, loomulikult. Mingil põhjusel, ma ei tea kuidas, ma sattusin  3. Keskkooli\index{Koolid!Tallinna 3. Keskkool}, kus oli sihuke suur klass mingisuguste masinatega, tõenäoliselt ka need olid MSX'id\index{Arvutid!Yamaha MSX}. Seal BASIC\index{Keeled!BASIC} oli ees ja võis midagi hakata klõpsima. MSX selle pärast  üldse oli pull masin, et  ta tegelikult vist mõeldi välja, et ühtlustada koduarvutite standardit ja BASIC'uid, mida nad jookseksid.  See initsiatiiv tuli vist isegi  mingisuguse Jaapani Microsofti \emph{executiv}'i poolt.

\question{Ta \emph{boot}'is BASIC'usse otse, eks?}

Jah. Sa põhimõtteliselt hakkasidki peale niimoodi, et esimesel ekraanil sa võid  kirjutada \verb|10| ja siis kirjutada ühe rea. Kirjutad \verb|20|, kirjutad teise rea, ütled \verb|list|, siis ta  näitab, mis sul on. Kirjutad uuesti \verb|20|, kirjutad selle rea üle ja kirjutad \verb|run| ja \emph{that's that}, ta sul kohe käib. 

Seal oli kari tegelasi, paari tükki ma tundsin ja nende kaudu ma vist kuidagi sain sellest klassist teada. Ma mäletan, et üks mu koolikaaslane nägi sihukest masinat esimest korda ja meile öeldigi, et hakake midagi tegema. No ja siis ta kirjutaski \verb|please draw me a circle|. Ütleme, et NLP\sidenote{\emph{Natural Language Processing - NLP}} ei olnud veel nii kaugel ja sealt midagi ei tulnud.

Tolle klassi ümber sagis palju rahvast, Karel Kannel\index[ppl]{Karel Kannel} seal kuidagi toimetas ja lisaks veel hulk toonaseid väikseid tattnokki. Aga palju huvitavam oli tegelikult see, et Jaak Loondel\index[ppl]{Loonde, Jaak} oli ka üks masin, nimi oli MIR-2\index{Arvutid!MIR-2}\sidenote{\begin{russian}МИР\end{russian} oli varane Nõukogude miniarvutite sari, mille kolm generatsiooni (MIR, MIR-1 ja MIR-2) töötati välja aastatel 1965-1969. Sarja nimi oli lühend pikemast nimest \begin{russian}Машина для Инженерных Расчётов\end{russian} (Inseneriarvutuste Masin).}, mingisugune nõukogude aparaat. See masin oli põhimõtteliselt sihuke pikk kapp, ikka pikem kui viis meetrit, poisikesest oli ta kõrgem, Jaagule võib-olla ninani. Masin tegi meeletut, sihukese villa kraasimise masina sarnast häält, millest põhiline tuli jahutusest. Ilge müra. Aeg-ajalt, kui inimestel viskas  kopa ette,  nad lülitasid masina jahutuse välja ja siis Jack\index[ppl]{Jack}\index[ppl]{Jack|see{Loonde, Jaak}} tuli muidugi ja tohutult röökis, sest too masin oleks ilma jahutuseta kokku küpsenud. 

Selles mõttes oli ta ka nagu geniaalne vana, et kust ta sellise \emph{advanced} masina üldse kätte oli saanud? Seal oli ikkagi võimalik klaviatuurist käske sisestada, kusjuures klaviatuur oli elektronkirjutusmasin, mis  põhimõtteliselt oli nagu klaver ja printer ühekorraga. Nii et \emph{hard copy} tuli ka samast masinast. Ja temaga oli ühendatud mustvalge telekas, aga tal oli ka valguspliiats. Ehk siis sa said nagu ekraanil tabada mingisuguseid punkte ja  masin tundis selle ära. Ilmselt ta luges seda  kineskoobi kiirt ja selle järgi pani asukoha kokku. Ja ta suutis ka mingisugust rudimentaarset graafikat kuvada, ehk et tal ei olnud ainult tekstiekraan, vaid ta suutis ekraanile  mingit punkti kuvada. See viimane oli arvuti jaoks tohutu ülesanne, punkt ei püsinud hästi paigal, õrnalt ujus, aga sai hakkama. 

See programmeerimiskeel, mida MIR-2 peal kasutati, oli vene keeles, vene tähtedega, kõik olid mingid lühendid, superluks, onju. Mu esimene programm, ma mäletan, oli mingisugune graafiline nelja-tipuline täht. Ma arvan, et ta koosnes ütleme siis  umbes kuueteistkümnest punktist võib-olla ja see ikka tõmbas selle arvuti täiesti kooma. Kõik see ekraan ujus, aga väga uhke tunne oli. 

Aga mida Jack\index[ppl]{Loonde, Jaak} meile veel õpetas, oli näiteks perfolindi lugemist. Masin sõi kahte moodi meediat. Üks oli paber, perfolint, siuke õhukene, mis lasti vurinal masinast läbi. Teoorias teravamad vennad, ma arvan, suutsid nõela või augurauaga  perfolindi peale proge torkida. Mul on selline tunne, et äkki ma oskasin seda kunagi. Aga aga siis olid seal veel mingisugused vahvad magnetkaardid. 

\question{Magnetkaart? Perfokaarti ma tean aga magnetkaardist ei ole kuulnud.}

Magnetkaart oli selline  tänapäeva  telefonist suurem, ma arvan, et mingi kaheksa senti korda mingi viisteist senti sihukene pruun latakas, mis põhimõtteliselt meenutas oma materjalilt seda, mis flopi diski sees on. Ja sa põhimõtteliselt panid selle mingist \emph{slot}'ist sisse, masin tõmbas selle surtsti läbi ja luges sealt midagi. Aga see oli nagu sihuke müstika, seda enam niisama lugeda ei saadud. 

\question{Noorel nagamannil peab olema ikka päris änksa tahtmine, et ekraani peale tähe joonistamine huvitav oleks?}

See oli  müstiliselt äge. Sa kirjutasid midagi ja see  sulle  sinna ekraanile tekkis pilt.

\question{See huvitav asi oli just see, et mina andsin käsu ja masin tegi midagi?}

Noh, täpselt, et kui see oleks nii lihtne, et \verb|please draw me a circle|, siis  ilmselt oleks huvi kaotanud, aga see oli ikkagi \emph{complicated} värk. Selles oli mingi alkeemiline element, see oli ülikõva. Väikestele poistele meeldivad salakeeled, koodid, lipukirjad ja igasusugused niisugused asjad. See  oli nagu kõike seda  ja veel midagi,  see oli ikkagi super, noh!

Selge oli see, et too masin oli  meeletult piiratud, et kaua sa seal ikka jändad. Ja lisaks see, et seda MIR'i oli ainult üks, õnneks MSX'i klass oli suurem. Aga klassiga olid vist jälle mingisugused piirangud, kuna see oli keskkooli all. Ja ilmselt sellepärast Jack\index[ppl]{Loonde, Jaak} sebiski Roopa tänavale selle ÕTK\index{Tallinna Oktoobrirajooni Õppetootmiskombinaat}\sidenote{Täpsemalt Tallinna Oktoobrirajooni Õppetootmiskombinaat. Sellest asutusest on natuke rohkem juttu leheküljel \pageref{content!OTK}.}, kus oli siis ka terve klass. ÕTK's oli  üks niisugune nagu juhtarvuti, millel oli mingisugune draiv (ma eeldan, et see oli mingi flopidraiv) ja terve klassitäis arvuteid, mis said sellest peaarvutist omale asju alla laadida. Sa võisid ka lihtsalt oma programme kirjutada, aga kuna kuna draive oli ainult üks, siis kui sa tahtsid mida salvestada, pidi selle juhtarvutisse saatma. 

Kuna Jackil\index[ppl]{Loonde, Jaak} ei olnud aega seal klassis väga  hängida, siis möllas seal kogu aeg mingi poistekari ja tal oli paar nutikamat venda pandud seda vedama. Üks legendaarne tegelane oli Mukats\index[ppl]{Mukats|see{Edesi, Linnar}} ehk Linnar Edesi\index[ppl]{Edesi, Linnar}, kes täna vist kuskil Soomes toimetab. Tema oli selgelt minu esimene guru, keda ma nägin. Ta oli, ma arvan, umbes minuvanune, aga ta oli omandanud kõik need arvutiasjanduse peenemad alged. Põhiline, mida ta oskas oli  kahest programmijupist  ühe terviku  kokku panek ja  selle paketeerimine nii, et tulemust sai programmina laadida. Point oli selles, et väga suur osa softi levis tavalistel magnetofoni kassettidel. Ja vist oli see kuidagi nii, et šeffimad mängud olid 32 kilobaiti umbes pikad, noh see oli ikka \emph{massive}. Aga selleks, et nad kasseti peale ära mahuks, olid nad tehtud pooleks: 16+16 kilo, sa pidid vahepeal kasseti ümber keerama. Ühesõnaga kogu see kassetimajandus oli keeruline. Aga kui sul oli juba nii kõva asi nagu flopidraiv, siis sa said selle kasseti pealt kuusteist kilo sisse lugeda, tõsta see kuskil mälus mujale,  lugeda teise kuusteist kilo ja selle esimese jupiga kokku panna. Tekkis  tervik, mida sai kuidagimoodi \emph{launch}'ida. Ühesõnaga, see kõik oli täielik supermaagia. 

\question{Järelikult tekkis sul niisugune teadmistel põhinev eeskuju. Keegi inimene, kellele sa vaatasid alt üles, sest ta teadis rohkem kui sina?}

Oo jaa, selliseid tegelasi oli veel. Üks vend, kelle nimi oli Kont\index[ppl]{Kont} (ma ei mäleta, mis ta eesnimi oli). Tal oli väikene metallkohvrikene, mille sees oli kogu MSX'i manual. See oli fotokopeeritud sihuke \emph{stack of paper}. Ta käis sellega väga uhkelt ringi, aegajalt tegi kohvri lahti ja siis midagi selle alusel kirjutas. Sihukesi tegelasi oli veel seal. 

Kuna sa istusid seal ilma otsese \emph{access}'ita sellele draivile, võib-olla  mängisid või siis kirjutasid oma BASIC'ut\index{Keeled!BASIC}. Sellele tegevusele tulid mingid piirid ette. Loomulikult mind huvitas graafika pool, ma üritasin mingisuguseid pilte ekraanile manada. MSX'il\index{Arvutid!Yamaha MSX}  tegelikult ei olnud graafika ekraani,  seda emuleeriti tekstiekraaniga. Ehk siis  põhimõtteliselt iga pilt, iga mäng, mis MSX'il toona jooksis, oli tegelikult otse mälus tähegeneraatorei ümber programmeerimine. 

\question{Jukuga\index{Arvutid!Juku} oli sama lugu, et sa said kuskile mällu oma nii-öelda fondi laadida. Iga tähe asemele panid \emph{bitmap}'i ja nendest sai mida tahes kokku laduda}

Just, põhimõtteliselt sama laks. MSX'i ekraan oli otse aadresseeritav. Kui sa teadsid, et selle režiimi ekraan algas aadressil heksas \emph{whatever-whatever} onju, siis sa said sinna järjest kirjutada. Iga bait oli üks rida ja võis olla kas läbipaistev, taustavärv või esivärv ja mingites režiimides sai rida-realt neid värve vahetada, see oli eriti \emph{advanced}. Mängude puhul on niisugune mõiste, nagu sprait, ma ei tea, kas sa oled kokku puutunud. Need on siis graafikatükid, mis tausta ees liiguvad. MSX'il neid ka emuleeriti sellesama tähegeneraatoriga, seda programmeeriti jooksvalt ringi. Ekraan kirjutati sümboleid täis, ja siis neid kogu aeg adresseeriti ja kirjutati ringi. Ekraan vist oli jagatud kolmeks osaks, igas osas sa said eri tähestiku väänata.

Minu jaoks oligi see võlu, kui ma sain selgeks, et on olemas assembler\index{Keeled!Assembler}, assembleris saab ühele mälu aadressile ühe baidi laadida ja siis ma põhimõtteliselt veetsin suure osa oma ärkveloleku ajast mingisuguseid tegelasi  millimeetripaberile joonistades, neid heksaks tõlkides ja kuskile mällu laadides. 

\question{Kui ma nüüd tagasi peegeldan, siis ega tänapäevane arvutigraafika ei ole ka lihtne aga keerukus tundub olevat teises kohas. Sa pead 3D geomeetriast aru saama ja spetsiifilisi API'sid tundma jne.}

No täna ikkagi \emph{layers of stuff} on sinu ja pildi vahel, aga MSXil oli just see, et sa põhimõtteliselt sa toorelt toppisid otse ekraanile midagi. 

\question{Sa pidid ikkagi välja mõtlema selle, mida sinna mällu toppida, ja hoolitsema, et see värskendatud saaks ja nii edasi}

Ütleme, et seal oli igasugused trikid, et ta töötaks. Aga kuna keelestik ja kõik see asi oli nii lihtne, oligi tulemus super elegantne, super  lihtne.  Ma arvan, et ega minu progemise aeg jäigi sinna kaheksakümnendate keskele. Pärast ma olen võib-olla natuke HTML'i ja võib-olla CSS'i nokkinud, aga see võlu läks nagu üle kohe, kui asjad läksid keerulisemaks. Aga õnneks tulid tasemele igasugused graafikapaketid ja ja muud asjad. 

\question{Aga millal see oli? Juba keskkooli ajal?}

Olles enda jaoks kaardistanud ära kõik kohad, kus sai midagi arvutitega näppida, jõudsin ma läbi

Ma olin enda jaoks ju ära kaardistanud kõik kohad, kus sai midagi arvutitega näppida. TPI's\index{Tallinna Tehnikaülikool} oli ka üks klass, kus olid MSX'id\index{Arvutid!Yamaha MSX}. Aga seal oli igal masinal juba draiv taga. See oli ka juba super \emph{advanced} ja seal see guru staatus oli ka nagu juba järgmisele \emph{level}'il. Seal olid  laborandid, Aare Tali\index[ppl]{Tali, Aare} nimi tuleb kuidagi ette aga ma ei ole kindel, kas see on õige nimi õige näo juures\sidenote{Aare tegutses TTÜ-s küll ja temast on ka varasemalt juttu olnud (vt. lk. \pageref{sisu!aare_tali}).}. Aga igal juhul toimetasid seal juba üliõpilased või isegi juba \emph{post-graduate},  palju kõvemad vennad ja loomulikult see nagade jada seal ukse taga neid selgelt tüütas. Nad kehtestasid siis oma  reegleid, olid suured jumalused. Näiteks, kui info levis ja järjest rohkem kutte tekkis sinna värava taha,  oli vaja reglementeerida, et kes saab ligi ja kes mitte. Siis nad võtsid ühe kõige popima mängu, mis seal parasjagu oli, Kings Valley\index{Mängud!King's Valley}, trükkisid välja kogu selle \emph{source} koodi. See oli \emph{stack of paper}, ilusti perforeeritud ja niimoodi. Nad lugesid seda koodi, punase pastakaga tõmbasid ringe ja progesid selle mängu ringi. Minu jaoks oli see nagu jumaluse tase! Nad progesid selle niimoodi ringi, et nad said väikeste ise-ehitatud \emph{joystick}-idega juhtida selle mängu kolle. Reegel oli põhimõtteliselt see, et kui sa said nende jumalate vastu ühe taseme läbi siis sa said ühe päeva klassis käia. Iseenesest see mängufaktor  oli põnev aga just see, et nad tõesti võtsid selle mängu, mis minu jaoks tundus superkeeruline ja lihtsalt kirjutasid \emph{binary} ringi. Nad mitte lihtsalt ei teinud seal mingile tegelasele mütsi pähe,  vaid nad lihtsalt nagu tegidki kõik ringi, käitumine muutus. Tänapäeval muidugi tagasi vaadates tundub, et see kõik oli tegelikult väga lihtne.

\question{Eks see oli ju \emph{gamification}, mis praegugi populaarne on ja üksiti kindlustas veelgi lugupeetavate jumala-staatust poistekamba silmis}

Aga igal juhul lõppkokkuvõttes ma lõpetasin kuskil Kullos\index{Kullo}, kus oli ka üks klass, kus olid vist juba natukene kõvemad MSX'id\index{Arvutid!Yamaha MSX}. Neil oli juba  mingi graafikarežiim ja igasugused muud asjad, ehkki, kui sa midagi kiiresti liigutada tahtsid, pidid ikkagi kasutama tekstiekraani. Toda klassi majandas selline legend nagu Räni Meister\index[ppl]{Meister, Räni}, kes toona oli selgelt sihuke tore punkar, kes oli tulnud kuskilt Valga Gaasianalüsaatorite tehasest umbes ja viitsis poistega jahmerdada. Aga ta vaikselt seal hakkas tegelema Commodore Amigadega\index{Arvutid!Amiga}\sidenote{Amiga oli Commodore poolt 1985. aastal turule toodud personaalarvutite sari. Teistest põlvkonnakaaslastest eristas seda perekonda spetsiaalse graafika- ja heliriistvara lisamine ning väljatõrjuva mitmetegumilisuse realiseerinud AmigaOS.}, mis oli juba sihukene \emph{super advanced} raud. 

Kuidagi mahtus see kõik videotootmise ja selliste asjade tähe alla, tänu sellele ta oli ka loomulikult kaardistanud, et kus niisugune asi veel toimub. Eesti Televisioon\index{Eesti Rahvusringhääling!Eesti Televisioon} oli selgelt üks ja veel oli mingisuguste vene metalliärikate turundusharu. Ilmselt keegi vend oli piisavalt palju lobi teinud ning kuskil Kristiines  keldris püsti pannud väikse nii-öelda reklaamistuudio kus ta siis tootis värki ja tal oli seal ka üks Amiga\index{Arvutid!Amiga}. 

\question{See pidi siis olema üheksakümnendate algus juba, eks?}

Jah, kuskil sealkandis. Kullos me hakkasime ka juba mingeid mänge tegema ja nii. Markus Klessman\index[ppl]{Klessman, Margus} toimetas seal näiteks. Raul Keller\index[ppl]{Keller, Raul}, kelle aka oli \enquote{Killer}\index[ppl]{Killer|see{Keller, Raul}}, üritas MSX-i mänge vist kuidagi publitseerida, aga see (vähemalt mulle ja toona) tundus kuidagi väga kahtlane ja naiivne tegevus.

Aga siis juba Räni\index[ppl]{Meister, Räni}, kuidagi nähes minus potentsiaali, meelitas mu Eesti Televisiooni\index{Eesti Rahvusringhääling!Eesti Televisioon}. Põhimõtteliselt ma ei olnud isegi veel keska viimases klassis, kui ma töötasin juba Aktuaalses Kaameras. Uudistetoimetuse kõrval oli sihuke väike kubrik, kus me siis tegime Aktuaalse Kaamera infonurki, mis olid  diktori taga seina peal. Ja kuna Amiga\index{Arvutid!Amiga} oli selline tore masin, et  sinna sai lasta videosignaali sisse ja sealt tuli videosignaal välja, sai temaga juba toona digimiksi teha. 

\question{Ossa, see oli PC peal jõhkralt kallis riistvara toona}

Ongi. Miks need Amigad siinkandis selles vallas levisid, oli just see, et PC jaoks selliste võimalustega videokaart oli Hollywoodi hinnaga asi. Ja PC-del oli enamasti, mingisugune CGA ja neli värvi onju, samas kui Amiga\index{Arvutid!Amiga} oli \emph{full video}. Põhimõtteliselt polnud sul vaja isegi arvuti monitori, sa võisid talle teleka järele panna. Ta oli ju  kodukodutarbimisest arenenud. 

Seal tegime oma ilmakaarte ja panime videopilte sinna ja põhilise osa ajast muidugi mängisime arvutimänge, sest  Amigal olid superšefid mängud. 

\question{Aga mis tarkvaraga te tegite seda kõike? Ega te ju nullist ei kirjutanud kogu seda kraami?}

Olid olemas täitsa viisakad graafikapaketid, Deluxe Paint\index{Deluxe Paint}\sidenote{Deluxe Paint on rastergraafika redaktorite sari, mille lõi Electronic Arts'i jaoks Dan Silva. Programm alustas elu majasisese graafikaprogrammina, kuid sai pärast avaldamist \emph{de facto} standardiks Amiga platvormil.} on üks šefimaid graafikasofte, mis oli igasugustest Photoshoppidest ja kurat teab millest ikka kümme aastat ees. 

Me olime nagu sellised \emph{in-the-know} amiga-vennad ja vaatasime  kõikide PC-de  ja muude vendade peale ikka väga ülevalt alla, sest et nad ikka ei teadnud, milles nad seal sorkisid, onju. Paraku see Amiga bisness oli kehva ja läks lõpuks nurja, aga tehnika iseenesest oli superäge. 

Meil tekkis mingi väikese punt tegelastest, kellel kas oli kodus Amiga või kes nendega kuidagi tegelesid. Näiteks Martin Rinne\index[ppl]{Rinne, Martin}, kes täna teeb Directot\index{Directo} tema juba kuidagi tekkis  sinna telesse ja siis Margus Kliimask\index[ppl]{Kliimask, Margus}, kes tegeles Eesti Videos\index{Eesti Video} Siilatsi mingite asjadega ja Mati Veermets\index[ppl]{Veermets, Mati} kellest pärast sai Tallinna linna disainer.  Kõigil oli nagu mingisugune \emph{access} Amiga-te juurde. 

Jällegi loomulikult seal ka võlus mind pigem see, et sa tekitasid mingisuguse elava pildi, sul ei  pidanud olema kaamerad ja näitlejad ja nii edasi, vaid sa võisid teha mingeid väikseid animatsioone otse arvutis teha.

\question{Isegi animatsiooni ta vedas välja?}

Noh, selles mõttes, et sa said seda teha põhimõtteliselt \emph{stop motion}'iga. Ütleme,  reaalajas animatsioon kippus ikka nõgisema juba, kuigi me tegelikult ikka telepäid tegime reaalajas ka, sest keegi ei viitsinud \emph{stop motion}'iga lasta. Aga minu meelest ikka enamus sellised asju käisid reaalajas. Deluxe Paint-is\index{Deluxe Paint} olid sisse ehitatud igasugused nutikad asjad. Näiteks  liikumise aeglustamine või kiirendamine. Sa andsid talle põhimõtteliset ette, et siin on sulle kast, nüüd see kast peab liikuma mingisuguse viiekümne kaadriga siia, siis ta automaatselt täitis need viiskümmend kaadrit ära. Ja vajadusel, kui sa ütlesid, et \emph{ease in}, siis ta tõmbas lõpus hoo maha ja kõik oli väga \emph{fine}.  

Ma mäletan, kui ma alles läksin sinna telesse (selle järgi võib muidugi aasta paika panna), me tegime Öölaulupeole\index{Öölaulupidu}\sidenote{Esimene Öölaulupidu toimus 1987. aasta juunis Tallinna Vanalinna päevade ajal, aga toona meediakajastus puudus ja üritus toimus spontaanselt. 1988. aastal oli Öölaulupidu juba ametlikult Vanalinna päevade programmi lülitatud.}  mingisuguseid valgusklippe, see oli jällegi \emph{super advanced}.

\question{Võru poisina ma Öölaulupidudest ei tea midagi, aga Öötelevisioonil\sidenote{1990. aastal aset leidnud omas ajas mitmes mõttes innovatiivne teleprojekt, mille käigus Eesti Televisioon\index{Eesti Rahvusringhääling!Eesti Televisioon} öö läbi katkematult otse-eetris oli.} oli väga äge graafika}

Jajah, see oli ka meie tehtud. Tegelikult oli kogu tele sihuksesed asjad meie rida, sest  alternatiiv oli tiitrimasin, mis oli mingi räme pool-analoog pult ja mis  eriti koledat jälge tootis. Meil oli ikka \emph{super-advanced} animatsioonidega ja värviline, sai teha mida iganes. Vahest tegime mingit haltuurat mingite reklaamide jaoks ja igasugu lollusi sai tehtud.

\question{Kas see oli puhas ise-õppimise värk või hakkas kusagilt mingit informatsiooni ka juba tulema?}

Ei, see oli ikka puhas iseõppimise teema. Need vahendid olid suhteliselt piiratud ja ega seal midagi väga keerulist ei olnud. Kunagi hiljem tulid ka esimesed 3D paketid, nendega sai pusserdatud. Tase oli nendega ikka hoopis teine, sa pidid ikkagi punkt punkti haaval mingisuguseid pindu konstrueerima ja siis nendega kuidagi opereerima. Tänapäeval vaatad, kuidas väänatakse mingeid \emph{bump mapping}-uid\sidenote{Arvutigraafika tehnika, mille abil kolmemõõtmelise objekti pinnale simuleeritakse kühme ja kortse. Lihtsalt öeldes, oranžist kerast tehakse usutava väljanägemisega apelsin.} ja mingisuguseid asju \emph{layer}-ite kaupa ja see kõik annab kuidagi tulemuse,  see on täiesti müstika. 

\question{Mis sa siis tegid, kui sa Eesti Televisioonis\index{Eesti Rahvusringhääling!Eesti Televisioon} enam ei olnud?}

Kuidagi tundus, et see videograafika oli  väga põnev, aga hakkas tekkima mingisugune \emph{business}. Sõbrad, kes kuidagi olid rohkem sattunud trükigraafika peale, kes  kujundas Eesti Ekspressi ja kes seal tegi mida, see tundus kuidagi nagu rohkem \emph{business}. Kuidagi ma sain aru, et, ahah, videot teeme Amigaga aga selle selle \emph{business}'i tarvis peaks ennast kuidagi PC-de peale  sebima. Sealsamas telemajas kuidagi tekkisid potentsiaalsed kliendid ja ma pidin hakkama tootma trüki-kõlbulikku kujundust. Ma ei olnud  kunagi näinud sellist programmi nagu Corel Draw\index{Corel Draw} aga mul oli see töö vaja  ära teha ja ma istusin öö läbi ja tegin ta endale selgeks. Mis oli tohutult frustreeriv, sest ta oli täiesti teine maailm. Tänapäeval on nii, et sa joonistad ja su joonistatud pilt on  ekraanil. Siis oli niimoodi, et sa  konstrueerisid mustvalgelt mingisuguse \emph{vector mesh}'i, panid sinna mingid värvid peale, vaatasid \emph{preview}-d ja siis ta joonistas sulle selle pildi aeglaselt ette. Ja alles siis sa läksid uuesti selle pildi kallale.

\question{Kuidas sul Corel Draw õppimine välja tuli, sest minu mälestuste järgi ta ei olnud kuigi töökindel: aegajalt tegi faile katki ja nii?}

Olles kasvanud nende arvutitega üles, sa arvestasid ju, et nad aeg-ajalt jooksid kokku ja aeg-ajalt nad tegid rumalusi. Aga võib-olla siin mängis natuke rolli ka see  poisikesepõlves ÕTK's\index{Tallinna Oktoobrirajooni Õppetootmiskombinaat} õpitud arvutist üleolek läbi ühe lihtsa fakti. MSX-il\index{Arvutid!Yamaha MSX} oli paremas nurgas port, mille sisse käis kas siis kettaseade või mingi mälu \emph{cartridge}. See oli sihuke päris suur sahtel. Selleks, et mitte \emph{cartridge} sisse lükkamise hetkel midagi tuksi keerata, oli sahtli sees üks väike lüliti, mis tegi masinale reseti. Ja loomulikult õpiti kiirelt ära, et kui sa oled midagi tuksi keeranud, näiteks olid kirjutanud programmi, mis jäi tsüklisse, siis selle asemel, et voolu välja võtta, panid kohe näpud sinna auku ja masin oli surnud. Alati sa teadsid, et mingi valemiga sa saad temast jagu. See teadmine on olnud minuga siiani, et kui ma kuskilt seinast ikka lõpuks juhtme kätte saan, siis on ta surnud. Ma ei pea teda pelgama.

\question{Selle koha peal ma pean järgi andma kihule ja ära küsima küsimused, mida ma väga tahan küsida. Me jõuame Microlinki\index{Microlink} ja .EXE-ni\index{.EXE}. Kuidas sa nende juurde jõudsid?}

Kui ma olin juba selle prindiga alustanud, siis ma vahepeal kuidagi sattusin mingisse niisugusesse maailma, kus ma peamiselt sellega tegelesingi. Ma olin Margusega\index[ppl]{Kliimask, Margus}\sidenote{Tanel peab ilmselt silmas Margus Kliimaskit} televisioonis varem suhelnud ja tema omakorda suhtles sellise tegelasega nagu Lõvi\index[ppl]{Lõvi}. Lõvi on muidugi kõige olulisem tegelane üldse, tema juurest ilmselt algab kogu Eesti arvuti \emph{business}. Kui Jaak Loondest\index[ppl]{Loonde, Jaak} algab kogu Eesti arvutiteadvus, siis ma arvan, et Lõvist algab kogu arvuti-\emph{business} kuigi ta ise pole vist bisnest kunagi teinud. No ja Rainer Nõlvak\index[ppl]{Nõlvak, Rainer} ja kõik see nagu klikkis kokku. Ma saan aru, et Rainer oli Margusele teinud ettepaneku toimetada mingisugust ajakirja. Tema siis võttis mul varrukast kinni ja ütles, et \enquote{davai, nüüd on vaja ajakirja teha}. Mina muidugi pigem oleks mänginud arvutimänge, nagu ma olin teles harjunud, kus ikkagi üheksakümmend protsenti meie tegevusest oli arvutimängude mängimine. Aga noh, ma sain omale väga korrektse 486-e, ma arvan, ja selle peal jooksis Ultima Underworld\index{Mängud!Ultima Underworld}\sidenote{1992. aastal Blue Sky Productions'i poolt üllitatud Ultima Underworld oli väga mitmes mõttes (kolmemõõtmeline keskkond, simuleeritud mittelineaarne mängu käik jne.) teedrajav rollimäng.} ja oli täitsa tore. 

Toimetustegevusega mitte tuttava inimesena ma mõtlesin, et  alustama peaks ikkagi ajakirja esikaanest. Ja siis ma Corel Draw-s seda esikaant  hiirega joonistasin. Praegu tagantjärgi mõeldes mulle tundub, et ma joonistasin seda kuude kaupa. Tõenäoliselt see nii ei olnud aga sinna läks tohutu aur. Pronto\index[ppl]{Pronto} luges kokku, et neid numbreid nii väga palju ei olnud ja nendega läks suhteliselt palju aega. Ja kuna tegu polnud nagu otseselt ka äriline ettevõtmisega vaid .EXE oligi pigem promo, siis keegi nagu väga ei survestanud seda aja poolt ka. Meil ei olnud kohustust, meil ei olnud tellijaid, et ta peab nüüd iga kuu ilmuma.

\question{Aga kuskil Võrus istus üks nohik, kes kurvastas, et \enquote{miks ei ole tulnud veel .EXE't}!}

No vot, me ei adunud, et meil on \emph{impact}.

\question{Mõju oli kindlasti olemas. Võin omal näitel kinnitada, ja et ta Pronto panduna praegu niimoodi Internetis on\sidenote{\url{punktexe.ee}}, on ka selgesti märk mõjukusest. Seetõttu ma ka küsin.}

Ta oli oluline igas plaanis. Olles selles asjas sees, siis minu jaoks ei olnud  küsimus, et kas arvutid tulevad maailma muutma. Ma isegi ei mõelnud sellele, nendega oli lihtsalt hea asju teha ja  tõenäoliselt inimesed, kes ei teinud, olid ikka täiesti rumalad. Kõrvalt vaadates ma isegi ei saanud aru, kuivõrd vähe tegelikult arvuteid toona kasutati, sest me istusime MicroLinki peakontoris ja seal käis kogu aeg mingisugune sebilung. Telemajas ja igal pool, mul oli \emph{access} arvutitele päris hea. Aga ma mäletan, et .EXE  esimeses numbris  oli arhitekt Kalle Rõõmuse\index[ppl]{Rõõmus, Kalle} büroo niisugune väikene tutvustus  läbi selle, et nad hakkasid kasutama arvuteid projekteerimisel. See oli see midagi täiesti epohhiloovat. Ja ma isegi toona ei saanud sellest aru, kuivõrd imelik see  üldse on, et keegi teeb  paberil midagi. Ega  ma üldiselt  laksisin artiklid paika ja panin pildid külge ja mind võib olla väga ei huvitanudki, mis seal kirjas oli, välja arvatud need, mis ma ise kirjutasin. Aga tolles Kalle Rõõmuse büroo artiklis kirjutati, et üks arhitekt käis Kanadas stažeerimas. Kanadas tegeleti aga just sellega, et osteti personaalarvutid ja töö muutus efektiivsemaks võrreldes sellega, kui arhitektid ja konstruktorid päevad läbi kalka peale midagi joonistasid.  Järsku panid selle kõik arvutisse ja kõik oli nagu hästi. Tegelikult on huvitav vaadata seda, et tänaseks on meil see nii-öelda BIM-modelleerimine\sidenote{\emph{Building Information Modeling - BIM.} Protsess, mille käigus füüsilisi ruume käsitletakse digitaalsete vahenditega.} ja siis sa kuuled, mis väljakutsed sellega seoses on. Mul üks sõber töötab startupis, mis tegeleb BIM-mudelite konfliktide analüüsiga. Üritavad aru saada, et näiteks ventilatsioonitoru ei tohi läbi akna minna. Ja siis sa mõtled, et \enquote{issand jumal, millega need inimesed on tegelenud, miks nad seda arvutit pole varem kasutusele võtnud?}. Kui palju on aega raisatud!

.EXE\index{.EXE}, olgugi, et temast jäi mulje, et ta on ikka \emph{super advanced} ja mingi häkkerite värk, üritas anda pilti sellest, mis tegelikult toimub. Et arvuti ei ole ainult raamatupidaja kalkulaator. 

\question{Kuidas sa joonistamise juurest kirjutamise juurde jõudsid?}

Oli vaja ju \emph{content}'i toota ja ega keegi toona ei olnud arvutiajakirjanik. Ja  mulle meeldis arvutimänge mängida ja ma arvan, et kirjutamine on iseenesest tore tegevus. 

\question{Kas sul juba kooli ajal lõi kirjutamise ja kirjandi soon kuidagi välja?}

Ei, ma olen võimeline kirjutama okeilt. Mulle joonistada meeldib võib-olla rohkem, sest kirjutamine on selline raske asi, et sa pead laused läbi mõtlema ja siis sulle tundub, et nad ei ole head. On nagu liiga
konkreetne formaat.

\question{Teema jätkuks veel üks oluline küsimus. Mõni aeg tagasi Tõnis Kahu (keda tuleb ilmselt uskuda)\index[ppl]{Kahu, Tõnis} lükkas ümber mu arusaama sellest, misasi on küberpunk. Aga minu vastav arusaam tuleb ühest konkreetsest .EXE artiklist, kus on sinu ja Pronto\index[ppl]{Pronto} nimed all\sidenote[][-5cm]{Kaspar Loit. (1994). Kes sa selline oled, küberpunk? .EXE, (3), 60-63. Kontrollides Pronto nime artikli juurest ei leia. Küll aga leiab sedastuse, et \enquote{\ldots õiged küberpungid lasevad selliste loetelude peale suht laias kaares}.}. Räägi nüüd ära, kuidas te tolle sisu produtseerisite}

Väga raske öelda tagantjärgi. Aga eks meil oli mingi ettekujutus. Ega küberpunk ei ole mingisugune geneetiline  organism, mis on välja arenenud ja siis pärast on hea klassifitseerida, et  pool on hüljes ja pool on mingisugune gepard. Ma eeldan, et me toona juba teadsime juba Gibsoni \enquote{Neuromancer}'it\sidenote[][-5.5cm]{\enquote{Neuromancer} on William Gibsoni 1984. aastal ilmunud romaan, esimene tema \emph{sprawl}'i triloogiast. Romaani loetakse \v{z}anri üheks mõjukamaks ning on ainus romaan, mis on võitnud nii Nebula, Hugo kui ka Philip K. Dick'i auhinna.}. Kui kõik räägivad Hichiker Guide'st\sidenote[][-3cm]{Douglas Adams, \enquote{The Hitchhiker's Guide to the Galaxy}. 1978. aastal  raadiokuuldemänguna alustanud komöödia, mis avaldati viieosalise raamatutriloogiana ning millele kuuenda osa lisas pärast autori surma avaldamata materjali põhjal Eoin Colfer. Sarja raamatud levisid tekstifailidena laialt BBSide ja interneti vahendusel olles seega ka siinmail kergesti kättesaadavad.}, siis see oli väga oluline teos. Aga noh, minu jaoks Gibsoni \enquote{Burning Chrome}\sidenote{\enquote{Burning Chrome} on William Gibsoni 1982. aastal ilmunud novell, kus tutvustatakse \emph{sprawl}'i triloogia maailma ja mille sündmusi ning tegelasi mainitakse triloogias korduvalt.} ja \enquote{Neuromancer} lasid ikkagi aju täiesti välja.

\question{Ma siiamaani loen neid asju regulaarselt üle. Härra Gibson kirjutas need raamatud trükimasinaga paberi peale ja aastal 2019 täpselt nii ongi!}

Oled sa tema uuemaid raamatuid ka lugenud? Need lähevad veel hirmuäratavamalt tõepärasemaks  ja ajahorisont tuleb üha lähemale.

\question{Siit siis loogiline küsimus, et pidi ju olema mingi allikas, te ei mõelnud ju küberpungi mõistet (mida Gibson ei maini) ise välja? Olid teil välismaa BBS-id, internet?}

Ma arvan, et kõik see nimetatu klikkis kuidagi kokku, tõenäoliselt. Ma ei oska Pronto\index[ppl]{Pronto} eest rääkida, aga \enquote{Blade Runner}\sidenote{\enquote{Blade Runner} on 1982. aastal linastunud Ridley Scott'i film, milles peaosa mängib Harrison Ford ja unustamatu improviseeritud lõpumonoloogi esitab Rutger Hauer. Film toetub lõdvalt  Philip K. Dick'i 1968. aasta, samuti klassikaks peetavale, novellile \enquote{Do Androids Dream of Electric Sheep?}.} on, eksole, eepiline nurgakivi, Sid Meier\sidenote{Sid Meier on küll legendaarne arvutimängude autor, kuid mitte futuroloog. Ei ole selge, keda Kaspar silmas peab.} oli  see futuroloog, kes joonistas ilusaid düstoopilisi pilte .See kõik kujundas meil välja mingisuguse düstoopilise arusaama tehnilisest maailmast, kus kõik on kõige külge ühendatav. Kasvõi \enquote{Battle Angel Alita}\sidenote{\enquote{Alita: Battle Angel} on 2019. aastal linastunud  Robert Rodriguez'i film, mis tugineb Jaapani mangakunstniku Yukito Kishiro 1990. aastate sarjal \enquote{Battle Angel Alita}}, mis täitsa juhuslikult praegu kinno jõudis, onju. Ma arvan, et vähesed inimesed Eestis teavad seda originaallugu, ma olin selle toonane totaalfänn. Ma käisin aeg-ajalt Helsingis Akadeemilises Kirjakauppa's\sidenote{
Akateeminen Kirjakauppa on Helsingi kesklinnas asuv kuulsusrikas raamatupood, mis just kõikvõimalike servapealsete huvidega eestlasi pikki aastaid raamatutega varustas. Palverännak sellesse poodi oli ka minu Helsingi-käikude lahutamatu osa.} ja seal kogu aeg vaatasin, et kas uus osa on tulnud. Üheksa raamatut, mul on nad kõik olemas. Kuidas mingid metalltorud lähevad su silmamuna sisse ja ajust on  järgi ainult mingisugused \emph{chip}'id ja natukene pudru. See kuidagi kujundas meid, me elasime selle asja sees, ma arvan. Mängumaailmas tõenäoliselt olid paar mängu jälle, mis kuidagi sinna kontributeerisid. 

Pluss on see, et me muidugi üritasime  siis ka mänge teha. Enne veel, kui me Bluemoon'iga\index{Bluemoon} \sidenote{Kaspar peab ilmsesti silmas Bluemoon Interactive nimelist ettevõtet, millest on juttu leheküljel \pageref{sisu!bluemoon}, ja mitte samanimelist Londoni ööklubi (mille ees Ahtit ja Jaani korduvalt pildistatud on)} midagi tegime üritasime Amiga maailmas  Ott Aaloe\index[ppl]{Aaloe, Ott} ja Juhan Soonetsaga\index[ppl]{Soomets, Juhan} midagi korda saata. Tegime Rocketsi-nimelise\index{Mängud!Rockets} mängu, mille Bluemoon pärast keeras PC peale palju ägedamana, aga võib-olla vähem ägedanda. Selle mängu intro, ma mäletan, oli väga selgelt kantud kõigist neist vee kaheksatest ja rakettidest ja nii edasi. Väga tähtis oli kindlasti see, et päikseprillid olid õige kujuga. Andrus Aaslaid\index[ppl]{Aaslaid, Andrus} sinna kõrvale rääkis või kirjutas lugusid, kuidas plinkiva valgusega saab sul aju ümber programmeerida. See kõik nagu absorbeerus ja tekitas mingisuguse omaette alternatiivse reaalsuse, ma arvan, et see on see meie arusaam küberpungist. 

\question{Kui teie tuleviku ettekujutuses oli ajust järel natuke putru ja palju kiipe ja tulevik oli muidu ka düstoopiline, siis miks te sellele vaatamata pika sammuga tolle tuleviku suunas astusite?}

No aga seda tagasi hoida on tõenäoliselt mõttetu, ludiidid ka üritasid. Aga parem on olla kohal enne teisi. Et sa paned juba õiged \emph{chipid} omale õigesse kohta  ära ja võtad selle pudru osakaalu väiksemaks. 

\question{Et üheksakümnendatel selliseid mõtteid mõelda, oli ikka korralikku visiooni tarvis. Aga räägime Bluemoonist: kuidas sa Ahti\index[ppl]{Heinla, Ahti}, Jaani\index[ppl]{Tallinn, Jaan} ja tolle pundiga kokku sattusid?}

Toona igaüks mõtles, et ta on super-häkker. Nii see, kellel oli kastis see MSX-i manual kui see, kes oskas neid faile kokku panna. Ma olin kogu selle niiöelda super-häkkerite seltskonnas üks väheseid tegelasi, kes joonistas pilte. Ma  tegelikult oskan ka ilma arvutita päris hästi joonistada, arvutis tundus see lihtsalt  kuidagi nagu lahedam, seal sai asju salvestada. Ja sai \emph{undo} teha, see on nagu põhiline. Kui sa lihtsalt joonistad, siis \emph{undo} teha on väga raske, isegi võiks öelda, et peaaegu võimatu. Jällegi, see seltskond ei olnud nagu nii suur ja nad kõik nagu \emph{connect}'isid kuidagi. 

Teen korra kiire kõrvalehüpe. Tuli meelde, kuidas BBS-ide ja värkidega suheldi, et meil sealsamas Telemajas oli täpselt samasugune ambitsioon. Meil olid Amigad aga mänge ju ei olnud, siis pidi neid mänge kuskilt pirama. Ma loodan, et tagantjärgi mingid piramisasjatundjad ei hakka peale lendama. Aga igal juhul üks viis neid mänge saada oligi see, et sa pidid jõudma kuskile mingisugusesse BBS-i ja kuidagi sinna sisse pääsema. Ega seal ei olnud niimoodi, et \enquote{astu sisse}. Seal tavaliselt istusid mingid vennad, kes monitoorisid tegevust. Eesti tundus eksklusiivne, sihukene veider koht, sama hea kui eskimod. Mingi hetk meil isegi tekkis mingisugune \emph{trading capacity}, et meil juba oli midagi, mida  vastu pakkuda. Aga tavaliselt me ikka mängisime sellist vaest sugulast ja siis me isegi nagu \emph{bluebox}'isime\sidenote{Ennevanasti liikusid instruktsioonid telefonikeskjaamadele konkreetse kõne kohta samas kanalis, kui kõne ise. Seega, tõstes toru ja vilistades  õigeid signaale, võis muuta kõne teekonda keskjaamade vahel ja, mis kõige olulisem, saada mööda kõnetasudest. Kõvemad spetsialistid, nagu Joe \enquote{Joybubbles} Engressia suutsid kaugekõneliini lähtestamiseks vajalikku 2600 Hz signaali suuga vilistada. Natuke nõrgemad, nagu John \enquote{Captain Crunch} Draper, vajasid tehnilisi vahendeid. Lihtsurelikud aga kasutasid elektroonilisi seadmeid. Neist esimene, mille ehitas 1960. aastal Robert Barclay, oli pakendatud sinisesse kesta, sealt ka mõiste \enquote{blueboxing}.} ennast sinna sisse. 

\question{Oot, räägime nüüd sellest lähemalt. Kas see tähendas seda, et sa pidid toonasele telefonikeskjaamale kõrva vilistama midagi, mida too tingimata kuulda ei tahtnud?}

Kusjuures nüüd, kui ma hakkan mõtlema, siis Margus\index[ppl]{Kliimask, Margus} oli põhiline \emph{bluebox}'i spetsialist, aga kas me ka reaalselt \emph{bluebox}-imiseni ka jõudsime, see on nüüd hea küsimus.

\question{Ma mäletan, .EXE's ilmus manuaal selle kohta\sidenote{Mark Tabas. (1993). Blueboxing parimates peredes. .EXE, 1-2. Tegu on nähtavasti tõlkeartikliga, Mark Tabas oli legendaarse häkkerirühmituse Legion of Doom asutajaliige}, mis oli jube huvitav lugeda}

Jaa. Põhimõtteliselt see on ju iseenesest üsna lihtne, kuna toona nood keskjaamad olid suhteliselt rumalad.  Võtame kasvõi selle, kui kiired olid modemid. Ma mäletan, et oli kolmesajane Hayes. Ja seda, et minu meelest kas Mast\index[ppl]{Kaal, Madis} või keegi, oli väidetavasti suuteline selle modemi \emph{handshake}-i ära vilistama sellele modemile. Ta  oli  piisavalt aeglane, talle sai põhimõtteliselt suusõnaliselt selgeks teha, mis sa tahtsid.

Võib-olla oli see, et igaühel meist oli oma fookus, on ju. Kes tahtis rohkem seal mingit \emph{network}'i häkkida, kes tahtis rohkem lihtsalt häkkida, kes tahtis progeda. Mind huvitasid selgelt mängud, liikuvad pildid, värvilised pildid, kuidas neid ise teha, 3D, kõik niisugune värk. Ma pigem nagu otsisin neid võimalusi. Eks see viis meid ka tegelikult kokku siis lõpuks Bluemoon'i\index{Bluemoon} pundiga, kellel oli  kindel soov, et nad tahavad teha mängu. Minu jaoks oli see lihtsalt natukene niisugune nõme ülesanne, kuna mul oli Amiga\index{Arvutid!Amiga}, seal olid miljonid värvid ja neil oli mingisugune EGA (alguses vist isegi CGA) ja sinna pidi mingi nelja värviga midagi valmis nikerdama. \emph{Why not}, teeme ära. Sellest sündis siis Kosmonaut\index{Mängud!Kosmonaut}. 

See, kuidas nad turustasid ja toimetasid, ma nagu lihtsalt vaatasin ja imestasin. Jube lahe oli. Mul on meeles kuivõrd \emph{dedicated} need vennad toona olid ja on vist siiamaani. Nad olid nagu tõesti  \emph{focused} oma asja tegemisele. 

Aga see graafiline pool oli super lihtne. Nokkisin selle valmis, siis nad tegid oma selle musa \emph{editor}-i, sinna ma nokkisin ka mingisugused ikoonid, mingid kitarred ja  trummid ja väga lahe oli.

\question{Seda ma mäletan küll, et inimesi, kes oskasid arvutiga joonistada, oli vähe. Kas sul kõigepealt tuli arvutiga joonistamine ja siis joonistamine või oli sul enne ka joonistamise huvi?}

Ma ikka enne joonistasin ka. Kes ikka ennast kiidab, kui mitte ise, eks ole. Akadeemilist joonistamist ma valdan  suhteliselt väga hästi. Selles mõttes mul ei olnud nagu keeruline neid oskusi omandada. Toona ei joonistatud tabletitega, vaid oli seesama munaga hiir, mille muna aeg-ajalt jooksis mingit pahna täis ja sa pidid teda küünega puhastama. Ma ikkagi endale muidugi otsisin mingi hiire, mis nagu enam-vähem jooksis. Selles mõttes minu jaoks ei ole vahet, kas see on pliiats või hiir või tablet või mis iganes. 

\question{Arvuti oli sinu kunsti tegemise nii-öelda laiendus}

Jah, ta oli lihtsalt nagu mingi teistsugune tehnika ja tunduvalt andeksandvam, kui näiteks akvarell. Täna sa vaatad, et  kõik kunstnikud kasutavad mingit Cintiq'u tabletti, neil on kõik super ägedad toolid. Ma siiamaani aeg-ajalt, kui mul on vaja midagi nikerdada kasutan oma läppari \emph{touchpad}'i ja kõik vaatavad, et ma olen peast soe. See on mugav ja käe järgi \emph{tool} tegelikult, kui sa ta ära omandad ja keerad kursori piisavalt kiireks.

\question{Üks hetk sul tekkis mõte, et võiks hakata veebi tegema?}
 
See oli pigem läbi selle, et ma olin aru saanud, et ma ei ole piisavalt järjepidev ja see progemise osa  tundus toona liiga kuiv. Mul olid sõbrad, kes sellega tegelesid ja tulemus ei olnud nagu seksikas. Veeb oli alguses ka super \emph{boring}, see Mosaic, või mis see esimene brauser oli, oli ikka \emph{ugly as hell}. Aga siis kui ma sain aru, et tabelitel saab keerata ported maha ja  mingite üksikute ühe-piksliste tükkidega hakata mingit \emph{layout}'i tegema,  ma olin müüdud mees. Ma sain mitte lihtsalt enam selle pildi oma käe seest ekraanile, vaid ma sain selle pildi nagu pauh kõigile nina ette, eks ole.  Töötasin ühes reklaamibüroos ja midagi  seal katsetasin. Mindworks\index{Mindworks} oli juba olemas ja selle asutajad siis vaatasid, et ma jagan natuke sellest reklaami-bisnesist ka, et paneme seljad kokku. Aasta oli 1996 või 1997, me olime umbes kakskümmend viis, me polnud enam niiväga poisikesed ja sai juba bisnest teha. Lõikasid neid pikseleid \ldots Ma mäletan, meil oli selline klient nagu Reval Hotel Group, mingid nagad tulid ja võtsid sihukesed kliendid ja tegid neile ägedaid asju.

\question{Ja jätkuvalt oli sind liigutav faktor see, et sa said oma pildi inimestele silma ette?}

Tegelikult mul ei oleks isegi vahet, kas inimestele silma ette, vaid just nimelt see, et sul oli mingisugune distsipliin, HTML. Ja sa teadsid, kuidas optimeerida GIF-e, sul oli mingisugune \emph{toolset} ja sa valdasid seda suhteliselt hästi. Tekitas rõõmu, et  sai teha mingisuguseid asju, mida võib-olla teised ei osanud teha. Sihuke \emph{job satisfaction}'i värk. Ma kujutan ette, et muru niitmine on ka selles mõttes lahe, et sa näed, kuidas niidetud muru jääb su taga maha. Sihuke suhteliselt \emph{instant gratification}.

\question{Mis sa praegu teed?}

Ma olen kuidagi liigselt distantseerunud sellest disaineri rollist aga samas mitte. Ajapikku ma olen aru saanud, et selle pildi tegemine on mõnes mõttes niisugune käsitöölise töö. Tegelikult need lõikelauad, mida minu lapsepõlves turul müüdi, kus  põletiga oli tehtud \begin{russian}ну погоди\end{russian} peale, on ju nagu veidi sarnane. Palju  šeffim on tegelikult võtta ja aru saada mingitest äriprotsessidest või mingisugusest inimeste mõttemallidest ja disainida neist midagi. Progemine minu jaoks on see, et kui asi õigesti sõnastada, siis mingisugused vennad teevad selle valmis ja see muutub nagu päriseks, see  protsessi toetav  asi seal masina sees toimetab täpselt nii nagu sa oled talle  öelnud, et toimeta. 

\question{Nüüd mitte ei tule käe seest pilt vaid tuleb sinu pea seest mõte, kuidas see kupatus võiks käia, programmeerijad teevad selle valmis ja siis käibki nii}

Just. Lapsepõlves kuidagi, ma mäletan selgelt, mind võlus mõte, et tehas on tore asi, sest et ta võtab mingisuguse toorme ja detailid ja siis neist pannakse kokku mingi asi. Ja jällegi me jõuame sedasama  Nintendo \emph{device} juurde, et füüsilisel kujul seda toota on jõle tüütu. Palju lihtsam oleks teha seesama asi nii, et oleks bittide jada, mis kõik grupeeruvad, moodustavaid mustreid ja sellest peaaegu nagu võluväel tekivad mingisugused asjad, mis inimestele tegelikult on tänaseks sama reaalsed tööriistad kui haamer ja höövel.
