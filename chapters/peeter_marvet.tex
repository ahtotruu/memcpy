\index[ppl]{Marvet, Peeter}
\index[ppl]{Tehnokratt|see{Marvet, Peeter}}

\question{Kuidas ja umbes millal sa jõudsid arvutite juurde?}

Ma arvan, et see oli umbes täpselt aastal 1985, ma pidin siis olema 15 aastat 
vana. Ilmselt ma olin arvuteid eelnevalt näinud Soome telekast, ma usun, et 
seal reklaamiti ilmselt Commodore 64 ja Spectrum masinaid. 

\question{Sa oled siis Tallinna poiss, ma järeldan?}

Jah. Ma olen sündinud Tartus aga pikalt olnud Tallinnas. 

Aga kui ma veel tagasi krutin, siis ma arvan, et kokkupuude arvutitegaa oli 
veidi varem, mu papsi laboris. Laboriks oli toonase TPI Veekvaliteedi 
Labor\index{Tallinna Tehnikaülikool!Veekvaliteedi labor}, asus selle koha peal, 
kus keset Järvevana teed on praegu Maru ehituse maja. Seal oli olemas üks 
terminal, mis käis Datasaabi\index{Arvutid!Datasaab} nimelise 
arvuti\sidenote{Datasaab oli Rootsi lennukitootja Saab arvutustehnoloogia 
eraldi ettevõtteks kasvanud divisjon. Toodeti nii tsiviil- kui 
militaarkasutuseks mõeldud arvuteid.} külge, mis asus kusagil Mustamäe teel. 
See masin oli ostetud ühelt nii-öelda Rahvamajandussaavutuste Näituselt, kus 
meil vahetevahel käisid väljakad ka kohal. Oli saadud raha ja ostetud valuuta 
eest välismaa arvuti, mille külge käisid modemitega sellised oranžid või 
\emph{amber} värvi terminalid. 

Datasaabi lugu oli selline, et see osteti väidetavasti ilma 
operatsioonisüsteemi ja igasuguse rakendustarkvarata. Selle jaoks nagu rohkem 
rutsi sellel ajal ei olnud. Aga Nõukogude insenerid olid vinged, nõukogude 
insenerid kirjutasid sinna ise mingi operatsioonisüsteemi peale. Või siis pandi 
midagi kusagilt tuuri. Ei, ma arvan, et tuuri ei pandud, sest mulle meenub, et 
vist õnnestus sellest kirjutatud tarkvarast veel mingisugune jupp Datasaabile 
tagasi müüa ja saada selle eest ilmselt siis vastu mingisugust mälu või 
lisakomponente. 

Tollel Datasaabi terminalil, millega ma esimesena kokku sattusin õnnestus 
lihtsalt ilma ühenduseta niimoodi \emph{backspace}-i ja  tühikuga 
\enquote{ronge kokku haakida}. See on minu esimene mälestus sellest, kuidas ma 
olen arvutiga suhtestunud. 

Aga järgmine mälestus on  aastast 1985, kui ma ise olin viieteist aastane, 
ilmselt seitsmendas klassis. Toonases Pedas toimus üks Tallinna koolide 
füüsikavõistlus ja meid viidi lisaks sellele füüsika asjale ka arvutisaali. Ma 
arvan, äkki oli see  Minsk\index{Arvutid!Minsk}. Seal ma sain tuttavaks ühe 
koolivennaga, kes oli minust aasta vanem, ehk siis juba kaheksandas klassis, ja 
kellel oli kaasas isiklik perfolint programmiga. Selleks koolivennaks oli ei 
keegi muu kui Sulo Kallas\index[ppl]{Kallas, Sulo}. Oma perfolindi peal oli tal 
mingisugune selline mängulaadne asi, mis minu arust ise  arendas mingisuguseid 
organisme või midagi sellist\sidenote{Tõenäoliselt oli  lindil Briti 
matemaatiku John Horton Conway poolt välja mõeldud rakuautomaat, mida tuntakse 
nime all \emph{Game of Life}. Tegu on mängijateta mänguga, mis ainsa sisendina 
vajab algseisu määratlemist. Automaat on ühest küljest levinud 
programmeerimisülesannne ja teisalt  põnev uurimisobjekt, seetõttu võis selle 
realiseerimine olla noorele arvutihuvilisele  nii huvitav kui ka jõukohane.}. 
Sulo lugu  oli selles, et tema vend oli Raadiomaja 
Arvutuskeskuses\index{Raadiomaja Arvutuskeskus}, nii et selleks ajaks oli Sulo 
juba mõnda aega arvutitega tegelenud. 

Ja see oligi nagu esimene kord, kus sattusin arvutiga kokku ja mõtlesin 
\enquote{oo, vinge!}.

\question{Mis seal vinget oli, mis konks see oli, mille külge sa jäid?}

Tol hetkel  oli see rohkem selline nagu \enquote{ahaa, vau, et teebki 
mingisugust asja!}. 

\question{Aga Sulo oli kõva mees küll oma perfolindiga?}

Nojaa, sul on nagu kaheksandik ees,  nagu vanem koolivend, kellel on,  kujutad 
sa ette, isiklik perfolint! Vau! Sellised kutid on ümberringi! No siis peab 
ikka ka vaatama, et mis seal tehakse. Ma arvan, et  midagi Soome telekast 
arvutitega seoses nähtut oli tegelikult see, mis tekitas soovi, et olgu või 
nõukogude oma ja perfolindiga, aga ikkagi arvuti. 

Ja järgmine asi tegelikult ei olnud sellest kaugel. See oli, vast mõni kuu või 
natukene hiljem, kui kooli tulid paar djuudi, kes tekid mingit arvutiklubi ja 
kutsusid sina osalema. 

\question{Kas see oli legendaarne arvutiklubi Ahhaa?}

Ei. See on legendaarne arvutiklubi Juta\index{Arvutiklubi!Juta}, mida vedas 
juudi papi nimega Lev Moišeejevitš Šoroht\index[ppl]{Šoroht, Lev}. Klubi 
tegutses Raua ja  Kreutzwaldi tänava nurga peal. Kui mööda Raua tänavat  
kesklinna poolt tulla, siis vasakut kätt on ühel majal keldrikorrusel kaares 
aknad. Vaat sealsamas kaares akende taga vedas ennast Arvutiklubi Juta. Kui 
Ahhaa puhul võiks kujutada ette, kust see nimi tuleb, siis kust tuleb nimi 
\enquote{Juta}? See tuleb vene keelest: \begin{russian}Юный Техник 
Автомат\end{russian}. Ma eeldan, et Lev Moišeejevitš Šoroht ei satu seda  
kuulama, aga kui  kellelgi meenub, et aastal 1985  on ta kas TPI või ka Peda 
üliõpilasena vedanud mingisuguseid noori arvutiklubisse, siis ma suurima hea 
meelega saaks kokku ja teeks väikesed õlled välja. Sest sealt see alguse sai. 

Klubis õpetati meile programmeerimist keeles PL/I\index{Keeled!PL/I}.

\question{Ja klubisse käidi kutsumas, mitte ei joostud ust maha, et arvuti ligi 
saada?}

Ma arvan küll. Ega ma täpselt ei mäleta, aga ma arvan, et see sõnum jõudis 
meieni kuidagi kooli või õpetajate kaudu. Ja ma arvan, et Sulo oli ka seal, 
sest et kui oli võimalik kusagil veel arvutisse saada, siis loomulikult seda 
kasutati.

\question{Seda ma mõtlengi, et tol ajal ju otsiti tikutulega neid kohti, kus 
\enquote{arvutisse saada}?}

Kusagil seitsmenda klassi klassi lõpupoole tuli  arutuskeskusest ühe minu 
programmi väljatrükk, laia aukudega paberi peal, \emph{line} printeri peal 
välja lastud. Keegi mu klassivendadest, kes siis ilmselt oli ka meiega seal 
koos käinud, tõi selle väljatrüki mulle klassi ja siis kõik vaatasid, et 
\enquote{oo, Soome telekavad}. Sest et tollel ajal oli inimestel selline kõige 
üldisem seos arvutitega see, et arvutuskeskustes trükiti välja Soome 
telekavasid. Neis olid näiteks tabuleeritud kujul eraldi väljavõtted, kust 
seriaalid jooksid, parimatel vendadel olid olemas nädalakavad. Aastal 1985 
keskmine teadlikkus arvutitest umbes selline oligi.

\question{Kust need kavad saadi?}

Nad liikusid arvutuskeskuste vahel ja ma tean, et (jällegi nüüd täpselt 
ajaloolisi hetki kokku panemata), vähemasti mingil hetkel oli üheks selliseks 
kohaks Postimaja Arvutuskeskus\index{Postimaja Arvutuskeskus}. Mis on 
iseenesest üks väheseid kohti, kus ma ise pole käinud. Seal oli mäletamist 
mööda SM-4\index{Arvutid!SM-4}, mille külge oli ehitatud mingisugune teksti-TV 
vastuvõtja\sidenote{Kavade allikaid oli rohkem kui üks. Mäletatakse, et vastav 
riistvara oli olemas TPI Raadiotehnika Kateedris\index{Tallinna Tehnikaülikool!Raadiotehnika kateeder} 
Apple II küljes. Räni Meister\index[ppl]{Meister, Räni} olla selleks otstarbeks kasutanud 
ka Eesti Televisiooni\index{Eesti Rahvusringhääling!Eesti Televisioon} Amigat.}  
ja sealt see kava tuli. Ma arvan, et see oli mõni aasta hiljem: kui 
ühes kohas oli Minsk ja teises SM-4, äkki ma ajan mingisugused ajajooned 
natukene sassi. Selle SM-4  küljes oli kolmesajaboodine modem, millega siis 
pumbati neid kavasid mööda linna laiali. Ma mäletan mingisuguste järgmist 
etappi sellest, kus  minu päralt oli üks PC, välja arvatud vist kolmapäeviti, 
kui lõuna paiku saabus Postimajast üle modemi Soome telekava ja siis maatriks 
printeri peal enam-vähem nähtamatuks kulunud lindiga trükiti seda välja. Siis 
ma pidin omale  muud tegevust leidma, muudel pärastlõunatel ma sain seda 
arvutit kasutada.

\question{Ma katkestasin sind seal, kus sa PL/I keeles programmeerisid\ldots}

Meile õpetati natukese programmeerimist ja seal oli palju hästi segaseid ja 
täiesti arusaamatuid asju. Oli programmeerimiskeel mingisuguste  muutujatega, 
mis nagu mõnevõrra koitis. Ja  ilmselt esimene programm, mida meile seletati, 
kuidas teha, oli ruutvõrrandi lahendamine. Annad paar muutujat sisse ja siis 
trükitakse sulle tulemus mingisuguse käraka  paberi peal välja. Alguses meid 
vist nagu päriselt arvuti juurde ei lastudki: Keegi  \emph{punch}-is meie 
programmid sisse ja pärast saime väljundi kätte. 

Pärast meile leiti võimalus arvutitega tegeleda veel kahes kohas. Üks oli 
Tihnikus, kus oli ETKVL-i Arvutuskeskus\index{ETKVLi Arvutuskeskus}. Järgmised 
põlvkonnad teavad seda kohta kui esimest Maksimarketit, seal ühes majas oli üks 
vinge ES\index{Arvutid!ES EVM}. Teine koht oli see, kus Endla tänaval asub (või 
vähemalt mõni aeg tagasi asus) Maksuameti üks ots\sidenote{2013. aastani asus 
Maksu- ja Tolliameti teenindussaal aadressil Endla 8.}. Seal  üleval kolmandal 
korrusel olid Ehituskomitee\index{Ehituskomitee} ES-id\index{Arvutid!ES EVM}. 

Minuga läks sealt edasi läks umbes niimoodi, nagu õpetatakse tööõpetuses, et on 
oluline  anda lastele mingisugune selline asi, mille nad saavad valmis voolida, 
tulla koju ja perele näidata. Siis laps saab kiita, tal  läheb edaspidi väga 
hästi ja ta teeb paremaid puulusikaid. Minuga läks niimoodi, et olles teinud 
oma esimese  kolmeteist-realise programmi ruutvõrrandi lahendamiseks, laekusin 
ma  selle väljatrükiga koju ja köögis näitasin siis  lapsevanematele, et 
\enquote{näe sihukse raha eest tegin sihukse asja}. Paps, kes oma 
teadustegevuse poole pealt tegeles elektrokeemia ja hapnikuanduritega ja teise 
poole pealt oli džässpianist vaatas mu tööd ja ütles, et \enquote{mul oli siin 
just mingisugune tudeng, aga kadus ära. Temast jäid ainult mingid listingud 
järgi. Kas sa saad sotti,  mul oleks vaja mingisuguseid  teadusandmeid 
töödelda.} Ülesandeks oli mingisuguste anduri toimekõverate kokkuajamine 
mingite matemaatiliste valemitega, et õnnestuks digitaalseid mõõteriistu teha 
või midagi sellist. Ja niimoodi juhtuski, et olles  programmeerinud oma 
esimesed kolmteist rida esimeses mulle täiesti tundmatus keeles, 
\emph{switch}-isin ma koheselt järgmisele keelele.

Nii ongi mu pea selles mõttes nagu puder ja kapsad, et ma suudan kirjutada 
ainult dokumentatsiooni abil, kaasa arvatud keeli, mida ma igapäevaselt kipun  
kirjutama nagu  PHP\index{Keeled!PHP} ja JavaScript{Keeled!JavaSctipt}. Need 
süntaksid on peas nii segi selles mõttes, et ma kunagi ei mäleta, täpselt mis 
oli PHP-s  \verb|for| tsüklis  parameetrite järjekord. Õnneks on tänapäeval 
olemas kõikvõimalikud IDE-d, mis teevad mõningase töö ära ja aitavad \emph{auto 
complete}-da ja mida iganes. 

\question{Aga siis su puulusikas mitte ainult ei saanud kiita vaid pandi kohe 
ka tööle!}

Puulusikas võeti kohe tööle, sealt edasi ma olen olnud lapstööjõud. Mingil 
hetkel juu papsil hakkas nagu kahju ka, et laps võiks  lisaks 
ekspluateerimisele natukese ka raha saada. Ehk ma olin siis ametlikult kirjas 
mingi veerandkohase laborandina või midagi sellist.  Parem pool oli muidugi 
see, et tänu sellele oli mul ligipääs kõikide papsi sõprade arvutuskeskustele. 
Ja kuna paps tegeles selle oma hapnikuga TPIs, siis loomulikult olid nende 
sõprade hulgas TPI ja ports seltskondi, kes olid seotud mingisuguste 
anduritega. Näiteks see pool, mis tegutses kunagi Pirital Masti tänaval,  kus  
aretati sportlaste mõõtmise lahendusi. Selle mingisugune teine ots asus 
Kiirabihaigla arvutuskeskuses\index{Kiirabihaigla arvutuskeskus}. See oligi see 
koht, kus mul oli üks arvuti pidevalt kasutusel, välja arvatud kolmapäeviti. 
Sanyo PC, väga vinge. Seal oli muuseas ka Apple II\index{Arvutid!Apple II} 
olemas, mille peal sai mängitud Karatekat\index{Mängud!Karateka}, minu arust 
mingeid muid muid toredaid rakendusi tolle masina peal ei olnud. Ja kui ma 
hästi mäletan, oli seal ka mingisugune Labtam-i\sidenote{Labtam oli suhteliselt 
obskuurne Austraalia arvutitootja, kes tegutses aastatel 1972 kuni 1990. 
Miskipärast olid neil oma lõpu-aastatel Nõukogude Liiduga head suhted: aastal 
1984 disainis Novosibirski Riikliku Ülikooli tudengite Kronos Research Group 
neile emaplaadi, URAL-LABTAM OOO tegutseb Venemaal siiani ning nende arvuteid 
leidub lisaks Austraalia arvutimuuseumidele ka Tartu Ülikooli omas. Labtami 
arvuteid osteti naftadollarite eest ka Küberneetika 
Instituuti\index{Küberneetika Instituut}} nimeline \emph{kone}. Lisaks, oli 
seal mingisugune selliste suurte trumlitega andmetöötlus-\emph{kone}, mille 
nime ma ei mäleta. Seda äkki oskab Kalle Lotamõis\index[ppl]{Lotamõis, Kalle} 
või keegi selline  meenutada, et mis see oli. Mina selle suure masinaga sellel 
hetkel ei suhtestunud. 

\question{Kas sulle see andurite maailm ja elektroonika ei  pakkunud huvi?}

Mitte liigselt. Progemise pool oli  huvipakkuvam. 

Siis oli üks keskus, mis oli TPI Santehnika Kateedri\index{Tallinna 
Tehnikaülikool!Santehnika Kateeder} SM-4\index{Arvutid!SM-4}. Ehituse all oli 
selline kateeder, vee kvaliteet ja kõik selline kuulus  sinna alla. Mingil 
hetkel tekkis sinnasamasse Järvevana teele, kus asus ka Läänemere 
Instituut\index{Läänemere Instituut}\sidenote{Ei ole selge, mis asutust Peeter 
silmas peab. Eestis on Läänemere Instituut tegutsenud eelmise sajandi 
kolmekümnendatel ja praegu tegutseb sellenimeline asutus Soomes.}, ka 
SM-4\index{Arvutid!SM-4}, mis oli  enam-vähem niisugune masin, mille ma kohale 
minnes  lülitasin ise sisse ning pärast töö tegemist jälle viisakalt välja. 

\question{Neid arvutid siis ikkagi ju oli?}

Neid oli selles mõttes, et kui sa nagu sattusid õigesse kohta ja ilmselt 
oskasid õigel ajal vait olla ja mitte liiga palju täiskasvanuid  segada nende 
tähtsas töös,  siis üldiselt neid nagu jagus. 

Tulles korraks veel tagasi  alguse, ehk Juta\index{Arvutiklubi!Juta} juurde 
siis selle asutajal oli endal ka paar niisugust  huvitavat projekti, millega ta 
üritas Nõukogude Liidu tasemel kuulsaks saada. Üks neist võiks olla  võrreldav 
Facebookiga. Selline üleliiduline projekt, kuidas inimesed, kirjasõbrad, 
saadaksid oma andmed kõik kokku, need sisestatakse perfokaartide peal 
\emph{mainframe}-i ja see kuidagi teeb mingisugust \emph{match-making}-ut ja 
siis saadetakse kirjad leitud \emph{match}-idele laiali. 

Mina tegelen igasuguste muude projektidega ja siis mingisugusel hetkel ma 
arvan, et me olime selleks ajaks juba jõudnud keskkooli, tekkisid arvutid ka 
meile kooli, ma olen  algusest lõpuni käinud Reaalkoolis.  Nendeks masinateks 
oli klassitäis Yamaha MSX-e\index{Arvutid!Yamaha MSX}. Tolle klassiga seoses on 
meeles, et kuna erinevate koolide vahel oli nende masinate saamiseks 
konkurents,  olla kool saanud ka mingisuguse ähvarduskõne. Mille peale 
loomulikult vaprad raadioruffi ja füüsikaklassi tagaruumi noored organiseerisid 
öö läbi valve koolimajja. Arvutikastid olid kusagil kas direktori kabinetis ja 
kus iganes ja siis me ööbisime koolis ja valvasime neid. Pärast see 
arvutiklassis sai suhteliselt meie selliseks nagu koduks.

\question{Mis seltskond see raadioruffi ja füüsikaklassi tagaruumi oma oli?}

Seal olime mina ja Sulo Kallas\index[ppl]{Kallas, Sulo} ja Heiki 
Savitš\index[ppl]{Savitš, Heiki} ja Vallo Veinthal\index[ppl]{Veinthal, Vallo}  
ja Reimo Mesipuu\index[ppl]{Mesipuu, Reimo} ja no ma kindlasti jätan kedagi 
ebaviisakalt mainimata.  Avo Nappo\index[ppl]{Nappo, Avo} tiirles meie ümber 
rohkem nagu arvutiseltskonna poolest, raadioruumis olime rohkem vist mina, Sulo 
ja Reimo.

\question{Ma küsin pigem seda, et mille alusel too seltskond moodustus. 
Klassivennad? Tehnikahuvi?}

Otseseid klassivendi oli  vähe, me olime kõik sama mingi paariaastane vahemik. 
Sulo oli kõige vanem, Vallo ja Heiki  olid meist aasta nooremad. Ehk et see oli 
selline parasjagu  klass üles, klass alla seltskond, kes siis oli kooli aktiiv 
niisuguse tehnilise poole pealt. Kuidas me täpselt sattusime raadioruumi? Ju me 
siis hängisime füüsika kandis ringi ja  tuli välja, et seal on raadioruum, kus 
on ka mingeid nuppe, mida saab keerata. Kuidagi nagu sealt ümbert tekkis kogu 
seesama punt, kellest nagu väiksem osa oli  alguses raadioruumi ümber ja 
kellest siis suurem seltskond arvutiklassi ümber tekkis. 

\question{Kas programmerimine ja raadioruumitamine koolitööd ei hakanud segama?}

Ei tea. Ma usun, et ega ma mingisugune medaliga lõpetaja oleks niikuinii 
viitsinud olla. See nagu ei olnud kuidagi minu maailmavaates sees. Noh, 
neljade-viitega lõpetasin, selles mõttes probleemi ei olnud.

\question{Eks see ongi tunnetuse küsimus, kumb primaarne oli tollel hetkel?}

Ma arvan, et eks arvuti pool oli põhiline. Keskkool möödus üleüldse enam-vähem 
niimoodi  kuidagi, et vahepeal sai käidud kohvikus ja siis vahepeal sai käidud 
olümpiaadidel. Kui olid olümpiaadid, olid jälle head hinded, sest õpetajad ei 
saanud ometi olümpiaadil esinejatele halbu hindeid panna. Aga kui olümpiaadil 
ei käinud, siis kippusid hinded kehvemaks minema sest et oli ununenud ära 
koolis käimine ja kõik muu selline. Ma siiamaani näen unenägusid sellest, et on 
tulekul eksam ja ma olen unustanud terve veerandi käigus tunnis käia. Aga 
sellest on  tekkinud  mõtteviis, et ma ei pea olema midagi õppinud. Ehk et kui 
TPI-sse läksin ja mata eksam tehti koos raamatutega, siis ma jõudsin eksami 
käigus alati ära õppida selle, mida oli eksamiks vaja. Ma nagu ei pidanud  
eelnevalt liiga palju mingisugusele loengus käimisele pühenduma, ma võisin 
lihtsalt tulla ja eksamid ära teha. Ülejäänud semestri võis lihtsalt arvutitega 
tegeleda. Ma kindlasti ei soovitaks seda kellegile noortest, aga noh, näed, 
niimoodi on see juhtunud. 

Selline lähenemine on tekitanud mingisugune väga, sellise, kuidagi teistsuguse 
arusaamise ümbritsevast tehnikast. Ma nagu  ilmselt ei karda midagi selles 
mõttes, et ma kui on midagi vaja, siis lihtsalt tuleb võtta \emph{manual}  või 
kood ette. Läheb aega selles mõttes, et ma loomulikult läksin eksamile 
esimesena sisse ja tulin viimasena välja. Aga  põhimõtteliselt kolme tunniga 
sain nagu õige asjaga hakkama. Tundus nagu efektiivne lähenemine. Võib-olla ma  
oleksin saanud targemaks, kui ma oleksin süsteemsemalt õppinud. 

\question{Mida sa sinna TPI-sse õppima läksid?}

See oli TI ehk majandusinfo töötlus\index{Tallinna Tehnikaülikool!TI}. Linnar 
Viik\index[ppl]{Viik, Linnar} on lõpetanud sama asja mõned aastad enne mind. 
Aasta sattus 1989 olema, kus päris nagu Pol.Ök.-i ja Kompartei 
Ajalugu\sidenote{Nõukogude ajal kuulus ülikoolihariduse juurde Poliitökonoomia, 
Kommunistliku Partei Ajaloo ja teiste niiöelda \enquote{punaste ainete} 
läbimine.} ei tahaks õppida sinna kõrvale, aga nad ei olnud veel päriselt välja 
mõelnud, et mis see teine asi on, mida õpetada. Oli ka muid asju, millest ma 
päris täpselt toona aru ei saanud, miks seda peaks tegema. Kaasa arvatud see, 
et miks ma peaksin tegema tansistoritest valmis 8080 protsessori paar käsku. 
Eriti, kui normaalsed inimesed kasutavad vähemasti Z80-t ja mitte 8080-t. 
Teismelise värk, ei ole \emph{cool} piisavalt. \enquote{Intel 8008? 
Zilog\sidenote{Zilogi toodetud 8-bitine Z80 protsessor oli Inteli 
8080 protsessoriga ühilduv aga märkimisväärselt odavam} on normaalne!} No 
täpselt selline asi, nagu on täna hipsteri habe või mingisugused muud välised 
tundemärgid. Pean takkajärgi tunnistama, et kuigi ma olin selle suhtes 
kriitiline, siis ma hetkel käin koolitusel, kus räägitakse sellest, kuidas 
\emph{fuzz}-imisega\sidenote{\emph{Fuzzing} on tarkvara (turva) testimise 
meetod, kus programmile söödetakse süsteemselt juhuslikku sisendit.} 
mälukorruptsiooni juhtumeid leida. Kui lektor ütles, et see on maru  keeruline, 
räägime hästi aeglaselt ja mitu korda nagu miilitsatele, siis minu arust midagi 
nii rasket seal polnud, \emph{stack} on \emph{stack}. Protsessoril on 
registrid, eks ju, ma põhimõtteliselt olen neid transidest teinud. Et kui sa 
pead nagu protsessori arhitektuuri tasemel läbi mõtlema, kuidas käskude 
töötlemine toimib, kuidas seal inkremenditakse mingisuguseid pointereid ja 
kuidas need asjandused  mäluga seotud on, sa tegelikult saad aru, kuidas arvuti 
masinkoodi tasemel töötab. Mul on  väga tore kuulata, et kui mu vanem poeg on 
Tartu Ülikoolis, siis neid sunnitakse ka aru saama protsessori sise-ehitusest. 
Tõsi küll,  raamatu tasemel, aga progevad ka assemblerit, see on hästi oluline. 

\question{Kas sind akadeemiline maailm ei tõmmanud, kuigi sa seal servapidi 
juba sees olid?}

Ei, sest et ma olin keska ajal sattunud sellisesse seltskonda, nagu seda oli 
Vabariiklik Õpilasstaap\index{Vabariiklik Õpilasstaap}, mis oli selline 
Komsomoli Keskkomitee juures tegutsev mittekommunistlik vastupanuliikumine. 
Tiina Tšatšua\index[ppl]{Tšatšua, Tiina} oli näiteks üks selle eestvedajaid. 
Sellest sai üks selliseid toonaseid orgunn tiime vabariigis, kes korraldas 
suurüritusi, milleks alustuseks olid komsomoli ja EKP kongressid. Aga orgunni 
mõttes on ju suht savi, kas on EKP kongress või Eesti Kongress või Rahvarinde 
oma. Inimesed tulevad kohale, sul on mingisugused tegevused nagu  
registreerimine ja kusagil tuleb neid toita. Kui on dokumentidega üritus (mida 
tänapäeval eriti toimu, aga kõikide Eesti Kongressi ja Rahvarinde kongressid 
olid sellised), siis on sul näiteks mingisugune redaktsioonitoimkond. Me olime 
seal kui arvuti-tiim, kes orgunnis seda, et registratuur toimiks mingite 
listidega ja samuti toetasime redaktsioonitoimkondi  kõikvõimalike 
tekstitöötluste pooltega ja  väljatrükkimiste ja vormistamistega ja millega 
iganes. 

Kui mul  keskkool sai läbi 1989, siis meenub, et suveks oli tööots. Tallinnas 
toimus ÜRO inva-ekspertide mingisugune hüper tipptaseme kokkusaamine. Sellega 
seoses mäletan, et Tallinnas lõigati sel puhul esimesed äärekivid faasi ja minu 
arust oli Jack Lippmaa\index[ppl]{Lippmaa, Jaak}\sidenote{Peeter peab ilmselt 
silmas Jaak Lippmaad} see, kes isiklikult ehitas ümber paar 
Ikarus-bussi\sidenote{Ungari tootja Ikarus bussid olid Eestis laialt kasutusel 
liinibussidena} nii, et neisse  kuidagi  ratastooliga sisse saaks. Kuidas see 
võimalik oli, ma ei kujuta ette. Meie see Reaalkooli tiim toetas ürituse 
redaktsioonitoimkonda,  kes seal ÜRO-le kohaselt kõigis põhikeeltes 
mingisuguseid dokumente vormistas. Mis tähendas, et oli ilge posu tõlke. Aga 
noh, aastal 1989 ükski tõlki ei olnud ilmselt arvutit näinud rohkem kui 
võib-olla Soome telekast reklaamist. Meid oli piisavalt palju, kümmekond tükki, 
 ja hoidsime siis neil tõlkidel kätt ja jalga, kogu aeg oli olemas mehitus  
praktiliselt igaühe jaoks. Kui kellelgil tekkis niisugune kivistunud pilk, siis 
keegi tuli ja \emph{reboot}-is selle tõlgi arvuti taga või arvuti enda, kumb 
igatahes parasjagu oli rohkem kinni jooksnud. 

Minu enda niisuguses hilisemas eluloos on see episood huvitav sellepärast, et 
tolle ürituse jaoks, olles parasjagu keska lõpetanud,  õnnestus mul lihtsalt 
omaenda sõna peale linna pealt toatäis PC-sid kokku laenata. Ma lihtsalt 
läksin, ütlesin et mul oleks nagu vaja, ja siit sai jälle paar tükki, sealt sai 
paar tükki ja niimoodi sai need umbes kaheksa arvutit kokku. Kõige kihvtim 
neist tuli surnukuurist. See oli üks PC, aga tema peal oli selline asi nagu 
Xerox Ventura Publisher koos Xeroxi graafilise kasutajaliidesega, milleks oli 
GEM, nägi põhimõtteliselt välja nagu MacOS\sidenote{GEM (\emph{Graphics 
Environment Manager} oli tõepoolest üks varastest graafilistest 
kasutajaliidestest. Liigne sarnasus Apple tarkvaraga viis ka kohtuasjani.}. GEM 
sai DOS-ist üles \emph{boot}-itud,  läks ilusti siukseks mustvalgeks ja halliks 
kasutajaliideseks ja seal peal jooksis minu esimene küljendusprogramm. Ja me 
saime neilt ka ühe laserprinteri kasutada, mis ei olnud küll PostScript tollel 
hetkel, aga mis oli täiesti  laserprinter. 

\question{See oli ju nõukogude aeg veel?}

Ilmselt siis meditsiin oli ikkagi saanud mingeid asju kusagilt valuuta eest 
osta. Tegelikult Kivilo\index[ppl]{Kivilo, Ago} plaanis oma  
diagnostikakeskust\index{Diagnostikakeskus}\sidenote{1988. aastal asutatud Diagnostikakeskus oli omal 
ajal märgilise tähendusega. Ühest küljest oli tegemist kõrgtehnoloogiliste 
teenustega, keskuse algusaegadel asus seal Eesti ainus kompuutertomograaf. 
Teisalt oli aga tegemist väga innovatiivse organisatsioonilise 
konstruktsiooniga, milline asjaolu viis hiljem mitmete keskust ümbritsenud 
kõrge profiiliga afäärideni.} kesklinna ja meditsiini poolel olid  tegelikult  
väga kõvad tegijad. Eesti arvutinduse arendusest teatakse rohkem seda 
seltskonda, kes oli nagu Tartu poole pealt ja seotud geeniga ja võib-olla 
Küber\index{Küber}. Mina olen tulnud  nagu meditsiiniliini pidi sisse, seal 
valdkonnas tegeldi päris kõvasti teadus- ja arendustegevusega. 

\question{Ja sealt said sa oma küljendamise-konksu?}

Jah. Otse loomulikult sai hunnik flopikettaid tolle Venturaga ära kopeeritud, 
eks ju. Mis toona oli nagu igati tavapärane \emph{standard operating 
procedure}:  kõigest, mis kätte satub, tehti koopia. Ja niimoodi siis sattuski, 
et kusagil sealsamas 1989-1990  oli minu jaoks ülikoolis käimisest palju 
huvitavam asi see, et arvuti peal on  võimalik küljendust või kujundust või 
disaini teha. 

\question{Kas sul muidu ka niisugune joonistamise soon oli?}

Ei ole. Ma kahtlustanud, et inimesed, kes on pidanud minu küljendatud raamatuid 
tarbima, on kindlasti selle all kannatanud, nii et ma väga vabandan. Kunagi  
Avita\index{Avita kirjastus} kirjastuse  algalgusaegade raamatutest suur hulk 
oli minu tehtud, näiteks.

\question{Aga mis sind selle küljendamise juures niimoodi köitis, kui sul muidu 
sellist visuaalkunstide huvi ei olnud?}

See oli pisut teistmoodi, mingisugune selline hoopis teistmoodi  arvutiga 
tegelemine, kui seda  oli  programmeerimine ja andmetööstus mis olid ka tore. 
Aga see, et sul õnnestub mingeid asju ekraanil teha, see oli see, mis mind 
kuidagi väga sellel hetkel tõmbas. 

Ehk siis 1989. aasta suvel too üritus sai läbi, mina läksin ülikooli. Ja siis 
Mart Siilmann\index[ppl]{Siilman, Mart}, kes oli äsja lõppenud ürituse orgunni 
pealik, ütles, et \enquote{Kuule, mul järgmine suvi on ka mingi asi, et 
arvutiabi vaja, et tule}. Ja see oli aastal, siis 1990 toimunud European 
Nuclear Disarmament Convention, ehk suur rahuvõitlejate ja roheliste üritus. 
Sellega seoses tekkis meil ühte kontorisse, mis asus enam-vähem nüüdseks 
lõpetanud No-teatri ruumides, kusagil teisel korrusel, üks PC, ma arvan, et 
äkki oli Sanyo. Ja selle küljes oli ma arvan, et 1200 boodine või bps-ine 
modem. Mingi koha peal lähevad boodid ja bps-id vist lahku, kui ma hästi 
mäletan\sidenote{Bps (\emph{bits per second}) on bittide hulk, mida sekundis 
edastatakse. \enquote{Boodid} (\emph{baud rate}) näitavad aga, mitu korda 
sekundis signaal muutub. Kuniks kasutatakse tavalist jadaporti, kus signaalil 
on kaks taset, on väärtused võrdsed. Keerulisemate skeemide korral aga võib ühe 
signaalimuutusega edastada rohkem, kui ühe biti ning kiiruseühikud lahknevad.}. 
 Meie ametlik tegevus oli suhtestumine Orgkomiteega. Ja see toimus niimoodi, et 
sai  helistatud kaugekõnega Tallinnast Helsingisse. Eestis oli otsevalimine, 
meil oli selles mõttes väga vinged positsioon, Baltikumis mujal otse välismaa 
numbreid valida ei saanud. Ka Tallinnas igal ei olnud, aga meil oli, sest see 
oli ürituse jaoks oluline. Mart Siilman, kes on endine Fila\sidenote{Peeter 
peab silmas Eesti NSV Riiklikku Filharmooniat\index{Eesti Riiklik 
Filharmoonia}, mille järeltulijaks on alates 1989. aastast Sihtasutus Eesti 
Kontsert. Tegu oli tohutult mõjuka asutusega, mille korraldada oli kogu 
kontserdi-elu Eestis, sealhulgas ka levi- ja jazzmuusika ning estraad. Seega 
oli \enquote{Fila endine direktor} äärmiselt mõjukas inimene, kelle jaoks Soome 
otsevalimise korraldamine kindlasti võimalik oli.} direktor ja muud sihukesed 
asjad, siis üldiselt juba oletan, et ta orgunnis, mida vaja. Kuidas, ei tea. 
Igatahes saime helistada Datapakki X.25 võrku,  X.25 kaudu oli siis võimalik 
suhelda ühe Rootsi serveriga, teine server oli Kanadas. Sealtkaudu me siis nagu 
suhtlesime tolle ürituse orgkomiteega aga  hakkasime ka vaatama, et kuhu veel 
õnnestub helistada.

\question{Kuidas te \emph{bootstrap}-isite seda asja? Mida kliendi poole vaja 
oli, et sinna võrku saada?}

Tavaline modem ja tavaline modemiga suhtlev mingisugunegi terminaliproge. 
Sellega modemiga helistas, siis Datapaki liidestuspunkti, kust edasi läks, asi 
pakettvõrguks või X.25-ks. Ja siis sealsamas terminali peal, nagu terminali 
peal ikka: lehekülg \emph{scrollib} ja siis seal on mingi menüü, valid mingi 
üks, kaks, üksteist, eks. Mingi meilboks oli seal, kus sai kirju vahetada, oli  
jututubade või listide alajaotus. Aga siis,  parafraseerides Heinleini, et 
\enquote{\emph{Have modem, will find BBS-es}}\sidenote{Peeter viitab Robert A. 
Heinlein-i 1958. aasta jutustusele \emph{Have Space Suit - Will Travel}.}. 
Loomulikult leidsime kusagilt üles ka selle, et on  BBS-id. Umbes samas 1989. 
aasta lõpus tekkis Lembit Pirnil\index[ppl]{Pirn, Lembit} esimene 
PirnBoxi\index{BBS!PirnBox} nimeline  BBS, mis asus seal kusagil, kus trammid 
väga kõva kriginaga keerasid toona ehk praeguse SEB vastas, seal oli 
Autotranspordi Arvutuskeskus\sidenote{Asutuse täpne nimi oli Eesti NSV 
Autotranspordi Arvutuskeskus (ATAK)} ilmselt. Ehk tal oli seal arvuti ja me 
kõik alguses helistasime  sinna Pirnile sisse. 

Natukese aja pärast tekkis selline asi nagu HNS ehk \emph{Hackers Night 
System}\index{BBS!HNS}. Ja siis kolmas oli Goodwin BBS\index{BBS!Goodwin} meil 
Suloga\index[ppl]{Kallas, Sulo}, mis  ilmselt jooksis sellesama 
väljahelistamise liini otsas. Öösel jätsime  arvuti sisse ja kõik said sinna 
sisse helistada. Kui sa tahad kuhugi sisse helistada aga liinid on kogu aeg 
kinni, siis ainuke võimalus olukorda parandada on see, et panna ise ka mingi 
\emph{box} püsti, eks ju. 

Sealt kusagilt tekkis siis ka Fido pool. Jällegi see sissehelistamise küsimus. 
Et kui meil on  võimalik teha see, et e-post ja jututoad oleksid omavahel nagu 
kuidagi  sünkroniseeritud eri masinates, siis pole ju vahet, kuhu me sisse 
helistame. Masinad vahepeal öösel või päeval käivad ja vahetavad omavahel need 
sõnumid ära. Fido oli selles mõttes nagu tõsiselt distributeeritud nett. See, 
mida nüüd räägitakse, kuidas  veeb kolm tuleb äkki nagu sarnane. 

\question{Igasugu võrgustike \emph{bootstrap}-imine on keeruline just inimeste 
mõttes. Selleks, et kuhugi külge minna, peaks seal olema huvitav. Selleks, et 
seal oleks huvitav, peaksid seal olema inimesed. Mis te näiteks seal PirnBoxis 
tegite, et huvitav oli?}

Eks ilmselt sai lihtsalt nagu lämisetud. Ma pean tunnistama, et ma ei mäleta, 
mida me  tegime, aga igal juhul väga huvitav oli. Ma oletan, et kusagilt  
pääses ligi mingisugustele faili kujul sci-fi  raamatutele ja mingitele muudele 
laiematele uudisgruppidele, mingi hetk  kusagilt igal juhul liidestusid need 
Fidonetiga ära. Ehk et seal informatsiooni nagu liikus. Ja lihtsalt selles 
mõttes oli ka huvitav kirjutadagi, et vau, et traadi kaudu see kõik liigubki! 
See oli nii \emph{amazing} selle aja kohta. Sellest ma sain aru, et 
programmeerida saab ja midagi kujundada aga et saab nagu  reaalselt suhelda ka!

\question{Mis tol ajal tegi ühe BBS-i populaarsemaks kui teise? Goodwin ja HNS 
olid ikkagi pikalt populaarsed, kuigi PirnBox oli esimene?}

Ta oli esimene, aga ta vist jooksis mingisugust asja, mis toona oli vist vähem 
levinud. Meil oli, kui ma hästi mäletan, äkki  Maximus. Ja ma ei mäleta, kuidas 
Pirni ja Fidonetiga läks, võib-olla oli tal palju tööd teha? Kuidagi ta nagu 
nende noorte poiste käe alla läks igatahes. Ma ei oska öelda, miks.

Sulo oli muidugi omamoodi nagu arvamusliider ehk selles mõttes, et tal olid  
kõikvõimalike asjade suhtes oma sellised väga toredad ja tugevad seisukohad. 
Mina olin ka niisuguse tutu-lutu taustaga, olles olnud muu hulgas 
Reaalkooli\index{Koolid!Tallinna 2. Keskkool} viimane komsomolisekretär.
Arvestades seda, et enne mind oli komsomolisekretär Karl Martin 
Sinijärv\index[ppl]{Sinijärv, Karl Martin} selles mõttes, me  ei võtnud seda 
asja väga tõsiselt.

Kuidagi me sattusime seda asja vedama  kuna meil oli tänu sellele 
tuuma-üritusele ressurssi käes. Meil mingil hetkel tekkis igal juhul  kaks 
telefoniliini. Võib-olla see oli mingi aastake hiljem, kui  üritus läbi sai ja 
me olime juba Eesti Instituudi\index{Eesti Instituut} ruumides, veidike enne 
seda kui Eesti Instituut osutus tegelikult Eesti välisesinduste ja iseseisvuse 
ettevalmistuslavaks. Näiteks sellel hetkel, kui kuulutati välja iseseisvus, 
tuli järsku välja, et Jüri Luiged\index[ppl]{Luik, Jüri} ja kõik muud, kes seal 
mööda maailma laiali olid, et neil olid juhuslikult kaasas ka pruunid ümbrikud 
esitamiseks kohalikule võimupealikule küsimusega, et kas teie ekstsellents 
lubaks meil asutada suursaatkonda. 

Eesti Instituudis olid meil ka mingid omad arvutid, jällegi ei mäleta täpselt 
kelle arvutid need olid, kust nad pärit olid. Äkki olid instituudi omad, äkki 
olid meie omad. Me olime Suloga\index[ppl]{Kallas, Sulo} need, kes öösiti  
faktse saatsid. Seal mitmed toredad kolleegid, vähemasti niimoodi huumoriga 
pooleks, olid sügavalt veendunud, et faks ongi selline  seade, et kui sinna  
peale panna paber koos kollase postitiga, kus on telefoninumber, siis on see 
hommikuks ennast ära saatnud, eks ju. Sest et tollased liinid olid sellised, et 
nad öösiti toimisid oluliselt paremini, kui päeval. 

Tingituna sellest, et seda välisühendust oli meil läbi modemi helistamise  
suhteliselt piiramatult käes ja liine oli seal ka mitmeid, siis oli meil  kaks 
modemiühendust. Mingil hetkel hakkas meie kaudu Fidoneti kaudu väljapoole 
ühenduma Läti.

\question{Ma teadsin, et Vene Fidonet käis läbi meie aga et Läti ka?}

Venemaa, tekkis ka millaski jah. Oli Läti, mis oli Eesti all  mis iganes see 
Fidoneti järgmine selline tase oli. Leedukad loomulikult ei oleks millegi 
selliseni laskunud, et nad on mingisugune Eesti regioon kusagil 
võrgustruktuuris. Nad selle asemel kord nädalas helistasid ja tõid enam-vähem 
nagu ämbriga e-posti. Välja arvatud, ma arvan, et see oli Kaunase 
Ülikoool\index{Kaunase Ülikool} ja Leedu parlament\index{Leedu Seim}, kes olid 
Goodwin BBS-i pointid. Seal oli hädasti vaja ja siis uhkus jäeti kõrvale. 

Lätlased käisid meil külas ka. Panid raha kokku ja tõid meile selle eest, et on 
ühendus. Saime mingisugune, ma ei tea, kakssada dollarit, mille meie 
investeerisime sellesse, et  ostsime endale kaks modemit. US robots\index{US 
Robotics}-i  HST-d, mis tegid vist kas neliteist kilo või midagi sihukest 
kiirust. Väga väärt aparaadid, niimoodi on lätlased panustanud  Eesti neti  
arengusse.

Samal ajal Internetiga nagu ametlikku postivahetust pidas seltskond 
Küberist\index{Küberneetika Instituut}. Aga neil seal Mustamäel oli ikka 
selline suhteliselt sant jaam,  mis  krabises ilmselt rohkem, kui sidet läbi 
lasi. Pluss veel see, et need akadeemilised tüübid olid millegipärast koledad 
UNIX-i sõbrad ja kasutasid PEP-i TrailBlazer-eid\index{Telebit TrailBlazer}. 
Mis esiteks olid 9.6 kilo, ehk mõttetult aeglased ja teiseks kuidagi nende 
post-sovieti liinidega HST suutis nagu paremini oma sidet vilistada. Me olime 
sügavalt veendunud, et nad olid ka oma reaalselt võimekuselt pikki seansse ja 
kiirust üleval pidada  oluliselt paremad. 

\question{Ja kui sa ütled \enquote{liin}, siis sa mõtled ikka telefoniliin?}

Jah, liinid olid kõik tavaline analoog, kus otsa käis kettaga telefon. Ja 
keskjaamas, ma arvan, et kui sa ikka numbri valisid, siis kusagil  mingisugused 
 releed jooksid kontakte mööda ringi. Kui modem valis, oli kuulda  klõbinat, 
tal oli seal mingisugune relee või herkon või mis iganes, millega ta katkestusi 
tekitas. Kõik oli reaalselt selline elektriline, sellepärast ma ka kujutan 
ette, kuidas andmeside  tegelikult toimub. Aga  kuidas on võimalik, et mingid 
vennad panevad läbi ADSLi kümme mega, me panime enam-vähem samasugust asjast 
läbi neliteist kilobaiti, hästi arusaamatu. Või wifi täpselt samuti. Ma ei 
kujuta ette kuidas see põhimõtteliselt saab üldse toimida. 

\question{Kas kujundamise rida käis sul kõige selle kõrvale?}

See käis kuidagi sinna jah, selles mõttes kõrvale, et ma seal mingis umbes 
samas ajajärgus sattusin seltskonda, kellel oli arvuti ja printer ja vaja  
midagi trükkida. See oli poistekoor\index{RAM-i poistekoor} ja Venno 
Laul\index[ppl]{Laul, Venno}\sidenote{Venno Laul asutas 1971. aastal Riikliku 
Akadeemilise Meeskoori juurde poistekoori ning oli kuni 1990. aastani selle 
kunstiline juht ja peadirigent.}. Neil oli esimene PostScript printer, mille 
ostus ma olen osalenud, kas Tectronicsi või millegi sellise A4 formaadis 
kolmesaja DPI-ga laserprinter. Aga see oli PostScript printer. Kui sina sai 
Ventura külge ühendatud, siis \ldots 

Põhimõtteliselt kõik toonased kujundusprogrammid olidki niimoodi, et sul ikkagi 
enam-vähem \emph{bitmap} fondid olid arvutis, eks ju. Kuidagi need 
\emph{bitmap}-id saadeti juhet pidi printerisse ja kõik see mõtles hästi pikka 
aega. Aga PostScript tegi nii, et sa said selle lehekülje nagu programmi saata 
printerisse ja printer oma tarkusega joonistas. Mis oli kunagi  Adobe ja Apple 
vendade poolt väga mõistlik valik, olles Silicon Valleys kokku saanud ja  
otsustanud, kes mis osa maailmast vallutama hakkab. Tõesti, kui sul on kontor, 
ilmselt igal vennal ei ole printerit ja selleks, et inimesed saaksid printida, 
võiks olla printer ka tark. Väga spetsiifiliselt tark, et ta suudaks 
joonistadanii-öelda lehekülje endal mälus valmis ja siis välja trükkida. 

Ja kusagil samal ajal ma sattusin kokku ka Sirbiga\index{Sirp} (ma ei tea, ka 
see toona oli juba Sirp või veel Sirp ja Vasar), mis oli üks esimesi 
ajakirjandusväljaanded, mis läks digitaalse töövoo peale. On raske öelda, kes  
täpselt see esimene oli, aga igatahes Sirp läks ka. Alguses protsess oli umbes 
selline, et toimus tinaladu. Tinalaoga tehti kas siis üks tõmme paberi peale ja 
see siis vist pildistati üles. Põhimõtteliselt sellel hetkel ofset-trükk toimus 
veel läbi tinalao. Ja nüüd, kui oli võimalik minna üle selle peale, et arvutist 
saaks välja trükkida, siis see oli ikka mega raju.

\question{Põhimõtteliselt PostScripti printerist lasti kile peale trükitavad 
asjad, eksole?}

Põhimõtteliselt jah, ja peegelpildis. Üks asju, mis ma mäletan, mis me 
Suloga\index[ppl]{Kallas, Sulo} koos tegime, või Sulo tegi, kui me Eesti 
Instituudis\index{Eesti Instituut} olime, oli PostScripti \emph{pre-header}. 
PostScripti puhul sageli oligi, et programmiga tuli kaasa mingisugune 
koodijupp, mis siis kirjeldas sellise nagu programmeerimiskeskkonna, defineeris 
täiendavad funktsioonid ja mingisugused muud oma käsud. \emph{Pre-header}? 
Preambul oli vist. Seejärel tuli  kood ise ja lõpus mingisugune 
koristusfunktsioon või midagi, mis välja trükis. PostScript oli tore selles 
mõttes, et ta oli nagu \emph{open source}. Mitte nii-öelda vaba tarkvara, aga 
ta oli nagu nähtava lähtekoodiga. Ehk oligi võimalik võtta sama Ventura ette, 
mis kusagilt laadis selle preambuli, mis oli tekstifail ja mida oli võimalik 
muuta. Ja oli võimalik kirjutada selline transformatsioon sinna ette, mis 
keeras pildi peeglisse. Too Sulo PostScripti preambul õnnestus meil maha müüa 
Avita\index{Avita kirjastus} kirjastusele Tiit Aunastale\index[ppl]{Aunaste, 
Tiit}, kes hakkasid tegema kooliraamatute kirjastamist. Ma ise sattusin ka  
mingi aeg hiljem  Avitasse tööle, mis oli ka mingi a'la 1991, asjad liikusid 
väga kiiresti tol ajal.

\question{Jah, sest umbes viis aastat hiljem, mina mäletan sind Eesti 
vaieldamatu autoriteedina teemal, kuidas arvutist värviline asi trükki saada}

Eks ma olin seda piisavalt praktiseerinud. Me olime teinud ilmselt 
mingisuguseid nii-öelda haltuura otsasid sellesama Ventura peal. Igast muud 
softi oli ka, näiteks Arts \& Letters, millega oli võimalik panna tähti ümber 
ringi käima. Toona, kui kõik asutasid endale aktsiaseltse ja börse, siis kõigil 
neil oli vaja endale pitsatit. See oli meeletu innovatsioon, et oli võimalik 
arvutist ühe matsuga põhimõtteliselt pitsat valmis teha ja ei pidanudki 
kujundajatädi mingisuguseid fotolao tähti välja lõikama ja kleepima. 

Sirbi  toimetus andis välja mingit Jutulehe\sidenote{Ilmus AS Kodamu väljaandel 
aasatel 1990-1992.} nimelist asja, mille  \emph{layout}-i vist mina tegingi 
tegelikult. 

Jah, ja nii edasi, kuidagi ma sattusin selle ala peale. Sai käidud vaatamas, 
kuidas Helsingis Helsingin Sanomat\index{Helsingin Sanomat} tehakse, kus olid 
mingisugused miniarvutid ja   rohelised terminalid. Seal oli ka 
Linotronic\sidenote{Mergenthaler Linotype Company poolt toodetud 
kõrgekvaliteediline printer. Tegu oli kalli seadmega, kuid võimaldas trükkida 
resolutsioonis kuni 2540 dpi.}, millega trükiti põhimõtteliselt veergu välja 
fotopaberi peale. Fotopaber oli kolmkümmend senti lai ja sinna lasti välja  üks 
ajaleheveerg. Ajaleheveerud lõigati kääriga sealt välja ja pandi sellise suure 
maketi peale, mis oli mingisuguse vaha või millegagi koos. Rastreeritud fotod 
pandi veergude vahele, niimoodi pandi leht kokku ja tulemus saadeti faksiga 
trükikotta. Faks ei olnud loomulikult see tavaline faks, vaid mingisugune 
selline \emph{industrial-grade} ajalehe formaadis kõrge-resolutsiooniline,  mis 
siis  skännis ühelt poolt sisse ja teisel pool siis  ilmselt trükis välja 
filmi, millega valgustati trükiplaadid. 

\question{Räägi, mis see \enquote{pull} oli? Kas tehnoloogiline keerukus või 
see, et protsessil oli palju samme või veel midagi?}

Kõige huvitavam on tegeleda asjadega, millega teised parasjagu ei tegele. Või 
siis, teistpidi, mingi asi, mis toimib nagu kuidagi teistmoodi, kui sa oled 
siiamaani arvanud, et asjad toimivad. \enquote{Aa, ongi niimoodi, et ma saan 
seda teha, okei?}. Ja niisama Pascalis programmeerida, ma ei tea, seda õpetati 
koolis, et see ei olnud nagu midagi väga huvitavat. Aga kuna mul oli ilmselt  
parasjagu olemas taust, mul oli arusaamine sellest, kuidas need asjad töötavad 
ja mis seal masinates on,  siis ma suutsin asju efektiivsemalt tööle panna. Ehk 
siis, kuidas siduda kirjastuse kontekstis see, et sul on  küljendusprogramm ja 
sul on tekstitöötlusprogramm. Tekstitöötluseks oli sul WordPerfect (Perfect? 
Prefect? Ford Prefect ja Word Perfect!\sidenote{Peeter viitab tõenäoliselt 
Douglas Adamsi loodud tegelaskujule, mitte omaaegsele populaarsele 
automargile.}),  me küll kasutasime rohkem Volkswriteri-nimelist 
asja\sidenote{Volkswriter oli üks esimese PC-platvormi tekstitöötlusprogramme, 
mida arendati peamiselt eelmise sajandi kaheksakümnendatel aastatel. 
Volkswriter oli saadaval ka eespool mainitud GEM platvormile, sellest ilmselt 
tema kasutamine kirjastustöös.}. Just see, et sa said lasta tekstitoimetajal 
võtta ette selle WordPerfecti faili, tema toimetab seal mingid asjad ära, seal 
kuidagi on juba sees see märgendus, mis ütleb ära, mis on  stiilid. Siis sa 
loed selle oma küljendusprogrammi tagasi ja sul on põhimõtteliselt võimalik 
teha teksti korrektuuri ilma et,  keegi oleks nagu kallima arvuti või 
keerulisem programmi taga. See oli see, mis meil õnnestus väga efektiivselt 
kuidagi juurutada. 

Kusagil sealsamas enne rubla-aja lõppu, 1992. aasta alguses, tekkis Prisma 
Printi\index{Prisma Print} esimene Linotronic ehk filmiprinter. Sinnamaani 
peeti kuuesaja DPI-st laserit juba väga heaks, nüüd tekkis järsku 1200 DPI-ne 
filmiprinter. Prismas  alumisel korrusel olid Crosfieldi suured 
trummelskännerid\sidenote{Crosfield Electronics oli Briti firma, mille toodetud 
skännereid peetakse siiani ühtedeks parimateks, mis iial tehtud.} millega sai 
teha värvilahutusi, filmi peale, juba rastrisse. Ja endiselt kogu montaaž 
toimus niimoodi, et sul olid teksti kile ja pildi kile või film ja need siis 
valgustati või füüsiliselt lõigati kuidagi kokku. 

Kuna mul oli  keskmisest parem ettekujutus sellest, kuidas need süsteemid 
töötavad, siis enamasti,  kui mina jõudsin oma failidega kohale, siis kõik seal 
nagu nägid vaeva, kuidas QuarkExpressist midagi välja printida ja muud 
sihukest. Minul olid kaasas oma flopid ja võib-olla hiljem mingid 
magnet-optilised või  \emph{syquest}-i kettad\sidenote{SyQuesti 88 megabaiti 
mahutavad eemaldatavad kõvakettad olid üheksakümnendatel \emph{de facto} 
standard suurte failide liigutamiseks Apple maailmas.} (vist mitte syquest, see 
on üks väheseid asju, mida mul pole kunagi olnud) ja ma tulin sinna vahele, et 
\enquote{laske minu omad vahepeal välja, ei viitsi oodata teiste järel}. Tehti 
ära. Põhjus oli selles,  et minu asjad käisid tõesti kiiresti läbi. Sest ma 
kujutasin kujundust tehes ette, kuidas see PostScriptiks läheb. Minu jaoks nii 
küljendusprogramm, kui, ütleme, seesama Ventura või graafikaprogrammid nagu 
Illustrator või Freehand oli tegelikult programm, mis aitas mul visuaalselt 
valmistada ette PostScripti. Ma põhimõtteliselt teadsin, kuidas see asi koodis 
välja näeb, ma võin võtta faili ette ja näha, kuskohas miski asi on. Ja tänu 
sellele ma teadsin ka seda, mis on printeri jaoks keerulised asjad ja ma 
oskasin neid asju lihtsustada ja mitte liiga keerulisi asju kasutada. Sest et 
see prose, mis seal taga oli, oli ikka suhteliselt vaene. Kui sa suudad nagu 
tekitada olukorra, kus programmil on tsükkel tsüklis tsüklis (tänapäeval tuleb 
sinna otsa veel SQL-i päring), siis üldiselt on see asi suhteliselt ebapädev.  

Arvuti taust ja siis kuidagi kokku sattumine selle kujundusega on tekitanud 
selle, et kuidagi ka sõprade hulgas on suur hulk igasuguseid disainereid ja, 
ütleme, kõiki kes tollest ajast on tegevad olnud, ma kuidagi nagu tunnen 
sellest samast Prisma Prindi väljatrükijärjekorrast. 

Sealt ma sattusin edasi Uniprinti. Alustuseks töötades andetu disainerina aga 
siis leides tasapisi võimalusi selliseid, noh, nagu vähem disainimaid asju 
teha, kus mängis rolli just see, et ma suutsin võtta, ette mingi Eesti Näituste 
näituse kataloogi andmebaasi (äkki oli Microsoft Accessis?) ja sealt 
genereerida väljundiks tekstifaili, mis oli juba märgendatud stiilidega ja mis 
oli võimalik lihtsalt küljendusse sisse lugeda. Jällegi asi, mida väga palju 
sellises \emph{desktop publishing}-us ei olnud tavaks kasutada: sa  valmistad 
stiilid ette ja siis tekst lihtsalt kasutab neid stiile, kohapeal midagi tegema 
ei pea.

\question{Põhimõtteliselt CSS?}

\emph{Right}, täpselt. Põhipõhimõtteliselt nagu CSS, ainult et tekst ja paber 
ja vanasti. Need kõik töötavad siiamaani niimoodi, aga see oli selline meetod, 
kuidas nagu mina tegin. See  tekitas võimaluse teha selliseid huvitavaid 
töövoogusid.  Minul oli põhimõtteliselt  see andmebaas käes, tüdrukud näitustel 
müüsid seal järgmisi bokse maha ja tegid ürituste korraldust ja tõstsid asju 
ümber ja täiendasid firmade andmeid  ja parandasid telefoninumbreid ja mida 
iganes. Mina mingisugusel hetkel trükkisin \emph{layout}-i välja, viisin neile 
ja nemad tegid andmebaasi korrektuuri, mina nii kaua joonistasin logosid 
puhtaks, ja siis niimoodi ma õppisin. Ma pean tunnistama, et ma olen andetu 
disainer. Aga nende tehniliste protsesside ja töökorralduste poole pealt 
ilmselt teadsin oluliselt rohkem, kui keegi teine jah, tollel hetkel. 

\question{Aga see tuleb ju kenasti selle juurde, mida sa praegu tundud tegevat? 
Millega sa praegu üldse tegeled?}

Jah, on igasuguseid seoseid. Kui ma veel trüki alal tegutsesin, oli mul mingil 
hetkel ilmselt liiga palju vaba aega tänu tänu sellele, et ma olin suutnud oma 
tööd ära optimeerida. Ja tänu sellele, et Uniprindi pealikud Sirje ja Andrus 
Reinsoo\index[ppl]{Reinsoo, Sirje}\index[ppl]{Reinsoo, Andrus}, kes on just 
mõlemad lahkunud\sidenote{Intervjuu Peetriga leidis aset märtsis 2019.} jätsid 
mulle piisavalt vabadust tegeleda. Ja nii ma käisin ja kolasin Ameerikas 
konverentsidel. See oli ajal, kui enamasti oli suhtumine, et \enquote{Mis 
mõttes väljamaa ja konverentsid? Me oleme Eestist ja teame väga hästi}. Mina 
käisin, olid mingisugused \emph{cyber publishing} seminarid, mis just olid 
seotud selle trükipoole alaga, mis mulle huvi pakkus: plaaditrükkida ja kõik 
muu. Ja oli lugu selles, et kuna ma olin põhimõtteliselt nagu ajakirjanik, siis 
mul oli võimalik möllida ennast igale poole konverentsidele, mis muidu maksid 
mingi paar tuhat taala (röögatult kallis tolle aja mõttes) põhimõtteliselt 
ajakirjaniku passiga sisse. 
Vist aastast 1994 olen ma sattunud  kirjutama.

See sai alguse niimoodi, et koolivend ja paralleelklassivend Peeter-Eerik 
Ots\index[ppl]{Ots, Peeter-Eerik} oli Äripäevas ajakirjanik ja kirjutas 
mingisugused tehnoloogiateemalisi lugusid. Aga minul reaalikana hakkas nagu 
mõnevõrra piinlik, kirjutatu ei tundunud olema piisavalt pädev. No see oli ka 
kõik sellest post-BBS-i ajast, ma olin kindlasti ka võrdlemisi 
\emph{opinionated} noor inimene. Eelarvamuskindel,  omade kindlate 
eelarvamustega. Ma kirjutasin teisele Peetrile paar lugu ette, et avalda parem 
neid, vähemasti  kirjutatud kellegi poolt, kes enam-vähem saab aru,  millega 
tegemist on. Peeter ütles, et kuidagi nagu väga imelik, et võiks ikka minu nime 
alt ka hakata avaldama ja saaks honorari ka maksta. Nii ma sattusin Äripäeva 
kirjutama. Sealt sattusin jälle igale poole mujale kirjutama, 
Arvutimaailma\index{Ajakiri!Arvutimaailm} ja kuhu iganes. 

\question{Tehnoloogia tehnoloogiaks, aga mis sind kirjutamise juurde tõmbas?}

Kirjandite kirjutamisega sain suhteliselt nagu hakkama juba kooliajal. Minu 
esimene avalikustatud töö oli Pikri\index{Ajakiri!Pikker} mingisugune noorte 
huumorivõistluse võidutöö. Ilmselt ma olin midagi lugenud ka,  selline sõprus 
sõnaga oli nagu olemas, ma olin juba teinud  kooli omavalitsust ja muud muud 
sellist. Ilmselt olin niisugune parasjagu jutukas ka, eks ju. Kirjutamine ei 
olnud keeruline. Võib-olla meeldis mulle ka õpetada,  läbi kirjutamises on 
võimalik teisi inimesi õpetada ja panna midagi teisiti tegema. Pluss, 
klassikaline küsimus, miks ma olen väga tänulik Peeter-Eerik Otsale on see, et 
ta tegi midagi valesti. See tüüpiline küsimus interneti puhul, et 
\enquote{\emph{wait, somebody is just wrong on the Internet!}}. Kirjutamine 
ilmselt sai alguse sellest, et \emph{somebody was wrong} ja mul oli vaja 
kaitsta oma  seisukohta, ja noh, Reaalkooli au loomulikult ka, eks. 

Sealt sai asi alguse ja edasi inimesed ütlesid, et ma võiksin neile kirjutada. 
Noh, ja  olles õppinud, palju asju, ma siiamaani ei oska \enquote{ei} öelda. 
Ilmselt mingi edevus ka, et \enquote{oh, keegi tahab, et ma midagi teeksin!}. 
Eks ma siis kirjutasin ja hetkel, kus selle valdkonna kohta suuresti nagu ei 
olnud kirjutatud, siis see kõik nagu kuidagi hakkas toimima. Mingil hetkel seal 
kusagil üheksakümmend kas mingisugune viis-kuus, kui Avo Raup\index[ppl]{Raup, 
Avo} tegi Raadio 2-s saadet \enquote{Võrgutaja}, kutsus ta mind kui juba 
kirjutanud ja tuntud inimest, saatesse külaliseks. See asi hakkas meil kuidagi 
niimoodi klappima, et minust sai resident-saatekülaline. Esimene inimene, keda 
ma sattusin üldse esimesena üksinda tegema  (Avo oli haigeks jäänud või 
midagi), oli Abobase Systems-ist\index{Abobase Systems} Kaido 
Saarma\index[ppl]{Saarma, Kaido}. 

Kusagil sealsamas 1999 tuli mingil hetkel minu juurde 
Sarvik\index[ppl]{Sarvik|see{Sarv, Henn}}\sidenote{Legendaarne IT-mees Henn 
Sarv\index[ppl]{Sarv, Henn}.} ja ütles, et Kukust kas siis Lang või Tiido või 
keegi oli öelnud et on vaja teha arvutisaadet. Istusime sealsamas Uniprindi 
lähedal Pärnu maanteel Westmani poe vastas keldris ühes mingisugune Hollandi 
õlletoas ja mõtlesime välja, et võiks olla selline asi nagu Tehnokratt. Ja 
hakkasime tootma raadiosaadet. Juba esimesel hooajal sattusime Kukus kokku 
tegelastega, kellel oli mõte ETV-sse ka midagi teha\index{Eesti 
Rahvusringhääling!Eesti Televisioon}. \emph{Whatever}, toodame! Nii ma sattusin 
telesaatesse olema korraga enam-vähem toimetaja, saatejuht ja (mis muidugi 
mõnevõrra üllatuseks tuli) pidin panema kokku ka montaažiriba.

\question{Ja nüüd sa oled tagasi ringiga\ldots} 

Kas nüüd tagasi või edasi. Praegu ma olen Zone-s\index{Zone}, mis on täiesti 
juhuslikult ajaloos esimene kord, kus ma olen töötanud mingit otsa pidi 
IT-firmas. Ma olen vahepeal olnud reklaamiagentuuris, küll digi-tiimi juht, 
võiks öelda, et ka natuke IT poole, aga ta oli ikka nagu reklaamiagentuur, 
eksju. Trüki poole peal ja kus iganes, ma olen koolitanud ja kõike muud teinud 
aga see on esimene kord, kus mingid IT-tüübid mõtlesid, et palkaks Marveti siia 
tuututama. Ametlikult mu  müts on seotud turunduse ja kommunikatsiooniga. Aga 
ma tegelen ka selle poolega, et kui on  keegi ütleb, et midagi ei tööta ja kõik 
ütlevad, et ei noh, aga töötab, siis kuidas saada aru, et mida inimene 
tegelikult tahab. Äkki tal on õigus, et tal ei tööta. Äkki on olemas võimalus, 
et see asi, mida meie oleme nunnutanud ja silunud ja teinud maailma kõige-kõige 
paremaks, et see tema kontekstis ei tööta. Ja täiesti üllatavalt tuleb välja, 
et kui sul on piisavalt keerulised süsteemid, siis  neid olukordi, kus on mingi 
asi, mis vaatamata kõige suurematele ja parematele püüdlustele vajaks ikkagi 
teisiti toimima panemist või siis võib-olla seletamist,  et seda on uskumatult 
palju. 

\question{Küll sa selle turunduse asja ka ära optimiseerid, nagu sa kõik asjad 
ära oled optimiseerinud!}

Jah, ma üritan. Mul see lootus on natuke teistpidine. 

Kunagi Andres Kulli\index[ppl]{Kull, Andres} ja Kroonpressi\index{Kroonpress} 
seltskond tuli küsima, et kuidas panna reklaami ajalehte. Mina rääkisin, et on 
olemas PDF. Teeme parem nii, et kõik  teeksid korraliku PDF-i, leheküljendaja 
tõstab selle küljendus-softi sisse ja kõik töötab. Kull selle peale, ikkagi 
suure trükikoja juht, ütles \enquote{Väga hea, siis  me otsustame niimoodi. 
Kõik peavad saatma PDF-ina asju Postimehesse}. Ja üllatus-üllatus, nii läkski. 
Mu enda roll selle kõige juures oli, et ma olin olnud pikka aega Prismas ja 
muudes reprodes selline majasõber, kes sageli tolkneski seal ja üritas endale 
tegevust leida ja saada aru, kuidas need asjad käivad. Kuni, kaasa arvatud see, 
et Eesti esimese Linotronicu me oleme pulkadeks lahti võtnud ja midagi seal 
jootnud, sest ta  otsustas lõpetades parasjagu töö kui oli vaja midagi välja 
lasta. 

Ma olin näinud, millist roppu vaeva näevad kõik minu  sõbrad, kes on sellised 
repropealiku või  sellise repro tehniku rollis,  kehvasti ette valmistatud 
originaalidega. Ja kui see PDF-i asi hakkas meile endale majja tulema, ma 
mõtlesin, et no mina selle ussipurgi avamist küll enda peale võtma ei hakka 
(tänapäeval räägitakse rohkem surströmmingust, kui Pandora laekast). Et ainuke 
asi, mis ma saan teha, on õeptada kliendid paremini originaale saatma. Mis 
loomulikult tundus äärmiselt lihtne. See on ju nii lihtne teha:  ma lihtsalt 
ütlen teile, et seal on vaja nagu mõned linnukesed panna ja siis see kõik 
lähebki nii, nagu vaja. Aga tuleb välja, et ei. Ma olen õppinud, et on päris 
kõva pingutus aru saada,  mida teised inimesed teavad, mis on nende taust. Ja 
siis viia nad selleni, et nad saaksid aru millestki, millest sina aru saad 
seejuures võimalust mööda ise mitte liigselt kas siis masendumata või siis 
nende peale kurjaks saamata. Nii ma sattusingi õpetama  kõiki neid Pagemakereid 
ja InDesigne ja Photoshopoe ja kõiki muid asju just sellise töökorralduse poole 
pealt. Ja hetkel ma lihtsalt Zones näen, et kui vaadata kogu seda veebiga 
seonduvat, siis ilmselt tuleb selle kõigega proovida rohkem edasi minna. 