\index[ppl]{Marvet, Peeter}
\index[ppl]{Tehnokratt|see{Marvet, Peeter}}

\question{Kuidas ja millal sa jõudsid arvutite juurde?}

See oli umbes täpselt aastal 1985, pidin siis olema 15 aastat 
vana. Eelnevalt olin arvuteid näinud Soome televisioonist, 
seal reklaamiti ilmselt Commodore 64 ja Spectrumi masinaid. 

\question{Sa oled järelikult Tallinna poiss?}

Jah. Ma olen sündinud Tartus, aga pikalt Tallinnas elanud. 

Kui ajas veel tagasi krutin, siis üks kokkupuude arvutitega oli 
veidi varem, papsi laboris. Ta töötas TPI Veekvaliteedi 
Laboris\index{Tallinna Tehnikaülikool!Veekvaliteedi labor}, mis asus selle koha peal, 
kus keset Järvevana teed on praegu Maru Ehituse maja. Seal oli olemas üks 
terminal, mis käis Datasaabi\index{Datasaab}-nimelise 
arvuti\sidenote{Datasaab oli Rootsi lennukitootja Saab arvutustehnoloogia 
eraldi ettevõtteks kasvanud divisjon, kus toodeti nii tsiviil- kui ka
militaarkasutuseks mõeldud arvuteid.} külge, mis asus kusagil Mustamäe teel. 
See masin oli ostetud ühelt rahvamajandussaavutuste näituselt, kus 
vahetevahel käisid ka välismaalased kohal. Kuskilt oli saadud valuutat ja ostetud selle
eest välismaa arvuti, mille külge käisid modemitega oranžid terminalid. 

Datasaab osteti väidetavasti ilma 
operatsioonisüsteemi ja igasuguse rakendustarkvarata, sest rohkem 
rutsi sellel ajal ei olnud. Aga Nõukogude insenerid olid vinged, kirjutasid sinna ise operatsioonisüsteemi peale. Sellest tarkvarast õnnestus veel mingisugune jupp Datasaabile 
tagasi müüa ja saada ilmselt vastu mälu või 
lisakomponente. 

Datasaabi terminalil õnnestus 
lihtsalt, ilma ühenduseta \emph{backspace}'i ja tühikuga 
\enquote{ronge kokku haakida}. See on minu esimene mälestus arvutiga suhestumisest. 

Järgmine mälestus on samuti aastast 1985, kui olin 
ilmselt seitsmendas klassis. Toonases Pedas toimus Tallinna koolide 
füüsikavõistlus ja meid viidi ka arvutisaali, kus oli Minsk\index{Minsk}. Seal sain tuttavaks ühe aasta vanema
koolivennaga, 
kellel oli kaasas isiklik perfolint programmiga. See ei olnud
keegi muu kui Sulo Kallas\index[ppl]{Kallas, Sulo} ja tema perfolindi peal oli 
üks mängulaadne asi, mis arendas mingisuguseid 
organisme\sidenote[][-2cm]{Tõenäoliselt oli lindil Briti 
matemaatiku John Horton Conway välja mõeldud rakuautomaat, mida tuntakse 
nime all Game of Life. Tegu on mängijateta mänguga, mis ainsa sisendina 
vajab algseisu määratlemist. Automaat on ühest küljest levinud 
programmeerimisülesannne ja teisalt põnev uurimisobjekt, seetõttu võis selle 
realiseerimine olla noorele arvutihuvilisele nii huvitav kui ka jõukohane.}. 
Sulo vend oli Raadiomaja 
Arvutuskeskuses\index{Raadiomaja Arvutuskeskus}, nii et selleks ajaks oli Sulo 
juba mõnda aega arvutitega tegelenud. 

Ja see oligi esimene kord, kui sattusin arvutiga kokku ja mõtlesin 
\enquote{oo, vinge!}.

\question{Mis seal vinget oli? Mis konksu külge sa jäid?}

Tol hetkel oli see rohkem \enquote{ahaa, vau, teebki 
mingisugust asja!}. 

\question{Ja Sulo oli kõva mees oma perfolindiga?}

Nojaa, ikkagi kaheksandik, vanem koolivend, kellel on, kujutad 
ette, isiklik perfolint! Vau! Sellised kutid on ümberringi! Siis peab 
ikka ise ka vaatama, mida seal tehakse. Küllap ka Soome televisioonist 
arvutitega seoses nähtu tekitas soovi, et olgu või 
Nõukogude oma ja perfolindiga, aga ikkagi arvuti. 

Mõni kuu hiljem tulid kooli paar djuudi, kes tegid arvutiklubi ja 
kutsusid mind osalema. 

\question{Kas see oli legendaarne arvutiklubi Ahhaa?}

Ei, see oli legendaarne arvutiklubi Juta\index{Juta}, mida vedas 
juudi papi Lev Moišeejevitš Šoroht\index[ppl]{Šoroht, Lev}. Klubi 
tegutses Raua ja Kreutzwaldi tänava nurga peal, ühe maja keldrikorrusel, kus on kaarega 
aknad. Nende akende taga asuski arvutiklubi Juta. Kui 
Ahhaa puhul võiks ette kujutada, kust see nimi tuleb, siis 
Juta nimi tuleneb vene keelest: \begin{russian}Юный Техник 
Автомат\end{russian}. Ma eeldan, et Lev Moišeejevitš Šoroht ei satu seda 
lugema, aga kui kellelegi meenub, et aastal 1985 vedas ta TPI või Peda 
üliõpilasena noori arvutiklubisse, siis ma suurima hea 
meelega saaksin kokku ja teeksin väiksed õlled välja, sest sealt see suurem arvutihuvi alguse sai. 

Klubis õpetati meile programmeerimist PL/I\index{PL/I} keeles.

\question{Kas klubisse käidi kutsumas, mitte ei joostud ust maha, et arvuti ligi 
saada?}

Ma arvan küll. Ma täpselt ei mäleta, aga sõnum jõudis 
meieni vist kooli või õpetajate kaudu. Küllap Sulo oli ka seal, 
sest kui oli võimalik kusagil veel arvutisse saada, siis loomulikult seda võimalust 
kasutati.

\question{Seda ma mõtlengi, et tol ajal ju otsiti tikutulega kohti, kus 
\enquote{arvutisse saada}.}

Seitsmenda klassi lõpupoole tuli arvutuskeskusest ühe minu 
programmi väljatrükk laia aukudega paberi peal, \emph{line}-printeril 
välja lastud. Keegi klassivendadest, kes oli ilmselt meiega seal 
koos käinud, tõi väljatrüki klassi ja siis kõik vaatasid, et 
oo, Soome telekavad. Tollal oligi inimestel kõige 
üldisem seos arvutitega see, et arvutuskeskustes trükiti välja Soome 
telekavasid, kus olid näiteks tabuleeritud kujul eraldi väljavõtted 
seriaalide kohta. Parimatel vendadel olid olemas nädalakavad. Aastal 1985 
keskmine teadlikkus arvutitest umbes selline oligi.

\question{Kust need kavad saadi?}

Need liikusid arvutuskeskuste vahel ja vähemasti millalgi oli üks selline 
koht Postimaja Arvutuskeskus\index{Postimaja Arvutuskeskus}, mis on 
üks väheseid kohti, kus ma ise pole käinud. Seal oli SM-4\index{SM EVM!SM-4}, mille külge oli ehitatud teksti-TV 
vastuvõtja\sidenote{Kavade allikaid oli rohkem kui üks. Mäletatakse, et vastav 
riistvara oli olemas TPI raadiotehnika kateedris\index{Tallinna Tehnikaülikool!Raadiotehnika kateeder} 
Apple II küljes. Räni Meister\index[ppl]{Meister, Räni} olla selleks otstarbeks kasutanud 
ka Eesti Televisiooni\index{Eesti Rahvusringhääling!Eesti Televisioon} Amigat.}, 
ja sealt see kava tuli. SM-4 küljes oli 300boodine modem, millega 
pumbati kavasid mööda linna laiali. Mäletan üht
etappi, kui minu päralt oli üks PC, välja arvatud vist kolmapäeviti, 
kui lõuna paiku saabus Postimajast üle modemi Soome telekava, mis trükiti maatriksprinteril enam-vähem nähtamatuks kulunud lindiga välja. Siis 
pidin endale muud tegevust leidma, aga muudel pärastlõunatel sain seda 
arvutit kasutada.

\question{Ma katkestasin sind seal, kus sa PL/I keeles programmeerisid \ldots}

Meile õpetati natuke programmeerimist ja seal oli palju segaseid ja 
täiesti arusaamatuid asju. Oli programmeerimiskeel mingisuguste muutujatega, 
mis mõnevõrra koitis. Ja ilmselt esimene programm, mida meile õpetati, oli ruutvõrrandi lahendamine. Annad paar muutujat sisse ja siis 
trükitakse tulemus paberil välja. Alguses meid 
päriselt arvuti juurde ei lastudki -- keegi toksis  meie 
programmid sisse ja pärast saime väljundi kätte. 

Hiljem leiti meile võimalus arvutitega tegeleda veel kahes kohas. Üks oli 
Tihnikus, kus asus ETKVLi Arvutuskeskus\index{ETKVLi Arvutuskeskus}. Järgmised 
põlvkonnad teavad seda kohta kui esimest Maksimarketit, seal ühes majas oli üks 
vinge ES\index{ES EVM}. Teine koht oli Endla tänaval, kus asus Maksuameti maja\sidenote{2013. aastani asus 
Maksu- ja Tolliameti teenindussaal aadressil Endla 8.}. Seal kolmandal 
korrusel olid Ehituskomitee\index{Ehituskomitee} ESid\index{ES EVM}. 

Minuga läks edasi umbes niimoodi, nagu õpetatakse tööõpetuses, et on 
oluline anda lastele midagi, mille nad saavad valmis voolida, näiteks puulusika,
et nad saaksid tulla koju ja seda perele näidata. Siis laps saab kiita, tal läheb edaspidi väga 
hästi ja ta teeb paremaid puulusikaid. Kui ma olin teinud 
oma esimese kolmeteistrealise programmi ruutvõrrandi lahendamiseks, laekusin 
selle väljatrükiga koju ja köögis näitasin vanematele, et 
näete, sihukese raha eest tegin sihukese asja. Paps, kes oma 
teadustegevuses tegeles elektrokeemia ja hapnikuanduritega ning teisalt 
oli džässpianist, vaatas mu tööd ja ütles: \enquote{Mul oli 
just üks tudeng, aga ta kadus ära ja temast jäid ainult mingid listingud 
järele. Kas sa saad nendest sotti? Mul oleks vaja teadusandmeid 
töödelda.} Ülesandeks oli anduri toimekõverate kokkuajamine 
matemaatiliste valemitega, et õnnestuks digitaalseid mõõteriistu teha. Ja nii juhtuski, et olles programmeerinud oma 
esimesed kolmteist rida esimeses mulle täiesti tundmatus keeles, 
läksin kohe üle järgmisele.

Nii ongi mu pea nagu puder ja kapsad selles mõttes, et ma suudan kirjutada 
ainult dokumentatsiooni abil, kaasa arvatud keeli, mida ma igapäevaselt 
kirjutan, nagu PHP\index{PHP} ja JavaScript{Keeled!JavaSctipt}. Need 
süntaksid on peas nii segi, et ma ei mäleta kunagi täpselt 
PHPs \verb|for|-tsüklis parameetrite järjekorda. Õnneks on tänapäeval 
olemas kõikvõimalikud IDEd, mis teevad mõningase töö ära ja aitavad \emph{auto 
complete}'ida. 

\question{Nii et su puulusikas mitte ainult ei saanud kiita, vaid pandi kohe 
ka tööle!}

Puulusikas võeti kohe tööle, sealt edasi olin lapstööjõud. Ühel
hetkel hakkas papsil kahju -- laps võiks lisaks 
ekspluateerimisele natuke ka raha saada. Mind pandi ametlikult kirja 
veerand kohaga laborandina. Tänu sellele oli mul ligipääs kõikide papsi sõprade arvutuskeskustele. 
Ja kuna paps tegeles oma hapnikuga TPIs, siis loomulikult olid nende 
sõprade hulgas TPI ja teisi seltskondi, kes olid seotud mingisuguste 
anduritega. Näiteks Pirital Masti tänaval 
arendati sportlaste mõõtmise lahendusi ja selle teine ots asus 
kiirabihaigla arvutuskeskuses\index{Kiirabihaigla arvutuskeskus}. Seal saingi 
pidevalt üht arvutit kasutada, välja arvatud kolmapäeviti. 

Arvutiks oli Sanyo PC ja see oli väga vinge. Seal oli muuseas olemas ka Apple II\index{Apple II}, mille peal sai mängitud Karatekat\index{Karateka}. Ja kui ma 
õigesti mäletan, oli seal ka üks Labtami\sidenote{Austraalia arvutitootja aastatel 1972--1990, kellel 
olid Nõukogude Liiduga head suhted. Aastal 
1984 disainis Novosibirski Riikliku Ülikooli tudengite Kronos Research Group 
neile emaplaadi. URAL-LABTAM OOO tegutseb Venemaal siiani ning nende arvuteid 
leidub lisaks Austraalia arvutimuuseumidele Tartu Ülikooli omas. Labtami 
arvuteid osteti naftadollarite eest ka Küberneetika 
Instituuti\index{Küberneetika Instituut}.}-nimeline \emph{kone}. Lisaks oli seal
suurte trumlitega andmetöötlus-\emph{kone}, millega mina ei suhestunud ja mille  
nime ma ei mäleta. 

\question{Kas andurid ja elektroonika ei pakkunud sulle huvi?}

Mitte eriti. Progemine oli huvitavam. 

TPI santehnika kateedris\index{Tallinna 
Tehnikaülikool!Santehnika kateeder} oli olemas SM-4\index{SM EVM!SM-4}. Ehituse all oli 
selline kateeder, vee kvaliteet ja kõik selline kuulus sinna alla. Mingil 
hetkel tekkis sinnasamasse Järvevana teele, kus asus ka Läänemere 
Instituut\index{Läänemere Instituut}\sidenote{Ei ole selge, mis asutust Peeter 
silmas peab. Eestis on Läänemere Instituut tegutsenud eelmise sajandi 
kolmekümnendatel ja praegu tegutseb sellenimeline asutus Soomes.}, ka 
SM-4\index{SM EVM!SM-4}, mille ma kohale 
minnes lülitasin ise sisse ning pärast töö tegemist jälle viisakalt välja. 

\question{Arvuteid oli siis ikkagi piisavalt?}

Kui sattusid õigesse kohta ja 
oskasid õigel ajal vait olla ning mitte liiga palju täiskasvanuid segada nende 
tähtsas töös, siis üldiselt jagus. 

Tulles korraks veel tagasi alguse ehk Juta\index{Juta} juurde, 
siis selle asutajal oli endal ka paar huvitavat projekti, millega ta 
üritas Nõukogude Liidu tasemel kuulsaks saada. Üks neist võiks olla võrreldav 
Facebookiga: kirjasõbrad kogu Nõukogude Liidust 
saadaksid oma andmed, mis sisestatakse perfokaartidel 
\emph{mainframe}'i ja see teostab \emph{match-making}'u ning 
siis saadetakse kirjad leitud \emph{match}'idele laiali. 

Mina käisin algusest lõpuni Reaalkoolis ja keskkooliajal tekkisid arvutid ka meie kooli. Saime klassitäie Yamaha MSXe\index{Yamaha MSX}. Kuna erinevate koolide vahel oli masinate saamiseks 
konkurents, olla Reaalkool saanud ka ähvarduskõne, mille peale 
vaprad raadiorufi ja füüsikaklassi tagaruumi noored organiseerisid 
öö läbi valve koolimajja. Arvutikastid olid vist direktori kabinetis, me ööbisime koolis ja valvasime neid. Pärast sai
arvutiklass meie teiseks koduks.

\question{Mis seltskond seal raadiorufi ja füüsikaklassi tagaruumis koos käis?}

Seal olime mina, Sulo Kallas\index[ppl]{Kallas, Sulo}, Heiki 
Savitš\index[ppl]{Savitš, Heiki}, Vallo Veinthal\index[ppl]{Veinthal, Vallo} 
ja Reimo Mesipuu\index[ppl]{Mesipuu, Reimo} -- kindlasti jätan kedagi 
ebaviisakalt mainimata. Avo Nappo\index[ppl]{Nappo, Avo} tiirles meie ümber 
rohkem arvutiseltskonna poolest, raadioruumis olime põhiliselt vist mina, Sulo 
ja Reimo.

\question{Mille alusel seltskond moodustus? 
Klassivennad? Tehnikahuvi?}

Otseseid klassivendi oli vähe, jäime paariaastasesse vahemikku. 
Sulo oli kõige vanem, Vallo ja Heiki olid meist aasta nooremad. Tegu oli pigem kooli aktiiviga, keda huvitas tehniline pool. 
Füüsikaklassi juures oli raadioruum, kus me hängisime, sest seal sai 
nuppe keerata. Sellest pundist tekkis hiljem suurem seltskond arvutiklassi ümber. 

\question{Kas programmerimine ja raadioruumitamine koolitööd ei hakanud segama?}

Lõpetasin kooli neljade-viitega, nii et selles mõttes probleemi ei olnud. Medaliga lõpetajat poleks minust nagunii saanud, see ei olnud minu maailmavaates.

\question{Eks see ongi tunnetuse küsimus, kumb oli tol hetkel primaarne.}

Eks arvutipool oli põhiline. Keskkool möödus üleüldse enam-vähem 
niimoodi, et vahepeal sai käidud kohvikus ja vahepeal 
olümpiaadidel. Kui olid olümpiaadid, olid hinded head, sest õpetajad ei 
saanud ometi olümpiaadil esinejatele halbu hindeid panna. Aga kui olümpiaadil 
ei käinud, siis kippusid hinded kehvemaks minema, sest kooliskäimine ununes. Näen siiamaani unenägusid sellest, et eksam on 
tulekul ja ma olen unustanud terve veerandi tunnis käia. 

Sellest tekkis mõtteviis, et ma ei pea olema midagi õppinud. Kui läksin
TPIsse ja mataeksam tehti koos raamatutega, siis jõudsin eksami 
käigus alati ära õppida selle, mida oli eksamiks vaja. Ma ei pidanud 
eelnevalt liiga palju loengus käimisele pühenduma, vaid võisin 
lihtsalt tulla ja eksamid ära teha. Ülejäänud semestri sai arvutitega 
tegeleda. Ma kindlasti ei soovitaks seda noortele, aga minul juhtus
niimoodi. 

Selline lähenemine tekitas mul teistsuguse 
arusaama ümbritsevast tehnikast. Ma ei karda midagi selles 
mõttes, et kui on vaja asi ära teha, siis tuleb võtta \emph{manual} või 
kood ette. Loomulikult võtab see aega -- läksin eksamile 
esimesena sisse ja tulin viimasena välja, aga sain kolme tunniga 
õige asjaga hakkama. Tundus nagu efektiivne lähenemine. Võibolla 
oleksin saanud targemaks, kui oleksin süsteemsemalt õppinud. 

\question{Mida sa TPIsse õppima läksid?}

See oli TI ehk majandusinfo töötlus\index{Tallinna Tehnikaülikool!TI}. Linnar 
Viik\index[ppl]{Viik, Linnar} lõpetas sama ala mõned aastad enne mind. 
Aasta sattus olema 1989, kui päris pol-ök-i ja kompartei 
ajalugu\sidenote[][-2.5cm]{Nõukogude ajal kuulus ülikoolihariduse juurde kohustuslikus korras \enquote{punaste ainete} 
läbimine: kaheksakümnendatel olid nendeks kommunistliku partei ajalugu, dialektiline ja ajalooline materialism, kapitalismi ja sotsialismi poliitökonoomia, filosoofia ajalugu ja teaduslik kommunism. Lisaks veel teaduslik ateism ja marksistlik eetika. Ka kooli lõpetamisel tuli üks riigieksamitest sooritada mõnes nendest ainetest} ei oleks tahtnud õppida, aga nad ei olnud veel välja 
mõelnud, mida nende asemel õpetada. Oli ka muid asju, mille vajalikkusest ma 
päris täpselt ei saanud toona aru. Näiteks
miks ma peaksin tegema transistoritest valmis 8080 protsessori paar käsku, 
eriti kui normaalsed inimesed kasutavad vähemasti Z80-t, mitte 8080-t. 
Teismelise värk -- ei olnud piisavalt \emph{cool}. \enquote{Intel 8008? 
Zilog\sidenote[][-3mm]{Zilogi toodetud 8-bitine Z80 protsessor oli Inteli 
8080 protsessoriga ühilduv, aga märkimisväärselt odavam.} on normaalne!} 
Täpselt sama lugu, nagu täna on hipsteri habe või muud välised 
tundemärgid. 

Pean tagantjärele tunnistama, et kuigi olin algul selle suhtes 
kriitiline, siis hetkel käin koolitusel, kus räägitakse sellest, kuidas 
\emph{fuzz}'imisega\sidenote{\emph{Fuzzing} on tarkvara (turvalisuse) testimise 
meetod, kus programmile söödetakse süsteemselt juhuslikku sisendit.} 
mälukorruptsiooni juhtumeid leida. Kui lektor ütles, et see on maru keeruline, 
räägime hästi aeglaselt ja mitu korda nagu miilitsatele, siis minu arust midagi 
nii rasket seal polnud, \emph{stack} on \emph{stack}. Protsessoril on 
registrid, ma olen neid transistoritest teinud. Kui 
pead protsessori arhitektuuritasemel läbi mõtlema, kuidas käskude 
töötlemine toimib, kuidas pointereid inkremenditakse ja 
kuidas see mäluga on seotud, siis saad aru, kuidas arvuti 
masinkoodi tasemel töötab. Mul on väga tore kuulata, kuidas mu vanem poeg räägib, et
Tartu Ülikoolis sunnitakse neid ka aru saama protsessori siseehitusest. 
Tõsi küll, raamatu tasemel, aga nad programmeerivad ka assemblerit ja see on väga oluline. 

\question{Kas sind akadeemiline maailm ei tõmmanud, kuigi servapidi olid
juba selle sees?}

Ei, sest ma sattusin keskkooliajal sellisesse seltskonda nagu 
vabariiklik õpilasstaap\index{Vabariiklik õpilasstaap}, mis oli 
komsomoli keskkomitee juures tegutsev mittekommunistlik vastupanuliikumine. 
Tiina Tšatšua\index[ppl]{Tšatšua, Tiina} oli näiteks üks selle eestvedajaid. 
Sellest sai vabariigis üks toonaseid orgunnitiime, kes korraldas 
suurüritusi, milleks alustuseks olid komsomoli ja EKP kongressid. Organiseerimise
mõttes on ju savi, kas tegu on EKP kongressi või Eesti Kongressi või Rahvarindega. Inimesed tulevad kohale, neid tuleb 
registreerida ja toita. Kui on dokumentidega üritus (mida 
tänapäeval eriti ei toimu, aga kõik Eesti Kongressi ja Rahvarinde kongressid 
olid sellised), siis on olemas näiteks redaktsioonitoimkond. Meie olime 
arvutitiim, kes organiseeris seda, et registratuur toimiks 
listide alusel, ja samuti toetasime redaktsioonitoimkonda kõikvõimaliku 
tekstitöötluse, väljatrükkimise ja vormistamisega. 

Kui keskkool sai läbi 1989. aastal, siis oli mul suveks üks tööots. Tallinnas 
toimus ÜRO invaekspertide tipptasemel kokkusaamine. Tallinnas lõigati sel puhul esimesed äärekivid faasi ja minu 
arust Jack Lippmaa\index[ppl]{Lippmaa, Jaak}\sidenote{Peeter peab ilmselt 
silmas Jaak Lippmaad} isiklikult ehitas ümber paar 
Ikaruse bussi\sidenote{Ungari tootja Ikarus bussid olid Eestis laialt kasutusel 
liinibussidena.} nii, et neisse kuidagi ratastooliga sisse saaks. Kuidas see 
võimalik oli, ma ei kujuta ette. Meie ehk Reaalkooli tiim toetas ürituse 
redaktsioonitoimkonda, kes vormistas ÜRO-le kõigis põhikeeltes 
dokumente. See tähendas, et oli posu tõlkijaid, aga aastal 1989 ei olnud ilmselt ükski tõlkija näinud arvutit rohkem kui 
võibolla Soome reklaamides. Meid oli piisavalt palju, kümmekond inimest, ja hoidsime tõlkijatel kätt ja jalga. Kui kellelgi tekkis kivistunud pilk, siis 
keegi meist tuli ja \emph{reboot}'is tõlkija arvuti taga või arvuti enda, kumb 
parasjagu oli rohkem kinni jooksnud. 

Minu enda hilisemas eluloos on see episood huvitav sellepärast, et 
olles parasjagu keskkooli lõpetanud, õnnestus mul tolle ürituse jaoks lihtsalt 
omaenda sõna peale linna pealt toatäis PCsid kokku laenata. 
Paar tükki siit, paar tükki sealt ja kokku sain umbes kaheksa arvutit. Kõige kihvtim 
tuli surnukuurist -- üks PC, mille peal oli 
Xerox Ventura Publisher koos Xeroxi graafilise kasutajaliidesega, milleks oli 
GEM ja mis nägi välja nagu MacOS\sidenote{Graphics 
Environment Manager oli üks varastest graafilistest 
kasutajaliidestest, mille liigne sarnasus Apple'i tarkvaraga viis ka kohtuasjani.}. GEM 
sai DOSist üles \emph{boot}'itud, läks ilusti mustvalgeks ja halliks 
kasutajaliideseks ning seal peal jooksis minu esimene küljendusprogramm. Lisaks
saime neilt ühe laserprinteri kasutada, mis ei olnud küll PostScript, aga siiski laserprinter. 

\question{Siis oli ju veel Nõukogude aeg!}

Ilmselt meditsiin oli saanud üht-teist valuuta eest 
osta. Tegelikult Kivilo\index[ppl]{Kivilo, Ago} plaanis kesklinna oma 
diagnostikakeskust\index{Diagnostikakeskus}\sidenote{1988. aastal asutatud Diagnostikakeskus oli omal 
ajal märgilise tähendusega. Ühest küljest pakuti kõrgtehnoloogilisi 
teenuseid, keskuse algusaegadel asus seal Eesti ainus kompuutertomograaf. 
Teisalt oli tegu väga innovatiivse organisatsioonilise 
konstruktsiooniga, mis viis hiljem mitme keskust ümbritsenud 
kõrge profiiliga afäärini.}, meditsiinis olid 
väga kõvad tegijad. Eesti arvutinduse arendusest teatakse rohkem Tartu
seltskonda, kes on seotud geneetikaga, ja võibolla 
Küberit\index{Küber}. Mina sisenesin meditsiiniliini pidi, selles 
valdkonnas tegeldi päris kõvasti teadus- ja arendustegevusega. 

\question{Kas sealt said ka oma küljendamiskonksu?}

Jah. Otse loomulikult sai hunnik flopikettaid Venturaga ära kopeeritud, mis toona oli igati tavapärane, \emph{standard operating 
procedure}: kõigest, mis kätte sattus, tehti koopia. Ja nii juhtuski, 
et aastatel 1989--1990 oli minu jaoks ülikoolis käimisest palju 
huvitavam arvutiga küljendamine ja kujundamine. 

\question{Kas sul on muidu ka joonistamise soon?}

Ei ole. Kahtlustan, et inimesed, kes on pidanud minu küljendatud raamatuid 
tarbima, on kindlasti selle all kannatanud, nii et ma väga vabandan. Näiteks
Avita\index{Avita kirjastus} kirjastuse algalgusaegade raamatutest oli suur hulk 
minu tehtud.

\question{Mis sind küljendamise juures köitis, kui sul muidu ei olnud
visuaalkunsti huvi?}

See oli hoopis teistmoodi arvutiga 
tegelemine kui programmeerimine ja andmetöötlus, mis olid ka toredad. 
Mind tõmbas see, et õnnestus asju ekraanil teha. 

Pärast 1989. aasta suve üritust läksin ma ülikooli. Ja siis 
Mart Siilmann\index[ppl]{Siilman, Mart}, kes oli äsja lõppenud ürituse orgunni 
pealik, ütles, et kuule, järgmisel suvel on ka üks üritus, kus on 
arvutiabi vaja, tule ka. Aastal 1990 toimuski European Nuclear Disarmament Convention ehk suur rahuvõitlejate ja roheliste üritus. 
Sellega seoses tekkis meil ühte kontorisse, mis asus nüüdseks 
lõpetanud NO-teatri ruumides, üks PC, vist Sanyo. Selle küljes oli 1200boodine või -bpsine 
modem. Mingi koha peal lähevad boodid ja bpsid vist lahku\sidenote[][-3.6cm]{Bps (\emph{bits per second}) on sekundis edastatavate bittide hulk. \enquote{Boodid} (\emph{baud rate}) näitavad aga, mitu korda 
sekundis signaal muutub. Kuniks kasutatakse tavalist jadaporti, kus signaalil 
on kaks taset, on väärtused võrdsed. Keerulisemate skeemide korral võib ühe 
signaalimuutusega edastada rohkem kui ühe biti ning kiiruseühikud lahknevad.}. 

Meie ametlik tegevus oli suhtlus orgkomiteega ja selleks 
sai helistatud kaugekõnega Tallinnast Helsingisse. Eestis oli otsevalimine -- 
meil oli selles mõttes väga vinge positsioon, et mujal Baltikumis välismaa 
numbreid otse valida ei saanud. Ka Tallinnas ei olnud seda võimalust igal pool, aga meil oli, sest see 
oli ürituse jaoks oluline. Mart Siilman, endine Fila direktor\sidenote[][-3cm]{Eesti NSV Riiklik Filharmoonia\index{Eesti Riiklik 
Filharmoonia}, mille järeltulija on alates 1989. aastast Sihtasutus Eesti 
Kontsert. Tegu oli mõjuka asutusega, mille korraldada oli kogu Eesti
kontserdielu, sealhulgas levi- ja jazzmuusika ning estraad. Seega 
oli \enquote{Fila endine direktor} äärmiselt mõjukas inimene, kelle jaoks Soome 
otsevalimise korraldamine oli kindlasti võimalik.}, organiseeris, mida vaja. Kuidas, ei tea. 
Igatahes saime helistada Datapakki X.25 võrku, mille kaudu oli võimalik 
suhelda ühe Rootsi serveriga, teine server oli Kanadas. Sealtkaudu 
suhtlesime ürituse orgkomiteega, aga hakkasime ka vaatama, kuhu veel 
õnnestub helistada.

\question{Kuidas te seda \emph{bootstrap}'isite? Mida kliendi poolel vaja 
oli, et võrku saada?}

Tavalist modemit ja tavalise modemiga suhtlevat terminaliproge. 
Modemiga helistasime Datapaki liidestuspunkti, kust edasi läks asi 
pakettvõrguks või X.25ks. Terminali peal oli nagu ikka: lehekülg skrollib ja menüüst tuleb valida 
\enquote{üks, kaks, üksteist}. Lisaks oli meiliboks, kus sai kirju vahetada, ja
jututubade või listide alajaotus. Ja siis, parafraseerides Heinleini: 
\enquote{\emph{Have modem, will find BBSs}}\sidenote{Robert A. 
Heinleini 1958. aasta jutustus \enquote{Have Space Suit -- Will Travel}.}. 
Loomulikult leidsime üles ka selle, et on olemas BBSid. 1989. 
aasta lõpus tekkis Lembit Pirnil\index[ppl]{Pirn, Lembit} esimene 
PirnBoxi\index{PirnBox}-nimeline BBS, mis asus praeguse SEB taga, kus trammid toona kõva kriginaga keerasid, 
Autotranspordi Arvutuskeskuses\sidenote{Eesti NSV 
Autotranspordi Arvutuskeskus (ATAK).}. Nii et 
alguses helistasime kõik sinna Pirni BBSi sisse. 

Peagi tekkis HNS ehk \emph{Hackers Night 
System}\index{HNS}. Kolmas oli Goodwin BBS\index{Goodwin} meil 
Suloga\index[ppl]{Kallas, Sulo}, mis ilmselt jooksis sellesama 
väljahelistamise liini otsas. Öösel jätsime arvuti sisse ja kõik said 
sisse helistada. Kui tahtsid kuhugi sisse helistada, aga liinid olid kogu aeg 
kinni, siis ainuke võimalus olukorda parandada oli panna ise ka üks
\emph{box} püsti. 

Sealt tekkis siis ka Fido pool ja jällegi sissehelistamise küsimus -- 
kui meil oli võimalik e-post ja jututoad omavahel 
kuidagi sünkroniseerida eri masinates, siis polnud ju vahet, kuhu me sisse 
helistame. Masinad käivad päeval ja ööl ning vahetavad omavahel 
sõnumeid. Fido oli selles mõttes korralikult distributeeritud nett. Seesama, 
mille kohta nüüd öeldakse \enquote{veeb kolm}. 

\question{Võrgustike \emph{bootstrap}'imine on keeruline just inimeste 
mõttes. Selleks, et kuhugi külge minna, peaks seal olema huvitav, ning selleks omakorda 
peaksid seal olema inimesed. Mida te näiteks PirnBoxis huvitavat
tegite?}

Ilmselt lämisesime niisama. Pean tunnistama, et ei mäleta, 
aga väga huvitav oli igal juhul. Oletan, et kuskilt
pääses ligi faili kujul \emph{sci-fi} raamatutele ja 
laiematele uudisegruppidele, mis kuskil liidestusid 
Fidonetiga, nii et informatsiooni liikus. Lihtsalt kirjutada
oli ka huvitav, et vau, kõik liigubki traadi kaudu! 
See oli tollal nii \emph{amazing}. Sellest ma sain aru, et arvutiga saab
programmeerida ja midagi kujundada, aga et ka reaalselt suhelda!

\question{Mis tegi ühe BBSi populaarsemaks kui teise? Goodwin ja HNS 
olid pikka aega populaarsed, kuigi PirnBox oli esimene.}

See oli esimene jah, aga jooksis toona vähem 
levinud softi peal. Meil oli vist Maximus. 

Sulo oli omamoodi arvamusliider, kuna tal olid 
kõikvõimalike asjade suhtes väga toredad ja tugevad seisukohad. 
Mina olin niisuguse tutu-lutu taustaga, olles olnud muu hulgas 
Reaalkooli\index{Tallinna 2. Keskkool} viimane komsomolisekretär.
Arvestades et enne mind oli komsomolisekretär Karl Martin 
Sinijärv\index[ppl]{Sinijärv, Karl Martin}, siis me ilmselgelt ei võtnud seda 
asja väga tõsiselt.

Kuidagi me sattusime seda asja vedama, kuna meil oli tänu sellele 
tuumaüritusele ressurssi käes. Ühel hetkel tekkis meil igatahes kaks 
telefoniliini, võibolla aastake hiljem, kui üritus läbi sai ja 
olime juba Eesti Instituudi\index{Eesti Instituut} ruumides, veidi enne 
seda, kui Eesti Instituut osutus tegelikult Eesti välisesinduste ja iseseisvuse 
ettevalmistuslavaks. Näiteks kui kuulutati välja iseseisvus, 
tuli järsku välja, et Jüri Luigel\index[ppl]{Luik, Jüri} ja kõigil teistel, kes 
mööda maailma laiali olid, olid juhuslikult kaasas ka pruunid ümbrikud 
esitamiseks kohalikule võimupealikule küsimusega, kas teie ekstsellents 
lubaks meil asutada suursaatkonda. 

Eesti Instituudis olid meil ka oma arvutid, aga ma ei mäleta, kas meie enda või instituudi omad. Saatsime Suloga\index[ppl]{Kallas, Sulo} öösiti 
fakse. Mitu toredat kolleegi 
oli, vähemalt huumoriga pooleks, sügavalt veendunud, et faks ongi selline seade, et kui sinna 
peale panna paber koos kollase post-itiga, kus on telefoninumber, siis on see 
hommikuks ennast ära saatnud. Tollased liinid toimisid öösiti oluliselt paremini kui päeval. 

Tingituna sellest, et välisühendust oli meil läbi modemi helistades 
suhteliselt piiramatult käes ja liine oli ka mitu, siis oli meil kaks 
modemiühendust. Ühel hetkel hakkas meie ja Fidoneti kaudu väljapoole 
ühenduma Läti.

\question{Ma teadsin, et Vene Fidonet käis läbi meie, aga et ka Läti?}

Venemaa tekkis ka jah millalgi. Läti oli Fidonetis Eesti all, aga leedukad loomulikult ei oleks millegi 
selliseni laskunud, et nad on mingi Eesti regioon kusagil 
võrgustruktuuris. Nemad selle asemel helistasid kord nädalas ja tõid e-posti enam-vähem 
nagu ämbriga, välja arvatud Kaunase 
Ülikool\index{Kaunase Ülikool} ja Leedu parlament\index{Leedu Seim}, kes olid 
Goodwin BBSi pointid. Seal oli hädasti vaja ja uhkus jäeti kõrvale. 

Lätlased käisid meil külas ka. Panid raha kokku ja tõid selle meile 
ühenduse eest. Investeerisime need kakssada dollarit kahte modemisse -- ostsime US Roboticsi\index{US 
Robotics} HSTd, mille kiirus oli vist neliteist kilobaiti. Väga väärt aparaadid, nii et lätlased panustasid Eesti neti 
arengusse.

Samal ajal ametlikku postivahetust internetiga pidas 
Küberi\index{Küberneetika Instituut} seltskond, aga neil oli Mustamäel suhteliselt sant jaam, mis krabises ilmselt rohkem, kui sidet läbi 
lasi. Ühtlasi olid akadeemilised tüübid millegipärast suured
UNIXi sõbrad ja kasutasid PEPi TrailBlazereid\index{Telebit TrailBlazer}, 
mis esiteks olid 9,6 kilobaiti ehk mõttetult aeglased ja teiseks suutsid HSTd 
postsovetlike liinidega paremini sidet vilistada. Olime 
sügavalt veendunud, et need olid ka oma reaalselt võimekuselt pikki seansse ja 
kiirust üleval pidada märksa paremad. 

\question{Kas sa mõtled \enquote{liini} all ikka telefoniliini?}

Jah, need olid tavalised analoogliinid, mille otsa käis kettaga telefon. 
Keskjaamas numbrit valides jooksid releed kontakte mööda ringi. Kui modem valis, oli kuulda klõbinat, kui see
releega katkestusi tekitas. Kõik oli elektriline, sellepärast ma kujutangi 
ette, kuidas andmeside tegelikult toimub. Aga kuidas on võimalik, et mingid 
vennad panevad läbi ADSLi kümme megabaiti? Meie panime enam-vähem samasugust asjast 
läbi neljateistkümne kilobaidi, nii et täiesti arusaamatu. Wifi täpselt samuti. Ma ei 
kujuta ette, kuidas see põhimõtteliselt saab üldse toimida. 

\question{Kas kujundamisega tegelesid kõige selle kõrval?}

See käis jah kõrval. Umbes 
samal ajajärgul sattusin seltskonda, kellel oli arvuti ja printer ning vajadus 
midagi trükkida. See oli poistekoor\index{RAMi poistekoor}, mida juhtis Venno 
Laul\index[ppl]{Laul, Venno}\sidenote{Venno Laul asutas 1971. aastal Riikliku 
Akadeemilise Meeskoori juurde poistekoori ning oli kuni 1990. aastani selle 
kunstiline juht ja peadirigent.}. Neil oli esimene PostScripti printer, mille 
ostus ma osalesin -- kas Tektronix või muu säärane A4-formaadis 
300 dpi laserprinter. Ühendasin Ventura selle külge.

Põhimõtteliselt kõik toonased kujundusprogrammid olid sellised, et arvutis olid 
\emph{bitmap}-fondid, mis saadeti juhet pidi printerisse, ja see kõik võttis kaua 
aega. Aga PostScriptiga sai lehekülje nagu programmi saata 
printerisse, mis siis oma tarkusega joonistas. See oli Adobe ja Apple'i 
vendade poolt väga mõistlik valik, kui nad kord Silicon Valleys kokku said ja 
otsustasid, kes mis osa maailmast vallutama hakkab. Tõepoolest, kontoris ei pruugi 
igal vennal printerit olla ja selleks, et inimesed saaksid printida, 
võiks olla printer ka tark. Väga spetsiifiliselt tark, et suudaks 
joonistada lehekülje endal mälus valmis ja siis välja trükkida. 

Millalgi samal ajal puutusin kokku ka Sirbiga\index{Sirp} (ma ei tea, ka 
see toona oli juba Sirp või veel Sirp ja Vasar), mis oli üks esimesi 
ajakirjandusväljaandeid, mis läks üle digitaalsele töövoole. Alguses oli protsess umbes 
selline, et toimus tinaladu, millega tehti kas siis üks tõmme paberi peale ja 
see vist pildistati üles. Nii et ofsettrükk toimus 
veel läbi tinalao. Ja nüüd, kui oli võimalik minna üle sellele, et arvutist 
saaks välja trükkida, siis see oli megaraju.

\question{Kas PostScripti printerist lasti trükitavad 
asjad kilele?}

Põhimõtteliselt jah, peegelpildis. Üks asi, mille koos 
Suloga\index[ppl]{Kallas, Sulo} Eesti 
Instituudis\index{Eesti Instituut}tegime, oli PostScripti \emph{pre-header}, mille nimi oli vist Preambul. 
PostScripti puhul tuli programmiga kaasa 
koodijupp, mis kirjeldas programmeerimiskeskkonna, defineeris 
täiendavad funktsioonid ja muud oma käsud. Seejärel tuli kood ise ja lõpus 
koristusfunktsioon või midagi sellist, mis välja trükkis. PostScript oli tore selles 
mõttes, et see oli nagu \emph{open source}. Mitte küll vabatarkvara, aga 
nähtava lähtekoodiga. Ehk oligi võimalik võtta sama Ventura ette, 
mis kusagilt laadis selle Preambuli, mis oli tekstifail ja mida oli võimalik 
muuta. Ette sai kirjutada transformatsiooni, mis 
keeras pildi peeglisse. Meil õnnestus see Sulo PostScripti preambul maha müüa kooliraamatute kirjastusele
Avita\index{Avita kirjastus}, mille eesotsas oli Tiit Aunaste\index[ppl]{Aunaste, 
Tiit}. Hiljem, vist 1991. aastal sattusin ise ka Avitasse tööle, asjad liikusid tollal
väga kiiresti.

\question{Jah, sest umbes viis aastat hiljem mäletan mina sind Eesti 
vaieldamatu autoriteedina teemal, kuidas arvutist värviline asi trükki saada.}

Eks ma olin seda piisavalt praktiseerinud. Tegime Ventura peal ka 
haltuuraotsasid. Liikus muudki tarkvara, näiteks Arts \& Letters, millega oli võimalik paigutada tähti ringikujuliselt. 
Toona asutasid kõik aktsiaseltse ja börse ning 
neil oli vaja pitsateid. See oli meeletu innovatsioon, et oli võimalik 
arvutist ühe matsuga pitsat valmis teha ja ei pidanudki 
kujundajatädi fotolao tähti välja lõikama ja kleepima. 

Sirbi toimetus andis välja Jutulehte\sidenote{Ilmus AS Kodamu väljaandel 
aastatel 1990--1992.}, mille \emph{layout}'i tegin mina. 

Nii ma tasapisi sattusingi selle ala peale. Käisin ka Helsingis vaatamas, 
kuidas Helsingin Sanomati\index{Helsingin Sanomat} tehakse. Neil olid 
miniarvutid ja rohelised terminalid ning 
Linotronic\sidenote{Mergenthaler Linotype Company toodetud 
kõrgekvaliteediline printer. Tegu oli kalli seadmega, mis võimaldas trükkida 
resolutsioonis kuni 2540 dpi.}, millega trükiti veerge 
fotopaberile. Fotopaber oli 30 cm lai ja sellele lasti välja üks 
ajaleheveerg. Siis lõigati veerud kääridega välja ja pandi suure 
maketi peale, mis oli vaha või millegi säärasega koos. Rastreeritud fotod 
pandi veergude vahele, leht sai kokku ja tulemus saadeti faksiga 
trükikotta. Faks ei olnud loomulikult tavaline faks, vaid 
\emph{industrial-grade} ajaleheformaadis kõrgeresolutsiooniline masin, mis 
skännis ühelt poolt sisse ja teiselt poolt trükkis välja 
filmi, millega valgustati trükiplaadid. 

\question{Mis selle juures köitis? Kas tehnoloogiline keerukus või 
see, et protsessil oli palju samme, või veel midagi?}

Kõige huvitavam on tegeleda asjadega, millega teised parasjagu ei tegele. Või 
ka asjadega, mis toimivad teistmoodi, kui olen 
siiamaani arvanud. Niisama Pascalis programmeerida ei olnud väga huvitav, seda õpetati 
koolis. Aga kuna mul oli ilmselt olemas arusaam, kuidas asjad töötavad 
ja mis masinates on, siis ma suutsin asju efektiivsemalt tööle panna. Näiteks kirjastuses siduda küljendus- ja 
tekstitöötlusprogramm. Tekstitöötluses oli levinud WordPerfect (Perfect? 
Prefect? Ford Prefect ja Word Perfect!\sidenote{Peeter viitab tõenäoliselt 
Douglas Adamsi loodud tegelaskujule, mitte omaaegsele populaarsele 
automargile.}), meie küll kasutasime rohkem Volkswriterit\sidenote{Volkswriter oli üks esimese PC-platvormi tekstitöötlusprogramme, 
mida arendati 1980ndatel. 
Volkswriter oli saadaval ka eespool mainitud GEM-platvormile, sellest ilmselt 
selle kasutamine kirjastustöös.}. Tekstitoimetaja toimetas 
WordPerfecti faili, kus olid stiilid juba ära märgendatud. Küljendaja luges
selle oma küljendusprogrammi tagasi ja teksti korrektuuri oli võimalik teha ilma kallima arvuti või 
keerulisema programmita. Seda õnnestuski meil väga efektiivselt juurutada. 

Enne rublaaja lõppu, 1992. aasta alguses, tekkis Prisma 
Printi\index{Prisma Print} esimene Linotronic ehk filmiprinter. Seniajani
peeti 600 dpi laserit väga heaks, nüüd tekkis järsku 1200 dpi 
filmiprinter. Prisma alumisel korrusel olid suured Crosfieldi\sidenote{Crosfield Electronics oli Briti firma, mille toodetud 
skännereid peetakse siiani ühtedeks paremateks, mis iial tehtud.} trummelskännerid, millega sai 
teha värvilahutusi, filmi peale, juba rastrisse. Ja kogu montaaž 
toimus endiselt nii, et tekstikile ja pildikile või film 
valgustati või lõigati füüsiliselt kuidagi kokku. 

Mul oli keskmisest parem ettekujutus, kuidas need süsteemid 
töötavad. Kui mina jõudsin oma failidega kohale, nägid enamasti kõik vaeva, kuidas QuarkXPressist midagi välja printida. Minul olid kaasas oma flopid ja hiljem 
magnetoptilised ja ma trügisin vahele, et 
minu omad vahepeal välja lastaks, kuna ma ei viitsinud teiste järel oodata. Ja lastigi, sest minu asjad käisid tõesti kiiresti läbi, kuna ma 
kujundust tehes kujutasin ette, kuidas see PostScriptiks läheb. Nii 
küljendusprogramm kui ka seesama Ventura või graafikaprogrammid, nagu 
Illustrator või Freehand, aitasid mul visuaalselt 
valmistada ette PostScripti. Teadsin, kuidas see koodis 
välja näeb, ning sain võtta faili ette ja näha, kus miski on. Tänu 
sellele teadsin ka, mis on printeri jaoks keerulised asjad, 
oskasin neid lihtsustada ja mitte liiga keerulisi asju kasutada, sest 
see prose, mis seal taga oli, oli suhteliselt vaene. Kui suudad 
tekitada olukorra, kus programmil on tsükkel tsüklis (tänapäeval tuleb 
sinna otsa veel SQLi päring), siis üldiselt on see asi suhteliselt ebapädev. 

Tänu arvutitaustale ja kujundamisele tekkis mul hulk disaineritest sõpru ja teisi sel alal tegutsejaid, keda ma 
Prisma Prindi väljatrükijärjekorrast teadsin. 

Edasi sattusin tööle Uniprinti. Algul töötasin andetu disainerina, aga 
siis leidsin tasapisi võimalusi vähem disaini nõudvaid asju 
teha, kus mängis rolli just see, et ma suutsin võtta ette näiteks Eesti Näituste 
näitusekataloogi andmebaasi (see oli vist Microsoft Accessis) ja 
genereerida väljundiks tekstifaili, mis oli stiilidega märgendatud ja mida 
oli võimalik küljendusse sisse lugeda. Jällegi asi, mida tollal ei olnud 
\emph{desktop publishing}'is tavaks kasutada: valmistasin 
stiilid ette ja tekst kasutas neid, kohapeal midagi tegema 
ei pidanud.

\question{Nii et põhimõtteliselt oli tegu CSSiga?}

Täpselt. Põhipõhimõtteliselt nagu CSS, ainult et tekst ja paber. Veeb töötab siiamaani niimoodi, aga see oli minu meetod, mis võimaldas teha huvitavaid 
töövoogusid. Minul oli andmebaas käes, näitusetüdrukud, kes müüsid bokse ja korraldasid üritusi, tõstsid asju 
ümber, täiendasid firmade andmeid ja parandasid telefoninumbreid. Mina trükkisin \emph{layout}'i välja, viisin neile 
ja nemad tegid andmebaasi korrektuuri, samal ajal kui mina joonistasin logosid 
puhtaks. Niimoodi ma õppisin. Nagu ma ütlesin, siis ma olen andetu 
disainer, aga tehniliste protsesside ja töökorralduse poolt 
teadsin tollal rohkem kui keegi teine. 

\question{See klapib kenasti sellega, mida sa praegu tundud tegevat. 
Millega sa üldse tegeled?}

Jah, seoseid on igasuguseid. Kui ma veel trükialal tegutsesin, oli mul palju vaba aega tänu sellele, et olin suutnud oma 
tööd optimeerida. Ja ka tänu sellele, et Uniprindi pealikud Sirje ja Andrus 
Reinsoo\index[ppl]{Reinsoo, Sirje}\index[ppl]{Reinsoo, Andrus}, kes on just 
mõlemad lahkunud\sidenote[][-2mm]{Intervjuu Peetriga leidis aset märtsis 2019.}, jätsid 
mulle piisavalt vabadust. Käisin ja kolasin Ameerikas 
konverentsidel. Sel ajal oli valdav suhtumine, et mis 
mõttes väljamaa ja konverentsid? Me oleme Eestist ja teame väga hästi. Mina 
käisin \emph{cyber publishing} seminaridel, mis olid 
seotud just trükipoolega, mis mulle huvi pakkus: plaaditrükk ja 
muu selline. Ja kuna ma olin põhimõtteliselt nagu ajakirjanik, siis oli
mul võimalik möllida ennast konverentsidele, mis muidu maksid 
paar tuhat taala (tolle aja mõistes röögatult palju), 
ajakirjaniku passiga sisse. 

Aastast 1994 hakkasin ka kirjutama. Paralleelklassivend Peeter-Eerik 
Ots\index[ppl]{Ots, Peeter-Eerik} oli Äripäevas ajakirjanik ja kirjutas 
tehnoloogiateemalisi lugusid. Mul reaalikana hakkas 
mõnevõrra piinlik, sest kirjutatu ei tundunud olema piisavalt pädev. Post-BBSi ajastu inimesena olin kindlasti ka võrdlemisi 
\emph{opinionated} noor inimene oma kindlate 
eelarvamustega. Kirjutasin teisele Peetrile paar lugu ette, et avalda parem 
neid, vähemasti kirjutatud kellegi poolt, kes enam-vähem saab aru, millega 
tegu. Peeter ütles, et võiks need ikka minu nime 
alt avaldada ja saaksin honorari ka. Nii sattusingi Äripäeva 
kirjutama, hiljem ka mujale, näiteks
Arvutimaailma\index{Arvutimaailm}. 

\question{Tehnoloogia tehnoloogiaks, aga mis sind kirjutamise juurde tõmbas?}

Kirjandite kirjutamisega sain suhteliselt hästi hakkama juba kooliajal. Minu 
esimene avalikustatud töö oli Pikri\index{Pikker} noorte 
huumorivõistluse võidutöö. Olin üht-teist ka lugenud, sõprus 
sõnaga oli olemas, lisaks olin teinud kooli omavalitsust ja muud 
sellist. Ilmselt olin parasjagu jutukas ka. Kirjutamine ei 
olnud keeruline ja võibolla meeldis mulle ka õpetada -- läbi kirjutamise on 
võimalik teisi õpetada ja panna midagi teisiti tegema. Olen Peeter-Eerik Otsale väga tänulik, et 
ta tegi midagi valesti. See on ka interneti puhul tüüpiline: 
\enquote{\emph{Wait, somebody is just wrong on the Internet!}}. Kirjutamine 
ilmselt sai alguse sellest, et \emph{somebody was wrong} ja mul oli vaja 
kaitsta oma seisukohta ning loomulikult ka Reaalkooli au. 

Sedasi see algas ja hiljem palusid ka teised mul kirjutada. 
Ma siiamaani ei oska ei öelda ja eks edevus mängis ka rolli. 
Pealegi oli see valdkond suuresti katmata. 1995. või 1996. aastal kutsus Avo Raup\index[ppl]{Raup, 
Avo} mind kui juba kirjutanud ja tuntud inimest Raadio 2 saatesse \enquote{Võrgutaja} külaliseks. Meil klappis nii hästi, et minust sai resident-saatekülaline. Esimene inimene, keda 
ma sattusin saates üksinda intervjueerima (Avo oli vist haige), oli Kaido 
Saarma\index[ppl]{Saarma, Kaido} Abobase Systemsist\index{Abobase Systems}. 

1999. aastal tuli minu juurde 
Sarvik\index[ppl]{Sarvik|see{Sarv, Henn}}\sidenote{Legendaarne IT-mees Henn 
Sarv\index[ppl]{Sarv, Henn}.} ja ütles, et Kukust kas Lang või Tiido oli öelnud, et on vaja teha arvutisaadet. Istusime sealsamas Uniprindi 
lähedal Pärnu maanteel Westmani poe vastas keldris Hollandi 
õlletoas ja mõtlesime välja saate Tehnokratt. Juba esimesel hooajal sattusime Kukus kokku 
tegelastega, kellel oli mõte ka ETVs\index{Eesti 
Rahvusringhääling!Eesti Televisioon} midagi sellist toota. \emph{Whatever}, toodame! Nii sattusingi 
telesaatesse, kus pidin olema korraga toimetaja ja saatejuht ning panema kokku ka montaažiriba (mis tuli muidugi mõnevõrra üllatusena).

\question{Ja nüüd oled ringiga tagasi\ldots} 

Kas nüüd tagasi või edasi, aga praegu olen Zone'is\index{Zone}, mis on täiesti 
juhuslikult ajaloos esimene kord, kui töötan mingit otsa pidi 
IT-firmas. Olen küll vahepeal olnud reklaamiagentuuris digitiimi juht, 
mis on ka natuke IT, aga ikkagi reklaamindus. Nii et olen töötanud trükinduses, teisi koolitanud ja kõike muud teinud, 
aga see on esimene kord, kui mingid IT-tüübid mõtlesid, et palkaks Marveti siia 
tuututama. Ametlikult on mu müts seotud turunduse ja kommunikatsiooniga, aga 
tegelen ka sellega, et kui keegi ütleb, et midagi ei tööta, ja kõik 
väidavad, et töötab ju, siis kuidas saada aru, mida inimene 
tegelikult tahab. Äkki tal on õigus, et tal ei tööta. Äkki on võimalik, 
et see asi, mida meie oleme nunnutanud ja silunud ja teinud maailma kõige 
paremaks, tema kontekstis ei tööta. Ja täiesti üllatavalt selgub, 
et kui on piisavalt keerulised süsteemid, siis olukordi, kus tuleks 
kõige suurematele ja parematele püüdlustele vaatamata midagi 
teisiti toimima panna, on uskumatult 
palju. 

\question{Küll sa turunduse ka ära optimeerid, nagu sa kõik asjad 
ära oled optimeerinud!}

Jah, ma üritan. Mul see lootus on natuke teistpidine. 

Kunagi tuli Andres Kulli\index[ppl]{Kull, Andres} ja Kroonpressi\index{Kroonpress} 
seltskond küsima, kuidas panna reklaami ajalehte. Mina rääkisin, et on 
olemas PDF. Teeme parem nii, et kõik teeksid korraliku PDFi, leheküljendaja 
tõstab selle küljendussofti sisse ja kõik töötab. Kull, ikkagi
suure trükikoja juht, ütles seepeale: \enquote{Väga hea, nii teemegi. 
Kõik peavad saatma oma asjad PDFina Postimehesse}. Ja üllatus-üllatus, nii läkski. 

Mu enda roll selle kõige juures oli, et olin olnud pikka aega Prisma Prindis ja 
muudes reprodes selline majasõber, kes sageli tolknes seal ja üritas endale 
tegevust leida ning saada aru, kuidas asjad käivad. Näiteks võtsime
Eesti esimese Linotronicu pulkadeks lahti ja jootsime seal midagi, sest masin otsustas töö lõpetada parasjagu, kui oli vaja midagi välja 
lasta. 

Teadsin, millist roppu vaeva kõik mu repropealikest või -tehnikutest sõbrad olid näinud kehvasti ette valmistatud 
originaalidega. Kui PDFindus hakkas meile endale majja tulema, siis 
mõtlesin, et mina küll ei hakka selle ussipurgi avamist enda peale võtma
(tänapäeval räägitakse rohkem \emph{surströmming}'ust kui Pandora laekast). Ainuke 
asi, mida ma saan teha, on õpetada kliendid paremaid originaale saatma, mis 
loomulikult tundus äärmiselt lihtne:
ütlen neile, et seal on vaja mõned linnukesed panna ja siis kõik 
lähebki nii, nagu vaja. Aga tuleb välja, et ei. Olen õppinud, et päris 
kõva pingutus on aru saada, mida teised inimesed teavad, ja 
panna nemadki aru saama millestki, millest mina aru saan, 
seejuures ise liigselt masendumata või 
nende peale kurjaks saamata. Nii sattusingi õpetama Pagemakerit, 
InDesigni, Photoshopi ja muud säärast just töökorralduse poole 
pealt. Hetkel Zone'is näen ma, et kui vaadata kogu veebiga 
seonduvat, siis ilmselt tuleb proovida selle kõigega veel rohkem edasi minna. 