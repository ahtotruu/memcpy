\index[ppl]{Veldre, Anto}

\question{Kuidas sina arvutite juurde 
said?}

Minu ema töötas ülikooli arvutuskeskuses\index{Tartu Ülikool!Arvutuskeskus}.

\question{Millise ülikooli?}

Tartu Ülikooli. Eestis ei ole rohkem ülikoole, ülejäänud on 
\enquote{tech}'id. Ema õppis ülikoolis matemaatikat ja 1959. aastast läks 
arvutuskeskusse. Ma ei tea, kas ta töötas seal juba põhikohaga või 
katsetas niisama, aga igatahes tegeles ta Ural-1\index{Ural!Ural-1} juures 
programmeerimisega. 

\question{Kas arvutuskeskus asus Liivi tänaval?}
Ei, ma arvan, et see oli ülikooli kõrval majas, kus praegu on biofüüsika ja 
kohvik. Ma usun, et see oli 1959. aastal seal. Ma ei tea, mida nad mu 
isaga vahepeal tegid, abiellusid ja midagi veel, aga 1961. aasta augustis 
sündisin mina. 

Minust on esimene pilt, mida näinud olen, arvuti taustal 1962. aasta 
jõuludest. Õigemini nääridest. Usun, et pildil oli Ural-1, mitte Ural-2. 
Ühesõnaga mingi imelik aparaat, mille taustal mind näidati ja mida ma 
tegelikult ise ei mäleta. 

Arvutuskeskus kolis mitu korda: küll Gagarinisse\sidenote{Praegu Jaan Tõnissoni tänav.}, siis Burdenkosse, mis 
on praegu Aia tänav\sidenote{Siiski Veski.}. Erinevad osad olid laiali. Kui hakkasin 
juba teadlikult masinatest aru saama, oli arvutuskeskus seal, kus praegu on 
punane korporatsioonihoone\sidenote{Eesti Üliõpilaste Seltsi maja 
Jaan Tõnissoni 1.}. Seal majas oli Ural-4\index{Ural!Ural-4}, aasta pidi olema umbes 1973\sidenote{Ürikute järgi kolis Tartu Ülikooli arvutuskeskus Liivi tänava 
hoonesse 1972. aastal.}. 

\question{Kas su vanemad olid programmeerijad? Mida nad tegid Uraliga?}

Ema oli arvutikeskuses põhikohaga matemaatik. Kogu arvutuskeskus oli täis
kõrgema haridusega naisterahvaid, kes puistasid varrukast 
korrelatsioonimaatrikseid ning arvutasid lehmade boniteeti ja teab mis 
jubedusi veel. Isa oli bioloog, 
ta muidu töötas zooloogiamuuseumis, aga käis hobiprogrammeerijana haltuurat tegemas. Profid olla teda vihanud, sest ta leidis alati 
mingisuguse lokaalse optimumi. Näiteks kõvaketta \emph{interleaving}'ut siis 
veel ei tuntud, aga oli magnettrummel, mille peal Ural-4 oma mälu pidas. 
Vanamees arvutas välja trumli pöörlemiskiiruse ja hakkas oma 
progesid tegema niimoodi, et seni, kuni tema muid asja tegi, jõudis trummel sama koha peale tagasi. Ja kuigi tema programm nii-öelda 
struktuurilt ei kõlvanud kodulooma istmikku ka, siis töökiiruselt oli vist 
13 korda kiirem kui profiprogejate oma.

Nii et minu lapsepõlv oli huvitav. Isal oli kapis kaks ülikonda. 
Üks, natuke kehvem, oli Vanemuises käimiseks -- teatris käidi tollal ülikonnas, 
mitte nagu praegu. Teine, natuke parem, oli öösel \enquote{Masinasse 
minekuks}, kusjuures Masin kirjutati suure tähega ja see oligi 
Ural-4\index{Ural!Ural-4}. Mind sinna öösiti ei lastud, aga päeval ma seal 
ikka kooserdasin. 

\question{Millal sai hakkasid seal teadlikumalt käima? 
Kas juba põhikooli ajal?} 

Iseasi, kui teadlik see oli. Vene ajal oli ju oluline 
päritolu. Kui oled see neetud intelligent ja sellest kihist pärit, 
siis pead tegelema mille kõigega. Ma käisin muusikakoolis, noorte trummarite 
ringis ja oi jumal, kus kõik veel! Ma ei enam väga täpselt ei mäleta, aga 
ilmselt 1973. aasta lõpus, kui arvutuskeskust juhatas vist Tapfer\sidenote{Jüri 
Tapfer\index[ppl]{Tapfer, Jüri} oli Tartu Ülikooli Arvutuskeskuse juhataja 
aastatel 1971--1995.}, kutsuti mind juhataja kabinetti ja anti pidulikult kätte kasutajatunnus, vist
viiekohaline number. Masinaga polnud sellel midagi pistmist, 
see oli aruandluse jaoks. Pidin paksu žurnaali allkirja 
andma nagu nõukaajal ikka. Ja see tähendas, et kui masinas 
vaba hetk oli, siis tehniliselt oli lubatud ka minu programmi sealt läbi 
jooksutada. Ega ma väga edukas ei olnud, paar programmi hakkasid
enam-vähem tööle. Eks isa aitas natukene siluda.

Urali\index{Ural} küljes oli niisugune 
ese nagu laitrükkal, millel oli minu arust 128 märki reas 
ja sealt tuli paberit päris koleda kiirusega. Ühesõnaga, kiirkirjutusmasin. Umbes nelja-viieaastaselt ema töö
juures käies istusin selle laiprinteri peal, sest see kurinahk oli soe! 
Istusin seal, kõlgutasin jalgu ja vaatasin, mida operaator 
eespool teeb. Nad kodeerisid seal tähtede 
trükitihedusega Mona Lisasid ja muid pilte, kusjuures Mona Lisad 
olid suhteliselt alasti. Mind see ei häirinud, aga mäletan, et
neid hoiti nurga taga. Tehti ka muidugi Leninist ja millest kõigest veel. Üritasin 
ka mingit pilti teha, millel olid loomulikult igasugused vead sees, aga ma ei 
mäleta, kas sain selle lõpuni tehtud või mitte. 

\question{See ju tähendab, et pidid kuskilt programmeerimist õppima. 
Või korjasid selle lihtsalt õhust üles?}

Asjalikke õpikuid ei olnud. Urali\index{Ural} kohta midagi oli, aga 
see ei olnud isegi assembler, puhas masinkood. Käsukoodid: 
null-üks oli liitmine, null-kaks lahutamine ja siis 
sinna midagi järele. Aga seal olid muidugi trikid nagu assembleriski ja nende 
selgekssaamine oli ainult läbi vaeva. 

Asi algas sellest, et tegid lihtsama joonise valmis, mis ei olnud 
muidugi õige. See tuli saada perfokaardi või perfolindi peale, kaks 
võimalikku sisendit oli ja millegipärast oli lihtsam perfokaardiga. Siis tuli 
tagaruumis mõnd telegrafistitädi painata, kes käisid Kesktelegraafist 
lisatööd tegemas. Nad tagusid sisse ma ei tea mitu 
märki sekundis. See üks või poolteist perfokaarti perforeeriti ära, seda ei 
olnud palju. Programm tuli alguses ju kirjutada 
rohelise värviga trükitud plankidele, kus olid operandid ja kommentaarid, 
panna sinna oma kasutajatunnus ja mida kõike veel. 

Kui programm sai 
perfokaardile, siis tuli see viia
masinasaali ukse juurde lahterdatud kasti, nagu on pioneerilaagris 
hambaharjade jaoks. Perfokaart käis tühja lahtrisse ja kui masinal kas mõni 
perifeeriaseade ei töötanud, tähtsamaid programme ei saanud teha või 
operaatoril oli öösel igav, võttis ta need naljaasjad ja lasi läbi kuni 
esimese veani. Siis kirjutas perfokaardile jõleda käekirjaga mingi jõleda 
kommentaari ja võibolla pani väljatrükitud paberi ka sinna. 

\question{Kogu selle vaeva läbimiseks pidi sul ju mingisugune põhjus olema.}

Mina ei tea. Miks mõned poisid käivad näiteks jalgpallis käivad? Ma ei usu, 
et sellele on ratsionaalset selgitust, ma lihtsalt kasvasin selle asutuse seinte vahel 
üles. Muidugi sai seal ka lollusi tehtud. Aia tänaval oli 
vahepeal üks ruum, kus oli paarkümmend tädi Robotroni ja Rheinmetalli 
mehaaniliste arvutitega. Mulle meeldis neid jagamistehtega kinni lasta, aga 
pärast tuli mehaanik kutsuda ja siis sain sõimata. 

\question{Kuidas mehaanilist asja kinni jooksutada?} 

Jagamistehe ei lõpe kunagi. See üritab kelguga kogu aeg edasi jagada, kuni 
kelk jookseb ühele poole kinni ja midagi läheb nässu.

\question{Jagamistehe lõpeb ju millalgi ära!?}

Ei lõpe, nulliga jagamine näiteks ei lõpe kunagi. 

\question{Miks sa panid mehaanilise arvuti nulliga jagama?}

Põnev oli lolli masinat kinni jooksutada. Kes piinab kasse, 
kes paneb bensiinitünni põlema ja kes laseb Rheinmetalli kokku.

\question{Rheinmentall muidugi ei kiunu \ldots} 

Ei, see ragises ja logises, kuna see ei olnud ettenähtud olukord. Minu arvates oli 
nalja kõvasti. Tädide arvates mitte. 

\question{Kas mehaanik hammustas läbi, mis juhtunud oli?}

Muidugi, ja ma sain sõimata. Ega ma ainuke olnud, lapsi oli seal
teisigi.

Ühte programmi tegin veel. Üks insener 
tekitas Urali\index{Ural} külge heli{\-}generaatori. Hiljem kuulsin, et 
see oli praktiliselt igal tolleaegsel arvutil -- Covox Speech 
Thingi\sidenote{Lihtne väline audioseade, mis võimaldas arvutil läbi paralleelpordi 
heli väljastada, koosnedes hulgast takistitest, mis 
moodustasid primitiivse digitaal-analoogmuunduri. Koolipoisid jootsid neid
üheksakümnendatel ise kokku ja pusisid neile ka sobilikud draiverid. Näiteks üks suursaavutus oli \enquote{NetHacki}\index{Nethack} paaritamine \enquote{Warcraft II} 
heliklippidega. Tekstipõhise mängu ekraanipuhvrist loeti kindlast kohast tekst, 
sõnadega olid vastavusse pandud helifailid ja need mängiti Covoxi abil maha.} 
eellane. Andsin arvutile lolle käske 
mõttetute argumentidega, käsukood loeti välja ja kui see oli näiteks 01, tehti 
madalat häält, 02 oli juba natuke kõrgem hääl ja niimoodi sai laulukesi 
teha. Kuna ma käisin muusikakoolis ka, siis üritasingi programmi teha. 
Seegi ei saanud kunagi valmis, alati oli viga sees. 

\question{Midagi see ju ometigi tegi, vähemalt piiksus?}

Muidugi, lihtsalt mõni noot oli vale. Ega see ei olnud 
Sibelius, Cubase või Fruity Loops, millega kuuled kohe! Oi ei -- operaatoriga pidi kokku leppima, et lähed ja kuulad. 
Selline raske elu oli arvutuskeskusse sündinud lapsel. 

\question{Kui vana sa olid, kui neid programme tegid?}

Ilmselt 12--13, maksimum 
14. Kui olin 13, sai Liivi\index{Tartu Ülikool!Liivi 
õppehoone} tänava arvutuskeskus lõpuks valmis. Uued masinad koliti üle 
ja Ural\index{Ural} visati üldse välja. Igatahes 1975. 
aastaks oli see kõik pidulikult läbi. 

\question{Mis Uralist sai? Kas läks lihtsalt utiili?}

Paraku jah. 

Algul läks Ural-1\index{Ural!Ural-1} 
Nõosse\sidenote{1965. aastal, sellest sai alguse Nõo Keskkooli\index{Nõo 
Keskkool} arvutiõpe.} ja siis ka Ural-2\index{Ural!Ural-2}. Selle
plokke vedeles veel Tartu vahel, kui Ural-4\index{Ural!Ural-4} töötas. Jõe 
ääres, keskmise silla juures, oli füüsikamaja, kust sai onude 
käest plokke kaubelda. Nii et olen ka trigeriplokke 
näinud, 6N9S või 6N8S lambi peal. 

\question{Mida sa edasi tegid? Sebisid end Liivi tänavale?}

Ei. Ma olin juba siis kuulus isemõtleja, aga Nõukogude Liidus 
isemõtlemist ei sallitud, nii et selle etapi võib arvutite 
koha pealt vahele jätta. 

Nõukaajal elasin, nagu suutsin. Lihtne ei 
olnud, ülikooli ei lastud ja mõned muud probleemid olid veel. Raha
teenisin aparaatide parandamisega. Tulin 16aastaselt 
Tallinnasse ja läksin polütehnikumi\index{Tallinna Polütehnikum} 
raadiotehnikat õppima. Sealt tuli kirg tinutuskolvi vastu, nii et vahepeal ei tegelenud ma arvutitega väga 
pikalt ja lõbustasin ennast 
elektroonikaga. Segastel aegadel tagas see
äraelamise. Kõik meenutavad, kui raske oli murdehetkedel, kui 
poes ei olnud midagi. Mina seda ei mäleta selles mõttes, et 
mul olid härrased vorstiga ukse taga, sest hommikul kell 
kuus algas jalgpalli-MM ja telekas oli katki. 

\question{See tähendab, et sa pidid kuskilt teadmisi üles korjama? Polütehnikumist?} 

See oli veel üks hobi. Ural-4\index{Ural!Ural-4} 
taga oli ju ka toatäis insenere ja ostsillograaf 
oli kogu aeg arvutil ligi. Ma ei oska öelda, kust ma selle täpselt üles 
korjasin. Kusagilt sealt. 

\question{Paljud räägivad, et süsteemset õpet 
on vähe olnud, aga kuskilt teadmine tuli.}

Nõukaajal algas süsteemne ju sellest, et pidid olema kodumaale lojaalne ja 
igatpidi väga standardne ja siilisoenguga. Siis sind võibolla lasti kuhugi 
õppima ja lõpetasid kuskil salajase töö instituudis. See oli ametlik \emph{track}. 

Kui ma poisikesena Tallinnasse tulin, siis sai just Küberneetika Instituudi\index{Küberneetika Instituut} maja valmis. Arvutuskeskus valmis esimesena, see oli kõige kallim. Käisin seal
mikroskeemide käsiraamatuid nuiamas. Läksin tagauksest, \emph{social 
engineer}'isin ennast Kevin Mitnicku moodi sisse\sidenote{Kevin Mitnick oli aegade alguses tuntud häkker ja tegutseb praegu turvakonsultandina. Tema ja tema juhitud seltskonna oluliseks häkkimisvahendiks oli seadmetele ligipääsu hankimine lihtsalt inimestega suhtlemise või siis ka näiteks prügikastidest ära visatud manuaalide otsimise teel.} ja 
seletasin, kuidas mul on tähtis konstruktsioon pooleli, aga ainult 
kahe mikroskeemi parameetrid on veel puudu. Sain salajase 
käsiraamatu nii-öelda kohapeal kasutamiseks kätte. Nii see asi käis. 

\question{Kas Küberneetika Instituudis olid need raamatud olemas?} 

Jah, see oli üks koht, kust neid sai. Selliseid kohti oli Tallinnas veel, 
näiteks sõjatehased. 

\question{Nii et sa olid põhimõtteliselt vabakutseline, vaba mees?} 

Ei, käisin siis tehnikumis ja õppisin raadiotehnikat. 

\question{Millal moodsad arvutid sinu juurde jõudsid?}

Sinna vahepeale jääb veel segane aeg, kui üritasin 
Vene arvuteid parandada, näiteks Iskra-555\index{Iskra!Iskra-555}. Pärast 
tehti ka Iskraid Intel 8086 kloonide peale, aga Iskra-555 oli ise 
leiutatud magnetkaardi pealt töötav raamatupidamisarvuti. Olen remontinud 
ka suuri mehaanilisi Robotroni raamatupidamisarvuteid, mis on 
Rheinmetalli moodi. Aga kui sa nii-öelda ametlik mehaanik ei ole ja 
sul ei ole kogu dokumentatsiooni, on see õnnemäng. Kuna mehaanikuid telliti 
Moskvast ja ma ei tea kust, siis aeg-ajalt lasti mind ligi. Kogemuse sain, aga 
head mälestust ei ole. 

Vist 1989. aastal sattus vend CeBITile. 

\question{1989 oli ju veel nõukogude aeg!}

Oli jah nõukaaja lõpp, aga ma aastaarve täpselt ei mäleta. Igatahes pani ta kõik oma elusäästud kokku 
ja tõi sealt valge portatiivse Taiwani läpaka Bondwell, millel oli kaks 730 flopit. Vend tegi sellega tööd, aga aeg-ajalt sain seda näppida ja imelikke asju teha. Sellega sai isegi Turbo 
Pascalit\index{Turbo Pascal} käivitada, aga selleks oli vaja 
kahte flopit. Ühe peale ei mahtunud ära. 

Ega must väga Pascali\index{Pascal} progejat polnud, tegin paar näidet 
ja keegi teine silus need ära ning andis tulemuse. Idee oli 
õige, aga näpud lühikesed, sest ma ei olnud seda õppinud. Ma ei ole kunagi
progemises kõva olnud.

1993. aasta alguses, kui oli selge, et nüüd on Eesti riik, ja vahepealne segane periood sai otsa, sattusin tööle 
õpetajaks Tallinna 
43. kooli\index{Tallinna 43. Keskkool}, praegusesse 
tehnikagümnaasiumi\index{Tallinna Tehnikagümnaasium}\phantomsection\label{sisu:43kool}. 
Selles majas oli kaks juriidilist asutust: kool ja 
neljandal korrusel kadunud Ants Reili\index[ppl]{Reili, Ants} 
tehtud ETEK ehk Eesti teaduslik-tehniline 
ettevalmistuskeskus, nagu see nõukaajal käis. 

Reili Ants oli 
kihvt vanamees, õpetas tööõpetust ja tal olid telekas 
tööõpetuse saated. Ta tagus kuskilt välja mingi eksperimentaalse raha 
ja tegi kooli neljandale korrusele keskuse, võttis TPIst vanad elektroonikud tööle ning saavutas selle, et 
kooliõpilastele hakati seal tehnilisi aineid andma. Seal olid arvutid, näiteks vana 
Elektronika\index{Elektronika}, ja isegi mehed, kes neid parandada oskasid, mis oli tollal 
täiesti kriitilise tähtsusega. Aga need vanad mehed ei saanud lastega suurt hakkama ja 
mina olin nii-öelda päästerõngas, kes pidi hakkama tunde andma. 

\question{Kuidas sa sinna sattusid? Kas tutvuste kaudu?}

Ema töötas seal kunagi psühholoogina, see on keeruline lugu. Selle taga on 
tegelikult Keevalliku\sidenote{Andres 
Keevallik\index[ppl]{Keevallik, Andres}, Tallinna Tehnikaülikooli rektor 
aastatel 2000--2005 ja 2010--2015.} pärastine stiil, miks TPIst nii palju välja 
langetakse. Poisid lähevad kooli, et oma eriala kätte saada, aga elama ei õpeta neid keegi -- kuidas õlut korralikult 
juua ja õhtul klubis käia. Lisaks tuleb tööandja, kes võtab esimeselt 
kursuselt juba inimesi ära, ja kui kogu seda asja kokku miksida, siis kukutaksegi välja. 
Enne Keevallikut oli Ants Reili\index[ppl]{Reili, Ants} üks selline, kes 
sai sellest põhimõttest aru ja saavutas selle, et 43. kooli\index{Tallinna 
43. Keskkool} lõpueksamid olid põhimõtteliselt ühtlasi 
tehnikaülikooli\index{Tallinna Tehnikaülikool} sisseastumiseksamid. 

Need, kes tehnika eriala valivad, ei vali seda 
mitte sellepärast, et nad on lollid ja jobud. Tark inimene läheb ju arstiks ja 
advokaadiks. Tegelik põhjus on verbaalne võimekus -- 
kui sa ei suuda seletada kiiresti ja korralikult, mida sa tahad, siis 
arvatakse, et ah, mingi pagana tehnikanohik. Reili aga
ajas sinna kooli kokku inimesi, kes nende poistega kolme aasta vältel 
tegelesid ja õpetasid neid oma mõtteid inimese moodi väljendama. Ja minu ema sattus ühel hetkel sinna psühholoogiks. 

\question{Ta oli ju programmeerija?}

Ta tegeles vahepeal kutsehariduse testidega. Aga 
see kukkus kuidagi nii välja, et mina soovitasin alguses 
Reilile\index[ppl]{Reili, Ants} oma ema ja ema soovitas pärast mind -- 
ühesõnaga see juhtus kuidagi rekursiivselt tutvuste kaudu. 

1993. aasta jaanuaris olin ma seal igatahes paigas ja mulle öeldi, et neljandast 
veerandist pean hakkama juba kellelegi midagi 
õpetama. See oli teisejärguline, kellele ja mida,
projekti eesmärgid pidid olema täidetud.

Põhimõtteliselt mind pandi olukorda, kus oli mingisugune Unix. Mu käest küsiti: \enquote{Tead, mis Unix on?} Ma ütlesin: \enquote{Jaa, ma olen 
vähemalt ühte raamatut lugenud ja umbes saan aru.} Kaks 
kuud läks selleks, et ise aru saada, mis asi see ikkagi on. Seejärel aeti lapsed 
ette ja tuli neile õpetada. Tegu oli SCO 3.2.2\index{SCO UNIX} masinaga, kuhu oli 
Tõraverest kaks Urania\index{Urania} muxi kaarti ostetud, nii et 
sellele sai kaks korda kaheksa terminali taha võtta, 
pluss oma konsool. 

\question{Mis masin see oli?}

386 -- 8 mega mälu, 40 mega ketast ja SCO Unix 3.2.2\index{SCO UNIX}. 

\question{Huvitav kombinatsioon! Kas selle organiseeris seesama koolidirektor?}

Ants Reili\index[ppl]{Reili, Ants} ja tema sõbrad ja sugulased. Kooli direktor 
oli proua Errit\index[ppl]{Errit, Anneli}, tema õpetas vene keelt. 

Eks see oli nii-öelda mentaalne ülekanne vanast \emph{mainframe}'i ajastust. Need mehed mõtlesid lihtsalt niimoodi, sest nad olid sellega üles 
kasvanud. 

\question{Kas 386 vedas tõesti kuutteist terminali?}

Oi, väga hästi! Ega siis ei progetud nagu praegu, et \emph{include}'itakse 
kogu eelnev maailm. Siis kirjutati asju asmis ja korralikult. 

Mulle anti kaks seltskonda: kaheteistkümnendikud ja viiendikud. 
Kaheteistkümnendikega oli veel nii ja naa, aga mida sa nendele viiendikele seletad 
aastal 1993? Aga, nagu öeldud, 
seal olid vanakooli mehed ja süsteemiülem
Sven Turnau\index[ppl]{Turnau, Sven} proges ANSI Cs 
nagu issand jumal ning tegi mulle paar abivahendit. Üks oli mäng, millega 
sai terminali ekraanile tärne joonistada, nagu ma isegi kunagi arvutuskeskuses 
tegin, nii et see oli tuttav asi. Ja see proge töötas, ei kiilunud kinni, lapsed said aru ja pärast sai printerist välja lasta. 

TPI laost saime kilode kaupa vana murdekohtadega paberit, selle 
eest maksma ei pidanud. Printeri lindi eest küll pidi, aga Reili eelarve elas 
selle kuidagi üle, nii et lastel oli praktiline väljund. 
Joonistasid oma jubeduse ekraani peal valmis ja trükkis välja -- ühe tunniga 
tehtud. Teine põnev asi oli see, et SCO Unixil\index{SCO UNIX} oli meil
sees ja üksteise masinate piires sai kirju saata. Mari sai Jürile 
teatada, mida ta temast arvab ja tema emast ja nii edasi, ja seda nad ka väga 
aktiivselt tegid. 

Järgmisel õppeaastal see kõik jätkus. Olukord läks huvitavaks 
aprillis, kui mind veeti Nõkku\index{Nõo Keskkool} Unixi 
koolitusele -- juhuks kui ma veel millestki aru ei saanud, siis et nüüd ikka 
tõesti ise ka aduksin, mis see on. Koolituse korraldas observatooriumi\index{Tõravere 
Observatoorium} all tegutsev Urania\index{Urania}-nimeline firma, Margus Liiv\index[ppl]{Liiv, Margus}, Kaiti 
Kattai\index[ppl]{Kattai, Kaiti} ja kes nad seal olid. Muide, sellest päevast, kas 2. või 5. aprillist, 
hakkab mu digiarhiiv peale.  

Pärast Urania koolitust käisime Nõost\index{Nõo Keskkool} 
ka läbi, et vaadata, mis seal koolis tehakse. Nõos oli 
selline härrasmees nagu Kill Kask\index[ppl]{Kask, Kalju}\index[ppl]{Kask, 
Kill|see{Kask, Kalju}}, kes rääkis: \enquote{Ah, me ripume interneti küljes, vahetame kirju ja 
trillallaa-trullallaa.} Neil oli laual mingisugune karbike ja ma küsisin, mis see on. 
\enquote{See on modem!} \enquote{Ahah.} Rohkem ma ei julgenud küsida. Kui 
modem, siis modem. 

Netti ei olnud ja ma ei tea, kuidas ma selle 
selgeks tegin, aga mõne päevaga oli kontekst selge, mis aparaat see modem on 
ja mida sellega saab teha. Kirjutasin Avatud Eesti Fondi\index{Avatud Eesti Fond} 
taotluse, et tahan ka seda saada. Kool 
saigi raha, nii et 1993. aasta sügisel saime modemi. 
SCO Unix\index{SCO UNIX} võttis selle ilusasti taha ja selgus, et meil 
on relv! Kuidas normaalsetes koolides tollal meili saatmine käis?
Näiteks Tartu Treffner\index{Hugo Treffneri Gümnaasium}, kus sai ka külas 
käidud, oli väga hästi varustatud kool. Neil oli palju arvuteid, üle kümne, aga modem 
oli taga ainult ühel ja selle arvuti taga oli järjekord. Üks õpilane toksis
kahe näpuga oma kirja, saatis minema ja teised ootasid järjekorras.

\question{Sul olid ju terminalid, see on arhitektuurselt palju parem!} 

Absoluutselt! Ja see oli relv. 

Anne Villems\index[ppl]{Villems, Anne} korraldas Tartus
õpilastele mänge, et nad internetiga ära harjuksid. Näiteks üks oli \enquote{Gaia}\sidenote{Gaia käivitus 1994. õppeaasta alguses ja sellest 
võttis osa 20 gruppi 17 keskkoolist või gümnaasiumist.}, kus olid väljamõeldud riigid. Meie kooli nimetati Barbariaks, 
väga õige nimetus!

Nagu öeldud, meil ei pidanud keegi järjekorras seisma, 17 inimest (tegelikult 16, aga 
kui Turnaul\index[ppl]{Turnau, Sven} oli hea tuju, lubati keegi konsooli taha 
ka) võisid korraga oma kirju trükkida. Masin korjas need kokku ja saatis ära
siis, kui oli aega. Aga sellega asi ei piirdunud -- SCO 
Unixis\index{SCO UNIX} sai modemi sissehelistamise peale häälestada. Saime Eesti Energiast\index{Eesti 
Energia} paar vana 1200boodist modemit (ilma veakorrektsioonita, 
ürgaegne värk), Soome tähtedega terminale ja muid
koledaid asju. Soome tähed ju muudavad ära ASCII lõpu, kus on nurksulud, sinna 
tulevad nende ö-d ja ü-d. Aga selle tulemusena õnnestus aktiivile, 
kolmele-neljale kõige aktiivsemalt arvutiklassis käivale poisile saada koju 
modemid ja terminalid, sest kes see kannatas siis endale arvutit osta. Siis
läks asi hulluks kätte, sest poiss logis ennast öösel kooli SCO 
Unixisse ja trükkis kirja valmis. Seepeale ütles Anne Villems\index[ppl]{Villems, 
Anne}, et oleme tehnoloogiliselt teistest nii palju üle, et 
see ei ole enam aus. 

\question{Huvitav on see, et sa ei olnud üle mitte tehnoloogiliselt, ägedama arvutiga, vaid just \emph{setup} oli äge!} 

Jah, organisatoorne pool, sest need vana kooli mehed teadsid, kuidas 
seda püsti panna, ja see sissehelistamine oli väärtuslik. Näiteks Microsofti 
masinatega ei olnud kellelgi mingit sissehelistamist. Ja kui poisil öösel 
kell kolm und ei olnud, aga tuli inspiratsioon peale ja ta tahtis 
\enquote{Gaia} mängus kaasvõitlejad teise planeedi peal saata, siis tal oli selleks 
täielik tehniline võimalus. 

\question{Legend räägib, et umbes sel ajal tehti Eestis esimesed veebmasterite 
kursused. Sina olla seal ka tembutanud.} 

Ei mäleta, aga sealt hakkas üks teine rida. Nimelt sain kiiresti aru, et see 2400ne 
modem on nabanöör. Selle asemel et öösel koju magama minna (mul 
ei olnud kodus siis veel terminali), istusin 1993. aastal ja 1994. aasta alguses 
öösel koolis ja rippusin horos.kbfi.ee\index{horos.kbfi.ee} küljes, kuna sinna sai 
sisse logida. Tallinna välisühendus oli tollal 64 kilobitti sekundis. 
Kuna normaalsed inimesed öösel magasid, siis viis ja pool 
kilobaiti sekundis oli maksimumkiirus, mille kätte sain. See masin 
tõmbas muidu uudiseid, \emph{news group}'e, aga mina sain mööda netti ringi 
kolistada ja terminali ekraaniga asju tõmmata. Öö jooksul suutsin tavaliselt 
umbes viis flopitäit kohale tõmmata. Kui järgmine päev tunde ei olnud, magasin 
välja ja läksin KBFIs\index{KBFI} Andres Baumani\index[ppl]{Bauman, Andres} juurde 
ning anusin, kas saaks tema masinast asjad flopile ära kopeerida. Teine variant oli järgmisel ööl oma 
modemiga tõmmata. 

\question{Mida sa tõmbasid?}

Tollal oli väga palju häid materjale liikvel. Ülikoolidel olid Gopheri ja FTP-saidid ning veeb just 
hakkas tulema. Minu lemmikmeetod oli see, et läksin 
mõnda uudisgruppi, nagu näiteks alt.sex, kus inimesed ikka 
käisid, ja siis mingi progejupiga, mida 
Turnau\index[ppl]{Turnau, Sven} mulle õpetas, otsisin välja ülikoolide 
aadressid. \emph{Strip}'isin nimed eest ära ja läksin käsitsi ülikooli 
FTP-serverisse kolama. Kuna tol ajal andmekaitset ei 
olnud, siis olid absoluutselt kõik asjad ripakil. 

\question{Ütlesid \enquote{anonymous} ja 
\enquote{ftp}\sidenote{Levinud ja kasutajate hulgas hästi teada viis 
avalikke FTP-teenuseid pakkuda oli kasutaja \enquote{anonymous} puhul aktsepteerida kas 
parooli \enquote{ftp} või ka mis iganes sisendit.} ning saidki sisse?}

Ja, aga tollal USAkad ei osanud seda veel karta, nii et 
harvesteerisin vastuoluliste nimedega gruppidest, sest need olid kõige 
suuremad. Sealt sain maksimumkoguse ülikoolide nimesid ja nende järgi tõin FTP-saidist ära kõik, mis tundus lugemisväärne.

\question{Just lugemismaterjali, mitte programme?}

Nii palju kui mul üldse mingisugust krüptoteadmist on, sellest tugevam 
osa on sealt pärit. Seal oli ripakil küll õppematerjale, küll 
igasuguseid teadustöid. Ja tol ajal näiteks arvutiõpetus koolis polnud 
teadus, vaid šamanism. Leidsin esimesed teadustööd, mis 
seda USAs käsitlesid -- neetult huvitav oli lugeda! 

\question{Kas need olid PDFid?}

Enamasti olid tekstifailid, PDFid hakkasid hiljem tulema. 
ASCII printeriga trükkisin välja ja nii see elu käis. 

\question{Kas \LaTeX-i kraami ka leidus?}

Mina ei olnud tollal selle ala inimene, hetkel juba kirjutan selles. Ilmselt leidus. 
Emacsit ei ole ma ka ära õppinud ja ei õpigi. Vi-ga kirjutan küll. 

Pisut hiljem sain tuttavaks mehega, kes rippus samamoodi öösiti Tartu 
satelliiditaldriku taga, tema nimi on Marek Tiits\index[ppl]{Tiits, Marek}. 
Tuli välja, et mina ekspluateerisin siin Tallinna oma ja tema Tartu oma. 

Millalgi sain aru, et ühest 
modemist ei jätku ja et \enquote{otseinternet} on ka olemas. 
Jätan praegu vahele selle, kuidas Andres Bauman\index[ppl]{Bauman, Andres} 
KBFIst kirjutas grandiraha eest UUCPd ja Pegasuse meili siduva 
proge, mida koolid ja teisedki kasutasid, see on eraldi teema. 

Mis veel tehnoloogilise üleoleku salarelva puutub, siis 
loomulikult olid Treffneri omad mõnes mõttes haritumad kui 
meie, vähemalt maailmavaate poolest, aga kahur oli meil jämedam. 

Ühel hetkel hakkasin suhtlema nendega, kes õpetajaid 
koolitasid ja koolides internetti levitasid -- Anne Villemsi\index[ppl]{Villems, 
Anne} ja tema koolkonnaga. Sealt tuli arusaam, et kuidagi on vaja 
püsiühendus saada. 1994. aastal oli see \emph{mission impossible} alates 
rahast ja lõpetades kõige muuga. EENeti poisid aitasid kirjutada 
haridusprojekti, et paneme 117 kooli internetti ja tulevikus otse ka ning seega on kusagil vaja nimeserverit. 

Andres Bauman\index[ppl]{Bauman, Andres} tegi viltuse näo pähe, et koole on nii palju 
ja et tema väike vaene MicroVAXist nimeserver ei jaksa neid enam pidada. 
Arvata võib, mis selle väite tõeväärtus on.\sidenote{Nimeserver on üks 
väiksema ressursivajadusega interneti tuumtehnoloogiaid, jutuks olnud riistvara 
oli selle teenuse pakkumiseks koolidele vähemalt suurusjärgu võrra 
üledimensioneeritud.} Aga niisiis oli vaja teha eraldi koolide nimeserver. Haridusministeeriumi teadusrahadest eraldati selleks
mingisugune Sun, vist SPARCStation\index{SPARC!SPARCStation} 10. 

Ühelt koolikonkursilt õnnestus ka kolm arvutit saada. Rohkem ei antud, 
öeldi, et teistel on ka vaja. Arvuti all mõtlen ma eraldiseisvat 
Microsofti masinat. Kolmest masinast üks läkski nimeserveri alla ja 
EENeti SPARCStationist sai hariduse veebiserver. 

See oli vist 1993. aasta lõpp, kui see Sun oli juba olemas, 
peak.edu.ee\index{peak.edu.ee}, tuksus Baumani juures riiulil ja 
sain sellele kaugelt ligi. Olime kord Liivi tänava\index{Tartu Ülikool!Liivi 
õppehoone} keldris saunas ja Toomas Soome\index[ppl]{Soome, Toomas} 
rääkis, et on olemas mingi jubetumalt kihvt asi, mida nimetatakse veebiks. 
\enquote{Ahaa, rõõm kuulda!} \enquote{Tartu Ülikoolil olla ka!} 

Pärast istusime mingi terminali taha ja Toomas Soome nõidus seal 
natuke, kompileeris. Siis ma ei saanud sellest tegevusest absoluutselt aru, hiljem
poisid koolis õpetasid. Igatahes üks proge sai 
kokku: faili kirjutati rida \enquote{killadi, kolladi, 
\emph{coming soon}, siia varsti tuleb midagi}. Vaadata sai sellise
programmiga nagu Mosaic\index{Mosaic} ja oli isegi kiri, mis 
ütles, et midagi tuleb \ldots Põhimõtteliselt kompileeris Toomas Soome selle laiendatud 
sauna vältel mulle veebiserveri ja tegi sinna esimese faili ühe 
reaga, ja minul lasus nüüd kohustus. Ei jäänud muud üle, kui pidin selle 
selgeks õppima. Väidetavalt oli see Eestis seitsmes veebiserver ja 
niimoodi see www.edu.ee\index{www.edu.ee} sündis. Sisu tekkis alles palju 
hiljem.

\question{Selle pidi ju keegi kirjutama!} 

Mis seal kirjutada, nuiasin haridusministeeriumi 
administraatorilt nende andmebaasi välja ja panin kõik avalikult netti. Praegu ei tohi sellist asja üldse teha, kõik isikuandmed on 
saladus.

\question{Mis andmebaas see oli?}

Tol ajal inimesed ei teadnud, kui palju Eestis koole on. Selles andmebaasis 
oli isegi koolidirektori nimi olemas! Aadress ja üheksa-kümme rida 
infot iga kooli kohta. Koole oli kokku ligi 600, aga sealt tuli valik 
teha, erikoole ei hakanud panema, ja see oligi veebiserveri esimene 
versioon. 

EENetiga\index{EENet} sai
tehtud rahastamise kokkulepe, milleks oli järgmine projekt: tegime
43. kooli\index{Tallinna 43. Keskkool} juurde koolide 
sissehelistamiskeskuse. Poisid olid teenusega hästi rahul, sest nad said 
aru, et modemeid saab muuks ka pruukida -- ise öösel sisse helistada, 
kuhu vaja. Aga trass, vaskkaabel tuli ise välja ajada, 
keldris juhtmeid kokku mässida ja takistusi mõõta. 

Ja siis oli küsimus, millise tehnoloogia saame. Vendomar\index{Vendomar}, kodanik 
Kingissepp\index[ppl]{Kingissepp, Meelis}, üritas meile 
RADi\sidenote[][-4mm]{Tõenäoliselt peab Anto silmas Iisraeli samanimelise firma 
modemeid, mis toona uudse kontseptsioonina ei vajanud toimimiseks eraldi 
toiteallikat.} müüa. See oleks olnud 64kilobitise kiirusega, aga RAD ei hakanud 
meie liini peal tööle. Meile sobis US Robotics\index{US Robotics} ja 
ka mitte päris 33,6 peal, vaid kiirus oli vist 28 või 29. Millalgi juulis saime lõpuks püsiliini kooli ja KBFI\index{KBFI} vahel tööle: 
mõlemas otsas olid modemid ja võttiski \emph{carrier}'i\sidenote[][-4mm]{Kui liinil oli olemas \emph{carrier} signaal, oli ühendus loodud.} üles! 

Siis tekkis õnnetu moment, kui olime otse interneti küljes 
kiirusega 28 kilobitti sekundis ja nurgast astus välja Antti 
Andreimann\index[ppl]{Andreimann, Antti}, kes ütles, et ta on seda hetke 
oodanud sajandeid ja et meil koolis on kõik asjad tuksis, sest meil on ainult 
vana räpane ANSI kompilaator, aga vaja oleks GNU C kompilaatorit, millega saaks 
maailma päästa. Viis ja pool tundi kõik ülejäänud ootasid, internetti kasutada 
ei saanud, sest Antti tõmbas GNU kompilaatorit. Kõik, kes olid pühitsemiseks 
kokku tulnud ja arvasid, et nüüd saab midagi tõmmata \ldots {} See oli GNU 
kompilaator, mis sealt tuli! Aga usun, et Anttil oli õigus 
ja elu läks oluliselt paremaks, sest nüüdsest õnnestus kompileerida asju, mida 
varem ei õnnestunud. See asi ei käinud nii, et tõid 
funet.fi-st\index{funet.fi} või mõnest muust pirasaidist, Unixi puhul oli 
teisiti. 

See kuu aega, mis koolini jäi, poisid ainult koolis elasid. Augustis 
hakkasid ka esimesed nähtused tulema. Sellise internetikiirusega satuti paratamatult mõne kõrgkooli bugistesse 
veebidesse. SCO Unixil\index{SCO UNIX} oli üks 
knihv, kuidas modemit konfida, ja kui seda knihvi ei teadnud, siis 
\emph{carrier detect} jäi püsti, kui meie kehva telefoniliini pealt maha 
kukkusid. Näiteks Peda ja TTÜ adminnid kukkusid oma ruudu 
\emph{prompt}'i otsast päris sageli maha. Kuna meie poisid teadsid seda 
nippi, siis mõnikord sai niiviisi toimiva terminali otsa kätte. 

Noil aastatel juhtus igasuguseid artefakte. Seadusi hakati tegema alles hiljem, vist 1995 
ja 1996 kuulutati välja\sidenote{Ei ole selge, millist seadust Anto silmas peab, tõenäoliselt on tegu mõne arvutitega toimuvat reguleeriva seadusmuudatusega.}, ning oli loodud ka töögrupp, kus ma käisin 
Tartus Riigikohtu\index{Riigikohus} omadega midagi arutamas. No mida ma oskasin 
neile rääkida?! Nii palju kui olin netist lugenud, kes läänes mille ja kust 
ära varastas, rohkem ei osanud. Oma õpilaste käest sain ka üht-teist teada.

\question{Kas siis puutusidki esimest korda kokku infoturbega?}

Jah. Vist 1995. aastal\sidenote{Vastav lugu Äripäevas, kus ka Antot 
tsiteeritakse, on pärit 1996. aasta aprillist.} ilmus see kuulus \emph{no name} 
lugu. Eestis oli selline pisipank nagu EVEA\index{EVEA Pank}, kus töötas süsadminnina Igor 
Švets\index[ppl]{Švets, Igor}. Tehniliselt väga 
kvaliteetne häkker, kes kolistas kogu Eestit pidi ringi, kuhu aga sisse sai. Ma 
ei tea, kellele ta töötas. Ega minagi kõva käsi ei 
olnud, aga vähemalt olin võimeline aru saama, et keegi elab mul 
masinas. Eks poisid natuke aitasid ja ega Sven Turnau\index[ppl]{Turnau, Sven}, 
ametlik masinaülem, polnud ka loll mees. Sai aru, mis toimub, ja siis sai see 
skandaaliks keeratud ja jõudis ajalehte. Postimees tegi mister \enquote{\emph{no name}'ist} veel mingi noaga pildi. Press on press. 
Politseisse polnud üldse mõtet helistada, need ei saanud sellest aru ja tollal 
polnud selle kohta paragrahve ega midagi. KAPOsse helistasin, nemad ei saanud ka aru. Ühel hetkel suutsime ennast ise ära kaitsta ja aduda, et see maailm 
on tegelikult olemas. 

Keerulisem kui enda kaitsmine oli tegelikult see, 
kuidas poisse seiklustest eemale hoida. Poisid tulid ja
ütlesid: \enquote{Meil juhtus õnnetus}. Ma küsisin, mis 
õnnetus teil täna juhtus. \enquote{Saime kogemata Pedas ruuduks!} 
Pedas\index{Pedagoogika Instituut} oli Cadmus\index{Cadmus}\sidenote{Saksa arvutitootja 
Periphere Computer Systeme (PCS) toodetud tööjaam, mis käis toona kehtinud 
võimeka riistvara kolmandatesse riikidesse eksportimise embargo alla. Kuidagi 
aga õnnestus ühte Teaduste Akadeemia allasutusse selline masin hankida, kus 
selle nimeks sai konspiratsiooni huvides \enquote{Muscad}\index{Muscad}.}, mingi 
tumba. Eks nad läksid netti ja küsisid, mis bugid sel on, ja 
loomulikult sellel olid bugid ja nad said ruuduks. \enquote{Mis me nüüd teeme?} Ma 
ütlesin, et mis siis ikka, parandage bugid ära ja kinkige masin 
tagasi. Parandasidki bugi ära, kompileerisid sellele uue kerneli ja mida kõike veel (ma üldse imestan, et nad seda masinat õhku ei lasknud) ning
saatsid süsadminnile lõpuks kirja, kus nad seletasid, et on bugid ära 
paiganud ja et ta võib nüüd oma masina tagasi saada. See oli hästi 
tüüpiline juhtum.

\question{Kas need olid 12. klassi poisid, kes seal möllasid?}

Ei olnud. See aktiiv, kes üldse arvutitunde ei saanud, olid üheksandikud või 
kümnendikud. See oli lihtsalt niisugune tore aastakäik. 

\question{Nii et üheksakümnendate keskel üheksandikud võtsid ruutu?!}

Jaa! Mõnest asjast ei julge tänaseni rääkida, inimesed on ju skeenel 
alles. Õnnetusi juhtub ikka. Näiteks käis meil ühe teise kooli poiss (ta on 
ka siin\sidenote{Vestlus toimus Cybernetica\index{Cybernetica} kontoris.} 
töötanud), kelle ema oli teadustöötaja ja peres oli raha rohkem või saadi aru, mis on 
oluline, ja temal oli tollal 486. Nähes SCO Unixit\index{SCO UNIX} ja seda igavest 
muret seerianumbritega, kirjutas ta programmi, mis neid 
seerianumbreid nagu küllusesarvest väljastas. Ta ei olnud veel keskkooligi ära 
lõpetanud. Tema oli ka esimene, kes õppis Jack the Ripperiga\sidenote{John (mitte 
Jack) the Ripper on tuntud paroolide murdmise töövahend.} ümber käima. 

Tallinnas Tehnikaülikoolis\index{Tallinna Tehnikaülikool} oli Suni klass, kus olid 
mingid vanad hädaabi masinad ja arvutiklassi pealikuks Rebane\sidenote{Tõenäoliselt peab Anto silmas Enn 
Rebast\index[ppl]{Rebane, Enn}.}. Tol ajal oli
tsentraalne lahendus NISi peal -- tänapäeval ei ütle see sõna mitte midagi --, 
aga tolles lahenduses oli paroolifail kättesaadav. See mainitud
kodanik, kellel oli natuke parem masin (tal oli kodus ka SCO 
Unix\index{SCO UNIX}, kujutad ette!), lasi Jack the Ripperi käima ja sai vist kolme päevaga 70 protsenti paroole lahti. Olid ajad!

Ja siis oli tollal ju veel läbi telefonide helistamine. Modemeid oli maru palju, aga 
turvat mitte mingisugust. Oli olemas selline proge nagu Tone Loc -- kui praegu pingid 
läbi IP-aadresside plokke, siis selle progega sai telefoninumbreid läbi pingida. Kohalik kõne ei maksnud ju midagi. Nii et mina 
olen ka korra elus näinud, kuidas ühest autost, kus on kolm \emph{laptop}'i, 
lähevad krokodillidega juhtmed öösel telefonikappi ja lastakse kümnetuhandeseid 
plokke läbi. Sinna etappi jäävad Eesti Telefoni digikeskjaamade esimesed 
häkid. Mingid transpordikooli poisid said aru, kuidas digijaam töötab ja et paroolid on nirud, ning võtsid detsembris, kui kõik olid jõulupühal või 
välismaal komandeeringutes, mitu jaama üle. Midagi hirmsat küll ei tehtud. 

\question{Kuidas neid poisse siis kantseldati? Neil pidi ju olema mingisugune 
eetiline arusaam, et \enquote{pätti ei tehta ja puruks ei lasta}?}

Ma ei tea, kes need transpordikooli poisid olid. Ma mõnda õpetajat 
tundsin, aga see oli pikantne teema, sest kui saad liiga 
targaks ja täpselt teada, kes need olid, siis 
tekib endal moraalne kohustus. Mina lahendasin selle nii, nagu mulle õige 
tundus: püüdsin neid suurest jamast eemale hoida ja ega see tavatarkus, 
et mis sa nüüd tegid, ei aidanud. Pidid ta ära kuulama ja 
siis väga ettevaatlikult õige tee peale tagasi patsutama. 

\question{Sest tol hetkel oli murdeealisel poisil võimekus 
märksa suurem kui mõistus või arusaam elust!} 

Võimekus on tal suur sellepärast, et ta veel õlut ei joo, selle peale aega, 
energiat ja raha ei kulu, ja tütarlastega ka veel ei semmi. Järelikult, kui parasjagu pole koolipäev, siis on kuusteist tundi vaba aega, 
sest kaheksa tuleb magada. Ja kui sul on inimene, kellel on 16 tundi päevas 
aega istuda ja murda, siis see on väga toores jõud. Rääkimata sellest, kui neid on mitu ja 
nad suhtlevad omavahel.

\question{Kui kaua sa koolis töötasid?}

Täpselt ei mäleta, aga kolm aastat olin seal 
kindlasti. Ühel hetkel, kui hakkasin EENeti\index{EENet} keskust 
tegema, kujunes EENet teiseks tööandjaks. See võis olla 1994. aasta lõpus või 1995. aasta algul. 

Minu ülekandumine 
EENeti toimus kuidagi väga sujuvalt. Ametikoha nimi oli \enquote{insener}, 
põhjaregioonis, tipphetkedel vastutasin ligi 150 kooli 
UUCP side eest. Öösel kell 11 helistas mõni tädi, kooliõpetaja, ja rääkis, 
kuidas tal mitte miski ei tööta. Pidin peas looma mentaalse mudeli ja 
nagu \emph{helpdesk} tema katkendite järgi tegutsema. \enquote{See sinine lätakas} 
-- ahah, selge, Norton Commander. 

Tegin ka mõned koolitused uute metoodikate teemal, näiteks vist 1995. aastal Anne Villemsi\index[ppl]{Villems, Anne} juures 
Tartu Ülikoolis\index{Tartu Ülikool}. Sissehelistamiskeskuseks oli meil minu läpakas. Telefoni keskjaamast tõime 
pikad otsad (kaks päeva oli ettevalmistamist), mida sai klassis iga arvuti 
juurde viia. Ja kooliõpetajatega koolitust alustasime niimoodi, et kõigepealt 
käskisime neil arvutit lüüa, et saaks hirmust üle ja tekiks kohe õige tunne, kes on peremees. Tegelikult oli klassi valdajaga kokku 
lepitud, mis juhtub, kui mõni arvuti \enquote{ära lüüakse}. Aga nad nii kõvasti 
ei julgenud lüüa. 

Teine asi oli hoida tädisid ülearuse info eest. Näiteks modemi konfigureerimisel 
ütlesin, et ära süvene sellesse, mis see on, siin on täpselt ainult 
nii palju valikuid, vali neist üks ära! Tulemus oli see, et kõik koolitusel olnud said oma modemid konfitud, läksid oma koolidesse laiali 
ja said seal ka hakkama. Aga selle ettevalmistamine oli viiskümmend ühele: ühe tunni 
hoolitsuse kohta, mis nad said, läks 50 tundi ettevalmistust. See on üks kõige 
jubedamaid asju olnud, aga metoodika mõttes toimis! 

Kui Ants Reili\index[ppl]{Reili, Ants} suri, siis lagunes ETEK ära. 
Kauplesime endale tehnikaülikoolist\index{Tallinna Tehnikaülikool} Jüri 
Kaljundi\index[ppl]{Kaljundi, Jüri} käest vana Väksu\index{VAX}, ainult et 
transa käigus sai midagi pihta ja see ei läinud enam tööle. Koolidirektor tundis suurt
huvi, kust ma masina toiteks elektrit kavatsen võtta, nii et käima see 
ei läinudki, vaid viidi lõpuks kuhugi utiili. Ühesõnaga, see \emph{business} vajus 
sellisel kujul koost ära. Tipphetk oligi natuke aega enne mind ja minu 
ajal ning mina kandusin rohkem EENeti. 

Nii et enne, kui 1997. aastal järgmise pöörde tegin, istusin 
Tatari tänaval MicroLinki kõrval üleval (seal oli ühes toas EENeti kontor), kus 
üks Fixi kunagine helitehnik, Antti Andreimann\index[ppl]{Andreimann, 
Antti} ja mina tegime kolme peale kokku, mida suutsime. Antti 
sai aru, mis tegelikult toimus, ja suutis kernelit kompileerida ning meie tegime 
muid asju. Sellega lõppes koolide saaga ära ja tekkisid 
Tiigrihüpped\index{Tiigrihüpe} ja muud asjad, sellisel kujul lähenemist ei 
olnud enam vaja. 

\question{Ühesõnaga sa jõudsid infoturbe juurde väga praktiliselt ja 
samas väga inimesekeskselt -- pidid tegelema kaakide 
ohjeldamisega, kes 16 tundi päevas igale poole auke torkisid.} 

Nad ei olnud kaagid, vaid lugupeetud inimesed, kes kõik tegutsevad siin skeenel. 

\question{Mida sa praegu teed?}

Nagu näed, istume Cybernetica-nimelises aktsiaseltsis\index{Cybernetica} ja 
ma olen hetkel kirjaneitsi. See on maru raske küsimus, mida sa oskad. 
Tööle võttes küsitakse alati, mida sa oskad. Mina oskan mõnikord (mitte 
alati) asju suhteliselt selgelt kirja panna. Siin 
majas on segaseid asju, mida on vaja pisut selgemini kirja panna, suurtes 
kogustes. 

\question{Seda sa oskasid küll väga selgesti öelda, ma sain kohe aru, mida 
sa teed!} 

Jah, sõltumata sellest, mis töökoha nimetus või milline järjekordne 
käimasolev projekt on. Tekib natuke teine vaade, 
parem sõnastus ja järgmine seltskond saab sellele tuginedes juba järgmise 
lati ära võtta -- näiteks mõne asja kuhugi riiki maha müüa. 