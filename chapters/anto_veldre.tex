\index[ppl]{Veldre, Anto}

\question{Alustame asjade algusest, nagu ikka. Kuidas sina arvutite juurde said?}

Minu ema töötas ülikooli arvutuskeskuses\index{Tartu Ülikool!Arvutuskeskus}.

\question{Millise ülikooli?}

Tartu ülikooli. Eestis ei ole rohkem ülikoole,  ülejäänud on mingisugused \enquote{techid}. Ema oli ülikoolis, õppis matemaatikat ja 1959. aastast läks arvutuskeskusse. Mina ei tea,  oli ta seal juba põhikohaga juba tööl, või niisama katsetas. Igatahes tegeles ta Ural-1\index{Ural!Ural-1}  juures mingisuguse programmeerimisega. 

\question{Arvutuskeskus asus Liivi tänaval?}
Ei. See oli ülikooli, kõrval, ma arvan. Seal, kus praegu see biofüüsika on, kohviku majas.  Ma usun, et see oli 1959. aastal seal. Mina ei tea, mis nad mu isaga  vahepeal tegid,  abiellusid ja midagi veel, aga 1961. aasta augustis sündisin mina. Ja seda ma ka enam ei mäleta, kas siis oli Ural-1 või Ural-2. Minust on esimene pilt arvuti taustal, mida ma näinud olen, 62. aasta jõuludest. No mis jõuludest, siis olid näärid. Usun, et see oli ikka Ural-1. Ühesõnaga, mingi imelik aparaat, mille taustal mind näidati ja mida ma tegelikult ise ei mäleta, olen ainult pilti näinud. Edasi  seda arvutuskeskust koliti, kus ta seal igal pool ei olnud, küll Gagarinis, mis iganes selle tänava nimi praegu on\sidenote{Praegu on tegu Jaan Tõnissoni tänavaga}, Burdenkos, mis on praegu Aia\sidenote{Siiski Veski.}. Erinevad osad olid laiali. Kui hakkasin juba  teadlikult masinatest aru saama,  oli arvutuskeskus seal, kus praegu on see punane korporatsioonihoone\sidenote{Eesti Üliõpilaste Seltsi maja aadressiga  Jaan Tõnissoni tänav 1.}. Seal majas oli Ural Neljaks nimetatud masin\index{Ural!Ural-4}, aasta pidi siis olema mingisugune 1973 või umbes nii\sidenote{Ürikute järgi kolis Tartu Ülikooli arvutuskeskus Liivi tänava hoonesse 1972. aastal.}. 

\question{Kas su esivanemad olid programmeerijad? Mis nad tegid selle Ural-iga?}

Ema oli matemaatik, kogu see arvutuskeskus oli niisugune kahtlane koht. Seal oli mingeid kõrgema haridusega naisterahvaid, kes puistasid varrukast  korrelatsioonimaatrikseid ja arvutasid  lehmade poniteeti ja ma ei tea mis jubedusi veel. Ema oli, jah, põhikohaga arvutuskeskuses tööl. Isa oli bioloog, ta muidu töötas zooloogiamuuseumis, aga  käis haltuurat tegemas. Oli hobi-programmeerija, profid olla teda vihanud, sellepärast et ta leidis alati mingisuguse lokaalse optimumi. Üks näide. Kõvaketta \emph{interleav}-ingut siis veel ei tuntud, aga oli magnettrummel, mille peal see Ural-4 oma mälu pidas. Ja vanamees, kurat, arvutas välja selle trumli pöörlemiskiiruse ja hakkas oma progesid tegema niimoodi, et täpselt seni, kuni tema muid asja teeb, jõuab see neetud trummel sama koha peale tagasi. Ja kuigi tema programm nii-öelda struktuurilt ei kõlvanud kodulooma istmikku ka, siis töökiiruselt olid vist kas 13 korda kiirem, kui profiprogejate oma.

Nii et see minu lapsepõlv oli niisugune huvitav. Isal oli kapis kaks ülikonda. Üks, natuke kehvem, oli Vanemuises käimiseks. Teatris käidi tollal ülikonnaga, mitte nagu praegu. Teine, natuke parema ülikond, oli öösel \enquote{Masinasse minekuks}, kusjuures Masin kirjutati suure tähega, See oli siis Ural-4\index{Ural!Ural-4}. Mind sinna öösiti ei lastud, aga  päeval ma seal ikka kooserdasin. 

\question{Millal sai siis hakkasid seal nagu teadlikumalt käima hakkasid? Põhikooli ajal juba?} 

Nojah, kui teadlik ta oli. Tead, kuidas öelda, Vene ajal oli ju oluline päritolu, eks ole. Kui sa oled see neetud intelligent ja sellest kihist pärit,  siis sa pead tegelema mille kõigega. Ma käisin muusikakoolis, noorte trummarite ringis ja, oi jumal, mida kõike! Ma ei suuda enam väga täpselt meenutada, aga see pidi ilmselt mingi 73 lõpp olema. Jälle ma ajan seal kaks arvutuskeskuse\index{Tartu Ülikool!Arvutuskeskus} juhatajat segamini, üks tuli, teine läks, aga ma arvan, et siis oli veel Tapfer\sidenote{Jüri Tapfer\index[ppl]{Tapfer, Jüri}, oli Tartu Ülikooli Arvutuskeskuse juhatajaks aastatel 1971 –- 1995}. Igatahes kutsuti mind arvutuskeskuse juhataja, ülemuse,  mis ta iganes oli, kabinetti ja anti pidulikult kätte kasutajatunnus. Minu arust oli mingi viiekohaline number. Ja sellel polnud masinaga midagi pistmist, see oli lihtsalt aruandluse jaoks. Pidin mingisse paksu žurnaali allkirja andma, nagu nõuka ajal ikka. Ja see siis tähendas seda, et kui masinas mingi vaba hetk oli, siis tehniliselt oli lubatud minu programmi sealt ka läbi jooksutada. Ega ma nüüd väga edukas ei olnud, mingi paar programmi oli, mis enam-vähem tööle hakkasid. Eks isa aitas natukene siluda. Vot seda sõna ka enam ei ole. 

Aga kuidas see siis välja nägi. 

Esimene liin oli see, et selle paganama Ural-i\index{Ural} küljes oli niisugune ese nagu lai-trükkal. Minu arust oli tal 128 märki reas, igatahes kole palju, ja sealt tuli paberit päris koleda kiirusega. Ühe sõnaga, kiirkirjutusmasin. 

Ma usun, et ma olin palju väiksem, mingi nelja-viieaastane, kui ma ema või isa juures käies istusin selle lai-printeri peal. See kurinahk oli soe, saad aru! Sihuke kõrge koht, kuhu, poisikese värk, päris ise ei saanudki, keegi pidi aitama. Istud seal, kõlgutad jalgu ja vaatad, et mis operaator seal natuke eespool teeb. Aga selle printeriga seoses siis esimene liin oli, et tahtsin ka mingeid tabeleid, mingeid pilte trükkida. Nad seal tähtede nii-öelda  trükitihedusega kodeerisid  igasuguseid Mona Lisa-sid ja pilte. Need Mona Lisad olid suhteliselt alasti, ma mäletan. Mind see ei häirinud aga ma mäletan, et neid hoiti nurga taga. Tehti ka muidugi Leninist ja millest kõigest. Üritasin ka mingit pilti teha, loomulikult olid selles igasugused vead sees ja ma ei mäleta, kas ta lõpuni sai või ei saanud. 

\question{See ju tähendab, et sa pidid ikka kuskilt programmeerimist õppima. Või korjasid sa selle lihtsalt õhust üles?}

Asjalikke õpikuid ei olnud. Urali\index{Ural} kohta üks mingisugune oli. Aga põhimõtteliselt ta ei olnud isegi assembler, puhas masinkood. Käsukoodid, null-üks oli liitmine, null-kaks oli lahutamine, äkki oli niimoodi, ja siis sinna midagi järele. Aga seal olid muidugi trikid,  nagu Asmiski, ja nende selgeks saamine oli ainult läbi vaeva. 

Algaski asi sellest, et tegid mingisuguse lihtsama joonise valmis,  ta ei olnud muidugi õige. Sa pidid ta saama kas perfokaardi või perforlindi peale, kaks võimalikku sisendit. Millegipärast oli lihtsam perfokaardiga. Siis tuli kuskil tagaruumis mingit telegrafistitädi painata, kes Kesktelegraafist käisid nii-öelda lisatööd tegemas. Neile maksti raha, tagusid sisse ma ei tea mitu  märki sekundis. See üks või poolteist perfokaarti perforeeriti ära, seda ei olnud palju. Nojah, programm tuli alguses ju kirjutada  plankide peale, rohelise värviga trükitud plangid, operandid ja kommentaarid ja, oh jeesus,  panna sinna oma kasutajatunnus ja ma ei tea, mida kõike. Programm sai perfokaardi peale ja ega siis ei olnud nii, et läksid masinale ligi. Masinasaali ukse juures oli sihuke lahterdatud kast, nagu pioneerilaagris hambaharjade jaoks. Viis korda kuus või kaheksa korda kaheksa, kes seda enam mäletab. Tühja lahtrisse panid oma programmi ja kui masinal kas mõni mõni perifeeriaseade ei töötanud, tähtsamaid programme ei saanud teha või operaatoril öösel igav oli, võttis ta need nalja-asjad ja lasi läbi kuni esimese veani. Kirjutas perfokaardile jõleda käekirjaga mingi jõleda kommentaari ja võib-olla pani välja trükitud paberi ka sinna juurde. 

\question{Kogu selle vaeva läbimiseks pidi sul ju mingisugune põhjus olema?}

Mina ei tea. Miks mõned poisid jalgpallis käivad või kuskil? Nojah, ma ei usu, et seal ratsionaalseid selgitusi on, mina lihtsalt selle  asutuse seinte vahel üles kasvasin. Ja, no muidugi seal sai ka muid lollusi tehtud. Aia tänaval oli vahepeal üks ruum, kus oli paarkümmend tädi Robotroni ja  Rheinmetalli  mehaaniliste arvutitega. Mulle meeldis neid jagamistehtega kinni lasta, aga pärast muidugi tuli mehaanik kutsuda ja siis ma sain sõimata. 

\question{Kuidas sa mehaanilist asja kinni jooksutad?} 

Jagamistehe ei lõpe kunagi. Ta üritab selle kelguga kogu aeg edasi jagada, kuni kelk jookseb ühele poole kinni ära ja siis kas midagi läheb\ldots

\question{Jagamistehe ju lõpeb millalgi ära?}

Ei lõpe, nulliga jagamine näiteks ei lõpe kunagi. 

\question{Miks sa, kulla mees, panid mehanilise arvuti nulliga jagama!?}

Loll masin, ta saab sellega kinni jooksutada, põnev lihtsalt. Kes piinab kasse, kes  paneb bensiinitünni põlema ja kes laseb Rheinmetalli kokku.

\question{Rheinmentall ei kiunu, muidugi\ldots}  

Ei, ta ragises ja logises, see ei olnud ikka mingi ettenähtud olukord. Et seda nalja oli nii, et vähe ei olnud. Minu arvates. Tädide arvates muidugi mitte. 

\question{Kas mehaanik, kes kutsuti, hammustas läbi, mis juhtunud oli?}

Muidugi, ega ma sain sõimata selle eest. Ega ma ainuke olnud,  neid lapsi oli teisigi. See oli teada asi.

Nojah, vot ja teine programm, mille ma unustasin rääkimata. Keegi inseneridest tekitas Ural-i\index{Ural}  külge heligeneraatori. Pärast ma olen kuulnud, et see oli praktiliselt igal tolleaegsel arvutil, see oli mingi Covox Speech Thing-i\sidenote{Väline audioseade, mis võimaldas arvutil läbi paralleelpordi heli väljastada. Seade oli väga lihtne koosnedes hulgast takistitest, mis moodustasid primitiivse digitaal-analoogmuunduri. Selliseid jootsid koolipoisid üheksakümnendatel ise kokku ja pusisid neile ka sobilikud draiverid. Mäletan, et suursaavutuseks oli Nethacki\index{Nethack} paaritamine Warcraft II heliklippidega. Tekstipõhise mängu ekraanipuhvrist loeti kindlast kohast tekst, sõnadele olid vastavusse pandud helifailid ja need mängiti Covoxi abil maha. \emph{Someone to ax?}} eellane, põhimõtteliselt. Andsid arvutile lolle käske mõttetute argumentidega, käsukood loeti välja ja kui ta oli näiteks 01, tehti mingit madalat häält, 02 oli juba natuke kõrgem hääl ja  niimoodi sai laulukesi teha. Ja kuna ma muusikakoolis käisin ka, siis ma üritasin. Aga minu mälu järgi see ei saanud ka kunagi valmis, alati oli mingi viga sees. 

\question{Midagi ta ju ometigi tegi, mingit piiksu ju sai?}

Muidugi, muidugi. Lihtsalt mingi noot oli jälle vale. Ega see ei olnud siis mingisugune Sibelius või Cuebase või Fruity Loops, eks ole, millega sa kohe kuuled! Oi ei! Sa pidid   operaatoriga kokku leppima, et sa lähed ja kuulad. Vot selline raske elu oli lapsel, kes arvutuskeskusse oli sündinud. 

\question{Kui vana sa olid, kui sa neid programme tegid?}

No kindlasti see lõppes otsa 12-13, praegu viitsi ei viitsi arvutada, maksimum 14. Aga minu arust, kui ma 13 olin, sai Liivi\index{Tartu Ülikool!Liivi õppehoone} tänava arvutuskeskus lõpuks valmis. Alguses uued masinad koliti üle, pärast see Ural\index{Ural} visati ju üldse välja. Nii et igatahes 1975. aastaks oli see kõik pidulikult läbi. 

\question{Mis tollest Uralist sai? Lihtsalt utiili?}

Paraku jah. 

Kui mina veel Tartus elasin, 76. aastani\ldots Algul Ural-1\index{Ural!Ural-1} läks Nõosse\sidenote{1965. aastal, sellest sai alguse Nõo Keskkooli\index{Nõo Keskkool} arvutiõpe.}, siis läks Ural-2\index{Ural!Ural-2} Nõosse. Ural-2 blokke vedelas veel Tartu vahel,  kui Ural-4\index{Ural!Ural-4} töötas. Jõe ääres, selle keskmise silla juures,  oli kunagi mingi füüsikamaja. Sealt onude käest sai veel mõningaid kaubelda. Nii et ma olen neid trigeri blokke ka näinud, mis nad olid, 6N9S või 6N8S lambi peal. 

\question{Nojah, toonane arvuti oli väga modulaarne asi.}

Jah. Aga neid praktiliselt ei ole järel,  kõik on  metalliks läinud. 

\question{Kuskil Venemaal kindlasti midagi on!}

Ma usun, et Nõos ka mõnel õpetajal kuskil tagatoas äkki on mõni triger alles. 

\question{Mis sa edasi tegid, sebisid end Liivi tänavale?}

Ei. Vaata, mina olin juba siis kuulus isemõtleja, aga Nõukogude Liidus isemõtlemist ei sallitud. Nii et selle etapi võib  põhimõtteliselt arvutite koha pealt vahele jätta. Nõuka ajal lihtsalt elasin nagu suutsin, lihtne ei olnud, ülikooli ei lastud, mõned muud probleemid veel. Aga nõukaaeg sai raha teenitud igasuguste aparaatide parandamisega. Põhimõtteliselt ma olin 16, kui Tallinnasse tulin. Läksin alguses  polütehnikumi\index{Tallinna Polütehnikum} raadiotehnikat õppima. Sealt tuli kirg tinutuskolvi vastu, nii et vahepeal väga pikalt ei olnud mingisuguseid arvuteid kuskil. Lihtsalt sai ennast elektroonikaga lõbustatud. Mingitel segastel aegadel see muidugi tagas äraelamise. Kõik meenutavad, kui raske oli mingisugustel murdehetkedel, kui poes ei olnud midagi. Ja ma ütlen, et ma ei mäleta seda hetke selles mõttes, et mul lihtsalt härrased olid vorstiga ukse taga, sellepärast et hommikul kell kuus pidi algama mingi jalgpalli MM  ja telekas pidi mängima sel kellaajal. 

\question{See tähendab, et sa pidid kuskilt teadmised üles korjama?} 

See oli veel üks hobi muidugi, see oli veel üks hobi. Ural-4\index{Ural!Ural-4} taga oli ju ka toatäis insenere, mingisugune tüki viis, ja eks ostsillograaf oli kogu aeg arvutil ligi. Ei mina oska öelda, kust ma selle täpselt selle üles korjasin. Kusagilt sealt. 

\question{See on huvitav, et seda räägivad paljud, et sihukest süsteemset õpet on vähe olnud, aga kuskilt teadmine tuli?}

Nõuka ajal süsteemne algas ju sellest, et sa pidid olema kodumaale lojaalne ja igatpidi väga standardne ja siilisoenguga ja siis sind võib-olla lasti kuhugi õppima ja siis sa lõpetasid kuskil mingisuguses salajase töö instituudis, eks ole. See oli  ametlik \emph{track}. Aga mitteametlik oli\ldots Mäletan kui ma Tallinnas olin poisikesest peast,  16 või mis ma olin, Küberneetika Instituudi maja\index{Küberneetika Instituut} sai just valmis. Samamoodi, arvutuskeskus läks alati esimesena, see oli kõige kallim. Ja  ma mäletan, et ma käisin  mikroskeemide käsiraamatuid nuiamas. Läksid kuskilt tagauksest, \emph{social engineer}-isid ennast sisse Kevin Mitnicku\index[ppl]{Mitnick, Kevin} moodi ja siis seletasid, et kuidas sul on ikka tähtis konstruktsioon pooleli, aga ainult kahe mikroskeemi parameetrid on veel puudu. Said siis selle salajase käsiraamatu nii-öelda kohapeal kasutamiseks kätte. Nii see asi käis. 

\question{Küberneetika instituudis, olid olemas need raamatud?} 

Jah, see oli üks koht, kus neid sai. Selliseid kohti oli Tallinnas veel,  mingid sõjatehased ja asjad. 

\question{Sa oli põhimõtteliselt vabakutseline, nihukene vaba mees?} 

Ei, siis ma ikka veel käisin tehnikumis, õppisin raadiotehnikat. 

\question{Mingi hetk jõudsid ikkagi moodsad arvutid ka sinu juurde, millal see oli?}

Sinna vahepeale jääb veel mingi segane aeg. Ma mäletan, ma üritasin mingeid vene arvuteid ka parandada, näiteks Iskra-555\index{Iskra!Iskra-555}.  Pärast tehti Iskraid, mis olid Inteli 8086 kloonide peale, aga see oli mingi ise leiutatud magnetkaardi pealt töötav raamatupidamisarvuti. Ma olen ka remontinud neid mehaanilisi suuri Robotroni raamatupidamisarvuteid. Põhimõtteliselt samasugune, nagu see Rheinmetall. Aga kui sa nii-öelda ametlik mehaanik ei ole, kui sul ei ole kogu dokumentatsiooni, on see õnnemäng. Ütleme nii, et enamasti ei õnnestu teha, tõenäosusega 20 prossa. Aga kuna neid mehaanikuid telliti ka Moskvast ja ma ei tea kust, siis aeg-ajalt lasti ligi. Mingi kogemuse sain, aga head mälestust ei ole. 

Vist 1989 või millal see oli, sattus vend mingitel asjaoludel CeBIT-ile. 

\question{89? See oli ju puha nõukogude aeg!}

Oli jah, nõukaaja lõpus, igatahes, mul need aastaarvud natuke ujuvad, äkki oli 90? Ei mäleta, kusagil seal kandis. Igatahes pani ta kõik oma elusäästud kokku ja tõi endale sealt portatiivse Taiwani läpaka, kahe flopiga. Isegi Turbo Pascal-it\index{Turbo Pascal} sai sellega käivitada, selleks oli vaja millegipärast kahte flopit. Ühe peale ei mahtunud ära. 

Ega must väga Pascali\index{Pascal} progejat polnud, jälle  tegin paar näidet  ja keegi teine silus nad mul ära ja andis tulemuse. Ütleme nii, et idee oli õige, aga  näpud olid lühikesed. Õppinud ma ikka ei olnud seda asja. Ega ma progemises ei ole kunagi mingi kõva käsi olnud. Aga jah, too masin käis kahe 720-se flopiga,  kolm pool tolli küll. Bondwell, mingi sihuke valge Taiwani XT. Vend tegi tööd sellega, aga siis aeg-ajalt sai näppida, imelikke asju teha. 

Ja järgmine koht oli\ldots No vot, ma ei teagi, mida nüüd järgmiseks lugeda. 

93. alguses suht, kui oli ikka selge, et nüüd on Eesti riik, et nüüd ei ole enam Vene riik (seal vahepeal oli segane periood), sattusin ma kuidagi tööle õpetajaks. See oli selline, kuidas ma nüüd ütlen, õnnetu juhus. Tallinnas on vana 43. kool\index{Tallinna 43. Keskkool}, praegune Tehnikagümnaasium\index{Tallinna Tehnikagümnaasium}\label{sisu:43kool}. Tegelikult oli ühes majas kaks juriidilist asutust. Üks oli kool ja  üleval neljandal korrusel oli vana kadunud Ants Reili\index[ppl]{Reili, Ants} poolt tehtud, mis ta nimi oli, ETEK. Eesti mingi teaduslik-tehniline ettevalmistuskeskus, nagu see nõuka ajal käis. Aga Reili Ants oli  sihuke sihuke kihvt vanamees. Ta õpetas ju tööõpetust, tal olid telekas mingisugused tööõpetuse saated. Igatahes  tagus ta kuskilt välja mingi eksperimentaalse raha ja tegi kooli neljandale korrusele nii-öelda keskuse ja saavutas selle, et kooliõpilastele hakati seal mingeid tehnilisi aineid andma. Võttis Tipi-koolist vanad elektroonikud  tööle ja nii edasi. Seal olid arvutid, mingisugune vana Elektronika\index{Elektronika} klass ja mida kõike, see oli tal hästi püsti pandud. Tal olid isegi mehed, kes neid parandada oskasid, mis oli tollel ajal täiesti kriitiline. Aga need vanad mehed ei saanud lastega suurt hakkama. Ja mina olin siis see nii-öelda päästerõngas, kes siis pidi hakkama tunde andma. 

\question{Kuidas sa sinna sattusid? Lihtsalt tutvuste kaudu?}

Ema töötas seal kunagi psühholoogina, see on keeruline lugu. Selle taga on tegelikult see stiil, mis  Keevallik\sidenote{Andres Keevallik\index[ppl]{Keevallik, Andres}, Tallinna Tehnikaülikooli rektor aastatel 2000–2005 ja 2010–2015.} pärast tegi, et miks TPI-st nii palju välja langeb. Selle taga on ju see, et ega elama ju keegi ei õpeta, eks ole. Et poisid lähevad  kooli ju et saada oma eriala kätte, aga kuidas õlut korralikult juua ja õhtul klubis käia? Ja siis tuleb  veel tööandja, kes võtab esimese kursa pealt juba ära ja kui kogu seda asja kokku miksida, siis kukuvad välja. Ja, vot, veel enne Keevallikut Ants Reili\index[ppl]{Reili, Ants} oli üks, kes sai sellest põhimõttest aru. Nii et põhimõtteliselt 43. kooli\index{Tallinna 43. Keskkool} lõpueksamitele ta saavutas staatuse, et need ühtlasi olid Tehnikaülikooli\index{Tallinna Tehnikaülikool} sisseastumiseksamid. 

Seal on veel pikk-pikk jutt. Need, kes tehnika eriala valivad ei vali seda mitte sellepärast, et nad lollid ja jobud on. Tark inimene läheb ju arstiks ja advokaadiks, sa tead küll. Aga tegelik põhjus on verbaalne võimekus. Ehk siis, kui sa ei suuda seda seletada kiiresti ja korralikult, mida sa tahad, siis arvatakse, et ah, mingi pagana tehnika-nohik. Aga mida Reili tegi oligi see, et ta ajas sinna kooli kokku inimesi, kes nende poistega kolme aasta vältel tegelesid ja õpetasid neid oma mõtteid inimese moodi väljendama. Ja minu ema siis mingil hetkel sattus sinna kooli psühholoogiks. 

\question{Ta oli ju programmeerija?}

No mis siis? Ta vahepeal tegeles kutsehariduses testidega, pikk lugu jälle. Aga see oli kuidagi nii naljakalt, et kas mina soovitan alguses Reilile\index[ppl]{Reili, Ants} oma ema ja minu ema soovitas pärast mind, ühesõnaga see oli kuidagi rekursiivselt tutvuste kaudu. 

Jaanuaris 1993 olin ma seal igatahes paigas ja mulle öeldi, et neljandast veerandist (see, mis suvega lõpeb) ma pean hakkama juba kellelegi midagi õpetama. Oli teisejärguline, kellele ja mida, aga noh, lihtsalt, et oleks projekti eesmärgid ära täidetud. Ma vist isegi hakkasin natuke varem. Põhimõtteliselt mind pandi olukorda, kus oli mingisugune Unix. See \enquote{mingisugune} osutus pärast SCO 3.2.2\index{Unix!SCO}, masinasse oli kusagilt Tõraverest kaks Urania\index{Urania} muxi kaarti siis ostetud, nii et põhimõtteliselt sellele masinale sai kaks korda kaheksa terminali taha võtta, pluss oma konsool.  

Mu käest küsiti \enquote{Tead, mis Unix on?}. Ma ütlesin \enquote{Jaa, ma olen vähemalt ühte raamatut lugenud ja umbes saan aru}. Ja siis tuli tegelda. Kaks kuud läks selleks, et ma ise aru sain, mis asi see on. Seejärel aeti lapsed ette ja tuli neile õpetada.

\question{Mis masin see oli?}

386, tal oli 8 mega mälu. 40 mega ketast ja SCO Unix 3.2.2\index{Unix!SCO}. 

\question{Huvitav kombinatsioon! Kas selle organiseeris seesama koolidirektor?}

Ants Reili\index[ppl]{Reili, Ants} ja tema sõbrad ja sugulased. Kooli direktor oli proua Errit\index[ppl]{Errit, Anneli} ja tema tegelase vene keele õpetamisega. Temast me täna rohkem ei räägi. 

Eks  ta oli niisugune nii-öelda mentaalne ülekanne vanast \emph{mainframe} ajastut, no need mehed mõtlesid lihtsalt niimoodi, nad olid sellega üles kasvanud. 

\question{386 vedamas kuuteteist terminali\ldots?}

Oi, väga hästi! Ega siis ei progetud, nii nagu praegu, et \emph{include}-takse kogu eelnev maailm. Siis ikkagi kirjutati asju asmis ja korralikult, aga see selleks. 

Mulle anti kaks seltskonda, anti kaheteistkümnendikud ja anti viiendikud. Ja kaheteistkümnendikega veel nii ja naa, aga mida sa nendele viiendikele seletad aastal 1993? Ja  mingi Unix on ka veel, eks ole. Päris kole. Aga, nagu öeldud, seal olid vanakooli mehed, mina progemisest jälle ei jaga suurt mõhkugi, aga Sven Turnau\index[ppl]{Turnau, Sven} oli  süsteemiülem ja see proges ANSI C-s nagu issand jumal ja proges mulle paar abivahendit. Üks oli mingi mäng, millega sai terminali ekraanile tärne joonistada. No vot, mida mina arvutuskeskuses kunagi tegin, väga tuttav asi oli. Ja see proge töötas, ei kiilunud kinni kusagil, lapsed said aru, kuidas ta on, ja pärast sai printerist välja lasta. Kuskilt TPI laost saime seda vana murdekohtadega paberit kilode kaupa, selle eest maksma ei pidanud. Printer lindi eest küll pidi, aga Reili eelarve elas selle kuidagi  üle. Nii et põhimõtteliselt lastel oli praktiline väljund. Joonistas oma jubeduse  ekraani peal valmis ja trükkis välja, ühe tunniga tehtud. Ja teine põnev asi oli, et SCO Unixil\index{Unix!SCO Unix} on email sees, saab üksteise masina piires kirju saata. Ja eks siis Mari sai Jürile teatada, mis ta tast arvab ja tema emast ja nii edasi ja seda nad ka väga aktiivselt tegid. 

Järgmine õppeaasta see kõik lihtsalt jätkus. Aga olukord läks huvitavaks aprillis, kui mind veeti Nõkku\index{Nõo Keskkool}, kus oli mingisugune Unixi koolitus. Juhuks, kui ma veel millestki aru ei saanud, siis et ma nüüd ikka tõesti ise ka aru saaks, mis see on. Observatooriumi\index{Tõravere Observatoorium} all tegutses Urania\index{Urania} nimeline firma, nemad selle korraldasid. Margus Liiv\index[ppl]{Liiv, Margus} ja  Kaiti Kattai\index[ppl]{Kattai, Kaiti} ja kes nad seal olid. Muide, sellest päevast hakkab mu digiarhiiv peale.  Minu  arust oli kas 2. või 5. aprill, see on mul juba digitaalselt alles. Tegelikult neid Bondwelli programme on ka kuskil flopide peal, aga tühi nendega. 

Pärast seda Urania  koolitust,  ütlesin et lähme siis Nõkku\index{Nõo Keskkool} ka juba, see siin lähedal, vaatame, mis need seal koolis teevad. Ja Nõos oli selline härrasmees nagu Kill Kask\index[ppl]{Kask, Kalju}\index[ppl]{Kask, Kill|see{Kask, Kalju}}. Räägib \enquote{Ah, me ripume mingi interneti küljes ja mingi trillallaa-trullallaa, mingi kirjade vahetamine} ja väga meeldiv, eks ole. Neil oli laual mingisugune karbike. Ma küsin, \enquote{mis see on}, \enquote{see on modem!}, \enquote{Ahah.} Rohkem ma ei julgenud küsida. Kui modem, siis modem. Nagu öeldud, netti ei olnud, ma ei tea, kust ma selle selgeks tegin, aga mõne päevaga oli nagu kontekst selge, et mis aparaat see on ja mida sellega saab. Kirjutasin Avatud Eesti Fondi\index{Avatud Eesti Fond} taotluse, et \enquote{ma tahan nüüd ka seda saada}. Siis ma olin veel Sorosega sõber, nad tegelesid veel tõsiste asjadega, mitte nendega, millega praegu. Kool sai selle raha kätte, nii et põhimõtteliselt 1993 sügisel, kuupäeva ei mäleta (mingid projektid on kuskil trükitult alles, aga vahet pole), sai selle modemi. SCO Unix\index{Unix!SCO} võttis selle ilusasti taha ja vot nüüd selgus, et meil on relv! Kuidas käis normaalsetes koolides tollel ajal emaili saatmine? No näiteks Tartus Treffneris\index{Hugo Treffneri Gümnaasium}, seal sai ka külas käidud,  oli palju arvuteid, üle 10. Ikka väga hästi varustatud kool. Aga modem oli neil taga ainult ühel ja selle arvuti taga oli järjekord. Üks õpilane siis kahe näpuga toksis oma kirja ära, saatis minema ja teised ootasid järjekorras. Oligi järjekord kogu aeg! 

\question{Aga sul olid ju terminalid, see on arhitektuurselt palju parem!} 

Absoluutselt! Ja see oli relv. 

Kuidas see välja tuli. Anne Villems\index[ppl]{Villems, Anne} Tartust korraldas õpilastele igasugu mängusid, et nad internetiga  ära harjuksid. Näiteks Gaia, kus olid mingid välja mõeldud riigid. Mäletan, meie kool nimetati Barbariaks, üks väga õige nimetus\sidenote{Gaia käivitus 1994. õppeaasta alguses ja sellest võttis osa 20 gruppi 17 keskkoolist või gümnaasiumist.}! 

Nagu öeldud, meil ei pidanud keegi järjekorras seisma. Meil 17 tükki (16 aga kui Turnaul\index[ppl]{Turnau, Sven} hea tuju oli, lubati keegi konsooli taha ka) võisid korraga oma kirju trükkida. Masin korjas nad kokku ja ära saatis siis, kui tal sellega tegelemiseks aega oli. Aga sellega asi ei piirdunud. SCO Unixis\index{Unix!SCO} sai modemi sisse helistamise peale häälestada. Ma ei mäleta enam, mis pagana laos me käisime, me saime Eesti Energiast\index{Eesti Energia} paar vana modemit, 1200 baudi (ilma vea korrektsioonita, mingi ürgaegne värk) ja mingisuguseid soome tähtedega terminale ja ühesõnaga mingeid koledaid asju. Soome tähed ju muudavad ära ASCII lõpu, kus nurksulud on, sinna tulevad nende ööd ja üüd. Aga selle tulemusena õnnestus aktiivile, ehk siis kolm neli poissi, kes kõige aktiivsemad arvutiklassis käijad olid, saada koju modemid ja terminalid, sest kes see kannatas siis endale arvutit osta. Nüüd läks asi hulluks kätte, sellepärast et poiss logis ennast öösel kooli SCO Unixisse ja trükkis kirja valmis.  Seepeale Anne Villems\index[ppl]{Villems, Anne} ütles, et te olete lihtsalt tehnoloogiliselt teistest nii palju üle, et see ei ole enam aus. 

\question{Huvitav on see, et sa ei olnud üle mitte tehnoloogiliselt selles mõttes, et sul oleks ägedam arvuti olnud aga just \emph{setup} oli äge!} 

Jah, organisatoorne pool, sest seal olid vana kooli mehed, kes teadsid, kuidas nad selle püsti panevad ja see sissehelistamine, vot see oli väärtuslik. See on omaette jutt, seda võib ka natuke rääkida, aga ütleme nii, et  Microsofti masinatega ei olnud kellelgi mingit sissehelistamist. Ja see, kui poisil öösel kell kolm und ei ole aga tuleb nii-öelda  inspiratsioon peale ja tahab seal Gaia mängus kaasvõitlejad teise planeedi peal saata, siis tal oli selleks täielik tehniline võimalus. 

\question{Legend räägib, et umbes sel ajal tehti Eestis esimesed veebmasterite kursused. Sina olla seal ka tembutanud?} 

Ei mäleta, see on minu jaoks praegu tühi koht. 

Aga sealt hakkas üks, teine rida. Nimelt ma sain kiiresti aru, et see 2400-ne  modem on nabanöör. Sest mina ise, selle asemel, et öösel koju magama minna (mul ei olnud kodus terminali tookord veel)  istusin 1993. aastal ja  1994. alguses öösel koolis ja rippusin, horos.kbfi.ee\index{horos.kbfi.ee} küljes, sinna sai sisse logida. Tallinna välisühendus oli tollel ajal 64 kilobitti sekundis. Ja kuna öösel normaalsed inimesed magasid, siis põhimõtteliselt viis pool kilobaiti sekundis oli maksimaalkiirus, mida ma sealt kätte sain. See masin muidu tõmbas uudiseid, \emph{news group}-e. Aga mina sain mööda netti ringi kolistada,  terminali ekraaniga asja tõmmata. Öö jooksul ma suutsin tavaliselt umbes viis flopitäit kohale tõmmata. Kui  järgmine päev tunde ei olnud, magasin välja ja läksin Andres Baumani\index[ppl]{Bauman, Andres} juurde KBFIs\index{KBFI} ja anusin, et kas ma nüüd saaks tema masinast asjad flopi peale ära kopeerida. Teine variant oli, et järgmisel öösel, tõmbasid oma modemiga. 

\question{Mida sa tõmbasid?}

Tollel ajal oli tõmmata väga palju. Internetis paska veel ei olnud ja häid materjale oli palju. Ülikoolidel olid gopher-i saidid, FTP saidid. Veeb just hakkas tulema, ei olnud veel tulnud. Ja minu lemmikmeetod oli see, et ma läksin mingisse uudisgruppi nagu mingi alt.sex või mis iganes, noh, kus inimesed ikka käivad. Ja siis mingisuguse progejupiga, ma ei mäleta mingi sed ja awk, mida Turnau\index[ppl]{Turnau, Sven} mulle õpetas, otsisin välja  ülikoolide aadressid. \emph{Strip}-isin nimed eest ära ja läksin käsitsi mööda ülikooli FTP servereid kollama. Ja kuna tollel ajal mingit andmekaitset ega midagi ei olnud, siis absoluutselt kõik asjad olid ripakil. 

\question{Anonüümne FTP! Ütlesid \enquote{anonymous} ja \enquote{ftp}\sidenote{Levinud (ja kasutajate hulgas laialt teada) viis avalikke FTP teenuseid pakkuda oli kasutajale \enquote{anonymous} kas teadaolevat parooli \enquote{ftp} või aktsepteerida paroolina mis iganes sisendit.} ja saidki sisse}

Ja, aga tollel ajal USA-kad ei osanud seda veel karta, nii et põhimõtteliselt ma harvesteerisin vastuoluliste nimedega gruppidest, sest need olid kõige suuremad. Sealt sain maksimumkoguse ülikoolide nimesid, ja nende järgi (ega keegi pole mind õpetanud, mis ülikoolide USA-s on) käisin tõin FTP saidist kõik  ära, mis mulle tundus, et on lugemisväärne.

\question{Just lugemismaterjal, mitte programmid?}

Niipalju kui mul üldse mingisugust krüpto-teadmist on, sellest ikkagi tugevam osa on sealt pärit. Asjad, mis olid ripakil: küll õppematerjalid, küll igasugused teadustööd. Ja tollel ajal näiteks koolis arvuti õpetamine polnud teadus, see oli šamanism. Ja sealtkaudu ma leidsin esimesed teadustööd, mis seda USA-s käsitlesid, neetult huvitav oli lugeda! 

\question{Need ei olnud ju PDF?}

Enamasti olid tekstifailid, PDF-id hakkasid pigem ikka natuke hiljem tulema. ACII printeriga trükkisid välja, nii see elu käiski. 

\question{\LaTeX-i kraami ka leidus?}

Mina ei olnud tollel ajal selle ala inimene, hetkel juba kirjutan. Ilmselt oli. Ja Emacsit ma ei ole ka ära õppinud, ja ei õpi. Vi-ga kirjutan. 

Pisut hiljem ma sain tuttavaks mehega, kes rippus samamoodi öösiti Tartu satelliiditaldriku taga, tema nimi on Marek Tiits\index[ppl]{Tiits, Marek}. Tuli välja, et mina ekspluateerisin  seda Tallinna oma siin ja tema Tartu oma. 

Aga jah, siis läksid ajad huvitavamaks. Ma sain kuidagi aru, et sellest ühest modemist ei jätku ja et mingisugune \enquote{otse internet} on ka olemas. Ma jätan praegu vahele selle jutu, kuidas Andres Bauman\index[ppl]{Bauman, Andres} KBFI-st jälle mingi grandiraha eest kirjutas UUCP-d ja Pegasus maili siduva proge, mida koolid ja kõik kasutasid, aga see selleks, see on eraldi \emph{thread}. 

Nii koolis see salarelv käis, see tehnoloogiline üleolek nii-öelda. Loomulikult, Treffneri omad, ma usun, olid ikka mõnes mõttes haritumad, kui meie, vähemalt maailmavaate poolest, aga kahuri jämedus oli meil suurem. 

Ja mingi hetk siis sai suhtlema hakatud nendega, kes neid va õpetajaid koolitasid ja koolides internetti levitasid. Anne Villems\index[ppl]{Villems, Anne} ja tema  koolkond. Ja sealt kusagilt tuli see arusaamine, et ikka on vaja kuidagi püsiühendus sisse saada. Aga no 1994. aastal ja püsiühendus! Alates rahast ja lõpetades kõige muuga, see oli ikka \emph{mission impossible} tollal. EENeti poisid aitasid mingeid projekte kirjutada, tuli kirjutada mingi jube haridusprojekt, et paneme 117 kooli internetti ja tulevikus otse ka ja kui otse ka, siis tähendab, et kusagil on vaja nimeserverit. Ühesõnaga Andres Bauman\index[ppl]{Bauman, Andres} tegi viltuse näo pähe, et koole on nii palju saanud, et tema väike vaene MicroVAX-ist nimeserver ei jaksa neid enam pidada. Noh, sa saad aru, mis selle väite tõeväärtus on\sidenote{Nimeserver on üks väiksema ressursivajadusega Interneti tuumtehnoloogiaid, jutuks olnud riistvara oli selle teenuse koolidele pakkumiseks vähemalt suurusjärgu võrra üledimensioneeritud.}. Aga oli vaja teha eraldi koolide nimeserver. No ja kui vaja, siis vaja, eks ole. Kuskilt Haridusministeeriumi teadusrahadest eraldati mingisugune Sun selle jaoks. Kas see oli nüüd SPARCStation\index{SPARCStation} 10 või midagi taolist, issand, kes seda mäletab! 

Ja kusagilt kooli konkursilt õnnestus ka kolm arvutit saada. Rohkem ei antud, öeldi, et teistel on ka vaja. Arvuti all ma  mõtlen siis eraldiseisvat Microsofti masinat. Ja nendest  tegelikult üks masin läkski nimeserveri alla. Põhimõtteliselt tegime diili, et mina kooli konkursiga saadud masina panin nimeserveriks. EENeti SPARCStation-ist, oli ta nüüd viis või kaks või kümme, enam ei mäleta, sai hariduse veebiserver. 

Ma arvan, et see oli 1993 lõpp, kui see Sun oli juba olemas, peak.edu.ee\index{peak.edu.ee},  tuksus seal Bauman juures  riiuli peal, ma sain talle kaugelt ligi. Olime kord  Liivi tänava\index{Tartu Ülikool!Liivi tänava õppehoone} keldris saunas ja Toomas Soome\index[ppl]{Soome, Toomas} rääkis, et  mingi jubetumalt kihvt asi on olemas, veebiks nimetatakse. \enquote{Ahaa, rõõm kuulda!} \enquote{Tartu Ülikoolil olla ka} \enquote{Ei no tore on!}. Pärast istusime mingi terminali taha, Toomas Soome nõidus seal natuke, kompileeris. See on tegevus, millest ma siis absoluutselt aru saanud, poisid, õpilased, hiljem koolis õpetasid. Olgu. Aga igatahes mingi proge sai  kokku ja  mingisse faili kirjutati mingi üks rida  \enquote{killadi, kolladi \emph{coming soon}, siia varsti tuleb midagi}. No ja siis ma vaatasin, mingi programm oli, Mosaic, sellega sai täitsa vaadata, oli isegi mingi kiri, mis ütles, et midagi tuleb\ldots Põhimõtteliselt Toomas Soome selle laiendatud sauna vältel kompelleeris mulle veebiserveri ja tegi sinna esimese faili, ühe reaga, ja minu  peal  lasus nüüd kohustus. Ei jäänud muud üle, ma pidin selle selgeks õppima. Aga see väidetavalt oli see Eestis seitsmes veebiserver. Ma arvan, et see saun oli 1993. detsember aga mälu võib siin petta. Igatahes niimoodi see www.edu.ee\index{www.edu.ee} tekkis. Sisu tekkis alles palju hiljem.

\question{Selle peab ju keegi kirjutama!} 

Mis seal kirjutada, ma lihtsalt läksin nuiasin Haridusministeeriumist administraatori käest nende andmebaasi välja ja sai kõik avalikult netti pandud. Asi, mida praegu nagu üldse teha ei tohi. Kõik isikuandmed on ju saladus.

\question{Mis andmebaas see oli?}

No ega tol ajal inimesed ei teadnud, palju Eestis koole on. Selles andmebaasis oli isegi kooli direktori nimi olemas! Aadress ja mingi üheksa või kümme rida infi iga kooli kohta. Koole oli kokku mingisugune 600, aga sealt tuli  valik teha, päris erikoole ei hakanud panema ja see oligi veebiserveri kõige esimene versioon. 

Aga nüüd on see järgmine oluline \emph{thread}. Sai EENetiga\index{EENet} tehtud rahastamise kokkulepe, milleks oli siis järgmine projekt: teeme sinnasamma 43. kooli\index{Tallinna 43. Keskkool} juurde koolide sissehelistamiskeskuse. Poisid olid selle teenusega hästi rahul, sest nad said aru, et neid modemeid saab muuks ka pruukida. Saavad ise öösel sisse helistada, kuhu vaja. Aga see trass, see vaskkaabel, tuli ise välja ajada. Tuli ise käia keldris juhtmeid kokku mässimas ja takistusi mõõtmas. Ja siis oli küsimus, mis tehnoloogia saame. Vendomar\index{Vendomar}, kodanik Kingissepp\index[ppl]{Kingissepp, Meelis}, üritas meile RAD-i\sidenote{Tõenäoliselt peab Anto silmas Iisraeli samanimelise firma modemeid, mis toona uudse kontseptsioonina ei vajanud toimimiseks eraldi toiteallikat.} müüa. See oleks muidugi olnud 64k kiirusega, aga RAD ei hakanud meie liini peal tööle. Mis tööle hakkas, oli US Robotics\index{US Robotics} ja ka mitte päris 33,6 peal, aga ma ei mäleta, kas mingi 28 või 29, mingi sihuke kiirus. No vot. Ma võin kuupäevaga eksida, paar päeva siia sinna, minu arust juuli lõpp. Suveaeg, ta võis olla mingi 29. juuli või 28. või 27., mis on see päev, kui sai selle püsiliini  lõpuks tööle kooli ja KBFI\index{KBFI} vahel. Mõlemas otsas modemid ja võttiski \emph{carrier}-i üles! 

Nojah, ja nüüd on see õnnetu moment, kus sa oled otse internetti küljes kiirusega 28 kilobitti sekundis ja nurgast astus välja Antti Andreimann\index[ppl]{Andreimann, Antti}. Antti ütles, et ta on seda hetke oodanud sajandeid ja et meil koolis on kõik asjad tuksis, sest meil on ainult vana räpane ANSI kompilaator ja meil on vaja GNU C kompilaatorit, millega saaks maailma päästa. Ja viis pool tundi kõik ülejäänud ootasid, internetti kasutada ei saanud, sest Antti tõmbas GNU kompilaatorit. Kõik, kes olid pühitsemiseks kokku tulnud, ja arvasid, et nüüd saab midagi tõmmata\ldots {} See oli GNU kompilaator, mis sealt tuli! Aga ma usun, et Anttil oli lõigus, et  nii oligi ja elu läks oluliselt paremaks, sest nüüdsest õnnestus kompileerida asju, mida varem ei õnnestunud. See asi ei käinud nii, et läksid ja tõid funet.fi-ist\index{funet.fi} või kuskilt mingist pirasaidist, Unixi puhul oli teisiti. 

See kuu aega, mis koolini jäi, poisid ainult koolis elasid. Tolles augustis hakkasid ka esimesed nähtused tulema. Sellise kogemusega, kui sul on juba nihuke interneti kiirus,  siis sa paratamatult satud mõne kõrgkooli bugistesse veebidesse ja\ldots Kuidas ma nüüd ütlen. SCO Unixil\index{Unix!SCO} oli üks knihv, kuidas  pidi modemit konfima ja kui sa seda knihvi ei teadnud, siis \emph{carrier detect} jäi püsti, kui sa meie kehva telefoniliini pealt maha kukkusid. Aga, ütleme niimoodi, et  Peda ja TTÜ adminid kukkusid oma ruudu \emph{prompt}-i otsast päris sageli maha. Ja kuna meie poisid teadsid seda nippi, siis mõnikord sai niiviisi toimiva terminali otsa kätte. Noil aastatel juhtus igasugu artefakte. Seadusi hakati tegema alles hiljem, 1995 vist hakati ja 1996 kuulutati välja. See oli ka mingi töögrupp, ma mäletan, ma käisin Tartus Riigikohtu\index{Riigikohus} omadega midagi arutamas. No mis ma oskasin neile rääkida! Nii palju, kui ma olin netist lugenud, et kes läänes mille kust ära varastas, ega ma ka rohkem ei osanud. Oma õpilaste pealt ka midagi. 

\question{Siis sa puutusidki esimest korda infoturbega kokku?}

Nojah. Kas oli vist 1995\sidenote{Vastav lugu Äripäevas, kus ka Antot tsiteeritakse, on pärit 1996. aasta aprillist.}, oli see kuulus \emph{noname} lugu. Mingi pisipank oli meil Eestis, EVEA pank\index{EVEA Pank}? Igor Švets\index[ppl]{Švets, Igor} töötas seal süsadminina. Ta oli tehniliselt väga kvaliteetne häkker, kolistas kogu Eestit pidi ringi, kuhu aga sisse sai. Mina ei tea, kellele ta töötas. Iseendale või oma rahvuslikule päritolule või, nojah, hea küll, ei võta seda teemat ette. Ega mina mingisugune kõva käsi ei olnud, aga ma vähemalt olin võimeline aru saama, et keegi kurat elab mul masinas. Eks poisid natuke aitasid ja ega Sven Turnau\index[ppl]{Turnau, Sven}, ametlik masinaülem, polnud ka loll mees. Sai aru, mis toimub ja siis sai see skandaaliks keeratud. Ajalehes ära trükitud, kuidas \enquote{\emph{mister noname}},  Postimees tegi temast mingit noaga pildi. Press on press. Politseisse polnud üldse mõtet helistada, need ei saanud aru sellest. Tollal polnud ei paragrahve ega midagi. KAPO-sse helistasin ka, need ei saanud ka aru. Kõvasti on nad ikka arenenud ja õppinud! 

Aga jah, mingi hetk me suutsime iseennast ära kaitsta, saada aru, et see värk on tegelikult olemas. Keerulisem, kui enda kaitsmine, oli tegelikult see, kuidas poisse seiklustest eemale hoida. Üks tüüpiline lugu oli selline. Poisid tulevad ja ütlevad, et \enquote{meil juhtus õnnetus}. Ma küsin siis, et mis õnnetustel teil siis täna juhtus. \enquote{Me saime kogemata Pedas ruuduks!} Pedas\index{Peda} oli Cadmus\index{Cadmus}\sidenote{Saksa arvutitootja Periphere Compter Systeme (PCS) toodetud tööjaam, mis käis toona kehtinud võimeka riistvara kolmandatesse riikidesse eksportimise embargo alla. Kuidagi aga õnnestus ühte Teaduste Akadeemia allasutusse selline masin hankida, kus tema nimeks sai konspiratsiooni huvides \enquote{Muscad}\index{Muscad}.}, mingi tumba  eks ole. Eks nad läksid netti ja küsisid, et mis bugid tal on, ja loomulikult tal olid ja  nad said ruuduks ja \enquote{mis me nüüd teeme?} Ma ütlesin, et \enquote{mis siis ikka, parandage bugid ära ja kinkige masin tagasi}. Parandasidki bugi ära, kompileerisid talle uue kerneli ja ma ei mäleta, mida veel (ma üldse imestan, et nad seda masinat õhku ei lasknud) ja saatsid süsadminile lõpuks kirja, kus nad seletasid, et nad on ta bugid ära paiganud ja et ta võib nüüd oma masina tagasi saada. See oli niisugune hästi tüüpiline juhtum.

\question{Need olid mingi 12. klassi poisid kes seal möllasid?}

Ei olnud. See aktiiv, kes üldse tunde ei saanud, olid mingid üheksandikud või kümnendikud. See oli lihtsalt niisugune tore aastakäik. 

\question{Ehk, üheksakümnendate keskel üheksandikud võtavad ruutu?}

Jaa! Mõnest asjast ei julge tänaseni rääkida, inimesed on ju \emph{scene} peal alles. Ikka juhtub õnnetusi ja, ütleme niimoodi, et oli ka üks poisslaps (ta on ka siin\sidenote{Vestlus toimus Cybernetica\index{Cybernetica} kontoris.} töötanud, vahet pole), kelle ema oli teadustöötaja.  Ma ei tea, kes ta isa oli, aga tundus, et igatahes seal peres kas oli raha rohkem või saadi aru, mis on oluline.  Ja temal oli 486 tollel ajal. Ta ei olnud meie kooli poiss,  käis külaliseks. Aga näiteks tema, nähes SCO Unixit\index{Unix!SCO} ja seda igavest muret, mis seal seerianumbritega oli, kirjutas programmi, mis neid seerianumbreid nagu küllusesarvest väljastas. Ta ei olnud veel keskkooli ära lõpetanud. Tema oli ka esimene,  kes Jack the Ripperiga\sidenote{John (mitte Jack) the Ripper on tuntud paroolide murdmise töövahend.} õppis ringi käima. Tallinnas Tehnikaülikoolis\index{Tallinna Tehnikülikool} oli SUN-i klass. Mingid vanad hädaabi masinad, Rebane\sidenote{Tõenäoliselt peab Anto silmas Enn Rebast\index[ppl]{Rebane, Enn}} oli arvutiklassi pealik. Aga tollel ajal tsentraalne lahendus oli NIS-i peal, tänapäeval see sõna ei ütle mitte midagi, aga tolles lahenduses oli paroolifail kättesaadav. No ja, eks ole, see räägitud kodanik, kellel oli natuke parem masin (tal oli kodus ka SCO Unix\index{Unix!SCO}, kujutad ette!), lasi Jack the Ripperi käima ja, kuidas ta ütles,  kolme päevaga sai vist 70 protsenti paroole lahti, mingi niisugune lugu. 

Olid ajad, olid ajad!

Ja siis oli ju veel  telefonide läbi helistamine. Modemeid oli maru palju, aga turvat mitte mingisugust. Sihuke proge oli nagu Tone Loc. Nagu sa praegu pingid läbi mingisuguseid IP aadresside plokke, tollel ajal sa selle progega nii-öelda pingisid läbi telefoninumbreid.  Kohalik kõne ei maksnud ju midagi. Nii et mina olen ka korra elus näinud olukorda, kui  ühest autost, kus on kolm laptopi, lähevad krokodillidega juhtmed öösel telefonikappi ja  lastakse kümnetuhandesi plokke läbi. Sinna etappi jäävad ju Eesti Telefoni digikeskjaamade esimesed häkid. Mingid transpordikooli poisid said aru, kuidas digijaam töötab ja said aru, et paroolid on nirud ja detsembris, kui kõik olid kas jõulupühal või välismaal komandeeringutes või kuskil mujal, siis võeti päris mitu jaama üle. Ka midagi hirmsat ei tehtud, kust oligi aru saada, kellega tegemist on. 

\question{Kuidas neid poisse siis kantseldati? Neil pidi ju olema mingisugune eetiline arusaam, et \enquote{pätti ei tehta ja puruks ei lasta}?}

Ma ei kujuta ette, kes need seal transpordikoolis olid. Ma mõnda õpetajat tunnen, aga see oli ka selline pikantne teema, sest, vaata, kui sa liiga targaks saad, kui sa saad lõpuks nagu täpselt teada, kes see oli, siis seal tekivad omal moraalsed kohustused. Mina lahendasin selle nii nagu mulle õige tundus. Ma püüdsin neid suurest jamast eemale hoida ja ega seal see tavatarkus, et \enquote{mis sa raisk nüüd tegid},  ei aidanud.  Sa pidid ta ära kuulama ja siis kuidagi väga ettevaatlikult õige tee peale tagasi patsutama. 

\question{Sest tol hetkel sihukesel murdeealisel poisi nii-öelda võimekus on märksa suurem, kui mõistus või arusaama elust!} 

No võimekus on tal suur sellepärast, et ta veel õlut ei joo, selle peale aega ja energiat ja raha ei kulu ja tütarlastega ka veel ei semmi ja selle peale ka raha ei kulu. Järelikult, kui parasjagu pole koolipäev, siis on 16 tundi aega, sest kaheksa tuleb magada. Ja kui sul on inimene, kellel on 16 tundi päevas aega istuda ja murda, siis see on väga toores jõud. Ja kui neid veel mitu on ja nad omavahel suhtlevad\ldots

\question{Kaua sa seal koolis olid?}

Vot nüüd ma jään nende aastanumbritega\ldots Kolm aastat olin ma seal kindlasti. Mingil hetkel, kui sai seda EENeti\index{EENet} keskust hakatud tegema, kujunes teiseks tööandjaks EENet. Ma võin praegu eksida, millal see oli, äkki oli juba 1994 lõpp, äkki oli 1995 algul. See minu üle kandumine EENeti toimus kuidagi väga sujuvalt. Ametikoha nimi oli \enquote{insener}. Põhja regiooni oma, põhimõtteliselt ma vastutasin  tipphetkedel mingi 150 kooli UUCP side eest. Öösel kell 11 helistab mõni tädi,  kooliõpetaja, ja räägib, kuidas tal mitte miski ei tööta ja  siis sa pead peas looma mentaalse mudeli ja nagu helpdesk  tema katkendite järgi tegutsema. \enquote{See sinine lätakas} --- ahah, Norton Commander, selge. Kui sa seda, Apollo lugu tead, kuidas neil seal midagi plahvatas ja maa peal tükikestest\ldots {} Vot samamoodi. 

Sel hetkel sai tehtud ka mõned koolitused,  olid mõned uued metoodikad. Käisime, see oli küll vist 1995, Anne Villemsi\index[ppl]{Villems, Anne} juures Tartu Ülikoolis\index{Tarttu Ülikool} koolitust tegemas. Sissehelistamiskeskuseks oli meil minu läpakas. Telefoni keskjaamast tõime pikad otsad (kaks päeva oli ette valmistamist), mida sai klassis iga arvuti juurde viia. Ja  kooliõpetajatega koolitust alustasime niimoodi, et kõigepealt käskisime neil arvutit lüüa.  Noh, et tekiks kohe selline õige tunne, et saaks hirmust üle, \enquote{kes on peremees?}. Tegelikult oli klassi valdajaga kokku lepitud, mis juhtub, kui mõni arvuti \enquote{ära lüüakse}. Aga nad nii kõvasti ei julgenud lüüa. Teine asi oli, kuidas hoida neid tädisid ülearuse info eest. Tollel ajal ju modemi konfigureerimine, eks ole, baasaadress oli mingisugune 0H370, ma ei mäleta, mida iganes. IRQ2 ja IRQ3. See on mõttetu! Ja mina nii ütlesingi, et \enquote{ära süvene sellesse, mis see on, siin on täpselt ainult nii palju valikuid. Vali neist üks ära!}. Ja tulemus oli siis see, et kõik kes seal koolitusel olid, said oma modemid konfitud. Läksid oma koolidesse laiali ja said seal ka hakkama. Aga selle ette valmistamine oli 50 ühele: ühe tunni hoolitsuse kohta, mis nad said, läks 50 tundi ettevalmistust. See on üks kõige jubedamaid asju olnud, aga metoodika mõttes toimis! 

Ants Reili\index[ppl]{Reili, Ants} suri ära, ma ei mäleta enam seda täpset surmapäeva, ma mäletan, et ma Keilast taksoga jõudsin täpselt selleks hetkeks kohale, kui ta Keila surnuaial maa sisse pandi. Aga siis lagunes ETEK ära. Me kauplesime näiteks endale Tehnikaülikoolist\index{Tallinna Tehnikaülikool} Jüri Kaljundi\index[ppl]{Kaljundi, Jüri} käest vana Väksu\index{VAX}. Ainult et transaga midagi sai pihta, ei läinud enam tööle. Ja kooli direktor tundis väga huvi, et kust ma selle masina toiteks elektrit kavatsen võtta, nii et käima ta ei läinudki, läks lõpuks kuhugi utiili. Ühesõnaga, see \emph{business} vajus seal sellisel kujul koost ära. Tipphetk oligi seal enne mind natuke aega, minu ajal ja mina kandusin rohkem nagu EENeti. 

Nii et enne, kui ma 1997. aastal järgmist korda mingi pöörde tegin, istusin  Tatari tänaval Microlinki kõrval üleval (seal oli üks tuba EENeti kontor), kus siis üks Fix'i kunagine helitehnik, Antti Andreimann\index[ppl]{Andreimann, Antti} ja  mina  kolme peale kokku tegime, mida me siis suutsime teha. Antti sai aru, mis tegelikult toimus ja suutis kernelit kompileerida ja meie tegime mingeid muid asju. Sellega see koolide saaga lõppes ära ja siis tekkisid juba mingid Tiigrihüpped\index{Tiigrihüpe} ja asjad, sellisel kujul lähenemist ei olnud enam vaja. 

\question{Ühesõnaga sa jõudsid infoturbe juurde kuidagi väga praktiliselt ja samas väga inimesekeskselt. Sa pidid tegelema nende kuramuse kaakide ohjeldamisega, kes 16 tundi päevas igale poole auke torkisid} 

Ei, mis nüüd kaagid, nad on kõik lugupeetud inimesed, nad on kõik skeene peal ja minu teada pole\ldots {} Vat ühe suhtes ei tea ja üks töötas siin aastaid ühe asutuse adminina  ja ma ei teagi täpselt, mille eest ta sealt lõpuks lahti tehti. Aga kellegi teise asjus ei ole midagi paha kuulda, kõik on väga lugupeetud inimesed. Osad on siin üldse mingis sõjaväes kuskil x riigis ja nii edasi. 

\question{Mis sa praegu teed?}

Näed, me siin istume mingis Cybernetica-nimelises AS-is\index{Cybernetica} ja ma olen hetkel kirjaneitsi. Vot see on maru raske küsimus, et mida sa oskad. Alati, kui tööle võetakse küsitakse, et mida sa oskad. Ja mina mõnikord, mitte alati, aga mõnikord, oskan mõne asja suhteliselt selgelt kirja panna. Siin majas neid segaseid asju, mida on vaja pisut selgemini kirja panna on suurtes kogustes. 

\question{Vot seda sa oskasid küll väga selgesti öelda, ma sain kohe aru, mida sa teed} 

Sõltumata sellest, mis selle töökoha nimetus on, milline see järjekordne käimasolev projekt on, aga, jah, sedamoodi. Mingi natuke teine vaade, natuke parem sõnastus ja mõni järgmine seltskond saab sellele tuginedes juba järgmise lati ära võtta, mis iganes see siis oleks. Mõni asi kuhugi riiki maha müüa või midagi muud. 