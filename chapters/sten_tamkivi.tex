\index[ppl]{Tamkivi, Sten}

\question{Kas sa oled Tartu poiss?}

Olen Tartus sündinud ja kasvanud. 

\question{Sa oled natuke noorema põlvkonna inimene, kui teised, kellega rääkinud 
olen. Mis aastal sa sündisid?}

1978. Olen noorem jah, sest kui olen memcpy saateid kuulanud, siis minu jaoks 
enamik saates esinenuid olid siis juba legendaarsed ja \emph{established} nimed, kui 
mina 1980ndate lõpus arvutite ja interneti juurde jõudsin. Mõnega neist olen hiljem tuttavaks saanud ja avastanud näiteks, 
et ohoo, Madis Kaal\index[ppl]{Kaal, Madis} on päriselt ka olemas 
ja ei olegi nii palju vanem, kui ma arvasin.

\question{Palun räägi oma kujunemisloost. Mõned on olnud hirmsad 
olümpiaadihundid, teisi on huvitanud raamatud. Kuidas sina 
arvutivärgi juurde jõudsid?}

Oli paar erinevat suunda. Esiteks olen pärit teadlaste 
suguvõsast, isa ja vanaisa olid mõlemad füüsikud ja ma kasvasin üles 
Tartus füüsika instituudi järgi nime saanud FI rajoonis. Mis 
tähendab seda, et kõik lasteaiakaaslased ja naabripoisid, kellega õues 
mängisin, olid kuidagi füüsika instituudiga seotud. Oma osa mängis ehk ka see, et 1980ndate lõpus, 1990ndate algul pani instituudi elektroonik majadesse piraatkaabeltelevisiooni ja 
instituudis sai ligi esimestele arvutitele. 

Teiseks õppisin ma Miina Härma Gümnaasiumis\index{Miina Härma 
Gümnaasium}, 1985. aastal 1. klassi minnes oli see Tartu 2. 
Keskkool\index{Tartu 2. Keskkool|see{Miina Härma Gümnaasium}}. See kool 
oli mõnes mõttes Tartu tollastest esikoolidest humanitaarsem. 
Treffner\index{Hugo Treffneri Gümnaasium}, toonane 1. keskkool, oli 
selgelt reaalainete kallakuga, kuigi ka Miina Härmas oli 
reaalainete suund täiesti olemas. Käisin isegi olümpiaadidel. 

Üks varasemaid pilte kuskil vanemate või vanavanemate fotoalbumis 
on mustvalge foto sellest, kuidas keegi tõi 2. keskkooli 
algklassilastele näha kooliarvuti Juku\index{Juku}. Ka mustvalgel pildil 
on juba näha, et silmad läigivad arvutit lähedalt nähes.

\question{Ühesõnaga sinu esimene arvutikogemus oli FIs?\index{Tartu 
Ülikool!Füüsika Instituut}}

Ma arvan küll. Tekkis selline \enquote{mina ja arvuti} aeg, kui isa 
lasi mind ühel õhtul kellegi kabinetti. Neid kohti oli veel, 
näiteks pinginaaber Amish Mody\index[ppl]{Mody, Amish} isa töötas Tartu Ülikooli raamatukogus\index{Tartu 
Ülikool!Raamatukogu}, kus saime arvuti taga käia, ja hakkas 1990ndate alguses 
isegi arvutikaupu müüma või kooperatiivipoodi pidama. 

Amishil oli kodus ka arvuti. Esimene koht, kus ma Amigat nägin, oli aasta või kaks noorema 
koolivenna Lemmit Kaplinski\index[ppl]{Kaplinski, Lemmit} juures, kelle isa oli
kirjanikuna ilmselt kuskilt maailma pealt Amiga kätte saanud. Ka 
Tähetornis\index{Tartu Tähetorn} olid mingisugused kuulsad arvutid. 
Noorte Tehnikute Majas\index{Tartu Noorte Tehnikute Maja} olid 
Yamahad. Kui niimoodi loendama hakata, siis võibolla 
iseloomustabki toda perioodi see, et kellelgi ei olnud püsivat kohta, vaid 
otsiti erinevaid võimalusi arvuti taha saada.

\question{Neid kohti oli Tartus palju!}

Tartu on nagu Eesti Cambridge või Berkeley, kus 
suhteliselt väikeses asulas on nii domineeriv ülikool. Kogu internet 
hakkas ju akadeemilistest võrkudest peale. Mäletan, et Tartu püsiühendused 
olid ka satelliidiga Tähetornist Rootsi. 
Ehk enne, kui tekkis Tallinna-Tartu link, tekkis Tartu link mingisse Rootsi 
ülikooli. 

\question{Kui said \enquote{sina ja arvuti} aega, siis kui 
palju ja mis õpetust sulle anti, mida arvutiga teha?}

Küllap see muster on enamikul inimestel täpselt sama: alguses 
tahad mängida, siis aru saada, kuidas mänge tehakse, ja seejärel hakkad natuke 
programmeerima. Mina hakkasin 9. klassis 15aastasena pärast kooli 
programmeerijana tööl käima. Tolleks hetkeks olin 
kaks-kolm aastat omal käel progenud. Isa oli Tartu 
Teaduspargi\index{Tartu Teaduspark} asutaja ja seal tegutses 
mitmeid tehnoloogiafirmasid. 

Ilmselt ta küsis, kas keegi leiaks poisile mingit kasulikku 
tegevust, ja leidus sihuke hulljulge mees nagu Valentin 
Abramov\index[ppl]{Abramov, Valentin}. Teatavasti toimus 1990ndate alguses 
Eestis ohjeldamatu metalliäri. Tartus oli metallikonglomeraat 
Primex ja sellel tütarfirma Primex Data\index{Primex Data}, kus 
tehti igasuguseid asju, põhiliselt pandi PC kloone kokku. Mõned inimesed progesid projektijuhtimistarkvara, teised
raamatupidamistarkvara, näiteks Tarmo 
Tali\index[ppl]{Tali, Tarmo}. Valja\index[ppl]{Valja|see{Abramov, 
Valentin}}\sidenote{Valentin Abramov.} palkas mind nii-öelda programmeerijaks, 
aga tegelikult oli see selline koolipoisi pärastlõunane ajaviide. Midagi 
vist progesin ka, aga ma ei usu, et sealt midagi \emph{production}'isse 
jõudis. Küll aga haldasin kohalikku arvutivõrku ja aitasin arvuteid kokku 
panna, lisaks hakkasin BBSi pidama -- oli selline tee-mida-tahad maailm.

\question{Järelikult pidi sul olema piisavalt alust väita 
end programmeerija olevat. Kas sa õppisid raamatute järgi? Internetti ju 
veel ei olnud.}

Ega raamatuid ka ei olnud päris alguses saada. Käisin hooti
Noorte Tehnikute Majas\index{Tartu Noorte Tehnikute Maja} 
arvutiringis, aga korralikku progemise algharidust sealt siiski ei saanud. Mäletan, et 
olen kirjutanud ka paberi peal koodi, kui parajasti 
ei olnud ühelegi arvutile ligipääsu, aga üritasin pabermaterjalide pealt 
midagi tuletada või teha, enne kui jälle arvuti taha sain. Mäletan sellist asja nagu
\enquote{Arvutustehnika \& Andmetöötlus}\index{Arvutustehnika \& Andmetöötlus}\sidenote{Vt lk \pageref{sisu:aa}.}. 

\question{Kui praegu on küsimus, 
kuidas lapsi programmeerima õpetada, siis nendest lugudest tuleb läbivalt
välja, et keegi ei oska öelda, kuidas tollal programmeerima õpiti. See kuidagi imbus 
õhust või läbi naha. Kuidas see nii on?}

Olen mõelnud, et mis puudutab 
nii-öelda kooliarvuteid, nõukogudeaegseid Agate\index{Agat}, 
Yamahasid\index{Yamaha} ja Jukusid\index{Juku}, siis seal oli 
ikkagi arenduskeskkond esimene asi, kuhu ennast alguses sisse 
\emph{boot}'isid. Suhteliselt raske oli arvutit kasutada ilma
arendusvahendite otsa komistamata. iPhone'i puhul pead kurja 
vaeva nägema, et saada üles keskkond, millega saaksid midagi teha. See 
on kindlasti niisugune muutus. 

Kooliarvutite ajastu oli nii 
palju põnev, et kui mu onu ostis endale DOSiga \emph{laptop}'i, ilmselt Compaqi, siis käisin tal külas seda kasutamas. Aga kuna sellel ei olnud ühtegi arendusvahendit, siis no mida sa 
teed seal? Kaua sa seal DOSi \emph{directory} puus ringi surfad, see ei ole väga huvitav. See oli selline äriarvuti, tekstiredaktori ja muude asjadega. Ja see oli
esimene kord, kui 
sattusin kasutama arvutit, mis ei olnud ennekõike arendamiseks mõeldud. 

Kui me sinuga kunagi tuttavaks saime, sattusin sulle Võrru 
külla, kui sa olid laenanud kooli arvutiklassist suveks koju ühe 
Agati\index{Agat}. Agat oli ekstreemne juhtum -- selleks et üldse midagi teha, pidid kõigepealt heksis või BASICus 
sisestama koodi, et saada \emph{prompt}, kuhu saaks hakata midagi 
kirjutama. Kui koolipoiss istub suvel arvuti taga ja toksib sisse 
kuueteistkümnendsüsteemis koodi, et arvutiga saaks midagi 
mõistlikku teha, siis on tema suhe arvutiga selgelt teistsugune kui 
lihtsalt meedia tarbimine. 

Lisaks teeb arvutite lihtsus või piiratus (kui 
visuaalne mängumaa on 25 rida korda 80 tähemärki või hiljem EGA- või 
VGA-graafika) selle kättesaadavamaks. Nii palju rohkem jäetakse 
fantaasia hooleks, et laps suudab tegelikult ka midagi progeda. Ehk kui keegi 
teeb tekstirežiimis mängu, siis see ongi nii-öelda selle arvuti tipptase. 
Kui täna keegi võtab koduse mängu-PC ja teeb seal midagi 
tekstirežiimis, siis \dots Ühesõnaga, kõik, mis ei ole tohutult videokaardi 
võimalusi kasutav 3D-renderdus reaalajas, tundub nüüd naeruväärne, aga tollal 
kõik see, mida suutsid ise oma kätega teha, ei olnud naeruväärne. 

\question{Kas sul oli ka raamatute või muusika vastu huvi? 
Selles seltskonnas, kus sa liikusid, pidi ju liikuma ägedat ingliskeelset 
kirjandust.}

Liikus ikka. Miina Härma\index{Miina Härma 
Gümnaasium} oli selles suhtes 
äge kool, et enamik asju, mis seal toimusid ja jälje jätsid, olid pigem 
seotud 
teiste õpilastega. Koolibände oli kõvasti, ma ise üheski küll ei olnud. 
Näiteks 1990ndate lõpus tekkinud Bizarre\index{Bizarre}, mille liikmetega ma väga palju hängisin 
ja osadega 
siiamaani läbi käin, oli Miina Härma koolibändist välja kasvanud. Otsapidi ka väga elektrooniline ning avas minu jaoks arvuti 
ja muusika seoste maailma. 

Raamatuid loeti, aga küberpungi ja \emph{science fiction}'i 
juurde jõudsin mina 1990ndate teises pooles, kui Ameerikasse 
sattusin. Enne seda lugesin pigem Tolkieni \enquote{Kääbikut} kui 
\emph{science fiction}'it.

\question{Kas Primexis programmeerijana töötasid enne, kui Ameerikasse 
läksid?}

Jah, see oli aastal 1993.

\question{Miks
Primexis seda softi progeti? Kas enda tarbeks või taheti sellega äri 
teha?}

Eesti IT-tööstuse ajalugu on käinud paari lainena. 
1980ndate lõpul, 1990ndate alguses, kui alustati 
täiesti tühjalt lehelt ja kõigil oli arvuteid vaja, siis kõik tõid juppe ja panid 
arvuteid kokku. Oli Primex Data\index{Primex Data}, kuskil kõrvalmajas
Ordi\index{Ordi}, samuti Astrodata\index{Astrodata} ja Tallinnas Microlink\index{Microlink} -- kõik tegid sama asja. Siis liiguti 
tasapisi tarkvara kihti. Riigil pigem raha ei olnud, pankadel oli, aga 
võibolla huvi ei olnud, ja tekkisid firmad, kes arendasid tarkvara teenusena. Kogu 
see Helmeste ja Webmediate laine on selle kõige tugevam näide. 
Tänaseks on \emph{mainstream}'iks muutunud toodete ehitamine. 

Primex Data oli naljakas hübriid. Ühest küljest oli seal arvutiäri, 
mida tegid ka kõik teised ja kust tuli põhiline käive. Teisest 
küljest hakati tarkvara tegema ikkagi tootena, tehti asju, mida loodeti ilmselt flopi peal müüa. Tekkis mingisugune turg ja sellised raamatupidamise tarkvara firmad nagu
Merit Tarkvara\index{Merit Tarkvara}\sidenote{Ehitab raamatupidamis- ja 
personalitarkvara, tegutseb siiani.} ja 
Eetasoft\index{Eetasoft}\sidenote{Ehitab Eeva-nimelist 
raamatupidamistarkvara, tegutseb siiani.}, 
osa neist tegutseb siiamaani. Projektijuhtimise tarkvara tegid kaks 
progejat nimega Urmas ja Jürgen, kes minu teada kirjutasid sellest samal ajal Tartu 
Ülikoolis oma magistritööd. Projektijuhtimine, Gantti graafikud ja selle kohta 
eestikeelne tarkvara -- ilmselt selline akadeemiline asi, mida nad 
lootsid ka müüa. Samas ma ei mäleta, et sellest oleks suurt äri 
tekkinud. 

Sellise hübriidi pidamine oli suhteliselt jabur. Ma ei mäleta, kas see juhtus minu 
või Tarmo Tali\index[ppl]{Tali, Tarmo} arvutiga või mõlemaga (me 
istusime kõrvuti), et tuled pärastlõunal tööle ja hakkad
progema, aga selgub, et keegi on su arvutist mälu ära müünud. Ja siis 
tegeled alustuseks sellega, et enne oli kaheksa megabaiti mälu, aga äkki leiab kuskilt
laost neljamegabaidise mooduli.\sidenote{Nii oli ka Korelis. Vt lk \pageref{sisu:jupimyyk}.}

Üks näide tarkvaramaailma läbipõimumisest muu maailmaga: ühel päeval istusin ja 
nokitsesin midagi teha ja sisse astus Jaan Tallinn\index[ppl]{Tallinn, Jaan}, 
kellest ma olin kuulnud. \enquote{Kosmonaut} ja nii, legendaarne mängutiim! Ja Jaan tuli 
monitori ostma! Kuigi olin riistvara müügiga tegelenud, siis põhikooli- või keskkoolipoisina käed värisesid, 
jube põnev oli!

\question{Seda on mitmest loost kosta, et tol ajal oli arvutiäri nii võimsa
marginaaliga, et selle kõrval sai näiteks pidada tudengeid 
projektijuhtimise tarkvara kirjutamas. Ilmselt kuulub siia alla ka sinu amet?}

Täpselt. Tänu sellele võtan ma kindlasti täna märksa
parema meelega endale praktikante, interne ja töövarjusid. Kui palju mind see 
võimalus mõjutas, mille Valja \index[ppl]{Valja} mulle 
andis! Nad maksid mulle isegi palka, aga see oli selgelt olukord, kus poleks 
mingit vahet olnud, kui nad ei oleks mulle palka maksnud.

Teiseks oli seal huvitav, et sattusin esimest korda võrkude 
juurde. Enne seda oli arvuti nagu iseseisev, eraldiolev asi. Täpselt umbes 1993. aastal ilmus
Tartusse internet ja Primex Data 
või vähemasti Teaduspark\index{Tartu Teaduspark} võis olla esimene koht, kus 
mitteülikooli asjad sattusid võrku. See oli jadaühendusega võrk, mille otsas oli 
terminaator ja said ilge siraka, kui arvuti oli maandamata. Välja nägi see võrk
nii, et kuskilt läbi seina tuli üks ots ja sa ei teadnud, mis 
masinad selles jadas veel on. Kõik olid ühes võrgus maja peal laiali. 

Meil oli 
1993. aastal püsiühendus internetti ja Lynxi-põhine veeb enne, kui Mosaic\index{Mosaic} 
välja ilmus! Ühest otsast pidasin Primex Data nime all BBSi ja teistpidi 
oli meil olemas püsiühendus, kust oli võimalik 
\enquote{Doomi}\index{Doom} demo või \enquote{Wolfenstein}\index{Wolfenstein 3D} 
FTPga kätte saada ja BBSi üles panna. Paljude teiste 
BBSide jaoks oli failide levitamine modemiga 
\emph{peer-to-peer} loksutamine, aga meie olime nii-öelda pumba juures. 

Teise näitena sellest, kuidas arvutifirma hobina tehes hoopis 
teistsugune välja nägi, meenub meie sisevõrk. Selles oli Novelli server, 
millel oli 300-megabaidine kõvaketas\sidenote{300 megabaiti oli ulmeliselt suur 
andmemaht. Tol ajal oli normiks 3,5" flopiketas mahutavusega 
1,44 megabaiti. Laial kasutusel olid ka 5,25" kettad mahutavusega mõnisada 
kilobaiti, milledele kääridega(!) lisasälk lõigati, et oleks võimalik 
salvestuseks tarvitada ketta mõlemat poolt nii mahtu kahekordistades. Neid kettaid kanti spetsiaalses  
karbis endaga kaasas, sellisesse karpi mahtus kogu kellegi digitaalne elu.}, millest 
tööasjadeks (kood, mida seal 
kirjutati, ja hinnakirja Wordi fail) oli kulutatud umbes paarkümmend megabaiti. Ülejäänu laadisid 
mingisugused Hollandi tüübid öösiti tarkvara täis, sest 
kuskil Ida-Euroopas, kus ei olnud ka intellektuaalse 
omandi kaitset, oli uks lahti tehtud \ldots

\question{Tollal polnud Euroopaski isegi ligilähedaselt nii rangeid 
intellektuaalse omandi reegleid, rääkimata Eestist!}

Jah. Hommikul tuulasin tolle ketta läbi ja vaatasin, mis asju BBSis ülejäänud 
Eestiga jagada.

\question{Päris nii ikka ei olnud, et avad FTP pordi või telefoninumbri ja 
muudkui hakatakse väärt kraami laadima. Sul pidi järelikult mingi võrgustik olema. Kuidas 
see tekkis?}

Usenet ja selle uudisgrupid võis olla esimene rahvusvaheline
kogukond, kuhu ma sattusin. Eesti Fidoneti grupid ka. \emph{Overlap} oli ilmselt täitsa olemas, kuna Eesti Fidoneti 
gruppides juba arutati, kus internetis käia ja kus keegi istub ning tarkvarale 
ligi pääseb. Minu jaoks tulid reaalajas 
\emph{chat-room}'id, Random\index{Random} ja teised 
jututoad, hiljem. Istusin ka IRC kanalites, aga ma ei mäleta, et sealt 
oleks tohutu side või sõpruskond tekkinud. See oli rohkem 
\emph{ad hoc}. 

\question{Kuidas sul üldse tekkis mõte hakata BBSi pidama\index{Primex 
Data}? Kas sulle anti tööülesanne?}

See on hea küsimus! Mäletan, et idee müüsin küll Primexile maha väitega, 
et jube kasulik on tärkavas võrgus nähtaval olla. Pagan teab, võibolla oli 
laos olemas modem ja tekkis küsimus, mida sellega teha saab. 
Kodus ei olnud mul arvutit 1990ndate lõpuni. Ei olnud nii, et 
oleksin olnud BBSide kasutaja ja nüüd tekkis võimalus üks ise püsti panna. 
Ju see oli ikka sedapidi, et laos oli modem ja sellega sai helistada sisse 
teistesse BBSidesse, mis juba Tallinnas olid.

Enne kui sa külla tulid, hakkasin mõtlema, et mäletan jätkuvalt Fidoneti 
\emph{node} numbrit peast: 2491/2.2. Tartu tsoon oli vist 
kaks, kõik Tallinna asjad olid ühe all. Ja siis selguski, et Tartu tsoon on veel 
suhteliselt hõre ja on võimalik olla teine BBS Tartus. Esimene oli vist Jaan 
Pruulmann\index[ppl]{Pruulmann, Jaan}.

\question{Järelikult oli Primexi BBS olemas enne 
Luciferi\index{Lucifer BBS} oma?}

Mis tema number oli? Veikot\index[ppl]{Tamm, 
Veiko}\sidenote{Veiko Tamm, Lucifer BBSi pidaja. Vt lk 
\pageref{chptr:lucifer}.} tundes võis see olla 666, kui ta endale
sellise nime pani. Võibolla need numbrid ei olnud puhtalt lineaarsed. 

\question{Kas sinu BBSil oli üks liin ja üks modem?}

Jaa. Alguses oli 2400boodine modem, hiljem kiirem.

\question{Kui olen küsinud, mida inimesed BBSis hoidsid, siis tavaliselt on 
öeldud, et seal hoiti endale huvitavana tundunud asju. Mis sorti kola sul seal 
oli?}

See värk oli suhteliselt kaootiline. Huvitav, kas see \emph{file 
list} oleks kuskil alles? Ilmselt olid mängud ja
\emph{utility}'d nagu kõigil teistelgi. Ressursikitsikusest tingitud
asjad, mis olid huvitavad või vajalikud, ja kiiresti arenev tehnoloogia, näiteks 
pakkimisalgoritmid, nagu zip ja arj\sidenote{ARJ (\emph{Archived by Robert Jung}) 
oli 1990ndatel väga levinud efektiivne pakkimisformaat, mis praeguseks 
on enamasti unustatud. Selle eelis oli võimekus suuri faile sujuvalt mitme 
flopi peale laiali jagada.}. Pluss nendega 
oli see rõõm, et need olid väiksed ja neid sai kiiresti liigutada, kui midagi uut 
tuli.

\question{See oli ju osa rutiinist, et esimesena asjana pidi arvuti juures 
käepärast olema mõni pakkija ja vahendid mälu laiendamiseks!}

Ruum ja modemi aeg oli ju kallis. Ja kui 
öine meili sünkimine venis tunniajaliseks, siis vaatasid, et 
võibolla kõiki neid gruppe pole vaja, mida ise ei loe. 

\question{Kui palju sul sellest aimu oli, kes sul BBSi küljes käisid? Kas
lihtsalt mingid numbrid helistasid?}

Mäletan seda minimaalselt. 
Pigem mäletan seda tunnet, et kui ise juhtumisi seal olin (ma öösiti 
käisin ikkagi kodus magamas) ja modem kõne vastu võttis, siis vaatasin
põnevusega, mida see inimene teeb. Ma ei mäleta, kas see oli 
Windows või OS/2\index{OS/2}, millega tollal katsetasin. Aga 
vist OS/2 peal sai kuskil aknas jooksutada DOSi virtuaalmasinaid või programme. Mul multitaskis see asi taustal isegi siis, kui tööd tegin.

Teine asi oli meil väljahelistamine. Minu arust istus Tartu 
Teaduspark\index{Tartu Teaduspark} linna kõige vanema telefonijaama taga, mis 
oli legendi järgi 1950ndatel püsti 
pandud analoogjaam. Üritasime teda ühte- ja teistpidi ka lahti 
häkkida. Üks asi, mis meil vist korra töötas, oli see, et kettaga telefonil on ühest kuni üheksani klõpsude arv 
ja kümme klõpsu on null. Lugesime kuskilt, et kui teed üksteist klõpsu, siis 
saad kätte kaugekõne \emph{trunk}'i. Tegime üksteist klõpsu, saime teise tooniga 
heli, helistasime mingisse Hongkongi BBSi ja minu arust ei tulnud selle 
eest kunagi arvet. Aga see lõbu kestis jube vähe, sest jaam oli tõesti 
muldvana ja kasutajaid oli ilmselt palju taga ning see vahetati esimeste seas 
digikeskjaama vastu välja. 

Üheksakümnendate lõpus, kui nägin esimest korda 
häkkeriajakirju, näiteks 2600\sidenote{\enquote{2600: The Hacker 
Quarterly}. Enamasti lugejate enda poolt sisuga täidetud kultuslik perioodiline 
ajakiri, mis käsitles kõikvõimalikke arvuti-, interneti- ja 
telefonisüsteemidega seotud tehnilisi teemasid ning üldisi 
\emph{underground} arvutiuudiseid. Nagu eespool (vt lk 
\pageref{sisu:2600}) jutuks tuli, oli 2600 Hz toonaste telefonijaamade 
jaoks oluline sagedus, sealt ka publikatsiooni nimi.}, ja lugesin 
kaheksakümnendate lõpu ja üheksakümnendate alguse \emph{phone phreaking}'u laine 
kohta USAs, siis see oli korraks relevantne ka Eestis. 

Teine asi, mis kindlalt töötas, oli see, et kui telefonikaardil (selle 
pooleaastase perioodi jooksul, kui Eesti Telefoni juurutas kiipkaardiga 
telefoniautomaadid\sidenote{Tõenäoliselt andis 
Eesti Telefon kiibiga kaarte välja aastatel 1995--2010. Iseküsimus on, kui 
palju sajandivahetuseks kaarti aktsepteerivaid taksofone järel oli.}) üks klemm 
kinni teipida, sai tasuta helistada. See oli ka tehnoloogiahuviliste 
noorte rõõm.

\question{Kust te sellest kõigest teada saite ja kuidas 2600d Eestisse 
sattusid?}

Kui said Fidonetis esimese ringi Eesti gruppidele peale ja 
lisaks tellisid mõne USA grupi või Usenetis häkkerigrupi, siis 
seal oli osa asju ASCII tekstina olemas. Paberkoopiat nägin 
USAsse minnes ja see oli šokeeriv kogemus. Ilmselt usklikel 
on mõne piibelliku teksti originaali 
juurde sattudes samasugune tunne. 

\question{Kas Ameerikasse sattusid vahetusõpilasena keskkooli ajal?}

Jah. Ma tegin avalduse Rotary klubi vahetusprogrammi, mille 
ankeet keskendus sellele, kes see konkreetne kooliõpilane on. Rotary 
vahetus oli kahesuunaline: klubi saatis kellegi kuskile välja ja 
samal ajal võttis mujalt vahetusõpilasi vastu. 
Ma ei tea, kas sellepärast, et mu ankeet oli nii arvutiasjade 
keskne, aga igatahes sattusin Eesti mõttes 11. klassi Silicon Valley keskele. Elasin
sellises õrnas eas aasta aega Cupertinos, kus on muu hulgas 
Apple'i peakontor, ja käisin Monta Vista High nimelises keskkoolis. 

Juhtumisi oli USAs just \emph{information 
superhighway} hullus puhkenud ja Al Gore oli aasta enne 
minu USAsse minekut kuulutanud viis kooli interneti 
pilootkoolideks. Monta Vista High oli üks nendest. Kool asus
Apple'i peakorterist paari kilomeetri kaugusel ja meil oli 1400 õpilast ja 800 arvutit. Selgus, et arvutilaboris 
assistendiks olemise eest saab ainepunkte ja ühe tunni asemel võis iga 
päev istuda arvutilaboris, kus olid Macid, Sunid ja Silicon Graphicsi 
asjad. Yahood kasutasin aadressil yahoo.cs.stanford.edu, sest see ei olnud siis veel 
firmaks muutunud.

\question{Ühesõnaga, sattusid paradiisi!}

Põhimõtteliselt küll jah. See mõjutas kindlasti tohutult seda, mis edasi 
sai. 

\question{Kuidas sa sellega toime tulid? Nõukogude 
liiduvabariigist sellisesse kohta sattumine võttis ilmselt jalust nõrgaks?}

Hea küsimus. See oli mu esimene lend Eestist välja. Üksi. Ilmselt vanemad 
pidid lennupileti jaoks raha laenama ja see kõik oli ka 
nende jaoks üsna hullumeelne. Aga eks vanemate 
asi ongi muretseda. Ise sellises vanuses lihtsalt teed ja lähed, 
oled nagu käsn ja võtad kõike seda sisse, mis tuleb. 

Koolikogemuse mõttes oli lihtne
see, et Eestist tulnuna ja matemaatika-füüsikaolümpiaadil käinuna olid sealses
11. klassis kõik mu reaalained 12. klassi \emph{honours}-taseme ained. Olin kõik selle Eestis juba läbinud -- koolisüsteemid olid nii palju erinevad. 
Teistmoodi oli see, et esimest korda elus pidin 
mitte üksiküritajana testi ära lahendama, vaid moodustama grupi kolme inimesega, 
kellega ma ei olnud enne koos töötanud -- midagi koos välja mõtlema ja klassi ees 
ette kandma. Eestis reaalainete tugevus oli olemas, aga selliseid asju pidin esimest korda tegema. Õppeviis oli seal 
ebamäärasem. 

Ühiskondade kontrast oli küll. Kui ma ise olin enne seda ju lausa tööl käinud, 
siis sinna jõudes küsisid vahel ka väga heasoovlikud klassivennad, kas meil 
Eestis telekaid on. Nende arusaam raudse eesriide taga 
toimuvast oli üsna hägune. 

\question{Kas käisid seal ka tööl?}

Ma ei tohtinud, vahetusõpilase värk. Alguses see kurvastas mind 
väga, aga siis sain aru, et seda defineeritakse läbi palga. Leidsin sellise lahenduse, et käisin 
ühes arvutipoes pärast kooli abiks, aga ma ei tohtinud palka saada, ja kui 
Eestisse tagasi kolisin, kinkis pood mulle arvuti. Aga mind käsitleti tõesti pigem nagu 
koolipoissi, kellel lubati arvutit kokku panna, kuna sealne IT-äri oli ju palju 
reglementeeritum võrreldes sellega, mis Eestis samal ajal toimus. 

Enne Ameerikasse minekut, 9. või 10. klassis juhtus veel selline asi, et tekkis rühmitus nimega Interactive 
Aspelungs\index{Interactive Aspelungs}, kuhu kuulusime Mina, Mark Tehver\index[ppl]{Tehver, Mark}, 
Kristjan Jansen\index[ppl]{Jansen, Kristjan} ja natuke hiljem Alari 
Aho\index[ppl]{Aho, Alari}. Mark ja Kika\sidenote{Kristjan 
Jansen.}\index[ppl]{Kika|see{Jansen, Kristjan}} olid 
Treffneris\index{Hugo Treffneri Gümnaasium}, mina ja Alari Miina 
Härmas\index{Miina Härma Gümnaasium}. Miks ma 
ei ole tänapäeval programmeerija, oli see, et mul õnnestus väga õrnas eas näha 
lähedalt inimesi, kes tegelikult oskavad programmeerida. Mark oli juba koolipoisina selline inimene, kes hommikul hakkas tekstifaili assemblerit 
kirjutama ja õhtul pani selle käima ning see töötas. Näiteks 
graafikamootor. Meil tekkis selline sümbioos, et 
Mark proges, Kika disainis, Alari tegi muusikat ja mina korraldasin igasuguseid asju, näiteks laenasin Primexist SoundBlasteri kaardi, et Mark 
saaks ka sellele audiodraiveri kirjutada. Toode, 
mida me ehitasime, oli arvutimäng \enquote{Drunkard}\index{Drunkard}.

Mina hakkasin seda mängu maha 
müüma. Olin tihedas kirjavahetuses selliste ettevõtetega nagu Epic Megagames\sidenote{Praegu tuntud kui 
lihtsalt Epic.} ja Apogee\sidenote{Praegu tuntud kui 3D Realms. Mõlemad mainitud ettevõtted olid omal ajal 
mängutööstuse absoluutsed gigandid.}, kes olid valmis meiega rääkima. 
Kui ma USAsse läksin, siis tekkis veider olukord, et sain USA 
postiaadressilt saata flopiga demosid ja jätta mulje, nagu meil oleks 
päris firma. 

Paraku ei teinud me seda mängu kooli kõrvalt lõpuni valmis, ainult demod olid 
olemas. Panime ka paari aastaga selles suhtes mööda, et kirjutasime 2D 
platvormikat, mis 1991. aastal oleks olnud selle taseme juures, mis 
Mark\index[ppl]{Tehver, Mark} ja Kika\index[ppl]{Jansen, Kristjan} välja
töötasid, ilmselt täiesti vabalt müüdav, nii nagu 
Bluemooni\index{Bluemoon} kutidki oma mänge maailma viisid. Aga meie 
komistasime täpselt sinna hetke, et kui saatsime demosid, siis ID 
Software \enquote{Wolfenstein}\index{Wolfenstein 3D} oli juba 
väljas\sidenote{Avaldati 5. mail 1992.} ja \enquote{Doom}\index{Doom} 
tulemas\sidenote{Avaldati 1993. aastal.}. Ehk graafika tase jõudis sinna, kus Tartu koolipoiste 
platvormikas ei paistnud enam säravalt silma. Aga põhimõtteliselt see asi isegi töötas.

\question{Ise mängu kirjutamine tundus toona täiesti hoomamatu ettevõtmine: 
muusika mängib taustal, kuidas sa seda teed?!}

Tundub isegi tänapäeval, aga tollal alustas iga mängukirjutaja sellest, et 
proges endale töövahendid. Nelja või 
kaheksa kanaliga \emph{tracker}'id taustmuusika tegemiseks olid olemas, mida jällegi 
Bluemoon\index{Bluemoon} tootis, ent \emph{level}'i disainimiseks või isegi 
animeerimiseks tööriistu ei olnud. Pildi-\emph{editor}'iga tegid spraidi\sidenote{Ingl 
\emph{sprite} -- kahemõõtmeline ühik rastergraafikat, mis integreeritakse 
suuremasse stseeni.} valmis, aga selle animeerimiseks pidid 
jälle oma töövahendi tegema. Meil oli kogu see \emph{stack} olemas.

\question{Miks te seda tegite? Kas tundus äge või tahtsite rikkaks saada?}

See tundus lihtsalt äge. Mark\index[ppl]{Tehver, Mark} ja 
Kika\index[ppl]{Jansen, Kristjan} tegutsesid juba enne, neil oli hoog 
sees. Mitu aastat tagasi kogus Kika kokku kõik meie tollase 
kirjavahetuse ja pildifailid ning pani avalikult internetti -- see oli hea
nostalgiarännak. 

Mängu peategelasel Drunkardil oli oluline, et alkoholitase 
veres ei langeks. Selleks pidi ta korjama pudeleid ja siis ta sai 
tühjade pudelitega loopida. Olid ka teatud olukorrad, kus ta pidi ajutiselt 
saama natuke kainemaks. Markil\index[ppl]{Tehver, Mark} ja 
Kikal\index[ppl]{Jansen, Kristjan} oli Treffneris üks klassivend, kes, kui ta 
koolipeol liiga palju õlut jõi, hakkas kükke tegema, et kainemaks saada. Ja 
Drunkardiga oli ka nii: allanoolt all hoides tegi ta kükke ja alkoholitase veres langes.

\question{Mingi loovuse element oli sees, sest ei tulnud lihtsalt joosta ja 
kirvestega loopida?}

Täpselt, meil oli idee teha selgelt teistsugune mäng, kus ei käi 
tulistamine ja tapmine. Ma ei mäleta, kui 
teadlikult me seda mõtlesime, et sihtida täitmata turunišši ja teha nii-öelda 
täiskasvanute mäng. Muidugi alaealiste idaeurooplaste arusaam sellest, mis 
on \emph{adult entertainment}, oli mõnevõrra teistsugune kui päris 
\emph{adult}'idel, aga vastas selgelt üheksakümnendate Tartule.

\question{Kas tegid Californias veel midagi huvitavat?}

Seal ma progesin ka. Ma ei mäleta, kas ma BBSi pidasin, aga 
eraisikuna koduliini peal BBSi pidamine ei ole nagu see, 
pluss satud USA telefoninumbrite ruumi. Tegelikult olin ikkagi pigem 
BBSide kasutaja. Ka graafiline veebibrauser ilmus orbiidile ja pilt hakkas 
muutuma. 

Kirjutasin hobiprojektina ka ühte BBSi softi, kuna tundus, et see 
võiks olla vajalik. Kui parajasti mängu ei proge, siis ju ikka mõtled, mida endal vaja on. Selle käigus 
uurisin, mis veel turu peal saada on, ja leidsin ühe ägeda BBSi softi, mis oli \emph{shareware} ja kinni keeratud. Ma murdsin selle lahti -- 
noorele häkkerile ei midagi komplitseeritut. Jooksutasin seda 
\emph{debugger}'is ja vaatasin, et ootamatute kohtade peal hüpatakse mälus 
ühele aadressile ja tehakse seal mingi väga lihtne tehe. Muutsin masinkoodi 
tasemel ära, et sinna enam ei hüpataks, ja ootamatult selgus, et oligi 
koopiakaitse maas. Raporteerisin sellest autorile ja ta andis mulle eluaegse tasuta litsentsi. See võttis mul oluliselt 
motti maha oma BBSi softi kirjutada, sest mul oli see nüüd olemas. 

Sain ka õudselt hea õppetunni enda tasemel progejana. 
Ma kohutavalt abstraheerisin selle asja üle: kirjutasin C++-s\index{C++} 
BBSi, kus katsusin hoida väga puhtalt eri kihtidena näiteks seda, kuidas 
käib modemi ja terminali händlimine, olles valmis selleks,
et tulevikus võib olla asju, millega liidestuda, mitmeid. 
Ühesõnaga ma olin kogu aeg jube kaugel sellest, et asi töötaks 
minimaalses skoobis. Hilisema elu tarkvaraarendusprojektideks oli see väga hea 
õppetund. Täna tegelen rohkem start-up'idega, MVP\sidenote{\emph{Minimum viable product} -- vähim elujõuline toode.} on kuidagi 
armsam. Pigem teha vähem, aga varem.

\question{Ja ühel päeval tulid Ameerikast tagasi, kaasas arvuti ja raamatud?}

Jah, paar sellist kasti oligi. 

\question{Mida sa siis tegid?}

Läksin keskkooli 12. klassi. Kuna mul oli juba natuke hoog sees, 
siis asusin ka tööle sellisesse Tartu firmasse nagu Triip\index{Triip}, mis oli 
algselt trükikoda ja disainibüroo. Mulle on eluaeg, isegi siis, kui 
progesin, meeldinud see, kuidas tehniline ja visuaalne osa kokku saavad. 
Olen alati töötanud koos progejate 
ja disaineritega, ka hiljem.

USA perioodist meenub veel üks asi, mille me tegime. Mäletad, olid kunstirühmitused 
kes tegid ASCII \emph{art}'i ja hiljem VGA \emph{art}'i, ning koos ühe koolivennaga me komistasime
varjunimede all paari neist sisse. Minu ASCII ja ANSI \emph{art} on isegi olnud mõnedes distributsioonipakkides. USAs oli üks kooliaine \emph{commercial art} ---
tootedisain ja pakend. Ilmusin nende näidistega Triipu ja ütlesin, et 
tahaksin pärast kooli natuke arvuti taga istuda, mis tähendas 
disainimist, ja siis sattusingi sinna tööle.
 
Teine mind hästi palju mõjutanud inimene tol ajal oli Marek 
Tiits\index[ppl]{Tiits, Marek}, kes pidas Balti Uuringute 
Instituuti\index{Institute of Baltic Studies}, mille alt ta tõstis\sidenote{Raha \enquote{tõstmise} all peetakse startup-kogukonnas silmas mingile ettevõtmisele rahastuse hankimist.} edukalt 
eurorahasid ja tegi ägedaid projekte, näiteks Eesti seaduste 
otsingumootori.

Marek andis ka mulle kui lihtsalt ringi hängivale 
koolipoisile võimaluse tulla ja aeg-ajalt midagi teha. See tähendas 
ligipääsu Tähetorni\index{Tartu Tähetorn} arvutitele. Seal oli üks Silicon Graphicsi\index{Silicon Graphics} 
arvuti, millel oli veebikaamera. Üheksakümnendate keskel! Arvutil oli oluline 
funktsioon: sinna oli võimalik sisse logida ja vaadata veebikaamerast, kas 
kohvimasin on täis jooksnud, et ei peaks tühja tassiga alumiselt korruselt 
teisele tulema. Samuti oli seal üks Zyxeli\index{Zyxel} modem, 
millel oli ka faksi funktsionaalsus, ja internet. Ma 
kirjutasin Perlis\index{Perl} veebipõhise faktide saatmise ja vastuvõtmise programmi: kui keegi saatis sellele numbrile faksi, siis võttis 
modem vastu, kirjutas failid Suni serverisse maha ja üle veebi oli 
võimalik neid näha. Ma ei ole kindel, kas mõtlesin ise, et võiks selle teha, ja Marek ütles, et tee muidugi, või oli see 
mõne projekti jaoks vajalik.

\question{Veel aastaid hiljem räägiti ühes suures ettevõttes, mis 
kindlasti ei olnud telekommunikatsiooniettevõte, et äge oleks 
üle interneti faksi saata!}

Minu arust kaks aastat tagasi\sidenote{Ajasime Steniga juttu novembris 2019.} 
tegid Twilio\index{Twilio} progejad aprillinaljana Twilio faksi API ja 
nüüd on see oluliselt kasvav ärisuund, sest võrgus on 
miljoneid inimesi, kes tahavad kogu aeg faksi saata. Mõtle, mis kõik oleks 
võinud olla! 

\question{Räägi palun Triibust.}

Tundsin sealt Jussi, Juhan Peedimaad\index[ppl]{Peedimaa, Juhan}, Evat, 
kes on nüüd ka Peedimaa\index[ppl]{Peedimaa, Eva}, ja Priit 
Jagomäge\index[ppl]{Jagomägi, Priit}. Jagomägede perekond on kuulus Regio ja 
kartograafia poolest, aga Priit oli selline \emph{rebel} vend, kes tegi oma firmat,
mitte ei töötanud Regios. 

Päris mitmed asjad Eestis juhtusid sellepärast, et 
meil oli täiesti tühi maa. Kui ma keskkooli 
lõpetasin, oli Eestis umbes 40 panka ja keskmine panga CEO vanus oli umbes 28. 
Kaheksateistaastaselt tekkis tunne, et oled mingist rongist maha jäänud. Tol hetkel me veel ei teadnud, et 
maandume tehnoloogia- ja internetiettevõtluse lainele. Aga oli selge, 
et see rong ei olnud veel läinud ja et saame endale ise rongi ehitada. 

Triip\index{Triip} oli täpselt selline ettevõte. Nad alustasid sellest, et kuna Eestis oli 
täiesti tühi maa, iga päev tehti kümneid firmasid ning igale firmale oli vaja logo 
ja visiitkaarti, siis hakkasidki neid tegema. Disainist tuli nii palju raha peale, 
et osteti oma trükikoda ja veel mingisuguseid asju kokku, ning siis selle 
sisse komistasime internetiasjaga. Kui algul kujundati trükiseid ja 
reklaame, siis mina hakkasin samadele klientidele ka veebilehti tegema. Ja 
kui ma keskkooli lõpetasin, siis läksin Triibust 
ära ja tegin oma esimese ettevõtte 
Halo\index{Halo Interactive DDB}, mis tegeles veebiga \emph{full time}. 

Halo aegadest oli ilusamaid mälestusi näiteks see, et ühel hommikul läksin 
kontorisse ja nägin, et üks disainer poeb valgete linade vahelt välja ---
selgus, et ta tõsteti juba kolm kuud tagasi ühikast välja. Lihtsalt mina 
arvasin, et ta on kogu aeg esimesena ööl ja lahkub viimasena. Tegelikult ta 
elas seal.\sidenote{Lugu vastab tõele ja see 
ei olnud ainus juhtum, kus Halo kontor kellelegi ajutist peavarju 
pakkus. Üheksakümnendate Tartu oli mõnevõrra teistsugune start-up'i-keskkond kui 
selle sajandi kahekümnendate Tallinn.}

\question{Sa rääkisid, et oli võimalus nii-öelda oma rong ehitada. 
Kas sul oligi selge visioon lainest ja endast selle peal või lihtsalt 
tegelesid asjadega, mis tundusid mõnusad?}

Ettemõtlemist oli selgelt liiga vähe. Halo läks nelja 
aastaga pankrotti, sest teatud piirini on võimalik asju ehitada intuitsiooni 
pealt, aga ühel hetkel peaks läbi arvutama ja mõtlema, mida sa 
teed. 

See oli mingil määral avastus, et veebilehed on asi, mida vajatakse. 
Juba Triibus hakkasid kliendiks saama ja Halos olid 
klientideks kõik suured pangad, Eestisse jõudnud rahvusvahelised 
brändid, nagu Audi, ja ESS, mis tol hetkel tohutult 
kasvas. Sellised kaubamärgid, mida kõik teadsid. Ja internetivärk 
oli nende jaoks arusaamatu. Meie jaoks oli see intuitiivne ja lihtne -- 
milles probleem, teeme ära! -- ja suured kaubamärgid olid 
nõus alla neelama selle, et ostavad mingite kaheksa- ja 
üheksateistaastaste Tartu kuttide käest teenust. Ei osanud me seda hinnastada 
ega midagi, iga kord mõtlesime, et see number kõlab liiga 
suur. Alguses kõik toimis liiga hästi või liiga lihtsalt. 

Alustasime Haloga\index{Halo} 1996. aastal, 1997. aasta kevadel ostis üks suur reklaamikett, 
DDB, kontrolli Halos ära ja me sattusime ootamatult päris 
ärikeskkonda, kus olid päris inimesed, mingid Soome juhid, kes olid nõukogus ja 
tahtsid eelarveid näha. Ühest küljest oli 
see kõik nagu varases nooruses kokkupuude päris asjadega. Teistpidi läksid ka 
need reklaamiinimesed ehk investorid
parajasti internetibuumi sisse. Ka nende kliendid loopisid raha vasakule ja 
paremale. Näiteks selline agentuur nagu Razorfish võttis 100 000 dollarit esimese 
kohtumise eest 100 000 dollarit. See tähendas, et nemadki ei 
suutnud kõike ette näha. 

Me näiteks palkasime selgelt liiga palju inimesi, 
sest kohe-kohe pidid lääne kliendid tulema Eestisse asju arendama -- meil olid 
tõsised jutud DDB keti sees. Ja kui see mull 2000. aastal 
lõhkes\sidenote{NASDAQ Composite indeks tõusis 1995. ja 2000. aasta märtsi 
(mida loetakse mulli lõhkemise hetkeks) vahel 400\% ning kukkus 2002. aasta 
oktoobriks 78\%, kaotades kogu mulli jooksul kogutud kasvu.}, siis 
kõigi jaoks korraga: lõhkes meil Eestis ja lõhkes seal. Ja seda ei suutnud
me täiesti roheliste tüüpidena ette näha. Et majanduses on 
tõusud ja langused, oli täielik müstika. 

\question{Sest sa ei olnud ühtegi varem näinud!}

Ei olnud. Ja mitte ühelgi hetkel ei olnud seal tunnet, et oleme 
ettevõtjad või start-up. Tegime asju, mida oskasime ja 
mille jaoks oli tõmme turult ja nõudlus. Homme 
küsitakse multimeedia CD-ROMi -- \enquote{oh, teeme neid ka!} 

\question{See multimeedia CD-ROM\sidenote{Tänapäeval küllalt raskesti seletatav 
asi. Kujutage ette, et teil on interaktiivne veebileht, mis sisaldab videoid, 
muusikat ja teksti, aga ei ole kättesaadav mitte läbi brauseri ja interneti, 
vaid levib CD-plaadi kujul ning on realiseeritud täiesti teistsuguse 
tehnoloogia abil.} oli täiesti müstiline asi. Kas tõesti tuli Ühispank\sidenote{Üks 
Halo suuremaid projekte oli Ühispanga\index{Ühispank} aastaraamatu väljaandmine CD-l.} seda meie käest küsima? Andres Aarma\index[ppl]{Aarma, 
Andres}\sidenote{Andres tegeles tol ajal Ühispanga avalike suhetega.} samas 
võis tulla küll \ldots}

Kaks tükki me neid suuri tegime: ühe Ühispangale 
ja selle lainel müüsime teise Eesti Telekomile\index{Eesti Telekom}. 
Idee seisnes selles, et kuulge, varsti on aasta 2000, miks te oma 
aastaraamatut paberile trükite! Mäletan, et üks neist 
maksis selle eest 200 000 Eesti krooni. Täna ei saa 15 
000 euro eest ühtegi progejat liigutama.

\question{Tagantjärele mõeldes hämmastab
see, et me ei teadnud midagi versioneerimisest, testimisest ega üldse 
mingist süsteemsest tarkvaraarendusest ja siiski suutsime punuda 
softi, mis enam-vähem töötas. Ja meil oli häbematust see kliendile tarnida ja 
klient maksis arved ära!}

Olen sellele paar korda mõelnud. Üheksakümnendate internetilainel räägiti kogu aeg uuest majandusest ja hästi palju oli neid kohti, 
kus arvati, et uus majandus ei allu vana majanduse reeglitele. Mõnes asjas ei 
allu ka, nagu turu suuruse mõõtmine või füüsiline kaugus, aga põhiasjades, näiteks et tulud võiksid ületada kulusid, allub. Samamoodi 
vastandati e-kommertsi ja päris kaubandust ning kõike käsitleti 
eraldi. Me tegime mitu aastat seda äri selles suhtes 
valesti, et mõtlesime, et veebiehitamine ei ole tarkvaraarendus. 
Tegelikult ei olnud meil tiimis kedagi, kes oleks mõelnud veebisaidi 
ehitamisest nagu tarkvaraarendusprojektist või lugenud läbi mõne raamatu, kuidas suuri projekte teha. Veebivärk tundus 
nii lihtne. Kui kogemata ehitad sinna taha ka sisuhaldussüsteemi, 
mida me üritasime tootestada, siis sellel hetkel oleks pidanud hakanud teadlikumalt tarkvara arendama. 

Tagantjärele arvan, et see, mida Taavi Kotka\index[ppl]{Kotka, Taavi} 
Webmedias\index{Webmedia}\sidenote{Praegu Nortal.} tegi -- nad hakkasid 
mõtlema oma tarkvaraarenduse protsessi peale ---, hoidis neid elus. 
Nad alustasid meist natuke hiljem, aga hammustasid kiiremini läbi, et kui 
lähevad suurt maksuameti infosüsteemi tegema, siis ei peaks seda veebilehena 
käsitlema. 

\question{Jah, me ehitasime ikkagi veebilehti \ldots}

\ldots millel oli kogemata mingeid skripte taga.\sidenote{Kuna andmebaas ja 
selle haldus tundus keeruline, siis hoiti kõike tekstifailides, 
mida muudeti Perli abil.}

\question{Kas sa olid tol ajal ka Tiigrihüppega\index{Tiigrihüpe} seotud?}

See oli ka üks kooliaegne asi. Tartus oli füüsika 
instituudi\index{Tartu Ülikool!Füüsika Instituut} füüsikute seas Jaak 
Aaviksoo\index[ppl]{Aaviksoo, Jaak}, kellega mu isa käis samal ajal 
ülikoolis ja kes oli ka Miina Härma\index{Miina Härma Gümnaasium} või 
2. keskkooli vilistlane. Kui Tiigrihüpet tegema hakati, siis Toomas Hendrik 
Ilves\index[ppl]{Ilves, Toomas Hendrik} ja Jaak Aaviksoo\index[ppl]{Aaviksoo, 
Jaak} mõtlesid selle välja. Nagu nad ise räägivad, kolmekesi: Ilves, Aaviksoo 
ja Johnny Walker. Mõtlesid selle oma rollides välis- ja 
haridusministrina välja ning moodustasid selle ümber Tiigrihüppe peakomitee. 
1996. aastal, kui ma olin keskkooli viimases klassis, kutsusid nad mind sellesse 
komiteesse õpilaste esindajaks -- iseenesest ilus žest, 
haridusprojekti tehes võiks mõni õpilane ka sellega seotud olla! Sõitsingi Tartust bussiga Tallinnasse haridusministeeriumisse 
koosolekule, kus olid Peeter Marvet\index[ppl]{Marvet, Peeter}, Marju 
Lauristin\index[ppl]{Lauristin, Marju}, Ants Sild\index[ppl]{Sild, Ants} ja 
teised korüfeed laua ümber. Istusin seal 
lihtsalt vait ja kuulasin, see tundus täiesti müstiline. 

Minu suurim panus Tiigrihüpesse oli see, kui käisime otsesaates, kus toimus Tiigrihüppe 
teemal väitlus: Marju Lauristin\index[ppl]{Lauristin, Marju} \emph{versus} 
Lauri Leesi\index[ppl]{Leesi, Lauri} ja mina. Olin keskkoolipoisina otse-eetris, esimest korda 
televisioonis, ja minu vastas oli kaks oma 
valdkonnas tunnustatud inimest. Kuulasin, kuidas Lauri Leesi 
jahus, et arvutit pole kooli vaja ja et krihvel ja tahvel 
on aastasadu vastu pidanud, ning sain aru, kui läinud rong tema
jaoks on, sest ma ise juba tunnetasin, kuhu maailm liigub ja mis juhtub. Umbes sellest hetkest tekkis mul ka tunne, 
et ma ei lähe ülikooli arvutiteadusi õppima, kuna ma juba progesin, pidasin
BBSi ja hängisin internetis, ja et tegelik lünk on see, miks need 
asjad toimivad. Võrgustikud ja sotsiaalne pool. Saatest lahkudes ütles Lauristin mulle: \enquote{Kuule, me teeme 
Tartusse uut eriala, kus hakkame kommunikatsiooniteooriat ja asju õpetama -- 
tule sinna.} Nii ma sattusingi sotsiaalteadusi õppima, mis 
oli paljude ja ka minu enda jaoks suhteliselt üllatav.

\question{Üheksakümnendatel juhtus igasuguseid üllatavaid asju!}

Jah, võrreldes kõigi nende progejatega, kes olid teoloogid, läks mul isegi 
hästi.

\question{Millega sa praegu tegeled? Kuhu see tee sind tänaseks on toonud?}

Põhiliselt ehitan ettevõtteid, pigem nende algfaasis. 
Mõnes mõttes võib öelda, et teen täpselt sama, mida me üheksakümnendatel 
tegime, aga nüüd juba iga kord uue kogemusega ja
natuke tean ka, mida ma räägin. 

Hakkasin viis 
aastat tagasi tegema sellist ettevõtet nagu Teleport ja kaks aastat tagasi ostis Topia
meid ära ja nüüd tegutsen seal edasi. Asutajana tekkis see hetk, et kas tahan suurt tükki 
väikesest pirukast või väikest tükki suurest pirukast, ja Teleporti müük 
Topiale andis võimaluse paar aastat vahele jätta ning hüpata trepil 
kõrgemale. Kaheteist inimesega start-up'ist sai 170 inimesega tiim, mul on tootearenduses 70 
inimest, kellega saab sama visiooni kiiremini ehitada. 

Ja kui mul aega üle on, siis ka investeerin start-up'idesse ja annan mõne
nõu. 

Sarnasus üheksakümnendatega on see, et ma tean täpselt, mida ma suudan üksi 
ehitada -- on see siis BBSi soft või mitte ---, ning mis väärtus ja võlu 
on sellel, kui progeja, disainer ja äriinimene koos midagi valmis teevad, alates mängust \enquote{Drunkard} ja 
Halo veebiprojektidest. Tean ka, et ma ei taha kunagi elus müüa oma aega 
tunnipõhiselt, sest tunde on lõplik hulk. Järelikult tuleb ehitada tooteid. 

See läheb küll nüüd meie ajaraamist välja, aga hiljem 
Skype'i\index{Skype} puhul oli näha, kuidas paar inimest ehitavad mõne kuuga asja, 
mida kuu aega hiljem kasutab miljon inimest. Üheksakümnendate 
rahmeldamine seda võibolla õpetaski ja nüüd otsin alati kohta, kus 
sissepandud töötundide hulk konverteeruks võimalikult suureks väärtuseks. Tol 
hetkel kippusin väga palju õhinapõhiselt tegutsema ja 
oleksin kõiki neidsamu asju teinud ka siis, kui ei oleks palka saanud. Kui 
müüd oma töötunde, siis on sul lihtsalt väga pikad päevad ja väga lühikesed 
ööd. Üheksakümnendatel põles osa tüüpe ka kohe läbi, osa põles hiljem. 

Austan sellest ajast väga näiteks
Taavi Talvikut\index[ppl]{Talvik, Taavi}, kes ehitas sama tausta ja huvide 
pealt Unineti\index{Uninet} ning võib ka magada, sest bitid müüvad 
ennast ise. See oli küll teenuseinfrastruktuuri äri, aga seal oli see alge 
olemas, kuidas ehitada midagi, mis saab hakkama ka siis, kui ei ole ise kogu 
aeg näppupidi juures.