Kuna tegu on inimeste endi lugudega on neid raske kuidagi üheselt kokku võtta: kõik lood on unikaalsed, need põimuvad, segunevad kummalistel ja vahel ebaloogilistel viisidel, viivad kuskilt kuhugi ja igasugune katse midagi üldistada teeb lugudele ja inimestele ülekohut. 

Mõnda  torkab siiski silma. 

Kõigepealt see kummaline tõmme, mis arvutitel inimeste suhtes oli. Seejuures on huvitav, et tegu ei ole lihtsalt tehnikahuviliste noorte huviga tehnika vastu. Pigem vastupidi: mitmel juhul öeldakse, et arvutid üldiselt ja programmeerimine spetsiifiliselt olid pigem vahendid millegi muu saavutamiseks, kui eesmärk iseeneses. Ka täiesti teiste huvidega (Jaanuse ja Tarvi puhul näitehuviga) inimesi tõmbas miskipärast tugevalt arvuti poole. Jaanuse kasutatud metafoor \enquote{lendamise trennist}, millest on võimatu niisama mööda minna, kajab igalt poolt vastu. 

Seejuures on see miski, mida arvutite abil saavutada, tugevalt humanistlik, üldinimlik, ja ehk seletab mõnel määral meie toonase arvuti-kogukonna teket. Ahti sõnastab seda kui sarnaselt mõtlevate noorte inimeste püüet koos, üksteisele toetudes, inimeseks saada. Priit ja Jaan aga ütlevad, et nende jaoks oli võluv asjaolu, et kõik, mis on võimalik inimese peas, on võimalik ka arvutis. Eks iga teismeline on kogenud frustratsiooni oma võimetuse üle viia ellu oma suurepäraseid ideid. Ühtäkki aga asendus kontrolli puudumine  täiusliku kontrolliga arvuti üle koos piiramatu vabadusega suhelda teiste omasugustega.  Ja mis võiks olla veel paeluvam, kui võimalus oma unistusi koos teistega ellu viia?

Just koos. Võiks ju arvata, et arvuti-inimesed tegelevad pigem arvutite, kui inimestega, kuid koos tegutsemine ja võimalus suhelda teiste omasugustega on oluline teema pea kõigis lugudes. Üksteiselt õpitakse, saadakse abi. Koos tehakse suuri asju ja ühel või teisel moel jookseb rõõm headest kaasteelistest läbi enamusest lugudest. Kindlasti on ka rivaalitsemist, tülisid. Tallinna ja Tartu asetsesid mingil hetkel hea põhjusega üksteisest 70 000 kilomeetri kaugusel\sidenote{Vt. lk. \pageref{sisu!70k}}. Siiski domineerib arusaam, et oluline on olla osa kogukonnast ja et kogukond toimib vaid kõigi osapoolte heast tahtest. Skype lugugi on ju vaadeldav kui lugu sõprusest.

Tugev tõmme arvuti poole võib  aga ilma sobiva keskkonnata lihtsasti vaid platooniliseks igatsuseks jääda. Lugudude alusel  võis see keskkond võtta mitmeid vorme näiteks mitmel puhul maagilise paigana mainitud kooli raadioruumi või vanemate arvutitega seotud töökoha näol ning vahel oli kodus olemas elektroonika-huvi. Samas on ka näiteid, kui inimene ületab teel arvutiteni hoomamatuid takistusi ning jõuab kaugele. Ehk, keskkond kahtlemata toetab arvuti-huvi kuid ei ole ilmtingimata vajalik.

Lugedes lugusid arvuti juurde jõudmisest ja nende juurde jäämisest, torkab silma tänasest dramaatiliselt eriline suhe nendega. Juba ammu ei ole arvuti ja internet asjad, millele ligipääs on probleemiks. Kuid toona olid arvuti ja temal toimiv tarkvara väga lihtne, täna on nii arvuti kui tarkvara ühele inimesele terviklikuks mõistmiseks selgelt liiga keerulised. Nii näiteks Arne kui Meelis ütlevad aga, et nad said oma arvutist lõpuni aru: BASICu detailidest kuni riistvarani välja. Ühelt poolt andis see põhimõtteline erinevus kogemuse kontrollist ja teisalt saavutuselamuse. Tihti oli koolipoisil puht praktiliselt vaja luua olemasolevaga samaväärset või isegi paremat tarkvara. Nii Jaan kui Andres kirjutasid  toimiva ja kasuliku tekstideraktori\sidenote{Sama lugu on olnud mujalgi (\url{https://corecursive.com/058-brian-kernighan-unix-bell-labs/}), arvutite algusaegadel kulus väga palju auru võimaldamaks arvutisse teksti sisestada. Donald Knuth ja tema \LaTeX, tänu millele ka see raamat sünnib, lahendab samuti tekstiga seotud probleeme.}, sest seda oli vaja. Täna ei ole sellisteks ettevõtmisteks ei praktilist vajadust ega ka sisulist võimalust. 

Need lihtsamad masinad paigutusid mõnes mõttes märksa lihtsamasse sotsiaalsesse konteksti, kus segavaid faktoreid oli vähe ning võimalusi keskendumiseks palju. Jah, kindlasti on teatud vanuses noor inimene juba piisavalt nutikas huvitavateks programmeerimisülesanneteks kuid veel mitte takerdunud täiskasvanu-ellu. Kuid teadlik keskendumine on siiski nii mitmeski loos läbivaks teemaks. Ja on selge, et tänases kommunikatsioonile vaikimisi avatud keskkonnas nõuab keskendumine teistsuguseid ning kindlamaid oskusi kui toonases suletud kontekstis. 

Aruvtite ja tarkvara lihtsus võimaldas kindlasti luua väga kiiresti väga kasulikku tarkvara. Jaani tekstiredaktorit sai juba mainitud, aga Masti ja Marguse kahe kuuga kirjutatud modemipank oleks samuti tänapäeval küllalt ennekuulmatu asi. Teisalt aga jookseb juttudest läbi terviku tajumise teema, mis tänaste arvutite puhul on raskem. Tõnis ja Tõnu mainivad, kuidas nad ei saa keerulistest asjadest aru, ning kui oluline on võime taandada keeruline probleem lihtsamale kujule. Sellist hoomatavat tervikpilti arvuti ja arvutivõrgu toimimisest on lihtsamate arvutite puhul kindlasti suhteliselt lihtsam luua. Samas jääb loodud mudel aga adekvaatseks ka keerukamate süsteemide puhul: Tõnul ei ole probleem tegelda mikroelektroonikaga ja Vilve ehitab ülikeerulisi finantssüsteeme, sest neil on olemas lihtsate toimivate süsteemidest pärinev toimiv mõttemudel.

Kujutage endale ette, et teil on töö juures ülemuse kabinetis umbes pool miljonit eurot maksev aparaat. Ja teie varateismeline laps avaldab soovi selle aparaadiga veidi mängida. Kõlab hullumeelselt? Ometi toimiti kaheksakümnendatel täpselt nii kõikvõimalikes asutustes üle Eesti lubades kõikvõimalikke jõnglasi toonases mõistes hirmkalleid arvuteid näppima. Veelgi enam, sagedasti võeti rüblik lausa palgale, kuna osundus, et ta suudab arvutist üle käia (sest need olid suhteliselt lihtsad!) ning temast on kasu. Mõnda sellist motiivi sisaldab peaaegu iga ära toodud lugu.

Ma usun, et selline usaldus inimeste vahel, kes saavad aru probleemdiest ja nende vahel, kes saavad aru lahendustest on Eesti IT eduloos põhimõttelise tähtsusega. Mõlemad osapooled ju mõistavad, et nende huvides ei ole usaldust kuritarvitada: kui IT-kutti liiast nöökida, läheb ta mujale, ning kui öise mängimis-sessiooni tagajärjed päevatööd häirivad, võetakse võtmed käest. Sel samal vastastikusel usaldusel ja sellest tuleneval koostööl põhinevad nii ID-kaart kui X-Tee kui Hansapank kui kõik teised meie eduloo peatükid. Võib ju olla visioon teistmoodi pangast, aga tuleb uskuda, et IT-inimesed selle ka valmis ehitavad. Ükski riigiametnik ei ärka ühel hommikul mõttega XML-sõnumite liikumisest asutuste vahel. See on inseneri mõte ja vajab realiseerumiseks usku sedalaadi mõtete kasulikkusse. Omavahelised usalduslikud suhted olid kindlasti olulised ka kogukonna sees, kus suhteliselt väikesearvuline seltskond üksteist vähemalt nime pidi tundis ning \enquote{letihinnast ikka allahindlust tegi}. 

Usaldusel on kindlasti ka teine pool. Enamus siin raamatus toodud lugudest oleksid oluliselt lühemad, kui toona oleks rakendatud tänapäevases mõistes infoturvet. Kindlasti oleksid suured tükid meie IT-edulugu olemata, kui tarkvarapiraatlusele oleks vaadatud samamoodi, kui praegu. Ometi ei kosta lugudest usalduse kuritarvitamist, pigem räägitakse üle võetud masinate paikamisest ja omanikule tagastamisest. Samamoodi tekib ilmselt küsimus, et kui legaalne oleks toonane suhteliselt kinnise seltskonna \enquote{käsi peseb kätt} lähenemine riigi- ja erasektori piiril tänase hankeregulatsiooni kontekstis. Kuid ka siin kostab pigem lugusid riigi raha eest võimalikult hea tulemuse toomisest (Tarvi ja sidemastide lugu, näiteks), kui seitsme naha koorimisest. 

Hea küll. Maagiline kast tõmbab maabilise jõuga noore inimest enda juurde. Kuid mida, kohale jõudnuna, selle kastiga ette võtta? Kust tulevad selleks vajalikud oskused? Läbivaks jooneks on siin selgelt ise õppimine. Seejuures on tähelepanuväärne, et institusionaliseeritud õppimist meenutatakse sisu mõttes kasulikuna pigem harva kuid vaimsuse, seltskonna ja kultuuri mõttes valgustavana pigem sageli. On üksikuid erandeid, nagu Ahti ja Vilve, kuid reeglina inimesed ei oska vastata, kuidas nad programmeerima või elektroonikaga tegelema õppisid. Vastupidiselt tänasele, kus tundub suund olevat võimalikult paljude inimeste programmeerima õpetamisele, võtavad toonase suhtumise ehk kenasti kokku Tõnise ütlus, et \enquote{õppida tuleb raskeid asju, lihtsad tulevad iseenesest} ning Andruse oma, et  \enquote{programmeerimine sünnib vajadusest}. 

Õppimise meetodina räägitakse palju kas plokkskeemide abil või niisama paberil programmeerimisest ning ega perfokaartide abil programmi loomine sellest palju ei erinenud. Võib arvata, et ülimalt kõrge barjäär (arvutil kas puudus üldse interaktiivne konsool või oli ligipääs sellele väga piiratud) programmi sisestamisel sundis inimesi rohkem süvenema ning oma koodi läbi mõtlema viies programmeerimise kunsti metoodilisema ja sügavama mõistmiseni kui internetist koodijuppide kopeerimine annab.

Samas on roll tolles ebamäärases ja seletuseta õppeprotsessis väga selge ja suur roll kogukonnal. Reeglina puudus arvutite kohta ametlik kirjandus, teadmine levis folkloorina suust suhu, seda kasutati väärtusliku kaubana, seda jagati vaid valitutega, seda kirjutati märkmikesse. Kogukonnaks võis olla arvutiklassis kogunev poistekamp, mõnd arvutifirmat ümbritsev seltskond aga ka kooliklass, konkreetne institutsioon (KBFI) või lihtsalt füüsiline koht (Tartu Tähetorn). Anto ütleb mitmel puhul, et õppis üht või teist asja oma kooli poistelt. Siit koorub ehk ka võti mõistmaks, miks kujunes reeglina tugevalt introvertsest arvuti-rahvast Eestile hoo andnud tugev kogukond. Kuna suurem osa teadmisest tuli kellegi teise käest, muutus suur suhtevõrgustik isikliku arengu mõttes hädavajalikuks. Tippudel pidi olema väga hea suhtevõrgustik ja, kuna kõigil suhetel on vähemalt kaks otsa, aitasid nad arendada ka teiste kogukonna liikmete võrgustikku ning oskusi. Üllatavalt sageli näeme inimesi tegutsemas mingit sorti müügifunktsioonis, mis jällegi rõhutab sotsiaalsete oskuste olulisust. 

Kogukonnad võivad olla isetekkelised, kuid reeglina mainitakse mõnda konkreetset inimest, kelle ümber koonduti. Keegi ei meenuta, et nad oleksid Jaak Loonde käest midagi konkreetselt õppinud. Küll aga meenutatakse tema hindamatut rolli arvutiklasside tekitamisel ning, mis veelgi olulisem, sinna kogunenud seltskonna jaoks katalüsaatorina toimimisel. Lõvi, Antot, Annet, Tarmot ja teisi meenutatakse soojalt lisaks nende teadmistele ka kogukonna loojatena. 

Siin kaante vahel toodud lugudega tegeldes torkab silma tugev kallutatus eestlastest meesterahvaste poole. Kindlasti tuleneb see osalt ka autorist, kuid ka lugudes tegutsevad reeglina eesti keelt rääkivad mehed. Seejuures, kui mõni naisterahvas pildile ilmub, teeb ta seda võimsalt mõjutades paljusid ja liigutades metafoorseid mägesid (Vilve ja Anne) või olles peategelase oluliseks suunajaks (Anto ja Ahti emad). Eesti ja vene kogukondade omavaheline suhe on aga keerulisem. Ainsana loob nende vahele tõsisema silla Sergei, kelle jutust avaneb tõeline paralleelmaailm oma seltskondade ning õpetajatega a la Jaak Loonde. Vilve jutust läbi jooksev keerulise nimega Moskvale allunud asutus annab aimu, et eksisteeris ka terve eraldiseisev enamasti vene töökeelega arvutitega tegelevate organisatsioonide võrgustik. Mõlemal puhul tundub, et ühel või teisel põhjusel oleme jätnud suure hulga tarku inimesi tähelepanuta ja sellest on kahju.

Lisaks juba mainitud müügitööle on mõnevõrra üllatav meedia, sealhulgas trükimeedia, oluline roll inimeste lugudes. Pangandus kui Eesti tehnoloogia taimelava on teada-tuntud fenomen, meediast on selles kontekstis vähem räägitud. Ometi olid Kaspar, Peeter, Sten ja Taavi ja teised üht või teistpidi seotud pabermeediaga ning Kaspar toimetas teles. Ilmselt oli meedia valdkond, kuhu esimesel võimalusel liikus raha ning kus tehnoloogia abil oli võimalik saavutada oluline kvaliteedihüpe. Tehnoloogia aga tõmbas ligi teatud liiki inimesi.

Teiseks mõningaseks üllatuseks oli lugude tugev rahvusvaheline mõõde. Eesti NSV oli juba Nõukogude liidus teistest erinevas rollis Soome füüsilise läheduse ning telekomi infra suhtelise kvaliteedi tõttu. Meilt oli teatud tingimustel võimalik \enquote{päris} välismaale helistada ning too side oli tänu lühikesele distantsile isegi arvutisideks kasutatav! See võimaldas toetada side osas näiteks Leedut ning toimida teatud väravana kogu Nõukogude Liidu arvuti-rahva jaoks. Lugu Vladivostokist flopidega Tallinna tarkvara järele lennanud inimestest kõlab uskumatuna, kuid on ilmselt siiski tõsi. Seejuures saime ka meie olulist abi Soomest ja Rootsist. Rootsi loodi meie esimesed satelliitühendused, Soome aitas Tallinna Tehnikaülikoolil modemeid hankida, nii Soome kui Rootsi tehti tööd. Ja kindlasti tuleb ära märkida Ron Dwight, kelle rolli Eesti Fido kogukonna arengul ei saa kuidagi üle hinnata. 

Kuidas siis võtta kokku \verb|print(memcpy[])|? 

Kugi lugudest saab aimu, kuidas ja miks toonane arvuti-kogukond kujunes, ei saa me täit vastust küsimusele \enquote{miks just Eesti IT-edulugu?}. Kindlasti mängisid oma rolli suure visiooniga inimesed kuid palju oli ka pragmaatilist asjade ära tegemist ja ka lihtsat lustimist. Õpetajad olid olulised, kuid enamasti mitte teadmiste edastajatena. Akadeemilised asutused olid olulised, kuid pigem üksikute kogukonna-kollete võimaldajate kui institutsioonidena. Eraettevõtted olid olulised, kuid olles lugenud toonase kauboi-kapitalismi kohta, valdab aknast Eesti elu vaadates kergendustunne. 

Küll aga koondab see raamat 29 suurepärase inimese lood. Ja ehk on sellest praeguseks küllalt.
