Kuna tegu on inimeste isiklike lugudega, on neid raske kuidagi üheselt kokku võtta: 
kõik lood on unikaalsed, need põimuvad, segunevad kummalistel ja vahel 
ebaloogilistel viisidel, viivad kuskilt kuhugi ja igasugune katse midagi 
üldistada teeb lugudele ja rääkijatele ülekohut. 

Mõnda torkab siiski silma. Kõigepealt see kummaline tõmme, mis inimestel arvutite suhtes oli. Seejuures 
ei ole tegu lihtsalt tehnikahuviliste noorte huviga tehnika vastu. 
Pigem vastupidi: mitmel juhul öeldakse, et arvutid üldiselt ja programmeerimine 
spetsiifiliselt olid pigem vahendid millegi muu saavutamiseks kui eesmärk 
iseeneses. Ka täiesti teiste huvidega (Jaanuse ja Tarvi puhul näitlemine) 
inimesi tõmbas miskipärast tugevalt arvuti poole. Jaanuse kasutatud metafoor 
\enquote{lendamise trennist}, millest on võimatu niisama mööda minna, kajab 
igalt poolt vastu. 

Seejuures tundub see miski, mida arvutite abil saavutada, tugevalt humanistlik, 
üldinimlik, ja ehk seletab mõnel määral meie toonase arvutikogukonna teket. 
Ahti sõnastab seda kui sarnaselt mõtlevate noorte inimeste püüet koos, 
üksteisele toetudes inimeseks saada. Priit ja Jaan ütlevad, et neid võlus asjaolu, et kõik, mis on võimalik inimese peas, on võimalik ka 
arvutis. Eks iga teismeline on kogenud frustratsiooni võimetuse üle viia 
ellu oma suurepäraseid ideid. Ühtäkki aga asendus kontrolli puudumine 
ümbritseva üle täiusliku kontrolliga arvuti üle koos piiramatu vabadusega 
suhelda teiste omasugustega. Ja mis võiks olla veel paeluvam, kui võimalus oma 
unistusi koos teistega ellu viia?

Just koos. Võiks ju arvata, et arvutiinimesed tegelevad pigem arvutite kui 
inimestega, kuid koos tegutsemine ja võimalus suhelda teiste omasugustega on 
oluline teema pea kõigis lugudes. Üksteiselt õpitakse, saadakse abi. Koos 
tehakse suuri asju ja ühel või teisel moel jookseb rõõm headest kaasteelistest 
läbi enamikust lugudest. Kindlasti on ka rivaalitsemist ja tülisid. Tallinna ja 
Tartu asetsesid mingil hetkel hea põhjusega üksteisest 70 000 kilomeetri 
kaugusel.\sidenote{Vt lk \pageref{sisu!70k}.} Ometigi domineerib arusaam, et 
tähtis on olla osa kogukonnast ja et kogukond toimib vaid kõigi osapoolte 
heast tahtest. Skype'i lugugi on ju vaadeldav kui lugu sõprusest.

Tugev tõmme arvuti poole võib aga ilma sobiva keskkonnata lihtsasti vaid 
platooniliseks igatsuseks jääda. Lugude alusel võis see keskkond võtta 
mitmeid vorme, näiteks mitmel puhul maagilise paigana mainitud kooli raadioruumi 
või vanemate arvutitega seotud töökoha näol, ning vahel oli kodus olemas 
elektroonikahuvi. Samas on ka näiteid, kuidas inimene ületab teel arvutini 
hoomamatuid takistusi ning jõuab kaugele. Ehk keskkond kahtlemata toetab 
arvutihuvi, kuid ei ole kuuldud lugude põhjal ilmtingimata vajalik.

Lugedes arvuti juurde jõudmise ja jäämise kogemustest, torkab 
silma tänasega võrreldes drastiliselt erinev ja eriline suhe arvutiga. Juba ammu ei ole arvuti ja 
internet asjad, millele ligipääs on probleemiks. Kuid toona olid arvuti ja 
sellel toimiv tarkvara erinevalt praegusest väga lihtsad; tänapäeval on nii arvuti kui ka
tarkvara ühele inimesele terviklikuks mõistmiseks selgelt liiga keerulised. Näiteks Arne ja Meelis ütlevad, et nad said oma arvutist lõpuni aru: alates BASICu 
detailidest kuni riistvarani välja. Ühelt poolt andis see põhimõtteline 
erinevus kogemuse kontrollist ja teisalt saavutuselamuse. Tihti oli koolipoisil 
puhtpraktiliselt vaja luua olemasolevaga samaväärset või isegi paremat 
tarkvara. Nii Jaan kui ka Andres kirjutasid toimiva ja kasuliku 
tekstiredaktori\sidenote{Sama lugu on olnud mujalgi 
(\url{https://corecursive.com/058-brian-kernighan-unix-bell-labs/}), arvutite 
algusaegadel kulus väga palju auru, võimaldamaks arvutisse teksti sisestada. 
Donald Knuthi \LaTeX, tänu millele ka see raamat sündis, lahendab samuti 
tekstiga seotud probleeme.}, sest seda oli vaja. Tänapäeval ei ole sellisteks 
ettevõtmisteks sageli ei praktilist vajadust ega ka sisulist võimalust.

Need lihtsamad masinad paigutusid mõnes mõttes märksa lihtsamasse sotsiaalsesse 
konteksti, kus segavaid faktoreid oli vähe ning keskendumisvõimalusi palju. 
Jah, kindlasti on teatud vanuses noor inimene juba piisavalt nutikas 
huvitavateks programmeerimisülesanneteks, kuid ei ole veel takerdunud 
täiskasvanu ellu. Samas on teadlik keskendumine nii mitmeski loos 
läbiv teema. Ja on selge, et tänases kommunikatsioonile vaikimisi avatud 
keskkonnas nõuab keskendumine teistsuguseid ja kindlamaid oskusi kui toonases 
suletud kontekstis.

Arvutite ja tarkvara lihtsus võimaldas luua väga kiiresti väga 
kasulikku tarkvara. Jaani tekstiredaktorit sai juba mainitud, aga Masti ja 
Marguse kahe kuuga kirjutatud modemipank oleks samuti tänapäeval küllaltki 
ennekuulmatu asi. Teisalt jookseb juttudest läbi terviku tajumise teema, 
mis tänaste arvutite puhul on raskem. Tõnis ja Tõnu mainivad, kuidas nad ei saa 
keerulistest asjadest aru ning kui oluline on võime taandada keeruline 
probleem lihtsamale kujule. Hoomatavat tervikpilti arvuti ja 
arvutivõrgu toimimisest on lihtsamate arvutite puhul kindlasti suhteliselt 
kergem luua. Samas jääb loodud mudel adekvaatseks ka keerukamate 
süsteemide puhul: Tõnul ei ole probleem tegelda mikroelektroonikaga ja Vilve 
ehitab ülikeerulisi finantssüsteeme, sest neil on olemas lihtsatest toimivatest 
süsteemidest pärinev mõttemudel.

Kujutage ette, et teil on töö juures ülemuse kabinetis umbes pool 
miljonit eurot maksev aparaat ja teie varateismeline laps avaldab soovi sellega 
veidi mängida. Kõlab hullumeelselt? Ometi toimiti kaheksakümnendatel 
täpselt nii kõikvõimalikes asutustes üle Eesti, lubades igas vanuses jõnglasi 
toonases mõistes hirmkalleid arvuteid näppima. Veelgi enam, sagedasti võeti 
rüblik lausa palgale, kuna selgus, et ta jõud käib arvutist üle (sest need 
olid suhteliselt lihtsad!) ning temast on kasu. Mõnda sellist motiivi sisaldab 
peaaegu iga lugu, joonides alla vastastikust usaldust tehnoloogide ja 
mittetehnoloogide vahel. Veelgi enam, kuna arvutiteemalist praktilist haridust 
nappis (väga vähesed meenutavad ülikooli kui olulist arvutiteadmise 
allikat), eksisteeris vahe tehnoloogide ja mittetehnoloogide vahel vaid 
nominaalselt, sõltudes pigem isiklikust huvist kui institutsionaalsest 
määratlusest. Teisisõnu võis arvutiga sina peale saada kes iganes, tehnoloogid olid 
inimesed meie endi keskelt ja seega olid neid lihtsam usaldada.

Ma usun, et usaldus inimeste, kes saavad aru probleemidest, ja 
nende vahel, kes saavad aru lahendustest, on Eesti IT eduloos põhimõttelise 
tähtsusega. Mõlemad osapooled ju mõistavad, et nende huvides ei ole usaldust 
kuritarvitada: kui IT-kutti liialt nöökida, läheb ta mujale, ning kui öise 
mängusessiooni tagajärjed päevatööd häirivad, võetakse võtmed käest. Selsamal 
vastastikusel usaldusel ja sellest tuleneval koostööl põhinevad nii ID-kaart, X-Tee, Hansapank kui ka kõik teised meie eduloo peatükid. Võib ju olla 
visioon teistmoodi pangast, aga tuleb uskuda, et IT-inimesed selle ka valmis 
ehitavad. Ükski riigiametnik ei ärka ühel hommikul mõttega XML-sõnumite 
liikumisest asutuste vahel. See on inseneri mõte ja vajab realiseerumiseks usku 
sedalaadi mõtete kasulikkusse. Omavahelised usalduslikud suhted olid kindlasti 
olulised ka kogukonna sees, kus suhteliselt väikesearvuline seltskond teadis üksteist 
vähemalt nime pidi ning \enquote{letihinnast ikka allahindlust tegi}. 

Usaldusel on kindlasti ka teine pool. Enamik siin raamatus toodud lugudest 
oleksid märksa lühemad, kui toona oleks rakendatud tänapäevases mõistes 
infoturvet. Kindlasti oleksid suured tükid meie IT edulugu olemata, kui 
tarkvarapiraatlusele oleks vaadatud samamoodi kui praegu. Ometi ei kosta 
lugudest usalduse kuritarvitamist, pigem räägitakse üle võetud masinate 
paikamisest ja omanikule tagastamisest. Samamoodi tekib ilmselt küsimus, kui 
legaalne oleks toonane suhteliselt kinnise seltskonna \enquote{käsi peseb kätt} 
lähenemine riigi- ja erasektori piiril tänase hankeregulatsiooni kontekstis. 
Kuid ka siin kostab pigem lugusid riigi raha eest võimalikult hea tulemuse 
toomisest (näiteks Tarvi ja sidemastid) kui seitsme naha koorimisest. 

Hea küll. Maagiline kast tõmbab maagilise jõuga noort inimest enda juurde. Kuid 
mida kohale jõudnuna selle kastiga ette võtta? Kust tulevad vajalikud 
oskused? Läbiv joon on siin selgelt ise õppimine. Seejuures on 
tähelepanuväärne, et institutsionaliseeritud õppimist meenutatakse sisu mõttes 
kasulikuna pigem harva, kuid vaimsuse, seltskonna ja kultuuri mõttes 
valgustavana pigem sageli. On üksikuid erandeid, nagu Ahti ja Vilve, kuid 
üldjuhul inimesed ei oska vastata, kuidas nad programmeerima või elektroonikaga 
tegelema õppisid. Vastupidiselt tänapäevale, kui suund tundub olevat võimalikult 
paljude inimeste programmeerima õpetamisele, võtavad toonase suhtumise 
kenasti kokku Tõnise ütlus: \enquote{Õppida tuleb raskeid asju, lihtsad 
tulevad iseenesest} ning Andruse oma: \enquote{Programmeerimine sünnib 
vajadusest.}

Õppimismeetodina räägitakse palju kas plokkskeemide abil või niisama paberil 
programmeerimisest ning ega perfokaartide abil programmi loomine sellest palju 
ei erinenud. Võib arvata, et ülimalt kõrge barjäär (arvutil kas puudus üldse 
interaktiivne konsool või oli ligipääs sellele väga piiratud) programmi 
sisestamisel sundis rohkem süvenema ja oma koodi läbi mõtlema, viies 
programmeerimiskunsti metoodilisema ja sügavama mõistmiseni, kui internetist 
koodijuppide kopeerimine anda saab.

Samas on tolles ebamäärases ja seletuseta õppeprotsessis väga selge ja suur 
roll kogukonnal. Enamjaolt puudus arvutite kohta ametlik kirjandus, teadmine 
levis folkloorina suust suhu, seda kasutati väärtusliku kaubana, jagati 
vaid valitutega ja kirjutati märkmikesse. Kogukonnaks võis olla 
arvutiklassis kogunev poistekamp, mõnd arvutifirmat ümbritsev seltskond, aga ka 
kooliklass, konkreetne institutsioon (KBFI) või lihtsalt füüsiline koht (Tartu 
Tähetorn). Anto ütleb mitmel puhul, et õppis üht või teist asja oma kooli 
poistelt. Siit koorub ehk ka võti, mõistmaks, miks kujunes enamasti tugevalt 
introvertsest arvutirahvast Eestile hoo andnud tugev kogukond. Kuna suurem osa 
teadmisest tuli kellegi teise käest, muutus suur suhtevõrgustik isikliku arengu 
mõttes hädavajalikuks. Tippudel pidi olema väga hea suhtevõrgustik ja kuna 
kõigil suhetel on vähemalt kaks osalist, aitasid nad arendada ka teiste kogukonna 
liikmete võrgustikku ning oskusi. Üllatavalt sageli näeme inimesi tegutsemas 
mingit sorti müügifunktsioonis, mis jällegi rõhutab sotsiaalsete oskuste 
olulisust. 

Kogukonnad võivad olla isetekkelised, kuid üldjuhul mainitakse mõnda 
konkreetset inimest, kelle ümber koonduti. Keegi ei mäleta, et nad oleksid 
Jaak Loonde\index[ppl]{Loonde, Jaak} käest midagi konkreetselt õppinud. Küll 
aga meenutatakse tema hindamatut rolli arvutiklasside tekitamisel ning, mis 
veelgi olulisem, sinna kogunenud seltskonna jaoks katalüsaatorina toimimisel. 
Lõvi\index[ppl]{Lõvi}, Antot\index[ppl]{Veldre, Anto}, Annet\index{Villems, 
Anne}, Tarmot\index[ppl]{Mamers, Tarmo} ja teisi meenutatakse soojalt lisaks 
nende teadmistele ka kogukonna loojatena. 

Nende kaante vahele kogutud lugudes torkab silma tugev kallutatus 
eestlastest meesterahvaste poole. Kindlasti tuleneb see osalt ka autorist, kuid 
ka lugudes tegutsevad tavaliselt eesti keelt rääkivad mehed. Seejuures, kui pildile ilmub mõni 
naine, teeb ta seda võimsalt, mõjutades paljusid ja 
liigutades metafoorseid mägesid (Vilve\index[ppl]{Vene, Vilve}, 
Anne\index[ppl]{Villems, Anne} ja kindlasti Kersti\index[ppl]{Kaljulaid, 
Kersti}) või olles peategelase oluliseks suunajaks (Anto ja Ahti emad). Eesti 
ja vene kogukondade omavaheline suhe on aga keerulisem. Ainsana loob nende 
vahele tõsisema silla Sergei\index[ppl]{Anikin, Sergei}, kelle jutust avaneb 
tõeline paralleelmaailm oma seltskondade ja õpetajatega, nagu Jaak Loonde. 
Vilve jutust läbi jooksev keerulise nimega Moskvale allunud asutus annab aimu, 
et eksisteeris ka terve eraldiseisev, enamasti vene töökeelega arvutitega 
tegelevate organisatsioonide võrgustik. Mõlemal puhul tundub, et ühel või 
teisel põhjusel oleme jätnud suure hulga tarku inimesi tähelepanuta, ja sellest 
on kahju.

Lisaks juba mainitud müügitööle on mõnevõrra üllatav meedia, sealhulgas 
trükimeedia, oluline roll inimeste lugudes. Pangandus kui Eesti tehnoloogia 
taimelava on teada-tuntud fenomen, meediast on selles kontekstis vähem 
räägitud. Ometi olid Kaspar, Peeter, Sten ja Taavi ning teised üht- või teistpidi 
seotud pabermeediaga ning Kaspar toimetas teles. Ilmselt oli meedia valdkond, 
kuhu esimesel võimalusel liikus raha ja kus tehnoloogia abil oli võimalik 
saavutada suur kvaliteedihüpe. Tehnoloogia aga tõmbas ligi teatud liiki 
inimesi.

Teiseks mõningaseks üllatuseks oli lugude tugev rahvusvaheline mõõde. Eesti NSV 
oli juba Nõukogude Liidus teistest erinevas rollis Soome füüsilise läheduse 
ja telekomi infra suhtelise kvaliteedi tõttu. Meilt oli teatud tingimustel 
võimalik \enquote{päris} välismaale helistada ning too side oli tänu lühikesele 
distantsile isegi arvutisideks kasutatav! See võimaldas toetada side osas 
näiteks Leedut ja toimida väravana kogu Nõukogude Liidu arvutirahva 
jaoks. Lugu Vladivostokist flopidega Tallinna tarkvara järele lennanud 
inimestest kõlab uskumatuna, kuid on ilmselt tõsi. Seejuures saime ka 
meie suurt abi Soomest ja Rootsist. Rootsis loodi meie esimesed 
satelliitühendused, Soome aitas Tallinna Tehnikaülikoolil modemeid hankida ja nii 
Soome kui ka Rootsi tehti tööd. Kindlasti tuleb ära märkida Ron 
Dwight\index[ppl]{Dwight, Ron}, kelle rolli Eesti Fido kogukonna arengus ei saa 
kuidagi üle hinnata. 

Kuidas siis võtta kokku \verb|print(memcpy[])|? 

Kuigi lugudest saab aimu, kuidas ja miks toonane arvutikogukond kujunes, ei saa 
me täit vastust küsimusele \enquote{miks just Eesti IT edulugu?}. Kahtlemata
mängisid oma rolli suure visiooniga inimesed, kuid palju oli pragmaatilist 
asjade ärategemist ja ka lihtsat lustimist. Õpetajad olid olulised, ent
enamasti mitte teadmiste edastajatena. Akadeemilised asutused olid olulised, 
kuid pigem üksikute kogukonnakollete võimaldajate kui institutsioonidena. 
Eraettevõtted olid olulised, aga olles lugenud toonase kauboikapitalismi 
kohta, valdab aknast Eesti elu vaadates kergendustunne. Kõikidel vastustel 
tundub olevat oma \enquote{aga}.

Niisiis, head vastust algsele küsimusele ei ole me leidnud. Küll aga koondab 
see raamat 29 suurepärase inimese lood. Ja ehk on sellest praeguseks küllalt.
