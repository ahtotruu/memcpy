\index[ppl]{Samuel, Tõnu}
\label{sisu:tonu}

\question{Kuidas jõudsid arvutid sinu juurde ja sina arvutite juurde?}

Arvutite juurde ma sain kolmeteistaastaselt (aastal 1985), kui mul tekkis 
esimest korda ligipääs arvutile, mis oli tegelikult programmeeritav 
taskuarvuti. Aga see oli ikka mäekõrguselt rohkem, kui mingisugune 
MK-51\sidenote{Elektronika MK-51 oli alates 1982. aastast 
Zelenogradis, Billuris ja Rodonis toodetud Nõukogude 
taskuarvuti, mis oli modelleeritud Casio FX-2500 alusel.}, mis oli tollal 
tavalisel koolijütsil kõige kõrgem unistus-arvuti. 

\question{MK-51 oli nõukogude kalkulaator ju?}

MK-51 oli jah nõukogude kalkulaator, mis oskas trigonomeetriat, aga mina sain 
ligi välismaa Casio PB-100-le\sidenote{PB-100 oli Casio üks esimesi ja 
lihtsamaid samme taskuarvuti juurest päris arvuti poole. See toodi nime PB-100 
all turule 1982. aastal ning 1983. aastal ka kui TRS-80 PC-4 (Tandy Radio 
Shack) ja OP-544 (Olympia). Tegu oli QWERTY klaviatuuriga päriselt 
programmeeritava arvutiga, kuigi üherealine ekraan tegi CASIO 
BASIC-us\index{BASIC!Casio BASIC} programmeerimise küllalt vaevarikkaks.}. 

\question{See on ju klassika?}

Nojah, mingis mõttes. Ta on taskuarvuti, tal on pool kilobaiti mälu (mis 
tänapäeval tundub ulmeliselt vähe), oskab ainult BASIC-ut. Aga tollal mul 
võttis silme eest kirjuks, sellepärast et klaviatuuril oli terve tähestik peal. 
Tundus ikka, et see on täielik kosmos. 

\question{Kust niisugune asi õndsasse nõukogude vabariiki sattus?}

Kõige esimene ligipääs oli lihtsalt selline, et ma nägin kuskil komisjonipoest 
sellist asja. Ja kuna sellel kalkulaatoril olid tähed, siis ta jättis minusse 
nii sügava mulje, et ma käisin rääkisin sellest igale sõbrale. Ja siis paar 
päeva hiljem tuli üks tuttav, kes juhuslikult oli suht rahakast perest, selle 
sama asjaga mulle nina alla liputama. Et \enquote{Mäletad, sa rääkisid mulle}. 
Siis ma sain seda natukene veel näppida. Aga see ei olnud minu oma, ma ei 
saanud seal suurt midagi teha. Läks mingisugune, ma ei tea, nädal kuu, mingi x 
aeg, mööda ja juhtus selline ime, mis juhtub ainult filmides. Noh, NSV Liidu 
ajal oli see täpselt sihukest laadi sündmus. Tuli info, et mul on välismaal 
rikas tädi. See \enquote{rikas} ma mõtlesin sinna juurde vist ise välja, tollal 
tundus iga asi rikas, mis oli välismaal. Aga selles mõttes et, \enquote{mul on 
vanatädi Šveitsis!} Täitsa juhuslikult see vanatädi saatis meile veel kirja ka, 
et \enquote{aga äkki tahate midagi siit}. Ja mina teadsin täpselt, ma tahan 
täpselt samasugust asja\sidenote{Tõnu peab oma armastatud Casio-t silmas.} 
saada. Ma kirjutasin kohe ära talle, et ma tahan täpselt sellist asja saada.

Ma olin kolmteist ja, ma  tagantjärgi just vaatangi, et tollal tundsin ennast 
väga täiskasvanuna, aga kui ma praegu vaatan kolmeteistaastaseid, siis ma saan 
aru küll, miks ma nihukese kirja kirjutasin. Väga lakooniliselt, et kui sa juba 
küsisid, siis üks arvuti oleks mulle just nagu vaja. Ja tuligi. See tuli küll 
pool aastat hiljem. Tollal käis niimoodi, et kiri käis siit Šveitsi kaks kuud. 
Siis tuli kaks kuud hiljem vastus, et unustasin tüübi kirjutada. Kusjuures, nii 
nagu oli paberkandja peal, ega ma ka ei tea, kas ma kirjutasin tüübi või ei 
kirjutanud. Igatahes pool aastat hiljem tuleb kiri, et \enquote{vabanda, tüüpi 
olnud kirjas, ma oleks sulle hea meelega saatnud}, ma olin nii solvunud. Siis 
ma kirjutasin jälle, siis ootasin veel pool aastat, siis ühel hetkel tuli 
tollist kiri, et  saatke viiskümmend rubla tollimaksuks. See oli tollal nii 
kosmiliselt suur raha\sidenote{Kooli raamatukoguhoidja kuupalk oli suurusjärgus 
100 rubla ja pudel viina maksis 10 rubla, pudel Žiguli õlut aga 33 kopikat.}, 
et ma kuskilt laenasin selle. 

\question{Tollimaks tol ajal? Loogiline, tegelikult, et kui importida, siis 
mingi maks rakendub.}

Ma olin tegelikult väga vaesest perest, viiskümmend rubla oli minu jaoks 
ulmeliselt suur raha. Ja kui ma lõpuks nägin tolli hinnakirja tuli välja, et 
välismaine kalkulaator on viiskümmend rubla. Aga sellel asjal oli kirjutatud 
karbi peale \emph{personal computer} ja selle puhul oleks maks olnud vist 
kolmsada või kolm tuhat või igatahes mingi nihuke number, mida ma kohe 
kindlasti ei oleks suutnud kuskilt leida. 

Sealt kuskilt hakkas asi minema. Kui ma olin viisteist,  mul õnnestus esimest 
korda kuhugi Normasse tööle saada. Ja kui ma olin kuusteist, siis selleks ajaks 
ma olin täiesti veendunud, et ma tahan arvutitega tööd teha. Haridust mul ei 
olnud, ma teadsin, et keegi mind tööle ei võta. Aga oli selline unistus, et ma 
lähen kuhugi koristajaks. Ma teadsin tollal, et on olemas arvutuskeskused. Need 
olid nihukesed kohad, kus oli tohutu suur arvuti, terve maja oli seda täis. Ja 
ma lootsin, et arvutuskeskuses äkki keegi võtab mind koristajaks. 

\question{Lähme korraks tagasi. Mind hakkas see personaalne kompuuter huvitama. 
Mis sa tegid sellega?}

Üks asi, ma lahendasin ruutvõrrandit koolis. Kolm numbrit sisse, kaks välja, 
väga kiire. Teised kõik tagusid oma kalkulaatorit, aga vigaselt ja said tihti 
valesid vastuseid. Aga, ma ei tea, see seitsmes klass oli selles suhtes imelik, 
me tegime vist pool aastat ruutvõrrandeid matemaatikas. See viskas kõigil  väga 
väga ära ja mina olin ainuke, kes kunagi ei eksinud. Sellepärast et mul oli 
kolm numbrit sisse, kaks välja. Kõige naljakam efekt oli see, et 
matemaatikaõpetaja lõpetas kodus asjade lahendamise, sest et tunnis käis 
niimoodi, et \enquote{davai, kodanik, see, tõuse püsti, loe oma vastused ette}. 
Ja pärast seda õpetaja lihtsalt vaatas küsivalt minu poole: kui ma noogutasin, 
siis  järelikult oli õige. Ma suutsin enam-vähem nagu reaalajas neid numbreid 
toksida, sel ajal, kui ta vastuseid luges. 

\question{Sest ega õpetaja tegi ju ka käsitsi, temal ka ei olnud sellist 
arvutit!}

Ei olnud jah, ta pidi tööd tegema selle nimel. Ja ka eksis. Oligi kuidagi 
niimoodi, et ma sain ükskord õpetaja vea kätte,  enam-vähem pärast seda ta vist 
loobuski. 

Teine asi, muidugi, inglise keele õpetajat sai trollitud. Koolis rumal küsimus, 
küsida inglise keele õpetajalt, kas eksamil arvutit võib kasutada. Nihuke 
natuke vanemapoolsem daam vaatab tõsiselt hämmastunult, ütleb, et \enquote{kui 
sul tast kasu on, eks kasuta}. Ja siis käis laua alt  klaviatuur mürtsuga laua 
peale,  terve sõnastik oli sees. Ja tal oli nagu väga piinlik siis  öelda 
ikkagi, et enam ei tohi. 

Aga noh, nad õppisid üsna ruttu muidugi.

\question{Mis sind selle aruvuti juures niimoodi paelus. Lihtsalt see 
klaviatuur, või et teda sai programmeerida või?}

Midagi oli. Ma tagantjärgi ei oska seletada, aga ma olen lapsest saati olnud 
selline laps, kes absoluutselt kõik asjad ära lõhub. Näiteks kodus oli raadio 
ja vot minule ei mahtunud pähe, kuidas saab olla inimene selle kasti sees. 
Selge see, et inimest seal  ei ole, aga sa ju kuuled, et on. Ja vot see 
konflikt tekitas mul selle, et ma lammutasin kõlari ära ja sain selle eest vist 
peaaegu peksa kodus. Tollal suur lampraadio, selline kast,  ja  ma uuristasin 
ennast  seal ees olevast võrgust läbi, et teada saada. Raadio oli tollal asi, 
mis osteti üks kord elus. Kui sa selle ära lõhud, siis \ldots. Pahandust oli 
palju. Kõik äratuskellad, lahti oskad ikka võtta, kokku panna enam ei oska. Ega 
ma selle taskuarvuti ka lõpuks lõhkusin ära. Ma üritasin teda lahti võtta, aga 
see oli kõik kleepsuga kokku pandud. Ja siis ma niimoodi venitasin teda lahti 
ja see kleeps muudkui venis ja venis ja ma kasutasin kääre, et nagu natukene 
kaasa aidata, aga seal oli üks lintkaabel sees, mida ma ei märganud ja suutsin 
selle ka läbi hammustada. Niimoodi. 

Aga see on meeletu huvi, kuidas asjad töötavad. 

Kuigi mul on üks väga oluline mentor ka elus olnud, naabrimees, kes viitsis 
mulle seletada. Lapsest saati olen igasugu asju pidanud ehitama, mingeid 
väikseid raadiosaatjad ja nihukesi vidinaid. Ma ei tea, sellel naabrimehel olid 
närvid ühelt poolt läbi, aga teiselt poolt aeg-ajalt ta viitsis. 

\question{Sa olid Tallinna poiss?}

Jah, ma olen Tallinnas kogu aeg olnud. 

\question{Siis neid juppe ja pudinaid, mille neid raadiosaateid teha, ikka 
liikus?}

Väga raske oli tegelikult. Tavaliselt asi läks ikka niimoodi, et said kuskilt 
mingi skeemi. Skeemid millegi hea tegemiseks olid tavaliselt just sellised, et 
kui sa jooksid sellega poodi, et nüüd ma hakkan  ehitama, siis selgus, et seda 
kõige põhilisemat asja ei ole. Näiteks käis tollal 
raadioajakirjas\sidenote{Tõenäoliselt peab Tõnu silmas ka mujal jutuks olnud 
ajakirja \begin{russian}Радио\end{russian}.\index{Radio}} läbi arvuti skeem. 
Tunduski täpselt nii, et  nüüd ma teen ise arvuti. Hoobilt jooksid, et kust ma 
need asjad kõik saan ja selgus, et neid mikroskeeme lihtsalt ei ole. 
Videokontroller ja CPU ja muu oli defitsiit, ma ei omanud ühtegi kanalit, kust 
midagi saab. Kindlasti, tuttavatel ma tagantjärgi tean, kellel töötasid  
vanemad sõjatööstuses, need suutsid kõike hankida. Ühel tuttaval olid näiteks 
vanemad keemikud,  kui poisikesena tahtsime igasugu plahvatavaid asju teha, 
siis me käisime nende keemikute juures ükshaaval aineid pinnimas, muidugi 
varjates nende tegelikku otstarvet. Ma ei tea, kas nad saidki aru. Tagantjärgi 
saan aru, et see oli väga rumal küsimus, nad pidid tegelikult aru saama, 
milleks me  näiteks salpeetrit vajame. Või, teine variant, just olid nii targad 
inimesed, et nad teadsid veel sadat otstarvet ja ei suutnud seal vahel enam 
ohtu näha. 

\question{Või mõtlesid, et kui poiss oskab juba salpeetrit küsida, siis vast 
näppe päris küljest ei lase?}

Vot, ei tea, meil keemiaõpetaja läks väga närvi ükskord sellepärast et me 
hakkasime nitroglütseriini valmistama\sidenote{Ka mina mäletan seda kihu. 1956. 
aastal ilmus Eesti Riikliku Kirjastuse sarjas \enquote{Seiklusjutte maalt ja 
merelt} Jules Verne \enquote{Saladuslik saar} ja seal oli juttu nii 
nitroglütseriini plahvatusjõust kui ka selle valmistamise protsessi 
detailidest.}. Ma tagantjärgi mäletan seda, kuidas keemiaõpetaja meile 
innustunult seletas, et see kodustes tingimustes ei õnnestu. Aga tagantjärgi 
saan aru, et see oligi täpselt see, et ta üritas nagu öelda, et ärge tehke, 
sest teisiti meid ei veena. 

\question{Nüüd on ju teistpidi. Kui läheb keegi eetrisse ja teatab, et nende 
süsteemid ei ole häkitavad, siis kohe palju inimesi proovib!}

Tavaliselt ta käib just sellise ülbuse noodiga sealjuures, et meie oleme nüüd 
paremad kui nemad. Ja kui ükskõik kellest väidad end parem olevat, siis 
tavaliselt see keegi solvub. Igatahes tavaliselt on mingi motivaator, miks 
selline väide toob sellise tulemuse. 

\question{Kust sul see arvutuskeskuse mõte tuli, miks just arvutuskeskus?}

Ma ei mäleta, kust ma selle info sain, et nihuke asi üldse olemas on. Meil vist 
üks tuttav käis kuskil arvutuskeskuses  mingi onu juures, õnnestus korra ennast 
sinna kaasa nihverdada ja see tundus nii põnev. Tuba oli arvutit täis. Igatahes 
teadsin, et ma tahan arvutuskeskusesse. Ja kuna ma midagi ei osanud, siis ma, 
jah, käisin mööda  uksetaguseid. Tagantjärgi saan aru, et ääretult naiivne 
mõte, et ma lähen sinna tööle. Käia viieteistaastaselt, ilma igasuguse 
hariduseta, uksi kulutamas. Aga võeti tööle. Kuusteist olin.

\question{Soh. Tegema mida?}

Arvutioperaator. 

\question{Keskkool jäi sul tegemata?}

Üheksanda, tollal kaheksanda, tegin lõpuks tagantjärgi ära. Aga sealt edasi ei 
ole midagi teinud. 

\question{Mis see arvutioperaatori töö endast kujutas?}

Oi, see oli väga ülbe töö selles mõttes, et ma mul oli lubatud isiklikult 
arvuti käima panna ja klahve vajutada. See ületas mu ootusi tugevalt, olin sel 
hetkel tõsiselt iseendaga rahul. Mitte midagi ma sellest ei teadnud, algne 
ülesanne oli lihtsalt mingit teksti sisse panna. Vene keelt ma oskasin hästi. 
Arvutuskeskuses üks osa tööst oli  tarkvara tegemine mingitele Venemaa 
asutustele. Meil oli asutusetäis tädisid, tollal mul oli tunne, et täiesti 
surmalähedased, vanamammid, sellised neljakümnesed. Kuueteistaastase poisi pilk 
on selline. Aga nemad siis programmeerisid  midagi, millest me (seal tuli paar 
poissi veel lõpuks)  nagu väga aru ei saanud. Kui nad kirjutasid kasutaja 
juhendi, nad kirjutasid selle käsitsi paberi peale ja meie pidime selle 
arvutisse sisestama ja välja printima. Aga see paratamatult tõi juba õiguse 
printerit näppida ja\ldots

\question{Mis süsteemi too arvutuskeskus siis kuulus?}

Sideministeeriumi. Ja see oli hästi edev selles mõttes, et asutuse nimi oli 
Sideministeeriumi Info- ja Arvutuskeskus\index{Sideministeeriumi Info- ja 
Arvutuskeskus}, mis oli äravahetamiseni sarnane Siseministeeriumiga. 
Töötõendiga anti kaasa punased kaaned, mis tegid nii mõnegi sellise vajaliku 
töö ära, kui oli vaja midagi kuskilt läbi suruda. Lõid oma kaaned lauale 
\ldots. See nägi välja nagu Siseministeeriumis töötaksid, tõsised punased 
kaaned olid, ma ei mäleta, kas Lenin ka peal oli. 

\question{Küllap oli. Sina sisestasid tekste, aga mida need programmid tegid?}

Ma lõpupoole pidin neid ise tegema. Oli palgaarvestus, mingisugune põhivara 
arvestus. Mul oli tollal poisikesena väga raske aru saada, et mis asi on 
põhivara ja mis asi on väikevaras, siis üks mammi seletas, et \enquote{vot sina 
istud väikevara peal, aga mina istun põhivara peal}. Sest kuskil viiskümmend 
rubla oli see piir, temal oli  viiekümne viie rublane tool. Mäletan seda, et 
kirjutasin mingi programmi, milleks oli postitoodangu arvestus. Löödi sisse, et 
 näiteks kirja kohale tassimine kellelegi koju on null koma null üks kopikat ja 
postkontor arvutas niimoodi oma toodangu mahtu. 

\question{Põhimõtteliselt jooksis seal arvutuskeskuses kellegi ERP?}

Jah, midagi niisugust. Kuigi tollal olid kõik need sõnad võõrad. See oli 
muidugi mingis mõttes väga nõme, mis tollal sai tehtud, tänapäeva mõistes väga 
lihtsad asjad. 

\question{See tekstide toksimine võis üsna nüri tegevus olla, kas see huvi ära 
ei võtnud? Või sa said millalgi nonde masinatega omi asju ka teha?}

Mul oli huvi nii suur, et ma istusin seal, jah, nii kaua kui  üldse kannatas 
olla. See venekeelne tekst, ega ma tegelikult sain sellega hästi hakkama. Vene 
keele klaviatuuri on ju teistsugune, kui ladina klaviatuur. Aga sellega harjub 
ära ja kui vene keel on käpas\ldots Eesmärk oli saada töö kaelast ära, et teha 
endale huvitavaid asju. Kui ma sinna majja läksin, siis ma tegelikult ei osanud 
ma midagi. Ja pool aastat hiljem oli umbes niimoodi, et ma teadsin kõikidest 
nendest tädidest rohkem. Ja kuna meil tekkis sinna mingisugune paaripoisine 
seltskond, meid oli  peale minu neli tükki veel, siis me  hakkasime omavahel 
infot jagama. Tollal internetti ei olnud, mitte midagi ei olnud kuskilt võtta, 
siis enamus käis kuulujuttude ja ise katsetamise teel. 

Me õppisime näiteks residentseid programme kirjutama. Tänapäeval võiks seda 
isegi viiruseks nimetada.  See oli tollal nii \emph{high tech}\ldots

\question{Kuidas te sihukese asja välja uurisite? See oli ju nõukogude arvuti?}

Meil olid erinevad. Arvutuskeskuses suured masinad oli ES-1022\index{ES 
EVM!ES-1022} ja ES-1045\index{ES EVM!ES-1045}, kaks arvutit. Kuus megabaiti 
mälu oli, terve korrus oli selle jaoks. Ferriitmälu, bitid olid ükshaaval 
silmaga kõik näha. Aegajalt läks mõni bitt tuksi,  insenerid parandasid neid. 
24 inseneri oli kokku, kes seda kõike siis lappima pidid. Ja kuna seda mälu 
pidi kord nädalas piiritusega puhastama, siis firmapeod möödusid nagu ilma muu 
alkoholita.  

Aga meil oli teine osakond ja meil olid kõik personaalarvutid. Meil olid 
kaheksabitised Robotronid, 1715\index{Robotron!Robotron 1715}. Siis tulid 
Iskra-1030-d\index{Iskra!Iskra-1030}, mis olid PC kloonid, kohutavalt halvad 
arvutid. Kuskilt tuli mingi DVK-2\index{DVK!DVK-2}, mis oli mingi IBM-i kloon. 
Igatahes asi muudkui arenes ja näiteks DVK-2 on selles mõttes naljaka arvutina 
meelde jäänud, et arvutuskeskus oli selline suhteliselt külm maja. Endla 16, 
tänapäeval Eesti Telefoni maja\sidenote{Selles majas tõesti asus üksvahe Eesti 
Telekom, aga praeguseks on maja põhjalikult renoveeritud ja seal asub midagi 
muud.}, mis ei pidanud sooja. Hommikul oli tubades pluss viisteist kraadi, või 
vähemgi, ja arvuti ei lähe selle külmaga käima. Flopi draivi rihmad olid 
kuidagi nihukesed jäigad. Mootor käis rihma sees ringi, flopi ringi ei lähe. 
DVK-l oli nii suur auk seal ees flopi jaoks, et sinna mahtus käsi sisse. 
Pistsid käe sisse, tõmbasid käega kogu selle asja ringi käima, et sai hoo sisse 
ja lükkasid floppi ruttu järele ja luugi kinni. Kui seda piisavalt kärmelt 
teha, sai arvuti käima. Hiljem, päeva peale, ei olnud enam probleemi. 

\question{Ikkagi, kust te infot saite? Manuaalid?}

Manuaalid olid kasutud. Täiesti kasutud. Kui Iskrad\index{Iskra!Iskra-1030} 
tulid, me  veel eraldi mõnitasime, sest et selle peae oli kirjutatud 
\begin{russian}электронная персональная вычислительная профессиональная 
машина\end{russian}. Personaalne professionaalne elektrooniline arvutuslik 
masin. 

\question{Kuidas te siis oskasite? Ei ole ju nii, et paned aga suvalisi käske 
ja äkki jääb programm residentseks?}

Üks asi on see, et tollel ajal kui mina arvutuskeskuses olin, seal võrku ei 
olnud, mitte mingisugust. Ja enamus asju käis nii, et keegi käis teises 
arvutuskeskuses külas, seal jälle keegi kuskilt oli midagi nuuskinud, kas ta 
oli välismaalt kuulnud või mujal käinud, aga tavaliselt info liikus koos 
inimesega. Käis kuskil külas,  tuli tagasi ja  oli lõpuks õppinud, kuidas teha 
mingit asja, mida me olime juba pool aastat mõelnud. Kusjuures see probleem oli 
tavaliselt väike. Meil näiteks printeril ei olnud täpitähti. Ja siis keegi tuli 
väga-väga kavala asjaga, kuidas \emph{map}-ida  klaviatuuril kandilised sulud 
täpitähtedeks. Tuli selle infoga  näiteks Tervishoiuministeeriumi 
arvutuskeskusest, sest keegi seal oli jälle mingi häkiga hakkama saanud. Või, 
teine häda oli see, et klaviatuurid olid aeglased. Meil oli tihti vaja ju 
mingeid jooni teha niimoodi, et tuli hästi palju miinuseid panna. See miinus, 
hoiad teda  nagu ma ei tea mida, aga ta ei jookse. Keegi õppis ära seda 
kiirendama, selleks tuli kuhugi porti kirjutada mingi number. Aga selline info 
liikus ainult suust suhu. Assemblerit\index{Assembler} ja nihukesi asju me juba 
oskasime kõik ise teha, aga muu oli nihuke müstika. Mingid pordid ja mida sinna 
kirjutada oli dokumenteerimata. Kuskilt tuli see info, et kui sa sinna porti 
kirjutad, läheb see klaviatuurile ja klaver on kiirem. 

\question{Kas te seda folkloori kuidagi üles ka kirjutasite või see lihtsalt 
jäi inimeste pähe?}

Ei, täiesti suuline. Seda ei olnud nii palju tollal, ei tulnud nagu pähe üles 
kirjutada. Teine asi oli, et see oli kõik selline info,  mida sa niisama ära ei 
andnud, sest see tõstis sind teistest kõrgemale. Näiteks ega me nendele 
tädidele ei rääkinud, kuidas residentseid programme kirjutada. Üks asi, et me 
pidasime neid madalamaks kassiks, kes nagunii aru ei saaks. Teine asi, et see 
andis meile võimaluse tehaseal arvutis  mida iganes ilma, et nad aru oleksid 
saanud. Oleks tollal tahtnud mingit kräkki jooksutada, me oleks seda 
jooksutanud \emph{background}-is.

\question{Aga nende teiste arvutuskeskuste tüüpidega te oleks seda infot 
jaganud?}

Tavaliselt see käis, jah, vorst vorsti vastu.  Sul pidi olema mingi mõju 
inimese üle. Tavaliselt professionaalid hindavad üksteist ja jagavad sellist 
infot, aga kui tuleb mingi jobu küsima, ega sa ei ütle talle. Sina oled palju 
vaeva näinud, et see asi ära lahendada. Ringi jooksnud ja küsinud ja mõelnud ja 
sa ei anna seda infot niisama ära. Vahest antakse, aga see oli osa 
käitumismustrist. 

Fidonet tõenäoliselt oli üks esimesi arvutivõrke, millest ma kuulsin ja kuskil 
nägin. Praegu ma arvan, et see võis Tarmo Mamers\index[ppl]{Mamers, Tarmo} 
olla, aga ma ei ole kindel. Igatahes ta hakkas nihukeseks asjaks muutuma, et ma 
hakkasin töö juures rääkima, et ma tahan modemit saada, tahaks nagu võrku. Meie 
arvutuskeskus ei pidanud seda kuidagi vajalikuks ja ma ei osanud  kuhugi 
õigesse kohta vajutada ka. Mulle korra vist insenerid kuskilt tagatoast 
pakkusid mingit modemit, mis oli niisugune kingakarbi suurune kast ja mis ei 
olnud see mudel, mida sa saad telefoniliini otsa panna. Oli mingisugune 
\emph{leased line} modem või midagi, ta ei osanud helistada, näiteks. Igatahes 
minu jaoks oli see täiesti tarbetu,  mul oli vaja seda modemit, mis käib 
telefoniliini külge. Telefoniliin oli ka tol ajal väga suur ressurss. Meil oli 
asutuse peale piiratud arv telefone, kusjuures me olime Sideministeeriumi 
arvutuskeskus! Jaam oli meie enda majas, aga meie osakonnas oli kakskümmend 
inimest ja kaks numbrit. 

\question{Küsin nende inseneride kohta. Kui ma mõtlen, kas lasta noor inimene 
tarkvara või riistvara juurde, ma julgeks teda pigem riistvara juurde lasta. 
Miks sind tarkvara juurde lasti?}

Aga riistvara tollal meil, ma ei mäletagi, et midagi erilist oleks olnud.

\question{No oli ju, ma mõtlen, et oleks pandud piiritusega ferriitrõngaid 
nühkima või midagi?}

Tegelikult see vastus, miks ma sinna tööle sain, on mul tagantjärgi nagu 
tulnud. Tollal ma uskusin, et ma jätsin nii tõsise mulje, et ma olin oma 
taskuarvutiga mingeid programme teinud. Tagantjärgi ma saan aru, et sellel ei 
olnud mingit asja. Tegelikult oli probleem, tollal ei olnud IT eriala üldse 
populaarne. Keegi ei tahtnud seal töötada, palgad vist olid madalad. Tõsine 
mees tegi kuskil haltuurat. Näiteks kui sa töötasid kuskil viinapoest, sa said 
sealt müüa midagi või teha. Arvutuskeskus oli koht, kus sa said oma ametliku 
palga ja midagi varastada ei saanud. Mingi reputatsiooni häda oli, sinna läksid 
kõik sellised matemaatikaharidusega naisterahvad, kõik nihukesed teravamad 
tüübid läksid kuhugi traktoristiks, seal sai kütust varastada või midagi. 
Igatahes seal oli mingi häda, miks sinna tööle ei mindud. 

\question{Aga siis tuli üks, kes tahtis!}

Jah. Ja mul on mulje, et, vaata, osakonnas oli kakskümmend naist ja osakonna 
juhataja oli mees. Tagantjärgi ma olen enda jaoks selle dekodeerinud niimoodi, 
et ta oli andnud kaadriosakonnale käsu, et ükskõik mis meesterahvas tuleb, 
tuleb lihtsalt kinni võtta ja temale anda. Sest nende paari aasta jooksul, mis 
ma seal olin, ma nägin, millised kismad seal käisid, naised ikka kraapisid 
üksteise silmad verele. Midagi füüsilist ei olnud, aga  nihukest ussitamist oli 
korralikult. Ja, mida ma olen hiljem näinud, et kollektiivis peavad 
mehed-naised mingis tasakaalus olema, vastasel juhul, mõlemas suunas, tekib 
jama. Ja mul on tunne, et mina olin esimene meesterahvas, kes talle ukse taha 
tuli, teda ei huvitanud ükski muu asi peale selle, et ma olen mees.  

\question{BBS-i aeg sattus juba sinna arvutuskeskuses olemise aja tagumisse 
otsa, sa rääkisid, et tahtsid sinna modemit saada?}

Hakkasin tahtma aga ei saanudki. Tollal, 1989 või niimoodi, hakkasid 
kooperatiivid. Täpsemalt ei mäletagi, aga  kooperatiivinduse tüüpidel oli raha 
paksult käes. Kui sa  midagi väga tahtsid, siis nad sulle ostsid. Ja mul 
õnnestus saada modemi ligi niimoodi, et see oli ainult minu kasutada, see 
arvuti ja see modem, ja ma suutsin ennast Fidonetti ajada. Ja vot Fidonet oli 
kullaauk, see oli täpselt nii nagu tänapäeval Internet. Kus seal info jooksis! 
Kui kellelgi midagi oli, siis sellest räägiti ja see seltskond tundus nii 
kohutavalt suur võrreldes selle paari arvutuskeskuse inimesega, kellega ma 
tavaliselt lävisin ja keda ma nagu  vääristasin endaga võrdseks. Nüüd oli 
kümnete viisi inimesi. Tänapäeva Internetis ei kujuta ette, et kuidas 
\enquote{kümnete viisi},  aga Fidonet ise oli Eestis, ma ei tea, mingi 
viiskümmend aktiivset inimest või kuni sada, mitte rohkem. Ja need inimesed 
olid targad. Sa jooksid hommikul arvuti juurde, et panna see käima, tõmmata 
kõik viimased kirjad ära ja vaadata, mida nad räägivad. Sest oli terve hulk 
selliseid legendaarseid mehi, kelle iga kiri oli kulda täis. Sa ainult lugesid 
seda, mida nad räägivad. 

\question{Näiteks?}

Mulle kuidagi, Sulo Kallas\index[ppl]{Kallas, Sulo} oli nihuke inimene, kes 
jättis mulle mulje. Tollal CD plaat oli asi, mida me teadsime, et selline asi 
on olemas, aga oma silmaga polnud näinud. Seda reklaamiti, et  nüüd on lõpuks 
ometi puhas heli. Sulo Kallas oli mingi audio-friik, kes sai endale Sony CD 
mängija sel ajal, kui teised igatsesid endale vene oma. Ja  ta tunnistas heli 
kvaliteedi ebakõlblikuks, pildus selle kasti sisust tühjaks, tegi sinna uue 
elektroonika. Minule jättis see kustumatu mulje, ma mäletan seda mitukümmend 
aastat hiljem. Ja nüüd, kus ma olen selle härraga  ise koostööd teinud, ma 
endiselt austan teda samamoodi. 

\question{Kui sa alguses olid Fidos klient, siis ühel hetkel hakkasid sa oma 
\emph{node}-i ka pidama. Millal see tuli?}

See tuli üsna ruttu. Ma olin kellelegi \emph{point} alguses, aga ma olin mingil 
hetkel  \emph{node} numbriga 25. Ja minu jaoks Tarmo Mamers\index[ppl]{Mamers, 
Tarmo} on  alati olnud see vaimne isa, kelle käest ma olen väga palju vastuseid 
saanud, väga palju abi ja tõenäoliselt ma tõmbasin tema juurest alguses ka kogu 
oma meili. Hiljem muutusin ise nii suureks, et ma näiteks vahendasin kogu 
Venemaa meili Eesti vahel. 

\question{Kas Venemaal BBS-id ja Fido või mõlemad olid kuidagi vähem levinud?}

Millegipärast jah, ja ma ei teagi, miks. Oli kuidagi niimoodi, et Eesti päris 
algusaegadel, mina olen sellest otsast ilma jäänud, käis kõik Soome küljes. 
Eestit ei tunnustatud välismaal üldse, meil puudus oma aadressruum, kõik olid 
Soome \emph{node}-d. Millalgi Eestis hakkas see asi kasvama nagu seen pärmi 
peal ja siis Vene omad olid kõik Eesti \emph{node}-d. Venemaa võrk oli minu 
meelest umbes sama suur, kui Eesti oma. 

\question{Ma olen kuulnud legende sellest, kuidas kuskilt kaugelt Venemaa 
avarustest käidi lennukiga Eestisse Fidosse. Kohver flopisid kaasas ja muudkui 
kopeeriti öö läbi.}

Aga tollal oli see vist isegi kiirem, sest üle telefoni läksid asjad nii 
aeglaselt, et oli odavam kohale lennata,   lendamine oli odav. Minu meelest 
Moskva lend maksis kas üksteist rubla või mingi niisugust, mis oli üsna väike 
raha.

\question{Kuidas Eesti oma tsooni saamine käies? Oli ju Nõukogude Liit veel, 
või ei olnud enam?}

Vaat ei oska öelda. Mina jäin sellest otsast ilma. Äkki Sulo 
Kallas\index[ppl]{Kallas, Sulo} või keegi teine juur-guru vanematest aegadest 
oskab sellest rääkida. 

\question{Fido \emph{node} sa panid püsti, kas sul BBS ka olnud on?}

Mul ei ole kunagi otseselt BBS-i olnud. Kindlasti ma olen midagi mänginud ja ma 
olen paar tükki  äri-inimestele püsti pannud  kommertsasjade jaoks. Neid 
tegelikult \emph{impress}-is kohutavalt see, et  mingid tüübid olid neilt paar 
aastat raha küsinud, et midagi programmeerida ja nad ei olnud kunagi näinud, 
mis sellest programmeerimisest tulu sai.  Tuli Tõnu, küsis mingisuguse mõttetu 
raha ja kaks tundi hiljem asi töötas. Ma mäletan, et tollal see jättis 
kustumatu mulje, et \enquote{lõpuks ometi keegi, kes aitas}. Aga ma sealt edasi 
ei ole midagi teinud. 

\question{Oot, kuskil olid mingid inimesed, kellel oli äriline põhjus BBS püsti 
panna?}

Tollal oli see lootus, et äkki nüüd hakkab äri minema, sest tulid inimesed, kes 
ütlesid, et nüüd läheb kõik internetti. Kogu äri, kõik. Ma toon samast ajast 
võrdluseks sellise pisiasja, et  tol ajal kõige populaarsemal tarkvaral, 
Maximusel, mida kõik BBS-id kasutasid, oli \emph{user}-i  ID  ühebaidine. Ma 
tahan öelda, et tänapäeval  ei kujuta ette, et sa teeks nii mingit tarkvara. 
Keegi hoidis ruumi kokku ja ta tegi \emph{user ID} ühebaidse. Enamust BBS-e ei 
kasutanud nii palju kasutajaid, et nad oleks sellest välja jooksnud. 

\question{Järelikult tehti disainis õige otsus!}

Eestis oli üks või kaks BBS-i, keda see ühel hetkel tõsiselt häirima hakkas. Ma 
tahan öelda, et see asi ei olnud üldse nii suur. Mina tunnetan seda siiamaani, 
kui väga elitaarset seltskonda, sellepärast et tegelikult kõik, kes sinna 
suutsid tulla, olid targad inimesed. 

Näiteks Fidonetti saamiseks oli vaja kolm erinevat tarkvara koostöös tööle 
panna, et sa suudaksid üldse sinna ligi saada, see ei olnud üldse nii lihtne. 

\question{Sul hakkas siis Fidos tekkima mingi seltskond tuttavaid, kellega 
juttu rääkida?}

Jah. Ja huvitav on see, et need suhted kestavad nagu üle mitmekümne aasta. 
Näiteks Venemaale me hakkasime äri tegema, nendesamade Fido \emph{node}-dega, 
kes sealpool olid. Hüperinflatsioon oli selline kummaline asi, et kuna 
Nõukogude Liit oli suur, siis ühes otsas hinnad liikusid kiiremini kui teises, 
Eesti oli Moskvast kuni nädal aega hindadega maas. 

\question{Ahaa, ja sul oli infot ja sa said seda vahendada!}

Enamus inimesi ikkagi toimetas kuskil ajalehtede peal kuulutuste läbi või 
niimoodi. Aga mina  rääkisin tuttavaga Peterburis või Moskvas, et 
\enquote{kuule,  ma tahan printereid saada}. Tema ütleb, et hind on selline. 
Mina ostsin seesama õhtu pileti, hommikuks olin juba seal, ladusin kõik asjad 
peale, ülejärgmise öö ma tulin Eestisse, müüsin nad kõik 
Kinexisse\index{Kinex}\sidenote{Üks varaseid Eesti arvutifirmasid, hiljem 
tegeles äritarkvara ja sellega seotud konsultatsioonidega.} maha. Ja Kinex oli 
nii õnnelik, nad ostsid mult kõik kakskümmend printerit korraga ära. Raha oli 
päris palju tollal. Mis sealt vahelt sai, sellega panid jälle Venemaale, tõid 
järgmise kuhja.

\question{Ja tulidki arvutuskeskusest ära ja hakkasid sihukest äri tegema?}

No enam ei olnud mõtet. Ma mäletan, kuidas ma üldse eraärisse sattusin. No see 
on täitsa omaette lugu. Arvutuskeskuses ma sain 110 rubla miinus maksud. 
Eraäris sain juba päris alguses kaheksasada rubla päevas. Neil lihtsalt ei 
olnud mitte millegagi mind enam motiveerida. Nad motiveerisid mind ainult 
sellega, et \enquote{Tõnu, sinu programmid näevad paremad välja kui minu omad}, 
ma kirjutasin neile sihukesi jubinaid, mis käivitasid nende programme. Noh, et 
nad tegid oma moodulid igaüks eraldi binaarina ja minul oli mingisugune 
akendega asi, mis neid käivitas. Akendel olid varjud taga, ega seda ka igaüks 
teha ei osanud. 

\question{See oli siis tekstipõhine värviline terminal?}

Puhas tekstivärk. Meie arvutuskeskus oli sellepärast, muideks,  teistest  nagu 
halvem, et kõigil teistel oli mingid graafilised asjad. Meile oli neid kas vähe 
või ei olnud üldse. Ma tundsin ennast maru halvasti, sest teised mängisid 
värvilisi mänge ja mina ei saanud. 

\question{Mis mänge sa mängisid?}

Digger\index{Digger}\sidenote{Digger on 1983. aastast pärit arvutimäng, 
suhteliselt lihtsa graafikaga ja omal ajal väga levinud.} näiteks, mis tollal 
oli täisvärviline. Aga meil olid Iskrad\index{Iskra}, millel olid rohelised 
ekraanid. Ja nad olid kõik, muideks, veel null ja üks, polnud  pooltoonegi. Me 
häkkisime  jootekolviga, et saaks  vähemalt pooltoonid. 

\question{Järelikult see jootekolvi ja inseneri-huvi oli tol ajal olemas?}

Oli, aga tegelikult oskused olid madalad.  See info tuli  jälle mingist teises 
arvutuskeskusest, et vot sinna kohta tuleb panna kaks takistit. Praegusel ajal 
ma suudaksin absoluutselt ise selle esimese viie minuti jooksul nagu välja 
mõelda, et teeks sellise \emph{fix}-i sinna monitori. Aga tollal me kõigepealt 
pool aastat istusime nende monokroomsete täiesti üks-null monitoride taga. 
Alles siis keegi tuli ülimalt hea infoga, et kui nüüd monitor lahti teha, seal 
kaks juhet lahti võtta, takistid vahele panna, tekivad pooltoonid. Meil käed 
värisesid, kui me seda tegime. 

\question{Muidugi, monitor oli ju kallis!}

Sellel ei olnudki hinda. Lihtsalt, kui sa ta katki tegid, siis rohkem ei saanud.

\question{Aga oli piisavalt julgust, et kaas maha võtta?}

Kuidagi oli. Kui viis poissi koos on, küll see julgus tekib. Ei ole vist ilus 
öelda, aga me aeg-ajalt seal ikka jõime ka koos. Tollal tekkisid mingid 
arusaamad, et kui nagu natuke peale võtta, siis programmeerimine edeneb 
kiiremini. 

Kusjuures mingi huvitav nähtus oli see, et kui hommikul iseenda kirjutatud 
tarkvara vaatasid, oli tunne, et mingi väga tark inimene on kirjutanud. Aru ei 
saa, töötab, aga nagu puutuma läksid, läks katki. See oli nagu  kõrvalefekt. 

\question{Sa ütlesid, et Fidost hakkas kohe infot tulema. Mis infot? Manuaale?}

Manuaale kui selliseid tollal ei olnudki, sest kõik liikus prinditud info peal 
ja ei olnud ju OCR-imiseks mingeid lahendusi, mitte midagi. Üks osa oli see, et 
kui keegi midagi teada sai, siis ta seda levitas. Teine osa oli igasugune 
ostan-müün-vahetan asi. Nõukogude Liidus oli kõigest puudu. Oli väga oluline 
infot, et keegi midagi müüb. Keegi müüs oma vana tooli näiteks, üks jalg oli 
alt ära, või mida iganes, aga see oli NSV Liidus väga väärt info. Ostsin oma 
esimese auto  Indrek Sauli\index[ppl]{Saul, Indrek} käest, kes oli tollal 
aktiivne Fidokas. Tal oli  Žiguli eksportvariant. Eksportvariant oli tavaliselt 
1500-se mootoriga, aga temal oli 1600! No enam nagu kõvemat autot ei andnud 
ette kujutada ja ma ostsin selle ära. Ja kuna ma olin seal võrgus, ta reklaamis 
seda seal, kus ainult paarkümmend inimest nägi, siis oli mul eelis.

\question{Kas mingeid mänge, muusikat, graafikat, sellist kraami ka liikus?}

BBS-ides väga palju. Aga tollal oli arvutivõrk niivõrd aeglane, et poleks 
tulnud isegi mõte mingite binaaride saatmisest mailiga. Tollal  sa vaatasid, et 
\enquote{oi, siin on kümme kilobaiti suur asi} ja  panid modemi ööseks tõmbama. 
Ma ei oska head võrdlust tuua, sest poolel rahval olid mingi 1200-boodised 
modemid. Peter Marvet\index[ppl]{Marvet, Peeter} kirjutas  suhteliselt ülbe 
tooniga meili, et  9600 bps modem on see, millega sa tunned, et tegemist on 
\emph{communication}-iga. 9600 bps oli siis see asi, mille üle sa võisid uhke 
olla, eks.  Ja oligi! Mina ei jõudnud seda osta, aga temal oli selline kuskilt 
saadud, ta oligi minust kõrgemal. 

Peamine oli see, et  kui sa midagi väga otsisid, siis keegi teadis, et 
\enquote{vot seal BBS-is ma nägin seda}, sest enamus ikkagi taandus 
piraattarkvarale. Muusika tuli veel hiljem. Kui MP3 tuli, ma mäletan siiamaani, 
et MP3 korralikuks mängimiseks oli vaja 100Mhz 486-te, mis oli täpselt selle 
piiri peal, et kui sa hiirt liigutasid,  oli muusika kinni. Ja kui tulid 120Mhz 
486-d, siis sa võisid hiirt ka liigutada.

\question{Demoskene asjad ju liikusid?}

Oi, demod liikusid, see oli ilus! Ma siiamaani igatsen neid vidinaid, mis olid 
imeväikesed, aga kui käima tõmbasid, siis oli tuba muusikat täis, ekraan oli  
graafikat täis. Nagu kolmemõõtmelised mingisugused\ldots See tundus mulle 
tollal kosmiliselt ilus. CGA graafika\sidenote{Võimaldas 320x200 
ekraanilahutusega kuvada nelja ja 640x200 lahutusega kahte värvi.}, mida 
tänapäeval keegi ei vaata. Ma mäletan seda hetke, kui ma Diggerit\index{Digger} 
esimest korda nägin. Meil polnud arvutuskeskuses ühtegi helikaarti ja ma käisin 
Tervishoiuministeeriumi Arvutuskeskuses\index{Tervishoiuministeeriumi 
Arvutuskeskus}, kuskil  kellelegi selja tagant korra mingeid asju ajamas ja 
keegi mängis Diggerit Olivetti arvuti peal. See heli ja see kõik tundus nii 
võimas ja need värvid ja! 

Muideks see Tervishoiuministeeriumi Arvutuskeskus oli seal kõrval, kus on 
surnukuur. Minu meelest selle tänava nimi on Tervise tänav, mis surnukuuri 
viib. 

\question{Mõnikord käib arvuti-huviga ka ulme-huvi kaasas, kas sinul ka?}

Ma ei mäleta. Ma lugesin raamatuid lapsena palju väiksemana. Mul on kuidagi 
niimoodi, et ma olen igasugu seiklusjutte vee alt ja kuu pealt sarjad kõik läbi 
lugenud sest ma loen väga varajasest ajast ja väga  ulmeliselt palju. Aga 
selleks ajaks mul oli rohkem huvi mitte selle nagu väljamõeldud maailma vastu 
vaid mind meeletult  huvitas reaalne maailm. Sest see väljamõeldud maailm on, 
on mis on, ta on alati välja mõeldud. Ma olen täiskasvanueas kogu aeg fakte 
otsinud. Mind meeletult huvitavad faktipõhised asjad, näiteks ajalugu. 

\question{Ajaloo kohta mõni inimene ütleb, et see ei ole ju fakt. Puhas 
arvamus.}

Vot see on NSV Liidust tulnud inimese üks hea omadus, et sa suudad päris palju 
filtreerida. Ma selles mõttes kuulan Vene propagandat hea meelega selle nurga 
alt, et ma saan üsna hästi aru, mida nad üritavad näidata ja mis osa siis  
tegelik on. Mind, see ei häiri niivõrd kohutavalt, sest aeg-ajalt just see, 
kuhu nad täna propagandat suunavad,  annab ise juba vastuseid. 

\question{Pärast toda arvutuskeskust sa toimetasid iseseisvalt või oli sul 
miski kooperatiiv?}

Ma sattusin eraärisse niimoodi, et  hakkasin tooma Venemaalt arvuteid ja 
teenisin selle eest tolle aja kohta meeletut raha. Kuigi see oli imelik aeg, et 
see meeletu raha oli ikkagi väiksem, kui kellegi teise meeletu raha. Aga kuna 
mul  tollal praktiliselt kõiki Eesti igasugu arvutipoed ja kõik olid kliendid, 
siis  üks, kellele ma kogu aeg asju vedasin, ütles, et \enquote{kuule,  hakkame 
parem koos tegema, et mis sa siin jahmerdad}. Igatahes tal oli kuidagi see 
idee, et ta müüb kogu mu kolu maha, aga mina toon ainult talle. Mind see 
kuidagi huvitas, sest muidu ma käisin mööda linna ja otsisin, kellele ma oma 
kolu lükkan. 

\question{See tähendab, et sul pidi olema päris korralik suhtevõrgustik nii 
Venemaal kui Eestis}

Kuigi see arv oli väike, aga võrgustik oli korralik ja, ütleme, suhted kestavad 
siiamaani. Näiteks Venemaal ma olen püüdnud igasugu asju ajada ja olen alati 
lõpuks petta saanud. Ja  on üks inimene, keda ma usaldan seal siiamaani 
siiralt. 

\question{Ja see võrgustik tuli puhtalt ainult Fido pealt?}

Jah. Need vanad võrgustikud ongi erakordselt usaldusväärsed. Inimesed, kes 
kolmkümmend aastat on üksteist tundnud, üksteisele mingit jama kokku ei keera. 
Sest see kommuun on selline, et kui oled seal nagu pleki endale külge saanud, 
ma ei kujuta ette, kuhu siis enam  taganeda. 

\question{Ma ei kujuta hästi ette, et Fido seltskond oleks ideaalsetest 
inimestest koosnenud. Kindlasti mingil hetkel visati keegi välja ka?}

Mind  visati ka välja. Ma ei mäleta, mida ma halvasti ütlesin, aga Tarmo 
Mamers\index[ppl]{Mamers, Tarmo} viskas mu väga kiirelt välja ja see oli väga 
hea õppetund, et Fidos ei ole demokraatiat. Fidos on igaühel oma kuningriik ja 
sa oled alati kellegi kuningriigis. Sa pead tema reeglite järgi mängima ja 
kõik. Kui sa tahad, sa võid oma kuningriigi luua, sinna rahvast hakata 
meelitama, aga tavaliselt sa istud üksi seal. 

\question{Aga see seab selle esimese adminide sauna-õhtu ju hoopis teise 
valgusse!}

Neil inimesed oli reaalne võim sind informatsioonist ära lõigata, aga seda 
tegelikult ei kuritarvitatud. Kes seal asja eest kinga said, need said. Ja mina 
sain ka asja eest. Aga oligi täpselt see, et väga kiirelt said aru, et kus need 
piirid on. Kakskümmend neli tundi hiljem olin ma tagasi, sest ma olin kenasti 
õige inimese juures vabandamas läinud ja rohkem ei teinud. 

Kloune oli seal üht ja teistsuguseid ja võimuga inimesi oli erinevaid. Näiteks  
oli teada, et üks mees oskab karated, kui on vaja kellelegi peksa anda, siis 
oli teada, et ennemini räägiks temaga. Samas kui elektroonikat on vaja teha, 
tead tolle teise inimesega juttu rääkida. Näiteks Madis Kaal\index[ppl]{Kaal, 
Madis}, kes hiljem Skype'is töötas. Tema oli see vend, kes julges seda eriti 
kallist asja, nagu arvuti, ennast häkkida. Me enamus ei julgenud, arvuti oli 
sul üks elu jooksul. Aga tema siis kraapis seal vaibanoaga mingid rajad lahti, 
panin relee vahele, et modemit lahti ühendada, kui see lolliks läks. See tundus 
nii riskantne tegevus, et isegi teades, mida teha, ma ei julgenud seda teha. Sa 
teadsid, et kui sul on riistvara probleem, siis sa lähed temaga rääkima. 

Siis seal mingid tegelased kogu aeg midagi müüsid ja oli teada, kelle käest 
mida saab. Kõik need tollased Microlinkid ja asjad,  seal oli igaühel  üks 
inimene, kes seal töötas. Et kui sa teadsid, et sa tahad allahindlusega asja 
saada,  siis tuli temaga suhelda. Fidokatel omavahel tavaliselt tehti mingeid 
allahindlusi. 

\question{Sest kõik ju said aru, et homme on mul vaja, võrgustik oli tihe. Mis 
selle võrgustiku nii tihedaks tegi? Kas vastastikune respekt kõrge barjääri 
tõttu või midagi veel?}

Respekt kindlasti, sest sul ei olnud alternatiivi. Nagu su oma tsunft, et kas 
sa oled seal või sa ei ole seal, kaks varianti. Need olid targad inimesed ja ma 
ei tea, kas see  ka nagu oluline on, aga tänapäeva internetiga võrreldes on üks 
tohutu vahe. Kuskil aastast 2000 edasi, on internetti tulnud lollid. Ma ei 
mõtle kõiki. Aga varem oli see, et kui sa lugesid midagi võrgust, siis see oli 
kuld. Siis nii oli. Ja kuskil seal üheksakümnendate lõpus, kui Eestisse pandi 
püsti Delfi ja \ldots See ei olnud  selle portaali häda, aga igaüks sai endale 
neti koju. Kõik need horoskoobid ja muu jama hakkasid nagu nagunii hullusti 
levima, et nüüd enam ei tea, mis internetis on tõsi. 

\question{Targad inimesed oleksid ju võinud oma kogukonna kolida ju teise, 
kõrgema, barjääri taha?}

Fido on eksisteerinud  tükk aega, ta vist eksisteerib mingil kujul siiani. 
Seesama barjäär on ka väga tarkadel inimestel lihtsalt jalus, ega nad ei viitsi 
seda teha. Ja tollased piirangud ikkagi \ldots Näiteks see, et need asjad ei 
käinud reaalajas, sa pidid kuhugi helistama, saatma oma kirjad ära panema toru 
hargile. Keegi teine pidi helistama sinna sama numbri peale, kus sina olid toru 
hargile pannud (muidu ta ei saanud helistada), tõmbama kirjad ära. Aga tema ei 
teadnud, et ta peab täna helistama, et sina oled sinna saatnud. Kui ta otsustas 
sulle vastata, siis tavaliselt see käis kuskil kahekümne nelja tunnise 
tsükliga. Ma kirjutasin oma mure ära ja sain sama päeva sees kuidagi oma 
vastused kätte. 

\question{Mina tean sind ikka rohkem infoturbe-inimesena. Mis hetkel sa 
liikusid arvutite toomise juurest arvutitega seotud probleemide lahendamise 
juurde?}

Ma ei tea. Ma just mõtlengi, et kust see tekkis. Üks asi on see, et mul on 
alati olnud sügav huvi asjade vastu ja millegipärast mõte töötab ka alati 
tagurpidi, et mida veel selle asjaga teha saab. Mul ei ole sellist kindlat 
vastust, lihtsalt uitmõte. Sel ajal ma müüsin automaatvastajaid hästi palju, 
Eestis enamus Panasonicu brändi automaatvastajatest olid minu käest tulnud. 
Sama moodi Citizeni ja Casio kalkulaatorid olid kõik minu müüdud. Eesti 
Pank\index{Eesti Pank} kasutas valuutakursside teatamiseks Panasonicu 
automaatvastajat, see oli ainuke ametlik kanal, kust valuutakursse teada sai, 
seda muudeti vist kord päevas või kord nädalas. Igatahes, ma tahan öelda, et 
see oli väga-väga tõsine asi. Helistasid numbrile peale ja ta luges sulle maha, 
et \enquote{Ameerika dollar nii palju}. See oli väga tõsine infokanal. Ja  neil 
oli Panasonicu automaatvastaja vaikeparool ära muutmata, seal oli mingi 
kolmekohaline number. Vist oli 555, sõltus mudelist natukene. Igatahes kui sa 
valisid rääkimise ajal selle 555, siis  automaatvastaja tegid piiks, ja pärast 
seda, kui sa vajutasid 7, võisid sinna uue teate peale lugeda. Ma nägin kohe, 
et põhimõtteliselt ma saaks nii teha. 

\question{Aga kas sa tegid?}

Ei. Lihtsalt ütlen, et selline mõtteviis oli. 

\question{Aga kas sa ütlesid neile, et vahetage oma kood ära?}

Tollal ei olnud nagu kanalit selleks. Ma arvan, et  kindlasti ma ei öelnud, 
just sellepärast, et tollal kuidagi maailm töötas teistmoodi. Sul polnud isegi 
endal telefoni, sa pidi minema telefoniputkasse ja otsima raamatust ja\ldots  
Ma ei suuda nagu ajaliselt ära määrata, aga see maailm oli teistsugune, kõik 
asjad ei käinud nii nagu praegu. Sa tahtsid sõbrale helistada, siis mõtlesid, 
et \enquote{homme helistan talle, et siis ma lähen inimese juurde, kellel on 
telefon}, kõik töötas teist moodi. 

\question{Aga ometi ei saanud sinust riistvaraärimeest, millalgi sul see asjade 
toimimise huvi sai nii tugevaks, et hakkasid hoopis sellega tegelema?}

Ma vist olen kogu aeg jooksnud mingi huvi ja raha kombinatsiooni järgi. Näiteks 
ma sattusin haltuura tegemisest ja spekuleerimisest  Kinexi\index{Kinex} 
direktoriks, see oli tollal Eesti tuntuim arvutifirma ja küllaltki 
tõsiseltvõetav. See tegeles kõigega ja see tarkvara osa oli päris oluline. 

\question{Jumal hoidku, sa pidid siis ju inimesi juhtima hakkama!}

Jah, aga ma olin seda  tegelikult kogu aeg teinud, näiteks see meie enda 
erafirma, mida me kahekesi tegime. Alguses seisime kordamööda letis: kui tema 
jooksis kauba järgi seisin mina letis  ja kui mina olin Venemaal, siis seisis 
tema letis. Mingi hetk oli raha nii palju, et mõtlesime, et mida me siin 
seisame, võtame kellegi tööle. Võtsime kellegi tööle. Siis istusime kodus, 
vaatasime, kuidas keegi müüb. Ja mina kirjutasin poele  nullist tarkvara. See 
seltskond läks päris suureks, meil oli tegelikult palju poode Tallinnas ja 
päris arvestatav hulgiäri. Eestis tegelikult kontoritehnika oli enamus meie 
oma. 

\question{Inimeste juhtimine tuli siis kuidagi loomulikult?}

Jah, ja see oli sõpruskond. Mina ei ole seal kunagi mingites konfliktides 
olnud. Ma olen seda täheldanud, et kui ma ära lähen, vaat siis on aeg-ajalt 
olnud see, et hakkavad üksteisele jalga  taha panema või midagi ja need 
aeg-ajalt eskaleeruvad ikka päris käest ära. Alati on nagu olnud  see, et me 
usaldame teineteist, et meil ei ole mingeid suuremaid probleeme olnud. 

\question{Mis sa praegu teed?}

Praegu mul on nihuke firma nagu Tochimo Lab\index{Tochimo Lab}. See on nii uus 
firma, et sellest ei ole keegi veel kuulnudki, aga selle mõte on tegelikult 
Planet Way Corporation-i all teha nihukene Skunkworks-i moodi moodustis, kus me 
teeme uusi projekte. Sest Planet Way ise muutunud praeguseks selliseks, et meil 
on väga tõsised kliendid, kus tootmine peab igapäevaselt jooksma, seal ei tohi 
mitte midagi katsetada,  kõik peab  käima nagu kellavärk. 

\question{Aga katsetada sulle meeldib!}

Jah. Aga mul on vaja teha just asju, mida veel ei ole olemas. 

\question{Miks?}

Vot ei tea. Uute asjade tegemine on ääretult põnev ja ma  viimasel ajal olen 
kuidagi  aru saanud, et mida võimatum ülesanne, seda rohkem ta mulle meeldib. 
Ja see on osaliselt muutunud mu tugevuseks. Seesama pen testide\sidenote{Ingl. 
\emph{penetration test}, penetratsioonitest, on autoriseeritud küberrünnak, 
mille käigus testija üritab süsteemi siseneda nii, nagu päris häkker seda 
teeks.} tegemine on andnud sellise mõttemaailma, et sa lammutad süsteeme, mis 
on ehitatud kindlaks. Pen teste  ma lihtsalt ei taha enam teha, sest see on 
tõsiselt depressiivne töö. Aga ta on nagu andnud sellise lihtsa asja, et kui sa 
oled kaks nädalat olnud niimoodi, et sul ei tule ühtegi ideed ja sul on 
totaalne depressioon, siis aeg-ajalt tuleb läbimurre pärast seda. 
Inseneriteadustes üldiselt on see hea omadus, et kui sa paned piisavalt 
ressursse alla, siis hakkab iga asi juhtuma. Kuidas öeldakse, et kana läheb 
sinna pilve või mesipuusse.

\question{Ja nüüd sa tegeled uute asjade leiutamisega?}

On mingid asjad, millel on ärilised vajadused olemas (päris udu ei tee),  aga 
millel ei ole päris selgeid vastuseid. 

Pooltel inseneridel on see häda, et kui sa annad talle mingi väga uduse 
ülesande, siis nad ei suuda seda teha.

\question{No jaa, sest kuidas ma arvutan midagi, millest ma ei tea, mida ta 
tegema hakkab!}

See ongi \emph{skunkworks}-i asi. Vaata, kui nad seda 
SR-71-te\sidenote{Lockheed SR-71 \enquote{Blackbird} on pikamaa strateegiline 
luurelennuk, mille arendas välja Lockheedi \enquote{Skunk Works} osakond. 
Lennuki loomisest Clarence \enquote{Kelly} Johnsoni käe all on tema toonane 
alluv Ben R. Rich kirjutanud inseneride hulgas populaarse raamatu, mida 
loetakse nii innovatsiooniõpikuna kui inspiratsiooni saamiseks. Ka minul on 
nende ridade kirjutamise ajal SR-71 mudel laual just selle raamatu tõttu.} 
tegid, siis nende eesmärk ei olnud teha mitte kolme-machine-lennuk\sidenote{Ehk 
lennuk, mis suudaks  kolmekordselt helikiiruse ületada.}, nagu ta lõpuks välja 
kukkus. Nende eesmärk oli Venemaa kohal kuidagimoodi ära luurata. Ja keegi ei 
ütle sulle, mida sa selleks tegema pead. Ja tollal, ükskõik mis sa teed ei 
tundunud reaalne, sest  kõik, mis lendab, saab raketiga alla lasta. 

Peab olema sellise pea kujuga inimene, kes nii kaua mõtleb, kuni ta saab selle 
vastuse. Tehti nii kiire lennuk, et  selleks ajaks, kui rakett õhku tõuseb, on 
lennuk lihtsalt taevast kadunud. Ega see vastus tundub praegu elementaarne. Aga 
tollal  ei olnud see üldse elementaarne,  oli võimatu. Selleks peavad olema  
inimesed, kes ei mõtle esimese asjana, et \enquote{ma ei võta ette, see ei ole 
tehtav} vaid sa suudad kuidagimoodi kuu aega tarbida raha ja kõike muud ja 
tulla välja kõige hullumeelsemate mõtetega ja siis panna asjale hind külge. 
Siis on juba kliendi asi, kas ta tahab seda või ei taha. Näiteks SR-71 
ehitamisel oli see hind, et sinna läks vaja suuremas koguses titaani, kui terve 
läänemaailm seda tootis. See tähendas ärioperatsiooni kuskil Venemaa kõhus, 
sest titaani sai sealt osta. Mis tähendas, et CIA tegi erioperatsiooni puhtalt 
selleks, et varjata, milleks ostetavat titaani vaja läheb. Kusjuures titaan on 
nii haruldane materjal, et kuidas sa tema otstarvet varjad: teda ei olegi 
tollases aja kontekstis millekski vaja. 

\question{Nojah, sa ei tee ju sellises koguses  titaanist kelli!}

Väga-väga tõsine  ressursi probleem. Ja olla see hull, kes ütleb, et 
\enquote{kuulge, teeme ülikalli asja. Aga me suudame teha}! Kusjuures need 
vennad, kes lennukit ehitasid ei teadnud, kas nad suudavad. 

\question{Aga nad ütlesid ja neil oli usk.}

Jah, ja kui sa loed neid raamatuid, mida need inimesed on kirjutanud, 
siis\ldots Lennuk oli pooleldi ehitatud ja siis selgusid mingid hädad. Näiteks 
üks häda oli see, et lennuk venib kolmkümmend sentimeetrit, sel ajal, kui ta 
kuumeneb. Aga selgus, et  needsamad kütusevoolikud, mis mootorisse jooksevad, 
peavad ka venima. Ja kuna see asi läheb seal mingisuguse kuuesaja kraadini, 
siis voolikuid ei saa ühestki mitte-metallist teha aga metall ei veni. Tekkisid 
asjad, mida ei olnud võimalik lahendada. Nad tegidki niimoodi, et torud on 
üksteise sees ja kui see lennuk on maa peal, ta lekib kohutavalt. Lennuk 
tangitakse minimaalselt täis,  ta lendab üles, teeb paar ülehelikiiruselist 
sellist tõmmet, kuumeneb mõnisada kraadi ülespoole ja vaat siis pannakse 
tankurlennuki pealt paagid täis. 

\question{Ja sinul on see usk olemas, et sa mõtled välja ja ongi võimalik?}

No vot, see on see koht, et selleks peab täiesti hull olema, et  mitte tagasi 
põrkuda. Pen testide tegemine on andnud selle, et ma ei pelga väga hullusti 
enam meid probleeme. 

\question{Kas on nii ka olnud et ei tule välja?}

Oi kindlasti. Kindlasti on. Ma ei oska nii mõelda enam, aga see oli üks 
põhjuseid, miks ma pen testimise maha jätsin: iial ei tea, millal ja kas sa  
tulemuseni jõuad. Ja see on meeletult depressiivne. Pen testidega veel teine 
häda on see, et igal juhul saad nagu peksa. Kui sa seda ära ei lõhu, mida sulle 
ette antakse, siis sa oled nõrk ja kui sa  ära lõhud, siis on kõik su peale 
solvunud. Tavaliselt on see veel nii lihtne, kuidas sa ta  lõhkusid ja 
öeldakse, et \enquote{nojah, nii me oleks isegi osanud}. Kuidagi, ma ei tea, ta 
on tõsiselt ebameeldiv töö, ma ei soovita kellelegi. 

Aga uute asjade ehitamine on selles mõttes lahe, et kui sa sinna nagu aega 
investeerinud, siis ma olen tavaliselt sealt ise midagi saanud ja tavaliselt ka 
see kommuun on midagi nagu saanud. Vot see on see koht, kus mulle meeldib 
inseneride hulgas näidata, mida ma tegin. 

\question{Ja sa ju ei ehita triviaalseid asju! Kuidas sa oskad? Lihtsalt 
kogemus?}

Vist jah, ma isegi ei tea, mida sa silmas pead. 

\question{No näiteks see sõrmus, mille kallal sa töötasid.\sidenote{Tõnu on 
töötanud sõrmuse kallal, mis toimib žestikontrolleri, võtme, NFC maksevahendi 
ja märguandjana olemata palju suurem tavalisest gümnaasiumi lõpusõrmusest.}}

Sõrmusega oli selles mõttes lihtne. Sõrmus on, eksju \emph{bluetooth}-i saatja, 
patarei, mingi väike mikroprotsessor ja mingid andurid. Sellist asja suudab 
igaüks ehitada. Ainuke asi, et ei tea, kui suurt. Esimese asjana ma mõtlesin 
enam-vähem välja, et kui ma tahaks midagi sellist ehitada, kui suur ta umbes 
tuleks. Vaadates, mida poes müüakse, Arduino näiteks, siis tegelikult igaüks 
suudab selle väga lihtsalt kokku ehitada. Nüüd on see, et kas ma suudan selle 
väiksemaks teha. Ja esimene reegel, mille ma võtsin, et kuna ma elan Jaapanis, 
siis peksame igast suurfirmast välja lahenduse, mis on väiksem kui see, mida 
turult saab. Ükskõik kui palju väiksem. Kui ma tean, et mingi asi on näiteks 
kümme millimeetrit suur, ma lähen nende ukse taha ja ütlen, et ma ennem ära ei 
lähe, kui ma saan üheksa millimeetri suuruse. Ma tahtsin konkurentsieelist. Ja 
tuli lihtsalt ehitama hakata, siis tulid avastused. Esimene avastus oli näiteks 
see, et teatud asjad, mida ma pidasin oluliseks, polnud üldse olulised. Näiteks 
see, palju \emph{bluetooth} voolu tarbib. Selgus, et üldse ei loe. Luges hoopis 
see, et palju ta magades voolu sööb, sest \emph{bluetooth}-i saatmise hetk on 
niivõrd lühike, et ta võib tarbida, palju tahab, mind see ei sega. Aga kui ta 
paari päevaga magades tühjaks jookseb, see mind häirib. Meie suutsime näiteks 
sõrmuse ajada nii kaugele, et suudab viis aastat karbis voolu sees hoida. Et 
kui nüüd klient saab karbi kätte ja teeb lahti, siis sõrmus ärkab ellu,  kui 
see on toodetud viimase viie aasta sees. 

\question{Kui ma sind niimoodi kuulan, siis tundub, et sul on mingisugused 
sellised põhimõttelised printsiibid, millest lähtudes on võimalik ehitada mida 
iganes?}

Tavaliselt on mingi väga lihtne läbiv idee küll jah. Minu arust kuidagi tuleb 
endale teha mingi väga lihtne rusikareegel, et mis asi see on. Et näiteks auto 
on mootor, rool, pidurid ja  sa pead hakkama teda sealt kuidagi tükkideks 
tegema, minimaalse asja valmis tegema. Siis tekib arusaam, milline on su 
probleem, mida sa tegelikult lahendad. 

\question{Praktiline käega katsutav lahendus?}

Jah, sest tegelikult ma tunnistan, et ma tegelikult vist ei saa üldse 
keerulistest asjadest aru. Minu esimene samm on  asi maha lihtsustada. 

\question{Ega keegi ei saa keerulistest asjadest aru, seepärast nad ongi 
keerulised!}

Jah, aga mul on tunne, et see vist on see tee, kuidas ma nagu asju teen. 

Toon mingi täiesti teise näite. Mind millalgi hakkas huvitama  arvutiga 
nägemine: kuidas arvuti näha saab? Võtsin raamatu, hakkasin otsast lugema, et 
mis asi on see OpenCV teek, mis on \emph{computer vision}-iks, arvutiga 
nägemiseks, kõige levinum teek. Kui ma olin kuskil poole raamatu peal või isegi 
vähem, mul juba näpud sügelesid nii kohutavalt, sellepärast et ma olin kõik 
ideed kätte saanud, mida ta tegelikult teeb ja need printsiibid olid lihtsad. 

Kõik teavad, et arvutiga saab  nägusid otsida, seda tänapäeval teeb juba iga 
telefon. Aga mind hakkas huvitama, et kas ma  suudan kokku panna, et kui 
inimene on pooleks lõigatud, et milline on alumine, milline ülemine ots. 
Tõsiselt, siiras huvi, et nägusid me leiame, aga kas me leiame jalgu või midagi 
muud? Kas seesama printsiip on rakendatav? Üritasin midagi kokku käkerdada ja 
sain tulemuseks, et sa võid neid frankensteine ehitada nii palju, kui tahad, 
arvuti leiab inimesele täiesti sobiva alakeha. See näeb maru naljakas välja, 
aga ta näeb välja ülimalt loogiline, tegelikult selline inimene võiks isegi 
olemas olla. Ainult et ta ei ole õige. 

Vot see on see, et sa pead proovima. Tollal ma näiteks jõudsin selle projektiga 
nii kaugele, et ma sain aru, et tegelikult on taust palju olulisem kui inimene. 
Ma ei tea, kas sa tead sellest projektist, kus ma tõmbasin terve 
rate.ee\index{rate.ee}\sidenote{rate.ee oli esimene Eestis tõeliselt 
populaarseks saanud sotsiaalvõrgu laadne teenus, mille sisu seisnes peamiselt 
üksteise piltidele hinnangute andmises.} alla. Ta oli hästi lihtsasti 
kopeeritav ja  seal oli terve elanikkond praktiliselt sees. Ja igast ühest oli 
terve hulk pilte. Rate.ee sai alla tõmmatud ja sai avastatud, et on Eestis 
nihuke sait nagu sexinestonia.com. Väike kommuun, umbes tuhat kasutajat. Et kui 
sul on tuhat kasutajat Eestist, kes kasutab pornosaiti  iseenda reklaamiks, 
siis, noh, üks on su õpetaja, üks on su kolleeg, üks on su ülemus. Väga  
sensitiivne asi. Ma hakkasin siis üritama leidma  paralleele, millised 
profiilid kattuvad rate.ee omadega. 

\question{See on ju ohtlik või ebameeldiv küsimus, mida küsida?}

Mind huvitas tehniliselt, et kas ma olen võimeline neid kokku viima. Ja 
avastasin, et neid ei ole väga palju, kes kasutab mõlemas teenuses sama 
telefoninumbrit. Selle järgi kokku viia oleks nagu elementaarne. Aga seal on 
muid asju. Ma sain \emph{computer vision}-iga tehtud selle, et ma vaatan 
mustreid taustal. Tuli välja, et tapeedi muster on üsna unikaalne asi. Ja kui 
nüüd seal pildis on kaks erinevat mustrit, näiteks tapeedi muster, vaiba muster 
ja kolmas on näiteks  mööbli muster ja kui nende kombinatsioon on unikaalne, 
siis see ongi unikaalne ruum. Ja kui sa juba ruumil samasuse leiad, siis 
tavaliselt oli profiil ka kohe nagu arusaadav. Saad täpselt aru, et see, kes 
rate.ee-s  on nagu selline tore väikeste lastega on sama, kes seal sexinestonia 
peal kõiki neid muid asju teeb. 

\question{Need on ju küsimused, millele inimesed ei taha reeglina vastust 
saada, miks sina tahad?}

Mind huvitas, kas see on võimalik ja see oli võimalik. See oli minu jaoks väga 
vapustav avastus. Et nii saabki. Ja see tuli ka IT-inimestele tihti üllatuseks, 
isegi turva-ala inimestele. Kui ma siin käisin  pankuritele seda mingil hetkel 
näitamas, siis neile tuli see just selles mõttes ebameeldiva üllatusena, et 
pangas on tuhandeid tellereid. Ma ei tea, kui palju neid on, kohutav arv. Ja 
nüüd on niimoodi, et kui tema hoiab endast mingisugust alasti pilti kuskil, ta 
muutub santažeeritavaks. Ja panga jaoks on see probleem, kui mingisugune häkker 
seda teab, aga sina pangana ei tea. Ja sa ei saagi  teada. Tegelikult sellised 
tõsised mured. 

Tekkis selline mõtlemine, et tuleb ise-enda igasugu käitumist korrigeerida. 
Tuleb välja, et keegi teab sust alati rohkem, kui sa ise arvad. Aga lihtsalt 
tavaliselt on see nihukene vandenõuteooria, aga see on see koht, kus saad ise 
tunda. Et mina tegingi selle süsteemi, mina tean. 

Ma kasutasin seda ju millalgi spämmerite püüdmisel ju ära. Suutsin sealt nende 
kohta nii mõnegi pildi leida.

\question{Oleme küll BBS-ide juurest kaugele eemale läinud, aga sellest ei ole 
lugu, sest on jube huvitav! Aitäh!}

Sa küsid küsimusi, mida ma ei ole iseendalt korralikult küsinud. Et miks ma 
teen või kuidas ma teen. Ma ei tea. 

\question{Ega ei peagi teadma!}

Ma üritan olla ise ja iseenda vastu aus, et see on nagu põhiline reegel. 
