\index[ppl]{Ansper, Arne}
\question{Nagu ikka alustame sellest, kuidas asjad alguse said. Kuidas nad siis 
said sinu jaoks alguse?}

No minu jaoks need asjad alguse sellest, et kui ma põhikooli lõpetasin siis 
minu matemaatikaõpetaja arvas, et ma peaksin minema Nõkku\index{Koolid!Nõo 
Keskkool} edasi õppima. Ja suutis mu vanemaid ära veenda, et see on suurepärane 
mõte, siis ma sinna läksingi.

\question{Aga kus sa põhikooli lõpetasid?}

Jõõpres\index{Koolid!Jõõpre kool}\index{Jõõpre}, selline pisike koht Pärnu 
lähedal. Sada õpilast oli see põhikool meil vanas  mõisas, mitte mõisamajas 
endas aga koolimaja oli mõisa keskel. Niisugune väga mõnus koht oli. Ja siis 
mul matemaatika nagu sobis ja õpetaja oli väga usin, andis mulle lisaülesandeid 
ja lõpuks saatis olümpiaadile ja seal läks ka suht hästi.


\question{Sa siis tulid puhtalt matemaatika ja mitte arvutite nurga alt sinna 
Nõkku?}

Ei, mul oli null kokkupuudet arvutiga enne. Vanemad seejuures pigem nagu 
tahtsid, et ma läheks. Ma ise olin väga  kahtleval seisukohal, et kas kodust 
nii kaugele minek, et see on äkki kuidagi raske ja paha ja nii edasi. 

\question{Mis aastal see oli?}

1985. 

\question{Sel ajal oli juba logistiliselt ju keeruline Pärnu lähedalt Nõkku 
saada?}

See oli lihtne ja tüütu, selles mõttes, et olid bussid, mis sõitsid neli tundi 
ja olid tavaliselt maast laeni rahvast täis ja siis veel Pärnust koju kus buss 
käis kahe tunni tagant. Seal ikkagi võttis aega, ütleme nii.  

\question{Ja Nõos pandi kohe arvuti ette?}

Ei, Nõos see oli tavaline keskkoolielu selle väikse vahega, et tuli ühikas 
elada. Mina olin viimane aasta, kes elas poiste ühikas, mis on selline 
suhteliselt raju ja legendaarne koht. Ehitatud kuskil tsaariaja lõpus, Eesti 
aja alguses. Talvel oli niimoodi, et tulid  kodust, tõid sihukesed suured 
märjad puunotid, läksid oma tuppa, mis oli  null kraadi lähedal kütsid ta siis 
üles selleks, et magada saaks. Hommikul lõid ikkagi pesukausi pealt jää katki, 
kui hakkasid hambaid pesema, niisugune koht oli. Esimene aasta oli hästi lahe. 

Alguses oli tavaline keskkond ja siis tuli programmeerimise õpetamise lihtsalt 
ühe regulaarse ainena sisse ja hakati õpetama. See oli ikkagi matemaatika ja 
füüsika kallakuga kool aga programmeerimise õpetamine seal oli lihtsalt nagu 
aine nagu mida iganes muugi. Mahud, loomulikult, olid suuremad nii 
matemaatikal, füüsikal kui ka sellel, programmeerimisel, millel mujal oli null, 
et seal oli siis nagu mingi muu number.

\question{Räägi palun Nõo kooli taustast, kuidas sinna üldse sai?}

Tead, ma ei tea. Mina olin tollal niisugune inimene, et emaga koos me sinna 
läksime. Ma arvan, et me käisime direktori juures rääkimas. Et kuna mul oli 
tegelikult olümpiaadilt mingisugune koht ette näidata siis kuidagi ma sinna 
igatahes sisse sain. Kuidas täpselt, kas seal oli mingi konkurss või mingi muu 
süsteem, ei tea. 

\question{Kes Nõo kooli direktor tol ajal oli? See kool tundus kellegi 
entusiasmi peal käivat?}

Enn Liba\index[ppl]{Liba, Enn} oli minu meelest tol ajal direktor\sidenote{Nõo 
kooli arendas selliseks reaalteaduste ja programmeerimise õppe keskuseks, nagu 
me teda praegu tunneme, Kalju Aigro\index[ppl]{Aigro, Kalju}. Ta oli kooli 
direktoriks aastatel 1951---1982, talle järgneski selles ametis Enn Liba.}. Aga 
seda entusiasmi aspekti ja ajalugu, ma pean tunnistama,  ma ei oska 
kommenteerida tollal huvitusin  oluliselt muudest asjadest.

\question{Aga mis asjad need olid, millest sa huvitusid?}

Tegelikult mulle meeldis põhikoolis elektroonika. Aga see oli selline 
platooniline huvi, kuna juppe oli hullult raske kätte saada. Ja mulle meeldisid 
mudellennukid, mis oli ka suhteliselt platooniline. Aga Nõos tuli 
programmeerimine hästi kiiresti peale, kui hakkasime seal õppima. Seal oli suur 
Vene \emph{mainframe} Nairi-3-1\index{Arvutid!Nairi!Nairi-3-1}\sidenote{1964. aastal 
Jerevanis välja töötatud Nõukogude arvutiperekonna Nairi kõige võimekam liige. 
Kool sai selle arvuti 1977. aastal.}. 
KÕPS\index{Keeled!KÕPS} ja ROPS\index{Keeled!ROPS}\sidenote{\label{sidenote:ROPS}KÕPS ja ROPS on 1980. 
aastate teisel poolel Nõo Keskkooli arvutuskeskuses välja töötatud eestikeelsed 
programmeerimiskeeled, millede loomisel osales ka Arne esimene arutiõpetaja Nõos 
Uuno Puus\index[ppl]{Puus, Uuno}. KÕPS oli sarnane MIT-is välja töötatud keelega 
LOGO, võimaldas vaid graafikat ning tugines LOGO looja Seymour Papert-i ideoloogiale. 
ROPS oli KÕPS-i edasiarendus, mis olla sarnanenud Algolile ja võimaldas lisaks 
graafikale ka arvutusi.}, eesti keeles sai 
programmeerida, need olid  vahvad. Siis olid seal Agatid\index{Arvutid!Agat}, 
mille ligi suht ruttu sai, mis olid teistmoodi vahvad, kus sai mingit 
valmistarkvaraga ka kasutada. Ikkagi mingite mängude mängimine oli oluline ja  
siis ise mingite asjade proovimine. See nagu hakkas väga kiiresti meeldima.

a \question{Oskad sa takkajärgi kuidagi reflekteerida, mis sulle seal meeldima 
hakkas?}

Väga ei oska, ausalt öeldes. Ma üritasin mõelda, et mis ma siis tegin nende 
arvutitega toona. Mul on umbes kaks asja meeles mida ma Agatiga tegin. Esimene 
programm oli umbes see, et oli \verb|for| tsükkel: muutis värvi, trükis mingi 
teksti nagu, ütleme, \enquote{tere}. Kõigis keeltes ja siis veel vilkuva 
taustaga ka. Sellega sai vähemalt üks õhtu kui mitte kauem möllatud ja timmitud 
neid efekte, tekste ja asju. Ja siis teine asi, mis mul on meeles, ma püüdsin 
ühte Nintendo mängu (need pisikesed puldi mängud, mis olid\sidenote{\label{sidenote!gameandwatch}Arne peab 
ilmselt silmas Nintendo Game \& Watch\index{Nintendo Game \& Watch} seeria käes hoitavaid mänge. 
Originaalidest oluliselt rohkem oli liikvel nende Nõukogude kloone, mida müüdi 
Elektronika kaubamärgi all. Tegu polnud siiski alati täpsete koopiatega: 
Nintendo EG-26 kloonis IM-02 püüdis mune Miki Hiire asemel hunt tuntud 
Nõukogude multifilmist \begin{russian}Ну, погоди!\end{russian}}) taasluua, ma 
lõingi. Seal oli, nagu ta on, mingi fikseeritud arv positsioone, mingi tegelane 
liikus, mingid teised tegelased liikusid ja siis olid mingid surmasaamised ja 
mingid boonuste saamised. Probleem oli selles, et ma ei teadnud tollal, mis asi 
on massiiv. Põhimõtteliselt oli niimoodi, et iga objekti jaoks oli mul muutuja, 
mis ütles, et kas objekt on või ei ole. Ja kui seal mingid asjad liikusid, siis 
mul oli lehekülgede kaupa \verb|if| lauseid, et kui see muutuja omab seda 
väärtust, siis järgmisel sammul ta omab teist väärtust. Ja muidugi 
refaktoreerimis-tööriistu ei olnud. Kui ma kuskil vea tegin, siis ma nägin 
päevade kaupa vaeva, et ma nimetasin neid oma muutujaid ja \verb|if| lauseid 
ümber.


\question{Väga huvitav. Tol ajal tundus asjadest mitte rääkimine olevat 
õpetamise metoodika osa. Meile näiteks ei räägitud \texttt{for} tsüklist tükk 
aega}

Ütleme nii, et seda Agati\index{Arvutid!Agat} ei õpetanud meile keegi. Õpetati 
Kõpsi ja Ropsi. Kõik, mis Agati peal sai tehtud, see oli puhas enda välja 
võidetud ja võideldud  arvutiaeg, enda entusiasm. Ma isegi ei mäleta, \emph{by 
example} käis see asi vist, et vaatasid, mida keegi teine oli teinud. Mina küll 
ei mäleta, et oleksin ühtegi, Agati või BASICu\index{Keeled!BASIC}  kohta 
käivat raamatud lugenud kunagi. Kõik see oli lihtsalt nagu folkloor, 
katsetamise ja kõlakate tasemel. Et oleks keegi lekitanud selle info, et 
massiivid on olemas, oleks selle Nintendo mänguga palju rutem valmis saanud. 

\question{See oli suur töö ju, pidi ikka kihu olema?}

No aega oli palju, segavaid faktoreid oli vähe, eks ole. Ja ilmselt siis see 
arvuti alistamine meeldis, nagu välja tuleb. Agatiga\index{Arvutid!Agat} ma 
mäletan seda kindlasti, et ma hankisin endale selle 
assembleri\index{Keeled!Assembler} nii-öelda manuaali. Mis oli põhimõtteliselt 
paar-kolm ruudulist lehte, kuhu ma siis kirjutasin tähtsamad käsud ja registrid 
ja värgid üles ja siis studeerisin seda. Ja ma tean, et ma ikkagi nagu 
tuuseldasin seal Agati assembleri poole peal ringi. Aga mida ma tegin, seda ma 
kindlasti ei mäleta. Mäletan olulisimaid registreid, mida näppides käis piiks 
ja kust sai lugeda mingit vist klaviatuuri sümboleid või midagi sellist, aga 
\emph{that's it}.

\question{Kuidas Nõos tase oli, seal olid kõik sinusugused koos?}

Seal oli  selliseid inimesi, kes olid üle vabariigi kokku tulnud, kellel olid  
mingid huvid ja eeldused  reaalainetega tegelemiseks. Aga seal oli ka noh 
lähikonna inimesi. Et see on nagu päris, selline geto kuskil, see oli ikkagi 
nagu natukene spetsialiseeritud kohalik kool, et seal oli igasuguseid inimesi

\question{Kas sealkandis mingit äri tegemist ka juba käis, keegi raha eest 
programmi ei kirjutanud? Kaheksakümnendate lõpp ikkagi?}

Võib-olla keegi tegi, aga  ma julgeks öelda, et ma isegi ei huvitunud sellest 
ja ma ei tea sellest midagi. 

\question{Tartu vahet ka käisite?}

Jaa. Mingil hetkel, ma ei mäleta enam mis klassis, aga siis ma sain teada, et 
Tartu Ülikooli Raamatukogus\index{Tartu Ülikool!Raamatukogu} on mingisugune 
XTde\index{Arvutid!XT} klass. Kaheksa kuni kümme arvutit oli seal. Kuidagi ma 
sain sinna juurde, ma ei mäleta, mis alustel sinna seda aega sai reserveerida. 
Igatahes ma tean, et ma seal ikkagi jõlkusin päris mitu õhtut nädalas. Sa 
said seal mingisuguse tunni või kahese \emph{slot}i, mul oli umbes kaks flopit, 
millest ühe peal oli Turbo C\index{Keeled!Turbo C} ja teise peal oli siis tüüpi 
opsüsteemi oma asjad ja siis  midagi ma seal programmeerisin. 

\question{Aga kust sa said tolle Turbo C?}

Ma ei kujuta ette, kus ma selle saada võisin. Seal ma käisin päris tükk 
aega aga seal ma põhiliselt tegelesin ka sellega, et mängisin selle Turbo Cga. 
Aga kas mul ka mingi eesmärk oli, seda ma ei mäleta. Aga Turbo C see oli 
igatahes.


\question{On ikka paras hüpe Kõpsust ja Ropsust C ja mälu ja pointeriteni? 
Mille pealt too hüpe tuli sul?}

Jällegi nii kauge aeg, et ma kardan, et meile koolis ma isegi mäletan seda, mis 
meile üheksandas klassis  programmeerimist õpetati, aga ma ei mäleta, mis edasi 
sai, ausalt öeldes. Mida meil seal üldse räägiti. Ilmselt ise liikusin 
kiiremini edasi. Pärast TPIs\index{TPI|see{Tallinna Tehnikaülikool}} 
\index{TPI} ka see asi esimestel kursustel, et need  programmeerimise loengud 
olid  sellised, et sealt ei olnud midagi uut saada. Seal  mingid teised asjad 
olid pigem  need, mis olid uued, aga mitte see programmeerimise pool. 


\question{Kuidas sa sealt Nõost TPIsse\index{TPI} sattusid? Oleks ju loogiline, 
et sa lähed sealt Tartusse matemaatikasse?}

See oli ka suht \emph{random}iga selles mõttes, et ma mõtlesin, et võib sinna 
minna või tänna minna. Need argumendid, miks  Tallinnasse proovida, need olid 
niisugused väga otsitud ja õrnad, et miks ma just sinna Tallinnasse läksin 
proovima,  seda ma tegin. 

\question{Mida õppima?}

LI\index{Tallinna Tehnikaülikool!Automaatikateaduskond!LI}. Ma täpselt ei mäleta, kas oli arvutid ja 
arvutisüsteemid, tõenäoliselt võis olla.

\question{See LI lühend jookseb mitmelt poolt läbi aga keegi ei tundu teadvat, 
mida see tähendas}

Kas ta üldse midagi tähendas? Et \enquote{L} on tõenäoliselt mingi 
automaatikateaduskonna kood, eks ole, ja \enquote{I} on mingi muu asja kood. 
Seal oli LA, mis oli äkki rohkem automaatika teisi tähti ei mäleta, äkki on LS 
ka olemas olnud. LI  oli jah see, kus mina oma aega veetsin.

\question{Sa ütlesid, et programmeerimise õpe sind väga edasi ei aidanud, kas 
seal üldse midagi õpetati, mis sulle midagi juurde andis?}

Tagantjärgi  vaadates tundub, et  seal LI-s räägiti nagu laiuti alates sellest, 
kuidas transistori teha, kuidas transistoridest saaks teha mingeid 
mikrolülitusi, kuidas saaks kõik see, mis sorti registrid meil on, kuidas 
registritest mingit automaatikat ehitada. Kuidas protsessorit teha, kui sul on 
neid registreid hulgi käes. Ja  teisele  poole minnes ka, kõik sellised asjad 
nagu siduteooria. Need asjad andsid, tagasivaates, need teadmised, et kui sa 
vaatad tänapäeval enda ümber, siis maagilisi asju, mille kohta ma ei tea, et 
kuidas seda saaks teha või ma pean uskuma midagi või ma vajaduse korral ei 
saaks sinna lõpuni välja kaevuda, neid on väga vähe. Ja see, ma arvan, on üks 
asi, mis mina olen leidnud, hästi kasulik. Tänapäeval on neid kihte sinna nii 
palju juurde tulnud, et vanasti oli ikkagi väga lihtne. See oli umbes nagu 
renessansiajastul, kui üks tüüp suutis  kõike, mida oli mõtet teada, teada. 
Natukene, kui  mina seal TPIs käisin, see aeg hakkas läbi saama. Ütleme 
niimoodi, et tänapäeval ilmselt ei ole võimalik, et sa tead kõike, mida oleks 
kasulik teada arvutiasjandusest. Ma mõtlen just tänapäeval seda, mis riistvara 
poole peal on juhtunud. Sinna on laotud neid kihte ja neid virtualiseerimise 
tasemeid ja mida iganes veel juurde. Ja siis \emph{soft}i poolel on ka vastu 
tuldud, sinna neli kihti virtualiseerimist vahele laotud ja nii edasi. See on 
nagu see, kus kipub nagu raskeks minema see järje pidamine.

\question{Kas TPIsse minek oli asjade loomulik käik või oli sul mingi plaan ka, 
mida tegema hakata?}

Mul niisugused pikaajalisi plaane ausalt öeldes ei olnud. Mulle meeldis teha, 
mulle meeldis nende arvutitega mässata, kas ma mässan Tallinnas või mässan 
Tartus, vahet pole. Ja siis ma mässasin nendega Tallinnas. Üks huvitav nüanss 
on veel see, et et umbes seal keskkooli lõpus ma sain isikliku arvuti ka. See 
oli midagi teistsugust, see oli Atari 520 STf\index{Arvutid!Atari 520 STf}. Mis 
oli siis Atari Motorola 68000 prosega tükk. 512kB oli tal mälu, selle ma 
\emph{upgradesin}  ühe megani mingil hetkel. Selle peal ma siis elasin ja 
siis selle peal ma püüdsin nagu süvitsi minna kogu sellega, mis seal nagu teada 
oli. 


\question{Kust sa sihukese aparaadi said kaheksakümnendate lõpus?}

Mul olid vanaonud, kes elasid Rootsis. Ema ja isa ükskord käisid seal ja siis 
sealtkaudu ma selle siis sain. 

\question{See pidi Agati kõrval ikka ulmeline aparaat olema}

Tegelikult oli niimoodi, et teised olid PCde peal. Kui ma nüüd vaatan, siis 
need inimesed, kellega me siis igal pool nagu koos ringi käisin, siis noh 
üheksakümnenda aasta paiku umbes, normaalsed inimesed said PCdele ligi ja siis 
toimetasid nendega.  Ja siis minul oli kodus Atari  ja tegelesin sellega 
põhiliselt. 

\question{Ataril on kihte vähem, sai lihtsamini sügavale välja minna}

Jaa, see oli nagu hoomatav täiesti,  mis seal toimus, midagi väga ulmelist 
polnud. Natuke mängisin ka, aga mitte liiga palju. Mul ikka see 
programmeerimine meeldis kõige rohkem selle asja juures. Selle Atari peal ma 
tegin igasuguseid imelikke asju.

Ma üritasin CAD programmi teha, joonistamisprogrammi. See isegi lõpuks selles 
mõttes töötas, et seal sai teha ringe ja jooni, igast värke, salvestada ja 
laadida. Ja siis mul oli, tagasi vaadates jälle hullumeelsus, et mulle nagu 
kohutav tegi muret see, et mälu saab otsa. Et kui sa teed dünaamilist 
mäluhaldust, eks ole, et siis saab mälu otsa. Üritasin seda siis minimeerida. 
Näiteks mulle tundus, et nagu lokaalsed muutujad, mis on \emph{stack}is, on 
kuradi ebaefektiivsed. Ja sisuliselt see CAD programm oli kirjutatud 
sajaprotsendiliselt globaalsete muutujate otsa. See oli täiesti hullumeelsus 
nagu tagasi mõeldes, seal tuli ikka kõvasti refaktoreerida, sest ma ikkagi panin 
täitsa puusse alguses. Seal seda loll ümberkirjutamist oli nii palju, sealt ma 
sain selgeks, et okei, nii ma mitte kunagi rohkem ja mitte ühtegi asja ei tee. 
Väga-väga palju vigu sai igatahes tehtud.

\question{Eks see on ju õppeprotsess, mõnda asja teoreetiliselt selgeks ei saa}

Jah, absoluutselt nõus. Ütleme, et nii võimekaid inimesi, kes kogu aeg teiste 
vigadest õpivad, et neid väga palju ei ole. Ikka enamus kipub oma vigadest 
õppima. 

\question{Kui sa TPIsse\index{TPI} jõudsid, kas sa seal teisi omasuguseid ka 
kohtasid?}

Meil oli hästi lahe kursus. Aga tegelikult oli niimoodi, et seal TPI ja alguses 
ma ikkagi õppisin, eks ole. Mis sest, et seal programmeerimise vallas mul ei 
olnud väga huvitav, aga neid muid ained ma ikka õppisin korralikult. Ma olen 
ikka väga usin õppur olnud. Ja mul juhtus niisugune asi, et mind 
Tarvi\index[ppl]{Martens, Tarvi} kutsus ühel hetkel Ektaco-sse\index{Ektaco}, 
ma arvan, et see oli üheksakümmend üks aasta. Ja see oli siis see 
\emph{community}, kus ma siis hakkasin nagu inimestega koos olema ja oli siis 
ka töise karjääri algus. Ma arvan, et see võis olla, see võis olla 1991, aga  
sada protsenti kindel ei ole. Mingi kolmas kursus äkki umbes.

\question{Kolmas kursus on üsna hilja ju?}

Tegelikult ongi see, et programmeerimise õppimine, üldse arvutiasjanduse 
õppimine võtab ikkagi aega. Ma tagasi vaadates mõtlen, et mis ma siis tookord 
oskasin või kuidas ma mõtlesin või  kuivõrd hästi ma siis programmeerisin.  
Ütleksin, et palju varem ei ole mõistlik seda tööd üritada teha. See võib  
frustratsiooni tekitada. Mis mul oli, ma olin ikka viis aastat nüüd innustunult 
selle asjaga tegelenud. Ma arvan, et kui ma  tööle sain, siis ma olin ka noh, 
enam-vähem miinimumtasemel, kus oleks  mõistlik, et keegi annab sulle 
ülesandeid, millele on ka mingi tähtsus ja tähendus ja sa teed nad  ära.

\question{Kas sul midagi sellist ei olnud, nagu inimesed on rääkinud, et 
lihtsalt arvutiaja saamiseks tekkis mingi arvutiklassi admini koht?}

Ei, mul ei ole ju midagi taolist. Ütleme tõesti mälu võib olla natuke petab, et 
mis aastal mul see Atari sinna täpselt tekkis, aga mul kuidagi oli alati 
mingisugune võimalus olemas, nii palju, kui mul seda tarvis oli ja sellest 
piisas. 

\question{Oskad sa mõnda näidet tuua, mida sa seal Ektacos alguses 
programmeerisid?}

Ektaco oli niisugune  firma, kus tehti riistvara ja tarkvara. Ta tegi 
tööstuskontrollereid, automatiseeris tehaseid, eks ole. Ja olid need 
sardsüsteemid, seal on väiksed mikroprotsessorid neid oli vaja programmeerida 
ja need programmaatorid olid kallid. Ja siis Ektaco hakkaks tegema oma 
programmaatorit. Põhimõtteliselt mingisugune lisaseade PCle, millega sa saad 
neid kivisid kõrvetada. Üks teine tüüp, kes oli nagu riistvara poole peal (ma 
ei tea, aga ma arvan, et ta oli umbes nagu mina, värskelt laekunud staatuses) 
ja mina tegin siis softi. See oli selles mõttes nagu päris huvitav, et meil oli 
PC/AT platvorm, seal oli ISA siin ja selle arvuti me süstemaatiliselt kogu aeg 
ajasime ikka täiesti lukku. Ja selleks, et saaks mingit sotti, siis meil oli 
seal siuke äge asi nagu loogikaanalüsaator. See on niisugune aparaat, et kui 
Ostsilloskoobiga saab visualiseerida mingit analoogsignaali, siis 
loogikaanalüsaatoril on palju-palju pisikesi klemme, mis sa paned kuskile prose 
või mingite digitaalsignaalide külge. Siis sul on teine arvuti mis  
visualiseerib, et kuidas need signaali mustrid on ja siis sa saad panna 
\emph{triggereid}, et kui mul tekib selline muster,  siis salvesta ja 
taasesita. Ehk et kui me ajasime selle selle PC täiesti hulluks, siis me saime 
sealt loogikaanalüsaatori pealt pärast vaadata, et mis siis juhtus, et mis me 
valesti tegime. Ühesõnaga tema siis tegi riista ja kirjutas siis sinna 
kontrolleri peale programmi ja mina kirjutasin PC peale siis põhimõtteliselt 
draiverite programmi vastu, mis omavahel suhtlesid. Ja siis tegin sellele ka 
kasutajaliidest.

Meil olid igasugused Inteli ja IBM-i \emph{manual}id laua peal, neid me siis 
seal sobrasime ja dekodeerisime, et mis me peame nüüd tegema, et siit 
midagigi läbi läheks. 


\question{See kõlab kuidagi hästi süsteemse ja korraldatud ettevõtmisena?}

Ei, see oli hull häkkimine. Nojah, Ektacos seda kraami, mille abil nagu häkkida, 
seda oli ja meil meil oli võimalus seda kasutada. Ja tegelikult ma tõesti selle 
teise tüübi  tausta ei tea, et võib-olla tema oli  kuidagi kogenum, tema tuli 
ju loogikaanalüsaatoriga sinna laua taha. Aga see oli suhteliselt niisugune 
kasulik ja kergesti omandatav seade, et noh kuidas sa seda pruugid. 

\question{Jah, aga võrreldes sellega, kui (nagu on räägitud) inimesed 
vaibanoaga emaplaadi pealt radu maha kratsisid, et modem tööle saada on tegu 
ikka \emph{high-tech} häkkimisega}

No me tegime ikka sinna radu juurde selleks et see kuidagi tööle saada, me ei 
kratsinud midagi maha! Mina ise seda riista-poolt tol ajal ei puutunud. Ehkki 
meil Ektacos programmeerija töövahendite hulgas oli kindlasti tinutus kolb, et 
nii raua lähedal oli seal see enamus sellest elust. 


\question{Kas te saite tööle ka selle kupatuse?}

Ja, loomulikult. Ja siis sellega seoses muidugi, kuna  see oli veel see aeg, et 
 see Borlandi\index{Borland}\sidenote{Borland Software Corporation oli 1983. 
aastal asutatud ja eri nimede all siiani toimetav tarkvaraettevõte, tuntud 
eelkõige arendajate töövahendite poolest. Neist kuulsaimad olid 
\enquote{Turbo-} eesliitega keeled Assembler, BASIC, C, C++, Pascal ning hiljem ka Delphi}
toodang, igasugused Turbo-blaahid, mis neil olid, need olid nagu  standard, eks 
ole. Siis loomulikult sai kirjutatud oma akendussüsteem, mis nägi välja nagu 
see Borlandi Turbo Vision\index{Turbo Vision}\sidenote{Borlandi poolt 
üheksakümnendate alul arendatud tekstipõhine kasutajaliidese raamistik Pascali 
ja C++ jaoks}, aga oli hoopis parem ja teistmoodi tehtud ja seega töötas väga 
kenasti. 

\question{Milles see väljendus, et ta parem oli?}

Ta oli nagu ägedamini struktureeritud. Siis mul hakkas juba 
C++\index{Keeled!C++}  meeldima, ta oli hullult objektorienteeritud. Tal olid 
mingid oma kontseptsioonid, et kuidas sa neid aknaid ja asju  esitad, kuidas sa 
sündmusi käsitled  selles mõttes, et sul on klaviatuur ja hiir. Mingi asi on 
fookuses, kuidas need sündmused jõuavad õige objektini, ja see on  klaviatuuri 
ja hiire puhul väga erinev loogika. Ja kõik see oli selliseks loogiliseks 
kompotiks keeratud, et sinna oli lihtne rakendusi teha. Sellel tükil oligi 
umbes üks programm, mis  seda ägedat raamistiku kasutas, see oli see sama 
programmaatori kasutajaliidese. Aga noh, selles mõttes oli Ektaco väga tore, et 
need tööülesanded ei olnud väga piiravad. Sa võisid ikkagi, ma ei tea, 
kuude või isegi aastate kaupa rahulikult häkkida ja sealt lõpuks tuli mingi asi 
välja. 

\question{Ja teistpidi, ega sul ei olnud neid akende joonistamise asju võtta 
riiulist kümneid?}

Ei, ikka oli. Sedasama Turbo Visionit oleks võinud pruukida ja seal oli 
igasuguseid teeke. Aga kuidagi, mis see siis on, nagu ametiuhkus ei lubanud 
teise mehe akna teeki kasutada. Tuleks ikka enda oma teha, sest et no mis 
mõttes, ma ei oska nüüd parimat akendusteeki teha. 

\question{Sellist suhtumist pannakse tänapäeval pahaks? Või ei panda?}

Seda tehakse teisel tasemel, eks ole. Tasemeid on juurde tulnud, seal 
nokitsetakse hoopis mingisuguste muude asjade juures, aga mina arvan, et see on 
nagu suht paratamatu, et see on hädavajalik, et inimesed heas mõttes 
leiutatakse jalgratast. Teeks asju, mis on juba tehtud, aga teeks teistmoodi, 
teeks paremini. Põhimõtteliselt olid ju opsüsteemid olemas, et mis mõte oli 
seda Linuxit hakata tegema, PC-Unix oli olemas. See oli olemas, et no mis siis 
häda oli sellel SCOl või millel iganes. 


\question{Jah, põhimõtteliselt oleks ju võinud olla, et siiamaani kõik 
kasutaksid sinu aknategijat}

Kindlasti need inimesed, kes on armunud kaheksakümmend korda kakskümmend viis 
teksti ekraanisse, need oleks olnud siiamaani selle andunud kasutajad. 

\question{Mäletan, FoxPro\index{FoxPro} joonistas lausa mingeid varjusid 
akende taha}

Ja, see on loomulik, varjud akendel pidid olema.

\question{Kas seda teie kiibikõrvetajat kasutati väljaspool 
Ektacot\index{Ektaco} ka?}

Need asjaolud muutusid nii kiiresti, et see, mis oli kallis ja kättesaamatu 
kaks aastat tagasi,  kaks aastat hiljem ei olnud enam seda. Ja ma arvan, et 
seda võib-olla tehti mingi üks või kaks eksemplari ja seda pruugiti Ektaco 
siseselt, aga sellest mingit edulugu ei tulnud. Ja see ei olnudki põhitegevus. 
Mina jälle ei tea, eks ole, et miks seda üldse tegema hakati, kas tõesti oli 
siis nii kättesaamatu või lihtsalt oli äge seda teha.  

\question{Jah, kui ma sind kuulan, see ei kõla suurepärase ärina}

Ektaco tegi ju  äri ka. Ja ma pean tunnistama ausalt, et  mind huvitas tollal 
programmeerimine. See, et mida  kolleegid nagu tegid, ma teadsin, aga ma väga 
ei süvenenud sellesse. See oli hästi selline fokusseeritud toimetamine.

\question{Kas tol ajal tekkis mingi kokkupuude arvutisidega ka juba?}

Seal Ektacos oli mul terve hulk toredaid kolleege. Olid 
Tarvi\index[ppl]{Martens, Tarvi}, Heiki Kask\index[ppl]{Kask, Heiki}, Jaak 
Niit\index[ppl]{Niit, Jaak}, Gunnar Valge\index[ppl]{Valge, Gunnar} oli seal 
minuga samas toas, kindlasti oli veel paar-kolm inimest. Ja siis meil oli 
Fido\index{FidoNet} \emph{point}, mis siis tekkis jälle seal Tarvi ja Heiki 
initsiatiivil, minu meelest ennekõike. Me olime alguses Lõvi point. 
Lõvi\index[ppl]{Lõvi|see{Lepp, Andres}}\sidenote{Lõvi, pärisnimega Andres 
Lepp\index[ppl]{Lepp, Andres}, on legendaarne TPI arvuti-mees, paljude meie 
põlvkonna inimeste sõber, teejuht ja eeskuju} oli siis TPI 
Arvutuskeskuses\index{Tallinna Tehnikaülikool!Arvutuskeskus}. Minu jaoks oli ta 
kunn, ma ei tea, mis ta seal tegelikult oli ja siis olime seal Lõvi 
\emph{point}. Jooksutasime seal FrontDoori\index{FrontDoor}\sidenote{FrontDoor 
oli üks populaarsemaid FidoNeti mailereid} ja mida iganes me jooksutasime. 

Ma arvan, et mingil hetkel me \emph{point}i staatusest \emph{upgrade}sime 
ennast \emph{node}ks. 71 oli meie number, julgeks arvata. Ja me helistasime 
kuhugi sisse ka, sest ma mäletan, et ma olen mingisuguse \emph{prompt}i otsas 
rippunud. Ja vaat seda jälle ei tea, et kust ma sain teada, mis käskudega seal 
Unixis\index{Unix} midagi teha. Ja kuidas mingi binaarne fail ära 
\emph{uuencode}da, selleks et ma saaks seda üle terminali endale 
\emph{dump}ida, selle \emph{dump}i salvestada, oma masinast \emph{decode}da ja 
mingit zipi sealt seest kätte saada. Kuidagi ma teadsin seda, kuidagi ma 
mingisuguseid asju imesin. Aga see on jälle niimoodi, et mingid asjad olid nagu 
õhus nagu mingisugused hallitusseene eosed laiali. Nii, kui kusagil pinnase 
sai, kohe läks kasvama. 

\question{Nii mitu sammu selleks, et midagi kätte saada, barjäärid olid jube 
kõrged toona.}

Info ikkagi liikus, see, ma arvan, ei olnud probleem. Küsimus oli ikkagi 
ennekõike riistvaras ja \emph{access}is ja  telefoniliinides ja niisuguses 
kraamis. Modemid olid ju roppkallid asjad, eks ole. Arvutid,kõik oli roppkallis 
välja arvatud aeg. Töö juures õnneks meil mingeid modemid olid, mitte küll 
kõige härjemad. Meil oli mingi 2400 ja MNP5\sidenote{\emph{Microcom Network 
Protocols (MNP)} on perekond (tähistatud numbritega ühest kümneni) 
veaparandusprotokolle, mida sageli kasutati varastes kiiretes (2400 bit/s ja 
rohkem) modemites} oli see meie lagi, millega me seal alguses toimetasime siis. 
Aga kõik olulised asjad liikusid ikka flopide peal, seda ei viitsinud keegi 
ära tõmmata, tõmmati mingeid pisikesi nublakaid. Tollal oli flopiga bussi peale 
minek reaalselt kiirem kui modemiga toimetamine.

\question{Mis sorti materjali te oma nodes hoidsite?}

Point oli meil puhas Fido point. Meil minu meelest küll BBSi ega midagi olnud. 
Meil oli ikkagi sõnumivahetus, \emph{Echomail} ja \emph{Netmail}, ehk siis 
privaatkirjad ja niisugused avalikud foorumid. See oli see, miks me nii-öelda 
suures pildis seda \emph{node}i pidasime. Kui keegi midagi tõmbas, siis ta 
tõmbas enda jaoks ja võib olla jagas  kolleegidega kuidagi midagi aga meil 
mingit sihukest varamut või niisugust ei olnud.

\question{Kellega te neid meile vahetasite, mis uudisgruppe lugesite? Kogukond 
ei olnud ju suur? Lõviga sai ju niisama ka juttu rääkida, ei pidanud kirja 
saatma?}

Mina lugesin põhiliselt \emph{Echomail}i, mul mingisuguseid kirjasõpru, kellega 
mingeid asju seal väga oleks olnud ajada, et tegelikult väga ei ei olnud. Minu 
jaoks oli see lihtsalt nagu foorum, kus sa saad huvitavat ja enamasti ka väga 
humoorikat  sisu. See väljendustase, see, kuidas inimesed, ükskõik mis teemal, 
viitsisid oma mõtteid sõnastada, need iroonia, sarkasm, huumor, kõik need 
tasemed, see oli niivõrd hea tekst valdavas osas, et seda oli  alati lust 
lugeda. Ükskõik mis oli, mingid autofoorumid, mul polnud  sooja ega külma 
nendest autodest. Aga lihtsalt need naljad, need vihjed, see oli lihtsalt hea 
meelelahutus, enamuses. Muidugi seal on ikka programmeerimised ja riistvara ja 
kõik muud teemad ka. See oli kasulik ja naljakas.

\question{No aga skaalal Tolkienist üle autode C++-ni?}

No kõike, absoluutselt. Kogu elu oli seal minu meelest. Seda jaksas tervikuna 
läbi lugeda sest inimesi oli vähe, palju sa ikka seda head kvaliteetset sisu 
suudad toota. Seda  oli vähe tegelikult, mis seal liikus minu meelest.

\question{Ühesõnaga, praeguses mõistes oli võimalik kogu sisuloomel silm peal?}

No sellel, mis Fido \emph{Echomaili} kaudu tuli, jah. Seal kuskil paralleelselt 
hakkasid arenema mingeid \emph{newsgroupid}, ka Eesti omad, millega mina 
alguses eriti ei puutunud  kokku. See oli natukene teine seltskond minu 
meelest, kes seal nii-öelda internetimaailmas hakkas toimetama. 

\question{Need olid kaks eri maailma, nende vahel mingit silda ei olnud?}

Nii ja naa, kontseptsiooni mõttes olid interneti uudisgrupid ja Fido omad 
samad, aga seal olid mingid ebamugavad erisused. Kunagi  hiljem, kui ma 
Ektacost Küberneetika Instituuti läksin\index{Küber|see{Küberneetika 
Instituut}}\index{Küberneetika Instituut|see{Cybernetica}} siis ma tegin oma 
\emph{node} Solarise\index{OS!Solaris} peale. Meil oli seal üks 
SPARC\index{Arvutid!SPARC}\sidenote{\emph{Scalable Processor Architecture 
(SPARC)} on Sun Microsystems'i poolt arendatud RISC-arhitektuur. Sun müüs 
sellele arhitektuurile tuginevaid, siinmail populaarseid, servereid ja 
tööjaamu} server ja siis ma ajasin seal peal käima kogu selle Fido softi. Üks 
venelane oli selle kirjutanud. Ja siis ma tegin \emph{news}i \emph{gateway}, 
mis nagu Fido Echomaili \emph{newsgroup}ideks köitis kahesuunaliselt ja siis 
ühtlasi ka Netmaili siis tavaliseks meiliks köitis. See oli päris popp, ma 
isegi ei mäleta, millal see maha sai võetud. Ma arvan, et seal juhtus see asi, 
et sellele Solarisele oli lõpuks vaja  korralik \emph{upgrade} teha ja siis ma 
ei viitsinud vist enam. Fido oli ära surnud selleks hetkeks ja siis ma tõmbasin 
ta maha. Aga mingil ajal  oli ta hästi popp, mul oli seal, ma arvan, ikkagi 
sadu sadu kliente oli oma personaalse \emph{account}iga seal minu \emph{news}i 
serveri küljes, kellel oli siis nii-öelda kirjutamisõigus Fido gruppidesse. 
Fidos oli see korrapidamine nagu olulisem, seal ei olnud sellist anonüümset 
kasutust, keegi vastutas alati kellegi eest. Keegi  kuskilt kaudu sai 
\emph{access}i ja kui see keegi oli nõme, siis see \emph{access} võeti talt 
ära. Kui ma hakkasin seda asja Newsi \emph{gate}ma, siis ma lubasin sedasama 
teha, eks ole. Ma ei andnud kellelegi Fido gruppidele  kirjutamisõigust, kui ma 
ei teadnud, kes ta on ja ma ei saanud seda \emph{access}i talt ära võtta. 

\question{Aga see on ju, ütleks, autoritaarne?}

See toimis. See oli nagu  endale olulise keskkonna  normaalsena hoidmise 
eeldus. Teistmoodi ei saa. 

\question{Aga mis on \enquote{nõme}?}

No, solvad teisi inimesi, trollid, ütled puhasti, eks ole. See ongi põhiline, 
et kui sa lähed isiklikuks, teed teisele haiget, halba emotsiooni, sihukest 
asja ei ole vaja. See kui sa vaidled, see on okei, seda peab olema, see ongi 
tähtis, eks ole. Aga sa ei tohi  teistele haiget teha. 

\question{Kõlab nagu lihtne, eluterve ja samas fundamentaalne definitsioon. Aga 
kui sa Ektacost ära tulid, kas sa veel õppisid?}

Ei, mu õppimised olid selleks hetkeks õpitud või noh, mitte päris lõpuni 
õpitud, aga ma olin oma inseneridiplomi kätte saanud, vist 1993 või 1992. Sain 
oma kraadiselt kätte. Magistrikraadiks \emph{upgrade}sin ma ta siin natuke 
hiljem. Mina õppisin, viis aastat, sain süsteemiinseneri diplomi, aga pärast 
hakati  kogu seda kraadi värki järjest lahjendama, eks ole. Kui nüüd õppeaastad 
järjest lühenesid, siis \emph{by default} oli mul bakalaureus, aga siis ma 
pidin veel natuke juurde õppima ja tegema magistritöö, ma saaks magistriks. See 
oli kunagi seal 2001. aastal umbes, kui ma selle  ette võtsin. 

\question{Aga tol hetkel sul ei olnud sellist tunnet, nutikas ja usin õppur 
nagu sa olid, et peaks teadusmaailma sukelduma?}

Ega ma seal TPIs ise teadusmaailma suurt kokku ei puutunud. Kuna ma sealt poole 
pealt hakkasin programmeerijana tööd tegema, eks ole, siis see  haaras  
enam-vähem täielikult. Ma arvan, et lõpus läks kas see õppimine natukene 
nigelamaks, sest töö juures oli palju huvitavam ja palju nagu väljakutseid 
pakkuvam. Viimased asjad mis seal Ektacos\index{Ektaco} sai tehtud, oli 
kontrollerite uue sideprotokolli disainimine. Ma olin hullult vaimustatud 
TCP/IPst ja  siis ma trükkisin välja kõik standardid, mis ma sain: TCP, IP, 
Etherneti. Aga kontrollerid on mingi 8051 peal, mis on umbes nagu, nagu väga 
väike. Aga siis ma lugesin need RFCd kõik läbi ja siis ma tegin mingi oma 
sideprotokolli, mis inspireerus siis kõigest: Ethernetist, IPst ja TCPst. Ehkki 
ta ei olnud nagu päris \emph{flow}le orienteeritud, aga pigem  selline 
\emph{datagram}i-põhine protokoll. Sihukesed vanad riistvaraässad Ektacos  olid 
väga nördinud ja solvunud, et mis mõttes ma kirjutan protokolli, mis ei ole 
deterministlik. Mitte \emph{master-slave}, vaid igaüks võib traadi peal 
lobiseda, kui mõte pähe tuleb, ja siis lahendatakse konfliktid ära ja tehakse 
re-transmissioon. Nad olid väga pahased minu katsetuste peale, aga ma arvan, et 
programmeerisin selle lõpuks sinna ära ja ta mingil määral töötas ka. See oli 
päris äge.

\question{Aga mille vahel see protokoll siis käis?}

Põhimõtteliselt oli see, et PC, mis siis juhtis neid tööstusarvuteid. Neil oli 
sihuke karp, mille nimi oli satelliit, mis oli siis tööstuskontroller, millel 
olid igasugused digi- ja analoogsisendid-väljundid, mis kuskil tehases keerasid 
mingit nuppu, et betooni teha või midagi. Ja siis sellel olid mingid
juht-programmid ja neid tuli konfida. Tüüpiline värk, eks ole. Sa pead teadma, mis 
sul tehases toimub. Sa pead käske andma, selleks on mingit võrku vaja. Ja neid 
satelliidikontrollereid võis seal korralikus tehases ikka palju olla. Ja siis 
ta tuli PCsse kokku tõmmata ja ma usun, et keegi kirjutas siis mingit softi 
sinna PC poolele, mis siis neid satelliite siis jälgis ja juhtis.

\question{See kupatus oli päriselt \emph{production}is ja Eesti Vabariigis 
tehti betooni niisuguste seadmetega?}

Jaa. Ma arvan, Palivere Ehitusmaterjalide Tehas\index{Palivere 
Ehitusmaterjalide Tehas} vist oli see, mis oli ära automatiseeritud Ektaco 
poolt ja ma millegipärast arvan, et midagi oli Tallinna 
Veepuhastusjaamas\index{Tallinna Veepuhastusjaam}.Aga seda ma väga kindlalt ei 
tea, aga seal oli neid veel. Neid objekte ikka oli.

Mida mina tegin, see oli järgmine generatsioon, need objektid juba töötasid 
mingisuguse muu protokolli ja mingi muu  tehnika peale, aga kõik see kasvas, 
eks ole. Ja siis algatati uue generatsiooni satelliidi väljatöötamise projekt, 
kus mina siis  protokolli kontributeerisin ja realiseerisin. 

\question{Kui sa võrgundusest juba nii palju teadsid, sind kuhugi interneti 
varasesse maailma ei tõmmatud kaableid vedama või midagi?}

Ei, mulle meeldis programmeerida. Nende muude asjadega ma tegelesin nii palju, 
kui nad olid kasulikud ja vajalikud selleks, et saaks midagi ägedat 
programmeerida. 


\question{Ja Küberis\index{Küber} sai ägedamalt programmeerida?}

Lõpuks jah. Jälle Tarvi\index[ppl]{Martens, Tarvi} kutsus mind sinna. 
Küberneetikasse oli tehtud infotehnoloogia osakond, mis peitis seda infot, et 
tegelikult tegeldi seal infoturbega ja siis oli seal mingi riiklik programm, 
mille eesmärk oli Eesti riigi infoturbe ja krüptograafia vajadusi rahuldada. 

\question{See oli juba enne, kui tekkis AS Cybernetica?}

Jaa, see oli enne seda. Mina läksin sinna 1994, aga see töögrupp tehti 1993, ma 
arvan. Ja siis seal oli terve hulk nutikaid inimesi koos, kes siis  selle 
missiooni elluviimisega tegelesid, et  kompetentsikeskust ehitada.

\question{Kes selle taga oli? Keegi pidi ju selle tellimuse formuleerima, et 
riiklikult on tarvis tegeleda krüpto ja infoturbega?}
Ülo Jaaksoo\index[ppl]{Jaaksoo, Ülo} oli siis Küberneetika 
Instituudi\index{Küberneetika Instituut} direktor. Minu vaates oli tema see, 
kes seda kõike lõi ja korraldas. Kuidas ja  kellega tema läbi rääkis või kust 
see mandaat tuli, seda mina ei oska küll öelda. Aga tema oli jah, kellel see 
visioon  oli, et seda on tarvis. 

\question{Arvestades, kui vähe vajas Eesti riik krüptot ja infoturvet praegu ja 
kui strateegiliselt oluline teema see praegu on, siis sellise visiooni jaoks on 
ju tarvis väga ägedat ettenägemisvõimet?}

No aga kaugemale vaatamine ongi teadlaste ja akadeemikute ülesanne. Kust mujalt 
see tulla saab? 

\question{Visioon visiooniks, mida see töö toona praktiliselt tähendas?}

Esiteks, ise õppida. Teiseks, teisi õpetada. Eestikeelne terminoloogia, 
standardid, profiilid, seminarid, koolitused mida iganes.  Ja  teistpidi 
hakkasid niisugused praktilised asjad tulema. Vaata, tollel ajal maailm oli 
nagu väiksem, ka krüpto ja infoturbemaailm oli väiksem ja mingil hetkel on 
ikkagi veel võimalik hoomata  kõike, mis oli oluline. Mitte küll päris üksi, 
aga sihukese väikese töögrupi sees nagu meil oli. Ma arvan, et mis meil  väga 
hästi läks, oli see, et meil olid inimesed, kes  tegelikult  huvitusid just  
sellest infoturbe süsteemsest poolest. Et mitte see, et mis on nagu see 
tehnika. Aga mis on see organisatsioon, need inimesed, need reeglid, eks ole, 
seadusandlus seal ümber. Ühesõnaga süsteemne lähenemine valdkonnale kui 
tervikule, mis on  väga tähtis ja  mis sellest meie grupist välja kasvas. 

Teiselt poolt oli see, et meil on seal sihukesed \emph{hardcore} häkkerid ja 
\emph{hardcore} krüptograafid, kes nagu olid valmis mida iganes tegema. See 
sümbioos oli minu meelest hästi lahe. Ma arvan, et minu esimene töö 
Küberneetika Instituudis oli see, et ma pidingi riigiasutustele kirjutama 
juhendi, kuidas KA9Q\index{KA9Q} otsas ehitada endale internetti ruuter. 

\question{Mille otsas?}

KA9Q on üks soft. \enquote{KA9Q} on mingi radistide kutsung, mis vastab mingile 
inimesele, kes selle softi kirjutas, on minu arusaamine. Ja see oli DOSi peal 
jooksev \emph{all singing all dancing} asi, mis realiseeris TCP, kõikvõimalikud 
sideprotokollid, võrgukaartide toed, SLIP, PPP, ruuterid, mida iganes. FTP 
deemonid. Täiesti müstilisi asju on tehtud maailmas.  Et kui sul oli üks  
üleliigne PC, modem, võrgukaart ja see soft, siis sa said teha endale ruuteri, 
millega oma organisatsioon kuhugi ära ühendada. Ja siis mina peksin selle käima 
ja kirjutasin eestikeelse lühijuhendi, kuidas seda asja  pruukida, hooldada ja 
nii-öelda käimas hoida. See oli mu esimene nii-öelda, ma ei tea, praktikandi 
töö või mis iganes töö seal Küberis. Aga siis hakkasid igasugused muud asjad
tulema.

Me olime mingis hästi varajases europrojektis, ma mäletan, see võis olla 1995. 
aastal. Ma tean, et ma käisin Darmstadtis\index{Darmstadt}. Sakslased olid 
kirjutanud sellise tarkvara nagu secu-d, mis oli,  ma ei kujuta ette, et ma 
pakun mingi kümme mega haljast C koodi väga halvasti kirjutatud, mis  
realiseeris kogu krüpto, mis tolleks hetkeks oli teada. Kõik sertide töötlus, 
särk-värk. Ja siis me üritasime seda secu-d'd kuidagi rakendada ja kuidagi 
käima peksta. Ütleme niimoodi, et selline \emph{cross-platform} arendus tollal, 
et sul on kood, mida sa kompileerid mingi UNIXi jaoks ja mingi PC jaoks ja siis 
tulid Windowsid, eks ole. Ja teha nii, et see kuidagi enam-vähem  töötab ja 
piisavalt vähe mälu lekib ja piisavalt harva sama mäluplokki kaks korda 
vabastab on  raske ülesanne. Ja siis ma selle secu-d najal ehitasin 
mingisuguseid asju. Turvalist meiliklienti näiteks ja sertifitseerimiskeskust. 
Sertifitseerimiskeskused olid lahedad,  seal mingisugusel  ajaperioodil oli 
see, et me seal Küberneetika Instituudis iga aasta programmeerime vähemalt ühe 
sertifitseerimiskeskus valmis softi mõttes.

\question{Miks?}

See oli mingisugune \emph{blend} sellistest praktilistest vajadustest ja 
teadustöö eesmärkidest. Et üks  sertifitseerimiskeskus, mille me näiteks 
programmeerimine oli näiteks selline. Tollal ei olnud ju mingeid kiipkaarte ja 
riistvaralisi turvamooduleid kätte saada. Ja see oht, et kui sul 
sertifitseerimiskeskuse võti ära 
 kompromiteerub, et siis keegi annab võltssertifikaate välja, see oli suur. Või 
et keegi annab sellele operaatorile altkäemaksu, et annaks võltssertifikaadi 
välja. Sul oleks vaja mitmesilma printsiipi ja sihukest  topeltkaitset. Ja siis 
me realiseerisime selle, et me võitsime selle RSA võtme tükkideks. See on 
seesama, mida praegu SplitKey\index{SplitKey} ja SmartID\index{SmartID} teevad. 
Meil ei olnud küll seda turvalist mitmes osas võtme genereerimist, me lihtsalt 
RSA võtme, jagasime ta osakuteks ja siis meil oli sihuke m-n-ist skeem. 
Selleks, et sertifikaati välja anda, siis viiest operaatorist kolm pidid  
allkirja andma ja siis me kombineeris neist korrektse sertifikaadi kokku. Selle 
nii-öelda initsialiseerimisprotsessi käigus tekitati viis flopit,  millega need 
 operaatorid ringi oleks pidanud käima. Selles mõttes oli ta praktiline, et ta 
töötas,  tegi täitsa korrektseid X.509  sertifikaate ja oli kasutajajuhendiga 
varustatud.  
 
\question{Tundub, et kui sa enne seal ISA siini peal tegelesid väga madala 
taseme asjade katsetamise ja läbi mängimisega, siis nüüd sa tegid sedasama 
krüpto jaoks põhiolemuses olulisi primitiive ja protsesse läbi realiseerides?} 

Jah, et seda võib öelda küll, et mingis mõttes me tegelesime selliste hästi 
\emph{basic} asjadega. Me jõudsime ka rakendusteni välja. Meil oli ka 
hästi-hästi praktilisi asju, aga me kontrollisime tegelikult kogu seda pinu 
ülevalt alla välja. Et sellel ühel hetkel me tegime tulemüüre, mis oli väga 
hästi müüv toode Eesti turul, Barrikaad\index{Barrikaad} oli selle nimi, mul 
siiamaani barrikaadi T-särk alles. Siis me tegime VPN toote, mis oli veel 
ägedam. Selle VPNi teine versioon oli igasugustes Eesti riigiasutustes 
väga-väga pikalt kasutusel ka peale seda, kui selle tugi ametlikult õnnetuseks 
ära lõppes. Ja selle põhieelis oli see, et ta oli projekteeritud hästi 
turvaliseks, keskelt administreeritavaks, eriti töökindlaks. Ehk et see, et sul 
on  harukontorid, kust sa ei taha üldse interneti väljapääsu, vaid tahad läbi 
keskse tulemüüri (mis oli kallis) neid välja juhtida, see oli meil sinna sisse 
ehitatud. Igasugused paralleelsed ruutingud üle erinevate kanalite, eks ole. 
Seal tekivad probleemid, kui sul on VPN tunnel, sul on  sisemised aadressid, 
välimised aadressid, kuidas sa neid majandad niimoodi, et see ruutingu info ka 
seal sisevõrgus korrektselt leviks ja tegelikult ka töötaks. Et kasutajad 
ei peaks  ootama, kuni nende seanss katkisest kanalist tervesse kolib, eks ole, 
et see lihtsalt töötakski. Ja kogu see administreerimine. Meil oli tehtud see 
tükk, mis võimaldas süsteemi konfiguratsiooni muuta, see oli eraldi, see võis 
offlainis olla, see suhtles  muu maailmaga floppide kaudu, see ei olnud võrgus. 
Ja siis oli meil võrgus olev tükk, mis ainult monitooris, kogu sealt infot ja 
täitis neid käske, mis võrgust väljas olev tükk talle  ette pani. Niisugune 
eriti kõrgete turvanõuete jaoks tehtud haldussüsteem. Ja, ja seal me muuhulgas 
siis, kuna tollal ikkagi see PC krüpteerimisvõime oli nõrk, siis me 
realiseerisime ise  šifreid. Tollal just MMXi laiendused tulid prosele välja, 
mis võimalused sul näiteks IDEAt\sidenote{\emph{International Data Encryption 
Algorithm (IDEA)} on esmakordselt 1991. aastal kirjeldatud sümmeetriliste 
võtmetega plokkšiffer} paralleelselt arvutada, mitu plokki korraga. Ja siis 
Helger Lipmaa\index[ppl]{Lipmaa, Helger} oli veel Küberis tööl, kes 
programmeeris siis Linuxi tuuma jaoks MMXi \emph{extension}eid  kasutava 
AESi\sidenote{\emph{Advanced Encryption Standard (AES)} on Belgia 
krüptograafide poolt välja töötatud Rijndael plokkšifri alamhulk. 1997. aastal 
teatas NIST (\emph{National Institute of Standards and Technology of the United 
States (NIST)}) plaanist asendada avaliku protsessi abil tolleks ajaks 
ohtlikult nõrgenenud DES algoritm. Vincent Rijmen ja Joan Daemen esitasid oma 
ettepaneku valikuprotsessi ja see standardiseeriti NISTi poolt 2001. aastal}  
realisatsiooni. Meil seal Linuxi\index{OS!Linux} tuumas olid oma draiverid, mis 
seda VPNi asja haldasid, seal peal olid  oma deemonid võtmete vahetuseks, konfi 
levituseks, kõigeks muuks  ja siis niimoodi hierarhiliselt üles välja.

\question{See, mis sa räägid, et see ei kõla enam nagu programmeerimine, see 
kõlab nagu arhitekti töö. Kas sa liikusid programmeerija rollist arhitekti 
rolli või mõtlesite te neid asju kambakesi välja, kuidas see käis teil?}

Selles mõttes, et välja mõtlesin kogu aeg lihtsalt enamasti oli see teine tüüp, 
kes asju realiseeris,  sellesama peakolu sees. Lihtsalt seal tulid inimesed 
nagu appi. Meil ei olnud  väga selgelt nagu defineeritud rolle, eriti alguses, 
eks ole. Arhitekt, projektijuht, projektijuhid olid üldse väga haruldased 
nähtused, Me ei teadnud isegi, mis projekt on, me lihtsalt programmeerisime 
mingi hetkeni. Meil oli seal, jah, ikkagi terve hulk inimesi, kes arutasid 
intensiivselt praktiliselt kõigil teemadel. Kui asjad olid selged ja siis 
igaüks natukene läks oma  valdkonnas  süvitsi sellega.

\question{Nutikatel inimestel on vahel oma nutikusele vastav ego ka, keegi nina 
püsti ei ajanud ja ennast arhitektiks ei kuulutanud?}

Ei, päris nii ei olnud. Aga ma ise kardan tagantjärgi võib-olla mina ise 
kippusingi see tüüp olema, kes oma  arvamust teistele peale surus. Aga ma tol 
hetkel ei tajunud seda kindlasti niimoodi. 

\question{Ma arvan, et ega teised ka ei tajunud ja soft ju lõpuks ikkagi 
töötas ju}

Absoluutselt. Nii see tulemüür kui ka see VPN, olid meil ikkagi lõpuks ikkagi 
ääretult stabiilsed ja, ma ütleks, kvaliteetset tükid. 

\question{Privador kasvas ka ju sealt välja?}
Jah, Privador\index{Privador} oli siis Küberneetika Aktsiaseltsi spin-off 
firma, mis siis sai need nii-öelda infoturbetooteid, eesmärgiga need laia 
maailma viia, aga see kahjuks ei õnnestunud. Seal oli  kindlasti ports 
probleeme ja üks probleem, mida mina nägin oli see, et tollal hakkasid tekkima 
standardid, et mis asi on VPN, mis asi on standardne VPN. Ja IPSec oli 
enam-vähem ära standardiseeritud, IKE oli ära standardiseeritud ja see oli 
tegelikult see, mida oleks tahetud osta. \emph{Vendor lockin}i juba päris 
mõõdukalt kuni palju kardeti. Ja ehkki meie olime oma asja ehitanud, eriti need 
alumised kihid, need olid  standardite põhjal ehitatud aga mudel,  kuidas me 
nägime seda võrgu tervikut ette ja mida me pidime tegema, selleks, et neid häid 
omadusi saada,  seal tekkisid konfliktid IKE või ütleme, IPSeci, ideoloogiaga 
natukene. Meil  tegelikult oli töölaua peal  versioon kolm VPNist, mis oleks 
siis olnud täiesti standarditega ühilduv, mis loodetavasti selle  firmapärasuse 
probleemi oleks ära kõrvaldanud, aga see kahjuks ei läinud realiseerimisele. 
Selle asemel me tegime digiallkirja tarkvara ja ajatembeldustarkvara ja 
Notariseerimistarkvara ja kõike muud. Me nagu natuke ennustasime valesti, et 
mis on see \emph{killer} rakendus krüptomaailmas järgmise kümne aasta jooksul. 
Olime nagu natuke ajast seest selles mõttes.

\question{See lähenemine, et võtame alumise kihi standardid ja paneme nad 
kuidagi täitsa uut moodi ülemise kihi standarditeks kokku on ju seesama, mis 
sai digiallkirja konteineriga tehtud ja X-Teega ka}

Absoluutselt. Aga vaat seal ongi see, et standardid on ja peavadki olema 
tegelikult geneerilised, eks ole. Nad peavad olema sellised, et nad lahendavad 
paljude inimeste paljusid probleeme, siis nad on elujõulised. Nii. Aga aga kui 
sa võtad ühe konkreetse riigiasutuse, kellel on konkreetsed vajadused, mis ta 
peab ära lahendama efektiivsel viisil, siis sa ei pääse lihtsalt sellega, et sa 
võtad standarditele vastavat tüki ja evitad selle. See ei ole efektiivne. Ja 
see oli siis see, mida meie tegime. Aga seal oligi vaata natukene see, et me 
võib-olla ei tajunud seda, et kui suur see maailm on ja kui võimas ta on ja kui 
suure massiga ja kui kiiresti ta liigub. Me mõtlesime, et me teeme ikka rajult 
ägeda asja. Ja noh, see on nagu \emph{way}  parem ja praktilisem väga suure 
hulga klientide jaoks. Aga see teadmine, et miski asi on hea ja praktiline,  
seda on väga raske efektiivselt ja kiiresti ühest peast teise viia.  

\question{Arvestades, et samast pundist tulid ju ka X-Tee\index{X-Tee} ja 
ID-kaardi kontseptsioon, siis kahest kolm ei ole üldse mitte paha edu protsent}

X-Teega on muidugi see, et X-Tee omab selles meie VPNi tootes väga selgeid 
juuri. Tegelikult, kui me seda X-Teed tegime, see oli 2001.  Mais või juunis 
hakkas asi pihta või isegi natuke hiljem ja detsembris läks tootesse. Eks ole. 
See oli võimalik ainult tänu sellele, et me võtsime oma selle VPN toote kui 
substraadi. Meil oli kõik see olemas, et kuidas me teeme ühe Linuxi purgi 
turvaliseks, kuidas me sellele  Linuxi purgile paneme peale oma tarkvara 
\emph{patch}id, särgid-värgid, kuidas me seda Linuxit konfime, kuidas me hoiame 
konfi niimoodi, et see on efektiivne, kuidas konfi jagamine käib, see kõik oli 
olemas. Me lihtsalt selle asja peale ehitasime ühe natukene teistsuguse 
protokolli vahenduse tüki, eks ole. 

\question{Aga see kõik on natuke hilisem lugu. Kui mina sinuga esimest korda 
kliendina kohtusin, siis sa ikkagi juba juhtisid vägesid. Mina rääkisin oma 
mure ära ja sina tegid nii, et asjad sündisid. Kuidas sul inimeste juhtimine 
rollina esile kerkis ja kas sa üldse mõtestad seda tegevust niimoodi?}

See tekkis Barrikaadi\index{Barrikaad} või VPNi või Privadori\index{Privador} 
programmeerimise käigus, kui meeskond läks suuremaks. Eriti selle VPNi juures, 
ma arvan,  koordineeriv funktsioon oli ikkagi minu peale, et kes nüüd mida 
programmeerib, eks ole, mis ajaks. Ja kes neid asju evitamas käis, ikka meie 
ise, sealt tuli ka see klientidega suhtlus, eks ole. \emph{Helpdesk}, 
projektijuht, arhitekt, programmeerija, testija, tarneinsener, et mu roll oli 
natukene nagu kõik koos. 

\question{Aga ometi kuidagi jäi see koordineeriv roll just sinu peale?}

No ju siis selles pundis see  kõige paremini  mulle sobis, ei oska muud midagi 
arvata. Keegi pidi selle ära tegema, eks ole. Kui see olin mina, siis olin see 
mina, nii see läks.

\question{Ma selle pärast küsin, et ega sul mingisugust kihu ei olnud inimesi 
juhtida?}

Ei. See pigem oligi sedapidi, et, ma nägin seda, mis see asi võiks olla, mida 
me teeme,  päris detailselt päris paljudes aspektides. Ja siis ma nagu tahtsin, 
et see nii läheks, siis ma olin sunnitud  inimestele  ülesandeid või siis 
eesmärke püstitama. See tuli pigem sedapidi, et üksinda ei jaksa kõik ära 
progeda.

\question{Aga see on jällegi arhitekti vaatenurk. Minu peas on olemas täiuslik 
mudel süsteemist ja siis ma teen niimoodi, et see saaks teoks tehtud. Mis sa 
praegu teed?}

Mis sa praegu teed? Väga paljusid erinevaid asju. Ma suhtlen hästi palju 
klientidega ja potentsiaalsete klientidega, et aru saada, mis on  nende  mured 
ja vajadused, kuidas me saame   neid aidata. See on alates müügitööst, projekti 
juhtimiseni. Teistpidi ikkagi see, ütleme, arhitektuurne töö. Kui probleem on  
arusaadav, et mis oleks see lahendus. Ja need probleemid on keerulisemaks ja 
mastaapsemaks läinud. Mõnes mõttes ka vastutusrikkamaks selles mõttes, et me  
ikkagi tegutseme suuresti turvavaldkonnas. Ja see keskkond on nii palju 
vaenulikum ja nii palju keerulisem ja need panused on nii palju suuremad, et sa 
pead lihtsalt palju palju paremaid asju tegema kui me kunagi tegime. Sedasorti 
arhitektuurne  mõtlemine ja siis inimestele nende ideede jagamine. Nõustamine, 
mõnes mõttes ka võiks öelda isegi natukene koolitamise moodi asjad. 

\question{Sa oled kogenud arhitekt ja tead, mida on vaja selleks, et projekt 
välja tuleks. Kuidas sa viid entusiastlikult pihta hakanud meeskonnale kohale 
selle, et sinu arvates projekt ei saa välja tulla? Ja seda nii, et sind pärast 
tuppa tagasi ka lastakse?}

Samm üks on see, et sa pead aru saama. See võtab tegelikult päris kaua aega ja 
see on nagu see koht, kus tihti suhtled väga vähe. Ega seda, et vaatad peale, 
saad kohe aru,  mis valesti on, kuidas peaks olema, seda ei ole. Kõigepealt 
pead probleemist aru saama. Ja võib olla, et  sellepärast see see olukord ongi 
võib-olla keeruline või halb,  et see ongi olemuslikult keeruline probleem. 
Seal on mingisugused mingisugused põhjused, keegi on teinud mingeid otsuseid, 
mingeid probleeme on lahendatud ja selle käigus on tekkinud niisugune asi. Sa 
pead sellest aru saama. Sa ei saa lihtsalt minna, et \emph{sorry}, vanad, et 
siin on jama. Sa pead kõigepealt aru saama, mida on tehtud ja miks on tehtud, 
need probleemid endale selgeks tegema. Ja siis sa tõenäoliselt marineerid nende 
otsas päris kaua ja see ei tule niimoodi, et hops, homme hommikuks on valmis, 
eks ole. Sa mõtled ja kirjutad ja räägid. Ja ehkki tihti on niimoodi, et  sulle 
endale võib tunduda, et lõpuks kui sa mingeid asju hakkad tegema, et selline 
lahendus oli algusest peale selge. Aga kui sa lähed kontrollima fakte, et mida 
sa tegelikult rääkisid, mida sa oled ise kirjutanud, mis sa arvasid, siis 
selgub, et tegelikult see lõplik lahendus on sinu juurde väga suure kaarega 
tulnud. Sa pead selle lihtsalt välja kannatama ja selle ära tegema. Aga, aga 
point on lõpuks see, et kui sa oled jõudnud mingisuguse asjani, millest sa 
näed, et see ongi okei ja lahendab ära  selle probleemi ja selle probleemi ja 
selle probleemi. Võib-olla see lahendus on keeruline ja on kulukas nagu 
realiseerida ja on isegi riskantne aga ta on õige, ta on juba olemuslikult 
õige. Sa saad aru, et mis see probleem olemuslikult on, kuidas seda asja  
tükeldada, kuidas seda keerukust peita, kuidas seda asja üldistada. Ja siis sa 
pead väga kannatlikult väga paljudele inimestele seletama, miks me võiks teha 
just nii. Seda jõuga ei saa teha. Sa pead neid julgustama ja sa pead olema 
valmis nende eest viskuma džotile, juhul kui on vaja. Aga ma ise muidugi usun, 
et ei lähe vaja, või siis sealt džotist ei tule midagi surmavat välja, eks ole.
