\index[ppl]{Ansper, Arne}
\question{Kuidas sinu puhul arvutihuvi 
alguse sai?}

Asi sai alguse sellest, et kui ma lõpetasin põhikooli, siis 
matemaatikaõpetaja arvas, et peaksin minema Nõkku\index{Nõo Keskkool} edasi õppima. Ta suutis mu vanemaid veenda, et see on suurepärane 
mõte, ja ma läksingi.

\question{Kus sa põhikooli lõpetasid?}

Jõõpres, pisikeses ja väga mõnusas kohas Pärnu 
lähedal. Põhikool saja õpilasega asus mõisa keskel. Kuna mulle matemaatika 
sobis ja õpetaja oli väga usin, siis ta andis mulle lisaülesandeid 
ja lõpuks saatis olümpiaadile, kus mul läks ka suhteliselt hästi.


\question{Sa läksid siis puhtalt matemaatika, mitte arvutite põhjal 
Nõkku?}

Mul oli enne seda null kokkupuudet arvutiga. Pigem tahtsid vanemad, et ma läheksin sinna kooli. Ma ise olin väga kahtlev, kas minna kodust 
nii kaugele -- äkki seal on raske ja paha ja nii edasi. 

\question{Mis aastal see oli?}

1985. 

\question{Sel ajal oli ju logistiliselt keeruline Pärnu lähedalt Nõkku 
saada.}

See oli lihtne, aga tüütu -- buss sõitis neli tundi 
ja oli tavaliselt maast laeni rahvast täis. Lisaks tuli teise bussiga Pärnust koju saada, mis
käis kahe tunni tagant. Nii et see kõik kokku võttis aega.  

\question{Kas Nõos pandi sind kohe arvuti taha?}

Ei, Nõos oli tavaline keskkoolielu selle väikse vahega, et tuli ühikas 
elada. See poisteühikas oli 
suhteliselt raju ja legendaarne koht, ehitatud millalgi tsaariaja lõpus, Eesti 
aja alguses. Talvel oli toas null kraadi lähedal, seda tuli suurte 
märgade puunottidega kütta, et magada saaks. Hommikul lõin ikkagi pesukausi pealt jää katki, 
kui hakkasin hambaid pesema. Aga see esimene aasta oli hästi lahe. 

Alguses olid tavalised õppeained ja siis tuli programmeerimine 
regulaarse ainena juurde. See oli ikkagi matemaatika- ja 
füüsikakallakuga kool, programmeerimine oli lihtsalt üks
aine. Võrreldes teiste koolidega oli maht loomulikult suurem nii 
matemaatikas, füüsikas kui ka programmeerimises.

\question{Räägi palun Nõo kooli taustast -- kuidas sinna üldse sai?}

Ma ei teagi, kas seal oli konkurss või mõni muu süsteem. Mina olin tollal niisugune inimene, et läksin sinna koos emaga ja käisime direktori juures rääkimas. Kuna mul oli 
olümpiaadilt koht ette näidata, siis kuidagi ma 
sisse sain. Kuidas täpselt, ei tea. 

\question{Kes Nõo kooli direktor tol ajal oli? See kool paistis kellegi 
entusiasmist toimivat.}

Enn Liba\index[ppl]{Liba, Enn} oli minu meelest tol ajal direktor\sidenote{Nõo 
kooli arendas selliseks reaalteaduste ja programmeerimise õppe keskuseks, nagu 
me seda praegu teame, Kalju Aigro\index[ppl]{Aigro, Kalju}. Ta oli kooli 
direktor aastatel 1951--1982, pärast teda Enn Liba.}. Aga 
entusiasmi{\-}aspekti ja ajalugu ei oska ma
kommenteerida, huvitusin tollal muudest asjadest.

\question{Millest täpsemalt?}

Mulle meeldis põhikoolis elektroonika, aga see huvi oli 
platooniline, kuna juppe oli raske kätte saada. Ja mulle meeldisid ka
mudellennukid, mis oli samuti suhteliselt platooniline huvi. Nõos tuli 
programmeerimine hästi kiiresti peale. Seal oli suur 
Vene \emph{mainframe} Nairi-3-1\index{Nairi!Nairi-3-1}\sidenote{1964. aastal 
Jerevanis välja töötatud Nõukogude arvutiperekonna Nairi kõige võimekam liige. 
Kool sai selle arvuti 1977. aastal.}. 
Siis veel KÕPS\index{KÕPS} ja ROPS\index{ROPS}\sidenote{\phantomsection\label{sidenote:ROPS}KÕPS ja ROPS on 1980. 
aastate teisel poolel Nõo Keskkooli arvutuskeskuses välja töötatud eestikeelsed 
programmeerimiskeeled, mille loomisel osales ka Arne esimene arutiõpetaja Nõos 
Uuno Puus\index[ppl]{Puus, Uuno}. KÕPS oli sarnane MITis välja töötatud keelega 
LOGO, võimaldas vaid graafikat ja tugines LOGO looja Seymour Paperti ideele. 
ROPS oli KÕPSi edasiarendus, sarnanes Algoliga ja võimaldas lisaks 
graafikale arvutusi.}, kus sai eesti keeles 
programmeerida, need olid vahvad. Lisaks Agatid\index{Agat}, 
mis olid teistmoodi vahvad, kuna nendega sai ka 
valmistarkvara kasutada. Ühesõnaga, mängude mängimine ja  
ise asjade proovimine oli oluline.

\question{Oskad sa tagantjärele meenutada, mis sulle arvutite puhul täpsemalt meeldima 
hakkas?}

Eriti ei oska. Mul on umbes kaks asja meeles, mida ma Agatiga tegin. Esimeses 
programmis oli \verb|for|-tsükkel: sai muuta värvi, trükkida 
teksti, näiteks \enquote{tere}, kõigis keeltes, ja panna vilkuvat 
tausta. Sellega sai vähemalt üks õhtu, kui mitte kauem, möllatud --
efekte, tekste ja muid asju timmitud. Teiseks püüdsin 
üht Nintendo puldimängu\sidenote{\phantomsection\label{sisu:gameandwatch}Arne peab 
ilmselt silmas Nintendo Game \& Watch\index{Nintendo Game \& Watch} seeria käeshoitavaid mänge. 
Originaalidest rohkem oli liikvel nende Nõukogude kloone, mida müüdi 
Elektronika kaubamärgi all. Tegu polnud siiski alati täpsete koopiatega: 
Nintendo EG-26 kloonis IM-02 püüdis mune Miki Hiire asemel hunt tuntud 
Nõukogude multifilmist \enquote{\begin{russian}Ну, погоди!\end{russian}}.}) taasluua, ja 
lõingi. Seal oli fikseeritud arv positsioone, mingid tegelased 
liikusid ringi, keegi sai surma ja 
boonuseid sai ka. Probleem oli selles, et ma ei teadnud tollal, mis asi 
on massiiv. Mul oli iga objekti jaoks muutuja, 
mis ütles, kas objekt on või ei ole. Ja kui seal miski liikus, siis 
mul oli lehekülgede kaupa \verb|if|-lauseid, et kui see muutuja omab seda 
väärtust, siis järgmisel sammul omab see teist väärtust. 
Refaktoreerimistööriistu muidugi ei olnud. Kui ma kuskil vea tegin, nägin 
päevade kaupa vaeva, et muutujaid ja \verb|if|-lauseid 
ümber nimetada.


\question{Tol ajal tundus asjadest mitterääkimine olevat 
õpetamismetoodika osa. Meile näiteks ei räägitud \texttt{for}-tsüklist tükk 
aega.}

Agati\index{Agat} ei õpetanud meile keegi, õpetati 
KÕPSi ja ROPSi. Kõik, mis Agati peal sai tehtud, oli tänu enda välja 
võideldud arvutiajale ja entusiasmile. See käis vist \emph{by 
example}, vaatasin, mida keegi teine oli teinud. Ühtegi Agati või BASICu\index{BASIC} kohta 
käivat raamatut ma küll pole kunagi lugenud. See oli nagu folkloor, 
katsetamise ja kõlakate tasemel. Oleks keegi lekitanud infot, et 
massiivid on olemas, oleksin selle Nintendo mängu palju rutem valmis saanud. 

\question{See oli ju suur töö, sul pidi kange kihk sees olema!}

Aega oli palju, segavaid faktoreid vähe. Ilmselt mulle meeldis
arvuti alistamine. Agati\index{Agat} puhul
mäletan, et hankisin endale 
assembleri\index{Assembler} manuaali, mis oli 
paar-kolm ruudulist lehte, kuhu kirjutasin tähtsamad käsud, registrid 
ja värgid üles ja siis tudeerisin seda. 
Tuuseldasin Agati assembleri poolel ka kindlasti ringi, aga meeles on ainult olulisemad registrid, mida näppides käis piiks 
ja kust sai lugeda klaviatuurisümboleid.

\question{Kuidas Nõos tase oli, kas seal olid kõik sinusugused koos?}

Seal oli inimesi üle kogu vabariigi, kellel olid  
huvid ja eeldused reaalainetega tegelemiseks. Samas oli ka
lähikonna inimesi. Nii et mõnes mõttes geto ja mõnes mõttes
spetsialiseerunud kohalik kool.

\question{Kas seal kandis tehti kaheksakümnendate lõpul äri ka? Kirjutati raha eest 
programmi?}

Võib-olla keegi tegi, aga ma ise ei huvitunud ega teadnud sellest midagi. 

\question{Kas Tartu vahet käisite?}

Jaa. Ma ei mäleta, mitmendas klassis sain teada, et 
Tartu Ülikooli raamatukogus\index{Tartu Ülikool!Raamatukogu} on 
XTde arvutiklass. Seal sai aega reserveerida 
ja ma jõlkusin seal mitu õhtut nädalas. Seal 
sai tunni-kahese \emph{slot}'i, mul oli umbes kaks flopit, 
ühel Turbo C\index{Turbo C} ja teisel 
opsüsteemi asjad, ja midagi ma seal programmeerisin. 

\question{Kust sa said Turbo C?}

Ma ei kujuta ette, kust, aga sellega ma põhiliselt mängisingi.
Kas mul ka mingi eesmärk oli, seda ma ei mäleta.


\question{KÕPSist ja ROPSist Turbo C, mälu ja pointeriteni on paras hüpe. 
Mille pealt sul see hüpe tuli?}

Meile õpetati üheksandas klassis programmeerimist, aga mida täpsemalt ja mis edasi 
sai, seda ma ei mäleta mitte. Ilmselt liikusin ise
kiiremini edasi. Pärast TPIs\index{TPI|see{Tallinna Tehnikaülikool}} 
\index{TPI} esimestel kursustel ei saanud ma ka programmeerimise loengutes 
midagi uut teada. Midagi uut muidugi oli, aga mitte programmeerimise poolel. 


\question{Kuidas sa Nõost TPIsse\index{TPI} sattusid? Olnuks ju loogiline 
minna Tartusse matemaatikasse.}

Mõtlesin, et võib sinna või tänna minna. Need argumendid, miks Tallinnasse proovida, olid 
väga otsitud ja nõrgad, aga seda ma tegin. 

\question{Mida sa õppima läksid?}

LId\index{Tallinna Tehnikaülikool!Automaatikateaduskond!LI}. Ma täpselt ei mäleta, kas eriala nimetus oli arvutid ja 
arvutisüsteemid.

\question{See lühend LI jookseb mitmelt poolt läbi, aga keegi ei tundu teadvat, 
mida see tähendas.}

Kas see üldse midagi tähendas? \enquote{L} on tõenäoliselt mingi 
automaatikateaduskonna kood ja \enquote{I} mõne muu asja kood. 
Seal oli ka LA, mis võis olla automaatikaga seotud, ja ilmselt ka LS. Mina veetsin oma aega LIs.

\question{Sa ütlesid, et programmeerimise õpe sind eriti edasi ei aidanud. Kas 
seal üldse midagi õpetati, mis sulle midagi juurde andis?}

LIs räägiti sellest, 
kuidas transistorit teha, transistoritest 
mikrolülitusi luua, erinevat sorti 
registritest automaatikat ehitada ja registritest protsessorit teha, kui neid on 
hulgi käes. Teisalt saime siduteooriat. Tagasivaates andsid kõik need asjad selliseid teadmisi, et kui 
enda ümber ringi vaatan, siis maagilisi asju, mille kohta ma ei tea, 
kuidas neid teha, on väga vähe. See on kindlasti väga kasulik asi. Tänapäeval on kihte nii 
palju juurde tulnud, vanasti oli kõik väga lihtne -- nagu 
renessansiajastul, kui üks tüüp teadis kõike, mida oli mõtet teada. 
Kui mina TPIs käisin, hakkas see aeg läbi saama. Tänapäeval ilmselt ei ole võimalik teada arvutiasjandusest kõike 
kasulikku, eriti mis puudutab riistvara. Sinna on laotud kihte, virtualiseerimistasemeid ja mida iganes veel juurde. Ja siis on \emph{soft}'i poolel ka vastu 
tuldud, neli kihti virtualiseerimist vahele laotud ja nii edasi. Raske on järge pidada.

\question{Kas TPIsse minek oli asjade loomulik käik või oli sul kindel plaan, 
mida tegema hakata?}

Mul ei olnud pikaajalisi plaane. Mulle meeldis arvutitega mässata ja vahet polnud, kas teen seda Tallinnas või 
Tartus. Nii ma siis mässasingi Tallinnas. Üks huvitav nüanss 
on veel see, et millalgi keskkooli lõpus sain isikliku arvuti -- Atari 520 STf\index{Atari 520 STf}. See
oli Atari Motorola 68000 prosega. Mälu oli 512 kB, mille ma 
\emph{upgrade}'isin ühe megani. Selle peal ma siis elasin ja püüdsin minna süvitsi. 

\question{Kust sa sihukese aparaadi said kaheksakümnendate lõpus?}

Mul olid vanaonud, kes elasid Rootsis. Ema ja isa käisid ükskord seal ja nende
kaudu saingi. 

\question{See pidi Agati kõrval ikka ulmeline aparaat olema!}

Tegelikult teised inimesed, kellega ma läbi käisin, olid üheksakümnendate paiku PCde peal ja 
toimetasid nendega. Minul oli kodus Atari ja tegelesin 
põhiliselt sellega. 

\question{Ataril on vähem kihte, sellega sai lihtsamini sügavuti minna.}

Jaa, see oli täiesti hoomatav, midagi ulmelist 
polnud. Natuke mängisin ka, aga mitte liiga palju. Mulle meeldis ikka 
programmeerimine kõige rohkem ja ma tegin Atari peal igasuguseid imelikke asju.

Ma üritasin CAD joonistamisprogrammi teha ja lõpuks see
töötaski: seal sai teha ringe ja jooni, salvestada ja 
laadida. Kusjuures mulle tegi kohutavat muret see, et dünaamilist mäluhaldust tehes saab mälu otsa. Üritasin seda muret minimeerida. 
Näiteks mulle tundus, et lokaalsed muutujad \emph{stack}'is olid 
ebaefektiivsed. Sisuliselt oli see CAD programm kirjutatud 
sajaprotsendiliselt globaalsete muutujate otsa. Tagantjärele mõeldes oli see täiesti hullumeelsus: seal tuli kõvasti refaktoreerida, sest ma panin 
alguses täitsa puusse. Tegin väga palju vigu ja ümberkirjutamist oli nii palju, et 
sain selgeks, et nii ma enam kunagi ühtegi asja ei tee.

\question{Eks see on ju õppeprotsess, mõnda asja teoreetiliselt selgeks ei saa.}

Jah, absoluutselt nõus. Nii võimekaid inimesi, kes kogu aeg teiste 
vigadest õpivad, väga palju ei ole. Enamik õpib oma vigadest. 

\question{Kui sa TPIsse\index{TPI} läksid, kas kohtasid seal teisi omasuguseid ka?}

Meil oli hästi lahe kursus. Ja kuigi programmeerimise vallas ei 
olnud mul väga huvitav, siis muid ained õppisin korralikult. Ma olin väga usin õppur! Vist kolmandal kursusel, 1991. aastal kutsus 
Tarvi\index[ppl]{Martens, Tarvi} mind Ektacosse\index{Ektaco}. See oli 
\emph{community}, kus ma hakkasin inimestega koos olema ja kus 
ka töine karjäär alguse sai.

\question{Kolmas kursus on ju üsna hilja?}

Programmeerimise ja üldse arvutiasjanduse 
õppimine võtab ikkagi aega. Nüüd tagantjärele mõeldes, mida ma tookord 
üldse oskasin või kuivõrd hästi programmeerisin, ütleksin, et palju varem ei ole mõistlik seda tööd teha üritada. See võib  
frustratsiooni tekitada. Ma olin ligi viis aastat innustunult 
selle asjaga tegelenud ja tööle saades olin enam-vähem miinimumtasemel, kus on mõistlik, et keegi annab sulle 
ülesandeid, millel on ka mingi tähtsus ja tähendus, ja sa teed need ära.

\question{Kas sul ei olnud sedasi, et lihtsalt arvutiaja saamiseks tekkis arvutiklassi administraatori koht?}

Ei, mul ei olnud nii. Ma küll täpselt ei mäleta, millal 
Atari sain, aga kuidagi oli alati võimalus olemas ja sellest piisas. 

\question{Oskad tuua mõnda näidet, mida sa Ektacos alguses 
programmeerisid?}

Ektacos valmistati nii riistvara kui ka tarkvara: tehti 
tööstuskontrollereid, automatiseeriti tehaseid. Sardsüsteemides on väiksed mikroprotsessorid, neid oli vaja programmeerida, aga 
kuna programmaatorid olid kallid, hakkas Ektaco tegema oma 
programmaatorit. See oli lisaseade PC-le, millega sai
\enquote{kive kõrvetada}\sidenote{ROM mikrokiibile uut sisu kirjutada.}. Mina tegin tarkvara ja üks teine tüüp riistvara. Meil oli 
PC/AT platvorm ISA-siiniga ja me ajasime selle arvuti süstemaatiliselt kogu aeg 
täiesti lukku. Siis saime sellise ägeda aparaadi nagu
loogikaanalüsaatori pealt vaadata, mis juhtus ja mida me 
valesti tegime. Kui 
ostsilloskoobiga saab visualiseerida analoogsignaali, siis 
loogikaanalüsaatoril on palju pisikesi klemme, mille saab panna prose 
või digitaalsignaalide külge ja seega tekib põhimõtteliselt teine arvuti, mis  
visualiseerib signaalide mustreid. Saab ka panna 
\emph{trigger}'eid, et kui tekib muster, siis saab seda salvestada ja 
taasesitada. Ühesõnaga tema tegi riistvara ja kirjutas  
kontrolleri peale programmi, ja mina kirjutasin PC peale 
draiverite programmi, mis omavahel suhtlesid, ja tegin sellele ka 
kasutajaliidest.

Meil olid Inteli ja IBMi \emph{manual}'id, mida me 
sobrasime ja dekodeerisime, et aru saada, mida edasi teha. 


\question{See kõlab väga süsteemse ja korraldatud ettevõtmisena.}

Ei, see oli hull häkkimine. Ektacos seda kraami jätkus, mille abil häkkida. Ma selle 
teise tüübi tausta ei tea, võib-olla oli kogenum -- tema tuli 
loogikaanalüsaatoriga laua taha. Samas oli see 
kasulik ja kergesti omandatav seade. 

\question{Võrreldes sellega, kuidas inimesed kuuldavasti
vaibanoaga emaplaadi pealt radu maha kratsisid, et modem tööle saada, oli tegu 
\emph{high-tech} häkkimisega!}

Me tegime radu juurde selleks, et asi kuidagi tööle saada, me ei 
kratsinud midagi maha! Ma ise riistvara poolt tol ajal ei puutunud, ehkki 
Ektacos oli programmeerija töövahendite hulgas kindlasti ka tinutuskolb. 


\question{Kas saite kogu kupatuse lõpuks tööle?}

Jah, loomulikult. Kuna tollal oli standardiks Borlandi\index{Borland}\sidenote{Borland Software Corporation on 1983. 
aastal asutatud ja eri nimede all siiani toimetav tarkvaraettevõte, tuntud 
eelkõige arendajate töövahendite poolest. Neist kuulsaimad on 
Turbo-eesliitega keeled assembler, BASIC, C, C++, Pascal ja hiljem ka Delphi.}
toodang, siis loomulikult sai kirjutatud oma akendussüsteem, mis nägi välja nagu 
Borlandi Turbo Vision\index{Turbo Vision}\sidenote[][2cm]{Borlandi poolt 
1990ndate alul arendatud tekstipõhine kasutajaliidese raamistik Pascali 
ja C++ jaoks.}, aga oli hoopis parem ning töötas väga 
kenasti. 

\question{Milles see paremus väljendus?}

See oli ägedamalt struktureeritud. Mulle hakkas siis 
C++\index{C++} meeldima, kuna see oli hästi objektorienteeritud. Sel olid 
oma kontseptsioonid, kuidas aknaid ja asju esitada, klaviatuuri ja hiirega 
sündmusi käsitleda. Oluline on, kuidas sündmused jõuavad õige objektini, see loogika on klaviatuuri 
ja hiire puhul väga erinev. Kõik see oli selliseks loogiliseks kompotiks keeratud, mille peale oli lihtne rakendusi teha. Oligi umbes üks programm, mis seda ägedat kompotti kasutas, seesama programmaatori kasutajaliides. Ektacos oli tore see, et 
tööülesanded ei olnud väga piiravad. Võisid kuude või isegi aastate kaupa rahulikult häkkida ja lõpuks tuli mingi asi 
välja. 

\question{Samas ei olnud sul ilmselt neid akende joonistamise asju 
riiulist kümnete kaupa võtta?}

Oli ikka. Sedasama Turbo Visionit oleks võinud pruukida, seal oli 
igasuguseid teeke. Aga ametiuhkus ei lubanud 
teise mehe aknateeki kasutada, nii et tuli ikka enda oma teha. 

\question{Kas tänapäeval pannakse sellist suhtumist pahaks? Või ei panda?}

Seda tehakse teisel tasemel. Tasemeid on juurde tulnud, 
nokitsetakse hoopis muude asjade kallal. Minu arvates on see
paratamatu ja hädavajalik, et ei peaks uuesti
jalgratast leiutama. Tänapäeval tehakse asju, mis on juba tehtud, aga teistmoodi, 
paremini. Põhimõtteliselt olid ju opsüsteemid olemas, näiteks PC-Unix, nii et mis mõtet oli 
Linuxit hakata tegema?! 


\question{Jah, põhimõtteliselt võiksid ju kõik siiamaani 
kasutada sinu aknategijat.}

Need inimesed, kes on armunud 80 x 25 tekstiekraani, oleksid kindlasti siiamaani selle andunud kasutajad. 

\question{FoxPro\index{FoxPro} joonistas lausa varje 
akende taha.}

Jaa, loomulikult pidid akendel varjud olema!

\question{Kas teie kiibikõrvetajat kasutati ka väljaspool 
Ektacot\index{Ektaco}?}

Asjaolud muutusid nii kiiresti, et see, mis oli kaks aastat varem kallis ja kättesaamatu, enam hiljem ei olnud. Seda tehti vist üks-kaks eksemplari ja pruugiti Ektacos, aga edulugu sellest ei tulnud. Ja ega see olnudki põhitegevus. 
Ma ei teagi, miks seda üldse tegema hakati -- kas see oli tõesti 
nii kättesaamatu või oli lihtsalt äge teha.

\question{Kui ma sind kuulan, siis see ei kõla jah suurepärase ärina.}

Ektaco tegi ju äri ka, aga mind huvitas tollal 
programmeerimine. Teadsin küll, mida kolleegid tegid, kuid ma 
ei süvenenud sellesse.

\question{Kas tol ajal tekkis sul juba ka kokkupuude arvutisidega?}

Ektacos oli terve hulk toredaid kolleege: 
Tarvi Martens\index[ppl]{Martens, Tarvi}, Heiki Kask\index[ppl]{Kask, Heiki}, Jaak 
Niit\index[ppl]{Niit, Jaak} ja Gunnar Valge\index[ppl]{Valge, Gunnar} olid 
minuga samas toas, lisaks veel paar-kolm inimest. Meil oli 
Fido \emph{point}, mis tekkis ennekõike Tarvi ja Heiki 
initsiatiivil. Alguses aga olime Lõvi \emph{point}'i küljes 
(Lõvi\index[ppl]{Lõvi|see{Lepp, Andres}}\sidenote{Lõvi, pärisnimega Andres 
Lepp\index[ppl]{Lepp, Andres}, on legendaarne TPI arvutimees, paljude selle
põlvkonna IT-inimeste sõber, teejuht ja eeskuju.} oli siis TPI 
Arvutuskeskuses\index{Tallinna Tehnikaülikool!Arvutuskeskus} ja minu silmis täielik
kunn). Jooksutasime seal FrontDoori\index{FrontDoor}\sidenote{FrontDoor 
oli üks populaarsemaid FidoNeti mailereid.} ja mida kõike veel. 

Ühel hetkel \emph{upgrade}'isime ennast \emph{point}'i staatusest 
\emph{node}'iks. Meie number oli vist 71 ja helistasime ilmselt
kuhugi sisse ka, sest ma mäletan, et olen mingi \emph{prompt}'i otsas 
rippunud. Vaat seda küll ei mäleta, kust ma sain teada, milliste käskudega 
Unixis midagi teha ja kuidas binaarne fail ära 
\emph{uuencode}'ida, et seda üle terminali endale 
\emph{dump}'ida, \emph{dump}'i salvestada, oma masinast \emph{decode}'ida ja 
zipi sealt seest kätte saada. Kuidagi ma siiski teadsin ja neid teadmisi endasse
imesin. Need olid 
õhus laiali nagu hallitusseene eosed laiali -- nii kui pinnase 
leidsid, läksid kohe kasvama. 

\question{Nii mitu sammu selleks, et midagi kätte saada\ldots{ }Barjäärid olid toona väga kõrged.}

Info ikka liikus, see ei olnud probleem. Küsimus oli 
ennekõike riistvaras, \emph{access}'is, telefoniliinides ja muus säärase. Modemid olid roppkallid, samuti arvutid -- kõik oli roppkallis, 
välja arvatud aeg. Töö juures mingid modemid õnneks olid, mitte küll 
kõige härjemad. Meil oli üks 2400 ja MNP5\sidenote{\emph{Microcom Network 
Protocols} (MNP) on perekond (tähistatud numbritega ühest kümneni) 
veaparandusprotokolle, mida sageli kasutati varastes kiiretes (2400 bit/s ja 
rohkem) modemites.} oli see lagi, millega me alguses toimetasime. 
Kõik olulised asjad liikusid flopide peal, suuri asju ei viitsinud keegi 
ära tõmmata, tõmmati pisikesi nublakaid. Tollal oli flopiga bussi peale 
minek reaalselt kiirem kui modemiga toimetamine.

\question{Mis sorti materjali te oma \emph{node}'is hoidsite?}

Point oli meil puhas Fido point, BBSi ega midagi säärast meil polnud. 
Oli sõnumivahetus, Echomail ja Netmail ehk 
privaatkirjad ja avalikud foorumid. Selleks me 
suures pildis \emph{node}'i pidasimegi. Kui keegi midagi tõmbas, siis enda jaoks ja võib-olla jagas kolleegidega ka, aga n-ö varamut ei olnud.

\question{Kellega te meile vahetasite ja mis uudisgruppe lugesite? Kogukond 
ei olnud ju suur? Lõviga sai ju niisama ka juttu rääkida, ei pidanud kirja 
saatma.}

Mina lugesin põhiliselt Echomaili, mul kirjasõpru väga ei olnud. Minu 
jaoks oli see foorum, kust sai huvitavat ja enamasti ka väga 
humoorikat sisu. See, kuidas inimesed viitsisid ükskõik mis teemal 
oma mõtteid sõnastada -- see iroonia, sarkasm ja huumor -- oli niivõrd hea, et lust oli 
lugeda. Kasvõi autofoorumeid (mul polnud autodest sooja ega külma). Lihtsalt need naljad ja vihjed olid hea 
meelelahutus. Muidugi räägiti ka programmeerimisest, riistvarast ja 
muudel IT-teemadel.

\question{Nii et alates Tolkienist ja autodest kuni C++ni?}

Absoluutselt kõigest, kogu elu oli seal. Ja seda jaksas tervikuna 
läbi lugeda, sest inimesi oli vähe ja palju nad ikka head kvaliteetset sisu 
jõudsid toota.

\question{Ühesõnaga praeguses mõistes oli võimalik hoida kogu sisuloomel silm peal?}

Sellel jah, mis Fido Echomaili kaudu tuli. Paralleelselt 
hakkasid arenema ka \emph{newsgroup}'id, sh Eesti omad, millega mina 
alguses ei puutunud eriti kokku. See oli natukene teistsugune seltskond, kes hakkas n-ö internetimaailmas toimetama. 

\question{Need olid kaks eri maailma, nende vahel silda ei olnud?}

Nii ja naa, kontseptsiooni mõttes olid interneti uudisgrupid ja Fido omad 
samad, aga samas esines mingeid ebamugavaid erisusi. Kunagi hiljem, kui läksin
Ektacost Küberneetika Instituuti\index{Küber|see{Küberneetika 
Instituut}}\index{Küberneetika Instituut|see{Cybernetica}}, siis tegin oma 
\emph{node}'i Solarise\index{Solaris} peale. Meil oli seal üks 
SPARCi\index{SPARC}\sidenote{\emph{Scalable Processor Architecture} 
(SPARC) on Sun Microsystemsi arendatud RISC-arhitektuur. Sun müüs 
sellele arhitektuurile tuginevaid, siinmail populaarseid servereid ja 
tööjaamu.} server, mille peal ma ajasin käima kogu Fido softi. 
Ja siis tegin \emph{news}'i \emph{gateway}, 
mis nagu Fidogi köitis Echomaili kahesuunaliselt \emph{newsgroup}'ideks ja 
ühtlasi Netmaili tavaliseks meiliks. See oli pikka aega päris popp, kuni ma lõpuks enam ei viitsinud sellega tegeleda, kui Solarisele oli vaja korralik \emph{upgdade} teha. Nii et minu \emph{news}'i 
serveri küljes oli sadu kliente oma personaalse \emph{account}'iga, kellel oli kirjutamisõigus Fido gruppides. 
Fidos oli korrapidamine olulisem -- anonüümset 
kasutust ei olnud, keegi vastutas alati kellegi eest. Kui keegi sai kuskilt
\emph{access}'i ja kui see keegi oli nõme, siis talt võeti \emph{access} 
ära. Kui ma hakkasin seda asja \emph{news}'i \emph{gate}'ima, siis lubasin sedasama 
teha. Ma ei andnud kellelegi Fido gruppidesse kirjutamise õigust, kui ma 
ei teadnud, kes ta on, ega oleks saanud talt vajadusel \emph{access}'i ära võtta. 

\question{Kõlab päris autoritaarselt!}

See toimis, teistmoodi ei saanud. See oli endale olulise keskkonna normaalsena hoidmise 
eeldus. 

\question{Mis oli \enquote{nõme}?}

Teiste inimeste solvamine ja trollimine. Ei ole vaja minna isiklikuks, teisele haiget teha ega halba emotsiooni tekitada. Vaidlemine on okei, seda peab olema. Aga teistele ei tohi haiget teha. 

\question{Kõlab nagu lihtne, eluterve ja samas fundamentaalne definitsioon! Kas sa siis veel õppisid, kui Ektacost ära tulid?}

Ei, selleks hetkeks oli mu õppimine õpitud, inseneridiplomi sain kätte vist 1993. või 1992. aastal. Magistrikraadiks \emph{upgrade}'isin selle natuke 
hiljem, 2001. aastal. Mina õppisin süsteemiinseneriks viis aastat, hiljem 
hakati kraadiõpet järjest lahjendama.  

\question{Kas sul ei tekkinud tol hetkel tunnet, nutikas ja usin õppur 
nagu sa olid, et peaks teadusmaailma sukelduma?}

Ma ei puutunud TPIs teadusmaailmaga eriti kokku. Kuna ma hakkasin õpingute keskel programmeerijana tööd tegema, siis see haaras mind  
enam-vähem täielikult. Lõpus läks õppimine 
nigelamaks, sest töö oli palju huvitavam ja pakkus rohkem väljakutseid. Viimane asi, mille ma Ektacos\index{Ektaco} tehtud sain, oli 
kontrollerite uue sideprotokolli disainimine. Ma olin väga vaimustatud 
TCP/IPst ja trükkisin välja kõik standardid, mille kätte sain: TCP, IP, 
Etherneti. Aga kontrollerid olid 8051 peal, mis on, ütleme, väga 
väike. Lugesin kõik RFCd läbi ja tegin oma 
sideprotokolli, inspireerituna nii Ethernetist, IPst kui ka TCPst. See küll
ei olnud päris \emph{flow}'le orienteeritud, pigem 
\emph{datagram}'i-põhine protokoll. Vanad riistvaraässad Ektacos olid 
väga nördinud ja solvunud, et mis mõttes ma kirjutan protokolli, mis ei ole 
deterministlik! Mitte \emph{master-slave}, vaid igaüks võib traadi peal 
lobiseda, kui mõte pähe tuleb, ja siis lahendatakse konfliktid ära ja tehakse 
retransmissioon. Nad olid minu katsetuste peale väga pahased, aga 
programmeerisin selle lõpuks ära ja mingil määral see töötas ka. See oli 
päris äge.

\question{Mille vahel see protokoll käis?}

PC juhtis tööstusarvuteid. Neil oli sihuke karp, mille nimi oli satelliit (põhimõtteliselt tööstuskontroller) ja millel 
olid digi- ja analoogsisendid-väljundid, mis keerasid tehases 
nuppu, et näiteks betooni teha. Ja siis olid 
juhtprogrammid, mida tuli konfida. Tüüpiline värk: sa pead teadma, mis 
sul tehases toimub. Sa pead käske andma ja selleks on võrku vaja. Korralikus tehases võis
satelliidikontrollereid palju olla, need tuli PCsse kokku tõmmata. Ilmselt keegi kirjutas PC poolele softi, mis satelliite jälgis ja juhtis.

\question{Kas see kupatus oli päriselt tootmises ja Eesti Vabariigis 
tehti betooni niisuguste seadmetega?}

Jaa. Palivere Ehitusmaterjalide Tehas\index{Palivere 
Ehitusmaterjalide Tehas} oli Ektaco poolt ära automatiseeritud ja midagi oli vist ka Tallinna 
Veepuhastusjaamas\index{Tallinna Veepuhastusjaam}. Neid objekte ikka oli.

Varem töötasid need objektid mingi muu protokolli ja tehnika peal, aga see kõik kasvas: algatati uue generatsiooni satelliidi väljatöötamise projekt, 
kus mina panustasin protokolli tegemisse ja realiseerimisse. 

\question{Kui sa võrgundusest juba nii palju teadsid, kas sind ei tõmmatud kuhugi 
varasesse internetimaailma, näiteks kaableid vedama?}

Ei, mulle meeldis programmeerida. Muude asjadega tegelesin nii palju, 
kui vaja, et saaks midagi ägedat 
programmeerida. 


\question{Kas Küberis\index{Küber} sai ägedamalt programmeerida?}

Lõpuks jah. Mind kutsus sinna jälle Tarvi\index[ppl]{Martens, Tarvi}. 
Küberneetikasse oli tehtud infotehnoloogia osakond, mis peitis seda infot, et 
tegelikult tegeldi seal infoturbega. Seal oli üks riiklik programm, 
mille eesmärk oli Eesti riigi infoturbe ja krüptograafia vajadusi rahuldada. 

\question{Kas see oli juba enne, kui tekkis AS Cybernetica?}

Jaa, see oli enne seda. Mina läksin sinna 1994. aastal, aga töögrupp tehti vist 1993. aastal. Seal oli koos terve hulk nutikaid inimesi, kes 
viisid ellu missiooni ehitada kompetentsikeskus.

\question{Kes selle taga oli? Keegi pidi ju formuleerima tellimuse tegeleda
riiklikult krüpto ja infoturbega.}
Ülo Jaaksoo\index[ppl]{Jaaksoo, Ülo} oli siis Küberneetika 
Instituudi\index{Küberneetika Instituut} direktor. Tema seda kõike lõi ja korraldas. Kuidas ja kellega ta läbi rääkis või kust 
see mandaat tuli, seda ma ei oska öelda. Igatahes visioon oli temal. 

\question{Arvestades kui vähe vajas Eesti riik krüptot ja infoturvet siis ja 
kui strateegiliselt oluline teema see praegu on, siis sellise visiooni jaoks on 
ju tarvis väga ägedat ettenägemisvõimet.}

Kaugemale vaatamine ongi teadlaste ja akadeemikute ülesanne. Kust mujalt 
see tulla saab? 

\question{Visioon visiooniks, aga mida see töö toona praktiliselt tähendas?}

Esiteks tuli ise õppida. Teiseks teisi õpetada -- eestikeelne terminoloogia, 
standardid, profiilid, seminarid, koolitused. Tollal oli maailm 
väiksem, sealhulgas krüpto- ja infoturbemaailm, nii et oli
ikkagi veel võimalik hoomata kõike olulist. Mitte päris üksi,
aga meie väikese töögrupi sees küll. Meil läks väga 
hästi selles mõttes, et meie inimesed ei huvitunud niivõrd tehnikast, vaid just  
infoturbe süsteemsest poolest: organisatsioonist, reeglitest, 
seadusandlusest. Ühesõnaga meie grupist kasvas välja süsteemne lähenemine valdkonnale kui 
tervikule. 

Samas oli meil \emph{hardcore} häkkereid ja krüptograafe, kes olid valmis mida iganes tegema. See 
sümbioos oli hästi lahe. Oma esimese tööna 
Küberneetika Instituudis pidingi riigiasutustele kirjutama 
juhendi, kuidas KA9Q\index{KA9Q} otsas ehitada internetti ruuter. 

\question{Mille otsas?}

KA9Q on üks soft. \enquote{KA9Q} ise on üks radistide kutsung, millele anti nimi
selle softi kirjutaja\sidenote{Populaarne varane TCP/IP implementatsioon mitte-Unixitele, mille autoriks oli Phil Karn.} järgi. See oli DOSi peal 
jooksev \emph{all singing all dancing} asi, mis realiseeris TCP, kõikvõimalikud 
sideprotokollid, võrgukaartide toed, SLIP, PPP, ruuterid, mida iganes. FTP 
deemonid\sidenote{Deemon on programm, mis \enquote{kummitab} arvuti mälus iseseisvalt, ilma kasutaja sisendita, ning teeb midagi kasulikku: näiteks kuulab sisse tulevaid võrguühendusi.}. Maailmas on täiesti müstilisi asju tehtud. Ühe  
üleliigse PC, modemi, võrgukaardi ja selle softi abil sain teha ruuteri, 
millega oma organisatsioon kuhugi ära ühendada. Lisaks kirjutasin eestikeelse lühijuhendi, kuidas seda asja pruukida, hooldada ja 
käimas hoida. See oligi mu esimene töö Küberis. Pärast tulid igasugused muud asjad.

Me osalesime ühes varajases europrojektis, see võis olla 1995. 
aastal. Ma käisin Saksamaal Darmstadtis. Sakslased olid 
teinud sellise tarkvara nagu secu-d, mis oli väga halvasti kirjutatud, kuigi
realiseeris kogu krüpto, mis tolleks hetkeks oli teada. Üritasime seda secu-di kuidagi käima ajada. Ütleme niimoodi, et toona oli selline \emph{cross-platform} tarkvaraarendus, kus sa kompileerid Unixi, PC ja hiljem Windowsi jaoks, päris keeruline. Teha nii, et see kõik enam-vähem töötab ja piisavalt vähe mälu lekib ja piisavalt harva sama mälublokki kaks korda vabastab, oli raske ülesanne. Ma ehitasin selle secu-d najal näiteks turvalist meiliklienti ja sertifitseerimiskeskust. 
Sertifitseerimiskeskused olid lahedad, me programmeerisime Küberneetika Instituudis igal aastal vähemalt ühe 
sertifitseerimiskeskuse softi.

\question{Miks?}

See oli \emph{blend} praktilistest vajadustest ja 
teadustöö eesmärkidest. Tollal ei olnud ju kiipkaarte ja 
riistvaralisi turvamooduleid saada. Oht, et 
sertifitseerimiskeskuse võti saab kompromiteeritud ja keegi annab võltssertifikaate välja või 
operaatorile altkäemaksu, et too väljastaks võltssertifikaate, oli suur. Vaja oli mitme silma printsiipi ja topeltkaitset. Meie võtsime RSA võtme tükkideks. Praegu teevad sedasama SplitKey\index{SplitKey} ja SmartID\index{SmartID}. 
Meil ei olnud küll turvalist mitmes osas võtme genereerimist, jagasime lihtsalt
RSA võtme osakuteks ja kasutasime m-n-ist skeemi, kus  
sertifikaadi väljaandmiseks pidid viiest operaatorist kolm 
allkirja andma, ja me kombineerisime selle põhjal korrektse sertifikaadi. Selle 
n-ö initsialiseerimisprotsessi käigus tekitati viis flopit, millega operaatorid oleksid pidanud ringi käima. Igatahes süsteem 
töötas, tegi täitsa korrektseid X.509 sertifikaate ja oli varustatud kasutusjuhendiga.  
 
\question{Nii et kui enne tegelesid ISA siini peal väga madala 
taseme asjade katsetamise ja läbimängimisega, siis nüüd tegid sedasama 
krüpto jaoks põhiolemuselt sarnaseid tegevusi ja protsesse läbi viies?\nopagebreak[100000]} 

Võib öelda küll, et teatud mõttes tegelesime väga
algsete asjadega, kuid me jõudsime ka rakendusteni välja. Meil oli küll 
väga praktilisi asju, aga kontrollisime kõike 
ülevalt alla välja. Muu hulgas tegime tulemüüri, mis müüs Eesti turul väga 
hästi. Selle nimi oli Barrikaad\index{Barrikaad}, mul on 
siiamaani Barrikaadi T-särk alles. Ja ühe ägeda VPNi tegime ka. Selle teist versiooni kasutati Eesti riigiasutustes 
väga pikalt ka peale seda, kui selle tugi ametlikult kahjuks
lõppes. Meie VPNi põhieelised olid 
turvalisus, keskne hallatavus ja töökindlus. Meil oli see funktsioon, et kui 
on harukontorid, kust ei taha üldse interneti väljapääsu, vaid on soov neid läbi 
keskse tulemüüri (mis oli kallis) välja juhtida, sinna sisse 
ehitatud. Erinevad paralleelsed ruutingud üle erinevate kanalite tekitavad probleeme. VPN tunnelit, sise- ja 
välisaadresse on keeruline majandada niimoodi, et ruutinguinfo ka 
sisevõrgus korrektselt leviks ja töötaks. Tähtis on, et kasutajad 
ei peaks ootama, kuni nende seanss katkisest kanalist tervesse kolib. 
Kõrgete turvanõuete jaoks oli meil haldussüsteemis tekitatud eraldi 
võimalus süsteemi konfiguratsiooni muuta, mis ei olnud võrgus, vaid 
suhtles muu maailmaga flopide kaudu. Pidevalt võrgus olev osa ainult monitooris, kogus infot ja 
täitis võrgust väljas oleva osa käske. Muu hulgas realiseerisime ise šifreid, kuna tollal oli PC krüpteerimisvõime nõrk. Tollal tulid prosedele just MMXi laiendused välja, 
mis võimaldasid näiteks IDEAt\sidenote{\emph{International Data Encryption 
Algorithm} (IDEA) on esmakordselt 1991. aastal kirjeldatud sümmeetriliste 
võtmetega plokk{\v s}iffer.} paralleelselt arvutada, mitu plokki korraga. Helger Lipmaa\index[ppl]{Lipmaa, Helger} töötas siis Küberis ja 
programmeeris Linuxi tuuma jaoks MMXi \emph{extension}'eid kasutava 
AESi\sidenote{\emph{Advanced Encryption Standard} (AES) on Belgia 
krüptograafide välja töötatud Rijndaeli plokk{\v s}ifri alamhulk. 1997. aastal 
teatas NIST (National Institute of Standards and Technology of the United 
States) plaanist asendada avaliku protsessi abil tolleks ajaks 
ohtlikult nõrgenenud DESi algoritm. Vincent Rijmen ja Joan Daemen esitasid oma 
ettepaneku valikuprotsessi ja NIST standardiseeris selle 2001. aastal.}  
realisatsiooni. Meil olid Linuxi\index{Linux} tuumas oma draiverid, mis 
VPNi haldasid ja millel olid võtmete vahetuseks, konfiguratsiooni 
levitamiseks ja kõigeks muuks oma deemonid, hierarhiliselt üles välja.

\question{See ei kõla enam nagu programmeerimine, vaid
hoopis arhitektitöö. Kas sa liikusid programmeerija rollist arhitekti 
rolli või mõtlesite neid asju kambakesi välja?}

Välja mõtlesin kogu aeg, ellu viis enamasti teine tüüp ja kui vaja, tulid teised appi. Meil ei olnud selgelt defineeritud rolle, eriti alguses, nagu näiteks arhitekt või projektijuht -- projektijuht oli üldse väga haruldane 
nähtus, me ei teadnud isegi, mis projekt on, vaid lihtsalt programmeerisime. Oli terve hulk inimesi, kes arutasid 
intensiivselt kõigil teemadel. Kui asi oli selgeks räägitud, siis 
igaüks läks sellega oma valdkonnas süvitsi.

\question{Nutikatel inimestel on vahel ka oma nutikusele vastav ego. Kas keegi nina 
püsti ei ajanud ja ennast arhitektiks ei kuulutanud?}

Ei, päris nii ei olnud. Tagantjärele kardan, et võib-olla mina ise 
kippusingi see tüüp olema, kes oma arvamust teistele peale surus. Tol 
ajal aga ei tajunud ma seda kindlasti niimoodi. 

\question{Ega vist teisedki ja soft ju lõpuks 
töötas?}

Absoluutselt. Nii tulemüür kui ka VPN olid meil lõpuks 
ääretult stabiilsed ja kvaliteetsed. 

\question{Kas Privador kasvas ka sealt välja?}
Jah, Privador\index{Privador} oli siis Küberneetika Aktsiaseltsi \emph{spin-off} 
firma, mis sai need nii-öelda infoturbetooted eesmärgiga need laia 
maailma viia, aga see kahjuks ei õnnestunud. Seal oli ports 
probleeme, näiteks hakkasid tollal tekkima VPNi
standardid. Ja IPSec oli 
enam-vähem ära standardiseeritud, samuti IKE\sidenote{IPSec ja IKE on turvalist arvuti{\-}sidet võimaldavad, praeguseks standardsed ja laialt kasutusel olevad, protokollid.}, ja see oli 
tegelikult see, mida oleks tahetud osta. \emph{Vendor lockin}'i kardeti juba päris 
mõõdukalt kuni palju. Ja ehkki meie olime oma asja ehitanud, eriti 
alumised kihid, standardite põhjal, siis kogu mudeli või tervikvõrgu puhul tekkisid konfliktid IKE või IPSeci ideoloogiaga. Meil oli töölaual VPNi kolmas versioon, mis oleks 
olnud standarditega täielikult ühilduv ja oleks loodetavasti selle firmapärasuse 
probleemi kõrvaldanud, aga see kahjuks ei realiseerunud. 
Selle asemel tegime digiallkirja tarkvara, ajatembelduse tarkvara, 
notariseerimistarkvara ja kõike muud. Me ennustasime pisut valesti, 
mis on krüptomaailma \emph{killer}-rakendus järgmise kümne aasta jooksul. 
Olime selles mõttes natuke ajast ees.

\question{See lähenemine, et võtame alumise kihi standardid ja paneme 
täitsa uutmoodi ülemise kihi standarditeks kokku, on ju sama, mis 
sai tehtud digiallkirja konteineriga ja ka X-teega.}

Absoluutselt. Aga seal on see aspekt, et standardid on ja peavadki olema 
tegelikult geneerilised. Need on elujõulised siis, kui lahendavad 
paljude inimeste erinevaid probleeme. Kui aga võtta üks konkreetne riigiasutus, kellel on konkreetsed vajadused, mis tuleb 
lahendada efektiivselt, siis ei pääse lihtsalt 
standarditele vastava osa võtmisega. See ei ole efektiivne. Ent
just seda meie tegime. Me 
võib-olla ei tajunud maailma suurust, võimsust ja liikumiskiirust. 
Me mõtlesime, et teeme rajult 
ägeda asja, mis on parem ja praktilisem väga suure 
hulga klientide jaoks. Aga seda teadmist, et üks asi on hea ja praktiline,  
on väga raske efektiivselt ja kiiresti ühest peast teise viia.  

\question{Arvestades et samast pundist tulid ka X-tee\index{X-tee} ja 
ID-kaardi kontseptsioon, siis kahest kolm ei ole üldse paha eduprotsent.}

X-tee omab meie VPNi tootes väga selgeid 
juuri. Tegime X-teed 2001. aasta mais-juunis ja detsembris läks tootesse. 
See oli võimalik ainult tänu sellele, et võtsime oma VPN-toodet kui 
substraati. Meil oli kõik see teadmine olemas, kuidas teha üks Linuxi purk 
turvaliseks ja sellele oma tarkvara 
\emph{patch}'id peale panna, kuidas Linuxit efektiivselt konfigureerida ja jagada. Ühesõnaga, me ehitasime VPNi põhjal ühe natuke teistsuguse 
protokollivahenduse osa. 

\question{Kui ma sinuga esimest korda 
kliendina kohtusin, siis sa juhtisid juba vägesid. Mina rääkisin oma 
mure ära ja sina tegid nii, et sünniks lahendus. Kuidas sul 
juhiroll esile kerkis ja kas sa üldse mõtestad seda tegevust niimoodi?}

See tekkis Barrikaadi\index{Barrikaad} või VPNi või Privadori\index{Privador} 
programmeerimise käigus, kui meeskond kasvas. Eriti VPNi puhul oli
koordineerimine, kes mida programmeerib ja mis ajaks, minu peal. Ja kes see muu asju ikka käiku laskmas käis kui me 
ise, nii et tuli ka klientidega suhelda. \emph{Helpdesk}, 
projektijuht, arhitekt, programmeerija, testija, tarneinsener -- minu töö hõlmas kõike seda. 

\question{Ometi jäi ka koordineeriv roll just sinu peale?}

Ju see sobis meie pundist mulle kõige paremini. Keegi pidi seda tegema. Kui mina, siis mina, nii see läks.

\question{Küsin selle mõttega, kas sul oli kihk inimesi 
juhtida?}

Ei. Pigem oli sedapidi, et üksinda ei jõua kõike ära progeda. Mina nägin seda, mida 
tahtsime teha, päris detailselt ja erinevatest aspektidest. Et seda saavutada, olin sunnitud teistele ülesandeid ja 
eesmärke püstitama.

\question{See on jällegi arhitekti vaatenurk: peas on olemas täiuslik 
mudel süsteemist ja see peab teostuma. Mida sa 
praegu teed?}

Erinevaid asju: arhitektuurset mõtlemist, ideede jagamist, nõustamist ja mõnes mõttes ka koolitamist. Suhtlen palju 
klientide ja potentsiaalsete klientidega, et saada aru, mis on nende mured-vajadused ja kuidas me saame neid aidata. Nii et alates müügitööst kuni projekti 
juhtimiseni. Teisest küljest on see ikkagi arhitektuurne töö. Kui probleem on  
arusaadav, tuleb välja mõelda lahendus. Sealjuures on probleemid muutunud keerulisemaks ja 
mastaapsemaks, mõnes mõttes ka vastutusrikkamaks -- me  
tegutseme ju suuresti turvavaldkonnas. See ala on nii palju 
vaenulikum ja keerulisem ning panused niivõrd palju suuremad, et nüüd tuleb
lihtsalt märksa paremaid asju teha, kui me kunagi tegime.

\question{Sa oled kogenud arhitekt ja tead, mida on vaja projekti teostumiseks. Kuidas sa viid entusiastlikult pihta hakanud meeskonnale kohale 
selle, et sinu arvates projekt ei saa õnnestuda? Ja seda nii, et sind pärast 
tuppa tagasi ka lastakse?}

Kõigepealt peab probleemist aru saama. See võtab päris kaua aega ja sel ajal on suhtlust tihti väga vähe. Ei ole nii, et vaatad peale ja 
saad kohe aru, mis valesti on ja kuidas peaks olema. Probleem võib olla olemuslikult keeruline: tuleb mõista eelnevaid põhjusi, otsusi ja lahendusi. 
Ja siis tõenäoliselt marineerid nende 
otsas päris kaua, mitte et homme hommikuks on hopsti valmis. Tuleb mõelda, rääkida, kirjutada. Ja ehkki tihtipeale võib endale tunduda, et 
lahendus oli algusest peale selge, siis fakte kontrollides 
selgub, et tegelikult on lõplik lahendus tulnud väga suure kaarega. Sa pead selle lihtsalt välja kannatama ja ära tegema. Võib ka olla, et lahendus on keeruline, kulukas 
realiseerida ja isegi riskantne, aga see on õige. Saad aru probleemi olemusest, kuidas seda 
tükeldada, kuidas keerukust peita ja asja üldistada. Seejärel 
pead väga kannatlikult paljudele inimestele seletama, miks me võiksime teha 
just nii. Jõuga seda teha ei saa. Neid tuleb julgustada ja peab olema 
valmis ise nende eest dzotile viskuma, kui vaja. Ma ise muidugi usun, 
et ei lähe vaja või et sealt dzotist ei tule vähemalt midagi surmavat välja.
