Tere. See siin on memcpy. Nende sõnadega olen sisse juhatanud suurt hulka 
intervjuusid oluliste inimestega ja nüüd on teie ees see tekst. 

Aga miks? 

Põhjus, tuleb tunnistada, on lihtne. Nagu ütleb Villu Tamme loos 
\enquote{Paneme punki}:

\begin{verse}
Tahan kord saada selliseks, nagu \\
on Villu või Freddy või Rott või Striit\\
\end{verse}

See raamat räägib inimestest, kes on mulle oma tarkuse, oskuste ja olemusega 
eeskujuks olnud. Kui Toivo Annus\index[ppl]{Annus, Toivo} mu kunagi Skype'i 
tööintervjuule kutsus, kõndisime piki toonase kontori koridori mille ühele 
poole avanesid töö- ja teisele nõupidamisteruumid. Kõigist ustest paistis ja 
koridoris tuli vastu järjest inimesi, kellega mul kas oli alati olnud rõõm koos 
töötada või kellega ma olin alati tahtnud koos töötada. Memcpy on mingit pidi 
katse toda tunnet uuesti kogeda. Sellest ka pühendus.

Siiski ei ole isiklik emotsionaalne heaolu tingimata heaks põhjuseks inimesi 
tülitada või veeta tunde teksti transkribeerides ja toimetades. Laiem põhjus 
memcpy taga on vajadus dokumenteerida inimesi ja nende suhteid, kelle isiklike 
väikeste näppude alt on välja tulnud kõik suuremad Eesti IT-edulood. 

Riigi Infosüsteemi Ametis\index{Riigi Infosüsteemi Amet} töötades pidin aastate 
kaupa peaaegu igal nädalal rääkima riigi infosüsteemist, selle ülesehitusest ja 
ajaloost ning vastama küsimustele. Mul ei olnud pikalt vastust sagedasele 
küsimusele \enquote{Miks Eestis ja mitte mujal?}. Meil ei ole objektiivselt 
vaadates erilisi põhjusi olla oma naabritest edukamad, me isegi ei tee midagi 
eriti innovatiivset aga ometi oleme suutnud kiiresti edasi liikuda ja meil on 
kogu maailmas selge positiivne IT-teemaline imago. Miks? Vastust otsides 
jõudsin ikka ja jälle usalduse küsimuseni. Mingil põhjusel Eestis usaldatakse 
IT-inimesi, neid kaasatakse oluliste otsuste juurde ja IT-inimesed suudavad 
selle usalduse ka välja teenida. Sedalaadi suhetel on juured ja nende üle 
juureldes jõudsin aega natuke enne ja pärast Vabariigi taassündi. Ühtäkki 
hakkas meile jõudma arvuteid, kuid keskmisel inimesel puudus igasugune võimekus 
neid kasutada. Teisalt oli tekkinud toimekas seltskond kodanikke, kes oskasid 
arvutit kasutada, kuid kellel ei olnud neile ligipääsu. Ja nii sündiski 
arusaam, et koos on mõistlik. Et IT-st on kasu. Et kuskil on kellegi 
lahendamist vajavad elulised probleemid. Ja, mis peamine, et seda suhet ei ole 
mõistlik lõhkuda\sidenote{Konkreetne näide leheküljel \pageref{sisu:andrus_usaldus}. Sealsamas ka haruldane näide tolle usalduse kuritarvitamisest.}. 

Kust too seltskond tuli, kuidas toimis, kes sinna kuulusid? Nendele küsimustele 
otsib memcpy vastust, kui \emph{fanboy} roll vähegi võimaldab. Seetõttu ongi 
fookus inimestel ja mitte näiteks ettevõtetel\sidenote{Vähemalt Dateli, 
Proeksperdi ja Microlinki kohta on väikesetiraa\v{z}ilised ajalood ilmunud.} 
või kurioossetel seikadel. 

Ometi ei ole ma ajaloolane ega folklorist, kas memcpy-t ei võiks teha 
professionaalid? Kõik katsed leida keegi asjatundja asja läbi viima luhtusid 
sel lihtsal põhjusel, et kellelgi ei olnud teema vastu piisavat isiklikku huvi 
ja kõik katsed ettevõtmist kuidagi rahastada jooksid liiva. Pärast esialgse 
idee formuleerimist veetsin ma umbes aasta üritades edutult leida tegijaid ja 
rahastust. Seejärel veetsin umbes pool aastat veendes ennast, et memcpy ei pea 
olema täiuslik. Intervjuude ette valmistamine, salvestamine, toimetamine ja 
järeltöötlus on tehniliselt keerulised protsessid, mida ma ei vallanud siis ja 
ei valda ka praegu. Siiski oli selge, et Issanda päike enne looja läheb, kui ma 
neis mind ennast rahuldava taseme saavutan. Nii tuli süda kõvaks teha ja teha 
mitte nii hästi, kui teema vajaks, vaid nii hästi, kui suudan. Seetõttu on 
eriti esimeste memcpy episoodide helikvaliteet päris kole ja see häirib mind 
siiani.

Siiski sai sügisel 2018 purki esimene episood Prontoga\index[ppl]{Pronto} ja 
kevadeks veel kaheksa episoodi. Suvel on inimesed liikvel ja nii jätkasin 
sügisel 2019 juba märksa parema planeerimisega saades napilt enne COVID-19 
pandeemia Eestisse jõudmist 2020. aasta märtsis purki ka teise hooaja 
intervjuud. Episoode kokku lõigates jäi mind häirima, et need ei ole mugavalt 
otsitavad. Inimesed, ettevõtted ja kohad jooksid läbi eri lugudest, aga 
millistest? Väga raske on öelda midagi võrgustiku kohta, kui seda võrgustikku 
saab uurida vaid tipp-haaval. Kuna pandeemia tõttu uusi episoode salvestada ei 
saanud siis oli loogiliseks sammuks võtta aega olemasolevate episoodide 
transkribeerimiseks, toimetamiseks ja indeksiga varustamiseks. Ehk tekitamaks 
seda teksti siin. 

Seega on memcpy igate pidi väga isiklik projekt ning sellisena paratamatult 
piiratud. Ma ei saa ega kavatsegi toota kiretut ajaloodokumenti\sidenote{Mu 
enda peatükk on lisatud just võimaldamaks \enquote{autorifiltri} paremat 
mõistmist.} ning teisalt ei saa lootagi, et võiksin suuta rääkida kõigi 
huvitavate või oluliste inimestega. Kõik lihtsalt ei mahu raamatusse, mõned ei 
soovinud (minuga) rääkida ja mõned lihtsalt ei tulnud pähe. Andestust! 
Mõningased piirid intervjueeritavatele seab ka projekti ajaline määratlus just 
kaheksa- ja üheksakümnendatega. Nii on enamasti välja jäänud näiteks 
Mainori\index{Mainor} ümber tegutsenud seltskond ning natuke vanema põlvkonna, 
näiteks kadunud Ahto Kalja\index[ppl]{Kaljo, Ahto} ja Monika 
Oiti\index[ppl]{Oit, Monika} tegemised. Samuti on puht praktilistel põhjustel 
seltskonnas vähe Tartus tegutsenud ning venekeelse taustaga inimesi. Üle ega 
ümber ei saa ka asjaolust, et kunagi IT-rahva kohta laialt kasutusel olnud 
mõiste \enquote{patsiga poisid} ka memcpy-s otsest peegeldust leiab. Enamasti 
on tõesti tegu poistega. Kahju küll, aga uuritav kogukond paraku oli 
ebaproportsionaalselt maskuliinne ja selle teistsugusena kujutamine ei oleks 
päris õige. Samas olid Vilve Vene\index[ppl]{Vene, Vilve} ja Anne 
Villemsi\index[ppl]{Villems, Anne} ühed kõige huvitavamad salvestada.

Inimeste mälu on erinev. Seega lähevad inimeste lood, ja just nimelt lugude 
talletamine on memcpy eesmärk, omavahel detailides vastuollu. Otsesed 
kõrvalekalded teadaolevast reaalsusest on osundatud ning pisemad vead 
parandatud. Siiski ei maksa oodata, et järgnevatel lehekülgedel näiteks vana 
Tartu ja Tallinna koolkondade vastuolu kuidagi objektiivselt lahendatud saaks. 
Tegu on lugudega ja neid tuleb paratamatult võtta tera soolaga.

Samuti tuleb arvestada, et suuri asju tegevad huvitavad inimesed on harva 
lihtsad isiksused. Olen üritanud kunagisest küllalt keerulisest suhete taagast 
oskust mööda üle olla. Seetõttu on intervjuud järgnevatel lehekülgedel 
tähestikulises järjekorras ja mitte intervjuude toimumise või näiteks olulisuse 
omas. 

Transkribeeritud ja podcasti eetrisse läinud juttudest on mõned üksikud 
detailid ka välja jäetud, sest mõnest asjast ei taha inimesed väga rääkida ja 
mõnda asja ei ole paslik tiražeerida. Üheksakümnendad oli päris hull ja 
tänasest täitsa erinev aeg. Tegu on siiski detailidega, mis suurt pilti ei 
tohiks mõjutada. Muu jutt on enamasti täies mahus\sidenote{Erandiks on 
intervjuu Tarvi Martensiga\index[ppl]{Martens, Tarvi}. Temaga oli meil paljust 
rääkida ja salvestasime kaks episoodi, mis teemade poolest osaliselt kattusid. 
Seega tuli kirjalikus tekstis selguse huvides asju natuke ümber tõsta ja 
tihendada.} ja võimalikult originaalilähedase keelekasutusega ära toodud. 
Sellest ka anglitsismide ja võõrkeelsete terminite kahetsusväärselt suur hulk. 
Aga kuna keelekasutus annab huvitava akna inimesse, eelistasin autentsust 
ilusale emakeelele.  

Tekst on mõeldud olema ka mitte-arvutiinimestele üldjoontes arusaadav: 
konteksti mõistmiseks olulised terminid on lahti seletatud ning tänaseks ehk 
ununenud asjad viidatud, kuid detailid otsib huviline ise välja. Eesmärk ei ole 
olnud anda struktureeritud ülevaadet arvutustehnika ajaloost või vanade 
tehnoloogiate toimimisest. Intervjuudes olen üritanud küsida võhiku 
positsioonilt. Mis on seda lihtsam, et paljus ma seda olengi.

Kuigi fookus on inimestel ja nende lugudel, olen mõneti ajaloo säilitamise ja 
mõneti oluliste inimeste ära märkimise eesmärgil lisanud ka esimese Eesti regiooni
sisaldanud Fido nodelisti ja kõige varasema leitud Eesti BBS-ide nimekirja\sidenote{Vaata lk. \pageref{sisu:nodelist}.}.

Järgnevaid lehekülgi ei ole juba kasvõi nende mahu pärast ehk mõistlik kaanest 
kaaneni lugeda. Targem on teda lapata kasutades kas nimede või muud indeksit, 
alustades mõnest huvitavamast loost või lihtsalt alates juhuslikust leheküljest 
pea ees minevikku hüpata.

Kuigi memcpy on, nagu öeldud, isiklik projekt. Siiski olen tema tegemise käigus 
saanud hindamatut tuge ja soojad tänuavaldused lähevad teele:

\begin{description}
	\item[Meelis Roos]\index[ppl]{Roos, Meelis} kes aitas nii oma kui nii 
mõnegi teise teksti toimetamisega, parandas mu piinlikke kirjavigu, mõtles 
projektiga kaasa ja andis tehnilist tuge
	\item[Rein Rüüsak]\index[ppl]{Rüüsak, Rein} kes aitas ajakirja A\&A 
ajalugu välja uurida
	\item[Ott Köstner]\index[ppl]{Köstner, Ott} kes on memcpy podcasti 
kaanepildi autor
	\item[Vootele Voit]\index[ppl]{Voit, Vootele} kes kommenteeris 
asjalikult ZX Spectrumi kiibistikku puudutavat
	\item[Kõik intervjueeritavad] kes võtsid oma tihedast päevast tunni, et 
minuga juttu rääkida
	\item[Mart Palmas ja Tarmo Mamers]\index[ppl]{Palmas, Mart}\index[ppl]{Mamers, Tarmo} kes aitasid Soome telekavade teket mäletada 
	\item[Veebipõhine transkriptsioon] ilma milleta käesolev tekst 
kindlasti sündinud ei oleks. Alumäe, Tanel; Tilk, Ottokar; Asadullah. "Advanced 
Rich Transcription System for Estonian Speech" Baltic HLT 2018
\end{description}
