Tere. See siin on memcpy. Nende sõnadega olen sisse juhatanud suurt hulka 
intervjuusid oluliste inimestega ja nüüd on teie ees see tekst. 

Aga miks? 

Põhjus, tuleb tunnistada, on lihtne. Nagu ütleb Villu Tamme loos 
\enquote{Paneme punki}:

\begin{verse}
Tahan kord saada selliseks, nagu \\
on Villu või Freddy või Rott või Striit\\
\end{verse}

See raamat räägib inimestest, kes on mulle oma tarkuse, oskuste ja olemusega 
eeskujuks olnud. Kui Toivo Annus\index[ppl]{Annus, Toivo} mu kunagi Skype'i 
tööintervjuule kutsus, kõndisime piki toonase kontori koridori, mille ühele 
poole avanesid töö- ja teisele nõupidamisruumid. Kõigist ustest paistis ja 
koridoris tuli järjest vastu inimesi, kellega mul kas oli alati olnud rõõm koos 
töötada või kellega olin alati tahtnud koos töötada. Memcpy on mingitpidi 
katse toda tunnet uuesti kogeda.

Siiski ei ole isiklik emotsionaalne heaolu tingimata hea põhjus inimesi 
tülitada või veeta tunde teksti transkribeerides ja toimetades. Sügavam põhjus 
memcpy taga on vajadus dokumenteerida inimesi, kelle 
väikeste näppude alt on välja tulnud kõik suuremad Eesti IT-edulood. 

Riigi Infosüsteemi Ametis\index{Riigi Infosüsteemi Amet} töötades pidin aastate 
kaupa peaaegu igal nädalal rääkima riigi infosüsteemist, selle ülesehitusest ja 
ajaloost ning vastama küsimustele. Mul ei olnud pikka aega vastust sagedasele 
küsimusele \enquote{miks Eestis ja mitte mujal?}. Meil ei ole objektiivselt 
vaadates erilisi põhjusi olla oma naabritest edukamad, me isegi ei tee midagi 
eriti innovatiivset, aga ometi oleme suutnud kiiresti edasi liikuda ja meil on 
kogu maailmas selge positiivne IT imago. Miks? 

Vastust otsides jõudsin ikka ja jälle usalduse küsimuseni. Mingil põhjusel Eestis usaldatakse 
IT inimesi, neid kaasatakse oluliste otsuste juurde ja nad suudavad 
selle usalduse ka välja teenida. Sedalaadi suhetel on juured ja nende üle 
% TODO: Kas siin ei oleks parem "ja nad suudavad seda usaldust ka õigustada"?
% TODO: Minu keeletunde järgi on välja teenimine midagi, mis juhtub enne usaldamist, samas kui õigustamine või mitte õigustamine selgub pärast.
juureldes jõudsin aega natuke enne ja pärast vabariigi taassündi. Ühtäkki 
hakkas toona meile jõudma arvuteid, kuid keskmisel inimesel puudus võimekus 
neid kasutada. Teisalt oli tekkinud toimekas seltskond kodanikke, kes oskasid 
arvutit kasutada, kuid kellel ei olnud neile ligipääsu. Ja nii sündiski 
arusaam, et koos on mõistlik. Et ITst on kasu. Et kuskil on kellegi 
lahendamist vajavad elulised probleemid. Ja, mis peamine, et seda suhet ei ole 
mõistlik lõhkuda.\sidenote{Vt näide lk \pageref{sisu:andrus_usaldus} 
Sealsamas ka haruldane näide usalduse kuritarvitamisest.} 

Kust too seltskond tuli, kuidas toimis, kes sinna kuulusid? Nendele küsimustele 
otsib memcpy vastust määrani, mida \emph{fanboy} roll vähegi võimaldab. Seetõttu ongi 
fookus inimestel, mitte ettevõtetel\sidenote{Vähemalt Dateli, 
Proeksperdi ja MicroLinki kohta on väikesetiraažilised ajalood ilmunud.} 
või kurioossetel seikadel. 

Ometi ei ole ma ajaloolane ega folklorist, nii et kas memcpy-t ei võiks teha 
professionaalid? Kõik katsed leida keegi asjatundja asja läbi viima luhtusid 
sel lihtsal põhjusel, et kellelgi ei olnud teema vastu piisavat isiklikku huvi 
ja katsed ettevõtmist kuidagi rahastada jooksid liiva. Pärast esialgse 
idee formuleerimist veetsin umbes aasta, üritades edutult leida tegijaid ja 
rahastust. Seejärel veetsin umbes pool aastat ennast veendes, et memcpy ei pea 
olema täiuslik. Intervjuude ettevalmistamine, salvestamine, toimetamine ja 
järeltöötlus on tehniliselt keerulised protsessid, mida ma ei vallanud tööga alustades ja 
ei valda ka praegu. Siiski oli selge, et enne läheb issanda päike looja, kui ma 
neis valdkondades mind ennast rahuldava taseme saavutan. Nii tuli süda kõvaks teha ja teha memcpy-t 
mitte nii hästi, kui peaks, vaid nii hästi, kui suudan. Seetõttu on 
eriti esimeste memcpy episoodide helikvaliteet päris kole ja see häirib mind 
siiani.

Sügisel 2018 sai purki esimene intervjuu Prontoga\index[ppl]{Pronto} ja 
kevadeks veel kaheksa. Suvel on inimesed rohkem liikvel ja nii jätkasin 
2019. aasta sügisel juba märksa parema planeerimisega, saades napilt enne COVID-19 
pandeemia Eestisse jõudmist 2020. aasta märtsis linti ka teise hooaja 
intervjuud. Neid kokku lõigates jäi häirima, et need ei ole mugavalt 
otsitavad. Inimesed, ettevõtted ja kohad jooksid läbi eri lugudest, aga 
millistest täpsemalt? Väga raske on öelda midagi võrgustiku kohta, kui seda võrgustikku 
saab uurida vaid tipphaaval. Kuna pandeemia tõttu uusi intervjuusid salvestada ei 
saanud, siis oli loogiline samm võtta aega olemasolevate intervjuude
transkribeerimiseks, toimetamiseks ja indeksiga varustamiseks, et luua
seda teksti siin. 

Seega on memcpy igati isiklik projekt ning sellisena paratamatult 
piiratud. Ma ei saa ega kavatsegi toota kiretut ajaloodokumenti\sidenote{Mu 
enda peatükk on lisatud just võimaldamaks paremini mõista filtrit, mille ülejäänud 
sisu on läbinud.} ning teisalt ei saa lootagi, et võiksin suuta rääkida kõigi 
huvitavate ja oluliste inimestega. Kõik lihtsalt ei mahu raamatusse, mõned ei 
soovinud (minuga) rääkida ja mõned ei tulnud pähe. Andestust! 

Mõningased piirid intervjueeritavate valikule seadis ka projekti ajaline määratlus just 
kaheksa- ja üheksakümnendatega. Seetõttu on enamasti välja jäänud näiteks 
Mainori\index{Mainor} ümber tegutsenud seltskond ning natuke vanema põlvkonna, 
näiteks kadunud Ahto Kalja\index[ppl]{Kalja, Ahto} ja Monika 
Oiti\index[ppl]{Oit, Monika} tegemised. Samuti on praktilistel põhjustel 
valikus vähe Tartus tegutsenud ja venekeelse taustaga inimesi. Üle ega 
ümber ei saa ka asjaolust, et kunagi IT-rahva kohta laialt kasutusel olnud 
mõiste \enquote{patsiga poisid} leiab ka memcpy-s otsest peegeldust. Enamasti 
on tõesti tegu poistega. Kahju küll, aga uuritav kogukond paraku oli 
ebaproportsionaalselt maskuliinne\sidenote{Lk \pageref{sisu:tydrukud} on 
natuke lähemalt juttu tolle nähtuse põhjustest.} ja selle teistsugusena kujutamine ei oleks 
päris õige. Samas olid Vilve Vene\index[ppl]{Vene, Vilve} ja Anne 
Villemsi\index[ppl]{Villems, Anne} intervjuud ühed kõige huvitavamad salvestada.

Inimeste mälu on erinev. Seetõttu lähevad inimeste lood -- ja memcpy eesmärk on just nimelt lugude 
talletamine -- omavahel detailides aeg-ajalt vastuollu. Otsesed 
kõrvalekalded teadaolevast\sidenote{Teadmine paraku laieneb kogu aeg ja kindlasti 
jääb osa kaheldavaid detaile märkamata.} reaalsusest on osundatud ning pisemad vead 
parandatud. Samas ei maksa oodata, et järgnevatel lehekülgedel saaks näiteks vana 
Tartu ja Tallinna koolkondade vastuolu kuidagi objektiivselt lahendatud. 
Tegu on lugudega ja neid tuleb paratamatult võtta tera soolaga.

Samuti tuleb arvestada, et suuri asju tegevad huvitavad inimesed on harva 
lihtsad isiksused. Olen üritanud kunagisest küllaltki keerulisest suhete taagast 
oskust mööda üle olla. Seetõttu on intervjuud järgnevatel lehekülgedel 
tähestikulises, mitte intervjuude toimumise või olulisuse järjekorras. 

Transkribeeritud ja \emph{podcast}'i eetrisse läinud juttudest on üksikud 
detailid ka välja jäetud, sest mõnest asjast ei taha inimesed väga rääkida ja 
mõnda asja ei ole paslik tiražeerida. Üheksakümnendad oli päris hull ja 
tänasest täiesti erinev aeg. Tegu on siiski nüanssidega, mis suurt pilti ei 
tohiks mõjutada. Muidu on intervjuud enamasti täies mahus\sidenote{Erandiks on 
intervjuu Tarvi Martensiga\index[ppl]{Martens, Tarvi}, kellega meil oli väga paljust 
rääkida ja salvestasime kaks episoodi, mis teemade poolest osaliselt kattusid. 
Seega tuli kirjalikus tekstis selguse huvides asju natuke ümber tõsta ja 
tihendada.} ja küllalt originaalilähedase keelekasutusega edasi antud. 
Sellest ka anglitsismide ja võõrkeelsete terminite suur hulk. 
Aga kuna keelekasutus annab huvitava akna inimesse, eelistasin vahel autentsust 
ilusale emakeelele.\sidenote{Seesinane eelistus põhjustas toimetajale kahjuks palju meelehärmi.}

Tekst on mõeldud olema ka arvutikaugematele inimestele üldjoontes arusaadav: 
konteksti mõistmiseks olulised terminid on lahti seletatud ja tänaseks ehk 
ununenud asjad viidatud, kuid detailid otsib huviline ise välja. Eesmärk ei ole 
olnud anda struktureeritud ülevaadet arvutustehnika ajaloost või vanade 
tehnoloogiate toimimisest. Intervjuudes üritasin küsida võhiku 
positsioonilt, mis oli seda lihtsam, et seda ma paljuski olengi.

Kuigi fookus on inimestel ja nende lugudel, olen mõneti ajaloo säilitamise ja 
mõneti oluliste inimeste äramärkimise eesmärgil lisanud ka esimese Eesti regiooni
sisaldanud Fido \emph{nodelist}'i ja kõige varasema Eesti BBSide nimekirja, mille 
leidsin.\sidenote{Vt lk \pageref{sisu:nodelist}.}

Järgnevaid lehekülgi ei ole juba kasvõi nende mahu pärast ehk mõistlik kaanest 
kaaneni lugeda. Targem on lapata, kasutades kas indeksit, 
alustada mõnest huvitavamast loost või lihtsalt alates juhuslikust leheküljest 
pea ees minevikku hüpata.

Raamatu kujunduses on toetutud Edward Tufte tööle läbi vastava \LaTeX-i klassi.\sidenote{\url{https://github.com/Tufte-LaTeX/tufte-latex}.}
Tegu on teadliku valikuga: nii on lisaks ääremärkustele lehekülgedel piisavalt ruumi 
ka lugeja märkuste, arvamuste või joonistuste jaoks. Rõõmsat sodimist!

