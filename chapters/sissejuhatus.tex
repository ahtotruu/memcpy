Juhatame sisse. 
\begin{itemize}
	\item Miks ma seda teen
	\begin{itemize}
		\item \enquote{Tahan kord saada selliseks, nagu on Villu või Freddy või Rott või Striit.}\sidenote{Villu Tamme, "Paneme punki"}
	\end{itemize}
	\item Eesmärk: kujutada inimesi ja nende suhteid (mitte näiteks kurioosseid hetki või ettevõtteid)
	\item Lühike ajalugu: idee, otsing, podcast, siis analüüs ja raamat
	\item Sisu kohta
	\begin{itemize}
		\item Kõik ei mahtunud raamatusse, kõik ei soovinud rääkida ja kõik ei tulnud pähe. Andestust!
		\begin{itemize}
			\item Välja on enamasti jäänud näiteks Mainor ja natuke vanema põlvkonna (näiteks kadunud Ahto Kalja ja Monika Oit'i) tegemised
			\item Tartu seltskonda on kahetsusväärselt vähe hulgas
		\end{itemize}
		\item Lood lähevad omavahel vastuollu, see on OK. Samas on otsesed kõrvalekalded teadaolevast reaalsusest osundatud ning, kui intervjueeritav kahtles, on üritatud õige välja tuua
		\item Kõik on isiksused. Mõned kergemad, mõned raskemad. Olen üritanud suhte-taagast üle olla
		\item Nii \enquote{läbipaistev} vaade kui võimalik. Sealhulgas näiteks ka ehk liigselt anglitsismirohke keelekasutus, sugude tasakaalu puudus jne.
		\item Mõeldud olema ka mitte-arvutiinimesele üldjoontes arusaadav: konteksti mõistmiseks olulised terminid on lahti seletatud, kuid detailid otsib huviline ise välja. Samas ei ole eesmärk anda struktureeritud ülevaadet arvutustehnika ajaloost või vanade tehnoloogiate toimimisest. Olen üritanud küsida võhiku positsioonilt. Mis on seda lihtsam, et paljus ma olen võhik.
		\item Inimesed tähestikulises järjekorras	
		\item Oma jutt on ka, sest muidu jääks juttudesse kummaline auk, lisaks tuleks ju anda aimu, mis prisma läbi ülejäänud asjad on kirjutatud. Intervjueerisin ennast ise
		\item \enquote{Patsiga poisid} kui üldnimetus. Enamasti siiski poisid. Kahju küll, aga nii oli. Raamat on läbilõige toonasest seltskonnast ja oleks vale toda seltskonda kuidagi teistsugusena kujutada
		\item Jutt on enamasti täies mahus nii, nagu räägitud sai, kirjakeelde pandud. Mõnes üksikus kohas lühendasin\sidenote{Tarvi jutt on kahe intervjuu kombinatsioon, seal tuli selguse huvides asju natuke ümber tõsta ja tihendada} ja mõnda üksikut asja ei pidanud paslikuks sisse jätta - Mõnest asjast ei taha inimesed väga rääkida ja mõnda asja ma ei taha avaldada. Üheksakümnendad oli päris hull ja teistsugune aeg. Need on siiski detailid ja suurt pilti ei muuda.. 
	\end{itemize}
	\item Kuidas lugeda
	\begin{itemize}
		\item On indeks, eraldi inimeste oma
		\item On lühikesed selgitused mainitud riistvara ja arvutite osas
		\item Detailsema jutu leiab igaüks ise Internetist
	\end{itemize}
	\item Tänuavaldused
	\begin{itemize}
		\item Rein Rüüsak, A\&A ajaloo välja uurimine
		\item Ott Köstner, memcpy kaanepilt
		\item Vootele Voit, info ZX Spectrumi kiibistiku kohta
		\item Kõik intervjueeritud
		\item Veebipõhine transkriptsioon (Alumäe, Tanel; Tilk, Ottokar; Asadullah. "Advanced Rich Transcription System for Estonian Speech" Baltic HLT 2018)
		\item Mroos toimetamine, kaasamõtlemine ja tehniline tugi
		\item Wikipedia
	\end{itemize}
	\item Küsimused
	\begin{itemize}
		\item The C Programming language raamat
	\end{itemize}
 \end{itemize}
