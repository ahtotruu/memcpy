\index[ppl]{Sutt, Margus}

\question{Kust sa pärit oled?}
Ma olen Tallinnast.

\question{Kuidas arvutid 
sinu juurde said ja sina arvutite juurde?}

Kui mul oli aeg kooli minna, siis 
meie pere parajasti kolis sellise kooli nagu Tallinna 3. 
Keskkool\index{Tallinna 3. Keskkool}, mis praegu kannab nime
Lilleküla gümnaasium\index{Lilleküla gümnaasium|see{Tallinna 3. 
Keskkool}}, piirkonda. Ja selles koolis oli 
selline tüüp nagu Jaak Loonde\index[ppl]{Loonde, Jaak}.

\question{Jälle Jaak!}

Jah. Põhikooli esimese kaheksa aasta jooksul ei 
juhtunud veel midagi, aga keskkooli minnes toimusid 
katsed ja konkursid ning meil tehti juurde kolmas klass, 
nii-öelda informaatika-matemaatika eriklass. Aasta oli siis 1985 või 1986.

\question{Kas seda klassi vedas 
Jack\index[ppl]{Jack}?}

Jah, tema organiseeris selle klassi.

\question{Kas ta oli klassijuhataja ka?}

Ei, klassijuhataja ta ei olnud, ta ajas asju haridusministeeriumiga või 
mis iganes asutus see siis veel Vene ajal oli, ja igatahes sai 
selle klassi. Kaua see klass vist ei püsinud, kaks 
komplekti oli seda kindlasti, aga kolmandas ma kindel ei ole, kuna 
lõpetasin juba kooli. Jack ei pidanud seal koolis kaua vastu ja peale 
teda see asi hääbus. 

Jackil oli koolis selline tore külmkastide rivi nagu 
MIR-2\index{MIR-2}, kus olid perfolindid ja -kaardid.

\question{Kas see oli tal lausa koolis?}

See oli tema klassiruumis, ruume oli tal muidugi mitu. Lisaks 
klassiruumile ka algeline raadioruum, millest sai hiljem päris raadioruum. 
Meie klassi tegelased ehitasid selle, mitte küll mina, aga põhiaktivist oli Andrus Tamboom\index[ppl]{Tamboom, Andrus}. Tema ehitas kooli 
raadiovõrgu või pigem taastas, sest juhtmed olid vanast 
ajast seinas, aga ta kohendas seda ja ajas kuidagi käima.

\question{Kuidas perfolindi tingimustes arvutiõpe praktiliselt välja nägi?}

Ega õpet väga palju ei olnud. Võibolla pool aastat tegelesime
MIR-2ga\index{MIR-2} ja edasi juba natuke 
keerulisemate asjadega. Ekraani peal sai mingisugust parabooli
joonistatud. Perfolindiga oli nii, et oli lugeja, mille sisse tuli lint pista
ja kuskil mingit käsku vajutada ning see hakkas linti hästi 
kiiresti läbi vedama.

\question{Kas kooliõpilased perforeerisid linti või see usaldati kellegi teise 
kätte?}

Arvuti perforeeris ise ka, sellel olid nii sisend kui ka väljund olemas.

\question{Nii et nagu trükimasin – toksid sisse ja arvuti teeb perfolindi?}

Ja pärast on võimeline seda lugema. 

\question{Kuidas sul see mõte sündis, et võiks just sellesse klassi
minna, kus informaatikat õpetati?}

Ma olen selle peale mõelnud, aga ei mäleta, kuidas. Võibolla vanemad 
torkisid tagant või oli endal mingil määral huvi. Matemaatikaga mul probleeme 
ei olnud ja käisin isegi olümpiaadidel.

\question{Tavaliselt on arvutiõppes kaks osa: see, mis tunnis räägitakse, 
ja programmiväline tegevus, mille käigus ise pusitakse. Kuidas teil 
see vahekord oli?}

Selles mõttes tuleb jälle Jacki\index[ppl]{Jack} tänada. Tema õpetamismeetod 
oli klassikalisest kardinaalselt erinev. Matemaatikat 
ta raamatust ei õpetanud, vaid tal olid enda tehtud töölehed, kus oli 
muu hulgas üritatud tekitada ka siirdeid teistesse ainetesse. Füüsika oli 
kõige lihtsam, keemiat ma ei mäleta, aga eesti keelt oli ka
kuskil mainitud. 

\question{Nii et ta õpetas matemaatikat ka?}

Jah, meie klassile ta õpetas matemaatikat ka, järgmisele aastakäigule enam mitte. 

\question{Kas töölehed ei sobinud?}

Seal oli igasuguseid probleeme. Järgmist aastakäiku mainin sellepärast, et 
seal õppisid Priit Kasesalu\index[ppl]{Kasesalu, Priit}, Mikk 
Orglaan\index[ppl]{Orglaan, Mikk} ja Janno 
Ossaar\index[ppl]{Ossaar, Janno}. 

Jaagu\index[ppl]{Loonde, Jaak} õpetamismeetod oli suures osas vetteviskamine, vähemalt 
arvuti poolel. Et näete, siin on arvuti, tegelege. Muu hulgas õnnestus tal 
näiteks koos klassijuhataja Tiiu Neemega\index[ppl]{Neeme, Tiiu} viimases 
klassis minule ja veel paarile klassikaaslasele välja rääkida selline eriprogramm, et me osas 
tundides ei pidanudki käima, tegime ainult semestrite või trimeistrite 
lõpus töid ja arvestusi.

\question{Et sellist eriprogrammi võimaldada, pidi akadeemiline 
jõudlus tasemel olema?}

Sellega ilmselt ei olnud probleemi jah. 

Arvutitest puutusin kõigepealt kokku MSXidega\index{Yamaha 
MSX} Luise tänava ÕTKs\index{Tallinna Oktoobrirajooni 
Õppetootmiskombinaat}\sidenote{Tallinna Oktoobrirajooni Õppetootmiskombinaat.}. 
Algul käisime seal klassiga ja pärast käisin juba ise. 
MSXidega oli tore see, et Jackil\index[ppl]{Jack} õnnestus need 
vaheaegadeks 3. keskkooli laenata. Tänu sellele oli mul hulga parem ligipääs arvutitele, sest mul oli lisaks Jacki ruumi võtmele ka kooli välisvõti. Nii et
keskkooliajal tekkis võimalus suvevaheajal ööpäev läbi arvutis olla.

\question{Mida sa arvutiga tegid? Teiste juttu kuulates tundub olema kahte 
liiki inimesi: need, keda huvitas rohkem programmeerimine, ja need, keda tõmbas 
mängude poole. Kumb sul rohkem domineeris?}

Esialgu kindlasti mängimine. Yamahal\index{Yamaha MSX} olid head ilusad mängud, eks need meelitasid 
oma graafikaga ja sellega, et võrreldes MIRiga oli see ikka hoopis teine maailm. 
Isegi võrreldes Agatiga\index{Agat}. Keskkooli 
esimesel või teisel aastal tekkis meile üks Agat oma kooli\index{Tallinna 3. 
Keskkool} ka. 

Progemise mõttes oli mul Agatist kindlasti rohkem kasu, sest seal käis 
Tarmo Mamers\index[ppl]{Mamers, Tarmo} vahel asja 
uurimas ja üle tema õla kiibitsedes õnnestus mul ka üht-teist-kolmandat 
omandada.

Agati peal oli Basic\index{BASIC}, aga muu hulgas sai ka juba assemblerit 
vaadatud.

\question{Kas Agat oli Apple II kloon?}

See oli Apple II\index{Apple II} kloon, mille sees oli originaal-Apple'i \emph{chip}, kust oli info maha kraabitud. Vähemalt sellel konkreetsel Agatil oli 
nii.

\question{Mina mäletan Agatil veidrusi, mida 
Apple'il vaevalt et oli. Ilmselt see OS oli Nõukogude oma?}

Ma ei mäleta nii täpselt, sest ma pole päris Apple'it puutunud ega oska võrrelda. Tean, et Tarmol\index[ppl]{Mamers, Tarmo} olid Apple'ist 
väljatrükid BIOSi ja selle \emph{call}'ide kohta, ja ta üritas seda 
Agati peal rakendada.

\question{See on juba oluline teadmine, et Apple'i \emph{call}'id võiksid 
töötada!}

Jah, muu hulgas oli mul näiteks selline huviprojekt, et üritasin
Apple'i peale Norton Commanderit\index{Norton Commander} kirjutada. 
Assembleris, loomulikult.

\question{Huvitav, minul oli samasugune projekt Juku peal.}

See tee on vist jah aastate jooksul kõigil läbi käidud. 

\question{Kust teile ülesanded tulid? 
Liiga lihtne ülesanne ei ole huvitav, aga liiga keerulise ülesande puhul jooksed end algajana sodiks, enne kui lootusekiir paistma hakkab. 
Kas teid juhendas Jack või mõtlesid ise välja?}

Ei, Jack\index[ppl]{Jack} selles mõttes eriti ei juhendanud. Tema antud 
programmid olid pigem matemaatikateemalised. Ja ega Jack ilmselt ei 
küündinudki sellise juhendamiseni või kui küündis, siis
ei pidanud seda vajalikuks. Tuli ise vaadata, mida teed. 

Mängude puhul püüdsin spraite või muid selliseid asju 
liigutada. Mäletan veel ka, et võtsin 
sinusoidi ja nii-öelda nihutasin selle ruumi. Kui tekitad mõlema koordinaadi 
suhtes väikse nihke, siis tekib ruumiline efekt. Ja kui 
paned programmi tagant kustutama ja eest uuesti joonistama, on tulemuseks 
tänapäevane \emph{screensaver}. Tol ajal oli see huvitav.

\question{Kas arvutihuvi juurde käis ka 
spetsiifiline muusika- või kirjandushuvi? Päris mitmed on rääkinud, 
kuidas nad on Asimovi ja Gibsoni peal üles kasvanud.}

Ma olen lugenud küll fantastikat ja \emph{sci-fi}'d, aga ma ei 
tea, kui palju see nüüd arvutiga seotud on. Pigem on need
konfliktis, sest võistlevad aja pärast. Enne arvutiaega lugesin 
väga palju ja kiiresti, näiteks „Seiklusjutte maalt ja merelt“. Lugesin kõike, mida kätte 
sai. Aga kui arvuti tuli, siis ma enam nii palju ei 
lugenud. Umbes kümme aastat tagasi võtsin kätte ja lugesin 
läbi enamiku McCaffreyst\sidenote{Margus peab ilmselt silmas ameerika-iiri 
ulmekirjanikku Anne McCaffreyt. Ainuüksi tema loheratsurite sarjas on 23 
romaani ja see on vaid üks paljudest selle kirjaniku loodud 
maailmadest.}, alustades lohelugudest, neist kaks on isegi eesti keeles olemas.

\question{Mõni on rääkinud, et juba keskkooliajal kippus töötegemiseks 
minema. Kas sul ei olnud nii?}

Ei, töötegemiseks ei läinud, kuigi sidemed esimese töökohaga juba tekkisid.

\question{Mis su esimene töökoht oli?}

Seda töökohta iseloomustab kõige paremini Raivo Rebase\index[ppl]{Rebane, Raivo} 
nimi. Ma ei tea, kuidas ta Jackiga oli seotud, aga kuidagi igatahes oli. 
Ühel hetkel oli ta Jacki juures ja põhimõtteliselt otsis jüngreid ehk tegeles 
\emph{head-hunting}'uga.

\question{Vaat kus! Tänapäeval keskkoolist vist enam ei käida otsimas!}

Ta vist plaanis oma arvutifirmat teha, tol 
hetkel tal seda veel ei olnud. Ta tegutses Küberis\index{Küber} sellise 
härrasmehe nagu Raul-Roman Tavasti\index[ppl]{Tavast, Raul-Roman} juures. Neil 
oli vist mingi firma Küberi kõrval. Mäletan seda sellepärast, et seal sain tõenäoliselt esimest korda PCd katsuda. Pärast värbamist saime koolist paari-kolmekesi hakata käima 
kuskil Küberi majas, kus olid PCd. Vist 
lausa 386d, mille mälu oli neli mega. Kui tegin 
\emph{boot floppy}, kus oli mäludraiv peal, siis sain väga palju 
kõvakettaruumi, kus jooksutada Turbo C \emph{editor}'i.

\question{Turbo C on ikka juba meeste vahend, kuidas sa 
selleni jõudsid?}

Ma ei mäleta, kuidas ma C\index{C} juurde 
jõudsin, sest Pascalit ei ole ma kunagi õppinud, aga kuidagi mul tekkis 
Turbo C. Agati peal seda ei saanud olla, ilmselt tuli Rebase kaudu. 

Rebane pistis mulle pihku Kernighan-Ritchie\sidenote{Kuulus valgete kaante ja 
sinise Cga raamat, vt lk \pageref{sisu:richie}.} koopia, mille ma neelasin 
mõne nädalaga läbi.

\question{Kas MSXilt ja Agatilt C peale üle minek keeruline ei olnud?}

Ma ei mäleta, kuidas see täpselt oli. Seal Rebase juures ma käisin ja ühel
hetkel lõi ta Küberist lahku. Järgmine koht oli Liivalaia tänaval praeguses
Swedbanki majas.\sidenote{Liivalaia 8, Tallinn.} Seal oli 12. korrusel, kus 
praegu on nii-öelda ülemuste korrus, arvutuskeskus.\sidenote{Margus peab ilmselt silmas sel aadressil asunud EKE 
Projekti nimelist asutust.}. Seda vaadet ma nautisin üheksakümnendatel. Pärast õnnestus peaaegu samasse kohta tagasi kolida, Rebane sai sinna üheksandele 
korrusele ruumid. 

Arvutuskeskuses oli ka üks suur kastarvuti, millega meie õnneks kokku ei puutunud. Meil olid oma 
PCd, kus hakkasime muu hulgas tegelema ka Unixiga\index{Unix}.

\question{Kuidas te PC peal Unixit tegite?}

SCO\index{SCO UNIX} ja BSD\index{BSD} olid sel ajal olemas.

\question{Kuidas need tol ajal Eesti Vabariiki jõudsid?}

Ma ei tea, võibolla olid kuskilt ostetud. SCO-l olid originaalkirjadega plaadid. Või siis ikka flopid, 
plaadid tulid hiljem. 

\question{Millega te seal tegelesite? Firma pidi ju äri tegema?}

Äriga oli alguses kehvasti, näiteks ma ei mäleta, millal ma palka 
hakkasin saama. Ühel hetkel aga läks väga huvitav radariäri käima. Tegime koostööd Vene firmadega, 
ühe seltskonnaga Peterburist. Nemad tegid radarile riistvara ja kaardi, 
mis läks PC sisse, ja meie ehitasime sinna peale softi.

\question{Radarid on ju sõjaväega seotud ja salajane värk?}

Ei, sõjaväega meil kokkupuudet ei olnud, tegime tsiviilsuuna jaoks. Pulkovo 
radarijaamas sai näiteks korduvalt käidud.

\question{Kas teie soft võttis radarilt signaali ja 
joonistas kaardile mummud?}

Jah. See käis BSD peal, 
võibolla alguses isegi SCO peal, ja 
lõpuks läksime Linuxi\index{Linux} peale, kui see tekkis ja oli juba niivõrd 
kobe, et seda sai kasutada.

\question{Alles joonistasid ekraanile siinust ja siis juba lugesid radarilt signaale – siin tundub lünk olema. Ülesande keerukus on ju 
palju suurem!}

Ega mina seda kõike ei teinud. Selle taga oli terve grupp inimesi.

\question{Kui suur see grupp oli?}

Ei olnud väga suur, alla kümne. Meie firma ei läinud väga suureks. 
Seal olime mina, Raivo Rebane\index[ppl]{Rebane, Raivo} ja Mart 
Rüütel\index[ppl]{Rüütel, Mart}, kes on vist praegu ka veel seal. Firma nimi 
on nüüd R-Süsteemid\index{R-Süsteemid}, mõnda aeg oli Virumaa 
Tiivad\index{Virumaa Tiivad|see{R-Süsteemid}}, sest tegeles natuke ka 
lennundusega. 

Vahepeal oli meil tööl üks lennundusfänn ja muu hulgas sai 
korraldatud väikelennukite ülelennu üritus, mille raames mul 
õnnestus näiteks sõita mingi Piperiga Kuressaare lennujaamast Viljandi lennujaama. Kuna need, kelle lennukiga ma lendasin, olid Saksamaalt, siis 
ega nemad ei teadnud, kus lennujaam täpselt on, ja Viljandi 
lennujaama otsimisega oli natukene tegemist. Et kuhu me nüüd siis 
maandume?

\question{Kes klient oli? Needsamad venelased?}

See soft oli vist mõnda aega ka Tallinnas kasutusel. Softis olid mõlemad, nii 
primaar- kui ka sekundaarradar. Primaarradar on see nii-öelda loll radar, 
mille signaal põrkab lihtsalt kuskilt tagasi või siis ei põrka. Ja 
sekundaarradar on see info, mille lennuk ise välja saadab oma 
transponderiga.

\question{Kuidas te radari sofit töökindluse tagasite?}

Sellel oli mitu kihti loogikat peal. Radaritel sai isegi 
vahvaid maatriks-algoritme kasutatud. Üks tegelane, kelle nime ma 
paraku ei mäleta (keegi Kaido?), tegi nendest lausa TPI 
lõputöö. Nagu lõputöödega ikka vahel on, pannakse kuskile mõni
\emph{catch} sisse lootuses, et niikuinii keegi ei loe. Nende algoritmidega 
juhtus nii, et see koht koodis, kus seda kõike pidi 
välja kutsutama, oli pikka aega välja kommenteeritud. Pärast tema järeltulijad 
avastasid, et oleks hulga parem, kui seda funktsionaalsust ka
kasutataks.

\question{Radari riistvaraga suhtlemine oli ilmselt keeruline?}

Riistvara suhtlemisel oli mängus ajakriitilisus. Tänu sellele tuli paratamatult kerneli alasse ronida, kuna signaal liigub kiiresti ja sul on vaja täpselt 
teada, millal signaal sisse tuli. Sealt omatehtud kaardilt tuli 
puhvrid (mis ei olnud küll tol ajal suured) kähku ära lugeda.

\question{Kas see oli SCO ajal?}

See algas SCO\index{SCO UNIX} ajal ja meil olid selle jaoks
ametlikud raamatud ja ametlik \emph{dev kit} ostetud.

\question{Neid inimesi, kes seda oskasid, ei saanud Eestis palju olla.}

Jah, ega kellegi käest küsida väga ei olnud. Esimene inimene, kellele see asi 
huvi pakkus ja kes midagi teemast teadis ning suutis kaasa rääkida, oli 
aastaid hiljem (kui ma 1993. aastal Tartusse jõudsin) Meelis 
Roos\index[ppl]{Roos, Meelis}. Ta on minust muidugi kõvasti ette jõudnud, 
sest ta on praeguseks päris palju kerneli \emph{patch}'e \emph{post}'inud\sidenote{Linuxi tuuma, ehk kerneli, arendajad on oma tegevuse suure mõju ning kõrge keerukuse tõttu programmeerijate hulgas kõrgelt hinnatud. \emph{Patch}'i ehk paiga postitamine tähendab, et paiga autor ei ole nende töö tulemuses mitte ainult probleemi leidnud vaid suutnud ka probleemi lahendada.}. 

\question{SCO kohta olid teil raamatud, aga kuidas oli muu infoga? Räägi BBSidest, 
Tarmo Mamers juba jooksis korraks jutust läbi.}

Jah, Tarmo kõrval sai kogemust koguda ja koos 
pitsat süüa, Peetri Pizzas, sest tol ajal Tallinnas väga valikut ei 
olnud, see oli enam-vähem ainus pitsakoht. 

\question{Kas see kohtusite Skriiningu kontoris?}

Jah, Skriiningus\index{Skriining} olin ma sage külaline. Teine BBS, kus ma tihti 
külas käisin, oli Dark Corner\index{Dark Corner}.

\question{Tol ajal oli arvutifirma vist natuke klubi moodi asi. Sõltub 
firmast muidugi, aga kogu aeg käisid inimesed läbi, jõid kohvi ja vahetasid 
uudiseid.}

Tundus vist olevat küll jah. Vähemalt osas kohtades, ka Skriiningus, oli 
niimoodi.

\question{Millised \emph{hot-spot}'id peale Skriiningu veel olid?}

Mina väga palju mujal ei käinud. Microlink oli vist ka millalgi selline koht. 
Dark Corner BBSis – Priit Kasesalu\index[ppl]{Kasesalu, Priit}, Ahti 
Heinla\index[ppl]{Heinla, Ahti} ja kes seal teised olid – sai 
ainult õhtuti ja öösiti käidud, sest neil oli päevatöö ka.
Tarmo juures oli sotsiaalne osa vist pigem ikka ka
õhtupoolikutel, päevasel ajal ei olnud nii palju 
läbikäimist.

\question{See oli see aeg, kui pidi hakkama tööd tegema ja raha teenima. 
Ometi panid sa oma BBSi püsti. Ühel hetkel olid sa MamBoxi \emph{point} 
ja siis tegid enda oma?}

Jah, MamBoxi\index{MamBox} point olin mõnda aega ja siis sai 
firma abiga enda oma tekitatud. Isiklik see ei olnud, ikkagi firma 
riistvara peal ja firma kontoris. Kiirust väga ei olnud, alguses vist 
1200, hiljem 2400. 

\question{Mis selle nimi oli?}

Boksi nimi oli Flying Discs BBS. 

\question{Väga lennukas! Aga miks sa selle tegid?}

Tundus põnev. Eks sealt sai mänge ja muusikat tõmbasin ka, minu MP3ndus 
sai sel ajal alguse.

\question{1200 modemiga MP3 allatõmbamine võtab ju kaua aega!}

Selles mõttes oligi parem pitsa kaasa võtta ja külla minna, efektiivsem kui helistada. Millalgi saime sellise modemi nagu 
Zyxel\index{Zyxel}, mis suutis rohkem välja vilistada, aga vist 
ainult teise Zyxeliga. Siis tuli otsida lähikonnast Skandinaaviast 
kohti, kuhu sai niimoodi helistada.

\question{Nii et siis sai juba kaugekõnet teha?}

Sai jah, mina ei pidanud elama üle seda aega, kus pidi tädile 
telefonis kõigepealt ütlema, et ühendage mind sinna ja tänna. Minul oli EKE 
Projekti arvutuskeskuses välisliin algusest peale olemas. 

\question{Kuidas sa 1993. aastal Tartusse sattusid?}

Sellega oli nii ja naa. Mõnes mõttes oli see mugavustsoonist väljaminek. Kui ma 1989. aastal keskkooli lõpetasin, käisin aasta 
TPIs. Eriala oli vist LI\index{Tallinna Tehnikaülikool!Automaatikateaduskond!LI} ehk arvutid ja 
arvutivõrgud. Pidasin vastu aasta, sest see ei andnud mulle mitte 
midagi. Arvutiaine eksam 
või arvestus tuli teha Pascalis. Kuigi Pascalit\index{Pascal} polnud ma
kunagi õppinud, tegin selle töö esimese kuu lõpuks ära, esitasin õppejõule ja 
rohkem kohal ei käinud. Partei 
ajalugu ei pidanud küll enam õppima, aga see-eest oli füüsika, kus olid 
Rusalepad\sidenote[][-5cm]{Ilmselt peab Margus silmas Ervin ja Maret 
Rusaleppa\index[ppl]{Rusalep, Maret}\index[ppl]{Rusalep, Ervin}.} kahekesi 
vastas, ja sellest ma vist kukkusingi läbi.

\question{Misjärel jõudsid Tartusse?}

Olin kolm aastat nii-öelda tööl: tegelesin igasuguste põnevate või 
vähem põnevate asjadega või mängisin arvutiga. UNIXi peal avastasin enda 
jaoks mängu, millest ma ei ole siiamaani lahti saanud – 
Rogue\index{Rogue}\index{Nethack}\sidenote[][-6.2cm]{Esimene mäng, kus tuli tekstipõhisel ekraanil programmi poolt genereeritud 
koobastikest koosnevas fantaasiamaailmas seiklusi otsida. Peategelase surm 
oli seejuures permanentne ja mängija võis komistada ka oma varasemate 
tegelaste laipadele. Hiljem nimetatigi sedalaadi mänge (polulaarsemad Hack, 
Nethack, Moria, Angband) ühise nimetajaga \emph{roguelike} ehk 
\emph{rogue}'i-sarnased.} või Nethack\sidenote[][-2cm]{Vt ka lk 
\pageref{sisu:nethack}.}.

\question{Mis versiooni\sidenote[][-2.6cm]{Nethack on pigem kultuuriline 
fenomen kui arvutimäng, ka selle lähtekood ja andmefailid on mõnuga loetavad, 
sisaldades viiteid algmüütidele, tsitaate, luulet jne. Ühest küljest tähendab 
see pidevat arengut, kuid teisalt ka seda, et mängijad peavad vaid üht 
konkreetset versiooni selleks \enquote{õigeks}, täpselt nii, nagu suhtutakse 
vahel skepsisega uuematesse „Star Warsi“ filmidesse.} sa mängid?}

Pean tunnistama, et mängin seda ka viimasel ajal päris palju. Võtsin 
eesmärgiks kõikide rollidega lõpuni mängida. Üks roll on veel jäänud, 
\emph{priest}\sidenote[][-.2cm]{Nethackis on 13 rolli, neist igaühel unikaalsed 
võimed ja vaenlased. Juba ühe rolliga mängu läbimine on keeruline ettevõtmine, 
teha mäng läbi kõigi rollidega peale preestri (mis tähendab, et Margus on 
kõikvõimalikele ohtudele vastu astunud ka näiteks turisti rollis, kelle peamine 
võimekus on vastupanu mürkidele) on Nethacki austajate hulgas üsna eepilistes 
mõõtmetes saavutus.}.

Tegelikult on temaga ju lihtne: \emph{priest} näeb kohe ära, kas asja 
saab selga panna või mitte ehk kas see on \emph{blessed} või \emph{cursed}. 
Alguses tundub lihtne, aga võibolla see lihtsus maksabki kätte. Mul on 
temaga olnud ka väga pikki mänge, aga lõpuni ei ole veel jõudnud. Mängin 
viimast versiooni, samas ma kõiki neid kahe käe relvi ja muid 
uuendusi ei kasuta.

\question{Miks Tartu ja matemaatika?}

Seal olime ju koos sinuga ja igasuguste teiste huvitavate tegelastega, nagu Meelis 
Roos\index[ppl]{Roos, Meelis}\sidenote{Meelise lugu algab lk
\pageref{sisu:mroos}.} ja Asko Seeba\index[ppl]{Seeba, 
Asko}\sidenote{Asko lugu algab lk \pageref{sisu:asko}.}, ja meie 
gasell\sidenote{Margus viitab Asko juhitava firma Mooncascade'i saavutustele 
Äripäeva koostatavas gasellettevõtete pingereas.}. Ülo 
Kaasik\index[ppl]{Kaasik, Ülo} on ka tuntud nimi tänapäeval, kuigi mitte küll 
arvutimaailmas.

\question{Põnev seltskond oli tõesti. Aga ikkagi miks just Tartu ja 
matemaatika?}

Üks koolivend keskkoolist, kellega me hästi klappisime ja tänapäevalgi läbi 
käime, oli kohe pärast kooli läinud Tartusse rakendusmatemaatikat õppima. Kui ta oleks 
kaasa kutsunud, võibolla oleksin läinud. Pärast oleme sellest 
rääkinud, et ehk oleks parem olnud, kui oleksin kohe läinud. 
Aga selliseid asju ei tea ette. Valisin informaatika\sidenote{Matemaatikateaduskonnas sai baasteadmiste omandamise järel spetsialiseeruda informaatikale.}, kuna mulle on arvutiasi 
südamelähedane, ja mõtlesin, et äkki sealt koolist saab midagi rohkemat. 

\question{Kas sai?}

Aeg oli edasi läinud ja sealt ikka üht-teist juba sai, kuigi selle 
kooliga ma ka lõpuni ei jõudnud. See oli pikk protsess. Esimesed kolm aastat 
läksid ilusti, olin kõigi asjadega graafikus, aga siis sai raha otsa. Võtsin 
aasta akadeemilist ja tegin tööd, millest märgatava osa ajast olin 
Venemaal Peterburis. Tegime sealsete inimestega koostööd sellesama radari 
teemal. R-Süsteemidel oli teine suund merelokaatorid. Need on selles 
mõttes sarnased, et erilist vahet ei ole, kas kajalokatsioon on õhus või vees, kuigi merepõhja läheb üldiselt keerulisem signaal, kanaleid 
on rohkem.

\question{Kas sa Tartus pidasid BBSi edasi või oli see pausi peal?}

BBSindus jäi Tartus katki. Tartusse minek oli igatpidi
mugavustsoonist väljaminek. Helistamine jäi ära, kuigi järele 
mõeldes oli internetiühendus meil ka alguses ikkagi niimoodi, et tuli 
helistada. Pikka aega oli ka üks telefon kogu aeg modemi 
taga kinni, alguses BBSi taga ja pärast internetis. Püsiühendused tulid 
hiljem.

\question{Kas Tartu Ülikool\index{Tartu Ülikool!Matemaatikateaduskond} sind 
akadeemilisse maailma ei tõmmanud? Neid näiteid oli meie kursuselt ka.}

Ei, väga ei tõmmanud. Kursatöö juhendajaga käisin küll korra ühel 
välisreisil kaasas, eks ta üritas mind sellega meelitada. Norras 
Bergenis oli konverents, kus ta pidi oma pabereid esitama. Ta võttis mu kaasa, sain selle 
eest ilmselt ka mõned akadeemilise maailma \emph{exp}'i punktid, et tema rääkis ja mina vajutasin 
arvutiklahve. Tol ajal ei olnud pulte ega muid vahvaid asju. 

\question{Aga see ei tõmmanud sind?}

Ei tõmmanud. Igasugust vahvat riistvara, 
nagu SUN, oli seal küll \ldots 

\question{Tartu Ülikool oli tollal vist päris hästi varustatud?}

Jah. Ühikas\sidenote{Tartu Ülikooli Tiigi tänava ühiselamu\index{Tartu 
Ülikool!Tiigi ühikas}} me ju alustasime ise internetiga. Asko 
Tiiduma\index[ppl]{Tiidumaa, Asko} oli nii-öelda ühika sysop\sidenote{Asko mäletab, 
et ta oli küll idee tekkimise juures ja võttis hiljem sysopi rolli üle, 
kuid ühika interneti ehitas siiski Aldo Mett\index[ppl]{Mett, Aldo}.}, sest tema 
toas asus sisendpurk. Katusel oli antenn ja tema elas seal antenni 
all. Ega ühikas kaabli vedamine ei olnud lihtne! Kui vales kohas puurid, tuleb terve telliskivi välja!

\question{Lõpetuseks, kuhu see tee on sind tänaseks toonud?}

Tänaseks, või pigem viimaseks peaaegu kahekümneks aastaks, olen maandunud 
pangandusse ja muidugi IT-valdkonda. Kunagi oli see Hansapank, 
nüüd Swedbank. Hansasse tulin 2000. või 2001. aastal ja siin ma nüüd olen.

\question{Panganduses on ju ülesanded hoopis teist 
masti kui radarisignaali lugemine. Millest selline muutus?}

Oli muutus küll ja ega see ei tulnud kergelt. R-Süsteemides oli tore 
seltskond. Meil oli näiteks suviti kombeks arvutitega 
mere äärde minna. Võrk pandi kuskil mujal püsti, internet võeti näiteks kohaliku kooli juurest ja siis sai vaheldumisi tööd teha ja rannas käia . Aga palgaga oli kehvasti, nii et olid majanduslikud probleemid. Töö oli üldiselt 
huvitav, aga kui on väga väike firma, siis tellimusi ei ole kogu aeg. Ma ei 
ole ise müügiinimene, et läheksin ja otsiksin tänavalt tööd. Ja kuna seal 
kippus see pool natuke lonkama, siis ma mõnes mõttes läksin 
lihtsama vastupanu teed. 

Olen vahepeal päris palju baasi 
kirjutanud. Kui panka tööle tulin, siis vestlusel rääkisin, 
et baasi ma ei ole kirjutanud ja väga ei taha ka. Vahepeal hakkas 
see mulle täitsa meeldima, aga nüüd vaatan, et tuul on jälle sinnapoole, et \emph{micro-service}'i maailmas on baas jälle väga \emph{evil}. 

\question{See on vist pikalt arvutitega toimetamise hüve, et näed neid 
tsükleid ja ringe.}

Jah, eks vahepeal ikka kaldutakse äärmustesse ja tõde on seal kuskil 
keskel.