\index[ppl]{Sutt, Margus}

\question{Kust sa pärit oled?}
Ma olen Tallinnast.
                 
\question{Aga kuidas sul seal Tallinnas nende arvutitega oli, kuidas arvutid 
sinu juurde said ja sina arvutite juurde?}

Mul niimoodi pool juhuslikult juhtus nii, et kui mul oli aeg kooli minna, siis 
meie pere parajasti kolis. Ja kolis sellise kooli nagu Tallinna 3. 
Keskkool\index{Tallinna 3. Keskkool} piirkonda. See on tänapäeval 
Lilleküla gümnaasium\index{Lilleküla gümnaasium|see{Tallinna 3. 
Keskkool}}. Ehk, nagu siin juba mitu korda on mainitud, selles koolis oli 
selline tüüp nagu Jaak Loonde\index[ppl]{Loonde, Jaak}.
                 
\question{Jälle Jaak!}

Jah, ma sattusin sinna kooli, esimesed kaheksa aastat põhikooli nagu ei 
juhtunud veel midagi. Ja ka siis, kui oli keskkooli minek, tulid mingisugused 
katsed ja konkursid ja meil tehti sellest aastast alates, kui ma keskkooli 
läksin, juurde kolmas klass (muidu oli kaks keskkooliklassi), mis oli siis 
nii-öelda informaatika-matemaatika eriklass. Aasta oli siis 1985 või 1986.

\question{Lausa informaatika ja matemaatika? Seda vedas siis 
Jack\index[ppl]{Jack}?}

Jah, et see komplekt sinna juurde tuli, organiseeris Jack.

\question{Kas ta oli siis klassijuhataja ka?}

Ei, klassijuhataja ta ei olnud, ajas asju. Kuskilt haridusministeeriumist või 
mis iganes asutused tollal olid, oli ikkagi vene aeg veel, ta igatahes sai 
selle klassi. Ega see klass nüüd ajaliselt vist kaua ei püsinud,  kaks 
komplekti oli seda kindlasti aga kolmandas ma enam väga kindel ei ole, kuna ma 
tulin sealt juba ära. Jack ei pidanud seal koolis enam väga kaua vastu ja peale 
teda see asi hääbus. 

Aga, jah, Jackil oli seal koolis selline tore külmkastide rivi nagu 
MIR-2\index{MIR-2}, kus olid perfolindid ja perfokaardid ja\ldots

\question{See oli tal koolis lausa?}

See oli tema klassiruumis, ruume oli tal muidugi mitu. Tal oli lisaks 
klassiruumile ka mingi raadioruum, mis pärast läkski nagu päris raadioruumiks. 
Meie klassi tegelased ehitasid. Mitte mina, aga põhiaktivist oli selline 
tegelane nagu Andrus Tamboom\index[ppl]{Tamboom, Andrus}. Ta ehitas kooli 
raadiovõrgu, või nii-öelda taastas, sest mingisugused juhtmeid olid vanast 
ajast  seinas, aga ta kohendas seda ja ajas kuidagi käima.

\question{Kuidas perfolindi tingimustes arvutiõpe praktiliselt välja nägi?}

Ega seda õpet seal nüüd väga palju ei olnud. Mingi pool aastat võib-olla oli, 
kui me selle MIR-2-ga\index{MIR-2} tegelesime, edasi läks juba natuke 
advaantsimate asjade juurde. Aga mingisugust parabooli sai ekraani peal 
joonistatud. Perfolindiga oli see, et  oli lugeja,  sa pistsid lindi sinna 
sisse ja kuskilt tuli siis mingi käsk vajutada, ja ta hakkas seda linti hästi 
kiiresti läbi vedama.

\question{Ja kooliõpilased perforeerisid linti või see usaldati kellegi teise 
kätte?}

Arvuti perforeeris ise ka, tal olid mõlemad, nii sisend kui väljund, olemas.

\question{Trükimasin, toksid sisse, tema teeb perfolindi?}

Ja pärast on võimeline seda lugema. 

\question{Kust sul see mõte sündis, et võiks just sinna sellesse paralleeli 
minna, kus informaatikat õpetati?}

Ma olen selle peale mõelnud, aga ega  ei mäleta, miks. Võib-olla vanemad 
torkisid, võib-olla oli endal mingil määral huvi. Matemaatikaga mul probleeme 
ei olnud, iseenesest. Käisin olümpiaadidel, keskkoolis kindlasti juba, ma ei 
tea, kas ka põhikoolis juba, mata nagu väga ei häirinud.

\question{Olümpiaadidel käimise puhul on  \enquote{mittehäirimine} vist natuke 
pehmelt öeldud. Tavaliselt on arvutiõppes kaks osa: see, mis tunnis räägitakse 
ja nii-öelda programmiväline tegevus, mille käigus ise pusitakse. Kuidas teil 
selle vahekorraga oli?}

Selles mõttes tuleb jälle Jack'i\index[ppl]{Jack} tänada. Tema õpetamismeetod 
oli ikka kardinaalselt erinev, kui nüüd nii-öelda klassikaline. Matemaatikat 
tema raamatust ei õpetanud, tal olid mata jaoks tehtud oma töölehed, kus oli 
muu hulgas üritatud tekitada mingeid siirdeid ka teistesse ainetesse. Füüsika, 
eks ole, kõige lihtsam. Keemiat ma ei mäletagi, aga eesti keelt oli näiteks 
kuskil mingites kohtades mainitud. 

\question{Ta õpetas matemaatikat ka?}

Jah, meie klassile ta õpetas matemaatikat ka, järgmisele aastale enam ei 
õpetanud.  

\question{Töölehed ei sobinud?}

No seal oli igasuguseid probleeme. Järgmist aastakäiku ma mainin sellepärast et 
seal õppisid Priit Kasesalu\index[ppl]{Kasesalu, Priit}, Mikk 
Orglaan\index[ppl]{Orglaan, Mikk} (kes on ka juba teada-tuntud) ja Janno 
Ossaar\index[ppl]{Ossaar, Janno}. 

Aga selle õpetamise juurde veel tagasi tulles, siis Jaagu\index[ppl]{Loonde, 
Jaak}  õpetamismeetod oli, jah, suures osas sihukene vette viskamine. Vähemalt 
arvuti poolelt. Et, näe, siin on arvuti, tegelege. Muuhulgas õnnestus tal 
näiteks koos klassijuhataja Tiiu Neemega\index[ppl]{Neeme, Tiiu} viimases 
klassis  minule ja  veel paarile välja rääkida selline asi, et me osades 
tundides ei pidanudki käima. Tegime mingisuguseid semestrite või trimeistrite 
(mis iganes nad seal parajasti olid), lõpus töid ja arvestusi. Selline 
eriprogramm.

\question{Et sellist eriprogrammi võimaldada, pidi nii-öelda akadeemiline 
jõudlus tasemel olema?}

Ilmselt ei olnud, jah sellega probleemi. 

Aga muu arvutitegevusega oli nii. Kõigepealt tulid MSX-id\index{Yamaha 
MSX}, seal Luise tänava ÕTK-s\index{Tallinna Oktoobrirajooni 
Õppetootmiskombinaat}\sidenote{Tallinna Oktoobrirajooni Õppetootmiskombinaat.}. 
Seal sai mingiaeg klassiga käidud, ilmselt hakkasin pärast seal ka ise käima. 
MSX-idega oli tore näiteks see, et Jackil\index[ppl]{Jack} õnnestus need 
vaheaegadeks (suvevaheajaks eriti, ma teisi vaheaegu ma ei mäleta, kas oli) 
kolmandasse kooli nii-öelda laenata.
          
\question{Soh. Sai nad siis kindlasse kohta ära lohistatud!}

Jah, ligipääs oli hulga parem, sest kolmandas koolis\index{Tallinna 3. 
Keskkool} mingi aeg mul oli  lisaks Jacki ruumi võtmele ka kooli välisvõti. Ehk 
keskkooli ajal tekkis võimalus suvevaheajal ööpäev ringi arvutis olla.

\question{Mis te tegite nende arvutitega? Jutte kuulates tundub olema kahte 
liiki inimesi: need, keda huvitas rohkem programmeerimine ja need, keda tõmbas 
mängude poole. Kumb sul rohkem domineeris?}

Esialgu kindlasti mängimine. Väga täpselt ei mäleta, aga 
Yamahal\index{Yamaha MSX} olid head ilusad mängud, eks need meelitasid 
oma graafika ja sellega, et võrreldes MIR-iga oli ta ikka hoopis teine maailm. 
Isegi võrreldes Agatiga\index{Agat}. Ma ei mäleta, kas  keskkooli 
esimesel või teisel aastal tekkis meile üks oma kooli\index{Tallinna 3. 
Keskkool} Agat, mis kogu aeg koolis oli. 

Progemise poole pealt oli mul Agatist kindlasti rohkem kasu, sest seal käis 
selline tegelane nagu Tarmo Mamers\index[ppl]{Mamers, Tarmo} vahest  asja 
uurimas ja üle tema õla kiibitsedes õnnestus mul ka üht-teist-kolmandat 
omandada.

Agati peal oli Basic\index{BASIC}, aga sai muuhulgas ka juba assemblerit 
vaadatud.

\question{Räägi korra sellest Agatist palun. Ta oli Apple II kloon?}

Ta oli Apple II\index{Apple II} kloon, mille sees  oli originaal 
Apple'i \emph{chip}, kust oli info maha kraabitud. Vähemalt see konkreetne oli 
selline.

\question{Sest mina mäletan Agati pealt mingeid veidrusi, mida ma ei usu, et 
Apple'l oleks olnud. Tundus, et see OS oli seal Nõukogude oma?}

Vot ei mäleta nii täpselt, sest ega ma päris Apple'it ei olegi puutunud, mul ei 
ole võrdlust. Ma tean, et Tarmol\index[ppl]{Mamers, Tarmo} olid Apple pealt 
mingid väljatrükid BIOS-i ja BIOS-i \emph{call}-ide kohta ja ta üritas seda 
Agati peal rakendada.
                 
\question{See on juba oluline teadmine, et Apple-i \emph{call}-id võiksd 
töötada!}

Jah, muu hulgas oli mul ka näiteks selline nii-öelda huviprojekt, et sai 
üritatud Apple peale Norton Commander-it\index{Norton Commander} kirjutada. 
Asmis, loomulikult.
                 
\question{Huvitav, minul oli samasugune projekt Juku peal\ldots}

See tee vist on, jah, aastate jooksul kõigil läbi käidud. 

\question{See toobki järgmise küsimuse juurde, et kust teile  ülesanded tulid. 
Arvutiga on ju see, et kui sa valid lahendamiseks liiga lihtsa ülesande, siis 
see ei ole huvitav. Aga kui sa valid liiga keerulise ülesande, siis sa algajana 
jooksed ennast enne sodiks,  kui üldse mingit  lootusekiirt paistma hakkab. 
Kuidas teil see oli, kas Jack juhendas või ise mõtlesid või kuidas see käis?}
                 
Ei, Jack\index[ppl]{Jack} selles mõttes küll väga ei juhendanud. Tema antud 
programmid olid pigem  matemaatikateemalised. Ja ega Jack ilmselt ei 
küündinudki sellise juhendamiseni. Võib-olla oleks küündinud, aga ei suvatsenud 
või ei pidanud vajalikuks. Tuli ise vaadata, et mida sa teed. 

Mängudega seoses, eks ma mingeid spraite või mingeid selliseid asju ikka 
püüdsin liigutada. Ja mida ma mäletan,  et ma ka ehitasin, oli see, et võtsin 
sinusoidi ja nii-öelda nihutasin ta ruumi. Kui sa tekitad mõlema koordinaadi 
suhtes  väikse nihke, eks ole, siis tekib sihuke ruumiline efekt. Ja kui sa 
paned programmi tagant kustutama ja eest uuesti joonistama, on tulemuseks 
sihuke nagu tänapäeva \emph{screensaver}. Aga tol ajal oli see huvitav.
                 
\question{Programmeerimise mõttes tore ülesanne. Kas arvuti-huvi juurde käis ka 
mingisugune spetsiifiline muusika- või kirjandushuvi? Päris mitmed on rääkinud, 
kuidas nad on Asimovi ja Gibsoni peal kasvanud selle koha pealt.}

Ma olen lugenud küll fantastikat ja \emph{sci-fi}-id ja fantaasiat aga ma ei 
tea, kui palju see nüüd arvutiga seotud on. Pigem on nad selles mõttes nagu 
konfliktis, sest  võitlevad mõlemad aja pärast. Enne arvuti-aega ma lugesin 
väga palju ja väga kiiresti, kõik need seiklusjutud maalt ja merelt. Mis  kätte 
sai, kõik sai läbi loetud. Aga kui arvuti tuli,  ma enam vahepeal nii palju ei 
lugenud. Kuigi nüüd siin, umbes 10 aastat tagasi, ma võtsin kätte ja lugesin 
läbi enamuse  McCaffrey-st\sidenote{Margus peab ilmselt silmas Ameerika-Iiri 
ulmekirjanikku Anne McCaffrey-d. Juba ainuüksi tema Loheratsurite sarjas on 23 
romaani ja see on vaid üks paljudest selle kirjaniku poolt loodud 
maailmadest.}. Alustades lohelugudest, neist kaks on isegi eesti keeles olemas.

\question{Mõned on rääkinud, et juba keskkooli ajal kippus töö tegemiseks 
minema. Sul ei olnud nii?}
                 
Ei, töö tegemiseks ei läinud, kuigi  sidemed esimese töökohaga juba tekkisid.

\question{Mis see esimene töökoht oli?}

Töökohta õige paremini iseloomustab Raivo Rebase\index[ppl]{Rebane, Raivo} 
nimi. Ma ei tea, kuidas tema Jackiga seotud oli, aga igatahes kuidagi ta oli. 
Mingi hetk oli ta Jacki juures kohal ja põhimõtteliselt otsis jüngreid. 
Nii-öelda \emph{head-hunting}.

\question{Vaata, kus! Tänapäeval keskkoolis vist päris ei käida otsimas!}

Ma ei tea, kuidas see niimoodi. Eks ta vist plaanis  oma arvutifirmat teha, tol 
hetkel tal seda veel ei olnud. Ta tegutses Küberis\index{Küber} sellise 
härrasmehe nagu Raul-Roman Tavasti\index[ppl]{Tavast, Raul-roman}  juures. Neil 
oli ka vist mingi firma seal Küberi kõrval. Ma mäletan seda nii palju selle 
pärast, et seal ma sain tõenäoliselt esimest korda PC-d katsuda. Peale seda, 
kui see värbamine aset leidis, saime me paari-kolmekesi koolist hakata  käima 
kuskil Küberi majas (ma ei mäleta, mis hoones), kus olid mingid PC-d. Vist 
lausa 386-d, kus oli neli mega mälu. Ehk siis, kui ma tegin ilusti selle 
\emph{boot floppy}, kus oli mäludrive peal, siis ma sain ikka väga palju 
kõvakettaruumi, kus jooksutada Turbo C \emph{editor}-i.
                 
\question{Ohoh, räägi lähemalt, Turbo C on ikka juba meeste vahend, kuidas sa 
selleni jõudsid?}
                 
Kusjuures jälle ma  mõtlesin, et kuidas ma selle C\index{C} juurde 
jõudsin, sest Pascalit ma ei ole kunagi õppinud, aga kuidagi mul tekkis see 
Turbo C. Agati peal teda ei saanud olla, ilmselt seal Rebase kaudu ta kuidagi 
tuli. 

Kas Rebane või\ldots Rebane kindlasti, aga ma ei tea, millal, pistis mulle 
mingisuguse kopeeritud Kernighan-Ritchie\sidenote{Kuulus valgete kaante ja 
sinise C-ga raamat, vt. \pageref{sisu:richie}.} pihku. Selle ma neelasin siis 
mõne nädalaga ilmselt läbi, väga kaua sinna ei läinud.  

\question{Kas MSX-i ja Agati pealt C peale üle minek keeruline ei olnud? 
Pointerid ja värk ja igasuguse turvavõrgu puudumine?}

Ma ei mäleta, kuidas see see täpselt oli. Seal Rebase juures ma käisin ja mingi 
hetk ta lõi  Küberist lahku. Järgmine koht oli Liivalaia tänaval selles majas, 
kus praegu on Swedbank\sidenote{Liivalaia 8, Tallinn}.  Seal 12. korrusel, kus 
praegu on nii-öelda ülemuste korrus, oli arvutuskeskus, ETK või ETKVL või 
mingisugune\sidenote{Margus peab ilmselt silmas sel aadressil asunud EKE 
Projekti nimelist asutust.}. Seda vaadet ma nautisin üheksakümnendatel, ilus 
vaade oli. Pärast õnnestus peaaegu samasse kohta tagasi kolida, üheksandele 
korrusele. Sinna  sai Rebane ruumid. Arvutuskeskuses oli ka mingisugune suur 
nii-öelda kast-arvuti, millega meie õnneks kokku ei puutunud, meil olid  oma 
PC-d kus, kus muuhulgas sai hakatud näiteks ka Unixiga\index{Unix} tegelema.

\question{Kuidas te PC peal Unixit tegite?}

SCO\index{SCO UNIX} ja BSD\index{BSD} olid sel ajal olemas.

\question{Aga kuidas need tol ajal Eesti Vabariiki jõudsid?}

Ma ei tea, kuidas nad jõudsid, see ei olnud minu teha. Kuidagi oli SCO, kuidagi 
oli BSD. Äkki nad olid isegi kuskilt ostetud, SCO-l olid  ikka nagu originaal 
kirjade plaadid mingist hetkest, ma mäletan. Flopid ikka, flopid pidid olema, 
plaadid tulid hiljem. 

\question{Aga millega te seal tegelesite? Firmal pidi ju äri olema?}

Selle äriga oli  alguses kehvasti. Ega ma näiteks ei mäleta, millal ma  palka 
hakkasin saama. Alguses ei olnud väga äri, aga mingi hetk nagu läks käima. Ja 
käima läks  huvitav äri, radarid. Radaritega meil oli koostöö Vene firmadega, 
ühe seltskonnaga Peterburist. Nemad tegid  radarile riistvara, tegid kaardi, 
mis läks PC sisse, ja meie ehitasime sinna peale softi.
           
\question{Radarid on ju sõjaväe ja saladus ja puha?}      

Ei, sõjaväge me väga ei puutunud, me tegime nagu tsiviilsuuna peale. Pulkovo 
radarijaamas sai näiteks korduvalt käidud.

\question{Teie soft siis võttis mis iganes signaali ta radari käest sai ja 
joonistas kaardi peale mummud?}

Jah. See käis alguses BSD peal. Ei mäleta, kui kaugele me SCO-ga jõudsime, 
võib-olla ta alguses käis isegi SCO peal. Aga BSD oli kindlasti vahepeal ja  
lõpuks läksime Linuxi\index{Linux} peale, kui see tekkis ja oli juba niivõrd 
kobe, et sai kasutada.
                 
\question{Alles äsja joonistasid ekraani peale siinust ja nüüd loed radari 
pealt signaale, siin tundub mingisugune lünk olema? Ülesande keerukus on ju 
palju suurem?}

Ega seda  kõike mina ei teinud, eks ole. Seal oli ikkagi grupp inimesi taga.

\question{Kui suur see grupp oli? Kümmekond? Kakskümmend?}

Alla kümne,  ei olnud väga suur. See meie firma ei läinud kunagi väga suureks. 
Seal olime mina,  Raivo Rebane\index[ppl]{Rebane, Raivo}, Mart 
Rüütel\index[ppl]{Rüütel, Mart} (kes on vist praegu ka veel seal). Firma nimi  
on praegu R-Süsteemid\index{R-Süsteemid}, mingi aeg oli ta Virumaa 
Tiivad\index{Virumaa Tiivad|see{R-Süsteemid}}, sest ta tegeles natuke ka 
lennundusega. Mingi aeg oli meil tööl üks lennundusfänn ja  muu hulgas sai 
korraldatud mingisugune väikelennukite ülelend või üritus,  mille raames mul 
õnnestus näiteks sõita Kuressaare lennujaamast Viljandi lennujaama mingi Piperi 
peal. Muidugi, kuna need, kelle lennukiga ma lendasin, olid Saksamaalt, siis 
ega nemad ka ei teadnud, kus lennujaam seal täpselt on ja siis selle Viljandi 
lennujaama otsimisega oli natukene tegemist. Et \enquote{kuhu me nüüd siis 
maandume?}
                 
\question{Sihuke värk! Kes see klient oli? Needsamad venelased?}

See soft oli vist mingi aeg Tallinnas ka kasutusel. Softis olid mõlemad, nii 
primaar- kui sekundaarradar. Primaar on see nii-öelda \enquote{loll} radar, 
tema signaal põrkab lihtsalt kuskilt tagasi või siis ei põrka. Ja 
sekundaarradar on siis see info, mille lennuk  ise välja saadab oma  
transponderiga.
                 
\question{Radari juures on tihtipeale see oluline asi, et ta peab töötama. 
Kuidas te seda tegite, et teie soft robustne oli ja töötas?}

Eks seal oli mitu kihti nagu loogikat peal jah. Aga radaritele sai isegi 
mingisuguseid vahvaid maatriks-algoritme kasutatud. Üks tegelane, kelle nime ma 
paraku ei mäleta (ta eesnimi võis äkki Kaido olla?) tegi nendest lausa TPI 
lõputöö. Ja nagu nende lõputöödega ikka vahest on, et kuskile pannakse mingi 
\emph{catch} sisse lootuses, et niikuinii keegi ei loe. Ja nende algoritmidega 
sai täiesti juhuslikult niimoodi, et see koht koodis, kus seda kõike pidi nagu 
välja kutsutama, oli pikka aega välja kommenteeritud. Pärast tema järeltulijate 
poolt avastati mingi hetk, et oleks hulga parem, kui seda funktsionaalsust 
kasutataks ka.
                 
\question{Radari riistvaraga suhtlemine pidi ju keeruline olema?}

Riistvara suhtlemisel oli ajakriitilisus mängus. Tänu sellele tuli paratamatult 
 kerneli alasse ronida, kuna signaal liigub kiiresti ja sul on vaja täpselt 
teada, millal see signaal sisse tuli. Ja sealt omatehtud kaardi pealt tuli 
puhvrid (mis ei olnud küll suure tol ajal)  kähku ära lugeda.

\question{See oli SCO ajal, eks?}

See algas SCO\index{SCO UNIX} ajal,  selle jaoks minu meelest olid meil 
ametlikud raamatud ja ametlik \emph{dev kit} ostetud.
                 
\question{Neid inimesi, kes seda oskasid, ei saanud Eestis palju olla?}

Jah, ega väga küsida kellegi käest ei olnud. Esimene inimene, kellele ka asi 
huvi pakkus ja kes midagi teemast teadis ja suutis teemas kaasa rääkida, oli 
aastaid hiljem (kui ma 1993. aastal Tartusse jõudsin) Meelis 
Rools\index[ppl]{Roos, Meelis}. Aga ta on minust muidugi kõvasti ette jõudnud, 
sest ta on praeguseks päris palju kerneli \emph{patch}-e \emph{post}-inud . 

\question{SCO osas olid teil raamatud ja asjad aga muu info? Räägime BBS-idest? 
Tarmo Mamers juba jooksis läbi\ldots}

Jah, eks Tarmo körval kogemust kogudes ja vaadates, mis ta seal teeb ja koos 
pizzat süües. Peetri pizzat\sidenote{Umbes sel ajal Tartus toimetades oli ka 
seal Peetri pizza üks igati kuum koht. Tänapäevases mõttes pizzaga polnud tol 
rasvasel jahutootel suurt midagi pistmist aga, veidral kombel, ei tekkinud 
neile ka ühtegi konkurenti.}, sest tol ajal muidugi Tallinnas väga valikut ei 
olnud, see oli enam-vähem ainus pizza koht. 

\question{See toimus Skriiningu kontoris?}

Jah, Skriiningus\index{Skriining} ma olin külaline. Ja teine BBS, kus ma tihti 
külas käisin, oli Dark Corner\index{Dark Corner}.
                 
\question{Tol ajal oli arvutifirma vist natuke nagu klubi moodi asi. Sõltub 
firmast muidugi, aga kogu aeg käisid inimesed läbi, jõid kohvi ja vahetasid 
uudiseid?}

Ta tundus vist olevat küll, jah. Vähemalt osades kohtades, ka Skriiningus, oli 
küll niimoodi.

\question{Mis need \emph{hot-spot}-id veel olid peale Skriiningu?}

Mina väga palju mujal ei käinud. Microlink vist oli ka millalgi selline koht. 
Dark Corner BBS, Priit Kasesalu\index[ppl]{Kasesalu, Priit} ja Ahti 
Heinla\index[ppl]{Heinla, Ahti} ja kes seal teised olid, nende juures sai 
ainult õhtuti ja öösiti käidud, sest neil vist oli seal päeval töö ka.
Tarmo juures ka, tegelikult ikkagi see sotsiaalne osa oli vist pigem ikka 
õhtupoolikul, päevasel ajal võib-olla ei olnud nii palju seda sotsiaalset 
läbikäimist.
                 
\question{See oli see aeg, hakkas juba olema vaja tööd teha ja raha teenida. 
Ometi ühel hetkel sa panid omale BBS-i püsti. Sa olid MamBoxi point mingi hetk 
ja siis tegid enda oma?}
                 
Jah, MamBox\index{MamBox}-i juures ma olin point mõnda aega ja siis sai 
firma abiga enda oma tekitatud. Isiklik ta ei olnud, ta oli ikkagi firma 
riistvara peal, firma kontoris. Kiirust väga ei olnud, alguses oli vist isegi 
1200, mingi hetk sain 2400-se. 

\question{Ooo! Mis su purgi nimi oli?}

Boksi nimi oli Flying Discs BBS. 

\question{Väga lennukas! Aga miks sa seda tegid?}

Tundus põnev. Eks sealt sai mänge. Muusikat ma tõmbasin ka, minu empeekolmendus 
sai sel ajal alguse.
                 
\question{1200-se modemiga MP3-e alla tõmmata võtab ju hirmsa aja!}

Selles mõttes oligi parem nagu külla minna. Võtta pizza kaasa ja minna külla 
oli efektiivsem, kui helistada. Mingi hetk me saime sellise modemi nagu 
Zyxel\index{Zyxel}, see suutis juba natuke rohkem välja vilistada aga vist 
ainult teise Zyxeliga. Siis tuli jälle otsida lähikonnast Skandinaaviast 
selliseid kohti, kuhu nii helistada sai.

\question{Ah juba kaugekõnet sai teha?}

Kaugekõnet sai teha jah, mina ei elanud üle seda aega, kus pidi tädile 
telefonis kõigepealt ütlema, et ühendage mind sinna ja tänna. Minul seal EKE 
projekti arvutuskeskuses oli  välisliin ikka algusest peale olemas. 

\question{Kuidas sa, lilleke, siis 1993. aastal Tartusse sattusid?}

Sellega oli nii ja naa. Mõnes mõttes, eks ta sihukesest mugavustsoonist välja 
minek oli, muidugi. Kui ma 1989. aastal  keskkooli lõpetasin, käisin aasta ikka 
TPI-s ka. LI\index{Tallinna Tehnikaülikool!LI} oli äkki? Arvutid ja 
arvutivõrgud? Seal ma pidasin vastu aasta, sest see ei andnud mulle väga mitte 
midagi. Või noh, ütleme, ei andnud kohe üldse mitte midagi. Arvutiaine eksam 
või arvestus tuli teha Pascalis. Kuigi Pascalit\index{Pascal} ma polnud 
kunagi õppinud, tegin ma selle töö esimese kuu lõpuks ära, andsin ära, ja 
rohkem kohal ei käinud. Ehk ühesõnaga täiesti mõttetu minu jaoks. Partei 
ajalugu enam pidanud küll õppima, aga see-eest oli  mingi füüsika, kus olid 
Rusalepad\sidenote[][-5cm]{Ilmselt peab Margus silmas Ervin ja Maret 
Rusaleppa\index[ppl]{Rusalep, Maret}\index[ppl]{Rusalep, Ervin}.} kahekesi 
vastas, sellest ma vist kukkusingi läbi.

\question{Misjärel sa jõudsid Tartusse?}

Ma olin, jah, kolm aastat nii-öelda  tööl, tegelesin igasuguste põnevate  või 
vähem põnevate asjadega või mängisin arvutiga. UNIXi peal ma avastasin enda 
jaoks sellise mängu, millest ma ei ole siiamaani lahti saanud, 
Rogue\index{Rogue}\index{Nethack}\sidenote[][-6.2cm]{Rogue oli 
esimene mäng, kus tekstipõhisel ekraanil protseduurselt genereeritud 
koobastikest koosnevas fantaasimaailmas tuli seiklusi otsida. Peategelase surm 
oli seejuures permanentne ning mängija võis komistada ka oma varasemate 
tegelaste laipadele. Hiljem nimetatigi sedalaadi mänge (polulaarsemad Hack, 
Nethack, Moria, Angband) ühise nimetajaga \enquote{roguelike} ehk 
\enquote{rogue-sarnased}.} või Nethack\sidenote[][-2cm]{Vt. ka lehekülg 
\pageref{sisu:nethack}.}.

\question{Ooo! Mina jõudsin Nethackini hiljem kuid mängin samuti aega-ajalt 
siiamaani. Mis versiooni\sidenote[][-2.6cm]{Nethack on pigem kultuuriline 
fenomen kui arvutimäng, ka tema lähtekood ja andmefailid on mõnuga loetavad 
sisaldades viiteid algmüütidele, tsitaate, luulet jne. Ühest küljest tähendab 
see pidevat arengut kuid teisalt ka seda, et mängijad peavad vaid üht 
konkreetset versiooni selleks \enquote{õigeks} täpselt nii, nagu suhtutakse 
vahel skepsisega uuematesse Star Wars filmidesse.} sa mängid?}
       
Pean tunnistama, et ma mängin seda ka viimasel ajal päris palju. Võtsin omale 
eesmärgiks ta kõikide rollidega lõpuni mängida. Üks roll on veel jäänud, 
kellega ma ei ole lõpuni jõudnud, see on 
\emph{priest}\sidenote[][-.2cm]{Nethackis on 13 rolli, neist igaühel unikaalsed 
võimed ja vaenlased. Juba ühe rolliga mängu läbimine on keeruline ettevõtmine, 
teha mäng läbi kõigi rollidega peale preestri (mis tähendab, et Margus on 
kõikvõimalikele ohtudele vastu astunud ka näiteks turisti rollis, kelle peamine 
võimekus on vastupanu mürkidele) on Nethacki austajate hulgas üsna eepilistes 
mõõtmetes saavutus.}.

Tegelikult on temaga ju lihtne. Vaata, \emph{priest} näeb kohe ära, et kas asja 
saab selga panna või mitte, et kas see on \emph{blessed} või \emph{cursed}. 
Alguses nagu tundub lihtne, aga võib-olla see lihtsus  maksabki kätte. Mul on 
temaga olnud ka väga pikki mänge, aga ei ole veel lõpuni jõudnud. Mängin 
viimast versiooni, ma küll kõiki neid igasuguseid kahe käe relvi ja muid 
uuendusi ei kasuta, aga mängin viimast versiooni.

\question{Nüüd tuleb kiiresti Tartu juurde tagasi tulla! Miks Tartu ja 
matemaatika?}

No seal me olime koos sinuga ja koos igasugu huvitavate tegelastega. Meelis 
Roos\index[ppl]{Roos, Meelis}\sidenote{Meelise lugu algab leheküljelt 
\pageref{sisu:mroos}.} sai mainitud. Asko Seeba\index[ppl]{Seeba, 
Asko}\sidenote{Asko lugu algab leheküljelt \pageref{sisu:asko}.}, eks ole, meie 
gasell\sidenote{Margus viitab Asko juhitava firma Mooncascade saavutustele 
Äripäeva koostatavas gasellettevõtete pingereas.}. Ülo 
Kaasik\index[ppl]{Kaasik, Ülo} on ka tuntud nimi tänapäeval, mitte küll 
arvutimaailmas.

\question{Põnev seltskond oli tõesti. Aga ikkagi. Miks just Tartu ja miks just 
matemaatika?}

Mul üks koolivend keskkoolist, kellega me hästi klappisime ja tänapäevalgi läbi 
käime, oli kohe pärast kooli läinud Tartusse rakmatti õppima. Kui ta oleks 
kaasa kutsunud, võib olla oleksin läinud. Pärast me oleme sellest nagu 
rääkinud, vaaginud, et võib-olla oleks parem olnud, kui oleksin kohe läinud. 
Aga selliseid asju ei tea ette. Informaatikasse, kuna mul see arvuti-asi on  
südamelähedane, siis mõtlesin, et äkki sealt koolist saab midagi rohkem. 

\question{Kas sai ka?}

Eks aeg oli ka edasi läinud, sealt ikka üht-teist juba sai. Kuigi ega selle 
kooliga ma ka lõpuni ei jõudnud. See oli pikk protsess. Esimesed kolm aastat 
läksid ilusti, olin kõigi asjadega  graafikus, aga siis sai raha otsa. Võtsin 
aasta akadeemilist ja tegin tööd, millest ma päris märgatava osa ajast olin 
Venemaal, Peterburis. Tegime sealsete inimestega koostööd, seesama radari 
teema. R süsteemidel oli teine suund merelokaatorid. Need on sarnased selles 
mõttes, et kas sul on nüüd see kajalokatsioon on õhus või vees, et ega seal 
palju vahet ei ole. Kuigi merepõhja läheb üldiselt keerulisem signaal, kanaleid 
on rohkem.

\question{Kas sa Tartus pidasid BBS-i edasi või oli see pausi peal?}
                 
BBS-i asi jäi Tartusse minnes katki küll. See Tartusse minek oli ikka väga 
mugavustsoonist välja minek. Helistamine jäi nagu ära, kuigi kui ma järele 
mõtlen, siis internetiühendus oli meil ka alguses ikkagi niimoodi, et tuli 
helistada. Pikka aega oli ka seesama asi, et üks telefon on kogu aeg modemi 
taga kinni. Alguses siis BBS-i taga ja pärast internetis. Püsiühendused tulid 
kunagi hiljem.
                 
\question{Kas Tartu Ülikool\index{Tartu Ülikool!Matemaatikateaduskond} sind 
akadeemilisse maailma ei tõmmanud? Neid näiteid oli ka meil ka kursa pealt?}

Ei, väga ei tõmmanud. Kursatöö juhendajaga ma käisin küll korra ühel 
välisreisil kaasas, eks ta üritas mind ilmselt sellega meelitada. Norras, 
Bergenis, oli mingisugune konverents, kus ta pidi oma pabereid esitama. 
Akadeemilises maailmas on see, et sa pead esitama mingeid asju, et saada 
mingeid punkte, \emph{exp}-i ja \emph{level}-it. Ta võttis mu kaasa, sain selle 
eest ilmselt ka mõned \emph{exp}-i punktid, et tema rääkis ja mina vajutasin 
arvutiklahve. Sest tol ajal ei olnud igasuguseid vahvaid pulte ja asju. 

\question{Aga see sind ei tõmmanud?}

Ei, see mind väga ei tõmmanud. Oli küll jah, igasuguseid vahvaid riistvarasid 
nagu SUN-id ja\ldots 

\question{Tartu Ülikool oli vist päris hästi varustatud tollal?}
                 
Ja. Ühikas\sidenote{Tartu Ülikooli Tiigi tänava ühiselamu\index{Tartu 
Ülikool!Tiigi ühikas}} me ju alustasime ise internetiga. Asko 
Tiiduma\index[ppl]{Tiidumaa, Asko} oli meil nii-öelda ühika sysop\sidenote{Asko mäletab, 
et ta oli küll idee tekkimise juures ja võttis hiljem sysopi rolli üle, 
kuid ühika interneti ehitas siiski Aldo Mett\index[ppl]{Mett, Aldo}}, sest tema 
toas asus see sisendpurk. Katusel oli antenn, ma arvan, ja tema  elas seal antenni 
all. Aga jah, kaableid vedada ja ühikat lõhkuda\ldots Ega see ühikas kaabli 
vedamine ei ole lihtne! Kui vales kohas puurid, tuleb terve telliskivi välja!

\question{Lõpetuseks, kuhu see tee sind tänaseks toonud on?}

Tänaseks, või ütleme viimaseks peaagu kahekümneks aastaks, olen ma maandunud  
pangandusse,  IT-valdkonda ikka muidugi. Kunagi oli selle firma nimi Hansapank, 
nüüd on ta Swedbank. Hansasse ma tulin 2000 või oli see 2001, kuskil seal 
kandis ja nüüd ma olen siin olnud.

\question{Üritades nüüd norida, et panganduses on ju ülesanded hoopis teist 
masti, kui kuskilt radari signaali lugemine. Millest selline muutus?}

Oli muutus küll ja ega see muutus ei olnud kerge tulema. Ühelt poolt olid  
muidugi majanduslikud probleemid, sest R süsteemide seltskond oli tore 
seltskond, meil oli näiteks suveti kombeks, et võeti arvutid ja mindi kuskile 
mere äärde. Pandi võrk kuskil mujal püsti, kohaliku kooli juurest  või kust sai 
võeti mingi internet ja siis käisid vahepeal rannas ja siis tegid tööd ja 
selles mõttes oli tore. Aga palgaga oli nagu kehvasti. Töö oli üldiselt 
huvitav, aga kui on väga väike firma, siis kogu aeg ei olnud tellimusi. Ma ei 
ole ise selline müügiinimene, et ma läheks, otsiks tänavalt tööd. Ja kuna seal 
firmast see pool kippus natuke nagu lonkama, siis ma mõnes mõttes läksin 
lihtsama vastupanu teed. 

Et kuhu ma tänapäevaks olen jõudnud. Ma olen vahepeal päris palju baasi 
kirjutanud. Kui ma panka tööle tulin, siis ma mäletan, et vestlusel rääkisin, 
et baasi ma ei ole kirjutanud ja et ma väga ei taha ka. Aga vahepeal hakkas 
baas mulle  täitsa meeldima, aga nüüd ma vaatan, et tuuled on jälle sinna poole 
läinud, et \emph{micro-service} maailmas on baas jälle väga \emph{evil}. 

\question{See vist on pikalt arvutitega toimetamise hüve, et sa näed neid 
tsükleid ja ringe.}

Jah, eks vahepeal ikka kaldutakse äärmustesse ja eks see tõde on seal kuskil 
keskel.
