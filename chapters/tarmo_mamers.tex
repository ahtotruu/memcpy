\index[ppl]{Mamers, Tarmo}

\question{Kuidas sina arvutite juurde said?}

Mul oli üks klassikaaslane, kelle isa töötas Küberneetika 
Instituudis\index{Küberneetika Instituut!Juhtimissüsteemide osakond}. Millalgi
keskkooliaastate alguses, aastatel 1983--1985 
käisin seal mitu korda tutvumas Apple 
IIga\index{Apple II}. See oli mõnevõrra keeruline, sest Apple oli 
üsna koormatud, kuna seda kasutati
teadus- ja uurimistööks. 
Monitori asemel oli sel tavaline telekas ja mäletan, et kui talvel sinna esimesi kordi sattusin, kasutati seda põhiliselt 
suusahüppe MMi jälgimiseks. Nii et päris iga kord ei saanud 
arvutit näppida või kui sai, siis ilma pildita. 

\question{Mida sa tegid arvutiga?}

Kõigepealt vaatasin, mida sõbrad teevad, põhiliselt mängisid nad igasuguseid 
huvitavaid mänge. Kui hakkasin ise arvutit
näperdama, siis mind huvitas pigem see, et kui arvutimängud on 
teatud hulga elude, relvade ja 
abivahenditega, siis kas neid nii-öelda ära häkkida ei 
saaks, et oleks veel rohkem elusid ja abivahendeid ning 
saaks rohkem punkte. Mind huvitas mängude ümbertegemine. 

Ma ei teadnud midagi programmeerimisest ega ka arvutite 
tööpõhimõttest, see oli esimene ajendav tegur, mis tõi kokkupuute esiteks 
BASICu\index{BASIC} keelega ja järgmiseks Apple 
assembleriga\index{Assembler}. Tundsin tollal huvi ka 
elektroonika, eriti digitaalelektroonika vastu ja juhtumisi jäid mulle 
ette selle Apple'i manuaalid, mis sisaldasid ka 
elektroonikaskeeme. Ajasin näpuga järge, kuidas bitid liiguvad, kui midagi 
printida või nuppu vajutada, ja kuidas ekraani peale tekitatakse 
mälus olevatest bittidest pilt. See oligi algus.

\question{Kas mängud olid väljastpoolt tulnud või liikus ka isetehtud asju?}

Apple'is olevad mängud olid ilmselt tulnud välismaalt sama kanalit pidi, kust 
arvutid isegi Nõukogude Liitu tulid. Osa huvilisi, eeskätt 
Apple'i kasutajad Tartus või Nõos, kellel oli kontakte teiste Apple'i kasutajatega mujal 
maailmas, said vist neilt ka.

\question{Tehti ju ise ka algelisi mänge, näiteks ühes Juku\index{Juku} 
mängus sai mõisa majandada.}

Ma sellest ajast ei mäleta kodukootud mänge, 
küll aga mõnevõrra hilisemast ajast, kui mul oli kasutada Nõukogude 
päritolu arvuti 
Iskra-226\index{Iskra!Iskra-226}\sidenote{\begin{russian}Искра 
226\end{russian} oli Nõukogude Liidus toodetud arvuti Wang 2200 kloon, mis oli 
originaaliga binaarkoodi mõttes sajaprotsendiliselt ühilduv. 
Iskra-226 sisemine struktuur oli siiski oluliselt erinev ja sisaldas mitmeid 
täiendusi, mis muutsid selle sobilikumaks tööstusrakendusteks.}. Seal oli küll 
isetehtud mänge, mille idee oli võetud kuskilt mujalt, 
või siis oli mõni originaalne mänguidee arvutimänguks vormistatud. 

Venelastel oli Apple'i kloon, mille nimi oli Agat\index{Agat}. 
Kui neid hakkas Eestisse tulema, siis nende jaoks oli muidugi vene 
mänge. Osa olid selgelt Apple'i pealt maha lükatud ja osa olid 
\enquote{originaalid}. Sellest ajast ei mäleta ma samuti kodumaist päritolu mänge.

\question{Kas need, kes Apple II päriselt kasutasid, lasid sul 
sellele niisama lihtsalt kõhu alla vaadata ja kaant maha kruvida?}

Küberneetika Instituudis\index{Küberneetika Instituut} väga palju 
ei lastud, sest seal oli oluline ikkagi see, et masin oleks töökorras ja igal 
ajal uurimistööks kasutatav.

Hiljem keskkooliajal sattusin ka tollasesse TPIsse ehk 
praegusesse Tehnikaülikooli\index{Tallinna
Tehnikaülikool!Automaatikateaduskond!Raadiotehnika kateeder}, kus raadiotehnika 
kateedris oli ka üks Apple II\index{Apple II}. Seal olid 
raadiotehnikud, kelle igapäevane leib oligi vaadata pigem 
seda, mis kõhus on ja kuidas see käib. Seal sai masina sisse vaadata, 
kõrval jootekolb, millega sai pädevam seltskond teha ise Apple'i 
jaoks perifeeriakaarte\sidenote{St. arvutisse käivaid trükiplaate, mis võimaldasid arvutil suhelda perfeeriaseadmetega.}, et TPI majas midagi juhtida või mõõta. 

\question{Kas tollal oli üsna hästi teada, kus mõni Apple II 
või Iskra saadaval oli?}

Jah, sest neid olid piiratud hulk ja arvutihuvilised 
tundsid ja teadsid üksteist üsna hästi. Seltskond, kellega ma arvutiringides iga nädal kokku puutusin, 
võis olla kolmkümmend-nelikümmend inimest. 
Ma ise käisin kolmes-neljas arvutiringis ja seal liikus info, mis arvuteid kuskil näppida saab.

Näiteks Oktoobrirajooni õppetootmiskombinaadis\index{Tallinna 
Oktoobrirajooni Õppetootmiskombinaat} oli arvutiklass Yamaha 
MSX\index{Yamaha MSX} arvutitega, mis kindlasti on Prontol\index[ppl]{Pronto} 
paremini meeles, seal käis päris suur seltskond noori huvilisi koos. 

Oli ka TPI arvutiring\index{TPI arvutiring}, mida juhendas 
Vladimir Viies\index[ppl]{Viies, Vladimir} ja hulk teisi TPI õppejõude ning kus 
olid Robotronid\index{Robotron}. TPI arvutiringis 
puutusin esimest korda kokku Iskra-226ga\index{Iskra!Iskra-226}.

Tallinna 3. Keskkoolis\index{Tallinna 3. Keskkool} oli matemaatikaõpetaja Jaak Loonde\index[ppl]{Loonde, 
Jaak}. Tema oli haridussüsteemis omamoodi fanaatik, kes populariseeris 
arvutite jõudmist kooli nii riistvara kui ka arvutiõppe näol. Jaak Loondel oligi üks 
Agat\index{Agat} kasutada, mille nende kool sai ilmselt kuskilt Venemaalt. 

Lisaks arvutiringidele käisin 34. kooli 
tehnikaringis\index{Tallinna 34. Kool! 
Tehnikaring}, mille juhendaja oli Ants Reili\index[ppl]{Reili, Ants}. Seal oli 
päris mitu poissi, kes ei olnud otseselt üldise tehnika- 
või elektroonikahuviga, vaid just arvutihuviga, kuigi arvutiklassi
seal ei olnud.

Need olidki minu põhiseltskonnad, mille liikmeskond mingil määral kattus. Muidugi oli 
veel seltskondi või huviliste ringe.

\question{Kas need olid kõik Tallinnas?}

Mina puutusin kokku Tallinna omadega, Tartus koonduti ülikooli 
juurde. Seal oli Anne Villems\index[ppl]{Villems, Anne} ja hulk
Apple'eid. Tartus oli Apple olemas ka Füüsika 
Instituudis\index{Füüsika Instituut}, kus tegutses Jaan 
Pruulmann\index[ppl]{Pruulmann, Jaan}, kellega mina puutusin esimest korda kokku siis, 
kui käisin sealset Apple II\index{Apple II} vaatamas. Sattusime Tartusse talvisel ajal kooli- või 
ringikaaslastega ja otsustasime Füüsika Instituuti külla minna, 
et sooja saada ja ehk ka midagi
arvutis teha. Seal saime Pruulmaniga tuttavaks.

\question{Kooli ja ülikooli huvi arvutite vastu on arusaadav, aga mis ajendas
Oktoobrirajooni asutust arvutiklassi hankima?}\label{content!OTK}

Õppetootmiskombinaadid olid mitme kooli peale ehk rajooni 
kaupa (Tallinnas oli tol ajal neli rajooni\sidenote{Alates 1974. aastast 
jagunes Tallinn Lenini (endine Keskrajoon), Kalinini, Oktoobri- ja Mererajooniks.}). 
Asutuse nimetus oli õppetootmiskombinaat, mis ilmselt viitas 
sellele, et seal sai praktilisi asju proovida ja 
ametikogemusi. Arvutiring oli puhtalt huviring, kus ei 
olnud mingit tootmisväljundit, nagu õppetoootmiskombinaadi nimetusest võinuks 
eeldada.

\question{Seda minagi imestan, miks nad hankisid arvutid. See oli ju 
keeruline.}

Igal juhul oli neil arvutiklass tosina Yamaha arvutitega. Võimalik, et 
kuna klass oli otsapidi seotud Jaak Loondega\index[ppl]{Loonde, 
Jaak}, sebis tema selle sinna ja õppetootmiskombinaat oli lihtsalt \enquote{katus}. 
Sedasi ei olnud arvutiklass ühes konkreetses koolis, mis võinuks tekitada pingeid teiste koolidega.

\question{Kui sa arvutitega toimetama hakkasid, siis millele sa toetusid? Tühja 
koha pealt inimene ju ei vaata arvutiskeemilt, kust kuhu bitid liiguvad.}

Elektroonikatausta oli mul nii palju, et teadsin, mismoodi bitid 
liiguvad ja loogikatehted toimivad ning kuidas asju tööle panna. 
Näiteks kuidas teha LCD-ekraaniga elektronkella. 

\question{Kuidas sa seda oskasid?}

Ühelt poolt 34. kooli tehnikaringi\index{Tallinna 34. Kool! 
Tehnikaring} teadmiste baasil. Teisalt lunisin vanematelt 
endale küllaltki palju kirjandust, mis oli põhiliselt vene ja saksa 
keeles, sest ingliskeelset kirjandust ei olnud tol ajal saada. Või 
kui oli, siis ilukirjandust, aimet ja ulmet, mitte 
teaduskirjandust. Seda liikus saksa keeles. Ega ma ühtegi
raamatut otsast lõpuni läbi ei lugenud, aga 
natuke siiski sirvisin ja lugesin olulisemaid peatükke. Sealt 
tasapisi see teadmine tekkis.

\question{Kas sellest võib järeldada, et saksa ja vene keeles tehnilise 
teksti lugemine ei olnud keeruline?}

Minu jaoks saksa keel küll oli, sest koolis õppisin inglise keelt 
süvendatult. Inglise keel oli küll peaaegu nii selge nagu teine emakeel, aga saksa keelt ma ei osanud üldse. 
Tänapäeval on küll nii, et võtad raamatu lahti ja mis sest, et saksa 
keelt ei oska, aga paljud sõnad on inglise või muu tuntud keelega nii sarnased, et üldisest mõttest saab aru. 
Konkreetseid juhiseid või faktilist infot ma saksa keeles 
täisväärtuslikult ei loe. 

\question{Mis koolis sa käisid?}

44. keskkoolis\index{Tallinna 44. Keskkool}, mis on tänapäeval Mustamäe 
gümnaasium\index{Tallinna Mustamäe Gümnaasium|see{Tallinna 44. Keskkool}}. Kooliajal ei saanud inglise keelt väga palju rakendada, ka mitte 
IT-maailmas,
peale selle, et Basicus on käsk \verb|print| inglise keeles. Aga 
Basicus ei ole käske palju ja neid ei ole keeruline pähe 
õppida, juhul kui inglise keelt ei valda. 

\question{Kas tehnikakirjanduse juurde käis ka mõni muu 
kirjandus- või ulmehuvi?}

Kooliajal lugesin üsna palju ulmet inglise keeles ja 
mulle sattus kätte ka Douglas Adamsi Hitchhikeri raamat\sidenote{Vt 
märkust lk \pageref{sidenote!adams}.}.
Kuna õppisime inglise keelt süvitsi, siis meil oli ka
inglise keele kodulugemise tund. Pidime kodus lugema ingliskeelset
ilukirjandust ja tunnis õpetajale jutustama. Nägin raamatupoes 
üleval lae all riiulil Hitchhikeri esimest osa, 
vaatasin, et huvitav pealkiri, ostsin raamatu ära ja võtsin selle inglise keele kodulugemiseks. See ei olnud siiski väga mõistlik 
valik, sest kui mõelda väljamõeldud sõnadele, mida seal kasutatakse, siis need on 
eesti keelde üsna raskesti tõlgitavad, selleks peab väga hea fantaasiaga tõlkija 
olema. Aga mina hakkasin entusiastlikult raamatut lugema ja õpetajale jutustama. Ma 
küll ei tea, kui palju ta sellest aru sai, aga vähemasti jäi ta rahule. 

\question{Seal ei ole ju narratiivi, vaid mingisugune keeruline sõlm, mis viienda 
raamatu lõpuks umbsõlme läheb!}

Ega minagi saanud esimesest osast eriti aru. Hiljem lugesin ülejäänud osad ka läbi, siis loksus 
pilt paika.

\question{Kas ingliskeelset ilukirjandust oli tollal saada?}

Jaa, seda oli igal pool, isegi Tallinnas. 
Pärast kooli lõppu, kui töötasin TPIs\index{Tallinna Tehnikaülikool}, 
käisin palju Moskvas ja Leningradis komandeeringus, kus oli väga 
lai valik. Asimovilt ma ei lugenud esimesena tema tuntuimat sarja \enquote{Asum}\sidenote{Vt.
ka märkust lk. \pageref{sidenote!asum}.}, vaid 
üksiklugusid. Asimov ja Adams olid kaks ulmekirjanikku, 
kellega ma inglise keele vahendusel esimesena kokku puutusin. 

\question{Aga vene klassikud? Strugatskid?}

Jaa, Strugatskeid olin lugenud varem, sest neid oli tõlgitud eesti keelde: 
\enquote{Purpurpunaste pilvede maa}\label{sisu:purpur}\sidenote[][-1cm]{Arkadi ja Boris 
Strugatski \enquote{\begin{russian}Страна багровых туч\end{russian}} (1959), mis eesti 
keeles ilmus 1961. aastal Ralf Tominga tõlkes (värsid Lembe Hiedel) sarjas 
\enquote{Seiklusjutte maalt ja merelt}.}, \enquote{Amfiibinimene} oli vist ka 
Strugatskite oma\sidenote{\enquote{\begin{russian}Человек-амфибия\end{russian}} on siiski 
Aleksandr Beljajevi 1928. aastal ilmunud romaan, mis eesti keeles ilmus 1960. 
aastal sarjas \enquote{Seiklusjutte maalt ja merelt} koos romaaniga 
\enquote{Maailmavalitseja} (\enquote{\begin{russian}Властелин мира\end{russian}}, 
1926).}. Neelasin võimalust mööda ka populaarteaduslikku ehk 
aimekirjandust, näiteks raamatusarja \enquote{Mosaiik}\sidenote{\enquote{Mosaiik} oli kirjastuses 
Valgus aastastel 1973--1991 välja antud populaarteaduslike raamatute sari, mis 
käsitles äärmiselt laia teemaringi ajaloost ja psühholoogiast kuni topoloogiani.}. Muud
aimekirjandust eriti ei olnudki.

Kriminullid olid teine ilukirjanduse valdkond, mis tol ajal peale ulmekirjanduse 
huvi pakkus. Neid ma ka lausa neelasin.

Üldine lugemistempo oligi kiire. Näiteks Rootsis töötades lugesin ühe \enquote{Asumi} osa 
inglise keeles ühe ööga läbi. Järgmisel päeval ostsin raamatupoest järgmise osa ja nii edasi. Kool oli äsja lõpetatud ning sain töö ja muu 
elu kõrvalt seda lubada, selle asemel et öösel magada ja puhata.

\question{Kui sa keskkooli lõpetasid, kas läksid siis kohe TPIsse tööle või edasi õppima?}

Pärast keskkooli läksin TPIsse\index{Tallinna 
Tehnikaülikool} ikkagi tööle, mitte õppima. Töökoht sattus mulle kätte tänu 
Vladimir Viiese\index[ppl]{Viies, Vladimir} juhendatud arvutiringile. 
Asusin tööle 
elektronarvutite kateedris\index{Tallinna Tehnikaülikool!Elektronarvutite 
kateeder}, kus töötas ka Viies. Aitasin teha arvutihooldustöid ning 
häälestasin ja valmistasin ette arvutilaboreid, mida õppejõududel vaja oli. 
Hiljem lõin kaasa
tarkvaraprojektides, kus oli vaja programmeerida, 
sisend-väljundseadme jaoks draiver kirjutada. 

\question{Kas läksid õppima ka?}

Õppima läksin mõnevõrra hiljem, sest mind ei köitnud
punaste ainete küllus. Tundsin 
nende vastu nii suurt vastumeelsust, et ei tahtnud ülikooli 
õppima minna isegi tehnilist ala, kui seal on punased ained juures, näiteks NLKP ajalugu. Paar aastat hiljem, vist 1990. aasta sügisel läksin siiski TPI õhtusesse osakonda õppima. Olin 
küll üks enamikust, kes ei lõpetanud. Meie kursusele astus sisse vist 
kakskümmend viis inimest, kellest lõpetas kaks, aga õhtuses osakonnas lõpetanute madal protsent oligi üheksakümnendate algul üsna tavapärane. 

\question{Mis eriala see oli?}

Elektronarvutid. Sealt alates ongi tegelikult kõik töökohad ja üsna palju ka vaba aja tegemised olnud seotud programmeerimisega ja arvuti tehnilise poolega.

TPIs töötades asus elektronarvutite kateeder teisel korrusel. 
Samas korpuses neljandal korrusel oli raadiotehnika 
kateeder\index{Tallinna Tehnikaülikool!Automaatikateaduskond!Raadiotehnika 
kateeder}, kus oli see Apple II\index{Apple II}. Meil tekkis 
Mastiga\index[ppl]{Kaal, Madis} (Madis Kaal\sidenote{Madis Kaal toimetas sel ajal
raadiotehnika kateedris. Vt lk \pageref{sisu!mast_raadiotehnikas}.}) 
mõte ühendada teisel korrusel olevad PC arvutid, mis olid minu igapäevased tööriistad, 
Apple IIga, mis oli neljandal korrusel Masti igapäevane tööriist. 
Ehitasime nende vahele \emph{current loop}'i ehk RS-232, 
misjärel sai PCst kopeerida andmeid Apple II sisse ja vastupidi. 
Nagu pilv --- kasutad kellegi teise arvutit. 

Umbes 1990. aastal jõudis Eestisse info BBSide olemasolust. TPI majas oligi Mast\index[ppl]{Kaal, Madis} see 
entusiast, kes pani esimese BBSi jooksma. Mina esialgu vaatasin 
kõrvalt, tundmata selle vastu erilist huvi. Seal sai mänge vahetada, 
aga kuna ma ei ole kunagi mängufanaatik olnud, siis selle pärast
BBSindus mind ei tõmmanud. Hiljem küll 
leidsin, et BBSid võivad olla kasulikud --- 
neis leidub tekstifaile, mis on nagu
\emph{manual}'id, standardid või
programmeerimisõpikud IBMide või Apple'i jaoks.

\question{Kas need olid \emph{plain text} failid või \LaTeX?}

Need olid vormindatud tekstifailid 
tabulatsiooni ja leheküljevahedega ning neid sai maatriksprinteriga 
ilusti vormindatuna paberile trükkida. Rasvase 
ja kaldkirjaga teksti oli ka võimalik kasutada.
Maatriksprinterid olid ilmselt kättesaadava hinnaga, sest need olid enamiku arvutite taga. 
Suurtel arvutitel (ESid ja SMid, mis olid TPIs ja Küberneetika 
Instituudis) olid laiad ridaprinterid, selle sõna mõtles välja vist Ustus Agur\index[ppl]{Agur, 
Ustus}. Ühesõnaga koledat häält ja värinat tegevad 
printerid.

\question{Kas siis, kui said aru, et BBSidest saab igasuguseid dokumente, hakkasid need sulle huvi pakkuma?}

Jah, see oli see hetk ja ajend, kui leidsin, et sealt sai
peale mängude ja muu tilulilu ka midagi mõistlikku. Millalgi panin oma BBSi 
püsti ja selleks ajaks oli ka Fidonet Eestisse 
jõudnud\sidenote{Esimene Fidoneti Eesti regiooni 2:49 sisaldanud 
\emph{nodelist} on 271 28. septembrist 1990. Regiooni koordinaatorina on 
kirjas Andrus Suitsu\index[ppl]{Suitsu, Andrus} ja \emph{host} on Tarmo 
Ausing\index[ppl]{Ausing, Tarmo}. Vt lk \pageref{sisu:nodelist}.}. Paljud, kes on ajalooliselt tagasi vaadanud ja 
sellest ajast rääkinud, ei pruugi eriti olla vahet teinud BBSindusel ja Fidonetil, mis 
olid kaks eraldi maailma. BBS oli lihtsalt 
süsteem, kuhu sai modemiga sisse helistada ja siis seal ringi toimetada, 
andmeid failide näol tõmmata ja sõnumeid 
vahetada. Kogu info ja sõnumid olid salvestatud ühte 
konkreetsesse BBSi süsteemi.

Fidonet sai alguse BBSidest ja selle eesmärk 
oli BBSide ja muude Fidoneti liikmesüsteemide vahel sõnumeid
edasi-tagasi toimetada. 

\question{Kas see tähendab, et Fidoneti sisse helistamise kohad helistasid ka 
üksteisele sisse ja vahetasid andmeid?}

Jah, see oli siis juba automatiseeritud süsteem, kus olid vahendid 
sõnumite ehk meilide vahetamiseks. Sõnumeid oli kahte liiki: 
privaatmeilid ja konverentsmeilid (tänapäeva mõistes 
meiligrupid või -listid).

\question{Kas \emph{Usenet} tekkis ka sel ajal?}

Usenet oli olemas palju varem. Usenet ja UUCP protokoll 
on Unixi maailma päritolu,  
Unixi arvutite vahelise meilivahetuse protokoll. See Usenet, mis sinna 
ümber tekkis, oli konverentside või vestlusringide 
süsteem Unixiga töötavate arvutite kasutajate vahel. Fidonet ja BBSid töötasid
enamjaolt PCdel.

\question{Kas seda peegeldati Fidosse ka?}

Jah, seal olid lüüsid olemas. Usenetist sai konvertida kirju Fidoneti 
\emph{echo}'desse ehk konverentsidesse. Muu hulgas ka faile, neid vahetati
Usenetis väga palju ja neid oli võimalik konvertida 
PC failideks, mis kuskil BBSis üles pandi.

\question{Kas eestlased toimetasid Usenetis oma gruppides või olemasolevates?}

Usenetis ei olnud tollal Eesti-spetsiifilisi või 
regionaalseid gruppe, erinevalt Fidonetist, kus oli küll viisteist-kakskümmend lokaalset vestlustgruppi ehk \emph{echo}'t. 
Neist kaks-kolm olid liikmeskonna mõttes üsna populaarsed.

\question{Mida see tähendab? 50, 100, 500 liiget?}

Lugejaid võis seal olla palju, sest pidevalt tuleb välja 
inimesi, kellega mina ei ole kunagi kokku puutunud, 
aga kes räägivad, et nad on seal \emph{echo}'des midagi lugenud. Selleks, et \emph{echo}'sid lugeda, ei pidanud ise
omama BBSi ega Fidoneti süsteemi. BBSi sai sisse helistada, seal lugeda ja soovi korral kirjutada. Kui kellelgi oli Fidoneti süsteem 
püsti pandud, siis selle eelis seisnes selles, et kirjad tulid 
automaatselt koju kätte ja lugemiseks-kirjutamiseks ei olnud vaja ise kuskile 
kaugele helistada. See ring inimesi, kes ainult luges, 
võis olla päris suur. Aktiivselt suhtlesid ja ka kirjutasid 
võibolla kakssada inimest.

\question{Seda on päris palju. Kas sa oma Fido \emph{node}'i panidki püsti 
selleks, et asjad tuleksid koju kätte? Mis selle asja nimi oli?}

Eesmärk oli jah saada asjad piisavalt automatiseerituks, et 
ei peaks kulutama aega 
BBSi löögile saamisele, sest kui BBSi küljes oli 
välismaailmaga suhtlemiseks üks modem, siis sai 
seda BBSi kui teenust korraga kasutada üks inimene. 
See tähendas, et helistasid modemiga, telefon oli kinni, helistasid viie minuti 
pärast, ikka kinni. Milleks niimoodi vaeva näha ja pidevalt 
proovida? Tõsi küll, modemi sai panna automaatselt kordusvalima 
ja kui see lõpuks löögile sai, andis signaali. Aga ma leidsin, et parem on lasta teha
seda Fidoneti automaatikal. Siis sain rahumeeli endale sobival hetkel avada meililugemise programmi ja lugeda vahepeal masinasse tõmmatud meile. 

\question{Mis su \emph{node}'i nimi oli?}

Minu \emph{node}'i nimi oli MamBox. Ma ei mäleta, mis hetkest alates hakkasin oma perekonnanimest tulenevat eesliidet kasutama, aga BBSi tehes panin MamBox. 
Oma lõbuks programme kirjutades märkisin kaubamärgiks
\enquote{\emph{Copyright MamSoft}}\sidenote{Tegu oli levinud praktikaga. 
Sellest, \emph{misasi} üks firma on, oli arusaam ähmane, ent sellest, et firma 
\emph{nimi} tuleb kindlasti ära mainida ja kuulsaks teha, oli arusaam väga 
konkreetne.}.

Üsna tüüpiline oli see, et kui kellelgi oli BBS, siis lisas
ta ühel hetkel sinna Fidoneti funktsionaalsuse. Oli ka teistsugune suundumus, et kui kellelgi oli 
tekkinud Fidoneti \emph{node}, siis paljud omanikud 
leidsid, et võiks ka BBSi püsti panna. 
Muidugi oli ka Fidoneti \emph{node}'e, kelle omanike või 
\emph{sysop}'ide eesmärk oligi lugeda-kirjutada ja lasta automaatselt sõnumeid 
vahetada; nende huvi ei olnud BBSi üleval pidada.

\question{Teisisõnu, kui mõni BBS sai populaarseks, siis põhjuseks võis olla see, 
et aktiivne kogukond vahetas seal omavahel faile, kui ka see, et miskipärast 
otsustasid paljud kasutajad just sealtkaudu Fidonetile ligi pääseda.}

Fidoneti sisule pääses ligi kõikidest BBSidest, mis olid Fidoneti liikmed, sest 
kõigis oli ühesugune koopia konverentskirjadest ehk \emph{echo}'dest. 
Iseasi olid privaatkirjad --- siis oli vaja Fidoneti \emph{node}'i numbrit teada, 
et konkreetsele inimesele kirja saata. Konverentskirjad 
olid ühtmoodi saadaval igas BBSis.

Ega see muidugi mõnus ei olnud, et täna loed meili siit, homme hoopis 
teisest BBSist. On ju viidad, kui palju sul on loetud meile, kus 
su lugemisjärjekord on, kas oled millelegi vastanud või ei ole. See läheb 
sassi, kui ei ole oma nii-öelda kodu-BBSi. Ja oli ka selge, et 
populaarsed failide tõmbamise BBSid olid üsna hõivatud ja 
tihtipeale liinid kinni. 

\question{Faili tõmbamine võttis ju tükk aega!}

Jah. Algusaegadel, kui BBSid ja Fidoneti \emph{node}'id Eestis tekkisid, 
siis 14 400boodine (ümmarguselt 14 400 
bitti ehk 14 kilobitti sekundis) andmevahetuskiirus oli üsna tüüpiline.

\question{Ma isegi mäletan 9600boodiseid.}

9600 oli jah lihtne ja odav igamehe tehnoloogia, aga kõik püüdlesid ikka 14 400boodiste 
modemite poole. Edasi tulid
19 200, 26 600 ja veelgi suuremad kiirused. Minul oli kasutada isegi
33 600boodise töökiirusega modem, aga selline kiirus tuli 
kätte ainult juhul, kui teisel pool sideliini oli vastas täpselt 
sama tootja modem. Selle nimi oli TrailBlazer\index{Telebit 
TrailBlazer}\sidenote{USA tootja Telebit, kes Trailblazeri sarja tootis, 
kasutas üldlevinud V-seeria protokollide asemel omaenda protokolli Packetized 
Ensemble Protocol (PEP).}. USRoboticsid\index{US Robotics} töötasid BBSide 
põhiajastul kõige kiiremini vist 34,4 kiloboodi juures.

\question{Kas BBSidega majandamine tekitaski sul võrguhuvi? Sa rääkisid, 
kuidas te Mastiga Apple'it ja PCd paaritasite.}

Ilmselt Apple'i ja PC paaritamine tekitaski võrgunduse 
pisiku, sest TPIs ega ka kuskil mujal, kus ma alguses arvutitega kokku 
puutusin, ei olnud kohtvõrgutamise tehnoloogiaid 
kasutusel. Ainuke oli UUCP, mis käis Unixite vahel, aga see oli 
tõsisemate ja suuremate arvutite sidepidamine akadeemilistes ja teadusringkondades. Fidonet seevastu oli 
rohkem asjaarmastajalik. Alles pärast TPId, järgmises töökohas, puutusin esimest 
korda kokku päris kohtvõrgutehnoloogiaga ARCNet\sidenote{ARCNet oli 1980. aastatel levinud esimene laialdast 
kasutust leidnud mikroarvutite võrgusüsteem, mis on siiani kasutusel 
sardsüsteemide puhul.}.

\question{Kus see oli ja mis aastal?}

See oli aastal 1991 ettevõttes Skriining\index{Skriining}, mis 
tegutseb tänapäevalgi. Skriiningus puutusingi kokku ARCNetiga, mis jooksis 
kahe ja poole megabitisel kiirusel. See oli koaksiaalkaabli võrk, 
peaaegu nagu esimesed Etherneti võrgud, aga neli korda aeglasem. ARCNeti koaksiaalkaabel oli ka vist 75oomine versus Etherneti 50oomine kaabel. 

ARCNet oli üsna lühiajaline, sellega puutusin kokku peamiselt 
tänu sellele, et tollal käidi Soomest ja mujalt lähivälismaalt 
seljakotiga kraami toomas. Väga palju Soome kraami oli 
seal maha kantud, seda vist ei tahetud seal ära visata, 
sest utiliseerimine maksis ja nii antigi ära. Ma ei mäleta, et ARCNetiga midagi väga tõsist oleks tehtud, aga 
kokkupuuted sellega ikkagi olid. Selle järel tuli koaksiaalkaabli Ethernet, kümme megabitti sekundis. See hakkas reaalselt ettevõtetesse jõudma ja selle baasil 
hakati üsna palju kohtvõrke ehitama.

\question{Räägi palun pisut Skriiningust\index{Skriining}. Arvutiäri 
jaoks peaks nime järgi olema kaks poolt: arvuti ja äri. Aga et aastal 1991 
oleks kumbagi olnud, tundub natuke uskumatu.}

Skriiningu arvutid tulidki alguses seljakotis piiri tagant. Järgmises faasis tulid need endiselt seljakotiga piiri tagant, aga 
selleks oli vaja piiri taha kõigepealt seljakotiga 
sularaha viia. Keskmise arvuti hind võis olla kakskümmend tuhat rubla, kuid mina hindade ja müügitööga ei tegelenud. Nii et ma täpselt ei kujuta ette, kui palju arvutid tol 
ajal maksid, aga arvutustehnika oli meeletult kallis.

\question{Kuidas selline firma üldse võis tekkida tol ajal? Ei saanud ju panna
internetti kuulutust, et tulge meile tööle.}

IT-maailmas liikusid inimesed ilmselt tutvuste kaudu ühest kohast 
teise tööle. Ka mina jõudsin Skriiningusse\index{Skriining} tuttava 
kaudu, mu varasem kolleeg TPIs sattus 
Skriiningusse tööle ja kutsus paar aastat hiljem mindki sinna. 
Skriiningu nii-öelda vertikaal- või kliendisegment oli ja on ka tänapäeval 
meditsiiniasutused: võrgud, arvutibaas ja infosüsteemid, 
programmeerimine ja hooldamine. Ma arvan, et see on ka üks põhjus, miks 
Skriining on tänapäeval endiselt elus: tal on oma 
üsna kitsas kliendisegment ning kindlad ja pikaajaliselt välja kujunenud kliendisuhted.

\question{Sinu jutu järgi tundub, et esimesed arvutifirmad olid sõprus- 
või vähemalt tutvuskonnapõhised.}

Mitte päris. Too endine kolleeg, kes mind Skriiningusse
kutsus, oli ainuke, keda ma seal tundsin. Aga sellised arvutifirmad ei olnud suured, Skriiningus oli kõige rohkem viis-kuus 
inimest ja kõik tegid enam-vähem kõike. Võibolla 
mõni programmeeris rohkem ja mõni teine, nagu mina näiteks, vedas rohkem 
kaablit, keeras kruvisid ja timmis asju 
arvutikaane all. Teatud eelistused olid kindlasti inimestel olemas, aga 
üldjoontes võib öelda, et kõik käisid mingil määral vähemalt üle kõikidest süsteemidest, mida firmas kasutati või millega see tegeles.

Samas oli muidugi ka mitmeid sõprade seltskondi, kes üheskoos tegid mõne arvutifirma.

\question{Kas sa sel ajal pidasid veel oma BBSi ka?}

Jah. BBS oli mul üleval päris pikka aega, ma olen teda ühest töökohast teise kaasa vedanud, sest kodus ei saanud seda pidada. 
Esiteks ei olnud eriti kellelgi võimalik koju arvutit hankida, see oli kallis. 
Ja kui ka oli võimalus mõni niru arvuti saada, siis selle peale 
BBSi hästi püsti ei pannud. Teiseks ei olnud tol ajal kodus telefoniga 
väljahelistamine just odav lõbu. Pealegi, kui mõelda 
Fidoneti peale ja et see oli ülemaailmne süsteem, siis Fidoneti side 
hõlmas ka teatud hulka rahvusvahelisi kõnesid. Sel ajal ei olnud kodustelt numbritelt 
võimalik otse välismaale helistada, kaugvalimine toimus läbi 
inimoperaatori\sidenote{Jaan Tallinn\index[ppl]{Tallinn, Jaan} on rääkinud, et 
inimoperaatoritele oli täiesti võimalik arvutiside vahendamine selgeks 
õpetada. Tuli öelda, et \enquote{kui vilistama hakkab, ühendage ära, nii peabki 
olema}.}. Ja ega ka kõikidest ettevõtetest ei olnud võimalik välismaale 
otse helistada. Tihtipeale oli ettevõttes selleks kümne 
või saja telefoni peale ainult üks telefoninumber, mida püüti 
endale ära rääkida, et selle taha BBSi ühendada. Tihti olid ka 
BBSi omanikel kokkulepped, et nende BBS töötab ja telefoniliini saab öösiti kasutada 
ning päeval kasutatakse liini kontoritööks. 
See tekkis hiljem, et BBSi jaoks oli mõnes firmas võimalik 
saada 24 tundi ööpäevas eraldi telefoniliin, ja eriti heal juhul sai sealt ka 
välismaale helistada.

Liini oli võimalik jagada ka BBSi modemi ja ettevõtte faksiseadme vahel, siis said
mõlemad sõbralikult töötada 24 tundi ööpäevas.

Kui ma tol ajal ühest töökohast teise liikusin, siis hindasin
uut ettevõtet muu hulgas selle järgi, kas mul on võimalik 
BBS sinna kaasa võtta ja selle jaoks 
kaugvalimisega telefoniliin saada --- veel parem, kui liin oleks ööpäev läbi kasutatav.

\question{BBSi kaasavedamine pakkus sulle siis päris olulisi valikuid. Mis selle juures huvitav oli?}

Ikka see, et BBSidest saadav info tuli üsna lihtsalt kätte ja seda oli kerge maailmast üles otsida, kui juba Fidoneti \emph{node} püsti 
oli. Lisaks sai meili- ja 
failivahetust automatiseerida. Kui tahtsin kuskilt kaugelt BBSist mõnd 
faili kätte saada ja teadsin selle nime, siis polnud tarvis jälle endal käsitsi 
sinna BBSi sisse logida, vaid sain seda teha 
Fidoneti automaatika abil.

\question{Toonases Fidoneti maailmas toimetav seltskond oli suhteliselt suur 
ja sinu nimi jookseb nende juttudest päris palju läbi. Miks see nii on?}

Tegelikult on olemas palju nimekamaid 
BBSi pidajaid, kes BBSi maailma Eestis põhimõtteliselt alustasid.

Kui BBSid ja Fidonet olid Eestis 
levima hakanud ja üsna agarasti kasutusele võetud, siis peagi tekkis meil 
ühes Fidoneti seltskonnas äratundmine, et meid on 
küll sada kuni kakssada inimest, kes igapäevaselt Fidoneti 
kaudu suhtlevad, kirju vahetavad, nalja teevad ja vahetevahel üksteist ka sõimavad, aga me teame võibolla kümmet neist nime- ja nägupidi. Jõudsime
järeldusele, et sellele probleemile tuleks lahendus leida. 

1991. aasta suvel kasvas huvi teiste BBSi kasutajatega näost näkku kohtuda nii suureks, 
et enam-vähem seesama kümnene seltskond mõtles teha kokkutuleku. Ühel augusti nädalavahetusel saigi BBSummer Väänas teoks. Osavõtumaks oli ka, 
võibolla viiskümmend rubla, võibolla vähem\sidenote[][-5.4cm]{Sel ajal valitses Eestis 
hüperinflatsioon ja hinnad kerkisid kiiresti, mistõttu 50 rubla tollast väärtust on raske hinnata. Lisaks olid enne 1992. aastat 
teatud kaupade hinnad riikliku kontrolli all ja paljusid kaupu polnud üldse saada, valitses 
ka sularahapuudus ja toimis elav ning väga volatiilsete hindadega 
must turg. 50 rubla eest võis saada näiteks 20 kilo kartulit või ühe 5.25 tollise flopi.}. Plaan oli suhelda ja mängida 
IT-kalduvustega mänge: mitte arvutimänge, aga näiteks flopiheidet ja 
kõvakettaheidet. 

BBSummerist kujunes traditsioon, igasuvine kokkutulek. Mina aitasin 
esimest BBSummerit ette valmistada ja läbi viia ning panin ka hilisematel BBSummeritel õla alla.
Eks seepärast mu nimi BBSi seltskonna mällu on jäänud.

\question{1991. aastal andis ikka kõvaketast heita!}

Kõvaketas koosnes tollal suurtest 19- või 21tollise 
läbimõõduga plaatidest, mis ei olnud 
hermeetilises korpuses nagu tänapäevased pöörlevad kettad. 
Ühe käepidemega varre külge oli pandud kaheksa või kümme plaati ja neid sai 
kettaseadme seest välja tõsta ning vahetada\sidenote[][-6.8cm]{Sellised kettapakid, näiteks IBMi 1316, 
suutsid talletada mõned megabaidid infot ja olid tolleks ajaks 
selgelt iganenud. Eestisse sattusid sedalaadi seadmed tõenäoliselt 
humanitaarabina, mis tõi meie kanti hulganisti kummalist vananenud riistvara. 
Üks selline kettalugeja oli 1992. aastal näiteks Võru I 
Keskkoolis\index{Võru Kreutzwaldi Gümnaasium}. 
Arvutiklass asus teisel korrusel ja kui kettaseade sisse lülitati, oli undamist 
tänavale kosta --- selle järgi sai hinnata, kas klassis parajasti oli keegi või 
mitte.}. Sealt lahti lammutatud kettaid me lennutasime küll esimesel 
kokkutulekul. 

Kokkutuleku nimi BBSummer\index{BBSummer} tulenes 
\enquote{BB} lühendist BBS ja suvest. Üks aasta varem 
oli toimunud esimene Rock Summer\sidenote[][-3.2cm]{Rock Summer oli 1980. aastate lõpus 
ja 1990. aastatel Tallinnas lauluväljakul peetud muusikafestival. Tegu oli esimese suurema 
rokifestivaliga siin kandis ja platsil valitsenud atmosfäär avaldas keskmisele 
nõukogude noorele radikaalset mõju. Kuna tegu oli ühega esimestest 
võimalustest piiluda raudse eesriide taha, meelitas festival kohale ka 
nimekaid lääne ansambleid.}, aga nii palju, kui oleme erinevate inimestega 
meenutanud, ei olnud Rock Summer kuidagimoodi \enquote{Summeri} nimeosa 
eeskuju või põhjustaja --- meil olid sõltumatud kaubamärgid. 

Kui see 1991. aasta BBSummer toimus\sidenote[][-1.5cm]{Tarmo saatis esimese BBSummeri 
(ametliku nimetusega \enquote{Eesti amatöörarvutivõrgu kasutajate I seminar-laager}) 
kutse 12. juulil 1991 ja üritus toimus 26.-27. augustil Tugamanni tuulikus (ametlikult EPT Tallinna osakonna puhkekompleks).}, 
siis mõni päev varem tehti
Moskvas riigipööre ja Tallinnas tulid tankid tänavale. Mina ütlesin seepeale, et BBSummeri teeme ära igal juhul, kui just ei ole 
liikumiskeeldu. Olukord oli üsna pingeline. Esimesel BBSummeril 
oli vist viiskümmend kuus osalejat. Suur hulk oli seal puhtalt sellepärast, et nad olid Fidoneti 
\emph{sysop}'id, aga kõvasti üle poole 
olid BBSi lihtkasutajad, kes tõmbasid faile ja 
vahetasid meile, ilma et neil endal oleks olnud oma BBS või \emph{node}.

\question{See oli üsna korralik suhe teenuse pakkujate ja tarbijate vahel, 
BBSi pidamise barjäär oli kõrge ja seltskond seega üsna tehniline?}

See oli jah parajalt tehniline. Kes tundis, et tehnika on temast 
üle, tõenäoliselt ei pidanud BBSi. Selle
häälestamine, korralikult tööle panemine ja Fidoneti 
automaatika käivitamine ei olnud triviaalne tegevus. Internetist juhendvideot vaadata ju ka ei saanud, küll 
aga sai lugeda tekstifaile samm-sammult juhistega.

\question{Siis sündis ju FAQ, \emph{Frequently Asked Questions}, mis praegu on 
lihtsalt osa veebilehest. Toona oli tegu konkreetse eraldi leviva 
failiga, kuhu jõudsidki \emph{echo}'des ja uudisgruppides sagedasti küsitud 
küsimused koos pädevate vastustega.}

Jah, olid küsimused-vastused, kuidas asi käima panna ja milliseid sümptomeid vaadata, kui midagi ei tööta.

\question{Kas sedalaadi sisu Eestis ainult tarbiti või panustati ise
ka?}

Jah, panustati ikka, kui toodeti sisu ehk kui keegi kirjutas 
programmi, mis ei olnud mõeldud ainult oma tarbeks ega olnud mäng, 
vaid näiteks funktsioonide või alamprogrammide teek ehk 
\emph{library}. Näiteks Mast\index[ppl]{Kaal, Madis} kirjutas 
tekstiliideste tegemiseks ühe funktsioonide teegi, millega sai teha ekraanimenüüsid ja 
-kaste. Tekstirežiimis sai
hiirega menüüdes ringi klikata. Selliste asjade jaoks olid 
FAQd või lihtsad juhendid olemas igal vähegi mõistlikumal 
autoril. Mängude puhul tuli küll installida flopi, mäng käima panna
ja siis vaadata, kuidas see tööle hakkab ja mida mingi nupp 
teeb. Mängude manuaale ilmselt keegi eriti ei lugenud.

\question{Kas pärast Skriiningut jõudsid Uninetti ka?}

Jah, Uninet on olnud minu tööandja küll.

\question{Kas päris alguses või millalgi hiljem?}

See oli mul vist viies töökoht. Pärast Skriiningut ja aastast maasikakorjamist Soomes
(loe: C++ ja x86 asm programmeerimistööd Rootsis)\sidenote{Tol ajal oli levinud viis korraga palju raha teenida käia Soomes maasikaid korjamas. Töö oli füüsiliselt raske, suhteliselt nüri aga toonase Eesti mõistes väga hästi makstud. Ilmselt olid Tarmo tööl Rootsis samad tunnused.} sattusin Baltic 
Computer Systemsisse\index{Baltic Computer Systems}.
BCSis tegelesin konkreetselt arvutivõrkudega: 
meil oli arvutivõrkude osakond ja me tegelesime ühelt poolt 
kaabeldusega ja teisalt serverite ning mingil määral ka sellise 
tarkvaraga, mis oli vaja võrgus käima panna. Näiteks andmebaasid, mis olid 
mõeldud algselt ühes arvutis kasutamiseks, aga mida sooviti hiljem võrgus kasutada. Seejärel tuli Uninet.

\question{Tol ajal oli enamik andmebaase mõeldud käima ühes 
arvutis. See tähendas, et sinna sisse ei olnud ehitatud transaktsioone ega muud säärast.}

Seda võimalust ei olnud jah tihtipeale olemas, aga oli viise, kuidas 
sellest mööda hiilida, et andmebaasi avamisel arvutis ei oleks see võrgus kasutajate jaoks lukus, vaid et seal saaks midagi 
teha. 

\question{Mida sa praegu teed?}

Hiljem olen teinud erinevaid asju, mis ei ole olnud enam niivõrd seotud
võrgu tehnilise ülesehitusega, vaid võrgus 
töötavate rakenduste ja võrguturbega. Praeguses töökohas olen mõnes mõttes uuesti
sattunud tagasi sellise tegevuse juurde, mis on seotud andmesidevõrgu 
baasprotokollidega: IP, TCP, UDP ja DNS. Sel tööl on küll
endiselt seos rakendusprogrammide ja mobiiliäppidega, 
sest suur osa tööst vajab ka teadmist, kuidas äpid võrgus käituvad: mismoodi 
nende liiklus ja andmevahetus on võrgus üles ehitatud ja kuidas 
andmevahetust filtreerida. Kuidas hakkama saada lahendustega, 
mida Google ja teised suured tegijad välja pakuvad ja mille eesmärgiks 
peetakse üldiselt seda, et kasutajal oleks internetis mugavam toimetada ja turvalisem olla, aga mis samas võivad kaasa tuua
teatud negatiivseid nähte --- näiteks võrguliikluse tarbetu kasvu. Minu töö on 
aidata neid negatiivseid nähte teatud kasutuskohtades kõrvaldada.

\question{Siis on ju selles mõttes toredasti, et kui sa alguses rääkisid huvist 
mängus tegelasele kapoti all toimetades teist värvi müts pähe panna, siis 
praegu on see tegelane teistsuguse arvuti sees ja kapotialune on natuke 
keerulisem, aga ülesanne suuresti sama.}

Täpselt nii. Minu jaoks on oluline see, mis on karul kõhus ja kuidas see 
seal töötab. Kui ei tööta hästi, siis tuleb mõelda, kas ja mida paremaks teha. Ja 
kui töötab hästi, siis saab sellest hoolimata midagi teistmoodi teha.

