\index[ppl]{Mamers, Tarmo}

\question{Kuidas sina arvutite juurde said?}

Mul oli üks klassikaaslane, kelle isa töötas Küberneetika 
Instituudis\index{Küberneetika Instituut}. Ma arvan, et see võis olla kuskil 
keskkooliaastate alguses, ilmselt siis seitsmes-kaheksas-üheksas klass, kus sai 
käidud päris mitu korda järjest tutvumas sellise asjaga nagu Apple 
II\index{Arvutid!Apple II}. See oli  mõnevõrra keeruline, sest et see Apple oli 
üsna koormatud, kuna seal Küberneetika Instituudis  kasutati teda mingisuguseks 
teadus ja uurimistööks. Ma küll ei tea täpselt mil moel. Ja kuna selle Apple II 
monitori asemel oli tavaline telekas siis ma mäletan väga hästi seda, et kui ma 
esimesi kordi sinna sattusin mingil talvisel perioodil, siis põhiliselt seda 
telekat kasutati vist suusahüppe MM-i jälgimiseks. Ehk päris alati ei saanud 
seda arvutit näperdada või kui sai, siis ilma pildita. See oli  tõenäoliselt 
umbes aastal 1983 kuni 1985, umbes selline ajavahemik. 

\question{Mis sa tegid selle arvutiga?}

Kõigepealt vaatasin, mis teised teevad ja teised põhiliselt mängisid igasugu 
huvitavaid mänge, mis seal Apple peal tol aja olid. Kui ma ise hakkasin seda 
näperdama, siis mind pigem hakkas huvitama see, et kui need arvutimängud on 
mingi teatud hulga eludega ja teatud hulka relvadega ja teatud hulga mingite 
abivahenditega, mida saab kasutada, kas neid kuidagi nii-öelda ära häkkida ei 
saaks? Et võib-olla oleks rohkem elusid,  võib-olla oleks rohkem abivahendeid, 
kuidagimoodi saaks rohkem punkte näiteks. Millegipärast minu huvi oli nagu 
hoopiski selline. Mitte  mängimine, kui tegevus, vaid pigem millegi ümber 
tegemine, millegi kuidagi teistmoodi tegemine. Võib-olla, et mingi tegelase 
müts ei oleks mitte punane vaid oleks hoopis roheline, midagi sellist. Kuidagi  
nikerdada  et, midagi muutuks ja saaks teistsuguseks. 

Ma ei teadnud midagi ei programmeerimisest ega ka eriti arvutite 
tööpõhimõttest, see oli nagu esimene ajendav tegur, mis tõi kokkupuute esiteks 
BASIC-u\index{Keeled!BASIC} keelega ja järgmisena Apple 
assembleriga\index{Keeled!Assembler}. Kuna ma natukene tundsin huvi tol ajal ka 
elektroonika ja eriti digitaalelektroonika vastu, siis juhtus nii, et mulle 
jäid ette ka selle Apple arvuti manuaalid. Tollaste arvutitega olid alati 
manuaalides kaasas nende elektroonikaskeemid. Põhimõtteliselt minu huvi oli 
sealt siis  näpuga järge ajada,  et kuidas need bitid liiguvad, kui midagi 
printida või nuppu vajutada. Või kuidas ekraani peale pilt tekitatakse 
bittidest. See oli algus.

\question{Kas need mängud olid kuskilt väljastpoolt tulnud või liikus ka ise 
tehtud asju?}

Apple-i peal olevad mängud olid välismaalt tulnud kas sama kanalit pidi, kust 
arvutid ise olid Nõukogude Liitu tulnud. Või siis osad nendest, eeskätt need 
Apple-i kasutajaid, kes olid kuskil Tartus või Nõos, kellel oli kuuldavasti ja 
mäletatavasti ka mingisuguseid kontakte teiste Apple kasutajatega mujalt 
maailmast, said mänge kuskilt mujalt ka.

\question{Tehti ju ise ka algelisi mänge, mingis Juku\index{Arvutid!Juku} 
mängus sai mõisa majandada, ma mäletan?}

Sellest Apple ajast ma ei mäleta väga palju mingisuguseid kodukootud mänge. 
Küll aga mõnevõrra hilisemast ajast, kui mul oli kasutada selline Nõukogude 
päritolu arvuti nagu 
Iskra-226\index{Arvutid!Iskra!226}\sidenote{\begin{russian}Искра 
226\end{russian} oli Nõukogudemaal toodetud arvuti Wang 2200 kloon, mis oli 
originaaliga binaarkoodi mõttes sajaprotsendiliselt ühilduv. Siiski oli 
Iskra-226 sisemine struktuur oluliselt erinev ning ta sisaldas mitmeid 
täiendusi, mis muutsid ta  sobilikumaks tööstusrakendusteks.}. Seal oli küll 
kodukootud mänge, mille idee oli võetud kuskilt mujalt ja tehtud mäng valmis  
või oli nullist mingisugune mõte mänguks vormistatud. 

Venelastel oli selline Apple kloon, mille  nimi oli  Agat\index{Arvutid!Agat}. 
Kui neid Eestisse hakkas tulema, siis nende jaoks oli mingisuguseid vene 
päritolu mänge. Osa  olid selgelt Apple pealt maha lükatud ja osa olid  
nii-öelda originaalid. Sellest ajast ma ei mäleta, et oleks väga palju 
mingisuguseid kodumaist päritolu mänge eriti olnud.

\question{Kas need inimesed, kes seda arvutit päriselt kasutasid, lasid sul 
seal talle niisama lihtsalt kõhu alla vaadata ja kaant maha kruvida?}

No ega seal Küberneetika Instituudis\index{Küberneetika Instituut} väga palju 
ei lastud, sest et seal oli oluline ikkagi see, et masin oleks töökorras ja iga 
hetk kasutatav  uurimistöö jaoks.

Hiljem,  keskkooli ajal veel, sattusin ma ka tollasesse TPI-sse ehk siis 
praegune Tehnikaülikool\index{Tallinna
Tehnikaülikool!Automaatikateaduskond!Raadiotehnika kateeder} kus raadiotehnika 
kateedris oli ka üks Apple II\index{Arvutid!Apple II}.  Seal olid siis juba 
nii-öelda raadiotehnikud, kelle igapäevane leib ongi see,  et vaadata pigem 
seda, mis seal kõhus on ja kuidas käib. Seal sai siis masina sisse vaadata, 
kõrval oli olemas  jootekolb, millega sai pädevam seltskond teha ise Apple 
jaoks mingisuguseid perifeeriakaarte, et TPI majas mingisuguseid juhtimisi või 
mõõtmisi või asju teha.  

\question{Tollal, ma saan aru, oli üsna hästi teada, kus mingisugune Apple II 
või Iskra saadava oli?}

Jah, sest neid olid valitud hulk. Ja see nii öelda arvutihuviliste seltskond 
tundis ja teadis üksteist üsna hästi. Võib-olla ei tundnud, aga nad teadsid. Ma 
ju teadsin neid inimesi, kellega ma kuskil arvutiringis iga nädal kokku puutun. 
See seltskond võis olla mingisugune kolmkümmend, võib olla nelikümmend inimest. 
Neid arvutiringe,  millega mina kokku puutusin sel ajal oli kolm tükki 
põhiliselt või isegi neli. 

Oli selline koht nagu Oktoobrirajooni Õppetootmiskombinaat\index{Tallinna 
Oktoobrirajooni Õppetootmiskombinaat}, kus oli arvutiklass, kus  olid Yamaha 
MSX-id\index{Arvutid!Yamaha MSX}, mis kindlasti on Prontol\index[ppl]{Pronto} 
hästi meeles, seal käis päris suur seltskond noori huvilisi koos. 

Ja siis oli TPI Arvutiring\index{Arvutiklubi!TPI Arvutiring}, mida juhendamas 
Vladimir Viies\index[ppl]{Viies, Vladimir} ja mingi hulk muid õppejõude. Seal 
olid Robotronid\index{Arvutid!Robotron}, ma küll ei mäleta, mis see täpne tüüp 
oli aga mingeid  sellise kaubamärgiga masinad seal olid. Ja selles arvuti 
ringis käies ma puutusingi tegelikult ka esimest korda tegelikult kokku selle 
Iskra-226'ga\index{Arvutid!Iskra!226}.

Tallinnas oli selline kool nagu 3. Keskkool\index{Koolid!Tallinna 3. Keskkool}, 
kus oli üks selline matemaatikaõpetaja nagu Jaak Loonde\index[ppl]{Loonde, 
Jaak}. Tema oli haridussüsteemis selline omaette fanatt ja  populariseeris 
kõvasti tol ajal, et arvutid nii riistvara kui arvutiõppe näol koolidesse 
jõuaksid. Jaak Loondel  mingist hetkest alates oligi üks 
Agat\index{Arvutid!Agat} kasutada. Ma ei tea, kustkaudu nende kool selle 
kuskilt Venemaalt sai. 

Ja siis oli olemas selline koht nagu 43. kooli 
tehnikaring\index{Arvutiklubi!43. Kooli Tehnikaring}\index{Koolid!Tallinna 43. 
Keskkool} mille juhendaja oli Ants Reili\index[ppl]{Reili, Ants}. Seal käis ka 
päris mitu sellist poissi, kes ei olnud nagu otseselt sellise üldise tehnika 
või siis elektroonika huviga, vaid rohkem  just arvutihuviga.

Et need vist olid nagu need põhilised seltskonnad, kus siis igas ühes oli  ka 
mingisugune kattuvus. Mina teadsin  kõiki neid nelja seltskonda, võib-olla oli 
veel mingisuguseid seltskondi või nii-öelda  huviliste ringe.

\question{Need olid kõik Tallinnas, eksole?}

Jah, need olid kõik Tallinnas. Tartus oli hulk inimesi, kes koondus ülikooli 
juurde. Seal oli Anne Villems\index[ppl]{Villems, Anne} ja mingi hulk, kui 
õigesti mäletan,  Apple-id. Kus oli Tartus Apple olemas, oli Füüsika 
Instituut\index{Füüsika Instituut}. Seal oli selline mees nagu Jaan 
Pruulman\index[ppl]{Pruulman, Jaan} kellega mina puutusin ka tol ajal  kokku, 
kui  ma ükskord seda Apple II-te\index{Arvutid!Apple II} käisin vaatamas. Ma ei 
mäleta, ma sattusin mingil põhjusel Tartusse mingisuguste kooli- või 
ringikaaslastega, ja  me mõtlesime, et lähme sinna Füüsika Instituuti külla, 
sest seal esiteks saab sooja (ilmad olid tol hetkel väga külmad ma mäletan) ja 
äkki saab arvutis ka midagi teha. Nii me Pruulmaniga tuttavaks saime. Ma ei 
tea, kuidas need asjad toimisid selles mõttes, et üks hommik mingi seltskond 
koolijuntsusid võtab pähe, et lähme ja sõidame. Kellelegi mingeid kontakte ei 
ole, mingeid eelnevaid kokkuleppeid ei ole, tol ajal ka ei helista ja ei saada 
meili aga kõik nagu tuleb välja lõppkokkuvõttes, eks.

\question{Kui ma koolist saan aru ja ülikoolist ka aga mis selle 
Oktoobrirajooni asutuse huvi oli arvutiklassi hankida?}\label{content!OTK}

Need õppetootmiskombinaadid olidki sellised mitme kooli peale, ehk siis rajooni 
kaupa (Tallinnas oli tol ajal neli rajooni\sidenote{Alates 1974. aastast 
jagunes Tallinn Oktoobri-, Lenini (endine Kesk), Kalinini ja Mererajooniks.}). 
Ja noh, asutuse nimi oli õppetootmiskombinaat, mis ilmselt siis pidi viitama 
sellele, et seal annab mingit praktilist asja proovida teha ja mingeid 
ametikogemusi saada. See arvutiring oli muidugi puht selline huviring, seal ei 
olnud mingit  tootmisväljundit, nagu see õppetoootmiskombinaadi nimi võiks 
öelda.

\question{Seda minagi imestan, et miks nad hankisid need arvutid, see pidi ju 
keeruline olema?}

No igal juhul neil oli arvutiklass mingi tosinajao arvutitega. Võis ka olla, et 
kuna see klass oli otsapidi seotud sellesama Jaak Loondega\index[ppl]{Loonde, 
Jaak}, et tema kuidagi selle arvutiklassi sinna sebis ja et 
Õppetootmiskombinaat oli lihtsalt nii-öelda \enquote{katus}. Nii  ei olnud see 
klass ühes konkreetses koolis kus oleks olnud mingisugused poliitilised pinged, 
et näe, neil on aga meil ei ole. Võib-olla seda klassi oli nii lihtsam 
organiseerida. 

\question{Kui sa arvutitega toimetama hakkasid, siis millele sa toetusid? Tühja 
koha pealt inimene ju ei vaata arvuti skeemi pealt, kust kuhu bitid liiguvad?}

No elektroonika tausta mul oli nii palju, et seda ma teadsin, mismoodi bitid 
liiguvad ja mismoodi  loogikatehted toimivad ja kuidasmoodi asju tööle panna. 
Ja kuidas näiteks teha LCD displeiga elektronkella.  

\question{Kuidas sa oskasid?}

Ühelt poolt selle 43. kooli tehnikaringi\index{Arvutiklubi!43. Kooli 
Tehnikaring} teadmiste baasi tõenäoliselt. Ja teisalt ma lunisin vanematelt 
endale välja küllaltki palju kirjandust, mis oli põhiliselt vene ja saksa 
keeles, inglisekeelset kirjandust ei olnud tol ajal lihtsalt kuskilt saada. Või 
kui oli, siis see oli ilukirjandus ja niisugune aime ja ulme, eks, aga mitte 
mingisugune teadus või tehnikakirjandus. Tehnikakirjandus, kui ta oli 
mitte-nõukogude päritolu, siis ta oli ikkagi saksa keeles. Ega ma ausalt öeldes 
neid raamatuid ühtegi otsast lõpuni läbi ei lugenud, aga ma neid siiski 
natukene sirvisin ja  lugesin võib-olla mõned olulisemad peatükid läbi. Sealt 
tasapisi ilmselt see kogemus või  teadmine tekkis.

\question{Millest me järeldame, et saksa ja vene keeles tehnilise sisulise 
teksti lugemiseks ei olnud probleem tolleks hetkeks?}

Minu jaoks saksa keel küll oli, sest ma õppisin koolis inglise keelt 
süvendatult. Selles mõttes inglise keel oli  minu jaoks nagu nagu eesti keele 
kõrval teine emakeel pea-aegu. Aga saksa keelt ma ei purssinud eriti üldse tol 
ajal. Tänapäeval on küll nii, et võtad raamatu lahti, mis sest et ma saksa 
keelt ei oska, aga  inglise keelega on palju sõnu samad, mingite muude tuntud 
keeltega on palju sarnasusi, nii et mingist üldisest mõttest saab aru. Muidugi 
selliseid konkreetseid juhiseid või mingit faktilist infot ma saksa keeles 
ikkagi ei loe. 

\question{Mis koolis sa käisid?}

44. Keskkool\index{Koolid!Tallinna 44. Keskkool}, mis on tänapäeval Mustamäe 
Gümnaasium\index{Koolid!Mustamäe Gümnaasium|see{Koolid!Tallinna 44. Keskkool}} 
ja seal ma õppisin inglise keelt. Aga kooli ajal see inglise keel, eriti 
it-maailmas, ei olnud mingi asi, mida oleks saanud väga palju rakendada. No 
peale selle, Basicu-s on käsk \verb|print| jah, on küll inglise keeles. Aga 
Basic-us neid käske nii väga palju ei ole ja neid ei ole keeruline ka pähe 
õppida juhul, kui sa inglise keelt ei valda. 

\question{Kas tehnikakirjanduse juurde käis ka mingisugune mingisugune muu 
kirjanduse või ulme huvi? Filmid,  raamatud?}

Kooli ajal ma lugesin üsna palju ulmet inglise keeles. Ja kooliajal sattus 
mulle kätte ka Douglas Adams ja tema Hitchhiker-i raamatud\sidenote{Vaata ka 
märkust leheküljel \pageref{sidenote!adams}.}, mida tal tol hetkel oli viis 
tükki. Kuna inglise keel oli koolis tol ajal süvaõppes, siis meil oli üks 
selline tund inglise keeles nagu inglise keele kodulugemine. Pidi kodus mingit 
ilukirjandust lugema inglise keeles ja tunnis  jutustama. Läksin raamatupoodi 
ja nägin kuskil üleval lae all riiuli peal seda Hitchhiker-i kõige esimest osa. 
Vaatasin, et see on huvitav pealkiri, ostsin selle raamatu ära ja mõtlesin, et 
võtangi siis selle inglise keele kodulugemiseks. See ei olnud väga mõistlik 
mõte sest kui mõelda nende sõnade pele, mida seal kasutatakse -- välja mõeldud 
sõnad, välja mõeldud liiginimed, seadmete nimed ja nii edasi -- siis need on 
eesti keelde üsna raskesti tõlgitavad,  peab väga hea fantaasiaga tõlkija 
olema. Aga mina hakkasin seda raamatut lugema ja siis õpetajale jutustama. Ma 
küll ei tea, kui palju õpetaja tol ajal sellest aru sai, mis ma talle 
jutustasin, aga vähemasti ta jäi rahule. 

\question{Seal ju ei ole narratiivi, on mingisugune keeruline sõlm, mis viienda 
raamatu lõpuks umbsõlme läheb!}

Ega mina ka ei saanud  sellest eriti väga palju aru, kui ma seda esimest osa 
kooli ajal lugesin. Hiljem lugesin  ülejäänud osad ka läbim siis nagu loksus 
see pilt paika.

\question{Aga oli siis saada ingliskeelset ilukirjandust?}

Jaa, ingliskeelset ilukirjandust oli küll. Seda oli igal pool, isegi Tallinnas. 
Pärast kooli lõppu, kui ma töötasin TPI-s\index{Tallinna Tehnikaülikool}, 
käisin ma palju Moskvas ja Leningradis komandeeringus ja seal oli valik üsna 
lai. Asimovilt ma vist ei sattunud esimesena lugema Asumi sarja\sidenote{Vaata 
ka märkust lehekülhel \pageref{sidenote!asum}.}, vaid mõnda muud sarja või 
üksiklugusid. Aga Asimov ja Adams olid nagu kaks põhilist ulmekirjanikku, 
kellega ma esmajoones kokku puutusin. 

\question{Aga vene klassikud? Strugatskid?}

Jaa, Strugatskeid ma olin lugenud varem, sest need olid tõlgitud eesti keelde. 
\enquote{Purpurpunaste pilvede maa}\sidenote{Arkadi Strugatski; Boris 
Strugatski. (1959). \begin{russian}Страна багровых туч\end{russian}. Eesti 
keeles 1961 Ralf Tominga (värsid Lembe Hiedel) tõlkes sarjas 
\enquote{Seiklusjutte maalt ja merelt}.} vist oli,  Amfiibinimene vist oli ka 
Strugatskite oma\sidenote{\begin{russian}Человек-амфибия\end{russian} on siiski 
1928. aastal ilmunud Alexander Belyaev-i romaan. Eesti keeles ilmus 1960. 
aastal \enquote{Seiklusjutte maalt ja merelt} sarjas koos romaaniga 
\enquote{Maailmavalitseja} (vene k. \begin{russian}Властелин мира\end{russian}, 
1926).}. Ja siis ma neelasin võimalust mööda igasugust pop-teaduslikku ehk 
aimekirjandust. \enquote{Mosaiik}\sidenote{\enquote{Mosaiik} oli kirjastuses 
Valgus aastastel 1973–1991 välja antud populaarteaduslike raamatute sari, mis 
käsitles äärmiselt laia teemaderingi ajaloost ja psühholoogiast topoloogia 
problemaatikani.}, selline raamatusari oli olemas. Ega suurt mingit muud 
aimekirjandust ei olnudki võtta.

Kriminullid olid teine valdkond, mis tol ajal peale  ulmekirjanduse 
ilukirjandusest huvi pakkus. Neid ma tollel ajal neelasin. Tempo oli selline, 
ma mäletan, et see vist oli Asumi mingisugune kolmas või neljas osa, mille ma 
vist lugesin Rootsis töötades ühe ööga läbi. Ma ei tea, kui palju lehekülgi see 
võis siis olla, mingi \emph{paperback}, kolm või nelisada lehekülge 
tõenäoliselt. Selliseid asju juhtus, sai tol ajal tollaste tööde kõrvalt ja 
tollase elu kõrvalt, kui  kool oli äsja lõpetatud, lubada. Selle asemel et 
öösel magada ja puhata, võtsid järgmise raamatu riiulist.

\question{Kui sa keskkooli ära lõpetasid, siis sa läksid TPI-sse kohe tööle või 
ikka õppima ka?}

Enam-vähem kohe pärast keskkooli ma läksin TPI-sse\index{Tallinna 
Tehnikaülikool} tööle. See töökoht tegelikult sattuski mulle kätte tänu sellele 
Vladimir Viiese\index[ppl]{Viies, Vladimir} juhendatud arvutiringile. Ma 
töötasin selles samas kateedris, kus Viieski tegelikult, see oli siis 
elektronarvutite kateeder\index{Tallinna Tehnikaülikool!Elektronarvutite 
Kateeder}. Aitasin seal igasuguseid arvuti hooldustöid teha, mingeid laborite 
häälestamisi ja  ette valmistamisi, mis erinevatel õppejõududele vaja oli. Ja 
mõnevõrra hiljem siis sai kaasa löödud juba mingisugustes arvutit hõlmavates 
projektides,  kui oli vaja midagi programmeerida, kui oli vaja mingi 
sisend-väljundseadme jaoks mingisugune draiver kirjutada. 

\question{Kas sa läksid õppima ka?}

Õppima, ma läksin mõnevõrra hiljem, kui ma sinna tööle läksin, sest mind 
punased ained ei  meelitanud eriti, keda nad oleks meelitanud. Aga mina tundsin 
nende vastu niivõrd suurt vastumeelsust,  et ma leidsin, et ma ei taha  üldse 
õppima minna, ka mingit tehnilist asja, kui seal on  punased ained juures. Need 
\enquote{punased} ained olid siis  NLKP ajalugu ja mingid sellised asjad. Aga 
mingil hetkel ikkagi paar aastat hiljem läksin õhtusesse osakonda õppima. Olin 
küll üks  üks enamusest, kes ei lõpetanud. Meie kursusele ühe astus sisse vist 
umbes kakskümmend viis inimest, kellest lõpetas kaks. 

\question{Mis eriala see siis oli?}

Elektronarvutid. Aga õhtuses osakonnas selline lõpetanute protsent oli minu 
arust üheksakümnendate algul üsna tavapärane. Oli vist 1990. aasta sügis, kui 
TPI-s õppima asusin. Sealt alates ongi tegelikult kõik need tööd ja samamoodi 
mingisuguseid vaba aja tegemised üsna palju olnud seotud programmeerimisega ja 
sellise arvuti tehnilise või riistvara poolega.

Kui  TPI-s sai töötatud, siis Elektronarvutite kateeder asus teisel korrusel. 
Samas korpuses neljandal korrusel asus siis Raadiotehnika 
kateeder\index{Tallinna Tehnikaülikool!Automaatikateaduskond!Raadiotehnika 
kateeder}, kus oli see Apple II\index{Arvutid!Apple II}. Meil tekkis 
Mastiga\index[ppl]{Kaal, Madis}, ehk Madis Kaal\sidenote{Kes sel ajal toimetas 
Raadiotehnika kateedris. Vt. lehekülg \pageref{sisu!mast_raadiotehnikas}.}, 
ühel hetkel kuidagimoodi mõte, et võiks proovida PC arvuteid, mis siis olid 
meil teisel korrusel kasutada ja mis oli minu igapäevane tööriist, kokku 
ühendada  Apple II-ga, mis oli neljandal korrusel Masti igapäevane tööriist. 
Ehitasime sinna vahele \emph{current loop}-i, no see on RS-232 põhimõtteliselt 
ainult natuke teistsuguse elektrilise signaaliga. Misjärel tekkis meil  selline 
pilv, et PC seest sai \emph{backup}-ida  andmeid Apple II sisse ja vastupidi. 
Nagu pilv ikka, kasutad kellegi teise arvutit. 

Üheksakümnes aasta oli minu arust ka see, kui Eestisse jõudis mingisugune info 
sellest, et on olemas BBS-id. TPI majas oligi Mast\index[ppl]{Kaal, Madis}  see 
entusiast, kes pani sealkandis esimese BBS-i jooksma. Mina esialgu vaatasin 
seda lihtsalt kõrvalt, mul ei olnud selle kohta nagu mingit arvamust,  ma ei 
tundnud  väga palju huvi selle kõige vastu. Seal sai mingeid faile vahetada, 
aga  ma ei ole näiteks kunagi mingi eriline mängu-fanatt olnud, järelikult mind 
ei huvitanud BBS-id, kust sai mingisuguseid mänge tõmmata, eks. See, kus ma 
leidsin, et need BBS-id võivad olla kuidagi kasulikud,  oli vist see moment, 
kui tuli välja, et seal BBS-ides on olemas  tekstifaile, mis on mingisugused 
referents-dokumendid, mingisugused \emph{manual}-id, mingid standardid, mingid 
programmeerimisõpikud kas IBM-ide või Apple jaoks.

\question{Kas need olid mingid \emph{plain text} failid või \LaTeX või mis?}

Need olid tekstifailid, aga nad olid natuke formaaditud ikkagi. Failides olid 
tabulatsioonid ja lehekülje vahed  sees, neid sai maatriksprinteriga välja 
trükkida nii, et tulid ikka ilusti formaadituna paberi peal välja. 
Maatriksprinterid olid kättesaadava hinnaga ja need olid enamuse arvutite taga. 
 Suured arvutid ehk siis ES-id SM-id, mis olid TPI-s või Küberneetika 
Instituudis, seal olid need laiad printerid. Ma ei teagi, kuidas nende 
printerite kohta öeldi. Ridaprinter? \emph{Line printer} öeldi inglise keeles, 
aga  oli mingisugune eestikeelne sõna ka mille vist Ustus Agur\index[ppl]{Agur, 
Ustus} välja mõtles. Ühesõnaga mingit koledat koledat häält ja värinat tegevad 
printerid.

\question{Kui sa aru said, et sealt saab igasugu dokumente, hakkasid BBS-id 
sulle ka huvi pakkuma?}

Jah, ma arvan, et see oli see hetk ja see ajanend kui ma leidsin, et sealt  
peale mängude ja  tilulilu sai midagi mõistlikku ka. Mingi hetk panin oma BBS-i 
ka püsti ja selleks ajaks oli ka Fidonet otsapidi Eestisse 
jõudnud\sidenote{Esimene Fidoneti Eesti regiooni 2:49 sisaldanud 
\emph{nodelist} on 271 28. septembrist 1990. Regiooni koordinaatorina on seal 
kirjas Andrus Suitsu\index[ppl]{Suitsu, Andrus} ja \emph{Host} on Tarmo 
Ausing\index[ppl]{Ausing, Tarmo}. BBS-idest on loetletud Hacker's Night 
System\index{BBS!HNS} (Tarmo Ausing), P.O.Box Maximus\index{BBS!P.O.Box 
Maximus} (Andrus Suitsu), Goodwin BBS\index{BBS!Goodwin} (Sulo 
Kallas\index[ppl]{Kallas, Sulo}), Mail Shark\index{BBS!Mail Shark} (Madis 
Kaal\index[ppl]{Kaal, Madis}) ja MamBox (Tarmo Mamers\index[ppl]{Mamers, 
Tarmo}).}. Väga paljud, kes nagu ajalooliselt on tagasi vaadanud ja rääkinud 
sellest ajast, ei pruugi eriti olla vahet teinud BBS-indusel ja Fidonetil,  mis 
tegelikult  olid kaks eraldi maailma. Vahe oli see, et BBS oli lihtsalt mingi 
süsteem, kuhu sai modemiga sisse helistada ja siis seal ringi toimetada,  
mingeid andmeid failide näol endale tõmmata või siis mingisuguseid sõnumeid 
vahetada. Aga kogu see info ja need sõnumid olid salvestatud sinna ühte 
konkreetsesse BBS-i süsteemi.

Fidonet aga sai ühe otsaga  alguse nendest samadest BBS-idest, aga tema eesmärk 
oli sõnumite BBS-ide ja mingite muude Fidoneti liikmete süsteemide vahel 
edasi-tagasi toimetada. 

\question{Ehk, Fidonetis need kohad, kuhu sa sisse helistasid, helistasid ka 
üksteisele sisse ja vahetasid andmeid?}

Jah. Ja see oli siis juba automatiseeritud süsteem, kus olid automaatvahendid 
selleks, et sõnumeid, ehk  meile, valetada. Ja meile oli kahte liiki: olid 
privaat-meilid ja olid konverents-meilid, mis siis on tänapäeva mõistes 
meiligrupid või meililistid.

\question{Kas \emph{Usenet} tekkis ka sel ajal?}

Usenet oli varem, see on hästi vana asi. Usenet ja UUCP protokoll sellega 
seotuna on põhiliselt  Unixi-maailma päritolu, ehk siis see oli konkreetselt 
Unixi arvutite vahelise meilivahetuse protokoll. Ja see Usenet, mis  sinna 
ümber tekkis,  see oli siis ka nagu selline konverentside või vestlusringide 
süsteem.

\question{Kas seda peegeldati Fidosse ka?}

Ja seal olid lüüsid olemas. Usenetist sai konvertida ümber kirju Fidoneti 
\emph{echo}-desse või meilikonverentsidesse. Muu hulgas ka faile, sest et 
Usenetis vahetati ka väga palju faile,  neid oli siis võimalik ka konvertida 
tavalisteks failideks, mis siis kuskil BBS-is üles pandi.

\question{Kas eestlased toimetasid seal usenetis mingites oma gruppides või 
möllati olemasolevates?}

Usenetis ma mäletan küll, et ei olnud mingeid erilisi Eesti spetsiifilisi või 
regionaalseid gruppe. Erinevalt Fidonetist, seal oli küll mingi viisteist või 
heal ajal võib-olla kakskümmend lokaalset  vestlustgruppi ehk \emph{echo}-t. 
Neist kaks-kolm gruppi olid üsna populaarsed liikmeskonna mõttes.

\question{Mis see tähendab? 50, 100, 500 liiget?}

No ma arvan, et lugejaid võis seal oli väga palju, sest et pidevalt tuleb välja 
inimesi, kellega mina ei ole kunagi kokku puutunud, ma ei tea neid nimepidi, 
aga nad räägivad, et nad on kunagi sealt \emph{echo}-st  midagi lugenud. Sest 
tegelikult selleks, et neid \emph{echo}-sid või konverentse lugeda,  ei pidanud 
sa ise omama ei BBS ega mingit Fidoneti süsteemi. Sa said helistada BBS-i 
sisse, seal lugeda, kui sa tahtsid, ja kirjutada. Kui kellelgi Fidoneti süsteem 
oli püsti pandud, siis selle eelis oli selles, et siis talle need kirjad tulid 
automaatselt koju kätte ja tal ei olnud lugemiseks-kirjutamiseks vaja kuskile 
kaugele ise helistada. Tegelikult see ring  neid inimesi, kes  ainult luges 
võis olla päris suur. 

Kui püüda hinnata seda, kes seal aktiivselt suhtlesid ja kirjutasid ka, siis  
võib-olla see on mingi kakssada inimest. See on väga laest võetud number, 
suurusjärgus.

\question{Seda on ikkagi päris palju. Kas sa oma Fido \emph{node} panidki püsti 
selle jaoks, et asjad tuleksid koju kätte? Mis selle asja nimi oli?}

Ma arvan, et eesmärk oli jah see, et asjad oleks piisavalt automatiseeritud, et 
mul ei oleks endal vaja mingeid liigutusi teha  ja aega viita selle pärast, et 
kuskile BBS-i nii-öelda löögile saada. Sest kui BBS-i küljes oli  
välismaailmaga suhtlemiseks üks modem, siis see tähendab, et igal ajahetkel sai 
seda BBS-i kui teenust korraga kasutada üks inimene. Oli BBS-e, millel oli mitu 
modemeit küljes, siis sai mitu inimest seda paralleelselt kasutada.  Aga kõik 
see tähendaski, et helistasid modemiga, telefon oli kinni. Helistad viie minuti 
pärast, ikka kinni. Ja no miks ma pean niimoodi vaeva nägema ja pidevalt 
helistama? Tõsi küll, modem valis ise automaatselt, tegi kordusvalimist, eks, 
ja kui lõpuks löögile sai, andis mingi signaali. Aga ma leidsin, et parem on 
seda asja lasta  Fidoneti automaatikal teha. Ja siis  saab  rahumeeli hetkel, 
kui sa tahad, avada meililugemise programmi ja lugeda seda meili, mis on 
vahepeal sul sinna masinasse ära tõmmatud. 

\question{Aga mis su \emph{node} nimi oli?}

Minu \emph{node} nimi oli MamBox. Ma ei mäleta, mis hetkel see eesliide, mis on 
siis tulnud minu perekonnanime algusest,  hakkas mingisuguste asjade külge 
tekkima. Aga tol hetkel oli jah nii, et kui ma tegin BBS-i, siis ta oli MamBox, 
kui ma kirjutasin mingisugust programmi  oma lõbuks, siis siis ma kirjutasin 
\enquote{\emph{Copyright MamSoft}}\sidenote{Tegu oli levinud praktikaga, minu 
samal viisil kasutatud fiktiivne firmanimi oli \enquote{\emph{I \& I Company}}. 
Sellest \emph{misasi} üks firma on, oli arusaam ähmane. Sellest, et firma 
\emph{nimi} tuleb kindlasti ära mainida ja kuulsaks teha, oli arusaam väga 
konkreetne.}. See oli tol ajal selline kaubamärk, mida ma kasutasin selliste 
ühesuguste eesliidetega. Üsna tüüpiline oli see, et kui kellelgi oli BBS, siis 
ta mingil hetkel lisas sinna  Fiodneti funktsionaalsuse isegi, kui tal seda 
alguses ei olnud. Ja väga palju oli ka teistsuguseid suundumusi, et kui sul oli 
mingil põhjusel tekkinud Fidoneti \emph{node}, siis väga palju nende  omanikest 
mingil ajal leidsid, et võiks  ka BBS-i püsti panna. 

Muidugi oli väga palju ka Fidoneti \emph{node}-sid, kelle omanike või siis 
\emph{sysop}-ide eesmärk oligi lugeda-kirjutada ja automaatselt lasta sõnumeid 
vahetada,  nende huvi ei olnud mingisugust BBS-i üleval pidada.

\question{Ehk, kui mõni BBS sai populaarseks siis võis see olla nii seepärast, 
et seal vahetas aktiivne kogukond omavahel faile kui ka see, et miskipärast 
otsustasid paljud kasutajad just sealtkaudu Fidonetile ligi pääseda?}

Fidoneti juurde pääses kõikidest BBS-idest, kes olid Fidoneti liikmed, sest 
kõigis oli põhimõtteliselt ühesugune koopia nendest konverents kirjedest. 
Iseasi olid privaatkirjad, siis oli  vaja Fidoneti \emph{node} numbrit teada, 
kuhu saab kellelegi inimesele kirja saata. Iga inimene oli  mingisuguse 
Fidoneti \emph{node}-ga seotud, et  privaatmeili vahetada. Aga mis konverentse 
või \emph{echo}-sid puudutas,  siis need  olid ühtmoodi igas BBS-is  saadaval.

Aga ega muidugi ei olnud eriti mõnus ka see, et täna loed siit, homme  hoopis 
teisest BBS-ist seda meili. Olid ju viited, kui palju sul on loetud meile, kus 
su lugemisjärjekord on, kas sa oled millelegi vastanud või ei ole. See  läheb 
sassi, kui sul ei ole oma sellist nii-öelda kodu-BBS-i. Ja oli ka selge, et kus 
oli väga populaarne faile käia tõmbamas,  need BBS-id on  üsna hõivatud ja 
tihtipeale kinni nende failide tõmbamise pärast. 

\question{Faili tõmbamine võttis ju tükk aega!}

Jah. Alguses, kui BBS-id Eestisse tekkisid ja need Fidoneti \emph{node}-d, 
siis, ütleme niimoodi, et 14 400 boodi (ümmarguselt võib seda teisendada 14 400 
bitti või siis 14 kilobitti sekundis) andmevahetuskiirus oli üsna tüüpiline  
algupäevadel nende BBS-ide juures.

\question{Ma isegi mäletan 9600-seid miskipärast}

9600 oli jah selline lihtne, odav igamehe tehnoloogia. Aga, ütleme, 14.4 olid 
sellised modemid, kuhu poole kõik nagu püüdlesid. Ja sealt edasi tuli siis 
19.2, 26.6, mingid sellised numbrid. Minul ühel hetkel oli kasutada sellised 
üsna  härjad modemid, mille töökiirus oli 33 600 boodi. Aga see kiirus tuli 
kätte ainult sellisel juhul, kui teisel pool sideliini otsas on vastas täpselt 
sama tootja modem. Modemi  nimi oli Trailblazer\index{Telebit 
Trailblazer}\sidenote{USA tootja Telebit, kes Trailblazeri sarja tootis, 
kasutas standardsete V-seeria protokollide asemel oma protokolli Packetized 
Ensemble Protocol (PEP).}. US Roboticsid\index{US Robotics}  töötasid BBS-ide 
nii-öelda  põhiajastul kõige kiiremini vist 34.4 kiloboodi juures.

\question{Kas BBS-idega majandamine tekitaski sul võrgu-huvi? Sa rääkisid, 
kuidas te Mastiga Apple-t ja PC-d paaritasite?}

Ma arvan, et see Apple ja PC paaritamine oligi see, mis  võrgunduse kui sellise 
pisiku tekitas, sest ega TPI-s ega ka kuskil mujal, kus arvutitega sai kokku 
puututud, ei olnud mingisuguseid erilisi kohtvõrgutamise tehnoloogiaid 
kasutusel. Ainukene olid UUCP, mis käis Unixite vahel, see oli rohkem nagu 
selline tõsisemate ja suuremate arvutite sidepidamine  ja rohkem nagu teadus- 
ja akadeemilistes ringkondades, eks. Ja teisalt siis Fidonet, mis oli selline 
rohkem asjaarmastajalik. Pärast TPI-d järgmises töökohas ma  puutusin esimest 
korda kokku ARCNetiga\sidenote{ARCNet oli 1980. aastatel levinud esimene laia 
kasutamist leidnud mikroarvutite võrgusüsteem. ARCNet on siiani kasutusel 
sardsüsteemide puhul.}.

\question{Kus see oli ja mis aastal umbes?}

See oli  aasta 1991, selline ettevõte nagu Skriining\index{Skriining}, mis 
eksisteerib tänapäeval ka. Skriiningus ma puutusin kokku ARCNetiga, mis jooksis 
tol ajal kahe ja poole megabitise kiiruse peal. See oli koaksiaalkaabli võrk, 
pea-aegu nagu esimesed Etherneti võrgud aga, ütleme, neli korda aeglasem. Minu 
arust see koaksiaalkaabel, mida ta kasutas, oli ka vist seitsmekümne viie 
oomine, ma arvan, versus Etherneti viiekümne oomine kaabel. 

Aga noh, see ARCNet oli  üsna lühiajaline, temaga olid kokkupuuted peamiselt  
tänu sellele, et see oli  aeg, kus Soomest ja mujalt lähi-välismaalt 
seljakotiga kraami toomas käidi. Väga palju kraami, mis Soomest tuli oli 
selline kraam, mis Soomes oli maha kantud, seda ei tahetud seal ära visata, 
sest  utiliseerimine maksis, siis anti ära, et \enquote{kasutage, tehke 
midagi}. Ma ei mäleta, et ARCNetiga midagi väga tõsist oleks tehtud, aga 
mingeid kokkupuuted sellega ikkagi olid. Selle peale tuli siis Ethernet, mis 
oli tol ajal koaksiaalkaabli Ethernet, kümme megabitti sekundis. Mis oli juba 
selline asi, mis hakkas päris reaalselt ettevõtetesse jõudma ja mille peale 
hakati tegelikult üsna palju  kohtvõrke ehitama.

\question{Räägi korraks palun sellest Skriiningust\index{Skriining}. Arvutiäri 
jaoks peaks nagu nime järgi olema kaks poolt: arvuti ja äri. Aga et aastal 1991 
oleks kumbagi olnud, tundub natuke uskumatu.}

Noh, arvutid olidki sellised, mis alguses tulid seljakotis piiri tagant. Ja 
järgmine faas oli see, kus  nad tulid endiselt seljakotiga piiri tagant, aga 
selleks, et neid saada, selleks oli vaja sinna piiri taha seljakotiga 
kõigepealt sularaha viia. Sest kakskümmend tuhat rubla, ma arvan, võis olla  
selline keskmise arvuti hind. Ma ei tea, mina ei muutunud hindadega kokku, sest 
ma ei tegelenud müügitööga. Nii et ma ei kujuta ette, kui palju  arvutid tol 
ajal  numbriliselt maksid, aga arvutustehnika oli veel meeletult kallis.

\question{Kuidas sihuke firma üldse võis tekkida tol ajal? Ei saanud ju 
internetti kuulutust panna, et \enquote{tulge meile tööle}?}

Ma ei tea, IT-maailmas inimesed liikusid ilmselt tutvuste kaudu ühest kohast 
teise tööle. Ja mina sinna Skriiningusse\index{Skriining} jõudsin ka  tutvuste 
kaudu, sest et üks inimene, kes varem oli olnud minu kolleeg TPI-s sattus 
sattus Skriiningusse tööle ja kutsus paar aastat hiljem mind ka sinna. 
Skriiningu nii-öelda vertikaal või kliendisegment oli ja on ka tänapäeval 
meditsiiniasutused ja meditsiiniasutuste võrgud, arvutibaas ja infosüsteemid, 
nende kirjutamine ja hooldamine. Ma arvan, et see on ka üks põhjus, miks 
Skriining on tänapäeval  endiselt elus ja ilmselt elab väga hästi: tal on oma 
üsna kitsas kliendisegment ja kindlad ja väljakujunenud kliendisuhted.

\question{Sinu jutu järgi tundub, et need esimesed arvutifirmad olid sõprus- 
või vähemalt tutvuskonna põhised?}

Nad tegelikult väga ei olnud. Sest Skriiningus see Mart, kes enne mind sinna 
läks ja kes mind hiljem kutsus, oli ainukene inimene, keda ma seal tundsin. Aga 
jah, sellised arvutifirmad ei olnud suured. Skriining oli, ma ei tea, viis-kuus 
inimest tõenäoliselt, mitte rohkem. Kõik tegid  enam-vähem kõike. Võib-olla 
mõni jah programmeeris rohkem, võib-olla mõni, nagu mina näiteks, vedas rohkem 
kaablit või käis seal mingeid kruvisid keeramas või timmimas mingeid asju seal 
arvuti kaane all. Mingid eelistused olid kindlasti inimestel olemas, aga 
üldjoontes võib öelda, et kõik käisid nagu mingil määral vähemalt üle kõikidest 
 süsteemidest, mis firma  sees kasutusel olid või millega see firma tegeles.

\question{Kas sa sel ajal veel oma BBS-i ka pidasid?}

Jah. BBS  oli mul üleval päris pikka aega, ma olen teda nii-öelda kaasa vedanud 
 ühest kohast teise, sest ega tol ajal kodus ei saanud teda pidada. Noh, 
esiteks koju ei olnud kellelgi eriti võimalik arvuti hankida, see oli kallis. 
Ja kui oli ka võib-olla võimalik mingi niru arvuti hankida, siis selle  
võib-olla BBS-i hästi püsti ei pane. Ja teisalt, tol ajal kodus telefoniga 
välja helistamine ei olnud just mitte kõige odavam lõbu. Pealegi, kui mõelda 
Fidoneti peale ja et see Fidonet  oli ülemaailmne süsteem, siis Fidoneti side 
hõlmas ka mingit hulka rahvusvahelisi kõnesid. Sel ajal kodustelt numbritelt ei 
olnud reeglina võimalik otse  välismaale helistada, kaugvalimine toimis läbi 
inim-operaatori\sidenote{Jaan Tallinn\index[ppl]{Tallinn, Jaan} on rääkinud, et 
nondele inim-operaatoritele oli täiesti võimalik arvutiside vahendamine ära 
õpetada. Tuli öelda, et \enquote{kui vilistama hakkab, ühendage ära, nii peabki 
olema}.}. Ja ega ka kõikidest ettevõtetest  ei olnud võimalik välismaale 
helistada. Tihtipeale oli ettevõttes üks telefoninumber, võib olla mingi kümne 
või saja telefoni peale, kust sai otse välismaale helistada. Seda siis püüti 
endale ära rääkida, et sinna taha saaks BBS-i ühendada. Tihtipeale olid ka 
BBS-i omanikel kokkulepped, et nende BBS töötab ja saab telefoniliini kasutada 
öösiti, ja päeval saab seda liini kasutada kontoritööks, inimkonna teenimiseks. 
Sellised ajad tekkisid hiljem, et BBS-i jaoks oli mõnedes firmades võimalik 
saada kakskümmend neli tundi telefoniliin ja eriti hästi, kui sealt sai ka 
välismaale helistada. Selliseid kohti oli. 

Ja noh, tol ajal oli nii, et kui mina liikusin ühest ettevõttest teise, siis 
siis ma uut ettevõtet muuhulgas hindasin ka selle järgi, et kas mul on võimalik 
BBS sinna kaasa võtta ja kas mul on seal võimalik selle BBS-i jaoks saada  
kaugvalimisega telefoniliin ja veel parem, kui see liin oleks  kakskümmend neli 
tundi kasutav.

\question{Need on ju päris olulised valikud, mida see BBS-i kaasa vedamine 
sulle pakkus? Mis selle juures huvitav oli?}

No see, et info tuleb üsna lihtsalt kätte, mida on võimalik BBS-idest saada. 
See info on parajalt lihtsalt otsitav, kui sul juba Fidoneti \emph{node} püsti 
on. ja automatiseerida sai ühelt poolt meili-  aga teiselt poolt ka 
failivahetust. Kui ma tahan saada ka kätte kuskilt kaugelt BBS-ist mingit 
faili, ma tean selle faili nime, siis mul ei ole vaja endal käsitsi jällegi 
sinna BBS-i sisse logida, et seda faili endale tõmmata, vaid ma saan seda teha 
Fidoneti automaatika kaudu.

\question{Toonases Fido maailmas toimetav seltskond oli ikkagi suhteliselt suur 
ja sinu nimi jookseb nende juttudest päris oluliselt läbi. Miks see nii on?}

Seal ei ole väga palju midagi arvata. Tegelikult on olemas palju nimekamaid 
BBS-i pidajaid, kes seda  BBS-i maailma Eestis põhimõtteliselt  alustasid ja 
kes on BBS-i kontekstis palju tuntumad. Kui need BBS-id ja Fidonet  olid Eestis 
levima hakanud ja üsna agarasti kasutusele võetud, siis üsna pea tekkis meil 
mingis Fidoneti inimeste seltskonnas  selline äratundmine, et nojah, et meid on 
siin küll mingisugune sada kuni kakssada inimest, kes igapäevaselt   Fidoneti 
kaudu suhtlevat ja kirju vahetavad ja teevad nalju ja vahetevahel sõimavad 
üksteist ja mida iganes. Aga noh, meie siin  näeme kümmet inimest päevast 
päeva, võib-olla, keda me teame nime- ja nägupidi, aga teisi me ei tea. Et 
peaks sellele probleemile mingisuguse lahenduse otsima. 

Tegelikult oli juba 1991. aasta, kui kui see probleem muutus niivõrd teravaks 
et enam-vähem siis seesama  umbes kümne inimeseline seltskond mõtles, et võiks 
teha mingisuguse kokkutuleku. Ma küll ei mäleta, kuidas need mõtted käisid või 
liikusid või kes mingisuguseid ideid välja käis. Või kui palju me kuskil 
Fidoneti \emph{echo}-des neid asju arutasime või mõtlesime enne, kui me selle 
mõtte välja käisime, et, davai, teeme mingisuguse kokkutuleku.

Oli üheksakümne esimese aasta augusti esimene pool, ma ei mäleta täpset 
kuupäeva, kus me olime  paika pannud, et nii, teeme  ühel nädalavahetusel 
kokkutuleku, saame Väänas ühes ürituste kohas kokku. Mingi osavõtumaks oli ka,  
ma ei mäleta, mingi viiskümmend rubla,  võib-olla vähem\sidenote{Siinkohal 
oleks ehk lugejale kasulik selgitada, kui suur või väike raha oli 50 rubla 
1991. aastal. Paraku on see üsna keeruline, sest sel ajal valitses Eestis 
hüperinflatsioon ja hinnad kerkisid kiiresti. Lisaks olid enne 1992. aastat 
teatud kaupade hinnad riikliku kontrolli all ja teiste omad vabad. Kõigele 
lisaks ei olnud paljusid kaupu mis iganes hinna eest saada, puudus valitses muu 
hulgas ka näiteks sularahast ja toimis elav ning väga volatiilsete hindadega 
must turg. 50 rubla eest võis saada 20 kilo kartulit aga võis saada ka ühe 5.25 
tollise flopi.}. Plaan oli, et räägime  ja suhtleme ja mängime   
IT-kalduvustega mänge. Mitte arvutimänge, aga noh, flopi heide ja 
kõvakettaheide ja mingid sellised asjad on olnud nende ürituste kavas. 

\question{1991. aastal kõvaketast ikka andis heita!}

Tol aastal oli kõvakettaheide ilmselt kavas küll aga see ei olnud päris 
tänapäevane kõvaketas, vaid siis olid sellised suured 19 või 21 tolli 
läbimõõduga plaadid, mis moodustasid kõvaketta, aga mis ei olnud kuskil 
hermeetilises korpuses nagu tänapäevased pöörlevad kettad. Neid oli, ma ei tea, 
kaheksa või kümme kokku pandud ühe sellise käepidemega varre külge ja neid sai  
kettaseadme seest välja tõsta\sidenote{Sellised kettapakid, näiteks IBMi 1316, 
suutsid talletada suurusjärgus mõned megabaidid infot ja olid tolleks ajaks 
selgesti iganenud. Eestisse sattusid sedalaadi seadmed tõenäoliselt 
humanitaarabina, mis tõi meie kanti hulganisti kummalist vananenud riistvara. 
Mäletan ühte sellist kettalugejat 1992. aastal Võru I 
Keskkoolis\index{Koolid!Võru Kreutzwaldi Gümnaasium} ka päriselt toimimas. 
Arvutiklass asus teisel korrusel ja kui kettaseade sisse lülitati, oli undamist 
tänavale kosta - selle järgi sai hinnata, kas klassis parajasti oli keegi või 
mitte.}. Sealt lahti lammutatud kettaid me lennutasime küll sellel esimesel 
kokkutulekul. 

Me mõtlesime selle kokkutuleku nimeks välja BBSummer\index{BBSummer}. Ehk siis 
\enquote{BB} lühendist BBS ja siis \enquote{Summer} sinna taha. Üks aasta varem 
oli toimunud esimene Rock Summer\sidenote{Rock Summer oli 1980. aastate lõpus 
ja 1990. aastatel Tallinnas Lauluväljakul toimunud muusikafestival, mille 
mitmekülgset mõju ei saa kuidagi üle hinnata. Tegu oli esimese suurema 
rokifestivaliga siinkandis ja selle platsil valitsenud atmosfäär oli keskmisele 
nõukogude noorele, ütleme, radikaalse mõjuga. Kuna tegu oli ühega esimestest 
võimalustest piiluda raudse eesriide taha, meelitas festival kohale ka küllalt 
nimekaid Lääne ansambleid.}, aga nii palju, kui me oleme erinevate inimestega 
meenutanud, see Rock Summer ei olnud kuidagimoodi selle \enquote{summeri} osa 
algataja või põhjustaja sinna nime sisse, meil olid sõltumatud kaubamärgid. 

Kui see 1991. aasta BBSummer toimus\sidenote{Tarmo on saatnud esimese BBSummeri 
(ametliku nimetusega \enquote{Eesti amatöörarvutivõrgu kasutajate I seminar-laager}) 
kutse 12. juulil 1991 ja üritus ise pidi toimuma 26.-27. augustil Tugamanni Tuulikus
 (ametlikult EPT Tallinna osakonna puhkekompleks).}, 
 siis, kes teab natukene rohkem ajalugu või 
on ise tol ajal noor olnud või ka natukene vanem, siis põhimõtteliselt sel 
nädalal,  viis päeva enne, kui BBSummer oleks pidanud toimima, oli see aeg, kui 
Vilniuses  tulid tankid tänavale. Ja mina selle peale ütlesin, et  meie teeme 
oma BBSummeri ära ka juhul, kui kui ei ole mingisuguseid liiklust segavaid 
tanke tänavatele, olukord oli üsna pingeline ja keeruline. Esimesel BBSummeril 
oli vist viiskümmend kuus osalejat, pluss-miinus, umbes selline number. Suur 
hulk oli muidugi neid, kes seal olid puhtalt  sellepärast, et nad olid Fidoneti 
\emph{sysop}-id. Aga ma arvan, et  üle poole ilmselt oli seda rahvast, kes  
olid mingid BBS-i lihtkasutaja, kes lihtsalt igapäevaselt tõmbas faile ja 
vahetas meile ilma et tal endal oleks BBS-i olnud.

\question{See oli üsna korralik suhe teenuse pakkujate ja tarbijate vahel, 
BBS-i pidamise barjäär oli kõrge ja seltskond seega üsna tehniline?}

Ta oli jah ilmselt mõõdukalt tehniline. Kes tundis, et tehnika võib-olla temast 
üle on, ei pidanud tõenäoliselt BBS-i. Ja ei olnud päris triviaalne, et BBS 
häälestada ja korralikult tööle panna ja võib-olla sinna see Fidoneti 
automaatika käima panna. Internetist juhendvideot vaadata ka ei saanud. Küll 
aga sai lugeda mingisuguseid tekstifaile  selle kohta, et \enquote{tõmba see 
softi ja tõmbas see soft ja see soft ja siis pane nad kõik niimoodi kokku ja  
tee sellised ja sellised konfifailid ja siis läheb asi käima}.

\question{Siis sündis ju FAQ, \emph{Frequently Asked Questions}, mis praegu on 
lihtsalt mingi osa veebilehest. Siis oli tegemist konkreetse eraldi leviva 
failiga, kuhu jõudsidki \emph{echo}-des ja uudisgruppides  sagedasti küsitud 
küsimused koos pädevate vastustega.}

Jah, neid  olid küsimused-vastused, kuidas asi käima panna, kui midagi ei tööta 
või on sellised sümptomid, siis mida tuleks vaadata ja nii edasi.

\question{Kas sedalaadi sisu Eestis ainult tarbiti või panustati sinna tagasi 
ka?}

Jah, kui nii-öelda toodeti mingit sisu, ehk siis, kui keegi kirjutas mingit 
programmi, mis ei olnud  päris oma tarbeks mõeldud ja see ei olnud mingi mäng 
vaid see oli näiteks mingi funktsioonide või alamprogrammide teek ehk siis 
\emph{library}. Mäletan Mast\index[ppl]{Kaal, Madis} tol ajal kirjutas 
tekstiliideste tegemiseks ühe funktsioonide teegi, millega sai teha menüüsid ja 
kaste ja igasuguseid asju ekraani peal. Tekstirežiimis, aga hiir oli  abiks, 
sallega sai menüüdes ringi klikata. Jah, selliste asjade jaoks ikka oli 
mingisugused FAQ-d või mingid lihtsad juhendid olemas igal vähegi mõistlikumal 
autoril. Sest mängud olid küll sellised jah, et võtad flopi ja installid, või 
lased mängu käima ja siis vaatad, et kuidas ta tööle hakkab ja mida mingi nupp 
teeb. Ma arvan, et ega mängude manuaale ilmselt keegi eriti ei lugenud.

\question{Pärast Skriiningut sa jõudsid mingi hetk Unineti ka?}

Jah, mingisugusel ajal on Uninet  olnud minu tööandja küll.

\question{Kas päris alguses või kunagi hiljem?}

See vist oli umbes neljas töökoht. Pärast Skriiningut ma sattusin Baltic 
Computer Systems-isse\index{Baltic Computer Systems}, mis ka tänapäeval 
eksisteerib. Ja BCS-is ma tegelesin  nüüd üsna sihituna juba arvutivõrkudega, 
ehk siis meil oli arvutivõrkude osakond ja me  ühelt poolt tegelesime 
kaabeldusega ja teiselt poolt  ka serverite ja mingil määral ka sellise 
tarkvaraga, mis  oli vaja võrgus käima panna. Näiteks andmebaasid, mis olid 
mõeldud algselt ühes arvutis kasutamiseks, aga mis siis kuskil ettevõtetes oli 
vaja niimoodi käima panna, et nad töötaks võrgus. 

\question{Vahemärkusena, tol ajal enamus andmebaase olid mõeldud käima ühes 
arvutis. See tähendas, et mingisuguseid transaktsioone või midagi ei olnud 
keegi sinna sisse ehitanud}

Nojah, seda otseselt ei olnud olemas, aga oli mingisuguseid viise, kuidas 
sellest mööda hiiliti, et  kui andmebaas arvutis lahti teha, et ta ei oleks 
mitte lukus võrgus kõikide kasutajate jaoks vaid et seal saaks ikkagi midagi 
teha. 

\question{Mis sa praegu teed?}

Vahepeal ma olen teinud igasuguseid muid asju, mis ei ole olnud väga sellise 
võrgu tehnilise ülesehitusega seotud, vaid mis on rohkem seotud võrgu 
rakenduste ja turbega. Nüüd paar kuud ma olen uuesti jälle mõnes mõttes 
sattunud tagasi sellise tegevuse peale, mis on seotud taas kord võrgu 
baasprotokollidega. Ehk  ma peaks praegu une pealt teadma, kuidasmoodi erinevad 
IP,  TCP  ja UDP kihi protokollid töötavad. Nii palju on minu praegusel 
tegevusel ka endisel seos muidugi mingite rakendusprogrammidega ja äppidega  
mobiilide sees, et otsapidi minu töö on ka teada ja vastavalt sellele 
seadmisele siis toimetada sellega, kuidas need äpid võrgus käituvad. Mismoodi 
nende liiklus on või  andmevahetus on võrgus üles ehitatud, kuidas seda 
andmevahetust ohjes hoida, kuidas seda juhtida. Kuidas hakkama saada sellega, 
mida Google pidevalt uute protokollide näol välja pakub ja mille eesmärk 
loomulikult on see, et kasutajal oleks Internetis turvalisem. Minu eesmärk on 
see, et lisaks turvalisusele peab kasutajal seal Internetis ka olema mugav. Ehk 
siis andmed peaks jõudma ühest punktist teise nii kiiresti, et kasutaja 
teadvusta, et  kuskil seal vahepeal Internet on, mis võib olla ebatöökindel või 
aeglane.

\question{Siis on ju selles mõttes toredasti, et kui sa alguses rääkisid huvist 
mängus tegelasele kapoti all toimetades teist värvi müts pähe panna, siis 
praegu lihtsalt see tegelane teistsuguse arvuti sees ja kapotialune on natuke 
keerulisem aga ülesanne on suuresti sama}

Täpselt nii. Minu jaoks on oluline see, mis on karul kõhus. Kuidasmoodi see 
seal kõhus töötab. Kui ta ei tööta hästi, kas ja mida  saab paremaks teha. Ja 
noh, kui ta töötab hästi, siis  sellest hoolimata kindlasti saab midagi 
teistmoodi teha.

\question{Ja kui hästi töötab, on ju huvitav, et kuidas?}

Jah, kuidas töötab ja miks  ta nii hästi töötab.
