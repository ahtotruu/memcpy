\index[ppl]{Mamers, Tarmo}
\index[ppl]{MomraT|see{Mamers, Tarmo}}




\question{Kuidas sina arvutite juurde said?}

Mul oli üks klassikaaslane, kelle isa töötas Küberneetika Instituudis\index{Küberneetika Instituut}. Ma arvan, et see võis olla kuskil keskkooliaastate alguses, ilmselt siis tol ajal seitsmes-kaheksas-üheksas klass, kus sai käidud päris mitu korda järjest tutvumas sellise asjaga nagu Apple II\index{Arvutid!Apple II}. See oli küll mõnevõrra keeruline, sest et see Apple oli üsna koormatud, kuna seal Küberneetika Instituudis teda kasutati mingisuguseks teadus ja uurimistööks. Ma küll ei tea täpselt mil moel. Ja kuna selle Apple II monitori asemel oli tavaline telekas siis ma mäletan väga hästi seda, et kui ma esimesi kordi sinna sattusin mingil talvisel perioodil, siis põhiliselt seda telekat kasutati vist suusahüppe MMi jälgimiseks. Ehk päris alati ei saanud seda Apple II-te näperdada  või kui sai, siis ilma pildita. See oli  tõenäoliselt umbes aastal 1983 kuni 1985, umbes selline ajavahemik. 

\question{Mis sa tegid selle arvutiga?}

No mina kõigepealt vaatasin, mis teised teevad ja teised põhiliselt mängisid igasugu huvitavaid mänge, mis seal Apple peal tol aja olid. Kui ma ise hakkasin seda näperdama, siis mind pigem hakkas huvitama see et, kui need arvutimängud on mingi teatud hulga eludega ja teatud hulka relvadega ja teatud hulga mingit abivahenditega, mida saab kasutada, kas neid kuidagi nagunii-öelda ära häkkida saaks? Et võib-olla oleks rohkem elusid,  võib-olla oleks rohkem abivahendeid, kuidagimoodi saaks rohkem punkte näiteks. Millegipärast minu huvi oli nagu hoopiski selline. Mitte  mängimine kui, kui tegevus, vaid pigem millegi ümber tegemine, millegi kuidagi teistmoodi tegemine. Võib-olla, et mingi tegelase müts ei oleks mitte punane vaid oleks hoopis rohelin kuskil mängus. No midagi sellist. Kuidagi  nikerdada  et, midagi muutuks ja saaks teistsuguseks. 

Ja enne seda, ma ei teadnud midagi ei programmeerimisest ega ka eriti arvutite tööpõhimõttest. See oli nagu see esimene aiendav tegur, mis siis tõi kokkupuute esiteks BASIC-u\index{Keeled!BASIC} keelega ja järgmisena siis Apple assembleriga\index{Keeled!Assembler}. Kuna ma natukene tundsin huvi tol ajal ka elektroonika ja eriti digitaalelektroonika vastu, siis juhtus nii, et mulle jäid ette ka selle Apple arvuti on manuaalid. Tollaste Apple või muude arvutitega olid alati manuaalides kaasas nende elektroonikaskeemid. Põhimõtteliselt minu huvi oli sealt siis  näpuga järge ajada, et et kuidas need bitid liiguvad, kui midagi printida või nuppu vajutada. Või kuidas, kuidas ekraani peale pilt tekitatakse bittidest. See oli algus.

\question{Kas need mängud olid kuskilt väljastpoolt tulnud või liikus ka ise tehtud asju?}

Apple'i peal olevad mängud olid välismaalt tulnud kas sama kanalit pidi, kust need Apple-d kunagi olid Nõukogude Liitu tulnud. Või siis osad nendest, eeskätt need Apple'i kasutajaid, kes olid kuskil Tartus või Nõos, kellel oli kuuldavasti ja mäletatavasti ka mingisuguseid kontakte teiste Apple'i kasutajatega mujalt maailmast, said mänge kuskilt mujalt ka.

\question{Tehti ju ise ka algelisi mänge, mingis Juku\index{Arvutid!Juku} mängus sai mõisa majandada, ma mäletan}

Sellest Apple'i ajast ma ei mäleta väga palju mingisuguseid kodukootud mänge, küll mõnevõrra hilisemast ajast, siis kui mul oli kasutada selline nõukogude päritolu arvuti nagu Iskra-226\index{Arvutid!Iskra!226}\sidenote{\begin{russian}Искра 226\end{russian} oli Nõukogudemaal toodetud arvuti Wang 2200 kloon, mis oli originaaliga binaarkoodi mõttes sajaprotsendiliselt ühilduv. Siiski oli Iskra-226 sisemine struktuur oluliselt erinev ning ta sisaldas mitmeid täiendusi, mis muutsid ta oluliselt sobilikumaks tööstusrakendusteks.}. Seal oli küll kodukootud mänge, mille idee oli võetud kuskilt mujalt ja tehtud mäng valmis  või oli nagu nullist mingisugune mõte mänguks vormistatud. 

Venelastel oli selline Apple kloon, mille  nimi oli aga Agat\index{Arvutid!Agat}, kui neid Eestisse hakkas tulema, siis nende jaoks oli mingisuguseid vene päritolu mänge. Osa  olid selgelt Apple'i pealt maha lükatud ja osa olid  nii-öelda originaalid. Sellest ajast ma ei mäleta, et oleks väga palju mingisuguseid kodumaist päritolu mänge eriti olnud.

\question{Kas need inimesed, kes seda arvutit päriselt kasutasid, lasid sul seal talle niisama lihtsalt kõhu alla vaadata ja kaant maha kruvida?}

No ega seal Küberneetika Instituudis\index{Küberneetika Instituut} väga palju ei lastud, sest et seal oli oluline ikkagi see, et masin oleks töökorras ja iga hetk kasutatav  uurimistöö jaoks.

Hiljem,  keskkooli ajal veel, sattusin ma ka tollasesse TPIsse ehk siis praegune Tehnikaülikool\index{Tallinna
Tehnikaülikool!Automaatikateaduskond!Raadiotehnika kateeder} kus raadiotehnika kateedris oli ka üks Apple II\index{Arvutid!Apple II}.  Seal olid siis juba nii-öelda raadiotehnikud, kelle igapäevane leib ongi see, eks, et vaadata pigem seda, mis seal kõhus on ja kuidas käib. Seal sai siis masina sisse vaadata, kõrval oli olemas  jootekolb, millega sai pädevam seltskond teha ise Apple'ile mingisuguseid perifeeria kaarte. Selleks, et TPI majas mingisuguseid juhtimisi või mõõtmisi või asju teha.  

\question{Tollal, ma sain aru, oli üsna hästi teada, kus mingisugune Apple II või Iskra saadava oli?}

Jah, sest neid olid valitud hulk. Ja see nii öelda arvutihuviliste seltskond, tundsid ja teadsid üksteist üsna hästi. Võib-olla ei tundnud, aga nad teadsid. Ma ju teadsin neid inimesi, kellega ma kuskil arvutiringis iga nädal kokku puutun. See seltskond võis olla mingisugune kolmkümmend, võib olla nelikümmend inimest. Neid arvutiringe,  millega mina kokku puutusin sel ajal oli kolm tükki põhiliselt või isegi neli. Oli selline koht nagu Oktoobrirajooni Õppetootmiskombinaat\index{Tallinna Oktoobrirajooni Õppetootmiskombinaat}, kus oli arvutiklass ja tegelikult seal olid Yamaha MSX-id\index{Arvutid!Yamaha MSX}, mis kindlasti on Prontol\index[ppl]{Pronto} hästi meeles. 

Seal käis koos päris suur seltskond noori huvilisi. Ja siis oli TPI arvutiring\index{Arvutiklubi!TPI Arvutiring}, mida juhendamas Vladimir Viies\index[ppl]{Viies, Vladimir} ja mingi hulk muid õppejõude. Seal olid Robotronid\index{Arvutid!Robotron}, ma küll ei mäleta, mis see täpne tüüp oli aga mingeid  sellise kaubamärgiga masinat seal olid. Ja noh, TPI-s olid siis ka need Iskra-226-ed\index{Arvutid!Iskra!226} millega ma siis kokku puutusingi tegelikult selle arvuti ringi käigus esimest korda. 

Tallinnas oli selline kool nagu 3. Keskkool\index{Koolid!Tallinna 3. Keskkool}, kus oli üks selline matemaatikaõpetaja nagu Jaak Loonde\index[ppl]{Loonde, Jaak}, kes oli haridussüsteemis selline omaette fanatt ja kõvasti populariseeris tol ajal, et arvutid nii riistvara kui arvutiõppe näol koolidesse jõuaksid. Jaak Loondel  mingist hetkest alates oligi üks Agat\index{Arvutid!Agat} kasutada. Ma ei tea, kustkaudu nende kool selle ilmselt Venemaalt sai. 

Ja siis oli olemas selline koht nagu 43. kooli tehnikaring\index{Arvutiklubi!43. Kooli Tehnikaring}\index{Koolid!Tallinna 43. Keskkool} mille juhendaja oli Ants Reili\index[ppl]{Reili, Ants}. Seal käis ka päris mitu sellist poissi, kes ei olnud nagu otseselt sellise üldise tehnika või siis lähemalt elektroonika huviga vaid rohkem  just arvutihuviga.

Et need vist olid nagu need põhilised seltskonnad, kus siis igas ühes oli  ka mingisugune kattuvus. Mina teadsin  kõiki neid nelja seltskonda, võib-olla oli veel mingisuguseid seltskondi või nii-öelda arvutiringe või huvilisi.

\question{Need olid kõik Tallinnas, eksole?}

Jah, need olid kõik Tallinnas. Tartus oli hulk inimesi, kes koondus ülikooli juurde. Seal oli Anne Villems\index[ppl]{Villems, Anne} ja mingi hulk, kui õigesti mualetan,  Apple-id. Kus oli Tartus Apple olemas, oli Füüsika Instituut\index{Füüsika Instituut}. Seal oli selline mees nagu Jaan Pruulman\index[ppl]{Pruulman, Jaan} kellega mina puutusin ka tol ajal  kokku, kui kui ma ükskord seda Apple II-te\index{Arvutid!Apple II} käisin vaatamas. Ma ei mäleta, ma sattusin mingil põhjusel Tartusse mingisuguste kooli- või ringikaaslastega, ja  me mõtlesime, et lähme sinna Füüsika Instituuti külla, sest seal esiteks saab sooja, sest et ilmad olid tol hetkel väga külmad ma mäletan, ja äkki saab arvutis ka midagi teha. Ja siis me Pruulmaniga tuttavaks saime. Ma ei tea, kuidas need asjad toimisid selles mõttes, et üks hommik mingi seltskond koolijuntsusid võtab pähe, et lähme ja sõidame. Kellelegi mingeid kontakte ei ole, mingeid eelnevaid kokkuleppeid ei ole, tol ajal ka ei helista ja ei saada meili  aga kõik nagu tuleb välja lõppkokkuvõttes, eks.

\question{Kui ma koolist saan aru ja ülikoolist ka aga mis selle Oktoobrirajooni asutuse huvi oli arvutiklassi hankida?}\label{content!OTK}

Need õppetootmiskombinaadid olidki sellised mitme kooli peale, ehk siis rajooni kaupa (neli rajooni oli Tallinnas oli tol ajal\sidenote{Alates 1974. aastast jagunes Tallinn Oktoobri-, Lenini (endine Kesk), Kalinini ja Mererajooniks.}). Ja noh, nimi oli õppetootmiskombinaat, mis ilmselt siis pidi viitama sellele, et seal annab mingit praktilist asja proovida teha ja mingeid ametikogemusi saada. See arvutiring oli puht selline huviring, seal ei olnud mingit sellist tootmisväljundit, nagu see õppetoootmiskombinaadi nimi võiks öelda.

\question{Seda minagi imestan, et miks nad hankisid need arvutid, see pidi ju keeruline olema?}

No igal juhul neil oli arvutiklass mingi tosinajao arvutitega. Võis ka olla, et kuna see klass oli otsapidi seotud sellesama Jaak Loondega\index[ppl]{Loonde, Jaak}, et tema kuidagi selle arvutiklassi sinna sebis ja et Õppetootmiskombinaat oli lihtsalt nii-öelda katus. Nii  ei olnud see klass ühes konkreetses koolis kus oleks olnud mingisugused poliitilised pinged, et näe, neil on aga meil ei ole. Võib-olla seda oli lihtsam organiseerida. 

\question{Kui sa arvutitega toimetama hakkasid, siis millele sa toetusid? Tühja koha pealt inimene ju ei vaata arvuti skeemi pealt, kust kuhu bitid liiguvad?}

No elektroonika tausta mul oli nii palju, et seda ma teadsin, mismoodi bitid liiguvad ja mismoodi  loogikatehted toimivad ja kuidasmoodi asju tööle panna. Ja kuidas näiteks teha LCD displeiga elektronkella.  

\question{Kuidas sa oskasid?}

Ühelt poolt selle 43. kooli tehnikaringi\index{Arvutiklubi!43. Kooli Tehnikaring} teadmiste baasi tõenäoliselt. Ja teisalt ma lunisin vanematelt endale välja küllaltki palju kirjandust, mis oli põhiliselt vene ja saksa keeles, inglisekeelset kirjandust ei olnud tol ajal lihtsalt kuskilt saada. Või kui oli, siis see oli ilukirjandus ja niisugune aime ja ulme, eks, aga ta ei olnud mingisugune teadus või tehnikakirjandus. Tehnikakirjandus, kui ta oli mitte-nõukogude päritolu, siis ta oli ikkagi saksa keeles. Ega ma ausalt öeldes neid raamatuid ühtegi otsast lõpuni läbi ei lugenud, aga ma neid ikkagi natukene sirvisin ja  lugesin võib-olla mõned olulisemad peatükid läbi. Sealt tasapisi ilmselt see kogemus või  teadmine tekkis.

\question{Millest me järeldame, et saksa ja vene keel tehnilise sisulise teksti lugemiseks ei olnud probleem tolleks hetkeks?}

Minu jaoks saksa keel küll oli sest mina õppisin koolis inglise keelt süvendatult. Selles mõttes inglise keel oli  minu jaoks nagu nagu eesti keele kõrval teine emakeel peaaegu. Aga saksa keelt ma ei purssinud eriti üldse tol ajal. Tänapäeval on küll nii, et võtad raamatu lahti, mis sest et ma saksa keelt ei oska. Aga  inglise keelega on palju sõnu samad, mingite muude tuntud keeltega on palju sarnasusi, nii et mingist üldisest mõttest saab aru. Muidugi selliseid konkreetseid juhiseid või mingit faktilsit infot ma saksa keeles ikkagi ei loe. 

\question{Mis koolis sa käisid?}

44. Keskkool\index{Koolid!Tallinna 44. Keskkool}, mis on tänapäeval Mustamäe Gümnaasium\index{Koolid!Mustamäe Gümnaasium|see{Koolid!Tallinna 44. Keskkool}} ja seal ma õppisin inglise keelt. Aga kooli ajal see inglise keel, eriti it-maailmas, ei olnud mingi asi, mida oleks saanud väga palju rakendada. No peale selle, Basicu-s on käsk \verb|print| jah, on küll inglise keeles. Aga Basic-us neid käske nii väga palju ei ole ja neid ei ole keeruline ka pähe õppida juhul, kui sa inglise keelt ei valdaks. 

\question{Kas tehnikakirjanduse juurde käis ka mingisugune mingisugune muu kirjanduse või ulme huvi? Filmid,  raamatud?}

Kooli ajal ma lugesin üsna palju ulmet inglise keeles. Ja kooliajal sattus mulle kätte ka Douglas Adams ja tema Hichikeri raamatud\sidenote{Vaata ka märkust leheküljel \pageref{sidenote!adams}.}, mida tal tol hetkel oli viis tükki. Kuna inglise keel oli koolis tol ajal süvaõppes siis meil oli üks selline tund inglise keeles nagu inglise keele kodulugemine. Kus siis kodus pidi mingit ilukirjandust lugema inglise keeles ja tunnis pidi jutustamas. Ja mina läksin raamatupoodi ja nägin kuskil üleval lae all riiuli peal seda Hitchikeri kõige esimest osa. Vaatasin, et see on huvitav pealkiri, ostsin selle raamatu ära ja mõtlesin, et võtangi siis selle inglise keele kodu lugemiseks. Aga see ei olnud väga mõistlik mõte sellepärast, et inglise keeles, kui mõelda nende sõnade pele mida seal kasutatakse, välja mõeldud sõnad, välja mõeldud liiginimed, seadmete nimed ja nii edasi, mis on eesti keelde üsna raskesti tõlgitavad,  selleks peab väga hea fantaasiaga tõlkija olema. Aga mina hakkasin seda raamatut lugema ja siis õpetajale jutustama. Ma küll ei tea, kui palju õpetaja tol ajal sellest aru sai, mis ma talle jutustasin aga vähemasti ta jäi rahule. 

\question{Seal ju ei ole narratiivi, on mingisugune keeruline sõlm, mis viienda raamatu lõpuks umbsõlme läheb!}

Ega mina ka ei saanud  sellest eriti väga palju aru, kui ma seda esimest osa kooli ajal lugesin. Hiljem lugesin  ülejäänud osad ka läbim siis nagu loksus see pilt paika.

\question{Aga oli siis saada ingliskeelset ilukirjandust?}

Jaa, ingliskeelset ilukirjandust oli küll. Seda oli igal pool, isegi Tallinnas. Pärast kooli lõppu, kui ma töötasin TPIs\index{Tallinna Tehnikaülikool}, käisin ma palju Moskvas ja Leningradis komandeeringus ja seal oli valik üsna lai. Asimovilt ma vist ei sattunud lugema Asumi sarja\sidenote{Vaata ka märkust lehekülhel \pageref{sidenote!asum}.}, see oli mõnu muu sari või üksiklood. Aga Asimov ja Adams olid nagu kaks põhilist ulmekirjanikku, kellega me esmajoones kokku puutusin. 

\question{Aga vene klassikud? Strugatskid?}

Jaa, Strugatskeid ma olin lugenud varem, sestneed olid tõlgitud eesti keelde. \enquote{Purpurpunaste pilvede maa}\sidenote{Arkadi Strugatski; Boriss Strugatski. (1959). \begin{russian}Страна багровых туч\end{russian}. Eesti keeles 1961 Ralf Tominga (värsid Lembe Hiedel) tõlkes sarjas \enquote{Seiklusjutte maalt ja merelt}.} vist oli,  Amfiibinimene vist oli ka Strugatskite oma\sidenote{\begin{russian}Человек-амфибия\end{russian} on siiski 1928. aastal ilmunud Alexander Belyaev-i romaan. Eesti keeles ilmus 1960. aastal \enquote{Seiklusjutte maalt ja merelt} sarjas koos romaaniga \enquote{Maailmavalitseja} (vene k. \begin{russian}Властелин мира\end{russian}, 1926).}. Ja siis ma neelasin võimalust mööda igasugust popteadusliku ehk aimekirjandust. \enquote{Mosaiik}\sidenote{\enquote{Mosaiik} oli kirjastuses Valgus aastastel 1973–1991 välja antud populaarteaduslike raamatute sari, mis käsitles äärmiselt laia teemaderingi ajaloost ja psühholoogiast topoloogia problemaatikani.}, selline raamatusari oli olemas. Ega suurt mingit muud aimekirjandust ei olnudki võtta.

Kriminullid olid teine valdkond, mis tol ajal peale sellise ulmekirjanduse ilukirjandusest huvi pakkus. Neid ma tollel ajal neelasin. Ja tempo oli selline, ma mäletan, et see vist oli Asumi mingisugune kolmas või neljas osa mille maha vist lugesin Rootsis töötades ühe ööga läbi. Ma ei tea, kui palju lehekülgi see võis siis olla, mingi \emph{paperback}, kolm või nelisada lehekülge tõenäoliselt.Selliseid asju juhtus, sai tol ajal tollaste tööde kõrvalt ja tollase elu kõrvalt, kui  kool oli äsja lõpetatud, lubada. Selle asemel et öösel magada ja puhata, võtsid järgmise raamatu riiulist.

\question{Kui sa keskkooli ära lõpetasid, siis sa läksid TPI-sse kohe tööle või ikka õppima ka?}

Enam-vähem kohe pärast keskkooli ma läksin TPI-sse\index{Tallinna Tehnikaülikool} tööle. See töökoht tegelikult sattuski mulle kätte tänu sellele Vladimir Viiese\index[ppl]{Viies, Vladimir} juhendatud arvuti ringile. Ma töötasin selles samas kateedris, kus Viieski tegelikult, see oli siis elektronarvutite kateeder\index{Tallinna Tehnikaülikool!Elektronarvutite Kateeder}. Aitasin seal igasuguseid arvuti hooldustöid teha, mingeid laborite häälestamisi ja  ette valmistamisi, mis erinevatel õppejõududele vaja oli. Ja mõnevõrra hiljem siis sai kaasa löödud juba mingisugustes, arvutit hõlmavates projektides,  kui oli vaja midagi programmeerida, kui oli vaja mingi sisend-väljundseadme jaoks mingisugune draiver kirjutada. 

\question{Kas sa läksid õppima ka?}

Õppima, ma läksin mõnevõrra hiljem, kui ma sinna tööle läksin, sest mind punased ained ei  meelitanud eriti, keda nad oleks meelitanud. Aga mina tundsin nende vastu niivõrd suurt vastumeelsust,  et ma leidsin, et ma ei taha  üldse õppima minna, ka mingit tehnilist asja, kui seal on need punased ained juures. Punased ained olid siis  NLKP ajalugu ja mingid sellised asjad. Aga mingil hetkel ikkagi paar aastat hiljem läksin õhtusesse osakonda õppima. Olin küll üks  üks enamusest, kes ei lõpetanud. Meie kursusele ühe astus sisse vist umbes kakskümmend viis inimest, kellest lõpetas kaks. 

\question{Mis eriala see siis oli?}

Elektronarvutid. Aga õhtuses osakonnas selline lõpetanute protsent oli minu arust tol ajal üsna tavapärane üheksakümnendate algul, see oli vist 1990. aasta sügis, kui TPI-s õppima asusin. Sealt sealt, ongi tegelikult kõik need tööd ja samamoodi mingisuguseid vaba aja tegemised on üsna palju seotud alguses  programmeerimisega ja sellise arvuti tehnilise või riistvara poolega.

Kui kui TPI-s sai töötatud, siis Elektronarvutite kateeder asus teisel korrusel. Ja samas korpuses neljandal korrusel asus siis Raadiotehnika\index{Tallinna Tehnikaülikool!Automaatikateaduskond!Raadiotehnika kateeder}, kus oli see Apple II\index{Arvutid!Apple II}. Ja siis meil tekkis Mastiga\index[ppl]{Kaal, Madis}, ehk Madis Kaal, ühel hetkel kuidagimoodi mõte, et võiks proovida IBMi PC arvuteid, mis siis olid meil teisel korrusel kasutada ja mis oli minu igapäevane tööriist kokku ühendada  Apple II-ga, mis oli neljandal korrusel Masti igapäevane tööriist. Ehitasime sinna vahele \emph{current loop}-i, no see on RS-232 põhimõtteliselt ainult natuke teistsuguse elektrilise signaaliga. Misjärel oli meil nagu selline pilv, et PC seest sai \emph{backup}-ida  andmeid Apple II sisse ja vastupidi. Nagu pilv ikka, kellegi teise arvuti. 

Üheksakümnes aasta oli minu arust ka see, kui Eestisse jõudis mingisugune info sellest, et on olemas BBSid ja TPI majas oligi Mast\index[ppl]{Kaal, Madis}  see entusiast, kes pani sealkandis esimese BBSi jooksma. Mina esialgu vaatasin seda lihtsalt kõrvalt, mul ei olnud selle kohta nagu mingit arvamust. Ja ma ei tundnud  väga palju huvi selle kõige vastu. Seal sai mingeid faile vahetada, aga  ma ei ole näiteks kunagi mingi eriline mängu-fanatt olnud, ja siis järelikult mind ei huvita  mingisugused mängud, mida saab BBSide kaudu tõmmata, eks. Ja see, kus ma leidsin, et neid BBS-id võivad olla kuidagi kasulikud  oli vist see moment, kui tuli välja, et seal BBS-ides on olemas  tekstifaile, mis on mingisugused referentsdokumendid. mingisugused \emph{manual}-d, mingid standardid, mingid programmeerimisõpikud kas IBM-ide või Apple jaoks.

\question{Kas need olid mingid \emph{plain text} failid või \LaTeX või mis?}

Need olid tekstifailid, aga nad olid natuke formatitud ikkagi, neil olid tabulatsioonid sees, lehekülje vahe oli sees, neid sai maatriksprinteriga välja trükkida nii, et tulid ikka ilusti formatituna paberi peal. Maatriksprinterid olid kättesaadava hinnaga ja need olid enamuse arvutite taga.  Suured arvutid ehk siis ES-id SM-id, mis olid TPIs või Küberneetika Instituudis, seal olid need laiad printerid. Ma ei teagi, kuidas nende printerite kohta öeldi. Ridaprinter? \emph{Line printer} öeldi inglise keeles, aga  oli mingisugune eestikeelne sõna ka mille vist Ustus Agur\index[ppl]{Agur, Ustus} välja mõtles. Ühesõnaga mingit koledat koledat häält ja värinat tegevad printerid.

\question{Kui sa aru said, et sealt saab igasugu dokumente, hakkasid BBS-id sulle ka huvi pakkuma?}

Jah, ma arvan, et see oli see hetk ja see ajanend kui ma leidsin, et sealt  peale mängude ja  tilulilu sai midagi mõistlikku ka. Mingi hetk panin oma BBS-i ka püsti ja selleks ajaks oli ka Fidonet otsapidi Eestisse jõudnud\sidenote{Esimene Fidoneti Eesti regiooni 2:49 sisaldanud nodelist on 271 28. septembrist 1990. Regiooni koordinaatorina on seal kirjas Andrus Suitsu\index[ppl]{Suitsu, Andrus} ja \emph{Host} on Tarmo Ausing\index[ppl]{Ausing, Tarmo}. BBSidest on loetletud Hacker's Night System\index{BBS!HNS} (Tarmo Ausing), P.O.Box Maximus\index{BBS!P.O.Box Maximus} (Andrus Suitsu), Goodwin BBS\index{BBS!Goodwin} (Sulo Kallas\index[ppl]{Kallas, Sulo}), Mail Shark\index{BBS!Mail Shark} (Madis Kaal\index[ppl]{Kaal, Madis}) ja MamBox (Tarmo Mamers\index[ppl]{Mamers, Tarmo}).}. Väga paljud, kes nagu ajalooliselt on tagasi vaadanud ja rääkinud võib-olla sellest ajast, nad ei pruugi eriti olla vahet teinud BBS-indusel ja Fidonetil aga tegelikult need olid kaks eraldi maailma. Vahe oli see, et BBS oli lihtsalt mingi süsteem, kuhu saab modemiga sisse helistada ja siis saab seal süsteemis sees toimetada. Ja mingeid andmeid failide näol endale tõmmata või siis mingisuguseid sõnumeid vahetada. Aga kogu see info ja need sõnumid on salvestatud sinna ühte konkreetsesse BBS-i süsteemi.

Ja Fidonet ühe otsaga  sai alguse nendest samadest BBS-idest, aga tema eesmärk oli sõnumite BBS-ide ja mingite muude Fidoneti liikmete süsteemide vahel edasi-tagasi toimetada. 

\question{Ehk, Fidonetis need kohad, kuhu sa sisse helistasid, helistasid ka üksteisele sisse ja vahetasid andmeid?}

Jah. Ja see oli siis juba automatiseeritud süsteem, kus olid automaatvahendid selleks, et neid sõnumeid vahetada, ehk siis meile valetada. Ja meile oli kahte liiki: olid privaatmeilid ja olid konverents meilid, mis siis on tänapäeva mõistes meiligrupid või meililistid.

\question{Kas \emph{usenet} tekkis ka sel ajal?}

Usenet oli varem, see on hästi vana asi. Usenet ja UUCP protokoll sellega seotuna on põhiliselt  Unixi-maailma päritolu, ehk siis see oli konkreetselt Unixi arvutitevahelise meilivahetuse protokoll. Ja see Usenet, mis  sinna ümber tekkis,  see oli siis ka nagu selline konverentside või vestlusringide süsteem.

\question{Kas seda peegeldati Fidosse ka?}

Ja seal olid lüüsid olemas. Usenetist sai konvertida ümber kirju Fidoneti \emph{echo}-desse või meilikonverentsidesse. Muu hulgas ka faile, sest et Usenetis vahetati ka väga palju faile,  neid oli siis võimalik ka konvertida tavalisteks failideks, mis siis kuskil BBSis üles pandi.

\question{Kas eestlased toimetasid seal usenetis mingites oma gruppides või möllati olemasolevates?}

Usenetis ma mäletan küll, et ei olnud mingeid erilisi Eesti spetsiifilisi või regionaalseid gruppe. Erinevalt Fidonetist, seal oli küll mingi viisteist või heal ajal võib-olla kakskümmend lokaalset  vestlustgruppi ehk \emph{echo}-t. Neist kaks-kolm gruppi olid üsna populaarsed liikmeskonna mõttes.

\question{Mis see tähendab? 50, 100, 500 liiget?}

No ma arvan, et lugejaid võis seal oli väga palju, sest et pidevalt tuleb välja inimesi, kellega mina ei ole kunagi kokku puutunud, ma ei tea neid nimepidi, aga nad räägivad, et nad on kunagi sealt \emph{echo}-st  midagi lugenud. Sest tegelikult selleks, et neid \emph{echo}-sid või konverentse lugeda,  ei pidanud sa ise omama ei BBS ega mingit Fidoneti süsteemi. Sa said helistada BBS-i sisse, seal lugeda, kui sa tahtsid, ja kirjutada. Kui kellelgi Fidoneti süsteem oli püsti pandud, siis selle eelis oli selles, et siis talle need kirjad tulid automaatselt koju kätte ja tal ei olnud lugemiseks-kirjutamiseks vaja kuskile kaugele ise helistada. Tegelikult see ring  neid inimesi, kes võib-olla ainult luges võis olla tegelikult päris suur. 

Kui püüda hinnata seda, kes seal aktiivselt suhtlesid ja kirjutasid ka, siis  võib-olla see on mingi kakssada inimest. See on väga laest võetud number, suurusjärgus.

\question{Seda on ikkagi päris palju. Kas sa oma Fido \emph{node} panidki püsti selle jaoks, et asjad tuleksid koju kätte? Mis selle asja nimi oli?}

Ma arvan, et eesmärk oli jah see, et asjad oleks piisavalt automatiseeritud, et mulle ei oleks endal vaja mingeid liigutusi teha  ja aega viita selle pärast, et kuskile BBS-i nii-öelda löögile saada. Sest kui BBS-i küljes välismaailmaga suhtlemiseks oli üks modem, siis see tähendab, et üks inimene igal ajahetkel korraga sai seda BBS-i kui teenust kasutada. Oli BBS-e, millel oli mitu modemeit küljes, siis sai mitu inimest seda paralleelselt kasutada.  Aga kõik see tähendaski, ethelistasid modemiga telefoni oli kinni. Helistad viie minuti pärast, ikka kinni. Ja no miks ma pean niimoodi vaeva nägema ja pidevalt helistama? Tõsi küll, modem valis ise automaatselt, tegi kordusvalimist, eks, ja kui lõpuks löögile sai, andis mingi signaali. Aga ma leidsin, et parem on seda asja lasta sellel Fidoneti automaatikal teha. Ja siis see rahumeeli saab  hetkel, kui sa tahad, avada meililugemise programmi ja lugeda seda meili, mis on vahepeal sul sinna masinasse ära tõmmatud. 

\question{Aga mis su \emph{node} nimi oli?}

Minu \emph{node} nimi oli MamBox. Ma ei mäleta, mis hetkel see eesliide, mis on siis tulnud minu perekonnanime algusest,  hakkas mingisuguste asjade külge tekkima. Aga tol hetkel oli jah nii, et kui ma tegin BBS-i, siis ta oli MamBox, kui ma kirjutasin mingisugust programmi  oma lõbuks, siis siis ma kirjutasin \enquote{\emph{Copyright MamSoft}}\sidenote{Tegu oli levinud praktikaga, minu samal viisil kasutatud fiktiivne firmanimi oli \enquote{\emph{I \& I Company}}. Sellest \emph{misasi} üks firma on, oli arusaam ähmane. Sellest, et firma \emph{nimi} tuleb kindlasti ära mainida ja kuulsaks teha, oli arusaam väga konkreetne.}. See oli tol ajal selline kaubamärk, mida mina kasutasin siis selliste ühesugust eesliidetega. Üsna tüüpiline oli see, kellel oli BBS, kui tal alguses ei olnud, siis ta mingil hetkel lisas sinna  Fiodeneti funktsionaalsuse. Ja väga palju oli ka teistsuguseid suundumusi, et kui sul oli mingil põhjusel tekkinud Fideoneti \emph{node}, siis väga palju nende  omanikest mingil ajal leidsid, et võiks  ka BBS-i püsti panna. 

Muidugi oli väga palju ka Fidoneti \emph{node}-sid, kelle omanike või siis \emph{sysop}-ide eesmärk oligi see, et lihtsalt selleks, et  lugeda-kirjutada ja automaatselt lasta sõnumeid vahetada. Et nende huvi ei olnud mingisugust BBS-i üleval pidada.

\question{Ehk, kui mõni BBS sai populaarseks siis võis see olla nii seepärast, et seal vahetas aktiivne kogukond omavahel faile kui ka see, et miskipärast otsustasid paljud kasutajad just sealtkaudu Fidonetile ligi pääseda?}

Fidoneti juurde pääses kõikidest BBS-idest, kes olid Fidoneti liikmed, sest kõigis oli põhimõtteliselt ühesugune koopia nendest konverents kirjedest. Iseasi olid privaatkirjad, siis oli põhimõtteliselt vaja Fidoneti \emph{node} numbrit teada, kuhu saab kellelegi inimesele kija saata. Iga inimene oli  mingisuguse Fidoneti \emph{node}-ga seotud, et  privaatmeili vahetada. Aga mis konverentse või \emph{echo}-sid puudutas,  siis need  olid ühtmoodi igas BBS-is  saadaval.

Aga ega muidugi ei olnud eriti mõnus ka see, et tänane loed siit, homme  hoopis teisest BBS-ist seda meili, sest seal on viited, kui palju sul on loetud meile, kus lugemisjärjekord on, kas sa oled millelegi vastanud või ei ole. See  läheb sassi, kui sul ei ole oma sellist nii-öelda kodu-BBS-i. Ja oli ka selge, et kus oli väga populaarne faile käia tõmbamas,  need BBS-id on  üsna hõivatud ja tihtipeale kinni nende failide tõmbamise pärast. 

\question{Faili tõmbamine võttis ju tükk aega}

Jah. Alguses, kui BBS-id Eestisse tekkisid ja need Fidoneti \emph{node}-d, siis, ütleme niimoodi, et 14 400 boodi (ümmarguselt võib seda teisendada 14 400 bitti või siis 14 kilobitti sekundis) andmevahetuskiirus oli üsna tüüpiline  algupäevadel nende BBS-ide juures.


\question{Ma isegi mäletan 9600-seid miskipärast}

9600 oli jah selline lihtne, odav iga mehe tehnoloogia. Aga ütleme neliteist koma neli olid sellised modemid, kuhu poole kõik nagu püüdlesid. Ja sealt edasi tuli siis 19.2, 26.6, mingid sellised numbrid. Minul ühel hetkel oli kasutada sellised üsna nii-öelda härjad modemit, mille töö kiirus oli 33 600 boodi. Aga sellisel juhul, kui teisele poole sideliini otsas on vastas täpselt sama tootja modem. Modemi  nimi oli Trailblazer\index{Telebit Trailblazer}\sidenote{USA tootja Telebit, kes Trailblazeri sarja tootis, kasutas standardsete V-seeria protokollide asemel oma protokolli Packetized Ensemble Protocol (PEP).}. US Roboticsid\index{US Robotics}  töötasid BBS-ide nii-öelda  põhiajastul kõige kiiremini vist 34.4 kiloboodi juures.

\question{Kas BBS-idega majandamine tekitaski sul võrgu-huvi? Sa rääkisid, kuidas te Mastiga Apple-t ja PC-d paaritasite?}

Ma arvan, et see Apple ja PC paaritamine oligi see, mis  võrgunduse kui sellise pisiku tekitas, sest ega TPI-s ega ka kuskil mujal, kus arvutitega sai kokku puututud, ei olnud mingisuguseid erilisi kohtvõrgutamise tehnoloogiaid kasutusel. Ainukene olid UUCP, mis käis Unixite vahel, see oli rohkem nagu selline tõsisemate ja suuremate arvutite sidepidamine  ja rohkem nagu teadus- ja akadeemilistes ringkondades, eks. Ja teisalt siis just see Fidonet, mis oli selline asjaarmastajalik. Pärast TPId järgmises töökohas ma  puutusin esimest korda kokku ARCNetiga\sidenote{ARCNet oli 1980. aastatel levinud esimene laia kasutamist leidnud mikroarvutite võrgusüsteem. ARCNet on siiani kasutusel sardsüsteemide puhul.}.

\question{Kus see oli ja mis aastal umbes?}

See oli  aasta 1991, selline ettevõte nagu Skriining\index{Skriining}, mis eksisteerib tänapäeval ka. Skriiningus ma puutusin kokku ARCNetiga, mis jooksis tol ajal kahe ja poole megabiti kiiruse peal. See oli koaksiaalkaablivõrk, pea-aegu nagu esimesed Etherneti võrgud aga, ütleme, neli korda aeglasem. Minu arust see koaksiaalkaabel, mida ta kasutas, oli ka vist seitsmekümne viie oomine, ma arvan, versus Etherneti viiekümne oomine kaabel. 

Aga noh, see ARCNet oli  üsna lühiajaline temaga olid kokkupuuted peamiselt  tänu sellele, et see oli see aeg, kus Soomest ja mujalt läivälismaalt seljakotiga kraami toomas käidi. Väga palju kraami, mis Soomest tuli oli selline kraam, mis Soomes oli maha kantud, seda ei tahetud seal ära visata, sest  utiliseerimine maksis, siis anti ära, et \enquote{kasutage, tehke midagi}. Ma ei mäleta, et ARCNetiga midagi väga tõsist oleks tehtud, aga mingeid kokkupuuted sellega ikkagi olid. Selle peale tuli siis Ethernet, mis oli tol ajal koaksiaalkaabli Ethernet, kümme megabitti sekundis. Mis oli siis selline asi, mis hakkas päris reaalselt nagu ettevõtetesse jõudma ja mille peal  hakati tegelikult üsna palju tegelikult kohtvõrke ehitama.

\question{Räägi korraks palun sellest Skriiningust\index{Skriining}. Aruvtiäri jaoks peaks nagu nime järgi olema kaks poolt: arvuti ja äri. Aga et aastal 1991 oleks kumbagi olnud, tundub natuke uskumatu.}

Noh, arvutid olidki sellised, mis alguses tulid seljakotis piiri tagant. Ja järgmine faas oli see, kus  nad tulid endised seljakotiga piiri tagant, aga selleks, et neid saada, selleks oli vaja sinna piiri taha seljakotiga kõigepealt sularaha viia. Sest kakskümmend tuhat rubla, võib-olla arvuti eest, ma arvan, oli selline keskmine arvuti hind. Ma ei tea, mina ei muutunud hindadega kokku, sest ma ei tegelenud müügitööga. Nii et ma ei kujuta ette, kui palju  arvutid tol ajal nagu numbriliselt maksid, aga arvutustehnika oli veel meeletult kallis.

\question{Kuidas sihuke firma üldse võis tekkida tol ajal? Ei saanud ju internetti kuulutust panna, et \enquote{tulge meile tööle}?}

Ma ei tea, IT-maailmas inimesed liikusid ilmselt tutvuste kaudu ühest kohast teise tööle. Ja mina sinna Skriiningusse\index{Skriining} jõudsin ka  tutvuste kaudu, sest et üks inimene, kes varem oli olnud minu kolleeg TPI-s sattus sattus Skriiningusse tööle ja kutsus paar aastat hiljem mind ka sinna. Skriiningu nii-öelda vertikaal või kliendisegment oli ja on ka tänapäeval meditsiiniasutused ja meditsiiniasutuste võrgud, arvutibaas ja infosüsteemid, nende kirjutamine ja hooldamine. Ma arvan, et see on ka üks põhjus, miks Skriining tänapäeval elus endiselt ja ilmselt elab väga hästi. Tal on oma üsna kitsas kliendisegment ja kindlad ja väljakujunenud kliendisuhted.

\question{Sinu jutu järgi tundub, et need esimesed arvutifirmad olid sõprus- või vähemalt tutvuskonna põhised?}

Nad tegelikult väga ei olnud. Sest Skriiningus see Mart, kes enne mind sinna läks ja kes mind hiljem kutsus, oli ainukene inimene, keda ma seal tundsin. Aga jah, sellised arvutifirmad ei olnud suured. Skriining oli, ma ei tea, viis-kuus inimest tõenäoliselt, mitte rohkem. Kõik tegid kõike enam-vähem. Võib-olla mõni jah programmeerimis rohkem, võib-olla mõni, nagu mina näiteks, vedas rohkem kaablit või käis seal mingeid kruvisid keeramas või timmimas mingeid asju seal arvuti kaane all. Mingid eelistused olid kindlasti inimestel olemas, aga üldjoontes võib öelda, et kõik käisid nagu mingil määral vähemalt üle kõikidest nendest süsteemidest, mis firma  sees kasutusel olid või millega see firma tegeles.

\question{Kas sa sel ajal veel oma BBS-i ka pidasid?}

Jah. BBS  oli mul üleval päris pikka aega, ma olen teda nii-öelda kaasa vedanud  ühest kohast teise, sest ega tol ajal kodus ei saanud teda pidada. Noh, esiteks koju ei olnud kellelgi eriti võimalik arvuti tankida, sest see oli kallis. Ja kui oli ka võimalik hankida, siis võib-olla mingi niru arvuti, mille peal võib-olla BBS-i hästi püsti ei pane. Ja teisalt tol ajal kodus telefoniga välja helistamine, ei olnud just mitte võib-olla kõige odavam lõbu. Pealegi, kui mõelda Fidoneti peale ja et see Fidonet  oli ülemaailmne süsteem, siis see hõlmas ka mingit hulka rahvusvahelisi kõnesid, siis kodust ei olnud võimalik otse helistada välismaale. Kaugvalimine toimis läbi inim-operaatori. Ja ega ma kõikidest ettevõtetest ka ei olnud võimalik välismaale helsitada. Tihtipeale oli ettevõttes üks telefoninumber, võib olla mingi kümne või saja telefoni peale, kust sai otse välismaale helistada. Seda siis püüti endale ära rääkida, et sinna taha saaks BBS-i ühendada. Tihtipeale olid ka BBS-i omanikel kokkulepped, et nende BBS töötab öösiti, siis nad saavad seda telefoniliini kasutada, helistada välja, kui on vaja, ja päeval saab seda liini kasutada kontoritööks, inimkonna teenimiseks. Sellised ajad tekkisid hiljem, kus BBS-i jaoks oli mõnedes firmades võimalik saada kakskümmend neli tundi telefoniliin ja eriti hästi, kui sealt sai ka välismaale helistada. Selliseid kohti oli. 

Ja noh, tol ajal oli nii, et kui mina liikusin ühest ettevõttest teise, siis siis ma uut ettevõtet muuhulgas hindasin ka selle järgi, et kas mul on võimalik BBS sinna kaasa võtta ja kas mul on seal võimalik selle BBSi-i jaoks saada  kaubavalimisega telefoniliin ja veel parem, kui see nii oleks siis kakskümmend neli tundi kasutav.

\question{Need on ju päris olulised valikud, mida see BBS-i kaasa vedamine sulle pakkus? Mis selle juures huvitav oli?}

No see, et info tuleb üsna lihtsalt kätte, mida on võimalik BBS-idest saada. Esiteks on teda parajalt lihtne otsida, kui sul on juba Fidoneti \emph{node} püsti ja automatiseerida ühelt poolt meili vahetust aga teiselt poolt ka failivahetust. Kui ma tahan saada ka kätte kuskilt kaugelt BBS-ist mingit faili, ma tean selle faili nime, siis mul ei ole vaja endal käsitsi jällegi sinna BBS-i sisse logida, et seda endale tõmmata, vaid ma saan sedada Fidoneti automaatika kaudu teha.

\question{Toonases Fido maailmas toimetav seltskond oli ikkagi suhteliselt suur ja sinu nimi jookseb nende juttudest päris oluliselt läbi. Miks see nii on?}

Seal ei ole väga palju midagi arvata. Tegelikult BBS-i pidajaid on  palju nimekamaid olemas, kes seda teeb BBS-i maailma Eestis põhimõtteliselt siis alustasid ja kes on BBS-i kontekstis palju tuntumad. Aga kas see oli üsna pea tegelikult, kui need BBS-id ja Fidonet  olid Eestis levima hakanud ja üsna agarasti kasutusele võetud, kui meil mingis Fidoneti inimeste seltskonnas  tekkis selline äratundmine, et nojah, et meid on siin küll mingisugune sada kuni kakssada inimest, kes igapäevaselt nagu suhtlevat Fidoneti neti kaudu ja vahetavad kirju ja teevad nalju ja vahetevahel sõimavad üksteist ja mida iganes. Aga noh, meie siin  näeme kümmet inimest, võib-olla, keda me teame, päevast päeva nime ja nägupidi, aga teisi me ei tea. Et peaks sellele probleemile mingisuguse lahenduse otsima. 

Tegelikult oli juba 1991. aasta, kui kui see probleem muutus niivõrd teravaks et enam-vähem siis seesama mingisugune umbes kümne inimeseline seltskond mõtles, et võiks teha mingisuguse kokkutuleku. Ma küll ei mäleta, kuidas need mõtted käisid või liikusid või kes mingisuguseid ideid välja käis. Või kui palju me kuskil Fidoeneti \emph{echo}-des neid asju arutasime või mõtlesime enne, kui me selle mõtte välja käisime, et, davai, teeme mingisuguse kokkutuleku.

Oli üheksakümne esimese aasta augusti esimene pool, ma ei mäleta täpset kuupäeva, kus me olime siis paika pannud, et nii, teeme  ühel nädalavahetusel kokkutuleku, saame Väänas ühes ürituste kohas kokku. Mingi osavõtumaks oli ka,  ma ei mäleta, mingi viiskümmend rubla,  võib-olla vähem\sidenote{Siinkohal oleks ehk lugejale kasulik selgitada, kui suur või väike raha oli 50 rubla 1991. aastal. Paraku on see üsna keeruline, sest sel ajal valitses Eestis hüperinflatsioon ja hinnad kerkisid kiiresti. Lisaks olid enne 1992. aastat teatud kaupade hinnad riikliku kontrolli all ja teiste omad vabad. Kõigele lisaks ei olnud paljusid kaupu mis iganes hinna eest saada, puudus valitses muu hulgas ka sularaharubladest ja toimis elav ning väga volatiilsete hindadega must turg. 50 rubla eest võis saada 20 kilo kartulit aga võis saada ka ühe 5 tollise flopi.}. Plaan oli, et räägime  ja suhtleme ja nägime midagi  IT-kalduvustega mänge. Mitte arvutimänge, aga noh, flopi heide ja kõvakettaheide ja mingid sellised asjad on olnud nende ürituste kavas. 

\question{1991. aastal kõvaketast ikka andis heita!}

Tol aastal oli kõvakettaheide ilmselt kavas küll aga see ei olnud päris tänapäevane kõvaketas, vaid siis olid sellised suured 19 või 21 tolli läbiumõõduga plaadid, mis moodustasid kõvaketta aga ,mis ei olnud kuskil hermeetilises korpuses nagu tänapäevased pöörlevad kettad. Neid oli, ma ei tea, kaheksa või kümme kokku pandud ühe sellise käepidemega varre külge ja neid sai  kettaseadme seest välja tõsta\sidenote{Sellised kettapakid, näiteks IBMi 1316, suutsid talletada suurusjärgus mõned megabaidid infot ja olid tolleks ajaks selgesti iganenud. Eestisse sattusid sedalaadi seadmed tõenäoliselt humanitaarabina, mis tõi meie kanti hulganisti kummalist vananenud riistvara. Mäletan ühte sellist kettalugejat 1992. aastal Võru I Keskkoolis\index{Koolid!Võru Kreutzwaldi Gümnaasium} ka toimimas. Arvutiklass asus teisel korrusel ja kui kettaseade sisse lülitati, oli undamist tänavale kosta - selle järgi sai hinnata, kas klasis parajasti oli keegi või mitte.}. Sealt lahti lammutatud kettaid me lennutasime küll sellel esimesel kokkutulekul. Me mõtlesime selle kokkutuleku nimeks välja BBSummer\index{BBSummer}. Ehk siis BB lühendist \enquote{BBS} ja siis \enquote{Summer} sinna taha. Üks aasta varem oli toimunud esimene Rock Summer\sidenote{Rock Summer oli 1980. aastate lõpus ja 1990. aastatel Tallinnas Lauluväljakul toimunud muusikafestival, mille mõju ei saa kuidagi üle hinnata. Tegu oli esimese suurema rokifestivaliga siinkandis ja selle platsil valitsenud atmosfäär oli keskmisele nõukogude noorele, ütleme, radikaalse mõjuga. Kuna tegu oli ühega esimestest võimalustest piiluda raudse eesriide taha, meelitas festival kohale ka küllalt nimekaid Lääne ansambleid.}, aga nii palju, kui me oleme erinevate inimestega meenutanud, see Rock Summer ei olnud kuidagimoodi selle \enquote{summeri} osa algataja või põhjustaja sinna nime sisse, meil olid sõltumatud kaubamärgid. 

Kui see 1991. aasta BBSummer toimus, siis, kes teab natukene rohkem ajalugu või on ise siis tol ajal noor olnud või ka natukene vanem, siis põhimõtteliselt sel nädalal,  viis päeva, enne, kui BBSummer oleks pidanud toimima, oli see aeg, kui vist Vilnius oli see koht, kus tulid tankid tänavale. Ja mina selle peale ütlesin, et  meie teeme oma BBSummeri ära ka juhul, kui kui ei ole mingisuguseid liiklust segavaid tanke tänavatele, olukord oli üsna pingeline ja keeruline. Esimesel BBSummeril oli vist viiskümmend kuus osalejat, pluss-miinus, umbes selline number. Suur hulk oli muidugi neid, kes seal olid puhtalt  sellepärast, et nad olid Fidoneti \emph{sysop}-id. Aga ma arvan, et  üle poole ilmselt oli seda rahvast, kes  olid mingid BBS-i lihtkasutaja, kes lihtsalt igapäevaselt tõmbas faile ja vahetas meile ilma et tal endal oleks BBS-i olnud.

\question{See oli üsna korralik suhe teenuse pakkujate ja tarbijate vahel, BBS-i pidamise barjäär oli kõrge ja seltskond seega üsna tehniline?}

Ta oli jah ilmselt mõõdukalt tehniline, igaüks ei pidanud tõenäoliselt seda, kes tundis ja tehnika võib-olla temast kuskilt üle pea. Ja selleks, et BBS häälestada ja korralikult tööle panna ja võib-olla sinna see Fidoniti automaatika käima panna  ei olnud päris triviaalne. Internetist juhendvideot vaadata ka ei saanud. Küll aga sai mingisuguseid tekstifaile  selle kohta, et \enquote{tõmba see softi ja tõmbas see soft ja see soft ja siis pane nad kõik niimoodi kokku ja  tee sellised ja sellised konfifailid ja siis läheb asi käima}.

\question{Siis sündis ju FAQ, \emph{Frequently Asked Questions}, mis praegu on lihtsalt mingi osa veebilehest. Siis oli tegemist konkreetse eraldi leviva failiga, kuhu jõudsidki \emph{echo}-des ja uudisgruppides  sagedasti küsitud küsimused koos pädevate vastustega.}

Jah, neid  olid küsimused-vastused, kuidas asi käima panna, kui midagi ei tööta või on sellised sümptomid, siis mida tuleks vaadata ja nii edasi.

\question{Kas sedalaadi sisu Eestis ainult tarbiti või panustati sinna tagasi ka?}

Jah, kui nii-öelda toodeti mingit sisu, ehk siis, kui keegi kirjutas mingit programmi ja kui see ei olnud  päris oma tarbeks mõeldud ja see ei olnud mingi mäng vaid kui see oli näiteks mingi funktsioonide või alamprogrammide teek ehk siis \emph{library}. Mäletan Mast\index[ppl]{Kaal, Madis} tol ajal kirjutas tekstiliideste tegemiseks ühe funktsioonide teegi, millega sai teha menüüsid ja kaste ja igasuguseid asju ekraani peal. Tekstirežiimis aga hiir oli  abiks, sallega sai klikata sealt menüüdest midagi välja. Jah, selliste asjade jaoks ikka oli mingisugused FAQd või mingid lihtsad juhendid olemas igal vähegi mõistlikumal autoril. Sest mängud olid küll sellised jah, et võtad flopi ja installid, või lased mängu käima ja siis vaatad, et kuidas ta tööle hakkab ja mida mingi nupp teeb. Ma arvan, et ega mängude manuaale ilmselt keegi eriti ei lugenud.

\question{Pärast Skriiningut sa jõudsid mingi hetk Unineti ka?}

Jah, mingisugusel ajal on Uninet  olnud minu tööandja küll.

\question{Kas päris alguses või kunagi hiljem?}

See vist oli umbes neljas töökoht. Pärast Skriiningut ma sattusin Baltic Computer Systems-isse\index{Baltic Computer Systems}, mis ka tänapäeval eksisteerib. Ja BCS-is ma tegelesin  nüüd üsna sihituna juba arvutivõrkudega, ehk siis meil oli arvutivõrkude osakond ja me  ühelt poolt tegelesime kaubedusega ja teiselt poolt  ka serverite ja mingil määral ka sellise tarkvaraga, mis  oli vaja võrgus käima panna. Näiteks andmebaasid, mis võib-olla olid mõeldud algselt ühes arvutis kasutamiseks aga mis siis kuskil ettevõtetes oli vaja niimoodi käima panna, et nad töötaks võrgus. 

\question{Vahemärkusena, tol ajal enamus andmebaase olid mõeldud käima ühes arvutis. See tähendas, et mingisuguseid transaktsioone või midagi ei olnud keegi sinna sisse ehitanud}

Nojah, sed otseselt ei olnud olemas, aga oli mingisuguseid viise, kuidas sellest mööda hiiliti, et  kui andmebaas arvutis lahti teha, et siis see andmebaas ei oleks mitte lukus võrgus kõikide kasutajate jaoks vaid et seal saaks ikkagi midagi teha. 

\question{Mis sa praegu teed?}

Vahepeal ma olen teinud igasuguseid muid asju, mis ei ole olnud väga sellise võrgu tehnilise ülesehitusega seotud vaid mis on rohkem seotud võrgu rakenduste ja turbega. Nüüd paar kuud ma olen uuesti jälle mõnes mõttes sattunud tagasi sellise tegevuse peale, mis on seotud taas kord võrgu baasprotokollidega ühelt poolt. Ehk  ma peaks praegu une pealt teadma, kuidasmoodi erinevad IP,  TCP  ja UDP kihi protokolli töötavad. Nii palju on minu praegusel tegevusel ka endisel seos muidugi mingite rakendusprogrammidega ja äppidega  mobiilide sees, et otsapidi minu töö on ka teada ja vastavalt sellele seadmisele siis toimetada sellega, kuidas need äpid võrgus käituvad. Mismoodi nende liiklus on või  andmevahetus on võrgus üles ehitatud, kuidas seda andmevahetust ohjes hoida, kuidas seda juhtida. Kuidas hakkama saada sellega, mida Google pidevalt uute protokollide näol välja pakub ja mille eesmärk Google'; loomulikult on see, et kasutajal oleks Internetis turvalisem. Minu eesmärk on see, et lisaks sellele  peab kasutajal seal Internetis ka olema mugav. Ehk siis andmed peaks jõudma ühest punktist teise nii kiiresti, et kasutaja teadvustaja ei teavdusta, et  kuskil seal vahepeal Internet on, mis võib olla ebatöökindel või aeglane.

\question{Siis on ju selles mõttes toredasti, et kui sa alguses rääkisid huvist mängus tegelasele kapoti all toimetades teist värvi müts pähe panna, siis praegu lihtsalt see tegelane teistsuguse arvuti sees ja kapotialune on natuke keerulisem aga ülesanne on suuresti sama}

Täpselt nii. Minu jaoks on oluline see, mis on karul kõhus. Kuidasmoodi see seal kõhus töötab. Kui ta ei tööta hästi, kas ja mida  saab paremaks teha. Ja noh, kui ta töötab hästi, siis  sellest hoolimata kindlasti saab midagi teistmoodi teha.

\question{Ja kui hästi töötab, on ju huvitav, et kuidas?}

Jah, kuidas töötab ja miks  ta nii hästi töötab.
