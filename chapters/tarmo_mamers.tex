\index[ppl]{Mamers, Tarmo}

\question{Kuidas sina arvutite juurde said?}

Mul oli üks klassikaaslane, kelle isa töötas Küberneetika 
Instituudis\index{Küberneetika Instituut!Juhtimissüsteemide osakond}. Ma arvan, et see võis olla kuskil 
keskkooliaastate alguses, ilmselt siis seitsmes-kaheksas-üheksas klass, kus sai 
käidud päris mitu korda järjest tutvumas sellise asjaga nagu Apple 
II\index{Arvutid!Apple II}. See oli  mõnevõrra keeruline, sest et see Apple oli 
üsna koormatud, kuna seal Küberneetika Instituudis  kasutati teda mingisuguseks 
teadus- ja uurimistööks. Ma küll ei teadnud, mille jaoks täpselt. Selle Apple II 
monitori asemel oli tavaline telekas ja ma mäletan, et kui ma 
esimesi kordi sinna sattusin - see oli talvisel perioodil, siis põhiliselt seda 
telekat kasutati suusahüppe MM-i jälgimiseks. Ehk päris alati ei saanud 
seda arvutit näperdada või kui sai, siis ilma pildita. See oli  tõenäoliselt 
umbes aastatel 1983 kuni 1985. 

\question{Mis sa tegid selle arvutiga?}

Kõigepealt vaatasin, mis sõbrad teevad ja nad põhiliselt mängisid igasugu 
huvitavaid mänge, mis seal Apple peal tol ajal olid. Kui ma ise hakkasin seda 
näperdama, siis mind pigem hakkas huvitama see, et kui need arvutimängud on 
mingi teatud hulga eludega ja teatud hulga relvadega ja teatud hulga mingite 
abivahenditega, mida saab kasutada, siis kas neid kuidagi nii-öelda ära häkkida ei 
saaks? Et võib-olla oleks rohkem elusid,  võib-olla oleks rohkem abivahendeid, 
kuidagimoodi saaks rohkem punkte näiteks. Millegipärast minu huvi oli 
hoopiski selline. Mitte  mängimine, vaid pigem mängude ümber 
tegemine, nendes millegi kuidagi teistmoodi tegemine. Võib-olla, et mingi tegelase 
müts ei oleks mitte punane vaid oleks hoopis roheline, midagi sellist. Kuidagi  
nikerdada,  et midagi muutuks ja saaks teistsuguseks. 

Ma ei teadnud midagi ei programmeerimisest ega eriti ka arvutite 
tööpõhimõttest, see oli  esimene ajendav tegur, mis tõi kokkupuute esiteks 
BASIC-u\index{Keeled!BASIC} keelega ja järgmisena Apple 
assembleriga\index{Keeled!Assembler}. Ma sel ajal natukene tundsin huvi ka 
elektroonika ja eriti digitaalelektroonika vastu, ja juhtus nii, et mulle 
jäid ette ka selle Apple arvuti manuaalid. Tollaste arvutitega olid vist alati 
manuaalides kaasas nende elektroonikaskeemid. Põhimõtteliselt minu huvi oli 
sealt siis  näpuga järge ajada,  et kuidas need bitid liiguvad, kui midagi 
printida või nuppu vajutada. Või kuidas ekraani peale pilt tekitatakse 
mälus olevatest bittidest. See oli algus.

\question{Kas need mängud olid kuskilt väljastpoolt tulnud või liikus ka ise 
tehtud asju?}

Apple-i peal olevad mängud olid välismaalt ilmselt tulnud ikka sama kanalit pidi, kust 
arvutid ise olid Nõukogude Liitu tulnud. Või siis osadel huvilistest, eeskätt need 
Apple-i kasutajaid, kes olid kuskil Tartus või Nõos, kellel oli kuuldavasti ja 
mäletatavasti ka mingisuguseid kontakte teiste Apple kasutajatega mujalt 
maailmast, said vist mänge neilt ka.

\question{Tehti ju ise ka algelisi mänge, mingis Juku\index{Arvutid!Juku} 
mängus sai mõisa majandada, ma mäletan?}

Apple ajast ma ei mäleta väga palju mingisuguseid kodukootud mänge. 
Küll aga mõnevõrra hilisemast ajast, kui mul oli kasutada Nõukogude 
päritolu arvuti 
Iskra-226\index{Arvutid!Iskra!Iskra-226}\sidenote{\begin{russian}Искра 
226\end{russian} oli Nõukogudemaal toodetud arvuti Wang 2200 kloon, mis oli 
originaaliga binaarkoodi mõttes sajaprotsendiliselt ühilduv. Siiski oli 
Iskra-226 sisemine struktuur oluliselt erinev ning ta sisaldas mitmeid 
täiendusi, mis muutsid ta  sobilikumaks tööstusrakendusteks.}. Seal oli küll 
isetehtud mänge, mille idee oli võetud kuskilt mujalt ja tehtud mäng valmis  
või oli ka mingisugune originaalne mänguidee arvutimänguks vormistatud. 

Venelastel oli selline Apple kloon, mille  nimi oli  Agat\index{Arvutid!Agat}. 
Kui neid Eestisse hakkas tulema, siis nende jaoks oli muidugi mingisuguseid vene 
päritolu mänge. Osa  olid selgelt Apple pealt maha lükatud ja osa olid  
nii-öelda originaalid. Sellest ajast ma ka ei mäleta, et oleks eriti
mingisuguseid kodumaist päritolu mänge  olnud.

\question{Kas need inimesed, kes seda arvutit päriselt kasutasid, lasid sul 
seal talle niisama lihtsalt kõhu alla vaadata ja kaant maha kruvida?}

No ega seal Küberneetika Instituudis\index{Küberneetika Instituut} väga palju 
ei lastud, sest seal oli oluline ikkagi see, et masin oleks töökorras ja igal 
ajal kasutatav  uurimistöö jaoks.

Hiljem, ikka veel keskkooli ajal, sattusin ma ka tollasesse TPI-sse ehk siis 
praegune Tehnikaülikool\index{Tallinna
Tehnikaülikool!Automaatikateaduskond!Raadiotehnika kateeder} kus raadiotehnika 
kateedris oli ka üks Apple II\index{Arvutid!Apple II}.  Seal olid siis juba 
nii-öelda raadiotehnikud, kelle igapäevane leib oligi see,  et vaadata pigem 
seda, mis seal kõhus on ja kuidas käib. Seal sai siis masina sisse vaadata, 
kõrval oli olemas  jootekolb, millega sai pädevam seltskond teha ise Apple 
jaoks mingisuguseid perifeeriakaarte, et TPI majas mingisuguseid juhtimisi või 
mõõtmisi või asju teha.  

\question{Tollal, ma saan aru, oli üsna hästi teada, kus mingisugune Apple II 
või Iskra saadaval oli?}

Jah, sest neid olid üsna piiratud hulk. Ja arvutihuviliste seltskond 
tundis ja teadis üksteist üsna hästi. Ma 
ju teadsin neid inimesi, kellega ma kuskil arvutiringis iga nädal kokku puutun. 
See seltskond võis olla mingisugune kolmkümmend, võib olla nelikümmend inimest. 
Neid arvutiringe,  millega mina kokku puutusin sel ajal oli kolm tükki 
või isegi neli. Ja info liikus selle kohta, mis arvuteid kuskil näppida saab.

Oli selline koht nagu Oktoobrirajooni Õppetootmiskombinaat\index{Tallinna 
Oktoobrirajooni Õppetootmiskombinaat}, kus oli arvutiklass Yamaha 
MSX\index{Arvutid!Yamaha MSX} arvutitega, mis kindlasti on Prontol\index[ppl]{Pronto} 
paremini meeles, seal käis päris suur seltskond noori huvilisi koos. 

Ja siis oli TPI Arvutiring\index{Arvutiklubi!TPI Arvutiring}, mida juhendamas 
Vladimir Viies\index[ppl]{Viies, Vladimir} ja mingi hulk teisi TPI õppejõude. Seal 
olid Robotronid\index{Arvutid!Robotron}. Ma küll ei mäleta, mis see täpne tüüp 
oli, aga mingid  sellise kaubamärgiga masinad seal olid. TPI arvutiringis 
ma puutusingi tegelikult ka esimest korda  kokku selle 
Iskra-226'ga\index{Arvutid!Iskra!Iskra-226}.

Tallinnas oli  3. Keskkool\index{Koolid!Tallinna 3. Keskkool}, 
kus oli  matemaatikaõpetaja  Jaak Loonde\index[ppl]{Loonde, 
Jaak}. Tema oli haridussüsteemis selline omaette fanatt ja  populariseeris 
kõvasti tol ajal, et arvutid nii riistvara kui arvutiõppe näol koolidesse 
jõuaksid. Jaak Loondel  mingist hetkest alates oligi üks 
Agat\index{Arvutid!Agat} kasutada. Ma ei tea, kustkaudu nende kool selle 
kuskilt Venemaalt sai. 

Lisaks arvutiringidele käisin ma  34. kooli 
tehnikaringis\index{Arvutiklubi!34. Kooli Tehnikaring}\index{Koolid!Tallinna 34. 
kool}, mille juhendaja oli Ants Reili\index[ppl]{Reili, Ants}. Seal oli veel
päris mitu sellist poissi, kes ei olnud otseselt sellise üldise tehnika 
või siis elektroonika huviga, vaid rohkem  just arvutihuviga, kuigi arvutiklassi
seal ei olnud.

Eks need vist olid põhilised seltskonnad minu jaoks, kus siis igasühes oli  ka 
mingisugune kattuvus liikmeskonna osas. Muidugi oli 
veel mingisuguseid seltskondi või nii-öelda  huviliste ringe, omakorda mingi kattuvusega.

\question{Need olid kõik Tallinnas, eksole?}

Mina puutusin kokku Tallinna omadega. Tartus oli hulk inimesi, kes koondus ülikooli 
juurde. Seal oli Anne Villems\index[ppl]{Villems, Anne} ja mingi hulk, kui 
õigesti mäletan,  Apple-id. Kus oli Tartus veel Apple olemas, oli Füüsika 
Instituut\index{Füüsika Instituut}. Seal oli Jaan 
Pruulman\index[ppl]{Pruulman, Jaan}, kellega mina puutusin esimest korda  kokku, 
kui  ma ükskord sealset Apple II-te\index{Arvutid!Apple II} käisin vaatamas. Ma ei 
mäleta asjaolusid, aga ma sattusin mingil põhjusel talvisel ajal Tartusse mingisuguste kooli- või 
ringikaaslastega, ja  me mõtlesime, et lähme sinna Füüsika Instituuti külla, 
sest seal esiteks saab sooja (ilmad olid tol hetkel väga külmad ma mäletan) ja 
äkki saab arvutis ka midagi teha. Seal me Pruulmaniga tuttavaks saime. Ma ei 
tea, kuidas need asjad toimisid selles mõttes, et üks hommik mingi seltskond 
koolijuntsusid võtab pähe, et lähme ja sõidame. Kellelgi mingeid kontakte ei 
ole, mingeid eelnevaid kokkuleppeid ei ole, tol ajal ka ei helistatud taskusse ja ei saadetud 
meile, aga lõppkokkuvõttes tuli kõik nagu  välja.

\question{Kui ma koolist saan aru ja ülikoolist ka, aga mis selle 
Oktoobrirajooni asutuse huvi oli arvutiklassi hankida?}\label{content!OTK}

Need õppetootmiskombinaadid olid sellised mitme kooli peale, ehk siis rajooni 
kaupa (Tallinnas oli tol ajal neli rajooni\sidenote{Alates 1974. aastast 
jagunes Tallinn Oktoobri-, Lenini (endine Kesk), Kalinini ja Mererajooniks.}). 
Ja noh, asutuse nimi oli õppetootmiskombinaat, mis ilmselt siis pidi viitama 
sellele, et seal saab mingit praktilist asja proovida teha ja mingeid 
ametikogemusi saada. See arvutiring oli muidugi puht selline huviring, seal ei 
olnud mingit  tootmisväljundit, nagu see õppetoootmiskombinaadi nimi võiks 
eeldada.

\question{Seda minagi imestan, et miks nad hankisid need arvutid, see pidi ju 
keeruline olema?}

No igal juhul neil oli arvutiklass mingi tosinkonna Yamaha arvutitega. Võis ka olla, et 
kuna see klass oli otsapidi seotud sellesama Jaak Loondega\index[ppl]{Loonde, 
Jaak}, et tema kuidagi selle arvutiklassi sinna sebis ja et 
Õppetootmiskombinaat oli lihtsalt nii-öelda \enquote{katus}. Seal  ei olnud see 
klass ühes konkreetses koolis, kus oleks olnud mingisuguseid poliitilisi pingeid, 
et näe, neil on aga teistel koolidel ei ole. Võib-olla seda klassi oli nii lihtsam 
organiseerida. 

\question{Kui sa arvutitega toimetama hakkasid, siis millele sa toetusid? Tühja 
koha pealt inimene ju ei vaata arvuti skeemi pealt, kust kuhu bitid liiguvad?}

No elektroonika tausta mul oli nii palju, et seda ma teadsin, mismoodi bitid 
liiguvad ja mismoodi  loogikatehted toimivad ja kuidasmoodi neid asju tööle panna. 
Kuidas näiteks teha LCD displeiga elektronkella.  

\question{Kuidas sa oskasid?}

Ühelt poolt selle 34. kooli tehnikaringi\index{Arvutiklubi!34. Kooli 
Tehnikaring} teadmiste baasil. Ja teisalt ma lunisin vanematelt 
endale välja küllaltki palju kirjandust, mis oli põhiliselt vene ja saksa 
keeles, inglisekeelset kirjandust ei olnud tol ajal lihtsalt kuskilt saada. Või 
kui oli, siis see oli ilukirjandus ja niisugune aime ja ulme, eks, aga mitte 
mingisugune teadus- või tehnikakirjandus. Tehnikakirjandus, kui ta oli 
mitte-nõukogude päritolu, siis see oli ikkagi saksa keeles. Ega ma ausalt öeldes 
neid raamatuid ühtegi otsast lõpuni läbi ei lugenud, aga ma neid siiski 
natukene sirvisin ja  lugesin võib-olla mõnesid olulisemaid peatükke. Sealt 
tasapisi ilmselt see teadmine tekkis.

\question{Millest me järeldame, et saksa ja vene keeles tehnilise sisulise 
teksti lugemiseks ei olnud probleem tolleks hetkeks?}

Minu jaoks saksa keel küll oli, sest koolis ma olin õppinud inglise keelt, 
süvendatult. Inglise keel oli küll nii selge, et minu jaoks oli see eesti keele 
kõrval  pea-aegu nagu teine emakeel. Aga saksa keelt ma ei osanud üldse. 
Tänapäeval on küll nii, et võtad raamatu lahti, ja mis sest, et ma saksa 
keelt ei oska, aga  inglise keelega on palju sõnu samad, mingite muude tuntud 
keeltega on palju sarnasusi, nii et mingist üldisest mõttest saab aru. Muidugi 
selliseid konkreetseid juhiseid või mingit faktilist infot ma saksa keeles 
ikkagi täisväärtuslikult ei loe. 

\question{Mis koolis sa käisid?}

44. Keskkool\index{Koolid!Tallinna 44. Keskkool}, mis on tänapäeval Mustamäe 
Gümnaasium\index{Koolid!Mustamäe Gümnaasium|see{Koolid!Tallinna 44. Keskkool}} 
ja seal ma õppisin inglise keelt. Aga kooli ajal see inglise keel, ka 
IT-maailmas, ei olnud mingi asi, mida oleks saanud tegelikult väga palju rakendada. No 
peale selle, Basicus on käsk \verb|print| küll inglise keeles. Aga 
Basicus neid käske nii väga palju ka ei ole ja neid ei ole keeruline ka pähe 
õppida sel juhul, kui sa inglise keelt ei peaks valdama. 

\question{Kas tehnikakirjanduse juurde käis ka mingisugune mingisugune muu 
kirjanduse või ulme huvi? Filmid,  raamatud?}

Kooli ajal ma lugesin üsna palju ulmet inglise keeles. Ja kooliajal sattus 
mulle kätte ka Douglas Adamsi  Hitchhikeri raamat\sidenote{Vaata ka 
märkust leheküljel \pageref{sidenote!adams}.}.
Kuna inglise keel oli meil koolis  süvaõppes, siis meil oli 
inglise keele kodulugemise tund. Pidi kodus mingit 
ilukirjandust lugema inglise keeles ja tunnis õpetajale jutustama. Läksin raamatupoodi 
ja nägin kuskil üleval lae all riiuli peal seda Hitchhikeri kõige esimest osa. 
Vaatasin, et see on huvitav pealkiri, ostsin selle raamatu ära ja mõtlesin, et 
võtangi siis selle inglise keele kodulugemiseks. See ei olnud siiski väga mõistlik 
valik, sest kui mõelda nende sõnade pele, mida seal kasutatakse -- välja mõeldud 
sõnad, välja mõeldud liiginimed, seadmete nimed ja nii edasi -- siis need on 
eesti keelde üsna raskesti tõlgitavad,  peab väga hea fantaasiaga tõlkija 
olema. Aga mina hakkasin entusiastlikult seda raamatut lugema ja  õpetajale jutustama. Ma 
küll ei tea, kui palju õpetaja tol ajal sellest aru sai, mis ma talle 
ette jutustasin, aga vähemasti ta jäi rahule. 

\question{Seal ju ei ole narratiivi, on mingisugune keeruline sõlm, mis viienda 
raamatu lõpuks umbsõlme läheb!}

Ega mina ka ei saanud  sellest eriti väga palju aru, kui ma seda esimest osa 
kooli ajal lugesin. Hiljem lugesin  ülejäänud osad ka läbi, siis nagu loksus 
pilt paika.

\question{Aga oli siis saada ingliskeelset ilukirjandust?}

Jaa, ingliskeelset ilukirjandust oli küll. Seda oli igal pool, isegi Tallinnas. 
Pärast kooli lõppu, kui ma töötasin TPI-s\index{Tallinna Tehnikaülikool}, 
käisin ma palju Moskvas ja Leningradis komandeeringus ja seal oli valik väga 
lai. Asimovilt ma vist ei sattunud esimesena lugema tema tuntuimat sarja Asum\sidenote{Vaata 
ka märkust lehekülhel \pageref{sidenote!asum}.}, vaid tema
üksiklugusid. Aga Asimov ja Adams olid need kaks ulmekirjanikku, 
kellega ma inglise keele vahendusel esmalt kokku puutusin. 

\question{Aga vene klassikud? Strugatskid?}

Jaa, Strugatskeid ma olin lugenud varem, sest neid oli tõlgitud eesti keelde. 
\enquote{Purpurpunaste pilvede maa}\label{sisu:purpur}\sidenote{Arkadi Strugatski; Boris 
Strugatski. (1959). \begin{russian}Страна багровых туч\end{russian}. Eesti 
keeles 1961 Ralf Tominga (värsid Lembe Hiedel) tõlkes sarjas 
\enquote{Seiklusjutte maalt ja merelt}.} vist oli,  Amfiibinimene vist oli ka 
Strugatskite oma\sidenote{\begin{russian}Человек-амфибия\end{russian} on siiski 
1928. aastal ilmunud Alexander Belyaev-i romaan. Eesti keeles ilmus 1960. 
aastal \enquote{Seiklusjutte maalt ja merelt} sarjas koos romaaniga 
\enquote{Maailmavalitseja} (vene k. \begin{russian}Властелин мира\end{russian}, 
1926).}. Ja siis ma neelasin võimalust mööda igasugust pop-teaduslikku ehk 
aimekirjandust. \enquote{Mosaiik}\sidenote{\enquote{Mosaiik} oli kirjastuses 
Valgus aastastel 1973–1991 välja antud populaarteaduslike raamatute sari, mis 
käsitles äärmiselt laia teemaderingi ajaloost ja psühholoogiast topoloogia 
problemaatikani.}, selline raamatusari oli olemas. Ega suurt mingit muud 
aimekirjandust väga ei olnudki võtta.

Kriminullid olid teine valdkond, mis tol ajal peale  ulmekirjanduse 
ilukirjandusest huvi pakkus. Neid ma tollel ajal kah lausa neelasin.

Üldine lugemise tempo oli selline, 
ma mäletan, et see vist oli Asumi mingisugune kolmas või neljas osa, mille ma 
inglise keeles lugesin Rootsis töötades ühe ööga läbi. Ma ei tea, kui palju lehekülgi see 
võis siis olla, mingi \emph{paperback}, kolm või nelisada lehekülge 
tõenäoliselt. Järgmisel päeval ostsin raamatupoest järgmise osa ja nii edasi. Selliseid asju juhtus, sai tööde kõrvalt ja 
elu kõrvalt, kui  kool oli äsja lõpetatud, lubada. Selle asemel, et 
öösel magada ja puhata, võtsid järgmise raamatu ette.

\question{Kui sa keskkooli ära lõpetasid, siis sa läksid TPI-sse kohe tööle või 
ikka õppima ka?}

Enam-vähem kohe pärast keskkooli ma läksin TPI-sse\index{Tallinna 
Tehnikaülikool} ikkagi tööle, mitte õppima. See töökoht tegelikult sattuski mulle kätte tänu sellele 
Vladimir Viiese\index[ppl]{Viies, Vladimir} juhendatud arvutiringile. Ma 
läksin tööle samasse kateedrisse kus Viieski, see oli siis 
elektronarvutite kateeder\index{Tallinna Tehnikaülikool!Elektronarvutite 
Kateeder}. Aitasin seal igasuguseid arvuti hooldustöid teha, arvutilaborite 
häälestamisi ja  ettevalmistamisi, mis erinevatel õppejõududel vaja oli. 
Mõnevõrra hiljem sai kaasa löödud juba mingites arvutit hõlmavates 
tarkvaraprojektides,  kus oli vaja midagi programmeerida, kus oli vaja mingi 
sisend-väljundseadme jaoks mingisugune draiver kirjutada. 

\question{Kas sa läksid õppima ka?}

Õppima ma läksin mõnevõrra hiljem, sest mind 
punaste ainete küllus ei köitnud eriti. No keda nad oleks köitnud. Aga mina tundsin 
nende vastu niivõrd suurt vastumeelsust,  et ma leidsin, et ma ei taha ülikooli küll 
õppima minna, ka mingit tehnilist asja, kui seal on  punased ained juures. Need 
\enquote{punased} ained olid siis  NLKP ajalugu ja muud sellised asjad. Aga 
mingil hetkel ikkagi paar aastat hiljem läksin õhtusesse osakonda õppima. Olin 
küll üks enamusest, kes ei lõpetanud. Meie õhtusele kursusele ühe astus sisse vist 
kakskümmend viis inimest, kellest lõpetas kaks. 

\question{Mis eriala see siis oli?}

Elektronarvutid. Aga õhtuses osakonnas selline lõpetanute protsent oli minu 
arust üheksakümnendate algul üsna tavapärane. Oli vist 1990. aasta sügis, kui 
TPI-s õppima asusin. Sealt alates ongi tegelikult kõik töökohad ja 
 üsna palju ka vaba aja tegemised  olnud seotud programmeerimisega ja 
 arvuti tehnilise  poolega.

Kui  TPI-s sai töötatud, siis Elektronarvutite kateeder asus teisel korrusel. 
Samas korpuses neljandal korrusel asus  Raadiotehnika 
kateeder\index{Tallinna Tehnikaülikool!Automaatikateaduskond!Raadiotehnika 
kateeder}, kus oli see Apple II\index{Arvutid!Apple II}. Meil tekkis 
Mastiga\index[ppl]{Kaal, Madis} (Madis Kaal\sidenote{Kes sel ajal toimetas 
Raadiotehnika kateedris. Vt. lehekülg \pageref{sisu!mast_raadiotehnikas}.}) 
ühel hetkel  mõte, et võiks proovida PC arvuteid, mis siis olid 
meil teisel korrusel ja mis olid minu igapäevased tööriistad, kokku 
ühendada  Apple II-ga, mis oli neljandal korrusel Masti igapäevane tööriist. 
Ehitasime sinna vahele \emph{current loop}-i, see on RS-232 põhimõtteliselt. 
Misjärel tekkis meil  selline 
võimalus, et PC seest sai \emph{copy}-ida  andmeid Apple II sisse ja vastupidi. 
Nagu pilv, kasutad kellegi teise arvutit. 

Üheksakümnes aasta oli ka umbes aeg, kui Eestisse jõudis mingisugune info 
sellest, et on olemas BBS-id. TPI majas oligi Mast\index[ppl]{Kaal, Madis}  see 
entusiast, kes pani sealkandis esimese BBS-i jooksma. Mina esialgu vaatasin 
seda lihtsalt kõrvalt, mul ei olnud selle kohta nagu mingit arvamust,  ma ei 
tundnud  väga palju huvi selle teema vastu. Seal sai mänge vahetada, 
aga kuna ma ei ole kunagi mingi eriline mängu-fanatt olnud, siis selle pärast
BBS-indus mind ei tõmmanud. Hiljem ma küll 
leidsin, et need BBS-id võivad olla kuidagi kasulikud - oli vist moment, 
kui tuli välja, et seal BBS-ides on olemas  tekstifaile, mis on mingid 
referents-dokumendid, mingid \emph{manual}-id, mingid standardid, mingid 
programmeerimisõpikud kas IBM-ide või Apple jaoks.

\question{Kas need olid mingid \emph{plain text} failid või \LaTeX või mis?}

Need olid tekstifailid, aga nad olid natuke formaaditud ikkagi. Failides olid 
tabulatsioonid ja lehekülje vahed  sees, neid sai maatriksprinteriga välja 
trükkida nii, et tulid ikka ilusti formaadituna paberi peal välja. Rasvase kirjaga
ja kaldkirjaga teksti oli ka võimalik kasutada.
Maatriksprinterid olid vist kättesaadava hinnaga, sest need olid enamuse arvutite taga. 
 Suured arvutid - ES-id ja SM-id, mis olid TPI-s või Küberneetika 
Instituudis, seal olid need laiad printerid. Ma ei mäletagi, kuidas nende 
printerite kohta öeldi. Ridaprinter? \emph{Line printer} öeldi inglise keeles, 
aga  oli mingisugune eestikeelne sõna ka, mille vist Ustus Agur\index[ppl]{Agur, 
Ustus} välja mõtles. Ühesõnaga mingit koledat häält ja värinat tegevad 
printerid.

\question{Kui sa aru said, et sealt saab igasugu dokumente, hakkasid BBS-id 
sulle ka huvi pakkuma?}

Jah, ma arvan, et see oli see hetk ja see ajend, kui ma leidsin, et sealt  
peale mängude ja  tilulilu sai midagi mõistlikku ka. Mingi hetk panin oma BBS-i 
ka püsti ja selleks ajaks oli ka Fidonet otsapidi Eestisse 
jõudnud\sidenote{Esimene Fidoneti Eesti regiooni 2:49 sisaldanud 
\emph{nodelist} on 271 28. septembrist 1990. Regiooni koordinaatorina on seal 
kirjas Andrus Suitsu\index[ppl]{Suitsu, Andrus} ja \emph{Host} on Tarmo 
Ausing\index[ppl]{Ausing, Tarmo}. Vt. lehekülg \pageref{sisu:nodelist}.}. Paljud, kes  ajalooliselt on tagasi vaadanud ja rääkinud 
sellest ajast, ei pruugi eriti olla vahet teinud BBS-indusel ja Fidonetil,  mis 
tegelikult  olid kaks eraldi maailma. BBS oli lihtsalt mingi 
süsteem, kuhu sai modemiga sisse helistada ja siis seal ringi toimetada,  
mingeid andmeid failide näol endale tõmmata või siis mingisuguseid sõnumeid 
vahetada. Aga kogu see info ja need sõnumid olid salvestatud sinna ühte 
konkreetsesse BBS-i süsteemi.

Fidonet sai ühe otsaga  alguse nendestsamadest BBS-idest, Fidoneti eesmärk 
oli sõnumite BBS-ide ja mingite muude Fidoneti liikmes-süsteemide vahel 
edasi-tagasi toimetada. 

\question{Ehk, Fidonetis need kohad, kuhu sa sisse helistasid, helistasid ka 
üksteisele sisse ja vahetasid andmeid?}

Jah. Ja see oli siis juba automatiseeritud süsteem, kus olid vahendid 
selleks, et sõnumeid ehk  meile vahetada. Sõnumeid oli kahte liiki: olid 
privaat-meilid ja olid konverents-meilid (tänapäeva mõistes siis
meiligrupid või meililistid).

\question{Kas \emph{Usenet} tekkis ka sel ajal?}

Usenet oli olemas palju varem, see on hästi vana asi. Usenet ja UUCP protokoll 
on põhiliselt  Unixi-maailma päritolu,  see oli konkreetselt 
Unixi arvutite vahelise meilivahetuse protokoll. Ja see Usenet, mis  sinna 
ümber tekkis,  see oli siis ka nagu selline konverentside või vestlusringide 
süsteem Unixiga töötavate arvutite kasutajate vahel. Fidonet ja BBS-id töötasid
enamjaolt PC-arvutitel.

\question{Kas seda peegeldati Fidosse ka?}

Jah, seal olid lüüsid olemas. Usenetist sai konvertida ümber kirju Fidoneti 
\emph{echo}-desse ehk konverentsidesse. Muuhulgas ka faile, sest 
Usenetis vahetati ka väga palju faile,  neid oli siis võimalik ka konvertida 
PC-arvuti failideks, mis siis kuskil BBS-is üles pandi.

\question{Kas eestlased toimetasid seal usenetis mingites oma gruppides või 
möllati olemasolevates?}

Usenetis ma mäletan vist, ei olnud tollal mingeid erilisi Eesti spetsiifilisi või 
regionaalseid gruppe. Erinevalt Fidonetist, seal oli küll mingi viisteist või 
heal ajal võib-olla kakskümmend lokaalset  vestlustgruppi ehk \emph{echo}-t. 
Neist kaks-kolm gruppi olid üsna populaarsed liikmeskonna mõttes.

\question{Mis see tähendab? 50, 100, 500 liiget?}

No ma arvan, et lugejaid võis seal oli väga palju, sest ikka pidevalt tuleb välja 
inimesi, kellega mina ei ole kunagi kokku puutunud, ma ei tea neid nimepidi, 
aga nad räägivad, et nad on kunagi sealt \emph{echo}-dest  midagi lugenud. Sest 
tegelikult selleks, et neid \emph{echo}-sid või konverentse lugeda,  ei pidanud 
sa ise omama ei BBS ega mingit Fidoneti süsteemi. Sa said helistada BBS-i 
sisse, seal lugeda, ja kui sa tahtsid, siis kirjutada. Kui kellelgi Fidoneti süsteem 
oli püsti pandud, siis selle eelis oli selles, et siis talle need kirjad tulid 
automaatselt koju kätte ja tal ei olnud lugemiseks-kirjutamiseks vaja kuskile 
kaugele ise helistada. Tegelikult see ring  neid inimesi, kes  ainult luges 
võis olla päris suur. 

Kui püüda hinnata seda, kes seal aktiivselt suhtlesid ja kirjutasid ka, siis  
võib-olla see on mingi kakssada inimest. See on väga laest võetud number, 
suurusjärgus.

\question{Seda on ikkagi päris palju. Kas sa oma Fido \emph{node} panidki püsti 
selle jaoks, et asjad tuleksid koju kätte? Mis selle asja nimi oli?}

Ma arvan, et eesmärk oli jah see, et asjad oleks piisavalt automatiseeritud, et 
mul ei oleks endal vaja mingeid liigutusi teha  ega aega kulutada selle peale, et 
kuskile BBS-i nii-öelda löögile saada. Sest kui BBS-i küljes oli  
välismaailmaga suhtlemiseks üks modem, siis see tähendab, et igal ajahetkel sai 
seda BBS-i kui teenust korraga kasutada üks inimene. Oli ka BBS-e, millel oli mitu 
modemeit küljes, siis sai muidugi mitu inimest seda paralleelselt kasutada.  
See tähendaski, et helistasid modemiga, telefon oli kinni, helistasid viie minuti 
pärast, ikka kinni. Ja no miks ma pean niimoodi vaeva nägema ja pidevalt 
proovima? Tõsi küll, ka modemi enda sai panna automaatselt kordusvalima, 
ja kui lõpuks löögile sai, andis mingi signaali. Aga ma leidsin, et parem on 
seda asja lasta  Fidoneti automaatikal teha. Ja siis  saab  rahumeeli hetkel, 
kui sa tahad, avada meililugemise programmi ja lugeda seda meili, mis on 
vahepeal sul sinna masinasse ära tõmmatud. 

\question{Aga mis su \emph{node} nimi oli?}

Minu \emph{node} nimi oli MamBox. Ma ei mäleta, mis hetkel see eesliide, mis on 
siis tulnud minu perekonnanime algusest,  hakkas mingisuguste asjade külge 
tekkima. Aga tol hetkel oli jah nii, et kui ma tegin BBS-i, siis ta oli MamBox, 
kui ma kirjutasin mingisugust programmi  oma lõbuks, siis siis ma kirjutasin 
\enquote{\emph{Copyright MamSoft}}\sidenote{Tegu oli levinud praktikaga, minu 
samal viisil kasutatud fiktiivne firmanimi oli \enquote{\emph{I \& I Company}}. 
Sellest \emph{misasi} üks firma on, oli arusaam ähmane. Sellest, et firma 
\emph{nimi} tuleb kindlasti ära mainida ja kuulsaks teha, oli arusaam väga 
konkreetne.}. See oli tol ajal selline kaubamärk, mida ma kasutasin siis 
ühesuguste eesliidetega Mam.

Üsna tüüpiline oli see, et kui kellelgi oli BBS, siis 
ta mingil hetkel lisas sinna  Fiodneti funktsionaalsuse ka siis, kui tal seda 
alguses ei olnud. Ja väga palju oli ka teistsuguseid suundumusi, et kui sul oli 
mingil põhjusel tekkinud Fidoneti \emph{node}, siis väga palju nende  omanikest 
mingil ajal leidsid, et võiks  ka BBS-i püsti panna. 
Muidugi oli väga palju ka Fidoneti \emph{node}-sid, kelle omanike või siis 
\emph{sysop}-ide eesmärk oligi lugeda-kirjutada ja automaatselt lasta sõnumeid 
vahetada,  nende huvi ei olnud mingisugust BBS-i üleval pidada.

\question{Ehk, kui mõni BBS sai populaarseks siis võis see olla nii seepärast, 
et seal vahetas aktiivne kogukond omavahel faile kui ka see, et miskipärast 
otsustasid paljud kasutajad just sealtkaudu Fidonetile ligi pääseda?}

Fidoneti sisu ligi pääses kõikidest BBS-idest, kes olid Fidoneti liikmed, sest 
kõigis oli põhimõtteliselt ühesugune koopia konverentskirjadest ehk \emph{echo}-dest. 
Iseasi olid privaatkirjad - siis oli  vaja Fidoneti \emph{node} numbrit teada, 
kuhu saab kellelegi konkreetsele inimesele kirja saata. Iga inimene oli  mingisuguse 
Fidoneti \emph{node}-ga seotud   privaatmeili vahetamiseks. Aga mis konverentse 
ehk \emph{echo}-sid puudutas,  siis need  olid ühtmoodi igas BBS-is  saadaval.

Aga ega muidugi ei olnud eriti mõnus ka see, kui täna loed siit, homme  hoopis 
teisest BBS-ist seda meili. On ju viidad, kui palju sul on loetud meile, kus 
su lugemisjärjekord on, kas sa oled millelegi vastanud või ei ole. See  läheb 
sassi, kui sul ei ole oma sellist nii-öelda kodu-BBS-i. Ja oli ka selge, et kus 
oli väga populaarne faile käia tõmbamas,  need BBS-id on  üsna hõivatud ja 
tihtipeale liinid kinni nende failide tõmbamise pärast. 

\question{Faili tõmbamine võttis ju tükk aega!}

Jah. Alguses, kui BBS-id Eestisse tekkisid ja need Fidoneti \emph{node}-d, 
siis, 14400 boodi (ümmarguselt võib seda teisendada 14400 
bitti ehk siis 14 kilobitti sekundis) andmevahetuskiirus oli üsna tüüpiline  
algupäevadel nende BBS-ide juures.

\question{Ma isegi mäletan 9600-seid miskipärast}

9600 oli jah selline lihtne, odav igamehe entry-level tehnoloogia. Aga, ütleme, 14400 olid 
sellised modemid, kuhu poole kõik ikka püüdlesid. Ja sealt edasi tulid siis 
19200, 26600, veelgi suuremad kiirused. Minul ühel hetkel oli kasutada sellised 
üsna  härjad modemid, mille töökiirus oli 33600 boodi. Aga see kiirus tuli 
kätte ainult sellisel juhul, kui teisel pool sideliini otsas on vastas täpselt 
sama tootja modem. Modemi  nimi oli Trailblazer\index{Telebit 
Trailblazer}\sidenote{USA tootja Telebit, kes Trailblazeri sarja tootis, 
kasutas üldlevinud V-seeria protokollide asemel omaenda protokolli Packetized 
Ensemble Protocol (PEP).}. US Roboticsid\index{US Robotics}  töötasid BBS-ide 
nii-öelda  põhiajastul kõige kiiremini vist 34.4 kiloboodi juures.

\question{Kas BBS-idega majandamine tekitaski sul võrgu-huvi? Sa rääkisid, 
kuidas te Mastiga Apple-t ja PC-d paaritasite?}

Ma arvan, et see Apple ja PC paaritamine oligi see, mis  võrgunduse kui sellise 
pisiku tekitas, sest ega TPI-s ega ka kuskil mujal, kus alguses arvutitega sai kokku 
puututud, ei olnud mingisuguseid kohtvõrgutamise tehnoloogiaid 
kasutusel. Ainukene oli UUCP, mis käis Unixite vahel, see oli rohkem nagu 
selline tõsisemate ja suuremate arvutite sidepidamine  ja rohkem nagu teadus- 
ja akadeemilistes ringkondades, eks. Ja teisalt siis Fidonet, mis oli selline 
rohkem asjaarmastajalik. Alles pärast TPI-d, järgmises töökohas, puutusin ma esimest 
korda kokku niiöelda päris kohtvõrgutehnoloogiaga ARCNet\sidenote{ARCNet oli 1980. aastatel levinud esimene laia 
kasutamist leidnud mikroarvutite võrgusüsteem. ARCNet on siiani kasutusel 
sardsüsteemide puhul.}.

\question{Kus see oli ja mis aastal umbes?}

See oli  aasta 1991, ettevõttes Skriining\index{Skriining}, mis 
tegutseb veel tänapäevalgi. Skriiningus ma puutusin siis kokku ARCNetiga, mis jooksis 
tol ajal kahe ja poole megabitise kiiruse peal. See oli koaksiaalkaabli võrk, 
peaaegu nagu esimesed Etherneti võrgud, aga siis neli korda aeglasem. Minu 
arust see koaksiaalkaabel, mida ARCNet kasutas, oli ka vist seitsmekümne viie 
oomine, versus siis Etherneti viiekümne oomine kaabel. 

Aga noh, see ARCNet oli  üsna lühiajaline, temaga olid kokkupuuted peamiselt  
tänu sellele, et see oli  aeg, kus Soomest ja mujalt lähi-välismaalt 
seljakotiga kraami toomas käidi. Väga palju kraami, mis Soomest tuli, oli 
selline kraam, mis Soomes oli maha kantud, seda vist ei tahetud seal ära visata, 
sest  utiliseerimine maksis, siis anti ära, et \enquote{kasutage, tehke 
midagi}. Ma ei mäleta, et ARCNetiga midagi väga tõsist oleks tehtud, aga 
mingeid kokkupuuted sellega ikkagi olid. Selle järel tuli siis Ethernet, mis 
oli tol ajal koaksiaalkaabli Ethernet, kümme megabitti sekundis. See oli juba 
selline asi, mis hakkas päris reaalselt ettevõtetesse jõudma ja mille baasil 
hakati  üsna palju  kohtvõrke ehitama.

\question{Räägi korraks palun sellest Skriiningust\index{Skriining}. Arvutiäri 
jaoks peaks nagu nime järgi olema kaks poolt: arvuti ja äri. Aga et aastal 1991 
oleks kumbagi olnud, tundub natuke uskumatu.}

Noh, arvutid olidki sellised, mis alguses tulid seljakotis piiri tagant. Ja 
järgmine faas oli see, kus  nad tulid endiselt seljakotiga piiri tagant, aga 
selleks, et neid saada, selleks oli vaja sinna piiri taha seljakotiga 
kõigepealt sularaha viia. Sest kakskümmend tuhat rubla, ma arvan, võis olla  
selline keskmise arvuti hind. Ma ei tea, mina ei puutunud hindadega kokku, sest 
ma ei tegelenud müügitööga. Nii et ma ei kujuta ette, kui palju  arvutid tol 
ajal  numbriliselt maksid, aga arvutustehnika oli ikka meeletult kallis.

\question{Kuidas sihuke firma üldse võis tekkida tol ajal? Ei saanud ju 
internetti kuulutust panna, et \enquote{tulge meile tööle}?}

Ma ei tea, IT-maailmas inimesed liikusid ilmselt tutvuste kaudu ühest kohast 
teise tööle. Ja mina sinna Skriiningusse\index{Skriining} jõudsin ka  tutvuste 
kaudu, sest et üks inimene, kes varem oli olnud minu kolleeg TPI-s, sattus 
Skriiningusse tööle ja kutsus paar aastat hiljem mind ka sinna. 
Skriiningu nii-öelda vertikaal või kliendisegment oli ja on ka tänapäeval 
meditsiiniasutused - võrgud, arvutibaas ja infosüsteemid, 
programmeerimine ja hooldamine. Ma arvan, et see on ka üks põhjus, miks 
Skriining on tänapäeval  endiselt elus: tal on oma 
üsna kitsas kliendisegment ja kindlad ja pikaajaliselt väljakujunenud kliendisuhted.

\question{Sinu jutu järgi tundub, et need esimesed arvutifirmad olid sõprus- 
või vähemalt tutvuskonna põhised?}

No paljud ikka väga ei olnud. Sest Skriiningus see Mart, kes enne mind sinna 
läks ja kes mind hiljem kutsus, oli ainukene inimene, keda ma seal tundsin. Aga 
jah, sellised arvutifirmad ei olnud suured. Skriining oli, ma ei mäleta, viis-kuus 
inimest vist, mitte rohkem. Kõik tegid  enam-vähem kõike. Võib-olla 
mõni jah programmeeris rohkem, võib-olla mõni, nagu mina näiteks, vedas rohkem 
kaablit või käis seal mingeid kruvisid keeramas või timmimas mingeid asju seal 
arvuti kaane all. Mingid eelistused olid kindlasti inimestel olemas, aga 
üldjoontes võib öelda, et kõik käisid nagu mingil määral vähemalt üle kõikidest 
 süsteemidest, mis firma  sees kasutusel olid või millega see firma tegeles.

Samas oli muidugi ka mitmeid sõprade seltskondi, kes üheskoos tegid mõne arvutifirma.

\question{Kas sa sel ajal veel oma BBS-i ka pidasid?}

Jah. BBS  oli mul üleval päris pikka aega, ma olen teda nii-öelda kaasa vedanud 
 ühest töökohast teise, sest ega tol ajal kodus ei saanud teda pidada. Noh, 
esiteks koju ei olnud kellelgi eriti võimalik arvutit hankida, see oli kallis. 
Ja kui oli ka võib-olla võimalik mingi niru arvuti hankida, siis selle peale 
BBS-i hästi püsti ei pane. Ja teisalt, tol ajal kodus telefoniga 
välja helistamine ei olnud just mitte kõige odavam lõbu. Pealegi, kui mõelda 
Fidoneti peale ja et see Fidonet  oli ülemaailmne süsteem, siis Fidoneti side 
hõlmas ka mingit hulka rahvusvahelisi kõnesid. Sel ajal kodustelt numbritelt ei 
olnud  võimalik otse  välismaale helistada, kaugvalimine toimus läbi 
inim-operaatori\sidenote{Jaan Tallinn\index[ppl]{Tallinn, Jaan} on rääkinud, et 
nondele inim-operaatoritele oli täiesti võimalik arvutiside vahendamine ära 
õpetada. Tuli öelda, et \enquote{kui vilistama hakkab, ühendage ära, nii peabki 
olema}.}. Ja ega ka kõikidest ettevõtetest  ei olnud võimalik välismaale 
otse helistada. Tihtipeale oli ettevõttes ainult üks telefoninumber, võibolla mingi kümne 
või saja telefoni peale, kust sai otse välismaale helistada. Seda siis püüti 
endale ära rääkida, et sinna taha saaks BBS-i ühendada. Tihtipeale olid ka 
BBS-i omanikel kokkulepped, et nende BBS töötab ja saab telefoniliini kasutada 
öösiti, ja päeval saab seda liini kasutada kontoritööks. 
Sellised ajad tekkisid hiljem, et BBS-i jaoks oli mõnedes firmades võimalik 
saada kakskümmend neli tundi eraldi telefoniliin ja oli veel eriti hästi, kui sealt sai ka 
välismaale helistada. Selliseid kohti oli. 

Liini oli võimalik jagada ka BBS-i modemi ja ettevõtte faksiseadme vahel, siis said
mõlemad sõbralikult töötada 24 tundi ööpäevas.

Ja noh, tol ajal oli nii, et kui ma liikusin ühest töökohast teise, siis 
ma uut ettevõtet muuhulgas hindasin ka selle järgi, et kas mul on võimalik 
BBS sinna kaasa võtta ja kas mul on seal võimalik  BBS-i jaoks saada  
kaugvalimisega telefoniliin ja veel parem, kui see liin oleks  ööpäev läbi 
 kasutatav.

\question{Need on ju päris olulised valikud, mida see BBS-i kaasa vedamine 
sulle pakkus? Mis selle juures huvitav oli?}

No ikka see, et info tuleb üsna lihtsalt kätte, mida on võimalik BBS-idest saada. 
See info on ka parajalt lihtsalt maailma pealt otsitav, kui sul juba Fidoneti \emph{node} püsti 
on. Ja automatiseerida sai ühelt poolt meili-  aga teiselt poolt ka 
failivahetust. Kui ma tahan saada ka kätte kuskilt kaugelt BBS-ist mingit 
faili, ma tean selle faili nime, siis mul ei ole vaja endal käsitsi jällegi 
sinna BBS-i sisse logida, et seda faili endale tõmmata, vaid ma saan seda teha 
Fidoneti automaatika abil.

\question{Toonases Fido maailmas toimetav seltskond oli ikkagi suhteliselt suur 
ja sinu nimi jookseb nende juttudest päris oluliselt läbi. Miks see nii on?}

Seal ei ole väga palju midagi arvata. Tegelikult on küll olemas palju nimekamaid 
BBS-i pidajaid, kes seda  BBS-i maailma Eestis põhimõtteliselt  alustasid ja 
kes on BBS-i kontekstis palju tuntumad.

Kui  BBS-id ja Fidonet  olid Eestis 
levima hakanud ja üsna agarasti kasutusele võetud, siis üsna pea tekkis meil 
mingis Fidoneti inimeste seltskonnas  selline äratundmine, et nojah, et meid on 
siin küll  sada kuni kakssada inimest, kes igapäevaselt   Fidoneti 
kaudu suhtlevad ja kirju vahetavad ja teevad nalju ja vahetevahel sõimavad 
üksteist ja mida iganes. Aga noh, meie siin  näeme võibolla kümmet inimest päevast 
päeva, võib-olla, keda me teame nime- ja nägupidi, aga teisi me nägupidi ei tea. Et 
peaks sellele probleemile mingisuguse lahenduse otsima. 

Tegelikult oli juba 1991. aasta suvi, kui huvi teiste BBSi kasutajatega näost näkku kohtuda kasvas nii suureks, 
et enam-vähem siis seesama  umbes kümne-inimeseline seltskond mõtles, et võiks siis 
teha mingisuguse kokkutuleku. Ma küll ei mäleta, kuidas need mõtted 
liikusid või kes mingisuguseid ideid välja käis. Või kui palju me kuskil 
Fidoneti \emph{echo}-des neid asju arutasime või mõtlesime enne, kui me selle 
mõtte välja käisime, et, davai, teeme siis kokkutuleku, et need inimesed
üksteist näha saaksid.

Oli üheksakümne esimese aasta augusti esimene pool, ma ei mäleta täpset 
kuupäeva, mille me olime kuu lõpuks paika pannud, et nii, teeme  siis nädalavahetusel 
kokkutuleku, saame Väänas ühes ürituste kohas kokku. Mingi osavõtumaks oli ka,  
ma ei mäleta, võibolla viiskümmend rubla,  võibolla vähem\sidenote{Siinkohal 
oleks ehk lugejale kasulik selgitada, kui suur või väike raha oli 50 rubla 
1991. aastal. Paraku on see üsna keeruline, sest sel ajal valitses Eestis 
hüperinflatsioon ja hinnad kerkisid kiiresti. Lisaks olid enne 1992. aastat 
teatud kaupade hinnad riikliku kontrolli all ja teiste omad vabad. Kõigele 
lisaks ei olnud paljusid kaupu mis iganes hinna eest saada, puudus valitses muu 
hulgas ka näiteks sularahast ja toimis elav ning väga volatiilsete hindadega 
must turg. 50 rubla eest võis saada 20 kilo kartulit aga võis saada ka ühe 5.25 
tollise flopi.}. Plaan oli, et räägime  ja suhtleme ja mängime   
IT-kalduvustega mänge. Mitte arvutimänge, aga noh, flopiheide ja 
kõvakettaheide ja mingid sellised asjad on olnud nende ürituste kavas. 

BBSumerist kujunes traditisoon ja sellest sai iga-suvine kokkutulek. Mina aitasin seda 
esimest BBSummerit ette valmistada ja läbi viia ja olen hilisemate BBSummerite juures ka mitmel korral õla alla pannud.
Eks seepärast mu nimi BB-seltskonna mällu on jäänud.

\question{1991. aastal kõvaketast ikka andis heita!}

Tol aastal oli kõvakettaheide ilmselt kavas küll aga see ei olnud päris 
tänapäevane kõvaketas, vaid siis olid sellised suured 19 või 21 tolli 
läbimõõduga plaadid, mis moodustasid kõvaketta, aga mis ei olnud kuskil 
hermeetilises korpuses nagu tänapäevased pöörlevad kettad. Neid oli, ma ei tea, 
kaheksa või kümme kokku pandud ühe sellise käepidemega varre külge ja neid sai  
kettaseadme seest välja tõsta ja vahetada\sidenote{Sellised kettapakid, näiteks IBMi 1316, 
suutsid talletada suurusjärgus mõned megabaidid infot ja olid tolleks ajaks 
selgesti iganenud. Eestisse sattusid sedalaadi seadmed tõenäoliselt 
humanitaarabina, mis tõi meie kanti hulganisti kummalist vananenud riistvara. 
Mäletan ühte sellist kettalugejat 1992. aastal Võru I 
Keskkoolis\index{Koolid!Võru Kreutzwaldi Gümnaasium} ka päriselt toimimas. 
Arvutiklass asus teisel korrusel ja kui kettaseade sisse lülitati, oli undamist 
tänavale kosta - selle järgi sai hinnata, kas klassis parajasti oli keegi või 
mitte.}. Sealt lahti lammutatud kettaid me lennutasime küll sellel esimesel 
kokkutulekul. 

Me mõtlesime selle kokkutuleku nimeks välja BBSummer\index{BBSummer}. Ehk siis 
\enquote{BB} lühendist BBS ja siis \enquote{Summer} sinna taha. Üks aasta varem 
oli toimunud esimene Rock Summer\sidenote{Rock Summer oli 1980. aastate lõpus 
ja 1990. aastatel Tallinnas Lauluväljakul toimunud muusikafestival, mille 
mitmekülgset mõju ei saa kuidagi üle hinnata. Tegu oli esimese suurema 
rokifestivaliga siinkandis ja selle platsil valitsenud atmosfäär oli keskmisele 
nõukogude noorele, ütleme, radikaalse mõjuga. Kuna tegu oli ühega esimestest 
võimalustest piiluda raudse eesriide taha, meelitas festival kohale ka küllalt 
nimekaid Lääne ansambleid.}, aga nii palju, kui me oleme erinevate inimestega 
meenutanud, see Rock Summer ei olnud kuidagimoodi selle \enquote{summeri} nimeosa 
eeskuju või põhjustaja, meil olid sõltumatud kaubamärgid. 

Kui see 1991. aasta BBSummer toimus\sidenote{Tarmo on saatnud esimese BBSummeri 
(ametliku nimetusega \enquote{Eesti amatöörarvutivõrgu kasutajate I seminar-laager}) 
kutse 12. juulil 1991 ja üritus ise pidi toimuma 26.-27. augustil Tugamanni Tuulikus
 (ametlikult EPT Tallinna osakonna puhkekompleks).}, 
 siis, kes teab natukene rohkem ajalugu või 
on päris ise selle ajalooga kokku puutunud, siis põhimõtteliselt samal 
nädalal,  mõned päevad enne, kui BBSummer oleks pidanud toimima, oli see aeg, kus 
Moskvas toimus riigipööre ja Tallinnas tulid tankid tänavale. Mina selle peale ütlesin, et  meie teeme 
oma BBSummeri ära igal juhul, kuniks just ei ole mingit 
liikumiskeeldu. Olukord oli üsna pingeline ja keeruline. Esimesel BBSummeril 
oli vist viiskümmend kuus osalejat. Suur 
hulk oli muidugi neid, kes seal olid puhtalt  sellepärast, et nad olid Fidoneti 
\emph{sysop}-id. Aga ma hindan, et kõvasti üle poole ikka oli seda rahvast, kes  
olid BBS-i lihtkasutajad, kes lihtsalt igapäevaselt tõmbasid faile ja 
vahetasid meile, ilma et neil endil oleks oma BBS-e või \emph{node}-sid olnud.

\question{See oli üsna korralik suhe teenuse pakkujate ja tarbijate vahel, 
BBS-i pidamise barjäär oli kõrge ja seltskond seega üsna tehniline?}

See oli jah ikka parajalt tehniline. Kes tundis, et tehnika võib-olla temast 
üle on, ei pidanud tõenäoliselt BBS-i. Ja ei olnud päris triviaalne, et BBS 
häälestada ja korralikult tööle panna ja  sinna ka Fidoneti 
automaatika käima panna. Internetist juhendvideot vaadata ju ka ei saanud. Küll 
aga sai lugeda mingisuguseid tekstifaile  selle kohta, et \enquote{tõmba see 
softi ja tõmbas see soft ja veel see soft ja siis pane nad kõik vat niimoodi kokku ja  
tee sellised ja sellised konfifailid ja siis läheb asi käima}.

\question{Siis sündis ju FAQ, \emph{Frequently Asked Questions}, mis praegu on 
lihtsalt mingi osa veebilehest. Siis oli tegemist konkreetse eraldi leviva 
failiga, kuhu jõudsidki \emph{echo}-des ja uudisgruppides  sagedasti küsitud 
küsimused koos pädevate vastustega.}

Jah,   olid küsimused-vastused, kuidas asi käima panna, ja kui midagi ei tööta 
ja kui on vat sellised sümptomid, siis mida tuleks vaadata ja nii edasi.

\question{Kas sedalaadi sisu Eestis ainult tarbiti või panustati sinna tagasi 
ka?}

Jah, ikka, kui nii-öelda toodeti mingit sisu, ehk siis, kui keegi kirjutas mingit 
programmi, mis ei olnud  päris oma tarbeks mõeldud ja see ei olnud mingi mäng 
vaid see oli näiteks mingi funktsioonide või alamprogrammide teek ehk siis 
\emph{library}. Mäletan Mast\index[ppl]{Kaal, Madis} tol ajal kirjutas 
tekstiliideste tegemiseks ühe funktsioonide teegi, millega sai teha menüüsid ja 
kaste ja selliseid asju ekraani peal. Tekstirežiimis, aga hiir oli  abiks, 
sallega sai menüüdes ringi klikata. Jah, selliste asjade jaoks ikka oli 
mingisugused FAQ-d või mingid lihtsad juhendid olemas igal vähegi mõistlikumal 
autoril. Mängud olid küll sellised, et võtad flopi ja installid, siis 
lased mängu käima ja siis vaatad, et kuidas ta tööle hakkab ja mida mingi nupp 
teeb. Ma arvan, et ega mängude manuaale ilmselt keegi eriti ei lugenud.

\question{Pärast Skriiningut sa jõudsid mingi hetk Uninetti ka?}

Jah, ühel ajal on Uninet  olnud minu tööandja küll.

\question{Kas päris alguses või kunagi hiljem?}

See vist oli mul viies töökoht. Pärast Skriiningut ja üheaastat programmeerimistööd Rootsis sattusin ma Baltic 
Computer Systems-isse\index{Baltic Computer Systems}.
BCS-is ma tegelesin  nüüd üsna konkreetselt juba arvutivõrkudega, 
meil oli arvutivõrkude osakond ja me  ühelt poolt tegelesime 
kaabeldusega ja teiselt poolt  ka serverite ja mingil määral ka sellise 
tarkvaraga, mis  oli vaja võrgus käima panna. Näiteks andmebaasid, mis olid 
mõeldud algselt ühes arvutis kasutamiseks, aga mis siis kuskil ettevõtetes oli 
vaja niimoodi käima panna, et nad töötaks võrgus. Seejärel tuli Uninet.

\question{Vahemärkusena, tol ajal enamus andmebaase olid mõeldud käima ühes 
arvutis. See tähendas, et mingisuguseid transaktsioone või midagi ei olnud 
keegi sinna sisse ehitanud}

Nojah, seda võimalust ei olnud tihtipeale olemas, aga oli  viise, kuidas 
sellest mööda hiiliti, et  kui andmebaas arvutis lahti teha, et ta ei oleks 
mitte lukus võrgus kõikide kasutajate jaoks, vaid et seal saaks ikkagi midagi 
teha. 

\question{Mis sa praegu teed?}

Hiljem olen ma teinud igasuguseid erinevaid asju, mis ei ole olnud väga enam  
võrgu tehnilise ülesehitusega seotud, vaid mis on rohkem seotud võrgus 
töötavate rakenduste ja võrguturbega. Praeguses töökohas olen ma uuesti jälle mõnes mõttes 
sattunud tagasi sellise tegevuse juurde, mis on seotud andmesidevõrgu 
baasprotokollidega - IP, TCP ja UDP, DNS. Sel tööl on küll 
endiselt seos  rakendusprogrammidega ja mobiiliäppidega,  
sest suur osa sellest tööst vajab ka teadmist, kuidas need äpid võrgus käituvad - mismoodi 
nende liiklus ja  andmevahetus on võrgus üles ehitatud, kuidas seda 
andmevahetust filtreerida. Kuidas hakkama saada lahendustega, 
mida Google ja teised suured tegijad  välja pakuvad ja mille eesmärgiks 
üldiselt peetakse seda, et kasutajal oleks Internetis mugavam toimetada  ja turvalisem olla, aga mis samas võivad kaasa tuua
teatud negatiivseid nähte - näiteks võrguliikluse tarbetu kasvu. Minu töö sisuks on 
aidata neid negatiivseid nähte teatud kasutuskohtades kõrvaldada.

\question{Siis on ju selles mõttes toredasti, et kui sa alguses rääkisid huvist 
mängus tegelasele kapoti all toimetades teist värvi müts pähe panna, siis 
praegu lihtsalt see tegelane teistsuguse arvuti sees ja kapotialune on natuke 
keerulisem aga ülesanne on suuresti sama}

Täpselt nii. Minu jaoks on oluline see, mis on karul kõhus. Kuidasmoodi see 
seal kõhus töötab. Kui ta ei tööta hästi, kas ja mida  saab paremaks teha. Ja 
noh, kui ta töötab hästi, siis kas sellest hoolimata  saab midagi 
teistmoodi teha.

\question{Ja kui hästi töötab, on ju huvitav, et kuidas?}

Jah, kuidas töötab ja kas piisavalt hästi töötab.
