%!TEX TS-program = arara
% arara: myindex

\textbf{\enquote{Kuidas sa arvutite juurde jõudsid?}}

Sündisin 1975. aastal Võrus\index{Võru}. Millestki midagi aru saama hakkasin 
mälu järgi kaheksakümnendate teisel poolel. See oli mitmes mõttes üsna kole 
aeg. Noorukile kõige arusaadavam neist koledustest oli lihtlabane praktiline 
puudus. Päris nälga ei olnud aga midagi vähegi leivast ja piimast edevamat 
saada ei olnud. Kui linnakeses levis kuuldus, et olla toodud kast jäätist, oli 
poes veerand tunniga saba ning poole tunni pärast kõik otsas. Muu hulgas oli 
kaubandusvõrgus saada kahte tüüpi meeste talvejopesid. Mitte kahtekümmet ja 
mitte kahtesadat vaid kahte. Ühed olid hallid ja neid said lihtsurelikud 
osta\sidenote{Huvitaval kombel oli tolle jope põuetasku 5.25" lai, sinna mahtus 
üks flopi täpselt sisse} ja teised olid punase suure a-tähega ja neid said osta 
ainult inimesed, kes teadsid kedagi, kes teadis kedagi. Ajad olid sellised. 
Kõige hämmastavamal moel käisid ka seda viletsust inimesed Pihkvast bussidega 
uudistamas ja viimastki kaupa ära ostmas. 

Aga kogu selle halluse keskel suutis Nõukogude Liit meie Võru Kreutzwaldi 
Gümnaasiumile\index{Koolid!Võru Kreutzwaldi Gümnaasium} tarnida 
arvutiklassitäie arvuteid Agat\index{Arvutid!Agat}\sidenote{Agat oli 
Nõukogudemaal valmistatud arvuti, mis oli küll Apple II\index{Arvutid!Apple 
II}'st inspireeritud, kuid siiski mitte täpne kloon}. Kust nad tulid ja kes 
seda asja ajas, ei tea. Küll aga mäletan, et nende saabumine oli pikalt oodatud 
ja edasi lükatud. Miks ja mida oodatud sai, ei oska öelda. Tean ainult seda, et 
kui klass tekkis, läksin ma sinna sisse ja enam välja ei tulnud. 

Ega tolle purgiga palju teha ei olnud. Olid mõned mängud ja programmeerimiseks 
BASIC. Tolles meid programmeerima õpetatigi. Esimese hooga ei õpetatud 
seejuures mitte kõiki käske, näiteks for-tsükkel oli tükk aega saladus. Kui aga 
nohikud aru said, et nende eest tarkust varjatakse, kadus igasugune respekt ja 
läks lahti suuremaks isepusimiseks. Kõik muutus, kui kooli saabus noor, minu 
meelest värskelt ülikoolist tulnud, arvutiõpetaja Aivar 
Halapuu\index[ppl]{Halapuu, Aivar}. Temaga tekkis kohe mingisugune 
pool-kamraadlik side, mis siiski alati suurt kogust meiepoolset lugupidamist 
sisaldas. Tolleks ajaks oli meil tekkinud väiksem seltskond poisse, kes seal 
klassis toimetas ja kes kohe end \emph{in corpore} Aivarile sappa haakis. Aivar 
viitsis meiega tegeleda ja, kuigi ta meile suurt midagi arvutite mõttes ei 
õpetanud, sai tema käest midagi, mida vist kultuuriks nimetatakse. Meiega 
üritati bridži mängida, räägiti mänguteooriast ja nii. 

Kuna me seal klassis sisuliselt elasime, siis usaldati meie kätte üsna pea ka 
arvutiklassi võti. Aga \emph{kooli} võtit meie kätte keegi ei andnud. Seetõttu 
oli oluline hoida järjepidevust: keegi oli alati klassis olemas ja hõikamise 
või kivikese viske peale lasi tulija sisse. Mingitel tingimustel oli meie käes 
siiski ka välisukse võti aga tihti roniti ka aknast. 

Ühel hetkel avanesid kraanid ja saabus humanitaarabi. Võrus oli vist seoses 
rahvamuusikaga igasugu põnevaid suhteid välismaa asutustega, kes hakkasid meie 
suunas igasugu põnevat kola saatma. Saabus klassitäis mingeid rootsikeelsete 
paberite ja tarkvaraga masinaid, millega me mitte midagi teha ei osanud. Mis 
neist sai, ei tea. Aga tuli ka mingi iidne aparaat, mille külge käis neli-viis 
terminali ja kaks kokku külmkapisuurust kettaseadet. Seadmete sisse käisid 
hiigelsuured plastkarbis kettad. Tegu oli industriaalseadmega: kui tuurid sisse 
võttis, siis oli alla tänavale kuulda, et \enquote{arvuti töötab}. Tolle masina 
peal midagi tarka teha ei osanud keegi, tarkvara polnud. Sai mingeid mänge 
mängitud ja see oli ka kõik. Mäletan siiski, et seal puutusin esimest korda 
kokku Zorki\index{Mängud!Zork} nimelise mänguga\sidenote{Zork on üks varasemaid 
tekstipõhiseid arvutimänge. Mängija sisestas teksti ja talle ka vastati tekstiga 
vastavalt sellele, mis mängus parasjagu juhtus. Kuna mängu alguses sattuti 
lagendikule valge maja ette, oli meie puhul ilmselt tegemist Zork I-ga}.

Lõpuks tulid meile Jukud\index{Arvutid!Juku} ja üheksakümnendatel lõpuks ka 
PC-d. Jukusid oodati väga, sest Agat oli päris jube aparaat\sidenote{Ma ei ole 
kunagi hiljem kohanud arvutit, mis suudab flopiketta füüsiliselt ära rikkuda}. 
Ja Jukud olidki väga ägedad, ainsaks nõrgaks kohaks oli minu mälu järgi 
klaviatuur. Ainus, mis palju ei muutunud, oli tarkvara. Võru ei ole Tartu ega 
Tallinn. Meie seltskond ei suhelnud õieti kellegagi, ei uut tarkvara ega 
teadmist ei tulud eriti kuskilt peale. Ajakirjast \enquote{Arvutustehnika \& 
Andmetöötlus} võis küll lugeda Unicode võludest aga programmeerida tuli ikkagi 
kas assembleris või BASICus. Seejuures sain alles hiljem teada, et eksisteeris 
ka asi nimega makro-assembler. Tavalises pidi JMP käsule argumendiks andma 
suhtelise aadressi (mis muidugi kohe valeks osutus, kui kuskile mingi rea 
vahele panid)\sidenote{See oli probleemi minusugustele surelikele. Inimesed 
nagu klassivend Vallo Trell\index[ppl]{Trell, Vallo} suutsid ka otse BIOSi 
prompti peal mällu baite kirjutades masinkoodis programmeerida} aga tolles 
uuemas sai silte kasutada. Mingitel üritustel sai Tallinnas käidud (mäletan 
Pedas\index{Pedagoogikaülikool} asunud MSXide\index{Arvutid!Yamaha MSX} klassi) 
ja sealt ka mingit tarkvara kaasa toodud aga üldiselt olime üsna omaette. Isegi 
flopisid käisime ostmas Tallinnas, seal oli teada üks komisjonipood, kust 
selliseid sai. Tavaliselt kasutati ära mõnda käiku teatrisse, reeglina jäi 
kuhugi paar tundi linnas kolamise aega. 

Olin ka üks õnnelikest, kellele lõpuks arvuti suveks koju usaldati. Esmalt 
Agat, siis Juku. Kuna ekraanid olid mõlemal nigelad, veetsin kaks või kolm suve 
ette tõmmatud kardinate taga arvutiga toimetades. Juku peal mäletan kahte 
suuremat projekti. Esimene oli Norton Commanderi moodi failihaldur ja teine 
fondiredaktor. Jukul sai tähekujusid suhteliselt lihtsasti ümber teha, mälus 
olid vist kaheksabaidised bitimaatriksid ning teksti kuvamine käis kiiremini 
kui muu graafika. Mõlemat kirjutasin assembleris ja kumbki päris valmis ei 
saanudki, sest teatud mahust alates muutus kood hoomamatuks. Sel ajal omandasin 
ka pärast palju vaeva põhjustanud kombe \enquote{tunde järgi} koodi kirjutada. 
Teed muutuse, kompileerid, proovid, muudad pikalt mõtlemata uuesti. Kood oli 
nii kole, et selle iga kord uuesti läbi mõtlemine oli liiga keeruline ja mingid 
\emph{off-by-one} vead olid sagedased, reeglina sai mingi konstandi ühe võrra 
nihutamise peale koodi käima. Sellest rumalast kombest pole ma siiani lõpuni 
vabanenud. 

Aga Juku peal sai ka andmebaase teha, täitsa oli olemas dBASE\index{dBASE}. 
Selle abil õnnestus maik suhu saada kellelegi arvuti abil kasulik olemisest. 
Koolivend Aini dieedi-teemalise uurimistöö jaoks tegin andmestiku ja kirjutasin 
ka programmi kassetiümbriste trükkimiseks. Tollal käibis muusika kassettidel, 
mida ohtralt kopeeriti\sidenote{Eksisteeris ka tänapäeval mõeldamatu täiesti 
põrandapealne muusika kopeerimise asutus, selline oli ka Tartus. Läksid kohale, 
valisid kataloogist albumi välja, jätsid tühja kasseti maha ja mõni päev hiljem 
sai sobiva summa vastu muusikaga kasseti tagasi}. Seetõttu kirjutati lugude 
nimesid käsitsi ning see oli tüütu. Minu tarkvara võimaldas aga kiiresti eri 
plaatide jaoks kassetiümbrised trükkida. Selle teenuse eest sai vist ühelt 
klassivennalt isegi raha küsitud.

Linna peal eri kohtades sai ka PCdega tutvust teha. Mööblivabrikus oli kellelgi 
tutvusi, seal toimus isegi mõned korrad mingisugune õpe. Istusime ilmselt 
raamatupidamise masinate taga ja meile näidati, kuidas FoxPros\index{FoxPro} 
vorme joonistada ja andmeid hoida. 

Keskkoolis õnnestus käia väga murdelistel aastatel 1990-1993. Võrus möllas 
punkar Saare Ain\index[ppl]{Saar, Ain}\sidenote{Kodanikunimega Ain Saar, asutas 
Vaba Sõltumatu Noortekolonni number 1 ja tegi muid tükke}, Võru surnuaial 
taastati Vabadussõja mälestussammas ja miilits ajas koertega üritusi laiali. 
Ühe sellise intsidendi järel oli koolis näha kummalistes ülikondades 
seltsimehi, kes pingsalt vanemate klasside õpilaste nägusid jälgisid ilmses 
lootuses tuttavaid kohata. Aga tekkis ka äri. Leidsime sõpradega mingist 
ajalehest kuulutuse, milles otsiti meie jaoks ulmeliste palkadega (mahus umbes 
meie vanemate aasta palk paarinädalase projekti eest) meelitades C 
programmeerijaid. Kandideerimise tähtaeg oli suurusjärgus kaks nädalat, see 
tundus täiesti mõistlik aeg, millega omale C selgeks teha. Kuskilt sai hangitud 
klassikaline Brian Kernighan ja Dennis Ritchie \enquote{The C Programming 
Language}\index{The C Programming Language}. Seda sai siis kampas tudeeritud ja tundus 
sihuke loogiline. Kuna puudus juurdepääs C kompilaatorile, siis päris koodi 
kirjutada ei saanud. See meid ei heidutanud ja mingid kirjad me isegi välja 
saatsime. Vastust muidugi ei tulnud. Hiljem olen mõelnud, kas võis tegu olla 
tollesama legendaarse lehekuulutusega, mis viis kokku Bluemooni\index{Bluemoon} 
poisid ja Stefan Obergi\index[ppl]{Oberg, Stefan} aga ajastus vist ei klapi. 

Siiski saavad kõik head asjad otsa, nii ka keskkool. Tol hetkel sai mingites 
piirides omale lõpueksamit valida ning oleks olnud kummaline, kui meie 
seltskond ei oleks valinud arvutieksamit. Tolleks hetkeks olime Aivarist kaugel 
ees, sest meil sõna tõsises mõttes ei olnud mitte midagi muud teha kui arvutit 
torkida. Laulsin küll ka kooris\sidenote{Kooriga välisreisile (kas Saksamaale 
või Soome) minek oli ka põhjuseks, miks ma ei ole kunagi vabariiklikul 
informaatikaolümpiaadil käinud. Tol ühel kevadel, kui sinna õnnestus välja 
murda, oli ka reis plaanis. Otsustavaks sai, et ma ei tahtnud koori hätta 
jätta. Mitte, et ma seal mingit kandvat rolli oleksin mänginud, aga siiski.} 
aga põhimõtteliselt kogu muu vaba aeg oli arvutite päralt. Isegi õppetöö ei 
seganud, sest põhikoolis tegin endale kõva põhja alla. Aga see kõik ei 
vähendanud sugugi eksami pidulikkust. Sisenesime ruumi, võtsime pileti, 
lahendasime, vastasime komisjonile, kõik oli nii nagu peab. Aivar oleks võinud 
meile kõigile viied välja kirjutada aga ometi viidi eksam täie tõsidusega läbi. 

Kuna õnnestus kool nibin-nabin kullaga lõpetada, sain Tartu Ülikooli 
Matemaatikateaduskonda\index{Tartu Ülikool!Matemaatikateaduskond} eksamiteta 
sisse. Sinna minek tundus loogiline, sest Tallinn oli kaugel ja tundmata ning 
arvuti-värki tahtsin kindlasti õppida. Sõjaväega probleeme ei olnud. Esiteks 
olid segased ajad ning Eesti riik polnud veel päriselt välja mõelnud, mis moodi 
oleks mõistlik väeteenistust korraldada. Teiseks oli mu silmanägemine nii paha, 
et mulle öeldi Kaitseväe tohtrite poolt: \enquote{Kui venelane peale tuleb, 
siis paneme su laipu vedama, seniks mine koju}. Nii veetsingi suve Võru ja 
Tartu vahel hääletades, käisin näiteks ka Steni\index[ppl]{Tamkivi, Sten} 
juures\sidenote{Tema ema ja minu tädi olid juba ülikooli aegsed sõbrannad, 
Steni vanaisa elas Võrus ja nii me juba üsna õrnas eas tuttavaks saimegi.} 
Primexis\index{Primex Data} külas. 

Sügisest algas ülikool ja jäin pidevalt Tartusse. Kuna jäin paberite ajamisega 
töllerdama, siis teiste matemaatikutega Tiigi ühikasse kohta saada ei 
õnnestunud. Ühe või kaks talve olin sugulase juures üüriliseks, ühe talve 
elasime kambaga Tartu Kurtide Ühingus (!)\index{Tartu Kurtide Ühing}, kes 
tudengitele tuba välja üüris. Küll aga sai külas käidud klassivendadel, kes 
läksid enamuses Tartusse majandust õppima, ja kelle ühikaks olid Narva Maantee 
Tornid. Toona Tartu ühikates toimunu on omaette lugu, millesse süvenemine viiks 
meid teemast kõrvale.

Ülikoolis sain kohe piltlikult öeldes ägeda laksu silmade vahele. Esmalt 
selgus, et, erinevalt keskkoolist, on ülikoolis vaja päriselt õppida. Aga oskus 
selleks oli juba kadunud ja tuli uuesti tekitada. Teiseks selgus, et puhtast 
ropust tööst enam heade hinnete saamiseks ei piisanud, vaja oli ka annet. Aga 
seda on mul kogu aeg nappinud. Kolmandaks selgus, et teistel seda annet jagus 
ning see tegi egole haiget. Inimesed nagu Meelis Roos\index[ppl]{Roos, Meelis} 
ja Rene Prillop\index[ppl]{Prillop, Rene} seilasid igasugu matemaatikast läbi 
ilma nähtava pingutuseta ja kirjutasid koodi nagu jumalad. Margus 
Sutt\index[ppl]{Sutt, Margus} teadis arvutitest nähtavasti kõike ja oli tolleks 
ajaks juba tegelenud täiesti müstilisena tunduvate asjadega. Asko 
Seeba\index[ppl]{Seeba, Asko}, oli kõike seda \emph{ja} oli seejuures veel 
seltskondlik ning tüdrukute hulgas popp. Ei jäänud midagi üle, tuli tasapisi 
inimeseks õppima hakata. 

Igatahes oli vaja tööle minna, sest ema käest ei saanud ju jääda raha küsima. 
Proovisin saada baarmaniks, vast avatud Atlantise ööklubi valgustajaks ja isegi 
arvutigraafikuks, aga asjata. Lõpuks sattusin kuidagi ettevõttesse Korel 
IN\index{Korel IN} programmeerijaks, mu esimene tööpäev oli detsembri alguses 
aastal 1993. Mind ja kamraad Veljot\index[ppl]{Hagu, Veljo} võeti palgale 
eesmärgiga luua firmale arvetega majandamiseks vajalik tarkvara. Keeleks oli 
Visual Basic\index{Keeled!BASIC} ja ei läinud palju aega, kui meil mingid asjad 
juba töötasid. \enquote{Programmeerija} kõlab märkimisväärselt glamuursemalt 
kui asi tegelikult välja nägi. Tegime kõike alates kauba tassimisest (kontor 
asus viiendal või kuuendal korrusel, kahekümnetolline CRT-monitor on päris 
raske), kuni isegi mõningase müügitööni. Toonasele arvutiärile iseloomulikult 
ei teadnud eales, mis seisus su töökoht kontorisse jõudes oli. Mõnikord oli ära 
müüdud mälu, mõnikord võrgukaart või monitor. Mäletan end kirjutamas koodi 
üheksatollise must-valge kassamonitori ees taburetil istudes. 

Tartu ei ole suur linn ja nii puutusime Korelis töötades kokku suure osaga 
toonasest arvutiseltskonnast. Tarmo Tali\index[ppl]{Tali, Tarmo} oli meil 
müügimeheks ja aeg-ajalt käis tal külas Asko Oja\index[ppl]{Oja, Asko}, keda 
hellitavalt \enquote{Tarmo blondiiniks} kutsuti. Vahel astus Sorose sajalisi 
tuulutades läbi Marek Tiits\index[ppl]{Tiits, Marek}, kellele mingi ime läbi 
õnnestus isegi üks Suni tööjaam müüa. Kui ütlen, et puutusime, siis tegelikult 
mina ei puutunud eriti kellegagi kokku, olin toona ja olen siiani küllalt 
asotsiaalne. Igasugu toredat rahvast käis poest läbi, enamasti sai lihtsalt 
silmad punnis peas spetsialistide jutte kuulatud ilma nende nimesidki teadmata. 

Kuidagi tekkis Koreli lähedale aktiivne kodanik nimega Tanel 
Urbanik\index[ppl]{Urbanik, Tanel}. Ta pandi meile alguses ülemuseks aga üsna 
varsti vedas ta meid Korelist minema asutades uue ettevõtmise nimega HClub. 
Nimi tuli sellest, et meie tuba Koreli päris-ärimeeste hulgas veidi põlastavalt 
häkkeriklubiks kutsuti. Tanel tahtis tarkvaraäri teha, küllap seetõttu tal 
Koreliga teed lahku läksidki. Meie peamiseks leivanumbriks sai kassasüsteemide 
ehitamine, peamisteks klientideks erinevad tanklad, näiteks Favora omad. 
Kirjutasin muu hulgas ka näiteks Ravimiametile\index{Ravimiamet} nende ühe 
esimestest andmebaasidest. Selguse mõttes olgu üle korratud, et toona mingist 
klient-server arhitektuurist juttu ei olnud. Kõik lahendused hoidsid andmeid 
võrguketta peal Microsoft Accessi\index{Microsoft Access} andmebaasis ja selle 
poole pöördumine käis kliendi juurde paigaldatud \enquote{paksu} kliendi abil. 

Tollele ajale tagasi mõeldes tundub hämmastav, et meie tarkvara töötas. Meid 
olid ainult mõned inimesed, mingist testimisest või versioneerimisest ei 
teadnud keegi midagi. Mäletan, et korra pidin Tartust Võrru tanklasse tagasi 
sõitma, sest värsket versiooni flopi peal kohale viies olin midagi valesti 
teinud. Vähemalt minu kood püsis kindlasti koos peamiselt tati ja teibiga. 
Veljo oli märkimisväärselt pädevam programmeerija aga tarkvaratehnikast polnud 
ilmselt palju aimu temalgi. 

See mind lõpuks HClubist (päris suure tüliga, tuleb tunnistada) ära viiski. Ma 
ei jaksanud enam kõige selle kokku punutud ja päris kliente teenindava tarkvara 
eest vastutada. Põlesin läbi ja kõndisin Tanelit pipramaale saates ära. 
Toonaseid seiku nägin veel aastaid unes ja ärkasin keset ööd. Oma rolli mängis 
ilmselt ka see, et just tol ajal, kui õigesti mäletan, läksid põhja mu 
unistused saada arvutialane haridus. Nimelt oli toona matemaatikateaduskonnas 
esimesed paar aastat kõigile ühised, seejärel tuli valida kas arvutiteaduse, 
statistika või rakendusmatemaatika vahel. Valik käis seejuures õpitulemuste 
alusel. Minu õpitulemused võimaldasid napilt ennast arvutiteadlaseks pidada ja 
nii esitasin vajaliku avalduse ning asusin järgmisest semestrist hoogsalt 
arvutiteaduse aineid kuulama. Neid loeti enamasti Liivi tänava 
õppehoones\index{Tartu Ülikool!Matemaatikateaduskond!Liivi õppehoone}. 
Dekanaat oma teadetetahvliga asus aga Vanemuise õppehoones. Ja kuna ma ka oma 
ut.ee meiliaadressi ei jälginud, läks minust täiesti mööda dekanaadi mõte, et 
peaks ikka veel mingeid pabereid küsima. Kui ma ükskord jaole sain, olid 
arvutiteaduse õppekohad täis ja minust sai statistikaüliõpilane. See oli päris 
valus hoop. Kuigi arvutiteaduse ained olid minu jaoks rasked (mäletan end kolm 
korda kompileerimismeetodite eksamit tegemas), oli mul siiski mingi lootus 
sealtkaudu kuidagi paremaks programmeerijaks saada ning kamraadidele järgi 
jõuda. Toonane ülikooliharidus oli tänasest väga erinev ja asus praktilisest 
elust valgusaastate kaugusel, aga lootus jäi. Statistikast huvitusin ma vähe ja 
ei näinud mingit võimalust sellest oma töises elus kasu saada 
(masinõppe-revolutsioonini jäi veel paarkümmend aastat). Seetõttu tegin 
edaspidi minimaalse, et kuidagi koolist läbi saada ja keskendusin tööl 
käimisele. 

Kogu BBSindus läks minust üsna suure kaarega mööda. Võrus ei olnud kohalikku 
BBSi ja kaugekõne ei tulnud kõne allagi. Sten Primexis küll vist näitas kuhugi 
helistamist, aga tuhka ma aru sain. Korelis oli küll väline modem ja aegajalt 
sai kuhugi sisse helistatud, aga seda väga sporaadiliselt. Peamiseks 
selleteemalise info allikaks oli kursavend Mati Muts\index[ppl]{Muts, Mati}, 
peamiselt sai käidud Lucifer BBSis\index{BBS!Lucifer BBS}. Küll aga oli 
ülikoolil tol ajal juba täiesti korralik internetiühendus ja palju aega kulus 
Vanemuise õppehoones\index{Tartu Ülikool!Vanemuise tänava õppehoone} terminali 
taga FTPd pidi ringi kolades. Mäletan, et tõmbasin kas ftp.funet.fi või 
ftp.sunet.se serverist tükk aega mingi Metallica albumi kaanepilti ja olin väga 
rahul, kui see ka päriselt kohale jõudis. 

Selgelt mäletan ka seda, kuidas ma kohtusin HTMLiga. See oli Liivi 
tänaval\index{Tartu Ülikool!Matemaatikateaduskond!Liivi õppehoone} , seal oli mingi Suni 
klass\sidenote{Need pidid olema Sunid, sest mäletan ruudulist hiirepatja. Mis 
muidugi ei olnud mingi padi. Kuna Sun kasutas toona levinud palli asemel hiire 
liikumise lugemiseks eesrindlikke optilisi sensoreid aga tehnoloogia polnud 
veel kuigi arenenud, pidi sensoritele teadaolevate vahedega ruudustikku 
näitama. Seetõttu töötas hiir ainult spetsiaalse metallist mati peal, kuhu oli 
joonistatud peen ruudustik} ning seal sukeldusin ma veebilehe tegemise 
võrratusse maailma. Pärast pikka pusimist suutsin omale tekitada kodulehe, kus 
asju õiges kohas hoidis tabel! Ega sinna kodulehele midagi kirjutada ei olnud 
aga tabeli ridade ja lahtrite saladuste lahti pusimine oli põnev.

Ja kõik see osutus kasulikuks, sest HClubi järel võttis mu oma juurde tehnikuks 
klassivend Meelis Mäeots\index[ppl]{Mäeots, Meelis}. Ta tegeles tol ajal 
igasugu imelike asjadega, kuid muu hulgas asutas ka internetifirma. See koosnes 
alguses peamiselt minust ja temast. Firma tegeles Unineti\index{Uninet} 
\emph{dial-up} ühenduste edasi müümisega, tegi kodulehekülgi ja pidas isegi 
Infomeistri nimelist interneti infokataloogi. See viimane oli täiesti hämmastav 
äri. Meelis käis ja rääkis mingitele firmadele augu pähe. Mina kirjutasin firma 
andmed kuskil serveris asunud staatilisse (!) HTMLi. Mis kasu sellest kellelegi 
ammu enne otsingumootorite laia levikut tõusta võis, on mulle siiani 
arusaamatu. Ma ka ei mäleta, et seal lehel keegi väga käinud oleks. Ometi sealt 
mingi kopika sai ja ma väga loodan, et tolle tegevuse käigus antud lubadused 
ikka enam-vähem täidetud said. 

Kuna teadsin Steni juba varasemast ja Meelis vist ka puutus temaga kokku, 
lõpetasime ühel hetkel modemitega jantimise ja infokataloogi pidamise ning 
asusime Halo\index{Halo Interactive DDB} nime all kodulehekülgi tegema. Kampa 
võeti ka mõned kunstnikud (näiteks väga andekas Oliver 
Reitalu\index[ppl]{Reitalu, Oliver} ja mitte vähem andekas Alar 
Koort\index[ppl]{Koort, Alar}, keda ilmselt tema rajude elukommete tõttu 
Helbekeseks kutsuti) ja projektijuhiks Priit Sasi\index[ppl]{Sasi, Priit}, keda 
kõik tema joviaalse oleku ja suure habeme tõttu Sasuks kutsusid. Sasu õpetas 
mind briti punki ja Alar kurjemat sorti hiphoppi kuulama ja elu oli päris tore. 
Miskipärast mäletan, et minu käe alt tuli Eesti esimene kommertsalustel tehtud 
(st. ettevõte maksis kellelegi lehe tegemise eest raha) kodulehekülg, see sai 
tehtud Tartu Raadiole\index{Tartu Raadio}, kui mälu ei peta. Kunstnik joonistas 
pildid valmis ja lõikas tükkideks, mina kirjutasin Notepadiga HTMLi ja nii see 
töö käis. 

Mingil hetkel hakkasime lehekülgede tekitamist automatiseerima, kirjutasime 
Perli skripte. Mõnda aega ei olnud meil ei oma serverit ega üldse kuskil Perli 
jooksutada. Siis sai programmeeritud nii, et skript läks e-mailiga 
Unineti\index{Uninet} süsadminile, see kopeeris faili õigesse kohta, meie 
vajutasime brauseris nuppu, saime veateate, admin saatis e-mailiga konsooli 
veateated, mina parandasin koodi ja saatsin uue versiooni. Admini kannatus 
lõppes enne kui minu oma. 

Siiski jõudsime lõpuks päris kaugele oma tegemistega. Perli skriptid läksid 
järjest pikemaks ja, kuna andmebaasi pidamiseks ei olnud meil serverites 
piisavalt õigusi, hoiti andmeid enamasti lihtsalt tekstifailis. Üllataval moel 
kattis see ära päris suure hulga vajadusi. Perlilt liikusime ühel hetkel PHPle 
ja ühel hetkel tekkis ka levinud kui seetõttu mitte vähem rumal mõte endale ise 
oma sisuhaldussüsteem kirjutada. See vist sai isegi valmis aga konkreetsed 
mälestused tollest elukast puuduvad. 

Ma ei mäleta, et see äri kuidagi tänapäevases mõistes äri moodi välja oleks 
näinud. Raha oli alati vähe ja nii tuli teha kõike, mille eest maksti. Kuidagi 
müüs Sten Ühispangale maha mõtte anda nende aastaraamat välja CDl. Mis muud, 
õppisime selgeks Macromedia Director'i kasutamise ja video redigeerimise ja 
andsime minna. Ainus asi, millega hakkama ei saanud, oli heli. Õnneks oli Sten 
hea sõber Lauri Liivakuga\index[ppl]{Liivak, Lauri}, kelle Forwards 
Studio\index{Forwards Studio} asus meiega tol ajal sama koridori peal. Lauri 
tegi kenad kõllid ja plõnnid ja aitas selle kõik visuaaliga ära 
sünkroniseerida. Tulemus sai päris kena. 

Igatahes hakkas meile järjest rohkem Tallinna kliente sigima. Samuti müüs Sten 
suure tüki ettevõttest Brand Sellers DDBle\index{Brand Sellers DDB}. Too oli 
minusugusele Tartu nohikule täiesti müstiline kamp inimesi. Intelligentsed, 
säravad, jõukad (nii mulle tundus) ning andekad. Bruno Lill\index[ppl]{Lill, 
Bruno} oma terava ütlemisega on siiani meeles.  Nii tehti kampas otsus kolida 
Tallinna. Olin tegelikult ligi aasta üsna kahepaikne, pendeldades Tartu ja 
Tallinna vahel. Ülikoolis olid veel viimased sabad lõpetada ja 
Mari\index[ppl]{Kütt, Maria}, kellega toona juba koos elasime, käis samuti veel 
koolis. Lõpuks sai lõputöö kaitstud ja, kuna selliseks triviaalseks asjaks ei 
hakanud ju keegi Tartusse sõitma, käis Mari diplomit dekanaadist ära toomas. 
Prouad nõudsid allkirjastatud volitust, mis ukse taga ka kohe valmis tehtud sai 
ning nii omandasin ma oma esimese teaduskraadi. Tartu Ülikooli peahoone 
sammaste vahelt ei ole ma kunagi välja astunud ja, kuigi toonaseid õppejõude 
hindan siiani kõrgelt, pean oma alma materiks siiski Massachusettsi 
Tehnoloogiainstituuti. 

Tallinnasse kolimisega sai läbi üks etapp Halo kasvu loost. Senise boheemliku 
mis-võib-ikka-valesti-minna mentaliteedi asemel tuli hakata käibenumbritest 
rääkima. Samuti oli meeskond kasvanud. Veel Tartu päevadel olin saanud omale 
oma elu esimese alluva olles samal ajal ka tema esimeseks ülemuseks. Vist veel 
keskkooli lõpetav noor nutikas tüüp aitas mul koodi kirjutada ja hängis niisama 
ringi, ei mina teadnud, kuidas inimesi juhitakse või mida üks ülemus tegema 
peaks. Nimeks oli tüübil Taavet Hinrikus\index[ppl]{Hinrikus, Taavet}. Inimesi 
lisandus veelgi ja ma ei saanud enam aru, miks ja kuidas asju tehakse. Nii 
leidsingi ühel ilusal päeval kuskilt kuulutuse, et Hansapank\index{Hansapank} 
otsib oma internetipanga meeskonda inimesi. Läksin intervjuule. Mäletan siiani 
seda tunnet, kui Liivalaia tänava pangahoone tolle aja kohta ülišiki lifti 
uksed kaheksandal korrusel avanesid ja minu ees avanes hurmav vaade 
vanalinnale. Olin müüdud mees, õnneks arvas Vilve Vene\index[ppl]{Vene, Vilve}, 
kes toona arendust vedas, samuti. Nii sai minust veidi enne sajandivahetust 
Hansapankur. Mul vedas kohutavalt, pank oli praeguses mõistes ulmeliselt 
dünaamiline asutus. Vägesid juhatas Indrek Neivelt\index[ppl]{Neivelt, Indrek}. 
Vaata Maailma programm oli just käima minemas ja sellega tegeles Tiit 
Pekk\index[ppl]{Pekk, Tiit}. Marketsi tiim eesotsas Erkki 
Raasukesega\index[ppl]{Raasuke, Erkki} pidas ülejäänud panka talumatuteks 
venivillemiteks ja tootis Erik Jõgi\index[ppl]{Jõgi, Erik} juhtimisel imeilusat 
koodi. Aga see, nagu öeldakse, on juba üks teine jutt.

Ümber teha. Henn lubas intervjueerida. 

Maria Klenskaja ütles ilusti midagi umbes sellist, et mõni inimene on lavale sündinud
ja mõni teeb kõvasti tööd ja, kui noot, ees, hätta ei jää. Mõni on programmeerija ja 
mõni oskab koodi kirjutada, ma ise olen selgesti see viimane tüüp. Võib olla just 
seepärast ma väga hindan inimesi, kes mitte ei tee vaid on ja mulle väga meeldivad päris asjad.
Et kui kinnisvaraärimees või minu pärast programmeerija
läheb nädalavahetusel lavale bluusi laulma, siis ei ole mõtet etendada bluesman-i. 
Sest tegu on Eesti kinnisvaraärimehe ja mitte  purjakil musta vanamehega eelmisest sajandi algusest. 
Mitte, et kummagagi midagi valesti oleks, aga üks ei ole teine. Johnny Lee Hooker jalgu 
trampimas ja kurjalt mörisemas on ehe kraam, seda selgeks ei õpi.
Sama liini pidi, Villu Tamme ja Freddy Grenzman on päris, nad minu kogemuses ei ole laval 
väga palju teistsugused, kui elus. Lemmy samuti. Ma ei ole suurema osaga Varg Vikernes'i ideoloogiast
absoluutselt nõus, aga see mees pani rõngassärgi selga, kaks pikka nuga vööle, võttis 
ihukaitsjad kaasa ja kõndis politseijaoskonda nõudma, et temasuguste 
\enquote{kiusamine} mõrvade ja põletamiste uurimise näol ära lõpetataks. Päriselt. 