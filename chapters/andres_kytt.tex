%!TEX TS-program = arara
% arara: myindex

Sündisin 1975. aastal Võrus. Millestki midagi aru saama hakkasin 
kaheksakümnendate teisel poolel. See oli mitmes mõttes üsna kole 
aeg. Noorukile kõige arusaadavam neist koledustest oli lihtlabane praktiline 
puudus. Päris nälga ei olnud, aga midagi vähegi leivast ja piimast edevamat 
saada ei olnud. Kui linnakeses levis kuuldus, et olla toodud kast jäätist, oli 
poes veerand tunniga saba ning poole tunni pärast kõik otsas. Muu hulgas oli 
kaubandusvõrgus saada kahte tüüpi meeste talvejopesid. Mitte kahtekümmet või
kahtesadat, vaid kahte. Ühed olid hallid ja neid said osta lihtsurelikud\sidenote{Huvitaval kombel oli jope põuetasku 5,25 tolli lai, sinna mahtus 
üks flopi täpselt sisse.} ning teised olid punase A-tähega ja neid said osta 
ainult inimesed, kes teadsid kedagi, kes teadis kedagi. 
Hämmastaval kombel käisid ka seda viletsust inimesed Pihkvast bussidega 
uudistamas ja viimastki kaupa ära ostmas. 

Kogu selle halluse keskel suutis Nõukogude Liit meie Võru \linebreak[4]\mbox{Kreutzwaldi} 
Gümnaasiumile\index{Võru Kreutzwaldi Gümnaasium} tarnida 
arvutiklassitäie arvuteid Agat\index{Agat}\sidenote{Agat oli 
Nõukogude Liidus valmistatud arvuti, mis oli küll Apple IIst\index{Apple 
II} inspireeritud, kuid siiski mitte täpne kloon.}. Kust need tulid ja kes 
seda asja ajas, ei tea. Küll aga mäletan, et nende saabumine oli pikalt oodatud 
ja edasi lükatud. Miks ja mida täpselt oodatud sai, ei oska öelda. Tean ainult, et 
kui klass tekkis, läksin sinna sisse ja enam välja ei tulnud. 

Ega tolle purgiga palju teha ei olnud. Olid mõned mängud ja programmeerimiseks 
BASIC\index{BASIC}, milles meid programmeerima õpetatigi. Esimese hooga ei õpetatud 
seejuures mitte kõiki käske, näiteks for-tsükkel oli tükk aega saladus. Kui aga 
nohikud said aru, et nende eest tarkust varjatakse, kadus igasugune respekt ja 
läks lahti suuremaks isepusimiseks. Kõik muutus, kui kooli saabus noor, vist
värskelt ülikoolist tulnud arvutiõpetaja Aivar 
Halapuu\index[ppl]{Halapuu, Aivar}. Temaga tekkis kohe 
poolkamraadlik side, mis sisaldas siiski alati suurt kogust meiepoolset lugupidamist. Tolleks ajaks oli meil väiksem seltskond poisse, kes seal 
klassis toimetas ja end kohe \emph{in corpore} Aivarile sappa haakis. Aivar 
viitsis meiega tegeleda ja kuigi ta meile suurt midagi arvutite mõttes ei 
õpetanud, sai tema käest midagi kultuuritaolist. Ta
üritas meiega bridži mängida, rääkis mänguteooriast ja nii edasi. Ega me väga palju aru saanud, kuid targa inimese viitsimine meiega tegelda tekitas soovi
tolle viitsimise vääriline olla.

Kuna me sisuliselt elasime arvutiklassis (peale kooli kohe sinna, õhtul 
hilja koju, nädalavahetustel käisime samuti Aivari käest võtit palumas), siis 
usaldati arvutiklassi võti üsna pea meie kätte. Aga \emph{kooli} võtit meie kätte keegi ei andnud. Seetõttu 
oli oluline hoida järjepidevust: keegi oli alati klassis olemas ja lasi hõikamise 
või kivikese viske peale tulija sisse. Mõnikord oli meie käes 
siiski ka välisukse võti, aga tihti roniti sisse-välja akna kaudu. 

Ühel hetkel avanesid kuskil kraanid ja hakkas saabuma humanitaarabi. Võrul oli vist seoses 
rahvamuusikaga põnevaid suhteid välismaa asutustega, kes hakkasid meile 
igasugust huvitavat kola saatma. Kord saabus klassitäis rootsikeelsete 
paberite ja tarkvaraga masinaid, millega me ei osanud mitte midagi teha. Käima 
nad läksid, rootsikeelseid veateid väljastasid, aga sellega asi piirdus. Millega tegu oli ja mis neist sai, 
ei tea. Aga tuli ka üks iidne aparaat, mille külge käis neli-viis 
terminali ja kaks kokku külmkapisuurust kettaseadet, mille sisse käisid 
hiigelsuured plastkarbis kettad. Tegu oli industriaalseadmega: kui tuurid sisse 
võttis, siis oli alla tänavale kuulda, et \enquote{arvuti töötab}. Tolle masina 
peal ei osanud ka keegi midagi tarka teha, sest tarkvara polnud. Kuna masinasse asjade saamiseks 
olid ainult nimetet hiigelsuured kettad, ei olnud tarkvara ka kuskilt võtta. Mängisime mänge 
ja oligi kõik. Mäletan siiski, et seal puutusin esimest korda 
kokku Zorki\index{Zork}-nimelise mänguga\sidenote{\enquote{Zork} on üks varasemaid 
tekstipõhiseid arvutimänge. Mängija sisestas teksti ja talle ka vastati tekstiga 
vastavalt sellele, mis mängus parasjagu juhtus. Kuna mängu alguses sattuti 
lagendikule valge maja ette, oli meie puhul ilmselt tegemist \enquote{Zork I-ga}.}.

Lõpuks tulid meile Jukud\index{Juku} ja üheksakümnendate algul ka 
PCd. Jukusid oodati väga, sest Agat oli päris jube aparaat.\sidenote{Ma ei ole 
kunagi hiljem kohanud arvutit, mis suudab flopikettasse füüsilised sooned tõmmata.} 
Jukud olid väga ägedad, ainus nõrk koht oli 
klaviatuur. 

Mis aga palju ei muutunud, oli tarkvara. Võru ei ole Tartu ega 
Tallinn. Meie seltskond ei suhelnud õieti kellegagi, nii et uut tarkvara ja
teadmisi ei tulnud eriti kuskilt peale. Ajakirjast \enquote{Arvutustehnika \& 
Andmetöötlus}\index{Arvutustehnika \& Andmetöötlus}\sidenote{\phantomsection\label{sisu:aa}Alates aastast 
1987 Eesti esimese infotehnoloogiaettevõtte Algoritm\index{Algoritm|see{Tallinna 
Teadus-Tootmiskeskus}} (sellest põnevast asutusest loe lähemalt lk \pageref{sisu:algoritm}) 
algatusel ja rahastusel ilmunud esimene regulaarne IT-ajakiri. A\&A ilmus Eesti Teadus- ja Tehnikainformatsiooni ning Majandusuuringute 
Instituudi\index{Eesti Teadus- ja Tehnikainformatsiooni ning Majandusuuringute Instituut} 
(lühendatult Eesti Informatsiooni Instituut\index{Eesti Informatsiooni 
Instituut|see{Eesti Teadus- ja Tehnikainformatsiooni ning Majandusuuringute Instituut}}) infoseeriana.} 
võis küll lugeda Unicode'i võludest, aga programmeerida tuli ikkagi 
assembleris või BASICus. Seejuures sain alles hiljem teada, et eksisteeris 
ka makroassembler. Tavalises assembleris pidi JMP-käsule andma argumendiks 
suhtelise aadressi (mis muidugi osutus kohe valeks, kui kuskile mõne rea 
vahele panid)\sidenote{See oli probleem vaid minusugustele surelikele. Inimesed, 
nagu klassivend Vallo Trell\index[ppl]{Trell, Vallo}, suutsid ka otse BIOSi 
prompti peal mällu baite kirjutades masinkoodis programmeerida.}, aga 
uuemas assembleris sai silte kasutada. Käisime ka mõnel üritusel Tallinnas (mäletan 
Pedas\index{Tallinna Pedagoogikaülikool} asunud MSXide\index{Yamaha MSX} klassi) 
ja tõime sealt ka tarkvara kaasa, aga üldiselt olime üsna omaette. Isegi 
flopisid käisime ostmas Tallinnas ühest komisjonipoest. Tavaliselt kasutasime ära mõnda klassiga organiseeritud käiku teatrisse, kui jäi 
ka paar tundi linnas kolamise aega. 

Olin ka üks õnnelikest, kellele lõpuks arvuti suveks koju usaldati -- muidu pidime 
suvekuud veetma arvutiklassi akna all kurvalt kiibitsedes. Esmalt lubati
Agat, siis Juku. Kuna ekraanid olid mõlemal nigelad, veetsin kaks-kolm suve 
ettetõmmatud kardinate taga arvutiga toimetades. Juku peal mäletan oma tegemistest kahte 
suuremat projekti. Esimene oli Norton Commanderi moodi failihaldur ja teine 
fondiredaktor. Jukul sai tähekujusid suhteliselt lihtsasti ümber teha, mälus 
olid vist kaheksabaidised bitimaatriksid ning teksti kuvamine käis kiiremini 
kui muu graafika. Mõlemat kirjutasin assembleris ja kumbki päris valmis ei 
saanudki, sest teatud mahust alates muutus kood hoomamatuks. Sel ajal omandasin 
ka hiljem palju vaeva põhjustanud kombe \enquote{tunde järgi} koodi kirjutada: 
teed muutuse, kompileerid, proovid ja muudad pikalt mõtlemata uuesti, kuni 
asi pigem juhuse kui mõistuse tahtel tööle hakkab. Kood oli
nii kole, et seda oli liiga keeruline iga kord uuesti läbi mõelda. Mingid 
\emph{off-by-one} vead olid sagedased, aga üldjuhul sai mõne konstandi ühe võrra 
nihutamise peale koodi käima. Sellest rumalast kombest pole ma paraku siiani lõpuni 
vabanenud. 

Juku peal sai ka andmebaase teha, dBASE\index{dBASE} oli täitsa olemas. 
Tänu sellele õnnestus maik suhu saada kellelegi arvuti abil kasulik olemisest, kui tegin 
koolivend Aini dieediteemalise uurimistöö jaoks andmestiku ja kirjutasin 
ka programmi kassetiümbriste trükkimiseks. Tollal käibis muusika kassettidel, 
mida ohtralt kopeeriti\sidenote{Eksisteeris ka tänapäeval mõeldamatu täiesti 
põrandapealne muusika kopeerimise asutus, näiteks Tartus. Läksid kohale, 
valisid kataloogist albumi välja, jätsid tühja kasseti maha ja mõni päev hiljem 
said sobiva summa vastu muusikaga kasseti tagasi.\phantomsection\label{sisu!kassetid}}. 
Seetõttu kirjutati lugude 
nimesid käsitsi ning see oli tüütu. Minu tarkvara võimaldas aga kiiresti 
kassetiümbriseid trükkida. Selle teenuse eest sai vist ühelt 
klassivennalt isegi raha küsitud.

Linna peal tegin erinevates kohtades ka PCdega tutvust. Kellelgi oli mööblivabrikus 
tutvusi ja seal toimus isegi mõned korrad mingisugune õpe. Istusime ilmselt 
raamatupidamise masinate taga ja meile näidati, kuidas FoxPros\index{FoxPro} 
vorme joonistada ja andmeid hoida. 

Keskkoolis õnnestus käia väga murdelistel aastatel 1990--1993. Võrus möllas 
punkar Saare Ain\index[ppl]{Saar, Ain}\sidenote{Kodanikunimega Ain Saar, asutas 
Vaba Sõltumatu Noortekolonni number 1 ja tegi muid tükke.}, Võru surnuaial 
taastati Vabadussõja mälestussammas ja miilits ajas koertega üritusi laiali. 
Ühe sellise intsidendi järel oli koolis näha kummalistes ülikondades 
seltsimehi, kes pingsalt vanemate klasside õpilaste nägusid jälgisid ilmses 
lootuses tuttavaid kohata. 

Tekkis ka äri. Leidsime sõpradega 
ajalehest kuulutuse, milles otsiti meie jaoks ulmeliste palkadega (umbes 
vanemate aastapalk paarinädalase projekti eest) meelitades C 
programmeerijaid. Kandideerimise tähtaeg oli kaks nädalat ja see 
tundus täiesti mõistlik aeg, millega omale C selgeks teha. Kuskilt sai hangitud 
klassikaline Brian Kernighani ja Dennis Ritchie \enquote{The C Programming 
Language}\index{The C Programming Language}, mida kambaga tudeerisime ja mis tundus 
loogiline. Kuna meil puudus juurdepääs C kompilaatorile, siis päris koodi 
kirjutada ei saanud. See meid ei heidutanud ja saatsime isegi mingid kirjad välja. Vastust muidugi ei tulnud. Hiljem olen mõelnud, kas tegu võis olla 
tollesama legendaarse lehekuulutusega, mis viis kokku Bluemooni\index{Bluemoon} 
poisid ja Stefan Obergi\index[ppl]{Oberg, Stefan}, aga ajastus siiski vist ei klapi. 

Kõik head asjad saavad kord otsa, nii ka keskkool. Tollal sai lõpueksamit valida ning oleks olnud kummaline, kui meie 
seltskond ei oleks valinud arvutieksamit. Aivarist olime arvutiteadmiste poolest juba kaugel 
ees, sest meil ei olnud sõna tõsises mõttes mitte midagi muud teha kui arvutit 
torkida. Laulsin küll ka kooris\sidenote{Kooriga välisreisile minek oli ka põhjus, miks ma ei ole kunagi vabariiklikul 
informaatikaolümpiaadil käinud. Tol ühel kevadel, kui sinna õnnestus välja 
murda, oli ka reis plaanis. Otsustavaks sai, et ma ei tahtnud koori hätta 
jätta, mitte et oleksin seal kandvat rolli mänginud.}, 
aga põhimõtteliselt kogu muu vaba aeg oli arvutite päralt. Isegi õppetöö ei 
seganud, sest põhikoolis tegin endale kõva põhja alla. Kõik see ei 
vähendanud sugugi eksami pidulikkust. Sisenesime ruumi, võtsime pileti, 
lahendasime, vastasime komisjonile -- kõik oli nii, nagu peab. Aivar oleks võinud 
meile kõigile pikalt mõtlemata viied välja kirjutada, aga ometi viidi eksam täie tõsidusega läbi. 
See tundub siiani oluline.

Kuna mul õnnestus kool nibin-nabin kullaga lõpetada, sain Tartu Ülikooli 
matemaatikateaduskonda\index{Tartu Ülikool!Matemaatikateaduskond} eksamiteta 
sisse. Sinna minek tundus loogiline, sest Tallinn oli kaugel ja tundmata ning 
arvutivärki tahtsin kindlasti õppida. Sõjaväega probleeme ei olnud. Esiteks 
olid segased ajad ning Eesti riik polnud veel päriselt välja mõelnud, mismoodi 
oleks mõistlik väeteenistust korraldada.\sidenote{Hiljem on selgunud, et ülikooli 
minek vabastas väeteenistusest ning meie aastakäik hakkas ülikooli lõpetama just 
siis, kui otsustati siiski enne ülikooli väeteenistuse läbimise kasuks.} 
Teiseks oli mu silmanägemine nii paha, 
et kaitseväe tohtrid ütlesid mulle: \enquote{Kui venelane peale tuleb, 
siis paneme su laipu vedama, seniks mine koju.} Nii veetsingi suve Võru ja 
Tartu vahel hääletades, käisin näiteks ka Steni\index[ppl]{Tamkivi, Sten} 
juures\sidenote{Meie suguseltsid sõbrustasid, 
Steni vanaisa elas Võrus ja saime juba üsna õrnas eas tuttavaks.} 
Primexis\index{Primex Data} külas. Kohtasin seal elus esimest korda Photoshopi-nimelist tarkvara, laserprinterit ning morni näoga, kuid
huvitavat tüüpi, kes osutus Tarmo Taliks\index[ppl]{Tali, Tarmo}. Temaga puutusime
hiljem veel korduvalt kokku. Tarmo on üks neid inimesi, kelle puhul olen veendunud,
et olen temalt kohutavalt palju õppinud, suutmata siiski midagi konkreetset sõnastada. 
Olen tänulik. 

Sügisest algas ülikool ja asusin püsivamalt Tartusse. Kuna jäin paberite ajamisega 
töllerdama, siis ei õnnestunud koos teiste matemaatikutega Tiigi ühikasse kohta saada. Ühe või kaks talve olin sugulase juures üüriliseks, ühe talve 
elasime kambaga Tartu Kurtide Ühingus\index{Tartu Kurtide Ühing}, mis üüris
tudengitele tuba välja. Küll aga sai külas käidud klassivendadel, kellest enamik 
läksid majandust õppima ja kelle ühikaks olid Narva maantee 
tornid. Nii õnnestus ühikaelust maik suhu saada selles siiski kõrvuni osalemata. 
Sellest mul ülemäära kahju pole, sest õlu mulle ei maitse. Tiigi ühikas 
tegid kaastudengid kord koridori lõkke ning Narva maantee tornides pudenes regulaarselt 
keegi rõdult alla. Minu jaoks ei ületanud kahtlemata elava seltsielu paleus 
kommunaalhorrori veidi ligast reaalsust. 

Ülikoolis sain piltlikult öeldes kohe ägeda laksu otse ego pihta. Esmalt 
selgus, et erinevalt keskkoolist oli ülikoolis vaja päriselt õppida, aga vastav oskus 
oli juba kadunud (keskkool möödus arvutite seltsis ja põhikooli seljas 
liugu lastes) ning tuli uuesti tekitada. Teiseks selgus, et 
ropust tööst enam heade hinnete saamiseks ei piisanud, vaja oli ka annet, aga 
seda on mul kogu aeg nappinud. Teistel seevastu annet jagus 
ning see tegi egole haiget. Näiteks Meelis Roos\index[ppl]{Roos, Meelis} 
ja Rene Prillop\index[ppl]{Prillop, Rene} seilasid igasugusest matemaatikast läbi 
ilma nähtava pingutuseta ja kirjutasid koodi nagu jumalad. Margus 
Sutt\index[ppl]{Sutt, Margus} teadis arvutitest nähtavasti kõike ja oli tolleks 
ajaks juba tegelenud täiesti müstilisena tunduvate asjadega. Asko 
Seeba\index[ppl]{Seeba, Asko} oli kõike seda \emph{ja} seejuures veel 
seltskondlik, mängis kitarri ning oli tüdrukute hulgas popp. Ei jäänud 
midagi üle, tuli tasapisi hakata inimeseks õppima. 

Lisaks inimeseks saamisele oli vaja saada tööinimeseks, sest ema käest ei 
saanud ju jäädagi raha küsima. Proovisin saada baarmeniks, vast avatud Atlantise ööklubi valgustajaks ja isegi 
arvutigraafikuks, aga asjata. Lõpuks sattusin ettevõttesse Korel 
IN\index{Korel IN} programmeerijaks, esimene tööpäev oli 1993. aasta detsembri alguses. Mind ja kamraad Veljot\index[ppl]{Hagu, Veljo} võeti palgale 
eesmärgiga luua firmale arvetega majandamiseks vajalik tarkvara. Keeleks oli 
Visual Basic\index{BASIC!Visual Basic} ja ei läinud palju aega, kui meil mõned asjad 
juba töötasid. \enquote{Programmeerija} kõlab märkimisväärselt glamuursemalt, 
kui asi tegelikult välja nägi. Tegime kõike alates kauba tassimisest (kontor 
asus viiendal või kuuendal korrusel ja kahekümnetolline CRT monitor on päris 
raske) kuni isegi mõningase müügitööni. Toonasele arvutiärile iseloomulikult 
ei teadnud eales, mis seisus töökoht kontorisse jõudes oli. Mõnikord oli ära 
müüdud mälu, mõnikord võrgukaart või monitor. Mäletan end kirjutamas koodi 
üheksatollise must-valge kassamonitori ees taburetil istudes\phantomsection\label{sisu:jupimyyk}. 

Tartu ei ole suur linn ja nii puutusime Korelis töötades kokku suure osaga 
toonasest arvutiseltskonnast. Tarmo Tali\index[ppl]{Tali, Tarmo} oli meil 
müügimeheks ja aeg-ajalt käis tal külas Asko Oja\index[ppl]{Oja, Asko}, keda kutsuti
hellitavalt \enquote{Tarmo blondiiniks}. Vahel astus Sorose sajalisi 
luhvtitades läbi Marek Tiits\index[ppl]{Tiits, Marek}, kellele õnnestus mingi ime läbi 
isegi üks Suni tööjaam müüa. Kui ütlen, et puutusime, siis tegelikult 
mina ei puutunud eriti kellegagi kokku, sest olin toona ja olen siiani küllaltki 
asotsiaalne. Igasugust toredat rahvast käis poest läbi ja enamasti kuulasin lihtsalt, 
silmad punnis peas, spetsialistide jutte ilma nende nimesidki teadmata. 

Kuidagi tekkis Korelisse aktiivne kodanik nimega Tanel 
Urbanik\index[ppl]{Urbanik, Tanel}. Ta pandi meile alguses ülemuseks, aga üsna 
varsti vedas ta meid Korelist minema, asutades uue ettevõtmise nimega HClub. 
Nimi tuli sellest, et meie tuba kutsuti Koreli päris ärimeeste hulgas veidi põlastavalt 
häkkeriklubiks. Tanel tahtis tarkvaraäri teha, küllap seetõttu tal 
Koreliga teed lahku läksidki. Meie peamiseks leivanumbriks sai kassasüsteemide 
ehitamine ja põhiklientideks erinevad tanklad, näiteks Favora omad. 
Kirjutasin muu hulgas ka Ravimiametile\index{Ravimiamet} nende ühe 
esimestest andmebaasidest. Selguse mõttes olgu öeldud, et toona mingist 
klient-server arhitektuurist juttu ei olnud. Kõik lahendused hoidsid andmeid 
võrguketta peal Microsoft Accessi\index{Microsoft Access} andmebaasis ja selle 
poole pöördumine käis kliendi juurde paigaldatud \enquote{paksu} kliendi abil. 

Tollele ajale tagasi mõeldes tundub hämmastav, et meie tarkvara töötas. Meid 
olid vähe ja testimisest või versioneerimisest ei 
teadnud keegi midagi. Kord pidin Tartust Võrru tanklasse tagasi 
sõitma, sest värsket versiooni flopi peal kohale viies olin midagi valesti 
teinud ja kriitiline toiming läks hilisõhtul katki. Vähemalt minu kood püsis 
kindlasti koos peamiselt nätsu ja teibiga. Veljo\index[ppl]{Hagu, Veljo} oli märkimisväärselt pädevam programmeerija, aga tarkvaratehnikast polnud 
ilmselt palju aimu temalgi. 

See kõik mind lõpuks HClubist ära viiski (päris suure tüliga, tuleb tunnistada). Ma 
ei jaksanud enam selle kokkupunutud ja päris kliente teenindava tarkvara 
eest vastutada. Põlesin läbi ja kõndisin Tanelit valjusti (ja mõneti teenimatult) needes minema. 
Mõnega toonastest seikadest kohtusin veel aastaid halbades unenägudes. Oma rolli mängis 
ilmselt ka see, et just tol ajal läks põhja mu 
unistus saada arvutialane haridus. Nimelt olid matemaatikateaduskonnas 
esimesed paar aastat kõigile ühised, seejärel tuli valida arvutiteaduse, 
statistika või rakendusmatemaatika suundade vahel. Valik käis seejuures õpitulemuste 
alusel. Minu tulemused võimaldasid napilt ennast tulevaseks arvutiteadlaseks pidada ja 
nii esitasin vajaliku avalduse ning asusin järgmisest semestrist hoogsalt 
arvutiteaduse aineid kuulama. Neid loeti enamasti Liivi tänava 
õppehoones\index{Tartu Ülikool!Liivi õppehoone}. 
Dekanaat oma teadetetahvliga asus aga Vanemuise õppehoones\index{Tartu Ülikool!Vanemuise 
tänava õppehoone}. Ja kuna ma ka oma 
ut.ee meiliaadressi ei jälginud, läks minust täiesti mööda dekanaadi mõte, et 
peaks tudengite käest nende suunavaliku kohta veel mingeid pabereid küsima. 
Kui ma ükskord jaole sain, olid 
arvutiteaduse õppekohad täis ja minust sai statistikaüliõpilane. 

See oli päris 
valus hoop. Kuigi arvutiteaduse ained olid minu jaoks rasked (mäletan end kolm 
korda kompileerimismeetodite eksamit tegemas), oli mul siiski mingi lootus 
sealtkaudu kuidagi paremaks programmeerijaks saada ning kamraadidele järele 
jõuda. Toonane ülikooliharidus oli tänasest väga erinev ja ei omanud reaalse eluga 
suurt sidet, aga lootus jäi. Statistikast huvitusin ma vähe ja 
ei näinud mingit võimalust sellest oma töises elus kasu saada. Masinõppe revolutsioonini 
jäi veel paarkümmend aastat. Seetõttu tegin 
edaspidi minimaalse, et kuidagi koolist läbi saada, ja keskendusin tööle. 

Kogu BBSindus läks minust üsna suure kaarega mööda. Võrus ei olnud kohalikku 
BBSi ja kaugekõne ei tulnud ei hinna ega kättesaadavuse mõttes kõne allagi. 
Sten\index[ppl]{Tamkivi, Sten} Primexis\index{Primex Data} küll vist näitas kuhugi 
helistamist, aga tuhka ma aru sain. Korelis oli väline modem ja aeg-ajalt 
sai kuhugi sisse helistatud, aga väga sporaadiliselt. Peamine 
selleteemalise info allikas oli kursavend Mati Muts\index[ppl]{Muts, Mati} ja 
põhiliselt käisin Luciferi \mbox{BBSis}\index{Lucifer BBS}. Küll aga oli 
ülikoolil tol ajal juba täiesti korralik internetiühendus ja palju aega kulus 
Vanemuise õppehoones\index{Tartu Ülikool!Vanemuise tänava õppehoone} terminali 
taga FTPd pidi ringi kolades. Mäletan, et tõmbasin kas ftp.funet.fi või 
ftp.sunet.se serverist tükk aega Metallica albumi kaanepilti ja olin väga 
rahul, kui see ka päriselt kohale jõudis ning ekraanile ilmus.

Selgelt mäletan ka kohtumist HTMLiga. See oli Liivi 
tänaval\index{Tartu Ülikool!Liivi õppehoone}, kus asus Suni 
klass\sidenote{Need pidid olema Sunid, sest mäletan ruudulist hiirepatja, mis 
muidugi ei olnud mingi padi. Suni optiline hiir sõltus lihtsalt ruudulisest aluspinnast.} 
ning kus ma sukeldusin veebilehtede 
võrratusse maailma. Pärast pikka pusimist suutsin tekitada oma kodulehe, kus 
asju õiges kohas hoidis tabel! Ega sinna kodulehele midagi kirjutada ei olnud, 
aga tabeli ridade ja lahtrite saladuste lahtipusimine oli põnev.

Kõik see osutus kasulikuks, sest HClubi järel võttis 
klassivend Meelis Mäeots\index[ppl]{Mäeots, Meelis} mind enda juurde tehnikuks. Ta tegeles tol ajal 
igasuguste imelike asjadega, kuid muu hulgas asutas ka internetifirma. See koosnes 
alguses peamiselt minust ja temast. Firma tegeles Unineti\index{Uninet} 
\emph{dial-up}-ühenduste edasimüümisega, tegi kodulehekülgi ja pidas isegi 
Infomeistri-nimelist interneti infokataloogi. See viimane oli täiesti hämmastav 
äri. Meelis käis ja rääkis mingitele firmadele augu pähe, mina kirjutasin firma 
andmed kuskil serveris asunud staatilisse (!) HTMLi. Mis kasu sellest kellelegi 
ammu enne otsingumootorite laia levikut tõusta võis, on mulle siiani 
arusaamatu. Ma ei mäleta ka, et keegi seal lehel väga käinud oleks. Ometi maksti 
meile arved ära ja ma väga loodan, et tolle tegevuse käigus antud lubadused said
enam-vähem täidetud. 

Kuna teadsin Steni\index[ppl]{Tamkivi, Sten} juba varasemast ja Meelis vist ka puutus temaga kokku, 
lõpetasime ühel hetkel modemitega jantimise ja infokataloogi pidamise ning 
asusime Steni asutatud Halo\index{Halo Interactive DDB} nime all kodulehekülgi tegema. Kampa 
võeti ka mõned kunstnikud, näiteks väga andekas Oliver 
Reitalu\index[ppl]{Reitalu, Oliver} ja mitte vähem andekas Alar 
Koort\index[ppl]{Koort, Alar}, keda kutsuti ilmselt tema rajude elukommete tõttu 
Helbekeseks. Projektijuhiks oli Priit Sasi\index[ppl]{Sasi, Priit}, keda 
kõik tema joviaalse oleku ja suure habeme tõttu Sasuks kutsusid. Sasu õpetas 
mind briti punki ja Alar kurjemat sorti hiphoppi kuulama ning elu oli päris tore. 
Minu käe alt tuli Eesti esimene kommertsalustel tehtud 
(st ettevõte maksis kellelegi lehe tegemise eest raha) kodulehekülg, mis sai 
tehtud Tartu Raadiole\index{Tartu Raadio}, kui mälu ei peta. Kunstnik joonistas 
pildid valmis ja lõikas tükkideks, mina kirjutasin Notepadiga HTMLi ja nii see 
töö käis. 

Ühel hetkel hakkasime lehekülgede tekitamist automatiseerima, kirjutasime 
Perli skripte. Mõnda aega ei olnud meil ei oma serverit ega üldse kuskil Perli 
jooksutada. Siis sai programmeeritud nii, et skript läks meiliga
Unineti\index{Uninet} süsadminnile, kes kopeeris faili õigesse kohta, meie 
vajutasime brauseris nuppu, saime veateate, admin saatis meiliga konsooli 
veateated, mina parandasin koodi ja saatsin uue versiooni. Admini kannatus 
lõppes enne kui minu oma. 

Lõpuks jõudsime oma tegemistega siiski päris kaugele. Perli skriptid läksid 
järjest pikemaks ja kuna andmebaasi pidamiseks ei olnud meil serverites 
piisavalt õigusi, hoiti andmeid enamasti lihtsalt tekstifailis. Üllataval moel 
kattis see ära päris suure hulga vajadusi. Perlilt liikusime ühel hetkel PHP-le 
ja tekkis ka levinud, kuid seetõttu mitte vähem rumal mõte endale ise 
oma sisuhaldussüsteem kirjutada. See sai vist isegi valmis, aga konkreetsed 
mälestused tollest elukast puuduvad.

Ma ei mäleta, et see äri oleks kuidagi tänapäevases mõistes äri moodi välja 
näinud. Raha oli alati vähe ja seega tuli teha kõike, mille eest maksti. Kuidagi 
müüs Sten\index[ppl]{Tamkivi, Sten} Ühispangale\index{Ühispank} maha mõtte anda nende aastaraamat välja CD-l. Mis muud, kui
õppisime selgeks Macromedia Directori kasutamise ja video redigeerimise ning 
andsime minna. Ainus asi, millega me hakkama ei saanud, oli heli. Õnneks oli Sten 
hea sõber Lauri Liivakuga\index[ppl]{Liivak, Lauri}, kelle Forwards 
Studio\index{Forwards Studio} asus meiega sama koridori peal. Lauri 
tegi kenad kõllid ja plõnnid ning aitas selle kõik visuaaliga ära 
sünkroniseerida. Tulemus sai päris kena. 

Igatahes hakkas meile järjest rohkem Tallinna kliente siginema. Ühtlasi müüs Sten 
suure tüki ettevõttest Brand Sellers DDB-le\index{Brand Sellers DDB}, mis oli 
minusuguse Tartu nohiku jaoks täiesti müstiline kamp inimesi. Intelligentsed, 
säravad, jõukad (nii mulle tundus) ning andekad. Bruno Lill\index[ppl]{Lill, 
Bruno} oma terava ütlemise ja peene olekuga on siiani meeles. Nii tehti 
kampas otsus kolida kogu Halo Tallinna. 

Olin tegelikult ligi aasta üsna kahepaikne, pendeldades Tartu ja 
Tallinna vahel. Ülikoolis olid veel viimased sabad lõpetada ja 
Mari\index[ppl]{Kütt, Maria}, kellega peagi abiellusime, käis samuti veel 
koolis. Lõpuks sain oma lõputöö kaitstud ja kuna selliseks triviaalseks asjaks ei 
hakanud ju keegi Tartusse sõitma, käis Mari mu diplomit dekanaadist ära toomas. 
Prouad nõudsid allkirjastatud volitust, mis sai ukse taga kohe valmis tehtud, 
ning nii omandasingi oma esimese teaduskraadi. Tartu Ülikooli peahoone 
sammaste vahelt ei ole ma kunagi välja astunud ja kuigi toonaseid õppejõude 
hindan siiani kõrgelt, pean oma \emph{alma mater}'iks siiski Massachusettsi 
Tehnoloogiainstituuti. 

Tallinnasse kolimisega sai läbi üks etapp Halo kasvuloost. Senise boheemliku 
mis-ikka-valesti-võib-minna mentaliteedi asemel tuli hakata käibenumbritest 
rääkima. Samuti oli meeskond kasvanud. Veel Tartu päevil olin saanud omale 
elu esimese alluva, olles ühtlasi ka tema esimeseks ülemuseks. Vist veel 
keskkooli lõpetav noor nutikas tüüp aitas mul koodi kirjutada ja hängis niisama 
ringi -- ei mina teadnud, kuidas inimesi juhitakse või mida üks ülemus tegema 
peaks. Nimeks oli tüübil Taavet Hinrikus\index[ppl]{Hinrikus, Taavet}. 

Inimesi 
lisandus veelgi ja ma ei saanud enam aru, miks ja kuidas asju tehakse. Ühel ilusal päeval
leidsin kuulutuse, et Hansapank\index{Hansapank} 
otsib internetipanga meeskonda inimesi. Läksin intervjuule. Mäletan siiani 
seda tunnet, kui Liivalaia tänava pangahoone tolle aja kohta ülišiki lifti 
uksed kaheksandal korrusel avanesid ja minu ees laius hurmav vaade 
vanalinnale. Olin müüdud mees, õnneks arvas Vilve Vene\index[ppl]{Vene, Vilve}, 
kes seal majas tarkvara arendamist vedas, samamoodi. 

Nii sai minust veidi enne sajandivahetust 
hansapankur. Mul vedas kohutavalt, sest pank oli praeguses mõistes ulmeliselt 
dünaamiline asutus. Vägesid juhatas Indrek Neivelt\index[ppl]{Neivelt, Indrek}. 
Vaata Maailma programm oli just käima minemas ja sellega tegeles Tiit 
Pekk\index[ppl]{Pekk, Tiit}. Marketsi tiim eesotsas Erkki 
Raasukesega\index[ppl]{Raasuke, Erkki} pidas ülejäänud panka talumatuteks 
venivillemiteks ja tema jaoks toodeti Erik Jõgi\index[ppl]{Jõgi, Erik} juhtimisel imeilusat 
koodi. Ehitasime panga jaoks mõne aastaga mitu internetipanka ja panime ka e-riigile käed külge. 
Aga see, nagu öeldakse, on juba üks teine jutt.

Maria Klenskaja ütles ühes intervjuus ilusti umbes midagi sellist, et mõni inimene on lavale sündinud
ja mõni teeb kõvasti tööd ja, kui noot ees, hätta ei jää. Mõni on programmeerija ja 
mõni oskab koodi kirjutada. Kuulun kindlasti viimaste hulka. Võib-olla just 
seepärast ma hindan väga inimesi, kes erinevalt minust mitte ei \emph{tee}, vaid \emph{on}, ja mulle väga meeldivad päris asjad.
Samamoodi tunduvad näiteks Villu Tamme ja Freddy Grenzman päris -- minu kogemuse põhjal ei ole nad laval 
kuigi palju teistsugused kui elus. Vahest see seletab ka, miks 
seesinane raamat on sündinud.