\index[ppl]{Martens, Tarvi}

\question{Kuidas sina said arvutite ja arvutid sinu juurde?\sidenote{Kuna Tarviga rääkisime juttu mitmel 
korral, on jutulõng mõnevõrra hüplik. Katkemiskohad on tekstis markeeritud.}}


Ma olen pärit Pärnust ja seal arvuteid minu meelest tollal ei olnud, aga 
ma käisin olümpiaadidel, nii et matemaatika ei olnud minu jaoks 
mingi teema. Viiendas klassis
võitsin kuuenda klassi matemaatika linnaolümpiaadi, mille üle kõik olid suhteliselt 
jahmunud. Ühe riikliku olümpiaadi käigus viidi meid 
ekskursioonile Nõo Keskkooli\index{Nõo Keskkool}, kus oli suur arvuti. See oli teistsugune maailm, aga kui mind sinna õppima 
taheti viia, siis ma ei tahtnud väga minna. Mul oli Pärnus oma bänd.

\question{Sul oli oma bänd?}

Jah. Tegime punki nagu ikka sel ajal. Käisin Pärnus muusikakallakuga koolis ja bänditegemine oli 
elementaarne. Kooliteater tegi ka oma esimesi samme. Kadunud Aare Laanemets\index[ppl]{Laanemets, Aare} ja Elmar 
Trink\index[ppl]{Trink, Elmar} tegid esimese kooliteatri, kus ka mina osalesin. 
Kõik see oli nii tore ja ma mõtlesin, et ei viitsi kuhugi kaugele 
kooli minna. Aga matemaatikaõpetaja käis mu vanemate juures, rääkis nad 
pehmeks ja nii see läks. 

\question{Kas sel ajal Nõo legend alles kujunes või oli see juba tuntud paik?}

Jah, oli kindlasti tuntud. Oli teisigi tugevaid koole, 
Tartus-Tallinnas, aga Nõo kool oli üle kõige. Põhiliselt 
sellepärast, et neile oli oma arvutuskeskus ehitatud, nii et sinna tuldi üle 
vabariigi kokku. Samas enamik olid ümberkaudsed maalapsed, kes ei olnud võibolla väga suured geeniused. 

Nõo Keskkoolis oli Nairi 3-1\index{Nairi!Nairi-3-1}, niisugune 
\emph{mainframe}, millele sai perfolinti sisse sööta ja laiprinterist 
tulemuse välja printida. Aga see ei tundunud väga huvitav. Umbes üheksanda klassi poisina 
avastasin Tartu Ülikooli Vanemuise õppehoone\index{Tartu 
Ülikool!Vanemuise tänava õppehoone} keldrikorruselt kabineti, kus oli 
kaks ja pool Apple II\index{Apple II}. Kaks ja pool sellepärast, 
et üks oli kogu aeg katki ja Andres Peiker\index[ppl]{Peiker, Andres}, kes oli 
selle keldri kunn, remontis seda.

Koolipoisina konkureerisin arvutiaja pärast tõeliste 
üliõpilastega nagu Tanel Tammet\index[ppl]{Tammet, Tanel}, Margus 
Liiv\index[ppl]{Liiv, Margus} ja teised. Sain ennast kuidagi 
vahele pista ja enamiku ajast ei käinud enam väga palju 
koolis, vaid olin rohkem Tartus.

\question{Ometigi oli Nõo kool mõeldud sinusuguste harimiseks süvendatult. Kas sul oli vaja veel rohkem süvitsi minna?}

Mis sa seal Nairi juures perfolindiga harid! Saatuse vingerpussina saabus 
aasta hiljem, kümnendas klassis Nõo kooli hunnik 
Agate\index{Agat}, mis olid Apple II kloonid, 
ainult värvilised. Kõige naljakam oli see, et kohalikud arvutiõpetajaid ei 
teadnud nendest midagi ja siis tuli välja, et on üks Tarvi, kes tunneb Agati
protsessorit läbi ja lõhki. Sel olid küll oma operatsioonisüsteem ja 
venekeelsed programmeerimiskeeled, aga sellest polnud midagi. Nii et ühel hetkel 
oli mul arvutuskeskuses oma kabinet ja arvuti. 

\question{Kas selleks piisas Tartus Apple II uurimisest? Kas said sahibide 
vahel noka piisavalt märjaks, et Nõos kunn olla?}

Täpselt nii, pärast õpetasin õpetajaid. 

\question{Kas Agat oli Apple II kloon kuni riistavara disaini ja arhitektuurini 
välja?}

Vähemalt protsessori mõttes oli see kindlasti sama. Ma ei ole väga suur riistvara 
asjatundja, kuigi assembleris\index{Assembler} programmeerisin 
vabalt sel ajal. Küllap see oli üsna täpne kloon, aga 
värviline võrreldes Apple IIga. See tähendab, et pilt virvendas kogu aeg 
silme ees. 

\question{Kui sa omale kabineti said, kas siis oli uhke tunne?}

Mis seal ikka erilist oli. Hea oli see, et sain oma asja ajada ega pidanud enam Tartu vahet 
käima.

\question{Kas see õppimist ei hakanud segama?}

Ei hakanud. Mul ei ole sellega kunagi probleeme olnud. Tuleb 
lihtsalt kontrolltööd ja eksamid ära teha ja siis keegi ei õienda.

Nairi peal olid tõsiste inimeste keeled nagu Algol\index{Algol}, aga 
lastele õpetati programmeerimiskeeli ROPS\index{ROPS} ja 
KÕPS\index{KÕPS}\sidenote{Vt ka märkust lk
\pageref{sidenote:ROPS}.}, mis olid eestikeelsed. KÕPSis 
sai programmeerida joonistamist, näiteks kuidas plotter 
liigub: mine üles, mine alla, mine paremale; jäta joon, ära jäta. ROPS oli 
päris programmeerimiskeel. Ma tegin need keeled ka Agati peale ringi, et 
lapsed ei peaks Nairiga tegelema. 

\question{Matemaatika tuli sul lihtsalt, aga kuidas matemaatikahuvi läks üle nii suureks arvutihuviks, et käisid Nõost Tartus arvutis ja portisid programmeerimiskeeli? Mis 
sind selle puhul tõmbas?}

See on hea küsimus, aga mul ei ole head vastust. Arvuti oli selgelt täiesti 
teistmoodi, nagu praktiline matemaatika -- rehkendusmasin, mis on kalkulaatorist intelligentsem. Mõtlesin vist
juba siis, et see on paratamatu tulevik ja teistmoodi ei saagi olla. 

\question{Huvitav, et sul on matemaatika ja arvutite seos algusest peale selge 
olnud. Mõnel tekib see seos palju hiljem kui üldse.}

Matemaatiline loogika on olnud kogu aeg üks minu lemmikdistsipliine, arvutid 
ja muusika on väga loogilised asjad. 

Ühel hetkel lõpetasin kooli ära ja läksin TPIsse\index{Tallinna 
Tehnikaülikool}.

\question{Miks sinna? Tartu Ülikool oli ju sulle juba tuttav.}

Mulle tundus, et TPI oli natukene praktilisema hoiakuga, ja aastal 
1987 räägiti Tartu Ülikooli informaatika kohta, 
et seal rohkem ikka joonistatakse tahvli peale. Ja päris matemaatikuks ma kindlasti 
ei tahtnud saada.

Tegelikult olin Tallinna vahet enne käinud. Seal oli Õpilaste 
Teaduslik Ühing\index{Õpilaste Teaduslik Ühing}, kus Peeter 
Lorents\index[ppl]{Lorents, Peeter} tegi matemaatikasektsiooni. Käisin 
Lorentsi juures aeg-ajalt, ta andis mulle kaelamurdvaid 
ülesandeid. Kahekordsete integraalidega 
elu oli huvitav, nii et TPIsse minek tundus loogiline.

\question{Mida sa õppima läksid?}

Automaatikateaduskonda ja eriala oli
LI\index{Tallinna Tehnikaülikool!Automaatikateaduskond!LI} ehk arvutid ja 
arvutitehnika. Seal juhtus kohe mitu asja. 

Kõigepealt ütlesin esimeses programmeerimistunnis, et siia tundi ma rohkem 
ei tule. Õppejõud ei solvunud, sest kirjutasin sissejuhatavas tunnis salaja
ühe programmi valmis ja näitasin seda talle.

Teiseks oli Teaduste Akadeemia Küberneetika Instituudi 
Erikonstrueerimisbüroo\index{EKTA} juhtimissüsteemide osakonnas\index{Küberneetika 
Instituut!Juhtimissüsteemide osakond}\sidenote{Esineb ka nimekuju Arvutustehnika Erikonstrueerimisbüroo ja
Arvutustehnika Arendusbüroo, mis paistavad viitavat samale asutusele.} just
leiutatud kooliarvuti Juku\index{Juku}. Nad asusid sealsamas Küberneetika majas, kus olin juba käinud, ja 
septembri esimesel nädalal sadasin sinna sisse. Mul jäi 
õpilaste keskkondade pärast mure, et kui tuleb kooliarvuti, siis võiks olla ka 
õpilastele mõeldud programmeerimiskeeled, ja ROPSi\index{ROPS} portimine 
Jukule oli tegemata. Rääkisin Juku tegijatele, et oleks vaja vastavasuunalist 
arendust. Nad lubasid mul enda juures hängida ja nelja kuu pärast 
olin tööle võetud. 

\question{Kas ülikool jäi kõrvale?}

Ei jäänud, käisin korralikult eksameid tegemas. 
Vahepeal, pärast esimest kursust, käisin Vene kroonus ka. Olin viimane 
lend, kes sai kroonusse minna, ja olen selle üle väga õnnelik. Meid viidi Leningradi lähistele, aga kuna
sain puhkpilliorkestrisse ja tegelikult tegin jälle bändi, siis polnud häda midagi. 
Jälle üks kogemus juures. 

Kroonust tulles paljud langevad ülikoolist välja, sest leiavad, et võiks 
midagi praktilist teha ja ennast targaks ajamine ei tasu ära. Mulgi 
oli teise kursuse poole peal kriis, kui mõtlesin, et mul on kohal 
käimata ja et kui eksameid ära ei tee, siis on kõik. Aga tegin 
eksamid ära ja võtsingi selle elustiili, et pühendasin ülikoolile umbes 
kolm nädalat poole aasta kohta. Imesin materjali sisse, tegin eksamid ära ja 
kõik töötas. 

\question{Minu puhul möödus keskkool mängides ja lauldes, sest 
kõik oli lihtne, kuid ülikooli minnes lõppes lihtsus ära. Kas sinul ei 
lõppenud?}

Lihtsus lõppes tõesti. Õigemini olid keerukad esimesed poolteist või kaks 
aastat, kui taoti pähe fundamentaalset kõrgemat füüsikat ja matemaatikat, mis lööb kaane pealt ära. Aga edasi läks erialasemaks 
ja inimlikumaks, õppimine ei olnud enam nii teoreetiliselt tappev. 

\question{Kas ülejäänud aja tegelesid Jukudega?}

Ei, kui kroonust tulin, oli kontorisse toodud juba esimene 286. Oli huvitav aeg, et käisin küll 
tööl, aga tööd oli vähe. Kui 
leidsid endale haltuuraotsi, oli suhtumine väga soosiv. Kõige suurema haltuuraotsa puhul, 
mida mäletan, tuldi koos arvutiga. Sain personaalse arvuti ja 
tööandja eraldas ka kabineti. 

\question{Kes need haltuurapakkujad olid? Kas oskad mõne näite tuua?}

Igasugused. Arvutiga tuli Soome laevaehitaja. 
Pean seda siiamaani kõige vingemaks programmiks, mille ma olen teinud. Ülesanne 
oli selline, et on kümne tekiga sõjalaev, mis vajab 
elektrivarustust; kuskil on jõuallikad ja kuskil tarbijad. Ja nüüd tuleb
hakata nende asjade vahele erineva jämedusega kaableid vedama. Kaablirennid 
on olemas, aga ühel hetkel saab kaablirenn täis. Mis me teeme? Veame 
teistpidi. Aga kes ütleb, et kaablikulu on sealjuures kõige optimaalsem? 

\question{Kas siis oli veel sügav Nõukogude aeg?}

Ei, siis oli juba sula ja hell aeg. See oli pärast kroonut, 1990 või 1991.

\question{Sel ajal ei tohtinud isegi mitte arvuteid 
Nõukogude Liitu tuua, aga sina arvutasid sõjalaevade kaableid.}

Kes seda ikka teadis. 

\question{Kuidas see haltuurapakkuja oskas sinu juurde tulla?}

See õppejõud, kellele esimeses tunnis ütlesin, et 
ma rohkem ei käi sinu juures, leidis mulle otsi. Inimesed 
teadsid mind ja oskasid soovitada. Just ülikooliajal sai väga 
eripalgelisi asju tehtud. Ma olin siis kõva programmeerija, kirjutasin muu hulgas 
oma andmebaasisüsteemi, mis oli FoxProst kordades kiirem. Vanasti oli 
kõvaketta poole pöördumine ränk tegevus, mis võttis 
aega, mitte nagu praegu SSD puhul. Ma kirjutasin andmebaasisüsteemi, millel olid 
fikseeritud pikkusega väljade asemel sujuva pikkusega väljad. See tähendab, et andmeid oli ketta peal täpselt nii palju, kui oli, mitte ei 
olnud eraldatud kindel hulk megabaite. Tõmbasin
keskmise andmebaasi umbes kaheksa korda kokku ja vastavalt sellele suurenes 
töötlemiskiirus.

\question{Kuidas sa kirjeid pakid ja mis saab siis, kui välja pikkus 
muutub? See ei ole ju lihtne.}

Miks see peaks lihtne olema? Mis see geniaalsele programmeerijale ja 
matemaatikule ära ei ole välja rehkendada? Nagu sõjalaevade kaalutud 
graaf, milline on kõige optimaalsem kaablikulu. 

\question{See tegevus läheb otsapidi teadusse, mujal maailmaski ei olnud
andmebaase teab mis palju. Kas sa teadlaseks ei tahtnud saada?}

Ei, mulle meeldis praktiline pool. Lõpuks läksin pika hambaga 
magistrantuuri ja virelesin seal umbes kuus aastat. Siis kui
ainepunktid hakkasid ära kustuma, tegin jõuga lõputöö. Mulle kuiv teooria ei paku 
eriti midagi, mulle meeldib maailma muuta. 

\question{Kas sa olid kuulus ka?}

Ei olnud. Eks ühe või teise tehtud töö tõttu renomee levis ja ka õppejõud
Peeter Lorents\index[ppl]{Lorents, Peeter} levitas sõna, nii et kõik käis
tutvuste ja sidemete kaudu. See ei olnud massiline, tegin umbes kümmekond projekti, aga need olid päris suured.

Tööasju tegin ka loomulikult, aga tööd oli toona vähe 
ja mentaliteet oli selline, et parem olgu inimene olemas ja valmis. Kui tööd 
tuleb, siis saab seda teha. Too kontor, mis on tänase nimega 
Ektaco\index{Ektaco}, oli fantastiline koht. Seal oli umbes 
viiskümmend inimest, tehti riistvara ja tarkvara, \emph{fifty-fifty}. 
Juku oli muidugi nende tehtud. Muu hulgas tegi Elleri-papi 
ehtekarbist valmis esimese hiire maailmas\sidenote[][-1cm]{Arvo 
Eller\index[ppl]{Eller, Arvo} oli Juku loomise eestvedaja (Ants Vill (2010). 
Meenutusi aegadest, kui arvuteid tehti veel käsitsi. Linnaleht (Tallinn), 
46). Kas tema loodud hiir just maailma esimene oli, aga ehtekarbi lugu kordab 
ka viidatud allikas.}.

Pooled inimesed olid \emph{cum laude} TPI lõpetanud, nii et sealne 
ajupotentsiaal oli nauditav. Näiteks kui ülemusel oli sünnipäev, siis
vennad mõtlesid, et teevad kingiks rääkiva papagoi. Tegidki. Seal oli 
briljantseid ja lahedaid tüüpe. 

\question{Mis see töö sisu seal ikkagi oli? Kas ise mõeldi projekte välja?}

Nii ja naa. Üks põhiline valdkond oli 
tööstuskontrollerid: ise mõtlesid välja, ise tegid, ise programmeerisid. Need olid 
\emph{rack}'i-suurused, täna saab samasuguse asja osta Hiinast 
kiibisuurusena. Kontroller koosneb analoogsisenditest ja 
\mbox{-väljunditest}, digitaalsisenditest ja \mbox{-väljunditest} ning nendevahelisest 
loogikast. 
Tollal oli vaene aeg ja Ektaco\index{Ektaco} tehti ühisettevõttena ühe Soome partneriga. Tänase 
päevani teevad nad kassasüsteeme Compucash, mida võib 
baarides aeg-ajalt siiamaani näha. Toona tuli soomlane ja ütles, et tehke mulle 
proovitöö -- selline maatriksklaviatuur, et kui baarmen vajutab \enquote{õlu}, on 
kohe olemas. See tuli välja ja koostöö jätkus. Tollal ei olnud lihtne 
tellimusi leida, seetõttu suur osa
inimesi istuski pool aega jõude. 

\question{Ja sina muudkui programmeerisid?}

Mina muudkui programmeerisin. Ektacos\index{Ektaco} olin kokku viis aastat, enam-vähem kogu
ülikooliaja. Aastal 1992 läksin siiski tagasi
nii-öelda peamajja, Küberneetika Instituuti\index{Küberneetika 
Instituut}. Seal tekkis uus rakuke, mis esialgu alustas krüptograafia alusuuringuid. Seltskonnas 
olid mõned teadlase moodi ülikoolipoisid ka, näiteks Ahto Buldas. Ülo Jaaksoo\index[ppl]{Jaaksoo, Ülo} oli 
toonud välismaalt paksu raamatu krüptograafia aluste kohta\sidenote{Küberi\index{Küber} allikate andmetel oli tõenäoliselt tegu raamatuga Simmons, Gustavus J. Contemporary cryptology: The science of information integrity. IEEE press, 1994.} ja seda me siis koos 
lugesime. Keegi luges peatüki läbi, proovis aru saada ja seletas 
teistele ka. Krüptograafia kui teadus Eestis puudus arusaadavatel põhjustel. Kui Eesti
iseseisvus, oli plats lage ja kuskilt pidi alustama.

\question{Kuidas mujal maailmas oli krüptoga? Mis tolleks hetkeks juba 
olemas oli?}

RSA oli olemas, aastast 1978. Ma täpselt ei tea, sest ei ole ennast 
kunagi krüptoloogiks pidanud. Minu eriala on rohkem nii-öelda 
rakenduskrüptograafia, mitte süvakrüptograafia.

\question{Miks sa sinna läksid? Sul oli Ektacos ju mõnus oma projekte teha.}

Pooled inimesed olid suurepärased insenerid, lõpetanud \emph{cum laude}, aga firmas ei saadud aru, et nende arenguga peaks tegelema. 
Oli väga selge seisukoht, et igaühe areng on tema enda asi. 
Interneti panek firmasse, ajakirjade ostmine või 
inimeste saatmine konverentsile ei tulnud kõne allagi. Pinge 
kogunes ja mingil hetkel, oma sünnipäeval, saatsin kohalikku võrku essee, mis firmas valesti on, mida tsiteeriti
pärast aastaid. Kümme aastat hiljem võeti see välja ja vaadati, et ikka on sama lugu. 

\question{Kuidas see kamp ülejäänud Eesti kogukonnaga kokku käis? Tol ajal pidas osa inimesi juba BBSe.}

Mu hea sõber ja kolleeg Heiki Kask\index[ppl]{Kask, Heiki} pidas ühte 
BBSi ja ma liitusin sellega. Sealtkaudu sattusin lõpuks fidonautide 
sekka ja hakkasin nendega läbi käima. 

\question{Kas see ei olnud sinu jaoks tähtis asi?}

FidoNet ei olnud minu jaoks tähtis, see oli lahe ja andis 
esialgse maigu suhu, aga nii kui tuli Internet, armusin sellesse.

\question{Mis interneti juures nii armastusväärset oli? Meile ja uudiseid 
sai FidoNeti kaudu ka.}

Meil oli esialgu UUCP ja modemiga helistamine mitu aastat, 1991--1993, kui ma 
ei eksi. Sai meili saata, mis oli väga tore, aga mulle jõudis kohale, et kuskil on 
olemas nii-öelda püsiühendusega internet ja suhelda saab reaalajas\sidenote[][-.8cm]{Mõiste \enquote{püsiühendus} oli tol ajal maagilise 
tähendusega: ei unistatud mitte kiirest, vaid pidevalt ühendatud 
internetist. Võimalus kaugete arvutitega vahetult suhelda tundus imeline.}. 
See oli minu jaoks nii võluv, et 
loomulikult tahtsin seda ühel või teisel moel uurida. Nii et UUCP 
aegadel mäletan ennast pühapäeviti kuskil modemi küljes rippumas ja RFCsid\sidenote{\emph{Request For Comments} (RFC) on juba alates 1969. aastast kasutusel olev standardne viis kõiksugu internetiga seotud standardite avaldamiseks ja kokku leppimiseks, RFCd on nummerdatud ja tuntudki oma numbrite järgi. Need sätestavad sõna tõsises mõttes kõike alates Interneti alusprotokollidest kuni tuvide abil side korraldamise (RFC 1149 — A Standard for the Transmission of IP Datagrams on Avian Carriers, D. Waitzman, 4/1/1990, 2 pp.) ja kohvi keetmiseni (RFC 2324 — Hyper Text Coffee Pot Control Protocol (HTCPCP/1.0), L. Masinter, 4/1/1998, 10 pp.).} 
alla laadimas, et need kõik algusest peale läbi lugeda.

\question{Kas see oli tol ajal võimalik?}

Oli küll. RFCde ülemine ots oli kuskil tuhande kandis alles, nii et see ei olnud 
probleem. Osad olid lühikesed ja osad mõttetud, ja oli ilmselge maniakaalsus 
koguda endale hästi palju materjali, et küll ükspäev loen.

\question{Kas seal uues üksuses oli internet sinu jaoks siis infoallikas?}

Eks jah. Sai meili kirjutada, lahe värk. Enne veebi olid 
põhilised FTP-saidid -- ei pidanud mõtlema, mis \emph{node}'ist või kust 
mida saad. Mõnikord sai FTPst ka mõne mängu kätte, seal ikka liikus kraami. 
Seal sai ju samamoodi alla ja üles laadida nagu FidoNetis. 

\question{Kas sa mängisid arvutimänge ka?}

Suur mängumees ma ei olnud, aga noorest peast midagi ikka õhtuti põristasin 
ja täristasin. See oli lõõgastumisviis, mitte huvi. 

\question{Sinu fookus oli matemaatikal?}

Programmeerimisel, mulle meeldis arvutit oma pilli järgi tantsima panna, mitte 
arvuti pilli järgi tantsida. Kui Windows\index{Windows} tuli, 
siis ma kaotasin usu arvutitesse, sest ma ei suutnud enam igat 
bitti kontrollida. Kuni sinnamaani teadsin opsüsteemi, EEPROMi 
tasemel, mis sünnib, aga nii kui Windows tuli, siis kontroll kadus ja mul läks tuju ära.

\question{Kui tekkis Linux\index{Linux}, kas siis tuli tuju tagasi?}

Linux aitas jah Windowsi aja üle elada, aga hulluks 
Linuxi kasutajaks ma ikkagi ei hakanud. Kui läksin 
Ektacost\index{Ektaco} Küberneetikasse\index{Küberneetika 
Instituut}, siis jätsin programmeerimise maha. Viimane asi, mille 
tegin, oli 1996. aastal mail.ee\index{mail.ee}. 

\question{Miks sa selle tegid?}

UNDP\sidenote{ÜRO Arenguprogramm. Üheksakümnendatel läks Eesti veel üsna 
selgesti arengumaana kirja ja sai paljudest kanalitest igasugust abi. 
Tänaseks on humanitaarabi mõiste õnneks suuresti ununenud, kuid toona tuli 
seda kõikvõimalikul kujul päris palju ning oli tõesti abiks.} andis selle tegemiseks väikse grandi.
Kõigepealt tekkis hea mõte, et igal soovijal võiks olla meiliaadress. 

Pean alustama sellest, et 1994. aastal sai tehtud 
firma Teleport\index{Teleport} (mitte ajada segi selle sajandi 
Teleportiga!). Meid oli kaheksa tudengit, kellest kuus õppisid välismaal, sest 
neil oli raha. Eesti tudengitel raha ei olnud. Kaheksakesi panime rahad 
kokku, ostsime Soomest portsu modemeid ja tegime sissehelistamiskeskuse, kus 
sai ilma lepinguta 900-numbri\sidenote{Telefoninumbrid 
algusega 900, millele helistamisel kehtis eritariif. Tariifi 
jagati teenusepakkujaga ja see võimaldas tasulisi teenuseid osutada.} kaudu helistada. Saime 
tänu 900-teenusele kohe oma raha kätte. Kommertsiaalse interneti pakkumine oli sel 
ajal vaat et olematu ja laiadele massidele mõeldes täiesti 
puudulik. 

\question{Mis aastal Uninet\index{Uninet} meile tuli?}

Uninet oli juba olemas, aga selleks tuli leping sõlmida. 
EsData\index{EsData} oli ka olemas, me istusime tegelikult nende võrgu peal. 
Kuu hiljem tuli MicroLink Online\index{MicroLink!Online} ja sõi meid massiga 
ära. Teleportist sai mõnesid partnereid kaasates 
Meediamaa\index{Meediamaa} ehk www.ee\index{www.ee}. See oli Eesti 
esimene veebiäri, kus proovisime inimestele rääkida, et kui sind pole 
internetis, pole sind olemas, ja et tulevikus pole sul oma kaubaauto peal muud vaja kui URLi. Nad vaatasid meid nagu idioote, aga nüüd ainult URLiga 
kaubaautosid näebki. 

\question{Miks teie kui programmeerijad firma tegite?}

Pigem olime ikka tudengid. Tarvi selgitas, et niisugust teenust turul ei ole, ja see 
tundus väga lahe, et inimesed saavad juurdepääsu internetile. 

\question{Kas see oli siis puhas missiooniüritus?}

Eks mõttes lootsime raha ka teenida, sest see tundus olematu \emph{business}, 
kus on võimalik kanda kinnitada. Veebiga oli sama lugu. Samas oli see 
paljuski ka missiooni ja eestvedamise asi. Kirjutasin 1996. aastal internetist ka raamatu, mis oli esimene eestikeelne 
selleteemaline originaalteos\sidenote{Tarvi Martens, Vello Hanson. Internet. Ilo, 
1996.}. See oli interneti propageerimine. Samal ajal 
ehitasin riigile andmesidevõrkusid ja TCP/IP 
tehnoloogia laialdane levik tundus mulle sellel kümnendil väga tähtis.

\question{Miks?}

Saavutamaks seda olukorda, kus me täna oleme. 

\question{Kas sul oli peas olemas teadmine, et selline olukord peab ja hakkab olema ning see on hea?}

Ma teadsin, et see on hea. Ma ei teadnud, kui kiiresti ja kui massiliselt see levib, aga 
hüved olid ilmselged. 

\question{Kas su juttu keegi kuulas ka?}

Arvan, et jah. Me oleme näinud, et igasuguse uue tehnoloogia evitamine 
võtab palju aega. Siis on täitsa loomulik, et räägime kahekümne viie aasta tagusest ajast, mille järelmeid võib näha täna. Samamoodi
ei ole ID-kaardi ja e-hääletamise tulemused tulnud 
päeva, kuu või aastaga. Rääkisin kord 
ühele psühholoogile, mida ma teen, ja ta ütles: \enquote{Tarvi, sa oled 
hull. Need asjad, mida sa teed, on inimeste käitumise muutmine. Ühiskondliku 
käitumise muutumine võtab minimaalselt seitse kuni kaheksa aastat aega. Sa ei 
saa oma tibusid lugeda enne, kui jääd vanaks.}

\question{Vähe sellest, tagantjärele on too algne impulss sisuliselt tuvastamatu 
ja seega keegi aitäh ei ütle.}

Ma ei igatsegi seda, see on väga okei. Lihtsalt vaatan 
ringi ja naeratan. 

\question{Sa mainisid, et tegid riigile 
andmesideühendusi.}

Ojaa, see on üks tore lugu. Tegime sel ajal
riigiga palju koostööd standardite ja andmekogude 
vallas, näiteks disainisime Andmekaitse Inspektsioonile\index{Andmekaitse Inspektsioon}. 
Usun, et oli aasta 1993, kui Eesti toll\index{Tolliamet} ja piirivalve\index{Piirivalveamet} tulid Küberneetika Instituuti\index{Küberneetika Instituut} ja ütlesid, et 
oleks vaja piirivalve ja toll üles ehitada. Neil on 
ühised piiripunktid, kus pole mingit sidet, mõnikord isegi mitte 
telefonisidet, ja kas Küberneetika Instituut saaks aidata. 
Joonistasin projekti, klient tuli paari kuu pärast tagasi ja ütles, et mitte keegi ei 
suuda seda projekti ellu viia ja tehke see ise ära. 
Pidimegi hakkama paberimäärimisest tegudele üle minema. 
Koostöös Eesti Telefoniga\index{Eesti Telefon} 
said esimesed ühendused tehtud ja siis hakkas see tegevus mullina 
paisuma. Järgmisena tuli politsei ja riburada teised järel. Me 
tegutsesime Küberneetika Instituudi katuse all, mis oli väga hea ja
amorfne asutus: tahtsid, tegid teadust; tahtsid, tegid äri.

Raha hakkas liikuma, pidime ruutereid 
ostma (kulud jagasime tellijaga -- näiteks ostsime piiripunkti ruuteri ja tegime piirivalvega kulud pooleks) ja seega oli vaja moodustada mingi juriidiline keha. Tegime midagi niisugust, mida 
ei tohtinud tegelikult seaduse järgi teha, põhimõtteliselt MTÜ riigiasutustest. 
See MTÜ oli Andmeside Osakond\index{ASO}\index{Andmeside 
Osakond|see{ASO}}, mida juhtis nõukogu, kus oli iga riigiasutuse esindaja.
Raamatupidamistoimkond nurises iga aasta, et sellist asja ei tohi teha, aga 
ülemused ja ministrid ütlesid, et ärme lõhu 
toimivat asja.

\question{See eeldas, et keegi riigi poolel kuulas sind ja 
mõtles kaasa. Kas need olid tippjuhid või IT-juhid?}

Kõigepealt kuulasid IT-juhid, kes rääkisid oma tippjuhtidele. 
Mäletan selgelt, kuidas 31. detsembril istusid toonase 
piirivalve\index{Piirivalveamet} ülema Kõutsi\index[ppl]{Kõuts, Tarmo} 
kabinetis kõik asjaosalised -- politsei, piirivalve, toll ja Küberneetika 
Instituut -- laua ümber ja kirjutasid lepingule alla. Kõuts veel ütles: \enquote{Ma saan aru, et meil on siin juhikandidaat ka laua taga.} Ma olin siis alles kahekümne viie aastane naga. 

Edasi läks väga huvitavaks, sest meil oli tegelikult olemas selline asutus nagu 
Valitsusside\index{Valitsusside}, kes tegeles erivõrkudega.

\question{Kas nad su peale kurjaks ei saanud?}

Teatav konflikt tekkis jah erinevatel põhjustel, sealhulgas 
koolkondade vastasseis -- Jaaksood \emph{versus} Lippmaad. Vanemad inimesed 
teavad seda väga hästi.

Aga juhtus jah, et piirivalvel oli kagupiir täiesti lage, 
seal polnud mingit sidet. Ja selle asemel et minna 
Valitsussidesse, kes pidanuks seda tegema, tulid nad minu juurde ja ütlesid, 
et näed, Tarvi, siin on kümme miljonit\sidenote{Tegu on Eesti 
kroonidega. Arvestades valuutakursse ja  inflatsiooni, on tänases kontekstis 
tegu umbes 1,2 miljoni euroga. Arvutades protsenti riigieelarve (mis oli tänasega võrreldes väga pisike) 
kuludest, maksis too projekt tänases mõistes suurusjärgus 21 miljonit eurot.}, meil on seal lage 
plats, kaheksa piiripunkti on vaja ühendada, tee midagi. Ma ütlesin, et jaa, väga huvitav. Aasta oli 1994 või 1995.

\question{See oli tol ajal suur raha. Kõike oli ju vaja ehitada, kust tekkis 
idee see raha just sidele kulutada?}

Kui oled keset tühja platsi, kus ei ole 
mobiililevi ega mitte midagi, kuidas sa seda piiri pead? Jutt käib elementaarsest telefonisidest ja sõnumivahetusest, mitte suvalisest veebibrausimisest.

Minust sai projektijuht ja me ehitasime tühjale kohale 2,4gigase raadioside
kaheksa mastiga, taldrikud otsa.

\question{See ei ole raadioside disaini mõttes triviaalne ülesanne -- kas
õppisid seda kuskilt raamatust?}

Mõtlesin kasutada 
kõrget sagedust ja seega pidi olema otsenähtavus. Aga kuidas seda 
kindlaks teha? Lõuna-Eesti maastik, mäed ja orud. Leidsin 
Maa-ametist\index{Maa-amet} ühe tuttava, kes oli hakanud 
Vene ohvitserikaarte (kõige täpsemaid, mis tollal oli) digiteerima ja oli 
selle kõige huvitavama osa ehk Võrumaa sisse saanud. Ta suutis mulle 
väljastada profiili: andsin talle otspunktid ja tema mulle arvujada. Kirjutasin ise programmi, keerasin maa kumeraks, panin mastid kasvama ja 
vaatasin, kas on otsenähtavus. Selle järgi sai mastide kõrguse 
rehkendada ning mida kõrgem mast, seda kallim oli. 
EMT\index{EMT} ei 
vaadanud mingit profiili, pani 80 meetrit igale poole. Mul aga oli 
52 ja 54 meetrit, mille puhul pidi 
lennutuled lisama ja jälle oli kallim. Sattusin ühe teadjamehe peale 
Eesti Telefonist\index{Eesti Telefon}, kes vaatas tehtut ja ütles: 
\enquote{Kuule, mees, kas sa tegid nädalaga sellise asja? Trassi projekteerimiseks 
läheb poolteist aastat, tuleb jala kõik läbi käia, puud ära kaardistada!} Aga 
mul olid juba mastid tellitud. Ta rääkis, et on olemas Fresneli tsoon -- 
saatja ja vastuvõtja vahele ei teki mitte kiir, vaid vorsti moodi 
asi\sidenote{Fresneli tsoon on ellipsoidne tsoon, mida pidi raadiolained 
saatjast vastuvõtjani levivad. Tsooni võivad sattuda ja seega sidet segada 
ka otsenähtavusest väljapoole jäävad objektid.}. See võttis natuke jahedaks 
küll, kuid mastid olid tellitud ja side läks käima. Järgmisel aastal tegin Peipsi 
äärde sama viguri. 

\question{Ühesõnaga sa ei teadnud, et nii ei saa teha?}

Ei teadnud, mõtlesin inseneri mõistusega, kuidas see käib. 

\question{Miks sa üldse kulude optimeerimisega vaeva nägid, kui nii palju raha anti kätte?}

Vabariigi algusajal ei olnud raha palju. Igas vallas pidi olema optimaalne ja tegema parimat, mis teha 
annab. See ei olnud teab mis üleliia suur raha, kulus kõik ära. 

See oli väga tore aeg, kui sai tõesti käegakatsutavalt riigi arengut 
toetada, pealegi minu lemmiktehnoloogia ehk 
interneti osas. 

\question{Kui ma sind kuulan, siis sa olid programmeerija, kuni saabus internet 
ja leidsid, et tuleb hoopis sinna panustada, sest maailm läheb sellest 
paremaks.}

Jah. Programmeerida oskas sel ajal juba üha rohkem inimesi, ma ei olnud enam 
unikaalne ja kaua sa ikka programmeerid.

\question{Mõni programmeerib eluaeg.}

Arusaadav, aga kõrgemad ja üllamad mõtted tundusid 
järjest paremad. Võibolla see on ka isiksuse arenguga seotud. Ausalt öeldes, 
kui olin programmeerija, siis kartsin telefonihelinat, sest ma ei 
tahtnud inimestega suhelda. Ühel hetkel läks see üle. Linna peal teadsid kõik, et kui Martens tuleb jaurama, siis 
proovib kindlasti Küberneetikasse tööle meelitada. 

\question{Kas sa olid Küberneetika Instituudis\index{Küberneetika Instituut} juhtkonnas, et käisid teisi tööle meelitamas?}

Olin ASO\index{ASO} pealik, see sai üle antud 
informaatikakeskusele\index{Informaatikakeskus}, mis oli RIA\index{Riigi Infosüsteemi Amet} eelkäija\sidenote{Eesti Informaatikakeskus koos 
Riigihangete Keskusega liideti aastal 2003 Riigi Infosüsteemi Arenduskeskuseks, 
millest 2011. aastal sai Riigi Infosüsteemi Amet ehk RIA.}. 

Aastal 1997 toimus reformatsioon: instituudid kui eraldiseisvad institutsioonid 
kaotati ja pidid liituma ülikoolidega. Küberneetika Instituut jagunes kolmeks: kõige väiksem osa ehk 
andmesideosakond läks informaatikakeskusele, teisest osast sai aktsiaselts ja kolmas liikus
Tallinna Tehnikaülikooli alla. Kuna Küberneetika Instituudis oli 
praktilist tegevust hästi palju, siis kõigest praktilisest moodustati 
Küberneetika Aktsiaselts\index{Küberneetika AS}, mis on siiamaani alles. See 
asutati riigiettevõttena ja nüüd on vist erastatud. 

Küberneetika AS oli väga 
huvitav kombinatsioon. Oli osakond, kus programmeeriti tolliameti\index{Tolliamet} 
infosüsteeme. Minu osakond oli keskendunud infoturbele nii teoorias, 
praktikas, konsultatsioonides kui ka analüüsides. Ja seal kõrval oli 
meremärgindus ja -navigatsioon ning valgusfooride tegemine. Lisaks
kinnisvarahaldus, aga seda enam pole. 
 
\question{Sinu jutu sisse sigineb tasapisi juhiroll. Mõned inimesed saavad selle maigu suhu ja siis ainult sellega 
tegelevadki. Kas sul ei olnud nii?}

Pidin jõuga maigu suhu saama, sest tegevust oli vaja laiendada ja 
töö tahtis tegemist. Inimesi oli vaja, neid tuli meelitada. 
Küberneetika ASi\index{Küberneetika AS} moodustamisel sai minust selle
arendusdirektor. 

Mõeldi küll, et vaatan laiemat asja ning tegelen ka meremärkide 
ja poidega, aga selle õnge ma ei läinud. Hakkasin arendama infoturbetooteid. 1996. aastal tegime esimese tulemüüri valmis, siis 
VPNi toote ja SSLi \emph{proxy}'sid. 

\question{Kas see oli pärast Meediamaad?}

Jah, see oli hiljem. Infoturbetoodete arendamine läks esialgu väga
hästi. Tegime Linuxi peale veebipõhise liidese 
jubinatele, millest osav insener saab ise tulemüüri teha. Tegime selle veebiliidese kaudu lihtsamaks ja oligi jämedas plaanis 
toode valmis. Eesmärk oli teha keskmisest viis korda odavam toode -- keskmine 
tulemüür maksis tollal kolm tuhat dollarit. Ja tuli välja. 

Ilmselt siin oli seos, sest just 
riigiasutused ostsid meeleldi meie tehtud tooteid. \enquote{Tarvi 
tegi võrgud, nüüd müüb neile turva ka peale.}

\question{Enamasti tekib riikides soov teha omale privaatne turvaline
internet. Kas Eestis seda ei mõeldud või üritati teha ja ei tulnud välja?}

Loomulikult üritati, tegime 
VPNi toote, mis oli võrreldes praegustega unikaalne. Kui kast 
oli võrgul ees, siis ei saanud internetti, see lasi ainult teise omasuguse juurde. 
Näiteks igas maakonnas on kontorid, kus paned rohelise kasti võrgule ette ja kamba peale on üks tulemüür ka, näiteks Tallinnas, ja ainult läbi selle tulemüüri saab 
välja. Muidu on täielikult sisevõrk. 

\question{Sa kirjeldad ju X-teed. Arhitektuuri mõttes tundub 
väga sarnane.}

Ei ole, sellel pole andmete semantikaga mingit pistmist. 

\question{Kas see tähendab, et projektide vahel ei toimunud mingit risttolmlemist?}

Ei, see oli privaattorude ehitamine, X-tee on OSI tasemetes 
natuke kõrgemal.

\question{Kas sa tol ajal tegelesid interneti propageerimisega paralleelselt 
edasi või oli see lihtsalt üks faas?}

Siis oli turul juba piisavalt tegijaid ja ma ei tundnud vajadust 
sellega tegeleda. Pigem oli minu jaoks saabunud järgmine faas teha 
internet turvaliseks. Kolmas elementaarne faas 
oli osapooled internetis identifitseerida, et saaks ka
\emph{business}'it teha. 

\question{Kust tuli mõte, et internet peab turvaline olema?}

Hakkasime teoreetiliselt turvalisusega tegelema juba 
1992. aastal. Kontseptsioon, kuidas ja miks seda 
teha, oli mulle tuttav. Meie roheliste kastide puhul oligi 
eesmärk puhas ja turvaline andmeside, muud midagi. Minu sõnum oli see, 
et ärme teeme eraldi X.25 võrku, sest üle avaliku interneti toimetades on palju 
kuluefektiivsem.

\question{Kuidas sul ikkagi tekkis mõte, et interneti turvalisus on 
probleem, mida tuleb hakata lahendama? Kas keegi luuras või häkkerid kiusasid? Kust 
probleem tekkis?}

Probleem on olnud aegade algusest. Ja olles infoturbega algusest 
peale tegelenud, oli selge, et võrkudes on infoturve teemaks. See on 
elementaarne. 

\question{Kui mina oma ajaloo peale mõtlen, siis minu jaoks ei olnud. Ehitasin pikalt oma asju ja võrke, üldse mõtlemata, et need võiksid ka turvalised 
olla.}

Infoturve oli minu eriala, ükskõik mis 
ametis, ja see sai alguse tolle 
ühe raamatu kooslugemisest.

\question{Lisaks on sul matemaatiku, programmeerija ja antenniehitaja 
taust, nii et saad päris süvitsi minna.}

Jah, ma olen kirjutanud Jukule\index{Juku} püsimälu. 
Minu töö puudutas tähtede joonistamist ekraanile, EEPROMi tasemel 
sai ESC-käskudega aknaid teha. 

\bigskip
\noindent\rule{.3\textwidth}{.7pt}
\bigskip

Mõtlesin, mis lugusid veel võiks rääkida, ja mõned tulid meelde.

Ma ei olnud Tallinna poiss ja Jaak Loondet\index[ppl]{Loonde, Jaak}, keda 
mitmed varasemad rääkijad on maininud, ei tundnud. Küll aga kuulsin temast 
FidoNeti inimestelt. 

Juhtus niisugune lugu, et varajastel üheksakümnendatel, kui 
Eestis ei olnud isegi piisavalt leiba, see oli talongide peal\sidenote{1980ndate
lõpust kuni umbes 1993. aastani, kui vaba turg hakkas enam-vähem 
toimima, müüdi elementaarseid toidu- ja tööstuskaupu, sealhulgas 
periooditi leiba, üksnes talongide esitamisel.}, otsustas 
Soome Rotary klubi Eesti koolidele natuke arvuteid kinkida. Ilmselt oli PC-aeg peale tulnud ja ühel tehaseinimesel jäi komptuureid üle. 
Need olid kummalised masinad, aga lahe oli see, et need olid võrgus ja emaarvuti ka. Soome Rotary tegi haridusministeeriumile
ettepaneku kinkida need Eesti koolidele. 
Minu mentor Peeter Lorents\index[ppl]{Lorents, Peeter} oli sel ajal 
ministeeriumis mingi tegelinski ja sattus selle peale. 
Läksimegi kolmekesi -- autojuht, Peeter ja mina eksperdina -- 
kohapeale vaatama, mis arvutid need on ja kuidas töötavad. Tõime need Eestisse ja siis tekkis küsimus, mida me 
nendega peale hakkame. 

\question{Kui palju neid masinaid oli?}

Kuus-seitse tükki, terve klassitäis. Eesti peale ei olnud palju, aga 
Rotary klubi sai endale linnukese kirja: Eestit aidatud, heategevus tehtud. Ja 
siis meenuski mulle Jaak Loonde\index[ppl]{Loonde, Jaak}. Sain temaga kokku ja Jaak 
oli kohe nõus sellega tegelema, silmad peas põlemas nagu ikka. Mõne aasta pärast saime kokku 
ja küsisin, kas masinatel pruukimist ka oli, ja tuli välja, et need olid väga 
hästi vastu võetud ja nendega igasuguseid vigureid tehtud. 

\question{Nii et Jaak toimetas edasi ka pärast seda, kui 
enamik temast rääkinuid olid koolipoisieast välja kasvanud?}

Jaa, ta oli legendaarne, toimetas arvutitega elu lõpuni. Tema põhiline soov oli, et lapsed saaksid näpud arvuti külge.


\bigskip
\noindent\rule{.3\textwidth}{.7pt}
\bigskip

1993. aastal tegin ma esimese 
jututoa, mille nimi oli Anna\index{Anna jutukas}. See oli umbes samasugune asi nagu praegu Messenger: hulk inimesi logib sisse ja hakkab omavahel suhtlema. 

\question{Kas see käis sinu enda tehtud tarkvara peal või said selle kuskilt?}

Sain kuskilt tarkvara ja tõlkisin käsud eesti keelde, käsk algas 
punktiga. Olin tollal Göteborgis neli kuud asumisel ja mul polnud 
seal suurt midagi teha, nii et putitasingi seda jututuba. 

Anna jututoas kaitsti isegi üks Tallinna 
Tehnikaülikooli\index{Tallinna Tehnikaülikool} diplomitöö ära -- kaitsja asus 
Uus-Meremaal, õppejõud kogunesid jututuppa.

\question{Mis oli jutukate fenomen? Seal käis igasugust rahvast, mitte ainult tehnikud.}

See oli \emph{community building}, umbes samasugune grupp nagu FidoNet. Edasi 
tekkisid OK \index{OK jutukas} ja 
Cafe\index{Cafe jutukas}\sidenote{Cafe pärisnimi oli The Roadkill 
Cafe ja see asus aadressil \texttt{ns.uninet.ee:5555}. Selle pani 23. 
veebruaril 1996 NUTSi (\emph{Neil's Unix Talk Server}) versiooni 2.3 
lähtekoodist püsti Indrek Siitan\index[ppl]{Siitan, Indrek}.} jutukad. Meil oli 
isegi Anna kasutajate kokkutulek Viljandi lähistel, mida 
Jüri Ruut\index[ppl]{Ruut, Jüri} veab siiamaani, nüüd küll ee.kevade nime all.

Jutukates käis suvaline rahvas, seal ei olnud õnneks üksnes tehno{\-}friigid, vaid ka tütarlapsi. 

\question{See pidi siis olema väga vajalik teenus, sest 
mittetehnofriigile pidi see tehnika olema paras barjäär.}

See oli tegelikult lihtne, kui ainult terminalile ligi said. Panid 
\verb|telnet anna.ioc.ee| ja läks. 

\question{Kas sa hoidsid jutukat Küberneetika Instituudis\index{Küberneetika 
Instituut}?}

Pean tunnistama, et jah. Alustasime Küberis Unixi pruukimist aastal 
1992, kui tõime Soomest flopidega Linuxi\index{Linux}. Teistmoodi ei 
saanud seda kätte. 

\question{Kas otse Linuse käest?}

Enam-vähem. Proovisin tollal Unixi kultuuri aretada. Kord ostsime hirmsa
raha eest ühe Suni. Kui küsiti, mis sellele nimeks panna, siis ütlesin suvaliselt 
\enquote{keeks} ja tekkiski igavesti kuulus FTP-server keeks.ioc.ee\index{keeks.ioc.ee}. Pärast pidin \enquote{keeksi} lahti mõtestama ja 
arvasin, et see on Küberi Esimene Eestimeelsete Kasutajate Server.

\question{Tuleme korraks jutukate juurde tagasi. Selleks et sotsiaalvõrk 
lendu läheks, peaks olema algne seltskond. Kes need inimesed olid ja 
kuidas sa selle võrgustiku tekitasid?}

Ma täpselt ei mäleta, aga küllap rääkisin sõpradele, nemad oma 
sõpradele ja nii see vaikselt levis. Ühtegi erilist 
aktsiooni ei mäleta, piisas sõprade ringist, aga lõpuks läks 
ring väga laiaks -- üle poole või rohkemgi olid 
täiesti tundmatud inimesed. 

Annaga\index{Anna jutukas} juhtus nii, et ühel hetkel vaatasin, et 
teised jutukad hakkavad ka tekkima, ning panin selle pidulikult kinni. 
Anna matused olid eraldi sündmus. Asja peab ära lõpetama, mitte laskma 
sel lihtsalt hääbuda. 

Kui Unixi juurde tagasi tulla, siis oli meil 
Eesti Unixi Pruukijate Selts ehk EUPS\sidenote{Selts asutati 1994. aastal ja sellel oli 62 
asutajaliiget. Asutavasse toimkonda kuulusid lisaks Tarvile Andres 
Bauman\index[ppl]{Bauman, Andres}, Margus Liiv\index[ppl]{Liiv, Margus}, Jaanus 
Pöial\index[ppl]{Pöial, Jaanus} ja Anto Veldre\index[ppl]{Veldre, Anto}.}. Teised tahtsid panna \enquote{Kasutajate Selts}, aga EUKS kõlab 
halvasti ja mina ütlesin, et peab ikka pruukima. Meil oli 
Tõraveres isegi kokkutulek.

\question{Miks te Soomest Linuxi\index{Linux} tõite? Kas te ei tahtnud Sunile 
raha anda?}

Ühelt poolt ei tahtnud raha anda ja teiselt poolt oli see uus värske 
tuul, mis oli vaja ära proovida. Linuxi eelis oli see, et see käis 
PC peal. 

\question{Linux on praeguseni hädas oma kõrge sisenemisbarjääriga, inimestel on 
raske sellega liikuma saada. Kuidas toona oli?}

Me rääkisime Linuxist serveri kontekstis, tööjaama-Linux ei olnud teema. 
Tol hetkel pidi raha eest ostma mingi tarkvara, et failiserverit ringi 
ajada. Ma ütlesin, et ärme tee seda! Panen Linuxi püsti, kasutame 
seda. 

\question{Tol ajal taheti igasuguste asjade eest, nagu 
veebiserver, raha saada ja kommertstarkvara oli väga kallis.}

See oli ropult kallis, kuna kirjutajaid oli vähe ja see oli eksklusiivne asi. Kui hakkasime Küberis\index{Küberneetika 
Instituut} 1996. aastal tegema esimesi tulemüüre nimega 
Barrikaad\index{Barrikaad}, siis tol hetkel maksis keskmine tulemüür maailmas 
kolm tuhat dollarit. See on ju absurdne. Me võtsime Linuxi, tegime näo pähe ja 
müüsime viis korda odavamalt.

Seoses kogukondadega ei saa mainimata jätta 
sellist olulist \emph{community}'t nagu WC Fauna\index{WC Fauna}. 
Raske öelda, mis see täpselt oli või kes sinna kuulusid, see oli rohkem 
mõtte- ja eluviis. Selle liikmed tegid igasuguseid asju, pahatihti käisid 
lihtsalt kõrtsides või tegid niisama nalja ja ehitasid lumelinna.

Vanasti olid kompuutrimessid tähtsad.\sidenote{Aastatel 1993--1999 
korraldati Eestis igakevadist arvuti-, side- ja bürootehnika messi 
\enquote{Kompuuter}. Tegu oli olulise kogukondliku ja 
müügiüritusega, mida Päevaleht tituleeris lausa infotehnoloogia laulupeoks.} Ühel messil pakuti meile oma boksi ja pidime selle 
kuidagi sisustama. Boksis oli üks kompuuter, mis luges sekundeid tuleviku 
alguseni, ja WC Fauna leviala kaart, milleks oli punaste läbipaistvate 
vorstinahkadega kaetud Eesti kaart, politseilindiga ümber tõmmatud. 

\question{Tänapäeval läheks selline asi kunstiprojektina kirja.}

Jah, ilmselt küll. Eks see oli häppening, igasuguseid erinevaid asju sai tehtud. Näiteks oli 
IT-inimeste kokkutulek 
OK-fest\index{OK-fest}\sidenote{1994. aastast Eesti Infotehnoloogia- ja 
Telekommunikatsiooniettevõtjate Liidu\index{Eesti Infotehnoloogia- ja 
Telekommunikatsiooniettevõtjate Liit} korraldatud suvine kokkutulek.}, 
kus \emph{community} kokku sai. WC Fauna nimi sai alguse sellest, et ühel 
OK-festil oli vaja jalgpallimeeskond kokku panna. Mõtlesime, et FC Flora juba 
on, paneme siis WC Fauna. Aga see oli ka vist viimane kord, kui jalgpalli 
mängisime. 

\question{Sinu jutust kumab läbi palju 
ühistegevust, aga tavaliselt ei tegeleta arvutitega sellepärast, et 
meeldib teiste inimestega suhelda. Kuidas sul arvutite ja inimeste suhe 
kokku käib?}

Ma olengi imelik loom, kellest pole kunagi aru saadud. Üks tuttav 
psühholoog ütles: \enquote{On olemas insenerid ja on olemas kunstiteadlased, 
aga kumb sina oled, aru ei saa.} 

Inimene areneb vaikselt. Nagu ma mainisin, siis algusaegadel olin 
introvert, kes istus nurgas ja programmeeris ning kartis, kui telefon 
helises. Hiljem hakkasin inimestega suhtlema, seejärel ühiskonda nägema ja sealt tulid ka riigi- ja 
vaat et maailmalaiused asjad. 

\phantomsection\label{sisu:everyday}Üks lugu, milles maksab kindlasti rääkida, on see, kust Skype\index{Skype} tegelikult alguse sai ja kus see kamp 
kogunes. Ilmselt nii mõnigi mäletab, et umbes 1994. või 
1995. aastal oli lehes kuulutus \enquote{otsime programmeerijat, maksame 
viis tuhat krooni päevas}\sidenote{Teiste allikate alusel oli kuulutus lehes 
1999. aastal, mis on loogilisem -- muidu jääb Skype'i asutamise ja 
Bluemooni Tele2-seikluse vahele liiga pikk paus.}. Viis tuhat krooni oli kaks kuupalka. 
Kuulutuse tagamaa oli see, et Tele2\index{Tele2}, kes oli juba Eestis olemas, ja 
Bonnier Media sepitsesid Rootsis 
nii-öelda uue põlvkonna portaali
everyday.com\index{everyday.com}. Niipea kui nad uudise välja lasid, et 
niisugune portaal tuleb, tõusis nende turuväärtus poolteist miljardit. 
Absurdne, aga nii see oli. Eestisse tuldi jutuga, et meil on tiimid Itaalias, Rootsis ja Taanis ning kõik on juba tükk aega programmeerinud. Kahte 
programmeerijat Eestist on veel vaja, siis saab kõik korda.\sidenote{Eestis 
töötas toona Tele2s Stefan Öberg\index[ppl]{Öberg, Stefan}, kes hiljem täitis Skype'is 
mitmeid juhtivaid rolle. Tema juhataski viimase kahe tegija otsijad 
Eestisse.} 

Mina sattusin seda otsingut nõustama ja lõpuks projektijuhiks, kes 
pidi need inimesed välja valima ja asjad ära tegema. Valisin välja 
Bluemooni\index{Bluemoon} poisid. Sõitsin kõik need Itaalia, 
Taani ja Rootsi kontorid läbi ning sain aru, et peale Rootsi, kus oli tehtud väike 
andmebaasimootor, olid kõik teised tiimid tootnud täielikku kräppi. Nii ei 
jäänudki projekti päästmiseks muud üle, kui kogu värk ise teha. Bluemooni
poistel ei olnud probleem see käsile võtta ja nädala-paariga 
portaal kokku veeretada, kuigi nad PHPd\index{PHP} ei tundnud.

Tulevane miljardär\sidenote{Tarvi peab silmas Niklas 
Zennströmi\index[ppl]{Zennström, Niklas}.} oli Tele2s projektijuht ja talle 
hakkasid need poisid meeldima. 

\question{Sina olid portaalis projektijuht. Kui tegid portaali valmis, kas siis 
ei tekkinud mõtet, et peaks suures Rootsi kontsernis kosmilist karjääri tegema?}

Absoluutselt mitte, see oli kõrvaltegevus -- aitamisprojekt ja raha 
maksti ka.

\question{Mis su põhitegevus oli?}

Ehitasin riigivõrku ja juhatasin neid 
vägesid. Sinna kõrvale mahtus veel üks kõrvaltegevus, 
mail.ee\index{mail.ee}, mille omanikuks sai ka lõpuks Tele2. 

\question{Kas mail.ee all oli standardne SMTP-server?}

Täpselt nii. Alustuseks oli ilma näota 
meilboks. See tähendas, et igaüks sai endale aadressi luua, aga pidi enda 
meilerit kasutama. Teine arengufaas oli sellele veebi nägu pähe teha, seal oli veebimeiler ka. See sai täitsa ise kirjutatud, all 
oli loomulikult standardne kompott. 

\question{Nii et sa ei läinud ise sinna maailma midagi leiutama, vaid võtsid 
tükid ja ladusid kokku?}

Jaa, see on mul kogu aeg veres olnud. Ühel hetkel sain aru, et 
programmeerimine on üldse kurjast, sest kõik on juba ära tehtud. 
Tegelikult on kunst tükid üles leida ja oskuslikult kokku panna. 
Tänapäeval on tükkide arv muutunud hoomamatuks ja väga raske on neist midagi kokku panna. Ilmselt on 
tekkinud kildkonnad ja voolud. Kunst on muutunud.

\question{Mis üldse on tänapäeval sinu jaoks programmeerimine?}

See kipub olema järjest igavam asi, sest vanasti oli 
see selgelt loometöö. Nii kui hakkasid tulema igasugused 
mudelid ja RUPid\sidenote{\emph{Rational Unified Process} (RUP). RUP oli 
1990ndatel suurorganisatsioonides levinud tarkvaraarenduse raamistik, 
mis keskendus arendusprotsessi keerukuse vähendamisele läbi standardiseeritud 
rutiinide. Et samal ajal üritati keerukat tarkvara tarnida harva ja suure 
pauguga, võis RUP küll teha projektid paremini kontrollitavaks, kuid ei vähendanud kuigivõrd arendajate frustratsiooni.}, siis hakkas see
järjest rohkem tunduma kraavikaevamisena. Arhitektid joonistavad asja ette ja sina lihtsalt täidad 
funktsiooni. See ei ole eriti keeruline. 

\question{Ometigi ehitatakse igasuguseid hullusi, nagu tekstiterminalis 
video mahamängimine.}

Loomulikult, nalja pärast saab ikka teha. Ma räägin raha 
eest või tööstuslikust programmeerimisest, kus tuleb konkreetset asja teha. 
Vanasti olid mees nagu orkester ja mõtlesid ise välja, kuidas arhitektuur 
võiks välja näha. Tegid oma äranägemise järgi ja keegi ei kobisenud. Nüüd 
on arhitektid. Loovust on 
programmeerijatele jäänud kindlasti vähemaks. 

\question{Kui me juba selle teema juurde jõudsime, siis küsin ka sinu käest, 
milline on ilus kood?}

Ilus kood on loetav kood, siin ei ole kahtepidi mõtlemist. 

\bigskip
\noindent\rule{.3\textwidth}{.7pt}
\bigskip

\question{Kuidas sündis ID-kaart?}

Küberis\index{Küber} tegutsesin ma kahel rindel. Ühelt poolt ehitasin võrke, 
aga olin ka kogu aeg infoturbe ja krüptograafia keskel. Lisaks 
võrguturbele, mis oli sel ajal väga oluline, tundus avaliku võtme 
krüptograafia huvitav ala ja pakkus oma rakenduste poolel pinget. 
Küberis sai jälgitud, kuidas 1995. aastal vist Rootsi Post alustas oma 
ID-kaardi väljalaskmisega ja avaldas ID-kaardi profiili. Päris vara, 
üheksakümnendatel, toodi mulle Ektacosse\index{Ektaco} Schlumbergeri 
kiipkaardid ja paluti vaadata, mis elukad need on. 
Kirjutasin sinna peale programmi nimega \emph{Clevercard}.

\question{Kas see oli Java kaart?}

Javat polnud veel väljagi mõeldud, 
krüptokaarte ka mitte. Mälukaart see ei olnud, protsessor 
oli sees. Sellele kiipkaardile sai käske anda, näiteks \enquote{tee fail}. Kõige all oli 
kaardi operatsioonisüsteem. Baidid ajasid sisse, baidid tulid vastu ja ma kirjutasin 
PC-le programmi, millega seda sai mõnusalt teha. 

Aeg läks vaikselt edasi ja see oli umbes 1996.
aastal, kui tegin Äripäeva lahti ja esimesel leheküljel oli pildil Kaja 
Kuivjõgi\index[ppl]{Kuivjõgi, Kaja}, keda ma tundsin ja kes 
oli siis Kodakondsus- ja Migratsiooniameti\index{Kodakondsus- ja 
Migratsiooniamet} asedirektor. Pildi juures oli kirjas, et riik 
planeerib uut dokumenti ja et esimesed passid, mis võeti kasutusele 1992. aastal, saavad 2002. aastal läbi. 
Sinna on viis aastat aega ja KMAs on moodustatud töörühm, kes 
uurib variante millegi uuega välja tulla. 

Võtsin Kajaga ühendust ja ta 
näitas mulle töörühmas arutatud materjale. Kui olin need 
läbi vaadanud, sain aru, et nende tehniline teadmus on üsna allpool 
nulli. Seal räägiti kiibiga varustatud vöötkoodidest. 

\question{Mida nad teha tahtsid? Uut ja paremat passi?}

Nad mõtlesid ikkagi kaardi suunas, aga milline see võiks olla -- 
kas kiibiga varustatud vöötkood või mis -- ei olnud selge. 

\question{Mina olen kogu aeg arvanud, et kaardi pakkusid
välja tehnikud, mitte ametnikud.}

Soov oli tol hetkel väga hägune ja igasugused 
variandid olid laual. Aga oli selge, et kuna tekib suurem 
passivahetus, siis on võimalik inimesi üllatada millegi uuega ning vaadata, mis 
maailmas tehnoloogia vallas toimub. 

Oli päris selge, et KMA\index{Kodakondsus- ja Migratsiooniamet} 
töörühmal ei ole mõtet jätkata. Tehti ettepanek moodustada 
laiem töörühm ja võtta laua taha ka eksperte: pangad, 
telekomid, riigisektori ja Küberi\index{Küber} inimesed.

\question{Kas tänapäeval tundub veider, et riik võtab pangad ja 
telekomid laua taha sellist dokumenti arutama?} 

Absoluutselt mitte. Ei tundu praegu ega tundunud ka tol ajal. 
Laiapõhjaline koostöö riigi- ja erasektori vahel on meile alati edu 
toonud nii ühes kui ka teises. 

\question{See on haruldane asi, mida mujal sageli ei näe.}

Eesti on nii väike riik, et põhimõtteliselt tead kõiki, kes midagi teavad, ja 
ei ole mõtet kedagi kõrvale jätta sellepärast, et ta on parasjagu erasektoris. Me räägime ikkagi eksperditeadmisest ja 
ekspertide kogumist, mitte institutsionaalsest asjast. 

Tuligi töörühm kokku ja arutas asju. Telliti kaks tööd, 
KMA\index{Kodakondsus- ja Migratsiooniamet} maksis. Ühe töö viis läbi 
aktsiaselts Aprote\index{Aprote}, kes uuris, milleks kõigeks 
võiks seda kaarti kasutada. Nad läksid näiteks tanklaketti ja küsisid, 
mida nemad tahaksid. Tulemus oli muidugi väga ulmeline, aga turuootuste uurimine oli 
vajalik tegevus, vaat et kohustuslik samm. 
Teine töö, mida tegime meie Küberis\index{Küber}, oli 
tehnoloogiline ülevaade, milleks kiipkaardid on suutelised, kaasa
arvatud see, mida on Rootsis ja Soomes tehtud. 
Millised on profiilid ja tehnoloogiad, sealhulgas Microsofti 
PC/SC\sidenote{\emph{Personal Computer/Smart Card} -- spetsifikatsioon 
tarkade kaartide integratsiooniks arvutustehnikaga.}. 

1996. aastal joonistasin projektiplaani, et neljateist kuuga 
toome kaardi välja, kaasa arvatud pilootprojekt ja muu säärane. Võttis see siiski viis 
aastat, sest see oli väga oluline samm ühiskonnas 
ja vajas pikemat kaalumist. Peale selle tuli seadusi juurde ja ringi teha. 

\question{Kas kõike seda vedas KMA?}\index{Kodakondsus- ja Migratsiooniamet}

Ei, kindlasti mitte. Digiallkirja seadust näiteks vedas 
majandusministeerium\index{Majandusministeerium}.

\question{Kuidas nii? Asi ju algas 
dokumendi väljastamise vajadusest ja siis äkki tahtis majandusministeerium digiallkirja 
teha?}

Kindlasti oli suunanäitajaks Saksamaa, kes võttis esimesena vastu 
digiallkirja seaduse, mille pealt Eesti oma on paljuski maha viksitud. Meie 
digiallkirja seadus või vähemalt selle kavand nägi ilmavalgust enne, kui 
oli olemas Euroopa 1999. aasta direktiiv\sidenote{Euroopa Parlamendi ja nõukogu 
direktiiv 1999/93/EÜ.}. Seetõttu oli meie seadus mõnevõrra erinev. Euroopa 
direktiiv lubas igasuguseid lahjasid allkirju ja sellist koledust nagu 
näpuga ekraanile kirjutamist ning ei läinudki tööle. Seepärast tuli ka 
lõpuks eIDAS\sidenote{\emph{electronic IDentification, Authentication and trust 
Services - eIDAS} -- Euroopa Parlamendi ja nõukogu määrus 910/2014 e-identimise ja e-tehingute kohta.}, et direktiiv 
oli väga lahja. Kehitati õlgu ja ei kasutatud, tehti lahjasid allkirju ja 
öeldi, et nüüd ongi kõik hästi. 

Meie seadus ütles algusest peale, et ainult 
kvalifitseeritud allkirjad\sidenote{Lihtsalt öeldes on kvalifitseeritud 
elektrooniline allkiri selline allkiri, mida võib pidada võrdväärseks 
omakäelise allkirjaga. Keerulisemalt on öeldud eelviidatud eIDASi direktiivis ja 
selle rakendusaktides.} on aktsepteeritud, ja mingeid lahjasid allkirju ei 
tunnistatud. Neid seadus ei käsitlenudki. 

\question{Kas seaduse väljatöötamises osalesid ka eksperdid või oli 
see majandusministeeriumi tehtud?}

Eksperdid olid kaasatud. Oli töörühm, kus osalesid
inimesed krüptoloogist kuni juurateadlasteni. Nad tegid seda tööd ligikaudu
kaks aastat, nii et see ei tekkinud niisama, vaid mõeldi väga põhjalikult 
läbi. Kuskilt mujalt kui Saksamaalt ei olnud šnitti võtta. Nii mõnigi 
seadusepunkt oli inspireeritud nii-öelda krüptograafide mõtlemisest. 

\question{Sinu jutust ei kõla läbi kõikehõlmav õilis visioon 
sellest, kuidas ühel päeval sünnib Eesti digiühiskond ja kõik saab e-teenuste 
abil uueks loodud.}

Eks see võibolla kuskil ajusopis oli, aga mis sellest ikka rääkida, asju 
tuleb teha. 

\question{Seda ma peangi silmas, et liikumine toimus samm-sammult ja tegeldi 
konkreetsete asjadega.}

Jah, kasvõi seesama ID-kaardi väljatoomine. Võib ju digiallkirja seaduse vastu 
võtta (mis aastal 2000 ka vastu võeti), aga kui inimestel ei ole vahendit, 
millega digiallkirja anda, siis pole seadusel suuremat mõtet. 
Euroopas valitses ka selle direktiivi tegemise ajal nägemus, et 
kommertsfirmad hakkavad sertifikaati müüma ja seetõttu on vaja neid 
reguleerida. Kuidas see võiks käia? Teed turule putka ja hakkad sertifikaate 
müüma: suured ja väikesed sertifikaadid, punased ja kollased? 

See ettenägemisvõime oli meil küll, et niisugune visioon, et müüme inimestele 
sertifikaate, neid ostetakse ja kuidagi tekib kasutus, on üdini vale. Selles mõttes oli näiteks Soome, kes tegi ID-kaardi 
mittekohustuslikuks ja pani kohe hinnaks nelikümmend eurot. Siis juhtub see, et teenusepakkujad ei hakka ID-kaarti toetama, sest 
nad teavad, et inimestel ei ole seda (viiel protsendil võibolla on). Ja inimestel ei ole kaarti, sest teenuseid ei 
ole. Siis ongi nokk kinni, saba kinni ja mudel, mis ei toimi. 

Kõigepealt peab elektroonilise identiteedi 
taristu looma ja siis võibolla hakkavad asjad juhtuma. Samamoodi nagu ei 
saa proovida kuskil metsa sees müüa kilomeetrit maanteed kohalikule 
metsaelanikule. 

\question{Su jutust kõlab läbi üsna suur usaldus ekspertide vastu. Poliitika eest vastutava inimese ja krüptoloogi 
vahel pidi olema usalduslik vahekord, et viimasel lasti seadusesse punkte kirjutada.}

Skepsis on väga raske tekkima, kui laua taga on Eesti paremad pead -- misasja sa ikka kahtled või kõhkled. Mida targemaks inimesed saavad ja 
mida rohkem eksperte on, seda rohkem tekib diskussiooni. 

\question{Kui suur see ekspertide ring oli, kes töörühmades käis ja seda 
ideed kujundas?}

Viis kuni kümme võtmeinimest.

\question{Ja kogu tarkus tugines tollele salapärasele raamatule, mis 
Küberis oli?}

Oo ei, see raamat oli lihtsalt algus. Me ei räägi ainult 
krüptoloogiast või infoturbest. Näiteks ID-kaart ei puuduta ainult 
infoturvet, vaid väga paljusid rakenduslikke ja isegi sotsiaalseid aspekte. Ei saa rääkida, et krüptograafia päästis maailma. 

\question{Sagedasti inimesed arvavad, et kui asjad saaks 
ära krüptida, siis olekski maailm päästetud.}

Ma jään selle juurde, et ei saa kilomeetrit maanteed müüa 
külaelanikule, sest ta küsib: \enquote{Mis ma teen selle maanteega?} -- 
\enquote{Hakkad autoga sõitma.} -- \enquote{Aga mis see auto on?} See on 
ufo müümine, täiesti mõttetu tegevus. Sa ehitad teed valmis, lased autod müüki, 
paned sõidukoolid püsti ja siis ühel päeval võibolla inimesed avastavad, et 
transpordist on kasu. Aga kui hakkad sellest pihta, et proovid 
igaühele juppi maanteed müüa, siis see ei toimi. 

\question{Küberneetika Instituut\index{Küberneetika 
Instituut} ja selle järelmid on väga pikalt Eestis olulist rolli mänginud. 
Sina oled olnud seal sees ja, mis veel olulisem, ka sellest väljas. Kas sa oskad 
öelda, mis maagiline asi selle asutuse nii võimsaks teeb?}

Nagu ma mainisin, siis 1997. aastal jagunes Küberneetika Instituut kolmeks. Osa 
läks ülikooli alla, osast moodustus aktsiaselts ja andmeside osa läks riigile. 
Võibolla kõige nähtavam osa IT-inimeste jaoks ongi Küberi instituudi või 
aktsiaseltsi ehk infoturbe ja programmeerimise osa. Seal tehakse meremärke 
ka, aga need on merel ja ei paista välja.

Mul on olnud au omal ajal umbes kolmkümmend inimest sinna tööle võtta ja 
Tartu labor\index{Cybernetica!Andmeturbelabor} asutada, mis on nüüd inimeste arvu mõttes  isegi suurem kui Tallinna kontor. Need olid väga toredad ajad. Aga fenomen seisneb selles, et Küberit peetakse põhimõtteliselt ainukeseks firmaks 
Eestis, kes oskab turvaliselt programmeerida ja teab midagi infoturbest. Seetõttu 
on neile ka sattunud niisugused tegevused ja projektid alates X-teest ja 
lõpetades Smart-IDga\index{SplitKey}, kus turvalisuse ja krüptograafia komponent on omal 
kohal. 

Lisaks on seal tõsiseid inimesi, kes tegelevad puhtalt teadusega ja koodi ei 
kirjuta. Küberis on oma teadusosakond ja teadusdirektor. Ühtlasi käivad 
teadusega tegelevad inimesed Küberi ja ülikoolide vahet. Sellist sümbioosi 
otsitakse nagu spunki mööda Eestit taga ja ka Teaduste 
Akadeemia president ei väsi rääkimast, et Küber on fenomen ja suur erand. 

\question{Miks ei ole näiteks Helmes võtnud endale teadusdirektorit 
tööle ja hakanud sama tegema?}

Asi on selles, kas teed kõigepealt teadust ja siis hakkad seda 
rakendama ühiskonnas või proovid vastupidi teha: kõigepealt oled kõva
programmeerija ja siis mõtled, et teeks teadust ka kuidagi. Päris 
nii see ei käi, need juured on natuke sügavamal.

Juhtusin hiljuti nägema Küberi töökuulutust: otsitakse 
projektijuhti, nõutav CISA\sidenote{\emph{Certified Information Systems Auditor} -- sertifitseeritud infosüsteemide audiitor.} sertifikaat. 
Halleluuja!

Omal ajal kõik teadsid, et Martens tuleb jälle jaurama ja Küberisse tööle 
meelitama. Mul oli väga lihtne äriidee: ajan kõige targemad 
inimesed ühte suurde ruumi ja annan neile teema kätte. Nad asuvad 
plaksti tööle, ise panen jalad seina peal. Töötas! Väga hästi töötas! Pärast 
lugesin kuskilt raamatust, et niimoodi tuleb käituda, ja täpselt nii ma 
olengi käitunud.

%--------------------------


\question{Kas sa oled siis asjade käimalükkaja ja visionäär?}

Kui sa nii ütled.

\question{Ma ei ütle, ma küsin. Sa ütlesid, et programmeerija sa enam ei ole. Kes 
sa niisugune oled?}

Ma ei oska ennast sildistada. Mul on see häda küljes, et mõtlen kogu aeg kuidagi 
laiemalt.

\question{Miks see häda on?}

Võiks ju midagi näpu vahel teha, kaltsuvaiba või midagi. On vähemalt füüsiline tükk taga, 
suurtest sõnadest ei jää midagi\ldots

\question{Mida sa praegu teed?}

Olen endiselt elektroonilise hääletuse juht, juba aastast 2003. 
Hiljuti olid meil kümnendad valimised, kohe on algamas üheteistkümnendad, 
europarlamendi valimised\sidenote{Jutuajamine Tarviga leidis aset 2019. aasta mai algul, europarlamendi valimised toimusid 26. mail, edukas 
elektrooniline hääletamine 16.--22. mail.}. Aga valimised võtavad võibolla 
kaks-kolm kuud tähelepanu. Valimistevahelisel ajal ma palju 
suurt ei teegi. Jõudumööda, nii nagu kutsutakse, käin maailmas ringi ja proovin 
inimesi aidata nende arengus erinevates riikides -- nii 
elektroonilise identiteedi teemal kui ka IKT rakendamise alal 
nii-öelda valimismajanduses.

