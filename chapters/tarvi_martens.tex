\index[ppl]{Martens, Tarvi}\sidenote{Kuna Tarviga rääkisime juttu mitmel 
korral, on jutulõng mõnevõrra hüplik. Katkemiskohad on tekstis markeeritud.}

\question{Kuidas sina said arvutite juurde ja arvutid sinu juurde?}

Ma olen pärit tegelikult Pärnust ja seal nagu arvuteid ei olnud minu arust. Aga 
ma olin sihuke olümpiaadidel käia, et matemaatika ei olnud minu jaoks mingi 
mingi teema, eks ole. Hakkas kunagi sealt peale umbes, kus ma viiendas klassis 
võitsin kuuenda klassi matemaaatika linnaolümpiaadi ära. Kõik olid suhteliselt 
jahtunud selle peale. Mingi riikliku olümpiaadi käigus veeti meid 
ekskursioonile Nõo Keskkooli\index{Koolid!Nõo Keskkool}, seal oli suur arvuti 
olemas. See oli  kuidagi nagu teistsugune maailm aga kui mind sinna õppima 
taheti tassida, ma ei tahtnud enam väga minna. Mul oli juba oma bänd ja nii.

\question{Sul oli oma bänd?}

Jah. Tegime punki, nagu ikka sel ajal. Ma käisin muusikakallakuga koolis, oli 
elementaarne, et sul on bänd. Kooliteater ka tegi oma esimesi samme, see 
algaski Pärnust. Kadunud Aare Laanemets\index[ppl]{Laanemets, Aare} ja Elmar 
Trink\index[ppl]{Trink, Elmar} tegid  esimest kooliteatrit, osalesin seal ka. 
Kõik see oli nii tore ja ma mõtlesin, et mina ei viitsi küll kuhugi kaugele 
kooli minna. Aga matemaatikaõpetaja käis mu vanemate juures ja rääkis nad 
tursaks ja nii läkski. 

\question{Kas sel ajal Nõo legend alles kujunes või oli see juba tuntud paik?}

Jah, oli kindlasti tuntud. Oli teisi tugevaid koole ka siin-seal, eks ole, 
Tartus-Tallinnas,  aga põhimõtteliselt Nõo kool oli üle kõige. Põhiliselt 
sellepärast, et neile oli oma arvutuskeskus ehitatud, sinna tuldi ikka üle 
vabariigi kokku. Kuigi, peab ütlema, et enamus olid niisugused ümberkaudsed 
inimesed, kes ei olnud võib-olla väga suured geeniused,  pigem pigem 
maa-lapsed. 

Nõo Keskkoolis oli meil  Nairi 3-1\index{Arvutid!Nairi!Nairi-3-1}, niisugune 
\emph{mainframe}, kus sai perfolinti sisse sööta ja said laiprinterist oma 
tulemuse kätte. Aga see ei olnud nagu väga tore. Ma ei suuda meenutada, mis 
ajenditel ma leidsin Tartust ülikooli Vanemuise õppehoonest\index{Tartu 
Ülikool!Vanemuise tänava õppehoone}  keldrikorruselt üles kabineti, kus oli 
nii-öelda 2.5 Apple II-te\index{Arvutid!Apple II}. Kaks koma viis selle pärast, 
et üks oli kogu aeg katki ja Andres Peiker\index[ppl]{Peiker, Andres} (kes oli 
selle keldri kunn) remontis teda.

Mina olin üheksanda klassi poiss ja konkureerisin  arvutiaja pärast tõeliste 
üliõpilastega nagu Tanel Tammet\index[ppl]{Tammet, Tanel}, Margus 
Liiv\index[ppl]{Liiv, Margus} ja niisugused mehed. Aga eks ma sain sinna ennast 
vahele pista ka ja siis oligi, et ma ikkagi enamus ajast ei käinud väga palju 
koolis, vaid olin rohkem seal Tartus.

\question{Ometigi oli Nõo kool disainitud selleks, et sinusuguseid süvendatult 
harida. Oli sul siis vaja veel sügavamale minna?}

No aga mis sa seal Nairi juures perfolindiga harid. Saatuse vingerpuss oli see, 
et aasta hiljem, ehk siis kümnendas klassis, saabus Nõo kooli hunnik 
Agat-e\index{Arvutid!Agat}, mis olid siis põhimõtteliselt Apple II  kloonid,  
ainult värvilised. Kõige naljakam oli see, et  kohalikud arvutiõpetajaid  ei 
teadnud nendest ju midagi. Ja siis tuli välja, et on üks Tarvi, kes tunneb seda 
protsessorit  läbi ja lõhki. Seal oli küll oma operatsioonisüsteem ja omad 
venekeelsed programmeerimiskeeled, aga vahet polnud, on ju. Nii et ühel hetkel 
oli mul arvutuskeskuses oma  kabinet ja oma arvuti. 

\question{Seda siis puhtalt Tartus Apple II uurimise pealt? Said seal sahibide 
vahel noka piisavalt märjaks, et Nõos kunn olla?}

Täpselt nii, õpetasin õpetajaid pärast. 

\question{Agat oli  Apple II kloon kuni riistavara disaini ja arhitektuurini 
välja?}

Ta vähemalt protsessori mõttes oli kindlasti sama. Mina väga suur riistvara 
inimene ei ole,  kuigi ma assembleris\index{Keeled!Assembler} programmeerisin 
vabalt sel ajal. Ma arvan, et ta oli ikka üsna täpne kloon, aga, veel kord, ta 
oli värviline võrreldes Apple II-ga. See tähendab, et pilt virvendas kogu aeg 
silme ees.  

\question{Kui omale kabineti said, oli vist päris uhke tunne?}

Ei no mis seal ikka,  tasapisi. Sai  oma asja ajada, ei pidanud Tartu vahet 
enam käima, see oli hea.

\question{Õppimist ei hakanud segama?}

Ei hakanud. Mul ei ole sellega kunagi mingeid probleeme olnud. Tuleb käia 
lihtsalt oma kontrolltööd ja eksamid ära teha ja siis keegi ei mölise.

Nairi peal olid tõsiste inimeste keeled nagu Algol\index{Keeled!Algol}. Aga 
lastele õpetati programmeerimiskeeli nimega on ROPS\index{Keeled!ROPS} ja 
KÕPS\index{Keeled!KÕPS}\sidenote{Vaata ka märkus leheküljel 
\pageref{sidenote:ROPS}.}. Need olid eestikeelsed. KÕPS oli minu arust selline, 
kus sai programmeerida, ütleme siis joonistamist. Näiteks, kuidas plotter 
liigub: mine üles, mine alla, mine paremale; jäta joon, ära jäta.  ROPS oli 
päris programmeerimiskeel. Ma tegin need keeled ka Agat-i peale ringi, siis ei 
pidanud lapsed Nairi-ga tegelema. 

\question{Siin tundub meie jutus auk olevat. Et matemaaika tuli sul lihtsasti, 
seda ma mõistan. Aga kuidas see matemaatikahuvi läks niisuguseks arvutihuviks 
üle, et sa Nõost käisid Tartus arvutis ja portisid programmeerimiskeeli? Mis 
sind selle asja juures niimoodi tõmbas?}

See on hea küsimus, aga mul ei ole head vastust. Selgelt oli ta midagi täiesti 
teistmoodi, praktiline matemaatika kui tahad. Praktiline niisugune rehkendus 
massin, mis on  kalkulaatorist ikka  natukene intelligentsem. Ja noh, ma vist 
juba sel hetkel mõtlesin, et see on paratamatu tulevik. Et teistmoodi see  
lihtsalt kainelt mõeldes ei saagi olla. 

\question{Huvitav, et sul on matemaatika ja arvutite seos algusest peale selge 
olnud. Mõnel tuleb see seos palju hiljem, kui üldse.}

Matemaatiline loogika on üks  minu lemmikdistsipliine kogu aeg olnud, arvutid  
ja muusika  on väga loogilised asjad. 

Ja siis üks hetk ma lõpetasin kooli ära läksin TPI-sse\index{Tallinna 
Tehnikaülikool}.

\question{Aga miks sinna? Tartu Ülikool oli ju juba tuttav?}

Mulle tundus, et TPI oli natukene praktilisema hoiakuga ja, ütleme, aastal 
1987, oli Tartu Ülikooli informaatika  sihukene distsipliin, millest räägiti, 
et rohkem joonistatakse ikka tahvli peale. Ja päris matemaatikuks ma kindlasti 
ei tahtnud saada.

Tegelikult ma olin Tallinna vahet enne käinud, oli selline üritus nagu Õpilaste 
Teaduslik Ühing\index{Õpilaste Teaduslik Ühing} ja seal  Peeter 
Lorents\index[ppl]{Lorents, Peeter} tegi matemaatika sektsiooni. Ma käisin 
Lorentsi juures aeg-ajalt,  ta andis mulle niisuguseid kaelamurdvaid 
ülesandeid. See oli ka nagu huvitav,  niisugune kahekordsete integraalidega 
elu. 

Nii et kuidagi loogiline tundus sinna TPI-sse minna.

\question{Mida sa õppima läksid?}

Ega ma täpselt ei oska öelda. No eks ta  Automaatikateaduskond oli ja eriala 
LI\index{Tallinna Tehnikaülikool!Automaatikateaduskond!LI}. Arvutid ja 
arvutitehnika, midagi niisugust. Ja seal juhtus kohe  mitu asja. 

Kõigepealt ma ütlesin esimeses programmeerimise tunnis, et siia tundi ma rohkem 
ei tule. Mille peale õppejõud ei solvunud, sest ma sissejuhatavas tunnist 
kirjutasin mingi proge valmis salaja ja näitasin seda talle.

Teine asi oli see, et  hiljuti oli Teaduste Akadeemia Küberneetika Instituudi 
Erikonstrueerimisbüroo\index{EKTA} Juhtimissüsteemide Osakonnas\index{Teaduste Akadeemia 
Küberneetika Instituut|see{Küberneetika Instituut}}\index{Küberneetika 
Instituut!Juhtimissüsteemide Osakond}\sidenote{Esineb ka nimekuju \enquote{Arvutustehnika Erikonstrueerimisbüroo} ja
 \enquote{Arvutustehnika Arendusbüroo}, mis kõik tunduvad viitavat samale asutusele.} 
leiutatud kooliarvuti Juku\index{Arvutid!Juku}. Nad asusid seal 
samas Küberneetika majas, kus ma juba olin käinud. Ja põhimõtteliselt 
septemberi esimesel nädalal ma sadasin sinna sisse. Mul jäi nagu mure nende 
õpilaste keskkondade pärast. Et kui tuleb kooliarvuti, siis võiks olla ka  
õpilastele mõeldud programmeerimiskeeled ja ROPS-i\index{Keeled!ROPS} portimine 
Jukule oli tegemata. Rääkisin Juku tegijatele, et oleks vaja  vastavasuunalisi 
arendusi teha. Nad lubasid mind enda juures hängida ja kuskil nelja kuu pärast 
olin tööl, kuidagi nii ta läks. 

\question{Ja ülikool jäi kõrvale?}

Ei no miks ta kõrvale jäi, ma käisin ikka korralikult eksameid tegemas. 
Vahepeal, pärast esimest kursust, käisin vene kroonus ka.  Ma olin viimane 
lend, kes sai minna, ja olen väga õnnelik. Viidi Leningradi lähistele, aga olin 
seal puhkpilliorkestris ja tegelikult tegin bändi jälle ja polnud häda midagi. 
Jälle üks kogemus juures. 

Kroonust tulles paljud langevad ülikoolist välja, sest nad leiavad, et võiks 
midagi praktilist teha ja see ennast targaks ajamine ei tasu ära. Eks mul ka 
oli teise kursuse poole peal ka mingi kriis, kus ma mõtlesin, et on kohal 
käimata ja et kui ma neid eksameid nüüd ära ei tee, siis on kõik. Aga tegin 
eksamid ära ja  võtsingi selle elustiili, et ma pühendasin ülikoolile umbes 
kolm nädalat poole aasta kohta. Imesin materjali sisse, tegin eksamid ära ja 
kõik nagu täiesti töötas. 

\question{Minu puhul oli küll nii, et keskkool möödus mängides ja lauldes, sest 
kõik oli lihtne. Aga ülikooli minnes lõppes kõik lihtsus ära. Sinul ei 
lõppenud?}

Lihtsus tõesti lõppes. Õigemini, keerukus oli esimene poolteist või kaks  
aastat,  kui nad sulle fundamentaalset kõrgemat füüsikat ja matemaatikat 
taovad, mis lihtsalt lööb kaane pealt ära. Aga edasi läheb natuke erialasemaks 
ja inimlikumaks, see asi ei ole enam nii teoreetiliselt tappev, on nagu 
natukene lihtsam. 

\question{Ülejäänud aja tegelesid Jukudega?}

Ei, mis Jukudega. Kui ma kroonust tulin, oli juba esimene  286  kontorisse 
toodud ja edasi läks juba sedapidi. Olid niisugust huvitavad ajad, et käisid  
tööl küll aga tööd oli  vähe, on ju.  Ja oli väga soositud suhtumine, kui sa 
leidsid endale haltuuraotsi. See oli täiesti okei. Kõige jämedam haltuura ots, 
mida mäletan, oli see, et tuldi koos arvutiga. Sain personaalse arvuti, 
tööandja eraldas kabineti, väga okei. 

\question{Kes need haltuura pakkujad olid? Oskad sa mõne näite tuua?}

Igasugused. Antud juhtumil me räägime Soome laevaehitajast, kes arvutiga tuli. 
Ma pean seda siiamaani kõige vingemaks programmiks mis ma teinud olen. Ülesanne 
oli selline, et kujuta ette, et sul on sõjalaev. Kümme tekki. Me räägime 
elektrivarustusest, kuskil on jõuallikad ja kuskil on tarbijad. Ja nüüd me 
hakkame nende asjade vahele erineva jämedusega kaableid vedama. Kaabli rennid 
on olemas. Ja ühel hetkel saab kaablirenn täis. Mis me siis teeme? Veame 
teistpidi. Aga kes ütleb, et kaabli kulu on selle juures kõige optimaalsem? 

\question{See oli ju Nõukogude aeg veel?}

Ei, siis oli juba sula. Peale kroonu, see pidi siis olema mingi üheksakümmend, 
üheksakümmend üks, niisugune hell aeg.

\question{Sel ajal, inimesed on rääkinud, ei tohtinud isegi mitte arvuteid 
Nõukogude Liitu tuua. Aga sina arvutasid sõjalaevade  kaableid?}

No mis siis on sellest, kes seda teadis. Minu arust oli juba siuke sulaaeg 
täiesti, kui mitte päris iseseisvus, ma lihtsalt tõin ühe näite. 

\question{Kuidas see haltuura pakkuja oskas sinu juurde tulla? Kuidas see käis?}

Ütleme siis niimoodi, et see õppejõud, kellele ma esimeses tunnis ütlesin, et 
ma rohkem ei käi sinu juures, muu hulgas tema leidis mulle neid otsi. Inimesed 
ikka teadsid mind,  oskaksid soovitada. Just ülikooli ajal sai väga 
eripalgelisi asju tehtud. Ma olin siis hirmus progeja, kirjutasin muu hulgas 
oma andmebaasi süsteemi, mis oli FoxPro-st  kordades kiirem. Vaat vanasti oli 
niimoodi, et kõvaketta poole pöördumine oli sihukene ränk tegevus, see võttis 
aega. Praegu SSD-ga me ei räägi nagu sellest. Aga sel hetkel oli kettapöördus 
kõige aeglasem tegevus. Ja siis ma kirjutasin andmebaasi süsteemi, millel olid 
fikseeritud pikkusega väljade asemel  sujuva pikkusega väljad. Mis tähendab 
seda, et andmeid oli ketta peal täpselt nii palju kui andmeid oli, mitte, et 
sul oli eraldatud mingisugune megabaitide hunnik. Mis tähendab seda, et 
keskmise andmebaasi ma tõmbasin umbes kaheksa korda kokku. Ja vastavalt läks 
selle töötlemiskiirus  üles.

\question{Kuidas sa neid kirjeid pakid ja mis saab siis, kui välja pikkus 
muutub? See ei ole ju lihtne?}

No miks see peaks lihtne olema? Mis see geniaalsele programmeerijale ja 
matemaatikule ära ei ole siis välja rehkendada. Nagu sõjalaevade kaalutud 
graaf, et kuidas kõige optimaalsem kaabli kulu on. 

\question{See tegevus läheb juba otsapidi teadusse, ega maailmaski noid 
andmebaase teab mis palju ei olnud. Sa teadlaseks ei tahtnud saada?}

Ei, mulle meeldis praktiline pool kogu aeg. Lõpuks pika hambaga läksin 
magistrantuuri ja siis virelesin seal umbes kuus aastat. Siis hakkasid 
ainepunktid  ära kustuma, tegin jõuga lõputöö. Mulle see kuiv teooria ei paku 
eriti midagi, mulle meeldib nagu maailma muuta. 

\question{Kui ma su juttu niimoodi kuulan, siis tekib küsimus, et kas sa olid 
siis kuulus või?}

Ei olnud. Eks mingi ühe või teise tehtu renomee käis kuidagi ringi ja mul oli 
Peeter Lorents\index[ppl]{Lorents, Peeter}, see õppejõud. Siit-sealt linna 
pealt,  ikka tutvuste  või kuidagi sidemete kaudu käis see värk. Aga see ei 
olnud massiline, me räägime projektide hulgast, ma ei tea, mingi kümmekond või 
niimoodi. Aga päris suured asjad, ikka sai tehtud.

Tööasju oli ka loomulikult aeg-ajalt, aga tööd oli sel hetkel, veel kord, vähe 
ja poliitika oli selline, et parem inimene olgu olemas ja valmis. Kui tööd 
tuleb, siis saab seda teha. Too  kontor, mis on tänase nimega 
Ektaco\index{Ektaco}, oli omal hetkel  fantastiline koht. Seal oli umbes 
viiskümmend inimest, tehti riistvara, tehti tarkvara, umbes \emph{fifty-fifty}. 
Juku, mida ma mainisin, mis oli muidugi nende tehtud. Muuhulgas Elleri-papi 
tegi ehtekarbist valmis esimese hiire maailmas\sidenote[][-1cm]{Arvo 
Eller\index[ppl]{Eller, Arvo} oli Juku loomise eestvedaja (Ants Vill. (2010). 
Meenutusi aegadest, kui arvuteid tehti veel käsitsi. Linnaleht (Tallinn), 
(46)). Kas tema loodud hiir just maailma esimene oli, aga ehtekarbi lugu kordab 
ka viidatud allikas.}.

Pooled inimesed olid \emph{cum laude} TPI lõpetanud, nii et sealne 
ajupotentsiaal oli nauditav. See, mis seal tehti\ldots Ülemusel oli sünnipäev, 
vennad mõtlesid, et mis teeme, teeme rääkiva papagoi. Tegidki. Seal oli 
briljantseid, lihtsalt absoluutselt lahedaid vendi. 

\question{Mis see töö sisu seal ikkagi oli? Kas ise mõeldi projekte välja?}

See oli nii ja naa. Üks nende põhiline ajalooline rida olid 
tööstuskontrollerid. Ise välja mõeldud, ise tegid, ise progesid. Need olid 
siuksed korralikud \emph{rack}-i suurused, täna võid samasuguse asja  Hiinast 
osta kiibi peal või, noh, põhimõtteliselt miniatuurse. Analoog-sisendid, 
analoog-väljundid, digitaal-sisendid, digitaal väljundid ja siis seal vahel on 
mingi loogika. See ongi kontroller oma olemuselt. Aga jah, see oli vaene aeg ja 
 Ektaco\index{Ektaco} tehtigi ühisettevõttena mingi Soome partneriga. Tänase 
päevani teevad nad ju kassasüsteeme. Compucash on nimi, võid aeg-ajalt näha 
baarides siiamaani. Toona siis soomlane tuli ja ütles, et tehke mulle 
proovitöö, tehke  maatriksklaviatuur, et kui baarman vajutab \enquote{õlu}, on 
kohe olemas. See tuli välja ja siis mindi edasi. Ega see lihtne ei olnud sel 
ajal seda tellimust või tööd või raha leida. Ja  seetõttu oligi nii, et pool 
inimestest istusid pool  aega jõude. 

\question{Ja sina muudkui programmeerisid?}

Mina muudkui programmeerisin. Ektacos\index{Ektaco} ma olin kokku viis aastat, 
ülikooli aja suures plaanis. Siiski, aastal 1992  võis olla,  kui ma läksin 
ikkagi nii-öelda peamajja tagasi,  Küberneetika Instituuti\index{Küberneetika 
Instituut} siis sel ajal. Seal tekkis võimalus sihukese uue rakukese 
tekkimiseks, mis esialgu alustas krüptograafia alusuuringutest. Seltskonnas 
olid mõned ülikooli  poisid ka,  teadlase moodi inimesed nagu Ahto Buldas ja 
nii edasi.  Enam-vähem oli nii, et  Ülo Jaaksoo\index[ppl]{Jaaksoo, Ülo} oli 
toonud välismaalt paksu raamatu a la Krüptograafia Alused ja seda me siis koos 
lugesime. Keegi oli peatükki lugenud ja proovinud aru saada ja  seletas 
teistele, kuidagi niimoodi see vaikselt alguse sai. See teadus nagu puudus, 
krüptograafia teadmine Eestis oli täielik null arusaadavatel põhjustel. Kui 
iseseisvus tuli, oli plats lage ja pidi kuskilt alustama. No vot, niimoodi 
alustasimegi. 

\question{Kuidas maailmas tol hetkel krüptoga oli? Mis tolleks hetkeks juba 
olemas oli, kui sa tagasi mõtled?}

RSA oli olemas, see on 78. aastast. Ma ei tea, ma ei ole ennast tegelikult 
kunagi krüptoloogiks pidanud. Minu eriala on rohkem  nii-öelda 
rakendus-krüptograafia, mitte  süva-krüptograafia.

\question{Aga miks sa läksid? Sul oli Ektacos ju mõnus oma projekte progeda?}

Ma ei taha halvasti öelda, aga põhimõtteliselt oli nii, nagu ma ütlesin, et 
pooled inimesed olid suurepärased insenerid, \emph{cum laude} lõpetanud ja nii 
edasi. Aga firmas ei saadud aru, et  peaks tegelema nende inimeste arenguga. 
Oli väga selge seisukoht, et igaühe areng on tema enda asi. Et niisugust asja 
nagu Internet firmasse, seda küll ei saa kulutada või mingit ajakirja osta või 
mingit inimest kuhugi konverentsile saata, see ei tulnud kõne alla. Pinge 
kogunes ja mingil hetkel, oma sünnipäeval,  saatsin kohalikku võrku essee, mida 
ikka aastaid ja aastaid tsiteeriti pärast, et  mis nagu firmas valesti on. 
Kümme aastat hiljem võeti see välja ja vaadati, et ikka on sama lugu. 

\question{Kuidas see kamp ülejäänud Eesti kogukonnaga kokku käis? Tol ajal juba 
nagu pruulis küll, mingid inimesed juba  toimetasid ja pidasid BBS-e?}

Nojah, mu hea sõber ja kolleeg Heiki Kask\index[ppl]{Kask, Heiki} pidas ühte 
BBS-i ja ma kuidagi liitusin sellega. Sealtkaudu kuidagi sattusin  Fido-nautide 
sekka lõpuks ja kuidagi sai hakatud läbi käima. 

\question{Aga see ei olnud sinu jaoks kriitiline ja tähtis asi?}

Fido ei olnud minu jaoks kriitiline ja tähtis. See oli lahe ja andis nagu 
esialgse maigu suhu, aga nii, kui tuli Internet, ma armusin sellesse.

\question{Mis tolle interneti juures nii armastusväärset oli? Meili ja uudiseid 
sai Fido kaudu ka?}

Meil oli  UUCP ja modemiga helistamine mitu aastat esialgu, 1991-1993, kui ma 
ei eksi. Sai meili saata, väga tore. Aga mulle jõudis kohale, et kuskil on 
olemas nii-öelda püsiühendusega Internet. Et saab nagu reaalajas 
kuidagi\sidenote{Tõepoolest oli mõiste \enquote{püsiühendus} tol ajal maagilise 
tähendusega: ei unistatud mitte kiirest Internetist vaid pidevalt ühendatud 
Internetist. Võimalus kaugete arvutitega vahetult suhelda tundus imeline. 
Mäletan esimeselt kursuselt 1993. aastal üht oma esimest \emph{online}-kogemust 
Tartu Ülikooli Vanemuise tänava arvutiklassis\index{Tartu Ülikool!Vanemuise 
tänava õppehoone}, mille käigus tõin masinast 
\texttt{ftp.sunet.se}\index{Masinad!ftp.sunet.se} ära Metallica plaadi Master 
of Puppets kaanepildi. See võttis aega ja pilt oli madala resolutsiooniga aga 
ma tõin endale Rootsist midagi kohale!}. See oli nii võluv minu jaoks, et 
loomulikult tahtsin teda ühel või teisel moel uurida. Nii et veel nendel UUCP 
aegadel ma mäletan ennast pühapäeviti kuskil modemi külges rippuvat ja RFC-sid 
alla laadimas, et need kõik läbi lugeda algusest peale.

\question{See vist isegi oli võimalik tol ajal?}

Oli. RFC-de ülemine ots oli kuskil tuhande kandis alles, nii et see ei olnud 
mingisugune probleem. Osad on lühikesed ja osad on mõttetud ja nii edasi, eks 
ole. Veel kord, see oli lihtsalt niisugune mõte. On ilmselge maniakaalsus, et 
kogud endale hästi palju materjali, et küll ma ükspäev loen.

\question{Seal uues üksuses oli Internet sinu jaoks siis infoallikas?}

No eks ta ikka oli jah. Sai meili kirjutada, lahe värk lihtsalt. Enne veebi oli 
põhilised FTP saidid. Ei pidanud mõtlema, et mis \emph{node}-st või kust sa 
mida saad.  Mõnikord sai FTP-st ka mõne mängu kätte, seal ikka liikus kraami. 
Sa saad ju \emph{upload}-eda ja \emph{download}-eda samamoodi nagu Fidonetis. 

\question{FTP-des oli igasugu asju aga arvutimängimist sa küll ei ole maininud?}

Ega ma ei olnudki suur mängumees,  aga noorest peast ikka sai midagi põristatud 
või täristatud õhtuti. See on umbes niimoodi, et sul on mingisugune huvi ja 
siis sul on viis sellest lõõgastuda. Et kui on lõõgastumise aeg, siis sa võib 
olla mängid. Aga arvutimängud ei olnud mu huvi. 

\question{Sinu fookus oli matemaatika peal?}

Progemise peal, mulle meeldis arvutit oma pilli järgi tantsima panna, mitte 
see, et mina arvuti pilli järgi tantsin. Kui  Windows\index{OS!Windows} tuli, 
siis ma kaotasin usu arvutitesse, sellepärast et ma ei suutnud enam iga igat 
bitti kontrollida. Kuni sinnamaani ma teadsin opsüsteemi tasemel,  EEPROM-i 
tasemel, mis sünnib. Nii kui Windows tuli, kontroll kadus ja mul läks tuju ära.

\question{Kui Linux\index{OS!Linux} tuli, tuli tuju tagasi?}

Linux, jah, aitas mul üle elada selle Windowsi aja. Aga mingiks hulluks 
Linuxi-kasutajaks ma ikkagi ei hakanud. Kui ma läksin sealt 
Ektacost\index{Ektaco} progemast ära Küberneetikasse\index{Küberneetika 
Instituut}, siis ma põhimõtteliselt jätsin progemise maha. Viimane asi, mis ma 
tegin, oli 1996. aastal mail.ee\index{mail.ee}. 

\question{Miks sa selle tegid?}

UNDP-st sai  väikese \emph{grant}-i selle tegemiseks\sidenote{\emph{United 
Nations Development Programme - UNDP}. Üheksakümnendatel läks Eesti veel üsna 
selgesti arengumaana kirja ja sai paljudest eri kanalitest igasugu abi. 
Tänaseks on mõiste \enquote{humanitaarabi} õnneks suuresti ununenud, toona tuli 
seda abi kõikvõimalikes vormides päris palju ning oli tõesti ka abiks.}. 
Kõigepealt oli see hea mõte, et igaühel, kes tahab, võiks olla meiliaadress. 

Oo jaa, ma pean alustama muidugi sellest, et kuskil 1994. aastal sai tehtud 
sihukene firma nagu Teleport\index{Teleport}. Mitte ajada segi selle sajandi 
Teleportiga! Meid oli kaheksa tudengit, kuus õppisid välismaal. Miks nii, sest 
neil oli, vaata, raha. Eesti tudengitel raha ei olnud. Kaheksakesi panime rahad 
kokku, ostsime Soomest portsu modemeid ja tegime  sissehelistamiskeskuse kus 
sai ilma lepinguta nii-öelda 900-numbri\sidenote{900-numbrid olid telefoninumbrid, 
mis algasid numbritega 900 ning millele helistamisel kehtis eritariif. Toda tariifi 
jagati teenusepakkujaga ja nii sai suhteliselt lihtsasti tasulisi teenuseid osutada}
 kaudu helistada. Nii et saime kohe oma 
raha kätte läbi selle 900-teenuse. Kommertsiaalse Interneti pakkumine sellel 
ajal oli vaata et olematu ja laiadele massidele mõeldes muidugi täiesti 
puudulik. 

\question{Uninet\index{Uninet} tuli meil mis aastal?}

Uninet oli juba olemas, aga Uninetiga sa pidid lepingu sõlmima. 
EsData\index{EsData} oli ka olemas, me istusime nende võrgu peal tegelikult. 
Aga kuu hiljem tuli Microlink Online\index{Micolink Online} ja sõi meid massiga 
ära.  Teleportist sai mõnesid partnereid kaasates edasi 
Meediamaa\index{Meediamaa} ehk siis www.ee\index{www.ee}. See oli siis Eesti 
esimene nihuke veebiäri, kus me proovisime inimestele rääkida, et kui sind pole 
Internetis, pole sind olemas. Ja et tulevikus pole sul muud vaja kui et URL oma 
kaubaauto peal. Nad vaatasid mind nagu idiooti, eks ole, aga nüüd sa URL-iga ja 
kaubaautosid ainult näedki. 

\question{Aga miks te, programmeerijad, tolle firma tegite?}

Pigem  tudengeid. Sest Tarvi selgitas, et niisugust teenust turul ei ole ja see 
tundus väga lahe, kui inimesed saavad juurdepääsu Internetile. 

\question{See oli siis puhas missiooniüritus?}

Eks võib-olla me lootsime raha ka teenida mõttes, tundus nagu olematu bisness, 
kus on võimalik kanda kinnitada. Miks ei peaks? Veebiga oli sama lugu. Eks ta 
paljuski niisugune missiooni ja eestvedamise värk oli. Ma kirjutasin mingi 
raamatu ka 1996. aastal  Internetist, mis oli esimene eestikeelne 
originaalraamat\sidenote{Martens, Tarvi., and Vello Hanson. Internet. Ilo, 
1996.} sel teemal. See kõik oli sihuke Interneti propageerimine. Ma samal ajal  
töö mõttes ehitasin juba riigile andmesidevõrkusid ja kogu selle TCP/IP 
tehnoloogia  laialdane levik tundus mulle sellel kümnendil väga-väga tähtis.

\question{Miks?}

Saavutamaks seda olukorda, kus me täna oleme. 

\question{Ja sul oli pea sees olemas, et selline olukord peab olema ja saab 
olema ja see on hea?}

Ma teadsin, et see on hea. Ma ei teadnud, kui kiiresti ja kui massiliselt, aga  
need hüved olid ilmselged. 

\question{Kas su juttu keegi kuulas ka?}

Ma arvan küll, jah. Me oleme näinud, et igasuguse uue tehnoloogia evitamine 
võtab ikka ju palju aega. Siis on täitsa loomulik, et me räägime ajast, mis oli 
 kakskümmend viis aastat tagasi ja mille järelmeid võib täna näha. Nii nagu 
muude asjade nagu ID-kaardi  või e-hääletamise puhul,  tulemused tulevad  mitte 
järgmine päev, mitte järgmine kuu ja mitte järgmine aasta. Ma rääkisin üks kord 
ühele psühholoogile, mis ma teen. Ta ütles, et \enquote{Tarvi, et sa oled ikka 
hull. Need asjad, mis sa teed, on inimeste käitumise muutmine. Ühiskondliku 
käitumise muutumine võtab miinimum seitse kuni kaheksa aastat aega. Et sa ei 
saa enne oma tibusid lugeda, kui sa vanaks jääd}.

\question{Vähe sellest, takkajärgi on too algne impulss sisuliselt tuvastamatu 
ja seega ega keegi aitäh ka ei ütle}

Ma ei igatse selle üle ausalt öeldes, see on väga okei. Ma lihtsalt vaatan 
ümberringi ja ma naeratan. 

\question{Tuleme korra sinu juurde tagasi. Mainisid, et sa tegid riigile 
mingisuguseid andmesideühendusi?}

Oojaa, see on üks tore lugu. Usun, et oli aasta 1993, kui Eesti toll ja 
piirivalve tulid Küberneetika Instituuti\index{Küberneetika Instituut}. Meil 
oli palju suhteid riigiga, et tehke mingi standard või mõtleme andmekogude 
peale. Andmekaitse Inspektsioon\index{Andmekaitse Inspektsioon} sai seal 
näiteks disainitud koos oma segadusega,  niisugused asjad. 
Toll\index{Tolliamet} ja piirivalve\index{Piirivalveamet}  tulid ja ütlesid, et 
 vaadake, meil on vaja piirivalvet ja tolli üles ehitada. Ja meil on niisugused 
ühised piiripunktid, kus pole  mingit sidet, mõnikord isegi mitte 
telefonisidet. Et kas Küberneetika Instituut saaks aidata, et joonistate äkki 
mingi projekti. Okei, joonistasin projekti. Klient sai projekti kätte, käis 
linna peal ringi ja paari kuu pärast tuli tagasi ja ütles, et mitte keegi ei 
suuda seda projekti ellu viia. Et tehke nüüd  see asi ära ka. Ja siis kuidagi 
juhtuski niimoodi, et me pidime paberimäärimisest hakkama tegudele üle minema. 
Koostöös Eesti Telefoniga\index{Eesti Telefon}, kus mul olid mingid sidemed, 
sai  esimesed ühendused  ära tehtud ja siis hakkas kogu see tegevus mullina  
paisuma. Järgmisena tuli politsei ja siis riburada teised ka järgi. Me 
tegutsesime Küberneetika Instituudi katuse all. Instituut oli väga hea selline 
amorfne asutus. Tahtsid, tegid teadust. Tahtsid, tegid äri. Tahtsid, olid nagu 
riigi asi.

Sai siis tehtud seda asja, kuni juba  rahad läksid liikuma ja ruutereid pidi 
ostma ja oli vaja moodustada mingi formatsioon. Tegime midagi niisugust, mida 
ei tohtinud tegelikult seaduse järgi teha, põhimõtteliselt MTÜ riigiasutustest. 
See MTÜ oli siis Andmeside Osakond\index{ASO}\index{Andmeside 
Osakond|see{ASO}}, mida  juhtis nõukogu, kus oli iga riigiasutuse esindaja. Iga 
aasta raamatupidamistoimkond  olevat ürisenud, et sellist asja ei tohi teha ja 
siis kõik need suured ülemused ja ministrid ütlesid, et \enquote{ärme lõhu 
toimivat asja}. Me \emph{split}-isime  kulusid, seesama politsei ja piirivalve 
näide. Sul on vaja piiripunkti üks ruuter osta, teeme kulud pooleks, väga 
loogiline. 

\question{Aga see kõik ju eeldas, et keegi riigi poole pealt kuulas sind ja 
mõtles kaasa? Olid need tippjuhid, IT-juhid?}

Ma arvan, et kõigepealt kuulasid IT-juhid, kes siis rääkisid oma tippjuhtidele. 
Ma mäletan väga selgelt, kuidas 31. detsembril istusid toonase 
piirivalve\index{Piirivalveamet} ülema Kõutsi\index[ppl]{Kõuts, Tarmo} 
kabinetis kõik asjaosalised ümber laua ja kirjutati alla leping. Neid 
moodustajaid oli siis  neli: Politsei, Piirivalve, Toll ja Küberneetika 
Instituut. Kõuts veel ütles  \enquote{Ma saan aru, et meil on siin juhi 
kandidaat ka laua taga}, ma olin kahekümne viie aastane, sihuke naga. 

Edasi läks väga huvitavaks, sest meil oli selline  asutus nagu 
Valitsusside\index{Valitsusside} iseenesest olemas, kes tegeles erivõrkudega ja 
puha.

\question{Kas nad kurjaks ei saanud su peale?}

No vot, eks siin teatav konflikt  tekkis erinevatel põhjustel. Ka, ütleme,  
koolkondade vastasseisud,  Jaaksood versus Lippmaad, eks ole. Vanemad inimesed 
teavad seda väga hästi.

Aga juhtus jah niimoodi, et Piirivalvel ühel hetkel oli kagupiir täiesti lage, 
seal polnud mingit sidet, see oli vaja tekitada. Ja selle asemel, et minna 
Valitsussidesse, kes peaksid neid asju tegema, tulid nad minu juurde. Ütlesid, 
et \enquote{näed, Tarvi, siin on kümme miljonit\sidenote{Tegu on Eesti 
Krooniga. Arvestades  valuutakursse ja dollari inflatsiooni, on aasta 2021 
kontekstis tegu umbes 1.2 miljoni euroga. Arvutades protsenti riigieelarve 
kuludest, maksis too projekt tänases kontekstis suurusjärgus 21 miljonit eurot, 
toonane riigieelarve oli võrreldes tänasega väga pisike.}, et meil on seal lage 
plats, kaheksa piiripunkti on vaja ühendada, tee midagi}. Ma ütlesin, et 
\enquote{jaa, väga huvitav}. Aasta oli 1994 või 1995.

\question{See oli jõhker raha tol ajal. Kõike oli ju vaja ehitada, kust tekkis 
idee see raha just sidele kulutada?}

No aga kui sul ei ole ka telefonisidet, me ei räägi siin mingist 
\emph{fancy}-st veebibrausimisest. Me räägime ikkagi telefonisidest ja 
elementaarsest sõnumivahetusest. Kui sa oled keset platsi ja sul ei ole ei 
mobiilevi ega mitte midagi, kuidas sa seda piiri pead?

Nojah, päeva lõpuks olin mina see projektijuht, kes ehitas tühjale kohale  
kaheksa masti ja pani sinna taldrikud otsa. Ehitasime raadioside, nagu mobiilis 
käib, mingi 2.4 giga peal.

\question{See ei ole ju ka raadioside disaini mõttes triviaalne ülesanne, seda 
õppisid ka kuskilt raamatust?}

Seda minagi kuulatama jäin. Kõige hullem oli see, et ma mõtlesin, et kasutame 
kõrget sagedust ja seega peab olema otsenähtavus. Aga kuidas seda otsenähtavust 
kindlaks teha?  Lõuna-Eesti maastik, mäed ja orud. Leidsin 
Maa-ametist\index{Maa-amet} ühe tuttava, kes oli sihukene fänn, et oli hakanud 
vene ohvitserikaarte (kõige täpsemaid, mis sel hetkel oli) digiteerima ja oli 
selle kõige huvitavama osa, ehk siis Võrumaa,  sisse saanud. Ta suutis mulle 
väljastada  profiili: andsin talle otspunktid, tema andis mulle mingi arvude 
jada. Kirjutasin ise proge, keerasin maa kumeraks, panin mastid kasvama, 
vaatasin, kas on otsenähtavus. Selle järgi sai siis  mastide kõrgust 
rehkendatud. Mida kõrgem mast, seda kallim. EMT\index{EMT}, sel ajal ei 
vaadanud mingit profiili, pani 80 meetrit igale poole. Aga mul oli seal 
viiskümmend kaks ja viiskümmend neli. Kui on viiskümmend, peavad juba 
lennutuled olema ja jälle on kallim, eks. Sattusin mingisuguse teadjamehe peale 
Eesti Telefonist\index{Eesti Telefon}, kes vaatas tehtut ja ütles, et 
\enquote{Kuule mees, sa tegid nädalaga sihukese asja? Trassi projekteerimiseks 
läheb poolteist aastat, tuleb jala kõik läbi käia, puud ära kaardistada!}. Aga 
mul olid mastid tellitud juba. Ta rääkis mulle, et on olemas fresneli tsoon, et 
saatja ja vastuvõtja vahele ei teki mitte kiir, vaid niisugune vorsti moodi 
asi\sidenote{Fresneli tsoon on ellipsoidne tsoon, mida pidi raadiolained 
saatjast vastuvõtjani levivad. Tsooni võivad sattuda ja seega ka sidet segada 
ka otsenähtavusest väljapoole jäävad objektid.}. See võttis natuke jahedaks 
küll. Aga mastid olid tellitud ja käima ta läks. Järgmine aasta tegin Peipsi 
äärde sama viguri. 

\question{Ühesõnaga, sa ei teadnud, et nii ei saa teha?}

Ei, ma ei teadnud,  mõtlesin inseneri mõistusega, kuidas käib. 

\question{Miks sa selle optimiseerimisega vaeva nägid? Palju raha anti kätte, 
oleks võinud ju julgelt üle dimensioneerida?}

Me räägime ikkagi vabariigi algusaegadest, kus raha ei ole palju, eks ole. Sa 
pead loomulikult olema igas asjas optimaalne ja tegema parimat, mis mis teha 
annab. See ei olnud meil teab mis üleliia suur raha, kulus kõik ära. 

See oli väga tore tore aeg, kus sai tõesti käegakatsutavalt riigi arengut 
toetatud, mis oli väga lahe. Ja pealegi minu lemmiktehnoloogia ehk siis 
internetitehnoloogia osas. 

\question{Kui ma sind kuulan, siis sa olid programmeerija kuni saabus Internet 
ja sa leidsid, et tuleb hoopis sinna panustada, sest maailm läheb sellest 
paremaks?}

Jaa. Programmeerida oskas sel ajal juba üha rohkem inimesi, ma ei olnud enam 
väga unikaalne, ka niipidi võib mõelda. Aga kaua sa ikka programmeerid.

\question{Mõni programmeerib eluaeg?}

Ma saan aru. Aga niisugused kõrgemad ja üllamad mõtted tundusid tundusid 
järjest paremad. Võib olla see on isiksuse arenguga ka seotud. Ausalt öeldes,  
kui ma olin programmeerija, kui telefon helises, ma kartsin seda väga. Ma ei 
tahtnud inimestega suhelda. Aga mingil hetkel läks see kuidagi  teistpidi üle.

Linna peal teadsid kõik  Martensist, et  kui ta jaurama tuleb, siis ta 
kindlasti proovib sind  Küberneetikasse tööle meelitada. 

\question{Oot, sa seal Küberneetikas\index{Küberneetika Instituut} olid juba 
mingisugune meelitaja mees, äkki juhtkonnas juba?}

ASO\index{ASO} pealik ma olin, see sai üle antud 
Informaatikakeskusele\index{Informaatikakeskus}, mis oli RIA 
eelkäija\index{Riigi Infosüsteemi Amet}\sidenote{Eesti Informaatikakeskus koos 
Riigihangete Keskusega liideti aastal 2003 Riigi Infosüsteemi Arenduskeskuseks, 
millest 2011. aastal sai Riigi Infosüsteemi Amet ehk RIA}. 

Aastal 1997 toimus reformatsioon, kus instituudid, kui sellised eraldiseisvana 
kaotati ja  pidid liituma ülikoolidega. Küberneetika Instituudiga juhtus 
põhimõtteliselt see, et ta jagunes kolmeks. Kõige väiksem osa, ehk siis 
andmesideosakond,  läks Informaatikakeskusele. Ülejäänu läks kaheks, üks läks 
Tallinna Tehnikaülikooli alla. Aga kuna Küberneetika Instituudis oli seda 
praktilist tegevust hästi palju, siis kõigest praktilisest moodustati 
Küberneetika Aktsiaselts\index{Küberneetika AS}, mis on siiamaani alles. Ta 
moodustati riigiettevõttena, nüüd on  vist erastatud. Küberneetika AS  oli väga 
huvitav  kombinatsioon. Oli osakond, kes progesid  Tolliameti\index{Tolliamet} 
infosüsteeme. See, kus mina olin, oli keskendunud infoturbele  nii teoorias, 
praktikas, konsultatsioonides, analüüsides, milles iganes. Ja seal kõrval oli 
siis kõik see meremärgindus,  merenavigatsioon ja valgusfooride tegemine. 
Kinnisvarahaldus ka, aga see on nüüd läinud. 
 
\question{Kui ma su juttu kuulan, siis hakkab sinna sisse tasapisi sigima juhi 
roll. Mõnikord inimesed saavad sellest maigu suhu ja siis ainult sellega 
tegelevadki. Sul niimoodi ei olnud?}

Eks ma jõuga pidin maigu suhu saama, tegevust oli vaja ju kuidagi laiendada ja 
töö tahtis tegemist.  Inimesi oli vaja, inimesi oli vaja meelitada. Selle 
Küberneetika AS-i\index{Küberneetika AS}  moodustamisel sai minust 
arendusdirektor. 

Mõeldi küll, et ma vaatan nagu laiemat asja ja võib-olla tegelen ka meremärkide 
ja poidega, aga sinna õnge ma ei läinud. Tol ajal ma hakkasin juba infoturbe 
toodete arendamisega tegelema. Mingi 1996 tegime esimese tulemüüri valmis, siis 
tegime  VPN toote ja mingeid SSL-i \emph{proxy}-sid ja asju. 

\question{See oli pärast Meediamaad?}

Jah, see oli pärast. See infoturbe toodete arendamine läks esialgu hirmus 
hästi. Me tegime Linuxi peale põhimõtteliselt veebipõhise interfeisi nendele 
jubinatele, millest saab niikuinii ise tulemüüri teha, kui sa oled väga osav 
insener. Tegime selle veebiliidese kaudu lihtsamaks ja oligi jämedas plaanis 
toode valmis. Eesmärk oli teha viis korda keskmisest odavam toode, keskmine 
tulemüür  maksis  tol hetkel kolm tuhat dollarit. Ja tuli välja. 

Vot siin peab nüüd mõtlema, et kas siin oli seos, ilmselt oli. Just 
riigiasutused väga meeleldi ostsid neid tooteid, mis me tegime. \enquote{Tarvi 
tegi võrgud, nüüd müüb neile turva ka peale}. Eks ta praktikas niimoodi kukkus 
välja küll. 

\question{Enamasti riikides tekib üks hetk kihu teha omale sihuke privaatne 
internet. Et oleks turvaline või nii. Kuidas Eestis sihukest hullust ei tulnud? 
Või kas üritati teha ja ei tulnud välja?}

No ikka üritati, loomulikult üritati. Aga just läbi selle, et me tegime siukse 
VPN toote. Ta oli võrreldes praeguste VPN toodetega unikaalne. Kui sul see kast 
oli võrgul ees, sa ei saa Internetti,  ta lasi ainult teise omasuguse juurde. 
Nii et sul nägi asi välja niimoodi, et sul on kontorid, ütleme igas maakonnas.  
Igas maakonnas paned selle rohelise kasti võrgule ette ja siis kamba peale, no 
ütleme Tallinnas,  on üks tulemüür ka. Ja ainult  läbi ühe selle tulemüüri saab 
välja, kui sul millegipärast vaja on. Aga muidu on täiesti sisevõrk. 

\question{Kuule, see, mis sa kirjeldada on ju X-Tee? Arhitektuuri mõttes tundub 
ta väga sarnane?}

Ei ole. Sa võid ju nii mõelda, aga siin pole andmete semantika mingit pistmist. 

\question{Ehk, ei toimunud mingit rist-tolmlemist projektide vahel?}

Ei, see oli ikka puhas privaat-torude ehitamine, siin X-tee on OSI tasemetes 
natuke kõrgemal.

\question{Kas sa tol ajal tegelesid selle Interneti promoga veel paralleelselt 
edasi või see oli sul lihtsalt üks faas?}

Siis oli juba turul juba piisavalt tegijaid ja ma ei tundnud, et ma peaks 
sellega tegelema. Pigem oli minu jaoks oli saabunud faas number kaks, et nüüd 
tuleb see Internet turvaliseks teha. Ja siis on elementaarne, et on ka kolmas 
aste, ehk siis tuleb kõik osapooled internetis identifitseerida. Et saaks nagu 
\emph{business}-i ka teha. 

\question{Kust selline hull mõte, et Internet peab turvaline olema?}

Ma ju rääkisin sulle, et me hakkaksime teoreetiliselt turvaga tegelema juba 
1992. aastal. Kõik need kontseptid olid mulle vägagi tuttavad, et kuidas seda  
teha ja miks seda vaja on. Nonde meie roheliste kastide puhul oligi kogu 
eesmärk puhas ning turvaline andmeside, rohkem ei olnudki. Minu  sõnum oli see, 
et ärme teeme eraldi X.25 võrku, üle avaliku interneti toimetades on palju 
kuluefektiivsem.

\question{Ma seda uuringi, et kust sul tekkis mõte, et interneti turvalisus on 
probleem, mida tuleb hakata lahendama? Keegi luuras? Häkkerid kiusasid? Kust 
see probleem tekkis?}

Probleem on aegade algusest, mis sa mõtled, et kuskilt tuli ilmutus või? See on 
elementaarne, see on kogu aeg niimoodi olnud. Ja olles  infoturbega algusest 
peale tegelenud on selge, et  võrkudes on infoturve  teemaks. See on 
elementaarne. 

\question{Kui mina oma ajaloo peale mõtlen, siis minu jaoks ei olnud. Ma ikka 
pikalt ehitasin omi asju ja võrke üldse mõtlemata, et nad võiksid ka turvalised 
olla.}

Ma siiamaani räägin, et ma tulen nii-öelda infoturbe erialalt. Ükskõik mis 
ametis, see on minu \emph{background}, mille ehitamine sai alguse toona tolle 
ühe raamatu koos lugemisest.

\question{Aga lisaks on sul ka matemaatika ja programmeerija ja antenni-ehitaja 
taust ka, sa saad päris põhjani välja minna.}

Jah, ma olen kirjutanud Jukule\index{Arvutid!Juku} püsimälu põhimõtteliselt. 
Minu osa oli see, mis puudutas tähtede joonistamist ekraanile. EEPROM-i tasemel 
sai ESC-käskudega aknaid teha. 

\bigskip
\noindent\rule{.3\textwidth}{.7pt}
\bigskip

Ma siin mõtlesin, et mis lugusid veel võiks rääkida. Mõned tulid meelde.

Mina Tallinna poiss ei olnud ja Jaak Loondet\index[ppl]{Loonde, Jaak}, keda 
mitmed varasemad rääkijad on maininud, ei tundnud. Küll aga kuulsin ma temast 
teiste käest. Fido inimesed rääkisid. 

Aga juhtus niisugune lugu, et mingitel varajastel  üheksakümnendatel, kus 
Eestis  isegi leiba ei olnud, oli talongide peal\sidenote{Tarvi ei liialda, 
kaheksakümnendate lõpust kuni umbes 1993. aastani, kui vaba turg enam-vähem 
toimima hakkas, müüdi elementaarseid toidu- ja tööstuskaupu, sealhulgas 
periooditi ka leiba, vaid inimestele jagatud talongide esitamisel.},  otsustas 
Soome Rotary klubi Eesti koolidele natuke arvuteid kinkida. Ühel 
tehase-inimesel jäi kuus tükki üle. Ilmselt oli mingi PC-aeg peale tulnud, üle 
jäid  mingid väga kummalised kompuutrid, ma isegi ei mäleta, mis täpselt. Aga, 
mis oli lahe, nad olid võrgus ja ema-arvuti oli ka. Soome Rotary tegi 
ettepaneku Haridusministeeriumile, et me kingime Eesti koolidele  arvuteid. 
Peeter Lorents\index[ppl]{Lorents, Peeter}, minu mentor, oli sel ajal 
Haridusministeeriumis mingi tegelinski ja sattus sellele teemale peale. 
Läksimegi siis kolmekesi, autojuht, Peeter ja mina (mina olin nagu eksperdiks), 
koha peale vaatama, mis loomad need on ja aru saama, kuidas nad käivad. Läksime 
kohale, tõime need arvutid ära ja siis tekkis muidugi küsimus, et mis me 
nendega peale hakkame? 

\question{Palju neid masinaid oli?}

Kuus-seitse tükki umbes, niisugune klassitäis. Eesti peale ei olnud palju aga 
Rotary klubi sai endale linnukese kirja, Eestit aidatud, heategevus tehtud. Ja 
siis mulle meenus Jaak Loonde\index[ppl]{Loonde, Jaak}. Sain temaga kokku, Jaak 
oli kohe nõus, läksid silmad põlema, nagu ikka.  Mõne aasta pärast saime kokku 
ja küsisin, kas neil masinatel  pruukimist ka oli,  ja tuli välja, et olid väga 
hästi vastu võetud ja  igasuguseid vigureid tehtud nendega. 

\question{Aga see ju tähendab, et Jaak toimetas edasi ka pärast seda, kui 
enamik temast rääkinuid koolipoisieast välja olid kasvanud?}

Jaa, ta ikka toimetes minu arust nii-öelda lõpuni, ta oli legendaarne. Tema 
tahtis, et lapsed saaksid näppude arvuti külge, see oli tema põhiline mõte.


\bigskip
\noindent\rule{.3\textwidth}{.7pt}
\bigskip

Rääkides  suhtlemisest ma pean tunnistama, et ma tegin 1993. aastal esimese 
jututoa. Jututuba on siis selline asi, nagu meil parasjagu on Messenger või 
värk: hulk inimesi logivad sisse, teksti terminal\ldots See oli Anna 
jututuba\index{Jutukad!Anna}.

\question{Kas see käis su oma tehtud softi peal või sa said selle kuskilt?}

Ma sain kuskilt selle softi, ja tõlkisin selle käsud eesti keelde, käsk algas 
punktiga. Ma olin sel hetkel Göterborgis asumisel neljaks kuuks. Mul polnud 
seal suurt midagi teha ja siis ma putitasin seda jututuba ja pidasin teda 
üleval, See oli päris lahe. 

Selles Anna jututoas isegi kaitsti üks diplomitöö ära. Inimene oli 
Uus-Meremaal,  õppejõud kogunesid jututuppa. Tegu oli  Tallinna 
Tehnikaülikooli\index{Tallinna Tehnikaülikool},  TPIga tol ajal.

\question{Mis nende jutukate fenomen oli? Kui ma kuulen asjadest nagu 
tekstiterminal ja punktiga käsud, siis see on tehnikute värk. Aga kui ma 
mäletan, siis neis jutukates käis  igasugust rahvast.}

See oli tegelikult ikka \emph{community building}. Umbes samasugune grupp nagu, 
ütleme, Fido grupp, eks ole. Tekkis niisugune jutukate grupp. Sealt edasi 
läksid igasugused OK \index{Jutukad!OK} ja 
Cafe\index{Jutukad!Cafe}\sidenote{Cafe jutuka pärisnimi oli \emph{The Roadkill 
Cafe} ja see asus aadressil \texttt{ns.uninet.ee:5555}. Selle pani 23. 
veebruaril 1996 NUTS-i (\emph{Neil's Unix Talk Server}) versiooni 2.3 
lähtekoodist püsti Indrek Siitan\index[ppl]{Siitan, Indrek}.} jutukad. Meil oli 
isegi Anna  kasutajate kokkutulek mingil hetkel Viljandi lähistel ja siiamaani 
Jüri Ruut\index[ppl]{Ruut, Jüri} veab seda, nüüd küll ee.kevade nime all.

Kogunes täiesti suvaline, rahvas, seal ei olnud õnneks üksnes tehno-friigid, 
oli tütarlapsi ka. 

\question{See pidi siis mingit pidi olema hirmus vajalik teenus, sest 
mitte-tehnofriigile pidi kõik see tehnika parajaks barjääriks olema}

See on täiesti lihtne, tegelikult, kui sa ainult terminalile ligi saad. Panid 
\verb|telnet anna.ioc.ee|\index{Masinad!anna.ioc.ee} ja läks. 

\question{Sa hoidsid toda jutukat Küberneetika Instituudis\index{Küberneetika 
Instituut}?}

Seda peab tunnistama. Küberis sai alustatud Unixi pruukimisega aastal umbes 
1992, kui sai Soomest flopidega Linux\index{OS!Linux} toodud. Teistmoodi ei 
saanud teda kätte. 

\question{Otse Linuse käest?}

Enam-vähem.  Eks ma proovisin seda Unixi kultuuri aretada. Üks hetk sai hirmsa 
raha eest üks  Sun ostetud. Küsiti, et mis  nimeks panna. Ütlesin suvaliselt 
\enquote{keeks} ja siis oligi keeks.ioc.ee\index{Masinad!keeks.ioc.ee}, igavene 
kuulus FTP server. Pärast ma pidin selle \enquote{keeks}-i lahti mõtestama ja 
siis ma arvasin, et see on Küberi Esimene Eestimeelset Kasutajate Server. Ehk, 
KEEKS.

\question{Tuleme korra nende jutukate juurde tagasi. Selleks, et sotsiaalvõrk 
lendu läheks, peaks olema mingi algne seltskond. Kes need inimesed olid ja 
kuidas sa selle tekitasid? Lihtsalt porti kuulav protsess ju populaarseks ei 
saa?}

Seda ma nüüd täpselt ei mäleta. Küllap ma rääkisin sõpradele,  sõbrad rääkisid 
sõpradele ja kuidagi niimoodi ta vaikselt Levis. Ma küll mingit erilist 
aktsiooni ei mäleta. Lihtsalt sõprade ringist piisas, aga see sõprade sõprade 
sõprade ring läks väga-väga laiaks ja seal ikkagi üle poole või rohkemgi olid 
täiesti tundmatud inimesed. 

Annaga\index{Jutukad!Anna} juhtus tõesti niimoodi, et ühel hetkel vaatasin, et 
teised jutukad hakkavad ka tekkima ja ma panin ühel hetkel ta pidulikult kinni. 
Anna matused olid kohe nihuke eraldi asi. Asja peab ära lõpetama, mitte laskma 
tal lihtsalt hääbuda, et \enquote{no kaua võib}. 

Aga Unixi-jutu  juurde tagasi tulles,  meil oli kunagi üks niisugune asi nagu 
Eesti Unixi Pruukijate Selts\sidenote{Selts asutati 1994 aastal, sellel oli 62 
asutajaliiget. Asutavasse toimkonda kuulusid lisaks Tarvile veel Andres 
Bauman\index[ppl]{Bauman, Andres}, Margus Liiv\index[ppl]{Liiv, Margus}, Jaanus 
Pöial\index[ppl]{Pöial, Jaanus} ja Anto Veldre\index[ppl]{Veldre, Anto}.}. EUPS 
oli nimeks. Kõik  tahtsid panna \enquote{Kasutajate Selts} aga EUKS kõlab nagu 
halvasti, siis mina olin see, kes ütles, et pruukima peab ikka. Meil oli isegi 
Tõraveres mingi kokkutulek. Nihukesi rühmitusi ikka tekkis.

\question{Miks te Soomest tolle Linuxi\index{OS!Linux} tõite? Ei tahtnud Sunile 
raha anda?}

Ühelt poolt ei tahtnud raha anda ja teiselt poolt oli see niisugune uus värske 
tuul, mis oli vaja ära proovida. Loomulikult Linuxi eelis oli see, et ta käis 
PC peal  ja  selles mõttes oli väga hea. 

\question{Linux on praeguseni hädas oma kõrge sisenemisbarjääriga, inimestel on 
raske temaga liikuma saada. Kuidas toona oli?}

Me ikkagi rääkisime Linuxist serveris, tööjaama-Linux ei olnud nagu teema. 
Sellel hetkel pidi raha eest ostma mingisuguse softi, et failiserverit ringi 
ajada. Ma ütlesin, et \enquote{Ärme teeme nii! Panen Linuxi püsti, kasutame 
seda}. 

\question{Tuletame meelde, et sel ajal taheti igasugu asjade eest nagu 
veebiserver raha saada ja kommertstarkvara oli too hetk maru kallis.}

Ta oli jah ropult kallis. Kui me hakkasime Küberis\index{Küberneetika 
Instituut} 1996. aastal tegema  esimesi tulemüüre nimega 
Barrikaad\index{Barrikaad},  siis sel hetkel maksis keskmine tulemüür maailmas 
kolm tuhat dollarit. See on ju absurdne. Me võtsime Linuxi, tegime näo pähe, ja 
müüsime viis korda odavamalt. Selles oligi asja mõte. 

Nii et, jah, omal ajal oli tarkvara väga kallis, kuna kirjutajaid oli vähe ja 
see oli nagu eksklusiivne värk. 

Seoses igasugu kogukondadega nagu Anna ja EUPS ei saa muidugi mainimata jätta 
ka olulist \emph{community}-t, mille nimi oli WC Fauna\index{WC Fauna}. On 
raske öelda, mis ta täpselt oli v˜øi kes sinna kuulusid, see oli rohkem nagu 
mõtteviis ja eluviis. Hulk inimesi, kes tegid igasugu asju. Pahatihti käisid  
lihtsalt kõrtsides või tegid niisama nalja või ehitasid lumelinna.

Vot vanasti olid veel kompuutrimessid tähtsad\sidenote{Aastatel 1993-1999 
korraldati Eestis iga-kevadist arvuti-, side- ja bürootehnika messi 
\enquote{Kompuuter}. Tegu oli tõepoolest olulise kogukondliku- ja 
müügiüritusega, mida Päevaleht lausa infotehnoloogia laulupeoks tituleeris. 
Kuna päev otsa messiboksis seista oli füüsiliselt raske, rakendas messidel 
osalenud Korel IN\index{Korel IN}  selleks kõiki vahendeid, sealhulgas meid 
Veljoga\index[ppl]{Hagu, Veljo}. Andmaks nohikutele vastu kevadet 
kaubanduslikumat välimust, saatis tööandja meid mõned korrad isegi solaariumi. 
Kahtlen, et sellest palju tulu oli, kuid messil osalemine oli tore kogemus 
igatahes.}. Ühel kompuutrmessil pakuti meile oma messiboksi ja me pidime selle 
kuidagi sisustama. Boksis oli üks kompuuter, mis luges sekundeid tuleviku 
alguseni ja WC Fauna leviala kaart, milleks oli punaste läbipaistvate 
vorstinahkadega kaetud Eesti kaart. Ja kõik oli politseilindiga ümber tõmmatud. 

\question{Tänapäeval läheks sihuke asi ju kunstiprojektina kirja?}

Jah, ilmselt küll. Eks ta häppeningi värk oli, igasugu asju sai tehtud. Oli 
selline kõikide IT-inimeste kokkutulek nimega 
OK-fest\index{OK-fest}\sidenote{1994. aastast Eesti Infotehnoloogia- ja 
Telekommunikatsiooniettevõtjate Liidu\index{Eesti Infotehnoloogia- ja 
Telekommunikatsiooniettevõtjate Liit} poolt korraldatud suvine kokkutulek.}, 
kus \emph{community} kokku sai. WC Fauna nimi sai alguse sellest, et ühel 
OK-festil oli vaja jalgpallimeeskond kokku panna. Mõtlesime, et FC Flora juba 
on, paneme siis WC Fauna. Aga see oli ka vist viimane kord, kui me jalgpalli 
mängisime. 

\question{Kui ma sind niimoodi kuulan, siis kumab jutust läbi hästi palju 
ühistegevust. Aga inimene tavaliselt ei tegele arvutitega põhjusel, et talle 
õudsalt meeldib inimestega tegeleda. Kuidas sul see arvutite ja inimeste suhe 
kokku käib?}

Eks ma olengi sihuke imelik loom, minust pole kunagi aru saanud. Mul üks tuttav 
psühholoog ütles, et \enquote{On olemas insenerid ja on olemas kunstiteadlased. 
Aga kumb sina oled, aru ei saa}. 

Võib olla inimene areneb ka vaikselt, Nagu ma rääkisin, et algusaegadel ma olin 
sihuke introvert, istusin nurgas ja progesin ning  kartsin, kui telefon 
helises. Aga siis kuidagi  asjad muutusivad, hakkasin inimestega suhtlema. 
Pärast juba mingisugust ühiskonda nägema ja sealt tulid ka niisugused riigi ja 
vaata et maailma laiused asjad. 

Üks lugu, milles mida kindlasti maksab rääkida ja võib-olla seda on räägitud 
ka, on see, kust Skype\index{Skype} tegelikult alguse  sai ja kus see kamp 
kogunes. Ma arvan, et nii mõnigi inimene mäletab, et kuskil umbes 1995. või 
1994. aastal oli kuulutus lehes, et \enquote{otsime programmeerijat, maksame 
viis tuhat krooni päev}\sidenote{Teiste allikate alusel oli kuulutus lehes 
1999. aastal. See viimane number on loogilisem - muidu jääb Skype asutamise ja 
Tele2 seikluse vahele liiga pikk paus.}. Viis tuhat krooni oli kaks kuupalka. 
Kuulutuse taga oli see, et Tele2\index{Tele2}, kes oli juba Eestis olemas, ja 
Bonnier Media\index{Bonnier Media}, kui ma ei eksi, septsisesid Rootsis 
nii-öelda uue põlvkonna portaali. Portaali nimi oli 
everyday.com\index{everyday.com}. Niipea, kui nad uudise välja lasid, et 
niisugune portaal tuleb, nende turuväärtus tõusis mingi poolteist miljardit. 
Absurdne värk, aga sel ajal toimis. Eestisse tuldi jutuga, meil on Itaalias 
tiim ja Rootsis ja  Taanis ja kõik on juba tükk aega progenud. Aga kahte 
programmeerijat Eestist on veel vaja, siis saab kõik korda\sidenote{Eestis 
töötas toona Tele2-s Stefan Öberg\index[ppl]{Öberg, Stefan}, kes hiljem Skype-s 
mitmeid juhtivaid rolle täitis. Tema juhataski viimase kahe tegija otsijad 
Eestisse.}. 

Mina sattusin seda kuidagi nagu nõustama ja siis lõpuks projektijuhiks, kes 
pidi need inimesed siis valima ja asjad ära tegema, sai need 
Bluemooni\index{Bluemoon} poisid välja valitud. Sõitsin kõik need Itaaliad ja 
Taanid ja Rootsid läbi ja sain aru, et peale Rootsi, kus oli tehtud väike 
andmebaasi mootor, olid kõik teised tiimid tootnud täielikku kräppi. Ja nii ei 
jäänudki projekti päästmiseks muud üle, kui kogu värk ise teha. No ja nendel  
poistel ei olnud mingi probleem see probleem nagu pihku võtta ja nädala-paariga 
portaal kokku veeretada, kuigi nad PHP-d\index{Keeled!PHP} ei tundnud.

Tulevane miljardär\sidenote{Tarvi peab silmas Niklas 
Zennströmi\index[ppl]{Zennström, Niklas}.} oli Tele2-s projektijuht ja talle 
hakkaksid need poisid meeldima, eks ole. 

\question{Sina olid seal portaalis projektijuht, eks. Tegid portaali valmis ja 
ei tekkinud mõtet, et peaks suures Rootsi kontsernis kosmilist karjääri tegema?}

Absoluutselt mitte. See oli ikkagi minu jaoks niisugune aitamisprojekt ja raha 
maksti ka ja no las ta olla. See  oli ikka  kõrvaltegevus mul.

\question{Aga mis su põhitegevus oli?}

Ma arvan, et sel hetkel ma ikka ehitasin seda riigivõrku ja juhatasin neid 
vägesid. Sinna kampa käib üks  kõrvaltegevus, sai tehtud oli, oli 
mail.ee\index{mail.ee} ka muidugi, selle omanikuks sai ka lõpuks Tele2. 


\question{Selle mail.ee all oli standardne SMTP server, eks?}

Täpselt nii. Ta algas ka pihta sellest, et oli meilboks, aga ilma näota 
meilboks. See tähendas, et igaüks sai endale aadressi luua aga  pidi  enda 
meilerit  kasutama. Ja siis teine arengufaas oli veebi nägu pähe teha sellele  
asjale, et seal oli veebi meiler ka. See viimane sai täitsa ise kirjutatud, all 
oli loomulikult standardne kompott. 

\question{Niiet sa ei läinud ise sinna maailma midagi leiutama vaid võtsid 
tükid ja ladusid kokku?}

Jaa, see on mul kogu aeg nagu veres olnud. Mingil hetkel ma sain aru, et  
progemine  on üldse kurjast, sest selles maailmas on kõik juba ära progetud. Et 
tegelikult  kunst on need tükid üles leida ja oskuslikult kokku panna. Aga 
tänapäeval on see tükkide arv muutunud hoomamatuks ja väga raske on.  On 
ilmselt tekkinud mingid kildkonnad ja voolud. Kunst on muutunud.

\question{Aga mis üldse on tänapäeval sinu jaoks programmeerimine?}

Eks ta kipub sihukene järjest igavam asi olema. Sellepärast et vanasti oli 
progemine päris selgelt  niisugune loometöö. Nii kui hakkasid igasugused 
mudelid ja RUP-id\sidenote{\emph{Rational Unified Process (RUP)}. RUP oli 
üheksakümnendatel suurorganisatsioonides levinud tarkvaraarenduse raamistik, 
mis keskendus arendusprotsessi keerukuse vähendamisele läbi standardiseeritud 
rutiinide. Et samal ajal üritati keerukat tarkvara tarnida harva ja suure 
pauguga, võis RUP küll teha projektid paremini kontrollitavaks, kuid, pehmelt 
öeldes, ei vähendanud kuigivõrd arendajate frustratsiooni.} asjad tulema, siis 
järjest rohkem hakkas mulle tunduma, et see on nagu rohkem kraavi kaevamine. Et 
tegelikult mingit arhitektid joonistavad asja ette ja sina lihtsalt täidad 
funktsiooni. See ei ole eriti keeruline. 

\question{Ometigi inimesed ehitavad igasuguseid hullusi nagu tekstiterminalis 
video maha mängimine?}

Loomulikult, nalja pärast saab ikka teha. Ma räägin ikkagi niisugusest raha 
eest või tööstuslikust progemisest, et sa pead mingit konkreetset asja tegema. 
Vanasti olid sa mees nagu orkester ja mõtlesid ise välja, kuidas arhitektuur 
võiks välja näha. Tegid  oma äranägemist järgi ja keegi ei kobisenud. Ja nüüd 
on arhitektid. Ühesõnaga loovust on vähemalt progemise mõttes päris 
programmeerijatele jäänud kindlasti vähemaks kui seda vanasti oli. 

\question{Kui me juba selle teema juurde jõudsime, siis ma küsin sinu käest ka. 
Milline on ilus kood sinu arvates?}

Sa küsid minu käest asju, millega ma tegelesin tõesti üle kahekümne aasta 
tagasi. Ilus kood on loetav kood loomulikult, siin ei ole kahtepidi mõtlemist. 
Ma rohkem ei oska siin midagi öelda

\bigskip
\noindent\rule{.3\textwidth}{.7pt}
\bigskip

\question{Jõuame ehk nüüd ringiga tagasi selle meie va ID-kaardi juurde. Kuidas 
see sündis?}

Küberis\index{Küber} ma tegutsesin kahel rindel. Ühelt poolt ehitasin võrke, 
aga ma olin tegelikult infoturbe ja krüptograafia keskel kogu aeg. Ja lisaks 
võrgu turbele, mis oli ka väga oluline sel hetkel, tundus avaliku võtme 
krüptograafia selgelt huvitav ala ja pakkus oma rakenduste poole pealt pinget. 
Küberis sai jälgitud, kuidas 1995. aastal vist Rootsi Post alustas oma 
ID-kaardi välja laskmisega ja avaldas oma id-kaardi profiili.  Päris vara, 
üheksakümnendatel, toodi mulle Ektaco-sse\index{Ektaco} mingid Schlumberger 
kiipkaardid ja öeldi, et \enquote{vaata, mis elukad need on}. Mispeale ma 
kirjutasin sinna pele programmi nimega \emph{Clevercard}.

\question{See oli siis java kaart või?}

Kus, Javat polnud veel väljagi mõeldud sel hetkel, Java kaarte  ei olnud ja 
krüptokaarte ka mitte. No kuidas sulle öelda, mälukaart ta ei ole,  protsessor 
on ikka sees. Talle saab käske anda, et tee nüüd fail või midagi. Kõige all on 
kaardi operatsioonisüsteem. Baidid ajasid sisse, baidid tulid vastu, kirjutasin 
PC-le  mingi proge, millega seda mõnusalt teha sai. 

Niiet kiipkaart oli tuntud loom. 

Aeg läks vaikselt edasi ja ma usun, et see oli mingi üheksakümne kuuendal 
aastal, kui tegin Äripäeva lahti ja seal oli esimesel leheküljel oli Kaja 
Kuivjõgi\index[ppl]{Kuivjõgi, Kaja} pilt, keda ma millegipärast tundsin ja kes 
oli sel hetkel Kodakondsus- ja Migratsiooniameti\index{Kodakondsus- ja 
Migratsiooniamet} asedirektor. Pildi juures oli jutt is ütles, et riik 
planeerib uut dokumenti ja et esimesed passid võeti 1992, 2002 saavad need 
otsa, praegu on viis aastat aega ja et KMA-s on moodustatud töörühm, kes siis 
uurib, mis variante oleks, et miski uuega välja tulla. 

Võtsin Kajaga ühendust ja ta  rääkis, et neil on töörühm ja puha ning salaja 
näitas mulle natukene  materjale ka, mis töörühmas oli arutatud. Kui ma  need  
läbi olin vaadanud, sain aru, et nende tehniline teadmus on üsna  allapoole 
nulli. Seal räägiti asjadest nagu kiibiga varustatud vöötkoodid. 

\question{Aga mida na siis teha tahtsid? Uut ja paremat passi?}

Eks nad sinna kaardi poole mõtlesid ikkagi. Aga milline see kaart nüüd on, et  
kas on kiibiga varustatud vöötkood või mis, ei olnud selge. 

\question{Näedsiis. Mina olen kogu aeg arvanud, et tehnikud pakkusid kaardi 
välja ja mitte ametnikud.}

Soov oli sel hetkel väga hägune soov, eks seal loomulikult olid igasugu 
variandid laua peal. Aga oli selge, et kuna tekib niisugune suurem 
passivahetus, siis on võimalik inimesi üllatada millegi uuega ning vaadata, mis 
maailmas tehnoloogia vallas toimub. 

Oli päris selge, et sellel KMA-sisesel\index{Kodakondsus- ja Migratsiooniamet} 
töörühmal ei ole mõtet jätkata. Ja oli kohe kähku ettepanek, et teeme nüüd ühe 
natukese laiema töörühma ja võtame  eksperte ka nagu laua taha: pangad ja 
telco-d ja riigi inimesed ja Küberi\index{Küber} inimesed ja nii edasi. Arutame 
seda asja!

\question{See tundub tänapäeval kuidagi veider, et riik võtab pangad ja 
telekomid  laua taha nihukest dokumenti arutama?} 

Absoluutselt mitte. Praegu ei tundu ja tol ajal ei tundunud kindlasti mitte. 
Sihuke laiapõhjaline koostöö riigi ja erasektori vahel on meile alati edu 
toonud nii ühes kui teises. 

\question{See on haruldane asi, mida mujal sageli ei ole.}

Eesti on nii väike riik et põhimõtteliselt sa tead kõiki, kes midagi teavad ja 
ei ole nagu mõtet kedagi kõrvale jätta, sellepärast et on erasektoris 
parasjagu. Me räägime ikkagi niisugusesst  ekspert-teadmisest ja 
ekspert-kogumist mitte mingisugusest institutsionaalsest asjast. 

No ja tuli töörühm kokku, arutas asju. Telliti kaks tööd, 
KMA\index{Kodakondsus- ja Migratsiooniamet} maksis. Üks oli umbes sihuke töö, 
mida aktsiaselts Aprote\index{Aprote} läbi viis, kes uuris, et milleks kõigeks 
võiks seda kaarti kasutada. Läksid näiteks tanklaketi juurde ja küsisid, et 
\enquote{noh mis te tahaks?}. Tulemus oli väga ulmeline muidugi. Aga see oli 
vajalik tegevus,  vaata et kohustuslik samm, et sa pead uurima turu ootusi. 
Teine töö, mida siis meie Küberis\index{Küber} tegime, oli see, et me tegime  
tehnoloogilise ülevaate, et milleks tänapäeva kiipkaardid on suutelised. Kaasa 
arvatud see,  mida on Rootsis tehtud, mida on Soomes tehtud ja nii edasi.  
Millised need profiilid on, millised on tehnoloogiad. Microsofti tehnoloogiaid, 
PC/SC\sidenote{\emph{Personal Computer/Smart Card - PS/SC}. Spetsifikatsioon 
tarkade kaartide integratsiooniks arvutustehnikaga.}, tehnoloogiaid, niisugune 
tehnoloogiline dokument. 

Ja mingi 1996, ma mäletan, ma joonistasin projektiplaani, et neljateist kuuga 
toome kaardi välja, kaasa arvatud piloot ja kõik värgid. Aga võttis viis 
aastat, nagu ikka tavaliselt. Sest see oli ikkagi väga oluline samm ühiskonnas 
ja tahtis pikemat kaalumist. Peale selle tuli seadusi juurde ja ringi teha. 

\question{Ja seda kõike vedas KMA?}\index{Kodakondsus- ja Migratsiooniamet}

Ei, kindlasti mitte. Digiallkirja seadust näiteks vedas selgelt 
Majandusministeerium\index{Majandusministeerium}.

\question{Oot-oot. Selle koha peal on nüüd lünk. Asi ju algas sellest, et peaks 
dokumenti väljastama, ja nüüd äkki tahab Majandusministeerium digiallkirja 
teha?}

Kindlasti oli mingi suunanäitaja Saksamaa, kes võttis esimesena vastu 
digiallkirja seaduse ja kust Eesti oma on väga palju maha viksitud. Meie 
digiallkirja seadus nägi enne ilmavalgust, vähemalt \emph{draft} sellest, kui 
oli olemas Euroopa direktiiv aastal 1999\sidenote{Euroopa parlamendi ja nõukogu 
direktiiv 1999/93/EÜ}. Seetõttu oli meie seadus mõnevõrra erinev. Euroopa 
direktiiv  lubas igasuguseid lahjasid allkirju ja igasugust koledust nagu  
näpuga ekraanile kirjutamist ning lõpuks ei läinudki tööle. Seepärast  tuli ka 
lõpuks eIDAS\sidenote{\emph{electronic IDentification, Authentication and trust 
Services - eIDAS}. Euroopa parlamendi ja nõukogu määrus 910/2014}, et direktiiv 
oli väga lahja. Kehitati õlgu ja ei kasutatud, tehti lahjasid allkirju ja 
öeldi, et nüüd ongi kõik hästi. 

Aga meil  algusest peale seadus ütles, tänases mõttes, et ainult 
kvalifitseeritud allkirjad\sidenote{Lihtsalt öeldes on kvalifitseeritud 
elektrooniline allkiri selline allkiri, mida võib pidada võrdväärseks 
omakäelise allkirjaga. Keerulisemalt on öeldud eelviidatud eIDAS direktiivis ja 
selle rakendusaktides.} on aktsepteeritud, et mingid lahjasid allkirju  ei 
tunnistata, neid seadus ei käsitlenudki. 

\question{Kas selle seaduse tegemise juures olid eksperdid juba kaasas või oli 
see Majandusministeeriumi tehtud?}

Seal olid eksperdid kaasas. Oli töörühm, kus mina ei olnud aga kus olid 
inimesed nii-öelda krüptoloogist kuni juura-teadlasteni. Nad tegid seda tööd 
kaks aastat, ma pakun. Nii et see ei olnud niisama, väga põhjalikult mõeldi 
läbi. Sest  väga palju peale Saksamaa ei olnud kuskilt šnitti võtta. Nii mõnigi 
seaduspunkt oli seal ikkagi inspireeritud nii-öelda krüptograafide mõtlemisest, 
et nii võiks teha. 

\question{Kuule, sinu jutust ei kõla läbi niisugune kõikehõlmav õilis visioon 
sellest, kuidas ühel päeval sünnib Eesti digiühiskond ja kõik saab e-teenuste 
abil uueks loodud?}

Eks ta võib-olla kuskil aju taganurgas on, aga mis sellest ikka rääkida, asju 
tuleb teha. 

\question{Seda ma peangi silmas, et liikumine toimus samm-sammult ja tegeldi 
konkreetsete asjadega}

Jah. Kasvõi seesama ID-kaardi väljatoomine.  Sa võid digiallkirja seaduse vastu 
võtta (mis aastal 2000 ka vastu võeti). Aga  kui inimestel ei ole vahendit, 
millega seda digiallkirja anda, siis pole sellel seadusel suuremat mõtet. 
Euroopas valitses ka selle direktiivi tegemise ajal nägemus, et  tulevad 
kommertsfirmad, hakkavad  sertifikaati müüma ja seetõttu on vaja neid 
reguleerida. No kuidas see võiks käia? Teed turule putka ja hakkad sertifikaate 
müüma: suured sertifikaadid ja väikesed, punased ja kollased. See 
ettenägemisvõime oli meil küll, et niisugune nägemus, et me müüme inimestele 
sertifikaate, neid ostetakse ja kuidagi tekib kasutus, et see on üdini vale 
asi. Selles mõttes oli meil Soome väga heaks näiteks, kes ID-kaardi tegi 
mitte-kohustuslikuks ja pani mingi nelikümmend eurot kohe  hinnaks. Mis 
seepeale juhtub, on see, et teenusepakkujad ei hakka su ID-kaarti toetama, sest 
nad teavad, et inimestel ei ole seda. Noh, viiel protsendil on, ma ei hakka 
selle pärast nagu tõmblema.  Ja inimestel ei ole kaarti, sest  et teenuseid ei 
ole. Ja siis ongi nokk kinni, saba kinni ja see on mudel, mis ei toimi. 

Selgelt on ikkagi nii, et sa pead kõigepealt selle  elektroonilise identiteedi 
taristu looma ja siis võib-olla hakkavad asjad juhtuma.  Samamoodi nagu sa ei 
lähe ja ei proovi kuskil metsa sees müüa kilomeetrit maanteed  kohalikule 
metsa-elanikule. 

\question{Su jutust kõlab läbi üsna suur usaldus ekspertide vastu. Seal pidi ju 
usalduslik vahekord olema selle poliitika eest vastutava inimese ja krüptoloogi 
vahel, et viimasel lasti seadusesse punkte kirjutada?}

See skepsis on nagu väga raske tekkima, kui sul on ikkagi Eesti paremad pead 
laua taga, et misasja sa kahtled või kõhkled. Mida targemaks inimesed saavad ja 
mida rohkem eksperte on, seda rohkem võib-olla tekib diskussiooni. 

\question{Kui suur see ekspertide ring oli, kes neis töörühmades käis ja seda 
asja kujundas, see ei saanud väga suur olla? Suurusjärgus?}

Pigem ütleme viis kuni kümme inimest, võtmeinimesed olid selged.

\question{Ja kogu see tarkus tugines tollele salapärasele raamatule, mis 
Küberis oli?}

Oo ei. Ma rääkisin, et see oli lihtsalt algus. Me ei räägi ainult 
krüptoloogiast või infoturbest laiemalt. ID-kaart näiteks ei ole  puhas 
infoturve, seal on väga palju rakenduslikke ja isegi sotsiaalseid  momente 
juures. Ei saa nagu rääkida, et krüptograafia päästis maailma. 

\question{Jällegi oluline arusaam. Sagedasti inimesed arvavavad, et kui saaks 
asjad ära krüptida, siis olekski maailm ära päästetud}

Jah, ma jään selle juurde, et sa ei saa kilomeetrit maanteed müüa 
külaelanikule. Sest ta küsib \enquote{Mis ma teen selle maanteega?}. 
\enquote{Hakkad autoga sõitma}. \enquote{Aga mis see auto on?}. See on selline 
ufo müümine, täiesti mõttetu tegevus. Sa ehitad teed valmis, lased autod müüki, 
paned sõidukoolid püsti ja siis ühel päeval võib-olla inimesed avastavad, et 
transpordist on midagi kasu. Aga kui sa hakkad sellest pihta, et proovid 
igaühele  juppi maanteed müüa, siis see ei toimi. 

\question{Küsin Küberi kohta. Küberneetika Instituut\index{Küberneetika 
Instituut} ja selle järelmid on väga pikalt Eestis kriitilist rolli mänginud. 
Sina oled olnud seal sees ja, mis veel olulisem, ka sellest väljas. Oskad sa 
öelda, mis see maagiline asi on, mis selle asutuse nii võimsaks teeb?}

1997 toimus siis suur jagunemine, kus Küberneetika Instituut läks kolmeks. Osa 
läks ülikooli, osast moodustus aktsiaselts ja andmeside osa läks  riigile. 
Võib-olla kõige nähtavam osa IT-inimeste jaoks on see Küberi kas instituudi või 
aktsiaseltsi või just see infoturbe ja progemise osa. Seal tehakse meremärke 
ka, aga need on merel ja ei paista välja.

Mul on endal au olnud, umbes kolmkümmend inimest sinna tööle võtta omal ajal ja 
Tartu labor asutada\index{Cybernetica!Andmeturbelabor}, mis on nüüd suurem 
isegi kui Tallinn. Nii et need olid väga toredad ajad. Aga mis selle fenomen 
ikka on. Fenomen on see, et põhimõtteliselt peetakse Küberit ainukeseks firmaks 
Eestis, kes oskab turvaliselt progeda ja teab midagi infoturbest. Ja seetõttu 
on neile ka sattunud niisugused tegevused ja projektid,  alates x-teest ja 
lõpetades Smart-ID-ga\index{SplitKey}, kus  turvalisuse ja krüptograafia komponent on omal 
kohal. 

Lisaks on seal on tõsiseid inimesi kes tegelevad puhtalt teadusega ja koodi ei 
kirjuta. Küberis on oma teadusosakond  ja oma teadusdirektor. Lisaks käivad 
teadusega tegelevad inimesed Küberi ja ülikoolide vahet. Selline sümbioos on 
miski, mida otsitakse nagu spunki mööda mööda Eestit taga ja ka Teaduste 
Akadeemia president ei väsi rääkimast, et Küber on fenomen ja suur erand. 

\question{No aga miks ei ole näiteks Helmes kah võtnud endale teadusdirektorit 
tööle ja hakanud ka tegema?}

No vaat,  see ongi, et kas sa teed kõigepealt teadust ja siis hakkad seda 
rakendama ühiskonda või sa proovid seda vastupidi teha. Et kõigepealt oled ilge 
progeja ja siis mõtled, et \enquote{paneks ikka teadust ka kuidagi}. See päris 
niimoodi ei käi, need juured on natuke sügavamad.

Vaatasin just täiesti kogemata Küberi töökuulutust paar päeva tagasi. Otsitakse 
projektijuhti. Projektijuhil  peab olema CISA 
sertifikaat\sidenote{\emph{Certified Information Systems Auditor - CISA}.}. 
Halleluuja!

Omal ajal  kõik teadsid, et Martens tuleb jälle ja meelitab Küberisse tööle 
nagu vastik jauraja. Mul mul oli väga lihtne äriidee:  ajad kõige targemad 
inimesed ühte suurde ruumi ja annad neile mingisuguse teema kätte. Nad lähevad 
plaks tööle, ise paned jalad seina peal. Töötas! Väga hästi töötas! Pärast 
lugesin kuskilt raamatust, et niimoodi tuleb käituda ja täpselt niimoodi ma 
olen käitunud.

%--------------------------


\question{Sa oled siis nihuke asjade käima lükkaja ja visioneerija?}

Kui sa nii ütled.

\question{Ma ei ütle, ma küsin. Programmeerija, sa ütlesid, sa enam ei ole. Kes 
sa niisugune oled?}

Jaa. Ma ei oska ennast nagu väga palju sildistada, eks ma   mõtlen kuidagi 
laiemalt kogu aeg, mul on see häda külges.

\question{Aga miks see häda on?}

Võiks ju midagi näpu vahel teha, kaltsuvaiba või midagi. On vähemalt tükk taga, 
suurtest sõnadest \ldots

\question{Mis sa praegu teed?}

No ma endiselt olen elektroonilise hääletuse juht, juba siis aastast 2003. 
Hiljuti olid meil  kümnendat valimised, kohe on algamas üheteistkümnendad, ehk 
siis Europarlamendi valimised\sidenote{See jutuajamine Tarviga leidis aset mai 
algul 2019. aastal, Europarlamendi valimised toimusid 26. mail, edukas 
elektrooniline hääletamine 16.-22. mail.}. Aga valimised võtavad võib-olla 
kaks-kolm kuud tähelepanu per valimine. Valimistevahelisel ajal ega ma palju 
suurt teegi. Jõudumööda nii nagu kutsutakse, käin maailmas ringi ja proovin  
natukene  inimesi aidata nende arengutes erinevates riikides. Nii 
elektroonilise identiteedi alal kui ka, kui kedagi huvitab,  IKT rakendamisest 
nii-öelda valimismajanduses.

