\index[ppl]{Vene, Vilve}

\question{Hakkame algusest pihta. Kuidas sina said arvutite juurde ja arvutid 
sinu juurde?} 

Tegelikult algas see juba koolis. Käisin 1. Keskkoolis\index{Tallinna 1. 
Keskkool}, mida täna tuntakse kui GAGi\index{Gustav Adolfi Gümnaasium}, 
matemaatika-füüsika klassis. Meile õpetati ka programmeerimist. Mul oli väga 
hea matemaatika õpetaja ja füüsika õpetaja, mulle õudselt meeldis, aga ma 
ilmselt ei teadnud, kas see meeldib mulle nii palju või meeldib mulle midagi 
muud rohkem. Mulle meeldis jubedalt kirjutada ja tegelikult kuni kuskil 
keskkooli lõpuni ma pigem arvasin, et ma lähen eesti keelt ja kirjandust 
õppima. Aga kuidagipidi see mõte arenes. Jõudsin sinnamaani välja, et õpiks 
füüsikat. Läksin juba Tartusse, vaatasin, et seal oli üldse vist neli tüdrukut. 
Mõtlesin, et jube veider, ja andsin avalduse rakendusmatemaatikasse\index{Tartu 
Ülikool!Matemaatikateaduskond!Rakendusmatemaatika}. See oli nii naljakas, et 
kui ma seda avaldust sisse viisin, see tädi seal laua taga vaatas mulle otsa ja 
ütles, et \enquote{aga teie oma  tunnistuse ja kuld medaliga saaksite ju 
arstiteaduskonda sisse!} Ma ikka selgitasin talle, et ma ei taha ja nii see 
läks. 

\question{Tuleme korra tagasi GAGi juurde. Mille peal teid programmeerima 
õpetati?}

Jessukesed\index{Jessuke} ikka ja perfokaardid ja kogu kogu see lugu. 

\question{Ja koolis oli kõik see olemas?}

Ei, me käisime Teaduste Akadeemia Arvutuskeskuses\index{Teaduste 
Akadeemia!Arvutuskeskus}.

\question{Kooli poolt väga edasipüüdlik ettevõtmine! Kes seda asja kooli poolt 
ajas?}

Ma arvan, et too aeg oli direktoriks Helmi Viikholm\index[ppl]{Viikholm, 
Helmi}\sidenote{Helmi Viikholm oli 1. Keskkooli direktor 1962 -- 1982.}. 
Selline inimene, kes ka selles nõukogudeaja \emph{setup}is suutis 
organiseerida. Meil olid jube head õpetajad, meil oli hästi avatud mõtlemine ja 
nii see tuligi.

\question{Teaduste akadeemia lasi õpilased oma arvutitele ligi?}

Jah!

Aga eks ta oli nagu niimoodi, et kood läks perfokaardi peale, ise toksime 
perfokaardid, need läksid masinasse, tulemus tuli välja, nagu ta käis. 

\question{Lihtsalt huvi pärast, mida te kirjutasite, mis programmid need olid?} 

Lihtsaid asju tegime. Ma ei mäleta enam enam niimoodi detailis, igatahes oli 
see Fortranis\index{Fortran}, seda ma tean küll. Aga mis selle asja sisu oli, 
seda ma ei mäleta. 

\question{Seni on ruutvõrrandi lahendamine umbes igas teises loos läbi 
jooksnud\ldots} 

Vaat seda ma ei mäleta, minu meelest olid meil ikkagi nagu natukene sellised 
ärilisemad teemad. 

\question{Misjärel sa läksid Tartusse Vanemuise tänavale\index{Tartu 
Ülikool!Vanemuise tänava õppehoone} rakendusmatemaatikat õppima?}

Jah. Meid alustas 20. See oli tegelikult täiesti kummaline. Minul, kellel 
koolis oli kõik jube lihtne, oli algus hästi raske. Ja alustati tegelikult 
assemblerist\index{Assembler}. Pärast ma olen mõelnud, et see oli mega hea, 
sest sa said kohe alguses selle\ldots Assembleris  sa pead aru saama, mis seal 
sees toimub, eks ole, sa ei saa lihtsalt kirjutada mingeid koodiridu arus 
aamata, mis seal sügaval sees toimub. 

\question{Selle raskuse ületamise taga pidi olema mingisugune kihu, miks sa ei 
läinud arstiks õppima?} 

Mind ei huvitanud!

\question{Aga arvutid huvitasid?} 

Ma siiamaani mõtlen, et kui edasi tulid need intelligentsed 
programmeerimiskeeled, kus sa ei pea  aru saama, mis seal sees toimub, siis 
(see ei ole küll populaarne) ma mõtlen, et tänapäeval on minu meelest 
programmeerimine käsitöö. Selles mõttes käsitöö, et ma koon kampsunit või ma 
teen programme. Sa oskad mingeid võtteid,  kasutada mingeid \emph{tool}e. Siis 
ei olnud nii. Kihvt oli nagu alustada nii, et nagu mängisid läbi: 
\enquote{Okei, see rida teeb nüüd seda, paneb selle sinna ja selle sinna. Kui 
mul on nüüd vaja võtta see sealt ja teha sellega midagi, mis ma selleks tegema 
pean?} Said nagu sügavuti aru. See oli hästi vahva. 

See viis, kuidas nad õpetasid, oli karm, aga ma arvan, et see tegelikult 
selekteeris välja need inimesed, kes tõesti tahtsid seda teha ja olid nagu 
võimelised. Sest  alguses tuli loogikat hästi kõvasti, tuli mat analüüsi. Ikka 
\emph{hard core}, mitte nagu see, mis tänapäeval õpetatakse. Just rääkisin ühe 
oma sõbrannaga, kes õppis teoreetilist matti ja kelle kursuselt on ülikoolis 
hästi palju õppejõude. Ja nad ei saa enam sellel tasemel õpetada, peavad oma 
programme lihtsustama. 

\question{On räägitud, et mõni on keskkooli ajal juba kuskil tööl, sest 
kangesti oli vaja arvuti juurde pääseda, sul ei olnud niisugust kihu?}

Ei, seda ei olnud. Ma ei ole kunagi sihuke nagu hull olnud, ma arvan, et see on 
võib-olla rohkem nagu poiste teema. 

\question{Õige, mu valim on seni väga kallutatud olnud.} 

Aga Tartus oli küll nii, et asi hakkas tõsiselt meeldima ja ma sain aru, et see 
on õige, mis ma teen. 

\question{Millest sa aru said?}

Mulle meeldis, lihtsalt meeldis! Ja see läks mu jaoks kuidagi nagu ajaga 
lihtsaks. Selles mõttes nagu lihtsaks, et et ma ei saanud aru, kuidas ma saan 
aru. Sihuke hästi filosoofiline. Kuna ta on ikkagi rakendusmatemaatika, siis 
lisaks  programmeerimisele oli seal ka hästi palju teoreetilist matti. Ja see 
läheb ikka vahel, näiteks matemaatiline analüüs, nii  abstraktseks, et sa 
tõesti enam ei saa aru, kuidas sa saad aru. Aga ma kuidagi toimisin selles 
asjas nagu jube hästi. Meil tegelikult kukkus niimoodi välja, et  meid  20 
alustas ja esimese aasta järel olid meid 14. See natuke näitab. 

\question{Väike grupp aga see väljalangevus ei olegi nii suur, meie kursuselt 
läks ikka rohkem. Aga, huvitav küll, aga mis sul tol ajal mõte oli, et sa 
tegema hakkad, kui kool läbi saab?}

Aeg oli ju selline, et valikuid loomulikult oli, eks ju, aga enamus valikud oli 
nii, et sa lähed kuskil arvutuskeskusesse tööle. Ja too aeg käis  niimoodi, et 
kohti said valida vastavalt sellele, kuidas sa lõpetasid ja mis oli see 
pingerida. Hästi popid, olid sihukesed kohad nagu näiteks nagu Raadiomaja 
Arvutuskeskus\index{Raadiomaja Arvutuskeskus}, see oli number üks. 
Statistikaametisse ei tahtnud keegi minna.

\question{Miks?}

Ei tea, ma ei oska seda selgitada, sest tagantjärele mõeldes on ju 
Statistikaametis töö ju palju huvitavam, kui kuskil raadiomajas. Aga nii ta 
kahjuks oli.

Aga minul oli selline õnnelik juhus, et ma sattusin Küberneetika 
Instituuti\index{Küberneetika Instituut}  praktikale. Viiendal kursusel tegin 
diplomitööd, minu diplomitöö juhendaja oli Otto Vaarman\index[ppl]{Vaarman, 
Otto}, kes oli väga tunnustatud matemaatik ja nii ma olin Küberis matemaatika 
osakonnas praktikal. Uurisime  Newtoni tüüpi meetodid, mis kõige paremini 
lahendavad erinevaid võrrandeid. Otto oli selline vana meesteadlane, tema ise 
väga palju ei viitsinud, tal oli selline noor jünger, just lõpetanud, paar 
aastat tagasi, ja see oli siis Maarika Lomp\index[ppl]{Lomp, Maarika}. Tema oli 
mul nagu sisuline juhendaja, tegime koos temaga. See kollektiiv hästi tore, 
hästi huvitav oli teha. Too tulid juba esimesed variandid, kus sa enam ei 
pidanud perfokaartide pealt toksima, vaid sa ikkagi olid 
Jessukese\index{Jessuke} taga, aga said juba ise intelligentsel viisil oma 
koodi  sisse viia. 

Ja siis sattus mingi hetk täiesti ootamatult Otule\index[ppl]{Vaarman, Otto} ja 
Maarikale\index[ppl]{Lomp, Maarika} üks härrasmees, kes oli ääretult sümpaatne. 
Ja tema oli sellisest huvitavast organisatsioonist nagu Algoritm.  Tegelikult  
oli sellel asutusel hästi pikk nimi, ma ei mäleta enam 
täpselt\sidenote[][-8cm]{\label{sisu:algoritm} 1976. aastal asutatud Tallinna 
Teadus-Tootmiskeskus (TTTK), mis kuulus  Üleliidulise Teadus-Tootmiserikoondise 
\enquote{Algoritm} koosseisu, Eestis tuntud lühendnime all Algoritm. Kuna 
asutus allus NSVL-i tasemel kaitsetööstuse ministeeriumide gruppi, oli talle 
omistatud koodnimi Postkast A-3433 ja kehtestatud ka vastav tööre\v{z}iim. 
Üleliiduline alluvus seletab ka töökeelt. Asutus tegeles ES EVM\index{ES EVM} 
hoolduse ja remondiga, IT-alase koolituse ja selle metoodikaga ning mitmesuguse 
tarkvara (näiteks matemaatika, arvutidiagnostika ja automatiseeritud 
projekteerimine) arendamise ning tira\v{z}eerimisega. Lisaks Tallinnale olid 
asutusel filiaalid Tartus ja Kohtla-Järvel (!). 1980-te keskel töötas 
Algoritmis ligi 1000 inimest, lisaks kaasati allhankijatena inimesi 
TPI-st\index{Tallinna Tehnikaülikool} ja Tartust\index{Tartu Ülikool}. Algoritm 
lõpetas töö 1992. aastal}. Küber oli Mustamäel, Kui sealt sõita edasi, kus on 
täna ARK, sealt veel edasi, selle risti peal olid baraki tüüpi 
majad\sidenote{Toonase aadressiga Kadaka puiestee 165.}. Väga  kole koht oli. 
Teadusuurimis keskus Algoritm, milles oli eraldi matemaatika osakond. Täiesti 
venelaste organisatsioon, tegelikult sõjaline ja tehti uurimistööd mingitel 
väga kummalistel aladel. Aga Ants\sidenote{Ants Roose\index[ppl]{Roose, Ants}. 
Tegelikult oli tema juhitava osakonna nimi \enquote{matemaatika tarkvara 
osakond}, hiljem töötas ta ka Algoritmi teadusala asedirektorina. Temalt on 
pärit kogu Algoritmi puudutav info.} oli siis selle matemaatika osakonna 
juhataja, see oli ainukene eestlaste osakond. 

Ühesõnaga tema rääkis mu ära. Ma läksin sinna tööle,  ääretult tore kollektiiv. 

\question{Sul siis ikka akadeemiline siht oli silma ees?}

Ja, ja ma pidingi tegelikult minema minema doktorantuuri Gennadi 
Vainiko\index[ppl]{Vainikko, Gennadi} juurde. Aga kuidagipidi  hakkasid ajad 
nagu arenema niimoodi, et hästi palju tuli selliseid huvitavaid teemasid. 

Äkki tasub  rääkida, kuidas tarkvara hangiti. Tarkvara hangiti niimoodi, et 
keegi tundis kedagi näiteks Minskis, kes oli kuidagipidi saanud mingi softi 
paketi. Kõik teame, kuidas neid too aeg saadi. Meil on seda vaja. Ülemus saatis 
siis minu, kes ma olin  noor tüdruk ja vene keelt väga hästi rääkinud, 
Minskisse ja ütles, et \enquote{sa pead sellega tagasi tulema!}. Ja noh, tagasi 
ma tulingi. 

\question{Aga \emph{millega} sa tagasi tulid?}

Suure lindikettaga!. Nii me seda tarkvara hakkisime, testisime. Tegime tööd ja 
oli nagu jumalast huvitav aeg.

Aga siis hakkasid ajad natuke muutuma, mul sai seal majas kolm aastat täis. Ain 
Rasva\index[ppl]{Rasva, Ain} oli muide tollal sealsamas ja Ain soovitas mind 
ühele inimesele, kes juba toona tegi koostööd soomlastega. See oli 88, asjad 
hakkasid avanema. Ja siis ma läksingi Tööstusprojekti\index{Tööstusprojekt} 
tööle, kus  olid juba too aeg ikkagi miniarvutid ja koostöö soomlastega ja. 

\question{Need on ju projekteerijad, mis seal programmeerida oli?}

Oi, tegelikult palju! Ehitusprojekt koosneb  hästi paljudest asjadest. 
Tugevusarvutused, kõikvõimalikud\ldots 

\question{See tundub olevat sihuke lahmimise koht, et sul oli võimalik minna 
kas  akadeemilisse maailma või sealt  ära. Miks sa just sihukese valiku tegid?} 

Pakkumine oli nagu tohutu ahvatlev. See, et ma saan teha tööd ka kuskil mujal, 
pakkumine tuli tegelikult sellega, et me saame teha Soome projekte koos 
soomlastega. Ja ma sain õppida. Ma mäletan, mulle anti C keele õpikud ja 
esimene mingi proovitöö oli automaatsed konverterid,  programmid mis  
Fortrani\index{Fortran} koodi koodi tõlkisid C\index{C} koodiks. See kood, mis 
välja tuli, oli muidugi kohutav, aga töötas. Selliseid asju tegime. 1990 ma 
olin natuke aega lapsega kodus. Tegin Soome väikseid töid, teenisin seal oma 
saja marga kaupa  väga head raha. 

\question{Jõhker raha, sada marka!}

Jah!

Ja nii see arenes, kui siis 1991. aasta talvel tuli selline kuulutus. Ajalehes 
oli kuulutus, et Rootsi-Eesti ühisfirma otsib programmeerijaid. Ja et need 
peavad olema naisterahvad. Ma polnud elus niisugust asja näinud aga tundus väga 
huvitav. Mingi üle 600 oli minu meelest neid taotlusi. 

\question{600!? See tähendas, et 1991. aastal oli Eesti Vabariigis aktiivsel 
tööturul 600 naisterahvast, kes võisid enda kohta \enquote{programmeerija} 
öelda?}

Jah. Esimesel intervjuul selgus ka, väga lihtne loogika, et miks nad naisi 
otsisid. Selle firma taga oli üks rootslane, kes töötas Stockholmi linna ja 
lääni valitsuses. Minu arust oli ta lausa IT osakonna juhataja, või midagi 
sellist. Ja  üks Eestist Rootsi ära läinud mees, mäletasin teda ülikooli ajast, 
tema oli viiendal kursusel, mina olin esimesel, Kalle 
Kullmann\index[ppl]{Kullmann, Kalle}. Neil koos tekkis idee, et kui nüüd Eesti 
vabaks saab, saab sealt odavalt head tööjõudu. Tegid firma, mis pidi hakkama 
Stockholmi lääni valitsusele teenust osutama. Ja naisi otsiti selle pärast, et 
naised on korralikumad, leplikumad ja küsivad vähem raha. Meid see too hetk 
eriti ei häirinud, sest, kujuta ette, saada tööle firmasse, mis maksab palka 
valuutas ja teha Rootsi tööd! Fantastiline!


Välja valiti kolm naisterahvast, üks oli jällegi Maarika Lomp\index[ppl]{Lomp, 
Maarika}, teine oli mina ja kolmas oli siis üks kolmas naisterahvas. Meile 
üüriti kontoriruumid, kus vana ajakirjandusmaja taga. Me pidime ise panema 
püsti kõik kohtvõrgud ja asjad, kõik! 

\question{Tollal selline asi nagu kohtvõrk?}

Jah! Panime püsti. Üle modemite suhtlesime Rootsiga ja kõik nagu toimis. 

Muidugi kasutasime tuttavate meesterahvaste abi, kes olid juba võib-olla rohkem 
 võrgu ja selle poole peal, aga hakkama saime. 

Programmeerisime sellises huvitavas keeles nagu Magic\index{Magic}. Oled 
kuulnud? 

\question{MUMPS-ist olen aga Magicust mitte.}

See oli 4GL juba toona. Ta oli Iisraeli päritoluga\sidenote{Platvormi tootnud 
Magic Software Enterprises oli tõesti asutatud 1983. aastal Iisraelis.}. Üldse 
oli kasutada,  ma võin nüüd eksida, mingi kaheksa erinevat käsku, neile panid 
parameetrid taha ja nendest panid koodi kokku. Ja koodi sai täita vastavalt 
vajadusele kas eest tahapoole või tagant ettepoole. Täiesti müstiline asi! Ja 
sellega sai tegelikult teha kõike. Esimese projektina tegime nende varade ehk 
siis autopargi haldusprogrammi, kus oli siis kõik alates muruniitjatest lõpetas 
autodega. 

\question{Kui ma sind kuulan, siis kerkib esile huvitav kontrast. See vahend, 
millega te tegite, kõlab meeletult palju keerulisemana kui tänapäeval 
kasutatavad. Aga see ülesanne, mida te lahendasite, kõlab nagu meeletult palju 
lihtsam, kui tänapäeval tavaliselt lahendatakse. Kas teile see tegevus kahuriga 
kärbes tapmisena ei tundunud?}

Ma ei oska sulle öelda, miks nad valisid sellise. Tollal ju tegelikult 
selliseid\ldots No igatahes ei kaldunud asi mitte mingil juhul mingi FoxPro, 
Paradoksi või mõne muu sellise asja poole, mis toona juba olemas olid. 
Võib-olla see oli see natukene päritoluga seotud, kuna Kalle ise juut ja tal 
olid Iisraeliga tihedad sidemed. Ja kuna ta ise oli hästi kõva matemaatik, 
hästi selline keeruliste ülesannete lahendaja, siis talle endale ilmselt see 
süsteem sümpatiseeris. 

See oli tore aeg. Kasutajaliidesed olid rootsikeelsed ja siiamaani on mingit 
sõnavara meeles. Kolmekesi saime palka kaheksasada Rootsi krooni kuus, mis oli 
too aeg siin suur raha. Seal oli see, et kui meid viidi restorani kliendiga 
kohtuma, siis meie igaühe restoraniarve oli suurem kui meie kuu palk. Aga jah, 
muidugi tegime  pikki päevi ja oli päris karm. 

\question{Huvitav, et selline programmeerija amet oli olemas. Sest näiteks 
rääkis Henn Ruukel\index[ppl]{Ruukel, Henn}, et tema mälestuste järgi enamasti 
inimesed ei tegelenud igapäevatööna programmeerimisega. Et leiva lauale tõi 
ikka see, et sa vedasid kuskil kaablit või panid arvuteid kokku?}

No vot, meie tegime! Tegime algusest lõpuni: mõtlesime välja, disainis 
mudeleid, programmeerisime selle ära ja ka juurutasime Rootsis kliendi juures. 

\question{Kui teil arvuti oli ja modem oli\sidenote{Ja pidi olema ka võimalus 
välismaa numbritele helistada!}, kas teil huvi ei tekkinud, et mida nendega 
veel teha saab?}

Tead, ma arvan, et meil oli lihtsalt\ldots Ma  mäletan päevi, mil sa alustasid 
kell kaheksa ja lõpetasid kell 12 öösel, eks.  Sul võis hästi palju huvisid 
olla, aga lihtsalt ei jõudnud nendeni. Tegelikult too aeg tärkas mu huvi 
andmebaaside vastu. Sest meil anti üks raamat (\enquote{lugege, tüdrukud, 
läbi!}), mis minu jaoks esimest korda kirjeldas  relatsiooniliste andmebaaside 
teooriat. Rootslaste juures ma olin umbes aasta, siis  nägin 
Hansapanga\index{Hansapank} kuulutust. 

\question{Hansapangal oli kuulutus?}

Ta ei otsinud üldse IT-inimesi, ostsis ma isegi ei mäleta, keda. Parasjagu 
lõpetasin oma kaheteisttunnist tööpäeva ja ütlesin Maarikale, et \enquote{okei, 
ma kirjutan}. Saatsin oma CV, see oli 1992. aasta november. Järgmine päev 
helistas mulle Tõnis Sildmäe\index[ppl]{Sildmäe, Tõnis} ja kutsus intervjuule. 

Mäletan, et kui ma seal vestlesin (Liivi\index[ppl]{Kompus, 
Liivi}\sidenote{Liivi Kompus, üks Hansapanga legendaarseid IT-inimesi.} oli 
ka), rääkisin väga uhkelt oma Rootsi kogemusest ja teadsin juba 
relatsioonilistest andmebaasidest ja jätsin ikka tohutult muljet.  Jüri 
Mõis\index[ppl]{Mõis, Jüri} läks vahepeal mööda, ütles \enquote{Tõnis, kui sina 
ei taha, ma võtan ise selle tüdruku!}.  Ja nii mind siis tööle võeti. 

\question{Kui suur Hansapank\index{Hansapank} toona oli?}
Ma arvan, et  oli kuskil umbes 40 inimest. IT-s mina olin 13. See oli siis 
Crebit\index{Crebit}, tegelikult Spin Development\index{Spin 
Development|see{Crebit}} veel tol ajal, mis oli pangast eraldi. 

\question{Kui organisatsioonis on kokku 40  inimest ja neist 13 on IT-inimesed, 
siis see on ju päris oluline protsent!} 

Ta oli tõesti väga väike, sellepärast et too hetk oli ju see, et inimesed tegid 
hästi laia laia teemade ringi alates klienditeenindusest kuni raamatupidamiseni 
olid ühed samad inimesed. Ja sealt kasvasid siis välja nagu Agve 
Aasmaa\index[ppl]{Aasmaa, Agve} näiteks, kes tuli tellerina tööle, Tea 
Trahov\index[ppl]{Trahov, Tea} samamoodi\sidenote{Mõlemad olid legendaarsed 
Hansapankurid.}. Kõik  kasvasid sealt alt. Aga vähe oli. Ma mäletan, et kui 
olid mingid tähistamised, me mahtusime mingisse väikesesse ruumi. 

\question{Kust see  suhteliselt ikkagi kõrge IT osakaal ikkagi tuli? Praegu ei 
ole ju veerand  Swedbanka IT?} 

See visioon (nüüd ma imetlen!), millega omal ajal Hansapanka\index{Hansapank} 
tehti! Taheti teha tõesti midagi, nagu praegu räägitakse \emph{start-up}'idest, 
et jube \emph{cool}. Tegelikult need mehed tegid täielikku \emph{start-up}'i. 
Nad tahtsid teha täiesti teistsugust panka, kus  paberkataloogide asemel oli 
arvuti. 

\question{Kust neil tekkis sihuke arusaam, et nii üldse on võimalik? Kes neile 
selle pähe pani või korjasid ise kuskilt üles?} 

Ma ei ole kunagi küsinud, aga ma arvan, et see tuli kuidagi nagu koostöös. 
Hästi palju oli  ilmselt Tõnis Sildmäe\index[ppl]{Sildmäe, Tõnis} panust, et ta 
müüs seda ideed, et nii saab teha. Nad olid ju sõpruskond tegelikult, eksole. 
No näiteks kust tuleb idee, et ma panen kõik kontorid onlainis juba sel ajal. 
Mõned aastad hiljem, kui ma käisime kohtumas Inglismaal ja Iirimaal nende 
suurte ja vanade pankadega, siis kõik vaatasid, et \enquote{Issand, lapsed, mis 
te räägite! Et teil on mingi Paradoxi lahendus, ja see töötab onlainis?}. Aga 
see oli nagu selline julgete mõtete maailm, eks ole, aga samas ei tehtud nagu 
lollusi. Kogu aeg õpiti. Keegi ei teadnud, keegi ei osanud. Siis mõeldi, et 
kuidas me siis nüüd sellest üle saame? Kuidas ma hakkame oskama? Saadeti keegi 
kuskil õpime midagi\dots Minu esimene ülesanne oligi see, et 
Tõnis\index[ppl]{Sildmäe, Tõnis} ütles, \enquote{Näed, mul on siin neli 
programmeerijat kirjutanud. Kes on teinud laenu, kes on teinud kontosid. Vaata 
ja ütle, et mis võiks olla teistmoodi.} Tõesti, vaatasin nagu peale, stiili 
järgi oli kohe näha, et see on Liivi Kompus, see on Kadri Trahov, see on Tõnis 
Argus, igaühel oli täiesti oma stiil. Andmebaas oli selline, nagu ta oli aga 
töötas. Kõik nagu töötas. Kirjutasin siis ettepanekud. Loomulikult ei olnud nad 
selle tehnoloogia peal teostatvad, aga sealt sai alguse mõte, et viime oma 
süsteemi Oracle peale, teeme uue pangasüsteemi ja viime asja järgmisele 
tasemele.

Aga mis oli minu arust kogu selle eduloo \emph{point} oli ääretult avatud 
suhtlus ääretult avatud suhtlus, kõik oli  ära tegemisele suunatud. 

\question{Mis seda asja takistas nurja minemast? Kui hakata niisama nullist 
kõrgtehnoloogilist panka tegema, see võib ju vussi minna?}

Tead, mul alati usk, et inimesed on \emph{key}. Siiamaani. Inimesed, kellega sa 
mingit asja teed asi võib minna kas väga hästi või võib minna ka totaalselt 
tuksi. Üks pool muidugi on inimeste oskused ja see pool, aga isegi rohkem 
määrab ära suhtumine ja ambitsioon. 

\question{Milline  suhtumine siis olema peaks?}

See, et ma tahan midagi ära teha. Aga samas ma tean, et ma ei tee seda üksi, 
vaid me teeme seda koos. Soov koos midagi ära teha. Samal ajal ka arusaam 
sellest, miks ma midagi teen. Hästi palju initsiatiive on selliseid, et mul on  
hästi huvitav tehnoloogia. Aga kas see tehnoloogia lahendab mingeid probleeme 
või mitte, selle peale liiga palju ei mõelda. Just  koos ära tegemine ja mida 
kindlasti ei olnud, oli see, et \enquote{kes on kõvem}.

Inimestel, kes seal olid tol hetkel, oli hästi kõva ambitsioon. Aga see 
ambitsioon ei olnud kindlasti mitte isiklik karjäär.  Meeskonna ambitsioon ja 
meeskonna saavutus. Tänapäeval räägitakse hästipalju ja räägiti ka Swedpangas 
aastal 2000, kui rootslased tulid, et protsessid, kvaliteedijuhtimine ja peab 
olema terve tohutu hulk dokumente ja siis ma teen plaane ja  raporteerin ja 
kogu mu jõud lähebki  selle peale. Siis tegelikult ma võin öelda, et seda 
Hansapangas ei olnud. Me ei mõelnud niimoodi. Meil oli see, et kui pank tahtis 
välja tulla eraisiku pangakontodega, kutsus Jüri Mõis\index[ppl]{Mõis, Jüri} 
ühte ruumi kokku kõik, kellel võis olla sellega mingit pistmist või mingit 
arvamust. Rääkis, miks ta seda teha. Ja siis mõtlesime, et mis selleks tegema 
peab. Ja edasi igaüks hakkas tegema seda, mis tema osa oli. Tehti koos ja tehti 
hästi ruttu. 

\question{Kui ma peegeldan tagasi, siis Magicu moodi asjadega tegelemine annab 
ju päris hea immuunsuse tehnoloogia järgi jooksmise vastu. Et kui oled korra 
pidanud tagurpidi käivat programmi kirjutama, siis ei ole miski enam nagu väga 
uudne!}

Oluline on see, et sa kasutad õiget asja õiges kohas. Populistlikud jooksmised 
mingi asja järgi\ldots See on see, miks ma ei poolda näiteks seda initsiatiivi, 
mis riigis  kunagi oli, et kõik programmid tuleb iga 13 aasta järel ümber 
kirjutada. \emph{Sorry}, et võib-olla ma ei peaks neid 13 aasta tagant ümber 
kirjutama, kui ma kogu aeg teeks nendega midagi? Aitaks järgi ja muudaks? See 
oli see, mis me pangas tegime. Vaatasime täpselt seda,  kust meil tulevikus 
pigistama hakkab ja muutsime, vahetasime välja. See oli protsess. 

\question{Ometi pangas tehnoloogia lained ju pidid tulema, tulid uued 
tehnoloogiad ja nii edasi. Kuidas te otsustasite mida üles korjata?}

Ega me tegime valeotsuseid ka, ega keegi ei ole ju immuunne nende asjad suhtes. 
Kui tehnoloogia poolest rääkida, siis  esimene laine oli see, et meil oli 
Paradox\index{Paradox}, mis oli failipõhine süsteem ja millest me kindlasti 
nägime, et see hakkab meil takistuseks saama. Kas või näiteks see, et me ei 
suutnud olla 24h kättesaadavad, päeva vahetuse teema\sidenote{Päeva vahetus on 
pangas oluline (ja keeruline ning seetõttu arvutuslikus mõttes kaua aega 
võttev) toiming, mille käigus tehakse selle päeva seisuga raamatupidamiskanded, 
toimetatakse arveldused, esitatakse aruanded ja, lühidalt öeldes, üks pangapäev 
asendub teisega.} ja kõik see. Minul oli relatsiooniliste baaside  teoreetiline 
teadmine ja Tarmo Pajumets\index[ppl]{Pajumets, Tarmo} oli töötanud mingi pool 
aastat Soomes ja arendanud Oracle-põhiseid süsteeme. Oracle\index{Oracle} oli 
see hetk  Eestis ikkagi number üks, Elion, EMT, ta oli igal pool. Tekkis mingi 
teadmine Oracle kohta, kui me nüüd andmebaasidest räägime. Ja üks põhjus, miks, 
Oracle tol hetkel tugevalt Eesti turule tuli,  oli see, et meie tugi oli Soomes 
ja Soome Oracle oli väga tugev organisatsioon. Kui ma alustasin uue süsteemi 
arendamist,  mul oli tugi telefoni otsas, võisin helistada Soome ja küsida, et 
mingi bugi, mingi asi, et \enquote{kuule, miks see ei tööta?}. Mulle oli otse 
liin Soome. Võrdle sellega, mis on praegu!

Ja nii me konvertisime,  natuke nagu see Fortrani konvertimine, oma Paradoxi 
Oracle peale.  Oli täiesti olemas selline automaatse konverteerimise võimalus, 
et meil kasutajaliides jäi endiselt Paradoxi ja baas taga oli Oracle. Mis andis 
meile selle vabaduse, et me saime teha oma päevavahetuse protsessid kõik 
niimoodi, et see oli rohkem 24h, saime hakata kaarte sinna külge panema, saime  
tulevikus ka näiteks Telehansa\index{Telehansa} sinna külge, kanalid tulid 
külge juba too aeg. Ja hakkasime seda Paradoxi rakendust ennast ümber 
kirjutama. 

\question{Kas Telehansa tuli enne, kui Forexi modemipank või oli ta hiljem?}

Minu meelest kuskil samal ajal, ma ei mäleta, millal täpselt.

\question{Sest Mast, kes selle Forexi asja kirjutas, rääkis, et tolle 
käivitamisüritusel  istunud Hansapanga tütarlapsed esireas ja teinud ohtralt 
märkmeid!}

Ta võis olla niimoodi, ta tuli kuskil enam-vähem sinna otsa, ega seal palju 
vahet ei olnud. 

\question{Aga miks te Telehansa\index{Telehansa} tegite?}

Kogu see mõtlemine, et sul on küll kontorid, aga firmade raamatupidajad ei 
tahtnud oma maksetega kontorisse tulla. See hakkas tegelikult sealt peale. Meil 
oli palju firmasid, Hansa tegelikult ju, kui vaadata panga  arengut, siis ta  
kõigepealt võttis ju sellised eesrindlikumad firmad ja siis tulid eraisikud 
pangakontodega järele. 

Ikkagi see, et ühelt poolt elu mugavamaks tegemine ja kindlasti oli ta ju 
rahaliselt kasulik. Võtad kontori \emph{load}'i maha, hoiad kokku. Ja kindlasti 
kogu see innovatsiooni pool, kust see asi algas, et me oleme teistmoodi pank, 
teeme asju teistmoodi. 

\question{Kui palju Telehansa tuumsüsteemi muutust eeldas? Ta ju vajab hoopis 
teistsugust  interaktsiooni tuumaga?} 

Tegelikult ei olnud Telehansa tegemine väga keeruline, sellepärast, et see 
andmebaasi vahetus tegelikult lõi selle võimaluse. 

\question{Ja ta ajaliselt tekkis umbes samas kandis?}

Jah. Andmebaasi vahetus oli  kuskil 1994, kuskil sealkandis ta tuli. Telehansa 
oli väga kõva asi, tegelikult.

\question{Telehansa on siiamaani väga kõva asi!}

Ja kolme mehe poolt tehtud ja kirjutatud. Toomas Lassmann\index[ppl]{Lassmann, 
Toomas}, Madis Tapupere\index[ppl]{Tapupere, Madis} ja Riho-Rene 
Ellermaa\index[ppl]{Ellermaa, Riho-Rene}. Kolmekesi tegid. 

\question{Tolleks hetkeks oli IT-inimesi juba rohkem kui neliteist?} 

Jah. Toomas Rand ja mina  kirjutasime taga seda \emph{process}'imist, et kui 
maksed sisse tulid, siis mis nendega sai. Tiim kasvas väga kiiresti. 14 oli 
1992 alguses ja, ma praegu võin natuke spekuleerida, aga kahekordistus umbes 
aasta või pooleteisega. 

\question{Kuidas te seda kasvu kontrolli all hoidsite? Kui sa nii kähku kasvad, 
on ju suur tõenäosus, et sa mõne lolli ka kogemata palgale võtad?}

Ma nii selgelt mäletan, et kui me olime kasvanud kuskil 50-ni, siis me tegime 
oma esimesed kompromissid. Enne oli see, et me ikka valisime inimesi väga. 
Esiteks, et nad oleksid professionaalsed, tipp. Hansal oli ka võimalik valida, 
niisugune Eesti majanduse lipulaev! Ja teiseks, et ta inimesena sobiks väga 
hästi tiimi. Aga siis me jah tegime esimesed\ldots

Ja siis hakkasid tekkima küsimused,  et kuidas seda asja hallata, hakkasid 
tekkima esimesed probleemid. Ja see \emph{learning by doing}, see kogemus, mis 
sa sellest said! Tänapäeval, ma arvan, et väga vähestel inimestel on olnud 
selline võimalus kasvada koos organisatsiooniga väikesest suureks. Ja väga 
kiiresti kasvada. 

Näiteks üks teema, mis tekkis, oli see, et kui üks inimene oli arendanud 
näiteks laenu süsteemi või väärtpaberite süsteemi, siis selleks, et uut 
funktsionaalsust teha, tal enam ei jätkunud aega, kuna ta tuli nii palju 
probleeme kogu aeg teiselt poolt peale, mida ta pidi lahendama. Kes vajas 
raportit, kellel oli võib-olla mingi mingi \emph{case},  mis ei mahtunud sisse 
või mingi  probleem ja seesama inimene tegeles mõlemaga. See hetk me  
tekitasime halduse poole pealt rakendusete halduse osakonna, kus siis olid 
inimesed, kes oskasid teha lihtsama probleemi lahendust, oskasid genereerida 
raporteid, tundsid andmed, mis andmebaasis on, ühesõnaga olid arendandajal 
kõrval, et arendajal oleks rohkem aega. 

\question{See otsus oli ikkagi hästi praktiliste juurtega, kui ma nüüd tagasi 
peegeldan? Et mitte te ei olnud kuulnud, et kuskil peaks olema 
rakendusadministraatorid, vaid et oli vaja konkreetset asja teha, leiti 
inimesed, koolitati ja pandi seda asja tegema?}

Seal organisatsioonis sündis kõik too aeg praktilisest vajadusest. 

Mul on tohutu austus kadunud Tõnis Sildmäe\index[ppl]{Sildmäe, Tõnis} kui juhi 
vastu. Kuidas ta seda tiimi juhtis. Juhtkond moodustus inimestest, kes suutsid 
vedada mingeid teemasid või olid juhi potentsiaaliga aga spetsialistid. Ta 
usaldas sind täielikult. Ta ei tulnud sulle kunagi ütlema, mida sa tegema pead. 
Ainuke võib olla negatiivne asi oli see, et ta kaitses sind liiga palju. Kuigi 
keegi sulle kallale tuli, ta läks kohe võistlusse su eest ja teda pidi tagasi 
hoidma. Aga see lõigi  kultuuri, et kui oli mingi teema, tulime istusime kokku 
ja arutasime, kuidas on seda mõistlik lahendada. 

\question{Kui ma su juttu kuulan, siis see on ikkagi enamasti jutt nagu 
inimestest ja juhtimisest ja väga vähe jutt sellest, et Oracle indekseid peaks 
tegema niiviisi?} 

Ma arvan, et indeksite tegemine, on lihtsalt tehnika. Loomulikult peavad olema 
mingid oskused, sa pead asju teadma, eks ole, aga see kõik on õpitav. See, kui 
hästi ma neid indeksid teen, ei või ka see, et kui ilusat koodi ma teen (ilus 
kood peab olema!), ei ole see, mis toob edu. 

\question{Aga milline on ilus kood?}

Ma ütleks seda, et ilus kood on (ja see on nüüd hästi nagu populistlikult 
öeldud) kood, millest saab aru. 

Kui ma isegi ei valda täielikult seda programmeerimiskeelt, või seda 
tehnoloogiat, milles see on kirjutatud, ma vaatan koodile peale ja ma saan aru. 
No loomulikult, kui mul ei ole üldse mingi kogemust sellel alal, siis ma 
võib-olla ei saa aru. Aga kui ma oskan C-d  kirjutada, ma vaatan Java koodile 
peale ja ma suudan seda lugeda. 

\question{Millist me  siis järeldame, et ilus kood sõltub sellest, keda sa 
arvad seda koodi hiljem lugevat. Et kas esimese kursuse tudeng või 20 aastat 
progenud inimene?} 

Ma arvan, et ka esimese kursuse tudeng võiks aru saada!

Mul on endal just andmebaaside poole pealt hästi palju kogemust, PL/SQL ja see 
pool või PostgreSQL baasi protseduurid. Omal ajal ma vaatasin nelja inimese 
koodile peale siinsamas majas\sidenote{Meie jutuajamine toimus toonases Icefire 
kontoris aadressil Kauba 2a, Tallinn.}, väga head, väga tugevad tegijad sellel 
alal. Ja ma võin öelda, et kaks neist kirjutas väga ilusat koodi, kaks mitte. 
Kood töötas perfektselt ja kõik neli on tehniliselt väga head. 

\question{Aga mõnel on ilus kood ja mõnel ei ole. Kunst ju?}

Jah, see ongi kunst. See hakkab peale sellest, et kuidas ma mõtlen, kuidas ma  
oskan maailma abstraktselt kujutada. 

\question{Mis läheb kokku sellega, mida Ahti rääkis sellest, kuidas tema 
kõigepealt õpiski programmeerima just nimelt paberi peal just nimelt mõeldes 
sellest programmist. Tema rõhutas ka võimekust oma peas asja ette kujutada} 

Jah. Teine pool on see, et kujutuma nagu reaalset maailma. Vahel on see teine 
pool, et kui ma abstraheerin lahenduse liiga ära,m tulevad sellised nagu 
maailma mudelid. Ja need üldiselt kunagi ei toeta\ldots. 

\question{Oleme näinud\sidenote{Ja, paraku, ka teinud.}. Aga kuidas seda 
tasakaalu siis hoida? Et oleks piisavalt üldine, et ta ei pane arendust kinni 
aga ta ei ürita ka maailma  mudeldada?} 

Ma ei oska sulle öelda, see on ikkagi nende inimeste\ldots Ma arvan, et see on 
mõtteviisis kinni ja tuleb kogemusega loomulikult, aga igaühele ta ei tule. 
Võib-olla see analüütiline mõtteviis on ikkagi hästi\ldots 

\question{Kui ma sind kuulen ja asju kokku panen, siis kõlab nii, et kui sul on 
hästi kokku pandud tiim, siis see tiim jõuab selle tasakaaluni nagu loomulikult 
oma kogemuste ja oskuste ja parasjagu käes oleva ülesande  pealt} 

Just. Muidugi tiimi töö. Minu arust ei saa nagu üle väärtustada seda, et 
inimesed koos mõtlevad. See väärtus, mis koos mõeldes välja tuleb on tükk maad 
suurem. Meil on väga hea näide, eks ole, Jan\sidenote{Vilve peab silmas Jan 
Laksperet\index[ppl]{Lakspere, Jan}, legendaarset Hansapankurit, kellega ta on 
intervjuu hetkeks koos töötanud pealt kahe aastakümne.} versus mina. Jan on 
perfektsete lahenduste mees. Ta lahendused üldiselt katavad ära kõik asjad, 
kõik ääre-\emph{case}'d, mis lõpptulemusena võib tekitada situatsiooni, lahedus 
läheb liiga keeruliseks. Ja mina suudan jällegi tulla selle poole pealt, et kus 
on nagu mõistlik, me koos töötame nagu jube hästi. 

\question{Tulles tagasi Hansapanga aja juurde. Kui Tõnis\index[ppl]{Sildmäe, 
Tõnis} oma juhtkonda kokku pani, oli see ka ju sinu jaoks valiku koht. Kas 
hakata inimesi juhtima või jääda koodi kirjutama. Tihti ju öeldakse, et kui 
heast programmeerijast juht teha, saad omale kehva juhi ja saad heast 
programmeerijast lahti. Sul seda hirmu ei olnud?}

See oli see protsess, see ei juhtunud ühe päevaga. Kui ma mõtlen, siis ma 
kirjutasin ju Hansas peaaegu lõpuni ka koodi. See juhtus nagu protsessina ja 
balansseerituna. 

\question{Kuidas sa hoidsid seda tasakaalu? Jube lihtne on, olles ka natuke 
inimesi juhtinud, sinna sisse ära uppuda. Ja uppuda niimoodi, et sa ei kirjuta 
ühel hetkel enam üldse koodi ja minetad selle oskuse?} 

Praegu ma näiteks ei kirjuta.

\question{Aga tahaksid?}

Nojah, vahel on ikka nagu kurb, et ei saa. Aga ma annan endale aru. See on 
olnud mingid viimased kolm aastat. Lihtsalt juhtus niimoodi, et keegi pidi 
selle ülesande ka võtma. Ma ei ütle üdlse, et ma selles nagu õnnetu oleksin, 
aga eks vahel ikka kriibib natukene.

\question{Ikkagi, miks sa tol ajal otsustasid, et sa tahad juhtimise rolli ka 
juurde võtta?} 

Võib olla oli mul lihtsalt iga asja kohta öelda? Ja kui sul on iga asja kohta 
öelda, siis sind karistatakse selle eest. 

\question{Mäletades seda, kuidas Hansapank oli seesmiselt üles ehitatud, ja 
seda keskset Oracle baasi, kus kõik maailma asjad sees olid, siis ühel hetkel 
see süsteem läks ikkagi ka tehnoloogiliselt laiaks. Et kuidas  seda kontrolli 
all hoiti?} 

Alguses oli kõik väga lihtne. Oli oraakli baas. Tol ajal, ma arvan,  oli see 
väga õige otsus, et me kirjutasime koodi baasi. Aga see kood oli hästi 
struktureeritud, sa ei tohtinud midai segamini ajada, ja ta oli tegelikult 
hästi ilusti ja loogiliselt struktureeritud. 

Siis aga tulid igasugused tehnoloogilise \emph{switchid}. Javat veel  ei olnud 
nii palju, osaliselt kirjutati C-s. Kui tuli esimene internetipank, eks me siis 
otsisime  seda, kuidas seda teha. Tehnoloogia ei olnud veel päris sealmaal. Ja 
siis me ostsime  sisse ka selle BroadVisioni platvormi\sidenote{Tegu oli 
varajase internetitehnoloogiaga, mis lubas igale kasutajale täielikult 
personaliseeritud kogemust. Paraku selgus, et kõigi nende kogemuste pakkumise 
vältimatuks eelduseks on nende välja mõtlemine. Küll aga võimaldas BroadVision 
üsna mõistlikult kombineerida HTMLi ja C-s\index{C} kirjutatud komponente 
(hiljem ka serveripoolset JavaScripti\index{JavaScript}). Ja sellest 
internetipanga, tellerirakenduse jms. ehitamiseks juba piisas.}. Need on 
needsamad otsused, et kui sa oled  kuskil liiga vara, sa teed  otsuseid, mis 
ilmselt ei ole  jätkusuutlikud, sest uus teletehnoloogia tuleb peale. 

\question{Aga kui ma mõtlen selle peale, kuidas Sergei 
Anikin\index[ppl]{Anikin, Sergei} kirjutas Light Tellerit\sidenote{Lähemalt loe 
Sergei loost lk. \pageref{sisu:teller}.}, siis see ongi see, kuidas praegu 
tehakse. Tehnoloogia võis ju jätkusuutmatu olla aga lahendus oli vägagi 
jätkusuutlik!} 

Tegelikult meil  juba too hetk andmebaas pakkus teenuseid, loogika ei olnud 
segamini kliendis ja baasis. 

\question{Kes selle otsuse tegi, et niimoodi teha?} 

Meie tegime. Asi algas  sellest, et mina ei Ruta Joost\index[ppl]{Joost, Ruta} 
olime nagu esimesed, kes panid need mustrid tegelikult paika. See tundus 
kuidagi nagu ainuke võimalik viis teha. 

\question{See on päris hea viis arhitektuuri teha, et  võtad selle, mis tundub 
ainukesena võimalik!} 

Teatud asju ei saa nagu põhjendada, kuidas need sinu peas nagu sünnivad, eks 
ju. Et ilmselt see on analüütiline mõtlemine, et sa vaatad mida tähendab see, 
kui ma teen nii või kui ma teen naa. Mul oli kohe nagu see, et kui ma kirjutan 
mingi loogika klienti, siis ma ei saa seda ju korduvalt kasutada. Järelikult ma 
ei tee seda. Kirjutad sinna nii vähe, kui võimalik. Ja oligi. 
Sergei\index[ppl]{Anikin, Sergei} sai kiiresti teha Light Telleri, sest tal 
olid kõik teenused  olemas. 

\question{Too hetk Hansapank läks ka horisontaalselt suureks, marketsid, 
kindlustused ja niisugused asjad, kuidas te seda loomaaeda kontrolli all 
hoidsite, et seal keegi mingisuguseid lollusi ei teeks?} 

Oligi hästi keeruline. Ega me tegelikult kõike ei kontrollinud lõpuni. 
Marketsil\index{Hansapank!Markets} oligi  väga suur eripära, neil oligi oma IT, 
seal oli hästi palju asju Excelis, kogu see analüütika, mis nad tegid sealt 
pealt omade vahenditega. Ja need sisse ostetud lahendused: keegi ei hakanud 
tegema \emph{trading} lahendust ise, Condor osteti sisse. Ja eks seal samamoodi 
oli ka valesid otsuseid nagu see marketsi platvorm, mis üks hetk osteti ja mis 
osutus liiga tooreks ja liiga keeruliseks. Marketsit me tõesti keskselt palju 
ei hallanud. Nii palju, kui ta haakus meie poolt pakutavate  pakutavate 
süsteemidega. 

\question{Keegi pidi ju tegema selle otsuse, et \enquote{las nad toimetavad 
omaette}?}. 

Just see hetk, kui organisatsioon läks  laiemaks, tekkisid meil IT-s sellised 
valdkonnaga tegelevad inimesed, kes meie poole pealt olid sellel konkreetsel 
valdkonnal vastas. Ja paljuski oli tegu valdkonnajuhi ja Marketsi oma IT 
vahelise kompromissi, kokkuleppe ja ühise ostusega. 

\question{Põhimõtteliselt tõmmati neile organisatoorne \enquote{kast} ümber ja 
lasti neil seal sees vabalt toimetada?}

Jah, aga väljaspoole kasti nad ei saanud, välismaailmaga suhtlemiseks olid väga 
selged liidesed. Hansa Capital\index{Hansapank!Hansa Capital} oli teine näide. 
Kasvas  väga kiiresti, oli väga isepäine, väga efektiivne, äriliselt tegi super 
head tulemust. Kuni üks hetk tuldi meie juurde ja öeldi, et \enquote{Kuulge 
meil läks Excel katki, tehke midagi!}. Ja siis me vaatasime sellele loomaaiale 
peale, ta oli peaaegu sama suur, kui pank ju oma äridega! Panimegi projekti 
tiimi kokku ja hakkasime tegema. 

\question{See kõik ju räägib ka sellest, et üks asi on vedada inimesi, kes on 
tõepoolest \emph{hand picked}. Aga teine asi on olla, nagu sa juba olid, 
organisatsioonis kõrge taseme juht. Mis tähendas seda, et sa pidid aru saama, 
kuidas inimene töötab, nendega suhtlema, panema neid kuidagi moodi tegema asju, 
mida on vaja teha. Kuidas sulle see oskus tuli?}

No õpid! 

Loomulikult oli ka hästi palju koolitusi, nad tulevad kindlasti kasuks, panevad 
teatud asjade peale mõtlema. Aga eks sa ikka tehes õpid, tegelikult. 

See juhtimise teema on hästi keeruline, sinu isikuomadused määravad hästi 
palju, kuidas sa toime tuled. Aga kogemus on kindlasti teine asi. Ma arvan, et 
ma olen praeguseks palju parem, kui ma olin siis. Aga poleks ma siis seda 
protsessi läbi teinud,  ma ei oleks seal, kus ma olen. Kõige keerulisem oli 
see, kuidas saab hakkama inimestega, kes kõik sinult midagi tahavad. See  on 
kõige keerulisem! Ja sa tead, et sa ei saa kõigile \enquote{jah}, öelda. Kuidas 
sa teed, seda niimoodi, et saavad tehtud õiged valikud ja mitte ei saa 
tähelepanu see, kes kõige kõvemini karjub. 

\question{Aga kuidas saavad tehtud õiged valikud?}

Tekkis ka see, et vahel keegi ei olnudki sinuga rahul. Sa pidid sellega nagu 
hakkama saama, et kõik sind ei armastanud, eks ju. Aga no kindlasti räägid ja 
räägid inimestega, ega seal muu ei aita! No paned nad kokku\ldots

\question{Kuskil väga sügaval peab olema mingisugune huvi inimese vastu lisaks 
huvile arvuti vastu?} 

Ja, no ega muidu ei saa! Ma ütlen, et see huvi inimese vastu peab olema 
võib-olla isegi natuke suurem, kui huvi arvuti vastu! 

\question{Ja ometi sina läksid õppima rakendusmatemaatikat ja mitte 
psühholoogiat, eks ole? Kus sul see huvi inimese vastu siis tuli?}

Ma ei tea, mul see huvi inimese vastu on vist kogu aeg olnud, kui ma niimoodi 
nagu mõtlen. Seesama asi, mis ma rääkisin, et valisin tegelikult kirjanduse ja 
matemaatika vahel. See huvi oli, jah, kogu aeg olemas. 

\question{Kui palju te suhtlesite tol ajal selle panga-välised kogukonnaga? 
Kuskil seal pulbitses mingisugune BBS-ide kamp ja toimetas mingisuguseid oma 
toimetamisi, toimis mingi võrgustik. Kui palju te selles osalesite?}

Ma arvan, et meil oli inimesi, kes  seal suhtles, aga need olid kindlasti 
sellised inimesed nagu Toomas Lasmann\index[ppl]{Lassmann, Toomas}, kes oli 
nagu hästi tehnilised. Meie nagu mitte niivõrd. Pigem suhtelsime natuke 
kõrgemal tasemel: kes kuhu liigub, kes milliseid tehnoloogilisi valikuid teeb.

\question{Üks asi, mida ma tollest ajast igatsen, on see, kuidas sündis 
iPizza\sidenote{Hiljem tuntud kui Pangalink. Minu mälestuses sai see kohapeal 
välja mõeldud, teised ütlevad, et tegu oli Soome lahenduse üle võtmisega. 
Igatahes võimaldas see (lisaks algselt ideeks olnud maksete lahendusele 
eesmärgiga internetis pitsat tellida, sealt ka nimi) anda panka sisse loginud 
inimese identiteeti edasi teistele osapooltele. ID-kaart ei olnud veel levinud, 
keegi Maksuameti paroolikaarti endale ei võtnud, aga Maksuametil oli vaja saada 
inimesed internetis tulu deklareerima. Nii sündiski koostöö, sest pangal oli 
vaja anda inimestele hea põhjus nende internetipanka kasutada.}. Kogu protsess 
sellest hetkest, kui astuti uksest sisse, et nüüd hakkame tegema kuni selle 
hetkeni, kui maksuametisse sai sisse logida, võttis aega suurusjärgus kolm 
nädalat. Praegu isegi lepingu läbirääkimised võtaksid tõenäoliselt kauem.} 

Vaata, me oleme jõudnud sinnasamasse, kus on kogu vana maailm. Tegelikult on 
see nagu kurb, aga ma ütleks, et  kõik algab  inimestest. Miks on see lepingu 
läbirääkimine nii pikk? Me oleme ise teinud endale kõik need protseduurid, 
poliitikad. Me ise oleme teinud, mitte keegi teine. \emph{Sorry}! Osaleme 
sellises organisatsioonis, nagu Nordic Finance Innovation Forum. Selle eesmärk 
on täna panna põhjamaades pankades, kus miski ei liigu (täpselt sama asi, neli 
kuud räägitakse lepingut läbi), mingilgi viisil liikuma innovatsioon. Panna nad 
omavahel koostööd tegema. Osaleme, sest see on huvitav. Nad korraldavad 
selliseid kuskil päevaseid seminare mingis Põhjamaade pealinnas, kus siis 
erinevad inimesed räägivad erinevatest asjadest, mis on tehtud. Rääkis väike 
Soome firma, mingi 15 inimest, kes tegi AliPay integratsiooni Soome 
maksesüsteemiga. Hiinlasena sa saad praegu minna Helsingis igasse poodi ja 
maksta oma AliPay-ga. Kujutad sa ette! See projekt oli niimoodi, et inimesed 
tulid ja ütlesid \enquote{Nelja kuu pärast, novembri lõpus, lendab siia 
Rovaniemisse ma ei tea mitu suurt lennukitäit, palju, palju, palju, palju 
hiinlasi. Ja neil kõigil on vaja sellega maksta.} Neli kuud ideest 
\emph{live}'ni. Nad tegid selle ära. See kutt rääkis, et \enquote{Üldiselt on 
meil Soomes nüüd niimoodi, et see ma saaksid mingi Soome panga juurde mingi 
ideega rääkima, et saada  kohtumist, selleks kulub neli kuud}. Et ma saaksin 
esimese kohtumise. Aga see projekt tehti nelja kuuga ära!

\question{Ometigi tol ajal Hansa ei olnud enam tilluke organisatsioon?}

Aga ta toimis ikkagi! 

\question{Aga kuidas sai niimoodi, et läbirääkimised ei võtnud neli kuud?}

Selle asja nimi on kultuur! Kultuuri loovad inimesed. Kultuur ei sünni mitte 
millestki muust. Ja kui mul on nüüd organisatsioon, ma ei taha siin olla 
kriitiline, kelle juhil  on ainult  üks ambitsioon, omaenda isiklik ambitsioon 
olla suur juht. Ambitsioon rääkida ümber kõike, mida ta on lugenud raamatutest 
hoolimata sellest, mis tegelikult tehtud saab, siis sünnibki tema ümber 
samasugune kultuur. Ja Eesti häda täna on väga paljuski see. Ja mitte ainult 
riigis, ka erasektoris. 

\question{Ja tol ajal Hansas oli juhtkond, kust tuli teistsugune kultuur!}

Ja kui sa vaatad, mis seal toimus, siis ju kõik need inimesed, kes selle 
kultuuri olid ehitanud, lahkusid. Mingi põhjus pidi olema, ei makstud ju 
kehvasti palka.

See on väga kurb, aga seal ei olegi\ldots 

Ma siin räägin ühele pangale. Olen rääkinud umbes aasta. Et 
\enquote{Kallikesed, kui te tahate, et asjad hakkaksid kiikuma, te peate seda 
vana kultuuri kandvad inimesed lihtsalt ära saatma. Te ei saa enne üle ega 
ümber, kuid te ei julge teha seda otsust.} Inimesed ei julge tihtipeale 
otsustada, need on rasked otsused. 

\question{Lõpuks ma alati küsin, et kõik see teekond on sind kuhugi toonud. 
Millega sa praegu tegeled?} 

Noh, praegu me oleme viimased 16 aastat ehitanud Icefire't. Seitsmeteistkümnes 
aasta käib. Me kogu aeg muutume, areneme. Meil on samamoodi, eks ole, et kuidas 
maailm muutub, kes me võiksime olla, kuidas sinna jõuda. Aga  sinna jõuda, ja 
siit tuleb jällegi seesama kultuuri küsimus, niimoodi, et  me hoiame oma 
kultuuri. Seda, mida meie inimesed kannavad. Ja see on võib-olla ka see, miks 
me ei ole näiteks läinud seda teed, et me ostame kokku mingeid  ettevõtteid. Et 
lihtsalt kasvataks väärtust, aja suureks, lõpuks saada väga rikkaks. Oleme 
pigem hoidnud organisatsiooni sellise hoovatavana. Ja täna me lähme platvormi 
ärisse, mis tegelikult muudab täiesti paradigmat, kus me tegutseme. 

\question{Ja sina oled selle asja juht? Ja koodi ei kirjuta?} 

Praegu juba kolm aastat ei kirjuta. Mul on fantastiline tiim. See tulemus, mis 
me viimase kolme aastaga oleme teinud pärast seda, kui me vahepeal augukeses 
käisime! See, kuidas me oleme suutnud ennast kasvatada, samas säilitades 
efektiivsust, näitab, et mee midagi teeme hästi õigesti. Mis on natuke 
kummaline on see, mida ma olen märganud mingil viimasel aastal, see külvab nagu 
kadedust ümberringi. Et mingid inimesed, kellega sa  kunagi oled teinud koos 
asju, tekib nagu mingi selline\ldots Miks me tunneme kadedust? Mina tunnen küll 
rõõmu, kui teisel hästi läheb! Aga eks me sellega peame  elama. 