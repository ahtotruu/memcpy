\index[ppl]{Vene, Vilve}

\question{Kuidas sina said arvutite juurde ja arvutid 
sinu juurde?} 

See algas juba koolis. Käisin 1. keskkoolis\index{Tallinna 1. 
Keskkool}, mida täna tuntakse kui GAGi\index{Gustav Adolfi Gümnaasium}, 
matemaatika-füüsikaklassis. Meile õpetati ka programmeerimist. Mul oli väga 
hea matemaatikaõpetaja ja füüsikaõpetaja, mulle õudselt meeldis, aga ma 
ilmselt ei teadnud, kas see meeldib mulle piisavalt palju või meeldib mulle midagi 
muud rohkem. Mulle meeldis ka kirjutada ja tegelikult arvasin 
keskkooli lõpuni, et lähen pigem eesti keelt ja kirjandust 
õppima. Aga kuidagipidi arenes mõte sinnamaani, et õpiks 
füüsikat. 

Läksin Tartusse, vaatasin, et seal oli üldse vist neli tüdrukut, 
mõtlesin, et jube veider, ja andsin avalduse rakendusmatemaatikasse\index{Tartu 
Ülikool!Matemaatikateaduskond!Rakendusmatemaatika}. 
Kui avaldust sisse viisin, vaatas tädi laua taga mulle otsa ja 
ütles: \enquote{Teie oma tunnistuse ja kuldmedaliga saaksite ju 
arstiteaduskonda sisse!} Selgitasin talle, et ma ei taha, ja nii see 
läks. 

\question{Mille peal teid GAGis programmeerima 
õpetati?}

Jessukestel\index{Jessuke} ja perfokaartidel. 

\question{Kas koolis oli kõik see olemas?}

Ei, me käisime Teaduste Akadeemia Arvutuskeskuses\index{Teaduste 
Akadeemia!Arvutuskeskus}.

\question{Kooli poolt väga edasipüüdlik ettevõtmine! Kes seda asja koolis 
ajas?}

Ma arvan, et tollal oli direktoriks Helmi Viikholm.\index[ppl]{Viikholm, 
Helmi}\sidenote{Helmi Viikholm oli 1. keskkooli direktor aastatel 1962--1982.} 
Selline inimene, kes suutis ka nõukogude aja \emph{setup}'is 
organiseerida. Meil olid jube head õpetajad, avatud mõtlemine ja 
nii see tuligi.

\question{Kas Teaduste Akadeemia lasi õpilased oma arvutitele ligi?}

Jah! Aga eks kood läks perfokaardi peale -- ise toksisime 
perfokaardid, need läksid masinasse ja tulemus tuli välja. 

\question{Mis programme te kirjutasite?} 

Tegime lihtsaid asju. Ma sisu ja detaile ei mäleta, aga igatahes oli 
see Fortranis\index{Fortran}. 

\question{Seni on umbes igas teises loos läbi 
jooksnud ruutvõrrandi lahendamine \ldots} 

Vaat seda ma ei mäleta, minu meelest olid meil ikkagi natuke 
ärilisemad teemad.

\question{Nii et pärast keskkooli läksid Tartusse Vanemuise tänavale\index{Tartu 
Ülikool!Vanemuise tänava õppehoone} rakendusmatemaatikat õppima?}

Jah. Meid alustas 20. Minul, kellel 
koolis oli kõik jube lihtne, oli algus hästi raske. Ja alustati 
assemblerist\index{Assembler}. Pärast olen mõelnud, et see oli väga hea -- 
assembleris pead aru saama, mis seal 
sees toimub, ei saa lihtsalt kirjutada mingeid koodiridu, mõistmata, mis sügaval sees toimub. 

\question{Selle raskuse ületamise taga pidi olema mingisugune kihk. Miks sa ei 
läinud arstiks õppima?} 

Mind ei huvitanud!

\question{Ja arvutid huvitasid?} 

Mõtlen siiamaani, et kui edasi tulid intelligentsed 
programmeerimiskeeled, kus ei pea aru saama, mis seal sees toimub, siis 
tänapäeval on programmeerimine minu meelest 
käsitöö. Selles mõttes, et nii kampsunit kududes kui ka 
programme tehes tuleb osata võtteid ja kasutada tööriistu. Tollal
ei olnud sedasi. Kihvt oli alustuseks läbi mängida, mis juhtub: 
\enquote{Okei, see rida teeb nüüd seda, paneb selle sinna ja selle sinna. Kui 
mul on nüüd vaja võtta see sealt ja teha sellega midagi, siis mida ma selleks tegema 
pean?} Said sügavuti aru ja see oli hästi vahva. 

Viis, kuidas nad õpetasid, oli karm, aga tegelikult see 
selekteeris välja inimesed, kes tõesti tahtsid seda teha ja olid selleks ka
võimelised. Alguses tuli loogikat kõvasti, samuti mat-analüüsi. Ikka 
\emph{hard core}, mitte nagu tänapäeval õpetatakse. Rääkisin just
sõbrannaga, kes õppis teoreetilist matti ja kelle kursuselt on ülikoolis 
hästi palju õppejõude. Nad ütlevad, et ei saa enam sellel tasemel õpetada, vaid peavad oma 
programme lihtsustama. 

\question{Mulle on räägitud, et mõni oli keskkooli ajal juba kuskil tööl, sest 
kangesti oli vaja arvuti juurde pääseda. Kas sul ei olnud niisugust kihu?}

Ei, seda ei olnud. Ma ei ole kunagi arvutihull olnud, see on 
võib-olla rohkem poiste teema. 

\question{Õige, mu valim on seni olnud väga kallutatud.} 

Aga Tartus oli küll nii, et asi hakkas tõsiselt meeldima ja sain aru, et see 
on õige, mida ma teen. 

\question{Millest sa aru said?}

Mulle lihtsalt meeldis! Ja see läks minu jaoks ajaga 
lihtsaks, kuigi ma ei saanud aru, kuidas ma saan 
aru. Kõlab väga filosoofiliselt ... Kuna see oli ikkagi rakendusmatemaatika, siis 
lisaks programmeerimisele oli palju teoreetilist matti. Vahel läheb matemaatiline analüüs nii abstraktseks, et sa 
tõesti enam ei saa aru, kuidas sa saad aru. Miskipärast tuli mul see jube hästi välja, kusjuures meid alustas 20 
ja pärast esimest aastat oli järel 14. 

\question{Väike grupp, aga väljalangevus ei olnudki nii suur, meie kursuselt 
läks rohkem. Aga mida sa mõtlesid
tegema hakata, kui kool läbi saab?}

Aeg oli ju selline, et valikuid loomulikult oli, aga enamik neist kuskil arvutuskeskuses. Tollal sai
kohti valida vastavalt sellele, kuidas lõpetasid ja milline oli 
pingerida. Hästi popp koht oli näiteks Raadiomaja 
arvutuskeskus\index{Raadiomaja Arvutuskeskus}, see oli number üks. 
Statistikaametisse ei tahtnud keegi minna.

\question{Miks?}

Ei tea, tagantjärele mõeldes on ju
statistikaametis töö palju huvitavam kui raadiomajas. Aga nii see 
kahjuks oli.

Minul oli selline õnnelik juhus, et sattusin Küberneetika 
Instituudi\index{Küberneetika Instituut} matemaatikaosakonda praktikale. Viiendal kursusel tegin 
diplomitööd, juhendaja oli Otto Vaarman\index[ppl]{Vaarman, 
Otto}, kes oli väga tunnustatud matemaatik. Uurisime Küberis Newtoni tüüpi meetodeid, mis kõige paremini 
lahendavad erinevaid võrrandeid. Otto oli vana meesteadlane, kes ise 
väga palju ei viitsinud teha, aga tal oli paar 
aastat varem lõpetanud noor jünger Maarika Lomp\index[ppl]{Lomp, Maarika}, kes oli 
mul sisuline juhendaja. Kollektiiv oli hästi tore ja töö 
huvitav. Toona tulid juba esimesed variandid, kus enam ei 
pidanud perfokaartide pealt toksima, vaid olid ikkagi 
Jessukese\index{Jessuke} taga ning said juba ise intelligentsel viisil oma 
koodi sisse viia.

Ja siis sattus mingi hetk täiesti ootamatult Otule\index[ppl]{Vaarman, Otto} ja 
Maarikale\index[ppl]{Lomp, Maarika} külla üks ääretult sümpaatne härrasmees 
sellisest huvitavast organisatsioonist nagu Algoritm. Tegelikult 
oli sellel asutusel hästi pikk nimi.\sidenote[][]{\label{sisu:algoritm} 1976. aastal asutatud Tallinna 
Teadus-Tootmiskeskus (TTTK), mis kuulus Üleliidulise Teadus-Tootmiserikoondise 
\enquote{Algoritm} koosseisu, Eestis tuntud lühendnime all Algoritm. Kuna 
asutus allus NSVLi tasemel kaitsetööstuse ministeeriumide gruppi, oli sellele
omistatud koodnimi Postkast A-3433 ja kehtestatud ka vastav tööre\v{z}iim. 
Üleliiduline alluvus seletab ka töökeelt. Asutus tegeles ES EVMi\index{ES EVM} 
hoolduse ja remondiga, IT-koolituse ja selle metoodikaga ning mitmesuguse 
tarkvara (nt matemaatika, arvutidiagnostika ja automatiseeritud 
projekteerimine) arendamise ja tira\v{z}eerimisega. Peale Tallinna olid 
asutusel filiaalid Tartus ja Kohtla-Järvel(!). 1980ndate keskel töötas 
Algoritmis ligi 1000 inimest, lisaks kaasati allhankijatena inimesi 
TPIst\index{Tallinna Tehnikaülikool} ja Tartust\index{Tartu Ülikool}. Algoritm 
lõpetas töö 1992. aastal.}. Küber oli Mustamäel. Kui sealt sõita edasi, kus on 
täna ARK, siis ühe risti peal olid koledad baraki tüüpi 
majad\sidenote{Toonase aadressiga Kadaka puiestee 165.}, kus asuski 
teadusuurimiskeskus Algoritm, millel oli eraldi matemaatikaosakond. 

Algoritm oli täielikult
venelaste sõjaline organisatsioon, kus tehti uurimistööd 
väga kummalistel aladel. Matemaatikaosakond oli ainuke eestlaste osakond ja Ants\sidenote{Ants Roose\index[ppl]{Roose, Ants}, kes hiljem töötas ka Algoritmi teadusala asedirektorina. 
Tegelikult oli osakonna nimi \enquote{matemaatika tarkvara 
osakond}. Rooselt on 
pärit kogu Algoritmi puudutav info.} selle
juhataja. Ühesõnaga tema rääkis mu ära ja läksin sinna tööle, ääretult tore kollektiiv oli. 

\question{Järelikult oli sul ikka akadeemiline siht silme ees?}

Jah, pidin tegelikult minema doktorantuuri Gennadi 
Vainiko\index[ppl]{Vainikko, Gennadi} juurde, aga kuidagipidi hakkasid asjad 
arenema ja huvitavaid teemasid tuli palju. 

Muide, Algoritmis hangiti tarkvara niimoodi, et 
keegi tundis näiteks Minskis kedagi, kes oli kuidagipidi saanud mingi softipaketi, ja meil oli seda vaja. Ülemus saatis 
minu, noore tüdruku, kes vene keelt väga hästi ei rääkinud, 
Minskisse ja ütles, et pean sellega tagasi tulema! Ja tulingi. 

\question{\emph{Millega} täpsemalt sa tagasi tulid?}

Suure lindikettaga! Nii me seda tarkvara hankisime ja testisime. Tegime tööd ja 
oli väga huvitav aeg.

Kui mul sai seal majas kolm aastat täis, hakkasid ajad muutuma. Ain 
Rasva\index[ppl]{Rasva, Ain} oli ka tollal seal ja tema soovitas mind 
ühele inimesele, kes juba toona tegi koostööd soomlastega. Läksingi aastal 1988 Tööstusprojekti\index{Tööstusprojekt}, kus olid juba tol ajal miniarvutid ja koostöö soomlastega. 

\question{Need on ju projekteerijad, mis seal programmeerida oli?}

Oi, palju! Ehitusprojekt koosneb paljudest asjadest, sealhulgas 
tugevusarvutustest.

\question{Sul oli võimalik minna akadeemilisse maailma või sealt ära. Miks sa just sellise valiku tegid?} 

Pakkumine oli tohutult ahvatlev -- võimalus teha tööd ka kuskil mujal, 
Soome projekte koos 
soomlastega. Ja ma sain õppida. Mulle anti C õpikud ja 
esimene proovitöö oli automaatsed konverterid, programmid, mis tõlkisid
Fortrani\index{Fortran} koodi C\index{C} koodiks. See kood, mis 
välja tuli, oli muidugi kohutav, aga töötas. 

1990. aastal olin natuke aega lapsega kodus. Tegin Soome väikseid töid, teenisin 
saja marga kaupa väga head raha. 

\question{Jõhker raha, sada marka!}

Jah! 1991. aasta talvel avaldati ajalehes 
kuulutus, et Rootsi-Eesti ühisfirma otsib programmeerijaid ja just naisterahvaid. Ma polnud elus niisugust asja näinud, aga tundus väga 
huvitav. Avaldusi oli üle kuuesaja. 

\question{Kuuesaja!? See tähendab, et 1991. aastal oli Eesti Vabariigis aktiivsel 
tööturul 600 naist, kes võisid enda kohta öelda \enquote{programmeerija}?}

Jah. Esimesel intervjuul selgus ka väga lihtne loogika, miks nad naisi 
otsisid. Firma taga oli üks rootslane, kes töötas Stockholmi linna ja 
lääni valitsuses (minu arust oli ta lausa IT osakonna juhataja), ja üks Eestist Rootsi läinud mees, keda mäletasin ülikooli ajast, 
Kalle Kullman\index[ppl]{Kullman, Kalle}. Neil tekkis idee, et kui Eesti saab 
vabaks, saab sealt odavalt head tööjõudu. Nad tegid firma, mis pidi hakkama 
Stockholmi lääni valitsusele teenust osutama, ja naisi otsiti sellepärast, et 
naised on korralikumad, leplikumad ja küsivad vähem raha. Meid see 
ei häirinud, sest, kujuta ette, saada tööle firmasse, mis maksab palka 
valuutas, ja teha Rootsi tööd! Fantastiline!

Välja valiti kolm naist: üks oli Maarika Lomp\index[ppl]{Lomp, 
Maarika}, teine mina ja keegi kolmas oli veel. Meile 
üüriti kontoriruumid vana ajakirjandusmaja taga. Pidime ise panema 
püsti kohtvõrgud ja kõik muu! 

\question{Kas tollal oli selline asi nagu kohtvõrk?}

Jah! Ise panime püsti. Suhtlesime üle modemite Rootsiga ja kõik toimis. Muidugi kasutasime tuttavate meesterahvaste abi, kes olid juba võib-olla rohkem võrgu ja selle poole peal, aga saime hakkama. 

Programmeerisime sellises huvitavas keeles nagu Magic\index{Magic}. Oled 
kuulnud? 

\question{MUMPSist olen, aga Magicust mitte.}

See oli juba toona 4GL, Iisraeli päritolu.\sidenote{Platvormi tootnud 
Magic Software Enterprises oli tõesti asutatud 1983. aastal Iisraelis.} Üldse 
oli kasutada vist kaheksa erinevat käsku, neile panid 
parameetrid taha ja nendest koodi kokku. Koodi sai täita vastavalt 
vajadusele kas eest tahapoole või tagant ettepoole. Täiesti müstiline asi! Ja 
sellega sai teha kõike. Esimese projektina tegime nende varade ehk 
autopargi haldusprogrammi, kus olid kõik asjad alates muruniitjatest ja lõpetades 
autodega. 

\question{Kui sind kuulan, kerkib esile huvitav kontrast. See vahend, 
millega te tegite, kõlab palju keerulisem kui tänapäeval 
kasutatavad, aga ülesanne, mida lahendasite, palju 
lihtsam, kui tänapäeval tavaliselt lahendatakse. Kas teile ei tundunud see tegevus kahuriga 
kärbse tapmisena?}

Ma ei oska sulle öelda, miks nad valisid sellise. Igatahes ei kaldunud asi mitte mingil juhul FoxPro, 
Paradoksi või mõne muu sellise asja poole, mis toona juba olemas olid. 
Võib-olla see oli see natuke päritoluga seotud, kuna Kalle on juut ja tal 
olid Iisraeliga tihedad sidemed. Ja kuna ta ise oli hästi kõva matemaatik ja 
keeruliste ülesannete lahendaja, siis talle ilmselt see 
süsteem sümpatiseeris. 

See oli tore aeg. Kasutajaliidesed olid rootsikeelsed ja mäletan siiamaani mingit 
sõnavara. Saime palka 800 Rootsi krooni kuus, mis oli 
tollal siin suur raha. Kui meid viidi restorani kliendiga 
kohtuma, siis meie igaühe restoraniarve oli suurem kui kuupalk. 
Muidugi tegime pikki päevi ja töö oli päris karm. 

\question{Huvitav, et selline programmeerijaamet oli olemas. Näiteks 
Henn Ruukel\index[ppl]{Ruukel, Henn} rääkis, et tema mäletamist mööda enamasti 
inimesed ei tegelenud igapäevatööna programmeerimisega. Leiva tõi lauale 
ikka kaabli vedamine või arvutite kokkupanek.}

No vot, meie tegelesime! Tegime algusest lõpuni: mõtlesime välja, disainisime 
mudelid, programmeerisime ja ka juurutasime Rootsis kliendi juures. 

\question{Kui teil olid arvuti ja modem\sidenote{Ja pidi olema ka võimalus 
välismaa numbritele helistada!}, kas teil ei tekkinud huvi, mida nendega 
veel teha saab?}

Mäletan päevi, mil alustasin 
kell 8 ja lõpetasin kell 12 öösel. Huvisid võis hästi palju 
olla, aga lihtsalt ei jõudnud nendeni. Tegelikult tol ajal tärkas mul huvi 
andmebaaside vastu, sest meile anti üks raamat (\enquote{Lugege läbi, tüdrukud!}), mis minu jaoks esimest korda kirjeldas relatsiooniliste andmebaaside 
teooriat. 

Rootslaste juures olin umbes aasta, siis nägin 
Hansapanga\index{Hansapank} kuulutust. Nad ei otsinud üldse IT inimesi, ma isegi täpselt ei mäleta, keda. 
Lõpetasin parasjagu kaheteisttunnist tööpäeva ja ütlesin Maarikale, et okei, 
ma kirjutan. Saatsin CV, see oli 1992. aasta novembris. Järgmisel päeval 
helistas mulle Tõnis Sildmäe\index[ppl]{Sildmäe, Tõnis} ja kutsus intervjuule. 

Mäletan, et kui ma seal vestlesin (Liivi\index[ppl]{Kompus, 
Liivi}\sidenote{Liivi Kompus, üks Hansapanga legendaarseid IT inimesi.} oli 
ka), rääkisin väga uhkelt oma Rootsi kogemusest ja teadsin juba 
relatsioonilistest andmebaasidest ning jätsin ikka tohutult muljet. Jüri 
Mõis\index[ppl]{Mõis, Jüri} läks vahepeal mööda ja ütles: \enquote{Tõnis, kui sina 
ei taha, ma võtan ise selle tüdruku!} Ja nii mind tööle võeti. 

\question{Kui suur Hansapank\index{Hansapank} toona oli?}

Umbes 40 inimest. ITs olin mina kolmeteistkümnes. See oli tol ajal veel
Crebit\index{Crebit}, tegelikult Spin Development\index{Spin 
Development|see{Crebit}}, mis oli pangast eraldi. 

\question{Kui organisatsioonis on kokku 40 inimest ja neist 13 on IT inimesed, 
siis see on ju päris suur protsent!} 

Pank oli tõesti väga väike, sest toona valdasid inimesed 
hästi laia teemaderingi -- alates klienditeenindusest kuni raamatupidamiseni 
olid samad inimesed. Sealt kasvas välja näiteks Agve 
Aasmaa\index[ppl]{Aasmaa, Agve}, kes tuli tööle telleriks, samuti Tea 
Trahov\index[ppl]{Trahov, Tea}.\sidenote{Mõlemad legendaarsed 
hansapankurid.} Kokku oli meid aga jah vähe. Kui 
olid tähistamised, siis mahtusime väiksesse ruumi ära. 

\question{Kust see suhteliselt suur IT osakaal ikkagi tuli? Praegu ei 
ole ju veerand Swedbanka IT.} 

See oli imetlusväärne visioon, millega omal ajal Hansapanka\index{Hansapank} 
tehti! Taheti teha midagi tõeliselt \emph{cool}'i. Tegelikult need mehed tegid täielikku \emph{start-up}'i. 
Nad tahtsid teha täiesti teistsugust panka, kus paberkataloogide asemel oli 
arvuti. 

\question{Kust neil tekkis arusaam, et sedasi on üldse võimalik?} 

Ma ei ole kunagi küsinud, aga arvan, et see tuli koostöös ja nad olid sõpruskond. 
Hästi palju oli ilmselt Tõnis Sildmäe\index[ppl]{Sildmäe, Tõnis} panust, kes
müüs seda ideed, et nii saab teha. Näiteks kust tuli mõte panna kõik kontorid juba sel ajal \emph{online}'i? 
Mõned aastad hiljem, kui käisime Inglismaal ja Iirimaal kohtumas nende 
suurte ja vanade pankadega, siis need imestasid: \enquote{Issand, lapsed, mis 
te räägite! Teil on mingi Paradoxi lahendus ja see töötab \emph{online}'is?}

See oli julgete mõtete maailm, samas ei tehtud 
lollusi. Keegi ei teadnud, keegi ei osanud, aga kogu aeg õpiti. Mõeldi, 
kuidas me nüüd sellest üle saame? Kuidas me hakkame oskama? Keegi saadeti
kuskile midagi õppima \dots 

Minu esimene ülesanne oli see, et 
Tõnis\index[ppl]{Sildmäe, Tõnis} ütles: \enquote{Näed, mul on siin neli 
programmeerijat kirjutanud. Kes on teinud laenu, kes kontosid. Vaata 
ja ütle, mis võiks olla teistmoodi.} Vaatasin ja stiili 
järgi oli kohe näha, et see on Liivi Kompus, see on Kadri Trahov, see Tõnis 
Argus -- igaühel oli täiesti oma stiil. Andmebaas oli selline, nagu oli, aga kõik
töötas. Kirjutasin ettepanekud ja loomulikult ei olnud need 
selle tehnoloogia peal teostatavad, aga sealt sai alguse mõte, et viime oma 
süsteemi Oracle peale -- teeme uue pangasüsteemi ja viime asja järgmisele 
tasemele.

Kogu eduloo \emph{point} oli ääretult avatud 
suhtlus, kõik soovisid midagi ära teha. 

\question{Mis seda takistas nurjumast? Kui hakata niisama nullist 
kõrgtehnoloogilist panka tegema, siis see võib ju vussi minna.}

Tead, mul on alati olnud ja on siiamaani usk, et inimesed on \emph{key}. Inimesed, kellega sa 
mingit asja teed, isegi kui asi võib minna ka totaalselt 
tuksi. Üks pool on muidugi oskused, aga isegi rohkem 
määrab ära suhtumine ja ambitsioon. 

\question{Milline suhtumine olema peaks?}

See, et tahan midagi ära teha, aga samas tean, et ma ei tee seda üksi, 
vaid me teeme koos. Oluline on ka arusaam, miks ma midagi teen. Palju on selliseid initsiatiive, et on 
hästi huvitav tehnoloogia, aga kas see lahendab mõnd probleemi, selle peale liiga palju ei mõelda. Tollast aega iseloomustabki koos ärategemine. Mida 
kindlasti ei olnud, oli see, et \enquote{kes on kõvem}.

Inimestel, kes seal toona töötasid, oli hästi suur ambitsioon ---
kindlasti mitte isiklik karjäär, vaid meeskondlik ambitsioon ja 
saavutusvajadus. Tänapäeval räägitakse palju ja räägiti ka Swedbankis 
aastal 2000, kui rootslased tulid, protsessidest ja kvaliteedijuhtimisest. Peab 
olema tohutu hulk dokumente ja siis ma teen plaane, raporteerin ja 
kogu jõud lähebki selle peale. Hansapangas seda ei olnud, me ei mõelnud niimoodi. Näiteks kui pank tahtis 
välja tulla eraisiku pangakontoga, kutsus Jüri Mõis\index[ppl]{Mõis, Jüri} 
ühte ruumi kokku kõik, kellel võiks sellega mingit pistmist või 
arvamust olla. Tema rääkis, miks seda teha, ja meie mõtlesime, mis selleks tegema 
peab. Edasi hakkaski igaüks oma osa tegema. Tehti koos ja 
hästi ruttu. 

\question{Magicu moodi asjadega tegelemine annab 
ilmselt päris hea immuunsuse tehnoloogia järel jooksmise vastu. Kui oled korra 
pidanud tagurpidi käivat programmi kirjutama, siis ei ole miski enam väga 
uudne!}

Oluline on kasutada õiget asja õiges kohas. Populistlik jooksmine 
mingi asja järel\ldots{ }See on põhjus, miks ma ei poolda näiteks kunagist riiklikku initsiatiivi, 
et kõik programmid tuleb iga 13 aasta järel ümber 
kirjutada. \emph{Sorry}, aga võib-olla ei peaks neid 13 aasta tagant ümber 
kirjutama, kui kogu aeg teeks nendega midagi? Aitaks järele ja muudaks? See 
oli see, mida me pangas tegime. Vaatasime täpselt seda, kust meil tulevikus 
pigistama hakkab, ja muutsime ning vahetasime välja. See oli pidev protsess. 

\question{Ometi pidid ka pangas uute tehnoloogiate lained tulema. Kuidas te otsustasite, mida üles korjata?}

Me tegime ka valeotsuseid, keegi ei ole ju selle suhtes immuunne. 
Kui tehnoloogia poolelt rääkida, siis esimene laine oli see, et meil oli 
Paradox\index{Paradox}, mis oli failipõhine süsteem ja millest me kindlasti 
nägime, et see hakkab meil takistuseks saama. Kas või näiteks see, et me ei 
suutnud olla 24h kättesaadavad, või päeva vahetuse teema.\sidenote[][]{Päeva vahetus on 
pangas oluline ja keeruline ning seetõttu arvutuslikus mõttes kaua aega 
võttev toiming, mille käigus tehakse konkreetse päeva seisuga raamatupidamiskanded, 
toimetatakse arveldused, esitatakse aruanded ja, lühidalt öeldes, üks pangapäev 
asendub teisega.} 

Minul oli relatsiooniliste baaside teoreetiline 
teadmine ja Tarmo Pajumets\index[ppl]{Pajumets, Tarmo} oli töötanud pool 
aastat Soomes ning arendanud Oracle-põhiseid süsteeme. Oracle\index{Oracle} oli 
sel ajal Eestis ikkagi number üks, seda kasutasid näiteks ka Elion ja EMT. Üks põhjus, miks 
Oracle nii tugevalt Eesti turule tuli, oli see, et meie tugi oli Soomes 
ja Soome Oracle oli väga tugev organisatsioon. Kui alustasin uue süsteemi 
arendamist, siis mul oli otseliin Soome ja tugi telefoni otsas -- võisin helistada Soome ja küsida, miks see või teine asi ei tööta. Võrdle sellega, mis on praegu!

Ja nii me konvertisimegi oma Paradoxi 
Oracle peale. Oli olemas automaatse konverteerimise võimalus, 
et kasutajaliides jäi endiselt Paradoxi ja baas taga oli Oracle. See andis 
meile vabaduse teha oma päevavahetuse protsessid kõik 
niimoodi, et see oli rohkem 24h. Saime hakata sinna külge kaarte panema ja 
tulevikus ka näiteks Telehansa\index{Telehansa}. Ja hakkasime Paradoxi rakendust ennast ümber 
kirjutama. 

\question{Kas Telehansa tuli enne kui Forexi modemipank või hiljem?}

Minu meelest umbes samal ajal.

\question{Mast, kes Forexi asja kirjutas, rääkis, et tolle 
käivitamis{\-}üritusel istunud Hansapanga tütarlapsed esireas ja teinud ohtralt 
märkmeid!}

Võis nii olla, see tuli enam-vähem sinna otsa, palju 
vahet ei olnud. 

\question{Miks te Telehansa\index{Telehansa} tegite?}

See algas sellest, et olid küll kontorid, aga firmade raamatupidajad ei 
tahtnud oma maksetega kontorisse tulla. Meil 
oli palju firmasid. Kui vaadata Hansapanga arengut, siis ta 
kõigepealt võttis ju sellised eesrindlikumad firmad ja seejärel tulid eraisikud 
pangakontodega järele. 

Ühelt poolt tahtsime elu mugavamaks teha ja see oli ka
rahaliselt kasulik. Kontori koormust vähendades hoiad kokku. Teine oluline asi oli
innovatsioon -- oleme teistmoodi pank, 
teeme asju teistmoodi. 

\question{Kui palju Telehansa tuumsüsteemi muutust eeldas? See ju vajab hoopis 
teistsugust interaktsiooni tuumaga.} 

Tegelikult ei olnud Telehansa tegemine kuigi keeruline, selle võimaluse lõi
andmebaasi vahetus. 

\question{Kas ajaliselt tekkis see umbes samal ajal?}

Jah. Andmebaasi vahetus oli vist 1994. aastal ja Telehansa tuli ka siis. See
oli väga kõva asi.

\question{Telehansa on siiamaani väga kõva asi!}

Kolm meest -- Toomas Lassmann\index[ppl]{Lassmann, 
Toomas}, Madis Tapupere\index[ppl]{Tapupere, Madis} ja Riho-Rene 
Ellermaa\index[ppl]{Ellermaa, Riho-Rene} -- selle tegid ja kirjutasid. 

\question{Kas tolleks hetkeks oli IT inimesi juba rohkem kui neliteist?} 

Jah. Toomas Rand ja mina kirjutasime töötlust, et kui 
maksed sisse tulid, siis mis nendega sai. Tiim kasvas väga kiiresti. Neliteist oli 
1992. aasta alguses ja see arv kahekordistus umbes 
aasta või pooleteisega. 

\question{Kuidas te kasvu kontrolli all hoidsite? Kui nii kähku kasvad, 
siis on ka suur tõenäosus mõni loll kogemata palgale võtta.}

Mäletan selgelt, et kui olime kasvanud kuskil 50 inimeseni, siis tegime 
oma esimesed kompromissid. Enne ikka väga valisime inimesi. Hansal kui Eesti majanduse lipulaeval oli võimalik valida! 
Esiteks, et nad oleksid professionaalsed tipptegijad. Teiseks, et nad inimestena sobiksid väga 
hästi tiimi. Aga siis tegime jah esimesed kompromissid\ldots

Hakkasid tekkima esimesed probleemid ja küsimused, kuidas asja hallata. Tekkis \emph{learning by doing} kogemus. Tänapäeval on ilmselt väga vähestel olnud 
võimalus kasvada koos organisatsiooniga väikesest suureks ja sealjuures väga kiiresti. 

Näiteks üks teema oli see, et kui keegi oli arendanud 
laenu- või väärtpaberisüsteemi, siis uue 
funktsionaalsuse tegemiseks tal enam aega ei jätkunud, kuna tal tuli teiselt poolt kogu aeg nii palju 
probleeme peale, mida pidi lahendama. Kes vajas 
raportit, kellel oli mõni \emph{case}, mis ei mahtunud sisse, 
ja seesama inimene tegeles mõlemaga. Siis tegimegi halduse poolel rakenduste halduse osakonna, kus olid 
inimesed, kes oskasid lihtsamaid probleeme lahendada, raporteid genereerida 
ja tundsid andmebaasi andmeid -- ühesõnaga olid arendaja 
kõrval, et arendajal oleks rohkem aega. 

\question{Nii et see otsus sündis praktilisest vajadusest, mitte te ei olnud kuulnud, et peaks olema 
rakendusadministraatorid? Teil oli vaja konkreetset asja teha, leidsite
inimesed, koolitasite neid ja panitegi tegema.}

Seal organisatsioonis sündis kõik praktilisest vajadusest. 

Mul on tohutu austus kadunud Tõnis Sildmäe\index[ppl]{Sildmäe, Tõnis} kui juhi 
vastu. Kuidas ta seda tiimi juhtis! Juhtkonna moodustasid inimesed, kes suutsid 
vedada teatud teemasid või olid juhipotentsiaaliga spetsialistid. Ta 
usaldas meid täielikult. Ta ei tulnud kunagi ütlema, mida keegi peab tegema. 
Ainuke negatiivne asi oli võib-olla see, et ta kaitses liiga palju: kui 
keegi kallale tuli, siis läks ta kohe võitlusse ja teda pidi tagasi 
hoidma. Aga see lõigi kultuuri, et probleemi tekkides istusime koos
ja arutasime, kuidas seda oleks mõistlik lahendada. 

\question{Sinu jutust koorub põhiliselt välja 
inimeste ja juhtimise olulisus. Väga vähe on juttu sellest, et Oracle indekseid peaks 
tegema nii- või naamoodi.} 

Indeksite tegemine on lihtsalt tehnika. Loomulikult peavad olema 
oskused ja teadmised, aga see on õpitav. See, kui 
hästi ma indekseid teen või kui ilusat koodi kirjutan (kood peab ilus 
olema!), ei ole see, mis toob edu. 

\question{Milline on ilus kood?}

See kõlab nüüd hästi populistlikult, aga ilus on kood, millest saab aru. Kui ma isegi ei valda täielikult programmeerimiskeelt või 
tehnoloogiat, milles see on kirjutatud, siis vaatan koodile peale ja saan aru. 
Loomulikult, kui ei ole üldse kogemust sellel alal, siis 
võib-olla ei saa aru. Aga kui oskan Cd kirjutada, siis vaatan Java koodile 
peale ja suudan seda lugeda. 

\question{Järelikult sõltub ilus kood sellest, keda sa 
arvad seda hiljem lugevat -- esimese kursuse tudengit või 20 aastat 
progenud inimest?} 

Ka esimese kursuse tudeng võiks aru saada!

Mul on just andmebaaside poolelt -- PL/SQL või PostgreSQL \mbox{baasi} protseduurid -- palju kogemust. Omal ajal vaatasin nelja inimese 
koodile peale siinsamas majas\sidenote{Meie jutuajamine toimus Tallinnas toonases Icefire 
kontoris aadressil Kauba 2a.} ja võin öelda, et kaks neist kirjutas väga ilusat koodi, kaks mitte. 
Kood töötas perfektselt ning kõik neli on tehniliselt väga head ja tugevad tegijad. 

\question{Aga mõnel on ilus kood ja mõnel ei ole. Kunst?}

Jah, see ongi kunst. See hakkab peale sellest, kuidas ma mõtlen ja 
oskan maailma abstraktselt kujutada. 

\question{See haakub Ahti jutuga, kuidas tema 
õppiski programmeerima just nimelt paberil ja programmist mõeldes. Temagi rõhutas võimekust asja oma peas ette kujutada.} 

Jah. Teine pool on see, et vahel kui lahendus liiga ära abstraheerida, tulevad sellised 
maailmamudelid, mis üldiselt eriti ei toeta \ldots 

\question{Kuidas seda tasakaalu hoida, et oleks piisavalt üldine ja ei paneks arendust kinni, 
aga ei üritaks ka maailma mudeldada?} 

Ilmselt see on 
mõtteviisis kinni ja tuleb loomulikult, kogemusega. Igaühel ei tulegi. 
Võib-olla on asi analüütilises mõtteviisis \ldots 

\question{Sinu jutust jääb kõlama, et kui on 
hästi kokku pandud tiim, siis see tiim jõuab tasakaaluni loomulikult 
oma kogemuste, oskuste ja parasjagu käsil oleva ülesande pealt.} 

Just. Muidugi tiimitöö. Inimeste koosmõtlemist ei ole võimalik üle väärtustada, sest koos välja mõeldud väärtus on tükk maad 
suurem. Meil on väga hea näide -- Jan\sidenote{Vilve peab silmas legendaarset hansapankurit Jan 
Laksperet\index[ppl]{Lakspere, Jan}, kellega nad on 
intervjuu ajaks koos töötanud üle kahekümne aasta.} \emph{versus} mina. Jan on 
perfektsete lahenduste mees: tema lahendused katavad üldiselt ära kõik asjad, sealhulgas
ääre-\emph{case}'id, mis lõpptulemusena võib tekitada olukorra, et lahendus 
läheb liiga keeruliseks. Mina jällegi suudan läheneda selle poole pealt, kuidas 
oleks mõistlik -- me töötame koos jube hästi. 

\question{Tuleme korraks tagasi Hansapanga aja juurde. Kui Tõnis\index[ppl]{Sildmäe, 
Tõnis} juhtkonda kokku pani, oli see ju ka sinu jaoks valikukoht, kas 
hakata inimesi juhtima või jääda koodi kirjutama. Tihti öeldakse, et kui 
heast programmeerijast juht teha, saad omale kehva juhi ja heast 
programmeerijast lahti. Kas sul seda hirmu ei olnud?}

See oli balansseeritud protsess, see ei juhtunud ühe päevaga. 
Kirjutasin ju Hansas peaaegu lõpuni ka koodi. 

\question{Kuidas sa seda tasakaalu hoidsid? Inimesi juhtides on lihtne
sinna sisse ära uppuda, nii et ühel hetkel ei kirjuta 
enam üldse koodi ja minetad selle oskuse.} 

Praegu ma näiteks ei kirjuta.

\question{Aga tahaksid?}

Nojah, vahel on kurb ka, et ei saa. See on 
olnud nii viimased kolm aastat, lihtsalt juhtus sedasi. Aga ma annan endale aru, et keegi pidi 
selle ülesande võtma. Ma ei ütle, et oleksin selle pärast õnnetu, 
kuid vahel ikka kriibib natuke.

\question{Miks sa ikkagi otsustasid juhi rolli ka juurde võtta?} 

Võib-olla oli mul iga asja kohta midagi öelda? Ja siis sind karistatakse selle eest. 

\question{Mäletades seda, kuidas Hansapank oli seesmiselt üles ehitatud, ja 
keskset Oracle baasi, kus kõik maailmaasjad sees olid, siis ühel hetkel läks
see süsteem ikkagi tehnoloogiliselt laiaks. Kuidas seda kontrolli 
all hoiti?} 

Alguses oli kõik väga lihtne ja koodi baasi kirjutamine oli õige otsus. Kood oli ilusti ja loogiliselt
struktureeritud, midagi ei tohtinud segamini ajada. Siis aga tulid igasugused tehnoloogilised \emph{switch}'id. Javat veel ei olnud 
nii palju, osaliselt kirjutati Cs. 

Kui tuli esimene internetipank, siis 
uurisime, kuidas seda teha, sest tehnoloogia ei olnud veel päris sealmaal. Ostsime sisse BroadVisioni platvormi.\sidenote{Varajane internetitehnoloogia, mis lubas igale kasutajale täielikult 
personaliseeritud kogemust. Paraku selgus, et kõigi nende kogemuste pakkumise 
vältimatuks eelduseks on nende väljamõtlemine. Küll aga võimaldas BroadVision 
üsna mõistlikult kombineerida HTMLi ja Cs\index{C} kirjutatud komponente 
(hiljem ka serveripoolset JavaScripti\index{JavaScript}.) ja sellest 
internetipanga, tellerirakenduse jms ehitamiseks juba piisas.} Kui oled kuskil liiga vara, siis teed otsuseid, mis 
ilmselt ei ole jätkusuutlikud, sest uus teletehnoloogia tuleb peale. 

\question{Kui mõelda, kuidas Sergei 
Anikin\index[ppl]{Anikin, Sergei} kirjutas Light Tellerit\sidenote{Lähemalt loe 
Sergei loost lk \pageref{sisu:teller}.}, siis see ongi see, kuidas praegu 
tehakse. Tehnoloogia võis ju jätkusuutmatu olla, aga lahendus oli vägagi 
jätkusuutlik!} 

Tegelikult meil juba tol ajal andmebaas pakkus teenuseid, loogika ei olnud 
segamini kliendis ja baasis. 

\question{Kes tegi otsuse niimoodi teha?} 

Meie tegime. Asi algas sellest, et mina ja Ruta Joost\index[ppl]{Joost, Ruta} 
panime esimesena need mustrid paika, kuna see tundus ainuvõimalik viis. 

\question{See on päris hea viis arhitektuuri teha, et võtad selle, mis tundub 
ainukesena võimalik!} 

Teatud asju ei saa põhjendada, kuidas need peas sünnivad. Ilmselt on see analüütiline mõtlemine, et vaatad, mida see tähendab, 
kui teen nii või naa. Mul oli kohe see mõte, et kui kirjutan 
mingi loogika klienti, siis ma ei saa seda ju korduvalt kasutada. Järelikult ma 
ei tee seda. Kirjutan nii vähe kui võimalik. Ja nii oligi. 
Sergei\index[ppl]{Anikin, Sergei} sai kiiresti teha Light Telleri, sest tal 
olid kõik teenused olemas. 

\question{Hansapank läks ju ka horisontaalselt suureks -- Markets,
kindlustused ja muu. Kuidas te seda pusa kontrolli all 
hoidsite, et keegi lollusi ei teeks?} 

See oligi hästi keeruline ja ega me ei kontrollinudki kõike lõpuni. 
Marketsil\index{Hansapank!Markets} oli väga suur eripära ja neil oli oma IT. 
Seal oli palju asju Excelis, lisaks analüütika, mis nad sealt pealt tegid. Neil olid sisseostetud lahendused, näiteks Condor, sest keegi ei hakanud 
\emph{trading}-lahendust ise tegema. Ja eks tehti
ka valesid otsuseid, näiteks kui osteti Marketsi platvorm, mis 
osutus liiga tooreks ja keeruliseks. Marketsit me tõesti keskselt palju 
ei hallanud, ainult nii palju, kui see haakus meie pakutavate 
süsteemidega. 

\question{Keegi pidi ju tegema otsuse, et \enquote{las nad toimetavad 
omaette}?} 

Just sel ajal, kui organisatsioon laienes, tekkisid meil ITs kindla
valdkonnaga tegelevad inimesed, kes meie poolelt olid sellel konkreetsel 
valdkonnal vastas. Paljuski oli tegu valdkonnajuhi ja Marketsi IT 
vahelise kompromissi, kokkuleppe ja ühise otsusega. 

\question{Kas põhimõtteliselt tõmmati neile organisatoorne kast ümber ja 
lasti neil seal sees vabalt toimetada?}

Jah, aga väljapoole kasti nad ei saanud, välismaailmaga suhtlemiseks olid väga 
selged liidesed. 

Teine näide oli Hansa Capital\index{Hansapank!Hansa Capital}, mis 
kasvas väga kiiresti, oli väga isepäine ja efektiivne ning äriliselt tegi superhead tulemust. Kuni üks hetk tuldi meie juurde ja öeldi: \enquote{Kuulge, 
meil läks Excel katki, tehke midagi!} Me siis vaatasime sellele loomaaiale 
peale, see oli peaaegu sama suur kui pank oma äridega! Panime projektitiimi kokku ja hakkasime tegema. 

\question{See haakub sellega, et üks asi on vedada inimesi, kes on 
\emph{hand picked}, aga teine asi on olla 
organisatsioonis kõrge taseme juht. Järelikult pidid sa aru saama, 
kuidas inimesed töötavad, nendega suhtlema ja panema neid vajalikke asju tegema. Kuidas sul see oskus tuli?}

Õppisin! Loomulikult oli ka palju koolitusi, mis tulid kindlasti kasuks ja panid 
teatud asjade peale mõtlema, aga tegelikult õpitakse läbi tegemise. 

Juhtimine on keeruline ja sellega toimetulemise määravad paljuski isikuomadused. Teine asi on kindlasti kogemus -- arvan, et 
olen praegu palju parem, kui ma olin siis, aga kui ma ei oleks seda 
protsessi läbi teinud, siis ma ei oleks seal, kus ma olen. Kõige keerulisem on 
hakkama saada inimestega, kes kõik sinult midagi tahavad, ja sa tead, et ei saa kõigile jah öelda. Kuidas 
teha seda niimoodi, et sünniksid õiged valikud, mitte anda
tähelepanu sellele, kes kõige kõvemini karjub. 

Ja vahel ei olegi keegi sinuga rahul. Sa pead sellega 
hakkama saama, et kõik sind ei armasta. Inimestega tuleb kindlasti rääkida, ega muu ei aita!

\question{Nii et kuskil sügaval peab lisaks arvutihuvile olema huvi inimese vastu?} 

Jah, muudmoodi ei saa! Huvi inimese vastu peab olema 
võib-olla isegi natuke suurem kui huvi arvuti vastu! 

\question{Ometi läksid sa õppima rakendusmatemaatikat, mitte 
psühholoogiat. Kust sul huvi inimese vastu tuli?}

Mul on see huvi olnud vist kogu aeg, kui nüüd mõelda. Nagu ma rääkisin, siis valisin tegelikult kirjanduse ja 
matemaatika vahel. 

\question{Kui palju te suhtlesite tol ajal pangavälise kogukonnaga? 
Kuskil pulbitses BBSide kamp, toimis mingi võrgustik. Kui palju te selles osalesite?}

Meil oli inimesi, kes seal suhtlesid, aga need olid kindlasti tehnilisemad inimesed,
nagu Toomas Lasmann\index[ppl]{Lassmann, Toomas}. Meie mitte nii väga. Pigem suhtlesime natuke 
kõrgemal tasemel: kes kuhu liigub ja kes milliseid tehnoloogilisi valikuid teeb.

\question{Üks asi, mida mina tollest ajast igatsen, on see, kuidas sündis 
iPizza\sidenote{Hiljem tuntud kui Pangalink. Minu mäletamist mööda sai see kohapeal 
välja mõeldud, teised ütlevad, et tegu oli Soome lahenduse ülevõtmisega. 
Igatahes võimaldas see (lisaks algselt ideeks olnud maksete lahendusele 
eesmärgiga internetis pitsat tellida -- sealt ka nimi) anda panka sisse loginud 
inimese identiteeti edasi teistele osapooltele. ID-kaart ei olnud veel levinud, 
keegi maksuameti paroolikaarti endale ei võtnud, aga maksuametil oli vaja saada 
inimesed internetis tulu deklareerima. Nii sündiski koostöö, sest pangal oli 
vaja anda inimestele hea põhjus nende internetipanka kasutada.}. Kogu protsess 
hetkest, kui astuti uksest sisse, et nüüd hakkame tegema, kuni 
hetkeni, kui maksuametisse sai sisse logida, võttis aega kolm 
nädalat. Praegu võtaks isegi lepingu läbirääkimine tõenäoliselt kauem.} 

Me oleme jõudnud sinnasamasse, kus on kogu vana maailm. Tegelikult on 
see kurb, aga kõik algab inimestest. Miks on lepingu 
läbirääkimine nii pikk? Me oleme ise teinud endale kõik need protseduurid ja 
poliitikad. Meie ise, mitte keegi teine. 

On olemas selline organisatsioon nagu Nordic Finance Innovation Forum, mille eesmärk 
on panna Põhjamaade pankades, kus miski ei liigu (täpselt sama asi, neli 
kuud räägitakse lepingut läbi), mingilgi viisil liikuma innovatsioon. Panna nad 
omavahel koostööd tegema. Ka meie osaleme selles, sest see on huvitav. Nad korraldavad 
päevaseid seminare mõnes Põhjamaade pealinnas, kus 
erinevad inimesed räägivad erinevatest asjadest, mis on tehtud. 

Ükskord rääkis üks kutt väikesest 
Soome firmast, 15 inimest, kuidas nad tegid AliPay integratsiooni Soome 
maksesüsteemiga. Hiinlasena saad praegu minna Helsingis igasse poodi ja 
maksta oma AliPayga. Kujutad ette! Projekt sündis sellest, et inimesed 
ütlesid: \enquote{Nelja kuu pärast, novembri lõpus, lendab siia 
Rovaniemisse mitu suurt lennukitäit, palju-palju 
hiinlasi ja neil kõigil on vaja sellega maksta.} Ideest 
\emph{live}'ini kulus neli kuud, nad tegid selle ära! See kutt ütles: \enquote{Üldiselt on 
meil Soomes nüüd niimoodi, et selleks kulub neli kuud, et saada esimene kohtumine mõnes Soome pangas, et oma 
ideest rääkida.} Aga see projekt tehti nelja kuuga ära!

\question{Ometigi ei olnud Hansa tol ajal enam tilluke organisatsioon?!}

Aga toimis ikkagi! 

\question{Kuidas sai niimoodi, et läbirääkimised ei võtnud neli kuud?}

Selle asja nimi on kultuur! Kultuuri loovad inimesed, see ei sünni mitte 
millestki muust. Ja kui nüüd on organisatsioon, kelle juhil on ainult üks omaenda isiklik ambitsioon 
olla suur juht (ambitsioon rääkida ümber kõike, mida ta on raamatutest lugenud, 
hoolimata sellest, mis tegelikult tehtud saab), siis sünnibki tema ümber 
samasugune kultuur. Eesti häda täna ongi paljuski see. Mitte ainult 
riigis, vaid ka erasektoris. 

\question{Ja Hansas oli juhtkond, kellelt tuli teistsugune kultuur!}

Kui mõelda, mis seal hiljem toimus, siis kõik need inimesed, kes selle 
kultuuri olid ehitanud, ju lahkusid. Mingi põhjus pidi olema, palk ju kehv ei olnud. See on väga kurb.

Olen siin umbes aasta ühele pangale rääkinud, et 
kallikesed, kui tahate, et asjad hakkaksid liikuma, siis peate seda 
vana kultuuri kandvad inimesed lihtsalt ära saatma. Te ei saa enne üle ega 
ümber, kui te ei julge teha seda otsust. Inimesed ei julge tihtipeale 
otsustada, sest need on rasked otsused. 

\question{Kogu see teekond on sind kuhugi toonud. 
Millega sa praegu tegeled?} 

Viimased 16 aastat oleme ehitanud Icefire't. Me kogu aeg muutume ja areneme. Mõtleme, kuidas 
maailm muutub ja kes me võiksime olla ning kuidas sinna jõuda. Siin tuleb mängu jällegi seesama kultuuriküsimus -- me hoiame oma 
kultuuri, mida meie inimesed kannavad. Ja see võib olla ka põhjus, miks 
me ei ole näiteks läinud seda teed, et ostame ettevõtteid kokku, 
lihtsalt kasvatame väärtust, ajame asja suureks ja lõpuks saame väga rikkaks. Oleme 
pigem hoidnud organisatsiooni hoomatavana. Ja täna läheme platvormiärisse, mis tegelikult muudab täiesti seda paradigmat, kus me tegutseme. 

\question{Ja sina oled selle asja juht ja koodi ei kirjuta?} 

Praegu juba kolm aastat ei kirjuta. Mul on fantastiline tiim. Oleme viimase kolme aastaga teinud suurepärase tulemuse pärast seda, kui vahepeal augukeses 
käisime! See, kuidas oleme suutnud ennast kasvatada samal ajal efektiivsust säilitades, näitab, et teeme midagi väga õigesti. Kummalisel kombel olen viimasel aastal märganud, et see külvab 
ümberringi kadedust mingites inimestes, kellega olen kunagi koos 
asju teinud. Miks me tunneme kadedust? Mina tunnen küll 
rõõmu, kui teisel hästi läheb! Aga eks me peame sellega elama. 