\index[ppl]{Tallinn, Jaan}

\question{Kuidas ja millal sina jõudsid arvutite juurde?}

Mäletan aega, kui isa hakkas kaheksakümnendatel Soome vahet 
käima filmi- ja videorežiitöid tegemas. Ta tõi mulle sealt erinevaid ajakirju, neid oli 
hea, odav ja võibolla isegi tasuta tuua. Päris mitmed olid 
arvutiajakirjad ja tundusid kohe väga põnevad. Armumine esimesest 
pilgust. 

Algkooli viimases või eelviimases klassis juhtus selline asi, et 
üks kooli lapsevanem valis mind ja mõningaid mu klassivendi (sealhulgas näiteks 
Priit Kasesalu\index[ppl]{Kasesalu, Priit}) eksperimentaalkatsejänesteks, et 
viia meid õhtuti kuskil Kopli servas asuvasse Sideministeeriumi 
arvutuskeskusesse\index{Sideministeeriumi Info- ja Arvutuskeskus} ja lasta seal 
suurte \emph{mainframe}'ide peal lahti ning vaadata, mis juhtub. Nii et arvutini jõudsin inimkatse 
tulemusel. 

\question{Mis kool see oli?}

See oli Lasnamäel 60. keskkool\index{Tallinna 60. Keskkool}.

\question{Kust selline mõte tuli, et peaks inimkatseid tegema?}

Seda ma ei tea, aga sa võid ta enda käest küsida. Tema nimi on Jüri 
Malsub\index[ppl]{Malsub, Jüri}. 
Seal seltskonnas oli peale minu Priit Kasesalu ja veel kaks klassivenda, kellest 
ühest, Mikk Orglaanest\index[ppl]{Orglaan, Mikk}) sai ka arvutiettevõtja. Neljas 
oli Martin Kruusvall\index[ppl]{Kruusvall, Martin}, kellele sai selgeks, et 
numbrid teda väga ei paelu ja et ta on rohkem luuletaja tüüpi.

Keskkoolis liitus selle seltskonnaga Ahti Heinla\index[ppl]{Heinla, Ahti}. Siis 
ma olin juba läinud Tallinnas Gustav Adolfi Gümnaasiumi, toona 1.
keskkooli\index{Tallinna 1. Keskkool}, ja hakanud tõsiselt tegema 
olümpiaadidega. Meie füüsikaõpetaja, kadunud Vilma Kukrus\index[ppl]{Kukrus, 
Vilma}, rääkis ühel hetkel Ahti pehmeks (peale seda, kui Ahti oli vabariikliku füüsika olümpiaadi 
kinni pannud), et mis ta seal Õismäel passib, tulgu parem 
Gustav Adolfisse. Nii et ta tuli meile teise keskkooliklassi ja me saime suhteliselt kiiresti headeks sõpradeks. Ilmselt mina, kes 
see muu võis olla, kutsusin teda sellesse seltskonda, kellega olime juba 
mõned aastad seal Kopli piiril tegutsenud. 

\question{Kas te käisite kogu selle aja \emph{mainframe}'i näppimas? Mida te sellega 
tegite?}

\emph{Mainframe}'id said kiire lõpu, kuna arvutustehnika arenes. 
Esimene mitte-\emph{mainframe} platvorm, kuhu me kolisime, oli sealsamas 
keskuses õhtuti meisterdatud riistvaraplatvorm 
Entel\index{Entel}, mis oli CP/M masin ja kasutas 
Intel 8088 protsessorit või mingit Vene klooni sellest kuulsast 
kaheksabitilisest protsessorist. CP/M tarkvara oli peal, aga midagi 
spetsiifilist selle jaoks kirjutatud ei olnud ja siis saigi 
hakata mitte-\emph{mainframe}'ide peal kätt proovima. 

\question{Mida te arvutitega ikkagi tegite? Noorel inimesel on ju see 
probleem, et kui valid liiga raske ülesande, ei saa hakkama ja on halb, ja kui 
liiga kerge, siis on igav ja ka halb.}

See on väga relevantne küsimus, sellepärast et mõnes mõttes on meie 
generatsioonil arvutitega vedanud. Sel hetkel, kui arvutite juurde 
sattusime, ei toimunud veel midagi väga huvitavat. 
Arvutite peamine köitvus oli potentsiaal, mis neis selgelt sees tuksus, erinevalt tänapäevast,
kui Youtube ja Minecraft on ühe kliki kaugusel. Ükskõik, kui palju sa praegu
pingutaksid, midagi ligilähedastki sa võimeline tegema ei ole. Teiseks olid
arvutid toona miljon korda aeglasemad kui praegu. 
Kui tahtsin midagi ägedat teha, siis pidin kiiresti selle 
hingeelu endale põhjalikult selgeks tegema, et pigistada välja viimanegi 
efektiivsusepiisk.

\question{Kas jooksid kohe riistvarapiirangutesse sisse ja isegi 
lihtsa asja ekraanil liigutamiseks pidi hoolega mõtlema, kuidas see täpselt käib?}

Täpselt, mistõttu läksime suhteliselt kiiresti 
assemblerile\index{Assembler} üle. Kõigepealt olid kodukootud 
Enteli arvutid ja aasta-paari pärast tekkisid Eestis esimesed IBM PC 
kloonid. 

\question{Assembleri peale kolimine eeldab siiski, et programmeerimisest on 
mingi aimdus olemas. Kust see tekkis?}

See tekkiski \emph{mainframe}'ide peal. Vist oli Robotron.

\question{Aga kuidas? Kas lugesid raamatuid?}

Lugesin läbi \enquote{Programmeerimine 
Pascalis}.\sidenote{R. Jürgensoni \enquote{Programmeerimine Pascal-keeles} (1985).} 
Mul on see raamat siiamaani raamaturiiulis, esimeste programmide 
väljatrükid vahel. Kirjutasin 
BASICus\index{BASIC} programmi ja kirusin, et keel on Pascalist erinev. BASICut ma ei osanud, aga Pascalit natuke teoreetiliselt oskasin ja 
nende kahe peale siis hakkasin avastama. Esimene programm oli vist ruutvõrrandi 
lahendaja.

\question{See on klassika, ilmselt kuna seda on praktiliselt vaja. Aga 
assemblerisse minna on ikkagi pikk samm. Kust sa infot said? Kas keegi õpetas? Raamatud? Ajakirjad?}

\emph{Mainframe}'ide peal jäin BASICusse ja tegin seal isegi 
oma esimese mängu. Ja kui kolisime \emph{mainframe}'ide pealt ära nende 
kodukootud kaheksabitiste arvutite peale, siis oli näha, et seal on lihtsam 
riistvarale ligi saada. Üks asi, mis hakkas kohe paistma ja ahvatlema, 
oli programmeerimiskeel C\index{C}. Mäletan, et samas grupis näitas aeg-ajalt 
nägu selline sell nagu Hannu Krosing\index[ppl]{Krosing, Hannu}, 
endine Skype'i kolleeg, kes otseselt samas seltskonnas ei olnud. Ja tema oli 
selleks hetkeks kirjutanud assembleri õpiku või pigem pisikese 
brošüüri. Ta vist pistis selle mulle pihku,
igal juhul ma lugesin selle läbi ja vaatasin, et ohoo, päris 
huvitav asi. 

\question{Mis aastal see võis olla?}

See võis olla aastal 1987 või 1988.

\question{Nii et 1987. aastaks oli Hannu kirjutanud assembleri õpiku!?}

Jah, sellise brošüüri vormis 
samizdat'i\sidenote{\begin{russian}Cамиздат\end{russian} ehk 
iseavaldamine oli Nõukogude Liidus levinud keelatud või põrandaaluse 
kirjanduse levitamise viis. Tekstid trükiti läbi mitme kopeerpaberi 
õhukesele paberile ümber ja levisid käest kätte ning neid paljundati 
omakorda. Mäletan, et ka minu vanaema tegeles sellega, ning lapsena 
ei mõistnud ma, miks sellest rääkida ei tohi. Kuna kõik klahvidega asjad mind 
väga huvitasid, nuiasin välja võimaluse ka ise tekste ümber lüüa, näiteks aitasin paljundada üht budistlikku teksti.}. 

\question{Mida sa assembleriga tegid?}

Üks korralikumaid projekte oli tekstiredaktor\label{sisu!jaani_tekstiredaktor}. 
Sattusime Priit Kasesaluga\index[ppl]{Kasesalu, Priit} 
võistlusrežiimi. Mõtlesime, mida oleks hea sellele uuele kodukootud 
platvormile kirjutada, ja leidsime, et seal ei olnud korralikku tekstiredaktorit. 
Hakkasime mõlemad tegema, kõigepealt BASICus\index{BASIC}, ja üritasime 
üksteist üle trumbata, et kummal tuleb parem. Mäletan, et 
Hannuga\index[ppl]{Krosing, Hannu} rääkisime tehnikast --- kuidas 
tekstiredaktoris teha \emph{split buffer} arhitektuuri, et 
liikumine ja \emph{insert}'imine oleksid kiired.

Tuli koolivaheaeg ja meil jäi võistlus pooleli. Aga mina panin edasi, kirjutasin suvi 
otsa paberi peal tekstiredaktorit, assembleris. Ja kui tagasi tulin, 
polnud Priit viitsinud suvel midagi teha ja sellega oli võistlus 
läbi. Siis kirjutasin selle assembleri paberilt arvutisse.

\question{Kas töötas ka?}

Esimene kord muidugi ei töötanud, aga tööle ma selle igal juhul sain, asi 
toimis ja vaatasin, et oo, päris äge. Kiire, mugav ja 
palju parem kui ükskõik milline tekstiredaktor sellel arvutil. See andis mulle 
väga positiivse tagasiside. Peale seda hakkasin mänge kirjutama.

\question{Järelikult pidi sul olema oskus päris suuri ja 
keerulisi abstraktseid struktuure peas ette kujutada, et suutsid kogu
koodi paberil Assemblerisse valada. Kust see oskus tuli või on see sul kogu aeg 
olnud?}

Ma ei tea, mulle tundus see suhteliselt loomulik. Lihtsad instruktsioonid nagu sammude kirjeldus. On 
vaja täpselt üles kirjutada, mida sa tahad, et arvuti teeks. Alguses BASICus sain esimese 
tagasiside, kuidas tsükkel käib, ja ühel hetkel assembleris nägin, et see on 
lihtsalt natuke tülikam, aga teisalt pakub rohkem positiivset tagasisidet, kui selle käima saad. Ja käib väga muljetavaldavalt võrreldes 
BASICuga. 

\question{Kas see tekstiredaktor jõudis kuskile edasi või sai lihtsalt oma lõbuks 
tehtud?}

See seltskond, kes Enteli\index{Entel} arvutit tegi, 
vormistas kooperatiivi, niipea kui eraettevõtlus muutus seaduslikuks, ja 
hakkas neid arvuteid tootma ja müüma. Muu hulgas käis arvuti juurde 
selle jaoks toodetud tarkvara CP/M ja lisaks minu tekstiredaktor.

\question{Nii et esimene suurem projekt, mille sa kirjutasid, läks kohe 
müüki?}

Jah, ma ei tea, palju seda kasutati, aga kui keegi endale kaheksakümnendate lõpus 
selle Eestis toodetud arvuti ostis, siis oli sel minu tekstiredaktor kaasas. 
Tänu sellele saime esimese palga ja tekkis esimene 
sissetulek.

\question{See ju tahab tarkvaraarenduse mõttes küpsust, et mõtled kõik 
nurgatagused juhtumid läbi ja võibolla kirjutad abiteksti?}

Olen märganud, et mul on selline 
OCD, \emph{obsessive compulsive disorder}: kui midagi alustan, tahan selle 
kindlasti lõpule viia, panna i-dele punktid peale. Seetõttu kulub paljudele 
projektidele, kus ei ole ajasurvet taga, kole palju aega. Alates sellest 
tekstiredaktorist --- tahtsin, et kõik oleks väga ilus, kogu funktsionaalsus 
olemas, ja pusisin senikaua, kuni oli. 

\question{Ehk siis kombinatsioon täiuslikkuse soovist ja võimekusest see ka ellu viia. Inimesel võib ju olla soov täiuslik teemant lihvida, aga ta lihtsalt ei oska seda 
teha.}

Mul vedas, et sattusin arvutite juurde sellisel 
hetkel, kui kogu tarkvara, mis seal juba oli, oli väga lihtne. 
Seetõttu ei olnud see nii-öelda hingemattev kogemus, et ma olen nii 
pisikene selle tarkvara kõrval, vaid pigem: ahah, okei, saan enam-vähem 
aru, kuidas tehtud on, ja teeksin paremini.

\question{Seda on mitmed öelnud, et tundsid oma esimest arvutit põhjani.}

Ka autoentusiastidel, uunikumide austajatel, on sama lugu, neil on 
väga lihtsad riistapuud.

\question{See annab kontrollitunde, eks ole?}

No mul läks täna Tesla katki ja midagi ei ole teha, tuleb Soome saata.

\question{Tuleme korraks arvutiajakirjade juurde tagasi. Kas
sa oskad tagantjärele öelda, mis sind nende juures paelus?}

Mis seal kohe väga prominentselt silma paistsid, olid 
arvutimängude reklaamid, näiteks Atari omad. Nagu 
ikka, joonistati reklaamid natuke ilusamaks kui päris maailm, aga 
need andsid vaate oma seaduste järgi toimivasse fantaasiamaailma, 
mis paelus tohutult. Vaatasin mingite tegelastega ekraanitõmmiseid, et ahaa, see on vist väga äge asi!

\question{See oli nagu täitsa teine maailm, kuhu sai 
sisse minna.}

Veelgi enam --- mõne aja pärast tekkis teadmine, et saan neid maailmu ise luua ja et ma ei ole passiivne tarbija, vaid aktiivne looja.

\question{Kas selle aktiivsusega kirjutasidki tekstiredaktori valmis ja 
sind võeti palgale?}

Ma ei mäleta, mis järjekorras see täpselt oli. Võimalik, et meid võeti palgale 
kohe alguses, kui seal niisama katsetasime, aga võibolla alles
pärast seda, kui esimesed asjad ära tegime.

\question{Kas see oli keskkooliajal?}

Jah, vist keskkooli alguses, 1987. aastal. 1986 võis olla see aasta, kui üldse sinna 
sattusin, ja 1987. või 1988. aastal hakkasin palka saama. 

\question{Kas arvutis käimine olümpiaade ei hakanud segama või käis see 
õppimisega lihtsasti kokku?}

Üldse ei seganud, kool ja olümpiaadid olid arvuti taustal. Arvutivärk on mul kogu elu olnud põhiline asi, ülikooli lõpetasin ka nii-öelda kõrvalhobina ära.

\question{Kas juba siis hakkas moodustuma seltskond, millest sai hiljem 
Bluemoon\index{Bluemoon}?}

Just. Bluemooni süda oligi tegelikult seesama seltskond, mõned klassikaaslased. 

\question{Kas arvutikooperatiivist eraldusite kohe eraldi ettevõtteks 
või oli vahepeal veel mõni faas?}

Ühel hetkel tekkis meil Ahtiga\index[ppl]{Heinla, Ahti} 
mõte teha korralik arvutimäng, mis jookseks PC, mitte 
ainult kodukootud arvutite peal. Meil oli eeskuju ka, mille järgi 
mängu teha, see oli Yamaha MSXide\index{Yamaha MSX} peal, mis oli 
palju vähem populaarne platvorm kui PC. Oli näha, et PCd hakkavad juba jõudma 
sinnamaani, kus saab midagi huvitavat teha. Mulle jätsid toona väga sügava 
mulje Ahti matemaatikuvõimed --- kuidas ta jagas ära, et 
\enquote{siin tuleb tangensit kasutada, et perspektiivi luua}. 
Esimesi eksperimente tegime tema vanemate juures Küberis\index{Küber}, kus tal 
oli arvutitele ligipääs. Ahti hakkas palju varem programmeerima kui mina. 

Üks tõuge mängu tegemiseks oli see, et keskkooli viimases klassis tekkis võimalus minna klassiga Rootsi. See oli esimene 
välisreis üldse, aastal 1989 suhteliselt unikaalne võimalus. Läksime läbi 
Leningradi, kuna siis oli vaja teha imelikke trikke välismaale 
saamiseks. Seal ütles onutütre mees mulle, et välismaal tahetakse õudselt 
softi: \enquote{Kirjutage mingi lahe soft! Lähed sinna, müüd maha ja pole
probleemi!}. Mõtlesin, et miks mitte, ja hakkasimegi mängu tegema. 
Valmis me seda muidugi ei 
jõudnud. Nagu ma nüüd hiljem tean, tuleb softiga kõik ennustused 
piiga läbi korrutada ja aega kulub umbes kolm korda rohkem, kui alguses arvad. 

Valmis me seda niisiis ei saanud, aga samas oli piisavalt suur hoog sees ja 
ühel hetkel võtsime appi korraliku kunstniku Kaspar 
Loiti\index[ppl]{Loit, Kaspar} ehk BKnowsi, ja muusika- ning heliinimese Ott 
Aaloe\index[ppl]{Aaloe, Ott}. Ja tegime mitte ainult ühe mängu, vaid 
mängude seeria. Meil vedas --- esimene mäng 
õnnestus Rootsi müüa hoolimata sellest, et Rootsis käigu ajal ei olnud seda 
kuhugi pakkuda, isegi kui see oleks valmis olnud.

\question{Kuidas see teil õnnestus? Siis oli ju veel Nõukogude Eesti.}

Sideministeeriumi arvutuskeskuse\index{Sideministeeriumi Info- ja 
Arvutuskeskus} juhatajal Jüri Malsubil\index[ppl]{Malsub, Jüri} oli üks tuttav 
sell nimega Tiit Vasli\index[ppl]{Vasli, Tiit}, kellel oli välismaal suhteid, 
kuna ta vahendas sinna midagi, vist metalli. Ta oli selline mees, keda oli juba kaugelt näha, sellepärast et tal oli 
üks Eesti esimesi mobiiltelefone, mille antenn oli kolm meetrit kõrge. 
Tal oli Rootsis ka sidemeid 
ja ta müüski meie mängu sinna. Tema äripartnerid 
olid huvitatud sellisest eksootilisest asjast nagu raudse eesriide taga 
toodetud mäng. 

Mängu müügist teenisime rohkem, kui mu vanemad kunagi oma 
elu jooksul olid teeninud, vist viis tuhat dollarit --- muidugi nominaal-, mitte reaalväärtuses, arvestades 
inflatsiooni. 

Kui mäng müüki läks, tekkis meil tõsine küsimus, kuidas seda administratiivselt 
korraldada. Olime kooperatiivis ametlikult tööl, aga tegelikult oli
näha, et meie plaanid võivad kujuneda kooperatiivist suuremaks. Mäletan pingelist läbirääkimist Jüri Malsubiga\index[ppl]{Malsub, Jüri}, 
kuidas mängu tulu jagada. Nemad olid ühelt poolt ja meie 
teiselt poolt panustanud, aga nüüd tahtsime oma asja teha. Lõpuks jõudsime 
väga mõistlikule kokkuleppele ja aastal 1990 vormistasime oma ettevõtte. 

\question{Kas te mõtlesite nullist välja, et teil on vaja kunstnikku ja 
muusikut ning kuidas nende töö programmeerimisega siduda, või oli teil eeskujusid 
ka?}

Olime teisi mänge näinud ja need nägid paremad välja kui 
meie katsetus ilma kunstniketa. Ma ei mäleta, kes meid 
BKnowsiga\index[ppl]{Loit, Kaspar} tutvustas, vist Tanel 
Hiir\index[ppl]{Hiir, Tanel}. Igatahes jätsid Kaspari kunstnikuvõimed 
mulle väga sügava mulje. Teda oli küll raske tööle saada, tihtipeale 
pidi selja taga istuma, et tee nüüd, aga kui ta tööle asus, oli tulemus väga 
äge. 

\question{Tänaseks on välja kujunenud kindlad viisid graafikat kasutada, 
töödelda ja laadida. Kas toona mõtlesite need ise 
välja?}

Üsnagi jah, sest platvormid olid miljon korda aeglasemad kui 
praegu, mistõttu tööriistad olid Turbo Pascal\index{Turbo Pascal} ja 
Borland C\index{Turbo C}. Kaspar tegi asju Amigal, kus tal olid oma 
tööriistad.

\question{Kuidas te muusikaprogrammi tegema sattusite? Ükski teist pole ju muusikainimene, nii 
palju kui mina tean.}

Ükskord ülikooli arvutiklassis komponeerisid mingid tüübid 
SoundClubis\index{SoundClub} muusikat. 
Kiibitsesin natuke ja ütlesin, et see on minu programm. Nad ei uskunud. 

Ma ei mäleta, kuidas algtõuge tekkis. Tänu sellele, et olime juba mänge 
teinud, oli meil kindlasti kokkupuude taustmuusika loomisega. 
Toona, üheksakümnendate alguses, oli väga suur trend tracker'id ehk 
sämplite baasil muusika kirjutamise riistapuud. Ja 
sealt tuli mõte, et heli on väga hea, aga kasutajakogemus tundus vähemalt 
harjumatule silmale väga ebamugav. Mõtlesime, kuidas kasutada sedasama 
tehnilist võimekust, aga teha äge kasutajaliides, eriti sellistele inimestele, 
kes ei ole pidevalt muusikakirjutamise juures.

Teema hakkas järjest rohkem huvitama, kuna seal on mitmeid nüansse, nagu UI 
disain ja muusikapool (kuigi ükski meist ei olnud muusikud), ning
kuidas tehniliselt teha aeglastel arvutitel head heli. Seal puutusin esimest 
korda kokku matemaatiliste teoreemidega, mida üritasin
Ahti\index[ppl]{Heinla, Ahti} abil lahendada. Üks huvitav asi oli see, 
et kuna korjasime instrumendid kuskilt BBSidest kokku, olid 
need õudse kvaliteediga. Ahti kirjutas tarkvara, kus ta 
tegi Fourier' analüüsi, et need häälde viia. Ükskord 
häälestasin ise ka Tartus pille niimoodi, et endal suurt muusikaharidust ei olnud, 
natuke olin pilli õppinud. Fourier' analüüsiga sai väga hea häälestuse. 

\question{Selle tarkvaraga on tehtud igasugu asju, näiteks Vennaskonna \enquote{Disko}. Küll aga ei ole juttu olnud sellest, et keegi oleks 
selle tarkvara ostnud.}

Jaa! Saame siiamaani vähemalt kord kuus 
fännikirju, et olen SkyRoadsi\index{SkyRoads} peal üles kasvanud. Isegi mõned 
kloonid on tehtud, seda saab tänapäeval veebis mängida. Ja SoundClub oli teine 
suurem projekt. Meil oli siis juba firma Bluemoon ja kaks toodet: 
SkyRoads (mis tegelikult oli järg tollele esimesele Rootsi müüdud mängule, 
mille nimi oli Kosmonaut\index{Kosmonaut}) ja SoundClub. 

Nüüd tekkis küsimus, kuidas neid müüa. Mäletan
telefonide, fakside ja tšekkidega jamamist. Mõnikord 
saadeti lihtsalt ümbrikus sularaha, aga tavaliselt tšekke, mida ma 
käisin Eesti Maapangas või Rahvapangas lunastamas. 

Teine äge kogemus oli läbirääkimiste pidamine 
olukorras, kus teisel poolel ei ole mingit juriidilist motivatsiooni 
lepinguid järgida, mistõttu tuli tihtilugu ise
luua helget tulevikku, et partnerlusel oleks jumet. Mõnes mõttes 
selline \emph{iterated prisoner's dilemma}\sidenote{Mänguteoreetiline 
konstruktsioon, mille abil uuritakse osapoolte koostööstrateegiaid. Selle 
valdkonna üks teadustulemusi on, et indiviidile 
annavad pikas perspektiivis parema tulemuse altruistlikud, mitte egoistlikud 
strateegiad (eriti mängu iteratiivses ehk korduvalt 
mängitavas ja eelmisi tulemusi \enquote{mäletavas} versioonis).}: pead looma olukorra, kus teisel poolel, hoolimata sundmehhanismi puudumisest, on lihtsalt huvi olla osa sinu tulevikust ja 
seeläbi lepinguid järgida.

Alguses oli meil \emph{shareware}, aga inimesed hakkasid kirjutama, et tahaks 
seda ajakirja panna või kuskil mujal levitada. Näiteks üks
lahe sell hakkas Saksamaal meie asju levitama aastal 
1996, lõpuks käisin tal isegi külas. Omaette äge kogemus oli ka
müük Taiwani telefoni ja faksi abil, kuskil Tartu Estiko\index{Estiko} 
kontoris. 

\question{Kuidas sa Tartusse sattusid?}

Läksin ülikooli.

\question{Mida sa õppima läksid?}

Füüsikat. Nii mina kui ka Ahti läksime füüsikat õppima, aga Ahti kukkus sealt juba 
teisel aastal välja. Mina punnitasin lõpuni. 

\question{Miks just füüsikat? Ahti rääkis, et see ala tundus talle mõnes mõttes 
kõige puhtam.}

Ma arvan, et see on vähem puhas kui arvutiteadus või matemaatika. Mul oli füüsika valikuks kaks 
põhjust (Ahti põhjused olid ilmselt koreleeritud). Esiteks tundsin, et olen arvutites ja matemaatikas juba piisavalt sees, samas kui 
füüsika laiendaks silmaringi. Teiseks oli meie füüsikaõpetaja Vilma Kukrus\index[ppl]{Kukrus, Vilma} väga-väga äge õpetaja 
ja tekitas füüsika vastu sügava huvi või vähemalt sügava austuse. Mul on väga 
hea meel, et ma füüsika lõpetasin.

\question{See ilmselt mõjutas noore inimese maailmapilti päris olulisel määral?}

Absoluutselt. Füüsika on selles mõttes optimaalne teadus, et sa suhestud 
reaalse maailmaga niimoodi, et see tõmbab su alati maa peale tagasi, samal ajal kui matemaatikud võivad minna niivõrd 
abstraktseks, et kaugenevad reaalse maailma piirangutest. Ja seetõttu tekib füüsika puhul intuitiivne arusaam sellest, mis on teadus. 

\question{Teil on Ahtiga\index[ppl]{Heinla, Ahti} samasugune 
muster --- see, mida ma kuulen, ei kõla nagu keskmise 
\emph{teenager}'i jutt. See kõlab üsna küpse inimese moodi!}

Praegu ma enam \emph{teenager} ei ole!

\question{Nüüd jah, aga toonased otsused ja viis, kuidas te asju ajasite, peegeldavad
kainet ja arutlevat lähenemist. Kust see pärit on?}

Mina võlgnen väga palju tänu Ahtile. 
Meil oli superhea koostöö, samuti Priiduga. Lahendasime praktiliselt kõiki 
probleeme tiimiga. Ahti on nupukas ja ta mõtleb minuga võrreldes hoopis erinevalt, mistõttu 
temaga koostöös sündinud otsused olid ägedad, kuna need hõlmasid kaht 
väga erinevat vaatepunkti, mida otsus pidi rahuldama.

\question{Kui 
vaadata Bluemooni kodulehekülge, on seal loetletud üksjagu edukaid 
projekte, aga ka mitu asja, mis ei ole ühel või teisel põhjusel välja tulnud. 
Inimesed ei suuda mõnikord isegi suure edu korral tiimina toimima jääda, aga teie 
olete koos nii edu kui ka ebaõnnestumiste korral. Kuidas te seda 
teete?}

Edulugude kontekstis meenub vahemärkusena selline lahe kogemus, et mõni aasta tagasi sain kokku Sean Parkeriga\sidenote{Napsteri kaasasutaja ja muu hulgas 
Facebooki esimene president. Ta esineb ka tegelaskujuna Facebookist rääkivas 
filmis, kus kujutatud intriigidest ja tülist tõukub ka eelnev küsimus.}, kellega 
meenutasime Napsteri ja Kazaa aegu.

Ma arvan, et meie kohatised eduelamused olid piisavad, et läbi 
suruda ka mitte nii õnnestunud projektidest. Ja mõned hetked olid ikkagi 
jube rasked. Näiteks kui kogu mänguarendus läks 
ülesmäge, aga ühel hetkel lõpetas meie kirjastaja Ameerikas, Interactive 
Magic, \emph{milestone}'ide maksmise. Raha jaoks oli meil Exceli tabel, kus 
\emph{runway} oli kogu aeg kirjas ehk mitmeks kuuks meil raha on.
\emph{Runway} hakkas järjest kahanema ja ühel hetkel oli selge, et nad on 
pankrotis ning sealt enam midagi ei tule. Tekkis tõsine küsimus, mis edasi 
saab, ja just Ahti\index[ppl]{Heinla, Ahti} ütles, et ah, 
küll me välja ujume! Ujusimegi.

\question{Ühel hetkel suundusite mängu kirjutamiselt 
Everyday kirjutamisele. Kas legend, et see käis lehekuulutuse kaudu, vastab tõele?}

Vastab küll jah. See oligi see raske hetk, kui mõtlesin, mis 
me nüüd teeme.

\question{Mis aastal see oli?}

See oli aastal 1999. Ühel reede hommikul nägin Tartus 
lehekuulutust, kus pakuti ulmelist palka\sidenote{Teist 
perspektiivi sellele loole vt Tarvi jutust lk
\pageref{sisu:everyday}.}. Tele2 oli 
börsile lubanud, et kohe tuleb Everyday portaal välja, aga nad olid arendushädas. Nad olid juba aasta või 
paar arendanud ning väljatulekust oli asi kaugel. Kuulutuses oli pikk nimekiri 
nõuetest, millele arendajad peaksid vastama, asjad, millest 
ma ei olnud kuulnudki, nagu IMAP, POP3, PHP ja SQL. Tähtaeg oli esmaspäev, mina olin Tartus ja teised 
Tallinnas. 

Helistasin Ahtile\index[ppl]{Heinla, Ahti}, rääkisime 
läbi ja mõtlesime, et proovime ning vaatame, mis juhtub. Tegime kohe nädalavahetuse 
plaani ja esmaspäevaks oli prototüüp soovitud portaalist valmis. Kui läksime esmaspäeval intervjuule, oli meil dilemma, 
kas öelda, et see oli ühe nädalavahetusega kirjutatud, või mitte. Prototüüp nägi 
väga hea välja, isegi parem kui lõpptoode, kuna meil oli palgal arvutimängudega karastunud kunstnik. Ja toimis 
ka täitsa hästi: sai sisse logida, erinevaid paneele ringi lohistada, meili 
kirjutada, uudiseid lugeda, ilmateadet vaadata ja mida iganes. 

\question{Järelikult pidi teil nädalavahetusega kujunema päris hea 
arusaam sellest, kuidas HTML, brauseri renderdus ja muu selline töötab.}

Võtmesõna oli andmebaasid. Sellega tegelemine jäi Priidu\index[ppl]{Kasesalu, Priit} peale. Ta jäi hätta, ei saanud loogikast aru ja helistas laupäeva hommikul ühele andmebaasieksperdile. 
Tollel ei olnud paraku aega ja Priit hakkas ise uurima, kuidas 
SQL käib. Uuris välja ja sai tehtud.

\question{Kõlab üsna ulmelisena! Seal peab ju olema meetod taga, kuidas 
seda teadmist omandada?}

See oli jah väga äge kogemus. Tegelikult tehnoloogia ei olnud superkeeruline, vaid lihtsalt meile võõras. Tiim oli meil tõesti äge 
ning saime tööjaotuse tehtud: igaüks pidi kindla aspekti välja uurima. 
Magasime natuke, 48 tundi tegime tööd.

\question{Seega pidite oma tööd ka koordineerima. 
Kes kampa juhtis?}

Mina olin nii-öelda ametlik juht. Samas töötas tiim ise ka päris hästi, 
välja arvatud ehk kunsti pool --- kõige rohkem põhjustas
meelehärmi, kuidas Kasparilt\index[ppl]{Loit, Kaspar} lubatud asjad kätte 
saada. Kunstniku asi, rohkem boheemlane kui teised.

\question{Kui kuulutus ilmus 1999. aastal ja keskkoolis käisite 
1990ndate alguses, siis kas te kogu kümnendi kirjutasite mänge?}

Jah, tegime päris mitmeid mänge, kutsusime ennast Eesti mängutööstuseks. 

\question{Kui suur teie tiim oli?}

1999. aastaks see väga palju suuremaks ei läinud. \emph{Core} tiimis olime 
mina, Priit\index[ppl]{Kasesalu, Priit}, Ahti\index[ppl]{Heinla, Ahti} ja 3D-kunstnik Artur 
Vill\index[ppl]{Vill, Artur}. Tema on muide teinud 
sellise filmi nagu \enquote{Happy Feet}, kus on ägedad maastikud. Ta kolis 
pärast Bluemooni kokkukukkumist\sidenote{Ilmselt peab Jaan silmas raskusi Bluemooni USA levitajaga ning liitumist Everyday projektiga. \url{bluemoon.ee} toimib siiski ka aastal 2022.} Austraaliasse ja tegi seal tähelendu --- väga 
kõva vend 3D-modelleerimises ja -kunstis.

Lisaks tegid Kaspar\index[ppl]{Loit, Kaspar} kunsti ning Ott\index[ppl]{Aaloe, 
Ott} ja Glen Pilvre\index[ppl]{Pilvre, Glen} muusikat. Juhan 
Soomets\index[ppl]{Soomets, Juhan} tegi poole kohaga 3D-graafikat ja 
vist oligi kõik. Kui minna bluemoon.ee lehele, siis see tiim on seal siiamaani 
üleval.

\question{Isegi toonase tehnoloogia lihtsuse juures pidi teil siis 
väikse tiimi peale tööd palju olema?}

Tööd oli päris kõvasti jah. Põnev oli ka muidugi.

\question{Mis põnevust tekitas? Kui olite ühe mängu valmis teinud, kas siis 
igavaks ei läinud?}

Mängude tegemine ongi selles mõttes äge, et see pakub niivõrd palju rahuldust, kui mõni asi tööle läheb. Kui kirjutad andmeanalüüsi ja asi hakkab
tööle, tuleb ekraanile õige number. Aga kui mängus asi tööle läheb, tuleb 
näiteks vägev plahvatus või rahuldust pakkuvad 
stseenid, efektid või lood. 

\question{Sest tekib vaade mingisse teise maailma, millest oled nüüd järgmise 
tüki loonud?}

Just. Ja nüüd saad testimise käigus ringi käia ning vaadata
kohati väga vapustavaid vaateid ja toimunud sündmuseid.

\question{Kui sa tegid üksi 
tekstiredaktori ja seal pidid kõik asjad ilusti reas olema, sellest ma saan aru. 
Aga kui meeskonnana softi kirjutada, siis see eeldab ju ka \emph{software 
engineering}'u protsesse. Kust teil need tulid?}

Üheksakümnendatel olid lihtsalt zip-failid ja \emph{backup directory}'d. 
Versioneerimist või muud säärast ei teinud me üldse. 

\question{Kuidas te siis tagasite, et kogu kupatus teil kokku ei kukkunud?}

Me olime väga ettevaatlikud! Üks põnts, mis meil siiski juhtus, oli see, et meile murti 
kontorisse sisse ja varastati arvutid ära. Meil olid diskidel \emph{backup}'id ja 
midagi jäi alles, aga mõned asjad Bluemooni ajaloost läksidki lõplikult 
kaduma, näiteks mingi SkyRoadsi\index{SkyRoads} versioon.

\question{Kas te oma töökorralduse mõtlesite lihtsalt jooksu pealt 
välja?}

Istusime telefoni otsas. Mina olin Tartus, Ahti\index[ppl]{Heinla, Ahti} kolis ühel 
hetkel Tallinna tagasi. Lisaks telefonile käis koordineerimine meili 
teel. Eks meil tekkis spetsialiseerumine. Mina manageerisin tiimi ja 
kunstnikke ning kirjutasin mingisuguseid tarkvaralõike. Priit\index[ppl]{Kasesalu, 
Priit} spetsialiseerus operatsioonisüsteemidele, graafikale ja Windowsi API-le. Ahti\index[ppl]{Heinla, Ahti} tegi 
teadusmahukaid asju, kus oli vaja midagi AI-laadset või 
matemaatikat. Kui midagi oli vaja, siis tal oli kohe võtta: enquote{Ahaa, ma tean, selle jaoks 
on siin sellel leheküljel Knuthi \enquote{Art of Computer Programmingus}\sidenote[][-2cm]{Knuth, 
Donald E. Art of computer programming (TAOCP), tuntud ka kui Knuthi Piibel. 
Tegu on monumentaalse teosega, mille seitse köidet pidid katma kogu teadaoleva 
arvutiteaduse. Praeguseks on ilmunud kolm köidet ja esimene osa neljandast. 
Kuna teise köite küljenduse kvaliteet lugupeetud autorit ei rahuldanud, lõi ta 
oma raamatu ilusaks tegemiseks süsteemi TeX, mille derivaatide abil kirjutab 
praegu oma artikleid kogu teadusmaailm ja mille abil on kujundatud ka käesolev 
tekst.} õige algoritm, teeme selle!}

\question{Tundub, et kuna töötasite tiimina hästi, siis see lahendas 
üksiti ära ka \emph{software engineering}'u probleemid. Konflikte ei tekkinud, sest töötasite inimlikult nii hästi koos.} 

Tiim oli väike ka, kolm programmeerijat. 

\question{Kui mõtled enda kui programmeerija peale üheksakümnendatel, 
siis kas oskad kuidagi kirjeldada oma isiklikku arengut?}

Põhiliselt domineeris arengut see, et arvutid läksid iga 
kahe aasta tagant kaks korda kiiremaks, mistõttu oli vaja kogu aeg hoolitseda 
selle eest, et ajale jalgu ei jääks. Lõpuks me ikkagi jäime, mängutööstuse aeg 
sai läbi. Kümne aastaga läksid arvutid kolmkümmend korda kiiremaks, heli rikkalikumaks, 
mälu suuremaks, graafika ägedamaks ja võrgundus tuli juurde. Kogu aeg tuli 
ennast hoida aja tasemel. 

\question{Aga programmeerimise kunsti mõttes? Mitte see, kas tean üht või 
teist APIt, vaid et olen programmeerijana täna parem kui eile?}

See on huvitav küsimus, ma ei ole selle peale eriti mõelnud. Kindlasti 
kogemus õpetas, aga tagantjärele on seda raske kokku võtta. 
Tean, et programmeerijana arenesin pigem hiljem või siis lihtsalt mäletan hilisemat arengut paremini. 
Funktsionaalse programmeerimise peale kolisin ma pärast Skype'i. Kuni Skype'i 
lõpuni kirjutasin oma vanade tööriistadega.

\question{Kui kiiresti te pärast Everyday intervjuud tooteversiooni välja lasite?}

Ma hästi ei mäleta, aga see võis olla nii, et suvi otsa kirjutasime ja 
sügisel tuli välja. Esimesele versioonile võis kolm-neli 
kuud minna.

\question{Sellesama väikse tiimiga?}

Põhimõtteliselt küll, kuigi nüüd oli juures rootslaste 
tehtud asju ja see väike tiim oli osa palju suuremast organisatsioonist. 
Mistõttu läks asi ka oluliselt aeglasemaks. Osa asju oli vaja rootsi keele 
tõlkida. Ükskord sain öösel Niklaselt\index[ppl]{Zennström, 
Niklas}\sidenote{Niklas Zennström, hilisem Skype'i asutaja.} meili 
rootsikeelsete vastetega inglisekeelsetele fraasidele ja all oli 
\enquote{midnight translation services by Niklas Zennström}.

\question{Mida Niklas selles projektis tegi?}

Niklas oli everyday.com-i tegevjuht. 

\question{Kas Niklas ei suutnud siis kogu oma rootslaste tiimiga 
tarnida?}

Seda ma täpselt ei tea, kuidas see atributsioon seal täpselt oli, aga Niklase 
lõplik otsus oli Eestist arendaja otsida. 
Linnar Viik\index[ppl]{Viik, Linnar} vist oli pakkunud, et võtaks Eesti 
programmeerijad, ja Niklas langetas otsuse kaasata Bluemooni 
seltskond.

\question{Kuidas tiimi skaleerumine tundus? Kui seni olite 
kirjutanud kompaktses kõgproffide tiimis keerulist softi, siis veebiarenduses 
on rõhk ju mujal?}

Ma erilisi probleeme ei mäleta, peale 
selle, et kommunikatsiooniülesanne tuli kohe juurde. Nagu ikka, kui on kaks 
programmeerijate tiimi, siis esimene reaktsioon on kõigil \enquote{see on 
teise tiimi bugi}. Neid asju tekkis kõvasti. Aga 
mingi tohutu külma vee kaela saamist ei olnud. Saime hakkama küll. 

\question{See kõik toob meid 1999. aastasse ja seega otsapidi väljaspoole 
meie ajaraami, milleks on kaheksa- ja üheksakümnendad. Mitte et pärast ei 
oleks igasugu põnevaid asju veel juhtunud.}

Enamik asju juhtus hiljem!

\question{Mida sa praegu teed?}

Peamine ja kõige olulisem tegevus on hoolitsemine selle eest, et kui 
inimajastu peaks mõne aastakümne (loodetavasti mitte mõne aasta) jooksul 
lõppema, jääksid inimesed siia planeedile alles.

\question{Palun selgita seda mõtteliini arengut ekraanile 
plahvatuse joonistamisest selle teemani!}

Sinna mahub üks kuni kaks aastakümmet veel ehk see, millest me ei ole 
rääkinud. 

\question{Kas sa oled lihtsalt jõudnud selleni, et see on sinu jaoks oluline 
probleem?}

Jah, jõudsin selleni umbes aastal 2008, kui Skype'is hakkas hoog 
raugema ja seal enam erilist väljundit ei olnud. Sattusin rääkima 
inimestega, kellega olen viimased kümme aastat ehitanud 
\emph{community}'t, mis üritab teha ära AI uurijate kodutöö ehk 
asju, mida on vaja selleks, et AIga tulevik oleks inimestele soodne, aga 
millega AI arendajad ise ei ole näidanud mingit motivatsiooni tegeleda peale 
abstraktse motivatsiooni, et nad on ka inimesed.

Üks võimalus probleemi kirjeldada on see, et meil on fundamentaalne 
\emph{trade-off}, see tähendab, et ei saa 
kahte asja korraga: superkompetentset süsteemi ja sellist süsteemi, mille 
üle sul on täielik kontroll. See ei ole isegi arvutite spetsiifiline probleem, 
inimjuhtidega on sama lugu: mida rohkem ta delegeerib, seda 
kompetentsemaks muutub süsteem või suuremaks kasvab organisatsiooni võimekus. 
Aga tema isiklik kontroll toimuva üle väheneb. See on fundamentaalne 
printsiip. Ja mida inimkond praegu iga päevaga järjest rohkem teeb --- 
delegeerib oma otsuseid masinatele, mistõttu tegelikult 
väheneb inimeste kontroll tuleviku üle. Võib öelda, et juba praegu on inimeste 
kontroll tuleviku üle väiksem kui näiteks viiskümmend aastat tagasi ja see tendents tõenäoliselt jätkub. 

Nüüd ongi küsimus, kuidas me siiski 
säilitaksime kontrolli mingisuguste oluliste aspektide, näiteks atmosfääri 
koostise üle. Temperatuur tundub juba praegu keeruline teema --- 
räägime sinuga veebruarikuus, väljas on kolm kraadi sooja, sajab vihma. 
Inimestel on raske keskkonna üle kontrolli säilitada. Lisame siia entusiastliku 
delegeerimise arvutitele, kellel on keskkonnast täiesti ükskõik! Sellepärast 
saadamegi roboteid radioaktiivsetesse aladesse või kosmosesse, et neid keskkond 
ei huvita. Probleem on selles, et AI arendajatel on motivatsioon arendada just 
nimelt delegeerimist, et see oleks võimalikult lai ja tulemus 
mingi meetrika järgi võimalikult kompetentne. Palju vähem on 
motivatsiooni mõelda sellele, kuidas kogemata mitte 
delegeerida selliseid asju, mida meie elusolekuks on vaja.

\question{Ma südamest loodan, et see tuleb sul välja, sest muidu on pahasti!}

Ütlen tihtilugu inimestele: \enquote{Wish me luck, you are going to need 
it!}