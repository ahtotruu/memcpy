\index[ppl]{Tallinn, Jaan}

\question{Kuidas ja umbes millal sa jõudsid arvutite juurde?}

Ma mäletan nagu seda aega, kuskohas mul isa hakkas kaheksakümnendatel Soome vahet käima, seal mingisuguseid
filmi ja videorežiitöid tegemas. Ja ta tõi mulle erinevaid ajakirju, neid oli hea, odav ja võib-olla isegi tasuta tuua. Ja neist päris mitmed olid arvutiajakirjad ja see tundus kohe olema väga põnev. Armumine esimesest pilgust. Algkooli kas  viimases või eelviimases klassis juhtus selline asi, et mind ja mõningaid mu klassivendi (sealhulgas näiteks Priit Kasesalu\index[ppl]{Kasesalu, Priit}) valis üks kooli lapsevanem eksperimentaal-katsejänesteks, et viia  meid õhtuti  kuskil Kopli servas asuvasse Sideministeeriumi Arvutuskeskusesse\index{Sideministeeriumi Info- ja Arvutuskeskus} ja lasta seal suurte \emph{mainframe}-de peal lahti ja vaadata, mis juhtub. Nii et inimkatse tulemus. 

\question{Mis kool see oli?}

See oli Lasnamäel kuuekümnes keskkool\index{Koolid!Tallinna 60. Keskkool}.

\question{Kust selline mõte tuli, et peaks inimestega niimoodi tegema?}

Seda ma isegi ei tea, aga sa võid ta enda käest küsida. Ta nimi on Jüri Malsub\index[ppl]{Malsub, Jüri}, talle meeldib sellest väga pikalt rääkida. Seal seltskonnas olin mina, Priit Kasesalu ja veel kaks klassivenda, kelles ühest (Mikk Orglaan\index[ppl]{Orglaan, Mikk}) sai ka arvuti-ettevõtja. Neljas oli Martin Kruusvall\index[ppl]{Kruusvall, Martin}, kellele sai selgeks, et numbrid teda väga ei paelu, et ta on rohkem nagu luuletaja tüüp. 

Keskkoolis liitus selle seltskonnaga Ahti Heinla\index[ppl]{Heinla, Ahti}. Siis ma olin juba läinud Tallinnas Gustav Adolfi Gümnaasiumi, toona esimesse keskkooli\index{Koolid!Tallinna 1. Keskkool} ja hakanud tõsiselt tegema olümpiaadidega. Meie füüsikaõpetaja, kadunud Vilma Kukrus\index[ppl]{Kukrus, Vilma} ühel hetkel (peale seda, kui Ahti oli vabariikliku füüsika olümpiaadi kinni pannud) rääkis Ahti pehmeks, et mis sa seal Õismäel passid, tule parem Gustav Adolfisse. Nii, et ta mitte esimese keskkooli klassi, vaid  teise  tuli meile ja me saime suhteliselt kiiresti headeks sõpradeks ja siis ma (ilmselt mina, kes see muu võis olla), kutsusin teda sellesse seltskonna, kellega me olime juba  mõned aastad seal Kopli piiril tegutsenud. 

\question{Ja kogu selle aja te käisite \emph{mainframe}-i näppimas? Mis te tegite nendega?}

No \emph{mainframe}-d said muidugi kiire lõpu, kuna arvutustehnika arenes. Ja esimene mitte-\emph{mainframe} platvorm, kuhu me kolisime, oli sealsamas keskuses õhtuti meisterdatud riistvara platvorm nimega Entel\index{Arvutid!Entel}, mis oli selline CP/M masin. Ta kasutas mingisugust Intel 8088 protsessorit, või mingit Vene klooni sellest kuulsast kaheksabitilisest protsessorist. CP/M tarkvara oli, aga midagi sellist spetsiifilist tema jaoks kirjutatud ei olnud ja  siis oligi nagu koht, kus sai hakata mitte-\emph{mainframe}-de peal kätt proovima. 

\question{Aga mis te nende arvutitega siis tegite? Noorel inimesel on ju see probleem, et kui valid liiga raske ülesande, ei saa hakkama ja on halb ja kui liiga kerge, siis on igav ja ka halb?}

See on üks väga relevantne küsimus, sellepärast et mõnes mõttes meie generatsioonil on arvutitega vedanud. Sel hetkel, kui arvutite juurde sattusime, olid nad sellised, et nagu midagi väga huvitavat ei toimunud. Arvutite peamine köitlus oli potentsiaal, mis neis nagu selgelt sees tuksus. Versus see, et sul on Youtube ja Minecraft ühe kliki kaugusel. Ükskõik, kui palju sa  pingutataksid, midagi  ligilähedastki sa võimeline tegema ei ole. Ja teine asi, et arvutid olid toona aeglased,  umbes mingisugune miljon korda aeglasemad kui praegu. Mistõttu, kui tahtsin midagi ägedat teha, siis sa pidid kohe kiiresti selle hingeelu endale põhjalikult selgeks tegema, et pigistada välja viimane efektiivsusepiisk.

\question{Sa jooksid kohe mingitesse riistvara piirangutesse sisse ja isegi mingi lihtsa asja ekraanil liigutamiseks pidi hoolega mõtlema, et kuidas see ikka täpselt käib!}

Täpselt. Mistõttu suhteliselt kiiresti läksime assembleri\index{Keeled!Assembler} peale. Kõigepealt siis kodukootud Entel-arvutite peal ja siis aasta-paari pärast tulid tekkisid Eestis esimesed IBM PC kloonid. 

\question{Assembleri peale kolimine eeldab siiski, et programmeerimisest on mingi aimdus olemas. Kust see tekkis?}

See tekkiski nende \emph{mainframe}-de peal. Robotron või mis ta oli.

\question{Aga kuidas? Lugesite raamatuid või\ldots?}

Lugesin läbi, mis selle nimi oligi, Programmeerimine Pascalis\sidenote{Tõenäoliselt R. Jürgenson Programmeerimine Pascal-keeles} või midagi sellist. Mul on seal siiamaani mingisugused esimeste programmide väljatrükid  vahel, raamat on raamaturiiulis. Ja siis kirjutan Basicus\index{Keeled!BASIC} programmi. Kirusin, et keel on ikkagi erinev kui Pascal. Basicut ma ei osanud aga Pascalit natuke siis teoreetiliselt oskasin ja kahe peale siis hakkasin avastama. Esimene programm vist oli ruutvõrrandi lahendaja.

\question{See on klassika, ilmselt seetõttu, et teda on praktiliselt vaja. Aga ikkagi, sealt Assemblerisse minna on pikk samm, juba arusaam, et kuskil on Assembler on küsimus. Kust te infot saite? Keegi õpetas? Raamatud? Ajakirjad?}

Jah, seal \emph{mainframe}-de peal ma isegi jäin Basic-usse. Ma tegin isegi oma esimese mängu seal Basicus. Ja kui me kolisime nende \emph{mainframe}-de pealt ära nende kodukootud kaheksabitiste arvutite peale, oli näha, et seal on lihtsam  riistvarale ligi saada, eks. Ja üks asi, mis kohe ahvatlema ja paistma hakkas oli C programmeerimiskeel\index{Keeled!C}.  Mäletan, et samas grupis aeg-ajalt näitas oma nägu selline sell nagu Hannu Krosing\index[ppl]{Krosing, Hannu}, endine Skype kolleg, kes  otseselt samas seltskonnas ei olnud. Ja tema oli selleks hetkeks  kirjutanud Assembleri õpiku, oli mingi selline pisikene brošüür põhimõtteliselt.  Ja ta kas pistis selle mulle pihku või, ma ei tea, igal juhul ma lihtsalt lugesin selle läbi, et \enquote{ohoo, mingi päris huvitav asi}. 

\question{Oot, mis aastal see võis olla?}

See võis olla 1987 äkki? 1988?

\question{87. aastaks oli Hannu kirjutanud Assembleri õpiku!?}

Jah, mingi sellise brošüüri vormis, samizdat\sidenote{\begin{russian}Cамиздат\end{russian}, tõlkes umbes \enquote{iseavaldamine} oli Nõukogude Liidus levinud keelatud või põrandaaluse kirjanduse levitamise viis. Teksti trükiti läbi mitmete kopeerpaberite õhukesele paberile ümber, tulemused levisid käest kätte ning neid paljundati omakorda. Mäletan, et ka minu vanaema tegeles sellise toksimisega, ning lapsena ei mõistnud, miks sellest väga rääkida ei tohi. Kuna kõik klahvidega asjad mind väga huvitasid, nuiasin välja võimaluse ka ise tekste ümber lüüa, miskipärast olen veendunud, et olen aidanud paljundandada mingit budistlikku teksti.} umbes. Oli ikka Hanno õpik? Temaga ma mäletan, ma sellel teemal arutasin,  üsna kindel et tema oli selle autor.

\question{Ja mis te tegite selle Assembleriga?}

Mäletan, üks nagu selliseid korralikumad projekte, võib-olla tegin mingeid väiksemaid asju, ka, oli tekstiredaktor\label{sisu!jaani_tekstiredaktor}. Sattusime Priit Kasesaluga\index[ppl]{Kasesalu, Priit} sellisesse võistlusrežiimi. Mõtlesime, et mida olen hea sellele uuele kodukootud platvormile kirjutada ja leidsime, et tekstiredaktorit ei ole siin korraliku. Hakkasime mõlemad tegema, kõigepealt Basicus\index{Keeled!Basic}. Ja üritasime üksteist üle trumbata, et kellel tuleb parem. Mäletan, et Hannuga\index[ppl]{Krosing, Hannu} rääkisime  mingit tehnikat, kuidas teha seal mingisugust \emph{split buffer} arhitektuuri tekstiredaktoris, et liikumine ja \emph{insert}-imine kiired oleksid.

Ja siism mäletan, tuli koolivaheajal niimoodi, et meil nagu see võistlus jäi pooleli. Aga mina panin nagu edasi, suvi otsa kirjutasin paberi peal tekstiredaktorit, assembleris. Ja kui tagasi tulin, siis Priit polnud muidugi suvega midagi viitsinud teha ja sellega oli võistlus läbi. Siis kirjutasin selle Assembleri paberi pealt arvutisse.

\question{Töötas ka?}

Esimene kord muidugi ei töötanud, eks ole. Aga tööle ma ta igal juhul sain, asi toimimis ja ma vaatasin, et \enquote{oo, see on ikka päris äge}. Kiire, mugav ja palju parem, kui ükskõik milline tekstiredaktor sellel arutil. See andis mulle väga positiivse tagasiside. Peale seda hakkasin mänge kirjutama.

\question{Reflekteerides siit tundub, et sul pidi olema oskus päris suuri ja keerulisi abstraktseid struktuure peas ette kujutada, et sa suudaksid selle koodi kõik paberil asmi valada. Kust see oskus tuli või on see sul kogu aeg olnud või oskad sa natuke selle juuri avada?}

Ma ei tea, mulle tundus see suhteliselt loomulik. Sellised instruktsioonid lihtsalt, et nagu sammude kirjeldus. Et mida sa tahad, et arvuti teeks, eks, ja siis on vaja täpselt üles kirjutada, mida sa tahad. Alguses Basicus sain esimese tagasiside, et kuidas tsükkel käib, ja ühel hetkel assembleris nägin, et see on lihtsalt natuke tülikam, aga teisalt jälle rohkem positiivset tagasisidet pakkuv, kui sa ta käima saad. Käib nagu väga muljetavaldavalt võrreldes Basicuga. 

\question{Kas see tekstiredaktor kuskile jõudis ka või sai lihtsalt oma lõbuks tehtud?}

See seltskond, kes siis seda Entel\index{Arvutid!Entel} arvutit tegi, vormistasid niipea, kui eraettevõtlus seaduslikuks muutus, kooperatiivi ja hakkasid  neid arvuteid tootma ja müüma. Muu hulgas käis selle arvuti juurde tema jaoks toodetud tarkvara, eks. CP/M ja  lisaks see minu tekstiredaktor.

\question{Ehk esimene suurem projekt, mis sa kirjutasid, läks kohe kliendile müüki?}

Jah, ma ei tea, palju seda kasutati, aga kui sa endale kaheksakümnendate lõpus selle Eestis toodetud arvuti ostsid, siis oli seal minu tekstiredaktor kaasas. Selle tõttu me saime siis mingit esimest palka ja tekkisid esimesed mingisugused sissetulekud

\question{See ju tahab tarkvara arenduse mõttes küpsust, et sa mõtled kõik nurgatagused juhtumid läbi ja võibolla kirjutad abiteksti?}

Mõnes mõttes mitte väga optimaalne, aga ma olen märganud, et mul on selline OCD, \emph{obsessive compulsive disorder}. Et kui midagi alustan, ma tahan seal kindlasti lõpule viia, panna iidele punktid peale. Et seetõttu paljud projektid, kus ei aja survet taga, siis läheb nende peale kole palju aega. Alates selle tekstiredaktoriga. Ma tahtsin, et kõik oleks väga ilus, kõik funktsionaalsus oleks olemas ja pusin senikaua, kuni oli. 

\question{Ehk siis kombintasioon täiuslikkuse soovist ja võimekusest see ka ära täita. Mul võib olla soov täiuslik teemant lihvida aga ma lihtsalt ei oska seda teha.}

Ja jällegi, millega mul vedas, oli see, ma astusin arvutite juurde sellisel hetkel kuskohas kogu tarkvara, mis seal arvutites juba oli, oli väga lihtne. Mistõttu see ei olnud selline nii-öelda hingemattev kogemus, et ma olen nii pisikene selle tarkvara kõrval vaid \enquote{ahah, okei, ma saan enam-vähem aru, kas tehtud on, ma teeksin paremini}.

\question{Seda on mitmed öelnud, et oma esimest arvutit nad tundsid põhjani.}

Ka auto-entusiastidel, uunikumide austajatel, on samasugune lugu, neil on ikkagi väga lihtsad riistapuud.

\question{Aga see annab sulle kontrolli tunde, eksole.}

No mul täna luaks Tesla katki. Ja midagi ei ole teha, tuleb Soome saata.

\question{Ma korraks võtaks kinni noist alguses räägitud arvutiajakirjadest. Oskad sa takkajärgi öelda, oled sa mõelnud, mis sind nende juures paelus?}

Asjad, mis seal kohe väga prominentselt silma paistsid, olid mingisuguste arvutimängude reklaamid, mingid Atari reklaamid ja sellised asjad. Noh, nagu ikka, reklaamidel muidugi joonistati natuke ilusamaks, kui päris maailm, aga nad andsid vaate mingisugusesse oma seaduste järgi toimivasse fantaasiamaailma, mis tohutult paelus. Mingisugused ekraanitõmmised, kus mingid tegelased on peal ja ma vaatasin, et \enquote{ahaa, see on vist väga äge asi!}.

\question{Seda on ka räägitud, et see oli nagu täitsa teine maailm, kuhu sai sisse minna.}

Ja veelgi enam, sa said neid maailmu ise luua, see teadmine tekkis mõne aja pärast. Et sa ei ole passiivne tarbija vaid aktiivne looja.

\question{Ja selle aktiivsusega sa kirjutasid selle tekstiredaktori valmis ja sind võeti palgale?}

Ma ei mäleta, kuidas see järjekord täpselt oli, võimalik, et meid võeti palgale seal alguses, kui ma seal niisama katsetasime. Aga võimalik, et see tõesti oli pärast seda, kui me esimesed asjad ära tegime.

\question{See oli keskkooli ajal veel?}

Jah, see oli vist keskkooli alguses, ma arvan. Kaheksakümmend seitse oli keskkooli algus. Kaheksakümmend kuus võis olla see aasta, kus ma üldse sinna sattusin ja siis 87-88 oli see, kus palka hakkasin saama. 

\question{Kas arvutis käimine olümpiaade ei hakanud segama või käis see õppimisega kuidagi lihtsasti kokku?}

Üldse ei seganud. Arvuti värk on nagu ikka suhteliselt kogu aeg põhiline asi elus olnud, ülikooli lõpetasin ka nii-öelda kõrvalhobina ära.  Aga juba siis olid kool ja olümpiaadid arvuti nagu taustal.

\question{Kas juba siis hakkas moodustuma seltskond, mis pärast sai Bluemooniks\index{Bluemoon}?}

Just. Bluemooni süda tegelikult oligi see seltskond, mõned klassikaaslased. 

\question{Kas tollest arvutikooperatiivist eraldusite kohe eraldi ettevõtteks või oli seal vahepeal mingi faas veel?}

See oli niimoodi, et ühel hetkel meil Ahtiga\index[ppl]{Heinla, Ahti} tekkis mõte, et teeks ühe arvutimängu.  Nagu korraliku mängu, mis jookseb PC peale, mitte ainult seal kodukootud arvutite peal. Meil oli mingi eeskuju ka, mille järgi  mängu teha, mis oli Yamaha MSX-ide\index{Arvutid!Yamaha MSX} peal, mis oli palju vähem populaarsem platvorm kui, PC. Oli näha, et PC-d hakkavad juba jõudma sinnamaani, kuskohas saab juba midagi huvitavat teha. Ja väga sellise sügava mulje jätsid toona Ahti matemaatiku võimed. Kuidas ta jagas ära, et \enquote{siin tuleb tangensit kasutada, et  perspektiivi luua}. Mingisuguseid esimesi eksperimente tegime tema vanemate juures Küberis\index{Küber}, kus tal oli arvutitele ligipääs. Ahti hakkas palju varem programmeerima, kui mina. 

Üks tõuge selle mängu tegemiseks oli see, et meil keskkooli viimases klassis (oli vist ikka viimane klass?) tekkis võimalus minna klassiga Rootsi. Esimene välisreis üldse, Aastal 1989 suhteliselt unikaalne võimalus. Käksime sinna läbi Leningradi, siis oli vaja teha mingisuguseid imelikke trikke väljamaale saamiseks. Ja siis onutütre mees ütles, et \enquote{Väljamaal nad õudselt tahavad softi, kirjutage mingisugune lahe soft. Lähed sinna, müüd maha ja mingit probleemi pole!}. Mõtlesin, et \enquote{aga teeme} ja hakkasime tegema. Leidsime, et teeme mängu, teeme korraliku mängu, hakkasime tegema ja muidugi ei jõudnud valmis. Softiga, nagu ma nüüd hiljem tean, tuleb kõik ennustused  umbes piiga läbim korrutada, kulub umbes kolm korda rohkem aega kui alguses arvad. 

Muidugi me seda valmis ei saanud, aga samas oli juba piisavalt suur hoog sees. Et ühel hetkel võtsime sinna juurde korraliku kunstniku, Kaspar Loit\index[ppl]{Loit, Kaspar} ehk BKnows. Ja muusika ning heli-inimese Ott Aloe\index[ppl]{Aaloe, Ott}. Ja tegime mitte ainult ühe mängu vaid täitsa sellise mängude seeria. Millega meil vedas, oli see, et see esimene mäng õnnestus Rootsi müüa hoolimata sellest, et meil Rootsis käisime ajal seda mängu kuhugi kellelgi pakkuda ei olnud isegi, kui ta oleks valmis olnud.

\question{Kuidas? Meil oli ju veel Nõukogude Eesti?}

Selle Sideministeeriumi arvutuskeskuse\index{Sideministeeriumi Info- ja Arvutuskeskus} juhataja Jüri Malsubi\index[ppl]{Malsub, Jüri}  tuttav oli üks sell nimega Tiit Vasli\index[ppl]{Vasli, Tiit}, kellel oli väljamaal suhteid, kes vahendas metalli, mingeid sihukesi asju, ma isegi ei teadnud, mida ta vahendas. Ta oli selline mees, keda oli kaugelt näha, sellepärast et tal oli üks Eesti esimesi mobiiltelefone, mille antenn oli mingi kolm meetrit kõrge. Oli kaugelt näha, et tema läheb seal kuskil tänaval. Ja temal oli Rootsis sidemeid ja ta mõtles, et \enquote{noh, ma vaatan räägin} ja müüski. Tema kaudu müüsime, tema äripartnerid olid huvitatud sellisest eksootilisest asjast nagu raudse eesriide taga toodetud mäng. 

Selle mängumüümise tulemusena me teenisime rohkem, kui mu vanemad kunagi oma elu jooksul teeninud olid. Mis oli vist mingi viis tuhat dollarit. Arvestades muidugi inflatsiooni, mitte reaalväärtuses, vaid nominaalväärtuses. 

Ja kui see mäng nii-öelda müüki läks, tekkis meil tõsine küsimus, see oli siis keskkooli lõpp, ülikooli algus, et kuidas ma nüüd seda administratiivselt korraldame. Oleme selles kooperatiivis  ametlikult tööl, eks, aga on tegelikult näha, et meie plaanid võivad suuremaks kujuneda, kui see kooperatiivi. Rääkisime läbi, mäletan niisugust pingelist läbirääkimist Jüri Malsubiga\index[ppl]{Malsub, Jüri} kuidas seda mängu tulu jagada. Nemad on ühelt poolt panustanud ja meie oleme teiselt poolt panustanud, tahaks nagu oma asja teha. Lõpuks saime meie poolt vaadates väga mõistliku kokkuleppe ja leidsime, et nüüd on aeg vormistada asi mingiks oma ettevõtteks. Mida me siis ka tegime, aastal 1990, ma arvan. 

\question{Kas te mõtlesite nullist välja, et teil on vaja kunstnikku ja muusikut ja kuidas nende töö programmeerimisega siduda või oli teil eeskujusid ka?}

No me olime teisi mänge näinud ja nägime, et nad näevad paremad välja kui see meie katsetus ilma kunstniketa. Ma ei mäleta, kes meid tutvustas BKnowsiga\index[ppl]{Loit, Kaspar}, see võis olla isegi Tanel Hiir\index[ppl]{Hiir, Tanel}, ei mäleta. Kaspari kunstniku-võimed toona jätsid mulle väga sügava mulje. Teda oli raske tööle saada, ma mäletan, tihtipeale pidi selja taga istuma, et \enquote{Tee nüüd}, aga kui ta tööle sai, oli väga äge. 

\question{Ma just mõtlengi seda, et kindlad viisid graafikat kasutada, töödelda, laadida on ju tänaseks välja kujunenud, kas teie mõtlesite need ise välja?}

Üsna, jah, sest, jällegi, need platvormid olid miljon korda aeglasemad, kui praegu. Mistõttu tööriistad olid Turbo Pascal\index{Keeled!Turbo Pascal} ja Borland C\index{Keeled!Turbo C}. Kaspar tegi asju Amigal, seal olid tal oma tööriistad.

\question{Mind on painanud see küsimus, et te ju tegite muusikaprogrammi. Kuidas te sinna valdkonda sattusite, te pole ükski muusikainimene ju, nii palju, kui ma tean?}

Ükskord ülikoolis oli sihuke lahe hetk, kus olin arvutiklassis ja mingid tüübid istusid arvuti taga ja komponeerisid muusikat Soundclubis\index{SoundClub}. Kiibitsesin natuke ja ütlesin, et see on minu programm. Nad ei uskunud. 

Ma ei mäleta, kuidas see algtõuge sattus.  Tänu sellele, et me olime juba mänge teinud, oli meil kindlasti kokkupuude sellega, kuidas teha taustamuusikat. Ja toona, üheksakümnendate alguses, oli väga suur trend trackerid, ehk mingite sämplite baasil muusika kirjutamise väga sellised platoonilised riistapuud. Ja sealt tuli mõte, et  heli on väga hea,  aga kasutajakogemus tundus nagu vähemalt harjumatule silmale väga-väga ebamugav. Mõtlesime, et kuidas kasutada sedasama tehnilist võimekust, aga teha  kasutajaliides, mis oleks äge eriti inimestele, kes ei ole pidevalt muusika kirjutamise juures.

Teemale hakkas järjest huvitama, kuna seal on väga mitmeid nüansse, nagu UI disain, muusika pool asjas (kuigi ükski nendest autoritest ei olnud muusikud), kuidas tehniliselt teha aeglastel arvutitel head heli. Seal ma puutusin esimest korda kokku mingisuguste matemaatiliste teoreemidega, mida ma siis Ahti\index[ppl]{Heinla, Ahti} abil üritasin lahendada. Ja üks huvitav asi oli see, et et kuna me need instrumendid korjasime endale kuskilt BBS-idest kokku, mis olid õudse kvaliteediga, siis mäletan veel, et Ahti kirjutas mingisuguse tarkvara, kus ta tegi Fourier analüüsi, et nad häälde viia. Ükskord Tartus istusin ja häälestasin pille niimoodi, et endal väga suurt muusikaharidust ei olnud, natuke olin pilli õppinud. Aga Fourier analüüsiga sai ikka väga hea häälestuse. 

\question{Selle tarkvaraga on ju tehtud igasugu asju Vennaskona Diskost alates ja ta käib ka mõnest loost läbi. Küll aga ma ei mäleta, et keegi oleks rääkinud selle tarkvara ostmisest?}

Jaa!  Sellest mängust, mille nimi oli SkyRoads\index{Mängud!SkyRoads}. Siiamaani võib-olla vähem kui kord nädalas, kord kuus vähemalt mingi fännikirja, et näete ma olen on selle mängu peal üles kasvanud. On isegi mõned kloonid tehtud,  teda saab tänapäeval veebis mängida. Ja SoundClub oli teine suurem projekt. Meil oli siis juba Bluemoon firmana ja meil oli kaks toodet SkyRoads (mis tegelikult oli järg tollele esimesele Rootsi müüdud mängule, mille nimi oli Kosmonaut\index{Mängud!Kosmonaut}) ja SoundClub. 

Ja nüüd oli küsimus, kuidas neid müüa. Mäletan, et see oli mingisuguste telefonide ja faktidega ja tšekkidega jamamine. Mõlemad olid \emph{shareware}, osalt saadeti lihtsalt ümbrikus sularaha aga tavaliselt saadeti tšekke, mida ma käisin Eesti Maapangas või Rahvapangas lunastamas, selline kogemus. 

Teine asi, mis oli tegelikult väga äge kogemus, oli läbirääkimiste pidamine olukorras, kuskohas teisel poolel ei ole mingit juriidilist motivatsiooni lepinguid järgida. Mistõttu tuli tihtilugu tekitada selline olukord, kuskohas sa nagu lood sellise helge tuleviku, et partnerlusel oleks jumet. Mõnes mõttes selline \emph{iterated prisoner's dilemma}\sidenote{Mänguteoreetiline konstruktsioon, mille abil uuritakse osapoolte koostööstrateegiaid. Selle üks tulemusi on, et (eriti mängu iteratiivses, korduvalt mängitavas ja eelmisi tulemusi \enquote{mäletavas} versioonis) pikas perspektiivis annavad indiviidile parema tulemuse altruistlikud, mitte egoistlikud strateegiad.}, sa pead looma olukorra, kus teisel poolel, hoolimata sellest, et mingit sundmehhanismi ei ole, on lihtsalt huvi olla osa sinu tulevikust ja seeläbi lepinguid järgida.

Alguses oli meil \emph{shareware} aga inimesed hakkasid kirjutama, et tahaks seda mingisugusesse ajakirja panna või tahaks seda kuskil levitada. Mingi väga lahe sell tekiks meil Saksamaale, kes hakkas levitama mitmeid meie asju. Ükskord, aastal 1996, käisin tal lõpuks isegi külas. Samuti üks väga lahe omaette kogemus oli müük Taiwani telefoni ja faksi abil, kuskil Tartu Estiko\index{Estiko} kontoris. 

\question{Kuidas sa Tartusse sattusid?}

Ülikooli läksin.

\question{Mida sa õppima läksid?}

Füüsikat. Nii mina kui Ahti läksime füüsikat õppima aga Ahti kukkus sealt juba teisel aastal välja. Mina punnitasin lõpuni. 

\question{Miks just füüsika? Ahti rääkis, et see ala tundus talle mõnes mõttes kõige puhtam?}

Ma arvan, et ta on vähem puhtam kui arvutiteadus või matemaatika, eks. Kaks põhjust oli füüsika valikuks, Ahti põhjused olid ilmselt korelleeritud. Üks oli see, et ma tundsin, et  arvutites ja matemaatikas olen ma juba piisavalt sees, et füüsika oleks nagu silmaringi laiendav. Ja teine oli see, et füüsika õpetaja, Vilma Kukrus\index[ppl]{Kukrus, Vilma}, oli ikka väga väga äge õpetaja. see tekitas sügava huvi füüsika vastu. Või vähemalt süvaga austuse füüsika vastu. Mul on väga hea meel, et ma füüsika lõpetasin.

\question{See ilmselt mõjutas päris olulisel määral noore inimese maailmapilti ka?}

Absoluutselt. Füüsika on selles mõttes optimaalne teadus, et sa suhtestud reaalse maailmaga niimoodi, et kui matemaatikud võivad minna niivõrd abstraktseks, et nad kaugenevad reaalse maailma piirangutest, siis füüsikas reaalne maailm tõmbab su alati maa peale tagasi. Sõna otseses mõttes, tihtilugu. Ja seetõttu sul tekib intuitiivne arusaam sellest, misasi on teadus. 

\question{Ahtiga\index[ppl]{Heinla, Ahti} oli ka nii, sinu puhul on samasugune muster, seepärast küsin. See, mis ma kuulen ei kõla nagu keskmine \emph{teenager}. See kõlab nagu üsna küpse inimese jutt?}

No praegu ma enam \emph{teenager} ei ole!

\question{Nüüd jah, aga need otsused ja see viis, kuidas toona asju aeti on üsna kaine, arutlev lähenemine. Kust see pärit on?}

Üks oluline asi oli ikkagi, ma arvan, et Ahtile ma võlgnen väga palju tänu. Meil oli super hea koostöö. Priit ka, eks, aga praktiliselt kõiki selliseid probleeme lahendasime tiimiga. Minu minu ja Ahti vahel tekkis väga tihti selline asi, et Ahti on nupukas ja ta mõtleb väga erinevalt, kui mina. Mistõttu  sellised koostöös temaga sündinud otsused olid just nimelt ägedad, kuna nendes oli kaks väga erinevat vaatepunkti, mida see otsus pidi rahuldama.

\question{Ma olen alati tahtnud küsida. Võib olla ruttame natuke ette, aga kui me vaatame kasvõi Bluemooni kodulehekülge, on seal loetletud üksjagu edukaid asju aga ka päris mitu asja, mis ei ole ühel või teisel põhjusel välja tulnud. Inimesed ei suuda mõnikord isegi läbi suure edu tiimina toimima jääda aga teie olete koos läbi nii suure edu kui mitmete ebaõnnestumiste. Kuidas te seda teete?}

Vahemärkusena, mäletan, mõni aasta  tagasi sain Sean Parkeriga\sidenote{Sean Parker\index[ppl]{Parker, Sean} on Napsteri kaasasutaja ja, muu hulgas, Facebooki esimene president. Ta esineb ka tegelaskujuna Facebookist rääkivas filmis, kus kujutatud intriigidest ja tülist tõukub ka eelnev küsimus.} kokku ja meenutasime  Napsteri ja Kazaa aegu, tema tegi Napsterit. Selline lahe kogemus.

Ma arvan ikkagi, et sellised kohatised eduelamused olid piisavad, et nagu läbi suruda ka sellistest mitte õnnestunud projektidest. Ja mõned hetked olid ikkagi jube rasked. Konkreetselt mäletan mingit sellist hetke, kus kogu mänguarendus läks üles-suunas ja siis ühel hetkel lõpetas meie kirjastaja Ameerikas Interactive Magic \emph{milestone}-de maksmise. Raha jaoks oli meil Exceli tabel, kus \emph{runway} oli kogu aeg kirjas, mitmeks kuuks  meil raha on põhimõtteliselt. See \emph{runway} hakkas siis kahanema ja ühel hetkel oli selge, et nad on pankrotis, sealt enam midagi ei tule. Oli tõsine küsimus, et mis nüüd edasi saab. Ja Ahti\index[ppl]{Heinla, Ahti} oli just see, kes ütles, et \enquote{ah, küll me välja ujume!}. Ujusimegi.

\question{Ühel hetkel te läksite mängu kirjutamise juurest ära ja kirjutasite Everyday. Kas see legend, et see käis kuidagi lehekuulutuse kaudu vastab tõele?}

Vastab tõesti, jah. See oligi just see raske hetk, kuskohas ma mõtlesin, et mis me nüüd teeme, midagi ei ole teha

\question{Mis aastal see oli?}

See oli aastal 1999. Ühel reede hommikul vaatasin, ise olin Tartus, oli lehekuulutus, et pakutakse inimestele mingisugust ulmelist palka\sidenote{Loo teist vaadet vaata Tarvi loost leheküljelt \pageref{sisu:everyday}.}. Everyday portaal oli arendushädas ja Tele2 oli börsile lubanud, et kohe sihuke asi tuleb välja. Nad olid juba mingi aasta või paar arendanud ja olid välja tulekust kaugel. Kuulutuses oli  pikk nimekiri  nõuetest, mida arendajad peaksid olema osanud. Pikk nimekiri asjadest, millest ma elu sees kuulnud ei olnud. IMAP ja POP3 ja PHP ja SQL ja mingid niisugused asjad. Tähtaeg oli esmaspäev, oli reede, mina olin Tartus ja teised olid Tallinnas. Mäletan, et helistasin Ahtile\index[ppl]{Heinla, Ahti},  rääkisime läbi, mõtlesime, et proovime, vaatame, mis juhtub. Tegime kohe  nädalavahetuse plaani ja põhimõtteliselt esmaspäevaks oli valmis prototüüp sellest portaalist, mida nad tahtsid. Kui me esmaspäeval läksime intervjuule siis oli meil dilemma, et kas me ütleme, et see oli ühe nädalavahetusega kirjutatud või mitte. Ta nägi väga hea välja, kuna meil oli palgal arvutimängudega karastunud kunstnik, näiteks. Ma arvan, et see nägi  parem välja, kui lõpptoode. Ja toimis täitsa, võisid  sisse logida, erinevaid paneele ringi lohistada, võisid emaili kirjutada võisid uudiseid lugeda, ilmateateid, mida iganes. 

\question{See tähendab ju, et tolle nädalavahetusega pidi sigima päris hea arusaam sellest, kuidas HTML ja brauseri renderdus ja muu selline töötab?}

Andmebaasid. Mäletan, et Priidule\index[ppl]{Kasesalu, Priit} jäi andmebaasidega tegelemine. Ta sattus hätta, ei saanud  loogikast aru. Ja ma mäletan, et ta võttis telefoni, helistas mingile andmebaasieksperdile, kahjuks ei mäleta, kes see võis olla. Oli laupäeva hommik. Kuulsin seda kõnet kõrvalt, et \enquote{kuule, mul on üks niisugune kogemus, on sul nagu hetk aega? Aa okei, okei.} ja pani toru ära. Ei olnud aega. Hea küll, tagasi uurima, kuidas SQL  käib uuris välja. Sai tehtud.

\question{Kõlab üsna ulmelisena, seal peab ju olema mingi meetod taga, kuidas seda teadmist omandada?}

See oli väga äge kogemus, jah. Põhimõtteliselt ega tegelikult tehnoloogiad toona ei olnud super keerulised, nad olid meile lihtsalt võõrad. Kuna meil oli tiim tõesti äge tooma ja saime tööjaotuse tehtud. Igaüks pidi mingi kindla aspekti välja uurima. Magasime natuke, mitte eriti. 48 tundi tundi tööd.

\question{See tähendab, et te pidite kuidagimoodi oma tööd ka koordineerima, kes seda kampa teil juhtis?}

No mina olin nii-öelda ametlik juht. Samas see tiim töötas ise ka päris hästi. Välja arvatud, jah, võib-olla kunsti pool, mis põhjustas võib olla kõige rohkem meelehärmi, et kuidas Kasparilt\index[ppl]{Loit, Kaspar} saada lubatud asjad kätte. Kunstniku asi, rohkem boheemlane, kui teised.

\question{Arvutades leiame, et kui kuulutus oli 1999 ja keskkool üheksakümnendate alguses, siis te kogu kümnendi kirjutasite mänge?}

Jah, päris mitmeid mänge tegime, kutsusime ennast Eesti mängutööstuseks. 

\question{Kui suur see tiim oli?}

1999. aastaks ega ta väga palju suuremaks ei läinud. \emph{Core} tiim oli siis mina, Priit\index[ppl]{Kasesalu, Priit}, Ahti\index[ppl]{Heinla, Ahti}. Artur Vill\index[ppl]{Vill, Artur}, kes oli 3D-kunstnik ja kes muide on teinud sellise filmi nagu Happy Feet mingid \emph{landscape}-d ja maastikud. Ta kolis pärast Bluemooni kokku kukkumist Austraaliasse ja seal tõusis tähelennuna, väga kõva vend 3D modelleerimises ja kunstis.

Ja kõrvalt Kaspar\index[ppl]{Loit, Kaspar} tegi kunsti, Ott\index[ppl]{Aaloe, Ott} ja Glen Pilvre\index[ppl]{Pilvre, Glen} tegid muusikat. Juhan Soomets\index[ppl]{Soomets, Juhan} tegi ka nagu poole kohaga 3D-graafikat ja vist oligi kõik. Kui bluemoon.ee lehele minna, siis see tiim on seal siiamaani üleval.

\question{Isegi toonase tehnoloogia lihtsuse juures pidi teil siis ju selle väikse tiimi peale tööd palju olema?}

Tööd oli päris kõvasti jah. Põnev oli ka muidugi.

\question{Mis see põnevus oli? Kui ühe mängu valmis olite teinud, kas siis igavaks ei läinud?}

Mängude tegemine ongi selles mõttes äge, et see on nagu niivõrd palju rahuldust pakkuv, kui mingi asi tööle läheb. Kirjutad mingisugust andmeanalüüsi. Kui asi tööle läheb, tuleb ekraanile õige number. Aga kui mängus asi tööle läheb, tuleb vägev plahvatus, näiteks. Või tuleb mingid väga, sellist, rahuldust pakkuvad stseenid,  efektid või lood või midagi sellist. 

\question{Nojah, vaade mingisse teise maailma, millest sa oled nüüd järgmise tüki loonud.}

Just, jah. Ja nüüd sa saad seal testimise käigus  ringi käia ja mingisugused kohati väga vapustavaid vaateid, sündmuseid, mis on toimunud\ldots

\question{Ühte asja ma tahtsin veel küsida. Kui sa rääkisid, et sa tegid üksi tekstiredaktori ja seal pidi kõik asjad ilusti reas olema, sellest ma saan aru. Aga kui meeskonnana softi kirjutada, siis see vajab ju \emph{software engineering}-u protsesse ka, kust teil need tulid?}

Üheksakümnendatel olid lihtsalt zip-failid ja \emph{backup directory}-d. Versoneerimist või selliseid  asju me üldse ei teinud. 

\question{Aga kuidas te siis tagasite, et see kupatus teil kokku ei kukkunud?}

Me olime väga ettevaatlikud! Üks põnts, mis meil juhtus, oli see, et meil murti kontorisse sisse ja varastati arvutid ära. Sealt läkski mingisuguse SkyRoadsi või mingi asja mingi versioon. Meil olid diskide peal \emph{backup}-id ja midagi jäi alles, aga mingisugused asjad bluemooni ajaloost läksid lõplikult kaduma. 

\question{Siiski, kas te oma töökorralduse mõtlesite lihtsalt jooksu pealt välja?}

Istusime telefoni otsas, põhimõtteliselt. Mina olin Tartus, Ahti\index[ppl]{Heinla, Ahti} kolis ühel hetkel Tallinna  tagasi. Istusime telefoni otsas, koordineerimine käis ka meili teel. Eks meil tekkis spetsialiseerumine ka. Mina manageerisin tiimi, kunstnikke, kirjutasin mingisuguseid tarkvaralõike. Priit\index[ppl]{Kasesalu, Priit} spetsialiseerus operatsioonisüsteemi asjadele, graafikale, mingi Windows API ja sihukesed asjad. Ahti\index[ppl]{Heinla, Ahti} tegi sellest rohkem teadusmahukat asja, kuskohas oli vaja mingit AI-moodi asja või siis mingisugust matemaatikat.  Kui vaja, tal oli võtta. Et \enquote{Ahaa, ma tean, selle jaoks on siin sellel leheküljel Knuthi Art of Computer Programming-us\sidenote{Knuth, Donald E. Art of computer programming (TAOCP). Tuntud ka kui Knuthi Piibel. Tegu on monumentaalse teosega, mille seitse köidet pidid katma kogu teadaoleva arvutiteaduse. Praeguseks on ilmunud kolm köidet ja esimene osa neljandast. Kuna teise köite küljenduse kvaliteet lugupeetud prorfessorit ei rahuldanud, lõi ta oma raamatu ilusaks tegemiseks süsteemi TeX, mille abil kirjutab oma artikleid kogu teadusmaailm ja mille abil on kujundatud ka käesolev tekst.} on õige algoritm, teeme selle!}. 

\question{Tundub siis, et kuna meeskond töötas tiimina hästi, siis see lahendas ka üksiti ära \emph{software engineering}-u probleemid. Ei tekkinud mingeid merge konflikte ega probleeme, sest te töötasite inimlikult nii hästi koos.} 

Tiim oli väike ka, räägime kolmest programmeerijast. 

\question{Kui sa vaatad tagasi enda kui programmeerija peale üheksakümnendatel, oskad sa kuidagi kirjeldada enda arengut?}

Põhiasi, mis, ma arvan, domineeris seda arengut, oli see, et arvutid läksid iga kahe aasta tagant kaks korda kiiremaks. Mistõttu oli vaja kogu aeg hoolitseda selle eest, et sa ajale jalgu ei jää. Lõpuks me ikkagi jäime, mängutööstuse aeg sai läbi. Kümme aastat, siis see siis on, arvutid läksid mingi kolmkümmend korda kiiremaks sella aja jooksul. Ja  võimalused: heli läks rikkalikumaks,  mälu läks suuremaks, graafika ägedamaks, võrgundus tuli juurde. Kogu aeg tuli ennast hoida aja tasemel. 

\question{Aga programmeerimise kunsti mõttes? Mitte see, et kas ma tean üht või teist API-t vaid kas ma olen programmeerijana täna parem, kui eile?}

See on huvitav küsimus, ma  väga palju ei ole selle peale mõelnud. Kindlasti  kogemus õpetas. Ma ei oska praegu  tagantjärgi seda kuidagi kompresseerida. Ma tean, et programmeerijana arengud on mul pigem nagu hiljem olnud, võib-olla ma aga mäletan hilisemaid arenguid paremini. See, kus ma kolisin rohkem funktsionaalse programmeerimise peale, see asi juhtus peale Skype'i. Kuni Skype'i lõpuni ma ikka kirjutasin oma vanade tööriistadega.

\question{Pärast Everyday intervjuud, kui kiiresti te tolle production versiooni välja lasite?}

Ma hästi ei mäleta, aga see võis olla nii, et  suvi otsa kirjutasime,  minu arust kuskil sügisel või umbes nii tuli välja. Mingi esimese versiooni jaoks võis kolm-neli kuud minna.

\question{Sellesama väikse tiimiga?}

Põhimõtteliselt küll. Kuigi nüüd oli nii, et seal oli juures mingeid rootslaste tehtud asju ja see väike tiim oli osa mingist palju suuremast organisatsioonist. Mistõttu läks ka oluliselt aeglasemaks asi. Mingeid asju oli vaja rootsi keele tõlkida, ma mäletan ükskord öösel sain Niklaselt\index[ppl]{Zennström, Niklas}\sidenote{Niklas Zennström, hilisem Skype asutaja.}, meili rootsikeelsete vastetega inglisekeelsetele fraasidele ja all oli \enquote{midnight translation services by Niklas Zennström}.

\question{Sellised teenused siis. Mis Niklas tegi seal projektis?}

Niklas oli everyday.com-i CEO. 

\question{Niklas oli siis see mees, kes ei suutnud kogu oma rootslaste tiimiga tarnida?}

Seda ma ei tea täpselt, kuidas see atributsioon seal täpselt  oli, aga Niklas oli põhimõttelist see, kelle lõplik otsus oli see, et Eestist arendaja otsida. Linnar Viik\index[ppl]{Viik, Linnar} vist oli see, kes oli pakkunud, et võtaks Eestis programmeerijaid ja Niklas oli see, kes otsuse langetas, et Bluemooni seltskonda kaasata.

\question{Kuidas tiimi skaleerumine tundus? Kui te olite kõik see aeg kirjutanud kompaktses kõgproffide tiimis keerulist softi, siis veebiarenduses on rõhk ju mujal?}

Ma hästi ei mäleta, et seal mingisuguseid olulisi probleeme oleks olnud peale selle, et kohe tuli  kommunikatsiooni ülesanne juurde. Nagu ikka, kui kaks programmeerijate tiimi, siis esimene reaktsioon kõigil on, et \enquote{see on teise tiimi bugi}. Neid asju  tekkis kohe kõvasti, neid asju kindlasti oli. Aga ma ei mäleta, et oleks mingi tohutu külma vee kaela saamine olnud. Saime hakkama küll. 

\question{See kõik toob meid 1999. aastasse ja seega ka otsapidi väljapoole meie ajahorisonti, milleks on kaheksa- ja üheksakümnendad. Mitte, et pärast ei oleks igasugu põnevaid asju veel juhtunud.}

Enamus asju juhtus hiljem!

\question{Aga inimeseks said sa ju varem. Mis sa praegu teed?}

Peamine ja kõige olulisem tegevus on hoolitsemine selle eest, et juhul kui inimajastu peaks mõne aastakümne (loodetavasti mitte mõne aasta) jooksul lõppema, et inimesed siia planeedile alles jääksid.

\question{Ära pane pahaks, aga ma hästi ei näe mõtteliini ekraani peale plahvatuse joonistamisest selle teemani. Palun selgita!}

Sinna mahub üks kuni kaks aastakümmet veel, ehk see, millest me ei ole rääkinud. 

\question{Sa oled lihtsalt jõudnud selleni, et see on sinu jaoks oluline probleem?}

Jah, selleni jõudsin ma aastal 2008 või midagi sihukest, kui Skype'is juba hoog hakkas raugema, sealt enam väga väljundit ei olnud. Sattusin rääkima inimestega, kellega me viimased kümme aastat olen ehitanud sellist \emph{community}-t, kes üritavad teha ära AI-uurijate kodutöö. Ehk siis teha asju, mida on vaja selle jaoks, et AI-ga tulevik oleks inimestele soodne, aga millega AI-arendajad ise ei ole näidanud mingit motivatsiooni tegeleda peale selle abstraktse motivatsiooni, et nad on ka inimesed.

Üks võimalus probleemi kirjeldada on see, et meil on fundamentaalne \emph{trade-off}, ma isegi tea, kuidas seda eesti keeles öelda. Et sa ei saa nagu kahte asja korraga. Super kompetentset süsteemi ja sellist süsteem, mille üle sul on täielik kontroll. See ei ole isegi arvutite spetsiifiline probleem, inimjuhtidega on sama probleem: mida rohkem ta delegeerib, seda kompetentsemaks muutub süsteem või suuremaks kasvab organisatsiooni võimekus. Aga tema isiklik kontroll selle üle, mis toimub, väheneb. See on fundamentaalne printsiip. Ja mida inimekond praegu teeb, iga päevaga järjest rohkem, ta delegeerib oma otsuseid masinatele. Mistõttu sellise delegatsiooniga tegelikult väheneb inimeste kontroll tuleviku üle. Võib juba öelda, et praegu on inimeste kontroll tuleviku üle väiksem, kui see näiteks oli mingi viiskümmend aastat tagasi. Ja see tendents tõenäoliselt jätkub. Nüüd on küsimus see, et kuidas me siiski säilitaksime kontrolli mingisuguste oluliste aspektide üle. Näiteks atmosfääri koostis, mis on meile oluline. Temperatuur, mis tundub juba praegu, on keeruline. Räägime siin veebruarikuus, väljas on kolm kraadi sooja, sajab vihma. Juba inimestel on raske raske keskkonna üle kontrolli säilitada. Lisame siia entusiastliku delegatsiooni arvutitele, kellel keskkonnast on täiesti ükskõik! Sellepärast me saadamegi roboteid radioaktiivsetesse aladesse või kosmosesse, et neid keskkond ei huvita. Probleem on selles, et AI arendajatel on motivatsioon aretada just nimelt delegatsiooni poolt, et delegatsioon oleks võimalikult lai ja tulemus oleks mingi meetrika järgi võimalikult kompetentne. Ja palju vähem on motivatsiooni selle jaoks, et mõelda selle peale, et kuidas kogemata mitte delegeerida selliseid asju, mida meie elus olekuks on vaja.

\question{Ma südamest loodan, et sul tuleb välja, sest muidu on pahasti!}

Ma tihtilugu ütlen inimestele, et \enquote{wish me luck, you are going to need it!}
