\index[ppl]{Talvik, Taavi}

\question{Kuidas sa arvutite juurde jõudsid?}

Arvutite juurde jõudmine iseenesest on väga lihtne. Kodus sattusid olema paar 
põnevat raamatut. Isegi enam ei mäleta, mis see täpselt oli, kas see oli Ustus 
Aguri \enquote{Abakusest raalini}\sidenote{Sarja \enquote{Mosaiik} 1980. aastal 
ilmunud 28. teose autorid olid siiski Rafail Guter ja Juri Poljunov, tõlkijaks 
Madis Järv.} või Norbert Wiener'i \enquote{Küberneetika}\sidenote{Norbert 
Wiener. Küberneetika ehk juhtimine ja side loomas ning masinas. Eesti Riiklik 
Kirjastus, 1961.}. Igal juhul mingi niisugune raamat oli, tundus jube põnev. Ja 
 kuna esivanemad olid Tartu Ülikooli\index{Tartu Ülikool} juures keemikutena 
ametis ja olid kuulujutud, et ülikoolis mõni arvuti ikka on, siis hakkasin 
neile kohe pinda käima, et \enquote{kuulge, ma tahaks näha, missugune see 
arvuti päriselt välja näeb}.

\question{Siit me kohe järeldame, et sa oled Tartust?}

Jah, ma olen Tartust. Selles mõttes Tartu on jumala okei, sihuke  väike 
linnakene Elva lähedal ja lapsepõlves mulle seal väga meeldis. 

\question{Kui vana sa olid kui sa oma vanematele arvuti asjus pinda käima 
hakkasid?}

Ma arvan, et see oli kuskil, ütleme niimoodi, üheksas või kümnes klass, pigem 
üheksas. Ja tõepoolest neil seal ülikoolis arvutid olid, oli isegi väljamaa 
omi. See oli aasta umbes 1985. Sel ajal oli  väljamaa arvuti  suht  haruldus, 
aga kuna nad tegid mingisuguseid imelikke elliptiliste kilede mõõtmisi, siis 
selle kilede mõõtmise masinaga oli kogemata kaasa ostetud  arvuti, mille nimi 
oli Hewlett-Packard 85\index{HP-85}. 

\question{See oli laua-arvuti?}

See oli lauaarvuti, kus oli sees pisikene, ma ei tea, viietolline\sidenote{Eri 
allikate andmetel tuli HP-85 kas viie- või kuuetollise CRT ekraaniga.}, ekraan, 
klaviatuur, kassetimakk ja termoprinter ning taga oli hunnik juhtmeid, mis 
ühendasid teda mõõtmisseadmetega.

\question{Niisugune mudel ei ole küll kellegi jutust läbi käinud, kõlab täitsa 
eksootiliselt!}

See on iseenesest väga eksootiline mudel. Tal oli mingisugune oma Hewlett 
Packardi protsessor\sidenote{Koodnimi \enquote{Capricorn}, protsessor toimis 
taktsagedusel 0.6MhZ (!).}, mis  omas ajas oli isegi täitsa innovaatiline ja 
tore. Kuigi protsessor protsessoriks, millega see välja paistis, oli tavaline 
BASIC\index{BASIC} ja see, et ekraani peal sai jutte joonistada. Ja kui 
ise jutte joonistada ei osanud, siis sai pingpongi või kosmonautide maandumist 
mängida. 

\question{Ja sind lasti kohe sinna juurde, et \enquote{näe, poiss, näpi!}?}

Jah. Täiesti niimoodi. Eks vanemad inimesed, kui ma õieti mäletan, siis tema 
nimi oli Zirk, õpetasid ka, et kaua sa siin mängid, et proovi kokku liita arve 
ühest kümneni või midagi sihukest ja sealt need asjad  pihta hakkasid.

\question{Teil koolis ei olnud arvuteid? Mis koolis sa üldse käisid?}

Tartu 10. Keskkool\index{Tartu 10. Keskkool}, mis on tänapäeval  Mart 
Reiniku Gümnaasium\index{Mart Reiniku Gümnaasium|see{Tartu 10. 
Keskkool}}. Koolis ei olnud see aeg veel mitte midagi,  täitsa tühi maa. 
Tõenäoliselt samal ajal Nõos midagi oli, aga Nõo on Tartust nii kaugel ja 
selleks peab ikkagi tutvusi olema, et sinna keegi kutsuks.

\question{Kui ma mõtlen sellise üheksanda ja kümnenda klassi peale, siis seal 
kipuvad ju igasugused muud põnevad hobid olema Wieneri ja Aguri lugemise 
asemel. Miks sa lugesid neid?}

Tore oli, huvitav oli ja võib-olla vanemad sokutasid ka midagi, et 
\enquote{loe, poiss, järsku saad targemaks} või midagi nihukest. Eks  
tagantjärgi tarkusena enam ei mäleta, mis see täpselt see ajend oli.

\question{Ma seepärast küsin, et kas sul oligi populaarteaduslike asjade huvi 
või ulmehuvi või\ldots?}

Oli populaarteaduslike asjade huvi ja oli ulmehuvi. Kuna nagu nupp 
reaalteadustes jagas alates igasugustest  olümpiaadidest ja asjadest, siis see 
tundus nagu naturaalne. 

\question{Kuidas sul see reaalteaduste jagamine nagu esile kerkis? Kas kohe 
esimesest klassist alates tundsid ennast nende osas mugavalt, tegeles keegi su 
arendamisega spetsiifiliselt?} 

Tundsin suht mugavalt tänu sellele, et isa-ema olid ülikoolis õppejõud, 
aeg-ajalt nad ikka keemikutega midagi  rääkisid ja nende käest on ju alati 
võimalik küsida, kui ma kuskil füüsikas, keemias või matemaatikas hätta jäin. 
Ja kui sa hädast üle saad, keegi sind hädast üle aitab, siis  endal tekib mugav 
tunne ka ja ei saa nagu vastu näppe.

\question{Nojah, hea turvaline toimetada! Aga miks sa siis keemiku teed ei 
läinud?}

Arvutid olid põnevamad. Peale seda, kui ükskord näpp oli antud, põnevus järjest 
ja järjest lihtsalt kasvas.

\question{Too HP-85 ei saanud ju väga kauaks põnevaks jääda?}

Eks seal olid ka omad asjad. Sai natukene mingeid trips-traps-trulli-laadseid 
mänge kirjutada ja sai selle esimese hea edukogemuse kätte. Sealt edasi sai 
kuhugi järgmistesse kohtadesse, kus olid  veidi ägedamad ja võimsamad arvutid.

Keemiahoonele\index{Tartu Ülikool!Keemiahoone} lisaks oli Tartus olemas ka 
füüsikahoone\index{Tartu Ülikool!Füüsikahoone}, kus oli see selline inimene 
nagu Alo Raidaru\index[ppl]{Raidaru, Alo}, kellel olid aadressil Tähe 4 kuskilt 
saadud PC-laadsed arvutid. Kuskil seal keldrikorrusel ta tegutses ja 
PC-laadsete arvutitega sai juba teha väga palju rohkemat, kui selle väikse 
õnnetu HP-85-ga\index{HP-85}. Lõpptulemusena võeti mind umbes 
kümnendast klassist alates füüsikahoones nii-öelda laborandina tööle. 
Tööülesandeks oli üht või teist või kolmandat või neljandat programmeerida. 
Kümnenda klassi poisina.

\question{Ohoh. Kust sul see programmeerimisoskus tuli?}

Ma ei tea, tuli järjest. Kõik aitasid kõrvalt ja õpetasid ja see kuidagi 
naturaalselt kasvas.

Ma loodan, et ma tegin seal isegi midagi kasulikku. Ma sain selle eest palka ja 
sain palka ikka täitsa kõvasti. Laborandi palk oli mingisugune viiskümmend 
rubla kuus, mis tekitas kümnenda klassi poisile sihukese Kröösuse tunde. Eks 
see laborandi palk natuke toetas neid huvisid ja värke ja tulemus oli see, et 
enamasti peale koolitundide lõppu oli minek mitte koju, vaid füüsikahoonesse.

\question{Mind ei jäta rahule see, et sellise kiire loomuliku arengu jaoks peab 
olema mingi huvi?}

Absoluutselt!

\question{Mis selle asja sinu jaoks põnevaks tegi?}

Põnevaks tegi asja see, et kui sa arvutile mingisuguse programmilaadse asja 
selgeks teed, siis ta teebki midagi, mida sa arvasid, et ta võiks teha. 
Tihtipeale ei tee, aga väga tihti ta tegi ka ja see oli jube kihvt, kui midagi 
juhtus. Mingi asi allus  minu korraldusele, täiesti unikaalne situatsioon 
maailmas!

\question{Oskad sa mõnda näidet tuua, mida sa laborandina progesid?}

Üks asi, mis kindlasti meelde tuleb, on antiviirus. 

\question{Antiviirus aastal 80 millegagi?!}

1986 umbes. Oli sihuke avastushetk, et maailm hakkas vaikselt lahti minema ja 
vaikselt ilmusid Eestisse ka viirused. Tekkis see jama, et viirus jõudis sinna 
meie juurde ka, sellest oli vaja kuidagi lahti saada, kuna arvutid hakkasid 
imelikult käituma. Viiruse nimi oli, kui ma õigesti mäletan, Yankee 
Doodle\sidenote{Internet ütleb küll, et Yankee Doodle viirus avastati 1989. 
aastal ja tähti kukutav Cascade 1988. aastal.} mis tegi  piikse ja ekraani peal 
vist hakkasid tähed kukkuma. Sai uuritud, kuidas see käitub ja tehtud pisike 
programmikene, et sellest lahti saada. Oli täitsa võimalik. Aga see on 
niisugune lihtsalt tore mälestus. 

Põhiline, mida seal Alo Raidaru\index[ppl]{Raidaru, Alo} laboris tehti, oli 
elektroonika füüsikutele. Ehk mingisuguseid lisasid  mõõteseadmetele, 
katseeksperimentidele ja nii edasi. Seoses sellega nad tegid ise trükkplaate. 
Tükkplaatide tegemiseks olid esimesed SmartCadi laadsed programmid, millega 
õnnestus joonistada elektroonikaskeem, joonistada trükkplaat ja see ka välja 
printida. Alo ehitas arvuti külge freespingi juhtimise \emph{interface}, mis 
freesis selle trükkplaadi välja. Aga lisaks välja freesitud trükkplaadi 
radadele oli vaja, et puuritaks ka läbiviigu augud. Ja näiteks see läbiviigu 
aukude puurimise programm oli see, mis usaldati mulle.

\question{Kui ma nüüd moodsasse terminoloogiasse tõlgin, tegelesid sa kohe 
esimese hooga IoT-ga!}

Seda võib tänapäeval IoT-ks nimetada, aga tegelikult oli see trükkplaatidesse 
aukude puurimine. 

\question{No nimetame siis robootikaks!}

Nimetame asja ikkagi õigete nimedega. Tegelesin puurpingi puuri õigele kohale 
viimisega ja siis käsu andmisega, et mine alla ja tule üles tagasi.

\question{Kuna tõenäoliselt mingisuguseid teeke või draivereid ei olnud, siis 
sinu programm käis riistvarani välja?}

Põhimõtteliselt küll, sellest samast CAD-i programmist sai aukude koordinaadid 
ja nende koordinaatide peale lihtsalt tuli augud puurida. Sinna  vahepeale sai 
tehtud mingi puuride liigutamise keel, et \enquote{astu sada sammu siiapoole, 
mine alla, tule üles, astu sada sammu sinnapoole}.

\question{Selle keele mõtlesid ka sina välja?}

No eks need vanemad inimesed kõrvalt ikka aitasid. Nii kui kinni jooksid, tuli 
keegi ikkagi appi. Hästi turvaline keskkond kasvada.

\question{Teised on rääkinud, et nad said üsna varakult teiste omasugustega 
ninapidi kokku, vahetati infot ja tekkis kogukonna moodi asi. Sul seda ei 
olnud?}

Kooli ajal ei olnud aga pärast kooli ülikooli astudes see kogukonna tunne 
tekkis üsna kohe, esimesel kursusel.

\question{Keskkoolis lanoranditöö õppimist segama ei hakanud?}

Ei, otseselt ei hakanud. Kuna nupp natukene lõikas, siis võis mõne koha pealt 
üle nurga lasta. Koolis lõputult pingutada ei olnud vaja, võib-olla seal eesti 
keele kontrolltööd läksid kehvemaks, aga, ütleme, üldtase sinna nelja juurde 
jäi. Kooli lõpetamisel oli tunnistus umbes selline, et kõik olid neljad, välja 
arvatud üks viis ja üks kolm. 

\question{Mis aine see kolm oli?}

Enam ei mäleta, lihtsalt ei mäleta. Võimalik, et oli vene keel, aga ma hetkel 
enam ei mäleta.

\question{Mis ülikooli sa läksid?}

Tartu Ülikool ja füüsika, füüsikateaduskond.

\question{Sest sul seal oli laborandina juba käsi sees?}

Laborandina oli käsi sees ja tegelikult füüsika kui nähtus huvitas ka, 
tegelikult oluliselt rohkem kui matemaatika. Füüsikas oli nagu keerukusaste 
väiksem selles mõttes, et matemaatikud läksid mingi teise või seitsmenda 
tuletiseni välja, samas kui füüsikud ütlesid, et \enquote{teine tuletis on nii 
ebaoluline juba, et seda efekti sellel kursusel ei aruta, jääb kolmanda kursuse 
materjaliks} ja see mulle jumalast sobis.

\question{Väga huvitav. Ma just mõtleksin vastupidi, et füüsikas on päris 
maailm kogu oma keerukuse ja ebakorrapärasusega?}

Ei, vastupidi. On mingisugused suhteliselt lihtsad rusikareeglid. Kui nendest 
suhteliselt lihtsatest rusikareeglitest aru saada, siis peenhäälestamine  tuleb 
peale ja, nagu ma ütlesin, see tuleb nagu järgmise kursuse materjalist,  
esialgu võid kõrvale jätta.

\question{Nii võttes küll, jah. Kas sul mingi spetsialiseerumine ka tekkis?}

Vot spetsialiseerumisega läks natukene natukene sandisti, sellepärast et kohe 
tuli peale Vene sõjavägi ja peale Vene sõjaväge ma küll  jätkasin füüsikas kaks 
pool aastat, aga siis tuli ka muu elu kõrvale ja õppimisvaimustus vaikselt, 
kuidas seda viisakalt öelda, hajus.

\question{Kus sa teenisid?}

Valgevenes, selline koht nagu Borissov 13,  on niisugune super koht. 

Eks ta oli mõnes mõttes sihuke aja raiskamine, teisalt sa nägid maailma, et kui 
palju erinevaid inimesi tegelikult olemas on.

\question{Mis arvuti-inimesele on ilmselt üsna silmiavav!}

Ilmselt jah, et kui paljud meist 1988. aastal, kui ma sõjaväe läheksin, 
tegelikult reisinud olid? Võib-olla Nõukogude liidu piires siin-seal kuskil 
käinud, aga ega reisimine ei olnud niisugune teema, mida kõik on teinud.  Ja 
see uute inimeste nägemine oli tegelikult sõjaväest päris kasulik kogemus, 
tagantjärgi vaadates.

\question{Sõjaväest tagasi tulid sa millal?}

Tagasi tulin aastal 1989, mul õnnestus Vene sõjaväest pääseda ühe aastaga, kuna 
Gorbatšov ütles, et üliõpilased on meie sotsialistliku riigi tulevik ja mingu 
parem õppigu ülikoolis  edasi, mitte ärgu jooksku püssiga ringi. 

\question{Mispeale tulevik tuli Tartu Ülikooli füüsikat õppima!}

Tulevik tuli, jah, Tartu Ülikooli edasi füüsikat õppima ja proovis 
spetsialiseeruda astronoomia peale.

\question{Miks just astronoomia?}

See on jälle niisugune juhus. Tõenäoliselt sa tead sihukest ulmekirjanikku nagu 
Isaac Asimov ja suure tõenäosusega oled kuulnud, et lisaks sellele, et ta oli 
ulmekirjalik, oli ta ka jube hea teaduse populariseerija. Koduses 
raamaturiiulis oli raamat, mille pealkiri oli 
\enquote{Universum}\sidenote{Isaac Asimov. The Universe: From Flat Earth to 
Quasar, 1966. Vene keeles \begin{russian}Гуров П. С.\end{russian} tõlkes 
1969.}. See oli, tõsi küll, venekeelne, \begin{russian}Вселенная\end{russian}, 
aga ta kirjeldas niivõrd fantastiliselt seda, kuidas universum toimib, et see 
jäi kuklas kripeldama. Et järsku peaks seda teemat edasi uurima,  tundus nii 
kihvt olema. Ja siis ma proovisin astronoomiale spetsialiseeruda.

\question{Mis tähendab \enquote{proovisid}? Kas ei tulnud välja?}

Otseselt  ei tulnud välja, kuna kooli-eelsest tööelust kasvas välja järgmine 
tööelu, mis hakkas natuke õppimist segama. Seesama Alo 
Raidaru\index[ppl]{Raidaru, Alo} sokutas mind tööle ajalehte 
Edasi\index{Edasi|see{Postimees}}.

\question{Ohoh! Edasis ei olnud ju vaja auke puurida, ma usun?}

Auke puurida ei olnud vaja, aga kaheksakümne üheksas aasta oli juba see aasta, 
kus reaalselt tekkisid ka ettevõtetesse esimesed arvutid. Ja tekkis esimene 
\emph{desktop publishing}. Ja kuna üks Alo hobidest oli ülikooli teatmiku välja 
andmine, siis tema käest küsiti nõu, et \enquote{me saime Edasis ühe arvuti, et 
kas seda saaks kasutada kuidagi ajalehe väljaandmise abiks}. Mind sokutati 
sinna, et \enquote{kuule, Taavi, mine aita neid}. Ja siis ma aitasingi neid üks 
neli-viis aastat.

\question{See abi pidi olema puhtalt ainult tolle \emph{publishing} programmi 
käima ajamine või\ldots?}

Ei, mitte ainult  programmi käima ajamine. Reaalsuses on mingisugused töö 
rutiinid ja kui need lähevad lihtsamaks, käivad nad kiiremini. Sulle nüüd 
omakorda küsimus, et mis sa pakud, mis oli esimene asi Edasis\index{Edasi}, 
mida aastal 1989 arvutiga automatiseeriti?

\question{Eesti keele spellerit ju veel ei olnud\ldots}

Eesti keele speller oli ka juba olemas, aga see selleks. Mis võiks olla see 
teema, mida automatiseeriti?

\question{Ei tea, ei oska öelda!}

Väga lihtne, see oli see teema, kust ajalehte raha tuli. Surmakuulutused. 
Tänaselgi päeval on Postimehes populaarne paar eelviimast lehekülge, kus on  
surmakuulutused. Neil on see hea omadus, et nad on suhteliselt standardses 
formaadis, neile on mingisugune neli-viis, võib-olla kümmekond erinevat 
kujundust ja kui need kuidagi mallidena ära implementeerida, siis 
surmakuulutuse publitseerimise aeg kukkus drastiliselt. Ja kuna see oli 
sisuliselt ainuke allikas, kust lisaks tellimusele raha tuli, siis selle vastu 
oli lehe juhtkonnas ikka päris normaalne huvi.

\question{Mis väljundisse need mallid läksid?}

Väljundiks oli kile peale trükitud lehekülg, mis läks siis ofsettrükki.

\question{Sinu tarkvara optimeeris selle asja otse kilele?}

Jah, laserprinteriga lased paberi abil läbi kile ja see, mis sealt välja tuleb, 
on enam-vähem see, mida saab trükkalitele kätte anda, et kleepige õigesse kohta.

\question{Ehk sa produtseerisid PostScripti?}

Jah.

\question{See on ju päris keeruline ju?}

Eks see  \emph{publishing} tarkvara, kui ma õigesti mäletan, siis Ventura 
Publisher, tegi põhitöö ära ja väga mõned üksikud asjad vajasid otseselt 
PostScripti tasemele minekut. 

\question{See kõik tahab jälle teadmist saada, kust sa seda juurde hankisid?}

Istud ja nokid ja nii ta on. Küll ta lõpuks tuleb, kus ta pääseb!

\question{Sa ütlesid enne, et sul tekkisid kogukonna moodi asjad?}

Üliõpilastena sa käid ikkagi seltskonnaga ringi. Proovid ühes arvutiklassis, 
proovid teises arvutiklasis. See hetk olid juba Tartu Ülikooli 
matemaatikateaduskonda\index{Tartu Ülikool!Matemaatikateaduskond} ka 
arvutiklassid tekkinud ja  sealsete inimestega suheldes  see  kogukond vaikselt 
tekkis. Samamoodi tekkis kogukond nendest, kes olid mul füüsikas 
kursusekaaslased.

\question{Tol ajal mingit arvutisidet ei olnud?}

Tol ajal veel arvutisidet ei olnud. Aga eks see tuli ka suhteliselt kiiresti. 
Ühtedel meestel  oli ühte laadi arvuti ja teistel teistlaadi arvuti, mis 
omavahel flopikettaid ei lugenud. Pandi kaks või kolm traati omavahel kokku ja 
prooviti neid kuskilt saadud programme teisele mehele ka üle kanda.

\question{Ma küsin Edasi kohta veel. Kuidas too töövoog välja nägi? Ma olen 
ikka tahtnud küsida, et millega see ajakirjanik oma teksti kirjutas?}

Edasi\index{Edasi} aegadel ajakirjanik kirjutas ikkagi kirjutusmasinaga ja oli 
tinaladu. Aga kui tekkis rohkem arvuteid,  mindi tinalaolt üle kile peale 
trükitud väljundile. Seal oli terve hunnik etappe veel vahel, et ajakirjanikele 
arvutid saada. Arvuti oli tol ajal suhteliselt kallis asi, terminalid olid 
natukene odavamad. Postimehes\index{Postimees}, kes oli siis juba erastatud ja 
Postimeheks muutumas, sai pandud üles üks UNIXi server, kus oli küljes 
kuusteist terminali, mis sai ajakirjanikele maja peale laiali veetud. 
Terminalide ühenduseks vajalikud kaardid sai Tõraverest, seal oli mingi 
Urania\index{Urania|see{Astrodata}} nimeline firma, millest kasvas välja 
Astrodata\index{Astrodata}.

Teksti sisestamiseks oli ajakirjanikule terminal piisavalt lihtne, seda ei 
olnud vaja ilusaks ajada, pea-asi, et tekst olemas on. Seesama tekst võeti ja 
pandi mingisse \emph{publishing} tarkvarra ja lasti kile peale välja. Kiled 
kleebiti kleeplindiga kokku küljeks, mis läks  kunagi öösel trükikotta. 

\question{Mis Unix seal serveris jooksis?}

BSDi Unix\index{Unix!BSDi Unix}, mis sai täiesti ausalt ostetud, \emph{source} 
koodiga,  kõige värgiga. 

\question{Tol ajal oli ju lausa embargo, kuidas te selle serveri hankisite?}

Jah, embargod ja asjad olid ka, aga need 386-laadsed arvutid embargo alla ei 
kukkunud. Ülemine ots, nagu PDP, oli embargo all. 

Rahulikult jaksas need kuusteist terminali välja vedada. Eks arvutite hankimine 
tol ajal oli  sihuke keerukas tegevus. A la kui  Postimees sai oma tellimise 
rublad kätte,  veeti need kohvriga oskuslike ärimeeste juurde, kes kuskilt 
Moskvast said oskuslikult mingi arvuti. Oli nagu niisugune vorsti-kauba aeg. 
Aga lõpuks oli need vajalikud arvutid võimalik välja ajada.

\question{Veiko Tammega\index[ppl]{Tamm, Veiko} oleme sellest rääkinud 
pikalt\sidenote{Vt. lk. \pageref{sisu!veiko_moskvas}.}!}

Just. Mõned kohvrid jõudsid tema juurde ka, ta oli põhiline Postimehele 
arvutite hankija.

\question{Jällegi paneb mind imestama, kui sujuvalt läheb skoop laiemaks. Kui 
programmeerimisest ma saan aru, siis nüüd näeme Unixi servereid, töövoogusid, 
võrke ja nii edasi? Mis hoidis sind seda ringi laiendamas, oleks ju olnud 
lihtne programmeerimisele või millelegi muule keskenduda?}

Ma arvan, et  see seltskond ümberringi, Postimehe ajakirjanikud või Edasi 
ajakirjanikud, neil läks silm särama, et \enquote{näed, niisugune võimalus on  
oma tööd paremini teha} ja see tekitas nagu surve, et neid kuidagi aidata. Et 
kui inimesel silm särab midagi tehes, läheb sinul endal ka silm särama, kui sa 
teda aitad ja põhimõtteliselt nii lihtne see ongi.

\question{Aga selline asi eeldab, kui ma tohin nii elementaarset asja 
sedastada, mingit huvi inimeste vastu ka? Arvutiinimeste hulgas\ldots}

Absoluutselt. Aga kui sa igapäevaselt kellelegi kõrval istud, sul ju tekib see 
huvi tahes-tahtmata. Ei ole võimalik, et ei tekiks. Ja kui sa istud veel 
intrigeerivate inimeste kõrval, kes hoiavad nii-öelda  kätt elu pulsil, 
räägivad sulle, \enquote{oh,  Tallinnas Toompeal räägiti seda ja toda ja kas 
paneme selle lehte või pane}, hakkab kõrv liikuma küll ja tahad selle  melu 
sees olla.

\question{See võis äge aeg olla küll, igasugu asju hakkas ilmuma!}

Tartlasena ma neid kõiki Tallinna asju ei tea, hakkas ilmuma Eesti Ekspress, 
Liivimaa Kuller Kalle Mülleri\index[ppl]{Müller, Kalle} ja Väino 
Koorberg\index[ppl]{Koorberg, Väino} vedamisel. Kroonika, alustuseks Kalle 
Mülleri, siis Ingrid Veidembergi\index[ppl]{Veidemberg, Ingrid} vedamisel. 
Kõikide nende juures olid mingisugused momendid, mis olid mega kihvtid ja 
huvitavad. 

Alguses oli mustvalge, siis tekkis värviline logo ja nii edasi.

\question{See oli nii suur asi, kui logo oli värviline. Kõigi nende juures oli 
ju mingisugune \emph{publishing}-u või trüki inimene ametis, Peeter 
Marvet\index[ppl]{Marvet, Peeter} tembutas juba kuskil tõenäoliselt?}

Tõenäoliselt Tallinnas tembutas, aga Tallinn ja Tartu on täiesti erinevad asjad.

\question{Ma seepärast küsingi, et kas sedalaadi inimestel mingit oma kogukonda 
ei tekkinud, et programme vahetada või midagi?}

Kindlasti oli, aga vot see on see teema, mida enam ei mäleta. Lihtsalt enam ei 
mäleta, pärast tuli neid asju nii palju peale, et\ldots

\question{Ühel hetkel sai Edasi asi otsa, mis sa edasi tegid?}

Enne kui Edasi otsa sai, on kindlasti üks selline huvitav moment, millest ma 
tahaksin rääkida. 

Edasiga seoses ma sain umbes aastal kaheksakümmend üheksa või üheksakümmend (ei 
vastuta kummagi numbri õiguse eest) endale emaili. Selle aadress oli umbes 
niimoodi, et \verb|taavi@pm.ew.su|.

\question{EW nagu \enquote{Eesti Wabariik} ja SU nagu \enquote{Soviet Union}!? 
Selline domeen oli olemas?}

Jah, niisugune domeen oli olemas ja domeen SU on endiselt olemas.

\question{Vot siis. Kust sa selle aadressi omale said?}

Jälle mõnes mõttes tänu ülikoolile, kuna ülikooli psühholoogia 
teaduskondast\index{Tartu Ülikool!Psühholoogia teaduskond} Tiit 
Mogom\index[ppl]{Mogom, Tiit} oli Tallinnasse kas Küberisse\index{Küber} või 
KBFI-sse\index{KBFI} (seda enam ei mäleta) käima ajanud  modemega UUCP 
meilisideme. Sealt see meiliaadress tuli. 

Sellest ajast ma veel mäletan esimest niisugust suuremamahulist meili umbe 
ajamise intsidenti. Olid siuksed asjad nagu meililistid, mida sai tellida. Ja 
ma kogemata tellisin mingi siukse aktiivsema kirjavahetusega meililisti. Meile 
muudkui tuli ja tuli, modem ei pannud toru hargile ja ei pannud ja ei pannud ja 
ei pannud. Läheb tund ja teine ja kolmas, \enquote{no täitsa pekkis, kogu see 
maailm on umbes ja läheb katki!}.  Võtsin jalad selga, kõndisin üle Toome 
Tiidu\index[ppl]{Mogom, Tiit} juurde, \enquote{kuule, aita mul see meilivoog 
ära katkestada} Umbes oli ju modemiga helistamine minu juurest tema juurde ja 
tema juurest kuhugi Tallinnasse ja Tallinnast veel kuhugi Soome ja mis iganes.

\question{Kusjuures on ju tänapäeval täitsa unustatud, kui oluline asi oli 
e-mail, et kõik asjad käisid üle selle. Oli olemas ju FTP üle e-maili, kus 
failid keerati sobiva pikkusega juppideks, lasti BASE64-ga kokku ja saadeti 
sulle meili peale!}

Täpselt. Sellisel kujul oli võimalik kuskilt listserveritest või 
arhiiviserveritest endale tellida vaba tarkvara lähtekoodi. Levis muid asju ka, 
aga meie jaoks oli just see vaba tarkvara lähtetekstide kättesaamine huvitav.

\question{Miks see huvitav oli?}

Sa said jälle mingit  uut, võimsamat, asja teha!

Sellesama Edasi juures tuli järjest  automatiseerimise asju  peale. Mingi hetk 
tehti oma kojukanne. Kojukande jaoks oli jubedalt abiks, et postiljonidele 
jagataks pakid sihtrajoonide järgi, et õiged kleepsud oleks peal, et õigesse 
hunnikusse õige  kogus ajalehti saaks, et neil oleks nimekirjad, mille  järgi 
viia jne. Sai näiteks tehtud kojukande infosüsteem. 

\question{Jälle, mida tänapäeval ei\ldots}

Eesti Postis või Omnivas on kojukande infosüsteem raudselt olemas!

\question{Just! Aga mida tänapäeval sagedasti ei juhtu, et sa astud uksest 
sisse ja hakkad nullist sellist infosüsteemi programmeerima. Tüüpiliselt on 
täna midagi juba olemas.}

No see oli see aeg, kus ei olnud võimalik midagi aluseks võtta, lihtsalt ei 
olnud.  Äriliselt oli Postimehel  ainuke võimalus ise kojukande süsteem teha, 
kuna riiklik kojukanne ei toiminud. Ajaleht viidi kätte lõunaks, aga Postimees 
tahtis, et hommikuse kohvijoomise ajaks oleks leht olemas ja ehitas nullist 
üles oma kojukandesüsteemi, mis pärast vist liitus Express Posti omaga. Ja 
praegusel hetkel on vist alternatiivse kojukandesüsteemina  mingis ulatuses 
isegi toimiv.

\question{Ma mäletan seda küll, sest see, et Postimees oli hommikul vara 
postkastis oli \enquote{nagu Läänes}, nagu \enquote{päris}! Mis sa tolle 
e-mailiga peale listide lugemise veel tegid?}

Kaks asja, mis on eredalt meeles, on see, et üheksakümne esimesel aastal oli 
Moskvas sihuke putšilaadne asi, kus vahetati valitsusi ja tankid olid igal pool 
teletornide ees ja mis iganes. Oli jube infoauk, et mis toimub, kus toimub. 
Siis sai sobivate Moskva arvutihäkkeritega  kokku lepitud, et teeme mingi 
otseliini, paneme infolistid käima. Oli jube põnev, et sa said toimiva ja 
ajakirjanikele kasuliku info   emaili teel kätte natuke enne, kui ta kurat teab 
kust tekkis. 

\question{Ja sul need kontaktid olid olemas\ldots?}

Need kontaktid olid olemas  mingist konverentsil käimisest või midagi sihukest. 
Sai Moskva taga Vladimiris või mingis sihukeses linnas Unixi kasutajate 
konverentsil käidud. Kus, kusjuures,  olid kohal esinejad Berkeley ülikoolist, 
kaks Berkeley Unixi loojat või guru. Kui ma õigesti mäletan, oli üks neist 
Keith Bostic\sidenote{Keith Bostic on Berkeley Software Distribtion-i (BSD) 
ajaloo üks võtmetegijaid, kelle kõikvõimalikku panust vaba tarkvara ajalukku on 
raske ülehinnata.}. Selles mõttes ei tasu naerda, venelased suutsid need mehed 
enda juurde meelitada, tõenäoliselt oli neil ka huvitav ja see konverents oli 
megakihvt.

\question{Seda ma imestan, et see võis olla väga kõva sõna tol ajal!}

Jah, oli. Tagantjärgi, see tekitas jälle tunde, et see arvuti teema on hea 
teema, et sa hoiad nagu näppu pulsil, oled suhteliselt lähedal sellele, mis 
maailmas ägedat toimub.

\question{Ühel hetkel sa ikkagi tolle ägeduse juurest liikusid ära?}

Oota veel natuke! 

Ülikooli juures tekkis ka esimene moment, kus emailist nähti, et see on päris 
kihvt asi, ja et lisaks emailile on olemas ka püsiühendused ja asjad. 1992. 
aastal oli see moment, kus Jaak Lippmaa\index[ppl]{Lippmaa, Jaak} pani 
Tallinnas püsti taldriku Rootsi Tele-X-iga ja Tartus Tähetorni\index{Tartu 
Tähetorn} sai satelliidi abil püsiühenduse, mis oli 64 kilobitti. Selleks, et 
see ühendus sealt Tähetornist Toomel kuhugi mujale ka leviks, sai Postimehe 
eestvedamisel Tähetornist kõigepealt Keemiahoonesse\index{Tartu 
Ülikool!Keemiahoone}, sealt sealt ülikooli peahoonesse ja sealt Postimehe majja 
üle katuste veetud Tartu esimene püsiühendus. See  oli ehitatud vene 
sõjaväelaste käest viiesaja rubla eest ostetud üle jäänud kaabli rullist.

Selle kaabli ümber tekkis reaalselt see nii-öelda interneti kommuun. 

\question{Nojah, sest need kes sinna tee peale jäid, said ka endale interneti!}

Absoluutselt. Ja see oli täiesti online!

\question{Aga latents pidi äge olema?}

Kuskil kuussada millisekundit.

\question{Nii hull ei olnudki!}

Ega see 64 kilobitti ei ole tänapäeva mõistes mingisugune superkiirus, aga 
tollel ajal, kui internet oli veel tühi igasugustest kassipiltidest, oli see 
täitsa okei kiirus.

\question{Kes see kogukond siis oli, kes tolle kaabli ümber kogunes?}

Kogukond oli,  EENeti\index{EENet} inimesed, Enok Sein\index[ppl]{Sein, Enok}, 
Anne Villems\index[ppl]{Villems, Anne}, Richard Villems\index[ppl]{Richard, 
Villems}, Tiit Mogom\index[ppl]{Mogom, Tiit}, Marek Tiits\index[ppl]{Tiits, 
Marek}, Balti Uuringute Instituudist\ldots	 Ja tõenäoliselt neid inimesi 
on veel, keda praegu mälu meelde ei tuleta.

\question{Sel ajal oli EENet juba olemas?}

EENeti veel ei olnud, aga tuumik, inimesed, olid needsamad kes  selle kogukonna 
moodustasid. ja kes nägid vaeva selle nimel, et see interneti püsiühendus oleks 
olemas. Lisaks kogunes  modemiga sisse helistajaid ja vaikselt hakkas kasvama.

\question{Kui ma nüüd loogiliselt järeldan, siis käis Tartu ja Tallinna side 
üle satelliidi?}

Jah, alguses käis üle satelliidi, mingi aastapäevad hiljem tuli ka maapealne 
püsiühendus, mille  eestvedajaks oli Küber\index{Küber}. See on see koht, kus 
ilmselt Eesti internetimaailmas tekkis kaks nii öelda rististe suuskadega 
kommuuni, Küberi  KBFI\index{KBFI} oma.

\question{Mitte siis veel AS Cybernetica vaid Küberneetika Instituut?}

Jah, aktsiaselts tekkis  kunagi hiljem.

\question{Miks need suusad ikkagi risti läksid?}

Vat seda mina ei tea. Tartu inimesena ei saanud ma sihukestest Tallinna 
probleemidest  lihtsalt aru. Kui ressursiga on kitsas, mida ta Eesti Vabariigi 
alguses oli, siis võimalik, et seal olid lihtsalt mingisuguste rahade jagamise 
või teaduse finantseerimise mured. Et kes sai oskuslikumalt finantseerimisele 
ligi või\dots See oli tegelikult olelusvõistlus, mis jättis kõigile jälje.

\question{Mis edasi sai?}

Edasi sai Postimehe periood sai läbi, mingisuguse aastakese töötasin Tartu 
Ülikooli Raamatukogus\index{Tartu Ülikool!Raamatukogu}, kuhu sai Rootsi kunni 
abiga muretsetud esimene serveri moodi asi ja otsast hakatud kirjutama 
raamatukogu infosüsteemi.

\question{Sihuke pisike projekt! Kas see tähendab, et sinu ajast on seal need 
kuulsad kalanimega\sidenote{Kui kõigil serveritel veel oma nimi oli, oli 
kombeks anda ühe asutuse serveritele samalaadsed nimed. Halo kunagised masinad 
olid nimetatud näiteks kuulsate häkkerite järgi - woz, mitnick jne. Sain 
sellest kombest kuulda just Tartu Ülikooli serverite kaudu.} serverid?}

Enam ei mäleta. kilu.nlib.ee\index{kilu.nlib.ee} oli, ma arvan, 
SPARCStation 2\index{SPARC!SPARCStation 2}.

Kusjuures selle serveri põhiline funktsioon oli ikkagi esimene ülikooli 
raamatukogu  elektrooniline kataloog, mida kohapealsed inimesed ise 
programmeerisid.

\question{Jaa, selle süsteemi kasutamiseks olid raamatukogus isegi terminalid. 
Oma aja kohta, mäletan, oli funktsionaalne ja tore lahendus!}

Jaa, see oli jube töö, see andmesisestamine on  nullist tehes ikka väga raske.

Aga, kahjuks või õnneks, tagantjärgi ei oska öelda, see raamatukoguperiood jäi 
suhteliselt lühikeseks, kuna Jaak Lippmaa\index[ppl]{Lippmaa, Jaak} kutsus mind 
Tallinnasse. See tundus veel põnevam. 

Ta kutsus mind sellises riigiasutusse nagu Valitsusside\index{Valitsusside}. 
Umbes sellise mõttega, et \enquote{sina oled nüüd selle Internetiga natuke 
kokku puutunud, oskad ühest arvutist teise sümboleid saata, et siin mingisugune 
KGB sidekeskus tuleb Eesti riigil üle võtta, tule aita}. Ja, noh, tulingi.

Ütleme niimoodi, et, KGB-st võib rääkida mida iganes, aga tehnoloogilise poole 
pealt nägi asi välja ikka suhteliselt õnnetu. Ädala tänava 
majas\sidenote{Aadressiga Ädala 4d.} olid suured saalid täis mittetöötavaid 
telefonijaamu. Ma võin nüüd natukene valetada, aga oli suurusjärgus 200 
töötavat telefoni, mis olid  ette nähtud riigi kui nähtuse ja igasuguste 
organite jaoks. Aga, ütleme, kakssada töötavat telefoni on kogu valitsuse peale 
ikkagi suhteliselt nadi number.

Järgmine projekt oli Siemensi telefonijaamadega reaalse sisetelefonijaamade 
võrgu käima panemine Toompeal, Kadriorus, Välisministeeriumis ja kes nad seal 
kõik Pikal tänaval olid. Kokku mingi paar tuhat numbrit, mis tuli õnneks täitsa 
edukalt välja ja oli reaalselt ka mingisugune kasu olemas.

\question{Kas see oli veel analoogjaam?}

Digijaam, Siemens TopCom,  võrgustatav ja kõik ja puha. See  oli päris sidevõrk.

\question{Kust sihuke visioon tuli, et selline asi ehitada, see jaam pidi ju 
kallis olema? Odavam oleks olnud ju kuidagi analoogjaamadega hakkama saada, aga 
tehti investeering?}

See investeeringu lüke tuli Jaagu isiklikust initsiatiivist, mida ta  
tehnoloogilise poole pealt konsulteeris Aavo Pikofiga\index[ppl]{Pikof, Aavo}, 
kes oli Tallinna Telefonivõrgus. Sinna taha tulid tänu Jaagu ja Endel 
Lippmaa\index[ppl]{Lippmaa, Endel} tutvustele ka riigi funktsioonid. Seda oli 
tõepoolest vaja, visioon osteti ära ja finantseeriti ära. Ja oli reaalselt 
töötav sisetelefonivõrk, aga eks ta mingi aja pärast jäi ajale jalgu.

\question{Kindlasti. Kas füüsiline kaabeldus oli Eesti Telefoni oma?}

Osa oli Eesti telefoni käest ja osa oli vana KGB kaablivõrku, mida oli 
tegelikult Tallinnas mõnisada kilomeetrit täitsa.

\question{Ahjaa, sest kuigi keskjaam ei toiminud, olid kaablid ikkagi olemas!}

Olid korraliku tinakestaga kaablid, mis oli umbes nagu tuumasõja üle elamiseks 
ette nähtud ja meenutasid rohkem tanki kui kaablit.

\question{See kõlab projektina, mille käigus kohtub huvitavate inimestega?}

Absoluutselt. Telefonijaama installeerimine ja käima panemine Kadrioru lossis, 
kus Lennart Meri\index[ppl]{Meri, Lennart} vaatab sul üle õla ja õpetab, kuidas 
telefoni pistikuid ühendada, võib tagantjärgi tarkusena muigama panna, aga see 
oli niimoodi. 

\question{Kõlab väga Lennarti moodi!}

Absoluutselt, ta oskas kõikides asjades nõu anda. Et \enquote{vaadake, poisid, 
et te teete nii või naa}.

\question{Kaua sa valitsuse kaableid vedasid?}

Kokku kolm aastat. Peale esimest edukogemust telefonijaamadega, loomulikult me 
tahtsime järgmisi sihukesi edukogemusi veel. See interneti-nimeline asi ronis 
kogu aeg uksest ja aknast sisse. Kirjutasime siia-sinna järgmiseid pabereid, et 
nüüd oleks mõttekas selle jaoks kuidagi investeerida. Ei võtnud väga vedu. 
Kuigi see hetk juba ükskõik kelle käest, kes natukenegi internetiga tegeles 
nõuti, et \enquote{mina tahan ka}. Inimesi tuli uksest ja aknast sisse, et 
\enquote{tehke midagi, mul oleks Internetti vaja}. 

Aga paberite kirjutamine ei olnud edukas. Nii hakkasin kõikide oma tuttavate 
juurest läbi sõitma, et \enquote{kuulge, niimoodi asi ei toimi, et interneti 
kõik tahavad, võiks ju teha, aga välja ei tule. Et  võtaks pundi kokku, hakkaks 
nagu normaalselt ise tegema}. Ma arvan, et ma käisin kõik inimesed, kes vähegi 
Internetiga tegelesid, läbi, et \enquote{teeks midagi}. Ja reaalsusena 
tartlased vedu ei võtnud. Tartus on nii mugav ülikooli juures olla,  Pirogovi 
ja kõik muud asjad. Kuigi puht praktiliselt Marek Tiits\index[ppl]{Tiits, 
Marek} aitas päris palju ideed formuleerida ja nii edasi. Kõige rohkem võttis 
vedu KBFI-st\index{KBFI} Andres Bauman\index[ppl]{Bauman, Andres}. Ja siis 
Andresega koos tegimegi sihukese internetiettevõtte. 

Käisin poes, ostsin riiulifirma, mille nimi oli 
Nösper\index{Nösper|see{Uninet}}. Mis  on praeguseks siis Elisa, sellesama  
riiuli pealt ostetud firma ja Elisa\index{Elisa} registreerimiskood on sama. 

\question{Vot sulle juriidilist järjepidevust! Mis aastal see oli?}

See oli aastal 1995, 1994 lõpp 1995 algus.

\question{See oli ju üsna karm aeg, mitte midagi polnud saada ja kõik asjad 
tuli nullist ehitada?}

Oli küll. Aga teisalt inimesed olid ka leidlikud: Data Telekom\index{Data 
Telekom} Neeme Takise\index[ppl]{Takis, Neeme} eestvedamisel ehitas ise RS422 
elektrilise ühenduse peal modemeid. Kui ei olnud, siis teed ise. Kes kohandas 
mingeid asju\ldots Midagi oli võimalik igal juhul teha.

\question{Just nimelt, kui midagi saada ei ole, pead ise tegema!}

Mingisugused lisaboonused olid ka. Tänapäeval inimesed arvavad, et mingi kümne 
euro eest kuus peaks tulema nii jäme internet, kui maailmas üldse olemas on. 
Mis ta siis on, kolm kohvitassi kuus? Siis too aeg oli Internet nii  seksikas 
või uus või edev või mis iganes, et oldi tema eest veel nõus maksma. Sellised 
asjad, et ettevõte ostab ise seadmed välja, maksab kinni paigalduse ja hakkab 
kuutasuga ka veel maksma. Oli niisugune helge aeg. Ja  tänu sellele oli 
võimalik need esimesed seadme-investeeringud teha. Sellise mudeliga nagu täna 
on, et kõik on kümme või viis kohvitassi kuus, internet ei oleks Eestisse 
jõudnud mitte kunagi.

\question{Kui ma õigesti mäletan, siis ma hakkasin teie edasimüüjana toimetama 
aastal mingi 1996, võib olla? Mis tähendas, et üsna varakult oli teil üsna lai 
võrk teenuseid ja partnereid?}

Põhiline, oli sissehelistamine, ma arvan. Ma ei mäleta, mis süsteemiga sai 
püsti pandud füüsiline modemite \emph{pool} telefoninumbritega  ja sellest see 
tegevus tegelikult pihta hakkas. Tõenäoliselt telefoninumbritega, kuna need 
olid ka defka, aitas Jaak Lippmaa\index[ppl]{Lippmaa, Jaak} isa Endel 
Lippmaa\index[ppl]{Lippmaa, Endel}, et saaks kuidagi seal mingi sobiva 
järjekorra sobivasse punkti.

Ja sealt  hakkas see internetiäri vaikselt kasvama, kuna inimesed tahtsid, huvi 
kerkis. Ma arvan, et 1997.  aasta alguses oli see koht, kus  nii-öelda 
üldkasutatav Interneti välisühendus oli igasuguste puukide poolt, nagu oli meie 
puukfirma, nii täis aetud, et EENet\index{EENet} tegi otsuse, et \enquote{nii, 
nüüd ärikasutajad, muretsege endale oma välisühendus}.

\question{Isegi hilja? Üsna kaua saite akadeemilise traadi peal toimetada?}

Jaa. Internet ei olnud veel päris akadeemilisest maailmast välja pääsenud ja 
eks seal kõik toimetasid akadeemilise traadi peal. Senimaani, kuni EENet jalga 
maha ei pannud, senise oli see ka täiesti  loomulik nähtus. 

\question{Kuidas te siis ühenduse saite? Soome kuhugi?}

Mul oli mingist ajast jäänud kontakt Helsingi telefonivõrguga ja nende 
inimestega koos sai  kanal tellitud, hangitud ühendus ja seadmed ja meil oli 
juba  sedavõrd palju käivet, et me enam-vähem suutsime isegi selle väliskanali 
kuutasu kinni maksta. Mis oli siuke soliidsed pea sada tuhat krooni kuus. 

\question{See oli jõhker summa tollal!}

See oli väga jõhker summa\sidenote{Selle raha eest võis endale Kadriorgu 
ühetoalise korteri osta! Iga kuu!} ja kogu see internetitehnoloogia või 
ruuterid,  kogu see värk maksis ikka suhteliselt hingehinda. Nihuke kuue 
pordiga Cisco ruuter, marki enam ei mäleta,  maksis mingi 250 tuhat, ehk, ma ei 
tea, S-klassi Mersu hinna.

\question{Õudne! Sellest ma järeldan, et miski füüsiline kaabel oli Eesti ja 
Soome vahel olemas?}

Füüsiline kaabel oli olemas. Peale seda, kui Telekom\index{Eesti Telekom} sai 
Eesti riigilt kontsessioonilepingu,  vedas ta suhteliselt kiiresti Eesti-Soome 
vahele ka kaks valguskaablit. Ma ei mäleta, mis aastal see oli, aga ma arvan, 
et see oli ka kuskil üheksakümmend seitse, üheksakümmend kuus. 
Telial\index{Telia} oli see kogemus olemas, investeerimisvajadus kogu selle 
analoogvõrgu välja vahetamiseks oli väga selgelt olemas ja Telial tegelikult 
investeerimisjõudu oli. 

\question{Te ju tegite mingit \emph{hosting}-ut ka?}

Jaa, sai servereid pidada, et ei peaks traadi raha maksma. Mis, mis oli, ma 
arvan, täiesti normaalne teema.

\question{Kuidas see käis? Kui ma praegu ostan omale mingi virtuka, siis tol 
ajal ma sain mingi konto ja parooli ja\ldots?}

Said UNIXi \emph{account}-i endale ja oligi kõik\sidenote{Tuleb arvestada, et 
toona ei olnud Windows ja UNIX pooltki nii suured sõbrad, kui praegu. Olen 
koostöös Uninetiga nii veebi programmeerinud, et kirjutasin Windowsi all Perli 
skripti, laadisin selle FTP-ga serverisse, see ei töötanud, küsisin admini 
käest väljundi, parandasin skripti ära, laadisin uue versiooni ja nii edasi. Ei 
läinud väga kaua aega, kui admin keeldus seda tantsu kaasa tegemast ja tuli 
leida viis Perl Windowsi all käima saada. Windowsi all veebiserveri 
jooksutamine, muidugi, oli jätkuvalt välistatud, aga vähemalt süntaksivead 
tulid välja.}.

\question{See oli üks legendaarne ettevõtmine ja legendaarne aeg, aga mis sa 
praegu teed?}

Praegu olen sellises huvitavas ettevõttes nagu RebelRoam. 
Tegeleme transpordiettevõtetele wifi teenuste pakkumisega ja nende 
optimeerimisega. Kliendibaasiks on mööda Euroopat, Ameerikat, Inglismaad 
sõitvad bussid, kus siis on bussis wifi, mida  bussireisijad saavad kasutada. 
Või jõelaevad. Või kruiisilaevad, kus pensionärid ostavad mingi kolme-nelja 
tuhande dollarilise nädalase reisipaketi ja kui nad Pariisis Eiffeli torni 
pildistavad, siis nad peavad õhtul saama saata oma pildi lastelastele, muidu on 
see reisikogemus natukene nõrk. 

See võib tunduda  eestlasele imelikuna, meil on internet kogu aeg igal pool 
vabalt kättesaadav, aga näidetena ei, Prantsusmaal, Ameerikas, Inglismaal ei 
ole ei 4G ega 5G nii levinud, et selle kvaliteet oleks kõigile piisav. Ja seal 
on see teenuse optimeerimise vajadus täitsa olemas.

\question{Mis tähendab, et kõik need inimesed, kes Ameerika kiirteel Greyhound 
bussi järel sõidavad, et wifit kasutada, on teie kliendid?}

Greyhound on meie klient. Ma ei tea ka, kas kõikides bussiliinides, või rohkem 
Lääneranniku pool, ma lihtsalt ei tea. Aga Greyhound on tuttav nimi küll,  on 
kliendina läbi jooksnud ka tehnomeeste poole pealt.

\question{Ma vaatan, et nagu sageli tundub, ring on täis saanud? Kui sa Edasi 
puhul rääkisid, kuidas sulle tegi rõõmu sära ajakirjaniku silmas, siis nüüd sa 
tõid ka kohe esimese näite Eifeli torniga. Et lõpuks sa tegeled ikka inimeste 
rõõmsaks tegemisega?}

Kui sa teda rõõmsaks ei tee ja ta sinu teenust vihaga ostab, ega  ta raha sulle 
ka ei maksa. Kui teed rõõmsaks, siis saad tõenäoliselt mingisuguse raha 
sobivatel päevadel pangakontole. Ja on rõõm endal ka midagi paremaks teha.

\question{Koodi veel kirjutad?}

Jah.

\question{Võrku veel konfid ise?}

Võrku ise ei konfi, ma arvan, et ma korralikult enam ei oska seda teha. Võrk on 
tänapäeval  niivõrd ära virtualiseeritud, et enam ei oskaks seda konfigureerida.

\question{Mina küll märkasin, kuidas keegi küsis Facebookis ja sa mitte ei 
vastanud, et kuidas saab teha vaid vastasid ifconfig-i käsureaga, mis võtmest 
töötas!}

See on oma arvuti konfimine, mitte võrgu konf. Võrgu konfimise all ma mõtlesin 
ikkagi päris võrku, kus on jämedad ruuterid, vilkuvad sinised LED-id ja käib 
läbi pool Eesti Interneti \emph{traffic}-ust. No see oleks nagu võrgu 
konfimine! 

