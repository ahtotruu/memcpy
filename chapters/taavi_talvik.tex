\index[ppl]{Talvik, Taavi}

\question{Kuidas sina arvutite juurde jõudsid?}

Arvutite juurde jõudmine oli iseenesest väga lihtne. Kodus sattus olema paar 
põnevat raamatut, vist Ustus 
Aguri \enquote{Abakusest raalini}\sidenote{Sarja \enquote{Mosaiik} 1980. aastal 
ilmunud 28. teose autorid olid siiski Rafail Guter ja Juri Poljunov, tõlkijaks 
Madis Järv.} ja Norbert Wieneri \enquote{Küberneetika}.\sidenote{Norbert 
Wiener. Küberneetika ehk juhtimine ja side loomas ning masinas. Eesti Riiklik 
Kirjastus, 1961.} Igal juhul tundusid need jube põnevad. Ja kuna esivanemad olid Tartu Ülikoolis\index{Tartu Ülikool} keemikutena 
ametis ja käisid kuulujutud, et ülikoolis mõni arvuti ikka on, siis hakkasin 
neile kohe pinda käima, et kuulge, ma tahaksin näha, missugune 
arvuti päriselt välja näeb.

\question{Sa oled järelikult Tartust pärit?}

Jah, ma olen Tartust. Tartu on väga okei, väike 
linnakene Elva lähedal, ja lapsepõlves mulle seal väga meeldis. 

\question{Kui vana sa olid, kui hakkasid vanematele arvuti asjus pinda käima?}

Arvan, et olin
üheksandas klassis, aastal 1985. Ja tõepoolest neil seal ülikoolis arvutid olid, isegi välismaa 
omad. Sel ajal oli välismaa arvuti haruldus, 
aga kuna nad tegid mingisuguseid imelikke elliptiliste kilede mõõtmisi, siis 
oli kilede mõõtmise masinaga kogemata kaasa ostetud arvuti, mille nimi 
oli Hewlett-Packard 85\index{HP-85}. 

\question{Kas see oli lauaarvuti?}

See oli lauaarvuti, millel oli pisikene, vist viietolline ekraan,\sidenote{Eri 
allikate andmetel tuli HP-85 kas viie- või kuuetollise CRT ekraaniga.} 
klaviatuur, kassettmakk ja termoprinter ning taga hunnik juhtmeid, mis 
ühendasid arvuti mõõtmisseadmetega.

\question{Niisugune mudel ei ole küll kellegi jutust läbi käinud, kõlab täitsa 
eksootiliselt!}

See oligi väga eksootiline mudel. Sellel oli oma Hewlett 
Packardi protsessor,\sidenote{Protsessori koodnimeks oli  Capricorn ja see toimis 
taktsagedusel 0,6 MHz (!).} mis oli oma aja kohta täitsa innovaatiline ja 
tore. Väljapaistvaks tegi selle arvuti tavaline 
BASIC\index{BASIC} ja see, et ekraanil sai jutte joonistada. Ja kui 
ise jutte joonistada ei osanud, siis sai pingpongi või kosmonautide maandumist 
mängida. 

\question{Kas sind lasti kohe seda niisama näppima?}

Jah. Eks vanemate kolleegid (kui ma õigesti mäletan, siis ühe
nimi oli Zirk) õpetasid ka, et kaua sa siin mängid, parem proovi kokku liita arve 
ühest kümneni või midagi sellist, ja nii see asi pihta hakkas.

\question{Kas teil koolis ei olnud arvuteid? Mis koolis sa käisid?}

Tartu 10. Keskkoolis\index{Tartu 10. Keskkool}, tänapäeval Mart 
Reiniku Gümnaasium\index{Mart Reiniku Gümnaasium|see{Tartu 10. 
Keskkool}}. Koolis ei olnud sel ajal veel mitte midagi, täitsa tühi maa. 
Tõenäoliselt Nõos midagi oli, aga Nõo on Tartust nii kaugel ja 
selleks pidi tutvusi olema, et sinna keegi oleks kutsunud.

\question{Üheksandas-kümnendas klassis 
kipuvad ju igasugused muud põnevad hobid olema Wieneri lugemise 
asemel. Miks sa neid lugesid?}

Tore ja huvitav oli, ja võibolla vanemad sokutasid ka lugemist, et 
poiss saaks targemaks. Tagantjärele enam ei mäleta, mis see täpne ajend oli.

\question{Kas sul oligi populaarteaduslike asjade 
või ulmehuvi?}

Oli nii populaarteaduslike asjade kui ka ulmehuvi. Kuna mu nupp 
reaalteadusi jagas -- käisin ka olümpiaadidel --, siis see 
tundus loomulik. 

\question{Kuidas sul reaalteaduste jagamine esile kerkis? Kas kohe 
esimesest klassist alates tundsid ennast sellel alal mugavalt või tegeles keegi sinu 
arendamisega spetsiifiliselt?} 

Tundsin end suhteliselt mugavalt tänu sellele, et isa-ema olid ülikoolis õppejõud, 
aeg-ajalt nad keemikutega midagi rääkisid ja nende käest oli alati 
võimalik küsida, kui füüsikas, keemias või matemaatikas hätta jäin. 
Ja kui keegi hädast üle aitab, siis tekib endal ka mugav 
tunne ja ei saa vastu näppe.

\question{Miks sa keemiku teed ei 
läinud?}

Arvutid olid põnevamad. Kui kord näpp oli antud, siis põnevus järjest kasvas.

\question{Too HP-85 ei saanud ju väga kauaks põnevaks jääda?}

Sellel oli oma võlu. Sai natuke trips-traps-trulli-laadseid 
mänge kirjutada ja esimese hea edukogemuse kätte. Edasi sai 
järgmistesse kohtadesse, kus olid veidi ägedamad ja võimsamad arvutid.

Lisaks keemiahoonele\index{Tartu Ülikool!Keemiahoone} oli Tartus olemas 
füüsikahoone\index{Tartu Ülikool!Füüsikahoone}, mille keldrikorrusel tegutses 
Alo Raidaru\index[ppl]{Raidaru, Alo}, kellel olid aadressil Tähe 4 PC-laadsed arvutid. Nendega sai juba teha palju rohkemat kui selle väikse 
õnnetu HP-85ga\index{HP-85}. Lõpptulemusena võeti mind umbes 
kümnendas klassis füüsikahoonesse nii-öelda laborandina tööle. 
Tööülesandeks oli üht, teist, kolmandat või neljandat programmeerida. 

\question{Kust sul programmeerimisoskus tuli?}

Ma ei tea, tuli järjest. Kõik aitasid ja õpetasid kõrvalt ning see kasvas kuidagi 
naturaalselt.

Loodan, et tegin seal midagi kasulikku. Laborandi palk oli viiskümmend 
rubla kuus, mis tekitas kümnenda klassi poisis kröösuse tunde. Eks 
palk natuke toetas mu huvi ja 
enamasti läksin pärast koolitunde kodu asemel füüsikahoonesse.

\question{Mis selle valdkonna sinu jaoks põnevaks tegi?}

Põnev oli see, et kui tegin arvutile mõne programmilaadse asja 
selgeks, siis arvuti tegigi midagi, mida ma ootasin. 
Mõnikord ei teinud ka, aga väga tihti tegi ja see oli jube kihvt, kui midagi 
juhtus. Mingi asi allus minu korraldusele -- täiesti unikaalne situatsioon 
maailmas!

\question{Oskad sa tuua mõnd näidet, mida sa laborandina progesid?}

Üks asi, mis meelde tuleb, on antiviirus. 

\question{Antiviirus 1980ndatel?!}

Umbes aastal 1986. Maailm hakkas vaikselt lahti minema ja 
tasapisi ilmusid Eestisse viirused. Tekkis see jama, et viirus jõudis ka
meie juurde ja sellest oli vaja kuidagi lahti saada, kuna arvutid hakkasid 
imelikult käituma. Viiruse nimi oli, kui ma õigesti mäletan, Yankee 
Doodle,\sidenote{Internet ütleb küll, et Yankee Doodle avastati 1989. 
aastal ja tähti kukutav Cascade 1988. aastal.} mis tegi piikse ja ekraanil 
hakkasid vist tähed kukkuma. Sai uuritud, kuidas see käitub, ja tehtud pisike 
programmikene, et sellest lahti saada.

Põhiliselt tegime Alo Raidaru\index[ppl]{Raidaru, Alo} laboris füüsikutele
elektroonikat: lisasid mõõteseadmetele, 
katseeksperimentidele ja nii edasi. Seoses sellega tegid nad ise ka trükkplaate. 
Tükkplaatide tegemiseks olid esimesed SmartCADi-laadsed programmid, millega 
õnnestus joonistada elektroonikaskeem ja trükkplaat ning see ka välja 
printida. Alo ehitas arvuti külge freespingi juhtimise \emph{interface}'i, mis 
freesis trükkplaadi välja. Lisaks väljafreesitud trükkplaadi 
radadele oli vaja puurida läbiviigu augud ja see 
puurimisprogramm usaldati mulle.

\question{Moodsas terminoloogias tegelesid sa järelikult kohe IoT-ga!}

Seda võib tänapäeval IoT-ks nimetada, aga tegelikult oli see trükkplaatidesse 
aukude puurimine. 

\question{Nimetame siis robootikaks!}

Nimetame asja ikkagi õigete nimedega. Tegelesin puurpingi puuri õigele kohale 
viimisega ja siis käsu andmisega, et mine alla ja tule üles tagasi.

\question{Kuna tõenäoliselt mingisuguseid teeke või draivereid ei olnud, siis kas
sinu programm käis riistvarani välja?}

Põhimõtteliselt küll, CADi programmist sai aukude koordinaadid 
ja nende koordinaatide peale tuli augud puurida. Vahepeale sai 
tehtud puuride liigutamise keel: astu sada sammu siiapoole, 
mine alla, tule üles, astu sada sammu sinnapoole.

\question{Kas selle keele mõtlesid ka sina välja?}

Vanemad inimesed kõrval aitasid -- nii kui kinni jooksin, tuli 
keegi appi. Hästi turvaline kasvukeskkond.

\question{Teised on rääkinud, et said üsna varakult teiste omasugustega 
ninapidi kokku, vahetati infot ja tekkis kogukonna moodi asi. Kas sul seda ei 
olnud?}

Kooliajal ei olnud, aga ülikooli astudes tekkis kogukonna tunne 
üsna kohe, esimesel kursusel.

\question{Kas laboranditöö õppimist segama ei hakanud?}

Ei, otseselt ei hakanud. Kuna nupp natukene lõikas, siis võis mõne koha pealt 
üle nurga lasta. Koolis lõputult pingutada ei olnud vaja, võibolla eesti 
keele kontrolltööd läksid kehvemaks, aga üldtase jäi nelja juurde. Kooli lõputunnistusel olid kõik neljad, välja 
arvatud üks viis ja üks kolm. 

\question{Mis aine kolm oli?}

Enam ei mäleta. Võimalik, et vene keel.

\question{Mis ülikooli sa läksid?}

Tartu Ülikooli, füüsikateaduskonda.

\question{Kas sellepärast, et sul oli seal laborandina juba käsi sees?}

Laborandina oli käsi sees ja tegelikult füüsika kui nähtus huvitas ka, 
oluliselt rohkem kui matemaatika. Füüsikas oli keerukusaste 
väiksem selles mõttes, et matemaatikud läksid teise või seitsmenda 
tuletiseni välja, samas kui füüsikud ütlesid: \enquote{Teine tuletis on nii 
ebaoluline, et seda efekti sellel kursusel ei aruta, jääb kolmanda kursuse 
materjaliks,} ja see sobis mulle.

\question{Ma just mõtleksin vastupidi, et füüsikas on päris 
maailm kogu oma keerukuse ja ebakorrapärasusega?}

Ei, vastupidi. On suhteliselt lihtsad rusikareeglid ja kui nendest 
aru saad, tuleb peenhäälestamine selle
peale ja, nagu ma ütlesin, siis alles järgmisel kursusel, nii et
esialgu võib selle kõrvale jätta.

\question{Kas sul tekkis spetsialiseerumine ka?}

Spetsialiseerumisega läks natuke sandisti, sest 
Vene sõjavägi tuli peale. Hiljem küll jätkasin füüsikas kaks 
pool aastat, aga siis tuli ka muu elu kõrvale ja õppimisvaimustus vaikselt hajus.

\question{Kus sa teenisid?}

Valgevenes, sellises superkohas nagu Borissov 13. 

Mõnes mõttes oli see ajaraiskamine, aga teisalt nägin maailma ja seda, kui 
palju erinevaid inimesi on tegelikult olemas.

\question{Arvutiinimesele oli see ilmselt üsna silmiavav!}

Ilmselt jah -- kui paljud meist 1988. aastal, kui ma sõjaväkke läksin, 
tegelikult reisinud olid? Võibolla Nõukogude liidu piires siin-seal 
käinud. Nii et tagantjärele mõeldes oli uute inimeste nägemine sõjaväes tegelikult päris kasulik kogemus.

\question{Millal sa sõjaväest tagasi tulid?}

Tagasi tulin aastal 1989. Mul õnnestus Vene sõjaväest pääseda ühe aastaga, kuna 
Gorbatšov ütles, et üliõpilased on meie sotsialistliku riigi tulevik ja mingu 
parem õppigu ülikoolis edasi, mitte ärgu jooksku püssiga ringi. 

\question{Mispeale tulevik läks Tartu Ülikooli füüsikat edasi õppima?}

Tulevik läks jah Tartu Ülikooli füüsikat edasi õppima ja proovis 
spetsialiseeruda astronoomiale.

\question{Miks just astronoomiale?}

See oli juhus. Tõenäoliselt sa tead sellist ulmekirjanikku nagu 
Isaac Asimov ja suure tõenäosusega oled kuulnud, et lisaks oli 
ta hea teaduse populariseerija. Koduses 
raamaturiiulis oli mul raamat 
\enquote{Universum}.\sidenote{Isaac Asimov. The Universe: From Flat Earth to 
Quasar, 1966. Venekeelne tõlge 1969.} See oli küll venekeelne tõlge pealkirjaga \enquote{\begin{russian}Вселенная\end{russian}}, 
aga kirjeldas niivõrd fantastiliselt seda, kuidas universum toimib, et see 
jäi kuklas kripeldama. Mõtlesin, et järsku peaks seda teemat edasi uurima, ja proovisin astronoomiale spetsialiseeruda.

\question{Mis tähendab \enquote{proovisid}? Kas ei tulnud välja?}

Otseselt ei tulnud välja, kuna koolieelsest tööelust kasvas välja järgmine 
tööelu, mis hakkas natuke õppimist segama. Seesama Alo 
Raidaru\index[ppl]{Raidaru, Alo} sokutas mind tööle ajalehte 
Edasi\index{Edasi|see{Postimees}}.

\question{Järelikult edasi ei olnud vaja auke puurida?}

Auke puurida ei olnud vaja, aga 1989. aastal tekkisid 
reaalselt ka ettevõtetesse esimesed arvutid ja 
\emph{desktop publishing}. Kuna üks Alo hobidest oli ülikooli teatmiku väljaandmine, siis tema käest küsiti nõu, kas Edasis saaks arvutit 
kasutada kuidagi ajalehe väljaandmiseks. Mind sokutati neid
sinna aitama ja siis aitasingi neli-viis aastat.

\question{Kas abi seisnes ainult \emph{publishing}-programmi 
käimaajamises?}

Ei, mitte ainult programmi käimaajamises. Reaalsuses on teatud töörutiinid ja kui need lähevad lihtsamaks, siis käivad kiiremini. Sulle nüüd 
omakorda küsimus, mis võis olla esimene asi Edasis\index{Edasi}, 
mida aastal 1989 arvutiga automatiseeriti?

\question{Eesti keele spellerit ju veel ei olnud \ldots}

Eesti keele speller oli juba olemas, aga see selleks. Mis võiks olla see 
teema, mida automatiseeriti?

\question{Ei oska öelda!}

Väga lihtne, see oli see valdkond, kust ajalehte raha tuli -- surmakuulutused. 
Tänaselgi päeval on Postimehes populaarsed paar eelviimast lehekülge, kus on 
surmakuulutused. Neil on see hea omadus, et on suhteliselt standardses 
formaadis: neil on kümmekond erinevat 
kujundust ja kui need kuidagi mallideks teha, siis 
surmakuulutuse publitseerimise aeg vähenes drastiliselt. Kuna see oli 
sisuliselt ainuke allikas, kust lisaks tellimustele raha tuli, siis selle vastu 
oli lehe juhtkonnas päris korralik huvi.

\question{Mis väljundisse need mallid läksid?}

Väljundiks oli kilele trükitud lehekülg, mis läks siis ofsettrükki.

\question{Kas sinu tarkvara optimeeris selle otse kilele?}

Jah, laserprinteriga lased paberi abil läbi kile ja selle, mis sealt välja tuleb, 
saad trükkalitele anda, et kleepige õigesse kohta.

\question{Teisisõnu produtseerisid sa PostScripti?}

Jah.

\question{See on ju päris keeruline!}

Ventura Publisher, see \emph{publishing} tarkvara, tegi põhitöö ära ja mõned üksikud asjad vajasid otseselt 
PostScripti tasemele minekut. 

\question{See kõik eeldab jälle teadmisi, kust sa neid juurde hankisid?}

Istusin ja nokitsesin. Küll ta lõpuks tuleb, kus ta pääseb!

\question{Mainisid enne, et sul tekkis midagi kogukonnalaadset.}

Üliõpilasena käid ju ikka seltskonnaga ringi. Proovid ühes ja
proovid teises arvutiklassis. Selleks ajaks olid juba Tartu Ülikooli 
matemaatikateaduskonda\index{Tartu Ülikool!Matemaatikateaduskond} ka 
arvutiklassid tekkinud ja sealsete inimestega suheldes see kogukond vaikselt 
tekkis. Samamoodi tekkis kogukond füüsika 
kursusekaaslastest.

\question{Kas tol ajal arvutisidet ei olnud?}

Tol ajal veel ei olnud, aga eks see tuli ka suhteliselt kiiresti. 
Ühtedel meestel oli ühtelaadi arvuti ja teistel teistlaadi arvuti, mis 
omavahel flopikettaid ei lugenud. Siis pandi kaks-kolm traati kokku ja 
prooviti kuskilt saadud programme teisele mehele ka üle kanda.

\question{Kuidas Edasis töövoog välja nägi? Millega ajakirjanik teksti kirjutas?}

Edasi\index{Edasi} aegadel kirjutas ajakirjanik ikkagi kirjutusmasinaga ja oli 
tinaladu. Aga kui tekkis rohkem arvuteid, mindi tinalaolt üle kilele 
trükitud väljundile. Seal vahel oli veel terve hunnik etappe, enne kui ajakirjanikud 
arvutid said. Arvuti oli tol ajal suhteliselt kallis, terminalid 
natuke odavamad. Postimehes\index{Postimees}, mis oli siis juba erastatud ja 
Postimeheks muutumas, sai pandud üles üks Unixi server, mille küljes oli 
kuusteist terminali, mis sai ajakirjanikele maja peale laiali veetud. 
Terminalide ühenduseks vajalikud kaardid sai Tõraverest, seal oli 
Urania\index{Urania|see{Astrodata}}-nimeline firma, millest kasvas välja 
Astrodata\index{Astrodata}.

Teksti sisestamiseks oli terminal ajakirjanikule piisavalt lihtne, seda ei 
olnud vaja ilusaks ajada, peaasi, et tekst oli olemas. Kui tekst oli valmis, 
pandi see \emph{publishing}-tarkvarasse ja lasti kile peale välja. Kiled 
kleebiti kleeplindiga kokku küljeks, mis läks öösel trükikotta. 

\question{Mis Unix seal serveris jooksis?}

BSDi Unix\index{Unix!BSDi Unix}, mis sai täiesti ausalt ostetud, \emph{source} 
koodi ja kõige muu värgiga. 

\question{Tol ajal oli ju lausa embargo, kuidas te selle serveri hankisite?}

Jah, embargo oli, aga need 386-laadsed arvutid embargo alla ei 
kuulunud. Ülemine ots, nagu PDP\index{PDP},\sidenote{\emph{Programmed Data Processor} (PDP). Üldnimetus Digital Equipment Corporation'i poolt 1957--1990 toodetud mitmetele miniarvutite sarjadele.} oli embargo all. 

Need kuusteist terminali jaksas rahulikult välja vedada. Arvutite hankimine oli
tollal keerukas. Kui Postimees sai tellimuste 
rublad kätte, veeti need kohvriga oskuslike ärimeeste juurde, kes said 
Moskvast mingi arvuti. Siis oli niisugune vorstikauba aeg. 
Igatahes lõpuks oli võimalik vajalikud arvutid välja ajada.

\question{Rääkisime ka Veiko Tammega\index[ppl]{Tamm, Veiko} sellest 
pikalt.\sidenote{Vt lk \pageref{sisu!veiko_moskvas}.}}

Just. Mõned kohvrid jõudsid tema juurde ka, ta oli põhiline Postimehele 
arvutite hankija.

\question{Mind paneb jällegi imestama, kui sujuvalt sinu skoop laienes. Kui 
programmeerimisest ma saan aru, siis nüüd on teemaks Unixi serverid, töövood, 
võrgud ja nii edasi. Mis hoidis sind seda ringi laiendamas? Oleks ju olnud 
lihtne programmeerimisele või millelegi muule keskenduda!}

Arvan, et see ümbritsev seltskond. Edasi või Postimehe ajakirjanikel läks silm särama, kui tekkis niisugune võimalus
oma tööd paremini teha, ja see tekitas soovi neid kuidagi aidata. 
Kui teisel inimesel silm särab midagi tehes, läheb endal ka silm särama 
teda aidates. Nii lihtne see ongi.

\question{See eeldab ka huvi inimeste vastu.}

Absoluutselt. Kui sa igapäevaselt kellegi kõrval istud, tekib see 
huvi tahes-tahtmata. Ei ole võimalik, et ei tekiks. Ja kui istud veel 
intrigeerivate inimeste kõrval, kes hoiavad kätt elu pulsil ja 
räägivad sulle: \enquote{Oh, Tallinnas Toompeal räägiti seda ja toda, ja kas 
paneme selle lehte või ei pane?}, hakkab kõrv liikuma küll ja tahad selle melu 
sees olla.

\question{See võis äge aeg olla küll, igasuguseid väljaandeid hakkas ilmuma!}

Tartlasena ma kõiki Tallinna asju ei tea, aga ilmuma hakkasid Eesti Ekspress ja 
Liivimaa Kuller Kalle Mülleri\index[ppl]{Müller, Kalle} ja Väino 
Koorbergi\index[ppl]{Koorberg, Väino} vedamisel. Samuti Kroonika, algul Kalle 
Mülleri, siis Ingrid Veidenbergi\index[ppl]{Veidenberg, Ingrid} vedamisel. 
Kõigi nende juures olid megakihvtid ja huvitavad momendid. 

Alguses oli väljaanne mustvalge, siis tekkis värviline logo ja nii edasi.

\question{Värviline logo oli suur asi! Kõigi väljaannete juures oli ilmselt mõni
\emph{publishing}'u või trükiinimene ametis. Kas Peeter 
Marvet\index[ppl]{Marvet, Peeter} tembutas juba ka kuskil?}

Tõenäoliselt Tallinnas tembutas, aga Tallinn ja Tartu on täiesti erinevad asjad.

\question{Ma seepärast küsingi, et kas sedalaadi inimestel oma kogukonda 
ei tekkinud, et näiteks programme vahetada?}

Kindlasti oli, aga seda teemat ma enam ei mäleta, kuna pärast tuli igasuguseid muid asju nii palju peale.

\question{Ühel hetkel sai Edasi asi otsa, mida sa edasi tegid?}

Enne kui Edasi otsa sai, oli üks huvitav moment, millest 
tahaksin rääkida. 

Edasiga seoses sain ma umbes aastal 1989 või 1990 endale meili, mille aadress oli 
\verb|taavi@pm.ew.su|.

\question{Kas EW nagu \enquote{Eesti Wabariik} ja SU nagu \enquote{Soviet Union}!? 
Kas selline domeen oli olemas?}

Jah, niisugune domeen oli olemas ja domeen SU on endiselt alles.

\question{Kust sa selle aadressi said?}

Mõnes mõttes tänu ülikoolile, kuna psühholoogia 
teaduskonnast\index{Tartu Ülikool!Psühholoogia teaduskond} Tiit 
Mogom\index[ppl]{Mogom, Tiit} oli Tallinnas kas Küberis\index{Küber} või 
KBFIs\index{KBFI} käima ajanud modemiga UUCP 
meiliside. Sealt see meiliaadress tuli. 

Sellest ajast mäletan veel esimest suuremahulist meili umbe 
ajamise intsidenti. Olid meililistid, mida sai tellida, ja 
ma tellisin kogemata ühe aktiivsema kirjavahetusega meililisti. Meile 
muudkui tuli ja tuli ja modem ei pannud toru hargile. Läks tund, teine ja kolmas, \enquote{no täitsa pekkis, kogu see 
maailm on umbes ja läheb katki!} Võtsin jalad selga ja kõndisin üle Toome 
Tiidu\index[ppl]{Mogom, Tiit} juurde: \enquote{Kuule, aita mul see meilivoog 
ära katkestada!} Umbes oli ju modemiga helistamine minu juurest tema juurde, 
tema juurest kuhugi Tallinnasse ja Tallinnast veel kuhugi Soome.

\question{Kusjuures tänapäeval on täitsa unustatud, kui oluline asi oli 
meil -- kõik asjad käisid üle selle. Oli olemas FTP üle meili, kus 
failid keerati sobiva pikkusega juppideks, lasti \mbox{BASE64ga} kokku ja saadeti 
meili peale!}

Täpselt. Sellisel kujul oli võimalik list- või 
arhiiviserveritest endale tellida vaba tarkvara lähtekoodi. Levis muidki asju, 
aga meid huvitas just vaba tarkvara lähtetekstide kättesaamine.

\question{Miks see huvitas?}

Siis sai jälle mingit uut ja võimsamat asja teha!

Edasis tuli järjest uusi automatiseerimisi peale. Ühel hetkel 
tehti oma kojukanne. Selle puhul oli oluline, et postiljonidele 
jagataks pakid sihtrajoonide järgi, õiged kleepsud oleksid peal, õigesse 
hunnikusse saaks õige kogus ajalehti ja et neil oleksid nimekirjad, mille järgi 
viia. Sai tehtud kojukande infosüsteem. 

\question{Jällegi, tänapäeval sellele ei mõelda!}

Eesti Postis või Omnivas on kojukande infosüsteem raudselt olemas!

\question{Just! Aga tänapäeval ei juhtu just sagedasti, et astud uksest 
sisse ja hakkad nullist sellist infosüsteemi programmeerima. Tüüpiliselt on 
midagi juba olemas.}

Sel ajal ei olnud võimalik midagi aluseks võtta, sest mitte midagi lihtsalt ei 
olnud. Äriliselt oli Postimehel ainuke võimalus ise kojukandesüsteem teha, 
kuna riiklik kojukanne ei toiminud. Ajaleht viidi kätte lõunaks, aga Postimees 
tahtis, et hommikuse kohvijoomise ajaks oleks leht olemas, ja ehitas nullist 
üles oma kojukandesüsteemi, mis pärast vist liitus Express Posti omaga. 
Praegugi on see vist alternatiivse kojukandesüsteemina teatud ulatuses 
toimiv.

\question{Ma mäletan seda küll, sest see, et Postimees oli hommikul vara 
postkastis, oli nagu \enquote{läänes}, nagu \enquote{päris}! Mida sa 
meiliga peale listide lugemise veel tegid?}

Kaks asja on eredalt meeles. 1991. aastal oli 
Moskvas putš: vahetati valitsust, tankid olid
teletornide ees ja mis kõik veel. Oli suur infoauk, mis toimub ja kus toimub. 
Siis sai Moskva arvutihäkkeritega kokku lepitud, et teeme 
otseliini, paneme infolistid käima. Jube põnev oli saada toimiv ja 
ajakirjanikele kasulik info meili teel kätte natuke enne, kui see teab 
kust tekkis. 

\question{Ja sul olid need kontaktid olemas?}

Jah, ühel Unixi kasutajate konverentsil käimisest Moskva taga Vladimiris. Muide, seal olid kohal esinejad Berkeley ülikoolist, 
kaks Berkeley Unixi loojat või guru. Kui õigesti mäletan, oli üks neist 
Keith Bostic.\sidenote{Berkeley Software Distributioni (BSD) 
ajaloo üks võtmetegijaid, kelle kõikvõimalikku panust vaba tarkvara ajalukku on 
raske üle hinnata.} Selles mõttes ei tasu naerda -- venelased suutsid need mehed 
enda juurde meelitada, tõenäoliselt oli neil ka huvitav ja konverents oli 
megakihvt.

\question{See võis tõesti olla tol ajal väga kõva sõna!}

Jah, oli küll. Tagantjärele mõeldes tekitas see jälle tunde, et arvutiteema on hea 
teema: hoiad näppu pulsil ja oled suhteliselt lähedal sellele, mis 
maailmas ägedat toimub.

Ülikoolis tekkis ka esimest korda moment, kui nähti, et meilindus on päris 
kihvt asi ja et lisaks sellele on olemas püsiühendus ja muu säärane. 1992. 
aastal pani Jaak Lippmaa\index[ppl]{Lippmaa, Jaak} 
Tallinnas püsti taldriku Rootsi Tele-Xiga ja Tartu tähetorni\index{Tartu 
tähetorn} sai satelliidi abil püsiühenduse, mis oli 64 kilobitti. Selleks et 
ühendus tähetornist Toomel kuhugi mujale ka leviks, sai Postimehe 
eestvedamisel tähetornist kõigepealt keemiahoonesse\index{Tartu 
Ülikool!Keemiahoone}, sealt ülikooli peahoonesse ja edasi Postimehe majja 
üle katuste veetud Tartu esimene püsiühendus. See oli ehitatud vene 
sõjaväelaste käest 500 rubla eest ostetud ülejäänud kaabli rullist.
Selle kaabli ümber tekkis ka nii-öelda internetikommuun. 

\question{Sest need, kes tee peale jäid, said ka endale interneti?}

Absoluutselt. Ja see oli täiesti \emph{online}!

\question{Aga latents pidi äge olema?}

Kuskil kuussada millisekundit.

\question{Nii hull ei olnudki!}

64 kilobitti ei ole tänapäeva mõistes mingi superkiirus, aga 
tollal, kui internet oli veel tühi igasugustest kassipiltidest, oli see 
täitsa okei.

\question{Kes see kogukond oli, kes kaabli ümber kogunes?}

EENeti\index{EENet} inimesed, Enok Sein\index[ppl]{Sein, Enok}, 
Anne Villems\index[ppl]{Villems, Anne}, Richard Villems\index[ppl]{Villems, Richard}, 
Tiit Mogom\index[ppl]{Mogom, Tiit}, Marek Tiits\index[ppl]{Tiits, 
Marek} Balti Uuringute Instituudist ja tõenäoliselt oli neid inimesi 
veel, kes praegu ei meenu.

\question{Kas sel ajal oli EENet juba olemas?}

EENeti veel ei olnud, aga tuumik oli seesama, kes kogukonna 
moodustas ja kes nägi vaeva selle nimel, et interneti püsiühendus oleks 
olemas. Lisaks kogunes modemiga sissehelistajaid ja kasutajate hulk hakkas vaikselt kasvama.

\question{Kas Tartu ja Tallinna side käis
üle satelliidi?}

Jah, alguses käis üle satelliidi, aastapäevad hiljem tuli ka maapealne 
püsiühendus Küberi\index{Küber} eestvedamisel. See on see koht, kus 
Eesti internetimaailmas tekkis ilmselt kaks nii-öelda rististe suuskadega 
kommuuni: Küberi ja KBFI\index{KBFI} oma.

\question{Kas siis ei olnud veel AS Cybernetica, vaid Küberneetika Instituut?}

Jah, aktsiaselts tekkis hiljem.

\question{Miks suusad risti läksid?}

Seda mina ei tea. Tartu inimesena ei saanud ma sellistest Tallinna 
probleemidest lihtsalt aru. Kui ressursiga on kitsas, nagu Eesti Vabariigi 
alguses oli, siis võibolla olid seal rahade jagamise 
või teaduse finantseerimise mured, et kes sai oskuslikumalt finantseerimisele 
ligi. See oli olelusvõitlus, mis jättis kõigile jälje.

\question{Mis edasi sai?}

Postimehe periood sai läbi ja umbes aastakese töötasin Tartu 
Ülikooli raamatukogus\index{Tartu Ülikool!Raamatukogu}, kuhu sai Rootsi kunni 
abiga muretsetud esimene serveri moodi asi ja otsast hakatud kirjutama 
raamatukogu infosüsteemi.

\question{Kas see tähendab, et sinu ajast on seal need 
kuulsad kalanimega serverid?\sidenote{Kui kõigil serveritel oli veel oma nimi, oli 
kombeks anda ühe asutuse serveritele samalaadsed nimed. Näiteks Halo kunagised masinad 
olid nimetatud kuulsate häkkerite järgi: woz, mitnick jne. Kuuldus
sellest kombest levis just läbi Tartu Ülikooli kalanimega serverite.}}

Enam ei mäleta, kilu.nlib.ee\index{kilu.nlib.ee} oli vist 
SPARCStation 2\index{SPARC!SPARCStation 2}. Kusjuures selle serveri põhifunktsioon oli ikkagi esimene ülikooli 
raamatukogu elektrooniline kataloog, mida kohapealsed inimesed ise 
programmeerisid.

\question{Selle süsteemi kasutamiseks olid raamatukogus isegi terminalid. 
Oma aja kohta oli see funktsionaalne ja tore lahendus!}

Jaa, tööd oli sellega kõvasti, andmesisestust on nullist väga raske teha.

Kahjuks või õnneks jäi raamatukoguperiood 
lühikeseks, kuna Jaak Lippmaa\index[ppl]{Lippmaa, Jaak} kutsus mind 
Tallinnasse ja see tundus veel põnevam. 

Ta kutsus mind sellisesse riigiasutusse nagu Valitsusside\index{Valitsusside}, 
umbes sellise mõttega: \enquote{Sina oled nüüd internetiga natuke 
kokku puutunud, oskad ühest arvutist teise sümboleid saata. Eesti riigil tuleb mingisugune 
KGB sidekeskus üle võtta, tule aita!} Ja tulingi.

KGBst võib rääkida mida iganes, aga tehnoloogilise poole 
pealt nägi asi välja suhteliselt õnnetu. Ädala tänava 
majas\sidenote{Ädala 4d. Selles endises KGB sidekeskuses paikneb tänaseni hulk eri organisatsioonide serveriruume.} olid suured saalid täis mittetöötavaid 
telefonijaamu. Võin nüüd natukene valetada, aga seal oli suurusjärgus 200 
töötavat telefoni, mis olid ette nähtud riigiorganite jaoks. Samas kakssada töötavat telefoni kogu valitsuse peale 
on suhteliselt nadi number.

Järgmine projekt oli Siemensi telefonijaamadega sisetelefonijaamade 
võrgu käimapanek Toompeal, Kadriorus, välisministeeriumis ja kes nad seal 
kõik Pikal tänaval olid -- kokku paar tuhat numbrit. Õnneks tuli see täitsa 
edukalt välja ja oli ka reaalne kasu olemas.

\question{Kas see oli veel analoogjaam?}

Võrgustatav digijaam Siemens TopCom. See oli päris sidevõrk.

\question{Kust tuli visioon, et nii kallist jaama ehitada? Odavam oleks olnud analoogjaamadega hakkama saada, aga 
tehti investeering.}

Investeeringu lüke tuli Jaagu isiklikul initsiatiivil, 
tehnoloogilise poole pealt konsulteeris ta Aavo Pikofiga\index[ppl]{Pikof, Aavo}, 
kes oli Tallinna Telefonivõrgus. Sinna taha tulid tänu Jaagu ja Endel 
Lippmaa\index[ppl]{Lippmaa, Endel} tutvustele ka riigi funktsioonid. Seda oli 
tõepoolest vaja, visioon osteti ja finantseeriti ära. Tänu sellele saadi reaalselt 
töötav sisetelefonivõrk, aga eks see jäi mõne aja pärast ajale jalgu.

\question{Kas füüsiline kaabeldus oli Eesti Telefoni oma?}

Osa oli Eesti Telefoni käest ja osa oli vana KGB kaablivõrk, mida oli 
Tallinnas mõnisada kilomeetrit.

\question{Kuigi keskjaam ei toiminud, olid kaablid ikkagi olemas?}

Olid korraliku tinakestaga kaablid, mis oli umbes nagu tuumasõja üleelamiseks 
ette nähtud ja meenutasid rohkem tanki kui kaablit.

\question{See kõlab projektina, mille käigus kohtub huvitavate inimestega.}

Absoluutselt. Telefonijaama installeerimine ja käimapanek Kadrioru lossis, 
kus Lennart Meri\index[ppl]{Meri, Lennart} vaatas üle õla ja õpetas, kuidas 
telefonipistikuid ühendada, võib tagantjärele muigama panna, aga nii see 
oli. 

\question{Kõlab väga Lennarti moodi!}

Absoluutselt, ta oskas igas asjas nõu anda.

\question{Kui kaua sa valitsuse kaableid vedasid?}

Kokku kolm aastat. Peale esimest edukogemust telefonijaamadega tahtsime loomulikult 
järgmisi edukogemusi. See internetinimeline asi ronis 
kogu aeg uksest ja aknast sisse. Kirjutasime siia-sinna järgmiseid pabereid, et 
nüüd oleks mõttekas sellesse kuidagi investeerida. Sel ajal nõuti juba ükskõik kelle käest, kes natukenegi internetiga tegeles, et \enquote{mina tahan ka}, soovijaid tuli järjest juurde. 

Aga paberite kirjutamine ei olnud edukas ja niimoodi see asi ei toiminud. Siis hakkasin kõiki tuttavaid 
läbi sõitma, et võtaks pundi kokku ja hakkaks 
normaalselt ise tegema. Käisin läbi vist kõik inimesed, kes vähegi 
internetiga tegelesid. Tartlased vedu ei võtnud, seal on nii mugav ülikooli ja Pirogovi juures olla, kuigi Marek Tiits\index[ppl]{Tiits, 
Marek} aitas päris palju ideed formuleerida. Kõige rohkem võttis 
vedu Andres Bauman\index[ppl]{Bauman, Andres} KBFIst\index{KBFI} ja temaga 
koos tegimegi internetiettevõtte. 

Käisin poes, ostsin riiulifirma, mille nimi oli 
Nösper\index{Nösper|see{Uninet}} ja mis praegu on Elisa -- sellesama 
riiulilt ostetud firma ja Elisa\index{Elisa} registreerimiskood on sama. 

\question{Milline juriidiline järjepidevus! Mis aastal see oli?}

See oli aastal 1995.

\question{See oli ju üsna karm aeg -- mitte midagi polnud saada ja kõik asjad 
tuli nullist ehitada!}

Oli küll, aga teisalt inimesed olid leidlikud. Näiteks Data Telekom\index{Data 
Telekom} Neeme Takise\index[ppl]{Takis, Neeme} eestvedamisel ehitas ise RS422 
elektrilise ühenduse peal modemeid. Kui ei olnud, siis tehti ise või kohandati. Midagi oli võimalik igal juhul teha.

Mingisugused lisaboonused olid ka. Tänapäeval inimesed arvavad, et kümne 
euro eest kuus peaks tulema nii jäme internet, kui maailmas üldse olemas on. 
Mis see siis teeb, kolm kohvitassi kuus? Tollal aga oli internet nii seksikas 
ja uus asi, et oldi nõus selle eest maksma. Oli niisugune helge aeg, et ettevõte ostis ise seadmed välja, maksis kinni paigalduse ja tasus veel 
kuutasu ka. Tänu sellele oli 
võimalik teha esimesed seadmeinvesteeringud. Sellise mudeliga nagu täna, et kõik on kümme või viis kohvitassi kuus, ei oleks internet kunagi Eestisse 
jõudnud.

\question{Kui ma õigesti mäletan, siis hakkasin teie edasimüüjana toimetama vist
aastal 1996, mis tähendab, et teil oli üsna varakult lai 
võrk teenuseid ja partnereid.}

Põhiline oli sissehelistamine. Ma ei mäleta, mis süsteemiga sai 
püsti pandud füüsiline modemite \emph{pool} telefoninumbritega, aga sellest see 
tegevus tegelikult pihta hakkas. Tõenäoliselt aitas telefoninumbritega, kuna need 
olid defka, Jaak Lippmaa\index[ppl]{Lippmaa, Jaak} isa Endel 
Lippmaa\index[ppl]{Lippmaa, Endel}, et saaks kuidagi sobiva 
koha sobivas järjekorras.

Sealt hakkas internetiäri vaikselt kasvama, kuna inimesed tahtsid ja huvi 
suurenes. 1997. aasta alguseks oli 
üldkasutatav interneti välisühendus igasuguste puukide poolt, nagu meie 
puukfirma, nii täis aetud, et EENet\index{EENet} tegi otsuse, et
ärikasutajad peavad muretsema endale oma välisühenduse.

\question{Lausa nii hilja? Siis saite üsna kaua akadeemilise traadi peal toimetada.}

Jaa. Internet ei olnud veel päris akadeemilisest maailmast välja pääsenud ja 
eks kõik toimetasid akadeemilise traadi peal. Kuni EENet jalga 
maha ei pannud, oli see täiesti loomulik nähtus. 

\question{Kuidas te siis ühenduse saite? Kas kuhugi Soome?}

Mul oli mingist ajast jäänud kontakt Helsingi telefonivõrguga ja nende 
inimestega koos sai kanal tellitud ning ühendus ja seadmed hangitud. Meil oli 
juba sedavõrd palju käivet, et suutsime enam-vähem isegi väliskanali soliidse
kuutasu kinni maksta, mis oli pea 100 000 krooni kuus. 

\question{See oli tollal jõhker summa!}

See oli väga jõhker summa,\sidenote{Selle raha eest võis endale Kadriorgu 
ühetoalise korteri osta!} kogu see internetitehnoloogia maksis hingehinda. Kuue 
pordiga Cisco ruuter, marki enam ei mäleta, maksis 250 000 krooni, S-klassi Mersu hinna.

\question{Järelikult oli füüsiline kaabel Eesti ja 
Soome vahel olemas?}

Füüsiline kaabel oli olemas. Peale seda, kui Telekom\index{Eesti Telekom} sai 
Eesti riigilt kontsessioonilepingu, vedas ta suhteliselt kiiresti Eesti-Soome 
vahele ka kaks valguskaablit. See võis olla 1996. või 1997. aastal. 
Telial\index{Telia} oli see kogemus olemas -- investeerimisvajadus kogu
analoogvõrgu väljavahetamiseks oli väga selge ja Telial tegelikult 
investeerimisjõudu oli. 

\question{Kas tegite \emph{hosting}'ut ka?}

Jaa, pidasime servereid, et ei peaks traadi raha maksma.

\question{Kuidas see käis? Kui praegu ostan endale virtuaalmasina, siis tol 
ajal sain konto ja parooli.}

Said Unixi \emph{account}'i ja oligi kõik.

\question{See oli legendaarne ettevõtmine ja legendaarne aeg, aga mida sa 
praegu teed?}

Praegu olen sellises huvitavas ettevõttes nagu RebelRoam. 
Tegeleme transpordiettevõtetele wifi-teenuste pakkumisega ja nende 
optimeerimisega. Kliendibaasiks on mööda Euroopat ja Ameerikat 
sõitvad bussid, kus on wifi, mida bussireisijad saavad kasutada. 
Või jõe- ja kruiisilaevad, kus pensionärid ostavad 3000--4000dollarilise nädalase reisipaketi ja kui nad Pariisis Eiffeli torni 
pildistavad, siis peavad õhtul saama saata oma pildi lastelastele -- muidu on 
reisikogemus natukene nõrk. 

See võib tunduda eestlasele imelik, sest meil on internet kogu aeg igal pool 
vabalt kättesaadav, aga näiteks Prantsusmaal, Inglismaal või Ameerikas ei 
ole 4G ega 5G nii levinud, et selle kvaliteet oleks kõigile piisav. Seal 
on teenuse optimeerimise vajadus täitsa olemas.

\question{Järelikult kõik need inimesed, kes Ameerika kiirteel Greyhoundi 
bussi järel sõidavad, et wifit kasutada, on teie kliendid?}

Greyhound on jah meie klient. Ma ei tea küll, kas kõikidel bussiliinidel või rohkem 
lääneranniku pool.

\question{Paistab, et ring on täis saanud. Kui sa Edasi 
puhul rääkisid, kuidas sulle tegi rõõmu sära ajakirjaniku silmas, siis nüüd 
tõid ka kohe esimese näite Eiffeli torniga ja et lõpuks tegeled ikka inimeste 
rõõmustamisega.}

Kui sa teda rõõmsaks ei tee ja ta ostab sinu teenust vihaga, siis ega ta sulle raha
ka ei maksa. Aga kui teed rõõmsaks, siis tõenäoliselt laekub ka raha 
pangakontole. Ja on rõõm endal ka midagi paremaks teha.

\question{Kas koodi veel kirjutad?}

Jah.

\question{Kas võrku veel konfid ise?}

Võrku ise ei konfi, ma vist ei oskaks seda enam korralikult teha. Võrk on 
tänapäeval niivõrd ära virtualiseeritud, et enam ei oskaks konfigureerida.

\question{Mina küll märkasin, kuidas keegi küsis selle kohta Facebookis ja sa mitte ei 
vastanud, kuidas saab teha, vaid kirjutasid ifconfigi käsurea, mis võtmest 
töötas!}

See on oma arvuti, mitte võrgu konfimine. Ma pidasin silmas
ikkagi päris võrku, kus on jämedad ruuterid ja vilkuvad sinised LEDid ja kust käib 
läbi pool Eesti interneti \emph{traffic}'ust. See on võrgu 
konfimine! 