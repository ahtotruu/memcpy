\index[ppl]{Kaal, Madis}
\question{Kuidas arvuti Saaremaale sai?}
Arvuti saanudki Saaremaale. Minu esimene kokkupuude päris arvutitega oli 
Rahvamajanduse Saavutuste Näitusel\sidenote{Tänapäeva mõistes oli tegu 
messikeskusega, kus kas permanentsete või ajutiste näituste abil demonstreeriti 
kas liiduvabariigi (nagu Tallinnas asunud näituse puhul) või kogu Nõukogude 
Liidu majanduslikku võimekust. NSVLi Rahvamajanduse Saavutuste Näitusest arenes 
välja Eesti Näituste Messikeskus}. Toimub seal Pirita näitusehallis. Käisin 
seal kohal, oma emaga veel sealjuures. Ja ühes nurgas olid üles pandud 
terminalid. Terminale manageeris kaks imeilusat tüdrukut. Imeilusat, ütleme, 
ajaloolises perspektiivis, tõenäoliselt oli tegemist üsna keskmise 
operaatoriga, aga imeilusad ja targad tundusid. Seal terminalide peal oli mingi 
Nõukogudeaegne raamatukogude otsingu andmebaas, venekeelne. Terminalid olid ka 
venekeelsed. See oli esimene kord, kui ma reaalselt nägin, et ekraani peal olid 
tähed ja oli klaviatuur ja sai kirjutada. 

\question{Mis aastal see oli?}

1983, ma arvan. See jättis kustumatu mulje, võib öelda. Terminalide pärast, 
muidugi. 

\question{Pärast seda tekkis sul selge soov, et sa pead nende terminalide 
juurde pääsema?}

Pärast seda tekkis kuidagi väga selge mõte, et see on asi, mis on huvitav. 
Sellele järgnevalt sattusin ma kuidagi Tartusse, ostsin sealt venekeelse 
raamatu \enquote{Programmeerimine keeles PL/I\index{Keeled!PL/I}} ja lugesin 
seda. Ma ei teadnud arvutitest veel midagi, aga tasapisi hakkas selgeks saama, 
misasi see programmeerimine on. Sihukesed asjad nagu \verb|for| ja. Ma ei 
mäleta täpselt, see oli mingi imeline keel. Ta ei olnud päris vene keeles, 
kõlab nagu mingi varastatud versioon. Mingi struktuurkeel põhimõtteliselt, ma 
olen üsna kindel, et see oli mingi piraaditud asi.

Järgmine kord, kui ma arvuteid nägin, oli Tehnikaülikooli\index{Tallinna 
Tehnikaülikool}, tolleaegse TPI\index{TPI}, lahtiste uste päeval. Käisime 
pinginaabriga, kellega me pärast ka kooli sisse astusime, lahtiste uste päeval. 
Tehti meile ekskursiooni Automaatikateaduskonna kõigis 
kateedrites\index{Tallinna Tehnikaülikool!Automaatikateaduskond} neljal 
korrusel ja mõnes kohas olid arvutid. Mäletan selgelt, et Indrek 
Saul\index[ppl]{Saul, Indrek} oli minu meelest sel ajal tudeng, pärast oli ta 
meil kinnisvara-ärimees, ja tema näitas meile analoogarvutit. Põhimõtteliselt 
sa saad analoogpingetega ja skeemiga diferentsiaalvõrrandeid lahendada, ütleme 
siis nii.

\question{Vanasti sihiti ju õhutõrjekahureid analoogarvutitega}

See masin võis täiesti olla sedalaadi projekti osa. Tekkis kindel soov, et 
peaks õppima minema seda. Ja pinginaaber veenis mind ümber, et lähme parem 
raadiotehnikat õppima, sama maja ikkagi.

\question{Kas sinna esimest korda arvuti nägemise ja sisse astumise vahele veel 
midagi jäi arvutitega tegelemise mõttes?}

Ainult see üks raamat. Otsus arvuteid õppima minna sündis kohe, raamat tuli 
juba pärast seda. Ainuke imelik asi oli see, kuidas ma sinna raadiotehnikasse 
ikka otsustasin minna, aga see selleks. Selle vea parandasin ruttu ära. Sain 
ülikooli sisse. Siis oli teise korruse otsas arvutus-saal, kus oli kaks või 
kolm SM-nelja\index{Arvutid!SM-4}. Need olid PDP-11\index{PDP-11} vene 
versioonid. Pärast seda, kui ma sain aru,  kuidas sinna sisse saab, ma enam 
tundidesse ei jõudnud. Ja kuna maalt tulnud poiss ja raha ka üldse ei olnud, 
käisin lihtsalt kõik kateedrid läbi, küsisin iga ukse vahelt, kust sisse 
pääsesin, kas teil tööd on anda. Raadiotehnika kateedris\index{Tallinna 
Tehnikaülikool!Automaatikateaduskond!Raadiotehnika kateeder}\label{sisu!mast_raadiotehnikas} oli anda tööd, 
võeti mind sinna laborandiks tööle ja nii see läkski. Kool jäi pooleli, jäin 
sinna seitsmeks aastaks paika.

\question{Aga mis kursuse peal kool sul pooleli jäi?}

Esimesel kursusel. Algul olin Raadiotehnika kateedris laborant  ja pärast 
tehnik. Sattusin kummalisel kombel tuppa, kus olid väga toredad inimesed. Mart 
Palmas\index[ppl]{Palmas, Mart}, tema õpetas mulle kõik, mida ma 
programmeerimises tean, peaaegu. Ja siis oli seal veel Villem 
Vannas\index[ppl]{Vannas, Villem}, kes on Datelis\index{Datel} tänapäeval. Tema 
õpetas mulle enam-vähem kõik, mis ma rauast tean.

\question{Siis ju ei saanud haridus pooleli jäetud?}

Nojah, aga formaalselt siiski.  Igasugu huvitavaid projekte oli, aga tol ajal 
eriti see laborandi töö oli rohkem selline, et see oli nagu abitööline. 
Parandasid seda, mis oli vaja, aitasid seal, kus oli vaja. Mu esimene töö oli 
kolikamber tühjaks tõsta.
Algusaegadel oli selliseid üsna suvalisi projekte, hiljem tekkis nagu mingi 
suund  kommunikatsiooni poole. See tundus mulle sel ajal ka huvitav. 

Ja siis umbes üheksakümnenda aasta paiku, natuke siia-sinna, tekkis kohe nagu 
mitu sihukest huvitavat suunda Eestis. Kõigepealt hakkas tulema 
personaalarvuteid. Sinnasamasse, kus oli kunagi see SM-4 arvutiklass, tekkisid 
personaalarvutite klassid. Seal oli  neid mitu tükki ja erinevate portsudena 
toodi Austraaliast MicroBeesid\index{Arvutid!MicroBee}\sidenote{1982. aastal 
Austraalias algselt komponentide komplektina müügile tulnud koduarvuti. Tuntud 
huvitava videolahenduse ning patareitoitel mälu poolest. Neist viimane 
võimaldas arvutit teisaldada mälu seisu kaotamata}. Kuskilt tuli MSXi 
arvuteid\index{Arvutid!Yamaha MSX} terve klassi jagu. Siis tulid mõned 
Robotronid\index{Arvutid!Robotron}\sidenote{Robotron (originaalis VEB Kombinat 
Robotron) oli Ida-Saksamaa suurim arvutitootja}, selline kummaline komplekt. 
Aga Raadiotehnika Kateedris oli juba siis, kui mina sinna sattusin, olemas 
Apple II\index{Arvutid!Apple II}. Ja mõned aastad hiljem tekkis IBM 
PC\index{Arvutid!IBM PC} sinna. See oli omapärane kogemus. Apple II peal olid 
harjunud, et lülitad sisse ja pilt on ees. IBM-i lülitad sisse, midagi ei 
juhtu. Üks hommikul, või päeval, nagu ma tööle tulin, tulen tööle, vaatan, uus 
arvuti. Lülitan sisse. Midagi ei juhtu. Vaatasin natuke aega, lülitasin välja 
ja tegin näo, et midagi pole toimunud. Hiljem selgus, et ta tegi \emph{self 
test}i. Seal oli tublisti mälu sees ja testimine võttis ikka tublisti aega, ei 
suutnud oodata nii kaua. 

\question{Ta pidi ju ekraanile midagi näitama nii kaua?}

Seal oli veel see roheline long-fosfor monitor ka selline, see läks tükk 
aega käima ja ei jõudnud mina esimese \emph{boot}imise ajal ära oodata, millal 
midagi toimuma hakkab.

Üheksakümnendate paiku, ma ei mäleta, mis järjekorras, aga mingi projekti 
raames tekkis meile modem. Meest mäletan, kes kohale tõi, aga mingit tausta, 
mille raames see meile toodi, ei mäleta. Tuhande kahesajane modem, läks 
sinnasamasse PC sisse. Ja sel ajal olid just tekkinud esimesed BBSid. 
Ja umbes samal ajal siis toimus sihuke projekt, kus TPI 
automaatikateaduskond\index{Tallinna Tehnikaülikool!Automaatikateaduskond} 
otsustas, et tuleb arvutivõrk ehitada. Toodi kohale viiesajameetrine kaablirull 
kollast sõrmejämedust Etherneti kaablit ja umbes kümmekond komplekti nendest 
kobakatest, mis sinna kaabli peale käis. Mille külge käis teine jäme kaabel, 
mis läks siis võrgukaardi taha. See oli nagu esimene Etherneti tehnoloogia. 
Mäletan selgelt, et meile toodi ainult see kaabel ja need kobakad. Ei toodud 
tööriistu, ei toodud mingeid pistikuid ega terminaatoreid.

Ja siis tõmbasime selle Etherneti kaabli ära. Eterniidi tahvlitest lagi oli sel 
ajal seal, sinna lae peale ajasime. Ja siis et need  kobakad sinna külge saada, 
siis lihtsalt naaskliga augud sinna kaabli  kesta siss, nõel läbi ja 
arvutite külge ja terminaatorid, takistid, tinutasime otsa. 

\question{Tarmo Mamers\index[ppl]{Mamers, Tarmo} rääkis, kuidas te PC ja Maci 
vahele mingit traati vedasite, kas too kaabeldamine oli enne või pärast seda?}

See oli meil kahe PC vahel tegelikult. Sellesama Raadiotehnika PC ja Tarmol oli 
all mingi natuke änksam, vist juba AT arvuti. Ja siis tõmbasime lihtsalt serial 
kaabli, et saaks kokku leppida, et lähme kohvikusse. Ja tegime mingi väikese 
\emph{chat}i programmi, et sai nagu teineteisega suhelda. 

Aga arvutivõrk tekkis sellest hiljem. Toodi mingi Novell 2.15\index{Novell} 
server, mida ma installisin. Väheseid esimesi asju, millel olid manuaalid 
olemas, kõik oli just nagu ametlik. Ja panime selle käima. Ja siis tekkis sinna 
veel selline nüanss, et seal Novelli serveri peal panin käima Pegasus 
Maili\index{Pegasus Mail} nimelise asja, kuhu külge kirjutasin \emph{gateway}, 
millega sai UUCP meili, põhimõtteliselt Interneti meil, mida toimetati Küberi 
majja Soomest. Ma ei mäleta, kas Soomest siiapoole lükates või siit tõmmates 
üle telefoniliini. Ja sellesama pisikese modemiga tõmbasime selle sinna 
Tehnikaülikooli majja ja jagasime kasutajate vahel laiali.

\question{Siin tundub jälle suuremat sorti lünk olema selle vahel, kuidas sulle 
hakati programmeerimist õpetama ja kui sa naaskliga kaablit torkisid ja 
\emph{gateway}sid programmeerisid?}

Mõned aastad tuli õppida asjade kirjutamist, lihtsalt erinevaid asju tehes. 
Põhimõtteliselt noor inimene, peret polnud, elasin ikkagi seal Raadiotehnika 
Kateedris\index{Tallinna Tehnikaülikool!Automaatikateaduskond!Raadiotehnika 
kateeder}. Ja sihuke omamoodi seltskond oli tegelikult seal. See arvutus-saali 
kamp, Tarmo\index[ppl]{Mamers, Tarmo} oli kohe seal arvutus-saali kõrval sama 
koridori peal, mina olen üleval Raadiotehnikas. Ja siis vana kooli mees, 
Lõvi\index[ppl]{Lõvi} oli, kõrvalkorpuses, käis aeg-ajalt selle Apple II peal 
oma  projekte arendamas.

Tasapisi niimoodi harjutades, erinevaid asju ehitades, tehes,  katsetamiseks 
oli aega palju.

\question{Meetodiks oli siis peamiselt katsetamine, mitte niipalju mingite 
manuaalide tudeerimine?}

Manuaale ei olnud üldse, põhimõtteliselt ei olnud dokumentatsiooni. Riiklikul 
tasemel tarkvara varastamise programm provaidis küll ägedat tarkvara, aga 
enamasti ilma dokumentatsioonita. See oli selline nagu  infovaakumis 
tegutsemine. Ja no eks disassembler oli nagu sõber.

\question{Selleks pidi sulle ju keegi ütlema, et selline asi nagu disassembler 
on olemas, kuidas arvuti töötab}

Jaa, jaa. Selleks olidki head vanemad kolleegid, kes nagu hoidsid kätt, 
juhendasid. Lõviga\index[ppl]{Lõvi} me ikka pikalt tegutsesime seal, kindlasti 
oli tal väga suur mõju arengule, ütleme siis nii. Aga see lünk,  kust BBSidesse 
asi läks,  oligi see, et mul oli Raadiotehnika Kateedris\index{Tallinna 
Tehnikaülikool!Automaatikateaduskond!Raadiotehnika kateeder} koht, mul oli 
arvuti, mul oli seal sees modem, millega sai helistada ja BBSid olid juba 
olemas. Ja lähim oli siinsamas, see oli selles majas kus nüüd on 
Tehnopoli kontor. Küberneetika Instituudi otsas oli tolleaegne 
Proekspert\index{Proekspert}. Ja Andrus Suitsu\index[ppl]{Suitsu, Andrus} juba 
oli BBSi-mees, temal juba oli BBS. Käisin sealt oskusteavet ja tarkvara 
hankimas. Panin ka algul BBSi ja üsna peatselt pärast seda ka Fido, algul 
\emph{point}i vist.

Ja siis käitasin seda üsna mitu aastat. 

\question{Aga miks sa seda tegid?}

Huvist kommunikatsiooni vastu põhiliselt.

\question{Kas kommunikatsioon masinate või inimeste vahel?}

Kõik. See moment, kus kus sul täielikust infopuudusest saabub järsku täielik 
infovabadus, see on väga ergastav. Inimestel, kellel on internet olemas, nad ei 
kujuta seda ette, kuidas, kuidas saab olla ilma, aga ilma oli väga pime.

Üks asi oli see tehniline info, aga tegelikult need Fidoneti grupid ja Useneti 
(selle UUCP meiliga koos toodi ka Useneti gruppe) grupid, mis tulid, olid väga 
huvitavad tegelikult. Juba need diskussioonid, mis seal käisid, olid väga 
informatiivsed. Tollal oli oluliselt tehnilisem kogu see värk, suurem osa 
juttu, mida räägiti, oli tehniline jutt, sest seal käisid tehnikud. Need, kes 
said kanalile ligi.

\question{Kas too kollase kaabliga võrk hakkas tööle ka?}

Ikka, see töötas uhkelt. See Novelli server käis seal veel 1992. aastal, kui ma 
sealt ära läksin. Ja inimesed said meili omavahel kirjutada, sai välismaailmaga 
kirjutada. Ainukene probleem, ma arvan, oli see, et arvuteid, mille oli see 
Etherneti äge \emph{interface}, neid on suhteliselt vähe, paar tükki kateedri 
peale suudeti neid vist tekitada.

\question{Etherneti kaart oli siis defitsiit?}

Tol ajal oli kõik defitsiit. Ära unusta, rubla-aeg oli veel. Millise projekti 
raames see toodi, ei tea. Avo Ots\index[ppl]{Ots, Avo} tegi minu meelest 
doktoritöö näiteks selle peale, kuidas ehitada arvutivõrku. Ja see oli 
tegelikult oluline kogemus, et toimuks järgmine samm. Peale Tehnikaülikooli 
töötasin lühikest aega Microlinkis\index{Microlink}, kus ma olingi 
arvutivõrkude installeerija, ühtlasi .EXE\index{Ajakiri!.EXE} kirjutaja.

\question{Aga miks sa läksid sinna?}

Sinna ma sattusin niimoodi, et ühel päeval astus uksest sisse Margus 
Kliimask\index[ppl]{Kliimask, Margus}, keda ma olin näinud koos Rainer 
Nõlvakuga\index[ppl]{Nõlvak, Rainer}, keda ma ei olnud enne näinud. Ütles, et 
teeme ajakirja. Sellest sai .EXE. Ja siis tegimegi.

\question{Aga miks? Jälle kõlab suure lüngana, et ühel päeval tõmban kollast 
kaablit, teisel päeval teen ajakirja}

Kusjuures täpselt nii oligi. Ma arvan, et Rainer tahtis Microlinki promo teha. 
Kujutan ette, et see võiks olla suur motivaator, võib-olla Rainer ise räägib 
teinekord.

\question{Kust sul üldse tuli selline mõte, et ajakirja tegemine võiks olla 
asi, mida teha?}

Tundus huvitav. Elus ei olnud rohkem kõrgeid eesmärke kui see, et elu oleks 
huvitav.

\question{See on tegelikult kõige kõrgemaid eesmärke, mis üldse saab olla}

Ja minu minu mäletamist mööda algul oli jutt, et teeme ajakirja ja siis selgus, 
et mul oleks ka uut töökohta vaja. Nii sattusin ma Microlinki\index{Microlink} 
korraks. Aga Microlingis olin ma vähe, ma arvan see aeg oli kuudes, ja 
siis hakati tegema Eesti Forekspanka\index{Pangad!Eesti Forekspank}\sidenote{Eesti 
Forekspank sündis 1992. aastal, Forekspank ühines Raepangaga\index{Pangad!Raepank} 
1995.}. Pangal olid oma sidevajadused, kutsuti mind sinna tööle.

\question{Üheksakümnendate algus oli Eesti panganduses ju hull aeg}

Jah, ja Forekspank oli sel ajal pisikene valuutavahetuskontor, mis opereeris 
rubla-dollari börsi.

Tegutses tolleaegses hulgifirmas Abestok\index{Abestok}. Selle  ühes toas olid 
inimesed, kes otsustasid, et nad teevad panga. Margus 
Kliimask\index[ppl]{Kliimask, Margus} oli jällegi nendega seotud, vist IT-poisi 
staatuses. Temaga koos me siis läksime Rävala Puiesteele, istusime koos 
ehitusjuhiga ühte tuppa, mis ei olnud päris puruks, aga kus ühes nurgas 
ehitajad hoidsid oma tööriistu, ja ehitasime panka.

\question{Kust tekkis see mõte, et panga tegemiseks ei piisa kilekottidega 
sularaha edasi-tagasi lohistamisest?}

Need mehed, kes panga tegid, olid piisavalt targad, et aru saada, et pank käib 
teistmoodi. Kui palju teistmoodi, ma arvan, et see sai alles siis selgeks, kui 
Inglismaalt pangatarkvara osteti ja konsultandid rääkisid, kuidas panka 
tehakse. Aga see ei olnud kohe esimesel aastal. Esimestel aastatel ehitasime, 
tõmbasin kaablit, panime serveri püsti laua alla. Ühel ilusal päeval juhtus 
selline lugu, kus Margus Kliimask\index[ppl]{Kliimask, Margus} kogemata varbaga 
lükkas toite välja ja pank jäi seisma. Aga mitte kauaks. 

Aga Rein Usin\index[ppl]{Usin, Rein} ja Ivar Lukk\index[ppl]{Lukk, Ivar} olid 
väga visiooniga inimesed ja Margus Kliimask\index[ppl]{Kliimask, Margus} oli ka 
väga visiooniga inimene. 

Ma ei mäleta, mis ajal see oli, aga suhteliselt varane aeg, sest nagu BBSid ja 
Fidonet ja kõik oli veel kuum teema, kui Margus Kliimask ütles, et teeme 
modemipanga. Ja tal oli kindel mõte, et see peab olema Norton 
Commanderi\index{Norton Commander} F2 menüüs\sidenote{1986. aastal turule 
tulnud ja 1998. aastal viimase versiooni saanud Norton Commander oli 
ülipopulaarne failihaldur MS-DOSi platvormile. Ekraanil oli korraga kaks 
nimekirja faile ja käsurida, ekraani allservas oli nimekiri saadaolevatest 
klahvivajutusega käivitatavatest käskudest. Nii oli kasutajal ilma suurema 
koolituseta kohe selge, mida ja kuidas ta teha saab. Ohtralt kasutati F-klahve 
ja neist olulisemate funktsioonid on inimestel siiani peas (F3 --- faili sisu 
vaatamine, F5 --- faili kopeerimine)}. Sest kõik kasutasid Norton Commanderit 
ja kõigil oli, aga keegi ei ostnud, sest tol ajal ei ostetud tarkvara. 

\question{Jah, ma mäletan poes karpe, aga ma ei mäleta, et keegi neid kunagi 
ostnud oleks}

Aga see oli nagu hämmastav, kuidas sellest mõtte välja käimisest kuni 
modemipanga \emph{launch}ini, ma arvan, läks kaks kuud.

\question{Tegite kahe kuuga nullist modemipanga?}

See  oli programm, mis oli natukene Norton Commanderiga integreeritud. 
Niimoodi, et ta läks sealt menüüst käima, nägi välja nagu Norton Commanderi 
osa, said makseid ette valmistada, konto väljavõtteid saada, panga teateid 
saada ja enda maksed panka saata ja need tehti ära.

\question{Ja seal teisel pool võttis mingisugune asi kõned vastu, suhtles panga 
tuumaga ja tegi arveldused ära?}

Just. See panga tuumaga suhtlemine oli suht traagelniitidega asi, kuna selleks 
ajaks oli juba panga soft  Inglismaalt toodud, millel ei olnud nagu ühtegi head 
liidest peale terminali.

\question{Ja siis te tegite vana head terminaliemu}

Kirjutasin terminali emulaatori, üks tolleaegne kolleeg kirjutas asja, mis 
sealt terminali emulaatorist maksed pangasüsteemi lükkas ja see käis 
palju-palju aastaid niimoodi. Enne, kui  tekkisid tehnilised vahendid, et seda 
natukene viisakamalt teha. Mäletan, et  selle nagu \emph{launch} toimus 
tolleaegse  aja kohta pressi käraga. Sihukene meediaüritus tehti. Ja imekenad 
Hansapanga\index{Pangad!Hansapank} tüdrukud istusid ka seal ja tegid märkmeid. Ja 
Hansapangal ei läinud ka rohkem kui mõni kuu, kui 
Telehansa\index{Pangad!Hansapank!Telehansa} välja tuli.

\question{No heaküll, teed modemipanga, aga sel ajal terve too kamp, kes 
BBSides suhtles, võis üldse olla mingisugune paarsada inimest. Kust need 
kliendid siis tulid?}

Kliendid tulid umbes pooleks.  Tolleaegse Forekspanga klientuurist arvestatav 
protsent, ei ei oska isegi peast öelda, mitu protsenti, olid Venemaal. Ja suur 
raha oli ka muidugi Venemaal tol ajal. Aga Eesti ei olnud üldse nagu kehv. Pank 
müüs seda tolleaegses rahas suhteliselt suure raha eest. Aga Eesti firmad 
ostsid seda. Ma käisin seda ise Tallinnas installeerimas. Küsimus ei olnud 
selles, et inimesed ei saanud maalt linna tulla oma panga-asju ajama. Nad 
lihtsalt ei tahtnud kontorist välja tulla. See mugavus, mis see andis, oli nii 
suur. Sa said oma laua tagant püsti tõusmata pangas ära käia.

\question{Ja see kõik tasus ära selle, et võib-olla isegi hakata arvutiga 
makseid ette valmistama?}

Sel ajal oli igas firmas raamatupidamiseks arvuti olemas. Ja raamatupidajate 
arvutis need kõik olidki. Ilmselt lihtsalt see mugavus ja aja kokkuhoid oli 
see, mis  Eesti firmad ka sinna poole ajas.

\question{Nii, puhtalt mälestuse mõttes, palju seal telefoniliine küljes oli?}

Alustasime  mingi umbes kahega või nii ja  lõpus oli  umbes kuus. Kuna 
sideseanss oli nii lühike, enamik nendest sideseanssidest muutus paari minuti 
sisse. Kõik pakiti kohapeal  kokku ja saadeti ühe portsuna ära. Kõik oli 
Fidonetist õpitud tehnoloogia. Alguses oli nii, et mina tegin kliendi poole ja 
Margus Kliimask\index[ppl]{Kliimask, Margus} kirjutas serveri poole. Ja hiljem 
ma kirjutasin serveri poole ka natuke paremaks, et ta paremini skaleeruks. 

\question{Ja skaleerumine tähendas siis mida?}

Et sai ühe masina taha  mitu modemit panna.

\question{Kas sa siis oma BBSi pidasid veel püsti?}

Panga ajal minu meelest oli meil pangas ka BBS veel mingil ajal, 
Microlinkis\index{Microlink} kindlasti oli, mul on mingi mälestus, et ka pangas 
sai BBSi peetud. Kuna Forekspank oli Rävala puiesteel, siis  kohe, kui 
üheksakümnendate suhteliselt alguses tekkis Internet, oli selge, et meil on ka 
seda vaja. Ja oma valgete käekestega koos Andrus Aaslaiuga\index[ppl]{Aaslaid, 
Andrus} tõmbasime mööda majade katuseid sinna kõrvale KBFI\index{KBFI} majja, 
kus sündis Uninet\index{Uninet}, Unineti tuppa tõmbasime Etherneti kaabli.

\question{Te olite siis otse Unineti küljes?}
Otse Unineti küljes, ühed esimesed kliendid ja mingisuguse kodukootud 
\emph{router}i softiga, mis flopi pealt käima läks. Mõlemas otsas oli üks 
arvutikast, panime ennast Internetti. Ainult ühe korra, minu minu mäletamist 
mööda, lõi meil sinna välk sisse.

\question{Mis te tegite seal Internetis?}

Ma arvan, et algul me õppisime, mis asi see on. Ja meilivahetus tegelikult 
pangas oli hädavajalik, lihtsalt, et  suhelda. Üks esimesi asju, mis pangas sai 
ehitatud, olid teleksi \emph{gateway} Pegasus Maili\index{Pegasus Mail}. 

\question{Misasi on teleks?}

Teleks oli viiekümne boodine põhimõtteliselt telegraafisüsteem. Pangad, suures 
osas maailmas ma kahtlustan, kasutavad seda endiselt. Ei käi telefoniliini 
pidi, selleks on eraldi teleksi võrk. Põhimõtteliselt telefonitraate mööda 
hoopis teistsuguse signaallinguga kui tavaline telefon.

\question{Ta on \emph{circuit switched}, eks? Siis ta vajas eraldi keskjaama?}

Jah. Põhimõtteliselt sa ikkagi  nii-öelda tegid kõne, seadsid ühenduse püsti. 
See ehitati veel sel ajal, kui olid veel  teletaibid, kus oligi klaviatuur ja 
paberirull.

\question{Aga see \emph{gateway} ei saanud siis ju olla ainult tarkvaraline, 
seal oli riistvara ka vaja?}

Jah. Seal oli üks kast vahel, mis tegi sellest serial pordi. Esimese kasti tegi 
Küberneetika Instituudi\index{Küberneetika Instituut} majas üks Sass, minu 
meelest. Üks Aleksander\index[ppl]{Reitsakas, 
Aleksander} oli.

Hästi keeruline kast oli, ma hiljem tegin  sellest sihukese peopesa suuruse 
versiooni flopi karpi.

\question{Mind hämmastab su jutu juures see, et need asjad, mida sa valmis 
ehitad, lähevad järjest keerulisemaks. Aga seda kohta, kust sa teada saad, 
kuidas keerulisemaid asju ehitada, ei paista}

See on nagu Youtube'i videot vaadates, tundub, et  kõik asjad juhtuvad ise. 
Vahepeale mahtus kuude kaupa igasugu õppimist ja häkkimist ja katsetamist.

\question{Hirmus kihu pidi siis ju seda teha olema?}

Kindlasti oli. Põhimõtteliselt seesama põhjus, et oleks huvitav. Ma arvan, et 
panga ajal oli nagu esimest korda  see, kus hakkas nagu kohusetunne ka vaevama. 
Sest kui pank  hommikul ei käinud, olin natukene paha.

Ja noh, tunde kulus kõvasti, ütleme siis. Aga üksik inimene,  ei olnud  väga 
palju muid kohustusi ka.

\question{Ja siis sai muudkui juttu räägitud teistega kuskil BBSides}

Panga ajal enam mitte. Ilmselt lihtsalt töö võttis üle,  võttis aja ära. Et eks 
mingi aeg ikkagi käis BBSides suhtlus, aga siis, kui Internet tuli, siis oli 
kohe Interneti mail, see nagu võttis asja üle. Ja mailiga ka \emph{gateway} 
kohe sinnasamasse panga serverisse. Pank oli selles mõttes väga 
hästi kommunikeeruv.

\question{Legend räägib, et sina kirjutasid esimese eesti klaviatuuri draiveri, 
on see tõsi?}

Nii ja naa. Klaviatuuriga oli sihuke lugu, et Rainer Nõlvak\index[ppl]{Nõlvak, 
Rainer} oli esimene mees, kes leidis, et klaviatuuril võiks Eesti \emph{layout} 
olla. Veel enne, kui infotehnoloogid jaole said, tellis Rainer ära Eesti 
klaviatuuri. Nii et pärast, kui kehtestati see uus standard (EVS 8:1993), siis oli nagu see 
niimoodi, et oli olemas klaviatuur ja oli kirja pandud standard. Korraldati 
konkurss. Oleks vaja standardile vastavaid mitte ainult klaviatuuri vaid 
ka lokalisatsiooni ja eriti hull lugu olid Windowsi fondid. 
Windows\index{OS!Windows} oli sel ajal olemas, Windows kolm küll veel. Ei olnud 
veel üheksakümmend viite. Ja siis korraldati konkurss. Tehke. Kõik lähenemised 
olid lubatud.

\question{Kes korraldas selle konkursi?}

Ma ei mäleta selle organisatsiooni nime (tegemist oli Eesti Informaatikafondiga, sellest
sai hiljem Eesti Informaatikakeskus, RIA eelkäia). Aga  see oli mingisugune riiklik 
konkurss. Tolleaegses mõistes täitsa kohaliku raha eest, vist oli kakskümmend 
tuhat krooni. See oli kõva raha ja kuna meil oli juba kogemus olemas, me olime 
juba natukene Raineriga koostööd teinud sel alal. Meie koostöö Raineriga oligi 
see, et Microlink enda klaviatuure müües pani kaasa draiveri, mis seda 
\emph{layout}i  toetas ka nii et osa tööd oli juba tehtud. Kui see konkurss 
tehti siis Margus Kliimask\index[ppl]{Kliimask, Margus}, jällegi visiooniga 
mees, ütles, et me teeme nii nagu Microsoft teeb. Ja \emph{reverse 
engineer}isime  kogu selle DOSi lokalisatsiooni ja klaviatuuri draiverid. Ja 
tegime põhimõtteliselt sellise installeerimisprogrammi, mis installeris justkui 
standard komponendid.  \verb|KEYBOARD.SYS| ja \verb|COUNTRY.SYS| ja mingid 
sellised asjad. Ja kuskilt õnnestus hankida mingis soft, mis tegi Windowsi 
fondid ka. Ja siis joonistasin fondid ka. Ta küll ei olnud väga head soft, ta 
ei teinud neid TrueType'i \emph{hint}ingut ja mis iganes, \emph{kerning} vist 
on see teine, mis, mis fondid ilusaks teeb, kui sa nad väikseks teed. Eesti 
fondid olid karvased ekraani peal aga sinna me kahjuks ei saanud midagi parata, 
aga  see lähenemine oli teiste omadest nii palju paremini, et võitsime ära 
selle.

\question{Pank läks konkursile osalema?}

Ei olnud pank, see olin mina ja Margus Kliimask\index[ppl]{Kliimask, Margus}. 
Meil oli pisikene OÜ, mille me tegelikult modemipanga müümiseks pangaga koos 
tegime. Pangaga ühisfirma ja minu mäletamist mööda selle Forex Communications 
nime alt osalesime. 

\question{Ja osalesite seepärast, et jälle tundus huvitav?}

Ma arvan, et sinna me läksime raha järele. Ja võib-olla ka selle 
Näitusepaviljonis toimunud joomingu pärast, mis pärast piduliku sündmuse puhul 
seesama riiklik asutus korraldas.

\question{Kas sul seepärast saigi panga aeg otsa, et pank sai valmis?}

Ma ütleks, et  rohkem peab tänulik olema nagu pangajuhtidele. Et meil olid nagu 
hämmastavalt vabad käed igatsugu tehnoloogiat katsetada ja uurida ja mõelda 
uusi asju. Sellepärast oli see Forekspank  üks esimesi internetipanga tegijaid 
ka, Meil oli interneti ühendus olemas, me juba mõistsime natukene, mis toimub. 

\question{Millega tollast internetipanka siis tehti?}

Forekspanga esimene Internetipank oli minu meelest 
IISi\index{IIS}\sidenote{1995. aastal turule toodud \emph{Internet Information 
Server (IIS)} oli Microsofti veebiserver, mis üritas (mõnevõrra tulutult) 
konkurentsi pakkuda tol ajal domineerinud Apache veebiserverile} peal, 
Windowsi\index{OS!Windows} all käis. 

\question{Eksootiline valik tol hetkel}

Oli küll, imelik valik oli. Aga sel ajal oli meil juba arendus ja 
hooldusmeeskonnad eraldi. Margus Kliimask\index[ppl]{Kliimask, Margus} oli 
arendusmeeskonnas. 

\question{Ehk, te olite \emph{devopsist} astunud sammu tagasi?}

No panga opereerimine on natuke omapärane tegevus ka. Ja Margus 
Kliimask\index[ppl]{Kliimask, Margus} juhtis seda internetipanga arendust, seal 
oli ka Pronto\index[ppl]{Pronto|see{Raja, Tanel}} \index[ppl]{Pronto} tal kambas. Ja  veel paar 
hakkajat selli ja tegid internetipanka. 

\question{Ja sina olid ka kuidagi sellega seotud?}

Mina internetipangaga ei olnud praktiliselt üldse seotud. Sel ajal ikkagi 
modemipank oli see põhikanal. Internetti oli lihtsalt nii vähestel veel. See 
oli nagu suhteliselt värske kraam ja modemipank toitis veel ikka suhteliselt 
pikalt. Aga sellel oli Forekspank juba üsna suureks ka kasvanud, et meil oli 
ikkagi hooldusmeeskond juba kümmekond inimest.

\question{See on juba organisatsioon juba, kahe telefoniliiniga ei saanud enam 
hakkama}

Sel ajal tekkisid juba teised probleemid. Seal oli rohkem nagu see, et see 
panga tarkvara, mis osteti, käis kummalise IBM-i platvormi peal, mida aeg-ajalt 
tuli \emph{upgrade}da. Selle panga tarkvara enda jaoks COBOL oli ikka uus keel. 
See oli kirjutatud imelikus keeles nimega \emph{Report Generator Language}, ta 
oli pärit System/36\index{Arvutid!System/36}\sidenote{System/36 oli IBMi poolt 
1983. aastal turule toodud väike mitme kasutaja jaoks mõeldud mitmetegumiline 
server. Teda programmeeriti peamiselt platvormipõhises RPG II\index{Keeled!RPG 
II} (\emph{Report Program Generator - RPG}) keeles} ajast veel. Sellest keelest 
kumasid ikka perfokaardid veel kõvasti läbi.

\question{Vähe sellest, et visioon, raha pidi siis ka olema, et Brittide juurde 
minna?}

See oli sel ajal meeletu raha.  Ja kas see server oli meeletu raha, nii et 
algul pangal seda õiget masinat osta jaksu ei olnud, osteti üks karm 
PC\index{Arvutid!PC} ja selle peal käis System/36 emulaator, kus käis 
pangasoft. Aga õnneks kasvasime sellest suht kähku välja. Siis pärast oli 
AS/400\index{Arvutid!AS/400}\sidenote{AS/400, hiljem tuntud kui 
\enquote{\emph{System i}}, oli IBMi keskmise suurusega serverite platvorm, mis 
turule toodi 1988. aastal}, mida siis uuendati veel korduvalt. See oli 
unikaalne platvorm.

Ma arvan et pank tol ajal, kui ta ostis selle tarkvara, ta ei saanud ainult 
tarkvara. Ta sai ka teadmise sellest, kuidas panka teha. See oli võib-olla 
rohkem väärt.

\question{Teil oli Margusega kahe peale sel ajal üks pisike OÜ juba, aga mõni 
veedab terve elu akadeemilistes sfäärides oma huvi rahuldades. Kust sul see 
arusaam ärist?}

See OÜ tekkis siis, kui me tegime modemipanga. Siis, jällegi peab kiitma 
tolleaegseid pangajuhte, nad tegid kohe meiega ühisfirma,  mis sidus meid päris 
hästi panga külge. Ja seal ei olnud mingit äri tegemist kui sellist, me 
kirjutasime softi ja inimesed maksid selle softi eest sellele OÜle, pärast 
jagasime pangaga raha ära. Nagu klassikalise äri mõistes, sa ei pidanud midagi 
müüma, pank müüs. Kindlasti noh, mingisugune ettekujutus sellest, mis asi 
raamatupidamine on,  sellised asjad muidugi tekkisid, aga  ei ütleks, et 
sellest mingi eriline ärisoon oleks tekkinud.

Kogu see OÜ  opereerimine seal kõrval,  see  ei olnud asi, mis oleks üldse 
mingit tähelepanu nõudnud, kogu fookus oli tehnoloogial.

\question{Rääkisin siin Tõnuga\index[ppl]{Samuel, Tõnu}\sidenote{Tõnu Samuel} 
ja tema ütles, et Mast oli see mees, kelle juurde sai minna riskantsete 
asjadega. Et kui oli vaja emaplaadi peal vaibanoaga mingeid radu lahti kratsida 
ja sinna relee vahele panna, siis tema teadis, mida on vaja teha, aga ei 
julgenud. Aga Mast julges}

Ma arvan, et seal on see Raadiotehnika Kateedri kool abiks. Kui sa saad aru, 
mida sa teed, siis sa ei karda lõigata.

\question{Ehk, sellist aukartust masina ees ei olnud}

See kadus ära, suhteliselt vara juba. Kuna selles Raadiotehnika Kateedri Apple 
II\index{Arvutid!Apple II}s oli mitu tükki neid laienduskaarte sees. Kui sellel 
kaas peal oli, ta  kuumenes üle. Seal ei olnud kunagi kaant peal. Seal võis 
vabalt näppe pidi sees sobrada. Ja keegi  ei öelnud, et sa ei tohi seda kivi 
välja võtta. Kõik oli pesades, kõike võis välja võtta.  Ja kui katki läks, siis 
võtsidki, muidu väga ei tahtnud võtta. 

\question{Läks katki ka?}

Ikka läks. Aga Apple II\index{Arvutid!Apple II} oli kõik tavalisest 
lihtloogikast ehitatud, Vene kivid läksid sinna asemele ja taktsagedus oli üks 
megaherts. Seda sai parandada ja selle parandamine oli väga õpetlik. Ka 
esimene IBM PC\index{Arvutid!IBM PC} oli selline, kus tulid kaasa (meil olid 
kõik juhendid olemas)  BIOSi \emph{listing}ud, skeemid. Kõik olid 
standardtükid, kõike sai parandada ja ka parandati. 

\question{Mis sa pärast panka tegid?}

Põhimõtteliselt IT-d ühele väikesele investeerimiskontorile. Kirjutasin 
Exceli\index{MS Excel} Visual Basicus\index{Keeled!Visual Basic} väärtpaberite 
kauplemise. Tol ajal tehti väga paljusid asju Excelis. Intressi vaja arvutada, 
ikka teed Excelis. Tegin sihukesed suured Exceli makrod, millega sai 
väärtpaberiportfelle põhimõtteliselt hallata ja tehinguid jagada. 

\question{Selle pärast, et oli huvitav?}

See oli rohkem vajaduspõhine. Sellel meie enda investeerimiskontoril oli seda 
vaja ja ühe koopia müüsin maha ka. 

\question{Ikka oli mingisugune müügiasi ka?}

Ma ei  tegelenud müügiga. Enamasti oli see nii, et keegi tuli, ütles, et tal 
oleks ka vaja. 

\question{No kui sul on väärt asi, siis lõpuks ikka tullakse}

Jah, kui hind sobis, siis miks mitte?

\question{Ometi sa BBSummeri\index{BBSummer} tolle kuulsa grupipildi peal ju 
oled. Sa käisid tolle seltskonnaga läbi ka siis, kui töö enamuse ajast ära 
võttis?}

Neid BBSummereid ei olnud nii palju üldse. BBSummerite algus  oli siis, kui ma 
olin täitsa Tehnikaülikoolis veel. Ja see grupipilt, mida sa vist 
mõtled\sidenote{Pean silmas memcpy podcasti kaanepildiks olevat fotot, kus peal 
hämmastavalt paljude suurte asjade hilisemad või toonased algatajad} minu  
meelest ei ole esimesest BBSummerist. See on umbes teisest või kolmandast 
BBSummerist, kus meil käisid juba Soomest Fidoneti tublid mehed kohal. Seal 
pildi peal on üks habemega mees nimega Ron Dwight\index[ppl]{Dwight, Ron}, kes 
oli FidoNeti kunn Euroopas, regiooni pealik. Küll suri ära mõni aeg tagasi. Ron 
oli tore mees, ma olen tal isegi mõnikord külas käinud, tema juures Soomes 
ööbinud, kui piirid lahti läksid. Ja ei ole Eesti kambast ainukene, kes tema 
juures külas käis ja seal ööbis, on ka teisi olnud. Ron oli väga tore inimene. 
Soomlased tol ajal üldiselt, kes FidoNeti vedasid Soomes, olid väga toetavad. 
Sa oled siin teistega rääkinud, kuidas te Soome helistasite ja keegi ei ole 
maininud, et tegelikult algusaegadel helistasid soomlased siia. Ei olnud nii, 
et ainult sealt oleks tõmmatud. Hiljem muidugi, kui siin BBSid ja firmad jalad 
alla said, siis me saime nagu selle rinnapiima otsast lahti. Aga algusaegadel 
soomlased toetasid kõvasti. 

\question{Puhtalt siis missioonitundest? Hõimuvelled ja nii edasi?}

Ma ei tea, kui palju hõimuvelled, võib-olla see, et tehnoloogiat tuleb jagada. 
Et seal on ka huvitatud inimesed. 

Minul on väga head mälestused nendest aegadest ja sellepärast kutsusime neid ka 
 BBSummeritele\index{BBSummer}. Ron käis minu meelest kahel näiteks. Ja 
soomlasi nendel esimestel BBSummeritel käis. Ma mäletan, et nad olid väga 
hämmastunud selle grupipildi-BBSummeri ajal (minu meelest toimus see 
Lõuna-Eestis kuskil), et kõik võivad õlut juua. Ja teisel päeval ei mingeid 
kaklusi!

BBSummeri korraldamise juures alati oli see, et korraldustasu  põhimõtteliselt 
tagas sulle söögi ja joogi kõigiks nendeks päevadeks. Alguses oli muidugi see, 
et kõigile peab õlut jätkuma. Ühele BBSummerile toodi küll õlut Fanta tünnides. 
Tal oli kerge Fanta mekk juures. Need alumiiniumpütid, millega hulgimüüki 
tehakse. 

\question{Tundub, et sul on inimestega vedanud?}

Mul on sõpradega vedanud, jah. Sel ajal, kui  nagu üksi elasin ja 
põhimõtteliselt siin Tehnikaülikoolis\index{Tallinna Tehnikaülikool} lihtsalt 
vabakutseline olin, sel ajal oli väga palju inimesi, kellega sai suheldud. 
Hiljem võttis perekond nii palju aega ära, et kahjuks ei ole jõudnud nagu nii 
palju nende inimestega enam kontakti hoida, kui vaja.

\question{Jah, aga kriitilisel hetkel olid nad olemas}

Eks nad on siiamaani olemas. Ma ei mäleta, mitu aastat ma ei ole 
Lõvi\index[ppl]{Lõvi} kohanud. Umbes viis aastat tagasi kohtasin Selveri 
parklas. Nüüd käisin tal hiljuti külas Tehnikaülikoolis.

\question{Ahti\index[ppl]{Heinla, Ahti}\sidenote{Ahti Heinla} ütles kuidagi 
väga targasti selle kohta, et mingisugune seltskond noori inimesi kuidagi 
omavahel suheldes sai inimeseks koos Eesti riigiga. Kas sul on ka selline 
tunne?}

Jah, me olime kõik sellises vanuses lihtsalt. Ma arvan, et see on nende 
generatsioonide värk, et me olime täpselt see generatsioon, kes oli selles 
vanuses, kus oli huvi teha midagi uut, ja tekkis võimalus teha midagi uut. Ja 
kuna me olime kõik nagu suhteliselt üheealised, siis ilmselt sellepärast tekkis 
see klapp. Kuigi nagu on ka erandeid. Henn Ruukel\index[ppl]{Ruukel, Henn} 
esimesel BBSummeril oli selgelt alaealine. Aga õlletünni juures passi ei 
küsitud.

\question{Mis sa praegu teed?}

Pean pausi. Aitan ülikoolil satelliiti\sidenote{Masti panusega satelliit lendas 
2020. aastal ka edukalt kosmosesse} ehitada. 

\question{Sellepärast, et on huvitav?}

Sellepärast, et on huvitav. Kosmos on huvitav.

\question{Kosmos on ju suur ka, seal ei ole karta, et huvitavad asjad otsa 
saavad?}

Ma ütleks, et praegu enamus sebimist käib Maale väga lähedal. Orbiidid, kuhu 
neid väikseid satelliite lastakse, on viissada kuni seitsesada kilomeetrit, et 
see on ikka väga lähedal.

\question{Kas sul üldse on kunagi olnud nii, et järgmist huvitavat asja ei ole 
silmapiiril?}

Ei.

\question{Aga kuidas sa nii oled saanud?}

Sellepärast, et isegi kui tööl ei ole  huvitav, nagu sellel päevatööl, mul 
kodus kogu aeg käib mingi projekt. Kohe kui üks kas saab valmis või läheb 
sahtlisse (enamus läheb sahtlisse, lihtsalt kaob huvi ära nende vastu), on 
järgmine kohe laua peal käsil. Ei ole sellist asja, et mul ei ole midagi teha.

\question{Sul sahtel täis ei saa?}

Saab. Jubedalt. 

\question{Mis sa teed siis?}

Viskan ära. Suur osa nendest on ju eksperimendid. Võtad ära tükid, mis lähevad  
järgmise eksperimendi peale. Ülejäänu on prügi. Teadmised jäävad alles.
