\index[ppl]{Priisalu, Jaan}

\ldots ütles eestlaste kohta väga hästi. Et eestlased otsused saunas 
teevad, eks ole, on \emph{no-brainer}. Aga kuidas seda tehakse? 
Inimesed käivad saunas, on paljad, räägivad midagi. Ja  kui ära 
lähevad, kõik nagu teavad, mis otsus on. Seda ei hääldatud mitte ühtegi korda 
välja 
ja ei ole aru saada, kes on liider, kes selle \emph{move}-ga välja tuli. 
Sul on pikka aega olnud mingid võõrad sellid peal, 
kelle eesmärk on mingi muu, kui kohaliku rahva eesmärk. Ja sa tead, et võimu 
peale loota ei saa. Aga kui sa teed otsuseid sellise \emph{mode}-ga, et sa oma 
liidreid välja ei näita, on see tegelikult liidrite kaitsmise süsteem. Mis 
muuhulgas tähendab, et me oleme projektirahvas. Kui sa  Vabadussõda vaatad, 
siis see oli ka selgelt projekt.
                 
\ldots Mulle on seda küsimust mitu korda esitatud. Ameeriklased tulevad ja 
küsivad, et kui me ringi vaatame, siis need \emph{challenge}'d, mis te 
välja käite, on nagu pooltel maailma riikidel. Aga miks siin välja tuleb?

\bigskip
\noindent\rule{.3\textwidth}{.7pt}
\bigskip
                 
\question{Kuidas sa sattusid arvutite juurde?}

Oi.

Arvutite juurde  sattusin kas neljandas või viiendas klassis. Kermo 
Jaaksoo\index[ppl]{Jaaksoo, Kermo}  oli mul  klassivend ja me olime nagu mingi 
pioneerisalk, mind pandi seda salka juhtima ja me tegime ka 
mingeid asju. Pidi päevikut kirjutama, näiteks. Kermo pakkus välja, et, 
kuule, lähme sellega kambaga ta isa töö juurde, Ülo\sidenote{Jaan peab silmas 
Kermo isa, akadeemik Ülo Jaaksood\index[ppl]{Jaaksoo, Ülo}} töötas sellel ajal 
Estonia puiesteel ja tal oli seal ES1010\index{ES 
EVM!ES-1010}\sidenote{1010 oli ES EVM-i esimese alaseeria \begin{russian}ЕС 
ЭВМ-1\end{russian} esimene mudel.}. Ega me seal temaga midagi muud teha ei 
osanud, kui seda Kuule maandumise mängu mängida. 

Ma käisin 1. Keskkoolis\index{Tallinna 1. Keskkool}.  Kuidas ma sinna 
sattusin, oli  nõnda. Kui küsitakse, et kelleks sa saada tahad, siis tavaliselt 
öeldakse, et  tuletõrjuja, politseinik, autojuht või midagi. Mu vanemad 
väidavad, 
et kui minu käest seda esimest korda küsiti, siis mina ütlesin 
\enquote{inseneriks}. 
No vot. Seepeale 
vanemad otsustasid, et poiss tuleks panna matemaatikat ja füüsikat õppima. Aga 
isa on natuke samasugune nagu mina ja leidis, et 1. Keskkool on õige koht, see 
oli veel elukoha järgne kool ka, me elasime vangla hoovis,  Suur-Patarei 29. 
Läksime sinna katsetele, tegime katsed ära ja siis direktor räägib isaga, et 
aga miks te teda siia tahate panna. Isa rääkis ära, et näed, inseneril on 
matemaatikat ja füüsikat vaja teada. Direktor mõtleb natuke ja ütleb, et teate, 
et see kõik on tore, mis ta siin räägite, ja tõepoolest, poiss tegi katsed 
edukalt. Aga meil on see häda, et matemaatika-füüsika klass hakkab üheksandast 
klassist, mis te seni teete? Seepeale pandi mind prantsuse keelt õppima.

\question{Mis on ka ilmselt väga tarviline olnud?}                 

Prantsuse keelega on see lugu, et paarikümne aasta pärast on prantsuse keel 
kõige kõneldum keel maailmas. Kui sa vaatad, palju see keel on Aafrikas 
levinud, siis tal on seal sama funktsioon, mis Indias inglise keelel, eks. Ja 
kui sa 
vaatad, missugune  rahvastikuplahvatus neil on ja palju neil maad käes on, siis 
neil seda India inimeste tiheduse probleeme ei ole, Aafrikal on  paisumisruumi 
palju.
                 
\question{Seal pioneerirühmas kohtusite esimest korda essukesega, maandusite 
Kuule ja\ldots?}

Vaatasime, kuidas lindid ringi käivad ja oli põnev. Sellel ajal olid juba 
mingid vahetatavad kõvakettad. Nad näitasid meile trummel-salvestit ka, aga see 
ei olnud käigus, oli lahti ühendatud ma ei tea, mille küljest.

\question{Mis asutus see oli seal Estonia puiesteel?}

Mingi Teaduste Akadeemia värk, ma ei tea, ei mäleta enam. Ilmselt see asi, mis 
praegu Küber\index{Küber} on või midagi sellist. Ülo oli kunagi ju Küberi 
direktor.

\question{Tollest ajast jäidki käima sinna oma rühmaga?}

Ei, mitte nii hullusti. Järgmine kord oli siis, kui meil olid arvutiõppe tunnid 
ka ja õpetaja oli Loonde\index[ppl]{Loonde, Jaak}, temast sa kindlasti oled 
kuulnud. Ta viis meid Pedasse\index{Tallinna Pedagoogikaülikool}, 
teine arvuti mu elus oli sealne Nairi-2\index{Nairi!Nairi-2}. Nairi on 
transside peal arvuti, perfolint on viierealine ja mingi sihuke värk. Kui 
mata-füssi klassi läksime, siis Jako Bergson\index[ppl]{Bergson, Jako} tõi 
Kirovi Kalurikolhoosist\index{Kirovi Kalurikolhoos} MIR-2\index{MIR-2} 
ära. Mir-2 on tegelikult maailma esimene personaalarvuti, see tähendab esimene 
arvuti, mis oli tehtud inimese aitamiseks. Tal protsessor kaalus pool 
tonni, aga kogu ideoloogia oli selles, et inimene, kes seal  taga istub, saaks 
oma rehkendused tehtud. Muuhulgas integreerimine ja diferentseerimine olid 
rauas realiseeritud.

\question{Sest inimesel ju oli vaja integreerida ja diferentseerida, mille 
jaoks tal muidu üldse arvuti!}
                 
Ukrainas tehtud värk, eksole, eks nad seal arvutasid gaasiturbiine ja 
rakette ja no mida vaja oli. Programm hakkas sul peale (mingi 
Algoli sarnane keel oli) käsuga \verb|RAZR|, mis tähendas siis 
\begin{russian}разрядность\end{russian}, ehk kui pikad täna meil arvud on.

\question{Kui pikaks neid arve keerata sai?}

Me keerasime kas 300 või 400 peale, midagi nihukest. Kümnendkohtades.

Panime ta piid arvutama, eksole, ritta ajama, ja see lõppes sellega, et 
kuigi oli talveaeg ja me tegime aknad lahti, ta saadan ikka kuumenes üle. 
Pooleteist tunni pärast pidi selle värgi nagu katki jätma.
                 
\question{Kust teil selline mõte, et võiks piid arvutada?}

Tundus lahe lihtsalt.

\question{Ja kust te matemaatika saite?}
                 
Mul oli mingi venekeelne matemaatika õpik või  
entsüklopeedia või teatmik, hästi paks. Seal olid igasugused read, pii rida oli 
üks nendest.

\question{See siis tähendab, et tolleks hetkeks oli matemaatika juba pihta 
hakanud?}

Jah. Selles koolis oli nii, et kui sa käisid 
matemaatika-füüsika eriklassis, siis esimene asi, mida õpetaja 
Uudelepp\index[ppl]{Uudelepp, Helgi}\sidenote{Jaan peab silmas Gustav Adolfi 
Gümnaasiumi legendaarset matemaatikaõpetajat Helgi Uudeleppa.} tegi oli see, et 
ta jagas klassi pooleks. Pool klassi hakkas õppima selle tavalise 
keskkooliprogrammi järgi ja ülejäänud kambale anti olümpiaadi ülesanded. Kamp 
oli 
väga kõva, probleem oli koolist üldse välja jõuda. Et rajoonist läksid sa 
niikuinii puhtalt edasi, kui sa rajooni jõudsid, olid vabariigis 
väljas\sidenote{Toonased olümpiaadid olid organiseeritud kooli-, rajooni- ja 
vabariiklikeks. Vabariiklikult olümpiaadilt oli võimalik pääseda ka 
üleliidulisele olümpiaadile.}. Aga koolist välja saamine oli probleem. Mati 
Pentus näiteks\index[ppl]{Pentus, Mati}\sidenote{Mati Pentus on Eesti 
matemaatik, alates 2003. aastast Moskva Riikliku Ülikooli professor.}. 
               
\question{Koolis oli ka teil arvutitund, eks ja teil oli sel see MIR-2?}

See oli spordihoones küll aga koolis, jah. 

\question{Jaagul oli neid kohti mitu, kus ta toimetas, ma selle pärast 
küsin}

Selge see, eks ole, et 1. Keskkool seisis direktori Viikholmi\sidenote{Helmi 
Viikholm\index[ppl]{Viikholm, Helmi} oli kooli direktor 1962-1982.} peal ja kui 
Viikholm pensionile läks,  juhtus nagu ikka organisatsioonidega juhtub, et 
hakatakse rootsi keelt õpetama või midagi.

\question{Selleks ajaks sina olid sealt koolist läinud juba?}

Ma lõppu nägin, kui Viikholm ära läks. Mingisuguse energiaga käib see 
organisatsioon veel natuke aega edasi, tavaliselt lagunema minek võtab ikka 
mingisugune kaks-kolm aastat aega.

\question{Keskkooli ajal pusisite Jaaguga MIR-i peal või oli juba muid 
võimalusi ka?}

Mina oma esimese palga progemise eest sain 1984,  see pidi olema siis 
üks suvi enne lõpetamist. Paldiski maantee 1 oli niisugune asi, nagu Termo- ja 
Elektrofüüsika Instituut\sidenote{Asutuse täpne nimi oli Eesti NSV Teaduste 
Akadeemia Termofüüsika ja Elektrofüüsika Instituut, lühendina 
TEFI.\index{Teaduste Akadeemia!Termofüüsika ja Elektrofüüsika Instituut}}. See 
oli kontor, kelle käes oli muu hulgas Arnold Veimeri nimeline laev, seal 
oli see Läänemere kamp. Mina olin akadeemik Krummi 
juures\sidenote{Akadeemik Lembit Krumm\index[ppl]{Krumm, Lembit} (1928-2016).}, 
kes arvutas elektrivõrkude mingeid staatilisi režiime. Neil olid omal  ka 
arvutid käes, see arvuti, millel ma esimese töö tegin, oli 
Iskra-226\index{Iskra!Iskra-226} mis on oma sisult Wang 2200 koopia, kus muu 
hulgas on 
8080 video protsessoriks pandud. Üks vend tegeles sellega, et 
pani CP/M-i sinna videoprotsessori peale käima, sihuke värk. 

\question{Kust nad su leidsid?}

Ma ei mäleta, kust see  peale hakkas ja kuidas juhtus. Ilmselt kuidagi tutvuste 
kaudu. Või läksin sinna lihtsalt\ldots?  Ma ei mäleta.

\question{Aga arvutasid seal elektrivõrke?}

Esimese asjana ma pidin tegema rehkenduse mingisuguse 
ministeeriumi aruandluse jaoks. Tegin seda Basicus\index{BASIC}, 
põhimõtteliselt tabelarvutus. Tegin  programmi valmis ja  nägin esimest korda 
niisugust päris \emph{user}-i probleemi ka. Tegime piduliku üleandmise, 
komisjon tuli 
nagu kokku ja sellele naisterahvale, kes pidi seda mu 
programmi kasutama hakkama, öeldi, et \enquote{istu nüüd arvuti taha ja proovi 
siin midagi toksida}. See ütles, et tema ei istu. Keegi ei saa midagi aru, 
et mis nüüd juhtus, et kas programmi meeldi või mis. \enquote{Ei, ma kardan 
elektrit!}

Ma isegi ei  mäleta, kuidas see tsirkus lõppes. Aga järgmine töö, mis mulle 
anti, oli järgmine. Neil oli mingi programm, millega nad oma suuri jakobiaane 
arvutasid ja  see käis essukese\index{ES EVM} peal, 1055 äkki?. Neid 
oli Küberis kaks tükki, üks oli 360 koopia, teine 370 oma\sidenote{ES EVM-i 
esimese alaseeria oli IBM System/360 koopia, teine ja kolmas System/370 oma. 
Mudel 1055 kuulus teise ja 1066 kolmandasse seeriasse.}. Terminal käis neile 
mitte VT100, nagu mujal maailmas, vaid IBM-i VT52. Mis tegelikult näeb  üsna 
hüperteksti moodi välja. Sa kirjeldad ära sisendi  ja väljundi väljad ja terve 
ekraan tuleb korraga, põhimõttelist töötab nagu veebileht. Sinna peale 
ma tegin mingisuguse programmi, mis võimaldas sisendandmeid kuidagi mõistlikult 
 sisestada ja tulemusi vaadata. Kuna essuke oli \emph{patch}-arvuti, siis pidi 
sinna vahele tegema programmi, kes selle \emph{patch}-i tulemused 
sisse söötis või tulemused välja võttis ja siis kuidagi  selle 
terminaliga suhtles. Tolle vahejupi tegi vist Tarmo Mere\index[ppl]{Mere, 
Tarmo}. See kõik oli üsna aeglane, ma ei mäleta, kui palju see ringi tõstmine 
aega võitis, see käis kuidagi läbi ketta.

Teine Küberi essuke oli 1066. Selle peal nad andsid mulle mingisuguse 
assembleri makro, et \enquote{pane asi käima}. Aga see protsessor oli juba 32 
bitine. Ma Inteli assemblerit olin juba vaadanud, aga no üldse ei olnud 
sarnane. Kõiki registreid sai kõikides funktsioonides kasutada ja\ldots Kõige 
hullem, millest ma ei saanud aru, oli see, et kõiki aadresse võeti 
baasregistri suhtes. Mida kõik eeldasid, et sa lihtsalt 
tead.  Aga kuidas sa teed kindlaks, milline baasregister on ja kuskohast ta 
hakkab? Selle käsu, millega baasregister valida, ma leidsin üles, aga 
midagi oli ikka puudu. Tuli välja, et kust see baasregister  aadresse 
lugema hakkas, seal oli lihtsalt kõikide käsuridade ette vaja panna üks 
tärn.
                 
\question{Programmeerimisoskus pidi sul ju olema? Kas sa said selle Jaagu käest 
või pusisid ise või kust see sul tuli?}

Eks Jaak\index[ppl]{Loonde, Jaak} õpetas, MIR-2 peal me ikka mingit nalja 
tegime, 
eks ole. Kui sa juba nagu piid arvutad, siis paar rida pead juba nagu kirjutada 
oskama. Järgmisena tuli BASIC seal TEFI-s. Aga see VT52 asi, ma ei mäleta, 
milles ma selle programmi seal taga tegin. Kas oli Fortran? 

\question{Kuidas selle VT52 puhul progemine käis, kui ekraan oli lihtsalt 
mingisugused väljad?}                 

Ei mäleta, ega sul see interaktsiooni kirjeldus on lihtsalt see, et sa saadad 
terve 
andmepaketi korraga minema ja saad terve andmepaketi korraga tagasi. Andmete  
töötlus ja esitus päriselt ongi eraldatud. 

\question{Naljakaid aparaate on olemas!}

Ma mäletan, need vennad  presenteerisid modemit, millega me  Küberisse 
helistasime. 1200 boodine modem oli külmkapi suurune. Kui 2400-ne modem 
tuli, siis see oli juba tükk maad väiksem, pool külmkappi.

Ja millega vennikesed veel tegelesid, 1984 olid ju olümpiamängud LA-s. Mängude 
ajal nad jälgisid elektrivõrgu parameetrid, peamiselt sagedust, ja sealt pealt 
ütlesid, et palju tootmist seisab ja mitu inimest vaatab olümpiamänge. 

\question{Kas nad ütlesid ametlikult ka kellelegi seda?}

Oma lõbuks. Neil oli mingi kihlveokontor, vedasid kihla, et palju homme nagu 
vaatajaid on, niisugune mäng.

\question{Sa olid ju matemaatikas ja füüsikas ka tugev, aga arvuti asja sa 
pidid suuresti ise pusima, mis sind selle juures tõmbas?}
                              
Kui oma kätega midagi teha saab, see on nagu äge. See, et sa teooriat saad 
välja mõelda, on nagu ühte liiki inimesed ja see, et sa mingid asjad saad kokku 
ja käima panna, see on natuke teistsugune suhtumine. Täna ma olen selle 
mõtlemise ja teooria pole peal.
                 
\question{Eks asjad ole maailmas tasakaalus. Aga tol hetkel sulle meeldis, et 
vajutad arvuti klahve ja see muudkui teeb?}
                 
See, et asjad ise juhtusid, automatiseerimise värk, see oli nagu väga äge. Et 
mingi auto sõidab ise. Sellel ajal oli haruldus, kui natukenegi targem 
juhtimisalgoritm oli.

\question{Ja see sind pidas nii palju arvuti juures ja õppisid programmeerima 
ja õigetesse kohtadesse tärne panema?}

Jah.

\question{Kas sinna juurde käis niisugune laiem valdkonna huvi ka? Mõni on 
rääkinud, et raamatud, muusika ja muu selline suunas ka kõvasti arvutite poole?}

Kuule, see oli sügav Vene aeg, mis mõttes \enquote{raamatud ja muusika}. 
Loomulikult ma 
Asimovit lugesin, seal robotivärki räägiti päris palju, lugesin kõiki 
raamatuid, mida kätte saada oli. Neid, mida kätte ei saanud, neid ma lugesin 
juba Prantsusmaal, siis kui nad Toulouse'i  õppima läksin. Osutus, et ma panin  
kooli minekuga natukene puusse. Kui ma poole septembri pealt kohale läksin 
polnud koolis veel kedagi. No ja selleks, et keelt parandada,  
istusin ma raamatukogus ja lugesin mingit matemaatikat ja Asimovi jutte.

\question{Enne, kui me Toulouse'i jõuame, lõpetasid sa ju keskkooli ära. Kas 
tööl käimine õppimist ei seganud?}                 

Ei seganud, see oli üsna nagu eluviisi osa,  ma olen kooli kõrvalt alati tööl 
käinud.
        
\question{Mida sa ülikooli õppima läksid?}
         
Automaatikat, automaatjuhtimissüsteeme, Tehnikaülikooli\index{Tallinna 
Tehnikaülikool!Automaatikateaduskond}.

\question{Kuidas see valik tekkis? Oli see loomulik valik?}

Oli loomulik valik. Sugulaste hulgas on niisugune vend nagu Jaan 
Võrk\index[ppl]{Võrk, Jaan}, eks ole, kelle kaudu ma natuke kuulsin, et mis asi 
see automaatika on. Väga äge koht oli. Rühma vendi-õdesid meie klassist 
oli kas umbes kümme. 
                 
Pärast on Gibbs\index[ppl]{Kübbar, Heiki} mulle öelnud, et seda oli  
vastik vaadata, et sa higistad matemaatika ja füüsikaga loengutes ja asjades ja 
need vennad tulevad kuskilt, teevad  mingit pulli, lähevad eksamile ja saavad 
kõik viied. 

\question{Ta oli sul siis kursavend?}

Me oleme kindlasti koos loengus käinud. Kuidas need niimoodi sattusid, ma ei 
teagi. Ta on aasta noorem, ilmselt  sõjaväest tagasitulekuga kuidagi läks see 
asi sünki, mina istusin oma kaks aastat sõjaväes ära.
                 
\question{Sind võeti enne ülikooli kroonusse?}

Ülikooli esimeselt kursuselt.

Arvasin, et ei lähe. Mingi 12. detsember või midagi sellist oli tolle aasta 
viimane võtmine, arvasin, et ei juhtu midagi. Aga pandi rongi peale ja läks.

\question{Kus sa need kaks aastat veetsid?}

Põhikoht oli Rostov Doni ääres ja sisevägedes, ehk siis vangivalvurid ja nii. 
Alguses olin  mingisuguses isolaatoris, seal oli kolm varianti, mida saab teha 
veest ja hapukapsastest. Esimene oli hapukapsasupp ehk vesi hapukapsastega. 
Teine oli praad ehk hapukapsad ilma veeta. Kolmas oli siis kissell, ehk siis 
vesi ilma hapukapsasteta. Ma vaatsin, et ma suren ära siin, kui väga pikalt 
pean olema  ja muntserdasin ennast \emph{utšebka}-sse\sidenote{Õppeväeossa}, 
natuke luuletasin 
ka. Mispeale saadeti mind valveseadmete  inseneriks õppima, see oli 
Galatšis. Galatš  on niisugune koht, kus Stalingradi kott kokku murti. Et seda 
Volgogradi Venemaa Ema olen  Mamayevi kurgaani peal päriselt lähedalt näinud. 
Hirmus roostes kolakas oli. Kaugelt vaadates on ilus, aga lähedale lähed, siis 
on roostes.

\question{Nagu enamus asju}

Põhimõttelielt küll jah, ikka.
                 
\question{Mina küll Vene kroonu napilt ei jõudnud aga, nohik, nagu ma olin, 
kartsin, et kuidas ma seal füüsilise koormuse ja keelega hakkama saan. Sul seda 
hirmu ei olnud?}

Ei no see, et sa nagu ellu üritad jääda, see on igal juhul. Ja see on ka selge, 
et vene keelt sa koolis ära ei õpi. Ja seal sul ei ole muud valikut.
                 
\question{Jah, klassikalise haridusega vene proua sulle kolme- ja neljatähelisi 
ei õpeta}

Jah, ja need on asjad, millega sa õpid tõepoolest kõiki asju ära ütlema. Seal 
tehti viina, naftast. Läksin mingi hetk brigaadi töökotta, juhendasin seal 
järgmist venda, see oli mingi poolakas Leedust, kes teadis elektroonikast 
tegelikult rohkem, kui mina. Ja noh, 
prapporid\sidenote{\begin{russian}Пра́порщик\end{russian}, lepinguline 
allohvitser Nõukogude armees.} tulid oma viinapudeliga, tahtsid, et me selle 
treipingis ära tsentrifuugiksime. Paned pudelile rätiku ümber, treipinki ja 
pöörded peale. Seesama major, kes mind \emph{utšebka}-sse vajas, vaatas kõrvalt 
ja ütles \enquote{\begin{russian}ты уважай русский язык, ты хот \ldots\ 
скажи!\end{russian}}. See oli päris hull, ei olnud päris tavaline keel.
                 
\question{Mõnikord inimesed räägivad, et sellest kolmetäheliste maailmast on 
keeruline ennast tagasi teaduse maailma keerata, sul seda probleemi ei olnud?}

No kindlasti. Kui sa vaatad, mis sellel ajal tehti, siis sõjaväest tagasi 
tulnud loobiti  kõik  eraldi kursusele, neid niisuguste, noh, puutumatute 
inimestega kokku ei lastudki. Ilmselt see haldamine oli keeruline, ma usun 
küll.
             
\question{Sina tulid Tehnikaülikooli tagasi sama asja õppima?}    

Jah, isegi sain sama töökoha tagasi. Aga siis  mingil hetkel vahetasin 
EKTA-sse\index{EKTA}\sidenote{Arvutustehnika Erikonstrueerimisbüroo (EKTA) oli 
Eesti NSV Teaduste Akadeemia Küberneetika Instituudi autonoomne osakond}. 
Ektaco\index{Ektaco} on EKTA  sihuke \emph{spin-off}. EKTA direktor oli 
Märtin\sidenote{Kaarel Märtin\index[ppl]{Märtin, Kaarel} oli siiski EKTA 
tarkvara osakonna pealik, tema alluvuses Jaan ilmselt töötaski. EKTA  direktor 
oli Kalju Leppik\index[ppl]{Leppik, Kalju}, Ektaco oma Rein 
Haavel\index[ppl]{Haavel, Rein}.}, 
ma hakkasin seal mingeid andmebaase kirjutava FoxPro-s.

Näiteks huvitav kogemus oli käia putši ajal Moskvas. Me tegime sealsele 
juveelitehasele 
väärismetallide arvestusprogrammi, aga sel ajal pidi softile autor kaasa 
minema, väga töökindlad need asjad ei olnud. Ma olen nagu eluaeg lollustega 
maha saanud või hirmus otse asja öelnud. Tehases 
osakonna juhataja, pealik, oli tiba juudi verd.  Kui ta kuulis, et  
erakorraline komitee on võimu üle võtnud, ütles ta mulle kohe, et see kõlab 
halvasti, vedur teise otsa ja kohe kodu tagasi. Ma arvasin vastu, et mingi 
satiir: 
\enquote{ah, mis sa jamad, vaata kui hästi 
Levitani\sidenote{\begin{russian}Юрий Борисович Левитан\end{russian} oli  
Nõukogude diktor, kelle kanda oli peamiste oluliste uudiste edastamine Teise 
Maailmasõja ajal, tema iseloomulikku häält tunti hästi.} häält teevad järgi, 
nagu sõjaajast oleks pärit}. Läksime tänavatele, olidki 
BTR-e\sidenote{\begin{russian}Бронетранспортёр\end{russian}. Venekeelne 
üldnimetus soomustatud jalaväetranspordimasinale.} täis ja  madin käis. Seal on 
ju suured tänavad,  seitsmerealised, need kõik olid autodest tühjad. Vaatasin, 
inimesed korjavad sillutisekive ja ehitavad nendest barrikaade.  Vennikesed 
küsisid mu käest ka, et \enquote{oleme ju kõvad mehed, vaata kus me ehitame 
 barrikaade}. Vaatasin, nii kõrged on\sidenote{Jaan osundas umbes põlve 
kõrgusele}. Ütlesin neile, et \enquote{kas te teate, et tank T-72 tehnilises 
spetsifikatsioonis on kirjas, et see sõidab 70 kilomeetrit tunnis, kui maapinna 
ebatasasus ei ületa meetrit?}. Noh ja seal juveelimessil oli ka mingisugune 
tädike, selline korralik proua, tuli ja rääkis, kuidas kõik on \enquote{nii 
kohutav, nii kohutav}. Küsis, kuidas neil ikka komme on, et 
\enquote{\begin{russian}Молодой человек\end{russian}, mis teie sellest asjast 
arvate?}. Ma ütlesin, et kõik on hästi ju. \enquote{Mis mõttes hästi?} 
\enquote{No vaata, siiamaani on venelasest tapnud kõiki teisi rahvaid, nüüd 
tapavad venelased venelasi, kõik on hästi ju}. Populaarsust ei võitnud sellega, 
muidugi.

\question{Kas igav ei olnud andmebaase treida tulles matemaatiliselt keeruliste 
asjade juurest? See on ju rutiinne töö?}

Ei olnud, seal oli tegelikult sisendit ja väljundit palju, on vaja  erinevatele 
inimestele vaated teha, seal oli nagu  pusimist küll. Ja mul olid väga lahedad 
töökaaslased, Jüri Freiberg\index[ppl]{Freiberg, Jüri} ja  Ülle 
Heinla\index[ppl]{Heinla, Ülle}. Ahti\index[ppl]{Heinla, Ahti} ema näitas 
esimest korda seda mängu, mis tal poiss tegi, hästi uhke oli. 

\question{Kes ei oleks! Kaua sa tegid neid baase?}

Ma ei mäleta. Kui ma läksin Ektaco-sse\index{Ektaco}, siis seal ma tegin 
mingeid baase edasi. Ahjaa, siis ma tulin juba Prantsusmaalt tagasi, tegin juba 
niisuguseid baase, kus olid mingid füüsilised asjad ka taga. Lukud ja kassad ja.

\question{Aga kuidas sa Prantsusmaale sattusid?}

Prantsusmaaga on see lugu, et nagu kõik suured riigid, üritavad nad endale 
ajusid 
kokku vedada. Nõukogude Liidule oli eraldatud 300 stipendiumit ja kui liit 
laiali läks, 
siis kuus stipendiumit tuli Eestisse. Kuna sellel ajal prantsuse keele  
oskajaid oli suhteliselt vähe, siis nad ei osanudki muud teha kui korraldada 
avalik konkurss. TPI-st korjati ka inimesi ja sealne prantsuse keele õpetaja 
pani mu naise (kes oli ka esimesest keskkoolist) kirja. Aga dekanaat tõmbas ta 
maha, et naine on rase ja et kuidas ta sinna läheb. Naine oli suhteliselt kõva 
iseloomuga, et mis see  dekanaadi asi on, kas ma olen rase või ole. Mis me siis 
teeme, lähme saatkonda. Tuvi tänaval, seal ujula ees, jääb tal samm järsku 
aeglasemaks ja ütleb, et kuule, aga ma tõesti olen rase, mine sina.

Otsisime välja, et mis minekuks vaja on. Stipendiumi saamiseks ütles saatkond, 
peab paar tingimust täitma. Esiteks, võimalikult kõrgel õppima. Kuna mul oli 
neli aastat ülikooli taga, siis kohe magistrisse. Teiseks, soovitavalt  mitte 
Pariisi minema. Ja kuna Leo Mõtusel\index[ppl]{Mõtus, Leo} oli Toulousis 
mingisugune tegelane, keda ta tundis (mingi tehisintellekti vend),  siis läksin 
sinna.
                
\question{Üheksakümnendate algul tehintellektiga tegelda oli üsna huvitav aeg, 
see oli ju enne riistvara läbimurret}

Oi, aga selle aja peale oli igasuguseid asju juba  tehtud. Mingid 
produktsioonisüsteemid\sidenote{Produktsioonisüsteem (ingl. \emph{production 
system}) on tüüpiliselt tehisintellekti pakkumiseks rakendatud arvutiprogramm, 
mis koosneb formaalsetest reeglitest, mehhanismidest nende reeglite järgimiseks 
ning süsteemi olekut säilitavast andmebaasist.},  esimesed teoreemitõestajad 
olid juba olemas, muidugi  otsustuspuud ja asjad. Ma ei mäleta, millal see 
Rete\sidenote{Charles L. Forgy poolt 1974. aastal maailmale tutvustatud 
algoritm efektiivseks formaalsete reeglite rakendamiseks.}  algoritm tehti, see 
on põhimõtteliselt nende produktsioonisüsteemide indeks. Sellel ajal oli 
tegelikult selliseid põhialuseid laotud juba päris palju. Masinõpet vist väga 
ei tehtud, seda ei osatud, nii palju jõudu käes ei olnud.
                 
\question{Kaua sa olid Prantsusmaal?}

Aasta. See oli see aeg, kui nad esimese ruuteri tõid majja.

\question{Ahaa, sealt hakkas sul võrgu asi tekkima?}

Jah, see oli esimene koht, kus me Internetti nägin. Naine oli veel Tallinnas, 
tema käis siin Küberis\index{Küber}. Oli niisugune programm nagu \emph{talk}, 
ühel pool Unixi masinas kirjutad sina ja teisel pool teine. Ja sellel ajal
kaugekõnesid pidi kuidagi tellima ja see oli hästi keeruline protseduur,  
\emph{talk} võimaldas nagu vähe paremini suhelda.

\question{Sul sihukest mõtet ei olnud, et hakkakski teadust tegema?}
          
Oli. Aga sellega oli see lugu, et naine käis mul külas ja siis tuli meil teine 
laps ka ja siis oli vaja nagu koju tagasi tulla.
                 
\question{Aga Eestis saab ka ju teadust teha?}

Sellel ajal ei saanud. No sul ei ole raha lihtsalt. Üheksakümnendate algus.

Sul on vaja pere jaoks raha teenida, kuskilt on vaja korter kätte saada. 
Korteri hinnad olid selles mõttes nagu naeruväärselt madalad, et kuna sain 
Prantsusmaal magistristippi, mis oli kõrgem ka, siis pool stippi hoidsin kokku 
ja selle eest ostsime korteri. Sihuke värk, niisugused ajad olid. 

\question{Kui sa Prantsusmaalt tagasi tulid, mis sa tegema hakkasid? 
Programmeerisid jätkuvalt?}

Jah ikka. Läksin Ektaco-sse\index{Ektaco}. Ma ei mäleta, mida ma enne 
Prantsusmaad Ektacos tegin, aga pärast välismaad ma tegin lukkude juhtimist. 
Lukkude juhtimine oli muu hulgas ühes pullis kohas, Viimsi 
Talveaias\sidenote{Viimsi Talveaed asub Pringi külas ja valmis 1973. aastal 
Kirovi-nimelise näidiskalurikolhoosile. See konkreetne kolhoos (ja 
sealsed kolhoosnikud) olid selle aja mõistes põhjatult rikkad ning Talveaiast 
kujunes ka Tallinna 
peenema rahva peokoht. Hulludel üheksakümnendatel oli tegemist populaarseima 
paigaga, kus 
kiiresti ja kõikvõimalikel viisidel rikastunud inimesed käisid oma rikkust 
neile teada olevatel viisidel demonstreerimas.}. Mis tähendab seda, et (neid 
lukke küll mulle ei antud) garderoobis oli püramiid, kuhu korjati numbri vastu 
relvad ära. Kui oli avamispidu, oli mingi kama, kas lukud ei töötanud või 
midagi. Läksin sinna ja sauna põrandal oli põhimõtteliselt kiht paljaid purjus 
naisi\sidenote{Sedalaadi asju juhtus, on ka minul mälestusi kinniste baaride 
kassasüsteemide hoolduse ajast.}. Väga pull koht, väga imelik.
 
\question{Mis selles lukuprogrammeerimises huvitavat oli?}
                 
Seal ikka on pusimist, et sa kõik asjad nagu paika saad. Aga mul oli ka see 
probleem, et kui sa selle nüüd kokku võtad, mis ma  Prantsusmaal õppisin, siis 
ta on põhimõttelist diskreetne matt. Kuidas  kompilaatoreid tehakse, 
kategooriate teooria, mis on eri liiki (loogiline, denotatsiooniline ja 
operatsioon-) semantikad, mingeid sihukesi asju õppisin. Aga kus seda vaja 
läheb? Tuled tagasi ja raha saad ikka selle eest, kui sa kellelegi mingi päris 
probleemi ära lahendad. See luku-projekt läks nagu hulluks kätte. Me kõigepealt 
pidime tegema kassasüsteemi, sellele tulid lukud külge ja nii see pintsaku 
nööbi ümber õmblemine käis. Neid muutusi, mida tellija tahtis, oli  hästi 
palju. Aga ma suutsin  selle andmemudeli kohe niimoodi paika panna, et ma seda 
pärast enam muutma ei pidanud. Ainult juurde tuli panna. Seepeale ma järsku 
sain 
aru, et olen midagi õppinud ka.

\question{See vajab ju päris head rakendusvõimet saada, et nii abstraktset 
teemat kohe baasi mudelis kasutada. Kategooriate teooria ju ei ütle sulle, 
millised tabelid olema peavad?}

Jah ja ei. Mida  see kategooriate teooriad õpetab on see, kuidas  maailmas 
asjad nagu korrastatud on. Matemaatika point on üldse selles, et ta korrastab 
mõtlemist, ega seal mingit muud pointi väga ei ole.
                 
\question{Üheksakümnendate lõpuks sa tegelesid juba infoturbe ja riskidega, 
kuidas sa sinna jõudsid lukkude juurest?}

Oligi igav. Enn Lakspere\index[ppl]{Lakspere, Enn} oli niisugune vend, kes  
läks Küberisse\index{Küber} tööle. Monika\sidenote{Monika 
Oit\index[ppl]{Oit, Monika}} ja Ülo Jaaksoo\index[ppl]{Jaaksoo, Ülo} olid 
teinud selle turvaseltskonna enne, kui Eesti Vabariigi iseseisvus paistma 
hakkas, sest nad arvasid, et see on strateegiline oskus, mida on igal riigil 
vaja. Milles on neil õigus.
                 
\question{Neil oli \emph{selline} visioon juba tol ajal?}
                 
Jah, neil oli enne iseseisvumist juba sihuke visioon olemas, et see iseseisvus 
ühel hetkel tuleb ja selleks ajaks peab kompetents olema. Et riigi  
infoturve on riigi jaoks strateegiline asi ja see tuleb korda saada.

\question{Mõnes kohas ei ole sellest siiamaani aru saadud!}

Ei saadagi, ma arvan. Need olid väga suure visiooniga inimesed, jah kindlasti. 

\question{Ja sa läksid siis nende juurde?}

Enn Lakspere viis. Kokkulepe oli see, et  mina teen uurimusi ja tema otsib 
tööd. Mu eriala olid kiipkaardid. Teda kaardid huvitasid, kuna ta tuli 
Ektacost, kus kassasüsteemide külge käisid kaardid ka. Mina pidin  kiipkaarte 
uurima. No ja siis ma 
kirjutasingi raha eest uurimusi, kolm lehekülge puhast teksti 
päevas, seda on päris palju.

Vello Hanson\index[ppl]{Hanson, Vello} õpetas mind kirjutama. Osa tema 
õpetamistest olen täna ära unustanud, aga Vello Hanson on nagu tõsiselt kõva 
vend. 

Kirjutasin selliseid asju nagu näiteks Pankade Kaardikeskuse\index{Pankade 
Kaardikeskus} arhitektuur. Keskpank tahtis sellele asjale litsentsi anda ja 
litsentsi menetleda. Aga ma kirjutasin sinna  ühe asja, mis oli 
Sildmäe\index[ppl]{Sildmäe, Tõnis} jaoks uudis, et ärge \emph{settlement}-i ja 
raha liigutamist üldse sinna 
keskusse pange. Võtke lihtsalt info, et kes kellele kui palju võlgu on ja tehke 
bilateraalne \emph{settlement}. Ta käis seda üle küsimas, et 
kas nii saab. Nad hoidsid nii mingi paar aastat  
lihtsalt puhast regulatsiooni kokku. Grupivend Margus Aun\index[ppl]{Aun, 
Margus} 
läks seda värki seal juhendama. 

Ülo Jaaksoo tuli  ühel hetkel juurde ja küsis, et sa siin mingite 
kaartidega mässad, et  kas sellega ühiskonna jaoks ka midagi teha saab. Kõrval 
oli Ahto Buldas\index[ppl]{Buldas, Ahto}, kes rääkis mulle asümmeetrilisest 
krüptost ja et sellega saab digiallkirja teha. Lugesin seda värki veel juurde 
kuskilt ja mingil sise-seminaril (see oli 1994. aastal) pakkusin välja, et 
kiipkaardid võiks inimestele kätte anda ja sellega võiks digiallkirja teha. 
Esimest korda mingisugune niisugune avalik esinemine oli, mingi 1995. Pandi 
mind lava peale Küberis ja öeldi, et \enquote{räägi sellest}. 
Tarvi\index[ppl]{Martens, Tarvi}, lõpuks võitis sellesama idee ja läks müüs 
riigile maha ja no nii see tuligi. Tarvi on tegelikult selle asja juurutaja 
või innovaator.

\question{Kust see legend tuleb, et tegemist on soomlaste tehnoloogiaga? Või on 
soomlasete tehnoloogia ja teie idee?}
                 
Ei ole tõsi. Soomlaste tehnoloogia ei ole ka originaalne. Tarvi siis, kui  
digiallkirja seadust tegema hakati, tellis minu käest profiili, et missugune 
see kaart peaks olema. Vaatasin ringi, mis kuskil tehtud on, ja rootslastel oli 
kaardi profiil kirjeldatud, nad tegid kolm võtmepaari. Sommid kopeerisid 
rootslasi ja panid kaks võtmepaari kokku. Miks sul üldse eri võtmepaare vaja 
on, on see, et neil on täiesti erinevad poliitikad. Autentimise ja 
krüpteerimise võtmeid ei tohiks kokku panna, sest krüpteerimise võtmel  peaks 
olema taaste, kui sa tahad teda pikaajaliseks säilitamiseks kasutada. Allkirja 
võtmel ei tohi olla taastet. 
Autentimisvõtmel ka ei ole nagu mingit põhjust taastet tahta. Need on erinevad 
poliitikad, mis tegelikult ei sobi väga hästi kokku, aga nii on lihtsam inimesi 
õpetada. Ja turbes ongi see põhimõte, et ainult lihtsad asjad töötavad. No 
jaapanlased võib-olla saavad keerulise asjaga ka hakkama, aga meie ei saa. Ja 
kuna ükski inimene krüpteerida ei oska, siis talle tuleb anda mingi junn, mis 
seda tema eest teeb, mingi arvuti. Kiipkaardist lihtsamat arvutit ei ole olemas.
     
\question{Ja nii sa siis jõudsid infoturbeni?}

Jah

\question{Infoturve tundub olevat sinu juttu kuulates ideaalne kombinatsioon: 
sai asju ära teha, oli palju matemaatikat, arvuteid, kõik kenasti koos}
            
Küber iseenesest oli üsna selge teadusasutus. Mis tähendab seda, et väga palju 
tehti teooriat, natukene kirjutati programmi ka, Arne\sidenote{Arne 
Ansper\index[ppl]{Ansper, Arne}} oli juba sellel ajal kodeerimises kibe käsi.

Aga miks mina sealt ära läksin oli, et tekkis selline tunne, et kirjutad 
igasugu plaane ja värke ja siis teised mehed plaanide põhjal ehitavad. 
Ühispank\index{Ühispank} oli viimane pank, kellel ei olnud oma 
kaardiserverit, ma läksin seda 
tegema. Ja kuna turvainimesi ka ei olnud, siis pidin olema turbe ja 
maksekaartide peal. Edasi läksin Hansasse\index{Hansapank}. 

Mis oli nagu tore, oli pangasüsteemi ringi tõstmine. Oli vaja ühendada kõik 
need maapangad\sidenote{Ühispanga asutasid 15. detsembril 1992 kaheksa 
maapanka, Viljandi kommertspank ja Nordpank}, selleks oli vaja süsteemi. 
IT-direktor oli 
Novelli-mees. Ma rehkendasin talle \emph{roundtrip} aegadega, et mitu 
transaktsiooni ta tänu lukustamistele jõuab üldse päevas teha, see oli kas 30 
000 või 40 000, midagi sihukest. Mis tähendab seda, et tal Tallinnas oleva 
panga jaoks jätkub, aga terve Eesti peale on vähe. Siis sai Unix sinna alla 
valitud, et lukustamine käiks ühes masinas ära. Tema valis millegipärast HP, 
tegelikult HP-UX\index{HP-UX} oli nagu väga äge Unix. Inimesed arvavad, mis 
nad arvavad, aga väga robustne riist. Solaris, millest tavalised inimesed 
räägivad, oli tükk maad hellem. 

Toona oli Windowsiga selline õnnetus, et TCP \emph{stack} eest 
pidi eraldi raha maksma, Linuxi käima panek oli kaks korda odavam. Nii oligi, 
et 
kõik need maapangad, ühendati niimoodi ära, et pandi Linux \emph{front}-i ja 
ühe ööga keerati kontor ringi, süsteemi vahetus ja  \emph{front}-i vahetus. 

\question{Mis sa praegu teed?}

Praegu ma uurin kriitilisi sõltuvusi, see teema on tegelikult pärit 
RIA\index{Riigi Infosüsteemi Amet}\sidenote{Jaan oli 2011-2015 Riigi 
Infosüsteemi Ameti peadirektor} ajast. 
Kui sa pead vastutama niisuguse asja eest nagu massiivne küberrünnak, siis 
selle juhtimiseks sa lükkad kokku mingis staabi. Ja esimene küsimus on muidugi 
see, et kas see, mida sa näed, on õige. Niisugust  infosüsteemist 
sisse tulevat müra, kus  sa pead süsteemi enda käitumist rünnakuks, on nagu 
päris palju.  Teised küsimused, mis sa siis küsima hakkad, on, et mis edasi 
juhtub ja kust rünnak peale hakkas. Ja tavaliselt  inimesed ei oska  kummalegi 
vastata. Minu algne mõte oli see, et äkki nendel sõltuvustel on mingi 
võrestruktuur või võre moodi mingi osaline järjestus. Kui leiad  miinimumi  
selles osalises järjestuses, siis see võiks olla algpõhjus. Ja kui transitiivse 
sulundi võtad, saad kõik tuleviku asjad kätte. Mõte oli sihukest 
planeerimis-abi 
tegema hakata. Ja et sellest aru saada või niisugust asja teha, on vaja 
neid neid sõltuvusi kuidagi kirja panna. Nüüd ma olen nii palju targaks saanud, 
et see mu arvamus, et seal tsükleid ei ole, ei ole õige. Tsüklid on ja väga 
palju ja väga lühikesi. Kogu majandus on tegelikult tsükleid ja tagasisidet 
täis.  Sa ise oled dünaamilisi süsteeme õppinud, mida sellised asjad teevad, 
seda sa oled näinud küll.

Aga nüüd üritan neist asjadest aru saada. Umbes nii.
           
\question{Kõlab, nagu oleksid asjad jätkuvalt huvitavad ja see on üks väheseid 
olulisi asju. Või sa eelistad igavaid?}
                       
Ei, seda kindlasti mitte. Aga mõnikord nad võivad liiga huvitavaks minna. Et 
kui sa vaatad, kuhu see keerukus läheb, keerukus kipub kasvama, maha sa pead 
teda jõuga võtma jõudma, \emph{refactoring} on alati töö.
