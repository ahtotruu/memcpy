\label{sisu:asko}
\index[ppl]{Seeba, Asko}

\question{Kuidas jõudsid arvutid sinu juurde ja sina arvutite juurde?}

Mina elasin oma teadliku lapsepõlve 
Viljandimaal, Viljandist Riia maanteed pidi linnapiirist viis-kuus kilomeetrit välja sõita, ja käisin linnaservas Carl 
Robert Jakobsoni nimelises Viljandi 1. Keskkoolis\index{Viljandi 
1. Keskkool}. Hiljem oli see Jakobsoni 
gümnaasium ja nüüd peale riigigümnaasiumite tegemist gümnaasiumi enam ei ole, 
Jakobsoni kooli nime all on ainult põhikool. 

Arvutiteni jõudsin sealsamas koolis. Meil oli tore arvutiõpetaja 
Heiki Pettai\index[ppl]{Pettai, Heiki}, kes tegutseb vist praegugi 
IT vallas, küll enam ammu mitte õpetajana, aga rohkem 
spetsialistina. Käisin neljandas või viiendas 
klassis, kui sain teada, et naabripoiss Toomas Aas\index[ppl]{Aas, Toomas} (praegu üks kõvemaid tarkvaraarendajaid) käib arvutiringis. Ta oli sellest 
rääkinud, aga minu teadmised arvutist olid hästi lapselikud. Olin arvutit näinud telekast lastesaates ja sellega
tehti midagi naljakat, aga ma ei osanud sellest tol hetkel 
midagi arvata. 

See hetk, kui klõps käis, oli hästi ootamatu ja 
lühike, enam-vähem sekundiga. Sattusin ükskord nägema koolikoridori 
peal, kuidas seesama naabripoiss läks arvutiklassi pisikesest uksest sisse ja kui uks paotus, paistsid sealt arvutid!
Seal oli küll ainult kolm arvutit, Vene DVK-2d\index{DVK!DVK-2}, aga see oli minu jaoks
maagiline moment, et märkasin midagi, mida olin telekast näinud ja
väga kaugeks pidanud, ning nüüd oli see järsku kahe-kolme meetri 
kaugusel. Klõps käis ära ja ma küsisin täiesti spontaanselt, kas 
tohin ka sisse tulla. Arvutiõpetaja Heiki Pettai 
lubas ja ma olin hetkega müüdud ning tahtsin seal käima 
hakata. Sellest hetkest peale teadsin, et mu ülejäänud elu peab olema 
arvutitega seotud. 

\question{Oskad sa öelda, mis täpselt see maagiline asi oli?}

Mis see lapsel võis olla? Et näed mingisugust lahedat asja, millest aru 
ei saa. Tolleaegsed arvutid olid rohkem sellised -- inglise keeles 
on tore väljend \emph{exposed} -- füüsiliselt avatud, 
igasuguseid keerulisi asju sai silmaga näha: juhtmeid, trükiplaate ja muud värki ja möllu. Tolleaegse 
poisina köitsid mind mehhanismid, tehnika ja asjad oma 
koledas ilus. 

\question{Mis aastal see oli? Kaheksakümnendate keskel?}

1982. aastal läksin esimesse klassi, 
sealt loeme neli-viis aastat edasi, nii et 1986 või 1987. 

\question{Võrus selliseid arvuteid ei olnud, need tekkisid märksa hiljem. 
Teil pidi olema millegi poolest eriline kool, et suudeti arvutid 
välja rääkida. Või oli õpetaja eriline?}

Heiki Pettai\index[ppl]{Pettai, Heiki} oli suhteliselt noor õpetaja, kui ta 
meie kooli tuli, vist otse Tartu Ülikoolist. Ma täpselt ei tea, lihtsalt 
spekuleerin, et tal olid jätkuvalt aktiivsed suhted ülikooliaegse 
kontaktvõrgustikuga, näiteks Viljo Sooga\index[ppl]{Soo, Viljo}. Sealtkaudu võis infot liikuda ja ta 
võis olla õigel ajal õiges kohas, et sai Viljandi kooli midagi hankida. 

\question{Kas arvuteid kasutati ka õppetööks või käis seal ainult 
arvutiring? Kuidas nende kolme masina peal õpetada sai?}

Tagantjärele mõeldes nägi see hästi improviseeritud värk välja küll. 
Ta üritas ka keskkooliõpilastega väikestes rühmades tunde
läbi viia, sest ruumi ei mahtunud 
palju inimesi. DVK-2\index{DVK!DVK-2} masinatel oli olemas graafikakaardi \emph{slot}, 
aga graafikakaarti ühelgi sees ei olnud, jooksis ainult 
tekstipõhine režiim. Lapsed harjutasid, kuidas 
ASCII graafikas tekstiredaktoriga pilti joonistada. Nooleklahvidega ringi
sõites ja sümboleid vajutades sai joonistada 
ning õpetaja pani selle eest hindeid. Lisaks
lihtsamate asjadega tegelejatele oli seal
paar ägedamat, häkkerimat last. Kas Ivar Smolin\index[ppl]{Smolin, 
Ivar} oli juba seal olemas või tuli ta hiljem, kui see klass kolis
kutsekasse? Sealt kerkis järgnevate aastate jooksul teisigi tänapäeval tuntud inimesi, 
nagu Janek Hiis\index[ppl]{Hiis, Janek} ja Kaido 
Kärner\index[ppl]{Kärner, Kaido}.

\question{Kas sina joonistasid ka pilte või tegid midagi muud?} 

Alguses ei osanud ma muud teha, lihtsalt põnev oli arvutit 
katsuda. Jube äge oli klaviatuuri klõbistada ja vaadata, kuidas ekraanil toimub selle 
peale midagi. See oli püha emotsioon, mille nimel 
tasus istuda ja kannatlikult järjekorras oodata. Aeg-ajalt, kui rahvast oli vähem ja kas õpetaja ise või mõni 
edumeelsem õpilane teadis, millise flopiketta peal mängud asusid, 
sai Rottide\index{Rotid}-nimelist mängu mängida, mis oli sisuliselt 
\enquote{Pacmani}\index{Pacman} imitatsioon. Ka \enquote{Snake} jooksis 
kusagilt flopi pealt. Vahepeal sai niisiis mängida, aga õpetaja üritas 
mängimise fooni loomulikult natuke alla suruda -- see oli rohkem nagu 
preemia, kui midagi asjalikku ära tegid. 

\question{Mis oli \enquote{asjalik}? Kas koodi kirjutasite?}

Mõned ägedamad vennad juba kirjutasid koodi ka. Ma ise DVK-2\index{DVK!DVK-2} peal veel koodi 
kirjutamiseni ei jõudnud. Selleks oli vaja 
rohkem vaba aega, kui lapsed parasjagu ei rüselenud 
liiga palju ja sai süveneda. Need, kes käisid 
lähemalt kooli ja said seal hilisematel õhtutundidel istuda, olid 
eelistatud seisus, sest mina elasin linnast väljas 
ja pidin bussigraafikuga arvestama. Niisama lihtsalt seal hilja õhtuni hängida 
ei õnnestunud. 

Koodimiseni jõudsin aastake või paar hiljem, kui see klass 
liikus suuremasse ruumi ja tulid pisikesed 
BKd\index{Elektronika!BK}.\sidenote[][-2mm]{Nõukogude kuueteistbitiste 
koduarvutite sari, mida huvitaval kombel (sest Viljandisse sattusid need teises järjekorras) 
peetakse varem mainitud DVKde 
eelkäijaks.} Nagu tol ajal ikka, tekkisid need kusagilt Vene 
arvutitööstusest,\sidenote{Selle arvuti töötas 1983. aastal välja Zelenogradis 
asunud asutus \begin{russian}НПО \enquote{Научный Центр}\end{russian}, 
toonase Nõukogude Liidu juhtiv mikroprotsessorite disaini 
keskus.} neil olid hästi väikesed monitorid ja väike 
must kandiline aju või plokk, mis oskas kas makilindi pealt 
või siis võrgukaabliga ühendatult emaarvutist andmeid lugeda. Emaarvutiks oli pandud DVK-2\index{DVK!DVK-2}, mille 
ketaste pealt BKd said lugeda mingisuguse protokolliga, 
millest ma ei teadnud tol ajal ega ka tagantjärele midagi. Teadsin ainult, 
mis käske tuleb sisestada, et asjad toimiksid.

BK-l oli kuhugi püsimälusse sisse keevitatud 
BASICu\index{BASIC} interpretaator. Kui selle sisse lülitasid, siis 
esimese asjana tuli ette \verb|line number| \verb|10| -- hakka kirjutama. 
Nii et sisuliselt sai interaktiivselt BASICu käske kirjutada ja siis
hakkasingi esimest korda koodi kirjutama. 

\question{Mida need esimesed programmid tegid?}

Mis see teismelise koolipoisi kõige esimene programm ikka põnevat teeb? 
Kõigepealt oli BASICus käsk number 10, mis printis ekraanile midagi toredat, näiteks 
\enquote{loll}. Järgmine rida 
oli käsk number 20, mille peale oli \verb|goto 10|. Selline tore lõpmatu 
tsükli harjutus, aga sellele järgnesid kiiresti igasugused muud näpuharjutused. 
Seal oli juba graafika olemas, sai ekraanile jooni kuvada, ja siis esimene 
tsükli harjutus oli see, et sai joon kuidagi liikuma pandud ekraani ühest 
servast teise. 

\question{Ega ei saanud ju lihtsalt joont liigutada, eelmine positsioon tuli 
mustaga üle joonistada \ldots}

Jah, just, sealt hakkas vaikselt algoritmika, mis
sundis lapse aju algoritmiliselt mõtlema. Kõik vead paistsid kohe 
välja, kui olid midagi valesti mõelnud. 

\question{Kust see programmeerimisõpetus tuli? Kas õpetajalt või raamatutest?}

See tuli pigem kellegi käest, me lastena ei viitsinud väga
manuaale lugeda. Aeg-ajalt näidati küll, et näe, loe sealt. Need tekstid olid üldjuhul venekeelsed. Vaatad natukene tuima näoga nagu ahv kirjutusmasinat ja siis 
küsid ikka naabripoisi käest, et kuule, kuidas sa seda tegid. 
Mõningaid asju näitas õpetaja, teisi asju mõni teine targem laps -- niimoodi killuke siit ja sealt muudkui korjasid. 

\question{Ja esimene tunne ei läinud üle?}

Ei, üle see ei läinud. Psühholoogiline sõltuvus või 
vajadus arvuti taha istuda ja seal midagi teha oli kogu aeg olemas. 
Eks lastel mängib rolli mängudega jändamise võimalus. Mul oli motivatsioon kohale minna, et äkki saab mängida. Aga 
kuna vaikselt tekkis ka programmeerimise kihk, siis oli see 
piisavalt põnev, et kutsus sinna asju tegema. 

Põhiline oli see, et vahel lubas õpetaja lastel, keda ta rohkem tundis või 
usaldas, pisikest BKd\index{Elektronika!BK} kas koolivaheajaks võiks suveks koju viia. See
oli piisavalt väike, mahtus kotti. Aga sellega oli üks probleem: 
kuna salvestusseadet ei olnud, siis oli 
kaks varianti. Kas tõmbasid makilindilt programmi sisse ja selleks pidid olema
kõik vajalikud kaablid, juhtmed ja oskused, et sellega õigesti 
ümber käia. Või teine variant, et sul oli programmi \emph{printout} ja iga 
kord, kui tahtsid mängida, pidid kõigepealt kogu mängukoodi 
vigadeta sisse toksima. Ühesõnaga, tund-poolteist nägid vaeva ja 
järgmised tund-poolteist said mängida -- see oli päris huvitav kogemus. 
Tagantjärele mõeldes pidid need mängukoodid hästi ökonoomsed olema, et neid sai lühikese ajaga 
sisse toksida. 

\question{Kas teil 
häkkimist ei esinenud? Siin on räägitud,\sidenote{Vt lk 
\pageref{sisu!ylikooli_root}.} kuidas inimesed veel enne keskkooli ülikooli 
adminnide käest \emph{root}-õigused ära võtsid.}

Mul otsest spetsialiseerumist või 
spinni ei tekkinud, et oleks kursi mõne konkreetse 
asja peale võtnud. Olen eluaeg olnud tarkvaraarendaja, ma ei ole 
läinud kuhugi riistvara häkkima ega muud säärast tegema. 

\question{Kas see ei ole huvi pakkunud?} 

Eks vahel on olnud uudishimusähvatusi, aga minu jaoks on kogu aeg 
olnud piisavalt atraktiivne tegeleda mõne uue laheda 
tarkvarakeskkonnaga. 

Ma olin matemaatika-füüsika
süvaklassis ja meil oli keskkoolis eraldi arvutitund. Tegime Jukudes\index{Juku}
dBase'is programmeerimisülesandeid, kus oli vaja programmeerida
andmebaasi- või tabelarvutuselaadseid asju. dBase on FoxPro 
sugulane ja päris äge. Oluline oli see, et 
kogu aeg oli midagi uut avastada, ja midagi muud pole mul motivatsiooniks vaja olnud. Pidevalt on uue 
asja avastamise rõõm. Andi Hektor\index[ppl]{Hektor, Andi} oli mu 
klassivend -- paljud teavad teda kui Eesti üht tuntumat füüsikut. 
Arvutitunnis olime sisuliselt kaks ärksamat pead, istusime ja 
õpetasime vastastikku üksteist ning tegime keerulisemaid asju. 

\question{Kas arvuteid muude ainete, näiteks matemaatikaga ka seoti?}

Mainisin just Andi Hektorit\index[ppl]{Hektor, Andi}, tema 
jaoks olid ilmselt füüsika ja keemia väga köitvad ained. Ta oli 
väga terav ja käis olümpiaadidel, pani neid järjest kinni. Tartu Ülikooli 
füüsikasse sai ta ilma eksamiteta sisse tänu sellele, et oli vahetult enne 
keskkooli lõppu vabariikliku füüsikaolümpiaadi võitnud. Tema puhul 
kindlasti see pool toimis. Minul kukkusid füüsika ja matemaatika kuidagi
loomulikult välja, sain vajalikud hinded kätte ja osalesin isegi 
Tartu Ülikooli matemaatikakoolis, mida keskkooli õpilastele 
kirja teel korraldati. Üritasin valmistuda Tartu 
Ülikooli sissesaamiseks, aga ikkagi rohkem informaatika motiiviga. Mul ei olnud
otsest huvi füüsikavalemitesse kaevuda, tugev emotsioon oli 
arvutite vastu. 

\question{Kas sul muusika- või kirjandushuvi oli ka? Sa muusikamees oled ju olnud?}

Hobi korras mängisin jah kitarri. Nokkisin selle üles 
teismelisena isa kõrvalt, aga klassikalist muusikakooliharidust mul ei 
ole. Kõik, mida ma muusikast tean, olen ise üles korjanud. 

\question{Kas sellist mõtet, et üritaks Jukuga midagi lindistada või muusikat 
teha, ei tulnud?}

Nii kaugele ma tollal ei jõudnud. Mingisugused tüübid tulid ükskord 
arvutiklassi ja lasid päris äratuntava 
kvaliteediga Roxette'i muusikat läbi Juku. Aga see oli ka ainus selline 
moment. 

Üks moment meenub veel, olin siis keskkoolis. 
Lõpetasin keskkooli 1993. aastal, nii et see võis olla 1990ndate algul. Millal 
Bluemooni\index{Bluemoon} tüübid SoundClubi\index{SoundClub}\sidenote[][-.8cm]{SoundClubi hakkasid Ahti\index[ppl]{Heinla, Ahti} ja 
Jaan\index[ppl]{Tallinn, Jaan} kirjutama 1991. aastal Tartus füüsikat õppides 
ja see avaldati \emph{shareware}-litsentsiga 1993. aastal. Samas võisid 
selle versioonid ka varem ringelda.} arendasid? Keskkooli lõpuklassides 
tulid meil 286d -- ma ei mäleta, kas me 386 nägime. 
Igal juhul oli värviline graafiline keskkond juba mingil määral olemas ja 
SoundClub meile sinna kooliarvutitesse jõudis. Sellega sai küll mingit 
tehnomuusikat kokku tõstetud. Mitte et oleksin midagi hullult programmeerinud, oli lihtsalt lahe ja arusaadav kasutajaliides, kus sai
rütmiriffe ja asju kokku pandud, nii palju kui tol hetkel 
muusikalist arusaamist oli.\sidenote{Asko ei olnud ainus. Vennaskonna 
omaaegne hittlugu \enquote{Disko} on loodud SoundClubii abil ja, nagu tegijad on meenutanud, 
umbes samal meetodil.}

\question{Kas arvutiklassis hängiv seltskond oli muus mõttes ka sõpruskond 
või puutusite kokku ainult seal?}

Nii ja naa. Kujunes küll jah välja tuumik, kes sai omavahel ka väga 
hästi läbi. Keskkooli lõpus olid seal peale 
minu ja Andi Hektori\index[ppl]{Hektor, Andi} meist paar aastat 
nooremad Janek Hiis\index[ppl]{Hiis, Janek} ja Janek 
Palõnski\index[ppl]{Palõnski, Janek}. Keegi Kristjan, kelle teine 
nimi praegu ei meenu, oli ka arvutite peal päris kõva tegija. Samuti Raivo 
Kotov\index[ppl]{Kotov, Raivo}, kel on praegu Andrus 
Kõresaarega\index[ppl]{Kõresaar, Andrus} arhitektuuri- ja 
disainibüroo. Andrus oli ka mu klassivend.

\question{Kas sind keskkooliajal tööle ei võetud?}

Kahjuks või õnneks ei tulnud ette. 

\question{Kas Viljandis oleks tollal mõni koht olnud, kes 
oleks võinud programmeerija tööle võtta?}

Ma ei tea, et seal oleks tollal otseselt programmeerija väljavaateid 
olnud. Samas mingid arvutispetsialistid hakkasid küll juba ringi 
toimetama, sest järjest rohkem ettevõtteid võtsid arvuteid kasutusele. Tolleaegse nimega 
Eesti Telefon ja postiasutus panid juba IT-võrke püsti. Valdavalt vist Heiki 
Pettai koordineeriski neid asju. Võibolla kusagil andmeid sisestada oleks 
heal juhul saanud, mis oleks olnud keskkooliõpilase jaoks okei. 

\question{Kas pärast keskkooli lõppu läksid otsejoones Tartu Ülikooli matemaatikat 
õppima?\index{Tartu Ülikool!Matemaatikateaduskond}}

Jah, kuna olin kooliajal õppeedukuse
poolest suhteliselt lohh, siis sain nibin-nabin ülikooli sisse. 
Õnneks oli siis üks madalama konkurentsiga aastaid. Olin esialgses pingereas joone peal täpselt viimane, kes sisse sai. 
Siis olid ju veel sisseastumiseksamid, kombinatsioon 
eksamihinnetest ja lõpuhinnetest. Sain sisse, tekkis jess!-emotsioon ja edasi hakkas ülikoolielu. 

\question{Seal vahepeal oli kummaline periood, kui Nõukogude sõjaväeteenistus 
läks Eesti kaitseväeteenistuseks üle -- kas sa sõjaväes ei käinud?}

Jah, see aken oli hästi lühike ja ma sattusin täpselt sellesse aknasse: 
nõukaaegne armeekord oli 
lagunenud ja sinna ei võetud juba mitu aastat, aga Eestis kehtestati üldine 
sõjaväekohustus 1993. aasta sügisel. Ma olin suvel ülikooli sisse 
saanud ja kõik enne sügist sisse saanud olid justkui vabad, eeldusel et nad ülikooli ära lõpetavad. 

\question{Naljakas aken oli jah! Mina pääsesin tervisega, aga meie kursuselt 
ei käinud keegi sõjaväes.}

Kusjuures mul oli endal selline suhtumine, et 
oleks täitsa okei olnud minna. Tegelesin sel ajal maskuliinsemate spordialadega, näiteks harrastasin karated, ja 
arvasin, et mis see sõjavägi siis ära ei ole -- kui vaja, siis 
teen ära. Aga ära see minu jaoks jäi ja hiljem olin selle üle õnnelik. 
Tollal oli sõjaväesüsteem 
väga lapsekingades ja ei oleks tõenäoliselt midagi 
väga meenutamisväärset olnud. Minu 
klassivendadest kaotas ajateenistuse tõttu oma elu kaks inimest. See 
näitab tollast taset, õnnetusi ja korralagedust oli veel päris palju. 

\question{Kui ma su juttu kuulan, koorub välja suhteliselt haruldane 
kombinatsioon: teeks sporti ja olümpiaadidel pigem ei käiks, aga samas 
programmeeriks isuga.}

Neid asju, millega ma paralleelselt tegelesin, oli tegelikult mitu. Võibolla oli seetõttu raske otsustada, mille juurde jääda. 
Laulmise ja muusika mõttes mul kuulmist oli, aga määravaks sai see, et 
kuna mul muusikakooli haridust ei olnud, siis olin juba 
rongist maas. Kaalusin isegi 
kultuurikolledžisse minekut, aga olin ikkagi suhteliselt lahja 
vend. Spordiga sai tegeldud ja käisin ka kunstiringis, muuhulgas koos 
Kotovi\index[ppl]{Kotov, Raivo} ja Kõresaarega\index[ppl]{Kõresaar, Andrus}. Seda vedas Jakobsoni gümnaasiumis Grünbach.\sidenote{Asko peab ilmselt silmas õpetaja 
Rein Grünbachi\index[ppl]{Grünbach, Rein}.} Aga
lõpuks jäin ikkagi arvutite juurde, kuna tundus, et sellega läheb kõige paremini. Motivatsiooni mõttes need teised asjad 
ilmselt ei kinnistunud nii tugevalt kui arvutid. 

\question{Ja nii astusimegi sinuga koos 1993. aastal matemaatikateaduskonda. 
Esimesed kaks aastat tambiti meile haljast matemaatikat, kuidas see tundus?}

Ülikool oli minu jaoks omaette saaga. Tüütu oli see, et kõik pidid 
esimesed kaks aastat sama programmi õppima. Alguses ei saanud veel 
otsustada, kas minna informaatika, matemaatika või statistika suuna 
peale. Valikuvõimalus anti alles teise aasta keskel ja siis vaadati ka õunte 
pealt, kui hea sa oled ühes või teises asjas. Astusin Tartu Ülikooli 
matemaatikateaduskonda selle tõe pähe, et informaatikasuund on seal 
olemas, aga kas sinna saab, seda alguses ei teadnud. Selles 
mõttes oli ülikool minu jaoks paras \emph{challenge}, sest olin lohh edasi, vähemasti esimestel aastatel. Mul ei läinud
matemaatikaained just kõige paremini, kuna need ei olnud minu jaoks motivatsiooni 
põhipõhjus. Alguses oli päris palju keerulisi asju, näiteks matemaatiline analüüs I ja II.

\question{Matemaatiline analüüs I võttis ju lausa kolmandiku kursusest!}

Jah, see niitis rahvast korralikult, aga see ei olnud veel kõige hullem. Kõige 
hullem oli võibolla isegi algebra Mati Kilbi\index[ppl]{Kilp, 
Mati} väga karmi käe all, kõik asjad tuli korrektselt selgeks saada. Minu jaoks oli
algebra oma abstraktsuses kõige raskemini omandatav. 
Matemaatiline loogika seevastu, mida õpetas tollal levinud folkloori järgi 
üks karmimaid õppejõude Rein Prank\index[ppl]{Prank, Rein}, tuli 
lihtsasti, kuna oma mõttemudeli poolest haakus see programmeerimisega palju 
paremini. 

\question{See oli naljakas aine jah, otseselt keeruline ei olnud, aga ometigi
peeti raskeks.}

Ilmselt mingite inimeste jaoks oli see keeruline, aga meie, programmeerijate 
jaoks tuli see kuidagi loomulikult. 

\question{Vanemuise tänava õppehoones olid laiad aknalauad, mille peal istuti, 
sest kuskil mujal ei olnud istuda. Ja seetõttu värviti neid
regulaarselt üle. Sellele vaatamata oli alati kuskile 
sisse kratsitud \enquote{Prank on loll}.}

Meil oli rebaste vandes, kui sa mäletad, palju lauseid, mida me kõike 
tõotasime, ja üks neist oli \enquote{tõotan Prangile kõik eksamid ära teha 
hiljemalt seitsmendal katsel}. 

\question{Paljudel ilmselt nii läkski. Kas sul oli 
programmeerimis{\-}unistus nii tugevalt silme ees, et ronisid ikkagi matemaatikast läbi?}

Lohistasin ennast läbi, aga kriisimoment oli olemas 
küll, olin tegelikult matemaatika tõttu väljakukkumise äärel. Kuna puhas matemaatika mind väga ei motiveerinud, siis veetsin suurema osa 
ülikooliajast 
arvutuskeskuses\index{Tartu Ülikool!Arvutuskeskus}, 
nõndanimetatud Väksu klassis\index{Tartu Ülikool!Liivi õppehoone!Vase klass}\phantomsection\label{sisu:vase_klass}. Tolleaegse 
vask.ut.ee\index{vask.ut.ee} serveri VT100-terminalid olid ühendatud ühe 
VAX VMSi süsteemi taha. Sealt sain oma esimesed suuremad 
programmeerimise tuleristsed. 

Avastasin seal enda jaoks 
nii-öelda \emph{on the dark side} maailma ehk
mudamängud\index{Muda}. Need olid üheksakümnendate 
esimese poole internetipõhised arvutimängud, virtuaalsed maailmad, kus ei olnud 
midagi graafilist, kõik oli tekstipõhine. Kusagil jooksis server, 
kuhu võeti telnetiga ühendust, ja seal maailmas käis möll ja 
tagaajamine ja \emph{quest}'ide lahendamine. Sattusin üsna kiiresti ise ühe sellise mänguserveri 
programmeerimise meeskonda. Mäng on oma 
olemuselt päris keeruline elukas. Mängumootor peab 
seesmiselt maailma mudeldama, seal on tuhandete viisi 
ruume, liste ja muid asju. Algoritmikat, kuidas seda kõike 
struktureerida, on üksjagu ja see oli minu jaoks üks esimesi tõsisemaid 
C\index{C} programmeerimise kogemusi. 

\question{Kas sa selleks hetkeks juba oskasid Cd?}

Loengutes õpetati programmeerimist Pascali\index{Pascal} baasil, 
nagu sa mäletad. Selle korjasin suhteliselt kergesti üles, kuna see oli 
hea tüpiseeritud keel ja õpetati ka enam-vähem okeilt. Aga kusagil kripeldas, et mingid vennad panevad Cd. Mind
häiris, et ise ei saa. Ostsin eestikeelse C-õpiku, mis oli 
täiesti arusaamatu, kuna oli nii halvasti koostatud. Pealkirja ei mäleta, aga kaanekujundus oli kollane-punane. Selle asemel et näidata esimeses peatükis, kuidas \enquote{Hello 
Worldi} teha, hakati kohe baidi \emph{alignment}'i 
arutama. Kamoon, mis otsast te pihta hakkate! 

Ühel hetkel aga 
jõudis minuni info, et Kernighani ja Richie \enquote{The C Programming 
Language}\index{The C Programming Language}\sidenote{Dennis M. Ritchie, Brian 
W. Kernighan ja Michael E. Lesk. The C programming language. Englewood 
Cliffs: Prentice Hall, 1988. Samast raamatust on juttu ka
lk \pageref{sisu:richie}.} on hea raamat, otse C-keele autoritelt. 
Igatahes kuskilt ma selle endale hankisin. 

\question{Neid liikus ilmselt Venemaal piraadituna ka.\sidenote{Vt lk.
\pageref{sisu:richie_vene}.}}

Mina ostsin täiesti legitiimse raamatu, mitte 
piraaditud väljatrüki. Mul on see vist siiani riiulis alles, kuigi kapsaks 
muutunud ja natuke teibitud. See oli metoodiliselt hea raamat: 
hakkas lihtsatest asjadest pihta ja läks lõpuks \emph{hard core}'ini 
välja. Neelasin selle läbi, tegin kõik harjutused ära ja 
sain C-keele selgeks. Kui üritada näppu peale panna 
raamatule, mis on mu karjääri kõige rohkem mõjutanud, siis see on 
see. 

Ühesõnaga, olin selle teose läbi protsessinud, enne kui jõudsin mängu 
progemiseni.

\question{Mutta vajus meil kursa pealt ka mitu inimest, kes enam 
Vaxu klassist ei väljunudki.}

Mõned jah, rohkem vajus sinna aasta vanemaid. Muda\index{Muda} 
oli tulnud aasta enne seda, kui meie sisse astusime, internetiga umbes 
ühel ajal. Internet tuligi koos paari 
pahega ja see oli üks peamisi. Nii et osa inimesi sattus mängu haardesse.

\question{Kas sina ei sattunud?}

Mul kestis see periood ainult paar kuud, see konverteerus suhteliselt ruttu 
programmeerimise entusiasmiks. 

\question{Kas toda serverit kirjutanud tiim oli Eestis või välismaal?}

Mudast\index{Muda} oli arendatud palju erinevaid versioone, kõik 
olid tollaste vabavaraliste litsentsidega, avatud koodiga, mida sai 
FTP-saitidelt tõmmata. Nendest arenes siin ja seal erinevate tiimide 
käes igasuguseid \emph{fork}'e.\sidenote{\emph{Fork} on tarkvara-maailmas koopia tarkvara
lähtekoodist, mida arendatakse edasi sõltumatult algsest versioonist.} 
Kui \emph{fork}'ide hierarhiat 
joonistada, siis üks kuulsamaid ja levinumaid juur-\emph{fork}'e oli 
DikuMUD\index{Muda!DikuMUD}, millest oli tehtud haru Merc, 
millest omakorda oli tehtud \emph{fork} nimega ROM. Raul Tölp\index[ppl]{Tölp, Raul} 
oli see, kes võttis ühe ROMi-põhise versiooni ja hakkas sellest Eesti 
oma Estonia-nimelist asja arendama. Tolle versiooniga mina liitusingi. Sai seda maailma edasi arendatud ja kohendatud, vigu parandatud ja muid asju tehtud. 

\question{Too oli ju huvitav kogemus, sest erinevalt ülikoolis õpetatavast 
tavalisest programmeerimispraktikast oli tegu meeskonnatööga!}

Jah, spontaanne meeskonnaelement tuli sisse. Laiem meeskond kirjeldas
mängumaailma, maailmafaile oli hästi palju: 
ruumid, kollid, mobprogid,\sidenote{Lühikesed programmijupid, mis käivituvad mängija teatud tegevuste peale ja
võimaldavad kollidel neile reageerida. Samuti on olemas oprogid, mis võimaldavad 
sedasama objektide jaoks.} propsid ja mis seal sees kõik elasid. 
Ägedamad koodivennad läksid järjest rohkem koodi 
sisse. Ühel hetkel üritasime mingi bandega hakata täiesti nullist, 
hoopis uutel alustel MUDi tegema. 

ROMi-põhine oli C-keele baasil ja võrguga suhtlus 
käis \verb|select| \emph{loop}'iga, deskriptorid olid \verb|select| 
listis, kus on oma piirangud -- maksimum tuhatkond 
\emph{connection}'it saab korraga püsti olla ja kõike protsessiti ühes 
\emph{single-threaded} tsüklis. Hakkasime Toomas 
Soomega\index[ppl]{Soome, Toomas}, kes oli tollal arvutuskeskuse süsadmin, 
ja Peeter Lauaga\index[ppl]{Laud, Peeter}, kes oli minust aasta hiljem ülikooliga 
liitunud, tegema täiesti uut arhitektuuri, mis oli C++\index{C++}-põhine 
ja \emph{multi-threaded}, et saada paralleelsus paremaks. 

Üritasime 
\verb|select| deskriptorite listist ja piirangutest lahti saada ning
tegime igasugust ägedat \emph{hardcore} värki, kus asendasime tekstiga 
maailmafailide sisse parsimise mingi kiirema ja efektiivsem asjaga. Kuna 
maailm koosnes tuhandetest ruumidest, siis serveri \emph{boot} võttis 
mitu minutit aega ja kui server \emph{chrash}'is, närisid mängijad küüsi, et 
kaua läheb ja kas saab uuesti sisse tagasi. Püüdsime selle asendada 
mingisuguse asjaga, kus maailmafailid olid eelkompileeritud 
mälu \emph{dump}'ideks, ja need \verb|mmap|'iga faili sisse 
lugeda, et saaks kohe hoobilt, murdosa sekundiga kõik püsti. 
Moproge hakkasime kirjutama dünaamiliselt lingitud libradeks kompileeritud 
failidena, mida sai jooksev server \verb|dlsym|'iga käigu pealt sisse linkida ja 
käivitada. Ühesõnaga, ajasime kontseptsiooni päris keeruliseks ning omandasime Unixi keskkonnas
korraliku \emph{hardcore} C ja C++ häkkimise oskuse.

\question{Kuidas see kõik sulle külge jäi? Lihtsalt õhust?} 

Korjasime vastastikku järjest üles ja toetasime üksteist. Toomas 
Soome\index[ppl]{Soome, Toomas} tegi otsa lahti ja
viskas palju ideid lauale just arhitektuuri osas. Meie 
Peeter Lauaga\index[ppl]{Laud, Peeter} korjasime ideed suhteliselt kiiresti 
üles ja hakkasime üksteist täiendama. 

\question{Teil pidi siis palju vaba aega olema, kas sa tööl ei käinud?}

Sellepärast mul oligi 
ülikoolis püsimisega raskusi. Veetsin suurema osa ajast 
arvutuskeskuses, vahel varajaste hommikutundideni välja, ja tihtipeale 
loengutesse ei jõudnud. Vahepeal oli mahajäämus vajalikes ainepunktides nii 
suur, et oleksingi võibolla kolmanda aasta keskel välja kukkunud, kui ma 
ei oleks ennast kätte võtnud. Mul käis mingisugune klõps, lapsest sai täiskasvanu. 

Ülikoolistress läks hästi tugevaks ja ühel hetkel jätsin 
mängud ja programmeerimise kõrvale ning hakkasin 
järjest laduma ülikooliõpinguid. Panin ühe 
semestriga 40 ainepunkti\sidenote{See oli topelt 
tavapärasest semestri õpikoormusest ja nõudis ilmselt tõesti suurt tööd, sest 
kolmandal aastal väga palju lihtsaid aineid enam järel ei olnud.} jutti, 
et saada ree peale. Kui ma varem ei suutnud 
ennast kokku võtta, siis peale seda olen praktiliselt kogu aeg 
suutnud. Baka sai 1998. aastal lõpetatud. Hinded ei olnud head, sest käis sõna otseses mõttes toores tootmine, et kõik ainepunktid 
kätte saada. 

Kui magistrisse läksin (pidin minema tasulisele kohale, sest 
hinnete tõttu ma riigieelarvelisele kohale ei pääsenud), siis seal tegin
hindelised ained maksimumi peale. 

\question{Miks sa magistrisse läksid? See ei olnud tol ajal 
vaikimisi valik.\sidenote{Toonane bakalaureuseõping kestis nominaalselt neli 
aastat ja võrdsustati hiljem haridustaseme mõttes praeguse magistrikraadiga.}}

Ma ei mäleta, kuidagi tekkis tahtmine. Magistris sain keskenduda 
teemadele, mis mind ennast huvitasid. Kui bakatasemel oli hästi palju 
sunniviisilist programmi, siis magistris valisin ise, mida teha. Tööle 
läksin 1995. aastal ja esimese palga häkkimise eest sain Tartu 
Ülikooli arvutuskeskusest\index{Tartu Ülikool!Arvutuskeskus}. 

Viljo Soo\index[ppl]{Soo, Viljo} andis operatsioonisüsteemide loengut\sidenote{Viljo 
loetud operatsioonisüsteemide ainel oli väga hea maine, seda peeti keeruliseks 
ning huvitavaks ja seal silma paista oli väga kõva sõna.} ja kord 
ühe loengu vahepausi ajal tuli minu juurde ja küsis, kas tahaksin natukene tööd ka teha. Ta oli ilmselt 
tähele pannud, et korjasin ehk
mingeid asju ladnamalt üles. Arvutuskeskuses oli süsadminnimise 
kõrval ja käigus vaja aeg-ajalt erinevaid asju arendada ja nii mind võetigi 
sinna programmeerijaks. 

\question{Tol hetkel oli võimalik juba ka arvutitega äri teha, kas see ei tõmmanud 
sind?}

Jah, neid tegijaid ümberringi toimetas, aga 
õnneks või kahjuks suutsin niisugustest ahvatlustest kõrvale hoida. Näiteks Alo Toom\index[ppl]{Toom, Alo} ja Ülo Säre\index[ppl]{Säre, 
Ülo} tegid alguses A ja Ü\index{A ja Ü|see{PC Expert}}, millest arenes välja PC 
Expert\index{PC Expert}. Ja mõnda aega eksisteeris niisugune tarkvarafirma 
nagu Codewiser\index{Codewiser}, mis on ka samast ettevõtete perest. Minu lapsepõlvesõber Alo üritas mind A ja Üsse tööle ahvatleda, 
aga tollal oli see rohkem tehnika \emph{support}'i ja konsultatsiooni 
firma, nad aitasid näiteks klientidel katkiläinud kõvakettaid taastada. Mind see eriti ei köitnud, mul oli emotsioon rohkem 
programmikoodi suunas. 

Samuti üritati mind meelitada meditsiinivallas tegutsevasse 
AtFuti\index{AtFut}, mille asutas Jaan 
Pruulmann\index[ppl]{Pruulmann, Jaan}. Teda kutsutakse ka
papa Pruulmaniks, sest Pruulmannide dünastia on suur ja 
lai, neil on seitsmevennaline perekond, kellest suurem osa on täna 
teada-tuntud IT-tegijad. Üks tuttav nimega Elvar Vask\index[ppl]{Vask, Elvar} \emph{alias} 
Cuprum\index[ppl]{Cuprum|see{Vask, Elvar}} sebis mind sinna 
töövestlusele ja ma vist küsisin liiga kõrget 
palka ning sel hetkel lõppes meie diskussioon ära. Muidu tundus, et klikk oli. 

\question{Kas sind akadeemiasse teadust tegema ei tõmmanud?}

Seda, et oleksin hullult tahtnud teadlaseks saada, ei olnud. Pigem tõmbas mind ikka
praktiline programmeerimine ja häkkimine. 

\question{Sa pidid kiusatusele kõvasti vastu seisma, nii mõnigi libastus ja jäi koolist kõrvale. 
Aga sina suutsid keskenduda?}

Selle mänguga seoses tekkis aeg-ajalt ikka unistusi. Arutasime tiimiga, et võibolla õnnestub sellest midagi kommertsiaalsemat 
välja arendada, 
aga ilmselt ei saanud me tervikpilti piisavalt hästi 
kokku, et see tegevus kuhugi välja 
viiks. 

Oma karjääri alustasin ikkagi palgatöölisena. Kõigepealt töötasin Tartu Ülikooli arvutuskeskuses ja
vahetult enne baka lõppu, 1998. aasta talvel läksin
ProMedi\index{ProMed}, mis just samal kevadel liitus Magnum 
Medicaliga\index{Magnum Medical}, millest hiljem sai Magnum ProMed. Seega
olen kaudselt olnud seotud ka viimasel ajal palju kõneainet pakkuva 
Margus Linnamäe\index[ppl]{Linnamäe, Margus} tegutsemise algusaegadega. 

Selle ühinemise käigus kogu ProMedi seltskond koondati kohe. 
Sain jube mugava paketi: tegin paar kuud sisulist tööd ja siis tuli 
koondamisteade, mis tolle aja reeglite järgi tähendas seda, et kui koondatakse 
rohkem kui 20 inimest, siis makstakse neile nelja kuu raha. Sain ilusti 
oma baka ära lõpetada niimoodi, et ei olnud vaja 
muretseda lauale leiva saamise pärast. Suvel läksin tööle
Medisofti\index{Medisoft}, mis tegeleb tänaseni raviasutuste 
infosüsteemide arendusega ja kus tol hetkel oli valdavaks arenduskeskkonnaks 
Borlandi Delphi\index{Borland Delphi}. See oli ülikooli algusaegadest 
tuttava Pascali keele põhjal, aga täiesti uus graafiline keskkond, kus oli hästi mõnus 
\emph{desktop}-rakendusi teha. Seal tegelesin niisiis
\emph{desktop}-rakenduste ja andmebaaside kokkupanekuga. 

Mul oli veel paar alternatiivi, kuhu tööle 
minna. Tagantjärele mõtlen, et tegin huvitava otsuse -- 
teistpidi otsustades oleks mu elu võibolla praegu teistsugune. Üks variant oli Medisoft ja teine see punt, kus toimetasid kursusekaaslased Rene Prillop\index[ppl]{Prillop, Rene} 
ja Mati Muts\index[ppl]{Muts, Mati}; seesama 
tuumik, kes hiljem Eesti-poolset PlayTechi\index{PlayTech} asutasid ja 
tegid. Nad pakkusid kõrgemat palka, aga ma kaldusin tollal
alalhoidlikkusele ja Medisofti pakkumine tundus parem, kuna seal olid kindlama meelega
vanemad tegijad ja stabiilsemad asjad. Läksin 
selle peale välja. Alguses oli hästi huvitav, sest õppisin
iga päev midagi uut -- mul on kogu aeg millegi uue saamise 
motivatsioon. Olin Medisoftis 1998. aasta suvest 1999. aasta suveni, 
kui mind kutsuti tööle Küberisse\index{Küber}. 

Küberis hakkasid juba naljakamad ja põnevamad 
lood, mis võibolla haakuvad tänapäevaga rohkem. Sattusin 
sinna täpselt sellel kuumal momendil, kui põhituumik -- Tarvi 
Martens\index[ppl]{Martens, Tarvi}, Aarne Ansper\index[ppl]{Ansper, Arne}, 
Viljar Tulit\index[ppl]{Tulit, Viljar} ja Monika Oit\index[ppl]{Oit, Monika} 
-- oli koos ja parasjagu visioneeriti DigiDoci ja muud põnevat. Krüptoteadlaste kambast olid seal Ahto 
Buldas\index[ppl]{Buldas, Ahto}, Helger Lipmaa\index[ppl]{Lipmaa, Helger} ja Jan 
Villemson\index[ppl]{Villemson, Jan}, samuti
Meelis Roos\index[ppl]{Roos, Meelis}, Peeter Laud\index[ppl]{Laud, 
Peeter} ja Ville Hallik\index[ppl]{Hallik, Ville}. Niisugune ajutrust, kõik tuntud korüfeed! 
Esimese hooga sattusin kohe C++-s\index{C++} programmeerima, 
Visual Studios DigiDoci kliendi prototüüpi. 

Teadlased leiutasid ajatemplite linkimise skeeme 
räsiahelate otsas ja samuti arendati digitaalse notari kontseptsiooni. Tollal me veel ei teadnud, et kümme aastat hiljem 
hakatakse seda nimetama \emph{blockchain}'iks. Iga natukese aja tagant tuleb keegi 
välja järjekordse teooriaga, kes on Satoshi Nakamoto.\sidenote[][]{Bitcoini leiutaja või leiutajate poolt kasutatav pseudonüüm.} Ka \emph{blockchain}'i puhul otsitakse ikka veel selle leiutajat. Paar aastat 
tagasi\sidenote{Jutuajamine Askoga toimus 2020. aasta jaanuaris.} tuli üks USA jurist 
lagedale teooriaga, et see oli Helger Lipmaa\index[ppl]{Lipmaa, 
Helger}.\sidenote{Keegi Justin Sobaje 2018. aastal tuli sellise teooriaga, 
mida Helger visalt eitas, tõesti välja.} Nii et meil on oma 
Satoshi Nakamoto olemas. Ahto Buldaselt\index[ppl]{Buldas, Ahto} võeti ka 
selle jutu peale intervjuu ja tema muigas ning ütles, et põhimõtteliselt oleksime me kõik 
võinud Satoshid olla. 

\question{Lähme korraks üheksakümnendate juurde tagasi. Kuidas sul arvuti ja muu 
maailma tasakaalustamine käis? Matemaatikateaduskonna ja arvutuskeskuse 
ümber käis ju äge seltsielu.}

Arvutuskeskuses istus ööde viisi seltskond koos 
ja toimetas oma asju. Üks püsiv ja stabiilne kaader olid 
Muda\index{Muda} mängijad. Osa inimesi \emph{chat}'is 
tolleaegsetes varajastes telnetipõhistes jutukates -- 
tänapäeval on meil nende asemel Facebook ja muud kohad. Ja mingi seltskond tegi asjalikumaid asju, näiteks õpingutega seotult. 

\question{Tehnikaülikooli arvutuskeskuse kohta on 
öeldud, et see oli nagu pooleldi klubi, kus käisid ka need, kes enam ammu ülikoolis 
ei õppinud.}

Jah, ja Tartu Ülikooli arvutuskeskuse kohta võis üheksakümnendatel 
enam-vähem sama öelda, sest seal ei olnud ainult üliõpilased. Sealt 
käis läbi igasugust rahvast ka väljastpoolt ülikooli ning oli tekkinud 
kamp, kus kõik tundsid enam-vähem kõiki, kes seal stabiilselt käisid. 

\question{Tartus oli vähemalt üks selline kogunemiskoht ka tähetorn.}

Neid kohti oli isegi rohkem ja tuumikud puutusid omavahel 
kokku ka, aga kokkuvõttes oli mitu gravitatsioonikeskust. Tähetorn oli 
võibolla isegi üks suletumaid ja väiksemaid, ma ise sinna
eriti ei sattunud. Aga lisaks Tartu Ülikooli 
arvutuskeskusele\index{Tartu Ülikool!Arvutuskeskus} oli 
füüsikahoone\index{Tartu Ülikool!Füüsikahoone}, kus Ville 
Hallik\index[ppl]{Hallik, Ville} ja Otto Teller\index[ppl]{Teller, Otto} 
vägesid juhatasid ja asju kontrolli all hoidsid. Seal käisid
füüsikud ja ka sotsiaalteaduskonna 
tudengid. Veel üks
gravitatsioonikeskus oli ülikooli peahoone kõrval kunagine Marksu 
maja\index{Tartu Ülikool!Marksu maja}, mille all oli üks punkt, kus ma ka
ise mingil perioodil üpris tihti viibisin. Keemiahoone\index{Tartu 
Ülikool!Keemiahoone} all oli ka midagi, aga see toimis vist lühemat aega ja
oli natuke ebamäärasem. 

\question{Kui paljud arvutuskeskuse seltskonnast oli matemaatikud? Kui Anne Villems\index[ppl]{Villems, Anne} oma esimesed 
veebmasterite kursused tegi, käis seal kuuldavasti igasugust rahvast psühholoogidest 
usuteadlasteni.}

Teistest teaduskondadest tean ma üksikuid inimesi, nii et ma
statistilist pilti anda ei oska. Kuna arvutuskeskuse 
palgaline kaader oli valdavalt ikkagi 
matemaatika-informaatikateaduskonnaga tihedalt seotud, siis paratamatult oli 
selle teaduskonna tudengkond seal ka kõige rohkem esindatud. 
Praegu tuntud nimedest oli näiteks Jaanus Lillenberg\index[ppl]{Lillenberg, 
Jaanus}, kes praegu juhib ERRis IT-vägesid, mingil perioodil seal hästi 
aktiivne külaline.\sidenote{Jaanuse seiklustest Liivi tänaval loe lähemalt 
lk\pageref{sisu!jaanus_liivi_tn}.} Tema teaduskondlik taust oli vist
midagi muud, ta ei olnud IT vallast. 

\question{Mida sa praegu teed?}
Praegu olen ettevõtja. 

\question{Vist juba üle kümne aasta?}

Selgelt ettevõtjaks sain ma pärast Skype'i\index{Skype}. Enne 
seda olid aeg-ajalt mingisugused unistused ja visioonid ning
projektilaadsed eksperimendid, aga esimese OÜ, Mooncascade'i\index{Mooncascade} asutasin oma väriseva 
käega paar kuud enne Skype'ist lahkumist, 2007. aasta septembris. 
Sellega olengi kõige rohkem seotud olnud. 

Mooncascade'i visioon tekkis mul juba Skype'st lahkudes, aga 
eestlastest Skype'i asutajate punt, Ambient Sound 
Investmenti seltskond, kutsus mind kohe jutule, sest neil oli 
käsil üks inkubaatorilaadne eksperiment, kus jooksis neli 
erinevat projekti. Nad kutsusid mind ühte nendest projektidest vedama 
ja kuna pundis oli ka Ahti Heinla\index[ppl]{Heinla, Ahti}, kes 
täna veab Starshipi, siis minu jaoks oli piisavalt motiveeriv panna Mooncascade seniks riiulile. Nii me tegimegi paar aastat 
ühte suhteliselt jõhkrat andmekaevelaadset projekti, mis äriliselt 
lõpuks ikkagi lendu ei läinud ja sai ära konserveeritud. Seejärel sai Mooncascade 
reaktiveeritud. Mooncascade alustas aktiivset tegutsemist 2009. aasta lõpus või 2010. aasta 
alguses ja tegutseb tänaseni. 
