\label{sisu:asko}
\index[ppl]{Seeba, Asko}

\question{Kuidas jõudsid arvutid sinu juurde ja sina arvutite juurde?}

Siis ma ilmselt peaks rääkima sellest, et ma  elasin oma teadliku lapsepõlve 
Viljandimaal. Ma olin sealt Viljandist Riia maanteed pidi natukene välja sõita, 
 linna piirist viis-kuus kilomeetrit, ja  käisin sealsamas linna servas Carl 
Robert Jakobsoni nimelises Viljandi esimeses keskkoolis\index{Koolid!Viljandi 
1. Keskkool}, selline oli tema nõukogude aegne nimi. Hiljem oli ta Jakobsoni 
gümnaasium ja nüüd peale riigigümnaasiumite tegemist gümnaasiumi enam ei ole, 
Jakobsoni kooli nime all on ainult põhikool. 

Aga arvutiteni jõudsin ma sealsamas koolis. Meil oli selline tore arvutiõpetaja 
nagu Heiki Pettai\index[ppl]{Pettai, Heiki}, kes vist isegi praegugi on kusagil 
IT-vallas tegutsev, küll enam ammu mitte õpetajana aga rohkem  
spetsialistina. Ma arvan, et ma olin kusagil kas neljandas või viiendas 
klassis ja teadsin, et mul naabripoiss Toomas Aas\index[ppl]{Aas, Toomas} (praegu on üks kõvemaid arendajaid tarkvarafirmas) käib 
 arvutiringis. Ta oli sellest 
rääkinud, aga minu teadmised arvutist olid sellised hästi  lapselikud. Kuskilt 
telekast olin näinud, mingisuguses lastesaates näidati arvutit ja 
tehti midagi naljakat sellega, aga ma ei osanud sellest nagu tollel hetkel 
midagi arvata. Ta jäi minu jaoks nagu kaugeks teemaks. 

Aga see moment, kui mul see klõps käis, oli sihuke hästi ootamatu ja hästi 
lühike, enam-vähem sekundite moment. Ma ükskord sattusin nägema koolis koridori 
peal, kuidas seesama naabripoiss läks  arvutiklassi  pisikesest uksest sisse, 
seal oli ainult kolm arvutit, vene DVK kahed\index{Arvutid!DVK!DVK-2}. Ta läks 
 uksest sisse ja ma sellel hetkel just sattusin nägema, kui see uks nagu 
paotus, ma nägin, et arvutid! See oli minu jaoks tollel hetkel mingisugune 
maagiline moment, et ma nägin midagi, mida ma olin telekast näinud ja mida olin 
väga kaugeks pidanud ja see oli nüüd järsku minust kahe, kahe või kolme meetri 
kaugusel. See oli nagu mingisugune klõps moment. Ma küsisin, et \enquote{kas ma 
tohin ka sisse tulla?} Noh, täiesti spontaanselt. Arvutiõpetaja, Heiki Pettai,  
lubas, ja ma praktiliselt olin momentidega müüdud, et ma tahan siin käima 
jääda. Sellest hetkest peale ma teadsin, et minu ülejäänud elu peab olema 
arvutitega seotud. 

\question{Oskad sa öelda, mis täpselt see maagiline asi oli?}

Mis see lapsel võis olla. Et sa näed mingisugust lahedat asja, millest sa aru 
ei saa. Tolleaegsed arvutid, nad olid natukene rohkem sellised, inglise keeles 
on tore väljend \emph{exposed}, füüsiliselt rohkem nagu avatud, et sa näed seal 
igast keerulisi asju, juhtmed ja värgid ja trükiplaadid ja möllud. Tolleaegne 
poiss nagu ma olin,  igasugused mehhanismid, tehnika ja asjad oma sellises 
koledas ilus alati köitsid mind. 

\question{Mis aastal see oli? Kaheksakümnendate keskel?}

Mis ma siis võisin olla? Neljas, viies klass? 1982 ma läksin esimesse klassi, 
sealt loeme neli-viis aastat edasi, siis kuskil 1986 või 1987. 

\question{Võrus selliseid arvuteid ei olnud ja need tekkisid märksa hiljem. 
Teil pidi olema siis ikkagi millegi poolest eriline kool, et suudeti arvutid 
välja rääkida. Või oli õpetaja selline?}

Heiki Pettai\index[ppl]{Pettai, Heiki} oli suhteliselt noor õpetaja, kui ta 
meie kooli tuli, vist otse Tartu Ülikoolist. Ma täpselt ei tea, lihtsalt 
spekuleerin hilisemate mingite vihjete ja asjade järgi, et tal olid jätkuvalt 
mingisugused suhteliselt aktiivsed suhted  oma ülikooliaegse 
kontaktivõrguga. Viljo Sooga\index[ppl]{Soo, Viljo} ta suhtles aktiivselt, nii 
palju, kui ma tean, ja sealt kuskilt võis mingit infot liikuda ja ta lihtsalt 
võis olla õigel ajal õiges kohas, et sai sinna Viljandi kooli  midagi hankida. 

\question{Kas neid arvuteid kasutati õppetööks ka või käis seal ainult 
arvutiring? Kuidas sa nende kolme masina peal õpetad?}

Tagantjärgi mõeldes ta nägi hästi selline improviseeritud värk välja küll. 
Ta  üritas ka mingisuguseid tunde seal  keskkooli õpilastega väikeste rühmade 
kaupa läbi viia, sest sellesse ruumi ei mahtunud 
palju  inimesi sisse. Nendel DVK-2\index{Arvutid!DVK!DVK-2} masinatel oli olemas graafikakaardis \emph{slot}, 
aga  graafikakaarti neil ühelgi sees ei olnud,  jooksis ainult 
tekstipõhine režiim. Sisuliselt, mida tehti, et  lapsed harjutasid, kuidas 
ASCII graafikas mingisugust pilti joonistada tekstiredaktoriga. Nooleklahvidega 
sõitis ringi ja vajutas mingeid sümboleid, joonistas mingisugust pilti  
ja õpetaja pani mingeid hindeid selle eest. Need olid sihukesed, ütleme, 
lihtsamad asjad, mida seal lapsed tegid. Aga tal olid seal tekkinud ka juba 
paar ägedamat, sihukest häkkerimat last kah. Kas Ivar Smolin\index[ppl]{Smolin, 
Ivar} oli juba seal olemas, või ta tuli natukene hiljem siis kui see klass 
kutsekasse kolis? Sealt kerkis  järgnevate aastate jooksul teisi inimesi ka, 
keda me täna teame, nagu Janek Hiis\index[ppl]{Hiis, Janek}, Kaido 
Kärner\index[ppl]{Kärner, Kaido} on sealt pärit ja \ldots Mul on paar inimest 
veel silme ees, üritan nime esile manada.

\question{Kas sina joonistasid ka pilte või millega sina seal tegelesid} 

No alguses, ega ma ka muud teha ei osanud, oli lihtsalt põnev, et sain arvutit 
katsuda. Jube äge, klõbistan siin klaviatuuri ja ekraani peal toimub selle 
peale midagi. See oli nagu sihuke püha emotsioon, mille nimel 
tasus istuda ja kannatlikult järjekorras oodata. Vahepeal aeg-ajalt 
mingitel momentidel, kui  rahvast vähem oli, kas õpetaja ise või mõni 
edumeelsem õpilane teadis, kuskohas mingisuguse flopiketta peal  mängud asusid. 
Siis sai rottide\index{Mängud!Rotid} nimelist mängu mängida, mis oli sisuliselt 
Pacmani\index{Mängud!Pacman} imitatsioon ja Snake\index{Mängud!Snake} jooksis 
kusagilt flopi pealt. Vahepeal sai mängida ka, aga õpetaja üritas seda 
mängimise fooni loomulikult natuke alla suruda, see oli nagu rohkem nagu 
preemia millegi eest, kui midagi asjalikku ka ära tegid. 

\question{Mis see \enquote{asjalik} oli? Koodi kirjutasite?}

Mõned sihukesed ägedamad vennad juba kirjutasid koodi ka, jah. Oli  paar 
sellist asjalikumat tegelast tõesti. Ma ise tollel hetkel DVK-2\index{Arvutid!DVK!DVK-2} peal veel koodi 
kirjutamiseni ei jõudnud. Selleks oli ikkagi vaja 
natukene rohkem sellist vabamat aega,  kui parasjagu seda laste rüselemist 
liiga palju ei olnud, et sa saad  süveneda. Need, kes käisid 
lähemalt kooli ja said seal hilisematel õhtutundidel istuda, olid 
selles mõttes natuke eelistatud seisus, sest mina olin kusagil linnast väljas 
ja pidin bussigraafikuga arvestama.  Niisama lihtsalt seal hilja õhtuni hängida 
ei õnnestunud. Koodimiseni ma jõudsin aastakese või paar hiljem, kui see klass 
liikus kuhugi natukene suuremasse ruumi ja tulid sihukesed pisikesed asjad 
BK-d\index{Arvutid!Elektronika!BK}\sidenote{Nõukogude kuueteistbitiste 
koduarvutite sari, mida huvitaval kombel (sest Viljandisse sattusid nad teises järjekorras) 
peetakse varem mainitud DVK-de 
eelkäijaks.}. Nagu tol ajal ikka, kusagilt vene mingisugusest 
arvutitööstusest\sidenote{Selle arvuti töötas 1983. aastal välja Zelenogradis\index{Zelenograd} 
asunud asutus nimega \begin{russian}НПО "Научный Центр"\end{russian}, mida 
peetakse toonase Nõukogude Liidu juhtivaks mikroprotsessorite disaini 
keskuseks.}, neil olid sihukesed  hästi-hästi, väikesed monitorid ja  väikene 
must kandiline  aju või plokk, mis oskas kas  makilindi pealt 
või siis võrgukaabliga ühendatult ema-arvutist andmeid lugeda. Ema-arvutiks oli pandud DVK-2\index{Arvutid!DVK!DVK-2}, mille 
ketaste pealt BK-d said  lugeda mingisuguse protokolliga, 
millest ma ei teadnud tollel ajal ega ka tagantjärgi midagi. Ma teadsin ainult 
mis käske tuleb sisestada  selleks, et asjad toimiksid.

BK-l oli selilne huvitav asi, et tal oli kuhugi püsimälusse sisse keevitatud 
Basic-u\index{Keeled!BASIC} interpretaator. Kui ta sisse lülitasid, siis 
esimene asi, mis ette tuli, oligi kohe \emph{line number 10}, hakka kirjutama. 
Sai sisuliselt nagu interaktiivselt Basic-u käske kirjutada ja sealt ma 
hakkasin reaalselt esimest korda koodi kirjutama. 

\question{Mida need esimesed programmid tegi? Umbeski?}

Mis see kõige esimene programm ikka põnevat teeb ühel teismelisel koolipoisil? 
Kõigepealt, ma täpselt ei mäleta, mis see Basic-u käsk oli, aga kõigepealt oli 
mingisugune käsk, mis printis ekraanile midagi toredat, nagu näiteks 
\enquote{loll} või midagi sellist. See oli käsk number 10. Järgmine rida 
oli käsk number 20, mille peale oli  \verb|goto 10|. Selline tore lõpmatu 
tsükli harjutus, aga sellele kiiresti järgnesid igasugused muud näpuharjutused. 
Seal oli juba graafika olemas, sai jooni kuvada ekraanile ja siis esimene 
tsükli harjutus oli see, et sai joon kuidagi liikuma pandud ekraani ühest 
servast teise. 

\question{Ega ei saanud ju lihtsalt joont liigutada, eelmine positsioon tuli 
mustaga üle joonistada \ldots}

Jah, just, sealt hakkas vaikselt see algoritmika, selles mõttes päris huvitav 
asi, et sundis nagu lapse aju algoritmiliselt mõtlema. Kõik vead paistsid kohe 
välja, kus sa valesti olid midagi mõelnud. 

\question{Kust see programmeerimise õpetus tuli? Õpetaja või raamatud või veel 
midagi?}

See tuli pigem kellegi käest kogu aeg, ega me seal lastena väga ei viitsinud 
mingisuguseid manuaale või asju lugeda. Aeg-ajalt näidati küll, et 
\enquote{näed, loe sealt}. Need tekstid olid reeglina  venekeelsed ja nii 
edasi. Et vaatad nagu natukene tuima näoga, nagu ahv kirjutusmasinat, ja siis 
ikka lähed küsid naabripoisi käest, et \enquote{kuule, kuidas sa seda tegid}. 
Mõningaid asju näitas õpetaja, mõningaid asju näitas mõni teine targem laps, 
kes seal kõrval oli, niimoodi killuke siit ja killuke sealt  muudkui korjasid. 

\question{Ja esimene tunne ei läinud üle?}

Ei, see üle ei läinud. See niisugune psühholoogiline nii-öelda sõltuvus või 
vajadus nagu arvuti taha istuda, seal midagi teha, oli ikka kogu aeg olemas. 
Eks lastel muidugi  mängib mingisugune mängudega jändamise võimalus ka,  eks 
ole. Et motivatsioon kohale minna, et äkki saab midagi mängida või midagi. Aga 
kuna see programmeerimise  kihk ka vaikselt tekkis, siis oli  ka see 
piisavalt põnev, et  kutsus sinna mingeid asju tegema. Põhiline asi oli see, 
et vahest õpetaja lubas lastel, keda ta juba rohkem tundis või 
usaldas, seda pisikest BK-d\index{Arvutid!Elektronika!BK} kas koolivaheajaks võiks suveks koju viia. Ta 
oli sihuke  piisavalt pisike, mahtus koti ära ka. Aga sellega oli see probleem, 
et kuna tal mingit salvestusseadet omal loomu poolest ei olnud, siis oli  
kaks varianti. Kas tõmbasid makilindi pealt programmi sisse, selleks pidid 
kõik vajalikud kaablid ja juhtmed ja oskused olemas olema, et sellega õigesti 
ümber käia. Või teine variant oli see, et sul oli programmi \emph{printout} ja iga 
kord, kui sa mängida tahtsid, pidid kõigepealt kogu mängu koodi ilma 
vigadeta sisse toksima. Ühesõnaga, tund-poolteist nägid vaeva ja siis 
järgmised tund-poolteist said mängida, et see oli päris huvitav kogemus. Aga 
tagantjärgi mõeldes, need pidid hästi ökonoomsed mängu koodid olema, et ta oli 
niimoodi sisse toksitav teatud lühikese ajaga. 

\question{Mingil hetkel see arvuti huvi pidi tõsisemaks ka minema. Kas teil 
sellist häkkimise asja ei tekkinud? Siin on räägitud\sidenote{Lehekülg 
\pageref{sisu!ylikooli_root}.}, kuidas inimesed veel enne keskkooli ülikooli 
adminide käest root õigused ära võtavad?}

Mul nagu mingisugust sellist otsest väga spetsialiseerumist või mingit sihukest 
spinni otseselt ei tekkinud, et ma nagu oleks kursi mingisuguse väga konkreetse 
asja peale võtnud. Ma olen eluaeg olnud tarkvaraarendaja,  ma ei ole nagu väga 
läinud kuhugi mingist riistvara häkkima ega mingeid sihukesi asju tegema. 

\question{Huvi ei ole pakkunud?} 

Noh, eks vahest nagu mingeid uudishimu sähvatusi on, aga minu jaoks on kogu aeg 
olnud piisavalt atraktiivne see, et kui on mingisugune lahe uus  
tarkvarakeskkond jälle, et siis sellega tegelda. Ma mäletan hiljem jõuti juba 
Jukudeni\index{Arvutid!Juku}, ma olin matemaatika-füüsika nii-öelda 
süvaklassis, meil oli keskkoolis juba eraldi arvutitund. Neis me tegime 
dBase-ses mingeid programmeerimisülesandeid, kus oli vaja programmeerida
andmebaasi või tabelarvutuse laadsed asju. dBase on FoxPro 
sugulane või midagi sihukest. See oli päris äge, eks ole. Oluline oli see, et 
kogu aeg oli midagi uut, kogu aeg oli midagi avastada ja midagi muud minul nagu 
 motivatsiooniks ei ole kunagi väga palju vaja olnud.  Mingi uue 
asja avastamise rõõm on kogu aeg. Andi Hektor\index[ppl]{Hektor, Andi} oli mul 
klassivend tollel ajal, paljud teavad teda kui Eesti ühte tuntumat füüsikut. 
Arvutitunnis olime temaga sisuliselt kaks ärksamat pead seal, istusime ja 
õpetasime vastastikku üksteist ja tegime mingeid keerulisemaid asju. 

\question{Kas arvuteid muude ainetega, näiteks matemaatikaga, ka seoti?}

Nojah, sellega on omaette naljad. Ma mainisin siin just Andi Hektorit\index[ppl]{Hektor, Andi},  tema 
jaoks ilmselt  füüsika ja keemia olid väga köitvad ained. Oli näha, et ta oli 
väga terav seal ja käis olümpiaadidel, pani neid järjest kinni. Tartu Ülikooli 
füüsikasse sai ta ilma eksamiteta sisse tänu sellele, et ta oli vahetult enne 
keskkooli lõppu vabariikliku füüsikaolümpiaadi ära võitnud. Tema puhul 
kindlasti see pool toimis.  Minul füüsika ja matemaatika kukkusid kuidagi, noh, 
loomulikult välja, sain oma vajalikud hinded kätte ja isegi 
osalesin mingis Tartu Ülikooli matemaatika koolis, mida keskkooli õpilastele 
kirja teel  korraldati. Üritasin  valmistuda selleks, et suudaks  Tartu 
Ülikooli sisse saada, aga ikkagi rohkem informaatika motiiviga. Mul nagu 
otsest huvi ei olnud, et ma kuidagi oleks kaevunud füüsika valemitesse ja 
asjadesse. Ma sain sellest aru, see tuli kuidagi loomulikult, aga mul ei olnud 
niisugust väga tugevat emotsiooni, see oli kogu aeg ikkagi 
arvutite suunas. 

\question{Muusika? Kirjandus? Sellised asjad? Sa muusikamees oled ju olnud?}

Hobi korras ma jah, mängisin kitarri. Kitarri ma nokkisin üles kusagil 
teismelisena isa kõrvalt, aga  mul seda klassikalist muusikakooli haridust ei 
ole. Kõik, mis ma muusikast tean, ma olen ise üles korjanud. 

\question{Aga sellist mõtet, et üritaks Jukuga midagi lindistada või muusikat 
teha, ei tulnud?}

Nii kaugele tollel hetkel ei jõudnud. Mingisugused tüübid tulid ükskord  
arvutiklassi ja lasid  tolle hetke kohta päris ägedalt äratuntava 
kvaliteediga mingisugust Roxette\sidenote{1986. aastal moodustatud Rootsi 
pop-rock duo.} muusikat  läbi Juku. Aga see oli ka ainus selline 
moment. Üks moment tegelikult meenub, ma olin siis veel keskkoolis. Ma 
lõpetasin keskkooli 1993, eks ole, see võis olla mingi 91? 92? 93? Millal need 
Bluemoon-i\index{Bluemoon} tüübid seda SoundClub-i\index{SoundClub} 
arendasid\sidenote{SoundClubi hakkasid Ahti\index[ppl]{Heinla, Ahti} ja 
Jaan\index[ppl]{Tallinn, Jaan} kirjutama 1991. aastal Tartus füüsikat õppides 
ja see avaldati \emph{shareware} litsentsiga 1993. aastal. Mis ei tähenda, et 
selle versioonid ei võinud juba palju varem ringelda.}? Keskkooli lõpuklassides 
meil juba tulid mingisugused 286-d ja vot ma ei mäleta, kas me 386-t nägime. 
Igal juhul värviline  graafiline keskkond oli juba mingil määral olemas ja 
SoundClub meile sinna kooli arvutitesse jõudis. Sellega sai küll juba mingit 
tehnomuusikat kokku tõstetud. Mitte, et ma seal midagi hullult programmeerinud 
oleks, oli lihtsalt sihuke lahe arusaadav kasutajaliides ja  seal sai siis 
igasuguseid rütmi riffe ja asju kokku pandud  nii palju, kui tollel hetkel  
muusikalist arusaamist oli\sidenote{Asko ei olnud siin ainus. Vennaskonna 
omaaegne hittlugu Disko on loodud SoundClubi-i abil ja, tegijad on meenutanud, 
umbes samal meetodil.}. 

\question{Kas see arvutiklassis hängiv seltskond oli muus mõttes ka sõpruskond 
või ta puutus kokku ainult arvutiklassi ümber?}

Nii ja naa. Kujunes küll jah välja mingisugune tuumik, kes  omavahel ka väga 
hästi üksteisega läbi said. Keskkooli lõpus, kes seal tuumikus olidki, peale 
minu ja Andi Hektori\index[ppl]{Hektor, Andi}  olid seal meist paar aastat 
nooremad Janek Hiis\index[ppl]{Hiis, Janek} ja Janek 
Palõnski\index[ppl]{Palõnski, Janek}. Keegi Kristjan, mul ei tule  tema teine 
nimi meelde, kes oli ka tollel hetkel arvutite peal päris kõva tegija, 
hilisemas elus ta vist ei ole arvutite juurde jäänud. Raivo 
Kotov\index[ppl]{Kotov, Raivo} oli seal kõva tegija, neil Andrus 
Kõresaarega\index[ppl]{Kõresaar, Andrus}  on praegu arhitektuuri ja 
disainibüroo. Andrus Kõresaar oli ka mul klassivend,  Raivo Kotov oli ka tollel 
ajal nagu ärksam just arvutite peale aga ilmselgelt tema võttis suuna rohkem 
disaini peale. 

\question{Keskkooli ajal tööle ei võetud sind?}

Tollel hetkel, ei oska isegi öelda, kas kahjuks või õnneks, aga ei tulnud ette. 

\question{Kas Viljandis tol hetkel selline koht oleks olnud, kes 
põhimõtteliselt oleks võinud programmeerija tööle võtta?}

Ma ei tea, et seal kusagil tollel hetkel otseselt programmeerija-väljavaateid 
oleks olnud. Seal küll hakkasid mingisugused arvutispetsialistid juba ringi 
toimetama, sest ettevõtted järjest võtsid arvuteid kasutusele. Tolleaegse nimega 
Eesti Telefon ja postiasutus ja mingisugused asjad toimusid, pandi 
 IT-võrkusid ja asju juba püsti. Valdavalt vist oligi Heiki 
Pettai, kes  koordineeris  neid asju. Ütleme niimoodi, et ega ei oleks ilmselt 
väga palju võimalusi olnud,  võib-olla kusagil mingeid andmeid sisestada oleks 
heal juhul saanud. Mis oleks võib-olla keskkooli õpilase jaoks OK 
olnud. 

\question{Lõpetasid keskkooli ära ja tulid joonelt Tartu Ülikooli matemaatikat 
õppima?\index{Tartu Ülikool!Matemaatikateaduskond}}

Jah, kuna ma kooli ajal ma mingi maani olin õppe-edukuse ja hinnete seisu 
poolest suhteliselt lohh, nibin-nabin oli see Tartu ülikooli sisse saamine. Aga 
õnneks  too oli üks madalaima konkurentsiga aastaid. Ma sain sisse niimoodi, et 
ma selles esialgses pingereas olin joone peal täpselt viimane, kes sisse sai. 
Siis olid veel sisseastumiseksamid, mingi kombinatsioon 
sisseastumiseksamihinnetest ja lõpuhinnetest või kuidagi niimoodi. Sain sisse 
ära, mul oli sihuke \enquote{jess!} emotsioon ja siis sealt edasi hakkas juba 
kogu  ülikooli aegne elu. 

\question{Seal vahepeal oli kummaline periood, kus Nõukogude sõjaväe teenistus 
läks Eesti katiseväeteenistuseks üle, sa sõjaväes ei käinud?}

Jah, see aken oli hästi lühike ja ma sattusin täpselt sellesse aknasse 
niimoodi, et ma sain ülikooli sisse 1993. aastal, nõuka-aegne armee kord oli juba ära 
lagunenud, sinna enam ei võetud juba mõned aastad, aga Eestis üldine 
sõjaväekohustus kehtestatigi  1993. aasta sügisel. Ma olin suvel ülikooli sisse 
saanud, sügisel juba kehtestati ja kõik need, kes enne seda hetke olid ülikooli 
sisse saanud, olid justkui nagu vabad eeldusel, et nad ülikooli ära lõpetavad. 

\question{Mina pääsesin puhtalt tervisega aga ma teadsin, et meie kursuselt 
keegi ei käinud ju sõjaväes, naljakas aken oli.}

Tollel hetkel, kusjuures, mul oli isegi endal selline suhtumine, et mul 
oleks täitsa okei olnud minna. Ma sellel ajal natukene ka sportlikuma poole 
pealt tegelesin niisuguste maskuliinsemate asjadega, harrastasin karated ja 
olin sihuke  vend, et \enquote{mis see sõjavägi siis ära ei ole, kui vaja, siis 
lähme ja teeme}. Aga ära ta minu jaoks jäi ja hiljem hiljem isegi olin selle üle õnnelik. 
Sest tollel  hetkel see sõjaväesüsteem oli ikkagi 
väga lapsekingades, ütleme niimoodi, et ta ei oleks tõenäoliselt  midagi 
väga tagantjärgi meenutamisväärset olnud. Mu kooliaegsetest 
klassivendadest kaotas puhtalt  ajateenistuse tõttu oma elu kaks inimest. See 
nagu natukene näitab seda,  milline see tase tollel ajal oli, igasuguseid 
sihukesi õnnetusi ja  korralagedust oli tollel hetkel veel päris palju, see asi 
kujunes alles välja. 

\question{Kui ma su juttu kuulan, koorub välja suhteliselt haruldane 
kombinatsioon: teeks sporti ja olümpiaadidel pigem ei käiks aga samas 
programmeeriks kohe isuga?}

Neid asju, millega ma tegelesin, oli tegelikult paralleelselt mitu. Asju, 
millega nagu hakkama ei oleks saanud, oli mitmeid ja see tegi 
võib-olla isegi natukene nagu otsustamist raskeks, et mille juurde jääda. 
Laulmise ja muusika mõttes mul kuulmist oli, aga seal sai määravaks see, et 
kuna mul muusikakooli haridust all ei olnud, siis  sellega ma olin juba nagu 
rongi pealt maas, sinna kuhugi ma ei  trüginud. Ma isegi kaalusin mõtet, et 
minna kultuurikolledžisse midagi tegema, aga ma olin ikkagi suhteliselt lahja 
vend. Spordiga sai tegeldud, käisin ka kunstiringis muuhulgas koos  nendesamade 
Kotovite\index[ppl]{Kotov, Raivo} ja Kõresaartega\index[ppl]{Kõresaar, Andrus}. 
 Jakobsoni gümnaasis oli Grünbach\sidenote{Asko peab ilmselt silmas õpetaja 
Rein Grünbachi\index[ppl]{Grünbach, Rein}}, kes vedas seda poolt. Aga, jah, 
lõpuks ikkagi jäin arvutite juurde. 

\question{Aga miks?}

Tundus, et sellega läheb kõige paremini. Motivatsiooni mõttes need teised asjad 
ilmselt ei kinnistunud nii tugevalt. Ütleme niimoodi, et arvutiga seotud 
motiivid olid kõige tugevamini kinnistunud. 

\question{Ja nii me astusime sinuga 1993. aastal Matemaatikateaduskonda. 
Esimesed kaks aastat tambiti meile haljast matemaatikat, kuidas see tundus?}

See ülikooli saaga oli minu jaoks veel omaette nagu nagu stoori. Ta oli minu 
jaoks nagu sihuke tüütu, et 
kui meie  Tartu Ülikooli sisse astusime, oli selline süsteem, et kõik pidid 
esimesed kaks aastat sama programmi õppima. Sa ei saanud alguses veel 
otsustada, et kas sa lähed informaatika või matemaatika või  statistika suuna 
peale. See valikuvõimalus anti alles teise aasta poole peal ja seda ka õunte 
pealt vaadati, et kui hea sa oled ühe või teise asja jaoks. Tartu Ülikooli 
matemaatikateaduskonda sai astutud selle tõe pähe, et informaatika suund seal 
olemas on, aga kas sa sinna ka saad, seda kohe alguses ei teadnud. Ja selles 
mõttes oli ülikool minu jaoks paras selline \emph{challenge}, sest ma olin ka 
ülikoolis jätkuvalt  lohh edasi, vähemasti esimestel aastatel. Ja ega mul need 
matemaatikaained, mis ei olnud minu jaoks  nii-öelda see motivatsiooni 
põhipõhjus, ei läinud kõige paremini. Alguses oli päris palju keerulisi asju, 
eks ole, matemaatiline analüüs üks ja kaks\ldots 

\question{Matemaatiline analüüs I võttis ju lausa kolmandiku kursust!}

Jah, see niitis  rahvast korralikult, aga see ei olnud veel kõige hullem. Kõige 
hullem, ma arvan, oli võib-olla isegi algebra, Mati Kilbi\index[ppl]{Kilp, 
Mati} väga karmi käe all, kõik asjad tuli korrektselt selgeks saada. Minu jaoks 
algebra oma selles abstraktsuses oli nagu kuidagi kõige raskemini omandatav. 
Matemaatiline loogika seevastu, mida õpetas tollel ajal levinud folkloori järgi 
üks karmimaid õppejõude Rein Prank\index[ppl]{Prank, Rein}, tuli mul kuidagi 
lihtsasti, kuna ta oma mõttemudeli poolest haakus programmeerimisega palju 
paremini. 

\question{See oli naljakas aine jah, ta otseselt keeruline ei olnud aga teda 
peeti raskeks.}

Ilmselt ta mingite inimeste jaoks oli keeruline, aga meie, programmeerijate, 
jaoks tuli ta kuidagi loomulikult. 

\question{Vanemuise tänava õppehoones olid laiad aknalauad, kus peal istuti. 
Sest kuskil mujal ei olnud istuda. Ja kuna seal istuti, värviti aknalaudu 
regulaarselt üle. Aga kui palju neid ka üle ei värvitud, oli alati kuskile 
sisse kratsitud \enquote{Prank on loll}.}

Meil oli rebaste vandes, kui sa mäletad, palju lauseid, eks ole, mida ma kõike 
tõotan, ja üks neist oli see, et \enquote{tõotan Prangile kõik eksamid ära teha 
hiljemalt seitsmendal katsel}. 

\question{Paljudega ilmselt ka nii läks. Aga sul oli siis 
programmeerimisunistus nii palju silma ees, et sa ikkagi matemaatikast läbi 
ronisid?}

Ma kuidagi lohistasin ennast nendest läbi, aga mul nagu kriisimoment oli olemas 
küll, ma olin tegelikult suhteliselt välja kukkumise äärel  matemaatika asjade 
tõttu. Kuna puhas matemaatika mind väga ei motiveerinud, siis ma suurema osa 
oma ülikooliajast veetsin 
arvutuskeskuses\index{Tartu Ülikool!Arvutuskeskus}, me kõik teame, selles 
nõndanimetatud Väksu klassis\index{Tartu Ülikool!Liivi Õppehoone!Vase klass}\label{sisu:vase_klass}. Tolleaegse nimega 
vask.ut.ee\index{Masinad!vask.ut.ee} serveri VT100 terminalid, ühendatud ühe 
VAX VMS-i  süsteemi taha. Sealt, ma arvan, ma sain oma esimesed suuremad  nagu 
programmeerimise tuleristsed.  Seal ma avastasin enda jaoks tollel hetkel 
suhteliselt nii-öelda \emph{on the dark side} maailma, mis oli siis 
 mudamängud\index{Mängud!Muda}. Need olid nagu üheksakümnendate 
esimese poole internetipõhised arvutimängud, virtuaalsed maailmad, kus ei olnud 
midagi graafilist, kõik oli tekstipõhine. Kusagil jooksis mingisugune server, 
kuhu telnetiga võeti ühendust ja seal maailmas käis mingisugune möll ja 
tagaajamine ja mingite \emph{quest}-ide lahendamine. Ma jõudsin nagu 
suhteliselt kiiresti selleni, et sattusin ise ühe sellise mänguserveri 
programmeerimise meeskonda, sealt ma põhimõtteliselt sain oma kõige esimese 
tugeva C\index{Keeled!C} programmeerimise kogemuse. Vaata, mäng on oma 
olemuselt seesmiselt tegelikult päris keeruline elukas. Mängumootor peab 
seesmiselt maailma mudeldama, seal on kõik asjad, tuhandete viisi igasuguseid 
ruume ja liste ja asju. Algoritmikat on üksjagu, kuidas seda kõike 
struktureerida ja ta oli minu jaoks tollel hetkel üks esimesi tõsisemaid  
programmeerimise kogemusi. 

\question{Selleks hetkeks sa pidid C-d ka juba oskama?}

Loengutes meile programmeerimist õpetati Pascali\index{Keeled!Pascal} baasil, 
nagu sa mäletad. Selle ma korjasin nagu suhteliselt kergesti üles, oli sihuke 
hea tüpiseeritud keel ja teda enam-vähem okeilt õpetati ka, et oli lihtne ja 
arusaadav. Aga kusagil kripeldas, et \enquote{mingid vennad panevad C-d}, et 
ikka häirib, et ise ei saa. Ostsin mingisuguse eestikeelse C õpiku, mis oli 
minu jaoks täiesti arusaamatu, ta oli nii halvasti koostatud. Ma ei mäleta 
täpselt, mis ta pealkiri oli, selline kollase-punase kujundusega  kaantega 
raamat oli. Seal selle asemel, et näidata esimeses peatükis, kuidas Hello 
Worldi teha, hakati kohe mingisugust baidi \emph{alignment}-i 
 arutama.  Kamoon, et mis otsast te pihta hakkate! Aga siis mingi hetk 
jõudis minuni info, et Kernighani ja Richie The C Programming 
Language\index{The C Programming Language}\sidenote{Ritchie, Dennis M., Brian 
W. Kernighan, and Michael E. Lesk. The C programming language. Englewood 
Cliffs: Prentice Hall, 1988. Seesama raamat, millest on ka varem juttu olnud 
leheküljel \pageref{sisu:richie}.} on ka hea raamat, otse C keele autoritelt. 
Kuskilt ma selle tellisin või sain või ostsin  omale ära. 

\question{Neid liikus ilmselt Venemaal piraadituna ka\sidenote{Vt. lk. 
\pageref{sisu:richie_vene}.}!}

Ei, mina mingisuguse täiesti legitiimse raamatut, ei olnud mingi  
piraaditud väljatrükk või midagi. Mul on see vist siiamaani riiulis alles, 
täiesti ilus korrektne raamat on, lihtsalt praeguseks  muidugi väsinud, kapsaks 
muutunud, natuke teibitud. Aga see oli nagu metoodilises mõttes hea raamat,  
hakkas lihtsatest asjadest pihta ja läks lõpuks nagu \emph{hard core}-ks 
välja ära. Ma neelasin selle raamatu läbi, tegin kõik harjutused läbi ja 
sealt sain omale C keele reaalselt selgeks. Kui üritada näppu peale panna 
raamatule, mis on mu tänast karjääri kõige rohkem mõjutanud, siis siis see on 
see. 

Selle raamatu ma olin enne ühesõnaga läbi protsessinud, kui ma mängu 
progemiseni jõudsin.

\question{Tollesse Mutta vajus meil kursa pealt ka ikka mitu inimest, kes enam 
sealt Vaxu klassist ei väljunudki.}

Mõned jah. Ma ütleks isegi niimoodi, et rohkem vajus sinna aasta vanemaid. Muda\index{Mängud!Muda} 
oli  tulnud aasta enne seda, kui meie sisse astusime,  see oligi umbes 
enam-vähem täpselt see aeg, kui internet tuli. Tuli ta  aga koos oma paari 
pahega ja see oli üks peamisi. Mingid inimesed sattusid mängu haardesse.

\question{Sina ei sattunud?}

Mul oli lühike periood, mingi paar kuud, mul see suhteliselt ruttu konverteerus 
programmeerimise entusiasmiks. 

\question{Kas toda serverit kirjutanud tiim oli Eestis või välismaal?}

Mudast\index{Mängud!Muda} oli arendatud hästi palju erinevaid versioone, kõik 
olid tolle aja mingite vabavaraliste litsentsidega, avatud koodiga, neid sai 
kusagilt FTP saitidelt tõmmata. Nendest arenes siin ja seal erinevate tiimide 
käes igasuguse erinevaid \emph{fork}-e. Kui seda \emph{fork}-ide hierarhiat 
joonistada, siis üks kõige kuulsamaid ja levinumaid juur-forke oli 
DikuMUD\index{Mängud!Muda!DikuMUD}, sellest omakorda oli tehtud haru Merc, 
millest omakorda oli tehtud fork nimega ROM. Raul Tölp\index[ppl]{Tölp, Raul} 
oli see, kes võttis ühe ROM-i põhise  versiooni ja hakkas sellest eesti 
oma  Estonia-nimelist asja arendama. Ja vot tolle versiooniga siis mina mingi 
hetk liitusin. Sai seda maailma seal edasi arendatud ja kohendatud ja vigu 
parandatud ja igast asju tehtud. 

\question{Too oli ju huvitav kogemus, sest erinevalt ülikoolis õpetatavast 
tavalisest programmeerimise praktikast, oli tegu meeskonnatööga!}

Jah, kuidagi spontaanne meeskonna-element tuli sisse. Laiem meeskond oli see, 
kes seda mängumaailma seal kirjeldas, oli hästi palju neid maailma faile. Kõik 
need ruumid ja kollid ja mobprogid\sidenote{Mobprog, vahel ka mprog, on 
lühikesed preogrammijupid, mis käivituvad mängija teatud tegevuste peale ning 
võimaldavad kollidel neile rageerida. Samuti on olemas oprogid, mis võimaldavad 
sama objektide jaoks.} ja propsid ja mis seal sees kõik elavad. Aga kes juba 
ägedamad koodivennad olid, need nagu läksid  järjest rohkem sinna koodi 
sisse. Mingi hetk me üritasime mingi bandega  hakata täiesti nullist 
hoopis uutel alustel MUDi tegema. 

ROM-i põhine muda oli nii-öelda C-keele põhine  ja võrguga suhtlus 
käis \verb|select| \emph{loopiga}, deskriptorid olid kusagil \verb|select| 
listis ja seal on mingid omad piirangud, mingi maksimum tuhatkond 
\emph{connection}-it saab korraga püsti olla ja kõike siis  protsessiti ühes 
\emph{single-threaded} tsüklis. Aga me hakkasime Toomas 
Soomega\index[ppl]{Soome, Toomas}, kes oli tolleaegne Arvutuskeskuse süsadmin 
ja Peeter Lauaga\index[ppl]{Laud, Peeter}, kes oli  aasta peale mind ülikooliga 
liitunud, tegema täiesti uut arhitektuuri, mis oli C++\index{Keeled!C++} põhine 
ja \emph{multi-threaded}, et paralleelsus paremaks saada. Me üritasime sellest 
\verb|select| deskriptorite listist ja tema piirajatest lahti saada ja 
tegime veel igasugu ägedat \emph{hardcore} värki, kus me asendasime tekstiga 
maailmafailide sisse parsimise mingisuguse kiirema ja efektiivsem asjaga. Kuna 
see maailm koosnes tuhandetest ruumidest, siis alati serveri \emph{boot} võttis 
mitu minutit aega ja kui server \emph{chrash}-is närisid mängijad küüsi, et 
kaua läheb, et kas saab uuesti sisse tagasi. Me üritasime selle asendada 
mingisuguse asjaga, kus  maailmafailid olid eelkompileeritud mingisugusteks 
mälu \emph{dump}-ideks ja me üritasime \verb|mmap|-iga need faili  sisse 
lugeda, et saaks kohe hoobilt, praktiliselt murdosa sekundiga, kõik püsti. 
Moproge hakkasime kirjutama dünaamiliselt lingitud libradeks kompileeritud 
failidena, mida sai jooksev server \verb|dlsym|-ga käigu pealt sisse linkida ja 
käivitada. Ühesõnaga, me ajasime seda kontseptsiooni päris keeruliseks. Sai 
sihukest korralikku tõsist Unixi keskkonnas \emph{hardcore} C ja C++ häkkimist 
omandatud. 

\question{See kõik jäi sulle külge lihtsalt ka kuskilt õhust või?} 

Korjasime  vastastikku järjest üles, toetasime teineteist. Toomas 
Soome\index[ppl]{Soome, Toomas} oli see, kes tegi alguses otsa lahti, tema 
viskas hästi palju ideid  lauda just arhitektuuri osas. Aga  mina 
ja Peeter Laud\index[ppl]{Laud, Peeter} korjasime selle suhteliselt kiiresti 
üles ja siis juba hakkasime vastastikku üksteist järjest täiendama. 

\question{Teil pidi siis palju vaba aega olema, tööl ei käinud?}

No vot seal oligi see koht, et jõuame tagasi ringiga sinna, et mul oli  
ülikoolis püsimisega vahepeal täiesti raskusi. Ma suurema osa oma ajast veetsin 
arvutuskeskuses, vahest  varajaste hommikutundideni välja, ja tihtipeale  
loengutesse ei jõudnud. Mul vahepeal mahajäämus vajalikes ainepunktides oli nii 
tugev, et ma vist isegi kusagil seal kolmanda aasta poole  peal, kui ma 
seal ei oleks ennast kätte võtnud, siis võib-olla, et olekski välja kukkunud. 
Kuskil seal kolmanda aasta poole peal mul käis mingisugune klõps, mida ma olen 
niimoodi hiljem nimetanud, et see oli see moment, kus lapsest sai täiskasvanu. 
See käis mingisuguse hästi lühikese ajahetkega. 

Ülikooliga seotud stress läks hästi kõrgeks ja mingi hetk oli nii, et ma jätsin 
kõik selle selle mängude ja  programmeerimisega tegelemise kõrvale ja hakkasin 
lihtsalt järjest laduma ülikooliõpinguid. Jutiga, ma mäletan, panin mingi ühe 
semestriga vist mingi 40 ainepunkti\sidenote{See oli suurusjärgus topelt 
normaalsest semestri õpikoormusest ja nõudis ilmselt tõesti suurt tööd, sest 
kolmandal aastal väga palju lihtsaid aineid enam järel ei olnud.} mingi hetk, 
et saada ree peale. Kuskil käis mingisugune klõps, et kui ma varem ei suutnud 
ennast  kokku võtta, siis peale seda ma olen praktiliselt kogu aeg seda teha 
suutnud. Baka sai kusagil 1998 ära lõpetatud. Ega mul hinded enam nagu head ei 
olnud,  sõna otseses mõttes lihtsalt toores tootmine, et kõik ainepunktid 
kätte saada. 

Aga kui ma juba magistrisse läksin (ma pidin minema tasulise kohale, sest 
hinnete seisu tõttu ma riigieelarvele selle kohale ei pääsenud) siis seal ma 
juba  hindelised ained tegin kõik praktilised maksimumi peale. 

\question{Miks sa magistrisse läksid? See ei olnud tol ajal üldse mitte 
vaikimisi valik?\sidenote{Toonane bakalaureusekraad kestis nominaalselt neli 
aastat ja juba sisaldas umbkaudu poolt praegusest magistrikraadist.}}

Ma ei mäleta. Kuidagi oli see tahtmine. Magistris sa said juba keskenduda 
teemadele, mis sind ennast huvitavad. Kui baka tasemel oli hästi palju sellist 
sunniviisilist programmi, siis magistris sa juba valid ise, mis sa teed. Tööle 
ma läksin kuskil 1995, esimese palga häkkimise eest ma saingi 1995 Tartu 
Ülikooli Arvutuskeskusest\index{Tartu Ülikool!Arvutuskeskus}. Viljo 
Soo\index[ppl]{Soo, Viljo} andis operatsioonisüsteemide loengut\sidenote{Viljo 
loetud operatsioonisüsteemide ainel oli väga hea maine, seda peeti keeruliseks 
ning huvitavaks ja seal silma paista oli väga kõva sõna.} ülikoolis ja ükskord 
mingi loengu vahepausi ajal tuli minu juurde ja küsis, et \enquote{kuule, mis 
sa arvad, et tahaksid äkki natukene tööd ka teha või midagi?}. Ta oli ilmselt 
mind tähele pannud, et ma ilmselt olin keegi nendest, kes seal võib-olla 
mingeid asju  kuidagi ladnamalt üles korjas. Arvutuskeskuses oli süsadminimise 
kõrval ja  käigus oli aeg-ajalt erinevaid asju vaja arendada ja nii mind võeti 
sinna programmeerijaks. 

\question{Tol hetkel oli võimalik juba ka arvutitega äri teha, see ei tõmmanud 
sind?}

Jah, neid tegijaid ümberringi toimetas. Aga, jällegi ei oska nüüd öelda, kas 
õnneks või kahjuks, ma suutsin niisugustest ahvatlustest  kõrvale hoida. Ma 
mäletan, et näiteks Alo Toom\index[ppl]{Toom, Alo} ja Ülo Säre\index[ppl]{Säre, 
Ülo} tegid tegid alguses A ja Ü\index{A ja Ü}, millest arenes välja PC 
Expert\index{PC Expert}.  Ja mingi aja eksisteeriks niisugune tarkvarafirma 
nagu Codewiser\index{Codewiser}, mis on ka samast  ettevõtete perest. Alo, minu 
 lapsepõlvesõber, üritas mind mingi vahe korraks A ja Ü-sse tööle ahvatleda, 
aga tollel hetkel  oli see rohkem sihukene tehnika \emph{support}-i ja konsuldi 
firma, aitasid klientidel katki läinud kõvakettaid taastada ja mingid sihukesi 
asju. Too mind nagu liiga ei köitnud, mul  oli emotsioon  ikkagi rohkem  
programmikoodi suunas. 

Siis korraks üritati mind meelitada meditsiini vallas tegutsevate 
AtFuti\index{AtFut}, kus töötas papa Pruulmann, Jaan 
Pruulmann\index[ppl]{Pruulmann, Jaan}. Mingis seltskonnas kutsutakse teda 
\enquote{Papa  Pruulmaniks}, eks ole, sest Pruulmannide dünastia on suur ja 
lai, neil on seitsmevennaline perekond, kellest suurem osa kõik on täna 
teada-tuntud IT-tegijad. Aga Jaan Pruulman on  see nii-öelda seenior, kes 
algatas AtFuti. Üks tuttav nimega Elvar Vask\index[ppl]{Vask, Elvar}, alias 
Cuprum\index[ppl]{Cuprum|see{Vask, Elvar}} üritas mind korraks sinna 
töövestlusele sebida ja ma vist tollel hetkel küsisin  äkki liiga kõrget 
palka. Mäletan, et tolle punkti juures lõppes diskussioon ära. Muidu nagu 
tundus, et klikk oli. 

\question{Akadeemiasse teadust tegema ei kiskunud?}

Seda, et hullult teadlaseks oleks tahtnud saada, ei olnud. Pigem ikkagi sihuke  
praktiline programmeerimine, häkkimine, oli nagu see emotsioon. 

\question{Sa pidid kiusatusele ikka kõvasti vastu seisma, sest mina libastusin 
kohe 1993. aastal ja olin selleks ajaks tont teab kus omadega ja alles õppisin bakas. 
Aga sina suutsid keskenduda?}

Selle mängu ümber tekkis aeg-ajalt mingisuguseid unistusi. Tiimiga aeg-ajalt 
sai arutatud, et võib-olla õnnestub sellest mängust midagi kommertsiaalsemat 
välja arendada ja mingisuguseid mõtteid, \emph{brainstorm}-e, selles vallas käis, 
aga ilmselt me ei saanud ilmselt kuidagi seda tervikpilti piisavalt hästi 
kokku, et kogu tiimis püsivus sees oleks olnud, et see tegevus kuhugi välja 
viiks. 

Ütleme niimoodi, ma oma karjääri, pigem ettevõtjana ei alustanud, olin 
ikkagi palgatööline. Kõigepealt oli Tartu Ülikooli Arvutuskeskus, siis sealt ma 
läksin vahetult enne baka lõppu 1998 talvel   
ProMedi\index{ProMed} mis just samal kevadel liitus Magnum 
Medicaliga\index{Magnum Medical}, millest hiljem sai Magnum ProMed. Ehk siis ma 
olen natukene kaudselt paar kuud olnud seotud ka täna palju kõneainet pakkuva 
Margus Linnamäe\index[ppl]{Linnamäe, Margus} tegutsemise algusaegadega. Aga 
selle ühinemise käigus ProMedi poolne seltskond kõik kohe koondati. 
Ma sain jube mugava paketi, et ma paar kuud sain sisulist tööd teha ja siis tuli 
koondamisteade, mis tolle aja reeglite järgi tähendas seda, et kui koondatakse 
rohkem kui 20 inimest, siis on nelja kuu rahad. Polnud paha, ma sain ilusti 
mõnusalt oma baka lõpu kõik ära teha niimoodi, et ei olnud nagu väga palju vaja 
muretseda laua peale leiva saamise pärast.  Suvel ma läksin 
Medisofti\index{Medisoft} tööle, mis tegeleb tänaseni  raviasutuste 
infosüsteemide arendusega, tollel hetkel oli valdavaks arenduskeskkonnaks 
Borlandi Delphi\index{Keeled!Borland Delphi}. See oli ülikooli algusaegadest 
tuttava Pascali keele põhjal, aga täiesti uus graafiline keskkond, hästi mõnus 
\emph{desktop} rakenduste tegemiseks, jube äge igatepidi. Seal oli siis 
valdavalt teemaks \emph{desktop} rakenduste ja andmebaaside kokku panemine. 

Mul oli veel niimoodi, et  laual oli veel paar alternatiivi,  kuhu tööle 
minna. Ma tagantjärgi veel isegi mõtlen, et see on iseenesest huvitav otsus, 
mis ma tegin, et võib-olla oleks mu elu täna teistsugune, kui ma teistpidi 
otsuse oleksin teinud. Üks variantidest oli Medisoft ja teine oli 
 see punt, kus meie sõbrad Rene Prillop\index[ppl]{Prillop, Rene} 
ja Mati Muts\index[ppl]{Muts, Mati} toimetasid, seesama 
tuumik, kes hiljem eesti-poolset PlayTechi\index{PlayTech} otsa  asutasid ja 
tegid. Nad pakkusid nagu kõrgemat palka, aga ma olin tollel hetkel nagu 
alalhoidlikkusele kalduv ja mulle tundus parem Medisofti pakkumine. Et nagu natukene 
sihukesed vanemad mehed, kindlama meelega ja stabiilsemad asjad. Läksin 
selle peale ja aasta olin Medisoftis. Mis alguses oli hästi huvitav, sest 
iga päev õppisin kogu aeg midagi uut, mul on kogu aeg millegi uue saamise 
motivatsioon.  Aga seal ma olin aasta, 1998. aasta suvest 1999. aasta suveni, 
kui mind kutsuti Küberisse\index{Küber} tööle. 

Küberist hakkasid juba teistsugused lood, naljakamad ja põnevamad 
\emph{story}'d, mis võib-olla tänapäevaga rohkem haagivad. Küberis ma sattusin 
täpselt sellesse kuuma momenti, kui seal oli parasjagu see põhituumik --- Tarvi 
Martens\index[ppl]{Martens, Tarvi}, Aarne Ansper\index[ppl]{Ansper, Arne}, 
Viljar Tulit\index[ppl]{Tulit, Viljar} ja Monika Oit\index[ppl]{Oit, Monika} 
--- veel koos ja parasjagu visioneeriti kõiki neid igasuguseid DigiDoc-e ja 
asju. See oli hästi huvitav aeg Küberis, väga põnev. Ahto 
Buldas\index[ppl]{Buldas, Ahto}, Helger Lipmaa\index[ppl]{Lipmaa, Helger}, Jan 
Villemson\index[ppl]{Villemson, Jan}, kogu see krüptoteadlaste kamp oli seal, 
eks ole. Meelis Roos\index[ppl]{Roos, Meelis},  Peeter Laud\index[ppl]{Laud, 
Peeter} oli tolleks hetkeks seal, Ville Hallik\index[ppl]{Hallik, Ville} 
samuti. Järjest kõik on tuntud  korüfeed, niisugune ajutrust oli seal koos! 
Esimese hooga ma sattusingi kohe midagi C++-s\index{Keeled!C++} programmeerima, 
kusagil Visual Studios DigiDoci kliendi prototüüpi või mingit visiooni klienti, 
eks ole. 

Tollel hetkel siis teadlased leiutasid neid ajatemplite linkimise skeeme 
räsiahelate otsas, arendati digitaalse notari kontseptsiooni. See kõik oli 
selline asi, mille kohta me tollel hetkel veel ei teadnud, et 10 aastat hiljem 
hakatakse seda \emph{blockchain}-iks nimetama. Iga natukese aja tagant keegi 
tuleb jälle mingi uue teooriaga välja, et kes on Satoshi Nakamoto. Satoshit 
ikka otsitakse, et kes see oli, kes \emph{blockchain}-i leiutas. Ja paar aastat 
tagasi\sidenote{Jutt Askoga toimus 2020. aasta jaanuaris.} tuli, keegi jurist 
kusagilt USA-st välja teooriaga, et see oli Helger Lippmaa\index[ppl]{Lipmaa, 
Helger}\sidenote{Keegi Justin Sobaje 2018. aastal sellise teooriaga, 
mida Helger visalt eitas, tõesti välja tuli.}. Et meil on nagu oma nii-öelda 
Satoshi Nakamoto ka olemas. Ahto Buldaselt\index[ppl]{Buldas, Ahto} võeti ka 
selle jutu peale intervjuu ja tema ütles, et põhimõtteliselt me kõik oleks 
võinud Satoshid olla, muigas selle jutuga peale. 

\question{Lähme korra üheksakümnendate juurde tagasi. Kuidas sul arvuti ja muu 
maailma tasakaalustamise ümber käis? Matemaatikateaduskonna ja Arvutuskeskuse 
ümber käis ju äge seltsielu ka?}

Arvutuskeskus oligi jah siis see koht, kus ööde viisi mingi seltskond koos 
istus ja oma asju toimetas. Üks mingisugune püsiv ja stabiilne kaader olid need 
samased Muda\index{Mängud!Muda} mängijad. Mingisugune seltskond \emph{chat}-is  lihtsalt mingites 
tolleaegsetes varajastes telnetipõhistes jutukates, 
tänapäeval on meil nende asemel Facebookid ja muud asjad. Mingi seltskond tegi  asjalikumaid asju, kas siis oma õpingutega seotud asju või mida iganes. 

\question{Ma mõtlen just seda, et Tehnikaülikooli arvutuskeskuse kohta on 
öeldud, et see oli nagu pooleldi klubiline asutus. Et inimesed seal ülikoolis 
ammu enam ei õppinud aga arvutuskeskuses käisid kogu aeg?}

Jah, Ma arvan, et Tartu Ülikooli Arvutuskeskuses üheksakümnendatel toimunu 
kohta võib enam-vähem sama öelda, sest seal ei olnud ainult üliõpilased. Sealt 
käis läbi ikka igasugust rahvast väljaspoolt ülikooli ka,  oli tekkinud 
mingisugune  kamp, kes nagu kõik tundsid enam-vähem kõiki selles 
seltskonnas, kes seal stabiilseid käisid. 

\question{Tartus vähemalt üks selline kogunemiskoht, tuum, oli ju Tähetorni 
näol veel?}

Neid kohti oli isegi rohkem, ma ütleksin. Ja need tuumikud  puutusid omavahel 
kokku ka, mingisugused inimesed viibisid aeg-ajalt siin ja aeg-ajalt seal. Aga, 
ütleme, jah, oli mitu gravitatsioonikeskust. Ma ütleks, et Tähetorn oli 
võib-olla  isegi üks  suletumaid ja väiksemaid, ma sellest Tähetorni värgist 
isegi ei tea väga palju. Ma tean küll paari inimest, kes seal olid, aga ma ise 
nagu ei sattunud sinna väga. Aga lisaks  Tartu Ülikooli 
Arvutuskeskusele\index{Tartu Ülikool!Arvutuskeskus} oli ka 
füüsikahoone\index{Tartu Ülikool!Füüsikahoone}, kus Ville 
Hallik\index[ppl]{Hallik, Ville} ja Otto Teller\index[ppl]{Teller, Otto} 
vägesid juhatasid ja  asju kontrolli all hoidsid. Ja oli  mingi punt, kes siis 
seal käisid. Füüsikud ise, ja ma tean, et mingid sotsiaalteaduskonna 
tudengid olid seal stabiilselt istumas. Siis  mingisugune 
gravitatsioonikeskus oli Ülikooli peahoone kõrval kunagine Marksu 
Maja\index{Tartu Ülikool!Marksu Maja}, mille all oli mingisugune punkt, kus ma 
ise ka mingi periood üpris tihti viibisin. Keemiahoone\index{Tartu 
Ülikool!Keemiahoone} all oli ka, aga see oli minu arust  lühemat aega või 
sihuke natukene selline ebamäärasem. 

\question{Kui paljud tollest arvutuskeskuse seltskonnast matemaatikud olid? On 
meenutatud seda, et kui Anne Villems\index[ppl]{Villems, Anne} oma esimesed 
veebmasterite kursused tegi, käis seal igasugust rahvast psühholoogidest 
usuteadlasteni.}

Ütleme niimoodi, et teistest teaduskondadest ma tean üksikuid inimesi, et 
sellist statistilist pilti anda ei oska. Kuna arvutuskeskuse  nii-öelda 
palgaline kaader, kes seal opereeris, oli valdavalt ikkagi  
matemaatika-informaatikateaduskonnaga tihedalt seotud, siis paratamatult oli 
selle teaduskonna tudengkond seal ka kõige rohkem esindatud. Aga võib-olla 
praegustest jällegi tuntud nimedest  Jaanus Lillenberg\index[ppl]{Lillenberg, 
Jaanus}, kes praegu juhib ERR-is IT-vägesid, oli mingi periood seal hästi 
aktiivne külaline\sidenote{Jaanuse seiklustest Liivi tänaval loe lähemalt 
leheküljelt \pageref{sisu!jaanus_liivi_tn}.}. Tema teaduskondlik taust vist oli 
ka kusagilt mujalt, ta ei olnud IT-vallast. 

\question{Mis sa praegu teed?}
Praegu olen ettevõtja. 

\question{Just hakkasin mõtlema, et kui kaua juba. Üle kümne aasta?}

Selliseks selgelt ettevõtjaks ma sain peale peale Skype'i\index{Skype}. Enne 
seda olid nagu aeg-ajalt  mingisugused unistused ja visioonid ja  mingisugused 
projektilaadsed eksperimendid, aga reaalselt esimese OÜ asutasin oma väriseva 
käega paar kuud enne seda, kui ma Skype'st lahkusin, see oli 2007. september. 
Kui sa vaatad äriregistrisse, siis see ongi Mooncascade\index{Mooncascade} 
registreerimise kirje, mis on siis tänane ettevõte, millega ma ise kõige rohkem 
seotud olen. Mul Mooncascade visioon tekkis juba tollel hetkel, aga kui ma  
Skype'st lahkusin siis eestlastest Skype'i asutajate punt, Ambient Sound 
Investment-i seltskond, kutsus mind kohe jutule, sest  neil oli parasjagu 
käsil üks mingisugune inkubaatori-laadne eksperiment, kus neil mingi neli 
erinevat projekti jooksid. Nad kutsusid mind ühte nendest projektidest vedama 
ja kuna seal oli pundis ka Ahti Heinla\index[ppl]{Heinla, Ahti} isiklikult, kes 
täna veab Starship'i siis minu jaoks oli see juba piisavalt motiveeriv setup, 
et okei, ma panen selle Mooncascade praegu  riiulisse. Nii me paar aastat 
tegime ühte suhteliselt jõhkrat andmekaevelaadset projekti, mis äriliselt 
lõpuks ikkagi lendu ei läinud, sai ära konserveeritud ja  uuesti Mooncascade 
re-aktiveeritud.  Mooncascade alustas oma aktiivset tegutsemist 2009 lõpp 2010 
algus ja tegutseb tänaseni. 
