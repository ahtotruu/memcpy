\index[ppl]{Heinla, Ahti}
\question{Kuidas sina sattusid arvutite juurde?}
Minu ema ja isa olid 
programmeerijad. Nad said kokku pärast ülikooli lõpetamist tööle minnes, 
kuuekümnendate lõpus. See oli aeg, kus Eestis tekkisid 
esimesed arvutid, mis sel ajal olid muidugi kapisuurused.

\question{Kuuekümnendate lõpus ei saanud programmeerijaid ju palju olla?}

Jah, palju ei olnud, kuid 
mingil määral ikka. Muide, emal oli hiljuti  
suur juubel ja ta kutsus enda 
kursusekaaslased külla – üle seitsmekümneaastased
rakendusmatemaatikud, nii mehed kui ka naised, ja enam-vähem kõik  
professionaalsed programmeerijad. Nii et meil on juba 
suhteliselt vanu põlvkondi, kes on üles kasvanud programmeerijatena. Ja mina 
sündisin kahe sellise inimese järeltulijana.

\question{Kas see on pigem vedamine või vastupidi? See valdkond oleks võinud ju 
ka ära 
hirmutada.}

Mind see kindlasti ära ei hirmutanud, ma kasvasin üles perfolintide vahel. 
Kuna arvutiaeg oli piltlikult öeldes talongidega 
jagatav, käisid ema ja isa vahetevahel
õhtuti tööl, kui nad said arvutiaja kella kaheksaks õhtul. 
Nad võtsid aeg-ajalt meid õega kaasa, mina jooksin arvutikappide vahel ringi ja 
vaatasin, kuidas 
magnetlindid käisid vaikselt, ning see oli väga põnev. 
Hoopis teistsugune keskkond, isegi helid olid teistsugused – magnettrumlid 
vihisesid ja sahisesid vaikselt.

\question{Legend räägib, et selle põlvkonna rahvas korraldas Ameerikas 
lindikappide võidujookse ja muud sellist. Kas tolles arvutiruumis ka midagi 
sellist toimus?}

Mina selliseid asju ei näinud. Küllap tehti ka Eestis tol ajal 
sellist pulli, aga võib-olla tegid seda natukene nooremad inimesed, kellel 
lapsi ei olnud. Minu isa ja ema olid ontlikumad. Nad 
üritasid arvutiga teha konstruktiivseid asju, panna neid 
ühel või teisel moel käima, mitte ei olnud tänapäeva mõistes häkkerid. Nad ei 
mõelnud, mismoodi näiteks arvutiga pead pesta.

\question{Sind ju esialgu ei lastud linte perforeerima? Mis oli esimene asi, 
millele sa ise käed külge said?}

Ma olin lapsena vist küll matemaatiliselt andekas, aga esimene otsene kokkupuude
arvutitega oli kümneaastaselt. Ühel õhtul tuli ema koju ja 
ütles, et kuule, Ahti, ma õpetan sulle midagi. Istusime maha ja 
ta õpetas mind programmeerima. Ja ma sain 
kolme õhtuga aru, mismoodi see käib. Sealt 
alates hakkasin juba ise edasi mõtlema, proovima, katsetama, lugema ja 
natukene lolle küsimusi küsima. Nii et süstemaatilist 
programmeerimisõpetust olen saanud kolm päeva.

\question{Mida ta rääkis, et see kümneaastasele huvitav oli?}

Eks mind lihtsalt huvitasid sellised asjad. Aastal 1992 ei olnud ju ka lapsel 
tohutult palju ahvatlusi, nagu praegu on 
Facebook, Instagram ja kõikvõimalikud 
muud asjad. Sel ajal ei olnud ju isegi veel kodus telefoni, rääkimata arvutist 
– need kolm päeva õppides kirjutasin ma programmi  
alguses paberile. Ema jutt kümneaastasele lapsele ei pidanudki olema
kohutavalt põnev. Põnev oli lõpuks  
see, kui sain aru, kuidas asi töötab.

\question{Sul pidi olema korralik ettekujutusvõime, et said paberile 
kirjutades aru, kuidas asi päriselt töötab.}

Eks programmeerimise üks võtmeoskus ongi tegelikult 
oskus ette kujutada, mismoodi masin töötab. Lõppkokkuvõttes  
programmeerija ehitab ju masinat. Üks koodirida on 
piltlikult öeldes üks hammasratas, teine koodirida on mingi kangikene. 
Füüsilist või mehaanilist 
masinat ehitades näed füüsiliselt, kuidas see töötab: ratas keerleb, 
kang liigub, üks asi lükkab teist ja 
lint või tross tõmbab midagi kuskilt. 
Programmeerija näeb samamoodi, aga vaimusilmas. Ja see vaimusilmas nägemise 
oskus on 
programmeerijale ülivajalik. Tagantjärele võib öelda, et seda ema mulle 
kolm päeva õpetaski.

\question{Nii et see, kas \texttt{goto} käib nii- või naapidi ja kas tehete 
järjekord on 
selline või teine, on teisejärguline.}

Just. Põhiline on teada, kuidas \verb|goto| toimib ehk 
millise hammasratta \verb|goto| masinas teeb.

\question{Need kolm päeva tekitasid huvi ja sa said enam-vähem aru, kuidas 
arvuti 
töötab, aga mis edasi sai?}

Ühel õhtul läksime ema töö juurde ja  
tippisime programmi sisse. Kui ma õigesti mäletan, siis see
võttis mitu päeva aega, kuna seda oli mitu lehekülge ja mõnikord läks midagi 
valesti ka. Ema aitas vead ära parandada ja tuli välja, et programm 
töötas. See lahendas üht matemaatika keerdülesannet, mille
loogika oli selles, et sada ühikut raha tuli poes erinevatele ostudele ära 
kulutada. Näiteks lähed 
raamatupoodi ja pead 
kombineerima, et osta üks raamat, mis maksab viiskümmend seitse, ja teine, mis 
maksab kolmkümmend. Tähtis oli kombineerida niimoodi, et saada kokku summa, mis 
on võimalikult 
lähedane sajale, aga mitte üle selle. Sellist ülesannet lahendaski minu 
programm. Ei olnud kõige triviaalsem asi, et 
vajutad nuppu ja programm ütleb lihtsalt \enquote{tere}. Tänapäeval üritatakse 
kõik asjad, ka heal põhjusel, ehitada nii, et rahulolu tekiks 
kiiresti: näed kaks minutit vaeva ja 
juba midagi väikest töötab; pingutad veel viis minutit 
ja tuleb veel midagi. Mina pidin kolm päeva vaeva nägema, enne 
kui tulemus sündis. Enne seda oli kõik ainult vaimusilmas. Aga tõepoolest, kui 
Facebook taskus kogu aeg ei hüppa, siis on lihtsam ka seda kolme 
päeva leida. 

\question{Mis tolle arvuti nimi oli?}

Ma ei mäleta, see ei pruukinud isegi nõukogude masin olla, seal 
oli ka lääne aparaate.

Minu kirjutatud programmi sai isegi mitmel arvutil käitatud. Ei olnud nii, et 
\enquote{kuule, Ahti, see on nüüd sinu arvuti, millega sa võid mitu päeva 
tegeleda}. Pool programmi tippisin sisse ühel arvutil, mis nägi välja nagu suur 
must kapp, ja teise osa tippisin sisse järgmisel 
ühe hoopis läänelikuma, nõtkema ja moodsama välimusega asja taga. Sain ka 
kogemuse, et programm on hoopis midagi muud kui
füüsiline arvuti. Võin istuda ühe arvuti taga ja 
siis minna teisele korrusele teise arvuti taha, mis on terve toa suurune, 
ja seesama programm jookseb mõlemas.

\question{Millele sa programmi vahepeal kirjutasid? Kaartidele?}

Siis olid juba diskid olemas, mitte kolmetollised, vaid viie- või 
kaheksatollised. Aga mu esimene programm oli ainult 
algus, sellest tuli oskus ja huvi asja vastu. 
Edasi hakkasin ise vaatama ja sattusin kokku teiste poistega, 
kes analoogsest huvitusid. Lähemate aastate jooksul hakkasid 
tekkima ka personaalarvutid ja enam ei pidanud õhtul 
tingimata ema töö juurde minema, vaid oli ka muid kohti.

\question{Kust sellised tutvused tekkisid, internetti ju polnud?}

Internetti ei olnud, küll aga oli kaheksakümnendatel olemas näiteks
Raaliklubi\index{Arvutiklubi!Raaliklubi}, mida vedas Jaak 
Loonde\index[ppl]{Loonde, Jaak}. Mina olin ka selle klubi liige ja see koondas 
huvilisi poisikesi. Ma ei mäleta enam, mida täpselt Jaak Loonde tegi ja kas ta 
üldse midagi 
tegi peale selle, et poisid kokku tuua. Täiesti võimalik, et sellest
piisabki, et sama huviga poisid kokku tuua, ja siis nad vahetavad juba omavahel 
kogemusi. 

Mul oli üks emalt saadud reliikvia: Pascali programmeerimiskeele õpik. See oli 
inglise keeles ja 
ma ei saanud eriti midagi aru. Koolis õppisin saksa,
mitte inglise keelt\sidenote{Tol ajal jagunesid koolid kaheks: lisaks vene 
keelele õpetati kas inglise või saksa keelt}. Küll aga sain aru 
programmi näidetest, asjad olid loogilises järjekorras. Nii et kuigi ma inglise 
keelt ei 
osanud, suutsin siiski raamatust midagi õppida ja katsetamiseks ideid saada. 
Näiteks kui seal oli mingi 
\verb|goto| käsk, siis tippisin 
selle arvutisse, et
vaadata, mida see teeb, ning küsisin kelleltki, mida \verb|goto| tähendab. 
Konkreetne küsimus oli midagi muud, kui lihtsalt öelda, et \enquote{õpeta mulle 
programmeerimist}. Niimoodi läbi nii-öelda lukuaugu see 
õppimine käis. Internetti ei olnud, aga inimestel ei olnud seda ka siis, 
kui nad lennukeid ja autosid ehitasid, ja nad said hakkama.

\question{Kas arvutiklubis käisid seepärast, et programmeerimine pakkus 
huvi?}

Jah, mulle pakkus huvi just programmeerimine. Enne programmeerimise katsetamist 
sattus mulle kätte üks laste 
elektroonikaraamat ja ma natukene mõtlesin ka
elektroonika peale: näiteks kuidas transistorid ja muud säärased asjad 
töötavad. 
Nii et ka see pakkus mulle huvi, ent tagantjärele tuleb öelda, et 
elektroonikaga tegelesin ma tol ajal ülialgelisel tasemel. Ma 
nii-öelda õrnalt kõditasin elektroonikat ja 
üritasin sellest aru saada, aga programmeerimisega tegelesin  
päriselt.

\question{Kas see oli ema-isa eeskuju või midagi muud, mis 
sind pigem programmeerimise poole suunas?}

Üks asi oli kindlasti eeskuju, teine asi oli see, et 
elektroonikaga tegelemiseks on vaja komponente ja 
tööriistu, mida ju ei olnud. Tänapäeval on poes kõik 
olemas, aga ka siis tuleb asjad esmalt hankida. Ma olen hobi korras 
elektroonikaga tegelenud, näiteks üht-teist
tinutanud. Praegu, kui kõik on olemas – jootekolvid, suurendusklaasid, 
takistite 
komplektid, väiksed mikroprotsessorid, igasugused kivid, sensorid, andurid 
ja displeid – ning need tuuakse lausa koju kohale, siis kulukas on see ikkagi. 
Sellega ei ole lihtne alustada. Programmeerimise alustamiseks piisab 
paarist-kolmest päevast, arvutikohast, paberist ja pliiatsist.

Nagu mu sõber ja töökaaslane Jaan Tallinn\index[ppl]{Tallinn, Jaan} on  
öelnud, siis enamikus muudes  
valdkondades saab õppima hakates 
algse taseme kätte ja siis peab veel rohkem õppima, et saada järgmisele 
tasemele. Ja sa ei saa iseseisvalt õppida,  
on vaja õpetajat. Näiteks kui õpid klaverimängu, siis peab keegi seda pidevalt 
õpetama, kuigi YouTube ja
õpikud võivad muidugi ka abiks olla. 
Programmeerimine aga on selline asi, et kui oled algse idu
kätte saanud ja kui sind suletaks aastaks üksikule saarele koos arvutiga, siis 
tegelikult suudad ise ilma ühegi õppevahendita õppida 
ennast väga heale tasemele, kui tahad. Puhtalt ise katsetades ja mõeldes. 
Täpselt seda ma teismelisena tegingi.

\question{Kas sel ajal hakkas juba ka personaalarvuteid liikuma?}

Jah, personaalarvuteid hakkas tulema ja meie koju tekkis ka kaheksakümnendate 
lõpus üks Apple 
II\index{Arvutid!Apple II}, millega mina hakkasin toimetama. Olin siis 
viisteist-kuusteist ja oskasin 
juba korralikult programmeerida. Enamikul teismelistel on selles vanuses meri 
põlvini: peod, seltsielu ja nii edasi. Aga mina 
olen üldiselt introvertne inimene ja mitte eriti seltsiv, seltsielu ei tulnud 
mul hästi välja. Teiseks hakkab mul 
paarist klaasist veinist pea valutama, ma ei pea 
ühel korralikul peol kaua vastu ja lähen hiljemalt 
keskööks koju. Niimoodi on alati olnud, ka siis, kui olin 
kaheksateist. Aga alates kella kaheteistkümnest ju tegelik 
\emph{action} alles pihta hakkab, nagu mulle on räägitud. Nii et kui ülejäänud 
inimesed avastavad seltsielu 
ja pidusid, siis osa avastab arvutiasju ja 
seltsielu tuleb lihtsalt natuke hiljem.

\question{Mida sa programmeerisid? Jõukohase ja samas huvitava 
ülesande leidmine ei ole ju üldse lihtne.}

Eks poisikesi huvitavad põhiliselt ikkagi mängud ja kindlasti oli nii minul  
kui ka kõigil kaasvõitlejatel üks esimesi unistusi 
kirjutada oma arvutimäng. Sel ajal olid juba  
Yamaha arvutid tekkinud ja ka Apple II-s oli täitsa 
korralikke mänge. Kommertsiaalsed mängud olid ikka sellisel 
tasemel, mida poisikene hobi korras nädalaga valmis ei viska. Ja 
ega teismelisel ole tähelepanu ulatuski 
selline, et suudaks midagi pikemat ette võtta. Ehitasime väga 
lihtsaid mängukesi ja üritasime arvutile 
häkkerlikult läheneda. Katsetasime, mida arvutit on 
võimalik tegema panna, näiteks hääli ja visuaalseid kujundeid luua. 

Hiljem teismeeas sai igasuguseid asju proovitud. Järjekindlamalt 
hakkasime mänge programmeerima Jaan Tallinna\index[ppl]{Tallinn, Jaan} ja Priit 
Kasesaluga\index[ppl]{Kasesalu, Priit} keskkooli ajal. Selleks ajaks oli 
ka tähelepanu 
kasvanud ja võtsime ette ühe mängu kirjutamise projekti. Panime kõik oma seni 
õpitud väiksed kogemused 
ja oskused kokku ning tegime tiimitööd, mitte igaüks ei pusinud nurgas oma 
mängu. Jagasime 
ülesanded omavahel ära ja töötasime selle kallal kuude viisi. 

\question{Kust see mõte üldse tuli?}

Kuskilt iseenesest tuli, meil ei olnud isegi
mingit arutelu sel teemal. Lihtsalt sündis mõte, et proovime. 

\question{Mis keskkoolis sa käisid?}

Ma õppisin Gustav Adolfi gümnaasiumis\index{Koolid!Gustav Adolfi gümnaasium}
\index{Koolid!Gustav Adolfi gümnaasium|see{Tallinna 1. keskkool}}
ja Jaan Tallinn\index[ppl]{Tallinn, Jaan} oli minu pinginaaber. Priit 
Kasesalu\index[ppl]{Kasesalu, Priit} oli Jaan Tallinna pinginaaber eelmisest 
koolist, kus Jaan käis. Nii et olime mõlemad Jaani pinginaabrid olnud. Viimase 
keskkooliaasta jooksul kirjutasimegi kolmekesi ühe mängu, 
mille nimi oli Kosmonaut\index{Mängud!Kosmonaut}. Mina küll  
kirjutasin seda kui hobiprojekti, aga Jaan ütles, et see tuleks maha müüa ja 
rahaks teha. 

\question{See oli ju veel nõukogude ajal, selle eest võis kinni minna!}

Peaaegu. Oli nõukogude aja lõpp, kui igasuguseid 
metalliärikaid liikus juba ringi, käis üle piiri  
kaubandus ja tekkisid kooperatiivid. Me 
muidugi ei teadnud tuhkagi sellest, kuidas ettevõtlus 
üldse käib. Tegelikult ei teadnud seda ka vanemad inimesed, kes sel 
ajal ettevõtluses olid, aga metalliäri alal 
mõningate kogemuste ja 
sidemetega inimeste abil õnnestus meil Kosmonaudi mäng tõesti Rootsi müüa. 
See oli muidugi pöördeline sündmus: saime selle eest 
vist viis tuhat 
dollarit. See oli täiesti kosmiline number, aasta oli umbes 1990 ja rubla 
kurss oli vist üks dollar = kolmkümmend rubla. Viis tuhat dollarit 
oli umbes selline summa, mille meie vanemad olid elu jooksul teeninud. 
Loomulikult võib seda inflatsiooniga korrutada ja 
korrigeerida, aga suur oli see ikkagi. Iseasi, kui õigesti me seda raha 
kasutasime – märgatav osa kulus 
valuutapoele ja Coca-Colale, aga me Jaaniga ostsime endale näiteks kahe 
peale arvuti. 

Raha saime kätte üheksakümnendate alguses, kui Eesti 
kroon oli just tulemas. Olime mingis kooperatiivis ühe
386SX protsessoriga arvuti välja valinud, kui anti teada, et järgmisel 
päeval toimub rahareform. Ja minul oli kümneid tuhandeid arvuti jaoks mõeldud 
rublasid käes. Kooperatiivitegelane, kellele helistasin, ütles, et toogu 
ma raha järgmisel päeval, aga mul oli nii palju oidu, et viisin raha samal 
päeval ära ning saime arvuti kätte. 

\question{386 oli päris korralik aparaat. Enne seda kulus ilmselt mitu aastat 
pusimist teiste inimeste arvutite taga. Kus te seda mängu 
kirjutasite? Kas kellegi juures kodus?}

Mängu kirjutasime suurel määral Jaani\index[ppl]{Tallinn, Jaan} 
ja Priidu\index[ppl]{Kasesalu, Priit} töökohas, nad töötasid keskkooli 
kõrvalt ühes kooperatiivis programmeerijatena. Ma ise töötasin ka 
keskkooli ajal poole kohaga programmeerijana vanemate töökohas 
ehk Küberneetika Instituudis\index{Küberneetika Instituut}. Vanemate tööandja 
oli selles mõttes mõistlik. 
Kui ma ise tööandjana mõtlen, et kui seitsmeteistaastane poiss 
tahab tööle tulla ja alles õpib programmeerima, siis ma 
ei maksaks talle kuigi palju, ei võtaks teda väga tõsiselt ega  
annaks talle missioonikriitilisi asju. 

Aga Jaan ja Priit olid oma kooperatiivis vaat et juhtprogrammeerijad ja neil 
olid 
tunduvalt paremad võimalused käes. Tänapäeval oleks see
küllaltki ebamõistlik, aga siis olidki ebamõistlikud ajad. 
Nad ei saanud küll oma arvuteid töölt koju kaasa võtta, aga töökoht võimaldas 
neil
päeval olla koolis ja õhtul-öösel 
arvutis. Teismeeas suudad niimoodi vastu 
pidada, et magad kuus tundi päevas.

\question{Ühesõnaga, te käisite Gustav Adolfi Gümnaasiumis, samal ajal
töötasite programmeerijatena ja takkapihta 
kirjutasite mängu, mille kannatas pärast maha müüa?}

Jah. Peab ütlema, et vähemalt siis, kui mina programmeerijana töötasin, ei 
olnud ma
uhke oma tööpanuse üle Küberneetika Instituudis\index{Küberneetika Instituut}. 
Tõsi küll, midagi sain ikka valmis 
ja tööandja oli sellega rahul. Tegelikult oli seal
teisigi õppimise kõrvalt töötajaid ja instituut oli minuga rohkem rahul kui 
mõne teisega. Aga eks see ütleb rohkem nende kui 
minu kohta. Mina kulutasin enamiku ajast mängu tegemise ja koolis käimise 
peale.

\question{Kas sel ajal hakkasid tekkima ka esimesed BBSid?}

Jah, minu tutvusringkonnas tegeles BBSidega põhiliselt Priit 
Kasesalu\index[ppl]{Kasesalu, Priit}, kes 
ka ise ühe püsti pani, mille nimi oli vist \emph{Dark Corner}\index{BBS!Dark 
Corner} ja mille Fido \emph{node}'i number oli neliteist. 
Teda tõmbas see pool rohkem ja 
kindlasti mind ka, sest BBSiga tekkis järsku võimalus ekraani kaudu suhelda 
paljude teiste inimestega, kellega sa füüsiliselt koos ei istu. 
Teatud mõttes anti meile justkui 
Facebook kätte: 
järsku tekkis hulk sõpru, kellega olin suhelnud ainult interneti teel. Fidos 
vahetati mõtteid kõikide asjade, mitte ainult arvutite üle ja tekkis 
tolle aja kohta täiesti isevärki sotsiaalne seltskond. Tänapäeva netiajastul on 
täiesti tavaline, et näiteks mõni Facebooki grupi seltskond saab aeg-ajalt 
kokkugi, aga 1990ndate alguses oli selline elustiil midagi uut.

\question{See seltskond pidi siis olema ka teatavas mõttes homogeenne, sest 
Fido külge saamise barjäärid olid kõrged.}

Jah, eks muidugi oli ka palju selliseid, kes nii-öelda jõlkusid kaasas. Olid 
entusiastid, nagu Priit Kasesalu\index[ppl]{Kasesalu, Priit} 
ja Tarmo Mamers\index[ppl]{Mamers, Tarmo}, ja aeg-ajalt tekkis juurde nende 
sõpru, kellele Tarmo ja Priit võimaldasid 
ligipääsu. Selline sotsiaalne 
distantssuhtlus oli väga huvitav. 

Mäletan üht juhtumit sellest ajast, kui Eestis hakkas internet tekkima. 
Tehniliselt oli see olemas juba 
seitsmekümnendatel-kaheksakümnendatel, aga Eestisse see alles hakkas jõudma. 
Mul on hea sõber 
Sulo Kallas\index[ppl]{Kallas, Sulo}, kellel oli ka BBS ja kes praegu 
töötab minuga koos Starshipis\index{Starship Technologies}. Tema andis 
mulle kasutada oma kontot ühes Unixi arvutis, kus oli olemas 
programm \verb|talk|, millega said omavahel ekraani kaudu suhelda 
inimesed, kes olid samasse masinasse sisse loginud. See oli minu 
jaoks silmi avav elamus, mul ei olnud sel ajal 
kodus telefonigi. Toimetasin arvutis Sulo kontoga Sulo nime alt 
ja järsku hakkas keegi minuga \verb|talk|i kaudu 
rääkima. Ütles, et tema nimi on Epp, mille peale mina teatasin, et ma ei ole 
Sulo, vaid hoopis keegi
teine. Tema vastas, et sellest pole midagi, räägime ikka. Ma ei 
saanud täpselt aru, mis värk on – ta ju tahtis Suloga rääkida. Aga siis sain 
aru, et ta tahab tegelikult lihtsalt kellegagi 
rääkida.  
Jutuajamise käigus sain teada, et  
tegu oli Eesti tüdrukuga, kes asus füüsiliselt 
Ameerikas, ta oli ühe Ameerika ülikooli üliõpilane. 
Me rääkisime tund aega maast ja ilmast, väga kummaline kogemus oli suhelda 
reaalajas kellegagi, kes on väga kaugel. Ma siiamaani ei tea, 
kes Epp täpselt oli, ta küll ütles oma perekonnanime ka, aga ma ei ole kunagi 
hiljem temaga suhelnud. Naljakas, et 
tänapäeval on see niivõrd tavaline, kõigil Snapchatid taskus. Tol ajal aga oli 
võimalus 
inimestega üle maailma suhelda ainult internetihäkkeritel.

\question{Sa ütlesid, et BBSides räägiti igasugustel teemadel. Näiteks 
millest?}

\label{sisu!inimeseks}Kui ma õigesti mäletan, siis internetifoorumites 
käsitleti igasuguseid elulisi teemasid nagu tänapäevalgi. 
Näiteks oli filosoofiateemaline vestlusgrupp, kus enamasti 
kaheksateistaastased alles mõtestasid oma elu. See ongi selline aeg 
elus, kui inimene mõtleb, mida miski tähendab ja kas ikka 
peaks ühte või teise asja panustama. Neljakümneaastasena 
võib-olla ei viitsita sel teemal juttu vesta, elu 
tõekspidamised on juba välja kujunenud, aga tollal minul kindlasti veel ei 
olnud ja enamik 
BBSi seltskonnast olid umbes sama vanad. Räägiti ka psühholoogia teemadel, nii 
et mitte sugugi ainult tehnoloogiast. 

\question{See, mis sa ütled, kõlab väga oluliselt. See tähendab, et 
ports nutikaid samamoodi mõtlevate ja samade oskustega inimesi mõtestasid koos 
seda, 
mida tähendab olla inimene kõige laiemas mõttes.}

Absoluutselt. Fido seltskond oli virtuaalne sõpruskond, võibolla esimene 
omasugune 
Eestis üldse. Tänapäeval on igaühel virtuaalseid sõpruskondi taskus sada tükki.

\question{Kas selle kõige juurde käis ka mõni spetsiifiline raamatu-, muusika- 
või filmihuvi?}

Muusikateemalisi vestlusgruppe oli loomulikult ka, aga selles ringkonnas olid 
populaarsed pigem elektroonilise muusika bändid.  
Kraftwerk mulle ei meeldinud, Jean-Michel Jarre samuti 
mitte eriti, aga näiteks Tangerine Dream meeldis väga ja meeldib
siiamaani. Mul on kuskil viisteist nende plaati. 
Samas olen ma selline inimene, kes ei ole kunagi vaadanud "Star Warsi" ega 
lugenud raamatut \emph{Hitchhiker's Guide to the Galaxy}. Minu jaoks on 
esteetiline subkultuur ja arvutid natuke lahus seisnud.

\question{Kas sul endal ei olnud BBSi?}

Ei olnud. Ma vist ei tahtnud ka, selle üleval hoidmiseks tuli hullult 
jahmerdada. 
Mul oli väga hea meel, et sain Priidu BBSi kasutada.

\question{Kui te mängu maha müüsite, mis edasi sai?}

Kui anda üheksateistaastasele nii palju raha, nagu tema vanemad 
on kogu elu jooksul teeninud, siis on karjäärivalik kohe selge. Ei olnud 
küsimustki, mida ma 
tulevikus hakkan professionaalselt tegema – loomulikult programmeerima. Arvasin 
tollal, 
üheksakümnendate alguses, et Eesti ülikoolides eriti midagi 
kasulikku sel teemal ei õpetatud; iseasi, kui õigustatud see mõte oli. 
Keskkoolis oli küll arvutiõpetus, aga üldiselt 
teadsid paljud meie klassist rohkem kui õpetaja. Oletasin
miskipärast, et ülikooliga oli samamoodi. Tänapäeval see ei ole kindlasti tõsi, 
ent tol ajal pigem oli. 
Igatahes otsustasin, et ei lähe ülikooli õppima 
programmeerimise või arvutitega seotut, vaid hoopis füüsikat. 
Füüsika oli mu teine huviala, olin 
füüsikaolümpiaadidel käinud ja mulle see väga meeldis. 

\question{Mis sulle füüsika juures meeldis?}

Võib-olla filosoofiline aspekt, kuidas 
maailm toimib: tuumafüüsika, planeetide liikumine. Ehk milline see maailm meie 
ümber on ja kui suured või väikesed meie, inimesed, selles maailmas oleme. 

\question{Kuhu sa läksid füüsikat õppima?}

Ma läksin füüsikat õppima Tartusse\index{Tartu Ülikool}, koos pinginaaber Jaan 
Tallinnaga\index[ppl]{Tallinn, Jaan}. Tollal ma ei väärtustanud seda, 
et ülikool oleks   
edukalt lõpetatud ja paber taskus. Kui olin umbes poolteist aastat ülikoolis 
olnud, siis hakkas mulle aina rohkem kohale jõudma, et tegelikult tegelen ma 
kogu aeg programmeerimisega, töötan professionaalse programmeerijana. 
Samal ajal tegime järgmist mängu, mille kavatsesime maha müüa. Ma ei kavatsenud 
füüsikuna töötama hakata, õppisin seda hobi 
korras. Alles teisel aastal hakkasin aru saama, et õppejõud eeldavad, 
et tegeled füüsikaga tõsiselt ja panustad enamiku oma ajast õppimisse. 
Mõistsin, et see nõuab rohkem 
tööd, kui olen nõus panustama, ja tulin ülikoolist ära. Ma ei ole siiamaani 
ülikooli lõpetanud. Jaan Tallinn\index[ppl]{Tallinn, Jaan} aga käis ülikooli 
lõpuni. Kui õigesti mõletan, tegi ta oma lõputöö 
relatiivsusteooriast. Sellest, kuidas ruumi painutada selleks, et reisida 
valguse kiirusest suurema kiirusega ühest kohast teise. 
Tõenäoliselt ta väga suurt teadmist ühiskonnale sellega juurde 
ei lisanud nende nelja aastaga, aga sellise töö ta tegi. Ta 
on rääkinud, et kord, kui ta kuskil seltskonnas kirjeldas oma lõputööd, 
küsis vestluskaaslane vastu, kas see oli 
rohkem teoreetiline töö või tuli ka praktilisi laboratoorseid katseid teha.

\question{Kas selle asja nimi, mida te tol hetkel oma kambaga tegite, oli 
Bluemoon\index{Bluemoon}?}\label{sisu!bluemoon}?

Jah. Me panime oma mängutegijate pundile nimeks Bluemoon 
Software ja Bluemoon Interactive. Inimesed ikka tahavad panna 
kõlavaid firmanimesid.

\question{Miks just Bluemoon?}

Lihtsalt oli selline nimi. Ega me osanudki nimesid välja mõelda, mina 
sealhulgas, aga kui Starshipile\index{Starship 
Technologies} nime panin, siis ikkagi osalesin mõttetöös ja lõpuks panimegi 
minu pakutud nime.

\question{Programmeerimise juures pidi olema täpselt üks raske asi, nimede 
väljamõtlemine}\sidenote{Täpne tsitaat 
Netscape\index{Netscape}'i arhitektilt Philip Karltonilt\index[ppl]{Karlton, 
Philip} 
kõlab nii: \enquote{\emph{There are only two hard things in Computer 
Science: cache invalidation and naming things}}.}

Olen täitsa nõus, võibolla nüüd neljakümneaastasena on see juba 
natuke rohkem käpas. 

\question{Mida sa praegu teed?}

Olen firmas Starship Technologies ja ehitan 
pakiroboteid. Asutasime selle koos Skype'i\index{Skype} kaasasutaja 
Janus Friisiga\index[ppl]{Friis, Janus} neli ja pool aastat 
tagasi\sidenote{Intervjuu Ahtiga toimus jaanuaris 2019}. Ja meil oli selline 
visioon, et asjad võiksid maailmas liikuda automaatselt, samamoodi nagu 
elekter tuleb stepslisse ise, veevärk on olemas ja 
informatsioon tuleb läbi interneti. Asjad aga liiguvad ikkagi läbi meie maja 
või korteri ukse ja sa kas ise tood või maksad kellelegi, kes toob füüsiliselt 
kohale. See 
on hirmus raiskamine, asjad võiksid liikuda automaatselt, samamoodi nagu 
broneerime
lennupileteid üle interneti nii-öelda automaatselt, ilma et 
läheksime füüsiliselt reisibüroosse kohale.

\question{Starshipi tegemine on ju juhtimise töö. Kuidas sa jõudsid 
programmeerimise juurest selleni ja kui erinevad 
need tööd sinu jaoks on?}

Need on väga erinevad. Minu jaoks on see areng olnud selline, et 
olin üsna kaua aega programmeerija, ilma et oleksin 
üldse midagi kuskil juhtinud. Ja kui hakkasime koos Jaanus 
Friisi ja Niklas Zennströmiga\index[ppl]{Zennström, Niklas} startup'e tegema, 
siis olin 
tehnilise arhitekti rollis. See on
natuke rohkem juhtimisega seotud, aga sa ei juhi niivõrd inimesi, 
organisatsioone või protsesse, vaid tehnilist arhitektuuri. Seda,
milline see masin suures plaanis saab, mida suur 
tiim inimesi ehitab. See on nagu maja ehitamine: osa inimesi ehitab ja paneb 
kive üksteise peale ning teised vaatavad projekti 
suuremalt: kus peaks olema aken, mitu akent me üldse teeme, kas 
teeme ümmargused või kandilised aknad ja nii edasi. Skype'is olin ma alguses 
tehniline peaarhitekt ja 
mitmes teises startup'is samuti. Skype'is pooleldi juhtisin 
ka üht väikest tiimi: mõtlesin umbes viiele 
inimesele välja, mida nad tegema peaksid, ja koordineerisin nende tööd. 
Mõtlesin, mis on meie eesmärk, kuhupoole peaksime liikuma ja 
nii edasi. Väikse tiimi juhtimine oli nagu harjutus või sissejuhatus, mis andis 
mingisuguseid kogemusi. Hiljem olen juhtinud ka natuke suuremaid, umbes 
kümneseid tiime. Aga Starship oli esimene koht, kus 
ma võtsin üsna kiiresti, esimese kahe nädalaga tööle kümme inimest
ja esimese poole aastaga oli meid juba umbes kakskümmend. Eks ma käigu pealt 
õppisin, kuidas juhtimine käib, ja olen kindlasti alles
üsna alguses. Nii et Starship on olnud naljakas 
olukorras, et firmat juhib juhtimises üsna kogenematu juht. Neli aastat olin 
tegevjuht ja pool aastat tagasi jõudsime
nii kaugele, et palkasime professionaalse tegevjuhi Lex 
Bayeri\index[ppl]{Bayer, Lex} Californiast. Mina olen CTO ehk 
tehnikadirektor, kes peab ka üpris palju juhtima, aga mitte enam kahtsadat 
inimest, vaid natuke väiksemat hulka.

\question{See on siis olnud pikk ja just vajadusest ning huvist kantud 
õppimine?}

Jah, absoluutselt. Paljudele programmeerijatele, 
kaasa arvatud mulle, meeldib programmeerimine väga, see on niivõrd 
tore ja äge tegevus, et tahaks  
muudkui ehitada masinaid. Inimeste juhtimine on pigem selline asi, mida 
enamik programmeerijaid eriti teha ei taha ja ma ei ole päris kindel, kas ma 
isegi tegelikult tahan. Aga kui 
oled üksikprogrammeerija ja sul tekib arhitekti kogemus, siis 
oskad juba rohkem mõelda, mismoodi tarkvara ehitada ja mis 
on sealjuures oluline ja mis mitte. Siis on kaks võimalust: 
kas oled vait ja osaled protsessis kellegi teise juhtimisel või tekib üha 
rohkem tahtmine ise ja paremini 
teha kui see juht, kes meil on. Siis tahadki ise asja juhtida või sul on juba 
nii 
hea ettekujutus, kuidas teha, et ei suuda pealt vaadata, kui 
teine inimene, kes on võibolla väiksema kogemusega,  
kuidagi seda asja juhib ja mitte selles suunas, mida sina pead õigeks. Nii et 
see on tulnud vajadusest. Kui 
oled üksikprogrammeerija, siis saad aja jooksul aru, et jõuad
lõppkokkuvõttes rohkem tehtud, kui palkad 
endale tiimi ja hakkad seda juhtima. 

Minu jaoks on Starshipis kümnese 
tiimi juhtimisest kuni üle kahesaja inimesega firma juhtimiseni olnud küll nagu 
raketiga lendamine – võttis pea ringi käima. Ma kindlasti 
edutasin ennast oma ebakompetentsuse tasemele. Eks kohati öeldaksegi, et 
startup'id on sageli klassikalise juhtimise 
distsipliini ja teooria ning juhtimispraktika mõttes väga halvasti juhitud 
organisatsioonid. Tihtipeale ei ole see siiski olnud takistuseks nende edule, 
sellepärast et nad on värske mõtlemisega ja nende toode on 
piisavalt revolutsiooniline. Nii et sellest ei ole 
hullu, et nad on olnud halvasti juhitud. Ma pean nende kahesaja 
inimese ees, kes meie Starshipis töötavad, vabandust paluma, et nad on osalenud 
sellises loomkatses, 
et ma olen neid mitu aastat juhtinud. See ei ole võibolla olnud nende 
suhtes aus. Samas ei ole nad ka firmast minema jooksnud 
ja tunduvad olevat rahul. Järelikult väga hullusti ei olegi läinud.
