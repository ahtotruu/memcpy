\index[ppl]{Heinla, Ahti}
\question{Kuidas sa sattusid arvutite juurde?}
Ma tulen sellisest perekonnast, et minu ema ja isa olid mõlemad 
programmeerijad. Ja nemad olid siis ülikooli lõpetanud ja  said tööl kokku ka, 
see oli kuskil kuuekümnendate lõpp. Ja see oli see aeg, kus Eestisse tekkisid 
esimesed arvutid, mis sel ajal olid muidugi kapi suurused, aga ikkagi.

\question{Kuuekümnendate lõpus ei saanud neid programmeerijaid ju palju olla?}

Jah, kindlasti neid ei olnud palju, kuigi neid siiski ikkagi täiesti 
mingisugusel määral oli. Minul muide muide oli hiljuti selline asi, et kui emal 
oli selline suur juubel, üle seitsmekümne ja niimoodi, ja ta kutsus enda 
kursusekaaslased külla. Ja kes need kursusekaaslased siis on, need on 
rakendusmatemaatikud, praegu siis sellised üle seitsmekümne aastaseid  
inimesed, nii mehed kui naised. Põhimõtteliselt kõik programmeerijad, 
professionaalsed programmeerijad olnud, enam-vähem kõik, mehi ja naisi 
võrdselt. Ja, näed, meil on ikkagi asi juba nii kaugel, et meil on juba nagu 
suhteliselt kaugeid põlvkondi, kes on üles kasvanud programmeerijatena. Ja mina 
sündisin kahe sellise inimese järeltulijana.

\question{Kas see on pigem vedamine või vastupidi? Oleks võinud ju ka ära 
hirmutada?}

Mind see kindlasti ära ei hirmutanud, ma kasvasin üles perfolintide vahel. Ja 
niimoodi, et vahetevahel, kuna arvutiaeg oli ju piltlikult öeldes talongidega 
jagatav, arvuti pidi ikka õhtuti töös olema ja siis mõnikord  ema ja isa käisid 
õhtuti tööl, kui nad said arvuti aja kella kaheksaks õhtul, siis nad said 
arvuti aja kella kaheksaks õhtul. Ja siis nad võtsid vahepeal minu ja mu õe 
siis kaasa, mina jooksin ka arvutikappide vahel ringi ja vaatasin, kuidas seal 
magnetlindid vaikselt käisin nii ja naa ja see oli kindlasti hästi põnev. 
Hoopis teistsugune keskkond ja isegi helid on teistsugused ja vaatad, kuidas 
need masinad seal toimetavad, mingid magnettrumlid vaikselt vihisevad ja 
sahisevad ja kindlasti oli.

\question{Legendid räägivad, et selle põlvkonna rahvas korraldas Ameerikamaal 
lindikappide võidujookse ja muud sellist, tolles sinu arvutiruumis midagi 
sellist ka toimus?}

Mina selliseid asju ei näinud. Ma saan aru, et Eestis ka tehti selle sel ajal 
sellist  pulli, aga võib-olla  seda tegid natukene nooremad inimesed, kellel 
lapsi ei olnud. Minu isa ja ema olid ikka natuke siukse ontlikumad. Nad 
üritasid mingisuguseid konstruktiivseid asju arvutiga teha,  panna neid just 
ühel või teisel moel  käima, aga nad ei olnud sellised, kuidas öelda, häkkerid 
tänapäeva mõistes. Et  mismoodi arvutiga  pead pesta, näiteks, et selliseid 
asju nad ei mõelnud.

\question{Sind ju esialgu ei lastud linte perforeerima? Mis esimene asi oli, 
millele sa ise käed külge said?}

No mu vanemad olid programmeerijad aga mina ei olnud programmeerija, tavaline 
laps nagu ikka. Ma vist olin nagu natuke  matemaatiliselt  andekas, aga 
arvutitega otseselt minu tegelikult esimene kokkupuude oli ikkagi sellest, kui 
ma olin kümne aastane. Lihtsalt järsku päevapealt  ühel õhtul tuli ema  koju ja 
ütles, et kuule, Ahti,  ma õpetan sulle midagi, istume maha. Istusime maha ja 
ta õpetas mind programmeerima. Kolm õhtut niimoodi õpetas. Ja ma sain selle 
kolme õhtuga tegelikult sellest oast aru, et mismoodi see asi käib. Sealt 
alates  siis hakkasin juba ise edasi mõtlema, proovima, katsetama, lugema, 
natukene lolle küsimusi küsima. Kolm päeva ma olen sellist süstemaatilist 
programmeerimise õpetust saanud.

\question{Aga mis ta siis rääkis et see kümneaastasele huvitav oli?}

Eks mind huvitasid sellised asjad kindlasti. Sel ajal oli ju ka niimoodi,  
aasta oli 1992, et  lapsel on tohutult palju mingisuguseid ahvatlusi 
ümberringi, et Facebookid ja Instagrammid hüppavad siia-sinna ja kõikvõimalikud 
muud asjad käivad. Sel ajal oli ikkagi niimoodi, et ega meil kodus ju telefoni 
näiteks ei olnud. Arvutit ka ei olnud, sest mina kirjutasin programmi ikkagi 
alguses niimoodi, et kirjutasin paberi peale. Need kolm päeva õpet käis paberi 
peal. Ja kui  ema tuli õhtul koju ja sellist asja ütles, siis me ikkagi mitu 
tundi istusime maas, eks ole. Ei ole niimoodi, et mul oleks kogu aeg telefoni 
helisenud ja hüpanud, mingisugune asi, et kuule, Ahti, tule nüüd sinna, teeme 
seda. Selles mõttes võimalik, et ei olnudki nii väga vaja, et see oleks nagu 
hullult kuidagi põnev olnud kümne aastasele lapsele. Mul pigem oligi lõpuks  
põnev see, kui ma sain aru, kuidas see asi töötab.

\question{See peab olema päris korralik ettekujutusvõime, et sa paberi peale 
kirjutades saad aru, kuidas miski asi töötab. Sest paberil ei tööta sul midagi, 
seal on lihtsalt tekst}

Nojah, samas aga eks programmeerimise üks selline  võtmeoskus ongi tegelikult 
ju oskus ette kujutada, et mismoodi see masin  töötab. Lõppkokkuvõttes ju 
programmeerija ehitab ju masinat. Ja noh, piltlikult öeldes, ikka samasugust 
masinat nagu,  mingisugused hammasrattad kuskil käiksid. Üks koodirida on 
nii-öelda piltlikult öeldes üks hammasratas, teine koodirida on mingi kangikene 
kuskil seal, eks ole. Ja kui sa ehitad sihukest füüsilist või mehaanilist 
masinat, siis sa näed kõike seda, kuidas see töötab, et siin mingi ratas keerab 
ja siis kang liigub ja kuidas siis teine asi kuskilt midagi lükkab ja mingi 
lint või tross kuskilt midagi tõmbab. Sa näed seda kõike füüsiliselt. Ja 
programmeerija peab ka nägema. Aga ta peab nägema seda vaimusilmas, sest seda  
füüsilises maailmas  silmadega ei näe. Ja see vaimusilmas nägemise oskus on 
ülivajalik programmeerijale. Tagantjärele vaadates võib öelda, et eks ema mulle 
selle tegelikult õpetaski  see kolm päeva.

\question{Kas \texttt{goto} käib nii- või naapidi või tehete järjekord on 
selline või teine, on teisejärguline}

Just. Tegelikult oleks põhiliselt vaja teada, et sa \verb|goto| tegelikult teeb 
või et selles masinas, millise hammasratta, millise kujuga asja, see 
\verb|goto| seal teeb.

\question{See kolm päeva tekitas huvi, sa said enam-vähem aru, kuidas arvuti 
töötab, aga mis edasi sai?}

Siis läksime kuskil õhtul emaga siis sinna arvuti juurde, ema tööle. Ja siis me 
tippisime selle programmi sisse. Ja kui ma õieti mäletan, siis seda sisse 
tippimist võis juba mitu päeva olla. See oli ikka mitu lehekülge, see minu 
programm ja mõnikord läks midagi valesti ka ja nii edasi. Ema aitas mul siis 
seal mõned vead ära parandada ja siis tuli välja, et  tegelikult see programm 
töötas. See lahendas ühte väikest sihukest matemaatik  keerdülesannet, kus  
loogika oli  selles, et kui sul on  näiteks sada ühikut raha ja sa lähed 
raamatupoodi ja sa tahad seda sada ühikut raha ära kulutada. Siis sa pead 
kombineerima, et osta üks raamat, mis maksab viiskümmend seitse ja teine raamat 
nüüd maksab kolmkümmend, selline klassikaline  matemaatika keerdülesanne, 
kuidas kombineerida niimoodi, et kokku saada  summa, mis on võimalikult 
lähedane sajale aga mitte üle selles. Ja sellist ülesannet lahendas see minu 
programm. Ei ole nagu kõige triviaalsem asi, see ei ole nagu päris niimoodi, et 
vajutad nuppu ja programm ütleb lihtsalt \enquote{tere}. Tänapäeval ikkagi 
pigem kõik asjad üritatakse, ka heal põhjusel, ehitada niimoodi, et sul on 
selline nagu hästi kiire rahuldus või et sa nagu näed kaks minutit vaeva ja 
juba midagi hästi väikest nagu töötab ja siis sa näed veel viis minutit vaeva 
sealt tuleb veel midagi. Siis mina pidin vaeva nägema alguses kolm päeva, enne 
kui tulemust oli. Enne seda oli kõik ainult vaimusilmas.  Aga, tõepoolest, kui 
sul kogu aeg Facebook taskus ei hüppa, siis on nagu natuke lihtsam ka seda kolm 
päeva leida. 

\question{Mis tolle arvuti nimi oli?}

Ausalt öelda ma isegi ei mäleta, ei pruukinud isegi nõukogude masin olla,  seal 
oli tegelikult ka lääne aparaate.

Isegi minul sedasama seda ühte programmi, mis ma kirjutasin, seda minu meelest 
sai isegi  mitmel arvutil käitatud. Et see ei olnudki niimoodi, et 
\enquote{kuule Ahti see on nüüd sinu arvuti, millega nüüd sina  mitu päeva nii 
nagu tegeled}. See isegi vist nägi niimoodi välja, et ma pool programmi 
tippisin ühel arvutil sisse, mis oli sihuke suur must kapp ja siis järgmisel 
õhtul läksime ühe hoopis siukse läänelikuma välimusega siukse nagu nõtkema 
välimusega moodsama asja taha ja tippisin teise osa sisse. Et ma juba sain ka 
natuke kogemusi sellest, et see programm on ikka hoopis midagi muud, see ei ole 
see füüsiline arvuti, millega ma tegelen. Ma võin istuda ühe arvuti taha ja 
siis ma võin minna teisele kodusele teise arvuti taha, mis on terve toa suurune 
ja see minus seesama programm jookseb selles ühes jookseb selles teises.

\question{Mille peale sa vahepeal kirjutasid selle programmi? Kaartide peale?}

Siis olid ikka juba diskid olemas. Mitte need sihukesed, kolme tollised 
disketid, vaid sihukesed  kaheksa või viie tollised või mingid sellised asjad, 
pigem kaheksa tollised ilmselt. Aga kindlasti see esimene programm oli ainult 
selline algus, eks ole, sellest tuli mingisugune  oskus ja huvi asja vastu. 
Edasi hakkasin siis nüüd ise vaatama ja  sattusin kokku juba teiste poistega, 
kes siis analoogse asja vastu huvi tundsid. Lähemate aastate jooksul hakkasid 
tekkima ka personaalarvutid ja enam ei olnud alati niimoodi, et sa pead õhtul 
ema töö juurde minema tingimata vaid on kuskil juba muid kohti ka olemas.

\question{Kust sellised tutvused tekkisid, internetti ju polnud?}

Internetti ei olnud, aga küll oli olemas näiteks kaheksakümnendatel tekkinud 
selline asi, nagu Raaliklubi\index{Arvutiklubi!Raaliklubi}, mida vedas selline tegelane 
nagu Jaak Loonde\index[ppl]{Loonde, Jaak}. Mina olin ka selle klubi liige seal 
vahepeal ja see koondas sihukesi huvilisi poisikesi. Ega mul on raske öelda 
täpselt,  ise nagu poisikesena tol ajal süstemaatiliselt ei pannud tähele ja ei 
jätnud meelde ka täpselt, mida täpselt Jaak Loonde tegi ja kas ta üldse midagi 
tegi peale selle, et lihtsalt need poisid kokku tuua. Aga täiesti võimalik, et 
sellest piisabki, et need poisid kokku tuua, kellel on sama huvid ja siis nad 
omavahel juba vahetavad kogemusi, kellel kuskil jälle ema või isa töötab 
kuskil. 

Minul oli näiteks üks niisugune reliikvia, mille ema mulle andis: ta õpetas 
mind kolm päeva ja siis ta andis mulle ühe ingliskeelse raamatu, mis oli 
põhimõtteliselt Pascali programmeerimiskeele õpik. See oli inglise keeles, ehk 
siis ma ei saanud sellest eriti midagi aru. Ma koolis õppisin saksa keelt, 
mitte inglise keelt\sidenote{Tol ajal jagunesid koolid kaheks: kas lisaks vene 
keelele õpetatakse läbivalt inglise või saksa keelt}. Küll aga ma sain aru 
nendest  programmi näidetest, mis seal oli, eks ole, ja seal oli asjad ikkagi 
mingisugused loogilises  järjekorras. Tegelikult, kuigi ma inglise keelt ei 
osanud, ma siiski suutsin sellest raamatust kindlasti midagi õppida ja sealt 
tuli ideid, mida katsetada. Seal oli kuskil mingi, piltlikult öeldes, mingi 
\verb|goto| käsk, oletame. Ma ei teadnud, mis see tähendas, eksole, aga ma sain 
selle \verb|goto| käsu kuskil hiljem mingisse arvutisse sisse tippida ja 
vaadata, mis teeb ja küsida kelleltki teiselt, et mis see \verb|goto| tähendab. 
See on midagi muud kui lihtsalt öelda, et õpetaja mulle programmeerimist, et 
sul on ka konkreetne küsimus juba. Niimoodi läbi lukuaugu piltlikult öeldes see 
õppimine käis. Internetti ei olnud, aga  inimestel ei olnud ka internetti, siis 
kui nad  lennukeid ja autosid ehitasid, ja sai hakkama.

\question{Seal arvutiklubis sa käisid seepärast, et programmeerimine pakkus 
huvi?}

Jah, mulle pakkus see programmeerimise pool huvi. Mul tegelikult oli niimoodi, 
et isegi enne programmeerimist sattus kätte mingisugune lastele mõeldud 
elektroonika raamat ja ma  natukene nagu harjutasin või mõtlesin selle 
elektroonika peale ka, et kuidas näiteks transistorid töötavad ja muu selline. 
Ja see pakkus ka mulle kindlasti huvi. Aga tagantjärele vaadates siis ma 
ütleks, et minu elektroonika tegemine sel ajal oli  ülialgelisel tasemel. Ma 
nii-öelda  kuidagi nagu hästi õrnalt natuke nagu kõditasin elektroonikat ja 
üritasin sellest midagi aru saada, aga samas programmeerimisega ma tegelesin  
tegelikult. Selles mõttes oli seal väga suur vahe ja loomulikult oli ka väga 
suur vahe siis minu  professionaalsuse tasemes, mis  tekkis.

\question{Kas sa oskad öelda, kas see oli pigem eeskuju või midagi muud, mis 
sind pigem programmeerimise poole suunas?}

Üks asi oli kindlasti eeskuju, aga teine asi oli ka kindlalt puhtalt ju see, et 
selle jaoks, et elektroonikaga tegeleda, sul on vaja ikkagi mingisuguseid 
teatud füüsilisi asju. Sul on vaja elektroonikakomponente, sul on vaja 
tööriistu ja nii edasi, mida ju ei olnud. Isegi tänapäeval on ju poes  kõik 
olemas, aga sa pead minema ja üldse mitte vähe raha kulutama ja ostma need 
endale. Ma olen natukene ka hobi korras, elektroonikaga nagu tegelenud vahepeal 
kodus, tinutan üht-teist seal ja nii edasi. Ja noh, praegu, kui kõik on nagu 
justkui valla, kõik on olemas ja kahe päevaga tuuakse koju ära, raha ikka kulub 
selle jaoks. Mingi  üks jootekolb ja teine suurendusklaas, takistite 
komplektid, väiksed mikroprotsessorid, igasugused kivid ja sensoreid ja andurid 
ja displeid ja nii edasi. Sellega ei ole lihtne alustada. Programmeerimine on 
niimoodi, et sul tegelikult on vaja seda kohta, kus nii-öelda arvutis käia, eks 
ole, paber ja pliiats ja kolm päeva on täitsa piisavad alustamiseks.

Nagu mul sõber ja töökaaslane Jaan Tallinn\index[ppl]{Tallinn, Jaan} on  
öelnud, et programmeerimine on selline naljakas asi, et  enamikes muudes  
valdkondades on niimoodi, et kui sa hakkad   õppima, siis sa saad mingisuguse 
algse  taseme kätte ja siis sa pead rohkem õppima, et saada järgmisele 
tasemele,  sa pead veel rohkem õppima. Ja sa ei saa iseseisvalt õppida, vaid 
sul on vaja kedagi, kes õpetab. Kui sa õpid klaverimängu näiteks, siis sul on 
vaja tegelikult seda, et keegi sulle pidevalt õpetaks  klaverimängu, sa ei saa 
ise õppida klaverit mängima. Sul on mingisugused teatud käelised asjad, et 
mismoodi sa seda teed, parimal juhul sa saad või mingist YouTube'i, videost või 
mingist õpikust õppida. Aga sul on vaja seda YouTube'i videod või õpikut. 
Programmeerimine, aga, on tegelikult selline asi, et kui sa oled selle algse oa 
kätte saanud ja sind siis suletakse üksikule saarele aastaks niimoodi, et sul 
on arvuti käes, siis sa tegelikult suudad ise ilma ühegi õppevahendita õppida 
ennast väga heale tasemele, kui tahad. Puhtalt ise katsetamise,  ise mõtlemise  
teel. Ja eks tegelikult täpselt seda ma tegingi, sel ajal, kui ma teismeline 
olin.

\question{Kas sel ajal hakkas ka juba personaalarvuti moodi arvuteid liikuma?}

Jah, personaalarvuteid hakkas täiesti tulema ja meile koju tekkis ka üks Apple 
II\index{Arvutid!Apple II}. Sellega siis mina hakkasin toimetama, aga see oli 
üsna  kaheksakümnendate lõpus kuskil. Ma ei oska täpselt aastanumbrit öelda, 
aga ju ta võib olla võis juba 1988 olla või midagi niimoodi. Ma juba ikkagi  
nagu täiesti oskasin sel ajal programmeerida, ma ei olnud nagu päris enam kümne 
aastane, ma olin juba viisteist või kuusteist või midagi niimoodi. Inimestel, 
kes on seitsmeteist ja kaheksateist aasta vanused, enamikel inimestel on üsna 
kõvasti nagu meri põlvini  ja peod ja seltsielu ja asjad käivad. Aga minul on 
nagu paar asja teisiti. Esiteks ma olen üldiselt introvertne inimene ja mitte 
üli seltsiv, see seltsielu mul kuidagi nii hästi nagu välja ei tulnud. See on 
üks asi. Teine lihtne tõsiasi oli see, et vist kuni viimaste aastateni, umbes 
kolmekümne viienda neljakümnenda eluaastani oli mul elus selline asi, et kui ma 
joon kaks klaasi veini ära, siis hakkab mul pea valutama. Ma lihtsalt ei pea 
ühel  korralikul peol kaua vastu, lihtsalt ei pea,   ma lähen koju hiljemalt 
keskööks. Ja niimoodi on  kogu aeg olnud ja oli ka siis, kui ma olin 
kaheksateist, eks ole. Aga alates kella kaheteistkümnest ju nagu tegelik 
\emph{action} hakkab pihta, nagu mulle räägitud, ma nagu väga palju ise kogenud 
ei ole. Ja siis ongi see, et kui  ülejäänud inimesed avastavad seal seltsielu 
ja ja pidusid, siis osad teised avastavad arvutiasju ja siis avastavad 
seltsielu natukene hiljem lihtsalt.

\question{Aga mida sa programmeerisid? Sellise jõukohase aga samas huvitava 
ülesande leidmine ei ole ju üldse lihtne?}

No eks poisikesi ikkagi mängud huvitavad üsna palju ja kindlasti ma arvan  
minul ja minu kaasvõitlejatel kindlasti kõigil oli ju üks esimesi unistusi, et 
kirjutada oma üks arvuti mängida näiteks. Sel ajal oldi juba vahepeal need 
Yamaha arvutid juba tekkinud, eks ole, ja juba ka Apple II peal oli täitsa 
korralikke  mänge olemas. Need kommertsiaalsed mängud, need olid ikka sellisel 
tasemel, mida üks mingi hobistist poisikene ikkagi nädalaga valmis ei viska. Ja 
ega see tähelepanu ulatus  kolmeteistaastasel või viieteistaastasel ei ole väga 
selline, et suudaks midagi väga palju pikemat ette võtta. Selliseid väga 
lihtsaid mängukesi sai kindlasti ehitatud ja kindlasti ka proovitud üritada 
siis niimoodi häkkerlikult natukene läheneda sellele arvutile, et mida on 
võimalik arvutit tegema panna, mis hääli on võimalik arvutit tegema panna ja  
igasuguseid lollakaid visuaalseid kujundeid ette ette manada, ja muu selline. 
See pool kindlasti ka huvitas. 

Aga hiljem, tegelikult teismeeas sai igasuguseid asju proovitud. Ega täpselt ei 
teadnud, mida võiks  teha, aga valdkond kindlasti huvitas. Aga järjekindlamalt 
hakkasime mänge programmeerima Jaan Tallinna\index[ppl]{Tallinn, Jaan} ja Priit 
Kasesaluga\index[ppl]{Kasesalu, Priit}. Siis kui me olime keskkoolis. Siis oli 
juba niimoodi, et tähelepanu ulatus on juba nagu natukene inimesel juba 
kasvanud, eks ole, ja võtsime ette ühe mängu kirjutamise projekti. See sai 
natukene pikema vinnaga, et paneme nüüd kõik oma seni õpitud väiksed kogemused 
ja oskused kokku ja kohe mitu inimest paneme, mitte niimoodi, et igaüks oma 
nurgas pusib mingeid oma mängu, vaid teeme ikka sellise tiimi töö. Jagame 
ülesanded omavahel ära ja kuude viisi töötame selle kallal. 

\question{Kust selline mõtte üldse tuli või arvamus, et selline asi üldse 
võimalik võiks olla?}

Kuskilt iseenesest tuli, ma ei oska täpselt mõelda. Meil isegi ei olnud isegi 
mingit arutelu sel teemal. Lihtsalt sündis, et proovime midagi, midagi sellist. 

\question{Mis keskkool see oli?}

Mina õppisin Gustav Adolfi Gümnaasiumis\index{Koolid!Gustav Adolfi Gümnaasium}
\index{Koolid!Gustav Adolfi Gümnaasium|see{Tallinna 1. Keskkool}}
ja Jaan Tallinn\index[ppl]{Tallinn, Jaan} oli minu pinginaaber. Ja Priit 
Kasesalu\index[ppl]{Kasesalu, Priit} oli Jaan Tallinna pinginaaber eelmisest 
koolist, kus Jaan käis. Nii et me olime mõlemad, Jaani pinginaabrid olnud. Ja 
siis viimase keskkooliaasta jooksul niimoodi kolmekesi kirjutasimegi ühe mängu, 
millel oli nimeks Kosmonaut\index{Mängud!Kosmonaut}. Mina küll kogu selle 
kirjutasin seda kui hobiprojekti, aga Jaan ikka ütles, et see asi tuleb teha 
nagu äriks või see asi tuleks maha müüa ja selle eest  raha saada. 

\question{See oli nõukogude aeg ju veel, selle eest võis kinni minna ju?}

Peaaegu. See oli nõukogude aja lõpp küll, sel ajal, kus juba igasuguseid 
metalliärikaid juba juba käis ringi ja niisugune nagu väikene üle piiri  
kaubandus käis ja kooperatiivid ja asjad ja selline värk juba täitsa toimis. Me 
muidugi ei teadnud tuhkagi sellest, kuidas  see  ettevõtluse või selline maailm 
üldse käib. Ja ega tegelikult ei teadnud seda ka, need suured inimesed, kes sel 
ajal ettevõtluses nagu  olid. Aga mingi tegevus toimus ja siis osade  
täiskasvanud näiteks metalliäri alal mõningast mõningase kogemusega või 
sidemetega inimeste abil õnnestus meil tõesti see Kosmonaudi mäng müüa Rootsi. 
See oli selles mõttes muidugi pöördeline sündmus, et me saime selle eest 
lõppkokkuvõttes ikkagi, kui ma õigesti mäletan, siis oli see viis tuhat 
dollarit. See oli täiesti kosmiline number,  aasta oli mingi 1990 ja ja  rubla 
kurss oli seal selline, et vist kui ma õieti mäletan, ühe dollari eest sai 
kolmkümmend rubla juba. Ja siis kui arvutad kokku, siis see viis tuhat dollarit 
oli ikkagi umbes selline summa, mis meie vanemad olid elu jooksul teeninud või 
midagi umbes sellist. Loomulikult inflatsiooniga võib seal seal korrutada ja 
korrigeerida, aga ikkagi  oluline number, ikka väga-väga oluline number. Kas 
nüüd mõelda, et kui õigesti me seda summat  kasutasime, kokkuvõttes sa ikkagi 
ka valuutapoes käidud ja Coca-Colat ostetud, selle peale kulus ka ikkagi 
märgatav osa sellest ära. Aga mina ja Jaan ostsime enda endale näiteks kahe 
peale arvuti. Selle peale läks pool sellest  minu ja Jaani osas sellest rahast 


Selle raha me saime kätte kuskil, see oli juba üheksakümnendate alguses,  Eesti 
kroon oli just tulnud või tulemas. Sellega on selline lugu kohe, et see on just 
täpselt see aeg, kus Eesti kroon tuli niimoodi, et minul oli see raha rubladena 
käes. Ja siis oli mingi kooperatiiv või firmakene, kust me siis olime kokku 
lepitud ja välja valitud mingi 386SX protsessoriga arvuti ja me olime seda siis 
ostmas. Ja siis ma mäletan, see oli see hetk, kus meil oli teada, et järgmine 
päev on siis see rahareform ja minul olid need rublad käes, kümnete tuhandete 
kaupa neid rublasid, mis oli selle arvuti jaoks mõeldud. Ja meil oli kokku 
lepitud, eks ole, et me anname need nii palju rublasid ja saame siis selle 
arvuti. See kooperatiivitegelane, kellele helistasin, siis ütles midagi, et too 
homme see taha või midagi niimoodi. Aga siis mul ikka nii palju oidu oli, et ma 
ütlesin, et ei, on kokku lepitud, ma toon täna selle raha. Ja ma tõingi täna 
selle raha ja ta võttis selle täna vastu ja me saime selle arvuti kätte umbes 
homme või midagi, või veel samal päeval. Nii et jah, ma ei tea, mis oleks 
juhtunud, kui oleks tegelikult üritanud homme selle raha  maksta. 

\question{Eks ajalugu oleks läinud tonks teistmoodi. Aga see oli juba 386, mis 
oli juba päris korralik aparaat. Sinna vahele jääb ju õige mitu aastat 
puselemist mingite teiste inimeste arvutite juures. Kus te selle mängu 
kirjutasite? Kodus kellegi juures?}

Mängu me kirjutasime suurel määral tegelikult Jaani\index[ppl]{Tallinn, Jaan} 
ja Priidu\index[ppl]{Kasesalu, Priit} töökohas. Sest Jaan ja Priit keskkooli 
kõrvalt töötasid programmeerijatena ühes kooperatiivis. Mina tegelikult ka 
töötasin keskkooli kõrvalt programmeerijana poole kohaga minu vanemate töökohas 
ehk Küberneetika Instituudis\index{Küberneetika Instituut}. Aga  ütleme 
niimoodi, et ma arvan, et  minu vanemate tööandja oli selles mõttes mõistlik. 
Kui ma ise tööandjana mõtlen, et kui mingisugune seitsmeteistaastane poiss 
tahab tööle tulla,  alles õpib programmeerima või niimoodi, et ega esiteks ma 
ei maksaks talle väga palju või ma ei võtaks teda nagunii väga tõsiselt.  
Samuti ma võib-olla ei annaks talle nii palju mingeid võimalusi, ma vast ei 
annaks talle missioonikriitilisi asju. 

Aga Jaan ja Priit olid, olid tööl ühes kooperatiivis, kus nemad olid vaata et 
sihukesed peaaegu et juhtprogrammeerijad või midagi niimoodi.  Ja neil oli 
tunduvalt paremad võimalused  käes. Mis on noh, tänapäeval vaadates, ma ütleks, 
ikkagi küllalt ebamõistlik, aga need olidki  ebamõistlikud. See tähendas, et 
nad ei saanud oma arvuteid nii-öelda töölt koju kaasa võtta, aga neil oli 
tegelikult töökoht, kus nad said päeval  olla koolis, aga õhtud-ööd said olla 
arvutis. Ja sel ajal,  kui sa oled kuusteist ja seitseteist, siis võid vastu 
pidada niimoodi, et magad kuus tundi päevas, siis kui vaja.

\question{Kui ma nüüd kokku loen, siis te käisite Gustav Adolfi Gümnaasiumis, 
mis polnud lihtne asi, te töötasite programmeerijatena ja takkapihta 
kirjutasite mängu, mille kannatas pärast maha müüa. Kõike samal ajal?}

Jah, peab ütlema küll,  et vähemalt siis, kui mina töötasin programmeerijana, 
ma töötajana ei ole uhke tööpanuse üle, mille ma Küberneetika Instituudile 
andsin\index{Küberneetika Instituut}. Tõsi küll,  ma sain ikkagi midagi valmis 
ja mu tööandja oli rahul sellega. Ma ei olnud ka tegelikult ainus, oli natukene 
teisigi selliseid õppijaid ja mõni üliõpilane, kes oli seal niimoodi tööl ja ma 
sain isegi aru, et mu tööandja isegi oli pigem minuga rohkem rahul kui seal 
mõnede teistega. Aga ma arvan, et see ütleb rohkem nende teiste kohta kui 
minu kohta. Mina ikkagi kulutasin enamiku ajast selle mängu ja koolis käimise 
peale.

\question{Sel ajal hakkasid tekkima esimesed BBSid ka?}

BBSid hakkasid tekkima ja nii-öelda minu tutvusringkonnast siis Priit 
Kasesalu\index[ppl]{Kasesalu, Priit} oli see põhiline, kes meie kambas tegeles 
BBSidega ja ühe ka püsti pani, mille nimi oli \emph{Dark Corner}\index{BBS!Dark 
Corner}, kui ma õigesti mäletan. Ja mille Fido, kuidas seda siis nimetati, 
\emph{node} number või midagi sellist, oli, kui ma õieti mäletan, neliteist. Ja 
teda tõmbas nagu see pool kuidagi rohkem või kuidagi väga palju ja eks 
kindlasti mind ka, sest BBSiga tekkis järsku  võimalus  ekraani kaudu suhelda 
hästi paljude teiste inimestega, kellega sa võib-olla füüsiliselt ei istu koos. 
Teatud mõttes võiks isegi öelda, et järsku nendele inimestele anti natuke nagu 
Facebook kätte. Mitte taskusse otseselt, aga ikkagi kätte või niimoodi, et 
järsku tekkis hulk sõpru, kellega ma olin suhelnud ainult interneti teel. Ja 
Fidos vahetati mõtteid  kõikide asjade üle, mitte ainult arvutite üle ja tekkis 
järsku üks mingisugune  täiesti isevärki sotsiaalne seltskond. Tolle aja aja 
kohta oli see väga isevärki sotsiaalne seltskond. Tänapäeval on niimoodi, et 
sotsiaalne seltskond, kes on mingi Facebookis mingisuguse grupi, olgu mingi 
MMSi klubi või ma ei tea mis, liige,  siis nad võivad aeg-ajalt kokku saada. 
Netiajastul on see tegelikult väga-väga tavaline. Aga selline selline, kuidas 
öelda elustiil või tutvusringkonna ülesehitus järsku tekkis  meile kätte, kui  
aasta oli umbes 1990 või umbes kuskil sealkandis.

\question{See seltskond pidi siis olema ka teatavas mõttes homogeenne, sest 
Fido külge saamise barjäärid olid kõrged?}

Jah, eks muidugi oli palju ka inimesi, kes nii-öelda jõlkusid kaasas. Olid 
sellised entusiastid nagu näiteks Priit Kasesalu\index[ppl]{Kasesalu, Priit} 
või Tarmo Mamers\index[ppl]{Mamers, Tarmo} näiteks no nende muud sõbrad  
aeg-ajalt tekkisid ju ka sinna sisse, kellele siis  Tarmo või Priit võimaldasid 
ligipääsu. Ja see oli kindlasti väga huvitav. Tekkis selline  sotsiaalne 
distants-suhtlus. 

Ma mäletan ühte juhtumit, oli juba tegelikult siis, kui vaikselt Internet juba 
hakkas Eestisse tekkima. Internet kui selline tehniliselt oli juba olemas juba 
ju kuskil seitsmekümnendatel kaheksakümnendatel, aga  Eestisse  ta umbes sel 
ajal niimoodi natukene juba tekkima. Mul oli selline sõber, siiamaani väga hea 
sõber, nimega Sulo Kallas\index[ppl]{Kallas, Sulo}, kellel oli ka BBS ja kes 
töötab minuga koos Starshipis\index{Starship Technologies} praegu. Tema andis 
mulle kasutada ühte oma kontot ühes Unixi arvutis. Ja Unixis oli olemas selline 
programm nagu \verb|talk|, kus sai siis omavahel ekraani kaudu suhelda 
inimesed, kes olid sisse loginud samasse masinasse. Ja ma mäletan, et  minu 
jaoks oli üks ikkagi täiesti selline silmi avav  elamus.  Mul ei olnud sel ajal 
kodus telefonigi. Ja siis ma midagi toimetasin selle Sulo kontoga Sulo nime alt 
selles ühes arvutis ja järsku selle \verb|talk|iga  hakkab minuga keegi 
rääkima.  Ütleb, et minu nimi on Epp. Nii, ja mina siis esimese asjana, kuna ma 
teadsin, ma kasutatud Sulo kontot, eks ole, keegi Epp tahab Suloga rääkida. 
Siis ma selgitasin talle, et kuule, mina ei ole Sulo, et mina olen hoopis üks 
teine inimene. Tema ütleb vastu, et  sellest pole midagi, räägime ikka. Ma ei 
saanud täpselt aru, mis värk on nagu, mis mõttes, ta ju tahab Suloga rääkida, 
eks ole. Aga siis ma sain aru, et ta tahab tegelikult lihtsalt kellelegi 
rääkida, et tal  tegelikult on täitsa okei, et ta räägib  minuga. Sihuke 
jutuajamine tekkis sealt, ja ma sain teada selle jutuajamise käigus, et  
tegemist on ühe Eesti tüdrukuga, kelle nimi on Epp ja kes hetkel füüsiliselt 
asub Ameerikas. Ja ta läks Ameerikasse  ülikooli õppima, ta oli Ameerikas 
üliõpilane. Ja mina istun Eestis, eks ole, ja ma reaalajas räägin arvuti 
ekraani vahendusel  temaga juttu, eks ole. Me rääkisime maast ja ilmast 
mingisugune tund aega, see oli  väga-väga kummaline kogemus. Sa  suhtled 
kellegagi reaalajas, kes on nagu sinust väga-väga kaugel. Ma siiamaani ei tea, 
kes Epp täpselt oli, ta ütles oma perekonnanime ka, ma ei ole seda nime mitte 
kunagi hiljem kuulnud, mitte kunagi hiljem selle inimesega suhelnud. Aga see 
oli ikka väga kummaline kogemus minu jaoks. Ongi naljakas tegelikult et, 
tänapäeval ju selline asi on ju niivõrd tavaline, kõigil mingid Snapchatid ja 
asjad on kuskil taskus, eks ole. Ja tol ajal oli sotsiaalses mõttes see, et sa 
võid suhelda inimestega kuskil üle maailma,  oli nendele interneti häkkerite 
võimalik ja teistele inimestele ei olnud.

\question{Sa rääkisid, et BBSides räägiti igasugustel teemadel. Näiteks, 
millest räägiti?}

Kui ma õieti mäletan, seal oli igasugust, sellist elulist, nagu tänapäeva 
internetifoorumid, eks ole. Kõigest võidakse seal rääkida. Seal oli mingisugune 
filosoofiateemaline  vestlusgrupp, kus  inimesed olid ju enamasti sellised 
kaheksateist aastased, kes veel mõtestavad oma elu. Ongi selline aeg inimeste 
elus, kus kõik mõtlevad, mida tähendavad mingisugused asjad ja kas ikka inimene 
peaks panustama sellele või tollele. Tänapäeval neljakümneaastasena väga 
võib-olla ei viitsi sel teemal juttu vesta väga, kõigil on juba oma elu 
tõekspidamised välja kujunenud, aga tol ajal minul kindlasti ei olnud ja enamik 
sellest ülejäänud BBSi seltskonnast oli ka umbes sama vanad, eks ole. Siis oli 
seal igasuguseid psühholoogiateemalisi, neid vestlusi oli igasuguseid, see 
kindlasti ei olnud sugugi mitte ainult tehnoloogiateemaline. 

\question{See, mis sa ütled, kõlab väga oluliselt. Sest see tähendab, et 
mingisugune ports nutikaid inimesi mitte üksinda ja mitte juhuslike inimestega 
vaid koos sama moodi mõtlevate ja samade oskustega inimestega mõtestasid seda, 
mida tähendab olla inimene kõige laiemas mõttes}

Absoluutselt. See oli tegelikult üks niisugune virtuaalne sõpruskond.  Võib 
olla võib öelda, et see Fido seltskond oli kõige esimene virtuaalne sõpruskond 
Eestis üldse. Tänapäeval on  igaühel virtuaalseid sõpruskondi taskus sada tükki 
aga see võis olla võib olla täiesti esimene.

\question{Kas selle kõige juurde käis ka mingi spetsiifiline raamatu-, muusika- 
või filmihuvi?}

Ahaa, muusikakanaleid oli loomulikult ka, muusikateemalisi  vestlusgruppe. 
Minul ei käinud. Võib-olla natukene. Ma arvan, et  selles ringkonnas pigem olid 
populaarsed sihukesed elektroonilise muusika bändid. Nii, ja naa, ütleme. 
Kraftwerk mulle ei meeldinud ja ei meeldi siiamaani, Jean-Michel Jarre samuti 
mitte nii väga palju aga Tangerine Dream näiteks meeldis mulle väga ja 
siiamaani meeldib, mul on ikka mingi viisteist nende plaati ja nii edasi. Aga 
samas jälle ma olen inimene, kes ei ole kunagi vaadanud Star Warsi, ma ei ole 
kunagi lugenud \emph{Hitchhiker's Guide to The Galaxy}'t. Minu jaoks  on 
esteetiline subkultuur ja arvutid natukene lahus seisnud.

\question{Endal sul BBSi ei olnud?}

Minul endal BBS-i olnud. Ma vist nagu kuidagi ei tahtnud ka, see oli ikka hull 
jahmerdamine, mis oli vajalik selle BBSi üleval hoidmiseks ja sellega pidevalt 
toimetada. Mul oli väga hea meel, et ma sain  Priidu BBS-i kasutada.

\question{Selge. Aga siis te müüsite selle mängu maha, mis edasi sai?}

Noh, kui üheksateistaastasele inimesele anda nii palju raha, nagu tema vanemad 
on kogu elu jooksul teeninud, eks ole, siis tal karjäärivalik on nagu selge 
kohe, eks ole ju. Et noh, sellist küsimust nagu ei olnud, et mida ma siis 
tulevikus professionaalselt tegema hakkan. Loomulikult programmeerija.  Ja  mul 
oli ka selline mõtlemine, ma ei tea, kui õigustatud see oli, aga ma arvasin, et 
et noh, eriti üheksakümnendate alguses Eesti ülikoolides eriti midagi väga 
kasulikku sel teemal ei õpetatud. Ma ei tea,  kui õige või vale see on. 
Kindlasti vastas tõele see, et meil keskkoolis oli  arvutiõpetus ka ja üldiselt 
ikkagi meie klassist pigem paljud teadsid rohkem kui meie õpetaja. Ma 
miskipärast oletasin, et ülikool siin samamoodi, ma ei tea, kas see on tõsi või 
mitte. Tänapäeval see kindlasti ei ole tõsi aga  tol ajal  võib-olla pigem oli. 
Igatahes ma tegin selle otsuse, et ma ei lähe ülikooli õppima midagi 
programmeerimise või arvutitega seotut, vaid ma läksin hoopis õppima füüsikat. 
Füüsika oli kindlasti mul  niisugune teine selline huviala,  ma olin 
füüsikaolümpiaadidel käinud  ja mulle see kindlasti kindlasti väga meeldis. 

\question{Aga mis sulle füüsika juures meeldis?}

No võib-olla natuke sihuke filosoofiline aspekt, et ma sain kuidagi aru, kuidas 
nii-öelda maailm toimib teatud mõttes. See oli põnev. Mingid sihukesed asjad, kui 
 mingid tuumafüüsikad ja mingid planeedid, kuidas liiguvad ja niimoodi, see 
natukene andis võib just sellist filosoofilist mõõdet. Mis see maailm meie 
ümber on ja kui suured või väikesed meie, inimesed, selles maailmas  oleme.  Ja 
noh, pigem ikkagi väga väikesed oleme. 

\question{Kuhu sa läksid seda füüsikat õppima?}

Ma läksin  füüsikat õppima Tartusse\index{Tartu Ülikool}, koos Jaan 
Tallinnaga\index[ppl]{Tallinn, Jaan}. Pinginaabrid läksid mõlemad õppima 
füüsikat. Sellega läks niimoodi, ma kindlasti  tegelikult ei väärtustanud seda, 
et piltlikult öeldes mul oleks paber taskus, et mul ülikooli oleks  kuidagi 
edukalt lõpetanud. Ja kui ma olin ühe aasta või poolteist aastat ülikoolis ära 
olnud, siis mulle hakkas veel rohkem kohale jõudma see, et tegelikult ma ju 
tegelen programmeerimisega kogu aeg, töötan professionaalse programmeerijana. 
Samal ajal tegi mind järgmist mängu, mille me kavatsesime maha müüa ja nii 
edasi ja nii edasi. Ja ma kunagi ei kavatsenud füüsikuna töötada, ma hobi 
korras õppisin füüsikat. Kui esimese aasta sai hobi korras  füüsikat õppida 
siis teisel aastal hakkad aru saama, et tegelikult  õppejõud ikkagi eeldavad, 
et sa tõsiselt tegeled selle asjaga, panustanud enamiku oma ajast füüsiku 
õppimisse. 

Ja siis ma tulin ülikoolist ära. Ma sain aru, et see asi lihtsalt nõuab rohkem 
tööd, kui ma olen nõus sinna sisse parema ja siiamaani ma ülikooli lõpetanud ei 
ole. Jaan Tallinn\index[ppl]{Tallinn, Jaan} käis ülikooli lõpuni ja õppis 
füüsika siis siis lõpuni. Tegi oma oma lõputöö, kui ma õieti mäletan, 
relatiivsusteooriast. Sellest, kuidas ruumi painutada selle jaoks, et reisida 
valguse kiirusest suuremate kiirusega ühest kohast teise. Ma küll oletan, et 
tõenäoliselt  ta mingisugust väga suurt teadmist ühiskonnale sellega juurde nii 
väga ei lisanud selle nelja aastaga, mis ta õppis aga sellise töö ta tegi. Ta 
on rääkinud, et ükskord, kui ta kuskil seltskonnas kirjeldas oma seda tööd, 
mida ta tegi, siis tema vestluskaaslane küsis  vastu, et kas see oli nagu 
rohkem teoreetiline töö või tuli seal ka mingeid praktilisi laboratoorseid 
katseajale.

\question{Selle asja nimi, mida te tol hetkel kampas pidasite, oli juba 
Bluemoon\index{Bluemoon}?}\label{sisu!bluemoon}

Jah. See mängutegijate punt, me hakkasime ennast nimetama nimega Bluemoon 
Software ja Bluemoon Interactive. Inimesed ikka tahavad panna mingisuguseid 
kõlavaid firmanimesid.

\question{Aga miks just Bluemoon>}

Lihtsalt oli üks nimi. Ma arvan, et me ei osanud nimesid üldse välja mõelda ja 
ma olen kogu aeg pidanud ennast väga halvaks nimede väljamõtlejaks ja et ma ei 
valda seda valdkonda üldse ja niimoodi, aga kui Starshipile\index{Starship 
Technologies} nime panin, siis ikkagi osalesin selles kõvasti ja  lõpuks oli 
ikkagi minu pakutud nimi, mis selleks lõpuks sai.

\question{Programmeerimise juures pidi olema täpselt üks raske asi, nimede 
välja mõtlemine}\sidenote{Eksin tsitaadiga. Täpne tsitaat on 
Netscape\index{Netscape} arhitekti Philip Karltoni\index[ppl]{Karlton, Philip} 
poolt ja kõlab nii: \enquote{\emph{There are only two hard things in Computer 
Science: cache invalidation and naming things}}}

Ma olen täitsa nõus sellega, võib-olla nüüd neljakümne aastasena on juba 
natukene rohkem käppa seda saadud. 

\question{Mis sa praegu teed?}

Praegu ma olen sellises firmas nagu Starship Technologies ja ehitan 
pakiroboteid. Asutasime selle selle firma koos Skype'i\index{Skype} kaasasutaja 
Janus Friisiga\index[ppl]{Friis, Janus}  neli pool aastat 
tagasi\sidenote{Intervjuu Ahtiga toimus jaanuaris 2019}. Ja meil oli selline 
visioon, et asjad võiksid ju maailmas liikuda automaatselt samamoodi, nagu 
elekter tuleb meile stepslisse seina ise sisse ja veevärk on olemas ja 
informatsioon tuleb läbi interneti. Aga asjad liiguvad ikkagi  läbi meie maja 
või korteri ukse, tulevad füüsiliselt kohale ja alati mingisugune inimene toob 
seda, kas sa ise tood või siis sa maksad kellelegi inimesele, kes toob. Ja see 
on hirmus raiskav ja asjad võiksid liikuda automaatselt samamoodi nagu me 
lennukipileteid broneerime üle interneti nii öelda automaatselt, ilma et me 
läheksime füüsiliselt kohale kuskile reisibüroosse seda lennukipiletit ostma.

\question{Starshipi tegemine on ju juhtimise töö. Kuidas sa jõudsid 
programmeerimise juurest selle töö juurde, mida sa praegu teed ja kui erinevad 
nad sinu jaoks on?}

No need on ikka väga erinevad. Minu jaoks on see areng olnud selline, et ma 
olin programmeerija ja ma olin programmeerija üsna kaua aega, ilma et ma oleks 
üldse midagi kuskil juhtinud. Ja kui me hakkasime startuppe tegema koos Jaanus 
Friisi ja Niklas Zennströmiga\index[ppl]{Zennström, Niklas} siis ma olin 
nendest startuppides tehnilise arhitekti rollis. Arhitekti roll on juba rohkem 
natukene nagu  juhtimisega seotud, aga sa ei juhi nii väga  inimesi või 
organisatsioone või protsesse, vaid sa juhid just tehnilist arhitektuuri. Et 
milline see masin niisuguses suures plaanis kokku tuleb, mida siis terve suurem 
tiim inimesi ehitab. Nagu maja ehitamisega: osad inimesed ehitavad ja panevad 
kive üksteise peale ja on ka teisi inimesi, kes vaatavad seda projekti 
suuremalt, et kus peaks olema paneme aken ja mitu akent me üldse teeme ja kas 
me teeme rohkem ümmargused aknad või teeme kandilised aknad ja nii edasi ja nii 
edasi. Ja ma olin Skype'is, olin siis tehniline peaarhitekt alguses  ja 
mitmetest teistes startuppides samuti. Skype'is veel natukene pooleldi juhtisin 
ka ühte väikest tiimi, kus  ma tegelesin sellega, et mõelda umbes viiele 
inimesele välja seda, mida nad tegema peaksid ja koordineerida nende tööd. 
Mõtlesin välja, mis meie eesmärk peaks olema, kuhu poole me peaksime liikuma ja 
nii edasi, nii edasi. Sihukene viie inimeselise tiimi juhtimine oli selline 
nagu väike harjutus või  sissejuhatus, et mingisuguseid kogemusi natukene sain 
või natukene kujutasin ette. Hiljem olen juhtinud siis ka natuke suuremaid 
tiime, umbes kümneinimeselisi ja niimoodi. Aga Starship oli esimene koht, kus 
ma üsna kiiresti võtsin tööle kümme inimest, võtsin esimese kahe nädalaga tööle 
umbes ja esimese poole aastaga oli juba umbes kakskümmend inimest meil tööl ja 
nii edasi  läks juba natukene suuremaks see asi. Eks ma niimoodi käigu pealt 
natukene siis  õppisin, et  kuidas juhtimine käib. Ju ma olen kindlasti veel 
üsna  alguses seal, et me oleme siin Starshipis olnud sihukeses  naljakas 
olukorras, kus nagu juhtimises ikkagi üsna kogenematu juht on olnud sellel 
firmal. Neli aastat ma olin tegevjuht ja  nüüd jõudis pool aastat tagasi siis 
asi nii kaugele, et me palkasime  professionaalse tegevjuhi Lex 
Bayeri\index[ppl]{Bayer, Lex} Californiast. Ja mina olen CTO ehk 
tehnikadirektor, kus ka peab üsna palju juhtima, aga nüüd enam mitte kahtsadat 
inimest, vaid natukene väiksemat hulka inimesi.

\question{See on siis olnud pikk ja just vajadusest ja huvist kantud õppimine?}

Jah, absoluutselt.  Üldiselt ma ütleks niimoodi, et paljud programmeerijad, 
kaasa arvatud ka mina, meile programmeerimine meeldib nii palju see on niivõrd 
tore tegevus ja niivõrd äge tegevus, et selliseid masinaid ehitada, et tahaks  
muudkui eitada neid masinaid. Inimeste juhtimine on pigem selline asi, mida 
enamik programmeerijaid väga ei taha teha ja ma ei ole päris kindel ka ise, kui 
palju mina seda tegelikult teha tahan. Aga küll on lihtsalt asi selles, et kui 
sa oled  üksikprogrammeerija ja sul kogemus tekib ja sa oled  arhitekt ja sa 
oskad juba rohkem  arvata, mismoodi me seda tarkvara peaksime ehitama ja mis 
asjad on selle juures olulisi, mis need ei ole. Siis on nagu on kaks võimalust, 
kas sa  oled vait ja kellegi teise juhtimisel osaled selles protsessis või siis 
sa üha rohkem nagu vaatad seda, et ei, ma teen ise, ma teeksin seda paremini 
kui see juht, kes meil on. Ma tahaks ise seda asja juhtida või mul on juba nii 
hea ettekujutus, kuidas seda teha, et ma ei suuda pealt vaadata, kui 
mingisugune teine inimene, kes on võib-olla väiksema kogemusega kui mina,  
kuidagi seda asja juhib ja mitte selles suunas, kus mina olen täiesti 
veendunud, et  õige oleks. Ehk siis see on tulnud justkui nagu vajadusest. Kui 
sa oled üksikprogrammeerija, siis sa aja jooksul ikkagi saad aru, et sa saad 
tegelikult lõppkokkuvõttes rohkem tehtud, kui sa piltlikult öeldes palkad 
endale tiimi ja hakkad juhtima mingisuguseid suuremaid seltskondi. 

Minu jaoks küll nii-öelda raketiga lendamine nagu Starshipis, kümneinimeselise 
tiimi juhtimisest kuni selleni, et ma juhtisin üle kahesaja inimesega firmat 
tükk aega, see ikkagi võttis pea ringi käima. Et ma kindlasti kindlasti 
edutasin ennast  oma ebakompetentsuse tasemele. Aga eks kohati öeldaksegi, et 
starupid ongi asjad, mis on väga sageli on  klassikalise sellise juhtimise 
distsipliini ja teooria ja juhtimispraktikate mõttes väga halvasti juhitud 
organisatsioonid. Mis ei ole siiski tihti takistuseks olnud nende edule, 
sellepärast et nad on olnud nii piisavalt värske mõtlemisega, nende toode on 
olnud piisavalt selline värske ja revolutsiooniline, et sellest ei ole olnud 
hullu, et nad on olnud halvasti juhitud. Tegelikult ikkagi need kakssada 
inimest, kes meie Starshipis töötavad,  ma ikkagi vaatan nende peale küll nagu 
niimoodi, et palun vabandust nende ees, et nad on osalenud sellises loomkatses, 
et mina olen neid juhtinud mitu aastat. See ei ole võib-olla olnud aus nende 
suhtes. Aga samas nad ei ole ka sugugi mitte meil siin firmast minema jooksnud 
ja tunduvad olevat rahul, et võib olla väga hullusti, siis ei olegi läinud.
