%!TEX TS-program = arara
% arara: myindex

\index[ppl]{Anikin, Sergei}
\textbf{\enquote{Kuidas sina arvutite juurde said?}}

Ja see oli päris huvitav lugu. Ma olin, võib öelda, \emph{entitled} mu isa oli
elektroonikainsener töötas Kalinini tehases\index{Kalinini
tehas}\sidenote{Algselt Balti Raudtee Peatehased, mis ehitati 1870. aastal ja
mis aastatel 1902 kuni 1903 seal töötanud Nõukogude riigitegelase järgi 1940.
aastast alates M.I. Kalinini nime kandis. 2007. aastast alates asub samal
territooriumil ja osalt samades hoonetes Telliskivi Loomelinnak restoranide,
kohvikute, kontorite ja loominguliste ruumidega}, see koht, kus meil nüüd on
see kõige popim koht noorte seas. Seesama Kalamaja ja see Lendav Taldrik.
Tegelikult ma olen seal lapsena käinud koos isaga, seal oli valvur, valvurist
pidi läbi minema, et sinna territooriumile saada. Nad tegid rongidele
elektrimootoreid ja jõuelektroonikat, minu isa projekteeris neid. Aga
hobi korras ta on teinud igasugust raadiotehnikat ja mina ise olen proovinud
mingit väikest raadiot kokku panna. Kuigi mina olin täiesti võhik, selles osas kuigi
käisin raadiotehnika mingisuguses ringis.

Aga minu esimene arvuti sai siis minu isa poolt kokku pandud.

\textbf{\enquote{Aga kust ta jupid sai?}}

Isal  oli selline ajakiri Vene ajakiri nagu 
\begin{russian}Радио\end{russian}\index{Ajakiri!\begin{russian}Радио\end{russian}}\sidenote{Igakuine populaarteaduslik 
raadiotehnika ajakiri, mida andsid välja Nõukogude Liidu Siseministeerium ja 
DOSAAF (\begin{russian}Добровольное общество содействия армии, авиации и флоту 
России\end{russian} - Vabatahtlik Venemaa armee, lennunduse ja mereväe 
abistamise selts). Ilmus eri nimede all alates 1925. aastast, 1975. aastal oli 
ajakirja tiraažiks 850 000 eksemplari}. Ja siis aastal 1986 avaldati seal 
kõigepealt arvuti skeemid ja siis, kuidas seda kokku panna. See oli 
Nõukogudemaal välja töötatud arvuti, aga skeemid nad võtsid selle ZX Spectrumi 
pealt\sidenote{Tegemist on arvutiga 
\begin{russian}Радио-86РК\end{russian}\index{Arvutid!Radio-86RK}, mis oli üks 
edukamaid  koduseks kasutamiseks mõeldud Nõukogude arvuteid. Kuigi Nõukogudemaal 
kopeeriti ZX Spectrumit usinasti, oli selle arvuti puhul siiski väidetavasti 
tegu originaalse disainiga, autoriteks Dmitri Gorshkov, Yuri Ozerov, Gennady 
Zelenko ja Sergey Popov (Stachniak, Zbigniew. "Red clones: The soviet computer 
hobby movement of the 1980s." IEEE Annals of the History of Computing 37, no. 1 
(2015): 12-23.)}. Isa siis kõigepealt korjas need komponendid kokku, ise 
joonistas plaadi ja kuna tal oli juurdepääs siis tehases tegi plaadi valmis ja 
pani selle kokku. Ma mäletan, et tal läks ikka paar kuud, enne kui ta kõik need 
vigased kohad seal ostsilloskoobiga välja juuris. Siis pani käima. See käis 
teleka  külge,  telekas oli monitori asemel. See oli mustvalge telekas, 
värvitelekat  meil peres ei olnud.  Aga ega ma sellega  midagi väga teha ei 
saanud, sest tal ei olnud isegi opsüsteemi. Tollel arvutil oli \emph{interface} 
kassettmakiga, aga meil ei olnud ka kassette, mille pealt laadida seda 
opsüsteemi. Selles samas ajakirjas oli baitkoodis opsüsteemi kood trükitud. 
Kakskümmend lehekülge bait baidi haaval. See oli talvel. Pimedad õhtud, ma 
põhimõtteliselt istusin kaks nädalat ja trükkisin need kõik koodid sisse.

\textbf{\enquote{Miks sa tegid seda? Normaalne laps ju ei toksi niimoodi 
pimedatel õhtutel baitkoodi?}}

Kas sellele on eellugu. Isa sõber tõi mulle umbes aasta enne seda lasteraamatu  
programmeerimisest. Seal  mingisugused  tegelased siis õppisid programmeerima 
BASICus\index{Keeled!BASIC}. Lugesin selle raamatu läbi, sain aru, kuidas 
programmi kirjutada, kirjutasin BASICus umbes kümnerealise programmi, mis 
midagi arvutas. Ja siis kompileerimise osas, me ei saanud ju paberi peal
kompileerida, ma näitasin sellele isa sõbrale ja siis ta vaatas, kontrollis,  
ja ütles, et see töötab küll.

Aga noh, see programm oli olemas, aga ma ei saanud proovida seda, et mul oli 
vaja arvuti tööle panna. Siis ma trükkisin need baitkoodid sisse ja  lõpuks 
sain oma programmi umbes aasta pärast sisse trükkida.

See sissetoksimine käis plokkide kaupa. Seal oli, ma ei mäleta, kui suur, aga 
umbes poole leheküljeline plokk, millel oli kontrollkood. Ma sain seda 
kontrollkoodi valideerida, kui see klappis, siis ma salvestasin selle makile. 
Kui ei klappinud, siis ma pidin viga otsima. Põhimõtteliselt ma pidin algusest 
peale selle ploki sisse toksima, sest selle vea leidmine oli väga-väga 
keeruline. Aga, aga ma arvan, et juba sellest ajast, mul tekkis esiteks 
kannatus ja teiseks tähelepanu detailidele. Selle baitkoodide sissetoksimisega 
ma lõpuks sain aru, et mul on hästi oluline kõik need õigesti ja õiges 
järjekorras sisse toksida sest ümber tegemine oli nii piinlik.


\textbf{\enquote{Sulle tehti väiksest peale selgeks, et sa võid küll üle jala 
lasta, aga siis sa ise toksid neid samu asju kolm korda}}

Ja, aga loomulikult, enamik aega mis arvutiga sai veedetud, olid  mängud. Tol 
ajal alguses arvutis olid need tavapärased madu ja mingisugune tennis. Sai neid 
mängitud. Siis isale meeldis arvuteid kokku panna ja ta on pannud ka sellesama 
ZX Spectrumi\index{Arvutid!ZX Spectrum}, isegi mitu tükki, kokku. Tegelesime ka 
selle väliskorpusega. Tol ajal vaata, Eestis on kuiv õhk ju talveti ja siis 
meil olid plastmassist õhuniisutajad, mis käisid radika peale. Sellest sai väga 
hea korpusse sellele arvutile. Ta oli õige kujuga, sinna sai sisse lõigata 
selle klaveri, toiteplokk, plaat kõik, mida vaja. Makk oli eraldi.

\textbf{\enquote{Miks sulle see elektroonika osa huvi ei pakkunud?}}

Ega mul ei olnud arvutite vastu mingisugust suurt kirge, siiamaani ei ole
tegelikult. Minu arust see on ikkagi vahend. Tänapäeval on ju
teada, et need, kes arvutitega tegelevate teenivad päris korralikult raha,
onju. Tol ajal see oli ka mingis mõttes staatuse küsimus, et sul peres oli
arvuti. Kui paljudel peredes
oli arvutid? Alles mitmed aastad hiljem tekkisid  need arvutiklubid või
arvutiga  mängida kohad. Aga mul oli kodus selline. Me ei olnud
jõukas pere, meile polnud raha et osta niisugust asja. 

Arvuti on jah, pigem vahend. Ta on meeldiv hobi ka, aga ma ei ole selline, et
see oleks ainuke hobi. Mingi aeg ma üldse ei tegelenud arvutitega, mulle see
mängimine enam kirge ei pakkunud ja programmeerida lihtsalt enda jaoks ei
tundunud väga huvitav. Aga mul oli üks sõber, me mängisime koos. Ja siis ta
mainis, et \enquote{hoo, et, et ma nüüd käin arvutiklubis. Ja seal õpime programmeerima,
aga mina muidugi enamasti käin seal mängimas}. Ja siis ma mõtlesin, et tema ju
tegelikult ei oskagi midagi. Et mina ju oskan ja peaks koos temaga minema. Sa
ilmselt oled rääkinud paljude inimeste Eesti kogukonnast aga, aga mina sattusin
siis Vene kogukonda. Selle arvutiklubi nimi oli Interface\index{Arvutiklubi!Interface}.

\textbf{\enquote{Aga kes seda klubi pidas ja kuskohas?}}

Selle vedaja oli, ma mäletan, naisterahvas. Ta töötas vist Bioloogia
Instituudis siin Mustamäe teel ja vedas laste arvutiklubi. Ta nii nimi oli
Nina Botina\index[ppl]{Botina, Nina}. Me käisime seal
Reaalkoolis\index{Koolid!Tallinna 2. Keskkool}\index{Koolid!Reaalkool|see{Tallinna 2. Keskkool}},
seal olid arvutiklassid, tundides.

\textbf{\enquote{Mis koolis sa ise käisid?}}

See on kool number kakskümmend kuus\index{Koolid!Tallinna 26. Keskkool}. 
Viimases klassis ma läksin Tõnismäe Reaalkooli\index{Koolid!Tõnismäe Reaalkool} 
kus oli väga tugev matemaatika. Ja tegelikult seesama Nina Botina surus mind ja 
veel ühte klassiõde, et me läheksime teise kooli, et lõpetaksime selle 
matemaatika klassi. Tema pärast me läksime sinna kool ja seal oli hästi palju  
tuttavaid sellest samast arvutiklubist.

Ja hiljem sellest arvutiklubist on kasvanud venekeelne tehnikakool või
arvutitehnikakool, mis oli Erika tänaval. 

\textbf{\enquote{Ma teadsin, et Tartu ja Tallinna vahel on erinevus. Aga 
selgub, et ka Tallinna sees on kaks täiesti isesugust Tallinna?}}

See on huvitav jah. Ja seal oli ka niimoodi, et minu huvi arvutite vastu 
vaheldus. Üks aasta ma olin seal klubis aga siis, kui ma läksin uude kooli, mul 
ei olnud aega, et  sellega tegeleda. Aga siis Nina kutsus, et kuule,  mul ei 
jätku instruktoreid. Tule, mul on uued grupid, tule ja aita 
arvutiklassis. Siis kuidagi  tekitas uuesti huvi. Kui ma kooli lõpetasin ja  
ülikooli läksin majandust õppima\index{Tallinna 
Tehnikaülikool!Majandusteaduskond}. Aga seal jälle  esimese aasta lõpus tekkis 
võimalus spetsialiseeruda majanduslikku andmetöötlusesse. See oli hästi pisike 
grupp, mingi seitse inimest. Kui kõik, kes olid majanduses, õppisid majanduse 
aineid, siis meie enamik meie tunde olid  arvutitehnika gruppidega.

Ja see meie grupp oli eestikeelne. Ma läksin venekeelsesse majandusteaduskonda, 
aga see grupp oli eestikeelne. Aga
see oli huvitav jällegi, et me ei pidanud õppima arvutitehnika baasaineid. 
Esimese aasta arvutitehnikas nad õppisid füüsikat-keemiat, kõiki neid üsna 
keerulisi ained. Ma olen kuulnud õudseid lugusid, kuidas inimesed ülikooli 
lõpuni ei saanud neid tehtud. Aga meie õppisime mikro ja makroökonoomikat, 
inglise keelt. Ja alates teises aastast hakkasime koos arvutitega õppima. Ja ei 
olnud erilist jõudluse vahet, tundus.

Aga ma jällegi mõtlen, et see, kus ma praegu olen, ilmselt on  ka  
põhjustatud sellest, et ma ei läinud väga süvitsi  arvutitehnikasse, vaid pigem 
alati oli arvuti mul  vahend mingi probleemi lahendamiseks.

\textbf{\enquote{Sa mainisid, et sul matemaatika tuli välja. Kas sa kuskil 
olümpiaadidel ka käisid?}}

Käisin, aga ma olin niisugune keskmine. See nii palju sõltub õpetajast, onju. 
Ma mäletan, meil oli kas viies või seitsmes\sidenote{Selle põlvkonna inimestel, 
nii vene- kui eesti koolides, jäi üks klassi vahele, sest koolid läksid 
kaheksakümnendate teisel poolel üle aasta võrra pikemale õppele}, minu meelest 
seitsmes klass, kus hakkas juba geomeetria ja muud sellised asjad. Ja siis mul 
kuidagi klikkis, et iga teoreemi kohta, mida meile räägiti, mul tekkis teine 
viis, kuidas seda tõestada. Ma kuidagi sain nagu aru, et ei ole alati ainult 
ühtemoodi, saab teistmoodi ka. Ja siis jällegi see klikib õpetajaga. Kui 
õpetaja näeb, et õpilane mõtleb, siis ta pöörab rohkem võib-olla tähelepanu 
inimesele. Aga noh, siis tema läks ära ja järgmised õpetajad ei olnud väga head.

Siis meil oli üks väga hea füüsikaõpetaja, tal oli hästi palju kontrolltöid. 
Tema juures ma õppisin seda, et üldse ei pea neid valemeid meelde jätma. Piisab 
sellest, kui sa oskad neid rakendada. Loomulikult spikerdamine ei olnud 
lubatud, aga mul ikkagi need valemid olid spikrina vihiku tagakaanel. Sa pead 
aru saama probleemist, pead aru saama, mis vahendeid kasutada selle probleemi 
lahendamiseks. Ja see õpetaja vaatas läbi sõrmede nende valemite peale, sest 
kui sa  probleemist aru ei saa, siis füüsikas lihtsalt valemid ei aita. 

Ja kui me läksime sinna Tõnismäe Reaalkooli, oli seal legendaarne 
matemaatikaõpetaja Mihhail Vassiljevitš\index[ppl]{Vassiljevitš, Mihhail}, 
siiamaani õpetab. See, kuidas inimene, õpetaja on ju autoriteet, kohtleb 
inimesi! Selles mata klassis, seal selgelt olid kolm või neli  tippõpilast, kes 
võitsid kõik riiklikud olümpiaadid  käisid maailmaolümpiaadidel. Loomulikult ta 
tegeles nendega, aga ta tegeles ka kogu ülejäänud rahvaga. Seal olid ka need, 
kes ei saanud väga aru aga tema juures need nende tase tõusis. Ta oskas 
selgitada ka keerulisi asju nii lihtsasti asju, et kogu klass  põhimõtteliselt 
oli paar taset teistest koolidest üle. Lihtsalt see, et sa olid seal  
keskkonnas juba tõstis sinul taset nii kõvasti.


\textbf{\enquote{Jällegi tuleb välja, et matemaatika tunnis õpiti lisaks 
matemaatikale ka suhtumist ja just see viimane on aastate järel meeles}}

No see olümpiaadide küsimus jällegi, Mina ei saanud seal mingeid kohti. Aga 
klassis, mis oli meist aasta vanem,
 oli selline lugu, et umbes kümme inimest läksid keemiaolümpiaadile, kümme 
inimest läksid matemaatika ja kümme läks füüsika olümpiaadile. Kõik need 
riiklikud olümpiaadid olid ju Tartus. Põhimõtteliselt terve klass läks 
olümpiaadile, aga erinevatel aladel. Ja kuna nad olid juba seal kohal, siis 
neil oli lubatud  ka teiste ainete olümpiaadides osaleda. Mille tulemusena nad 
kõikidel aladel, isegi need, kes ei kvalifitseerunud alguses, said enam-vähem 
kõik esikümnesse kõikidel aladel. Saad aru, see oli selline nii võimas klass, 
täiesti hämmastav.

\textbf{\enquote{Miks sa majandust läksid õppima?}}

Sest mu vanemad ütlesid, et meil on peres juba kaks inseneri olemas, ema oli 
soojustehnik. Eks ma mõtlesin minna kuskile mujale ka õppima, aga  kodu juures 
on palju lihtsam. 

\textbf{\enquote{Kas sul oli mingi ettekujutus sellest ka, mis sa pärast oma 
haridusega ette võtta tahad?}}

Ega ega mul väga ei olnud mingit ettekujutust. Ma arvan, et mul lihtsalt ei 
olnudki mingit plaani. Ma tahtsin lihtsalt näha, et mis  see majandus siis 
õigupoolest on. Suvel ma korra proovisin töötamist müügiinimesena. Selgus, et 
see ei sobi mulle absoluutselt. Sest mulle ei sobinud see, et müügitöös 
üheksakümmend kaheksa protsenti inimestest ütleb \enquote{ei}. Aga mulle see ei 
sobinud, mulle ei meeldi feilida ja minu jaoks \enquote{ei} tol ajal oli feil. 
Tegelikult nüüd kui ma olen siin Pipedrive'is\index{Pipedrive} juba seitse 
aastat töötanud, ma saan aru, et see on osa protsessist, on statistika, see ei 
ole feil. Et feil on see, kus sa ei tee seda üheksakümne üheksandat korda
müüki, mis võib õnnestuda. Müük on see, et sa tead oma neid statistilisi 
numbreid ja plaanid vastavalt nendele. Mitte ees see, et kui esimene juhuslik 
inimene ütleb sulle, et mul ei ole seda teenust vaja, siis sa oled feilinud. 
Tegelikult ei ole.

\textbf{\enquote{Mida sa müüsid?}}

See oli tänavamüük, tegelikult. Tänava peal müüsime erinevaid tooteid a la 
tööriistakaste (mis läksid tegelikult päris hästi), mingisuguseid 
elektroonilisi hambaharju ja nii edasi.

\textbf{\enquote{See on ju igavene raske töö!}}

See on väga raske töö ja  see süsteem oli niimoodi, et igal hommikul me tulime 
sinna, kus oli ladu ja saime päeva kvoodi. Et pead näiteks müüma viisteist 
tööriistakasti. Ja kui sa kvoodi täitsid mingisuguse kahe nädala jooksul siis 
sa said järgmisse tiitli ja selle tiitliga sa said endale õpilasi. Ja kui viis 
õpilast said kvoodi nii-öelda täidetud mingi aja jooksul, siis said oma 
nii-öelda äri. Aga, jällegi, õppetund oli see, et see töö ei ole kindlasti minu 
jaoks. Ja ma teadsin, et kui ma lähen programmeerijaks, siis ma saan oluliselt 
rahulikuma töö eest oluliselt suuremat tasu. See sundis mind mingi hetk, umbes 
pool aastat hiljem, ütlema, et \enquote{Okei, ma lähen nüüd}. See oli ülikooli 
teise aasta poole peal umbes, siis kui ma sellesse informaatika gruppi läksin.

\textbf{\enquote{Kas sa siis juba programmeerisid mingeid tõsisemaid asju ka 
või lihtsalt loengus puutusid kokku?}}

Olen teinud kahte projekti, mis nii palju kui ma mäletan, tõi natuke raha.

Üks oli selline. Tol ajal olid hästi populaarsed need 
sat-tv\sidenote{Kaheksakümnendate lõpus ja üheksakümnendatel oli suhteliselt 
lühike periood, mille jooksul isiklik satelliidivastuvõtja oli ületamatult 
kallis, piraatlusele vaadati läbi sõrmede (õigupoolest keegi ka sealtkaudu 
eriti ei vaadanud), suuri teenusepakkujaid veel polnud aga väikestel oli juba 
võimalus tegutseda. Siis pandigi mõne kortermaja katusele satelliiditaldrik, 
hangiti piraat-kaart tasuliste kanalite jaoks, veeti üle katuste 
ümberkaudsetesse majadesse kaablid ja asuti teenust müüma} firmad. Mõnes 
väikeses rajoonis oli mingis oma kunn, kes pakkus seda sat-TV-d kuutasu eest. 
Siis oli üks tuttav, kes palus teha infosüsteemi, kus oleks kirjas, kes on 
liitunud, kes ei ole ja kui palju nad maksavad ja mis teenust nad kasutavad. 
Emal oli jällegi tööl arvuti, mille peal ma sain teha Accessi\index{Microsoft 
Access} andmebaasi ja selle peale väikese liidese. Ma ei tea, kas ta kasutas 
seda hiljem või mitte.

Teine oli veel huvitavam. Kui ma sain teada, kui palju raha ma selle töö eest 
saan, ma olin väga imestunud. Isa sõbrad, tegelesid valvesüsteemidega. Neil oli 
projekt, vanglaprojekt. Nad panid valvesüsteemi vanglasse ja neil oli 
põhimõtteliselt vaja joonistada selle vangla projekti järgi mingisugune skeem, 
kus oleks näha, kus on alarmid tööle läinud. Ta ei olnud otseselt 
programmeerimine, ta oli rohkem disain või midagi sellist. Ma pidin neid pilte 
joonistama ja siis ma sain mingisuguse kolme nädalaga selle tehtud ja see 
summa, mis ma sain, oli mu isa umbes poole aasta palk. Siis ma sain aru, et, et 
arvutitega tasub toimetada.

\textbf{\enquote{Kust sa infot said? Ega Accessis programmeerimine ei ole ka 
niisama lihtne, et muudkui otsast hakkad tegema?}}

Kusjuures Accessi koht ma ei mäletagi, eks ma vist lugesin dokumentatsiooni. 
Programmeerimist õppisin 
raamatutest. Mul oli üks üks Pascali raamat, mis õpetas objektorienteeritud 
programmeerimist, venekeelne raamat. See aitas mul mõista, just 
objektorienteeritust. Ja ülikoolis tegelikult mõned ained olid väga-väga 
kasulikud. Näiteks andmebaaside projekteerimine. Tänapäeval väga paljud 
inimesed ei oska relatsioonilist andmebaasi projekteerida ja see ja see on üks 
vajalikumaid oskusi, tegelikult, kui sa tahad isegi lihtsat süsteemi kokku 
panna. Tänapäeval lahendatakse selliseid asju tihti lihtsalt jõuga.

\textbf{\enquote{Kas sinu arvuti või reaalainete huvi juurde käis ka 
mingisugune spetsiifiline, näiteks ulme, raamatu-huvi? Vene keeles oli ju palju 
rohkem asju kättesaadavad, mina ei olnud suuteline tol ajal Strugatskeid 
originaalis lugema}}

Ega ma ei mäleta, väga oleks olnud. Raamatuid mulle meeldis lugeda, mulle 
meeldis ka ulme nii-öelda või fantastika. Aga mul ei tekkinud nagu arvutitega 
seost. Minu jaoks arvuti on nii praktiline asi kui olla saab. Eks 
Bulõtšovi\sidenote{Kir Bulõtšov (1934 --- 2003). Nõukogude 
ulmekirjanik}\index{Kir Bulõtšov} ja Strugatskeid\sidenote{Arkadi Strugatski 
(1925 --- 1991) ja Boris Strugatski (1933 --- 2012), Nõukogude ulmekirjanikud. 
Kirjutasid enamasti koos, seega tuntud kui \begin{russian}братья 
Стругацкие\end{russian} või lihtsalt Strugatskid}\index{Strugatskid} aga ka 
välismaa asju. Aga ma olen ka kõik need Barbar Conani\sidenote{Robert E. 
Howard'i poolt 1932. aastal loodud tegelane, kes on sellest ajast tembutanud 
kõikvõimalikes meediumides ajakirjadest ja raamatutest filmide ja 
videomängudeni } ja Tarzani\sidenote{Edgar Rice Burroughs' poolt 1912. aastal 
loodud tegelane, kes Nõukogude Liidus sai tuntuks kinodes näidatud 
trofeefilmide (Johnny Weissmuller'i kehastatud tegelane erines küll oluliselt 
raamatukangelasest) kaudu} raamatud läbi lugenud.

\textbf{\enquote{Mis su esimene päris programmeerija töö oli ja millal see 
oli?}}

Mul olid sõbrad seal juba ees, veebruaris 1996 ma läksin tööle Aeteci
Finantsvara ASi\index{Aeteci Finantsvara AS|see{Profit Software}} mis nüüdseks 
on Profit Software\index{Profit Software}. Nad tegid soomlastele igasugu 
finantskindlustussüsteeme. Ma mäletan, et esimese oma tööülesandega ma 
feilisin, sest mulle anti mingisuguse
valemi programmeerimine. See pidi Cs\index{Keeled!C} olema ja sellest pidi 
\emph{library} saama. No mul ei olnud teadmisi. Ma ei teadnud, kuidas Cs 
kirjutada, ma ei saanud sellest valemist aru (see oli kõrgem matemaatika). 
Ühesõnaga, sellega ma feilisin. Aga milles ma olin väga hea, oli meil Lotus 
Notes'i\index{Lotus Notes Domino} tarkvara, mida kasutati suhtlemiseks omavahel 
ja soomlastega. See oli dokumendianadmebaas tegelikult. Tal oli oma 
skriptimiskeel ja sellega siis ma kirjutasin reisikindlustuse süsteemi 
kindlustusagentidele, et nad saaksid välja arvutada, palju see reisimine maksab 
ja saaksid poliisi teha. Ja see oli internetipõhine aastal 1997. Selle Dominoga 
oli võimalik, need samad dokumendid, mida sa muidu Lotus Notes'i enda kliendiga 
nägid, oli võimalik ka veebiserveri kaudu, HTML dokumentidena näidata.

Aga see kogemus aitas mul saada Hansapanga\index{Pangad!Hansapank} internetipanga 
tiimi.

\textbf{\enquote{Kuidas sa sinna sattusid?}}

See oli ka naljakas. Tegelikult Hansapanga ITs või üldse pankades ilmselgelt 
oli rohkem raha kui mingis IT-firmas. Ja siis kaks aastat töötasin Aeteci
Finantsvaras ja tundsin, et võiks väikse nii-öelda karjääri teha. Ja tegelikult 
kõikidesse pankadesse proovisin tööle saada, et seal olid vabad kohad. 
SEBs\index{Pangad!SEB|see{Ühispank}} võis toona Ühispangas\index{Pangad!Ühispank} ma ei 
saanud isegi vist jutule, aga ma rääkisin Hoiupangas\index{Pangad!Hoiupank} Aleksei 
Bljahhiniga\index[ppl]{Bljahhin, Aleksei}. Hansas oli ka tööintervjuu, läksin 
Vilve Vene\index[ppl]{Vene, Vilve} ja Heiki Kübbariga\index[ppl]{Kübbar, 
Heiki}. Ja siis ma mõlemast pangast sain tööpakkumise umbes sama summa peale. 
Otsustasin Hansapanga kasuks, sest arvasin, et seal on võib olla natuke rohkem 
karjäärivõimalusi nii-öelda. Minu esimene tööpäev Hansapangas oli 
üheksateistkümnes jaanuar 1998. Kui ma läksin sinna fuajeesse, seal oli värske 
Äripäev, kus oli kirjas, et Hoiupank ja Hansapank ühinevad. See minu ise 
esimene tööpäev oli sama päev, kus teatati ühinemisest. Ja see määras kogu minu 
järgmist nii-öelda karjääri.


\textbf{\enquote{See tähendab, et sa pidid suhteliselt ruttu hakkama 
internetipanga asemel tegelema hoopis Light Telleri nimelise telleri 
töökohasüsteemiga?}}

Sinna läks veel natuke aega. Ma arvan, et see otsus hakata seda tegema sündis 
umbes viis-kuus kuud peale seda kui ühinemine pihta hakkas. Sest alguses ju  
ei olnud veel selge, et kumba süsteemi üldse hakatakse kasutama ja kuidas see 
otsus tehakse. Ja sellel ajal mina õppisin siis kuidas internetipanka teha.

\textbf{\enquote{See kõik on mulle üllatus. Mina sisenesin sinnasamma panka 
1999. aasta lõpus. Light Teller oli selleks hetkeks olemas ja laua taga oli 
vana kala nimega Sergei, kes selle omakäeliselt valmis oli teinud. Kui nüüd 
näppude peal arvutada, siis see tähendas, et sa tegid nullist 
täisfunktsionaalse veebipõhise telleri töökoha umbes kolme kuuga?}}

No ega ma üksi ei olnud. Aga astume sammu tagasi. See Hansapanga esimene 
internetipank oli jällegi ehitatud tehnoloogia peale, mis oli ajast ees. See 
oli Oracle\index{Oracle} mingisugune veebi veebikomponent või veebiserver kus 
sa said PL/SQLiga\index{Keeled!PL/SQL} tekitada HTML mida siis kliendid 
vaatasid. See oli omal ajal hästi lihtne, mitte mingit disaini ei olnud, sest 
vist disaineritest keegi ei teadnud tol ajal, et on selline amet nagu disainer. 
Trükidisainerid kindlasti olid aga kasutajaliidese disaineritest mitte keegi ei 
teadnud tol ajal. Ja siis kui ma tulin siis vaatasin, et, \enquote{oo milline 
ebavõrdsus!}. Et et see internetipank on ainult eesti keeles. Kohe ütlesin, et 
noh, mis ta siin teha on, ma võin näiteks teha niimoodi, et ta on 
mitmes keeles. Öeldi, et tee. Ja siis ma tegingi. Kaks nädalat tegelesin 
sellega, et võtsin kõik tekstid välja, asendasin \verb|lang| funktsiooniga, mis 
arvestas ka kasutajaprofiiliga. Ma veel õppisin samal ajal ülikoolis, see 
päev, kui Madis Ollisaar\index[ppl]{Ollisaar, Madis} asjad toodangusse pani, olin ma koolis. 
Logisin siis sisse, et vaadata, kas töötab. Eesti keel töötab, inglise keel 
töötab, vene keel näitab küsimärke. Ilmselt siiamaani inimesed mässavad nende 
\emph{encoding}utega, aga see oli minu esimene kokkupuude sellega, kui minu 
arvutis töötab aga serveris ei tööta.

Aga aga samal ajal hakkas juhtuma ju mitu asja nii et, toimus ühinemine, 
Aleksei Bljahhin\index[ppl]{Bljahhin, Aleksei} tegeles \emph{data} migraga. 
Tekkis probleem, et telleri programm oli kirjutatud Oracle Formsi. Igas 
kontoris oli Formsi server. Ja kõik tellerid siis kasutasid Formsi klienti mida 
serveeriti sealt serverist ja nad võtsid peaserveriga Oracle 
andmebaasiühenduse. Oracle'i litsentsid, teatavasti, maksid ühenduste arvu 
pealt. Siis kujuta ette, et Hansapangal on, no ma ei tea, mingi nelikümmend 
kontorit äkki? Nüüdseks see on juba suur number, aga Hoiupangal oli nelisada 
kontorit Eestis. Ja paljudes maakohtades ei olnud isegi nii head ühendust, et 
hoida seda pidevat ühendust baasi otsa. Ja kui nad arvutasid, kui palju need 
Oracle litsentsid maksnud oleksid, siis nad ütlesid, et võib-olla anname selle
Hoiupanga tagasi. 

Siis tegelikult oli see hästi julge otsus. Ma ei tea, kes selle nüüd võtsid 
vastu, ilmselt needsamad Vilve\index[ppl]{Vene, Vilve} ja 
Gibbs\index[ppl]{Gibbs|see{Kübbar, Heiki}}. Aga otsus oli, et teeme siis 
interneti telleri programmi ja samal ajal meile müüdi uut tehnoloogiat 
internetipanga tegemiseks, BroadVisioni\index{BroadVision} nimeline platvorm. 
BroadVisioni müügiargumendiks oli, et me saame põhimõtteliselt e-kommertsi 
platvorm,  seal sai igale kasutajale näidata personaalselt välja nägevat 
rakendust.

Aga jällegi iga kasutaja maksis. Mis tähendas, et me kunagi ei kasutanud neid 
võimalusi, kõik oli anonüümne selle süsteemi mõttes. Aga ta pakkus 
\emph{template}'mise võimalust, mis oli väga suur samm võrreldes selle Oracle 
PL/SQLiga, kus sa pidid oma HTMLi ise kokku panema. Selle peale me ehitasime. 
Aga ma arvan, et see telleri arhitektuur, kui sa praegu mõtled sellele tagasi, 
ta oli võimas, aga ta oli hästi lihtne. See võimaldas tegelikult 
funktsionaalsust hästi kiiresti hästi suures koguses toota.

\textbf{\enquote{Selle arhitektuuri kohta ma tahaks küsida. Ta ju oma olemuselt 
oli toonaseid vahendeid kasutades täpselt selline, nagu täna \emph{de facto} 
veebirakendused on. JavaScript\index{Keeled!JavaScript} jooksis brauseris ja 
tegi \emph{backend}i poole päringuid. See lahendus oli oma 20 aastat ajast ees, 
kuidas ta sündis?}}

Seal oli seesama piirang, et maakontorite ühendus oli hästi aeglane. Ehk, me
pidime optimeerima, kui palju me \emph{data}t liigutame kliendi ja serveri 
vahel. See oli üks nõudmistest, mis sundis mõtlema, et me peame palju tööd juba 
kliendis ära tegema. Aga kliendiks oli brauser ja JavaScripti versioon oli 
selline, et parimal juhul sai mingit validatsiooni teha. Midagi joonistada väga
ei saanud või mingid midagi dünaamiliseks teha. Aga siis tulli samal ajal 
Internet Explorer 4.0\index{Internet Explorer}, kus olid \emph{custom} 
JavaScripti võimalusel brauseris, mis võimaldasid väga palju dünaamilisemat 
lehte ehitada. Ei olnud mingisuguseid JavaScripti \emph{library}sid, nagu 
Reactid\index{React} ja muud, mis võimaldavad kõike teha. Sa kirjutasid puhast 
JavaScripti, isegi Github'i ega Stack Overflow'd. Oli Internet Exploreri 
dokumentatsioon.

Ja siis veel üks nõue oli, et kõik need Hoiupanga töötajad on harjunud 
terminaliga, kus  hiirt ei olnud. Hiire kasutamine aeglustab tegelikult tööd. 
Ja siis nõue oli see, et sa pead saama navigeerida rakenduses ilma hiireta. 
Brauseris. 

\textbf{\enquote{Põhimõtteliselt ju tehtav, aga kasutajaliidese disaini mõttes 
päris keeruline ülesanne}}

Võttes kõiki neid piiranguid, ma pidin välja tulema mingisuguse kliendipoolse 
raamistikuga. Ja noh, eks ma siis tulin. Seal tekkis päris palju koodi ja tol 
ajal tüüpilises brauseri rakenduses vajutad \emph{submit} nuppu ja siis sul 
terve leht laetakse uuesti. Ja meil ei olnud seda \emph{bandwidth}i kontorite 
vahel. Kujutan ette, et sul on viis tellerit ja  nad istuvad selle 28K 
modemi\sidenote{Sidet üle telefoniliinide (ja just seda peetakse siin silmas) 
reguleerisid \emph{International Telecommunication Union}'i V-seeria 
soovitused. V.34 kirjeldas sidet kuni 33.6 kbit/s, kuigi levinuim oli just siin 
mainitud 28.8 kbit/s kiirus.} peal ja igaüks iga nupuvajutusega sulle hakkab 
tulema mingisuguse sadades kilobaitides lehte. Tol ajal jällegi tulid 
\emph{frame}d ja \emph{frameset}id\sidenote{HTML 4.0, mis avaldati 1997. aastal 
W3C soovitusena, sisaldas eraldi variatsiooni \enquote{raamide} (ingl. 
\emph{frame}) toega. Raamid võimaldasid HTML lehe jagada eri aadressidelt 
laetavateks alamosadeks. HTML 5.0 enam raame ei toeta}, nende vahel sai andmeid 
vahetada brauseri sees. No ja oligi üks \enquote{menu} \emph{frame}, kus oli 
enamik JavaScripti loogikat, mida kunagi uuesti ei laetud, ja siis oli see 
\enquote{main} \emph{frame}, mille sees siis laeti iga konkreetne tegevus.

\textbf{\enquote{Seal tehti veel mingeid huvitavaid asju, olid peidetud raamid, 
mis käitusid nagu praegune brauserist algatatud REST päring}}

Eks see arenes. Rakenduses oli \enquote{main} \emph{frame},  siis 
kliendiandmete \emph{frame} sest tavaline \emph{workflow}  oli selline, et 
klient tuli, sa leidsid tema konto ja siis sa said selle kontoga teha makseid, 
teha hoiuseid, mis iganes. Alati oli see, et otsid klienti, siis me laeme 
kliendi andmed eraldi kliendi raami, kus on nähtavad kliendi nimi, konto nimi, 
kontonumber. Aga seal all oli veel brauseri poole peal kliendiandmed. Ja siis 
meil oli \enquote{foori} \emph{frame}. Selle kaudu me \emph{submit}isime vormi 
andmeid, sest jällegi valideerimine pidi toimuma koha peal. Nupp käivitas 
valideerimismeetodi  ja valideerimismeetodi lõpus andmed saadeti teise vormi 
kaudu serverisse. Ma ei mäleta, miks me nii tegime, ju see oli vajalik. Aga see 
oli nagu raam, mille sees kõik pangafunktsioonid said tehtud. Põhimõtteliselt 
mul läks mingi kuu aega, et see kõik niimoodi püsti panna ja esimene 
eestisisese makse vorm ära teha. Kui see oli valmis, siis põhimõtteliselt kõik 
ülejäänud funktsioonid tulid mingi kahe kuuga. Põhimõtteliselt 
\emph{copy-paste}, seal ei olnud enam midagi keerukat. Eks seal pärast vigade 
parandamist ja optimeerimist muidugi oli ka, aga ei midagi keerukat.

\textbf{\enquote{Ehk, põhiline arhitektuur sai õigesti paika ja see töötas. Kui 
sa nüüd tagasi mõtled, mis sulle andis põhja, et selline asi teha? Oli see 
ülikool või lihtsalt häkkerimentaliteet või veel midagi?}}

Ma arvan, et ei olnud mitte mitte midagi peale probleemi, mida oli vaja 
lahendada. Mingid muud nõudmised, mis olid nagu \emph{hard} nõudmised, me ei 
saanud neist üle ega ümber. Tol ajal me tegime hanza.net'i\index{Pangad!Hansapank!hanza.net} 
juba ja see oli värviline ja disaini mõiste oli juba olemas. Aga telleri 
rakenduse kohta oli spetsiifiline nõue et ta ei tohi väsitada inimest, et sa ei 
tohi kasutada erksaid värve, sest inimene teeb selle programmiga kaheksa tundi 
tööd. Ta oligi selline hall.

\textbf{\enquote{Sihukese asja peale isegi tänapäeval sageli ei mõelda, kust 
selline nõue tuli?}}

Meil oli ju tubli pangatehnoloogia osakond, kes mõtlesid, kuidas tellerid saavad 
hästi efektiivselt oma tööd teha. Ja jällegi ma ütlen, et mina olin ainult 
teostaja, seal oli terve tiim seal taga. Meil oli Toomas Rand\index[ppl]{Rand, 
Toomas}, kes tegelikult kirjutas kogu selle panga loogika, mina ju tegelesin 
ainult kasutajaliidesega ja andsin talle andmed. See, mis seal panga süsteemis 
toimus, oli tema teha et tema istus täpselt samamoodi kaksteist tundi päevas ja 
tegi. Aga tänu sellele projektile ka pangasüsteemi arhitektuuris tekkis 
korrastatus. Orcale Formsiga sa said kutsuda suvalisi funktsioone otse vormist. 
Aga meie arhitektuuri ütles, et üks nupuvajutus ja kogu tehing tehtud. Et see 
jällegi oli selline nõue pangasüsteemile. See õpetas, et liides ennekõike. 
Lepid liidese kokku ja siis osapooled saavad oma osaga  edasi tegeleda. See 
võimaldab sul testimist, testimise automatiseerimist, töö paralleliseerimist. 

Kitsendused tegelikult sunnivad inimesi tegema õigeid otsuseid. Ja me näeme 
järjest, et väga paljud inimesed ei oma kogemust sellistes piiratud  
ressurssidega olukorras toimetamisest. Eriti on seda näha välismaa inimeste 
puhul. Näiteks Silicon Valleyst inimene tuleb ja ta ei saa aru, et mis mõttes 
me ei palka juurde inimesi. Et ma ei saa ju kõiki oma ideid realiseerida, mis 
mõttes ma pean prioritiseerima? See on probleem nendele inimestele, nad ei saa 
aru, mis tähendab, et mul ei ole raha. Ma näen, et Eestis on tekkinud selline 
olukord, kus on palju ära tehtud hästi vähese ressursiga täpselt sellepärast, 
et inimesed oskavad teha õigeid valikuid, oskavaid prioritiseerima. Ja see on 
see oskus tuleb omakorda sellest, et sa pead alati prioritiseerime, sest sul ei 
ole ressurssi.

\textbf{\enquote{Ilusast arhitektuurist edasi minnes, milline on ilus kood?}}

Ilus kood on see, kus inimene ei pea küsima, mida see kood teeb. Väga 
paljud inimesed, kes oskavad programmeerida, millegipärast arvavad, et mida 
optimeeritum või lakoonilisem kood on, seda parem. Kuid see teeb halba. On piir, kust
edasi enam teine inimene ei saa aru, mida see kood teeb. Selline kood ei ole hea 
kood, isegi kui ta teeb õiget asja. See on üks asi. Aga teine asi on, et ma 
pean sulle suur aitäh ütlema selle eest, et sa tõid Eestisse Joshua 
Kerievsky\index[ppl]{Kerievsky, Joshua} omal ajal\sidenote{Joshua on USA firma 
Industrial Logic asutaja ja üks pikema kogemusega agiilse tarkvaraarenduse 
praktikuid ja koolitajaid maailmas. Tema Eestisse toomise Hansapanga arendajate 
koolitamiseks kas 2000. aasta lõpus või 2001. aasta algul algatas siiski Erik 
Jõgi\index[ppl]{Jõgi, Erik}}. Sul tekivad elus mingid hetked, kus sa saad aru, 
et see on nüüd \emph{step function}. Ja see koolitus (ta ei olnud pikk, nädal 
vist ja mitte täis päevad), mis me saime, lõi 
tegelikult väga paljud asjad oma kohtadele. Joshua on ju tegelenud koodi \emph{refactor}iga, 
 kuidas teha kehvast koodist ilusat. Ja me rääkisime temaga \emph{unit} 
testimisest \ldots.

See ongi, see, mis tegelikult aitab ilusat koodi kirjutada: sa 
pead seda mitu korda ümber kirjutama, enne kui ta loogiline välja näeb ja 
kood peab loogiline välja nägema.

\textbf{\enquote{Kui ma takkajärgi mõtlen, siis tolleks hetkeks kogu see 
agiilse arenduse liikumine, kogu see mõtteviis, oli veel väga noor}}

Me olime Skype'is\index{Skype}, ja siis ma tulin Pipedrive'i ja siin on meil 
igasugu \emph{agile coach}id. Ma mingi hetk ma mõtlesin, teeme eksperimendi. 
Oli mingisugune grupp. Seal olid \emph{agile coachid}, arendajad. Siis ma 
küsisin, et teeme eksperimenti. Rivistame grupi niimoodi, et kes on kõige kauem 
\emph{agile} liikumisega tegelenud  või vähemalt teadlik olnud. Ja enamasti, 
isegi \emph{coach}idel, oli see aeg mingi 7-8 aastat. Inimesed, ma olen 20 
aastat sellega tegelenud! \emph{Agile Manifesto} minu meelest oli 2001 või 
2002. Tegelikult me kõik saime seda maitsta enne, kui ta popiks muutus.

\textbf{\enquote{Mis sa praegu teed?}}

Ma isegi ma isegi ei saa öelda, et ma juhin \emph{engineering}u 
organisatsiooni, sest ma juhin ka muid organisatsioone nüüd 
Pipedrive'is\index{Pipedrive}. Ma olen siin juba seitse aastat olnud. Aastal 
2013 liitusin, see oli väike firma, ambitsioonikas. Tööintervjuul minult 
küsiti, et kas ma usun, et me saame Salesforce'iga võistelda. Ütlesin, et päris 
Salesforce'iks ei kasva, aga sihuke võib-olla veerand sellest on võimalik. Siis 
oli kakskümmend inimest, oli kümme inseneri. Ja nüüdseks, kuus ja natuke peale 
aastat hiljem, on meil on kuussada inimest.

Kõik need aastad olen tegelenud skaleerimisega. Nii infosüsteemi kui ka 
organisatsiooni skaleerimisega. Ja, jällegi, ei olnud kunagi sellist mõtet, et 
äkki meil ei õnnestu, äkki me ei kasva. Niipea, kui sa niimoodi hakkad mõtlema, 
sa ei kasva. Ma  siiamaani tegelikult ei ole kindel, et kumb on \emph{cause}, 
kumb on \emph{effect}. Et kas see, et me oleme skaleerinud \emph{engineering}ut 
aitas Pipedrive'il kasvada või see, et ta kasvas, aitas meil skaleerida 
\emph{engineering}ut.

Kui vaadata teisi osakondi, ütleme \emph{marketing} ei skaleerunud. 
\emph{Product} pidi skaleeruma koos \emph{engineering}uga, muidu inseneridel 
poleks midagi teha. \emph{Sales} ei skaleerunud, \emph{support} skaleerus 
nii-öelda natuke tagantjärgi. Et tegelikult \emph{engineering}u kasvatamine 
kasvatas firmat. Ma loodan. Aga samas, kui te ei kasvaks, siis me ei saaks  
inimesi juurde palgata, need kasvud on omavahel seotud. Aga küsimus ongi see, 
et mis tõukas seda. Ja ma väidan, et me nagu väga ei vaadanud. Me olime 
kindlad, et me peame skaleeruma, sest minu kõige suurem hirm on olnud, et  kui 
me jääme \emph{bottleneck}iks. Et kui \emph{engineering}u peale hakatakse 
näpuga näitama, et näed, me tahame või peame tegema seda ja toda 
\emph{engineering}ul ei ole ressurssi, või nad ei jõua või süsteemid hakkavad 
kokku kukkuma, kui kui kliente on liiga palju või me palkame inimesi juurde ja 
need inimesed ei saa tööd teha, sest kuskil protsessis on \emph{bottleneck}. 
Või me ei saagi inimesi palgata, sest inimesed ei taha meile tööle tulla. Neid 
pudelikaelu on nii palju, et ma pidi korraga kõikide nendega tegelemine. 

Kui kuidagi ei saa, siis kuidagi ikka saab.
