%!TEX TS-program = arara
% arara: myindex

\index[ppl]{Anikin, Sergei}
\question{Kuidas sina arvutite juurde sattusid?}

See oli päris huvitav lugu. Ma olin nii-öelda \emph{entitled}, mu isa oli
elektroonikainsener ja töötas Kalinini tehases\index{Kalinini
tehas}\sidenote{Algselt Balti Raudtee Peatehased, mis ehitati 1870. aastal ja
kandis aastatel 1902–1903 seal töötanud Nõukogude riigitegelase järgi 1940.
aastast alates M. I. Kalinini nime. 2007. aastast asub sellel
territooriumil ja osalt samades hoonetes Telliskivi Loomelinnak restoranide,
kohvikute, kontorite ja loomeruumidega}. Nüüd on seal kõige popim koht noorte seas, seesama Kalamaja ja Lendav Taldrik.
Lapsena käisin koos isaga tehases. Isa projekteeris rongidele
elektrimootoreid ja jõuelektroonikat. 
Hobi korras on ta teinud igasugust raadiotehnikat ja ma ise olen proovinud
väikest raadiot kokku panna, kuigi olin täielik võhik. Käisin küll raadiotehnikaringis. Minu esimese arvuti aga pani kokku isa.

\question{Kust ta vajalikud jupid sai?}

Isal oli selline Vene ajakiri nagu 
\begin{russian}Радио\end{russian}\index{Radio}\sidenote{Igakuine populaarteaduslik 
raadiotehnika ajakiri, mida andsid välja Nõukogude Liidu Siseministeerium ja 
DOSAAF (\begin{russian}Добровольное общество содействия армии, авиации и флоту 
России\end{russian} – vabatahtlik Vene armee, lennunduse ja mereväe 
abistamise selts). Ilmus eri nimede all alates 1925. aastast, 1975. aastal oli 
ajakirja tiraaž 850 000 eksemplari.}. Aastal 1986 avaldati seal 
kõigepealt arvutiskeemid ja siis kokkupanemise juhend. See oli 
Nõukogudemaal välja töötatud arvuti, aga skeemid võtsid nad ZX Spectrumi 
pealt\sidenote{\begin{russian}Радио-86РК\end{russian}\index{Radio-86RK}, populaarne
Nõukogude liidus loodud koduarvuti. Kuigi Nõukogudemaal 
kopeeriti ZX Spectrumit usinasti, oli see arvuti siiski väidetavasti 
originaalse disainiga, autoriteks Dmitri Gorškov, Juri Ozerov, Gennadi 
Zelenko ja Sergei Popov.}. Isa korjas komponendid kokku, 
joonistas ise plaadi, tegi tehases plaadi valmis ja 
pani arvuti kokku. Mäletan, et tal läks paar kuud, enne kui kõik 
vigased kohad ostsilloskoobiga välja juuris. Siis pani ta selle 
teleka külge, mis asendas monitori. See oli mustvalge telekas, 
värvitelekat meil ei olnud. Ega ma selle arvutiga midagi väga teha ei 
saanud, sel ei olnud isegi opsüsteemi. Oli küll \emph{interface} 
kassettmakiga, aga meil ei olnud kassette, mille pealt 
opsüsteemi laadida. Sellessamas ajakirjas oli trükitud baitkoodis opsüsteemi kood – 
kakskümmend lehekülge bait baidi haaval. Istusin kaks nädalat pimedatel talveõhtutel arvuti taga 
ja trükkisin kõik need koodid sisse.

\question{Miks sa seda tegid? Normaalne laps ju ei toksi niimoodi 
pimedatel õhtutel baitkoodi?}

Ka sellel on eellugu. Isa sõber tõi mulle umbes aasta varem lasteraamatu, kus tegelased õppisid programmeerima 
BASICus\index{BASIC}. Lugesin raamatu läbi, sain aru, kuidas 
programmi kirjutada, ja kirjutasin BASICus umbes kümnerealise programmi, mis 
midagi arvutas. Kuna aga paberi peal ei saanud ju
kompileerida, siis näitasin seda isa sõbrale, kes kontrollis ja  
ja ütles, et töötab küll.

Koodide sissetoksimine käis plokkide kaupa. Seal oli 
umbes poole leheküljeline plokk, millel oli kontrollkood. Sain seda 
valideerida ja kui see klappis, siis salvestasin makile. 
Kui ei klappinud, siis pidin viga otsima, mis oli väga 
keeruline. Ilmselt sellest ajast tekkis mul esiteks 
kannatus ja teiseks tähelepanu detailidele. Koode sisestades sain
lõpuks aru, et hästi oluline on need õigesti ja õiges 
järjekorras sisse toksida, sest ümbertegemine oli nii piinlik.


\question{Sulle tehti väiksest peale selgeks, et võid küll üle jala 
lasta, aga siis toksid ise neidsamu asju kolm korda.}

Jah, aga enamiku ajast veetsin loomulikult arvutiga mängides. Tollal 
oli olemas tavapärane maomäng ja ka tennis. Isale meeldis arvuteid kokku panna, sealhulgas sedasama 
ZX Spectrumi\index{ZX Spectrum}. Tegelesime ka 
selle väliskorpusega. Eestis on ju talviti kuiv õhk ja 
meil olid siis plastist õhuniisutajad, mis käisid radiaatori peale. Sellest sai väga 
hea korpuse arvutile: see oli õige kujuga ja selle sisse sai lõigata 
klaveri, toiteploki, plaadi ja kõik muu vajaliku. Makk oli eraldi.

\question{Miks sulle elektroonikaosa huvi ei pakkunud?}

Mul ei olnudki tegelikult arvuti vastu suurt kirge, siiamaani ei ole. Minu arust on see ikkagi vaid vahend. Tänapäeval on ju
teada, et arvutitega tegelejad teenivad päris korralikult raha. Tol ajal oli see ka mõnes mõttes staatuseküsimus, kui peres oli
arvuti. Kui paljudes peredes
üldse oli? Alles aastaid hiljem tekkisid arvutiklubid või
arvutimängukohad. Aga mul oli kodus olemas, kuigi me ei olnud
jõukas pere, kel oli raha niisugust asja osta. 

Arvuti on jah pigem vahend, ka meeldiv hobi, aga mitte ainuke. Mõnda aega ei tegelenud ma üldse arvutitega, 
mängimine enam kirge ei tekitanud ja programmeerida lihtsalt enda jaoks ei
tundunud väga huvitav. Aga mul oli üks sõber, kellega koos me mängisime. Tema
mainis: \enquote{Hoo, ma käin nüüd arvutiklubis. Me õpime seal programmeerima,
kuigi mina käin muidugi enamasti mängimas}. Siis ma mõtlesin, et tema ju
tegelikult ei oskagi midagi, aga mina küll, ja et peaksin koos temaga minema. Sa
ilmselt oled rääkinud paljude inimestega Eesti kogukonnast, aga mina sattusin
siis Vene kogukonda. Selle arvutiklubi nimi oli Interface\index{Interface}.

\question{Kes seda klubi pidas ja kus?}

Seda vedas Nina Botina\index[ppl]{Botina, Nina}, kes töötas vist bioloogiainstituudis Mustamäe teel. Me käisime 
Reaalkoolis\index{Tallinna 2. Keskkool}\index{Reaalkool|see{Tallinna 2. Keskkool}} tundides,
seal olid arvutiklassid.

\question{Mis koolis sa ise käisid?}

Koolis number kakskümmend kuus\index{Tallinna 26. Keskkool}. 
Viimasesse klassi läksin Tõnismäe Reaalkoolis\index{Tõnismäe Reaalkool} 
kus oli väga tugev matemaatika. Tegelikult seesama Nina Botina õhutas mind ja 
veel ühte klassiõde teise kooli minema ja 
matemaatikaklassi lõpetama. Tema pärast läksimegi sinna, seal oli hästi palju  
tuttavaid arvutiklubist.

Hiljem kasvas sellest arvutiklubist venekeelne tehnikakool või
arvutitehnikakool, mis asus Erika tänaval. 

\question{Ma teadsin, et Tartu ja Tallinna vahel on erinevus. Aga 
selgub, et ka Tallinna sees on kaks täiesti isesugust Tallinna.}

See on huvitav jah. Kusjuures minu huvi arvutite vastu 
vaheldus. Ühe aasta olin klubis, aga uude kooli minnes ei olnud mul 
selleks aega. Siis kutsus Nina mind appi, arvutiklassi instruktoriks, ja see 
tekitas uuesti huvi. Kui ma lõpetasin kooli ja läksin
ülikooli majandust õppima\index{Tallinna 
Tehnikaülikool!Majandusteaduskond}, tekkis seal esimese aasta lõpus 
võimalus spetsialiseeruda majanduslikule andmetöötlusele. Meil oli pisike 
grupp, seitse inimest. Kui kõik, kes olid majanduses, õppisid majandusaineid, siis enamik meie tunde olid arvutitehnika gruppidega.

Ma läksin küll venekeelsesse majandusteaduskonda, 
aga grupp oli eestikeelne. Huvitaval kombel ei pidanud me õppima arvutitehnika baasaineid. 
Esimese aasta arvutitehnikas õpiti nimelt füüsikat-keemiat, kõiki üsna 
keerulisi aineid. Ma olen kuulnud õudseid lugusid, kuidas inimesed ei saanud ülikooli 
lõpuni neid tehtud. Aga meie õppisime mikro- ja makroökonoomikat ning 
inglise keelt. Alates teises aastast hakkasime niisiis koos arvutitehnika omadega õppima ja
erilist jõudluse vahet ei olnud.

See, kus ma praegu olen, on ilmselt  
põhjustatud ka sellest, et ma ei läinud väga süvitsi arvutitehnikasse, vaid pigem 
oli arvuti alati vahend mõne probleemi lahendamiseks.

\question{Sa mainisid, et matemaatika tuli sul hästi välja. Kas käisid 
olümpiaadidel ka?}

Käisin, aga ma olin keskmiste seas. See sõltub palju õpetajast. 
Mäletan, et olin kas viiendas või seitsmendas klassis\sidenote{Selle põlvkonna inimestel jäi 
nii vene kui ka eesti koolides üks klass vahele, sest koolid läksid 
kaheksakümnendate teisel poolel üle aasta võrra pikemale õppele}, kui hakkasid geomeetria ja muud sellised ained. Ja siis mul 
klikkis, et iga teoreemi kohta, mida meile räägiti, tekkis mul teine 
viis, kuidas seda tõestada. Ma sain aru, et asjad ei ole alati ainult 
ühtemoodi, saab ka teisiti. See omakorda klikib õpetajaga: kui 
õpetaja näeb, et õpilane mõtleb, siis pöörab talle rohkem tähelepanu. Paraku läks see õpetaja ära ja järgmised ei olnud nii head.

Meil oli üks väga hea füüsikaõpetaja, kes tegi palju kontrolltöid. 
Tema juures õppisin seda, et valemeid ei pea üldse meelde jätma. Piisab, kui oskad neid rakendada. Loomulikult ei olnud spikerdamine 
lubatud, aga mul olid valemid ikkagi spikrina vihiku tagakaanel. Sa pead 
aru saama probleemist ja vahenditest, mida selle 
lahendamiseks kasutada. See õpetaja vaatas valemite teadmisele läbi sõrmede, sest 
kui probleemist aru ei saa, siis lihtsalt valemid füüsikas ei aita. 

Tõnismäe Reaalkoolis oli legendaarne 
matemaatikaõpetaja Mihhail Vassiljevitš\index[ppl]{Vassiljevitš, Mihhail}, kes õpetab seal
siiani. See inimene on tõeline autoriteet, kohtleb 
õpilasi ühtemoodi! Meie matemaatikaklassis oli kolm-neli tippõpilast, kes 
võitsid kõik riiklikud olümpiaadid ja käisid ka maailmaolümpiaadidel. Loomulikult ta 
tegeles nendega, aga ka kogu ülejäänud rahvaga. Oli neidki, 
kes ei saanud matemaatikast väga aru, aga tema juures nende tase tõusis. Ta oskas 
selgitada ka keerulisi asju nii lihtsalt, et kogu klass 
oli paar taset teistest koolidest üle. Ainuüksi selles  
keskkonnas olemine tõstis taset nii kõvasti.


\question{Jällegi tuleb välja, et matemaatikatunnis õpiti lisaks 
matemaatikale suhtumist, ja just see on sul aastate järel meeles.}

Mina ei saanud olümpiaadidel küll mingeid kohti, aga 
meist aasta vanemas klassis oli selline lugu, et umbes kümme inimest läksid keemia-, kümme 
matemaatika- ja kümme füüsikaolümpiaadile. Põhimõtteliselt terve klass osales Tartus riiklikel
olümpiaadidel, aga erinevatel aladel. Ja kuna nad olid juba seal kohal, siis 
neil oli lubatud ka teiste ainete olümpiaadidest osa võtta. Selle tulemusel said enam-vähem kõik, isegi need, kes algul ei kvalifitseerunud,
kõikidel aladel esikümnesse. Hämmastavalt võimas klass!

\question{Miks sa läksid majandust õppima?}

Sest mu vanemad ütlesid, et meil on peres juba kaks inseneri olemas, ema oli 
soojustehnik. Eks ma mõtlesin ka muid variante, aga kodu juures 
oli palju lihtsam. 

\question{Kas sul oli mingi ettekujutus ka sellest, mida sa tahad pärast oma 
haridusega ette võtta?}

Erilist ettekujutust ega plaani mul ei olnud. Tahtsin lihtsalt näha, mis see majandus 
õigupoolest on. Ühel suvel proovisin töötamist müügiinimesena ja selgus, et 
see ei sobi mulle absoluutselt. Müügitöös ütleb 
üheksakümmend kaheksa protsenti inimestest \enquote{ei}, aga mulle ei 
meeldi feilida ja minu jaoks oli \enquote{ei} tol ajal feil. 
Tegelikult nüüd, kui olen Pipedrive'is\index{Pipedrive} juba seitse 
aastat töötanud, saan aru, et see on osa protsessist, statistika. Feil on see, kui sa ei tee seda üheksakümne üheksandat 
müüki, mis võib õnnestuda. Müük on see, et tead neid statistilisi 
numbreid ja plaanid vastavalt nendele. See ei ole feilimine, kui esimene juhuslik 
inimene ütleb, et tal ei ole seda teenust vaja.

\question{Mida sa müüsid?}

See oli tänavamüük, müüsime erinevaid tooteid, näiteks 
tööriistakaste, mis läksid päris hästi, 
elektroonilisi hambaharju ja nii edasi.

\question{See on ju igavesti raske töö!}

See oli väga raske töö. Tulime igal hommikul lattu ja saime päevakvoodi, näiteks tuli müüa viisteist 
tööriistakasti. Kui täitsid kvoodi kahe nädala jooksul, siis 
said järgmise tiitli ja koos sellega endale õpilasi. Ja kui viis 
õpilast said omakorda kvoodi täidetud, siis said 
nii-öelda enda äri. Mina sain õppetunni, et see töö ei ole kindlasti minu 
jaoks. Teadsin, et kui lähen programmeerijaks, saan oluliselt 
rahulikuma töö eest oluliselt suuremat tasu. See sundiski mind umbes 
pool aastat hiljem ütlema: \enquote{Okei, ma lähen.} Ja nii ma läksingi ülikooli 
teise aasta keskel informaatikagruppi.

\question{Kas sa siis programmeerisid juba tõsisemaid asju ka 
või puutusid nendega ainult loengus kokku?}

Tegin kahte projekti, mis tõid natuke raha sisse.

Tol ajal olid hästi populaarsed 
SAT-TV\sidenote{Kaheksakümnendate lõpus ja üheksakümnendatel oli isiklik satelliidivastuvõtja ületamatult 
kallis, piraatlusele vaadati läbi sõrmede, suuri teenusepakkujaid veel polnud, aga väikestel oli juba 
võimalus tegutseda. Siis pandigi mõne kortermaja katusele satelliiditaldrik, 
hangiti piraatkaart tasuliste kanalite jaoks, veeti üle katuste 
ümberkaudsetesse majadesse kaablid ja asuti teenust müüma} firmad. Mõnes 
väikeses rajoonis oli oma kunn, kes pakkus SAT-TVd kuutasu eest. 
Mul oli üks tuttav, kes palus teha infosüsteemi, kus oleks kirjas, kes on 
liitunud, kes ei ole, kui palju nad maksavad ja mis teenust kasutavad. 
Emal oli tööl arvuti, millega sain teha Accessi\index{Microsoft 
Access} andmebaasi ja selle peale väikese liidese.

Teine projekt oli veel huvitavam. Kui sain teada, kui palju raha ma selle töö eest 
saan, olin väga imestunud. Isa sõbrad tegelesid valvesüsteemidega ja neil oli 
üks vanglaprojekt, valvesüsteemi panemine vanglasse. Neil oli 
tarvis joonistada vangla projekti järgi skeem, 
kus oleks näha, kus on alarmid tööle läinud. See ei olnud otseselt 
programmeerimine, rohkem disain. Mina pidingi selle skeemi
joonistama, sain selle kolme nädalaga tehtud ja tasuks
umbes isa poole aasta palga. Siis sain aru, et 
arvutitega tasub toimetada.

\question{Kust sa infot said? Accessis programmeerimine ei ole 
niisama lihtne, et hakkad muudkui otsast tegema.}

Accessi kohta ma ei mäletagi, eks vist lugesin dokumentatsiooni. 
Programmeerimist õppisin 
raamatutest. Mul oli üks venekeelne Pascali raamat, mis õpetas objektorienteeritud 
programmeerimist. Ka ülikoolis olid mõned ained väga-väga 
kasulikud, näiteks andmebaaside projekteerimine. Tänapäeval paljud 
inimesed ei oska relatsioonilist andmebaasi projekteerida, aga see on üks 
vajalikumaid oskusi, kui tahad kasvõi lihtsat süsteemi kokku 
panna. Tänapäeval lahendatakse selliseid asju tihti jõuga.

\question{Kas sinu reaalainete ja arvutihuvi juurde käis ka 
mõne spetsiifiline, näiteks ulme- või raamatuhuvi? Vene keeles oli ju palju 
rohkem asju kättesaadavad, mina ei olnud suuteline tol ajal Strugatskeid 
originaalis lugema.}

Ei mäleta, et oleks väga olnud. Raamatuid lugeda mulle meeldis, samuti
ulme või fantastika. Aga arvutite suhtes ei tekkinud mul tugevat
tunnet, minu jaoks oli arvuti nii praktiline asi, kui olla saab. Lugesin
Bulõtšovit\sidenote[][-3cm]{Kir Bulõt\v{s}ov (1934--2003), Nõukogude 
ulmekirjanik} ja Strugatskeid\sidenote[][-1.5cm]{Arkadi Strugatski (1925--1991)  ja Boris Strugatski (1933–2012). Nõukogude ulmekirjanikud, kes kirjutasid enamasti koos, seega tuntud kui \begin{russian}братья Стругацкие\end{russian} või lihtsalt Strugatskid}, aga ka 
välismaa asju. Olen ka kõik Barbar Conani\sidenote{Robert E. 
Howardi (1906--1936) 1932. aastal loodud tegelane, kes on tembutanud 
kõikvõimalikes meediumides ajakirjadest ja raamatutest filmide ja 
videomängudeni} ja Tarzani\sidenote{Edgar Rice Burroughsi (1875--1950) 1912. aastal 
loodud tegelane, kes sai Nõukogude Liidus tuntuks kinodes näidatud 
trofeefilmidest (Johnny Weissmulleri kehastatud tegelane erines küll oluliselt 
raamatukangelasest).} raamatud läbi lugenud.

\question{Mis su esimene päris programmeerijatöö oli ja millal?}

Veebruaris 1996 läksin tööle Aeteci
Finantsvara ASi\index{Aeteci Finantsvara AS|see{Profit Software}}, mis nüüdseks 
on Profit Software\index{Profit Software}. Mul olid seal sõbrad ees. Nad tegid soomlastele 
finantskindlustussüsteeme. Oma esimese tööülesandega ma 
feilisin, sest mulle anti mingisuguse
valemi programmeerimine. See pidi Cs\index{C} olema ja sellest pidi 
\emph{library} saama. Ma ei teadnud, kuidas Csi 
kirjutada, ma ei saanud sellest valemist aru (see oli kõrgem matemaatika). 
Ühesõnaga, sellega ma feilisin. 

See-eest olin väga hea Lotus 
Notesi tarkvaras, mida kasutati suhtlemiseks omavahel 
ja soomlastega. See oli dokumendiandmebaas, millel oli oma 
skriptimiskeel. Sellega ma kirjutasin reisikindlustuse süsteemi 
kindlustusagentidele, et nad saaksid välja arvutada, palju reisimine maksab, 
ja poliisi teha. Ja see oli internetipõhine aastal 1997. Dominoga 
oli võimalik samu dokumente, mida muidu nägi Lotus Notesis kliendi 
kohta, ka veebiserveri kaudu ehk HTML-dokumentidena näidata.

See kogemus aitas mul saada Hansapanga\index{Hansapank} internetipanga 
tiimi.

\question{Kuidas sa sinna sattusid?}

Hansapanga ITs või üldse pankades ilmselgelt 
oli rohkem raha kui mõnes IT-firmas. Kui olin kaks aastat Aeteci
Finantsvaras töötanud, tundsin, et võiks nii-öelda karjääri teha. Proovisin tegelikult 
kõikidesse pankadesse tööle saada, igal pool oli vabu kohti. 
SEBs\index{SEB|see{Ühispank}} ehk toonases Ühispangas\index{Ühispank} ma ei 
saanud isegi vist jutule, aga Hoiupangas\index{Hoiupank} rääkisin Aleksei 
Bljahhiniga\index[ppl]{Bljahhin, Aleksei}. Hansas oli ka tööintervjuu, läksime sinna koos 
Vilve Vene\index[ppl]{Vene, Vilve} ja Heiki Kübbariga\index[ppl]{Kübbar, 
Heiki}. Ja sain mõlemast pangast tööpakkumise umbes sama summa peale. 
Otsustasin Hansapanga kasuks, sest arvasin, et seal võib olla natuke rohkem 
karjäärivõimalusi. 

Minu esimene tööpäev Hansapangas oli 
19. jaanuaril 1998. Fuajeesse astudes märkasin värsket 
Äripäeva, kus oli kirjas, et Hoiupank ja Hansapank ühinevad. Nii et minu 
esimesel tööpäeval teatati ühinemisest ja see määras kogu mu järgneva karjääri.


\question{See tähendab, et pidid suhteliselt ruttu hakkama 
internetipanga asemel tegelema hoopis Light Telleri\label{sisu:teller} nimelise telleri 
töökohasüsteemiga?}

Sinna läks veel natuke aega. Otsus hakata seda tegema sündis 
umbes viis-kuus kuud peale seda, kui ühinemine pihta hakkas. Alguses  
ei olnud ju veel selge, kumba süsteemi üldse hakatakse kasutama ja kuidas see 
otsus tehakse. Sellel ajal õppisin mina, kuidas internetipanka teha.

\question{See kõik on mulle üllatus. Mina läksin sinna panka 
1999. aasta lõpus. Light Teller oli selleks ajaks olemas ja laua taga oli 
vana kala nimega Sergei, kes oli selle oma käega valmis teinud. Kui nüüd 
näppudel arvutada, siis järelikult tegid sa nullist 
täisfunktsionaalse veebipõhise telleri töökoha umbes kolme kuuga?}

Ega ma seda üksi teinud. Aga astume sammu tagasi. Hansapanga esimene 
internetipank oli üles ehitatud tehnoloogiale, mis oli ajast ees. See 
oli Oracle'i\index{Oracle} veebikomponent või -server, kus 
sai PL/SQLiga\index{PL/SQL} tekitada HTMLi, mida kliendid 
vaatasid. See oli omal ajal hästi lihtne, ilma igasuguse disainita, sest 
disaineritest ei teadnud tol ajal vist keegi, et on olemas selline amet nagu disainer. 
Trükidisainerid kindlasti olid, aga kasutajaliidese disaineritest polnud keegi kuulnud. 

Mina mõtlesin, et oo, milline 
ebavõrdsus, et internetipank on ainult eesti keeles. Ütlesin, et ma võin teha selle 
mitmekeelseks. Seepeale öeldi, et tee. Ja tegingi. Kaks nädalat tegelesin 
sellega, et võtsin kõik tekstid välja ja asendasin \verb|lang|-funktsiooniga, mis 
arvestas ka kasutajaprofiiliga. Samal ajal õppisin veel ülikoolis, olin sel
päeval, kui Madis Ollisaar\index[ppl]{Ollisaar, Madis} asja tootmisse pani, koolis. 
Logisin sisse, et vaadata, kas töötab. Eesti keel töötas, inglise keel 
töötas, vene keel aga näitas küsimärke. Ilmselt inimesed mässavad siiamaani nende 
\emph{encoding}'utega, aga see oli minu esimene kokkupuude sellega, et minu 
arvutis töötab, aga serveris mitte.

Samal ajal hakkas ühinemise tõttu juhtuma mitu asja korraga.  
Aleksei Bljahhin\index[ppl]{Bljahhin, Aleksei} tegeles \emph{data} migraga. 
Tekkis probleem, kuna telleriprogramm oli kirjutatud Oracle Formsis ja igas 
kontoris oli Formsi server. Kõik tellerid kasutasid Formsi klienti, mida 
serveeriti serverist, ja nad võtsid peaserveriga Oracle'i 
andmebaasiühenduse. Oracle'i litsentside eest maksti teatavasti ühenduste arvu 
pealt. Hansapangal oli tol hetkel, no ma ei tea, mingi nelikümmend 
kontorit. Nüüdseks see on juba suur number, aga Hoiupangal oli nelisada 
kontorit!. Paljudes maakohtades ei olnud isegi nii head sidet, et 
hoida pidevat ühendust andmebaasiga. Kui nad arvutasid, kui palju 
Oracle'i litsentsid oleksid kokku maksnud, siis nad ütlesid, et võib-olla anname 
Hoiupanga tagasi. 

Tegelikult tehti väga julge otsus teha interneti telleriprogrammi. Otsustajateks olid ilmselt needsamad Vilve\index[ppl]{Vene, Vilve} ja 
Gibbs\index[ppl]{Gibbs|see{Kübbar, Heiki}}\sidenote{Sergei peab silmas Heiki Kübbarat, kes oli toona paljude Hansapanga innovatiivsete ideede taga.}. Samal ajal müüdi meile internetipanga tegemiseks uus tehnoloogia, BroadVisioni\index{BroadVision} platvorm. 
BroadVisioni müügiargumendiks oli, et saame põhimõtteliselt e-kommertsi 
platvormi, millel sai igale kasutajale näidata personaalselt välja nägevat 
rakendust.
Samas iga kasutaja maksis, mis tähendas, et me ei kasutanudki kunagi seda
võimalust, süsteemi mõttes oli kõik anonüümne. Ühtlasi pakkus BroadVision
\emph{template}'imise võimalust, mis oli väga suur samm edasi võrreldes Oracle'i 
PL/SQLiga, kus tuli oma HTML ise kokku panna. Nii et selle peale me internetipanga ehitasimegi. 
Tagantjärele mõeldes oli see telleri arhitektuur lihtne, aga võimas. See võimaldas 
kiiresti ja suures koguses funktsionaalsust toota.

\question{See arhitektuur oli siis toonaseid vahendeid kasutades täpselt selline, nagu tänased \emph{de facto} 
veebirakendused on. JavaScript\index{JavaScript} jooksis brauseris ja 
tegi \emph{backend}'i poole päringuid. See lahendus oli 20 aastat ajast ees, 
kuidas see sündis?}

Meil tuli arvestada piirangut, et maakontorite ühendus oli väga aeglane. 
Pidime optimeerima, kui palju \emph{data}'t kliendi ja serveri 
vahel liigutada. See sundiski palju tööd juba 
kliendipoolel ära tegema. Kliendiks oli brauser ja JavaScripti versioon oli 
selline, et parimal juhul sai teha valideerimist. Midagi joonistada või dünaamiliseks teha eriti
ei saanud. Samal ajal tuli 
Internet Explorer 4.0\index{Internet Explorer}, kus olid \emph{custom} 
JavaScripti võimalused, mis lasid palju dünaamilisemat 
lehte ehitada. Tol ajal ei olnud ju mingisuguseid JavaScripti \emph{library}'sid, nagu 
Reactid\index{React} ja muud, mis võimaldavad kõike teha. Sa kirjutasid puhast 
JavaScripti, isegi Githubi ega Stack Overflow'd ei olnud. Oli Internet Exploreri 
dokumentatsioon.

Ja kuna kõik Hoiupanga töötajad olid harjunud ilma hiireta
terminaliga (hiire kasutamine aeglustab tööd), siis oli ka 
nõue, et kasutaja pidi saama navigeerida brauserirakenduses ilma hiireta. 

\question{Põhimõtteliselt ju tehtav, aga kasutajaliidese disaini mõttes 
päris keeruline ülesanne.}

Arvestades kõiki neid piiranguid pidin välja tulema mingisuguse kliendipoolse 
raamistikuga ja tulin ka. Seal tekkis päris palju koodi ja tol 
ajal tuli tüüpilises brauserirakenduses vajutada \emph{submit}-nuppu, mispeale 
terve leht laeti uuesti. Meil aga ei olnud kontorite vahel \emph{bandwidth}'i. Näiteks kui viis tellerit, kes istusid 28 K 
modemi\sidenote{Sidet üle telefoniliinide 
reguleerisid Rahvusvahelise Telekommunikatsiooni Liidu V seeria 
soovitused. V.34 kirjeldas sidet kuni 33,6 kbit/s, kuigi levinuim oli 
mainitud 28,8 kbit/s kiirus.} peal, vajutas nuppe, hakkas iga nupuvajutusega 
tulema sadades kilobaitides lehte. Tollal tekkisid 
\emph{frame}'id ja \emph{frameset}'id\sidenote{HTML 4.0, mis avaldati 1997. aastal 
W3C soovitusena, sisaldas eraldi variatsiooni \enquote{raamide} (ingl 
\emph{frame}) toega. Raamid võimaldasid jagada HTML-lehe eri aadressidelt 
laetavateks alamosadeks. HTML 5.0 enam raame ei toeta.}, mille vahel sai andmeid 
vahetada brauseri sees. Nii et oligi üks \enquote{menu} \emph{frame}, kus oli 
enamik JavaScripti loogikat, mida kunagi uuesti ei laetud, ja 
\enquote{main} \emph{frame}, mille sees laeti iga konkreetne tegevus.

\question{Seal tehti veel mõningaid huvitavaid asju, näiteks olid peidetud raamid, 
mis käitusid nagu praegune brauserist algatatud REST päring.}

Eks see arenes. Rakenduses oli \enquote{main} \emph{frame} ja 
kliendiandmete \emph{frame}, sest tavaline \emph{workflow} oli selline, et kui 
klient tuli, siis leidsid tema konto ja said seal teha makseid, 
hoiuseid ja mida iganes. Klienti otsides tuli laadida 
tema andmed eraldi kliendiraami, kus olid nähtavad kliendi nimi, konto nimi ja 
kontonumber, aga seal all olid veel ka brauseripoole peal kliendiandmed. Ja siis 
meil oli \enquote{foori} \emph{frame}, mille kaudu \emph{submit}'isime vormi 
andmeid, sest valideerimine pidi jällegi toimuma kohapeal. Nupp käivitas 
valideerimismeetodi, mille tulemusel saadeti andmed teise vormi 
kaudu serverisse. Ma ei mäleta, miks me nii tegime, ju oli vaja. Aga see 
oli nagu raam, mille sees said kõik pangafunktsioonid tehtud. Selle püstipanekuks ja esimese 
Eesti-sisese maksevormi tegemiseks kulus kuu aega. Kui see sai valmis, siis kõik 
ülejäänud funktsioonid tulid kahe kuuga. Põhimõtteliselt 
\emph{copy-paste}, midagi keerulist ei olnud, ainult pärast pisut vigade 
parandamist ja optimeerimist.

\question{Kui sa nüüd tagasi mõtled, siis mis sulle andis põhja, et selline asi teha? Oli see 
ülikool, lihtsalt häkkerimentaliteet või veel midagi?}

Ei olnud mitte midagi peale probleemi, mida oli vaja 
lahendada. Muidugi oli sealjuures ka muid nõudmisi, millest ei 
saanud üle ega ümber. Näiteks tellerirakenduse puhul oli spetsiifiline nõue, et see ei tohi inimest väsitada, st me
ei tohtinud kasutada erksaid värve, sest selle programmiga tehti päevas kaheksa tundi 
tööd. Sellepärast see saigi hall. Tol ajal me tegime ka hanza.net'i\index{Hansapank!hanza.net} 
ja see oli värviline, disaini mõiste oli juba olemas.

\question{Sellise asja peale tänapäeval sageli isegi ei mõelda, kust 
see nõue tuli?}

Meil oli tubli pangatehnoloogia osakond, kes mõtles, kuidas tellerid saaksid oma tööd teha
hästi efektiivselt. Kordan, et mina olin ainult 
teostaja, asja taga oli terve tiim. Meil oli Toomas Rand\index[ppl]{Rand, 
Toomas}, kes kirjutas kogu pangaloogika; mina tegelesin 
ainult kasutajaliidesega ja andsin talle andmed. Pangasüsteemis 
toimuv oli tema teha ja ta istus täpselt samamoodi kaksteist tundi päevas töölaua taga ja 
tegi. Tänu sellele projektile tekkis pangasüsteemi arhitektuuris 
korrastatus. Oracle Formsiga sai kutsuda suvalisi funktsioone otse vormist, seevastu kui 
meie arhitektuuri ütles, et üks nupuvajutus ja ongi kogu tehing tehtud. Ennekõike tuli kokku leppida liideses 
ja siis said osapooled oma osaga edasi tegeleda. See 
võimaldas testimist, testimise automatiseerimist ja töö paralleliseerimist. 

Kitsendused sunnivad tegelikult tegema õigeid otsuseid. Paljudel inimestel ei ole piiratud  
ressurssidega toimetamise kogemust, eriti välismaalastel. Näiteks tuleb Silicon Valleyst inimene, kes ei saa aru, miks
me ei palka inimesi juurde. Mis mõttes ei saa kõiki oma ideid realiseerida, vaid
peab prioritiseerima? See on tema jaoks probleem, kuna ta ei saa 
aru, mis tähendab, et raha ei ole. Näen, et Eestis 
on palju ära tehtud väga vähese ressursiga täpselt selle pärast, 
et inimesed oskavad teha õigeid valikuid. Prioritiseerima peab, sest ressurssi ei 
ole.

\question{Kui ilusast arhitektuurist edasi minna, siis milline on ilus kood?}

Ilus on kood siis, kus inimene ei pea küsima, mida see teeb. Väga 
paljud, kes oskavad programmeerida, arvavad millegipärast, et mida 
optimeeritum või lakoonilisem kood on, seda parem, kuid see teeb halba. On piir, kust
edasi teine inimene ei saa enam aru, mida kood teeb. Selline kood ei ole hea, isegi kui teeb õiget asja. See on üks asi. Teiseks pean ma 
ütlema sulle suur aitäh selle eest, et tõid omal ajal Eestisse Joshua 
Kerievsky\index[ppl]{Kerievsky, Joshua}\sidenote{Joshua Kerievsky on USA firma 
Industrial Logic asutaja ja üks pikema kogemusega agiilse tarkvaraarenduse 
praktikuid ja koolitajaid maailmas. Tema Eestisse toomise Hansapanga arendajate 
koolitamiseks kas 2000. aasta lõpus või 2001. aasta algul algatas siiski Erik 
Jõgi\index[ppl]{Jõgi, Erik}.}. Elus tekivad hetked, kui saad aru, 
et see on nüüd \emph{step function}. Tema koolitus viis
väga paljud asjad oma kohale. Joshua on tegelenud koodi \emph{refactor}'iga ehk kuidas teha kehvast koodist ilusat. Samuti rääkisime temaga \emph{unit}'i 
testimisest \ldots

See aitabki ilusat koodi kirjutada: sa 
pead seda mitu korda ümber kirjutama, enne kui see näeb loogiline välja.

\question{Tagantjärele mõeldes oli kogu see 
agiilse arenduse liikumine ja mõtteviis tol ajal veel väga noor.}

Kui ma tulin Skype'ist\index{Skype} Pipedrive'i, siis siin on meil 
igasugu \emph{agile coach}'e. Ma korraldasin sellise eksperimendi, et rivistasime oma 
\emph{agile coach}'id, arendajad selle järgi, kes on 
\emph{agile} liikumisega kõige kauem tegelenud või sellest vähemalt teadlik olnud. Enamiku puhul oli see aeg 7-8 aastat. Mina olen sellega 20 
aastat tegelenud! \emph{Agile Manifesto}\sidenote{Vt. \url{https://agilemanifesto.org/}} tekkis vist 2001. või 
2002. aastal. Tegelikult me kõik saime seda maitsta enne, kui see popiks muutus.

\question{Mida sa praegu teed?}

Ma isegi ei saa öelda, et juhin \emph{engineering}'u 
organisatsiooni, sest ma juhin ka muid organisatsioone. Olen 
Pipedrive'is\index{Pipedrive} juba seitse aastat olnud. Aastal 
2013 meeskonnaga liitudes oli see väike ja ambitsioonikas firma. Tööintervjuul küsiti minult, kas ma usun, et suudame Salesforce'iga võistelda. Ütlesin, et päris 
Salesforce'iks me ei kasva, aga võib-olla veerand sellest on võimalik. Siis 
oli meil kakskümmend inimest, kümme inseneri. Nüüdseks, kuus aastat hiljem ja natuke peale, on meid kuussada.

Kõik need aastad olen tegelenud skaleerimisega: nii infosüsteemi kui ka 
organisatsiooni skaleerimisega. Selle aja jooksul ei ole kordagi tekkinud mõtet, et 
äkki meil ei õnnestu, äkki me ei kasva. Niipea kui hakkad niimoodi mõtlema, siis 
ei kasvagi. Ma ei ole tegelikult siiamaani kindel, kumb on põhjus ja 
kumb tagajärg: kas see, et oleme skaleerinud \emph{engineering}'ut, 
aitas Pipedrive'il kasvada või see, et ta kasvas, aitas meil skaleerida 
\emph{engineering}'ut.

Kui vaadata teisi osakondi, siis näiteks turundus ei skaleerunud. 
\emph{Product} pidi skaleeruma koos \emph{engineering}uga, muidu inseneridel 
poleks midagi teha. Müük ei skaleerunud, \emph{support} skaleerus 
nii-öelda tagantjärele. Tegelikult \emph{engineering}'u kasvatamine 
kasvatas firmat. Samas, kui ettevõte ei kasvaks, siis ei saaks ju ka 
inimesi juurde palgata. Küsimus on, 
mis tõukas kasvu tagant. Me eriti ei mõelnud sellele, vaid olime 
kindlad, et peame skaleeruma. Minu kõige suurem hirm on olnud jääda pudelikaelaks. See, et \emph{engineering}'u peale hakatakse 
näpuga näitama, et tahaks küll seda või toda teha, aga 
\emph{engineering}'ul ei ole ressurssi või süsteemid hakkavad 
kokku kukkuma, kui kliente on liiga palju. Või et palkame inimesi juurde ja 
nad ei saa oma tööd teha, sest kuskil protsessis on pudelikael. 
Või me ei saagi inimesi palgata, sest nad ei taha meile tööle tulla. Neid 
pudelikaelu, millega korraga tegeleda, on olnud palju. Aga kui kuidagi ei saa, siis kuidagi ikka saab!
