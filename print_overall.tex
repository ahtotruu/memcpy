%!TEX TS-program = arara
% arara: latexmk: { clean: partial }
% arara: xelatex: { shell: true, synctex: true} 
% arara: makeindex
% arara: xelatex: { shell: true, synctex: true} 
% arara: xelatex: { shell: true, synctex: true} 
% arara: xelatex: { shell: true, synctex: true}
% arara: latexmk: { clean: partial }

\documentclass[a4paper, twoside]{tufte-book}
\usepackage[
    type={CC},
    modifier={by-nc-nd},
    version={4.0},
]{doclicense}


\ifxetex
  \newcommand{\textls}[2][5]{%
    \begingroup\addfontfeatures{LetterSpace=#1}#2\endgroup
  }
  \renewcommand{\allcapsspacing}[1]{\textls[15]{#1}}
  \renewcommand{\smallcapsspacing}[1]{\textls[10]{#1}}
  \renewcommand{\allcaps}[1]{\textls[15]{\MakeTextUppercase{#1}}}
  \renewcommand{\smallcaps}[1]{\smallcapsspacing{\scshape\MakeTextLowercase{#1}}}
  \renewcommand{\textsc}[1]{\smallcapsspacing{\textsmallcaps{#1}}}
\fi


\usepackage[T1]{fontenc}
%\usepackage[utf8]{inputenc}
\usepackage{polyglossia}
\setmainlanguage{estonian} 
\setotherlanguage{russian}
\newfontfamily\russianfont[Script=Cyrillic]{Linux Libertine}

\hypersetup{colorlinks}% uncomment this line if you prefer colored hyperlinks (e.g., for onscreen viewing)


%%
% Book metadata
%\title{print(memcpy[])\thanks{Thanks to Edward R.~Tufte for his inspiration.}}
\title{print(memcpy[])}
\author[Andres Kütt]{Andres Kütt}
\publisher{TeamConsulting}

%%
% If they're installed, use Bergamo and Chantilly from www.fontsite.com.
% They're clones of Bembo and Gill Sans, respectively.
%\IfFileExists{bergamo.sty}{\usepackage[osf]{bergamo}}{}% Bembo
%\IfFileExists{chantill.sty}{\usepackage{chantill}}{}% Gill Sans

%\usepackage{microtype}

%%
% Just some sample text
\usepackage{lipsum}

%%
% For nicely typeset tabular material
\usepackage{booktabs}

% Veiko keemia jaoks
\usepackage[version=4]{mhchem}

%%
% For graphics / images
\usepackage{graphicx}
\setkeys{Gin}{width=\linewidth,totalheight=\textheight,keepaspectratio}
\graphicspath{{graphics/}}

% The fancyvrb package lets us customize the formatting of verbatim
% environments.  We use a slightly smaller font.
\usepackage{fancyvrb}
\fvset{fontsize=\normalsize}

%%
% Prints argument within hanging parentheses (i.e., parentheses that take
% up no horizontal space).  Useful in tabular environments.
\newcommand{\hangp}[1]{\makebox[0pt][r]{(}#1\makebox[0pt][l]{)}}

%%
% Prints an asterisk that takes up no horizontal space.
% Useful in tabular environments.
\newcommand{\hangstar}{\makebox[0pt][l]{*}}

%%
% Prints a trailing space in a smart way.
\usepackage{xspace}

%%
% Some shortcuts for Tufte's book titles.  The lowercase commands will
% produce the initials of the book title in italics.  The all-caps commands
% will print out the full title of the book in italics.
\newcommand{\vdqi}{\textit{VDQI}\xspace}
\newcommand{\ei}{\textit{EI}\xspace}
\newcommand{\ve}{\textit{VE}\xspace}
\newcommand{\be}{\textit{BE}\xspace}
\newcommand{\VDQI}{\textit{The Visual Display of Quantitative Information}\xspace}
\newcommand{\EI}{\textit{Envisioning Information}\xspace}
\newcommand{\VE}{\textit{Visual Explanations}\xspace}
\newcommand{\BE}{\textit{Beautiful Evidence}\xspace}

\newcommand{\TL}{Tufte-\LaTeX\xspace}

% Prints the month name (e.g., January) and the year (e.g., 2008)
\newcommand{\monthyear}{%
  \ifcase\month\or January\or February\or March\or April\or May\or June\or
  July\or August\or September\or October\or November\or
  December\fi\space\number\year
}


% Prints an epigraph and speaker in sans serif, all-caps type.
\newcommand{\openepigraph}[2]{%
  %\sffamily\fontsize{14}{16}\selectfont
  \begin{fullwidth}
  \sffamily\large
  \begin{doublespace}
  \noindent\allcaps{#1}\\% epigraph
  \noindent\allcaps{#2}% author
  \end{doublespace}
  \end{fullwidth}
}

% Inserts a blank page
\newcommand{\blankpage}{\newpage\hbox{}\thispagestyle{empty}\newpage}

\usepackage{units}

% Typesets the font size, leading, and measure in the form of 10/12x26 pc.
\newcommand{\measure}[3]{#1/#2$\times$\unit[#3]{pc}}

% Macros for typesetting the documentation
\newcommand{\hlred}[1]{\textcolor{Maroon}{#1}}% prints in red
\newcommand{\hangleft}[1]{\makebox[0pt][r]{#1}}
\newcommand{\hairsp}{\hspace{1pt}}% hair space
\newcommand{\hquad}{\hskip0.5em\relax}% half quad space
\newcommand{\TODO}{\textcolor{red}{\bf TODO!}\xspace}
\newcommand{\ie}{\textit{i.\hairsp{}e.}\xspace}
\newcommand{\eg}{\textit{e.\hairsp{}g.}\xspace}
\newcommand{\na}{\quad--}% used in tables for N/A cells
\providecommand{\XeLaTeX}{X\lower.5ex\hbox{\kern-0.15em\reflectbox{E}}\kern-0.1em\LaTeX}
\newcommand{\tXeLaTeX}{\XeLaTeX\index{XeLaTeX@\protect\XeLaTeX}}
% \index{\texttt{\textbackslash xyz}@\hangleft{\texttt{\textbackslash}}\texttt{xyz}}
\newcommand{\tuftebs}{\symbol{'134}}% a backslash in tt type in OT1/T1
\newcommand{\doccmdnoindex}[2][]{\texttt{\tuftebs#2}}% command name -- adds backslash automatically (and doesn't add cmd to the index)
\newcommand{\doccmddef}[2][]{%
  \hlred{\texttt{\tuftebs#2}}\label{cmd:#2}%
  \ifthenelse{\isempty{#1}}%
    {% add the command to the index
      \index{#2 command@\protect\hangleft{\texttt{\tuftebs}}\texttt{#2}}% command name
    }%
    {% add the command and package to the index
      \index{#2 command@\protect\hangleft{\texttt{\tuftebs}}\texttt{#2} (\texttt{#1} package)}% command name
      \index{#1 package@\texttt{#1} package}\index{packages!#1@\texttt{#1}}% package name
    }%
}% command name -- adds backslash automatically
\newcommand{\doccmd}[2][]{%
  \texttt{\tuftebs#2}%
  \ifthenelse{\isempty{#1}}%
    {% add the command to the index
      \index{#2 command@\protect\hangleft{\texttt{\tuftebs}}\texttt{#2}}% command name
    }%
    {% add the command and package to the index
      \index{#2 command@\protect\hangleft{\texttt{\tuftebs}}\texttt{#2} (\texttt{#1} package)}% command name
      \index{#1 package@\texttt{#1} package}\index{packages!#1@\texttt{#1}}% package name
    }%
}% command name -- adds backslash automatically
\newcommand{\docopt}[1]{\ensuremath{\langle}\textrm{\textit{#1}}\ensuremath{\rangle}}% optional command argument
\newcommand{\docarg}[1]{\textrm{\textit{#1}}}% (required) command argument
\newenvironment{docspec}{\begin{quotation}\ttfamily\parskip0pt\parindent0pt\ignorespaces}{\end{quotation}}% command specification environment
\newcommand{\docenv}[1]{\texttt{#1}\index{#1 environment@\texttt{#1} environment}\index{environments!#1@\texttt{#1}}}% environment name
\newcommand{\docenvdef}[1]{\hlred{\texttt{#1}}\label{env:#1}\index{#1 environment@\texttt{#1} environment}\index{environments!#1@\texttt{#1}}}% environment name
\newcommand{\docpkg}[1]{\texttt{#1}\index{#1 package@\texttt{#1} package}\index{packages!#1@\texttt{#1}}}% package name
\newcommand{\doccls}[1]{\texttt{#1}}% document class name
\newcommand{\docclsopt}[1]{\texttt{#1}\index{#1 class option@\texttt{#1} class option}\index{class options!#1@\texttt{#1}}}% document class option name
\newcommand{\docclsoptdef}[1]{\hlred{\texttt{#1}}\label{clsopt:#1}\index{#1 class option@\texttt{#1} class option}\index{class options!#1@\texttt{#1}}}% document class option name defined
\newcommand{\docmsg}[2]{\bigskip\begin{fullwidth}\noindent\ttfamily#1\end{fullwidth}\medskip\par\noindent#2}
\newcommand{\docfilehook}[2]{\texttt{#1}\index{file hooks!#2}\index{#1@\texttt{#1}}}
\newcommand{\doccounter}[1]{\texttt{#1}\index{#1 counter@\texttt{#1} counter}}

% Generates the index
\usepackage{imakeidx}
\makeindex[name=ppl, title={Nimede register}]
\makeindex[title={Indeks}]

% See also
\makeatletter
\newcommand{\gobblecomma}[1]{\@gobble{#1}\ignorespaces}
\makeatother

\usepackage{csquotes}


%% Versioneerimine

\newcounter{run}
\InputIfFileExists{\jobname.runs}{}{}
\stepcounter{run}

\usepackage{atveryend}
\usepackage{newfile}
\AtVeryEndDocument{%
  \newoutputstream{runs}%
  \openoutputfile{\jobname.runs}{runs}%
  \addtostream{runs}{\string\setcounter{run}{\number\value{run}}}%
  \closeoutputstream{runs}%
}

\usepackage{needspace}
\raggedbottom
\addtolength{\topskip}{0pt plus 10pt}
%% Küsimuse vormistus
\newcommand{\question}[1]{\begin{samepage}\needspace{3\baselineskip}\textbf{#1\\}\end{samepage}}
%\newcommand{\question}[1]{\begin{minipage}{\textwidth}\textbf{\enquote{#1}}\end{minipage}}

% Reset the sidenote number each chapter
\let\oldchapter\chapter
\def\chapter{%
  \setcounter{footnote}{0}%
  \oldchapter
}


\usepackage{multicol}
\usepackage{pdfpages}
%\usepackage{verse}
\begin{document}

% Front matter
\frontmatter

% r.1 blank page
\blankpage
%\includepdf[fitpaper=true, pages=-]{mock_kaas.pdf}

% v.2 epigraphs
\newpage\thispagestyle{empty}
\openepigraph{%
Design and programming are human activities; forget that and all is lost.
}{Bjarne Stroustrup%, {\itshape Design, Form, and Chaos}
}
\vfill
\def\svgwidth{6cm}
%\input{barcode.pdf_tex}

\begin{fullwidth}
\sffamily\large
\begin{doublespace}
%\noindent\allcaps{Ärge valetage isad }\\ % The quote
%\noindent\allcaps{ära hoia kinni ema mind}\\ % The quote
%\noindent\allcaps{Need ei ole halvad sõbrad}\\ % The quote
\noindent\allcaps{\ldots}\\ % The quote
\noindent\allcaps{see on minu Vennaskond ja ring}\\ % The quote
\noindent\allcaps{Vennaskond. \enquote{Jumal kaitse vennaskonda}} % The author
\end{doublespace}
\end{fullwidth}
%\vfill
%\openepigraph{% 
%Ärge valetage isad ära hoia kinni ema mind Need ei ole halvad sõbrad see on minu Vennaskond ja ring}{Vennaskond. \enquote{Jumal kaitse vennaskonda}}
%\vfill
%\openepigraph{%
%\ldots the designer of a new system must not only be the implementor and the first 
%large-scale user; the designer should also write the first user manual\ldots 
%If I had not participated fully in all these activities, 
%literally hundreds of improvements would never have been made, 
%because I would never have thought of them or perceived 
%why they were important.
%}{Donald E. Knuth}


% r.3 full title page
\maketitle


% v.4 copyright page
\newpage
\begin{fullwidth}
~\vfill
\thispagestyle{empty}
\setlength{\parindent}{0pt}
\setlength{\parskip}{\baselineskip}
Copyright \copyright\ \the\year\ \thanklessauthor

\par\smallcaps{Toimetanud Kadri Põdra}
\par\smallcaps{Välja andnud \thanklesspublisher}

%\par\smallcaps{tufte-latex.googlecode.com}

\par \doclicenseThis 

\par\textit{\monthyear. Version V0.\therun}
\end{fullwidth}

% r.5 contents
\tableofcontents

%\listoffigures

%\listoftables

% r.7 dedication
\cleardoublepage
~\vfill
\begin{doublespace}
\noindent\fontsize{18}{22}\selectfont\itshape
\nohyphenation
Toivole, Meelisele ja teistele.
\end{doublespace}
\vfill
\vfill


% r.9 introduction
\cleardoublepage
\chapter{Sissejuhatus}
Tere. See siin on memcpy. Nende sõnadega olen sisse juhatanud suurt hulka 
intervjuusid oluliste inimestega ja nüüd on teie ees see tekst. 

Aga miks? 

Põhjus, tuleb tunnistada, on lihtne. Nagu ütleb Villu Tamme loos 
\enquote{Paneme punki}:

\begin{verse}
Tahan kord saada selliseks, nagu \\
on Villu või Freddy või Rott või Striit\\
\end{verse}

See raamat räägib inimestest, kes on mulle oma tarkuse, oskuste ja olemusega 
eeskujuks olnud. Kui Toivo Annus\index[ppl]{Annus, Toivo} mu kunagi Skype'i 
tööintervjuule kutsus, kõndisime piki toonase kontori koridori mille ühele 
poole avanesid töö- ja teisele nõupidamisteruumid. Kõigist ustest paistis ja 
koridoris tuli vastu järjest inimesi, kellega mul kas oli alati olnud rõõm koos 
töötada või kellega ma olin alati tahtnud koos töötada. Memcpy on mingit pidi 
katse toda tunnet uuesti kogeda. Sellest ka pühendus.

Siiski ei ole isiklik emotsionaalne heaolu tingimata heaks põhjuseks inimesi 
tülitada või veeta tunde teksti transkribeerides ja toimetades. Laiem põhjus 
memcpy taga on vajadus dokumenteerida inimesi ja nende suhteid, kelle isiklike 
väikeste näppude alt on välja tulnud kõik suuremad Eesti IT-edulood. 

Riigi Infosüsteemi Ametis\index{Riigi Infosüsteemi Amet} töötades pidin aastate 
kaupa peaaegu igal nädalal rääkima riigi infosüsteemist, selle ülesehitusest ja 
ajaloost ning vastama küsimustele. Mul ei olnud pikalt vastust sagedasele 
küsimusele \enquote{Miks Eestis ja mitte mujal?}. Meil ei ole objektiivselt 
vaadates erilisi põhjusi olla oma naabritest edukamad, me isegi ei tee midagi 
eriti innovatiivset aga ometi oleme suutnud kiiresti edasi liikuda ja meil on 
kogu maailmas selge positiivne IT-teemaline imago. Miks? Vastust otsides 
jõudsin ikka ja jälle usalduse küsimuseni. Mingil põhjusel Eestis usaldatakse 
IT-inimesi, neid kaasatakse oluliste otsuste juurde ja IT-inimesed suudavad 
selle usalduse ka välja teenida. Sedalaadi suhetel on juured ja nende üle 
juureldes jõudsin aega natuke enne ja pärast Vabariigi taassündi. Ühtäkki 
hakkas meile jõudma arvuteid, kuid keskmisel inimesel puudus igasugune võimekus 
neid kasutada. Teisalt oli tekkinud toimekas seltskond kodanikke, kes oskasid 
arvutit kasutada, kuid kellel ei olnud neile ligipääsu. Ja nii sündiski 
arusaam, et koos on mõistlik. Et IT-st on kasu. Et kuskil on kellegi 
lahendamist vajavad elulised probleemid. Ja, mis peamine, et seda suhet ei ole 
mõistlik lõhkuda. 

Kust too seltskond tuli, kuidas toimis, kes sinna kuulusid? Nendele küsimustele 
otsib memcpy vastust, kui \emph{fanboy} roll vähegi võimaldab. Seetõttu ongi 
fookus inimestel ja mitte näiteks ettevõtetel\sidenote{Vähemalt Dateli, 
Proeksperdi ja Microlinki kohta on väikesetiraa\v{z}ilised ajalood ilmunud.} 
või kurioossetel seikadel. 

Ometi ei ole ma ajaloolane ega folklorist, kas memcpy-t ei võiks teha 
professionaalid? Kõik katsed leida keegi asjatundja asja läbi viima luhtusid 
sel lihtsal põhjusel, et kellelgi ei olnud teema vastu piisavat isiklikku huvi 
ja kõik katsed ettevõtmist kuidagi rahastada jooksid liiva. Pärast esialgse 
idee formuleerimist veetsin ma umbes aasta üritades edutult leida tegijaid ja 
rahastust. Seejärel veetsin umbes pool aastat veendes ennast, et memcpy ei pea 
olema täiuslik. Intervjuude ette valmistamine, salvestamine, toimetamine ja 
järeltöötlus on tehniliselt keerulised protsessid, mida ma ei vallanud siis ja 
ei valda ka praegu. Siiski oli selge, et Issanda päike enne looja läheb, kui ma 
neis mind ennast rahuldava taseme saavutan. Nii tuli süda kõvaks teha ja teha 
mitte nii hästi, kui teema vajaks, vaid nii hästi, kui suudan. Seetõttu on 
eriti esimeste memcpy episoodide helikvaliteet päris kole ja see häirib mind 
siiani.

Siiski sai sügisel 2018 purki esimene episood Prontoga\index[ppl]{Pronto} ja 
kevadeks veel kaheksa episoodi. Suvel on inimesed liikvel ja nii jätkasin 
sügisel 2019 juba märksa parema planeerimisega saades napilt enne COVID-19 
pandeemia Eestisse jõudmist 2020. aasta märtsis purki ka teise hooaja 
intervjuud. Episoode kokku lõigates jäi mind häirima, et need ei ole mugavalt 
otsitavad. Inimesed, ettevõtted ja kohad jooksid läbi eri lugudest, aga 
millistest? Väga raske on öelda midagi võrgustiku kohta, kui seda võrgustikku 
saab uurida vaid tipp-haaval. Kuna pandeemia tõttu uusi episoode salvestada ei 
saanud siis oli loogiliseks sammuks võtta aega olemasolevate episoodide 
transkribeerimiseks, toimetamiseks ja indeksiga varustamiseks. Ehk tekitamaks 
seda teksti siin. 

Seega on memcpy igate pidi väga isiklik projekt ning sellisena paratamatult 
piiratud. Ma ei saa ega kavatsegi toota kiretut ajaloodokumenti\sidenote{Mu 
enda peatükk on lisatud just võimaldamaks \enquote{autorifiltri} paremat 
mõistmist.} ning teisalt ei saa lootagi, et võiksin suuta rääkida kõigi 
huvitavate või oluliste inimestega. Kõik lihtsalt ei mahu raamatusse, mõned ei 
soovinud (minuga) rääkida ja mõned lihtsalt ei tulnud pähe. Andestust! 
Mõningased piirid intervjueeritavatele seab ka projekti ajaline määratlus just 
kaheksa- ja üheksakümnendatega. Nii on enamasti välja jäänud näiteks 
Mainori\index{Mainor} ümber tegutsenud seltskond ning natuke vanema põlvkonna, 
näiteks kadunud Ahto Kalja\index[ppl]{Kaljo, Ahto} ja Monika 
Oiti\index[ppl]{Oit, Monika} tegemised. Samuti on puht praktilistel põhjustel 
seltskonnas vähe Tartus tegutsenud ning venekeelse taustaga inimesi. Üle ega 
ümber ei saa ka asjaolust, et kunagi IT-rahva kohta laialt kasutusel olnud 
mõiste \enquote{patsiga poisid} ka memcpy-s otsest peegeldust leiab. Enamasti 
on tõesti tegu poistega. Kahju küll, aga uuritav kogukond paraku oli 
ebaproportsionaalselt maskuliinne ja selle teistsugusena kujutamine ei oleks 
päris õige. Samas olid Vilve Vene\index[ppl]{Vene, Vilve} ja Anne 
Villemsi\index[ppl]{Villems, Anne} ühed kõige huvitavamad salvestada.

Inimeste mälu on erinev. Seega lähevad inimeste lood, ja just nimelt lugude 
talletamine on memcpy eesmärk, omavahel detailides vastuollu. Otsesed 
kõrvalekalded teadaolevast reaalsusest on osundatud ning pisemad vead 
parandatud. Siiski ei maksa oodata, et järgnevatel lehekülgedel näiteks vana 
Tartu ja Tallinna koolkondade vastuolu kuidagi objektiivselt lahendatud saaks. 
Tegu on lugudega ja neid tuleb paratamatult võtta tera soolaga.

Samuti tuleb arvestada, et suuri asju tegevad huvitavad inimesed on harva 
lihtsad isiksused. Olen üritanud kunagisest küllalt keerulisest suhete taagast 
oskust mööda üle olla. Seetõttu on intervjuud järgnevatel lehekülgedel 
tähestikulises järjekorras ja mitte intervjuude toimumise või näiteks olulisuse 
omas. 

Transkribeeritud ja podcasti eetrisse läinud juttudest on mõned üksikud 
detailid ka välja jäetud, sest mõnest asjast ei taha inimesed väga rääkida ja 
mõnda asja ei ole paslik tiražeerida. Üheksakümnendad oli päris hull ja 
tänasest täitsa erinev aeg. Tegu on siiski detailidega, mis suurt pilti ei 
tohiks mõjutada. Muu jutt on enamasti täies mahus\sidenote{Erandiks on 
intervjuu Tarvi Martensiga\index[ppl]{Martens, Tarvi}. Temaga oli meil paljust 
rääkida ja salvestasime kaks episoodi, mis teemade poolest osaliselt kattusid. 
Seega tuli kirjalikus tekstis selguse huvides asju natuke ümber tõsta ja 
tihendada.} ja võimalikult originaalilähedase keelekasutusega ära toodud. 
Sellest ka anglitsismide ja võõrkeelsete terminite kahetsusväärselt suur hulk. 
Aga kuna keelekasutus annab huvitava akna inimesse, eelistasin autentsust 
ilusale emakeelele.  

Tekst on mõeldud olema ka mitte-arvutiinimestele üldjoontes arusaadav: 
konteksti mõistmiseks olulised terminid on lahti seletatud ning tänaseks ehk 
ununenud asjad viidatud, kuid detailid otsib huviline ise välja. Eesmärk ei ole 
olnud anda struktureeritud ülevaadet arvutustehnika ajaloost või vanade 
tehnoloogiate toimimisest. Intervjuudes olen üritanud küsida võhiku 
positsioonilt. Mis on seda lihtsam, et paljus ma seda olengi.

Järgnevaid lehekülgi ei ole juba kasvõi nende mahu pärast ehk mõistlik kaanest 
kaaneni lugeda. Targem on teda lapata kasutades kas nimede või muud indeksit, 
alustades mõnest huvitavamast loost või lihtsalt alates juhuslikust leheküljest 
pea ees minevikku hüpata.

Kuigi memcpy on, nagu öeldud, isiklik projekt. Siiski olen tema tegemise käigus 
saanud hindamatut tuge ja soojad tänuavaldused lähevad teele:

\begin{description}
	\item[Meelis Roos]\index[ppl]{Roos, Meelis} kes aitas nii oma kui nii 
mõnegi teise teksti toimetamisega, parandas mu piinlikke kirjavigu, mõtles 
projektiga kaasa ja andis tehnilist tuge
	\item[Rein Rüüsak]\index[ppl]{Rüüsak, Rein} kes aitas ajakirja A\&A 
ajalugu välja uurida
	\item[Ott Köstner]\index[ppl]{Köstner, Ott} kes on memcpy podcasti 
kaanepildi autor
	\item[Vootele Voit]\index[ppl]{Voit, Vootele} kes kommenteeris 
asjalikult ZX Spectrumi kiibistikku puudutavat
	\item[Kõik intervjueeritavad] kes võtsid oma tihedast päevast tunni, et 
minuga juttu rääkida
	\item[Veebipõhine transkriptsioon] ilma milleta käesolev tekst 
kindlasti sündinud ei oleks. Alumäe, Tanel; Tilk, Ottokar; Asadullah. "Advanced 
Rich Transcription System for Estonian Speech" Baltic HLT 2018
\end{description}


\chapter{Tänusõnad}
Memcpy on, nagu öeldud, isiklik projekt. Ometi on põhjust tänulik olla. 
Eelkõige kaasteelistele, kes on minuga ja minu ümber olnud. Neile, kes võtsid 
oma tihedast päevast tunni, et minuga juttu rääkida on eriline tänu. Neile, keda saame tänada
kunagi ühes kohvikus, on pühendus.

Aga on ka konkreetseid inimesi ja ettevõtmisi, kelleta see raamat ei oleks sellisel kujul sündinud. 

Esmalt tänan transkriptsioonitarkvara autoreid\sidenote{Alumäe, Tanel; Tilk, Ottokar; Ullah, Asad. Advanced 
Rich Transcription System for Estonian Speech. Baltic HLT 2018.}, kelle tööta see raamat kindlasti ei oleks sündinud. Samuti tänan Kadri Põdrat, kes kogu teksti eestikeelseks tegi. 

Teiseks tahan tänada inimesi, kes panustasid ühel või teisel viisil lugude täiendamisse märkuste ja selgitustega ning olid muidu abiks:

\begin{description}
	\item[Tarmo Mamers]\index[ppl]{Mamers, Tarmo} aitas palju nimede ja muude detailidega
	\item[Ott Köstner]\index[ppl]{Köstner, Ott} tegi memcpy \emph{podcast}'i kaanepildi
	\item[Kaspar Loit]\index[ppl]{Loit, Kaspar} tegi selle raamatu hurmava esikaane\sidenote{Mis muu hulgas peidab endas mitmeid viiteid kunagistele .EXE\index{.EXE} esikaantele!}
	\item[Vootele Voit]\index[ppl]{Voit, Vootele} kommenteeris 
 ZX Spectrumi kiibistikku puudutavat
	\item[Mart Palmas]\index[ppl]{Palmas, Mart} aitas Soome telekavade teket mäletada 
	\item[Gert Silling]\index[ppl]{Silling, Gert} aitas jõuda Ants Roose ja Algoritmini 
	\item[Ants Roose]\index[ppl]{Roose, Ants} andis infot legendaarse Arvutustehnika ja Andmetöötluse ning Algoritmi kohta 
\end{description}

Hulk inimesi panustas Hooandja platvormil selle raamatu ilmumisele paberil ka rahaliselt. Aitäh teile kõigile!

\begin{multicols}{3}
\small
Andres Ääremaa\\
Raido Aarop\\
Henrik Aavik\\
Ain Aaviksoo\\
Kerti Alev\\
Raul Allikivi\\
Lauri Antalainen\\
Lauri Anton\\
Sille Arikas\\
Hannes Arus\\
Aho Augasmägi\\
Fredi Dorbek\\
Kaido Einama\\
Margus Ernits\\
Priit Haamer\\
Keio Hämäläinen\\
Alvar Hansen\\
Risto Hansen\\
Jaak Härma\\
Risto Hinno\\
Margus Holland\\
Kristiina Hunt\\
Raik Ilves\\
Tõnis Jaagus\\
Urmet Jänes\\
Marko Jõemets\\
Agur Jõgi\\
Kaarel Jõgi\\
Kaarel Julge\\
Ardi Jürgens\\
Madis Kaal\\
Kaia Kalberg\\
Jüri Kaljundi\\
Allan Kändmaa\\
Margus Kangur\\
Andrus Kanter\\
Kristjan Karmo\\
Kenert Karu\\
Jaanus Kase\\
Klemens-Augustinus Kasemaa\\
Evelin Kasenõmm\\
Heiki Kask\\
Anu Käver\\
Harri Kirik\\
Martin Kivi\\
Marko Kivimäe\\
Edgar Kivit\\
Tobias Johannes Koch\\
Gerri Kodres\\
Kristo Kooskora\\
Alek Kozlov\\
Tristan Krass\\
Aivar Kraus\\
Tiit Kriiska\\
Kristjan Krips\\
Ülle Kroon\\
Valve Krosing\\
Kristjan Kuhi\\
Rainer Kuhi\\
Martti Kuldma\\
Kaur Kullman\\
Kristi Küppar\\
Els Kütt\\
Maris Kütt\\
Hanno Kuus\\
Katrin Laas-Mikko\\
Jan Lakspere\\
Meelis Lang\\
Juhan Lasn\\
René Lasseron\\
Sander Laumets\\
Jüri Laur\\
Robert Laursoo\\
Elen Leigri\\
Hillar Leoste\\
Toomas Lepik\\
Priit Liivak\\
Aali Lilleorg\\
Martin Lillepuu\\
Mari-Liis Lind\\
Elo Lindi\\
Alvar Lumberg\\
Tarmo Luumann\\
Alar Mäerand\\
Ivo Mägi\\
Joonathan Mägi\\
Kristo Mägi\\
Rauno Mägi\\
Maili Mahlapuu\\
Kitty Mamers\\
Tarmo Mamers\\
Argo Mändmaa\\
Viljo Marrandi\\
Tarvi Martens\\
Allan Martinson\\
Peeter Marvet\\
Jaan Metsa\\
Janek Metsallik\\
Aldo Mett\\
Dmitri Mihhailov\\
Andres Mihkelson\\
Jan-Erik Moon\\
Hanno Mosov-Hallik\\
Aivar Naaber\\
Jaak Niit\\
Allar Olgo\\
Mart Oruaas\\
Rudolf Osman\\
Ehouse OÜ\\
Ideatest Oü\\
Priit Pääsukene\\
Martin Paljak\\
Meelis Palover\\
Martin Paroll\\
Mart Parve\\
Alo Peets\\
Kalev Pihl\\
Anu Piirisild\\
Madis Pink\\
Allan Poola\\
Priit Potter\\
Diana Poudel\\
Neeme Praks\\
Reet Prii\\
Jaak Pruulmann-Vengerfeldt\\
Martin Raag\\
Urmo Rae\\
Toomas Rand\\
Eero Ränik\\
Allan Rank\\
Tarmo Rasmann\\
Indrek Rebane\\
Kristjan Rebane\\
Jana Reikop\\
Raimo Reiman\\
Tõnis Reimo\\
Andrei Reinus\\
Martti Remmelgas\\
Henrik Roonemaa\\
Kristjan Roosild\\
Argo Roots\\
Holger Rünkaru\\
Peeter Russak\\
Maarja-Leena Saar\\
Silver Salla\\
Joonatan Samuel\\
Marti Schmidt\\
Asko Seeba\\
Andrus Seiman\\
Andres Selli\\
Martin Sepp\\
Priit Siilaberg\\
Indrek Siitan\\
Siim Sikkut\\
Hanno Sirkel\\
Mihkel Solvak\\
Tiia Sõmer\\
Tarmo Soodla\\
Alvar Soome\\
Vladimir Šor\\
Erik Suit\\
Urmo Sutermäe\\
Enn Sutting\\
Kalle Tabur\\
Gristel Tali\\
Tarmo Tali\\
Sten Tamkivi\\
Taavi Tamkivi\\
Asko Tamm\\
Taavi Tamm\\
Kalle Tammemäe\\
Taavi Tänavsuu\\
Lauri Tarend\\
Tikan Tarko\\
Villu Teearu\\
Janno Teelem\\
Hasso Tepper\\
Taavi Tiirik\\
Marek Tiits\\
Sten Tikerpe\\
Ragnar Toomla\\
Renee Trisberg\\
Ahto Truu\\
Ülle Tuulmägi\\
Teet Vaher\\
Laas Vahur\\
Lauri Väin\\
Margus Vaino\\
Tanel Vakker\\
Toomas Vaks\\
Härmo Väljaste\\
Sven Varkel\\
Aarne Vasarik\\
Ekke Vasli\\
Anti Veeranna\\
Renno Veinthal\\
Vallo Veinthal\\
Linnar Viik\\
Siim Viikman\\
Toomas Viirsalu\\
Rauno Villberg\\
Rainer Villido\\
Martin Villig\\
Kalle-Rasmus Volkov\\
Vahuri Voolaid\\
Ivar Zarans\\
\end{multicols}
 
%%
% Start the main matter (normal chapters)
\mainmatter


\chapter{Andrus Aaslaid}
%!TEX TS-program = arara
% arara: myindex

\index[ppl]{Aaslaid, Andrus}
\question{Kuidas sa arvutite juurde jõudsid?}

Tihti on nii, et me ei mäleta, kuidas me oma elu muutvad otsused  
tegime. Aga seda juhust ma mäletan täpselt. Mul oli juba toona 
raadiohobi. Olin põhikooli juntsu ja mulle meeldis hirmsasti mööda 
lühilainet ringi kammida. Meil oli kodus Melodija 101 stereo, Riia 
raadiotehase\sidenote{A. S. Popovi nimeline Riia Raadiotehas, alates 1951 Rigas 
Radio Rupnica.} toodang. Sellega ma siis seiklesin suviti, kui midagi targemat 
teha ei olnud, mööda eetrit. Tegelikult oli mul kaks raadiot: lisaks Melodijale 
detektorvastuvõtja, mille mu poolvend 
oli mulle ehitanud. Sellega ma istusin pööningul. Vanemad tegelesid 
põllumajandusega ja neil oli 
üks konkreetne põllumajandusnipp: raamatukogudest toodi vanu ajakirju, 
need rebiti lehtedeks, keerati ümber õõnsa 
põhjaga pudeli väikesteks pottideks, mille sisse istutati taimed. 
Paber lagunes mulla sees ära, taim pääses põllul vabaks. Neid ajakirju oli 
pööningul tohutu hunnik, muu hulgas mitu aastakäiku 
\begin{russian}Техника - молодёжи\end{russian}'t\sidenote{Aastast 1933 ilmuv 
algselt Nõukogude ja nüüd Vene populaarteaduslik ajakiri.}. Lappasin siis 
pööningul neid ajakirju, detektoriklapid peas. 

Igatahes ükskord astusin ma tuppa, lülitasin Melodija sisse ja sealt öeldi, et 
Tallinna 43. Keskkool\index{Tallinna 43. Keskkool}\sidenote{Praegune 
tehnikagümnaasium.\index{Tehnikagümnaasium|see{Tallinna 43. Keskkool}}} 
on otsustanud hakata 
eksperimentaalseks tehnikaülikooli\index{Tallinna 
Tehnikaülikool}\sidenote{Tallinna Polütehniline Instituut, praegune Tallinna 
Tehnikaülikool.} ettevalmistuskooliks ja 
nad võtavad kümnendasse klassi vastu õpilasi, kes tahaksid TPIsse edasi õppima 
minna. Kuulasin uudise ära, lülitasin raadio välja, läksin vanemate juurde ja 
teatasin, et lähen Tallinnasse kooli. Ma olin siis 14.


\question{Kust sa pärit oled, et tahtsid Tallinna kooli 
minna?}

Pärit olen ma tegelikult kahesaja meetri kauguselt sealt, kus ma täna elan, 
ehk siis Tallinnast. Aga kuna mu perekond otsustas evakueeruda 
Muhusse, kui ma olin kahe- või kolmeaastane, siis mind 
deporteeriti sinna. Nii et oma põrsapõlve veetsin Muhus ja siis ühel 
hetkel panin sealt tagasi tehnoloogia juurde putku. 

\question{Mõni ime, et te Mastiga\index[ppl]{Kaal, 
Madis}\index[ppl]{Mast}\sidenote{Vt lk \pageref{cptr:mast}.} 
hästi läbi saate!}

Me oleme Mastiga ühe kooli poisid, Mast oli keskkoolis, kui mina olin 
põhikoolis. Me oleme mõnda aega isegi sama 
raadiosõlme väisanud. Aga ega tollel ajal nooremad ja vanemad väga läbi käinud, 
eriti veel 
maakohtades. Mast oli hea 
poiss, ei peksnud nooremaid ega midagi. 

\question{Mis sealt lühilaine pealt kostis? Muusikat?}

Ei, muusikat kuulati Radio Luxembourgist. Lühilaine pealt tuli erinevaid 
hääli: morset, huvitavaid kahinaid ja sahinaid, keegi 
luges numbreid. Lühilaine on tegelikult siiamaani päris hea tervise 
juures, eeter on maast laeni sodi täis ja olemus 
ei ole väga palju muutunud. Võibolla propagandasaateid on vähemaks 
jäänud ja Hiina raadiojaamu vaikselt kinni pandud  
interneti pealetulekuga. Üldiselt on lühilaine ilmselt ikka samasugune nagu 
nelikümmend aastat tagasi.

\question{Kas nende ajakirjade hulgas oli arvutiajakirju ka?}

Esimest arvutit nägin tänu poolvennale. Ta tundis Guido 
Tammissaart\index[ppl]{Tammissaar, Guido} Eesti Energia 
arvutuskeskusest\index{Eesti Energia!Arvutuskeskus}. Ühel 
päeval tuli poolvend maale ja ütles: \enquote{Tule kaasa paariks päevaks, näed, 
mis asi 
see arvuti on. Sind see tehnikaasi huvitab.} Ja lubatigi mind paariks päevaks 
maalt 
linna. Estonia puiestee arvutuskeskuses olid tollal veel põhiliselt 
SMid\index{SM EVM}\sidenote{\begin{russian}Система Малых ЭВМ (СМ 
ЭВМ)\end{russian} oli mitut tüüpi Nõukogude Liidus toodetud, enamasti lääne 
analoogidel põhinevate arvutite üldnimetus.}. Ja 
üks CP/M\sidenote{CP/M oli 1974. 
aastal Inteli 8080/85 protsessorisarja tarvis turule toodud 
operatsioonisüsteem, mille 1980ndatel asendas mitmes mõttes sarnane MS-DOS.} 
masin, mis tagantjärele tundub oma sotsmaa disaini poolest täiesti kosmiline. 
Küllap Bulgaarias toodetud. Olen mõelnud, et 
peaks üles otsima, mis masin see selline võis olla. 

Sellel CP/M masinal ma klõbistasin niisama, aga 
SM-4\index{SM EVM!SM-4}\sidenote{SM-4 oli PDP-11/40\index{PDP-11} 
ühilduv 
Nõukogude päritolu ja terves idablokis toodetud arvutisüsteem.} peal 
kirjutasin selsamal päeval oma esimese BASICu\index{BASIC} programmi. 
See oli derivaat mingist asjast, mida mulle näidati, et näed, umbes nii 
käib. Ja edasi ma olin \emph{hooked}. Sellest ühest päevast piisas, et sõltlane 
tekitada. 

\question{See oli enne seda, kui otsustasid, et nüüd oled 
neliteist ja lähed Tallinnasse kooli?}

Ma ei oskagi öelda, ma ei ole sada protsenti kindel, kumb oli enne, kumb 
pärast, ja kas huvi tulla Tallinnasse mängis rolli. Ega nad ju 
arvutikallakut tegelikult ei propageerinud, suurem rõhk oli elektroonikal. 
Tarkvara osa nad väga ei reklaaminud. Minust pidi tegelikult elektroonik saama 
ja see minust ka sai, aga tollal tundusid ikkagi arvutid 
see päris asi. 

\question{Kas 43. keskkoolis valmistati päriselt ka ette 
ülikooliks? Oli sellest kasu?}

See oli selline kahe teraga mõõk -- valmistati ette ja 
väga hästi. Keskkooliprogrammi olid kokku pannud 
tollaste inseneride õpetajad, kes teadsid suhteliselt hästi, mida tuleks 
õpetada, et põhi alla saada. Saime 
läbisegi tavalisi keskkooliaineid ja siis ühel hetkel tuli härra 
Tiidemann\index[ppl]{Tiidemann, Tiit} meile rääkima võllide 
epüüridest\sidenote{Epüür (pr \emph{épure}) on teatava suuruse asukohast 
olenevate väärtuste graafiline esitus.}. Sisuliselt tegime käsitsi võllidele 
rakendavate jõudude arvutusi, näiteks kust läheb võll katki, kui see on siit
sellise ja sealt säärase jämedusega. Vahelduseks loeti meile 
teise kursuse elektrotehnikat ja 
inseneripsühholoogiat, mida andis Toomsalu\index[ppl]{Toomsalu, Arvo} ja mis ei 
olnud vist üldse TPI õppekavas. Meie õppekava lühinimetusega ETEK\index{ETEK}, 
mille koostasid Ants Reili\index[ppl]{Reili, Ants} ja 
Peeter Grossberg\index[ppl]{Grossberg, Peeter}, oli kõikide jaoks äge 
eksperiment ja täielik \emph{greenfield}, eriti kuna 
olime esimene lend.\sidenote{vt ka lk \pageref{sisu:43kool}.} 

Lahe oli ka see, et enne meid oli keskkool tühjaks löödud ja me olime kolm 
aastat keskkooli kõige 
vanem klass. Olime koolis nagu jumalad ja 
tänu sellele jäid olemata mitmed probleemid, mida tavalistes 
keskkoolides tol ajal veel eksisteeris. Keegi kedagi ei toginud ega 
nüginud ja samal ajal tekkis kõigil mingisugune väärikus. 

Kahe teraga mõõk oli see aga sellepärast, et nii kõva põhja pealt läksid paljud 
otse tööle. Me saime ju keskkooli lõpetades kõik  
automaatselt TPIsse sisse, sisseastumiseksamit ei olnud vaja teha. Nii et kõik 
meie vist kaheksateist õpilast marssis otse TPIsse. Nendest 
nominaalajaga lõpetas kooli vist paar inimest. Paljud läksid tööle, kuna aeg 
oli 
selline, et see, mida TPIs tollal arvutiteadusena õpetati, ei jõudnud 
päris elule veel järele. See pidi olema aasta 1991 või 1992, kui 
see \enquote{kambriumiplahvatus} siin Eestis toimus.

Mina istusin ööd-päevad arvuti taga ja kirjutasin 
ihuüksi tarkvara, mis pidi 
üleval hoidma tervet suurt autoparki. Samal ajal üritasin ennast kuidagi nügida 
läbi SuperCalci\index{SuperCalc}\sidenote{Varajane tabelarvutussüsteem, 
algselt loodud CP\textbackslash M operatsioonisüsteemile.} arvestusest TPIs, 
kus 
aeg-ajalt tuli õppejõule näidata, et \enquote{ära nii tee, nii see asi päris ei 
käi}. 
Mitte et nad oleksid rumalad olnud, nad õpetasid seda, mida olid kogu 
aeg õpetanud. Nüüd aga tekkis selline seis, kus reaalne elu liikus edasi palju 
kiiremini kui õppekava.

\question{Kuidas sa ikkagi programmeerimise juurde jõudsid? Sa 
pidid seda ju saama kuskil harjutada?}

Tänu 43. keskkoolile see eksperiment kestis ja kestab mõnes mõttes tänaseni. 
Seal oli 
põhimõtteliselt esimest korda selline päris arvutiinimese elu. Kuna 
IT-spetsialiste liiga palju ei olnud, siis juhtus selline hämar lugu, et meile 
Eero Tohvriga\index[ppl]{Tohver, Eero} ulatati kümnendas klassis arvutiklassi 
võtmed ja hakati 
koolist palka maksma. Tegelikult oli see vist seotud 
kerge koolipoolse kaastundega. Peale 
kaheksandat klassi tööstuskooli tulemise traditsiooni ei olnud enam juba 
mõnikümmend aastat ja kõigile tundus see kangesti hirmus, et laps tuleb üksi 
Tallinnasse. Ma arvan, et see oli pigem koolipoolne stipendium. 
Kahe peale maksti meile täisõppejõu palka, mis 
ei olnud ilmselt palju väiksem kui õpetajad 
ise said. Nii hästi kui keskkooli ajal ei ole ma kunagi ei varem ega 
hiljem elanud. 

\question{Mida te selle rahaga tegite?}

Käisime restoranis söömas ja mida ikka lapsed rahaga teevad. Aga kool sai 
selle, et nad ei pidanud rohkem arvutiklassiga tegelema. Klassis oli kolm-neli  
Iskrat\index{Iskra}\sidenote{\begin{russian}Искра\end{russian} oli 
mitmel pool Nõukogude Liidus eri modifikatsioonides toodetud arvutiseeria, mis
 omakorda jagunes erinevaid lääne süsteeme kopeerivateks mudeliperekondadeks.}, 
mida me püsti hoidsime. Meie asi 
oli hoolitseda, et masinad töötaksid ja nendel saaks midagi õpetada. Ühel
hetkel, kui olime ise juba natuke vanemad, tekkis arvutiklassi 
kamp nooremaid huvilisi, kes seal pidevalt hängisid. Arenes
tüüpiline arvutiklassi ökosüsteem. Ühel suvel ka remontisime 
klassi: värvisime ja panime uued põrandakatted. Ühesõnaga käitusime 
loodetavasti heaperemehelikult. 

\question{Tollal ilmselt ei olnud sarnastes situatsioonides hea{\-}peremeheliku 
käitumisega eriti 
probleeme?}

Aeg oli selline, inimeste usaldus oli suur. Arvuti oli müstiline ja teistmoodi 
asi, vanem 
generatsioon justkui kartis seda. Kunagi asus Rävala puiesteel, seal, kus 
praegu on Sakala 3 teatrimaja, turismibüroo Sarved ja Sõrad\index{Sarved ja 
Sõrad} (ma ei tea siiamaani, kellele see kuulus). Juhtusin nende akna alt mööda 
minema ja nägin, et neil on 
seal arvuti. See oli vist aastal 1991, igatahes ma veel ei töötanud 
Skriiningus\index{Skriining}. Keskkool 
oli läbi, sinna mind enam sisse ei lastud arvutit kasutama. Eks sõltlane käis 
mööda linna ja järsku nägi arvutit. Tundmatu värske keskkoolilõpetaja 
marssis tundmatusse firmasse hooga sisse, et 
\enquote{teil on siin arvuti, ma tahaksin seda kasutada}. Ja ilma mingisuguse 
tänapäeval heaks kiidetud taustauuringuta ja töövestluseta ütles firma omanik 
oma kirjutuslaua tagant: \enquote{Jah, loomulikult, me tahaksime seda ise ka 
kasutada.}  Pikema jututa anti mulle kontorivõtmed ja öeldi: \enquote{Tee 
see korda, et meie saaksime ka arvutit kasutada}. Ja avastasingi end
arvuti tagant, ilma et keegi oleks isegi dokumenti vaadanud või mõelnud, kas 
tegu on
vargaga, kes tahab terve firma ära varastada või 
ainult arvuti. Usaldus, mis tollal valitses inimeste vastu, kes 
oskasid arvuti sisse lülitada ja sellega midagi teha, oli 
\emph{enormous}.\phantomsection\label{sisu:andrus_usaldus} 
Tänapäeval ei ole võimalik seda ette kujutada. Värskel keskkoolilõpetajal oli 
põhimõtteliselt võimalik küsida ükskõik millise firma ükskõik millise  
arvuti \enquote{võtmed}. 

Noh, see lõppes muidugi sellega, et lõpuks tuli Imre Perli\index[ppl]{Perli, 
Imre}\sidenote{Imre Perli oli pehmelt öeldes raju elulooga Eesti 
arvutispetsialist, kes sai kuulsaks \enquote{Perli andmebaasi} koostajana. 
Kasutades ära ligipääsu mitmele andmebaasile, lõi ta üheksakümnendate keskel 
althõlma levinud \enquote{superandmebaasi}, mis sisaldas isikustatud andmeid autode, (toona üsna 
haruldaste) mobiiltelefonide, aadresside jms kohta.
 Perli hukkus segastel asjaoludel 15. aprillil 2000 
politseioperatsiooni käigus.} ja kopeeris kellelegi andmebaasid. Eks iga 
aeg saab lõpuks otsa. 

\question{Kuidas see programmeerima õppimise protsess ikkagi käis?}

See on eelmisel sajandil tekkinud paradigma, et 
programmeerimine on midagi, mida peab õppima ja millega tuleb 
spetsiaalselt vaeva näha. Programmeerimine juhtub. Vajadusest ja tahtmisest. 
Keegi ei ole mulle mitte kunagi õpetanud 
ridagi C-d ega assemblerit. 

\question{Ometi said ju kuskilt teada, kuidas \texttt{malloc} käib.}

See sündis tahtmisest teha. Mina hakkasin  
Pascalit\index{Pascal} õppima seepärast, et 
mulle sattus kätte Jürgensoni pruunide kaantega Pascali  
raamat\sidenote{Rein Jürgensoni 
\enquote{Programmeerimine Pascal-keeles} (1985), mis 
huviliste hulgas laialt levis.}, mis on tagantjärele mõeldes 
päris õudne algus programmeerimisele. Kui 
Turbo Pascal hakkas ära tüütama (selles 
keeles midagi normaalset teha oli väga keeruline), siis ühel hetkel 
leidsin, et assembler\index{assembler} on see päris asi. Kuna tol 
ajal oli popp kirjutada igasuguseid demosid ja häkkida kõiki tarkvarasid, mis 
kätte sattus, siis\ldots{ }Kuidas õppida x86 
assemblerit? Võtad raamatu ühte kätte ja AT86 teise kätte ning hakkad tegema.

\question{Kust sa selle raamatu said? Neid ju ei liikunud.}

Liikus küll. Selle eest tuleb tõenäoliselt varem või hiljem anda
presidendi auraha Tarmo Mamersile\index[ppl]{Mamers, 
Tarmo}\index[ppl]{Mamers, Tarmo}, kes oli tollal 
TTÜ-s\index{Tallinna Tehnikaülikool} üks arvutiasjanduse püstihoidjatest. 
Tarmo kaudu materjalid liikusidki, käest kätte. Tema oli raudselt minu varane 
mentor ja veel pikka aega ka siis, kui ma 
juba tööl käisin. Hiljem tuli 
FidoNet. Kui ma oma esimese FidoNeti \emph{point}'i 
püsti panin, siis oli kõik juba palju lihtsam, sest aken maailma oli
olemas. \emph{Point}'i püstipanemine käis ka loomulikult läbi TPI. Seal käis 
põhiline elu ja \emph{action}   
Aare Tali\index[ppl]{Tali, Aare}\phantomsection\label{sisu!aare_tali} ja Tõnu 
Raimla\index[ppl]{Raimla, Tõnu} toas. Tarmo juures teises ruumis oli natuke 
rahulikum 
õhkkond. 

Ühel  
hetkel (töötasin siis Skriiningus\index{Skriining}) tekkis mul kinnisidee teha 
endale FidoNeti \emph{point}, et 
lõpuks olla osa maailmast. Läksin Aare juurde: \enquote{Noh, Aare, 
sa oled siin \emph{sysop} ja värk} ning Aare ütles talle omase abivalmidusega: 
\enquote{Jah, masin on seal.} Leidsingi ennast seepeale BBSi masina tagant ja 
asusin
valmistama FidoNeti \emph{node}'i. Ilmselt Tõnu või keegi 
lõpuks halastas mu peale ja näitas, kuidas seda päriselt teha. 

Edaspidi oli materjal palju kättesaadavam, sai 
igasuguseid dokumente risti-rästi alla laadida. 

\question{Mida sa TPIsse õppima läksid?}

Ma läksin LIsse. Tollal nimetati seda vist informaatikaks. 
Kuna sain suhteliselt ruttu aru, et ma ei ole võimeline hommikul loengutes 
käima, siis läksime pundi inimestega, kes olid ka otsustanud, et nemad 
peavad õhtuõppes käima, dekanaati ja nõudsime õhtust vahetust. Kateedris öeldi, 
et jaa, väga tore mõte, aga 
meie kogemus näitab, et kui te juba sihukese jutuga tulete, siis vaevalt keegi 
teist seal õhtuses ka käima hakkab. Me ei hakka teie jaoks  
eraldi rühma püsti panema, käite ehitajatega esimese aasta koos koolis. Ja kui 
teisel aastal veel siin olete, siis vaatame seda asja. Kas nüüd osalt selle 
pärast või et dekanaadil oli õigus, nii või teisiti kukkusime 
sealt kõik robinal kolmanda kuu lõpuks välja ja läksime tööle. Nii 
et TPI on mul siiamaani lõpetamata. 

\question{Sa mainisid, et kirjutasid autobaasi softi. Kuidas 
sa seda tegema sattusid?}

Tol ajal \emph{start-up}-kultuuri ja ettevõtluse ehitamist veel ei 
eksisteerinud. Me lõpetasime kooli ajal, kui esimesi 
arvutikooperatiive oli väike käputäis. Minu esimene ametlik töökoht pidi 
olema tegelikult Noorsooteatri valgustaja. Kuna mulle juba 
tollel ajal meeldis audioga tegeleda, siis tahtsin sinna helimeheks minna, 
aga helimees oli värskelt tööle võetud ja valgustaja koht oli vaba. Paar päeva 
enne seda, kui pidin lepingu alla kirjutama, 
küsis Tarmo Mamers\index[ppl]{Mamers, Tarmo}, kas ma ei tahaks ikkagi päris 
tööd teha, kuna Skriining\index{Skriining} otsis programmeerijat. 

Nii sattusingi Skriiningusse Kalle Lotamõisa\index[ppl]{Lotamõis, Kalle} juurde 
tööle. Seal öeldi mulle esimese ülesandena, et \enquote{autopark on sellel 
aadressil}. 
Neil oli mingi eriti eksootilise asja peal jooksev andmebaasisüsteem, 
isegi mitte \emph{mainframe}, vaid mingi mini. Ja see tuli 
moodsale vahendile ümber kirjutada. Moodne vahend tähendas tol ajal Novell 
Netware'i\index{Novell} ja Paul Leis\index[ppl]{Leis, Paul} oli värskelt toonud 
Eestisse sellise asja nagu DataFlex\index{DataFlex}. Tegu oli päris 
 korraliku objektorienteeritud kõrgkeelega. Hakkasin ühest otsast õppima, 
kuidas DataFlexis programmeeritakse, ja 
teisest otsast, kuidas autopark töötab. 

\question{Ahaa, läksid kohe äriprotsessi ka sisse!}

Äriprotsessid olid seal paljuski olemas, st töötav tarkvara 
oli olemas. Pigem oli seal äriprotsesside seisukohast hea lastetuba, et ära 
kunagi eelda midagi. Näiteks mina oma IT-inimese mõistusega tegin oma 
arust mõned asjad paremaks ja siis selgus, et päris nii ei sobinud, nagu 
mina olin mõelnud. Raamatupidaja vaatas mind nagu idiooti ja küsis: 
\enquote{Kas sa ikka saad aru, kui palju ma pean numbreid siia päevas sisestama 
ja seda \emph{enter}'it, mille sa siia vahele toppisid, vajutama? Need arvud 
on neljakohalised. Ma sisestan neli numbrit ära ja 
need lähevad ise järgmisele väljale, mitte ma ei pea vajutama. Ma ei saa
vajutada \emph{tab}'i, mis on teises klaviatuuri otsas. Saad aru? Mul on ühes 
käes
paberid ja teise käega vajutan klaviatuuri. Kuidas ma sinna \emph{tab}'i juurde 
sinu 
meelest saan, kui mul on teine käsi kinni?} 

Nad olid väga innovatiivsed tegelikult selles mõttes, et nad olid 
sedasama andmetöötlust selleks ajaks juba aastat kuus-seitse kasutanud. See oli 
 meditsiinitehnika autobaas, Termak\index{Termak}, siiamaani elu ja tervise 
juures. 


\question{Kas nad olid juba nõukogude ajal arvutiasjandusega alustanud?}

Nad olid jah juba sügaval nõukogude ajal end täiesti ära automatiseerinud. 
Selleks 
ajaks, kui mina aastal 1992 sinna jõudsin, oli nende esimene IT-süsteem 
jõudnud moraalselt nii ära vananeda, et see tuli PCde peale ümber 
kirjutada. Neil oli siis juba \emph{legacy}, nad olid nii palju ajast 
ees.

\question{Kuidas Skriining jõudis selleni, et neil on programmeerijat 
vaja? Lihtsalt kasti sai ju ka edukalt müüa?}

Kalle\index[ppl]{Lotamõis, Kalle} hammustaski selle läbi, et kuna nad olid kogu 
aeg meditsiinitehnika ümber sebinud ja proovinud meditsiinisüsteemi arvuteid 
müüa, oli seal ka arendusvõimalusi. Nii saigi Skriiningust\index{Skriining} 
üheksakümnendate alguses arendusfirma. Arvutimüük käis ka, aga mina  
noore inimesena ei süüvinud sellesse, kust raha tuleb. Ilmselt päris palju tuli 
arendusest.


\question{Kas sa tehnikaülikoolis ka veel ringi hängisid?}

Ma hängisin seal pikalt, kuigi ma ei õppinud seal. Seal oli 
elu epitsenter, kuna seal töötasid kõik olulised inimesed: 
Mast\index[ppl]{Kaal, Madis} ülemisel korrusel, Tõnu\index[ppl]{Raimla, Tõnu}, 
Aare\index[ppl]{Tali, Aare} ja Tarmo\index[ppl]{Mamers, Tarmo} alumisel 
korrusel. Lisaks veel 
Martin Rinne\index[ppl]{Rinne, Martin}, Merle Alliksoo\index[ppl]{Alliksoo, 
Merle} ja kõik teised, kes hiljem MicroLinkis\index{MicroLink} lõpetasid. Tegu
oli sotsiaalse elu keskusega. 

\question{Mulle tundub see variant, et sa ei õpi, aga hängid, palju 
mõnusam, kui et õpid, aga ei hängi.}

Eks ma ise ikka soovitan teistele kool kohe 
ära lõpetada, sest pärast osutub see palju raskemaks. Mina ja mu sõbrad oleme 
hakanud 
neljakümnendates oma haridusega lõpuks tegelema. On 
tekkinud natuke rohkem vaba aega ja ka moraalne vajadus --
kuidas sa oled kõige väiksemate pagunitega mees ruumis \ldots

\question{Tol ajal ülikool kuigi palju praktiliselt 
kasulikku ei andnud. Tänapäeval on teistmoodi.}

Paljud ütlevad, et diplom ei olnud mitte tempel selle 
kohta, et tuled koolist välja targemana, vaid tõestus, et 
oled võimeline järjepidevalt, mitu aastat asjaga tegelema. See on pigem 
vastupidavuse ja hoolsuse proov kui koolitus.

\question{Räägi palun BBSidest. Kuidas sa selle \emph{node}'i ikkagi püsti 
said? 
Selleks tuli ju ennast kuskil registreerida?}

BBS oli varane arvutivõrk, mille mõte oli selles, et helistad 
kuhugi oma modemiga ja teises otsas on modem, kes vastab. Modemid saavad 
omavahel andmeühenduse ja siis saab teises arvutis, mille 
küljes teine modem on, ringi sobrada. Kusjuures tollal tõepoolest
sobrati, arvutiturvalisus oli pigem kokkuleppe 
küsimus. Üks suvaline BBSi omanik oleks võinud teise 
omaniku BBSi ilma mingi 
probleemita kaks korda tunnis neljaks tükiks lasta, aga seda lihtsalt ei 
tehtud. See oli nagu 
saarlase ukselukk: kui oled luku ukse ette paika pannud, siis kõik 
teavad, et sind ei ole kodus ja nii on. Ei ole vaja katsuda, kas uks 
on lahti või kinni, kedagi ei ole kodus. BBSidega turvalisusega oli sama lugu. 

BBSi teine ja palju kasulikum omadus oli see, et kui 
oli olemas modem ja arvuti, siis sai ennast FidoNeti 
\emph{node}'iks registreerida. BBS iseenesest ei eeldanud midagi sellist, vaja 
oli vaid
modemi ja vastava tarkvara olemasolu. Mingeid hämaraid teid pidi levisid 
telefoninumbrid, kuhu helistada ja end kohapeal ära registreerida.

FidoNet oli esimene üleilmne arvutivõrk selles 
mõttes, et modemid helistasid üksteisele automaatselt. See oli ka kaunikesti 
hästi toimiv elektronpostiteenus, mille üks eriline omadus 
oli veel see, et see liikus väljaspool KGB huviala. Eks küll 
kahtlustati, et seda kuulatakse pealt, ja aeg-ajalt mingid imelikud modemid 
üritasid sinu modemiga poole jutu pealt rääkida, aga üldiselt seda vist väga ei 
jälgitud. Ma vähemalt ei tea, et kellelgi oleks 
kaheksakümnendatel olnud modem-modemiga sidepidamisega probleeme, ei Eestis ega 
välismaaga. Mis on selles mõttes eriti huvitav, et kui kaugekõneliinid läksid 
nii palju lahti, et oli võimalik kuhugi automaatvalida, siis me ju helistasime 
igale poole välja.  
FidoNeti \emph{mail}, mis tuli Eestisse umbes aastal 1988 või 1989, oli esimene 
vaba ja 
demokraatlik sidekanal väljapoole.

Mina olin siis keskkoolis, esimese \emph{node}'i panin püsti umbes 1991. 
aastal. Ma olingi vist Aare \emph{point}. 
Omaenda \emph{point}'i numbrit ma enam ei mäleta, võibolla oli 
kaksteist-kakstest. \emph{Node}'i number oli
kolmkümmend viis. Eesti oli sel ajal ülemineku vabariik. 
Registreeritud postiaadress andis võimaluse foorumites 
kaasa rääkida. Eestis oli kümmekond gruppi, kus käis jutt erinevatel 
teemadel. Mõnes mõttes oli elu selline, nagu oleme täna 
harjunud, kuigi natuke teistsuguste tehniliste vahenditega. Post oli aeglasem 
ja 
saabus paar korda 
päevas, mitte reaalajas. Ei olnud nii, et kirjutan kirja ja see läheb kohe 
kõigile laiali. Samas täitis see kõik need ülesanded, millega täna tegeleme, 
ära. Nii et kaheksakümnendate lõpus, üheksakümnendate alguses oli see 
\enquote{ökosüsteem}, millega täna oleme harjunud, täiesti olemas ning 
väike käputäis inimesi Eestis omasid selle kasutamise privileegi. 

\question{Kas see väike käputäis olid pigem entusiastid, akadeemiline 
seltskond või kes?}

FidoNeti ökosüsteem koosnes sada 
protsenti entusiastidest. Akadeemilised inimesed läksid ärisse, panid püsti 
esimesed arvutifirmad ja üritasid raha teha. 

\question{Kas eksisteeris ka mõningane spetsialiseerumine, et siit saab 
tarkvara ja seal on huvitavaid jutte-raamatuid?}

BBSidel väike spetsialiseerumine oli, aga mitte eriti suur. Eks 
enam-vähem kõik proovisid endale kõhu alla korjata, mida vähegi said. 
See oli aeg, kus tekkisid esimesed suuremad kõvakettad. 
Lühikest aega valitses olukord, kus tarkvara 
oli vähem kui ruumi. Ruumi mõiste oli ka muidugi tollal huvitav. Kõige 
rohkem ruumi võtsid Sierra\sidenote{1979. aastal 
asutatud Sierra Entertainment (varem On-Line Systems ja Sierra On-Line) 
disainis paljud toonased hittmängud. Eriti populaarsed olid 
seiklusmängude sarjad \enquote{King's Quest}, \enquote{Space Quest} ja \enquote{Leisure 
Suit Larry.}\index{Larry (mängusari)}} mängud, mis olid flopiketaste peal. Neist suuremad, Space 
Questid\index{Space Quest} ja muud, tulid viie-kuue flopi 
kaupa. Mäletan, kuidas arutasime Eeroga\index[ppl]{Tohver, Eero}, et 
kui oleks võimalik panna kokku oma unelmate masin, siis kui suur kõvaketas sel 
peaks olema. Jõudsime järeldusele, et kui oleks umbes kaheksakümmend megabaiti, 
siis ilmselt jätkuks eluajaks, sinna saaks kõik mängud ja
tööasjad peale panna ning umbes pool jääks veel üle.

\question{Sierra oli omaette fenomen, seda mängiti palju. Kas keegi
seda müüs ka?}

Küsime laiemalt, kas Eestis üldse keegi tol ajal tarkvara müüs. 
Äritarkvara, nagu Novell, oli võimalik osta. Teoreetiliselt oli 
Windowsi või DESQview'd\index{DESQview}\sidenote{DESQview oli kaheksakümnendate 
lõpus ja üheksakümnendate algul populaarne tekstipõhine mitme{\-}tegumiline 
keskkond, mis toimis DOSi peal ja võimaldas korraga mitut programmi eri akendes 
käimas hoida.} kindlasti kuskilt võimalik osta. Aga peale Novelli serveri ja 
DataFlexi 
litsentside ei mäleta ma, et oleks üheksakümnendatel kellelgi 
legaalset tarkvara näinud. 

\question{Tuleme tagasi BBSinduse juurde. Kas selle sisu hulk, 
mida enda kõhu alla õnnestus kokku kuhjata, oli ka staatuse 
sümbol?}

Ma ei oska öelda, oskan ainult enda BBSide kohta rääkida. \mbox{Mina} 
korjasin kokku kõik, mida kätte sain, ja pakendasin ringi. See 
oli selline kultuuriküsimus, et tarkvara skaneeriti viiruste vastu 
kõige värskema skanneriga, mis parasjagu käeulatuses oli, ja see käis 
muidugi automaatselt. Siis lisasid arhiivi väikese faili, 
mis sisaldas sinu \emph{header}'it -- väikest 
failijuppi, kus oli graafiliselt (või tollal pseudo{\-}graafiliselt) sinu 
logo sisse punnitatud. Ja siis panid selle välja ja oma faililisti nupukese, 
millega tegu. 

See oli nagu \emph{basic housekeeping}. Kui sinu fail läks 
järgmisse BBSi, siis see viskas sinu logo välja ja pani enda oma 
asemele, \emph{tag}'iti ära nagu grafitiga, et see on 
minu käest tulnud asi. Vähemalt mul oli küll tunne, et välja läks 
kõik, mida olid ise endale mingil põhjusel hankinud. Mitte küll nii, et 
tõmbasid öösel HNSi\index{HNS} tühjaks ja 
panid enda lehekülje peale välja, küll aga mõned asjad, mille olid kätte 
saanud. Duplikaate ei olnud väga palju üllataval kombel.

\question{Tahtsingi küsida, et sedasi oleks pidanud ühel hetkel ju kõigil 
kõik olemas olema, aga seda siis ei tekkinud?}

Seda ei tekkinud. Kuna BBSid olid väga stabiilselt üleval, siis enda jaoks 
vajalikud asjad tõmmati
ära ja pandi omakorda enda juurde üles. Mõttetut \emph{leach}'imist ja püüet 
iga hinna eest oma failiandmebaas kõige suuremaks saada 
väga ei olnud. 

\question{Too mõni näide, mis laadi asjad sulle toona huvi pakkusid.}

Olin siis juba vihane \emph{nerd}, minu spetsialiteet oli 
programmeerimismaterjalid ja -vahendid, käsiraamatud ja
tööriistakesed. 
Kahjuks mul ei ole seda vana faililisti alles, sest kui ma Skriiningust ära 
läksin, lendas see vana SCSI-ketas, 
mille peal BBS jooksis, õhku. \emph{Backup}'i sellest ei olnud ja 
kogu FidoNeti \emph{node} koos failibaasiga läks hingusele.

Ma ise seda järgmisse kohta kaasa ei võtnud, sest läksin Skriiningust panka, 
kus 
olid ees sellised kõvad mehed nagu Mast\index[ppl]{Mast} ja 
Marx\index[ppl]{Marx|see{Kliimask, Margus}}\index[ppl]{Kliimask, Margus}, kes 
olid oma ökosüsteemi püsti pannud. Ühele BBSile seal rohkem ruumi ei olnud. 

\question{Mis panka sa läksid?}

Mina läksin sellesse panka, mille lõpupidu kohe 
kätte jõuab\sidenote{Intervjuu Andrusega toimus 2019. aasta novembri algul.} -- 
praegune Danske\index{Danske Pank}\index{Danske 
Pank|see{Forekspank}}, toona Forekspank\index{Forekspank|see{Eesti 
Forekspank}}. 

\question{Miks sa sinna läksid? 
Skriiningus said ju programmi kirjutada ja BBSi pidada.}

Nagu ma paljudesse kohtadesse olen läinud -- sellepärast et kutsuti. Ja 
kuna parasjagu jooksis Eestis teleseriaal Capital City, mis 
näitas panganduselu väga glamuurse \emph{highroller}'ina, siis mulle tundus, 
et mina tahan ka nii elada. Tuleb tunnistada, et üheksakümnendate panganduses  
ei pidanud väga pettuma, elu oli täitsa lill. Päris nii nagu 
teleseriaalis \enquote{Pank} elu meie majas küll ei käinud. 
Päris hulle pidusid sai peetud, aga et keegi oleks kokaiinise  
ninaga ringi käinud, seda mina ei tea. Meie kandis oli kokaiin täiesti 
tundmatu või ehk tehti seda salaja, mina küll
narkootikumidega pidusid ei näinud.

\question{Kas mäletan õigesti, et tollal tõmbasite panka 
püsiühenduse\sidenote{Enamik varasest internetiühendusest Eestis toimis kuhugi 
sisse helistades. See tähendas, et pidev side puudus ja side 
kvaliteet sõltus suuresti analoogtehnoloogial põhinevatest 
telefonikeskjaamadest. 
Püsiühenduseks kutsuti seda, kui asutusest jooksis füüsiline kaabel interneti 
külge 
ja kaabli olemasolu oli IT-inimeste unelmates kesksel kohal.} sisse?}

Püsiühenduse tõmbasime sisse väga konkreetsel päeval. 
Modemitega oli n-ö poolpüsiühendus juba pikemat aega olemas.  
Forekspank asus Rävala puiesteel, nagu 
juhtumisi ka KBFI\index{KBFI}\sidenote{Keemilise ja Bioloogilise 
Füüsika Instituut\index{Keemilise ja Bioloogilise Füüsika Instituut|see{KBFI}}
 (KBFI). 1979. aastal Endel Lippmaa\index[ppl]{Lippmaa, Endel} 
loodud teadusasutus, tuntud ka kui \enquote{Lippmaa instituut}. Just 
Lippmaade perekonna aktiivse ja laiahaardelise tegutsemise tõttu mängis 
instituut rolli paljudes toonastes olulistes protsessides (sh kohaliku 
interneti arengus).}. Baumaniga\index[ppl]{Bauman, Andres} 
oli läbi räägitud, kuidas internetti saab, ja meil oli suhteliselt 
rivitu ligipääs. Samas tundus ühel hetkel, et see võiks ikka päriselt 
permanentne olla. Võtsime Mastiga\index[ppl]{Mast} kaablirulli ja 
hakkasime üle Rävala puiestee katuste KBFI poole liikuma. Tähelepanuväärne oli, 
et see juhtus päeval, mil Eestit väisas esimest korda paavst.
\sidenote{Paavst Johannes Paulus II külastas Tallinna 10. septembril 1993.} 
Kõik katused olid snaipreid täis, kehtestati tohutu 
\emph{lockdown}, et keegi paavsti käigu pealt ära ei tapaks.  
Seletasime kõigile, et meil on vaja kaablit vedada ja paneme interneti 
püsiühendust. See oli maagiline valem, mis võimaldas ligipääsu 
kõikidele kesklinna katustele, ilma et keegi oleks midagi küsinud. Me küll 
otseselt snaiperitega samale katusele ei sattunud. Natukene tuli häkkida ka, 
et ühest koodlukust läbi minna, aga see ei olnud suur takistus. 

\question{Toona oli maailm järelikult teistsugune. Internet ei 
olnud veel kommertsiaalne, vaid pigem kogukondlik nähtus.}

Selle eest vististi keegi maksis ka kellelegi midagi, aga kui palju, 
seda jällegi ei mäleta. Eks see käis paljuski inimsuhete baasil. 
Kuna me tundsime Andres Baumani\index[ppl]{Bauman, Andres}, siis kuidas raha
seal tegelikult liikus, seda ma ei tea. Mast\index[ppl]{Mast} ajas seda asja. 
Millegipärast ma arvan, et maksime KBFI-le midagi. 
Tegelikult oli meil alates
üheksakümne viiendast aastast
Forekspangas\index{Forekspank} infotehnoloogiliselt selline elu nagu 
tänapäeval. 
Suhteliselt samal ajal tuli Mosaici\index{Mosaic}\sidenote{NCSA Mosaic oli üks 
esimesi internetibrausereid ja mängis WWW populariseerimisel olulist rolli. 
Sama meeskond lõi hiljem Netscape\index{Netscape} Navigatori, mis oli  
Firefoxi eelkäija.} brauser, hakkas veeb arenema ja 
tekkisid meile kõigile e-posti aadressid (need olid 
küll juba pisut varem KBFI kaudu korraks olnud, aga siis tekkisid need 
meie oma foreks.ee domeeni külge). Kogu see ökosüsteem, miinus Facebook, oli 
meil siis juba olemas. 

Tollal me ka täitsa tõsimeeli arutasime, 
et KBFI ühendus on ikkagi nii aeglane, et ehk peaks kogu 
veebi kohalikku serverisse kopeerima. Ja 
kuna see mahuks tõenäoliselt ühele DVD-le ära, siis ehk peaks tegema 
äri ja hakkama müüma internetiga DVDd. 

\question{Ka teistest intervjuudest käib läbi, et toonane maailm põhines 
suuresti 
inimsuhetel. Ometigi ei hakka inimesed arvutitega tegelema, kuna neile 
meeldib tegeleda inimestega. Samas tunduvad Eesti arvutiinimesed küllaltki 
suhtealtid ja -osavad. Miks see nii on?}

Kui inimesel on arvutihuvi, siis on ta
terve keskkooli ja pool ülikooliaega olnud sotsiopaat ning tal ei ole eriti 
olnud kellegagi millestki rääkida. Ja ühel hetkel leiab ta üles omasugused, 
samasuguste huvidega. Puhas \emph{nerd}'i ja nohiku käitumine, eks ole! 
Kui panna nohikud kõik ühte tuppa kinni, siis nad leiavad 
üksteist ja kõigil on järsku lõbus, sest kõik lõpuks ometi naeravad samade 
naljade üle. Pidudega on sama lugu. Kõige karmimad peod, kus ma 
olen osalenud, on ikkagi olnud inimestel, kelle igapäevatöö on kaunikesti 
\emph{boring}. Ma ei taha anda hinnangut teatud inimgruppidele, aga kui näiteks
 raamatupidajad ja andmesisestajad käima lähevad, siis see on 
ikka täiesti teine tase. Keskmised lõbusad inimesed on lõbusad 
kogu aeg. Aga kui nohkarid lõpuks lõbusaks muutuvad, siis juhtub asju.

Nii et see ökosüsteem toimis tänu sellele, et inimestel oli hea meel üksteist 
leida. 
Algul oli neid alla saja, 
võibolla isegi alla viie{\-}kümne inimese. Tegu oli uue 
laine arvutitegelastega, kelle seast suur osa meie tänasest 
\emph{start-up}-ettevõtlusest 
ongi välja kasvanud. Tänu tihedale suhtlusele hakkasid 
toimuma ka legendaarsed BBSummeri\index{BBSummer}-nimelised üritused. 

\question{Räägi lõpetuseks, mida sa praegu teed.}

See on võibolla masendav tõdemus, aga elu pole mind sellelt kursilt
kaugemale ega kuhugi mujale viinud. Laias laastus 
tegelen täna täpselt sama asjaga, millega kakskümmend viis aastat 
tagasi. Olen pendeldanud elektroonika ja tarkvara vahel, 
olnud mitme firma CTO, asutanud firmasid ja neid kihva keeranud, töötanud 
teiste juures ja endale. Ja kui keegi küsib, millega ma tegelen, 
siis tavaliselt ütlen, et annan masinatele hinge. 

\question{See on ilus ütlemine ja läheb kokku küsimusega, mis jäi enne 
küsimata. Tavaliselt inimesed tegelevad kas riist- või tarkvaraga, aga sinul 
tundub olevat üks jalg ühes ja teine teises?}

Mõeldes oma elu peale, siis ma muidugi tahaksin, et tarkvara oleks mu tõmmanud 
endasse. See on mõnes mõttes nii palju lihtsam ala. \mbox{Vigu} on palju 
lihtsam parandada ja katkiseid asju ei tule peaaegu üldse ära visata. 
Kettaruum ei maksa täna eriti palju erinevalt elektroonika valmistamisest ja
utiliseerimisest.

Mul on kuidagi juhtunud niimoodi, et kui panen 
tule vilkuma ja näen, kuidas minu tehtu manifesteerub päris asjades, 
siis mul läheb tuju paremaks. Mul tuleb elektroonika disain välja ka. Kuna ma 
olen ikkagi ka
programmeerija, siis olen sattunud sinna omamoodi side{\-}meheks. Ma suudan tõlkida 
riistvara tarkvara jaoks ja vastupidi. Selle konkreetne töönimetus on 
\emph{embedded engineering}. Vaadates, mis meil täna koolidest 
saabub, siis on see täiesti väljasurev kunst. Neid tegelasi, kes suudavad nii 
riistvara valmistada kui ka sellele tarkvara peale kirjutada, 
nimetatakse mehhatroonikuteks või kelleks iganes, aga fakt on see, et nende 
juurdekasv on järsult pidurdunud ja varem või hiljem hakkab see 
probleemiks muutuma. Tõsi küll, ka töömeetodid muutuvad. Me kasutame täna
töövahendeid, mis annavad näiteks tarkvaratiimile parema 
ettekujutuse riistvarast kui vanasti. Kirjeldused ja 
\emph{markup language}'id, millega seda tehakse, on paremad. Masinale hinge 
andmine tähendab seda, et kui sa näiteks lülitad oma pesumasina sisse, siis on 
oluline, mida see oskab või ei oska sinu 
heaks teha. Hea kasutajakogemus tuleb sellest, kui hästi raua ja tarkvara 
kooslus 
on välja mõeldud. 

\question{Sa ütlesid enne, et sa oled ka CTOna toimetanud. Järelikult tuleb
kolmas element juurde -- sa pead suutma selle kõik ka äriks tõlkida.}

CTO ametit on kaht sorti. Tavaliselt väikestes firmades tähendab 
CTO olek seda, et koosolekule on vaja kedagi kaasa võtta, ja kuidas sa 
ütled, et ta on mul programmeerija, eks. Sa pead talle andma 
visiitkaardi, millega ta näeb presentaabel välja. Väikefirma CTO 
teeb kõike, millel on tehnika maitse 
küljes. Suurema firma CTO tähendab, et ta ongi CTO. Tänases 
\emph{start-up}-maailmas on \emph{customer fit} ja \emph{market fit} kõva 
teema. 
Vanasti sellega väga ei tegeletud, aga nüüd, kus on tohutu kuhi 
investorite raha põlema pandud, ilma et sellest oleks isegi sooja saadud, on 
hakatud rääkima sellest, et toodetut peaks kellelegi päriselt ka
tarvis olema. See paistab olevat uus asi, viimase paari aasta 
paradigma. Kaks-kolm aastat tagasi hakkas Silicon Valleys 
pihta see kultuur, et laste kätte ei taheta raha enam hästi anda. Ehk 
nende kaheksateistaastaste imeettevõtjate aeg, kes suudavad väga suure kuhja 
raha 
korraga põlema panna, nii et sooja ei saa, on läbi saanud. Nüüd on selgunud 
innovatiivne lähenemine, et toodet peab 
kellelegi tarvis olema. See tähendab, et projektidele on erakordselt raske raha 
saada, sest kõik 
on järsku pirtsakas muutunud ja nõudnud, kust raha tagasi tuleb. 

\question{See läheb ju kokku sinu kunagise ettevõtte uksest sisse minekuga: 
seal pidid ka kohe kasulik olema ja ei tohtinud asju tuksi 
keerata.}

Kasumlikkus on tegelikult õudselt valus teema. Riistvaraga on  
asi selles mõttes selgem, et riistvara ei skaleeru, kui keegi seda ei osta. Sa 
ei 
saa valmistada sedasama \emph{recorder}'it, millega me siin praegu salvestame, 
miljon tükki, kui keegi ei osta. Sa lähed 
pankrotti. Tarkvara tiražeerimine ei maksa aga midagi. Ja täpselt samamoodi 
võib  
juhtuda, et tarkvara, millest mitte kellelegi mitte pennigi ei teki, on 
tegelikult väga kasulik. Seega kasulikkus ja ärimudel ei tähenda veel mitte 
midagi. Dotcomi- ja igasugu tarkusemullidega kipub tavaliselt juhtuma, et väga 
raske on tõmmata piiri selle vahel, kus asi ei teeni 
raha sellepärast, et väga head mõtet ei ole veel õpitud rahaks  
tegema, ja nende asjade vahel, mis ongi täiesti mõttetud. 
Seetõttu on väga palju tegelasi, kes suudavad maha müüa täiesti kasutu idee, 
öeldes, et tegu ongi monetariseerimiseelse faasiga ja see ei peagi midagi 
tootma. 
Unustades ära, et tegu on ühtlasi täieliku kräpiga. 
Viimasel ajal on tekkinud paar niisugust suuremat skandaali, näiteks
õnnetu Theranose \emph{case}, kus suudetakse endale nii veenvalt 
valetada. Terve ökosüsteem on üles ehitatud väga kasulikest 
asjadest, mille ainus viga on see, et fundamentaalne eeldus, millele süsteem 
rajati, oli täiesti vale. 

\question{Nii et selle kahekümne viie aastaga ei ole maailm väga muutunud,
aga toimib siiski natuke teisti?}

Üks asi on oluliselt erinev. Tollal valmistati tarkvara kahel põhjusel. 
Esiteks oli seda tarvis, mis tähendas tugevat kliendipoolset 
tõmmet. Teiseks taheti, et midagi sellist eksisteeriks 
maailmas, mis tähendab, et võeti lihtsalt kätte ja kirjutati tarkvara kas enda 
või teiste rõõmuks ning lasti maailma. Hästi palju väikesi ja 
kasulikke 
utiliite oli ju tegelikult kirjutatud kellelgi 
enda jaoks, siis pakendatud ja laiali saadetud. Eestis seda kontseptsiooni 
polnud, et teha tarkvaraga 
raha: kirjutada mõni vidin ja küsida selle eest tasu. \emph{Corporate} maailmas 
tollal 
küll juba osteti-müüdi igasuguseid raamatupidamissüsteeme väga 
edukalt ja see kõik töötas. Mujal maailmas tegeleti utiliitide 
pealt raha teenimisega ka väikest viisi. Eestis üldse mitte. Tänapäeval on 
tarkvara tootmine läinud niimoodi, et kellelgi tuleb mõni väga
hea idee ja ta tahab sellest teha raha tootmise masina. Asi on vastupidine: 
mitte 
vajadus-, vaid unistuspõhine. Nagu me 
aeg-ajalt Ivar Zaransiga\index[ppl]{Zarans, Ivar} naerame, et kui vanasti 
otsiti probleemidele lahendust, siis tänapäeva 
maailmas otsitakse probleeme neid vajavatele lahendustele. See on viimase 
kahekümne viie aasta jooksul kõige suurem paradigma muutus.

\chapter{Sergei Anikin}
%!TEX TS-program = arara
% arara: myindex

\index[ppl]{Anikin, Sergei}
\textbf{\enquote{Kuidas sina arvutite juurde said?}}

Ja see oli päris huvitav lugu. Ma olin, võib öelda, \emph{entitled} mu isa oli
elektroonikainsener töötas Kalinini tehases\index{Kalinini
tehas}\sidenote{Algselt Balti Raudtee Peatehased, mis ehitati 1870. aastal ja
mis aastatel 1902 kuni 1903 seal töötanud Nõukogude riigitegelase järgi 1940.
aastast alates M.I. Kalinini nime kandis. 2007. aastast alates asub samal
territooriumil ja osalt samades hoonetes Telliskivi Loomelinnak restoranide,
kohvikute, kontorite ja loominguliste ruumidega}, see koht, kus meil nüüd on
see kõige popim koht noorte seas. Seesama Kalamaja ja see Lendav Taldrik.
Tegelikult ma olen seal lapsena käinud koos isaga, seal oli valvur, valvurist
pidi läbi minema, et sinna territooriumile saada. Nad tegid rongidele
elektrimootoreid ja jõuelektroonikat, minu isa projekteeris neid. Aga
hobi korras ta on teinud igasugust raadiotehnikat ja mina ise olen proovinud
mingit väikest raadiot kokku panna. Kuigi mina olin täiesti võhik, selles osas kuigi
käisin raadiotehnika mingisuguses ringis.

Aga minu esimene arvuti sai siis minu isa poolt kokku pandud.

\textbf{\enquote{Aga kust ta jupid sai?}}

Isal  oli selline ajakiri Vene ajakiri nagu 
\begin{russian}Радио\end{russian}\sidenote{Igakuine populaarteaduslik 
raadiotehnika ajakiri, mida andsid välja Nõukogude Liidu Siseministeerium ja 
DOSAAF (\begin{russian}Добровольное общество содействия армии, авиации и флоту 
России\end{russian} - Vabatahtlik Venemaa armee, lennunduse ja mereväe 
abistamise selts). Ilmus eri nimede all alates 1925. aastast, 1975. aastal oli 
ajakirja tiraažiks 850 000 eksemplari}. Ja siis aastal 1986 avaldati seal 
kõigepealt arvuti skeemid ja siis, kuidas seda kokku panna. See oli 
Nõukogudemaal välja töötatud arvuti, aga skeemid nad võtsid selle ZX Spectrumi 
pealt\sidenote{Tegemist on arvutiga 
\begin{russian}Радио-86РК\end{russian}\index{Arvutid!Radio-86RK}, mis oli üks 
edukamaid  koduseks kasutamiseks mõeldud Nõukogude arvuteid. Kuigi Nõukogudemaal 
kopeeriti ZX Spectrumit usinasti, oli selle arvuti puhul siiski väidetavasti 
tegu originaalse disainiga, autoriteks Dmitri Gorshkov, Yuri Ozerov, Gennady 
Zelenko ja Sergey Popov (Stachniak, Zbigniew. "Red clones: The soviet computer 
hobby movement of the 1980s." IEEE Annals of the History of Computing 37, no. 1 
(2015): 12-23.)}. Isa siis kõigepealt korjas need komponendid kokku, ise 
joonistas plaadi ja kuna tal oli juurdepääs siis tehases tegi plaadi valmis ja 
pani selle kokku. Ma mäletan, et tal läks ikka paar kuud, enne kui ta kõik need 
vigased kohad seal ostsilloskoobiga välja juuris. Siis pani käima. See käis 
teleka  külge,  telekas oli monitori asemel. See oli mustvalge telekas, 
värvitelekat  meil peres ei olnud.  Aga ega ma sellega  midagi väga teha ei 
saanud, sest tal ei olnud isegi opsüsteemi. Tollel arvutil oli \emph{interface} 
kassettmakiga, aga meil ei olnud ka kassette, mille pealt laadida seda 
opsüsteemi. Selles samas ajakirjas oli baitkoodis opsüsteemi kood trükitud. 
Kakskümmend lehekülge bait baidi haaval. See oli talvel. Pimedad õhtud, ma 
põhimõtteliselt istusin kaks nädalat ja trükkisin need kõik koodid sisse.

\textbf{\enquote{Miks sa tegid seda? Normaalne laps ju ei toksi niimoodi 
pimedatel õhtutel baitkoodi?}}

Kas sellele on eellugu. Isa sõber tõi mulle umbes aasta enne seda lasteraamatu  
programmeerimisest. Seal  mingisugused  tegelased siis õppisid programmeerima 
BASICus\index{Keeled!BASIC}. Lugesin selle raamatu läbi, sain aru, kuidas 
programmi kirjutada, kirjutasin BASICus umbes kümnerealise programmi, mis 
midagi arvutas. Ja siis kompileerimise osas, me ei saanud ju paberi peal
kompileerida, ma näitasin sellele isa sõbrale ja siis ta vaatas, kontrollis,  
ja ütles, et see töötab küll.

Aga noh, see programm oli olemas, aga ma ei saanud proovida seda, et mul oli 
vaja arvuti tööle panna. Siis ma trükkisin need baitkoodid sisse ja  lõpuks 
sain oma programmi umbes aasta pärast sisse trükkida.

See sissetoksimine käis plokkide kaupa. Seal oli, ma ei mäleta, kui suur, aga 
umbes poole leheküljeline plokk, millel oli kontrollkood. Ma sain seda 
kontrollkoodi valideerida, kui see klappis, siis ma salvestasin selle makile. 
Kui ei klappinud, siis ma pidin viga otsima. Põhimõtteliselt ma pidin algusest 
peale selle ploki sisse toksima, sest selle vea leidmine oli väga-väga 
keeruline. Aga, aga ma arvan, et juba sellest ajast, mul tekkis esiteks 
kannatus ja teiseks tähelepanu detailidele. Selle baitkoodide sissetoksimisega 
ma lõpuks sain aru, et mul on hästi oluline kõik need õigesti ja õiges 
järjekorras sisse toksida sest ümber tegemine oli nii piinlik.


\textbf{\enquote{Sulle tehti väiksest peale selgeks, et sa võid küll üle jala 
lasta, aga siis sa ise toksid neid samu asju kolm korda}}

Ja, aga loomulikult, enamik aega mis arvutiga sai veedetud, olid  mängud. Tol 
ajal alguses arvutis olid need tavapärased madu ja mingisugune tennis. Sai neid 
mängitud. Siis isale meeldis arvuteid kokku panna ja ta on pannud ka sellesama 
ZX Spectrumi\index{Arvutid!ZX Spectrum}, isegi mitu tükki, kokku. Tegelesime ka 
selle väliskorpusega. Tol ajal vaata, Eestis on kuiv õhk ju talveti ja siis 
meil olid plastmassist õhuniisutajad, mis käisid radika peale. Sellest sai väga 
hea korpusse sellele arvutile. Ta oli õige kujuga, sinna sai sisse lõigata 
selle klaveri, toiteplokk, plaat kõik, mida vaja. Makk oli eraldi.

\textbf{\enquote{Miks sulle see elektroonika osa huvi ei pakkunud?}}

Ega mul ei olnud arvutite vastu mingisugust suurt kirge, siiamaani ei ole
tegelikult. Minu arust see on ikkagi vahend. Tänapäeval on ju
teada, et need, kes arvutitega tegelevate teenivad päris korralikult raha,
onju. Tol ajal see oli ka mingis mõttes staatuse küsimus, et sul peres oli
arvuti. Kui paljudel peredes
oli arvutid? Alles mitmed aastad hiljem tekkisid  need arvutiklubid või
arvutiga  mängida kohad. Aga mul oli kodus selline. Me ei olnud
jõukas pere, meile polnud raha et osta niisugust asja. 

Arvuti on jah, pigem vahend. Ta on meeldiv hobi ka, aga ma ei ole selline, et
see oleks ainuke hobi. Mingi aeg ma üldse ei tegelenud arvutitega, mulle see
mängimine enam kirge ei pakkunud ja programmeerida lihtsalt enda jaoks ei
tundunud väga huvitav. Aga mul oli üks sõber, me mängisime koos. Ja siis ta
mainis, et \enquote{hoo, et, et ma nüüd käin arvutiklubis. Ja seal õpime programmeerima,
aga mina muidugi enamasti käin seal mängimas}. Ja siis ma mõtlesin, et tema ju
tegelikult ei oskagi midagi. Et mina ju oskan ja peaks koos temaga minema. Sa
ilmselt oled rääkinud paljude inimeste Eesti kogukonnast aga, aga mina sattusin
siis Vene kogukonda. Selle arvutiklubi nimi oli Interface\index{Arvutiklubi!Interface}.

\textbf{\enquote{Aga kes seda klubi pidas ja kuskohas?}}

Selle vedaja oli, ma mäletan, naisterahvas. Ta töötas vist Bioloogia
Instituudis siin Mustamäe teel ja vedas laste arvutiklubi. Ta nii nimi oli
Nina Botina\index[ppl]{Botina, Nina}. Me käisime seal
Reaalkoolis\index{Koolid!Tallinna 2. Keskkool}\index{Koolid!Reaalkool|see{Tallinna 2. Keskkool}},
seal olid arvutiklassid, tundides.

\textbf{\enquote{Mis koolis sa ise käisid?}}

See on kool number kakskümmend kuus\index{Koolid!Tallinna 26. Keskkool}. 
Viimases klassis ma läksin Tõnismäe Reaalkooli\index{Koolid!Tõnismäe Reaalkool} 
kus oli väga tugev matemaatika. Ja tegelikult seesama Nina Botina surus mind ja 
veel ühte klassiõde, et me läheksime teise kooli, et lõpetaksime selle 
matemaatika klassi. Tema pärast me läksime sinna kool ja seal oli hästi palju  
tuttavaid sellest samast arvutiklubist.

Ja hiljem sellest arvutiklubist on kasvanud venekeelne tehnikakool või
arvutitehnikakool, mis oli Erika tänaval. 

\textbf{\enquote{Ma teadsin, et Tartu ja Tallinna vahel on erinevus. Aga 
selgub, et ka Tallinna sees on kaks täiesti isesugust Tallinna?}}

See on huvitav jah. Ja seal oli ka niimoodi, et minu huvi arvutite vastu 
vaheldus. Üks aasta ma olin seal klubis aga siis, kui ma läksin uude kooli, mul 
ei olnud aega, et  sellega tegeleda. Aga siis Nina kutsus, et kuule,  mul ei 
jätku instruktoreid. Tule, mul on uued grupid, tule ja aita 
arvutiklassis. Siis kuidagi  tekitas uuesti huvi. Kui ma kooli lõpetasin ja  
ülikooli läksin majandust õppima\index{Tallinna 
Tehnikaülikool!Majandusteaduskond}. Aga seal jälle  esimese aasta lõpus tekkis 
võimalus spetsialiseeruda majanduslikku andmetöötlusesse. See oli hästi pisike 
grupp, mingi seitse inimest. Kui kõik, kes olid majanduses, õppisid majanduse 
aineid, siis meie enamik meie tunde olid  arvutitehnika gruppidega.

Ja see meie grupp oli eestikeelne. Ma läksin venekeelsesse majandusteaduskonda, 
aga see grupp oli eestikeelne. Aga
see oli huvitav jällegi, et me ei pidanud õppima arvutitehnika baasaineid. 
Esimese aasta arvutitehnikas nad õppisid füüsikat-keemiat, kõiki neid üsna 
keerulisi ained. Ma olen kuulnud õudseid lugusid, kuidas inimesed ülikooli 
lõpuni ei saanud neid tehtud. Aga meie õppisime mikro ja makroökonoomikat, 
inglise keelt. Ja alates teises aastast hakkasime koos arvutitega õppima. Ja ei 
olnud erilist jõudluse vahet, tundus.

Aga ma jällegi mõtlen, et see, kus ma praegu olen, ilmselt on  ka  
põhjustatud sellest, et ma ei läinud väga süvitsi  arvutitehnikasse, vaid pigem 
alati oli arvuti mul  vahend mingi probleemi lahendamiseks.

\textbf{\enquote{Sa mainisid, et sul matemaatika tuli välja. Kas sa kuskil 
olümpiaadidel ka käisid?}}

Käisin, aga ma olin niisugune keskmine. See nii palju sõltub õpetajast, onju. 
Ma mäletan, meil oli kas viies või seitsmes\sidenote{Selle põlvkonna inimestel, 
nii vene- kui eesti koolides, jäi üks klassi vahele, sest koolid läksid 
kaheksakümnendate teisel poolel üle aasta võrra pikemale õppele}, minu meelest 
seitsmes klass, kus hakkas juba geomeetria ja muud sellised asjad. Ja siis mul 
kuidagi klikkis, et iga teoreemi kohta, mida meile räägiti, mul tekkis teine 
viis, kuidas seda tõestada. Ma kuidagi sain nagu aru, et ei ole alati ainult 
ühtemoodi, saab teistmoodi ka. Ja siis jällegi see klikib õpetajaga. Kui 
õpetaja näeb, et õpilane mõtleb, siis ta pöörab rohkem võib-olla tähelepanu 
inimesele. Aga noh, siis tema läks ära ja järgmised õpetajad ei olnud väga head.

Siis meil oli üks väga hea füüsikaõpetaja, tal oli hästi palju kontrolltöid. 
Tema juures ma õppisin seda, et üldse ei pea neid valemeid meelde jätma. Piisab 
sellest, kui sa oskad neid rakendada. Loomulikult spikerdamine ei olnud 
lubatud, aga mul ikkagi need valemid olid spikrina vihiku tagakaanel. Sa pead 
aru saama probleemist, pead aru saama, mis vahendeid kasutada selle probleemi 
lahendamiseks. Ja see õpetaja vaatas läbi sõrmede nende valemite peale, sest 
kui sa  probleemist aru ei saa, siis füüsikas lihtsalt valemid ei aita. 

Ja kui me läksime sinna Tõnismäe Reaalkooli, oli seal legendaarne 
matemaatikaõpetaja Mihhail Vassiljevitš\index[ppl]{Vassiljevitš, Mihhail}, 
siiamaani õpetab. See, kuidas inimene, õpetaja on ju autoriteet, kohtleb 
inimesi! Selles mata klassis, seal selgelt olid kolm või neli  tippõpilast, kes 
võitsid kõik riiklikud olümpiaadid  käisid maailmaolümpiaadidel. Loomulikult ta 
tegeles nendega, aga ta tegeles ka kogu ülejäänud rahvaga. Seal olid ka need, 
kes ei saanud väga aru aga tema juures need nende tase tõusis. Ta oskas 
selgitada ka keerulisi asju nii lihtsasti asju, et kogu klass  põhimõtteliselt 
oli paar taset teistest koolidest üle. Lihtsalt see, et sa olid seal  
keskkonnas juba tõstis sinul taset nii kõvasti.


\textbf{\enquote{Jällegi tuleb välja, et matemaatika tunnis õpiti lisaks 
matemaatikale ka suhtumist ja just see viimane on aastate järel meeles}}

No see olümpiaadide küsimus jällegi, Mina ei saanud seal mingeid kohti. Aga 
klassis, mis oli meist aasta vanem,
 oli selline lugu, et umbes kümme inimest läksid keemiaolümpiaadile, kümme 
inimest läksid matemaatika ja kümme läks füüsika olümpiaadile. Kõik need 
riiklikud olümpiaadid olid ju Tartus. Põhimõtteliselt terve klass läks 
olümpiaadile, aga erinevatel aladel. Ja kuna nad olid juba seal kohal, siis 
neil oli lubatud  ka teiste ainete olümpiaadides osaleda. Mille tulemusena nad 
kõikidel aladel, isegi need, kes ei kvalifitseerunud alguses, said enam-vähem 
kõik esikümnesse kõikidel aladel. Saad aru, see oli selline nii võimas klass, 
täiesti hämmastav.

\textbf{\enquote{Miks sa majandust läksid õppima?}}

Sest mu vanemad ütlesid, et meil on peres juba kaks inseneri olemas, ema oli 
soojustehnik. Eks ma mõtlesin minna kuskile mujale ka õppima, aga  kodu juures 
on palju lihtsam. 

\textbf{\enquote{Kas sul oli mingi ettekujutus sellest ka, mis sa pärast oma 
haridusega ette võtta tahad?}}

Ega ega mul väga ei olnud mingit ettekujutust. Ma arvan, et mul lihtsalt ei 
olnudki mingit plaani. Ma tahtsin lihtsalt näha, et mis  see majandus siis 
õigupoolest on. Suvel ma korra proovisin töötamist müügiinimesena. Selgus, et 
see ei sobi mulle absoluutselt. Sest mulle ei sobinud see, et müügitöös 
üheksakümmend kaheksa protsenti inimestest ütleb \enquote{ei}. Aga mulle see ei 
sobinud, mulle ei meeldi feilida ja minu jaoks \enquote{ei} tol ajal oli feil. 
Tegelikult nüüd kui ma olen siin Pipedrive'is\index{Pipedrive} juba seitse 
aastat töötanud, ma saan aru, et see on osa protsessist, on statistika, see ei 
ole feil. Et feil on see, kus sa ei tee seda üheksakümne üheksandat korda
müüki, mis võib õnnestuda. Müük on see, et sa tead oma neid statistilisi 
numbreid ja plaanid vastavalt nendele. Mitte ees see, et kui esimene juhuslik 
inimene ütleb sulle, et mul ei ole seda teenust vaja, siis sa oled feilinud. 
Tegelikult ei ole.

\textbf{\enquote{Mida sa müüsid?}}

See oli tänavamüük, tegelikult. Tänava peal müüsime erinevaid tooteid a la 
tööriistakaste (mis läksid tegelikult päris hästi), mingisuguseid 
elektroonilisi hambaharju ja nii edasi.

\textbf{\enquote{See on ju igavene raske töö!}}

See on väga raske töö ja  see süsteem oli niimoodi, et igal hommikul me tulime 
sinna, kus oli ladu ja saime päeva kvoodi. Et pead näiteks müüma viisteist 
tööriistakasti. Ja kui sa kvoodi täitsid mingisuguse kahe nädala jooksul siis 
sa said järgmisse tiitli ja selle tiitliga sa said endale õpilasi. Ja kui viis 
õpilast said kvoodi nii-öelda täidetud mingi aja jooksul, siis said oma 
nii-öelda äri. Aga, jällegi, õppetund oli see, et see töö ei ole kindlasti minu 
jaoks. Ja ma teadsin, et kui ma lähen programmeerijaks, siis ma saan oluliselt 
rahulikuma töö eest oluliselt suuremat tasu. See sundis mind mingi hetk, umbes 
pool aastat hiljem, ütlema, et \enquote{Okei, ma lähen nüüd}. See oli ülikooli 
teise aasta poole peal umbes, siis kui ma sellesse informaatika gruppi läksin.

\textbf{\enquote{Kas sa siis juba programmeerisid mingeid tõsisemaid asju ka 
või lihtsalt loengus puutusid kokku?}}

Olen teinud kahte projekti, mis nii palju kui ma mäletan, tõi natuke raha.

Üks oli selline. Tol ajal olid hästi populaarsed need 
sat-tv\sidenote{Kaheksakümnendate lõpus ja üheksakümnendatel oli suhteliselt 
lühike periood, mille jooksul isiklik satelliidivastuvõtja oli ületamatult 
kallis, piraatlusele vaadati läbi sõrmede (õigupoolest keegi ka sealtkaudu 
eriti ei vaadanud), suuri teenusepakkujaid veel polnud aga väikestel oli juba 
võimalus tegutseda. Siis pandigi mõne kortermaja katusele satelliiditaldrik, 
hangiti piraat-kaart tasuliste kanalite jaoks, veeti üle katuste 
ümberkaudsetesse majadesse kaablid ja asuti teenust müüma} firmad. Mõnes 
väikeses rajoonis oli mingis oma kunn, kes pakkus seda sat-TV-d kuutasu eest. 
Siis oli üks tuttav, kes palus teha infosüsteemi, kus oleks kirjas, kes on 
liitunud, kes ei ole ja kui palju nad maksavad ja mis teenust nad kasutavad. 
Emal oli jällegi tööl arvuti, mille peal ma sain teha Accessi\index{Microsoft 
Access} andmebaasi ja selle peale väikese liidese. Ma ei tea, kas ta kasutas 
seda hiljem või mitte.

Teine oli veel huvitavam. Kui ma sain teada, kui palju raha ma selle töö eest 
saan, ma olin väga imestunud. Isa sõbrad, tegelesid valvesüsteemidega. Neil oli 
projekt, vanglaprojekt. Nad panid valvesüsteemi vanglasse ja neil oli 
põhimõtteliselt vaja joonistada selle vangla projekti järgi mingisugune skeem, 
kus oleks näha, kus on alarmid tööle läinud. Ta ei olnud otseselt 
programmeerimine, ta oli rohkem disain või midagi sellist. Ma pidin neid pilte 
joonistama ja siis ma sain mingisuguse kolme nädalaga selle tehtud ja see 
summa, mis ma sain, oli mu isa umbes poole aasta palk. Siis ma sain aru, et, et 
arvutitega tasub toimetada.

\textbf{\enquote{Kust sa infot said? Ega Accessis programmeerimine ei ole ka 
niisama lihtne, et muudkui otsast hakkad tegema?}}

Kusjuures Accessi koht ma ei mäletagi, eks ma vist lugesin dokumentatsiooni. 
Programmeerimist õppisin 
raamatutest. Mul oli üks üks Pascali raamat, mis õpetas objektorienteeritud 
programmeerimist, venekeelne raamat. See aitas mul mõista, just 
objektorienteeritust. Ja ülikoolis tegelikult mõned ained olid väga-väga 
kasulikud. Näiteks andmebaaside projekteerimine. Tänapäeval väga paljud 
inimesed ei oska relatsioonilist andmebaasi projekteerida ja see ja see on üks 
vajalikumaid oskusi, tegelikult, kui sa tahad isegi lihtsat süsteemi kokku 
panna. Tänapäeval lahendatakse selliseid asju tihti lihtsalt jõuga.

\textbf{\enquote{Kas sinu arvuti või reaalainete huvi juurde käis ka 
mingisugune spetsiifiline, näiteks ulme, raamatu-huvi? Vene keeles oli ju palju 
rohkem asju kättesaadavad, mina ei olnud suuteline tol ajal Strugatskeid 
originaalis lugema}}

Ega ma ei mäleta, väga oleks olnud. Raamatuid mulle meeldis lugeda, mulle 
meeldis ka ulme nii-öelda või fantastika. Aga mul ei tekkinud nagu arvutitega 
seost. Minu jaoks arvuti on nii praktiline asi kui olla saab. Eks 
Bulõtšovi\sidenote{Kir Bulõtšov (1934 --- 2003). Nõukogude 
ulmekirjanik}\index{Kir Bulõtšov} ja Strugatskeid\sidenote{Arkadi Strugatski 
(1925 --- 1991) ja Boris Strugatski (1933 --- 2012), Nõukogude ulmekirjanikud. 
Kirjutasid enamasti koos, seega tuntud kui \begin{russian}братья 
Стругацкие\end{russian} või lihtsalt Strugatskid}\index{Strugatskid} aga ka 
välismaa asju. Aga ma olen ka kõik need Barbar Conani\sidenote{Robert E. 
Howard'i poolt 1932. aastal loodud tegelane, kes on sellest ajast tembutanud 
kõikvõimalikes meediumides ajakirjadest ja raamatutest filmide ja 
videomängudeni } ja Tarzani\sidenote{Edgar Rice Burroughs' poolt 1912. aastal 
loodud tegelane, kes Nõukogude Liidus sai tuntuks kinodes näidatud 
trofeefilmide (Johnny Weissmuller'i kehastatud tegelane erines küll oluliselt 
raamatukangelasest) kaudu} raamatud läbi lugenud.

\textbf{\enquote{Mis su esimene päris programmeerija töö oli ja millal see 
oli?}}

Mul olid sõbrad seal juba ees, veebruaris 1996 ma läksin tööle Aeteci
Finantsvara ASi\index{Aeteci Finantsvara AS|see{Profit Software}} mis nüüdseks 
on Profit Software\index{Profit Software}. Nad tegid soomlastele igasugu 
finantskindlustussüsteeme. Ma mäletan, et esimese oma tööülesandega ma 
feilisin, sest mulle anti mingisuguse
valemi programmeerimine. See pidi Cs\index{Keeled!C} olema ja sellest pidi 
\emph{library} saama. No mul ei olnud teadmisi. Ma ei teadnud, kuidas Cs 
kirjutada, ma ei saanud sellest valemist aru (see oli kõrgem matemaatika). 
Ühesõnaga, sellega ma feilisin. Aga milles ma olin väga hea, oli meil Lotus 
Notes'i\index{Lotus Notes Domino} tarkvara, mida kasutati suhtlemiseks omavahel 
ja soomlastega. See oli dokumendianadmebaas tegelikult. Tal oli oma 
skriptimiskeel ja sellega siis ma kirjutasin reisikindlustuse süsteemi 
kindlustusagentidele, et nad saaksid välja arvutada, palju see reisimine maksab 
ja saaksid poliisi teha. Ja see oli internetipõhine aastal 1997. Selle Dominoga 
oli võimalik, need samad dokumendid, mida sa muidu Lotus Notes'i enda kliendiga 
nägid, oli võimalik ka veebiserveri kaudu, HTML dokumentidena näidata.

Aga see kogemus aitas mul saada Hansapanga\index{Hansapank} internetipanga 
tiimi.

\textbf{\enquote{Kuidas sa sinna sattusid?}}

See oli ka naljakas. Tegelikult Hansapanga ITs või üldse pankades ilmselgelt 
oli rohkem raha kui mingis IT-firmas. Ja siis kaks aastat töötasin Aeteci
Finantsvaras ja tundsin, et võiks väikse nii-öelda karjääri teha. Ja tegelikult 
kõikidesse pankadesse proovisin tööle saada, et seal olid vabad kohad. 
SEBs\index{SEB|see{Ühispank}} võis toona Ühispangas\index{Ühispank} ma ei 
saanud isegi vist jutule, aga ma rääkisin Hoiupangas\index{Hoiupank} Aleksei 
Bljahhiniga\index[ppl]{Bljahhin, Aleksei}. Hansas oli ka tööintervjuu, läksin 
Vilve Vene\index[ppl]{Vene, Vilve} ja Heiki Kübbariga\index[ppl]{Kübbar, 
Heiki}. Ja siis ma mõlemast pangast sain tööpakkumise umbes sama summa peale. 
Otsustasin Hansapanga kasuks, sest arvasin, et seal on võib olla natuke rohkem 
karjäärivõimalusi nii-öelda. Minu esimene tööpäev Hansapangas oli 
üheksateistkümnes jaanuar 1998. Kui ma läksin sinna fuajeesse, seal oli värske 
Äripäev, kus oli kirjas, et Hoiupank ja Hansapank ühinevad. See minu ise 
esimene tööpäev oli sama päev, kus teatati ühinemisest. Ja see määras kogu minu 
järgmist nii-öelda karjääri.


\textbf{\enquote{See tähendab, et sa pidid suhteliselt ruttu hakkama 
internetipanga asemel tegelema hoopis Light Telleri nimelise telleri 
töökohasüsteemiga?}}

Sinna läks veel natuke aega. Ma arvan, et see otsus hakata seda tegema sündis 
umbes viis-kuus kuud peale seda kui ühinemine pihta hakkas. Sest alguses ju  
ei olnud veel selge, et kumba süsteemi üldse hakatakse kasutama ja kuidas see 
otsus tehakse. Ja sellel ajal mina õppisin siis kuidas internetipanka teha.

\textbf{\enquote{See kõik on mulle üllatus. Mina sisenesin sinnasamma panka 
1999. aasta lõpus. Light Teller oli selleks hetkeks olemas ja laua taga oli 
vana kala nimega Sergei, kes selle omakäeliselt valmis oli teinud. Kui nüüd 
näppude peal arvutada, siis see tähendas, et sa tegid nullist 
täisfunktsionaalse veebipõhise telleri töökoha umbes kolme kuuga?}}

No ega ma üksi ei olnud. Aga astume sammu tagasi. See Hansapanga esimene 
internetipank oli jällegi ehitatud tehnoloogia peale, mis oli ajast ees. See 
oli Oracle\index{Oracle} mingisugune veebi veebikomponent või veebiserver kus 
sa said PL/SQLiga\index{Keeled!PL/SQL} tekitada HTML mida siis kliendid 
vaatasid. See oli omal ajal hästi lihtne, mitte mingit disaini ei olnud, sest 
vist disaineritest keegi ei teadnud tol ajal, et on selline amet nagu disainer. 
Trükidisainerid kindlasti olid aga kasutajaliidese disaineritest mitte keegi ei 
teadnud tol ajal. Ja siis kui ma tulin siis vaatasin, et, \enquote{oo milline 
ebavõrdsus!}. Et et see internetipank on ainult eesti keeles. Kohe ütlesin, et 
noh, mis ta siin teha on, ma võin näiteks teha niimoodi, et ta on 
mitmes keeles. Öeldi, et tee. Ja siis ma tegingi. Kaks nädalat tegelesin 
sellega, et võtsin kõik tekstid välja, asendasin \verb|lang| funktsiooniga, mis 
arvestas ka kasutajaprofiiliga. Ma veel õppisin sama lõik ülikoolis, ja see 
päev, kui Madis Ollissaar\index[ppl]{Ollissaar, Madis} olin ma ülikoolis. 
Logisin siis sisse, et vaadata, kas töötab. Eesti keel töötab, inglise keel 
töötab, vene keel näitab küsimärke. Ilmselt siiamaani inimesed mässavad nende 
\emph{encoding}utega, aga see oli minu esimene kokkupuude sellega, kui minu 
arvutis töötab aga serveris ei tööta.

Aga aga samal ajal hakkas juhtuma ju mitu asja nii et, toimus ühinemine, 
Aleksei Bljahhin\index[ppl]{Bljahhin, Aleksei} tegeles \emph{data} migraga. 
Tekkis probleem, et telleri programm oli kirjutatud Oracle Formsi. Igas 
kontoris oli Formsi server. Ja kõik tellerid siis kasutasid Formsi klienti mida 
serveeriti sealt serverist ja nad võtsid peaserveriga Oracle 
andmebaasiühenduse. Oracle'i litsentsid, teatavasti, maksid ühenduste arvu 
pealt. Siis kujuta ette, et Hansapangal on, no ma ei tea, mingi nelikümmend 
kontorit äkki? Nüüdseks see on juba suur number, aga Hoiupangal oli nelisada 
kontorit Eestis. Ja paljudes maakohtades ei olnud isegi nii head ühendust, et 
hoida seda pidevat ühendust baasi otsa. Ja kui nad arvutasid, kui palju need 
Oracle litsentsid maksnud oleksid, siis nad ütlesid, et võib-olla anname selle
Hoiupanga tagasi. 

Siis tegelikult oli see hästi julge otsus. Ma ei tea, kes selle nüüd võtsid 
vastu, ilmselt needsamad Vilve\index[ppl]{Vene, Vilve} ja 
Gibbs\index[ppl]{Gibbs|see{Kübbar, Heiki}}. Aga otsus oli, et teeme siis 
interneti telleri programmi ja samal ajal meile müüdi uut tehnoloogiat 
internetipanga tegemiseks, BroadVisioni\index{BroadVision} nimeline platvorm. 
BroadVisioni müügiargumendiks oli, et me saame põhimõtteliselt e-kommertsi 
platvorm,  seal sai igale kasutajale näidata personaalselt välja nägevat 
rakendust.

Aga jällegi iga kasutaja maksis. Mis tähendas, et me kunagi ei kasutanud neid 
võimalusi, kõik oli anonüümne selle süsteemi mõttes. Aga ta pakkus 
\emph{template}'mise võimalust, mis oli väga suur samm võrreldes selle Oracle 
PL/SQLiga, kus sa pidid oma HTMLi ise kokku panema. Selle peale me ehitasime. 
Aga ma arvan, et see telleri arhitektuur, kui sa praegu mõtled sellele tagasi, 
ta oli võimas, aga ta oli hästi lihtne. See võimaldas tegelikult 
funktsionaalsust hästi kiiresti hästi suures koguses toota.

\textbf{\enquote{Selle arhitektuuri kohta ma tahaks küsida. Ta ju oma olemuselt 
oli toonaseid vahendeid kasutades täpselt selline, nagu täna \emph{de facto} 
veebirakendused on. JavaScript\index{Keeled!JavaScript} jooksis brauseris ja 
tegi \emph{backend}i poole päringuid. See lahendus oli oma 20 aastat ajast ees, 
kuidas ta sündis?}}

Seal oli seesama piirang, et maakontorite ühendus oli hästi aeglane. Ehk, me
pidime optimeerima, kui palju me \emph{data}t liigutame kliendi ja serveri 
vahel. See oli üks nõudmistest, mis sundis mõtlema, et me peame palju tööd juba 
kliendis ära tegema. Aga kliendiks oli brauser ja JavaScripti versioon oli 
selline, et parimal juhul sai mingit validatsiooni teha. Midagi joonistada väga
ei saanud või mingid midagi dünaamiliseks teha. Aga siis tulli samal ajal 
Internet Explorer 4.0\index{Internet Explorer}, kus olid \emph{custom} 
JavaScripti võimalusel brauseris, mis võimaldasid väga palju dünaamilisemat 
lehte ehitada. Ei olnud mingisuguseid JavaScripti \emph{library}sid, nagu 
Reactid\index{React} ja muud, mis võimaldavad kõike teha. Sa kirjutasid puhast 
JavaScripti, isegi Github'i ega Stack Overflow'd. Oli Internet Exploreri 
dokumentatsioon.

Ja siis veel üks nõue oli, et kõik need Hoiupanga töötajad on harjunud 
terminaliga, kus  hiirt ei olnud. Hiire kasutamine aeglustab tegelikult tööd. 
Ja siis nõue oli see, et sa pead saama navigeerida rakenduses ilma hiireta. 
Brauseris. 

\textbf{\enquote{Põhimõtteliselt ju tehtav, aga kasutajaliidese disaini mõttes 
päris keeruline ülesanne}}

Võttes kõiki neid piiranguid, ma pidin välja tulema mingisuguse kliendipoolse 
raamistikuga. Ja noh, eks ma siis tulin. Seal tekkis päris palju koodi ja tol 
ajal tüüpilises brauseri rakenduses vajutad \emph{submit} nuppu ja siis sul 
terve leht laetakse uuesti. Ja meil ei olnud seda \emph{bandwidth}i kontorite 
vahel. Kujutan ette, et sul on viis tellerit ja  nad istuvad selle 28K 
modemi\sidenote{Sidet üle telefoniliinide (ja just seda peetakse siin silmas) 
reguleerisid \emph{International Telecommunication Union}'i V-seeria 
soovitused. V.34 kirjeldas sidet kuni 33.6 kbit/s, kuigi levinuim oli just siin 
mainitud 28.8 kbit/s kiirus.} peal ja igaüks iga nupuvajutusega sulle hakkab 
tulema mingisuguse sadades kilobaitides lehte. Tol ajal jällegi tulid 
\emph{frame}d ja \emph{frameset}id\sidenote{HTML 4.0, mis avaldati 1997. aastal 
W3C soovitusena, sisaldas eraldi variatsiooni \enquote{raamide} (ingl. 
\emph{frame}) toega. Raamid võimaldasid HTML lehe jagada eri aadressidelt 
laetavateks alamosadeks. HTML 5.0 enam raame ei toeta}, nende vahel sai andmeid 
vahetada brauseri sees. No ja oligi üks \enquote{menu} \emph{frame}, kus oli 
enamik JavaScripti loogikat, mida kunagi uuesti ei laetud, ja siis oli see 
\enquote{main} \emph{frame}, mille sees siis laeti iga konkreetne tegevus.

\textbf{\enquote{Seal tehti veel mingeid huvitavaid asju, olid peidetud raamid, 
mis käitusid nagu praegune brauserist algatatud REST päring}}

Eks see arenes. Rakenduses oli \enquote{main} \emph{frame},  siis 
kliendiandmete \emph{frame} sest tavaline \emph{workflow}  oli selline, et 
klient tuli, sa leidsid tema konto ja siis sa said selle kontoga teha makseid, 
teha hoiuseid, mis iganes. Alati oli see, et otsid klienti, siis me laeme 
kliendi andmed eraldi kliendi raami, kus on nähtavad kliendi nimi, konto nimi, 
kontonumber. Aga seal all oli veel brauseri poole peal kliendiandmed. Ja siis 
meil oli \enquote{foori} \emph{frame}. Selle kaudu me \emph{submit}isime vormi 
andmeid, sest jällegi valideerimine pidi toimuma koha peal. Nupp käivitas 
valideerimismeetodi  ja valideerimismeetodi lõpus andmed saadeti teise vormi 
kaudu serverisse. Ma ei mäleta, miks me nii tegime, ju see oli vajalik. Aga see 
oli nagu raam, mille sees kõik pangafunktsioonid said tehtud. Põhimõtteliselt 
mul läks mingi kuu aega, et see kõik niimoodi püsti panna ja esimene 
eestisisese makse vorm ära teha. Kui see oli valmis, siis põhimõtteliselt kõik 
ülejäänud funktsioonid tulid mingi kahe kuuga. Põhimõtteliselt 
\emph{copy-paste}, seal ei olnud enam midagi keerukat. Eks seal pärast vigade 
parandamist ja optimeerimist muidugi oli ka, aga ei midagi keerukat.

\textbf{\enquote{Ehk, põhiline arhitektuur sai õigesti paika ja see töötas. Kui 
sa nüüd tagasi mõtled, mis sulle andis põhja, et selline asi teha? Oli see 
ülikool või lihtsalt häkkerimentaliteet või veel midagi?}}

Ma arvan, et ei olnud mitte mitte midagi peale probleemi, mida oli vaja 
lahendada. Mingid muud nõudmised, mis olid nagu \emph{hard} nõudmised, me ei 
saanud neist üle ega ümber. Tol ajal me tegime hanza.net'i\index{hanza.net} 
juba ja see oli värviline ja disaini mõiste oli juba olemas. Aga telleri 
rakenduse kohta oli spetsiifiline nõue et ta ei tohi väsitada inimest, et sa ei 
tohi kasutada erksaid värve, sest inimene teeb selle programmiga kaheksa tundi 
tööd. Ta oligi selline hall.

\textbf{\enquote{Sihukese asja peale isegi tänapäeval sageli ei mõelda, kust 
selline nõue tuli?}}

Meil oli ju tubli pangatehnoloogia osakond, kes mõtlesid, kuidas tellerid saavad 
hästi efektiivselt oma tööd teha. Ja jällegi ma ütlen, et mina olin ainult 
teostaja, seal oli terve tiim seal taga. Meil oli Toomas Rand\index[ppl]{Rand, 
Toomas}, kes tegelikult kirjutas kogu selle panga loogika, mina ju tegelesin 
ainult kasutajaliidesega ja andsin talle andmed. See, mis seal panga süsteemis 
toimus, oli tema teha et tema istus täpselt samamoodi kaksteist tundi päevas ja 
tegi. Aga tänu sellele projektile ka pangasüsteemi arhitektuuris tekkis 
korrastatus. Orcale Formsiga sa said kutsuda suvalisi funktsioone otse vormist. 
Aga meie arhitektuuri ütles, et üks nupuvajutus ja kogu tehing tehtud. Et see 
jällegi oli selline nõue pangasüsteemile. See õpetas, et liides ennekõike. 
Lepid liidese kokku ja siis osapooled saavad oma osaga  edasi tegeleda. See 
võimaldab sul testimist, testimise automatiseerimist, töö paralleliseerimist. 

Kitsendused tegelikult sunnivad inimesi tegema õigeid otsuseid. Ja me näeme 
järjest, et väga paljud inimesed ei oma kogemust sellistes piiratud  
ressurssidega olukorras toimetamisest. Eriti on seda näha välismaa inimeste 
puhul. Näiteks Silicon Valleyst inimene tuleb ja ta ei saa aru, et mis mõttes 
me ei palka juurde inimesi. Et ma ei saa ju kõiki oma ideid realiseerida, mis 
mõttes ma pean prioritiseerima? See on probleem nendele inimestele, nad ei saa 
aru, mis tähendab, et mul ei ole raha. Ma näen, et Eestis on tekkinud selline 
olukord, kus on palju ära tehtud hästi vähese ressursiga täpselt sellepärast, 
et inimesed oskavad teha õigeid valikuid, oskavaid prioritiseerima. Ja see on 
see oskus tuleb omakorda sellest, et sa pead alati prioritiseerime, sest sul ei 
ole ressurssi.

\textbf{\enquote{Ilusast arhitektuurist edasi minnes, milline on ilus kood?}}

Ilus kood on see, kus inimene ei pea küsima, mida see kood teeb. Väga 
paljud inimesed, kes oskavad programmeerida, millegipärast arvavad, et mida 
optimeeritum või lakoonilisem kood on, seda parem. Kuid see teeb halba. On piir, kust
edasi enam teine inimene ei saa aru, mida see kood teeb. Selline kood ei ole hea 
kood, isegi kui ta teeb õiget asja. See on üks asi. Aga teine asi on, et ma 
pean sulle suur aitäh ütlema selle eest, et sa tõid Eestisse Joshua 
Kerievsky\index[ppl]{Kerievsky, Joshua} omal ajal\sidenote{Joshua on USA firma 
Industrial Logic asutaja ja üks pikema kogemusega agiilse tarkvaraarenduse 
praktikuid ja koolitajaid maailmas. Tema Eestisse toomise Hansapanga arendajate 
koolitamiseks kas 2000. aasta lõpus või 2001. aasta algul algatas siiski Erik 
Jõgi\index[ppl]{Jõgi, Erik}}. Sul tekivad elus mingid hetked, kus sa saad aru, 
et see on nüüd \emph{step function}. Ja see koolitus (ta ei olnud pikk, nädal 
vist ja mitte täis päevad), mis me saime, lõi 
tegelikult väga paljud asjad oma kohtadele. Joshua on ju tegelenud koodi \emph{refactor}iga, 
 kuidas teha kehvast koodist ilusat. Ja me rääkisime temaga \emph{unit} 
testimisest \ldots.

See ongi, see, mis tegelikult aitab ilusat koodi kirjutada: sa 
pead seda mitu korda ümber kirjutama, enne kui ta loogiline välja näeb ja 
kood peab loogiline välja nägema.

\textbf{\enquote{Kui ma takkajärgi mõtlen, siis tolleks hetkeks kogu see 
agiilse arenduse liikumine, kogu see mõtteviis, oli veel väga noor}}

Me olime Skype'is\index{Skype}, ja siis ma tulin Pipedrive'i ja siin on meil 
igasugu \emph{agile coach}id. Ma mingi hetk ma mõtlesin, teeme eksperimendi. 
Oli mingisugune grupp. Seal olid \emph{agile coachid}, arendajad. Siis ma 
küsisin, et teeme eksperimenti. Rivistame grupi niimoodi, et kes on kõige kauem 
\emph{agile} liikumisega tegelenud  või vähemalt teadlik olnud. Ja enamasti, 
isegi \emph{coach}idel, oli see aeg mingi 7-8 aastat. Inimesed, ma olen 20 
aastat sellega tegelenud! \emph{Agile Manifesto} minu meelest oli 2001 või 
2002. Tegelikult me kõik saime seda maitsta enne, kui ta popiks muutus.

\textbf{\enquote{Mis sa praegu teed?}}

Ma isegi ma isegi ei saa öelda, et ma juhin \emph{engineering}u 
organisatsiooni, sest ma juhin ka muid organisatsioone nüüd 
Pipedrive'is\index{Pipedrive}. Ma olen siin juba seitse aastat olnud. Aastal 
2013 liitusin, see oli väike firma, ambitsioonikas. Tööintervjuul minult 
küsiti, et kas ma usun, et me saame Salesforce'iga võistelda. Ütlesin, et päris 
Salesforce'iks ei kasva, aga sihuke võib-olla veerand sellest on võimalik. Siis 
oli kakskümmend inimest, oli kümme inseneri. Ja nüüdseks, kuus ja natuke peale 
aastat hiljem, on meil on kuussada inimest.

Kõik need aastad olen tegelenud skaleerimisega. Nii infosüsteemi kui ka 
organisatsiooni skaleerimisega. Ja, jällegi, ei olnud kunagi sellist mõtet, et 
äkki meil ei õnnestu, äkki me ei kasva. Niipea, kui sa niimoodi hakkad mõtlema, 
sa ei kasva. Ma  siiamaani tegelikult ei ole kindel, et kumb on \emph{cause}, 
kumb on \emph{effect}. Et kas see, et me oleme skaleerinud \emph{engineering}ut 
aitas Pipedrive'il kasvada või see, et ta kasvas, aitas meil skaleerida 
\emph{engineering}ut.

Kui vaadata teisi osakondi, ütleme \emph{marketing} ei skaleerunud. 
\emph{Product} pidi skaleeruma koos \emph{engineering}uga, muidu inseneridel 
poleks midagi teha. \emph{Sales} ei skaleerunud, \emph{support} skaleerus 
nii-öelda natuke tagantjärgi. Et tegelikult \emph{engineering}u kasvatamine 
kasvatas firmat. Ma loodan. Aga samas, kui te ei kasvaks, siis me ei saaks  
inimesi juurde palgata, need kasvud on omavahel seotud. Aga küsimus ongi see, 
et mis tõukas seda. Ja ma väidan, et me nagu väga ei vaadanud. Me olime 
kindlad, et me peame skaleeruma, sest minu kõige suurem hirm on olnud, et  kui 
me jääme \emph{bottleneck}iks. Et kui \emph{engineering}u peale hakatakse 
näpuga näitama, et näed, me tahame või peame tegema seda ja toda 
\emph{engineering}ul ei ole ressurssi, või nad ei jõua või süsteemid hakkavad 
kokku kukkuma, kui kui kliente on liiga palju või me palkame inimesi juurde ja 
need inimesed ei saa tööd teha, sest kuskil protsessis on \emph{bottleneck}. 
Või me ei saagi inimesi palgata, sest inimesed ei taha meile tööle tulla. Neid 
pudelikaelu on nii palju, et ma pidi korraga kõikide nendega tegelemine. 

Kui kuidagi ei saa, siis kuidagi ikka saab.


\chapter{Arne Ansper}
\index[ppl]{Ansper, Arne}
\question{Nagu ikka alustame sellest, kuidas asjad alguse said. Kuidas nad siis 
said sinu jaoks alguse?}

No minu jaoks need asjad alguse sellest, et kui ma põhikooli lõpetasin siis 
minu matemaatikaõpetaja arvas, et ma peaksin minema Nõkku\index{Koolid!Nõo 
Keskkool} edasi õppima. Ja suutis mu vanemaid ära veenda, et see on suurepärane 
mõte, siis ma sinna läksingi.

\question{Aga kus sa põhikooli lõpetasid?}

Jõõpres\index{Koolid!Jõõpre kool}\index{Jõõpre}, selline pisike koht Pärnu 
lähedal. Sada õpilast oli see põhikool meil vanas  mõisas, mitte mõisamajas 
endas aga koolimaja oli mõisa keskel. Niisugune väga mõnus koht oli. Ja siis 
mul matemaatika nagu sobis ja õpetaja oli väga usin, andis mulle lisaülesandeid 
ja lõpuks saatis olümpiaadile ja seal läks ka suht hästi.


\question{Sa siis tulid puhtalt matemaatika ja mitte arvutite nurga alt sinna 
Nõkku?}

Ei, mul oli null kokkupuudet arvutiga enne. Vanemad seejuures pigem nagu 
tahtsid, et ma läheks. Ma ise olin väga  kahtleval seisukohal, et kas kodust 
nii kaugele minek, et see on äkki kuidagi raske ja paha ja nii edasi. 

\question{Mis aastal see oli?}

1985. 

\question{Sel ajal oli juba logistiliselt ju keeruline Pärnu lähedalt Nõkku 
saada?}

See oli lihtne ja tüütu, selles mõttes, et olid bussid, mis sõitsid neli tundi 
ja olid tavaliselt maast laeni rahvast täis ja siis veel Pärnust koju kus buss 
käis kahe tunni tagant. Seal ikkagi võttis aega, ütleme nii.  

\question{Ja Nõos pandi kohe arvuti ette?}

Ei, Nõos see oli tavaline keskkoolielu selle väikse vahega, et tuli ühikas 
elada. Mina olin viimane aasta, kes elas poiste ühikas, mis on selline 
suhteliselt raju ja legendaarne koht. Ehitatud kuskil tsaariaja lõpus, Eesti 
aja alguses. Talvel oli niimoodi, et tulid  kodust, tõid sihukesed suured 
märjad puunotid, läksid oma tuppa, mis oli  null kraadi lähedal kütsid ta siis 
üles selleks, et magada saaks. Hommikul lõid ikkagi pesukausi pealt jää katki, 
kui hakkasid hambaid pesema, niisugune koht oli. Esimene aasta oli hästi lahe. 

Alguses oli tavaline keskkond ja siis tuli programmeerimise õpetamise lihtsalt 
ühe regulaarse ainena sisse ja hakati õpetama. See oli ikkagi matemaatika ja 
füüsika kallakuga kool aga programmeerimise õpetamine seal oli lihtsalt nagu 
aine nagu mida iganes muugi. Mahud, loomulikult, olid suuremad nii 
matemaatikal, füüsikal kui ka sellel, programmeerimisel, millel mujal oli null, 
et seal oli siis nagu mingi muu number.

\question{Räägi palun Nõo kooli taustast, kuidas sinna üldse sai?}

Tead, ma ei tea. Mina olin tollal niisugune inimene, et emaga koos me sinna 
läksime. Ma arvan, et me käisime direktori juures rääkimas. Et kuna mul oli 
tegelikult olümpiaadilt mingisugune koht ette näidata siis kuidagi ma sinna 
igatahes sisse sain. Kuidas täpselt, kas seal oli mingi konkurss või mingi muu 
süsteem, ei tea. 

\question{Kes Nõo kooli direktor tol ajal oli? See kool tundus kellegi 
entusiasmi peal käivat?}

Enn Liba\index[ppl]{Liba, Enn} oli minu meelest tol ajal direktor\sidenote{Nõo 
kooli arendas selliseks reaalteaduste ja programmeerimise õppe keskuseks, nagu 
me teda praegu tunneme, Kalju Aigro\index[ppl]{Aigro, Kalju}. Ta oli kooli 
direktoriks aastatel 1951---1982, talle järgneski selles ametis Enn Liba}. Aga 
seda entusiasmi aspekti ja ajalugu, ma pean tunnistama,  ma ei oska 
kommenteerida tollal huvitusin  oluliselt muudest asjadest.

\question{Aga mis asjad need olid, millest sa huvitusid?}

Tegelikult mulle meeldis põhikoolis elektroonika. Aga see oli selline 
platooniline huvi, kuna juppe oli hullult raske kätte saada. Ja mulle meeldisid 
mudellennukid, mis oli ka suhteliselt platooniline. Aga Nõos tuli 
programmeerimine hästi kiiresti peale, kui hakkasime seal õppima. Seal oli suur 
Vene \emph{mainframe} Nairi-3-1\index{Arvutid!Nairi-3}\sidenote{1964. aastal 
Jerevanis välja töötatud Nõukogude arvutiperekonna Nairi kõige võimekam liige}. 
Kõps\index{Keeled!Kõps} ja Rops\index{Keeled!Rops}, eesti keeles sai 
programmeerida, need olid  vahvad. Siis olid seal Agatid\index{Arvutid!Agat}, 
mille ligi suht ruttu sai, mis olid teistmoodi vahvad, kus sai mingit 
valmistarkvaraga ka kasutada. Ikkagi mingite mängude mängimine oli oluline ja  
siis ise mingite asjade proovimine. See nagu hakkas väga kiiresti meeldima.

a \question{Oskad sa takkajärgi kuidagi reflekteerida, mis sulle seal meeldima 
hakkas?}

Väga ei oska, ausalt öeldes. Ma üritasin mõelda, et mis ma siis tegin nende 
arvutitega toona. Mul on umbes kaks asja meeles mida ma Agatiga tegin. Esimene 
programm oli umbes see, et oli \verb|for| tsükkel: muutis värvi, trükis mingi 
teksti nagu, ütleme, \enquote{tere}. Kõigis keeltes ja siis veel vilkuva 
taustaga ka. Sellega sai vähemalt üks õhtu kui mitte kauem möllatud ja timmitud 
neid efekte, tekste ja asju. Ja siis teine asi, mis mul on meeles, ma püüdsin 
ühte Nintendo mängu (need pisikesed puldi mängud, mis olid\sidenote{Arne peab 
ilmselt silmas Nintendo Game \& Watch seeria käes hoitavaid mänge. 
Originaalidest oluliselt rohkem oli liikvel nende Nõukogude kloone, mida müüdi 
Elektronika kaubamärgi all. Tegu polnud siiski alati täpsete koopiatega: 
Nintendo EG-26 kloonis IM-02 püüdis mune Miki Hiire asemel hunt tuntud 
Nõukogude multifilmist \begin{russian}Ну, погоди!\end{russian}}) taasluua, ma 
lõingi. Seal oli, nagu ta on, mingi fikseeritud arv positsioone, mingi tegelane 
liikus, mingid teised tegelased liikusid ja siis olid mingid surmasaamised ja 
mingid boonuste saamised. Probleem oli selles, et ma ei teadnud tollal, mis asi 
on massiiv. Põhimõtteliselt oli niimoodi, et iga objekti jaoks oli mul muutuja, 
mis ütles, et kas objekt on või ei ole. Ja kui seal mingid asjad liikusid, siis 
mul oli lehekülgede kaupa \verb|if| lauseid, et kui see muutuja omab seda 
väärtust, siis järgmisel sammul ta omab teist väärtust. Ja muidugi 
refaktoreerimis-tööriistu ei olnud. Kui ma kuskil vea tegin, siis ma nägin 
päevade kaupa vaeva, et ma nimetasin neid oma muutujaid ja \verb|if| lauseid 
ümber.


\question{Väga huvitav. Tol ajal tundus asjadest mitte rääkimine olevat 
õpetamise metoodika osa. Meile näiteks ei räägitud \texttt{for} tsüklist tükk 
aega}

Ütleme nii, et seda Agati\index{Arvutid!Agat} ei õpetanud meile keegi. Õpetati 
Kõpsi ja Ropsi. Kõik, mis Agati peal sai tehtud, see oli puhas enda välja 
võidetud ja võideldud  arvutiaeg, enda entusiasm. Ma isegi ei mäleta, \emph{by 
example} käis see asi vist, et vaatasid, mida keegi teine oli teinud. Mina küll 
ei mäleta, et oleksin ühtegi, Agati või BASICu\index{Keeled!BASIC}  kohta 
käivat raamatud lugenud kunagi. Kõik see oli lihtsalt nagu folkloor, 
katsetamise ja kõlakate tasemel. Et oleks keegi lekitanud selle info, et 
massiivid on olemas, oleks selle Nintendo mänguga palju rutem valmis saanud. 

\question{See oli suur töö ju, pidi ikka kihu olema?}

No aega oli palju, segavaid faktoreid oli vähe, eks ole. Ja ilmselt siis see 
arvuti alistamine meeldis, nagu välja tuleb. Agatiga\index{Arvutid!Agat} ma 
mäletan seda kindlasti, et ma hankisin endale selle 
assembleri\index{Keeled!Assembler} nii-öelda manuaali. Mis oli põhimõtteliselt 
paar-kolm ruudulist lehte, kuhu ma siis kirjutasin tähtsamad käsud ja registrid 
ja värgid üles ja siis studeerisin seda. Ja ma tean, et ma ikkagi nagu 
tuuseldasin seal Agati assembleri poole peal ringi. Aga mida ma tegin, seda ma 
kindlasti ei mäleta. Mäletan olulisimaid registreid, mida näppides käis piiks 
ja kust sai lugeda mingit vist klaviatuuri sümboleid või midagi sellist, aga 
\emph{that's it}.

\question{Kuidas Nõos tase oli, seal olid kõik sinusugused koos?}

Seal oli  selliseid inimesi, kes olid üle vabariigi kokku tulnud, kellel olid  
mingid huvid ja eeldused  reaalainetega tegelemiseks. Aga seal oli ka noh 
lähikonna inimesi. Et see on nagu päris, selline geto kuskil, see oli ikkagi 
nagu natukene spetsialiseeritud kohalik kool, et seal oli igasuguseid inimesi

\question{Kas sealkandis mingit äri tegemist ka juba käis, keegi raha eest 
programmi ei kirjutanud? Kaheksakümnendate lõpp ikkagi?}

Võib-olla keegi tegi, aga  ma julgeks öelda, et ma isegi ei huvitunud sellest 
ja ma ei tea sellest midagi. 

\question{Tartu vahet ka käisite?}

Jaa. Mingil hetkel, ma ei mäleta enam mis klassis, aga siis ma sain teada, et 
Tartu Ülikooli Raamatukogus\index{Tartu Ülikool!Raamatukogu} on mingisugune 
XTde\index{Arvutid!XT} klass. Kaheksa kuni kümme arvutit oli seal. Kuidagi ma 
sain sinna juurde, ma ei mäleta, mis alustel sinna seda aega sai reserveerida. 
Igatahes ma tean, et ma seal ikkagi jõlkusin päris mitu õhtut nädalas. Sa 
said seal mingisuguse tunni või kahese \emph{slot}i, mul oli umbes kaks flopit, 
millest ühe peal oli Turbo C\index{Keeled!Turbo C} ja teise peal oli siis tüüpi 
opsüsteemi oma asjad ja siis  midagi ma seal programmeerisin. 

\question{Aga kust sa said tolle Turbo C?}

Ma ei kujuta ette, kus ma selle saada võisin. Seal ma käisin päris tükk 
aega aga seal ma põhiliselt tegelesin ka sellega, et mängisin selle Turbo Cga. 
Aga kas mul ka mingi eesmärk oli, seda ma ei mäleta. Aga Turbo C see oli 
igatahes.


\question{On ikka paras hüpe Kõpsust ja Ropsust C ja mälu ja pointeriteni? 
Mille pealt too hüpe tuli sul?}

Jällegi nii kauge aeg, et ma kardan, et meile koolis ma isegi mäletan seda, mis 
meile üheksandas klassis  programmeerimist õpetati, aga ma ei mäleta, mis edasi 
sai, ausalt öeldes. Mida meil seal üldse räägiti. Ilmselt ise liikusin 
kiiremini edasi. Pärast TPIs\index{TPI|see{Tallinna Tehnikaülikool}} 
\index{TPI} ka see asi esimestel kursustel, et need  programmeerimise loengud 
olid  sellised, et sealt ei olnud midagi uut saada. Seal  mingid teised asjad 
olid pigem  need, mis olid uued, aga mitte see programmeerimise pool. 


\question{Kuidas sa sealt Nõost TPIsse\index{TPI} sattusid? Oleks ju loogiline, 
et sa lähed sealt Tartusse matemaatikasse?}

See oli ka suht \emph{random}iga selles mõttes, et ma mõtlesin, et võib sinna 
minna või tänna minna. Need argumendid, miks  Tallinnasse proovida, need olid 
niisugused väga otsitud ja õrnad, et miks ma just sinna Tallinnasse läksin 
proovima,  seda ma tegin. 

\question{Mida õppima?}

LI\index{Tallinna Tehnikaülikool!LI}. Ma täpselt ei mäleta, kas oli arvutid ja 
arvutisüsteemid, tõenäoliselt võis olla.

\question{See LI lühend jookseb mitmelt poolt läbi aga keegi ei tundu teadvat, 
mida see tähendas}

Kas ta üldse midagi tähendas? Et \enquote{L} on tõenäoliselt mingi 
automaatikateaduskonna kood, eks ole, ja \enquote{I} on mingi muu asja kood. 
Seal oli LA, mis oli äkki rohkem automaatika teisi tähti ei mäleta, äkki on LS 
ka olemas olnud. LI  oli jah see, kus mina oma aega veetsin.

\question{Sa ütlesid, et programmeerimise õpe sind väga edasi ei aidanud, kas 
seal üldse midagi õpetati, mis sulle midagi juurde andis?}

Tagantjärgi  vaadates tundub, et  seal LI-s räägiti nagu laiuti alates sellest, 
kuidas transistori teha, kuidas transistoridest saaks teha mingeid 
mikrolülitusi, kuidas saaks kõik see, mis sorti registrid meil on, kuidas 
registritest mingit automaatikat ehitada. Kuidas protsessorit teha, kui sul on 
neid registreid hulgi käes. Ja  teisele  poole minnes ka, kõik sellised asjad 
nagu siduteooria. Need asjad andsid, tagasivaates, need teadmised, et kui sa 
vaatad tänapäeval enda ümber, siis maagilisi asju, mille kohta ma ei tea, et 
kuidas seda saaks teha või ma pean uskuma midagi või ma vajaduse korral ei 
saaks sinna lõpuni välja kaevuda, neid on väga vähe. Ja see, ma arvan, on üks 
asi, mis mina olen leidnud, hästi kasulik. Tänapäeval on neid kihte sinna nii 
palju juurde tulnud, et vanasti oli ikkagi väga lihtne. See oli umbes nagu 
renessansiajastul, kui üks tüüp suutis  kõike, mida oli mõtet teada, teada. 
Natukene, kui  mina seal TPIs käisin, see aeg hakkas läbi saama. Ütleme 
niimoodi, et tänapäeval ilmselt ei ole võimalik, et sa tead kõike, mida oleks 
kasulik teada arvutiasjandusest. Ma mõtlen just tänapäeval seda, mis riistvara 
poole peal on juhtunud. Sinna on laotud neid kihte ja neid virtualiseerimise 
tasemeid ja mida iganes veel juurde. Ja siis \emph{soft}i poolel on ka vastu 
tuldud, sinna neli kihti virtualiseerimist vahele laotud ja nii edasi. See on 
nagu see, kus kipub nagu raskeks minema see järje pidamine.

\question{Kas TPIsse minek oli asjade loomulik käik või oli sul mingi plaan ka, 
mida tegema hakata?}

Mul niisugused pikaajalisi plaane ausalt öeldes ei olnud. Mulle meeldis teha, 
mulle meeldis nende arvutitega mässata, kas ma mässan Tallinnas või mässan 
Tartus, vahet pole. Ja siis ma mässasin nendega Tallinnas. Üks huvitav nüanss 
on veel see, et et umbes seal keskkooli lõpus ma sain isikliku arvuti ka. See 
oli midagi teistsugust, see oli Atari 520 STf\index{Arvutid!Atari 520 STf}. Mis 
oli siis Atari Motorola 68000 prosega tükk. 512kB oli tal mälu, selle ma 
\emph{upgradesin}  ühe megani mingil hetkel. Selle peal ma siis elasin ja 
siis selle peal ma püüdsin nagu süvitsi minna kogu sellega, mis seal nagu teada 
oli. 


\question{Kust sa sihukese aparaadi said kaheksakümnendate lõpus?}

Mul olid vanaonud, kes elasid Rootsis. Ema ja isa ükskord käisid seal ja siis 
sealtkaudu ma selle siis sain. 

\question{See pidi Agati kõrval ikka ulmeline aparaat olema}

Tegelikult oli niimoodi, et teised olid PCde peal. Kui ma nüüd vaatan, siis 
need inimesed, kellega me siis igal pool nagu koos ringi käisin, siis noh 
üheksakümnenda aasta paiku umbes, normaalsed inimesed said PCdele ligi ja siis 
toimetasid nendega.  Ja siis minul oli kodus Atari  ja tegelesin sellega 
põhiliselt. 

\question{Ataril on kihte vähem, sai lihtsamini sügavale välja minna}

Jaa, see oli nagu hoomatav täiesti,  mis seal toimus, midagi väga ulmelist 
polnud. Natuke mängisin ka, aga mitte liiga palju. Mul ikka see 
programmeerimine meeldis kõige rohkem selle asja juures. Selle Atari peal ma 
tegin igasuguseid imelikke asju.

Ma üritasin CAD programmi teha, joonistamisprogrammi. See isegi lõpuks selles 
mõttes töötas, et seal sai teha ringe ja jooni, igast värke, salvestada ja 
laadida. Ja siis mul oli, tagasi vaadates jälle hullumeelsus, et mulle nagu 
kohutav tegi muret see, et mälu saab otsa. Et kui sa teed dünaamilist 
mäluhaldust, eks ole, et siis saab mälu otsa. Üritasin seda siis minimeerida. 
Näiteks mulle tundus, et nagu lokaalsed muutujad, mis on \emph{stack}is, on 
kuradi ebaefektiivsed. Ja sisuliselt see CAD programm oli kirjutatud 
sajaprotsendiliselt globaalsete muutujate otsa. See oli täiesti hullumeelsus 
nagu tagasi mõeldes, seal tuli ikka kõvasti refaktoreerida, sest ma ikkagi panin 
täitsa puusse alguses. Seal seda loll ümberkirjutamist oli nii palju, sealt ma 
sain selgeks, et okei, nii ma mitte kunagi rohkem ja mitte ühtegi asja ei tee. 
Väga-väga palju vigu sai igatahes tehtud.

\question{Eks see on ju õppeprotsess, mõnda asja teoreetiliselt selgeks ei saa}

Jah, absoluutselt nõus. Ütleme, et nii võimekaid inimesi, kes kogu aeg teiste 
vigadest õpivad, et neid väga palju ei ole. Ikka enamus kipub oma vigadest 
õppima. 

\question{Kui sa TPIsse\index{TPI} jõudsid, kas sa seal teisi omasuguseid ka 
kohtasid?}

Meil oli hästi lahe kursus. Aga tegelikult oli niimoodi, et seal TPI ja alguses 
ma ikkagi õppisin, eks ole. Mis sest, et seal programmeerimise vallas mul ei 
olnud väga huvitav, aga neid muid ained ma ikka õppisin korralikult. Ma olen 
ikka väga usin õppur olnud. Ja mul juhtus niisugune asi, et mind 
Tarvi\index[ppl]{Martens, Tarvi} kutsus ühel hetkel Ektaco-sse\index{Ektaco}, 
ma arvan, et see oli üheksakümmend üks aasta. Ja see oli siis see 
\emph{community}, kus ma siis hakkasin nagu inimestega koos olema ja oli siis 
ka töise karjääri algus. Ma arvan, et see võis olla, see võis olla 1991, aga  
sada protsenti kindel ei ole. Mingi kolmas kursus äkki umbes.

\question{Kolmas kursus on üsna hilja ju?}

Tegelikult ongi see, et programmeerimise õppimine, üldse arvutiasjanduse 
õppimine võtab ikkagi aega. Ma tagasi vaadates mõtlen, et mis ma siis tookord 
oskasin või kuidas ma mõtlesin või  kuivõrd hästi ma siis programmeerisin.  
Ütleksin, et palju varem ei ole mõistlik seda tööd üritada teha. See võib  
frustratsiooni tekitada. Mis mul oli, ma olin ikka viis aastat nüüd innustunult 
selle asjaga tegelenud. Ma arvan, et kui ma  tööle sain, siis ma olin ka noh, 
enam-vähem miinimumtasemel, kus oleks  mõistlik, et keegi annab sulle 
ülesandeid, millele on ka mingi tähtsus ja tähendus ja sa teed nad  ära.

\question{Kas sul midagi sellist ei olnud, nagu inimesed on rääkinud, et 
lihtsalt arvutiaja saamiseks tekkis mingi arvutiklassi admini koht?}

Ei, mul ei ole ju midagi taolist. Ütleme tõesti mälu võib olla natuke petab, et 
mis aastal mul see Atari sinna täpselt tekkis, aga mul kuidagi oli alati 
mingisugune võimalus olemas, nii palju, kui mul seda tarvis oli ja sellest 
piisas. 

\question{Oskad sa mõnda näidet tuua, mida sa seal Ektacos alguses 
programmeerisid?}

Ektaco oli niisugune  firma, kus tehti riistvara ja tarkvara. Ta tegi 
tööstuskontrollereid, automatiseeris tehaseid, eks ole. Ja olid need 
sardsüsteemid, seal on väiksed mikroprotsessorid neid oli vaja programmeerida 
ja need programmaatorid olid kallid. Ja siis Ektaco hakkaks tegema oma 
programmaatorit. Põhimõtteliselt mingisugune lisaseade PCle, millega sa saad 
neid kivisid kõrvetada. Üks teine tüüp, kes oli nagu riistvara poole peal (ma 
ei tea, aga ma arvan, et ta oli umbes nagu mina, värskelt laekunud staatuses) 
ja mina tegin siis softi. See oli selles mõttes nagu päris huvitav, et meil oli 
PC/AT platvorm, seal oli ISA siin ja selle arvuti me süstemaatiliselt kogu aeg 
ajasime ikka täiesti lukku. Ja selleks, et saaks mingit sotti, siis meil oli 
seal siuke äge asi nagu loogikaanalüsaator. See on niisugune aparaat, et kui 
Ostsilloskoobiga saab visualiseerida mingit analoogsignaali, siis 
loogikaanalüsaatoril on palju-palju pisikesi klemme, mis sa paned kuskile prose 
või mingite digitaalsignaalide külge. Siis sul on teine arvuti mis  
visualiseerib, et kuidas need signaali mustrid on ja siis sa saad panna 
\emph{triggereid}, et kui mul tekib selline muster,  siis salvesta ja 
taasesita. Ehk et kui me ajasime selle selle PC täiesti hulluks, siis me saime 
sealt loogikaanalüsaatori pealt pärast vaadata, et mis siis juhtus, et mis me 
valesti tegime. Ühesõnaga tema siis tegi riista ja kirjutas siis sinna 
kontrolleri peale programmi ja mina kirjutasin PC peale siis põhimõtteliselt 
draiverite programmi vastu, mis omavahel suhtlesid. Ja siis tegin sellele ka 
kasutajaliidest.

Meil olid igasugused Inteli ja IBM-i \emph{manual}id laua peal, neid me siis 
seal sobrasime ja dekodeerisime, et mis me peame nüüd tegema, et siit 
midagigi läbi läheks. 


\question{See kõlab kuidagi hästi süsteemse ja korraldatud ettevõtmisena?}

Ei, see oli hull häkkimine. Nojah, Ektacos seda kraami, mille abil nagu häkkida, 
seda oli ja meil meil oli võimalus seda kasutada. Ja tegelikult ma tõesti selle 
teise tüübi  tausta ei tea, et võib-olla tema oli  kuidagi kogenum, tema tuli 
ju loogikaanalüsaatoriga sinna laua taha. Aga see oli suhteliselt niisugune 
kasulik ja kergesti omandatav seade, et noh kuidas sa seda pruugid. 

\question{Jah, aga võrreldes sellega, kui (nagu on räägitud) inimesed 
vaibanoaga emaplaadi pealt radu maha kratsisid, et modem tööle saada on tegu 
ikka \emph{high-tech} häkkimisega}

No me tegime ikka sinna radu juurde selleks et see kuidagi tööle saada, me ei 
kratsinud midagi maha! Mina ise seda riista-poolt tol ajal ei puutunud. Ehkki 
meil Ektacos programmeerija töövahendite hulgas oli kindlasti tinutus kolb, et 
nii raua lähedal oli seal see enamus sellest elust. 


\question{Kas te saite tööle ka selle kupatuse?}

Ja, loomulikult. Ja siis sellega seoses muidugi, kuna  see oli veel see aeg, et 
 see Borlandi\index{Borland}\sidenote{Borland Software Corporation oli 1983. 
aastal asutatud ja eri nimede all siiani toimetav tarkvaraettevõte, tuntud 
eelkõige arendajate töövahendite poolest. Neist kuulsaimad olid 
\enquote{Turbo-} eesliitega keeled Assembler, BASIC, C, C++, Pascal ning hiljem ka Delphi}
toodang, igasugused Turbo-blaahid, mis neil olid, need olid nagu  standard, eks 
ole. Siis loomulikult sai kirjutatud oma akendussüsteem, mis nägi välja nagu 
see Borlandi Turbo Vision\index{Turbo Vision}\sidenote{Borlandi poolt 
üheksakümnendate alul arendatud tekstipõhine kasutajaliidese raamistik Pascali 
ja C++ jaoks}, aga oli hoopis parem ja teistmoodi tehtud ja seega töötas väga 
kenasti. 

\question{Milles see väljendus, et ta parem oli?}

Ta oli nagu ägedamini struktureeritud. Siis mul hakkas juba 
C++\index{Keeled!C++}  meeldima, ta oli hullult objektorienteeritud. Tal olid 
mingid oma kontseptsioonid, et kuidas sa neid aknaid ja asju  esitad, kuidas sa 
sündmusi käsitled  selles mõttes, et sul on klaviatuur ja hiir. Mingi asi on 
fookuses, kuidas need sündmused jõuavad õige objektini, ja see on  klaviatuuri 
ja hiire puhul väga erinev loogika. Ja kõik see oli selliseks loogiliseks 
kompotiks keeratud, et sinna oli lihtne rakendusi teha. Sellel tükil oligi 
umbes üks programm, mis  seda ägedat raamistiku kasutas, see oli see sama 
programmaatori kasutajaliidese. Aga noh, selles mõttes oli Ektaco väga tore, et 
need tööülesanded ei olnud väga piiravad. Sa võisid ikkagi, ma ei tea, 
kuude või isegi aastate kaupa rahulikult häkkida ja sealt lõpuks tuli mingi asi 
välja. 

\question{Ja teistpidi, ega sul ei olnud neid akende joonistamise asju võtta 
riiulist kümneid?}

Ei, ikka oli. Sedasama Turbo Visionit oleks võinud pruukida ja seal oli 
igasuguseid teeke. Aga kuidagi, mis see siis on, nagu ametiuhkus ei lubanud 
teise mehe akna teeki kasutada. Tuleks ikka enda oma teha, sest et no mis 
mõttes, ma ei oska nüüd parimat akendusteeki teha. 

\question{Sellist suhtumist pannakse tänapäeval pahaks? Või ei panda?}

Seda tehakse teisel tasemel, eks ole. Tasemeid on juurde tulnud, seal 
nokitsetakse hoopis mingisuguste muude asjade juures, aga mina arvan, et see on 
nagu suht paratamatu, et see on hädavajalik, et inimesed heas mõttes 
leiutatakse jalgratast. Teeks asju, mis on juba tehtud, aga teeks teistmoodi, 
teeks paremini. Põhimõtteliselt olid ju opsüsteemid olemas, et mis mõte oli 
seda Linuxit hakata tegema, PC-Unix oli olemas. See oli olemas, et no mis siis 
häda oli sellel SCOl või millel iganes. 


\question{Jah, põhimõtteliselt oleks ju võinud olla, et siiamaani kõik 
kasutaksid sinu aknategijat}

Kindlasti need inimesed, kes on armunud kaheksakümmend korda kakskümmend viis 
teksti ekraanisse, need oleks olnud siiamaani selle andunud kasutajad. 

\question{Mäletan, FoxPro\index{FoxPro} joonistas lausa mingeid varjusid 
akende taha}

Ja, see on loomulik, varjud akendel pidid olema.

\question{Kas seda teie kiibikõrvetajat kasutati väljaspool 
Ektacot\index{Ektaco} ka?}

Need asjaolud muutusid nii kiiresti, et see, mis oli kallis ja kättesaamatu 
kaks aastat tagasi,  kaks aastat hiljem ei olnud enam seda. Ja ma arvan, et 
seda võib-olla tehti mingi üks või kaks eksemplari ja seda pruugiti Ektaco 
siseselt, aga sellest mingit edulugu ei tulnud. Ja see ei olnudki põhitegevus. 
Mina jälle ei tea, eks ole, et miks seda üldse tegema hakati, kas tõesti oli 
siis nii kättesaamatu või lihtsalt oli äge seda teha.  

\question{Jah, kui ma sind kuulan, see ei kõla suurepärase ärina}

Ektaco tegi ju  äri ka. Ja ma pean tunnistama ausalt, et  mind huvitas tollal 
programmeerimine. See, et mida  kolleegid nagu tegid, ma teadsin, aga ma väga 
ei süvenenud sellesse. See oli hästi selline fokusseeritud toimetamine.

\question{Kas tol ajal tekkis mingi kokkupuude arvutisidega ka juba?}

Seal Ektacos oli mul terve hulk toredaid kolleege. Olid 
Tarvi\index[ppl]{Martens, Tarvi}, Heiki Kask\index[ppl]{Kask, Heiki}, Jaak 
Niit\index[ppl]{Niit, Jaak}, Gunnar Valge\index[ppl]{Valge, Gunnar} oli seal 
minuga samas toas, kindlasti oli veel paar-kolm inimest. Ja siis meil oli 
Fido\index{FidoNet} \emph{point}, mis siis tekkis jälle seal Tarvi ja Heiki 
initsiatiivil, minu meelest ennekõike. Me olime alguses Lõvi point. 
Lõvi\index[ppl]{Lõvi|see{Lepp, Andres}}\sidenote{Lõvi, pärisnimega Andres 
Lepp\index[ppl]{Lepp, Andres}, on legendaarne TPI arvuti-mees, paljude meie 
põlvkonna inimeste sõber, teejuht ja eeskuju} oli siis TPI 
Arvutuskeskuses\index{Tallinna Tehnikaülikool!Arvutuskeskus}. Minu jaoks oli ta 
kunn, ma ei tea, mis ta seal tegelikult oli ja siis olime seal Lõvi 
\emph{point}. Jooksutasime seal FrontDoori\index{FrontDoor}\sidenote{FrontDoor 
oli üks populaarsemaid FidoNeti mailereid} ja mida iganes me jooksutasime. 

Ma arvan, et mingil hetkel me \emph{point}i staatusest \emph{upgrade}sime 
ennast \emph{node}ks. 71 oli meie number, julgeks arvata. Ja me helistasime 
kuhugi sisse ka, sest ma mäletan, et ma olen mingisuguse \emph{prompt}i otsas 
rippunud. Ja vaat seda jälle ei tea, et kust ma sain teada, mis käskudega seal 
Unixis\index{Unix} midagi teha. Ja kuidas mingi binaarne fail ära 
\emph{uuencode}da, selleks et ma saaks seda üle terminali endale 
\emph{dump}ida, selle \emph{dump}i salvestada, oma masinast \emph{decode}da ja 
mingit zipi sealt seest kätte saada. Kuidagi ma teadsin seda, kuidagi ma 
mingisuguseid asju imesin. Aga see on jälle niimoodi, et mingid asjad olid nagu 
õhus nagu mingisugused hallitusseene eosed laiali. Nii, kui kusagil pinnase 
sai, kohe läks kasvama. 

\question{Nii mitu sammu selleks, et midagi kätte saada, barjäärid olid jube 
kõrged toona.}

Info ikkagi liikus, see, ma arvan, ei olnud probleem. Küsimus oli ikkagi 
ennekõike riistvaras ja \emph{access}is ja  telefoniliinides ja niisuguses 
kraamis. Modemid olid ju roppkallid asjad, eks ole. Arvutid,kõik oli roppkallis 
välja arvatud aeg. Töö juures õnneks meil mingeid modemid olid, mitte küll 
kõige härjemad. Meil oli mingi 2400 ja MNP5\sidenote{\emph{Microcom Network 
Protocols (MNP)} on perekond (tähistatud numbritega ühest kümneni) 
veaparandusprotokolle, mida sageli kasutati varastes kiiretes (2400 bit/s ja 
rohkem) modemites} oli see meie lagi, millega me seal alguses toimetasime siis. 
Aga kõik olulised asjad liikusid ikka flopide peal, seda ei viitsinud keegi 
ära tõmmata, tõmmati mingeid pisikesi nublakaid. Tollal oli flopiga bussi peale 
minek reaalselt kiirem kui modemiga toimetamine.

\question{Mis sorti materjali te oma nodes hoidsite?}

Point oli meil puhas Fido point. Meil minu meelest küll BBSi ega midagi olnud. 
Meil oli ikkagi sõnumivahetus, \emph{Echomail} ja \emph{Netmail}, ehk siis 
privaatkirjad ja niisugused avalikud foorumid. See oli see, miks me nii-öelda 
suures pildis seda \emph{node}i pidasime. Kui keegi midagi tõmbas, siis ta 
tõmbas enda jaoks ja võib olla jagas  kolleegidega kuidagi midagi aga meil 
mingit sihukest varamut või niisugust ei olnud.

\question{Kellega te neid meile vahetasite, mis uudisgruppe lugesite? Kogukond 
ei olnud ju suur? Lõviga sai ju niisama ka juttu rääkida, ei pidanud kirja 
saatma?}

Mina lugesin põhiliselt \emph{Echomail}i, mul mingisuguseid kirjasõpru, kellega 
mingeid asju seal väga oleks olnud ajada, et tegelikult väga ei ei olnud. Minu 
jaoks oli see lihtsalt nagu foorum, kus sa saad huvitavat ja enamasti ka väga 
humoorikat  sisu. See väljendustase, see, kuidas inimesed, ükskõik mis teemal, 
viitsisid oma mõtteid sõnastada, need iroonia, sarkasm, huumor, kõik need 
tasemed, see oli niivõrd hea tekst valdavas osas, et seda oli  alati lust 
lugeda. Ükskõik mis oli, mingid autofoorumid, mul polnud  sooja ega külma 
nendest autodest. Aga lihtsalt need naljad, need vihjed, see oli lihtsalt hea 
meelelahutus, enamuses. Muidugi seal on ikka programmeerimised ja riistvara ja 
kõik muud teemad ka. See oli kasulik ja naljakas.

\question{No aga skaalal Tolkienist üle autode C++-ni?}

No kõike, absoluutselt. Kogu elu oli seal minu meelest. Seda jaksas tervikuna 
läbi lugeda sest inimesi oli vähe, palju sa ikka seda head kvaliteetset sisu 
suudad toota. Seda  oli vähe tegelikult, mis seal liikus minu meelest.

\question{Ühesõnaga, praeguses mõistes oli võimalik kogu sisuloomel silm peal?}

No sellel, mis Fido \emph{Echomaili} kaudu tuli, jah. Seal kuskil paralleelselt 
hakkasid arenema mingeid \emph{newsgroupid}, ka Eesti omad, millega mina 
alguses eriti ei puutunud  kokku. See oli natukene teine seltskond minu 
meelest, kes seal nii-öelda internetimaailmas hakkas toimetama. 

\question{Need olid kaks eri maailma, nende vahel mingit silda ei olnud?}

Nii ja naa, kontseptsiooni mõttes olid interneti uudisgrupid ja Fido omad 
samad, aga seal olid mingid ebamugavad erisused. Kunagi  hiljem, kui ma 
Ektacost Küberneetika Instituuti läksin\index{Küber|see{Küberneetika 
Instituut}}\index{Küberneetika instituut|see{Cybernetica}} siis ma tegin oma 
\emph{node} Solarise\index{OS!Solaris} peale. Meil oli seal üks 
SPARC\index{Arvutid!SPARC}\sidenote{\emph{Scalable Processor Architecture 
(SPARC)} on Sun Microsystems'i poolt arendatud RISC-arhitektuur. Sun müüs 
sellele arhitektuurile tuginevaid, siinmail populaarseid, servereid ja 
tööjaamu} server ja siis ma ajasin seal peal käima kogu selle Fido softi. Üks 
venelane oli selle kirjutanud. Ja siis ma tegin \emph{news}i \emph{gateway}, 
mis nagu Fido Echomaili \emph{newsgroup}ideks köitis kahesuunaliselt ja siis 
ühtlasi ka Netmaili siis tavaliseks meiliks köitis. See oli päris popp, ma 
isegi ei mäleta, millal see maha sai võetud. Ma arvan, et seal juhtus see asi, 
et sellele Solarisele oli lõpuks vaja  korralik \emph{upgrade} teha ja siis ma 
ei viitsinud vist enam. Fido oli ära surnud selleks hetkeks ja siis ma tõmbasin 
ta maha. Aga mingil ajal  oli ta hästi popp, mul oli seal, ma arvan, ikkagi 
sadu sadu kliente oli oma personaalse \emph{account}iga seal minu \emph{news}i 
serveri küljes, kellel oli siis nii-öelda kirjutamisõigus Fido gruppidesse. 
Fidos oli see korrapidamine nagu olulisem, seal ei olnud sellist anonüümset 
kasutust, keegi vastutas alati kellegi eest. Keegi  kuskilt kaudu sai 
\emph{access}i ja kui see keegi oli nõme, siis see \emph{access} võeti talt 
ära. Kui ma hakkasin seda asja Newsi \emph{gate}ma, siis ma lubasin sedasama 
teha, eks ole. Ma ei andnud kellelegi Fido gruppidele  kirjutamisõigust, kui ma 
ei teadnud, kes ta on ja ma ei saanud seda \emph{access}i talt ära võtta. 

\question{Aga see on ju, ütleks, autoritaarne?}

See toimis. See oli nagu  endale olulise keskkonna  normaalsena hoidmise 
eeldus. Teistmoodi ei saa. 

\question{Aga mis on \enquote{nõme}?}

No, solvad teisi inimesi, trollid, ütled puhasti, eks ole. See ongi põhiline, 
et kui sa lähed isiklikuks, teed teisele haiget, halba emotsiooni, sihukest 
asja ei ole vaja. See kui sa vaidled, see on okei, seda peab olema, see ongi 
tähtis, eks ole. Aga sa ei tohi  teistele haiget teha. 

\question{Kõlab nagu lihtne, eluterve ja samas fundamentaalne definitsioon. Aga 
kui sa Ektacost ära tulid, kas sa veel õppisid?}

Ei, mu õppimised olid selleks hetkeks õpitud või noh, mitte päris lõpuni 
õpitud, aga ma olin oma inseneridiplomi kätte saanud, vist 1993 või 1992. Sain 
oma kraadiselt kätte. Magistrikraadiks \emph{upgrade}sin ma ta siin natuke 
hiljem. Mina õppisin, viis aastat, sain süsteemiinseneri diplomi, aga pärast 
hakati  kogu seda kraadi värki järjest lahjendama, eks ole. Kui nüüd õppeaastad 
järjest lühenesid, siis \emph{by default} oli mul bakalaureus, aga siis ma 
pidin veel natuke juurde õppima ja tegema magistritöö, ma saaks magistriks. See 
oli kunagi seal 2001. aastal umbes, kui ma selle  ette võtsin. 

\question{Aga tol hetkel sul ei olnud sellist tunnet, nutikas ja usin õppur 
nagu sa olid, et peaks teadusmaailma sukelduma?}

Ega ma seal TPIs ise teadusmaailma suurt kokku ei puutunud. Kuna ma sealt poole 
pealt hakkasin programmeerijana tööd tegema, eks ole, siis see  haaras  
enam-vähem täielikult. Ma arvan, et lõpus läks kas see õppimine natukene 
nigelamaks, sest töö juures oli palju huvitavam ja palju nagu väljakutseid 
pakkuvam. Viimased asjad mis seal Ektacos\index{Ektaco} sai tehtud, oli 
kontrollerite uue sideprotokolli disainimine. Ma olin hullult vaimustatud 
TCP/IPst ja  siis ma trükkisin välja kõik standardid, mis ma sain: TCP, IP, 
Etherneti. Aga kontrollerid on mingi 8051 peal, mis on umbes nagu, nagu väga 
väike. Aga siis ma lugesin need RFCd kõik läbi ja siis ma tegin mingi oma 
sideprotokolli, mis inspireerus siis kõigest: Ethernetist, IPst ja TCPst. Ehkki 
ta ei olnud nagu päris \emph{flow}le orienteeritud, aga pigem  selline 
\emph{datagram}i-põhine protokoll. Sihukesed vanad riistvaraässad Ektacos  olid 
väga nördinud ja solvunud, et mis mõttes ma kirjutan protokolli, mis ei ole 
deterministlik. Mitte \emph{master-slave}, vaid igaüks võib traadi peal 
lobiseda, kui mõte pähe tuleb, ja siis lahendatakse konfliktid ära ja tehakse 
re-transmissioon. Nad olid väga pahased minu katsetuste peale, aga ma arvan, et 
programmeerisin selle lõpuks sinna ära ja ta mingil määral töötas ka. See oli 
päris äge.

\question{Aga mille vahel see protokoll siis käis?}

Põhimõtteliselt oli see, et PC, mis siis juhtis neid tööstusarvuteid. Neil oli 
sihuke karp, mille nimi oli satelliit, mis oli siis tööstuskontroller, millel 
olid igasugused digi- ja analoogsisendid-väljundid, mis kuskil tehases keerasid 
mingit nuppu, et betooni teha või midagi. Ja siis sellel olid mingid
juht-programmid ja neid tuli konfida. Tüüpiline värk, eks ole. Sa pead teadma, mis 
sul tehases toimub. Sa pead käske andma, selleks on mingit võrku vaja. Ja neid 
satelliidikontrollereid võis seal korralikus tehases ikka palju olla. Ja siis 
ta tuli PCsse kokku tõmmata ja ma usun, et keegi kirjutas siis mingit softi 
sinna PC poolele, mis siis neid satelliite siis jälgis ja juhtis.

\question{See kupatus oli päriselt \emph{production}is ja Eesti Vabariigis 
tehti betooni niisuguste seadmetega?}

Jaa. Ma arvan, Palivere Ehitusmaterjalide Tehas\index{Palivere 
Ehitusmaterjalide Tehas} vist oli see, mis oli ära automatiseeritud Ektaco 
poolt ja ma millegipärast arvan, et midagi oli Tallinna 
Veepuhastusjaamas\index{Tallinna Veepuhastusjaam}.Aga seda ma väga kindlalt ei 
tea, aga seal oli neid veel. Neid objekte ikka oli.

Mida mina tegin, see oli järgmine generatsioon, need objektid juba töötasid 
mingisuguse muu protokolli ja mingi muu  tehnika peale, aga kõik see kasvas, 
eks ole. Ja siis algatati uue generatsiooni satelliidi väljatöötamise projekt, 
kus mina siis  protokolli kontributeerisin ja realiseerisin. 

\question{Kui sa võrgundusest juba nii palju teadsid, sind kuhugi interneti 
varasesse maailma ei tõmmatud kaableid vedama või midagi?}

Ei, mulle meeldis programmeerida. Nende muude asjadega ma tegelesin nii palju, 
kui nad olid kasulikud ja vajalikud selleks, et saaks midagi ägedat 
programmeerida. 


\question{Ja Küberis\index{Küber} sai ägedamalt programmeerida?}

Lõpuks jah. Jälle Tarvi\index[ppl]{Martens, Tarvi} kutsus mind sinna. 
Küberneetikasse oli tehtud infotehnoloogia osakond, mis peitis seda infot, et 
tegelikult tegeldi seal infoturbega ja siis oli seal mingi riiklik programm, 
mille eesmärk oli Eesti riigi infoturbe ja krüptograafia vajadusi rahuldada. 

\question{See oli juba enne, kui tekkis AS Cybernetica?}

Jaa, see oli enne seda. Mina läksin sinna 1994, aga see töögrupp tehti 1993, ma 
arvan. Ja siis seal oli terve hulk nutikaid inimesi koos, kes siis  selle 
missiooni elluviimisega tegelesid, et  kompetentsikeskust ehitada.

\question{Kes selle taga oli? Keegi pidi ju selle tellimuse formuleerima, et 
riiklikult on tarvis tegeleda krüpto ja infoturbega?}
Ülo Jaaksoo\index[ppl]{Jaaksoo, Ülo} oli siis Küberneetika 
Instituudi\index{Küberneetika Instituut} direktor. Minu vaates oli tema see, 
kes seda kõike lõi ja korraldas. Kuidas ja  kellega tema läbi rääkis või kust 
see mandaat tuli, seda mina ei oska küll öelda. Aga tema oli jah, kellel see 
visioon  oli, et seda on tarvis. 

\question{Arvestades, kui vähe vajas Eesti riik krüptot ja infoturvet praegu ja 
kui strateegiliselt oluline teema see praegu on, siis sellise visiooni jaoks on 
ju tarvis väga ägedat ettenägemisvõimet?}

No aga kaugemale vaatamine ongi teadlaste ja akadeemikute ülesanne. Kust mujalt 
see tulla saab? 

\question{Visioon visiooniks, mida see töö toona praktiliselt tähendas?}

Esiteks, ise õppida. Teiseks, teisi õpetada. Eestikeelne terminoloogia, 
standardid, profiilid, seminarid, koolitused mida iganes.  Ja  teistpidi 
hakkasid niisugused praktilised asjad tulema. Vaata, tollel ajal maailm oli 
nagu väiksem, ka krüpto ja infoturbemaailm oli väiksem ja mingil hetkel on 
ikkagi veel võimalik hoomata  kõike, mis oli oluline. Mitte küll päris üksi, 
aga sihukese väikese töögrupi sees nagu meil oli. Ma arvan, et mis meil  väga 
hästi läks, oli see, et meil olid inimesed, kes  tegelikult  huvitusid just  
sellest infoturbe süsteemsest poolest. Et mitte see, et mis on nagu see 
tehnika. Aga mis on see organisatsioon, need inimesed, need reeglid, eks ole, 
seadusandlus seal ümber. Ühesõnaga süsteemne lähenemine valdkonnale kui 
tervikule, mis on  väga tähtis ja  mis sellest meie grupist välja kasvas. 

Teiselt poolt oli see, et meil on seal sihukesed \emph{hardcore} häkkerid ja 
\emph{hardcore} krüptograafid, kes nagu olid valmis mida iganes tegema. See 
sümbioos oli minu meelest hästi lahe. Ma arvan, et minu esimene töö 
Küberneetika Instituudis oli see, et ma pidingi riigiasutustele kirjutama 
juhendi, kuidas KA9Q\index{KA9Q} otsas ehitada endale internetti ruuter. 

\question{Mille otsas?}

KA9Q on üks soft. \enquote{KA9Q} on mingi radistide kutsung, mis vastab mingile 
inimesele, kes selle softi kirjutas, on minu arusaamine. Ja see oli DOSi peal 
jooksev \emph{all singing all dancing} asi, mis realiseeris TCP, kõikvõimalikud 
sideprotokollid, võrgukaartide toed, SLIP, PPP, ruuterid, mida iganes. FTP 
deemonid. Täiesti müstilisi asju on tehtud maailmas.  Et kui sul oli üks  
üleliigne PC, modem, võrgukaart ja see soft, siis sa said teha endale ruuteri, 
millega oma organisatsioon kuhugi ära ühendada. Ja siis mina peksin selle käima 
ja kirjutasin eestikeelse lühijuhendi, kuidas seda asja  pruukida, hooldada ja 
nii-öelda käimas hoida. See oli mu esimene nii-öelda, ma ei tea, praktikandi 
töö või mis iganes töö seal Küberis. Aga siis hakkasid igasugused muud asjad
tulema.

Me olime mingis hästi varajases europrojektis, ma mäletan, see võis olla 1995. 
aastal. Ma tean, et ma käisin Darmstadtis\index{Darmstadt}. Sakslased olid 
kirjutanud sellise tarkvara nagu secu-d, mis oli,  ma ei kujuta ette, et ma 
pakun mingi kümme mega haljast C koodi väga halvasti kirjutatud, mis  
realiseeris kogu krüpto, mis tolleks hetkeks oli teada. Kõik sertide töötlus, 
särk-värk. Ja siis me üritasime seda secu-d'd kuidagi rakendada ja kuidagi 
käima peksta. Ütleme niimoodi, et selline \emph{cross-platform} arendus tollal, 
et sul on kood, mida sa kompileerid mingi UNIXi jaoks ja mingi PC jaoks ja siis 
tulid Windowsid, eks ole. Ja teha nii, et see kuidagi enam-vähem  töötab ja 
piisavalt vähe mälu lekib ja piisavalt harva sama mäluplokki kaks korda 
vabastab on  raske ülesanne. Ja siis ma selle secu-d najal ehitasin 
mingisuguseid asju. Turvalist meiliklienti näiteks ja sertifitseerimiskeskust. 
Sertifitseerimiskeskused olid lahedad,  seal mingisugusel  ajaperioodil oli 
see, et me seal Küberneetika Instituudis iga aasta programmeerime vähemalt ühe 
sertifitseerimiskeskus valmis softi mõttes.

\question{Miks?}

See oli mingisugune \emph{blend} sellistest praktilistest vajadustest ja 
teadustöö eesmärkidest. Et üks  sertifitseerimiskeskus, mille me näiteks 
programmeerimine oli näiteks selline. Tollal ei olnud ju mingeid kiipkaarte ja 
riistvaralisi turvamooduleid kätte saada. Ja see oht, et kui sul 
sertifitseerimiskeskuse võti ära 
 kompromiteerub, et siis keegi annab võltssertifikaate välja, see oli suur. Või 
et keegi annab sellele operaatorile altkäemaksu, et annaks võltssertifikaadi 
välja. Sul oleks vaja mitmesilma printsiipi ja sihukest  topeltkaitset. Ja siis 
me realiseerisime selle, et me võitsime selle RSA võtme tükkideks. See on 
seesama, mida praegu SplitKey\index{SplitKey} ja SmartID\index{SmartID} teevad. 
Meil ei olnud küll seda turvalist mitmes osas võtme genereerimist, me lihtsalt 
RSA võtme, jagasime ta osakuteks ja siis meil oli sihuke m-n-ist skeem. 
Selleks, et sertifikaati välja anda, siis viiest operaatorist kolm pidid  
allkirja andma ja siis me kombineeris neist korrektse sertifikaadi kokku. Selle 
nii-öelda initsialiseerimisprotsessi käigus tekitati viis flopit,  millega need 
 operaatorid ringi oleks pidanud käima. Selles mõttes oli ta praktiline, et ta 
töötas,  tegi täitsa korrektseid X.509  sertifikaate ja oli kasutajajuhendiga 
varustatud.  
 
\question{Tundub, et kui sa enne seal ISA siini peal tegelesid väga madala 
taseme asjade katsetamise ja läbi mängimisega, siis nüüd sa tegid sedasama 
krüpto jaoks põhiolemuses olulisi primitiive ja protsesse läbi realiseerides?} 

Jah, et seda võib öelda küll, et mingis mõttes me tegelesime selliste hästi 
\emph{basic} asjadega. Me jõudsime ka rakendusteni välja. Meil oli ka 
hästi-hästi praktilisi asju, aga me kontrollisime tegelikult kogu seda pinu 
ülevalt alla välja. Et sellel ühel hetkel me tegime tulemüüre, mis oli väga 
hästi müüv toode Eesti turul, Barrikaad\index{Barrikaad} oli selle nimi, mul 
siiamaani barrikaadi T-särk alles. Siis me tegime VPN toote, mis oli veel 
ägedam. Selle VPNi teine versioon oli igasugustes Eesti riigiasutustes 
väga-väga pikalt kasutusel ka peale seda, kui selle tugi ametlikult õnnetuseks 
ära lõppes. Ja selle põhieelis oli see, et ta oli projekteeritud hästi 
turvaliseks, keskelt administreeritavaks, eriti töökindlaks. Ehk et see, et sul 
on  harukontorid, kust sa ei taha üldse interneti väljapääsu, vaid tahad läbi 
keskse tulemüüri (mis oli kallis) neid välja juhtida, see oli meil sinna sisse 
ehitatud. Igasugused paralleelsed ruutingud üle erinevate kanalite, eks ole. 
Seal tekivad probleemid, kui sul on VPN tunnel, sul on  sisemised aadressid, 
välimised aadressid, kuidas sa neid majandad niimoodi, et see ruutingu info ka 
seal sisevõrgus korrektselt leviks ja tegelikult ka töötaks. Et kasutajad 
ei peaks  ootama, kuni nende seanss katkisest kanalist tervesse kolib, eks ole, 
et see lihtsalt töötakski. Ja kogu see administreerimine. Meil oli tehtud see 
tükk, mis võimaldas süsteemi konfiguratsiooni muuta, see oli eraldi, see võis 
offlainis olla, see suhtles  muu maailmaga floppide kaudu, see ei olnud võrgus. 
Ja siis oli meil võrgus olev tükk, mis ainult monitooris, kogu sealt infot ja 
täitis neid käske, mis võrgust väljas olev tükk talle  ette pani. Niisugune 
eriti kõrgete turvanõuete jaoks tehtud haldussüsteem. Ja, ja seal me muuhulgas 
siis, kuna tollal ikkagi see PC krüpteerimisvõime oli nõrk, siis me 
realiseerisime ise  šifreid. Tollal just MMXi laiendused tulid prosele välja, 
mis võimalused sul näiteks IDEAt\sidenote{\emph{International Data Encryption 
Algorithm (IDEA)} on esmakordselt 1991. aastal kirjeldatud sümmeetriliste 
võtmetega plokkšiffer} paralleelselt arvutada,  mitu blokki korraga. Ja siis 
Helger Lipmaa\index[ppl]{Lipmaa, Helger} oli veel Küberis tööl, kes 
programmeeris siis Linuxi tuuma jaoks MMXi \emph{extension}eid  kasutava 
AESi\sidenote{\emph{Advanced Encryption Standard (AES)} on Belgia 
krüptograafide poolt välja töötatud Rijndael plokkšifri alamhulk. 1997. aastal 
teatas NIST (\emph{National Institute of Standards and Technology of the United 
States (NIST)}) plaanist asendada avaliku protsessi abil tolleks ajaks 
ohtlikult nõrgenenud DES algoritm. Vincent Rijmen ja Joan Daemen esitasid oma 
ettepaneku valikuprotsessi ja see standardiseeriti NISTi poolt 2001. aastal}  
realisatsiooni. Meil seal Linuxi\index{OS!Linux} tuumas olid oma draiverid, mis 
seda VPNi asja haldasid, seal peal olid  oma deemonid võtmete vahetuseks, konfi 
levituseks, kõigeks muuks  ja siis niimoodi hierarhiliselt üles välja.

\question{See, mis sa räägid, et see ei kõla enam nagu programmeerimine, see 
kõlab nagu arhitekti töö. Kas sa liikusid programmeerija rollist arhitekti 
rolli või mõtlesite te neid asju kambakesi välja, kuidas see käis teil?}

Selles mõttes, et välja mõtlesin kogu aeg lihtsalt enamasti oli see teine tüüp, 
kes asju realiseeris,  sellesama peakolu sees. Lihtsalt seal tulid inimesed 
nagu appi. Meil ei olnud  väga selgelt nagu defineeritud rolle, eriti alguses, 
eks ole. Arhitekt, projektijuht, projektijuhid olid üldse väga haruldased 
nähtused, Me ei teadnud isegi, mis projekt on, me lihtsalt programmeerisime 
mingi hetkeni. Meil oli seal, jah, ikkagi terve hulk inimesi, kes arutasid 
intensiivselt praktiliselt kõigil teemadel. Kui asjad olid selged ja siis 
igaüks natukene läks oma  valdkonnas  süvitsi sellega.

\question{Nutikatel inimestel on vahel oma nutikusele vastav ego ka, keegi nina 
püsti ei ajanud ja ennast arhitektiks ei kuulutanud?}

Ei, päris nii ei olnud. Aga ma ise kardan tagantjärgi võib-olla mina ise 
kippusingi see tüüp olema, kes oma  arvamust teistele peale surus. Aga ma tol 
hetkel ei tajunud seda kindlasti niimoodi. 

\question{Ma arvan, et ega teised ka ei tajunud ja soft ju lõpuks ikkagi 
töötas ju}

Absoluutselt. Nii see tulemüür kui ka see VPN, olid meil ikkagi lõpuks ikkagi 
ääretult stabiilsed ja, ma ütleks, kvaliteetset tükid. 

\question{Privador kasvas ka ju sealt välja?}
Jah, Privador\index{Privador} oli siis Küberneetika Aktsiaseltsi spin-off 
firma, mis siis sai need nii-öelda infoturbetooteid, eesmärgiga need laia 
maailma viia, aga see kahjuks ei õnnestunud. Seal oli  kindlasti ports 
probleeme ja üks probleem, mida mina nägin oli see, et tollal hakkasid tekkima 
standardid, et mis asi on VPN, mis asi on standardne VPN. Ja IPSec oli 
enam-vähem ära standardiseeritud, IKE oli ära standardiseeritud ja see oli 
tegelikult see, mida oleks tahetud osta. \emph{Vendor lockin}i juba päris 
mõõdukalt kuni palju kardeti. Ja ehkki meie olime oma asja ehitanud, eriti need 
alumised kihid, need olid  standardite põhjal ehitatud aga mudel,  kuidas me 
nägime seda võrgu tervikut ette ja mida me pidime tegema, selleks, et neid häid 
omadusi saada,  seal tekkisid konfliktid IKE või ütleme, IPSeci, ideoloogiaga 
natukene. Meil  tegelikult oli töölaua peal  versioon kolm VPNist, mis oleks 
siis olnud täiesti standarditega ühilduv, mis loodetavasti selle  firmapärasuse 
probleemi oleks ära kõrvaldanud, aga see kahjuks ei läinud realiseerimisele. 
Selle asemel me tegime digiallkirja tarkvara ja ajatembeldustarkvara ja 
Notariseerimistarkvara ja kõike muud. Me nagu natuke ennustasime valesti, et 
mis on see \emph{killer} rakendus krüptomaailmas järgmise kümne aasta jooksul. 
Olime nagu natuke ajast seest selles mõttes.

\question{See lähenemine, et võtame alumise kihi standardid ja paneme nad 
kuidagi täitsa uut moodi ülemise kihi standarditeks kokku on ju seesama, mis 
sai digiallkirja konteineriga tehtud ja X-Teega ka}

Absoluutselt. Aga vaat seal ongi see, et standardid on ja peavadki olema 
tegelikult geneerilised, eks ole. Nad peavad olema sellised, et nad lahendavad 
paljude inimeste paljusid probleeme, siis nad on elujõulised. Nii. Aga aga kui 
sa võtad ühe konkreetse riigiasutuse, kellel on konkreetsed vajadused, mis ta 
peab ära lahendama efektiivsel viisil, siis sa ei pääse lihtsalt sellega, et sa 
võtad standarditele vastavat tüki ja evitad selle. See ei ole efektiivne. Ja 
see oli siis see, mida meie tegime. Aga seal oligi vaata natukene see, et me 
võib-olla ei tajunud seda, et kui suur see maailm on ja kui võimas ta on ja kui 
suure massiga ja kui kiiresti ta liigub. Me mõtlesime, et me teeme ikka rajult 
ägeda asja. Ja noh, see on nagu \emph{way}  parem ja praktilisem väga suure 
hulga klientide jaoks. Aga see teadmine, et miski asi on hea ja praktiline,  
seda on väga raske efektiivselt ja kiiresti ühest peast teise viia.  

\question{Arvestades, et samast pundist tulid ju ka X-Tee\index{X-Tee} ja 
ID-kaardi kontseptsioon, siis kahest kolm ei ole üldse mitte paha edu protsent}

X-Teega on muidugi see, et X-Tee omab selles meie VPNi tootes väga selgeid 
juuri. Tegelikult, kui me seda X-Teed tegime, see oli 2001.  Mais või juunis 
hakkas asi pihta või isegi natuke hiljem ja detsembris läks tootesse. Eks ole. 
See oli võimalik ainult tänu sellele, et me võtsime oma selle VPN toote kui 
substraadi. Meil oli kõik see olemas, et kuidas me teeme ühe Linuxi purgi 
turvaliseks, kuidas me sellele  Linuxi purgile paneme peale oma tarkvara 
\emph{patch}id, särgid-värgid, kuidas me seda Linuxit konfime, kuidas me hoiame 
konfi niimoodi, et see on efektiivne, kuidas konfi jagamine käib, see kõik oli 
olemas. Me lihtsalt selle asja peale ehitasime ühe natukene teistsuguse 
protokolli vahenduse tüki, eks ole. 

\question{Aga see kõik on natuke hilisem lugu. Kui mina sinuga esimest korda 
kliendina kohtusin, siis sa ikkagi juba juhtisid vägesid. Mina rääkisin oma 
mure ära ja sina tegid nii, et asjad sündisid. Kuidas sul inimeste juhtimine 
rollina esile kerkis ja kas sa üldse mõtestad seda tegevust niimoodi?}

See tekkis Barrikaadi\index{Barrikaad} või VPNi või Privadori\index{Privador} 
programmeerimise käigus, kui meeskond läks suuremaks. Eriti selle VPNi juures, 
ma arvan,  koordineeriv funktsioon oli ikkagi minu peale, et kes nüüd mida 
programmeerib, eks ole, mis ajaks. Ja kes neid asju evitamas käis, ikka meie 
ise, sealt tuli ka see klientidega suhtlus, eks ole. \emph{Helpdesk}, 
projektijuht, arhitekt, programmeerija, testija, tarneinsener, et mu roll oli 
natukene nagu kõik koos. 

\question{Aga ometi kuidagi jäi see koordineeriv roll just sinu peale?}

No ju siis selles pundis see  kõige paremini  mulle sobis, ei oska muud midagi 
arvata. Keegi pidi selle ära tegema, eks ole. Kui see olin mina, siis olin see 
mina, nii see läks.

\question{Ma selle pärast küsin, et ega sul mingisugust kihu ei olnud inimesi 
juhtida?}

Ei. See pigem oligi sedapidi, et, ma nägin seda, mis see asi võiks olla, mida 
me teeme,  päris detailselt päris paljudes aspektides. Ja siis ma nagu tahtsin, 
et see nii läheks, siis ma olin sunnitud  inimestele  ülesandeid või siis 
eesmärke püstitama. See tuli pigem sedapidi, et üksinda ei jaksa kõik ära 
progeda.

\question{Aga see on jällegi arhitekti vaatenurk. Minu peas on olemas täiuslik 
mudel süsteemist ja siis ma teen niimoodi, et see saaks teoks tehtud. Mis sa 
praegu teed?}

Mis sa praegu teed? Väga paljusid erinevaid asju. Ma suhtlen hästi palju 
klientidega ja potentsiaalsete klientidega, et aru saada, mis on  nende  mured 
ja vajadused, kuidas me saame   neid aidata. See on alates müügitööst, projekti 
juhtimiseni. Teistpidi ikkagi see, ütleme, arhitektuurne töö. Kui probleem on  
arusaadav, et mis oleks see lahendus. Ja need probleemid on keerulisemaks ja 
mastaapsemaks läinud. Mõnes mõttes ka vastutusrikkamaks selles mõttes, et me  
ikkagi tegutseme suuresti turvavaldkonnas. Ja see keskkond on nii palju 
vaenulikum ja nii palju keerulisem ja need panused on nii palju suuremad, et sa 
pead lihtsalt palju palju paremaid asju tegema kui me kunagi tegime. Sedasorti 
arhitektuurne  mõtlemine ja siis inimestele nende ideede jagamine. Nõustamine, 
mõnes mõttes ka võiks öelda isegi natukene koolitamise moodi asjad. 

\question{Sa oled kogenud arhitekt ja tead, mida on vaja selleks, et projekt 
välja tuleks. Kuidas sa viid entusiastlikult pihta hakanud meeskonnale kohale 
selle, et sinu arvates projekt ei saa välja tulla? Ja seda nii, et sind pärast 
tuppa tagasi ka lastakse?}

Samm üks on see, et sa pead aru saama. See võtab tegelikult päris kaua aega ja 
see on nagu see koht, kus tihti suhtled väga vähe. Ega seda, et vaatad peale, 
saad kohe aru,  mis valesti on, kuidas peaks olema, seda ei ole. Kõigepealt 
pead probleemist aru saama. Ja võib olla, et  sellepärast see see olukord ongi 
võib-olla keeruline või halb,  et see ongi olemuslikult keeruline probleem. 
Seal on mingisugused mingisugused põhjused, keegi on teinud mingeid otsuseid, 
mingeid probleeme on lahendatud ja selle käigus on tekkinud niisugune asi. Sa 
pead sellest aru saama. Sa ei saa lihtsalt minna, et \emph{sorry}, vanad, et 
siin on jama. Sa pead kõigepealt aru saama, mida on tehtud ja miks on tehtud, 
need probleemid endale selgeks tegema. Ja siis sa tõenäoliselt marineerid nende 
otsas päris kaua ja see ei tule niimoodi, et hops, homme hommikuks on valmis, 
eks ole. Sa mõtled ja kirjutad ja räägid. Ja ehkki tihti on niimoodi, et  sulle 
endale võib tunduda, et lõpuks kui sa mingeid asju hakkad tegema, et selline 
lahendus oli algusest peale selge. Aga kui sa lähed kontrollima fakte, et mida 
sa tegelikult rääkisid, mida sa oled ise kirjutanud, mis sa arvasid, siis 
selgub, et tegelikult see lõplik lahendus on sinu juurde väga suure kaarega 
tulnud. Sa pead selle lihtsalt välja kannatama ja selle ära tegema. Aga, aga 
point on lõpuks see, et kui sa oled jõudnud mingisuguse asjani, millest sa 
näed, et see ongi okei ja lahendab ära  selle probleemi ja selle probleemi ja 
selle probleemi. Võib-olla see lahendus on keeruline ja on kulukas nagu 
realiseerida ja on isegi riskantne aga ta on õige, ta on juba olemuslikult 
õige. Sa saad aru, et mis see probleem olemuslikult on, kuidas seda asja  
tükeldada, kuidas seda keerukust peita, kuidas seda asja üldistada. Ja siis sa 
pead väga kannatlikult väga paljudele inimestele seletama, miks me võiks teha 
just nii. Seda jõuga ei saa teha. Sa pead neid julgustama ja sa pead olema 
valmis nende eest viskuma džotile, juhul kui on vaja. Aga ma ise muidugi usun, 
et ei lähe vaja, või siis sealt džotist ei tule midagi surmavat välja, eks ole.


\chapter{Ahti Heinla}
\index[ppl]{Heinla, Ahti}
\question{Kuidas sa sattusid arvutite juurde?}
Ma tulen sellisest perekonnast, et minu ema ja isa olid mõlemad 
programmeerijad. Ja nemad olid siis ülikooli lõpetanud ja  said tööl kokku ka, 
see oli kuskil kuuekümnendate lõpp. Ja see oli see aeg, kus Eestisse tekkisid 
esimesed arvutid, mis sel ajal olid muidugi kapi suurused, aga ikkagi.

\question{Kuuekümnendate lõpus ei saanud neid programmeerijaid ju palju olla?}

Jah, kindlasti neid ei olnud palju, kuigi neid siiski ikkagi täiesti 
mingisugusel määral oli. Minul muide muide oli hiljuti selline asi, et kui emal 
oli selline suur juubel, üle seitsmekümne ja niimoodi, ja ta kutsus enda 
kursusekaaslased külla. Ja kes need kursusekaaslased siis on, need on 
rakendusmatemaatikud, praegu siis sellised üle seitsmekümne aastaseid  
inimesed, nii mehed kui naised. Põhimõtteliselt kõik programmeerijad, 
professionaalsed programmeerijad olnud, enam-vähem kõik, mehi ja naisi 
võrdselt. Ja, näed, meil on ikkagi asi juba nii kaugel, et meil on juba nagu 
suhteliselt kaugeid põlvkondi, kes on üles kasvanud programmeerijatena. Ja mina 
sündisin kahe sellise inimese järeltulijana.

\question{Kas see on pigem vedamine või vastupidi? Oleks võinud ju ka ära 
hirmutada?}

Mind see kindlasti ära ei hirmutanud, ma kasvasin üles perfolintide vahel. Ja 
niimoodi, et vahetevahel, kuna arvutiaeg oli ju piltlikult öeldes talongidega 
jagatav, arvuti pidi ikka õhtuti töös olema ja siis mõnikord  ema ja isa käisid 
õhtuti tööl, kui nad said arvuti aja kella kaheksaks õhtul, siis nad said 
arvuti aja kella kaheksaks õhtul. Ja siis nad võtsid vahepeal minu ja mu õe 
siis kaasa, mina jooksin ka arvutikappide vahel ringi ja vaatasin, kuidas seal 
magnetlindid vaikselt käisin nii ja naa ja see oli kindlasti hästi põnev. 
Hoopis teistsugune keskkond ja isegi helid on teistsugused ja vaatad, kuidas 
need masinad seal toimetavad, mingid magnettrumlid vaikselt vihisevad ja 
sahisevad ja kindlasti oli.

\question{Legendid räägivad, et selle põlvkonna rahvas korraldas Ameerikamaal 
lindikappide võidujookse ja muud sellist, tolles sinu arvutiruumis midagi 
sellist ka toimus?}

Mina selliseid asju ei näinud. Ma saan aru, et Eestis ka tehti selle sel ajal 
sellist  pulli, aga võib-olla  seda tegid natukene nooremad inimesed, kellel 
lapsi ei olnud. Minu isa ja ema olid ikka natuke siukse ontlikumad. Nad 
üritasid mingisuguseid konstruktiivseid asju arvutiga teha,  panna neid just 
ühel või teisel moel  käima, aga nad ei olnud sellised, kuidas öelda, häkkerid 
tänapäeva mõistes. Et  mismoodi arvutiga  pead pesta, näiteks, et selliseid 
asju nad ei mõelnud.

\question{Sind ju esialgu ei lastud linte perforeerima? Mis esimene asi oli, 
millele sa ise käed külge said?}

No mu vanemad olid programmeerijad aga mina ei olnud programmeerija, tavaline 
laps nagu ikka. Ma vist olin nagu natuke  matemaatiliselt  andekas, aga 
arvutitega otseselt minu tegelikult esimene kokkupuude oli ikkagi sellest, kui 
ma olin kümne aastane. Lihtsalt järsku päevapealt  ühel õhtul tuli ema  koju ja 
ütles, et kuule, Ahti,  ma õpetan sulle midagi, istume maha. Istusime maha ja 
ta õpetas mind programmeerima. Kolm õhtut niimoodi õpetas. Ja ma sain selle 
kolme õhtuga tegelikult sellest oast aru, et mismoodi see asi käib. Sealt 
alates  siis hakkasin juba ise edasi mõtlema, proovima, katsetama, lugema, 
natukene lolle küsimusi küsima. Kolm päeva ma olen sellist süstemaatilist 
programmeerimise õpetust saanud.

\question{Aga mis ta siis rääkis et see kümneaastasele huvitav oli?}

Eks mind huvitasid sellised asjad kindlasti. Sel ajal oli ju ka niimoodi,  
aasta oli 1992, et  lapsel on tohutult palju mingisuguseid ahvatlusi 
ümberringi, et Facebookid ja Instagrammid hüppavad siia-sinna ja kõikvõimalikud 
muud asjad käivad. Sel ajal oli ikkagi niimoodi, et ega meil kodus ju telefoni 
näiteks ei olnud. Arvutit ka ei olnud, sest mina kirjutasin programmi ikkagi 
alguses niimoodi, et kirjutasin paberi peale. Need kolm päeva õpet käis paberi 
peal. Ja kui  ema tuli õhtul koju ja sellist asja ütles, siis me ikkagi mitu 
tundi istusime maas, eks ole. Ei ole niimoodi, et mul oleks kogu aeg telefoni 
helisenud ja hüpanud, mingisugune asi, et kuule, Ahti, tule nüüd sinna, teeme 
seda. Selles mõttes võimalik, et ei olnudki nii väga vaja, et see oleks nagu 
hullult kuidagi põnev olnud kümne aastasele lapsele. Mul pigem oligi lõpuks  
põnev see, kui ma sain aru, kuidas see asi töötab.

\question{See peab olema päris korralik ettekujutusvõime, et sa paberi peale 
kirjutades saad aru, kuidas miski asi töötab. Sest paberil ei tööta sul midagi, 
seal on lihtsalt tekst}

Nojah, samas aga eks programmeerimise üks selline  võtmeoskus ongi tegelikult 
ju oskus ette kujutada, et mismoodi see masin  töötab. Lõppkokkuvõttes ju 
programmeerija ehitab ju masinat. Ja noh, piltlikult öeldes, ikka samasugust 
masinat nagu,  mingisugused hammasrattad kuskil käiksid. Üks koodirida on 
nii-öelda piltlikult öeldes üks hammasratas, teine koodirida on mingi kangikene 
kuskil seal, eks ole. Ja kui sa ehitad sihukest füüsilist või mehaanilist 
masinat, siis sa näed kõike seda, kuidas see töötab, et siin mingi ratas keerab 
ja siis kang liigub ja kuidas siis teine asi kuskilt midagi lükkab ja mingi 
lint või tross kuskilt midagi tõmbab. Sa näed seda kõike füüsiliselt. Ja 
programmeerija peab ka nägema. Aga ta peab nägema seda vaimusilmas, sest seda  
füüsilises maailmas  silmadega ei näe. Ja see vaimusilmas nägemise oskus on 
ülivajalik programmeerijale. Tagantjärele vaadates võib öelda, et eks ema mulle 
selle tegelikult õpetaski  see kolm päeva.

\question{Kas \texttt{goto} käib nii- või naapidi või tehete järjekord on 
selline või teine, on teisejärguline}

Just. Tegelikult oleks põhiliselt vaja teada, et sa \verb|goto| tegelikult teeb 
või et selles masinas, millise hammasratta, millise kujuga asja, see 
\verb|goto| seal teeb.

\question{See kolm päeva tekitas huvi, sa said enam-vähem aru, kuidas arvuti 
töötab, aga mis edasi sai?}

Siis läksime kuskil õhtul emaga siis sinna arvuti juurde, ema tööle. Ja siis me 
tippisime selle programmi sisse. Ja kui ma õieti mäletan, siis seda sisse 
tippimist võis juba mitu päeva olla. See oli ikka mitu lehekülge, see minu 
programm ja mõnikord läks midagi valesti ka ja nii edasi. Ema aitas mul siis 
seal mõned vead ära parandada ja siis tuli välja, et  tegelikult see programm 
töötas. See lahendas ühte väikest sihukest matemaatik  keerdülesannet, kus  
loogika oli  selles, et kui sul on  näiteks sada ühikut raha ja sa lähed 
raamatupoodi ja sa tahad seda sada ühikut raha ära kulutada. Siis sa pead 
kombineerima, et osta üks raamat, mis maksab viiskümmend seitse ja teine raamat 
nüüd maksab kolmkümmend, selline klassikaline  matemaatika keerdülesanne, 
kuidas kombineerida niimoodi, et kokku saada  summa, mis on võimalikult 
lähedane sajale aga mitte üle selles. Ja sellist ülesannet lahendas see minu 
programm. Ei ole nagu kõige triviaalsem asi, see ei ole nagu päris niimoodi, et 
vajutad nuppu ja programm ütleb lihtsalt \enquote{tere}. Tänapäeval ikkagi 
pigem kõik asjad üritatakse, ka heal põhjusel, ehitada niimoodi, et sul on 
selline nagu hästi kiire rahuldus või et sa nagu näed kaks minutit vaeva ja 
juba midagi hästi väikest nagu töötab ja siis sa näed veel viis minutit vaeva 
sealt tuleb veel midagi. Siis mina pidin vaeva nägema alguses kolm päeva, enne 
kui tulemust oli. Enne seda oli kõik ainult vaimusilmas.  Aga, tõepoolest, kui 
sul kogu aeg Facebook taskus ei hüppa, siis on nagu natuke lihtsam ka seda kolm 
päeva leida. 

\question{Mis tolle arvuti nimi oli?}

Ausalt öelda ma isegi ei mäleta, ei pruukinud isegi nõukogude masin olla,  seal 
oli tegelikult ka lääne aparaate.

Isegi minul sedasama seda ühte programmi, mis ma kirjutasin, seda minu meelest 
sai isegi  mitmel arvutil käitatud. Et see ei olnudki niimoodi, et 
\enquote{kuule Ahti see on nüüd sinu arvuti, millega nüüd sina  mitu päeva nii 
nagu tegeled}. See isegi vist nägi niimoodi välja, et ma pool programmi 
tippisin ühel arvutil sisse, mis oli sihuke suur must kapp ja siis järgmisel 
õhtul läksime ühe hoopis siukse läänelikuma välimusega siukse nagu nõtkema 
välimusega moodsama asja taha ja tippisin teise osa sisse. Et ma juba sain ka 
natuke kogemusi sellest, et see programm on ikka hoopis midagi muud, see ei ole 
see füüsiline arvuti, millega ma tegelen. Ma võin istuda ühe arvuti taha ja 
siis ma võin minna teisele kodusele teise arvuti taha, mis on terve toa suurune 
ja see minus seesama programm jookseb selles ühes jookseb selles teises.

\question{Mille peale sa vahepeal kirjutasid selle programmi? Kaartide peale?}

Siis olid ikka juba diskid olemas. Mitte need sihukesed, kolme tollised 
disketid, vaid sihukesed  kaheksa või viie tollised või mingid sellised asjad, 
pigem kaheksa tollised ilmselt. Aga kindlasti see esimene programm oli ainult 
selline algus, eks ole, sellest tuli mingisugune  oskus ja huvi asja vastu. 
Edasi hakkasin siis nüüd ise vaatama ja  sattusin kokku juba teiste poistega, 
kes siis analoogse asja vastu huvi tundsid. Lähemate aastate jooksul hakkasid 
tekkima ka personaalarvutid ja enam ei olnud alati niimoodi, et sa pead õhtul 
ema töö juurde minema tingimata vaid on kuskil juba muid kohti ka olemas.

\question{Kust sellised tutvused tekkisid, internetti ju polnud?}

Internetti ei olnud, aga küll oli olemas näiteks kaheksakümnendatel tekkinud 
selline asi, nagu Raaliklubi\index{Arvutiklubi!Raaliklubi}, mida vedas selline tegelane 
nagu Jaak Loonde\index[ppl]{Loonde, Jaak}. Mina olin ka selle klubi liige seal 
vahepeal ja see koondas sihukesi huvilisi poisikesi. Ega mul on raske öelda 
täpselt,  ise nagu poisikesena tol ajal süstemaatiliselt ei pannud tähele ja ei 
jätnud meelde ka täpselt, mida täpselt Jaak Loonde tegi ja kas ta üldse midagi 
tegi peale selle, et lihtsalt need poisid kokku tuua. Aga täiesti võimalik, et 
sellest piisabki, et need poisid kokku tuua, kellel on sama huvid ja siis nad 
omavahel juba vahetavad kogemusi, kellel kuskil jälle ema või isa töötab 
kuskil. 

Minul oli näiteks üks niisugune reliikvia, mille ema mulle andis: ta õpetas 
mind kolm päeva ja siis ta andis mulle ühe ingliskeelse raamatu, mis oli 
põhimõtteliselt Pascali programmeerimiskeele õpik. See oli inglise keeles, ehk 
siis ma ei saanud sellest eriti midagi aru. Ma koolis õppisin saksa keelt, 
mitte inglise keelt\sidenote{Tol ajal jagunesid koolid kaheks: kas lisaks vene 
keelele õpetatakse läbivalt inglise või saksa keelt}. Küll aga ma sain aru 
nendest  programmi näidetest, mis seal oli, eks ole, ja seal oli asjad ikkagi 
mingisugused loogilises  järjekorras. Tegelikult, kuigi ma inglise keelt ei 
osanud, ma siiski suutsin sellest raamatust kindlasti midagi õppida ja sealt 
tuli ideid, mida katsetada. Seal oli kuskil mingi, piltlikult öeldes, mingi 
\verb|goto| käsk, oletame. Ma ei teadnud, mis see tähendas, eksole, aga ma sain 
selle \verb|goto| käsu kuskil hiljem mingisse arvutisse sisse tippida ja 
vaadata, mis teeb ja küsida kelleltki teiselt, et mis see \verb|goto| tähendab. 
See on midagi muud kui lihtsalt öelda, et õpetaja mulle programmeerimist, et 
sul on ka konkreetne küsimus juba. Niimoodi läbi lukuaugu piltlikult öeldes see 
õppimine käis. Internetti ei olnud, aga  inimestel ei olnud ka internetti, siis 
kui nad  lennukeid ja autosid ehitasid, ja sai hakkama.

\question{Seal arvutiklubis sa käisid seepärast, et programmeerimine pakkus 
huvi?}

Jah, mulle pakkus see programmeerimise pool huvi. Mul tegelikult oli niimoodi, 
et isegi enne programmeerimist sattus kätte mingisugune lastele mõeldud 
elektroonika raamat ja ma  natukene nagu harjutasin või mõtlesin selle 
elektroonika peale ka, et kuidas näiteks transistorid töötavad ja muu selline. 
Ja see pakkus ka mulle kindlasti huvi. Aga tagantjärele vaadates siis ma 
ütleks, et minu elektroonika tegemine sel ajal oli  ülialgelisel tasemel. Ma 
nii-öelda  kuidagi nagu hästi õrnalt natuke nagu kõditasin elektroonikat ja 
üritasin sellest midagi aru saada, aga samas programmeerimisega ma tegelesin  
tegelikult. Selles mõttes oli seal väga suur vahe ja loomulikult oli ka väga 
suur vahe siis minu  professionaalsuse tasemes, mis  tekkis.

\question{Kas sa oskad öelda, kas see oli pigem eeskuju või midagi muud, mis 
sind pigem programmeerimise poole suunas?}

Üks asi oli kindlasti eeskuju, aga teine asi oli ka kindlalt puhtalt ju see, et 
selle jaoks, et elektroonikaga tegeleda, sul on vaja ikkagi mingisuguseid 
teatud füüsilisi asju. Sul on vaja elektroonikakomponente, sul on vaja 
tööriistu ja nii edasi, mida ju ei olnud. Isegi tänapäeval on ju poes  kõik 
olemas, aga sa pead minema ja üldse mitte vähe raha kulutama ja ostma need 
endale. Ma olen natukene ka hobi korras, elektroonikaga nagu tegelenud vahepeal 
kodus, tinutan üht-teist seal ja nii edasi. Ja noh, praegu, kui kõik on nagu 
justkui valla, kõik on olemas ja kahe päevaga tuuakse koju ära, raha ikka kulub 
selle jaoks. Mingi  üks jootekolb ja teine suurendusklaas, takistite 
komplektid, väiksed mikroprotsessorid, igasugused kivid ja sensoreid ja andurid 
ja displeid ja nii edasi. Sellega ei ole lihtne alustada. Programmeerimine on 
niimoodi, et sul tegelikult on vaja seda kohta, kus nii-öelda arvutis käia, eks 
ole, paber ja pliiats ja kolm päeva on täitsa piisavad alustamiseks.

Nagu mul sõber ja töökaaslane Jaan Tallinn\index[ppl]{Tallinn, Jaan} on  
öelnud, et programmeerimine on selline naljakas asi, et  enamikes muudes  
valdkondades on niimoodi, et kui sa hakkad   õppima, siis sa saad mingisuguse 
algse  taseme kätte ja siis sa pead rohkem õppima, et saada järgmisele 
tasemele,  sa pead veel rohkem õppima. Ja sa ei saa iseseisvalt õppida, vaid 
sul on vaja kedagi, kes õpetab. Kui sa õpid klaverimängu näiteks, siis sul on 
vaja tegelikult seda, et keegi sulle pidevalt õpetaks  klaverimängu, sa ei saa 
ise õppida klaverit mängima. Sul on mingisugused teatud käelised asjad, et 
mismoodi sa seda teed, parimal juhul sa saad või mingist YouTube'i, videost või 
mingist õpikust õppida. Aga sul on vaja seda YouTube'i videod või õpikut. 
Programmeerimine, aga, on tegelikult selline asi, et kui sa oled selle algse oa 
kätte saanud ja sind siis suletakse üksikule saarele aastaks niimoodi, et sul 
on arvuti käes, siis sa tegelikult suudad ise ilma ühegi õppevahendita õppida 
ennast väga heale tasemele, kui tahad. Puhtalt ise katsetamise,  ise mõtlemise  
teel. Ja eks tegelikult täpselt seda ma tegingi, sel ajal, kui ma teismeline 
olin.

\question{Kas sel ajal hakkas ka juba personaalarvuti moodi arvuteid liikuma?}

Jah, personaalarvuteid hakkas täiesti tulema ja meile koju tekkis ka üks Apple 
II\index{Arvutid!Apple II}. Sellega siis mina hakkasin toimetama, aga see oli 
üsna  kaheksakümnendate lõpus kuskil. Ma ei oska täpselt aastanumbrit öelda, 
aga ju ta võib olla võis juba 1988 olla või midagi niimoodi. Ma juba ikkagi  
nagu täiesti oskasin sel ajal programmeerida, ma ei olnud nagu päris enam kümne 
aastane, ma olin juba viisteist või kuusteist või midagi niimoodi. Inimestel, 
kes on seitsmeteist ja kaheksateist aasta vanused, enamikel inimestel on üsna 
kõvasti nagu meri põlvini  ja peod ja seltsielu ja asjad käivad. Aga minul on 
nagu paar asja teisiti. Esiteks ma olen üldiselt introvertne inimene ja mitte 
üli seltsiv, see seltsielu mul kuidagi nii hästi nagu välja ei tulnud. See on 
üks asi. Teine lihtne tõsiasi oli see, et vist kuni viimaste aastateni, umbes 
kolmekümne viienda neljakümnenda eluaastani oli mul elus selline asi, et kui ma 
joon kaks klaasi veini ära, siis hakkab mul pea valutama. Ma lihtsalt ei pea 
ühel  korralikul peol kaua vastu, lihtsalt ei pea,   ma lähen koju hiljemalt 
keskööks. Ja niimoodi on  kogu aeg olnud ja oli ka siis, kui ma olin 
kaheksateist, eks ole. Aga alates kella kaheteistkümnest ju nagu tegelik 
\emph{action} hakkab pihta, nagu mulle räägitud, ma nagu väga palju ise kogenud 
ei ole. Ja siis ongi see, et kui  ülejäänud inimesed avastavad seal seltsielu 
ja ja pidusid, siis osad teised avastavad arvutiasju ja siis avastavad 
seltsielu natukene hiljem lihtsalt.

\question{Aga mida sa programmeerisid? Sellise jõukohase aga samas huvitava 
ülesande leidmine ei ole ju üldse lihtne?}

No eks poisikesi ikkagi mängud huvitavad üsna palju ja kindlasti ma arvan  
minul ja minu kaasvõitlejatel kindlasti kõigil oli ju üks esimesi unistusi, et 
kirjutada oma üks arvuti mängida näiteks. Sel ajal oldi juba vahepeal need 
Yamaha arvutid juba tekkinud, eks ole, ja juba ka Apple II peal oli täitsa 
korralikke  mänge olemas. Need kommertsiaalsed mängud, need olid ikka sellisel 
tasemel, mida üks mingi hobistist poisikene ikkagi nädalaga valmis ei viska. Ja 
ega see tähelepanu ulatus  kolmeteistaastasel või viieteistaastasel ei ole väga 
selline, et suudaks midagi väga palju pikemat ette võtta. Selliseid väga 
lihtsaid mängukesi sai kindlasti ehitatud ja kindlasti ka proovitud üritada 
siis niimoodi häkkerlikult natukene läheneda sellele arvutile, et mida on 
võimalik arvutit tegema panna, mis hääli on võimalik arvutit tegema panna ja  
igasuguseid lollakaid visuaalseid kujundeid ette ette manada, ja muu selline. 
See pool kindlasti ka huvitas. 

Aga hiljem, tegelikult teismeeas sai igasuguseid asju proovitud. Ega täpselt ei 
teadnud, mida võiks  teha, aga valdkond kindlasti huvitas. Aga järjekindlamalt 
hakkasime mänge programmeerima Jaan Tallinna\index[ppl]{Tallinn, Jaan} ja Priit 
Kasesaluga\index[ppl]{Kasesalu, Priit}. Siis kui me olime keskkoolis. Siis oli 
juba niimoodi, et tähelepanu ulatus on juba nagu natukene inimesel juba 
kasvanud, eks ole, ja võtsime ette ühe mängu kirjutamise projekti. See sai 
natukene pikema vinnaga, et paneme nüüd kõik oma seni õpitud väiksed kogemused 
ja oskused kokku ja kohe mitu inimest paneme, mitte niimoodi, et igaüks oma 
nurgas pusib mingeid oma mängu, vaid teeme ikka sellise tiimi töö. Jagame 
ülesanded omavahel ära ja kuude viisi töötame selle kallal. 

\question{Kust selline mõtte üldse tuli või arvamus, et selline asi üldse 
võimalik võiks olla?}

Kuskilt iseenesest tuli, ma ei oska täpselt mõelda. Meil isegi ei olnud isegi 
mingit arutelu sel teemal. Lihtsalt sündis, et proovime midagi, midagi sellist. 

\question{Mis keskkool see oli?}

Mina õppisin Gustav Adolfi Gümnaasiumis\index{Koolid!Gustav Adolfi Gümnaasium} 
ja Jaan Tallinn\index[ppl]{Tallinn, Jaan} oli minu pinginaaber. Ja Priit 
Kasesalu\index[ppl]{Kasesalu, Priit} oli Jaan Tallinna pinginaaber eelmisest 
koolist, kus Jaan käis. Nii et me olime mõlemad, Jaani pinginaabrid olnud. Ja 
siis viimase keskkooliaasta jooksul niimoodi kolmekesi kirjutasimegi ühe mängu, 
millel oli nimeks Kosmonaut\index{Mängud!Kosmonaut}. Mina küll kogu selle 
kirjutasin seda kui hobiprojekti, aga Jaan ikka ütles, et see asi tuleb teha 
nagu äriks või see asi tuleks maha müüa ja selle eest  raha saada. 

\question{See oli nõukogude aeg ju veel, selle eest võis kinni minna ju?}

Peaaegu. See oli nõukogude aja lõpp küll, sel ajal, kus juba igasuguseid 
metalliärikaid juba juba käis ringi ja niisugune nagu väikene üle piiri  
kaubandus käis ja kooperatiivid ja asjad ja selline värk juba täitsa toimis. Me 
muidugi ei teadnud tuhkagi sellest, kuidas  see  ettevõtluse või selline maailm 
üldse käib. Ja ega tegelikult ei teadnud seda ka, need suured inimesed, kes sel 
ajal ettevõtluses nagu  olid. Aga mingi tegevus toimus ja siis osade  
täiskasvanud näiteks metalliäri alal mõningast mõningase kogemusega või 
sidemetega inimeste abil õnnestus meil tõesti see Kosmonaudi mäng müüa Rootsi. 
See oli selles mõttes muidugi pöördeline sündmus, et me saime selle eest 
lõppkokkuvõttes ikkagi, kui ma õigesti mäletan, siis oli see viis tuhat 
dollarit. See oli täiesti kosmiline number,  aasta oli mingi 1990 ja ja  rubla 
kurss oli seal selline, et vist kui ma õieti mäletan, ühe dollari eest sai 
kolmkümmend rubla juba. Ja siis kui arvutad kokku, siis see viis tuhat dollarit 
oli ikkagi umbes selline summa, mis meie vanemad olid elu jooksul teeninud või 
midagi umbes sellist. Loomulikult inflatsiooniga võib seal seal korrutada ja 
korrigeerida, aga ikkagi  oluline number, ikka väga-väga oluline number. Kas 
nüüd mõelda, et kui õigesti me seda summat  kasutasime, kokkuvõttes sa ikkagi 
ka valuutapoes käidud ja Coca-Colat ostetud, selle peale kulus ka ikkagi 
märgatav osa sellest ära. Aga mina ja Jaan ostsime enda endale näiteks kahe 
peale arvuti. Selle peale läks pool sellest  minu ja Jaani osas sellest rahast 


Selle raha me saime kätte kuskil, see oli juba üheksakümnendate alguses,  Eesti 
kroon oli just tulnud või tulemas. Sellega on selline lugu kohe, et see on just 
täpselt see aeg, kus Eesti kroon tuli niimoodi, et minul oli see raha rubladena 
käes. Ja siis oli mingi kooperatiiv või firmakene, kust me siis olime kokku 
lepitud ja välja valitud mingi 386SX protsessoriga arvuti ja me olime seda siis 
ostmas. Ja siis ma mäletan, see oli see hetk, kus meil oli teada, et järgmine 
päev on siis see rahareform ja minul olid need rublad käes, kümnete tuhandete 
kaupa neid rublasid, mis oli selle arvuti jaoks mõeldud. Ja meil oli kokku 
lepitud, eks ole, et me anname need nii palju rublasid ja saame siis selle 
arvuti. See kooperatiivitegelane, kellele helistasin, siis ütles midagi, et too 
homme see taha või midagi niimoodi. Aga siis mul ikka nii palju oidu oli, et ma 
ütlesin, et ei, on kokku lepitud, ma toon täna selle raha. Ja ma tõingi täna 
selle raha ja ta võttis selle täna vastu ja me saime selle arvuti kätte umbes 
homme või midagi, või veel samal päeval. Nii et jah, ma ei tea, mis oleks 
juhtunud, kui oleks tegelikult üritanud homme selle raha  maksta. 

\question{Eks ajalugu oleks läinud tonks teistmoodi. Aga see oli juba 386, mis 
oli juba päris korralik aparaat. Sinna vahele jääb ju õige mitu aastat 
puselemist mingite teiste inimeste arvutite juures. Kus te selle mängu 
kirjutasite? Kodus kellegi juures?}

Mängu me kirjutasime suurel määral tegelikult Jaani\index[ppl]{Tallinn, Jaan} 
ja Priidu\index[ppl]{Kasesalu, Priit} töökohas. Sest Jaan ja Priit keskkooli 
kõrvalt töötasid programmeerijatena ühes kooperatiivis. Mina tegelikult ka 
töötasin keskkooli kõrvalt programmeerijana poole kohaga minu vanemate töökohas 
ehk Küberneetika Instituudis\index{Küberneetika Instituut}. Aga  ütleme 
niimoodi, et ma arvan, et  minu vanemate tööandja oli selles mõttes mõistlik. 
Kui ma ise tööandjana mõtlen, et kui mingisugune seitsmeteistaastane poiss 
tahab tööle tulla,  alles õpib programmeerima või niimoodi, et ega esiteks ma 
ei maksaks talle väga palju või ma ei võtaks teda nagunii väga tõsiselt.  
Samuti ma võib-olla ei annaks talle nii palju mingeid võimalusi, ma vast ei 
annaks talle missioonikriitilisi asju. 

Aga Jaan ja Priit olid, olid tööl ühes kooperatiivis, kus nemad olid vaata et 
sihukesed peaaegu et juhtprogrammeerijad või midagi niimoodi.  Ja neil oli 
tunduvalt paremad võimalused  käes. Mis on noh, tänapäeval vaadates, ma ütleks, 
ikkagi küllalt ebamõistlik, aga need olidki  ebamõistlikud. See tähendas, et 
nad ei saanud oma arvuteid nii-öelda töölt koju kaasa võtta, aga neil oli 
tegelikult töökoht, kus nad said päeval  olla koolis, aga õhtud-ööd said olla 
arvutis. Ja sel ajal,  kui sa oled kuusteist ja seitseteist, siis võid vastu 
pidada niimoodi, et magad kuus tundi päevas, siis kui vaja.

\question{Kui ma nüüd kokku loen, siis te käisite Gustav Adolfi Gümnaasiumis, 
mis polnud lihtne asi, te töötasite programmeerijatena ja takkapihta 
kirjutasite mängu, mille kannatas pärast maha müüa. Kõike samal ajal?}

Jah, peab ütlema küll,  et vähemalt siis, kui mina töötasin programmeerijana, 
ma töötajana ei ole uhke tööpanuse üle, mille ma Küberneetika Instituudile 
andsin\index{Küberneetika Instituut}. Tõsi küll,  ma sain ikkagi midagi valmis 
ja mu tööandja oli rahul sellega. Ma ei olnud ka tegelikult ainus, oli natukene 
teisigi selliseid õppijaid ja mõni üliõpilane, kes oli seal niimoodi tööl ja ma 
sain isegi aru, et mu tööandja isegi oli pigem minuga rohkem rahul kui seal 
mõnede teistega. Aga ma arvan, et see ütleb rohkem nende teiste kohta kui 
minu kohta. Mina ikkagi kulutasin enamiku ajast selle mängu ja koolis käimise 
peale.

\question{Sel ajal hakkasid tekkima esimesed BBSid ka?}

BBSid hakkasid tekkima ja nii-öelda minu tutvusringkonnast siis Priit 
Kasesalu\index[ppl]{Kasesalu, Priit} oli see põhiline, kes meie kambas tegeles 
BBSidega ja ühe ka püsti pani, mille nimi oli \emph{Dark Corner}\index{BBS!Dark 
Corner}, kui ma õigesti mäletan. Ja mille Fido, kuidas seda siis nimetati, 
\emph{node} number või midagi sellist, oli, kui ma õieti mäletan, neliteist. Ja 
teda tõmbas nagu see pool kuidagi rohkem või kuidagi väga palju ja eks 
kindlasti mind ka, sest BBSiga tekkis järsku  võimalus  ekraani kaudu suhelda 
hästi paljude teiste inimestega, kellega sa võib-olla füüsiliselt ei istu koos. 
Teatud mõttes võiks isegi öelda, et järsku nendele inimestele anti natuke nagu 
Facebook kätte. Mitte taskusse otseselt, aga ikkagi kätte või niimoodi, et 
järsku tekkis hulk sõpru, kellega ma olin suhelnud ainult interneti teel. Ja 
Fidos vahetati mõtteid  kõikide asjade üle, mitte ainult arvutite üle ja tekkis 
järsku üks mingisugune  täiesti isevärki sotsiaalne seltskond. Tolle aja aja 
kohta oli see väga isevärki sotsiaalne seltskond. Tänapäeval on niimoodi, et 
sotsiaalne seltskond, kes on mingi Facebookis mingisuguse grupi, olgu mingi 
MMSi klubi või ma ei tea mis, liige,  siis nad võivad aeg-ajalt kokku saada. 
Netiajastul on see tegelikult väga-väga tavaline. Aga selline selline, kuidas 
öelda elustiil või tutvusringkonna ülesehitus järsku tekkis  meile kätte, kui  
aasta oli umbes 1990 või umbes kuskil sealkandis.

\question{See seltskond pidi siis olema ka teatavas mõttes homogeenne, sest 
Fido külge saamise barjäärid olid kõrged?}

Jah, eks muidugi oli palju ka inimesi, kes nii-öelda jõlkusid kaasas. Olid 
sellised entusiastid nagu näiteks Priit Kasesalu\index[ppl]{Kasesalu, Priit} 
või Tarmo Mamers\index[ppl]{Mamers, Tarmo} näiteks no nende muud sõbrad  
aeg-ajalt tekkisid ju ka sinna sisse, kellele siis  Tarmo või Priit võimaldasid 
ligipääsu. Ja see oli kindlasti väga huvitav. Tekkis selline  sotsiaalne 
distants-suhtlus. 

Ma mäletan ühte juhtumit, oli juba tegelikult siis, kui vaikselt Internet juba 
hakkas Eestisse tekkima. Internet kui selline tehniliselt oli juba olemas juba 
ju kuskil seitsmekümnendatel kaheksakümnendatel, aga  Eestisse  ta umbes sel 
ajal niimoodi natukene juba tekkima. Mul oli selline sõber, siiamaani väga hea 
sõber, nimega Sulo Kallas\index[ppl]{Kallas, Sulo}, kellel oli ka BBS ja kes 
töötab minuga koos Starshipis\index{Starship Technologies} praegu. Tema andis 
mulle kasutada ühte oma kontot ühes Unixi arvutis. Ja Unixis oli olemas selline 
programm nagu \verb|talk|, kus sai siis omavahel ekraani kaudu suhelda 
inimesed, kes olid sisse loginud samasse masinasse. Ja ma mäletan, et  minu 
jaoks oli üks ikkagi täiesti selline silmi avav  elamus.  Mul ei olnud sel ajal 
kodus telefonigi. Ja siis ma midagi toimetasin selle Sulo kontoga Sulo nime alt 
selles ühes arvutis ja järsku selle \verb|talk|iga  hakkab minuga keegi 
rääkima.  Ütleb, et minu nimi on Epp. Nii, ja mina siis esimese asjana, kuna ma 
teadsin, ma kasutatud Sulo kontot, eks ole, keegi Epp tahab Suloga rääkida. 
Siis ma selgitasin talle, et kuule, mina ei ole Sulo, et mina olen hoopis üks 
teine inimene. Tema ütleb vastu, et  sellest pole midagi, räägime ikka. Ma ei 
saanud täpselt aru, mis värk on nagu, mis mõttes, ta ju tahab Suloga rääkida, 
eks ole. Aga siis ma sain aru, et ta tahab tegelikult lihtsalt kellelegi 
rääkida, et tal  tegelikult on täitsa okei, et ta räägib  minuga. Sihuke 
jutuajamine tekkis sealt, ja ma sain teada selle jutuajamise käigus, et  
tegemist on ühe Eesti tüdrukuga, kelle nimi on Epp ja kes hetkel füüsiliselt 
asub Ameerikas. Ja ta läks Ameerikasse  ülikooli õppima, ta oli Ameerikas 
üliõpilane. Ja mina istun Eestis, eks ole, ja ma reaalajas räägin arvuti 
ekraani vahendusel  temaga juttu, eks ole. Me rääkisime maast ja ilmast 
mingisugune tund aega, see oli  väga-väga kummaline kogemus. Sa  suhtled 
kellegagi reaalajas, kes on nagu sinust väga-väga kaugel. Ma siiamaani ei tea, 
kes Epp täpselt oli, ta ütles oma perekonnanime ka, ma ei ole seda nime mitte 
kunagi hiljem kuulnud, mitte kunagi hiljem selle inimesega suhelnud. Aga see 
oli ikka väga kummaline kogemus minu jaoks. Ongi naljakas tegelikult et, 
tänapäeval ju selline asi on ju niivõrd tavaline, kõigil mingid Snapchatid ja 
asjad on kuskil taskus, eks ole. Ja tol ajal oli sotsiaalses mõttes see, et sa 
võid suhelda inimestega kuskil üle maailma,  oli nendele interneti häkkerite 
võimalik ja teistele inimestele ei olnud.

\question{Sa rääkisid, et BBSides räägiti igasugustel teemadel. Näiteks, 
millest räägiti?}

Kui ma õieti mäletan, seal oli igasugust, sellist elulist, nagu tänapäeva 
internetifoorumid, eks ole. Kõigest võidakse seal rääkida. Seal oli mingisugune 
filosoofiateemaline  vestlusgrupp, kus  inimesed olid ju enamasti sellised 
kaheksateist aastased, kes veel mõtestavad oma elu. Ongi selline aeg inimeste 
elus, kus kõik mõtlevad, mida tähendavad mingisugused asjad ja kas ikka inimene 
peaks panustama sellele või tollele. Tänapäeval neljakümneaastasena väga 
võib-olla ei viitsi sel teemal juttu vesta väga, kõigil on juba oma elu 
tõekspidamised välja kujunenud, aga tol ajal minul kindlasti ei olnud ja enamik 
sellest ülejäänud BBSi seltskonnast oli ka umbes sama vanad, eks ole. Siis oli 
seal igasuguseid psühholoogiateemalisi, neid vestlusi oli igasuguseid, see 
kindlasti ei olnud sugugi mitte ainult tehnoloogiateemaline. 

\question{See, mis sa ütled, kõlab väga oluliselt. Sest see tähendab, et 
mingisugune ports nutikaid inimesi mitte üksinda ja mitte juhuslike inimestega 
vaid koos sama moodi mõtlevate ja samade oskustega inimestega mõtestasid seda, 
mida tähendab olla inimene kõige laiemas mõttes}

Absoluutselt. See oli tegelikult üks niisugune virtuaalne sõpruskond.  Võib 
olla võib öelda, et see Fido seltskond oli kõige esimene virtuaalne sõpruskond 
Eestis üldse. Tänapäeval on  igaühel virtuaalseid sõpruskondi taskus sada tükki 
aga see võis olla võib olla täiesti esimene.

\question{Kas selle kõige juurde käis ka mingi spetsiifiline raamatu-, muusika- 
või filmihuvi?}

Ahaa, muusikakanaleid oli loomulikult ka, muusikateemalisi  vestlusgruppe. 
Minul ei käinud. Võib-olla natukene. Ma arvan, et  selles ringkonnas pigem olid 
populaarsed sihukesed elektroonilise muusika bändid. Nii, ja naa, ütleme. 
Kraftwerk mulle ei meeldinud ja ei meeldi siiamaani, Jean-Michel Jarre samuti 
mitte nii väga palju aga Tangerine Dream näiteks meeldis mulle väga ja 
siiamaani meeldib, mul on ikka mingi viisteist nende plaati ja nii edasi. Aga 
samas jälle ma olen inimene, kes ei ole kunagi vaadanud Star Warsi, ma ei ole 
kunagi lugenud \emph{Hitchhiker's Guide to The Galaxy}'t. Minu jaoks  on 
esteetiline subkultuur ja arvutid natukene lahus seisnud.

\question{Endal sul BBSi ei olnud?}

Minul endal BBS-i olnud. Ma vist nagu kuidagi ei tahtnud ka, see oli ikka hull 
jahmerdamine, mis oli vajalik selle BBSi üleval hoidmiseks ja sellega pidevalt 
toimetada. Mul oli väga hea meel, et ma sain  Priidu BBS-i kasutada.

\question{Selge. Aga siis te müüsite selle mängu maha, mis edasi sai?}

Noh, kui üheksateistaastasele inimesele anda nii palju raha, nagu tema vanemad 
on kogu elu jooksul teeninud, eks ole, siis tal karjäärivalik on nagu selge 
kohe, eks ole ju. Et noh, sellist küsimust nagu ei olnud, et mida ma siis 
tulevikus professionaalselt tegema hakkan. Loomulikult programmeerija.  Ja  mul 
oli ka selline mõtlemine, ma ei tea, kui õigustatud see oli, aga ma arvasin, et 
et noh, eriti üheksakümnendate alguses Eesti ülikoolides eriti midagi väga 
kasulikku sel teemal ei õpetatud. Ma ei tea,  kui õige või vale see on. 
Kindlasti vastas tõele see, et meil keskkoolis oli  arvutiõpetus ka ja üldiselt 
ikkagi meie klassist pigem paljud teadsid rohkem kui meie õpetaja. Ma 
miskipärast oletasin, et ülikool siin samamoodi, ma ei tea, kas see on tõsi või 
mitte. Tänapäeval see kindlasti ei ole tõsi aga  tol ajal  võib-olla pigem oli. 
Igatahes ma tegin selle otsuse, et ma ei lähe ülikooli õppima midagi 
programmeerimise või arvutitega seotut, vaid ma läksin hoopis õppima füüsikat. 
Füüsika oli kindlasti mul  niisugune teine selline huviala,  ma olin 
füüsikaolümpiaadidel käinud  ja mulle see kindlasti kindlasti väga meeldis. 

\question{Aga mis sulle füüsika juures meeldis?}

No võib-olla natuke sihuke filosoofiline aspekt, et ma sain kuidagi aru, kuidas 
nii-öelda maailm toimib teatud mõttes. See oli põnev. Mingid sihukesed asjad, kui 
 mingid tuumafüüsikad ja mingid planeedid, kuidas liiguvad ja niimoodi, see 
natukene andis võib just sellist filosoofilist mõõdet. Mis see maailm meie 
ümber on ja kui suured või väikesed meie, inimesed, selles maailmas  oleme.  Ja 
noh, pigem ikkagi väga väikesed oleme. 

\question{Kuhu sa läksid seda füüsikat õppima?}

Ma läksin  füüsikat õppima Tartusse\index{Tartu Ülikool}, koos Jaan 
Tallinnaga\index[ppl]{Tallinn, Jaan}. Pinginaabrid läksid mõlemad õppima 
füüsikat. Sellega läks niimoodi, ma kindlasti  tegelikult ei väärtustanud seda, 
et piltlikult öeldes mul oleks paber taskus, et mul ülikooli oleks  kuidagi 
edukalt lõpetanud. Ja kui ma olin ühe aasta või poolteist aastat ülikoolis ära 
olnud, siis mulle hakkas veel rohkem kohale jõudma see, et tegelikult ma ju 
tegelen programmeerimisega kogu aeg, töötan professionaalse programmeerijana. 
Samal ajal tegi mind järgmist mängu, mille me kavatsesime maha müüa ja nii 
edasi ja nii edasi. Ja ma kunagi ei kavatsenud füüsikuna töötada, ma hobi 
korras õppisin füüsikat. Kui esimese aasta sai hobi korras  füüsikat õppida 
siis teisel aastal hakkad aru saama, et tegelikult  õppejõud ikkagi eeldavad, 
et sa tõsiselt tegeled selle asjaga, panustanud enamiku oma ajast füüsiku 
õppimisse. 

Ja siis ma tulin ülikoolist ära. Ma sain aru, et see asi lihtsalt nõuab rohkem 
tööd, kui ma olen nõus sinna sisse parema ja siiamaani ma ülikooli lõpetanud ei 
ole. Jaan Tallinn\index[ppl]{Tallinn, Jaan} käis ülikooli lõpuni ja õppis 
füüsika siis siis lõpuni. Tegi oma oma lõputöö, kui ma õieti mäletan, 
relatiivsusteooriast. Sellest, kuidas ruumi painutada selle jaoks, et reisida 
valguse kiirusest suuremate kiirusega ühest kohast teise. Ma küll oletan, et 
tõenäoliselt  ta mingisugust väga suurt teadmist ühiskonnale sellega juurde nii 
väga ei lisanud selle nelja aastaga, mis ta õppis aga sellise töö ta tegi. Ta 
on rääkinud, et ükskord, kui ta kuskil seltskonnas kirjeldas oma seda tööd, 
mida ta tegi, siis tema vestluskaaslane küsis  vastu, et kas see oli nagu 
rohkem teoreetiline töö või tuli seal ka mingeid praktilisi laboratoorseid 
katseajale.

\question{Selle asja nimi, mida te tol hetkel kampas pidasite, oli juba 
Bluemoon\index{Bluemoon}?}\label{sisu!bluemoon}

Jah. See mängutegijate punt, me hakkasime ennast nimetama nimega Bluemoon 
Software ja Bluemoon Interactive. Inimesed ikka tahavad panna mingisuguseid 
kõlavaid firmanimesid.

\question{Aga miks just Bluemoon>}

Lihtsalt oli üks nimi. Ma arvan, et me ei osanud nimesid üldse välja mõelda ja 
ma olen kogu aeg pidanud ennast väga halvaks nimede väljamõtlejaks ja et ma ei 
valda seda valdkonda üldse ja niimoodi, aga kui Starshipile\index{Starship 
Technologies} nime panin, siis ikkagi osalesin selles kõvasti ja  lõpuks oli 
ikkagi minu pakutud nimi, mis selleks lõpuks sai.

\question{Programmeerimise juures pidi olema täpselt üks raske asi, nimede 
välja mõtlemine}\sidenote{Eksin tsitaadiga. Täpne tsitaat on 
Netscape\index{Netscape} arhitekti Philip Karltoni\index[ppl]{Karlton, Philip} 
poolt ja kõlab nii: \enquote{\emph{There are only two hard things in Computer 
Science: cache invalidation and naming things}}}

Ma olen täitsa nõus sellega, võib-olla nüüd neljakümne aastasena on juba 
natukene rohkem käppa seda saadud. 

\question{Mis sa praegu teed?}

Praegu ma olen sellises firmas nagu Starship Technologies ja ehitan 
pakiroboteid. Asutasime selle selle firma koos Skype'i\index{Skype} kaasasutaja 
Janus Friisiga\index[ppl]{Friis, Janus}  neli pool aastat 
tagasi\sidenote{Intervjuu Ahtiga toimus jaanuaris 2019}. Ja meil oli selline 
visioon, et asjad võiksid ju maailmas liikuda automaatselt samamoodi, nagu 
elekter tuleb meile stepslisse seina ise sisse ja veevärk on olemas ja 
informatsioon tuleb läbi interneti. Aga asjad liiguvad ikkagi  läbi meie maja 
või korteri ukse, tulevad füüsiliselt kohale ja alati mingisugune inimene toob 
seda, kas sa ise tood või siis sa maksad kellelegi inimesele, kes toob. Ja see 
on hirmus raiskav ja asjad võiksid liikuda automaatselt samamoodi nagu me 
lennukipileteid broneerime üle interneti nii öelda automaatselt, ilma et me 
läheksime füüsiliselt kohale kuskile reisibüroosse seda lennukipiletit ostma.

\question{Starshipi tegemine on ju juhtimise töö. Kuidas sa jõudsid 
programmeerimise juurest selle töö juurde, mida sa praegu teed ja kui erinevad 
nad sinu jaoks on?}

No need on ikka väga erinevad. Minu jaoks on see areng olnud selline, et ma 
olin programmeerija ja ma olin programmeerija üsna kaua aega, ilma et ma oleks 
üldse midagi kuskil juhtinud. Ja kui me hakkasime startuppe tegema koos Jaanus 
Friisi ja Niklas Zennströmiga\index[ppl]{Zennström, Niklas} siis ma olin 
nendest startuppides tehnilise arhitekti rollis. Arhitekti roll on juba rohkem 
natukene nagu  juhtimisega seotud, aga sa ei juhi nii väga  inimesi või 
organisatsioone või protsesse, vaid sa juhid just tehnilist arhitektuuri. Et 
milline see masin niisuguses suures plaanis kokku tuleb, mida siis terve suurem 
tiim inimesi ehitab. Nagu maja ehitamisega: osad inimesed ehitavad ja panevad 
kive üksteise peale ja on ka teisi inimesi, kes vaatavad seda projekti 
suuremalt, et kus peaks olema paneme aken ja mitu akent me üldse teeme ja kas 
me teeme rohkem ümmargused aknad või teeme kandilised aknad ja nii edasi ja nii 
edasi. Ja ma olin Skype'is, olin siis tehniline peaarhitekt alguses  ja 
mitmetest teistes startuppides samuti. Skype'is veel natukene pooleldi juhtisin 
ka ühte väikest tiimi, kus  ma tegelesin sellega, et mõelda umbes viiele 
inimesele välja seda, mida nad tegema peaksid ja koordineerida nende tööd. 
Mõtlesin välja, mis meie eesmärk peaks olema, kuhu poole me peaksime liikuma ja 
nii edasi, nii edasi. Sihukene viie inimeselise tiimi juhtimine oli selline 
nagu väike harjutus või  sissejuhatus, et mingisuguseid kogemusi natukene sain 
või natukene kujutasin ette. Hiljem olen juhtinud siis ka natuke suuremaid 
tiime, umbes kümneinimeselisi ja niimoodi. Aga Starship oli esimene koht, kus 
ma üsna kiiresti võtsin tööle kümme inimest, võtsin esimese kahe nädalaga tööle 
umbes ja esimese poole aastaga oli juba umbes kakskümmend inimest meil tööl ja 
nii edasi  läks juba natukene suuremaks see asi. Eks ma niimoodi käigu pealt 
natukene siis  õppisin, et  kuidas juhtimine käib. Ju ma olen kindlasti veel 
üsna  alguses seal, et me oleme siin Starshipis olnud sihukeses  naljakas 
olukorras, kus nagu juhtimises ikkagi üsna kogenematu juht on olnud sellel 
firmal. Neli aastat ma olin tegevjuht ja  nüüd jõudis pool aastat tagasi siis 
asi nii kaugele, et me palkasime  professionaalse tegevjuhi Lex 
Bayeri\index[ppl]{Bayer, Lex} Californiast. Ja mina olen CTO ehk 
tehnikadirektor, kus ka peab üsna palju juhtima, aga nüüd enam mitte kahtsadat 
inimest, vaid natukene väiksemat hulka inimesi.

\question{See on siis olnud pikk ja just vajadusest ja huvist kantud õppimine?}

Jah, absoluutselt.  Üldiselt ma ütleks niimoodi, et paljud programmeerijad, 
kaasa arvatud ka mina, meile programmeerimine meeldib nii palju see on niivõrd 
tore tegevus ja niivõrd äge tegevus, et selliseid masinaid ehitada, et tahaks  
muudkui eitada neid masinaid. Inimeste juhtimine on pigem selline asi, mida 
enamik programmeerijaid väga ei taha teha ja ma ei ole päris kindel ka ise, kui 
palju mina seda tegelikult teha tahan. Aga küll on lihtsalt asi selles, et kui 
sa oled  üksikprogrammeerija ja sul kogemus tekib ja sa oled  arhitekt ja sa 
oskad juba rohkem  arvata, mismoodi me seda tarkvara peaksime ehitama ja mis 
asjad on selle juures olulisi, mis need ei ole. Siis on nagu on kaks võimalust, 
kas sa  oled vait ja kellegi teise juhtimisel osaled selles protsessis või siis 
sa üha rohkem nagu vaatad seda, et ei, ma teen ise, ma teeksin seda paremini 
kui see juht, kes meil on. Ma tahaks ise seda asja juhtida või mul on juba nii 
hea ettekujutus, kuidas seda teha, et ma ei suuda pealt vaadata, kui 
mingisugune teine inimene, kes on võib-olla väiksema kogemusega kui mina,  
kuidagi seda asja juhib ja mitte selles suunas, kus mina olen täiesti 
veendunud, et  õige oleks. Ehk siis see on tulnud justkui nagu vajadusest. Kui 
sa oled üksikprogrammeerija, siis sa aja jooksul ikkagi saad aru, et sa saad 
tegelikult lõppkokkuvõttes rohkem tehtud, kui sa piltlikult öeldes palkad 
endale tiimi ja hakkad juhtima mingisuguseid suuremaid seltskondi. 

Minu jaoks küll nii-öelda raketiga lendamine nagu Starshipis, kümneinimeselise 
tiimi juhtimisest kuni selleni, et ma juhtisin üle kahesaja inimesega firmat 
tükk aega, see ikkagi võttis pea ringi käima. Et ma kindlasti kindlasti 
edutasin ennast  oma ebakompetentsuse tasemele. Aga eks kohati öeldaksegi, et 
starupid ongi asjad, mis on väga sageli on  klassikalise sellise juhtimise 
distsipliini ja teooria ja juhtimispraktikate mõttes väga halvasti juhitud 
organisatsioonid. Mis ei ole siiski tihti takistuseks olnud nende edule, 
sellepärast et nad on olnud nii piisavalt värske mõtlemisega, nende toode on 
olnud piisavalt selline värske ja revolutsiooniline, et sellest ei ole olnud 
hullu, et nad on olnud halvasti juhitud. Tegelikult ikkagi need kakssada 
inimest, kes meie Starshipis töötavad,  ma ikkagi vaatan nende peale küll nagu 
niimoodi, et palun vabandust nende ees, et nad on osalenud sellises loomkatses, 
et mina olen neid juhtinud mitu aastat. See ei ole võib-olla olnud aus nende 
suhtes. Aga samas nad ei ole ka sugugi mitte meil siin firmast minema jooksnud 
ja tunduvad olevat rahul, et võib olla väga hullusti, siis ei olegi läinud.


\chapter{Madis Kaal}
\label{cptr:mast}
\index[ppl]{Kaal, Madis}
\index[ppl]{Mast|see{Kaal, Madis}}

\question{Kuidas arvuti Saaremaale sai?}

Arvuti ei saanudki Saaremaale. Minu esimene kokkupuude päris arvutitega oli 
Rahvamajanduse Saavutuste Näitusel\sidenote{Tänapäeva mõistes oli tegu 
messikeskusega, kus ajutistel või püsinäitustel demonstreeriti 
kas liiduvabariigi (nagu Tallinnas asunud näituse puhul) või kogu Nõukogude 
Liidu majanduslikku võimekust. NSVLi Rahvamajanduse Saavutuste Näitusest arenes 
välja Eesti Näituste Messikeskus.}, praeguses Pirita näitusehallis. Käisin 
seal koos oma emaga. Ühes nurgas olid üles pandud 
terminalid, mida manageeris kaks imeilusat tüdrukut. Seda siis ajaloolisest perspektiivist, tõenäoliselt oli tegemist üsna keskmiste 
operaatoritega, aga siis tundusid nad imeilusad ja targad. Terminalide peal oli 
nõukogudeaegne venekeelne raamatukogude otsingu andmebaas. Terminalid ise olid ka 
venekeelsed. See oli esimene kord, kui ma reaalselt nägin, et ekraanil olid 
tähed ja klaviatuuril sai kirjutada. 

\question{Mis aastal see oli?}

Arvatavasti 1983. Ja need terminalid jätsid kustumatu mulje. 

\question{Kas pärast seda tekkis sul selge soov terminalide 
juurde pääseda?}

Pärast seda tekkis väga selge mõte, et see asi huvitab mind. 
Seejärel sattusin Tartusse ja ostsin sealt venekeelse 
raamatu \enquote{Programmeerimine keeles PL/I\index{PL/I}} ning lugesin 
seda. Ma ei teadnud arvutitest veel midagi, aga tasapisi hakkas selgeks saama, 
misasi on programmeerimine ja näiteks \verb|for|. See oli mingi imeline struktuurkeel, mitte päris vene, 
vaid kõlas nagu piraatversioon.

Järgmine kord nägin arvuteid Tehnikaülikooli\index{Tallinna 
Tehnikaülikool}, tolleaegse TPI\index{TPI} lahtiste uste päeval, kus me käisime 
pinginaabriga, kellega koos pärast ka kooli sisse astusime. 
Meile tehti ekskursioon automaatikateaduskonna kõigis 
kateedrites\index{Tallinna Tehnikaülikool!Automaatikateaduskond} neljal 
korrusel ja mõnes kohas olid arvutid. Mäletan selgelt, et Indrek 
Saul\index[ppl]{Saul, Indrek}, kes oli minu meelest sel ajal tudeng ja hiljem 
kinnisvaraärimees, näitas meile analoogarvutit. Sellega
sai analoogpingete ja skeemiga diferentsiaalvõrrandeid lahendada.

\question{Vanasti sihiti ju õhutõrjekahureid analoogarvutitega.}

See masin võis täiesti olla sedalaadi projekti osa. Igatahes mul tekkis kindel soov seda valdkonda
õppima minna, aga pinginaaber veenis mind ümber, et lähme parem 
raadiotehnikasse, ikkagi sama maja.

\question{Kas esimest korda arvuti nägemise ja ülikooli sisseastumise vahele jäi veel 
midagi arvutitega tegelemise mõttes?}

Ainult see üks raamat. Otsus arvuteid õppima minna sündis esimesel korral ja raamat tuli 
pärast seda. Ainuke imelik asi oli otsus raadiotehnikasse 
minna, aga selle vea parandasin ruttu ära. Ülikooli teise korruse otsas oli arvutussaal, kus oli kaks või 
kolm SM-4\index{SM EVM!SM-4}. Need olid PDP-11\index{PDP-11} vene 
versioonid. Pärast seda, kui sain aru, kuidas sinna sisse saab, ma enam 
tundidesse ei jõudnud. Ja kuna olin maalt tulnud poiss ja raha ka üldse ei olnud, 
käisin lihtsalt kõik kateedrid läbi ja küsisin iga ukse vahelt, kas neil on tööd anda. Raadiotehnika kateedris\index{Tallinna 
Tehnikaülikool!Automaatikateaduskond!Raadiotehnika kateeder}\label{sisu!mast_raadiotehnikas} oli, 
mind võeti sinna laborandiks tööle ja nii see läks. Kool jäi pooleli, kateedrisse
jäin seitsmeks aastaks paika.

\question{Mitmendal kursusel kool pooleli jäi?}

Esimesel kursusel. Algul olin raadiotehnika kateedris laborant ja pärast 
tehnik. Sattusin tuppa, kus olid väga toredad inimesed: Mart 
Palmas\index[ppl]{Palmas, Mart}, kes õpetas mulle peaaegu kõike, mida ma 
programmeerimisest tean, ja Villem 
Vannas\index[ppl]{Vannas, Villem}, kes praegu töötab Datelis\index{Datel}. Tema 
õpetas mulle enam-vähem kõike, mida ma rauast tean.

\question{Siis ei jäänud ju haridus pooleli.}

Formaalselt siiski jäi. Tol ajal oli
laborant rohkem nagu abitööline. 
Parandasin seda, mida vaja, aitasin seal, kus vaja. Mu esimene töö oli 
kolikamber tühjaks tõsta.
Algusaegadel oli üsna suvalisi projekte, hiljem tekkis
suund kommunikatsiooni poole, mis tundus mulle sel ajal huvitav. 

1990. aasta paiku tekkis Eestis 
mitu huvitavat suunda. Kõigepealt hakkas tulema 
personaalarvuteid. Sinnasamasse, kus oli kunagi SM-4 arvutiklass, tekkisid 
personaalarvutite klassid. Neid oli mitu tükki ja erinevate portsudena 
toodi Austraaliast MicroBeesid\index{MicroBee}\sidenote{1982. aastal 
Austraalias algselt komponentide komplektina müügile tulnud koduarvuti. Tuntud 
huvitava videolahenduse ja patareitoitel mälu poolest, mis 
võimaldas arvutit teisaldada mälu seisu kaotamata.}. Kuskilt tuli terve klassi jagu MSXi 
arvuteid\index{Yamaha MSX} ja siis mõned 
Robotronid\index{Robotron}\sidenote{Robotron (originaalis VEB Kombinat 
Robotron) oli Ida-Saksamaa suurim arvutitootja.}. 
Raadiotehnika kateedris oli juba siis, kui mina sinna sattusin, olemas 
Apple II\index{Apple II} ja mõned aastad hiljem tekkis sinna IBM 
PC\index{IBM PC}. See oli omapärane kogemus. Apple II peal olid 
harjunud, et lülitad sisse ja pilt on ees. IBMi sisse lülitades ei juhtunud midagi. Ühel hommikul tööle tulles vaatasin, et uus 
arvuti, ja lülitasin sisse. Midagi ei juhtunud. Ootasin natuke aega ja lülitasin välja, ise 
tegin näo, et midagi pole toimunud. Hiljem selgus, et masin tegi \emph{self 
test}'i. Seal oli tublisti mälu sees ja testimine võttis palju aega -- ma ei 
suutnud nii kaua oodata. 

\question{Midagi pidi see ju ekraanil senikaua näitama?}

IBMil oli roheline long-fosfor\sidenote[][-1.3cm]{Katoodkiirtel põhinevates monitorides suunati laetud osakeste kiir fosforühendiga kaetud ekraanile. Kasutatud ühendi tüübist sõltus nii elektronkiire mõjul tekkinud värv kui ka see, kui kauaks ekraan peale kiire edasi liikumist helendama jäi. Selle viimase järgi liigitataksegi ekraanides kasutatavaid fosforühendeid  \enquote{pikkadeks} ja \enquote{lühikesteks} (ingl. \emph{long} ja \emph{short}), seda Mast ilmselt silmas peabki.} monitor, mis läks tükk 
aega käima, ja ma ei jõudnud esimese \emph{boot}'imise ajal ära oodata, millal 
midagi toimuma hakkab.

Üheksakümnendate paiku tekkis meile tuhande kahesajane modem, mis läks 
PC sisse. Sel ajal olid just tulnud esimesed BBSid ja umbes samal ajal otsustas TPI 
automaatikateaduskond\index{Tallinna Tehnikaülikool!Automaatikateaduskond} 
ehitada arvutivõrgu. Toodi kohale viiesajameetrine kaablirull 
kollast sõrmejämedust Etherneti kaablit ja umbes kümmekond komplekti 
kobakaid kaabli peale, mille külge käis teine jäme kaabel, 
mis läks võrgukaardi taha. See oli nagu esimene Etherneti tehnoloogia. 
Mäletan selgelt, et meile toodi ainult kaabel ja kobakad, ei mingeid tööriistu, pistikuid ega terminaatoreid.

Kateedris oli sel ajal eterniiditahvlitest lagi, mille peale me selle Etherneti kaabli tõmbasime. Et kobakad külge saada, tegime
naaskliga kaabli kesta sisse augud, ajasime nõela läbi ja ühendasime
arvutite külge ning tinutasime otsa terminaatorid ja takistid. 

\question{Tarmo Mamers\index[ppl]{Mamers, Tarmo} rääkis, kuidas te PC ja Maci 
vahele traati vedasite. Kas too kaabeldamine oli enne või pärast seda?}

See oli meil kahe PC -- sellesama raadiotehnika kateedri PC ja Tarmo oma -- vahel, Tarmol oli 
veidi vägevam AT arvuti. Ühendasime need 
kaabliga ja tegime väikese 
\emph{chat}'i programmi, et teineteisega suhelda. 

Arvutivõrk tekkis sellest hiljem. Tehnikaülikooli toodi Novell 2.15\index{Novell} 
server, mille ma installisin ja mis oli üks esimesi väheseid asju, millel oli manuaal 
olemas, nii et kõik oli justnagu ametlik. Novelli serveri peal panin käima Pegasuse 
Maili\index{Pegasus Mail}-nimelise asja, kuhu külge kirjutasin \emph{gateway}, 
millega sai UUCP meili, mida toimetati Küberi 
majja Soomest (ma ei mäleta, kas Soomest siiapoole lükates või siit 
üle telefoniliini tõmmates). Tõmbasime selle oma pisikese modemiga 
Tehnikaülikooli majja ja jagasime kasutajate vahel laiali.

\question{Siin tundub jälle suuremat sorti lünk olema selle vahel, kuidas sulle 
hakati programmeerimist õpetama ja kuidas sa naaskliga kaablit torkisid ja 
\emph{gateway}'sid programmeerisid.}

Mõned aastad tuli õppida asjade kirjutamist lihtsalt erinevaid asju tehes ja ehitades, aega katsetamiseks oli palju. 
Olin noor inimene, peret polnud ja praktiliselt elasin raadiotehnika 
kateedris\index{Tallinna Tehnikaülikool!Automaatikateaduskond!Raadiotehnika 
kateeder}. Meil oli seal omamoodi seltskond: arvutussaali 
kamp, Tarmo\index[ppl]{Mamers, Tarmo} kohe kõrval sama 
koridori peal ja mina üleval raadiotehnikas. Vana kooli mees 
Lõvi\index[ppl]{Lõvi} oli kõrvalkorpuses ja käis aeg-ajalt Apple II peal 
oma projekte arendamas.

\question{Kas meetodiks oli siis peamiselt katsetamine, mitte 
manuaalide tudeerimine?}

Manuaale ega dokumentatsiooni ei olnud üldse. Riiklikul 
tasemel tarkvara varastamise programm pakkus küll ägedat tarkvara, aga 
enamasti ilma dokumentatsioonita. See oli nagu infovaakumis 
tegutsemine ja disassembler\sidenote{Programm, mis teeb masinkoodist 
oluliselt loetavamat Assemblerit.} oli justkui sõber.

\question{Keegi pidi sulle ju ometi ütlema, et selline asi nagu disassembler 
on olemas.}

Jaa, seda tegid head vanemad kolleegid, kes hoidsid kätt ja 
juhendasid. Lõviga\index[ppl]{Lõvi} tegutsesime pikalt koos, temal oli kindlasti 
väga suur mõju minu arengule. Aga see lünk, kuidas ma BBSideni 
jõudsin, sai täidetud nii, et mul oli raadiotehnika kateedris\index{Tallinna 
Tehnikaülikool!Automaatikateaduskond!Raadiotehnika kateeder} 
arvuti, mille sees oli modem ja millega sai helistada. Lähim BBS
asus Küberneetika Instituudi otsas, kus tollal asus 
Proekspert\index{Proekspert} ja kus nüüd on Tehnopoli kontor. Andrus Suitsu\index[ppl]{Suitsu, Andrus} 
oli BBSi mees, käisin tema juures oskusteavet ja tarkvara 
hankimas. Panin algul BBSi ja peatselt pärast seda ka Fido, algul vist
\emph{point}'i, ja käitasin seda üsna mitu aastat. 

\question{Miks sa seda tegid?}

Huvist kommunikatsiooni vastu.

\question{Kas sa mõtled kommunikatsiooni masinate või inimeste vahel?}

Mõlemat. See moment, kui täielikust infopuudusest saab järsku täielik 
infovabadus, on väga ergastav. Tänapäeva inimestel, kellel on internet olemas, ei 
kujuta ette, kuidas saab olla ilma, aga ilma oli väga pime.

Üks asi oli tehniline info, aga Fidoneti ja Useneti grupid
(UUCP meiliga koos toodi ka Useneti gruppe) olid ka muidu väga 
huvitavad. Sealsed diskussioonid olid väga 
informatiivsed. Suurem osa 
juttudest olid muidugi tehnilised, sest seal käisid tehnikud ehk need, kes 
said kanalile ligi.

\question{Kas too kollase kaabliga võrk hakkas tööle ka?}

Ikka, see töötas uhkelt. Novelli server käis veel 1992. aastal, kui ma 
sealt ära läksin. Inimesed said omavahel meilida ja ka välismaailmaga 
suhelda. Ainukene probleem oli see, et arvuteid, millel oli see 
Etherneti äge \emph{interface}, oli suhteliselt vähe, paar tükki kateedri 
peale vist suudeti tekitada.

\question{Kas Etherneti kaart oli defitsiit?}

Tol ajal oli kõik defitsiit, siis oli veel rublaaeg. Millise projekti 
raames see toodi, ei tea. Avo Ots\index[ppl]{Ots, Avo} tegi minu meelest 
doktoritöö selle kohta, kuidas ehitada arvutivõrku. See oli 
oluline kogemus, et toimuks järgmine samm. Pärast tehnikaülikooli 
töötasin lühikest aega Microlinkis\index{Microlink}, kus ma olingi 
arvutivõrkude installeerija ja ühtlasi .EXE\index{.EXE} kirjutaja.

\question{Miks sa sinna läksid?}

Ühel päeval astus uksest sisse Margus 
Kliimask\index[ppl]{Kliimask, Margus}, keda ma teadsin Rainer 
Nõlvaku\index[ppl]{Nõlvak, Rainer} kaudu, ja tegi ettepaneku hakata 
tegema ajakirja. Sellest sai .EXE.

\question{Miks ikkagi? Jälle kõlab suure muutusena, et ühel päeval tõmbasid kollast 
kaablit ja järgmisel päeval tegid ajakirja.}

Täpselt nii oligi. Ma arvan, et Rainer tahtis Microlinki promo teha. 
See võis olla suur motivaator, aga seda peab Rainerilt endalt küsima.

\question{Kust sul üldse tuli mõte, et ajakirja tegemine võiks huvi pakkuda?}

Tundus huvitav. Mul ei olnud siis rohkem kõrgeid eesmärke kui see, et elu oleks 
huvitav.

\question{See on tegelikult kõige kõrgem eesmärk, mis üldse saab olla.}

Algul oli jutt, et teeme ajakirja, ja siis selgus, 
et mul oleks ka uut töökohta vaja. Nii sattusingi korraks Microlinki\index{Microlink}. Olin seal aga loetud kuud, sest 
siis hakati tegema Eesti Forekspanka\index{Eesti Forekspank}\sidenote{Eesti 
Forekspank sündis 1992. aastal ja ühines 1995. aastal Raepangaga\index{Raepank} 
1995.}. Pangal olid oma sidevajadused ja mind kutsuti sinna tööle.

\question{Üheksakümnendate algus oli Eesti panganduses ju hull aeg!}

Jah, ja Forekspank oli sel ajal pisikene valuutavahetuskontor, mis opereeris 
rubla-dollari börsi.
See tegutses tolleaegses hulgifirmas Abestok\index{Abestok}. Selle ühes toas olid 
inimesed, kes otsustasid panga teha. Margus 
Kliimask\index[ppl]{Kliimask, Margus} oli nendega seotud, vist IT-poisi 
staatuses. Temaga läksimegi Rävala puiesteele, istusime koos 
ehitusjuhiga ühte tuppa, mille ühes nurgas hoidsid
ehitajad oma tööriistu, ja ehitasime panka.

\question{Kust tekkis mõte, et panga tegemiseks ei piisa kilekottidega 
sularaha edasi-tagasi lohistamisest?}

Need mehed, kes panga tegid, olid piisavalt targad, mõistmaks, et pank käib 
teistmoodi. Kui palju teistmoodi, sai alles siis selgeks, kui 
Inglismaalt osteti pangatarkvara ja konsultandid rääkisid, kuidas panka 
tehakse. Aga see ei olnud kohe esimesel aastal. Esimestel aastatel ehitasime, 
tõmbasime kaablit ja panime laua alla püsti serveri. Ühel ilusal päeval lükkas
Margus Kliimask\index[ppl]{Kliimask, Margus} kogemata varbaga 
toite välja ja pank jäi seisma. Aga mitte kauaks. 

Nii Rein Usin\index[ppl]{Usin, Rein}, Ivar Lukk\index[ppl]{Lukk, Ivar} kui ka Margus Kliimask\index[ppl]{Kliimask, Margus} olid 
visiooniga inimesed. See pidi olema suhteliselt algusaastatel -- BBSid ja 
Fidonet olid siis veel kuum teema --, kui Margus Kliimask ütles, et teeme 
modemipanga. Tal oli kindel mõte, et see peab olema Norton 
Commanderi\index{Norton Commander} F2 menüüs\sidenote{1986. aastal turule 
tulnud ja 1998. aastal viimase versiooni saanud Norton Commander oli 
ülipopulaarne failihaldur MS-DOSi platvormile. Ekraanil oli korraga kaks 
nimekirja faile ja käsurida, allservas nimekiri saadaolevatest 
klahvivajutusega käivitatavatest käskudest. Nii oli kasutajal ilma suurema 
koolituseta kohe selge, mida ja kuidas teha. Ohtralt kasutati F-klahve 
ja neist olulisemate funktsioonid on inimestel siiani peas (F3 -- faili sisu 
vaatamine, F5 -- faili kopeerimine).}. Kõik kasutasid Norton Commanderit 
ja kõigil oli see olemas, aga keegi ei ostnud, sest tol ajal tarkvara ei ostetud. 

\question{Jah, ma mäletan poes karpe, aga ei mäleta, et keegi oleks neid kunagi 
ostnud.}

Hämmastav oli see, kuidas mõtte väljakäimisest 
modemipanga \emph{launch}'ini läks umbes kaks kuud.

\question{Tegite kahe kuuga nullist modemipanga?}

See oli programm, mis oli mingil määral Norton Commanderiga integreeritud: 
läks sealt menüüst käima, nägi välja nagu Norton Commanderi 
osa, võimaldas makseid ette valmistada, kontoväljavõtteid ja panga teateid 
saada ning enda makseid panka saata.

\question{Ja teisel pool võttis mingi asi kõned vastu, suhtles panga 
tuumaga ja tegi arveldused ära?}

Just. Panga tuumaga suhtlemine oli üsna traagelniitidega asi, kuna selleks 
ajaks oli juba toodud Inglismaalt panga tarkvara, millel ei olnud ühtegi head 
liidest peale terminali.

\question{Ja siis tegite terminali emulaatori?}

Mina jah kirjutasin terminali emulaatori ja üks kolleeg kirjutas programmi, mis 
lükkas emulaatorist maksed pangasüsteemi, ning see toimis 
aastaid niimoodi, enne kui tekkisid tehnilised vahendid, et seda 
natukene viisakamalt teha. \emph{Launch} toimus 
tolle aja kohta suure pressikäraga: tehti korralik meediaüritus, imekenad 
Hansapanga\index{Hansapank} tüdrukud istusid ka seal ja tegid märkmeid. Ja läks mööda vaid
mõni kuu, kui Hansapangal tuli välja
Telehansa\index{Telehansa}.

\question{See kamp, kes tollal
BBSides suhtles, võis olla kokku paarsada inimest. Kust tulid
kliendid modemipangale?}

Kliendid jagunesid umbes pooleks. Forekspanga klientuurist arvestatav 
protsent oli Venemaal, sest suur 
raha oligi tol ajal Venemaal, aga ka Eesti klientuur ei olnud sugugi kehv. Pank 
müüs seda suhteliselt suure summa eest ja Eesti firmad 
ostsid. Käisin seda ise Tallinnas installeerimas. Küsimus ei olnud 
selles, et inimesed ei saanud tulla maalt linna pangaasju ajama, vaid nad 
lihtsalt ei tahtnud kontorist välja tulla. Pangas sai mugavalt ära käia 
laua tagant püsti tõusmata.

\question{Ja see kõik tasus ära, et hakata isegi arvutiga 
makseid ette valmistama?}

Sel ajal oli igas firmas raamatupidamiseks arvuti olemas ja raamatupidajate 
arvutites maksed olidki. Ilmselt mugavus ja aja kokkuhoid tõukasid
Eesti firmad sinnapoole.

\question{Kui palju seal telefoniliine küljes oli?}

Alustasime kahega ja lõpus oli vist kuus. Kuna 
sideseanss oli nii lühike, mahtus enamik sideseanssidest paari minuti 
sisse. Kõik pakiti kohapeal kokku ja saadeti ühe portsuna ära -- Fidonetist õpitud tehnoloogia. Alguses tegin mina kliendipoole ja 
Margus Kliimask\index[ppl]{Kliimask, Margus} kirjutas serveripoole. Hiljem kirjutasin serveripoole veidi paremaks, et see oleks paremini eskaleeritav.

\question{Mida see tähendab?}

Ühe masina taha sai panna mitu modemit.

\question{Kas sa oma BBSi hoidsid siis veel püsti?}

Minu meelest oli meil pangas ka BBS veel mõnda aega, 
Microlinkis\index{Microlink} oli kindlasti. Kuna Forekspank asus Rävala puiesteel, siis kohe, kui 
üheksakümnendate alguses tekkis internet, oli selge, et meil on ka 
seda vaja. Tõmbasime koos Andrus Aaslaiuga\index[ppl]{Aaslaid, 
Andrus} oma valgete käekestega mööda majakatuseid Forekspanga kõrvale KBFI\index{KBFI} majja, 
kus sündis Uninet\index{Uninet}, Etherneti kaabli.

\question{Te olite siis otse Unineti küljes?}

Otse Unineti küljes, olime ühed esimesed kliendid, kodukootud 
ruuteri softiga, mis läks flopi pealt käima. Mõlemas otsas oli üks 
arvutikast ja nii me ennast internetti panime. Muide, ükskord 
lõi meil sinna välk sisse.

\question{Mida te internetis tegite?}

Algul õppisime, mis see on. Ja pangas oli hädavajalik meilivahetus, et suhelda. Üks esimesi asju, mis pangas sai 
ehitatud, oli teleksi \emph{gateway} Pegasus Maili\index{Pegasus Mail}. 

\question{Misasi on teleks?}

Teleks oli viiekümneboodine\sidenote{\emph{Baud rate}, eesti keeles lihtsalt \emph{boodid}, 
näitab, mitu korda sekundis signaal liinil muutub andes indikatsiooni side kiirusest.}  telegraafisüsteem. Kahtlustan, et paljud pangad maailmas kasutavad seda endiselt. Suhtlus ei käi telefoniliini 
pidi, vaid selleks on eraldi teleksivõrk, mis toimib mööda telefonitraate 
hoopis teistsuguste signaalidega kui tavaline telefon.

\question{Kas see oli \emph{circuit switched}\sidenote{Ahelkommuteeritud. 
On ju ilus eestikeelne sõna?}, eks? Siis see vajas eraldi keskjaama?}

Jah. Põhimõtteliselt tuli ikkagi kõne teha ja ühendus püsti seada. 
See ehitati veel sel ajal, kui olid teletaibid -- klaviatuur ja 
paberirull.

\question{See \emph{gateway} ei saanud siis ju olla ainult tarkvaraline, vaid
oli ka riistvara vaja?}

Jah. Seal oli üks kast vahel, mis tegi sellest jadapordi. Esimese kasti tegi minu meelest
Küberneetika Instituudi\index{Küberneetika Instituut} majas üks Sass, Aleksander\index[ppl]{Reitsakas, 
Aleksander}.

See oli väga keeruline kast, tegin hiljem sellest peopesasuuruse 
versiooni flopikarpi.

\question{Mind hämmastab see, et sa ehitasid järjest keerulisemaid asju, aga kust sul tulid selleks teadmised, seda ei selgu.}

See on nagu Youtube'i videot vaadates -- tundub, et kõik asjad juhtuvad ise. 
Vahepeale mahtus siiski kuude kaupa õppimist, häkkimist ja katsetamist.

\question{Sul pidi hirmus kihu seda teha olema.}

Kindlasti, peaasi, et oli huvitav. 
Pangas töötades hakkas esimest korda ka kohusetunne vaevama, sest kui pank hommikul ei toiminud, olin ju mina paha.
Töötunde kulus kõvasti, aga üksiku inimesema ei olnud mul eriti muid kohustusi.

\question{Lisaks rääkisid muudkui teistega juttu BBSides.}

Panga ajal enam mitte, siis võttis töö kogu aja ära. Varem toimus jah BBSides suhtlus, aga kui tuli internet, võttis meilindus asja üle. Meiliga tuli kohe ka  \emph{gateway} 
kohe panga serverisse. Pank oli selles mõttes väga hästi kommunikeeruv.

\question{Legend räägib, et sina kirjutasid esimese eestikeelse klaviatuuri draiveri, 
on see tõsi?}

Nii ja naa. Rainer Nõlvak\index[ppl]{Nõlvak, 
Rainer} leidis esimesena, et klaviatuuril võiks eestikeelne \emph{layout} 
olla. Veel enne, kui infotehnoloogid jaole said, tellis Rainer eestikeelse 
klaviatuuri ära.  Nii et pärast, kui kehtestati  uus standard (EVS 8:1993),  
olid olemas klaviatuur ja oli kirja pandud standard. Lisaks klaviatuurile oli aga vaja ka standardile vastavat 
lokalisatsiooni. Eriti hull lugu oli Windowsi fontidega -- sel ajal oli olemas
Windows 3\index{Windows}. Ja siis korraldati konkurss, kus kõik lähenemised 
olid lubatud.

\question{Kes konkursi korraldas?}

Ma ei mäleta organisatsiooni nimesidenote{Tegemist oli Eesti Informaatikafondiga\index{Eesti Informaatikafond}, sellest sai hiljem Eesti Informaatikakeskus\index{Eesti Informaatikakeskus}, Riigi Infosüsteemi Ameti\index{Riigi Infosüsteemi Amet} eelkäia.}, aga see oli riiklik 
konkurss, mille auhind oli tolle aja kohta täitsa korralik, vist kakskümmend 
tuhat krooni. Olime selleks ajaks Raineriga juba natuke sel alal 
koostööd teinud -- Microlink pani enda klaviatuure müües kaasa draiveri, mis seda 
\emph{layout}'i toetas ka, nii et osa tööd oli juba tehtud. Kui konkurss 
välja kuulutati, ütles Margus Kliimask\index[ppl]{Kliimask, Margus}, visiooniga mees,
et teeme nii, nagu Microsoft teeb. Me \emph{reverse 
engineer}'isime kogu selle DOSi lokalisatsiooni ja klaviatuuri draiverid ning 
tegime installeerimisprogrammi, mis paigaldas 
standadkomponendid: \verb|KEYBOARD.SYS|i, \verb|COUNTRY.SYS|i ja muud
sellised asjad. Kuskilt õnnestus hankida soft, mis tegi Windowsi 
fonte, ja ma joonistasin fondid ka. See ei olnud küll kuigi hea soft, 
ei teinud TrueType'i \emph{hint}'ingut; \emph{kerning} vist 
on see teine, mis teeb fondid ilusaks, kui need väikseks muudad. Eesti 
fondid paistsid ekraanil karvased, aga me ei saanud sinna kahjuks midagi parata. Igal juhul
oli meie lähenemine teistega võrreldes nii palju parem, et võitsime konkursi.

\question{Kas pank läks konkursile osalema?}

Ei, ainult meie Margus Kliimaskiga\index[ppl]{Kliimask, Margus}. 
Meil oli pisike OÜ, koos pangaga tehtud ühisfirma Forex Communications modemipanga müümiseks. 
Selle firma alt osalesimegi. 

\question{Ja osalesite seepärast, et tundus huvitav?}

Sinna läksime ilmselt raha pärast ja võibolla ka 
Näitusepaviljonis toimunud joomingu pärast, mille seesama riiklik asutus piduliku sündmuse puhul 
korraldas.

\question{Kas sul sellepärast saigi panga aeg otsa, et pank sai valmis?}

Pigem pean olema tänulik pangajuhtidele, kes andsid meile 
hämmastavalt vabad käed igasugust tehnoloogiat katsetada ja uurida ning mõelda 
uusi asju. Tänu sellele oli Forekspank ka üks esimesi internetipanga tegijaid -- meil oli olemas internetiühendus ja me juba mõistsime, mis toimub. 

\question{Millega tollast internetipanka tehti?}

Forekspanga esimene internetipank oli minu meelest 
IISi\sidenote{1995. aastal turule toodud \emph{Internet Information 
Server (IIS)} oli Microsofti veebiserver, mis üritas (mõnevõrra tulutult) 
konkurentsi pakkuda tol ajal domineerinud Apache'i veebiserverile.} peal ja töötas
Windowsis\index{Windows}. 

\question{Eksootiline valik tolle aja kohta ...}

Oli küll imelik valik. Aga sel ajal olid meil juba arendus- ja 
hooldusmeeskonnad eraldi. Margus Kliimask\index[ppl]{Kliimask, Margus} oli 
arendusmeeskonnas. 

\question{Ehk te olite \emph{DevOpsist}\sidenote{Arendusmetoodika, kus tarkvara 
ehitamine ja selle edasine käitamine korraga nime kaotavad ehk omavahel 
lahutamatult kokku saavad.} astunud sammu tagasi?}

Panga käigushoidmine ongi natuke omapärane tegevus. Margus 
\index[ppl]{Kliimask, Margus} juhtis internetipanga arendust, tema meeskonnas
oli ka Pronto\index[ppl]{Pronto|see{Raja, Tanel}}\index[ppl]{Pronto}\sidenote{Vt lk\pageref{sisu:pronto}.} ja veel paar 
hakkajat selli. 

\question{Kas sina olid ka sellega seotud?}

Mina ei olnud internetipangaga peaaegu üldse seotud. Sel ajal oli
modemipank veel põhikanal, kuna internet oli siis vähestel. Forekspank oli juba üsna suureks kasvanud, 
hooldusmeeskonnas oli kümmekond inimest.

\question{See kõlab juba nagu terve organisatsioon, kahe telefoniliiniga ei saanud enam 
hakkama?}

Sel ajal tekkisid teised probleemid. 
Pangale ostetud tarkvara käis kummalise IBMi platvormi peal, mida aeg-ajalt 
tuli \emph{upgrade}'ida. Selle tarkvara jaoks oli COBOL uus keel. 
Tarkvara oli kirjutatud imelikus keeles nimega \emph{Report Generator Language}, mis 
oli pärit System/36\index{System/36}\sidenote{System/36 oli IBMi poolt 
1983. aastal turule toodud väike mitme kasutaja jaoks mõeldud mitmetegumiline 
server, mida programmeeriti peamiselt platvormipõhises RPG II\index{Report Program Generator} (\emph{Report Program Generator - RPG}) keeles.} ajast. Sellest keelest 
kumasid perfokaardid ikka veel kõvasti läbi.

\question{Vähe sellest, et teil oli visioon, aga raha pidi ju ka olema, et brittide juurde 
minna.}

Server maksis sel ajal meeletu raha. Algul ei olnud pangal jaksu õiget masinat osta, hangiti üks karm 
PC ja selle peal käis System/36 emulaator, millel jooksis 
panga tarkvara. Õnneks kasvasime sellest üsna ruttu välja. Pärast oli meil selline unikaalne platvorm nagu
AS/400\index{AS/400}\sidenote{AS/400, hiljem tuntud kui 
\enquote{System i}, oli IBMi keskmise suurusega serveriplatvorm, mis 
toodi turule 1988. aastal.}, mida ka korduvalt uuendati.

Ilmselt sai pank tarkvara ostes 
ka teadmise sellest, kuidas panka teha. See oli võibolla 
rohkem väärt.

\question{Teil oli Margusega juba siis kahe peale pisike OÜ, aga mõni 
veedab terve elu oma huvi üksnes akadeemilistes sfäärides rahuldades. Kust sul tekkis
arusaam ärist?}

Nagu ma mainisin, siis OÜ sündis modemipanka tehes ja pean jällegi kiitma 
tolleaegseid pangajuhte, kellega koos me ühisfirma lõime. Otseselt äritegemist kui sellist ei olnud: meie 
kirjutasime tarkvara ja inimesed maksid selle eest OÜ-le, pärast 
jagasime pangaga raha ära. Klassikalise äri mõistes ei pidanud meie midagi 
müüma, pank müüs. Muidugi tekkis ettekujutus näiteks
raamatupidamisest, aga erilist ärisoont see minus ei arendanud.
OÜ käigushoidmine mingit tähelepanu ei nõudnud, kogu fookus oli tehnoloogial.

\question{Tõnu Samuel\index[ppl]{Samuel, Tõnu}\sidenote{Vt lk \pageref{sisu:tonu}.}  rääkis mulle, et Mastsidenote{Ehk siis käesoleva loo kangelane.} oli see mees, kelle juurde sai minna riskantsete 
asjadega. Kui oli vaja emaplaadi peal vaibanoaga radu lahti kratsida 
ja sinna relee vahele panna, siis Tõnu teadis, mida teha, aga ei 
julgenud. Seevastu Mast julges.}

Ilmselt oli abiks raadiotehnika kateedri kool. Kui saad aru, 
mida teed, siis sa ei karda lõigata.

\question{Nii et sul sellist aukartust masina ees ei olnud?}

See kadus suhteliselt vara ära, kuna raadiotehnika kateedri Apple 
II\index{Apple II}s oli mitu laienduskaarti sees. Kui sellel oli
kaas peal, siis kuumenes üle, aga kaas ei olnud kunagi peal. Seal võis 
vabalt näppupidi sees sobrada ja mitte keegi ei öelnud, et sa ei tohi seda kivi 
välja võtta. Kõik oli pesades, kõike võis välja võtta. Kui katki läks, siis 
tuligi võtta. 

\question{Kas läks katki ka?}

Ikka läks, aga Apple II\index{Apple II} oli 
lihtsa loogika järgi ehitatud, Vene kivid läksid sinna asemele ja taktsagedus oli üks 
megaherts. Seda sai parandada ja see oli väga õpetlik. Ka 
esimese IBM PC\index{IBM PC}ga tulid kaasa (meil olid 
kõik juhendid olemas) BIOSi \emph{listing}'ud ja skeemid. Kõik olid 
standardtükid, kõike sai parandada ja parandatigi. 

\question{Mida sa pärast panka tegid?}

ITd ühele väikesele investeerimiskontorile. Kirjutasin 
Exceli Visual Basicus\index{Visual Basic} väärtpaberite 
kauplemise programmi. Tol ajal tehti paljusid asju Excelis, näiteks arvutati intressi. Tegin suured Exceli makrod, millega sai 
väärtpaberiportfelle hallata ja tehinguid jagada. 

\question{Kas jälle selle pärast, et oli huvitav?}

See oli rohkem vajaduspõhine. Meie enda investeerimiskontoril oli seda 
vaja ja ühe koopia müüsin maha ka. 

\question{Nii et tegelesid siiski ka müügiga?}

Ma ei tegelenud müügiga. Enamasti oli nii, et keegi tuli ja ütles, et tal 
oleks ka vaja. 

\question{Kui on väärt asi, siis lõpuks ikka tullakse.}

Jah, kui hind sobis, siis miks mitte.

\question{Sa oled BBSummeri\index{BBSummer} kuulsa grupipildi peal. Kas käisid tolle seltskonnaga läbi, kuigi töö võttis enamiku ajast ära?}

BBSummerid algasid siis, kui olin alles tehnikaülikoolis, ja neid ei olnud üldse palju. See grupipilt, mida sina vist 
mõtled\sidenote{Memcpy podcast'i kaanepildiks olev foto, kus on peal 
hämmastavalt paljude suurte asjade toonased või hilisemad algatajad.}, ei ole esimesest BBSummerist, vaid teisest või kolmandast, kus käisid ka FidoNeti tublid mehed Soomest. Seal 
pildil on üks habemega mees nimega Ron Dwight\index[ppl]{Dwight, Ron}, kes 
oli FidoNeti kunn Euroopas, regiooni pealik. Ron 
oli väga tore mees, ma olen tal isegi paar korda külas käinud ja tema juures Soomes 
ööbinud, kui piirid lahti läksid. Ja ma ei ole Eesti kambast ainukene, kes tal
külas käis. 
Soomlased, kes FidoNeti Soomes vedasid, olid tol ajal üldiselt väga toetavad. 
Sa oled teistega rääkinud, kuidas te Soome helistasite, ja keegi ei ole 
maininud, et tegelikult algusaegadel helistasid soomlased siia. Ei olnud nii, 
et ainult sealt oleks tõmmatud. Hiljem, kui BBSid ja firmad said siin jalad 
alla, saime "rinnapiima" otsast lahti, aga algusaegadel 
soomlased toetasid meid tublisti. 

\question{Kas puhtalt missioonitundest? Hõimuvelled ja nii?}

Ma ei tea, kui palju hõimuvendlus rolli mängis, pigem arusaam, et tehnoloogiat tuleb huvitatud inimestega jagada. 

Mul on nendest aegadest väga head mälestused ja sellepärast kutsusimegi neid ka BBSummeritele\index{BBSummer}. Ron käis minu meelest kahel. Igatahes oli
soomlasi esimestel BBSummeritel palju ja ma mäletan, kuidas nad olid selle grupipildi aegsel BBSummeril äärmiselt
hämmastunud sellest, et kõik võivad õlut juua ja et teisel päeval ei toimunud mingeid 
kaklusi!

BBSummeri korraldamise juures oli veel tore see, et korraldustasu
tagas söögi ja joogi kõigiks päevadeks. Ja õlut pidi kõigile jätkuma. Ühele BBSummerile toodi küll õlut Fanta tünnides, nii et 
õllel oli kerge Fanta mekk juures.

\question{Tundub, et sul on inimestega vedanud.}

Mul on jah sõpradega vedanud. Kui ma üksi elasin ja 
tehnikaülikoolis\index{Tallinna Tehnikaülikool} 
vabakutseline olin, siis suhtlesin väga paljudega. 
Hiljem võttis perekond nii palju aega ära, et kahjuks ei jõudnud enam kõigiga 
kontakti hoida.

\question{Aga kriitilisel hetkel olid nad olemas?}

Nad on siiamaani olemas. Näiteks 
Lõvi\index[ppl]{Lõvi} kohtasin ma umbes viis aastat tagasi Selveri 
parklas, nüüd käisin tal hiljuti tehnikaülikoolis külas.

\question{Ahti\index[ppl]{Heinla, Ahti}\sidenote{Vt lk \pageref{sisu:ahti}.}  ütles 
väga targasti, et seltskond noori inimesi sai
omavahel suheldes inimeseks koos Eesti riigiga. Kas sul on ka selline 
tunne?}

Jah, me olime kõik suhteliselt üheealised. Täpselt selles 
vanuses, kui oli huvi teha midagi uut ja selleks tekkis võimalus ning ka omavaheline klapp. Oli ka erandeid, näiteks Henn Ruukel\index[ppl]{Ruukel, Henn} 
oli esimesel BBSummeril selgelt alaealine, aga õlletünni juures passi ei 
küsitud.

\question{Mida sa praegu teed?}

Pean pausi. Aitan ülikoolil satelliiti\sidenote{Masti panusega satelliit lendas 
2020. aastal ka edukalt kosmosesse.} ehitada. 

\question{Sellepärast, et on huvitav?}

Sellepärast, et on huvitav. Kosmos on huvitav.

\question{Kosmos on suur ka, seal ei ole karta, et huvitavad asjad 
saavad otsa.}

Praegu käib sebimine enamjaolt Maale väga lähedal. Orbiidid, kuhu 
väikseid satelliite lastakse, on viie- kuni seitsmesaja kilomeetri kaugusel.

\question{Kas sul üldse on kunagi juhtunud, et järgmist huvitavat asja ei ole 
silmapiiril?}

Ei.

\question{Kuidas see sul on õnnestunud?}

Isegi kui päevatööl ei ole huvitav, siis mul 
kodus käib kogu aeg mõni projekt. Kui üks saab valmis või läheb 
sahtlisse (sinna läheb enamik, sest huvi kaob ära), on 
järgmine kohe laual. Sellist asja ei ole, et mul ei ole midagi teha.

\question{Kas sul sahtel juba täis ei saa?}

Saab. Jube täis on. 

\question{Mida sa siis teed?}

Viskan ära. Suur osa neist on ju eksperimendid. Võtan ära tükid, mis lähevad  
järgmise eksperimendi peale, ja ülejäänu on prügi. Teadmised jäävad alles.


\chapter{Kain Kalju}
\index[ppl]{Kalju, Kain}
\question{Kuidas sina arvutite juurde jõudsid?}
See oli umbes aastal 1990--1991, kui mu sõpradele tekkisid esimesed arvutid, 
olime kaksteist kuni neliteist aastat vanad. Ühele sõbrale tekkis selline 
imelik asi nagu Texas Instruments TI-99\sidenote{Täpsemalt Texas Instruments 
TI-99/4\index{Texas Instruments TI-99/4}. Ärilistel ja arhitektuursetel 
põhjustel lühikese elueaga koduarvutite perekond. Oli koos samal 1979. aastal 
turule tulnud Atari 8-bitiste arvutitega üks esimesi omataolisi, millel oli 
audio- ja videoülesanneteks omaette protsessorid.}, see oli Commodore ja Apple 
II sarnane riistapuu selles mõttes, et ta oli 16-bitise protsessoriga ja 
\emph{boot}is otse BASICusse\index{BASIC}. 

See arvuti oli telekaga ühendatud ja seal olid mingisugused primitiivsed mängud 
Space Invaders\index{Space Invaders} ja muud sarnased. Ja siis 
loomulikult ka BASIC. Kogu programmi kood tuli kassetilindilt, nii nagu tollel 
ajal kombeks, mingeid flopisid polnud olemas. See oli minu esimene kokkupuude 
sellise arvutiga, millel oli klaviatuur, kuhu sai sisestada programmi koodi, 
kus me siis katsetasime ka esimest korda ise programme teha BASICUS toksides 
arvutisse ajakirjades ilmunud koodi ja mõeldes neid ka ise välja. 

\question{Mis linnas see oli?}

Ma olen Keilast pärit. Mul ei ole nagu kunagi olnud mingisugust 
sellist spetsiaalset ligipääsu kuhugi  teadusasutustele, koolidele ja nii 
edasi. Minu ligipääs arvutitele oli selles mõttes suhteliselt  piiratud 
võrreldes mõnede teistega.

\question{Kas sul seejuures mingit reaalainete huvi ka oli taustal?}

Koolis ma käisin reaalkallakuga klassis. Meil oli väga vahva lend 
gümnaasiumis\index{Keila Gümnaasium}, meil praktiliselt kõik poisid olid 
mingisuguse arvutihuviga ja nii palju kui ma nende elukäiku jälginud olen, on 
praktiliselt kõik  arvutimaailmas miskit pidi tegevad.

\question{Aga kust see tuli? Teil oli koolis nii korralik tase?}

Selles mõttes ongi väga huvitav, et gümnaasiumi esimestes klassides (me just 
olime läinud kaheteistkümne klassi süsteemile), meil olid kooli arvutiklassis  
Jukud\index{Juku}. Need loomulikult absoluutselt meid ei huvitanud, 
seal oli Pascal\index{Pascal}, meil oli juba ligipääs PC-dele tollel 
hetkel. 

\question{Juku oli ju igavesti äge aparaat omas ajas?}

Jah, aga nad tulid  selles mõttes  hilisemas faasis, pärast seda, kui meil oli 
juba PC ligipääs olemas ja kui mul endal oli ka kodus juba PC. Minu  kõige 
suurema arvutihuvi läkski sellest hetkest lahti, kui vanemad otsustasid mulle 
PC osta. Seda lugu peab natuke tagasi kerima selles mõttes, et seesama sõber, 
kellel oli see Texas Instrumentsi imepill, sai aasta hiljem 
  monokroomekraaniga 286-e. Tal isa käis Ameerikas ja  tõi 
sealt. Naljakas oli veel see, mis näitab seda ajastut, nad elasid esimesel 
korrusel kortermajas ja PC oli raudkapis, mis käis kinni. Oli nii suur hirm, et 
keegi murrab sisse ja varastab ära.

\question{See arvuti ju maksis rohkem kui korter tollel ajal. Mõni ime, et PC 
kappi pandi!}

Mu vanematele, käis see kohutavalt pinnale, et ma üldse ei viibi kodus, olen 
kogu aeg sõbra juures külas, hilisööni välja. Millalgi üheksakümnendatel, 
vahetult enne Eesti krooni tulekut oli aeg, kui rubla devalveerus hästi 
kiiresti. Ma isa käest olen küsinud, kuidas see täpselt oli, ja ta meenutas, et 
tollel hetkel tema sai millegipärast palka juba Ameerika dollarites  ja siis 
mingisugusest kooperatiivist või mis iganes tollel hetkel äriühingud olid, sai 
ostetud üks 286 dollarite eest.  Hinnaklass oli umbes tuhat dollarit. See oli 
siis VGA ekraaniga ja nii edasi. Täis mats, täiesti uus, väga äge, kuigi 286 
ilmselt oli tolleks ajaks juba \emph{outdated}  natukene,  oli juba 386-te 
ajastu.

\question{Ikkagi, võrreldes nende XT-dega, mille abil Tartu Ülikoolis 
programmeerimist õpetati, oli see ikkagi väga kõva sõna. Mis sa tegid tolle 
arvutiga?}

Nagu noor poiss ikka, tõenäoliselt mängisin, mind huvitasid kõikvõimalikud 
tarkvarad. 

Üks huvitav seik on veel, et me käisime sama sõbraga 1993. aastal Ameerikas. 
See oli umbes aasta pärast seda kui ma omale arvuti sain. See Ameerikasse 
minek oli väga kummaline. Ma mäletan seda, et meil oli  kolmene punt, kes me 
elasime üksteise lähedal ja  kõikidel oli juba kodus arvuti. Kas vanemate 
tööarvutid või siis isiklikud. Me sõitsime rongiga Tallinnast Keilasse ja 
millegipärast rongis hakkasime rääkima, et kuule, jube lahe oleks minna 
Ameerikasse. Ühel sõbral on tädi Ameerikas, et ta võtaks hea meelega vastu, aga 
kuidas me sinna saaksime. Minu isa töötas tollel hetkel Muuga 
sadamas. Kuidagi sai räägitud, et põhimõtteliselt saaks ka 
laevaga minna. Mina ei tea, kust see tuli, noored poisid, me olime kuskil 
viisteist, kuusteist aastat vanad. Kodus rääkisin sellest ja kuidagi hakkas see 
pall veerema niimoodi, et üks hetk me olime USA saatkonnas viisat taotlemas, 
järgmisel hetkel isal oli juba kokku lepitud, et me saame minna kaasreisijateks 
Ameerika suurele kaubalaevale ja me sõitsime üle Atlandi ookeani laevaga Muuga 
sadamast New Orleansi. Seal pani laevakompanii meid lennuki peale ja sealt 
edasi lendasime JFK lennuväljale New Yorki, kus siis sõbra tädi meid vastu 
võttis. 

Kusjuures me saime laeva peal palka, sest laevafirmale oli palju odavam 
vormistada meid töötajateks. Muidu oleks olnud vaja tasuda suuri 
kindlustusmakseid. Selles mõttes täiesti kreisi.

\question{Sellist asja ma kuulen esimest korda! Kaua te sõitsite sinna?}

Kaks nädalat, umbes neliteist-viisteist päeva võttis see laevasõit üle
ookeani.

\question{Kas te midagi kasulikku ka seal laeva peal tegite või sõitsite 
lihtsalt kaasa?}

Midagi kasulikku me tegelikult ei teinud. Hängisime ohvitseride nii-öelda 
piirkonnas. Meile küll näidati, kuidas laev töötab, aga me ei teinud selles 
mõttes mingit kasulikku tööd, et me oleks  koristanud tekki või midagi sellist. 
Ei, me lihtsalt hängisime. Võib-olla heal juhul saime mingisugust sellist  
ülevaatlikku õpet, umbes nagu sa muuseumis käid, et näed, siin on see asi, siin 
mootoriruumis on sellised nupud. Loomulikult keegi meil midagi teha ei lasknud 
välja arvatud see, et võib-olla ava-ookeanil me saime rooli keerata ja natukene 
nii öelda laeva juhtida.

\question{Millega  te tagasi tulite?}

Tagasi me tulime lennukiga juba. Aga miks ma sellest üldse räägin on see, et 
kui Ameerika pinnale astusime, oli meil päris palju raha, saime ju laevast 
palka. Meil oli stiilis tuhat viissada dollarit, mis oli tolle aja kohta üüratu 
summa. Ja siis mina isiklikult kulutasin selle raha loomulikult ära arvutipoes. 
Ma tõin endale Ameerikast, ma arvan, et elu ühe kõige tähtsama riistapuu, 
milleks oli modem. 

Ja vot pärast seda läheks elu lahti. 

See oli mingisugune 2400 boodine modem, ma täpselt ei mäleta tüüpi enam. Lisaks 
tõin veel Sound Blaster 16\index{Sound Blaster} helikaardi, mis oli täiesti 
tipp tollel hetkel\sidenote{Sound Blaster oli Singapuri firma Creative 
Technology (tuntud USAs kui Creative Labs) helikaartide perekond. Need kaardid 
olid PC-maailmas \emph{de facto} standardiks, kuni Windows 95 vastavad liidesed 
standardiseeris ja PC audio kommoditiseerus.}. See oli just välja tulnud, 
stiilis paar kuud varem. 

Üks asi, mille ma  hiljem avastasin, mis BBSides levisid, olid  helimoodulid,  
mul oli neid hästi palju, ma mingil hetkel kogusin neid. Ma arvan, et tollel 
ajastul paljud tegid seda. Need moodulid on  sellised helifailid, mida tollel 
ajal Amiga arvutites kokku pandi, koosnesid sämplitest. Põhimõtteliselt sul oli 
 mingisugune kaheksa \emph{track}i, kuhu sa siis miksid sämpleid niimoodi 
kokku, et sellest tekkis mingisugune \emph{meaningful} muusika. 

\question{Need liikusid siis BBSides?}

Jah. Loomulikult  sai üritatud ise ka neid teha, aga mul erilist muusikalist 
tausta ei ole, nii et sellest midagi välja ei tulnud.

\question{Tulid Ameerikast tagasi ja panid kohe BBSi püsti?}

Ei. Kui ma tulin Ameerikast tagasi, siis ma hakkasin avastama enda jaoks  BBSi 
maailma. Vanu asju üle vaadates selgus, et mu üks lemmik-BBS oli Dark 
Corner\index{Dark Corner}, mis oli Priit Kasesalu\index[ppl]{Kasesalu, 
Priit} veetud. Esmalt loomulikult sa üritad  alla laadida kõike, mida saad. 
Kõik on ju puhas kuld, kõik tarkvara, mida sul pole veel kunagi olnud ja nii 
edasi. Siis huvitav oli veel see, et tollel ajal eksisteeris selline asi nagu 
Kadaka Turg\sidenote[][-4cm]{Aastal 1991 avatud ja 2002. aastal 
kaubanduskeskusega asendatud Mustamäel asunud turg oli küllalt metsik 
müügikeskkond, kust oli võimalik hankida kõike alates karvamütsidest ja 
Nõukogude aurahadest kõikvõimaliku piraatkaubani. Sisuliselt oli tegemist 
endise Nõukogude Liidu territooriumil toiminud varimajanduse väljundiga 
Eestisse. Turg oli turistide seas hinnas, parematel aegadel käisid sinna 
Tallinna Sadamast eribussid.}, seal müüdi piraattarkvara. Ma arvan, et ma sain 
ka väga palju sealt tarkvara. BBSides, kusjuures minu mäletamist mööda 
tegelikult otseselt piraattarkvara väljas ei olnud. Seal oli rohkem sellist 
häkkimise stiilis tarkvara, aga mitte nii otseselt.

\question{Windowsi sealt vist keegi ei laadinud endale}

Jah, just, selliseid asju otse faililistides ei olnud, need olid taha 
nurkadesse ära peidetud. Aga seda ma mäletan küll, et mul oli kodus 
telefoniliin ja minu meelest ei olnud minutitasu tollel hetkel või see  
minutitasu oli nii odav. Igal juhul, mul oli kodus liin praktiliselt 
ööpäevaringselt kinni kogu aeg, sinna ei olnud võimalik helistada, sest minu 
arvuti helistas kogu aeg, laadis midagi alla.

\question{Kuidas sa alguses rea peale said? Kuidas sa teada said, mis numbri 
peale helistada?}

Väga võimalik, et see tuli stiilis .EXE ajakirjast\index{.EXE}, ma ei suuda 
seda enam meenutada. Aga kui sa oled ühte BBSi juba sisse pääsenud, siis kogu 
see maailm juba avaneb. Üks teema, mida BBS levitas, oli teiste BBSide 
aadressidega failid. Mingil hetkel Priit Kasesalu\index[ppl]{Kasesalu, Priit} 
pani kogu oma BBSi viimase versiooni veebi üles. Ma laadisin selle alla ja 
avastasin selle  ketta pealt, vaatasin just hiljuti läbi, oli  päris huvitav. 

\question{Mis seal siis leidus?}

Kõikvõimalikke häkkimisvahendeid, C-programmide näiteid, mingisuguseid raamatuid 
stiilis \enquote{Terrorist Handbook}\sidenote{Ilmselt peab Kain silmas William 
Powelli raamatut \emph{The Anarchist Cookbook}. Vietnami sõja vastaste 
protestide laineharjal 1971. aastal USAs ilmunud (ja mitmel pool keelatud 
olnud) raamat sisaldas kõikvõimalikku vastandkultuuriga seotud sisu 
\emph{Thermite}-i ja LSD  valmistusõpetustest õpetusteni telefonisüsteemide 
murdmiseks. Raamat levis tekstifailina laialt ülikoolide serverite ja FidoNeti 
kaudu ning teda täiendati aja jooksul pidevalt: eriti kuulsad on anonüümse 
autori \enquote{\emph{The Jolly Roger}} täiendused.} ja muud sarnased. Igasugune 
selline kraam, mis  noortele inimestele põnevust pakkus.

\question{Tulles korra veel sinu arvutihuvi alguse juure. Kas sa olid pigem 
seda tüüpi mees, kes mängis arvutiga, võrgutas arvutit või programmeeris 
arvutiga?}

Ma olen mõelnud selle üle, et kuidas see siis täpselt oli. Mulle tundub, et mul 
on olnud  mitu  ajajärku. Koduse 286 ja BBSide ajajärk oli pigem selline, et sa 
lihtsalt üritad endale sisse krahmata kõike, mida sa näed. Seal leidus ka 
arvutimänge, aga ma ei mäleta, et ma oleksin väga  kohutavalt mänginud. Siis, 
kui mul endal veel arvutit ei olnud,  sõbra juures me mängisime loomulikult 
kõik need ajad täis. Me ei tegelenud programmeerimisega, vaid pigem ikkagi 
mängimisega. Aga hiljem jäi see mängimine pigem taha taustale ja ikkagi 
üritasid aru saada, kuidas arvuti töötab. Näiteks üks teema, mis mind 
kohutavalt paelus, olid viirused. Mul oli alati kõige viimane viirusetõrje 
tarkvara. Ma usun, et mul oli selleks hetkeks juba ka mitu kõvaketast, ehk mul 
oli võimalus katsetada, mida viirused teevad. Ka  viiruse nii-öelda 
kollektsioone levitati BBSides. Ja siis sai uuritud, et kuidas selline asi 
põhimõtteliselt töötab. 

Ja siis järgmine ajastu tuli siis, kui ma avastasin enda jaoks 
Linuxi\index{Linux}, samal ajal tuli ka Internet. Sealtsamast gümnaasiumi 
kõrvalt kaheteistkümnendas klassis ma sattusin tööle Riigi Elektriside 
Inspektsiooni\index{Riigi Elektriside Inspektsioon|see{Tehnilise Järelevalve 
Amet}}, mis on täna Tehnilise Järelevalve Amet\index{Tehnilise Järelevalve 
Amet}. Sattusin selliseks, noh,  patsiga või arvutipoisiks, mul patsi pole 
kunagi olnud. Olin selline arvutipoiss nagu ikka, kellele antakse mingisugused 
arvutid, et palun seadista nüüd need ära, tee seda ja teist.

\question{Kuidas sa sinna sattusid niimoodi kooli kõrvalt?}

Seesama sõber, töötas Pennus\index{Pennu} ja kuidagi tema kaudu tuli kontakt, 
et otsitakse sellist arvutitüüpi, kes oskab arvutitega midagi teha. Ma läksin 
kohale  ja kuidagi võeti tööle poole kohaga.

\question{Teil klassist ikka mitmed töötasid siis keskkooli ajal?}

Jah, meil mitmed töötasid. Üks klassivend näiteks töötas Keila Linnavalitsuse 
juures. Ta oli selline kõva programmeerija juba tollel ajal, kes kinkis mulle 
mu esimese programmeerimisraamatu C Programming Language\index{The C 
Programming Language}, Brian Kernighan and Dennis Ritchie.

\question{See on seesama salapärane väljaanne\sidenote{\label{sisu:richie_vene}Kainil oli raamat 
jutuajamisel kaasas, selles puudus igasugune märge väljaandja ning trükkimise 
aja ning koha osas. Raamat oli 
korralikult köidetud ja kopeeris isegi värvilist kaanekujundust täpselt. 
Isegi peidetud \enquote{üllatusmunad} olid kopeeritud: indeksis viitas mõiste \enquote{recursion}  
samale indeksi lehele. Mart Palmas\index[ppl]{Palmas, Mart} mäletab, et raamatut 
olla trükitud Novosibirskis.}, mis minul oli}

Just, kui praegu minna Amazoni vaatama, siis täpselt selline raamat on müügil. 
See oli mu esimene programmeerimisraamat, aga see leidis kasutamist ikkagi 
aastaid hiljem, kui ma juba netit\index{neti.ee} tegin ja mul oli praktiline 
vajadus programmeerida otsingusüsteemi mis oleks suurema jõudlusega.

\question{Tahaks ikkagi aru saada, et kuidas teil juhtus selline klass olema, 
kus mitmed juba keskkooli ajal töötasid, kodudes olid arvutid ja inimesed 
programmeerisid}

Aga võib-olla see oligi see aeg, kus need arvutid ilmusidki rohkem koju ja 
kontorisse ja oli tohutu puudus sellisest nii-öelda oskusteabest. Vanemad 
inimesed võib-olla julgenud arvuteid veel kasutada ja noored julgesid nendega 
igasuguseid asju teha.

\question{Aga igas keskkooliklassis ei olnud see asi niimoodi, et neli-viis 
poissi töötasid arvutispetsialistidena, miks teil oli?}

Ma ei oska seda tagantjärgi öelda. Küll aga mäletan sellist huvitavat seika, et 
meil üks  eksam oli põhimõtteliselt arvutieksam ja see ei seisnenud meie puhul 
programmeerimises. Meie puhul tähendas see eksam, et  me sisustasime 
arvutiklassi.  Kool sai Tiigrihüppe või  mis iganes programmi kaudu peaaegu 
klassitäie arvuteid ja siis R-klassi\sidenote{R nagu Reaal.} poiste ülesanne 
oli võrgutada see klass 
füüsiliselt Etherneti kaabliga, installeerida need arvutid, installeerida 
võrguserver, milleks oli  Linuxi server. Serveri \emph{task} jäi minu peale, 
kuna ma olin tollel hetkel kõige suurem Linuxi\index{Linux} käpp võrreldes 
siis teiste poistega.  Meie kaheteistkümnenda klassi arvutieksam seisnes 
selles, et me põhimõtteliselt seadistasime koolile esimese PC klassi. See oli 
1995. aastal.

\question{Linux ei olnud selleks ajaks ju kuigi vana, kuidas sa selle otsa 
komistasid?}

Linuxi otsa ma komistasin siis, kui ma juba Riigi Elektriside 
Inspektsioonis\index{Riigi Elektriside Inspektsioon} töötasin. Kui ma sinna 
läksin, siis seal veel Internetti ei olnud, aga see tekkis sinna üsna pea, ma 
arvan, et mingisugune kuu-paar hiljem. See oli siis 1994. aasta lõpp.  
Elektriside Inspektsioon asus aadressil Ädala 4d, mis on ka siis selline 
legendaarne internetihoone.  Meie allkorrusel oli 
Valitsusside\index{Valitsusside}, kus toimetas Taavi Talvik\index[ppl]{Talvik, 
Taavi}. Ja Taavi andis Riigi Elektriside Inspektsioonile juhtmeotsa kätte, 
milleks oli  tolleaegne kümnemegabitine koaksiaalkaabel ja, palun, siin on 
Internet. See koaksiaalkaabel sai siis veetud kõikidesse ruumidesse. Ei mingeid 
hub'e ega täht-topoloogiat.

Siis ma avastasin enda jaoks Interneti. Koolis loomulikult poistele rääkisin, et 
see FidoNet on nüüd mingi  vana jama, aeglane, toimib üle modemi, et siin on 
üks palju uuem ja huvitavam asi. Kusjuures  Valitsussidest edasi olid kanalid 
üsna kiired. Mäletan, et Tartu Ülikooli FTP-serverist sai kahemegabitise 
kiirusega faile alla laadida, see oli  meeletu kiirus. Välislink oli 
loomulikult kuskil 64 või 128 kilobitti. 

\question{Mis sealt Tartu Ülikoolist siis tõmmata oli nii väga?}

Vot seda ma täpselt ei mäleta, aga ju seal midagi oli, sest mul on väga selgelt 
meeles kadri.ut.ee\index{kadri.ut.ee}\sidenote{Tartu Ülikooli masinad kadri.ut.ee ja madli.ut.ee said Toomas Soome\index[ppl]{Soome, Toomas} andmetel nimed Otto Telleri\index[ppl]{Teller, Otto} tütarde järgi.}  FTP-server. 

Aga see Valitsusside\index{Valitsusside} ja see ethernetikaabel, see oli nagu 
huvitav. Tollel ajal, nagu teisedki on rääkinud, arvutiturvalisus ei olnud 
eriti teemaks. 

FidoNet oli selles mõttes tohutu kulla-auk, et ta avas loomulikult kõik oma  
\emph{echo} kanalid. Aga Internet avas meililistid ja kusagilt meililistist ma 
lugesin, et Anto Veldre\index[ppl]{Veldre, Anto} teeb 43. 
Keskkoolis\index{Tallinna 43. Keskkool} mingisuguseid selliseid 
\emph{introduction} kursuseid. Tollel ajal ilmus ka ajakiri .EXE\index{.EXE}, 
kus Anto artikleid kirjutas. Ma ei mäleta, kumb kummale täpselt eelnes, aga 
igal juhul mäletan seda, et üks hetk olin ma seal 43. Keskkoolis, et 
\enquote{siin ma olen ma tahan teadmisi saada}. Seal olid koha peal veel tol 
ajal sellised legendaarsed koolipoisid nagu Indrek Mandre\index[ppl]{Mandre, 
Indrek} ja Heno Ivanov\index[ppl]{Ivanov, Heno} vist. Tagasi tulin ma sealt 
juba Slackware\index{Slackware} distributsiooni installeerimisflopidega, mida 
oli stiilis kuus tükki. Installeerimise protsess käis ikkagi niimoodi, et 
esimene flopi, teine flopi, kolmas-neljas ja nii edasi lõpuks sai 
installeeritud. 

\question{Aga siis jääb lisaks kõigele muule Anto peale ka Linuxi pisiku 
levitamine Eestis?}

Ma usun küll, jah. Ma arvan, et temal on väga suur roll selles osas, Linux 
Eestis käima läks. Igal juhul mina selle pisiku sealt sain. Kuna ma olin tollel 
hetkel juba mõnda aega Elektriside Inspektsioonis\index{Riigi Elektriside 
Inspektsioon} töötanud ja ka palka saanud, oli mul päris korralik nii-öelda  
taskuraha. Ja ma ehitasin  endale uue arvuti, 286 sai FidoNetis  kuskil maha 
müüdud (FidoNetis  käis ka suur riistvaraga hangeldamine) ja ehitasin endale 
486 arvuti. Kusjuures see ei olnud mitte lihtsalt 486 vaid 486DX4 
100 MHz\sidenote{Inteli nomenklatuuris olid \enquote{DX} tähistusega protsessorid 
need, millel oli kiibil eraldi matemaatika kaasprotsessor, see andis olulise 
jõudlusvõidu.}, see oli siis absoluutne tipp. 

See oli kõige kõvem 486, mis üldse kunagi tehti. See oli  juba siis see aeg, 
kui mul oli juba \emph{node} registreeritud. Sealtsamast Dark Corner 
BBSist\index{Dark Corner} sain ma esimese FidoNeti \emph{point}i, kus ma 
pääsesin ligi FidoNeti uudistekanalitele. Mingil hetkel tundus, et aga palju 
ägedam oleks \emph{node}. Sai kirjutatud Tarmo Mamersile\index[ppl]{Mamers, 
Tarmo} (sest ta oli Eesti regiooni \emph{manager}, tema neid aadresse jagas), 
et kas oleks võimalik registreerida \emph{node} number kuuskümmend kuus ja 
Tarmo vastas, et \enquote{tehtud}. Sealt edasi oli mul \emph{node}, mis mõnda 
aega eksisteeris mul kodus. 

Aga siis mingil hetkel sai seadistatud Elektriside Inspektsioonis 
Linuxi\index{Linux} server, sest meil on praktiline vajadus serveerida 
printerit, faksi ja faile. Ehk siis sai nurka tekitatud Linuxi server, kes siis 
šeeris faile üle Samba teenuse ja võttis vastu fakse. Mul õnnestus ka enda 
FidoNeti \emph{node} sellesse samasse serverisse sokutada. Kui muidu FidoNeti 
tarkvara oli MS-DOSi peal, siis  oli ka alternatiiv Unixitele 
Ifmaili\index{Ifmail} nimelise programmi näol.

\question{Räägime korra sellest riigiametist. Miks seal üldse Internetti vaja 
oli? Kas see oli puhas sinu huvi või nad tegid midagi kasulikku ka sellega?}

Jah, selleks oli praktiline vajadus  olemas, sellepärast et Elektriside 
Inspektsioon\index{Riigi Elektriside Inspektsioon} tegi tihedat koostööd 
ITUga\sidenote{\emph{International Telecommunications Union (ITU)}.}, kes siis 
juhib kõiki neid sageduste jaotust ja protokolle ja  kõike muud sellist. 
Nendega oli inspektsioonil tihe kirjavahetus ja ma arvan, et e-maili teel. Ma 
ei suuda meenutada, kuidas see meilivahetus enne kaabliga Interneti käis, aga  
pärast  seesama Linuxi masin oli ka loomulikult mailiserver. Sellest hetkest 
tekkis ka meil oma domeen nimega rei.ee. Või äkki domeen oli juba varem olemas, 
igal juhul pärast Linuxi server tuli, hakkas ta rei.ee domeeni  kirju vastu 
võtma ja ka mina sain endale isikliku esimese ülilühikese e-maili aadressi, mis 
oli tollel ajal ülikõva, kain@rei.ee\sidenote{Lühikesed meili ja muud aadressid 
olid staatusesümboliks, need näitasid kuulumist kas serveri-administraatorite 
kõrgesse kasti või neile väga lähedasse ringkonda.}.

\question{Ehk sa avastasid ennast suhteliselt õrnas eas Linuxi ruuduna 
riigiasutuses?}

Just. Miks ma seda kümne megabitist ethernetikaablit mainisin oli see, et seal 
kõik liiklus oli ju näha. Ja kui ma külastasin siis Anto 
Veldre\index[ppl]{Veldre, Anto} arvutiklassi 43. 
Keskkoolis\index{Tallinna 43. Keskkool},  jäi mulle sealt üks asi elu 
lõpuni meelde. Kuidas kõik need noored tüübid, kes seal siis 
siil.edu.ee\index{siil.edu.ee} nimelise SCO\index{SCO UNIX} masina 
taga istusid, oli tohutu kõvad häkkerid. Nad demonstreerisid, mida nad siin 
teevad, näitasid, et kuidas nad  suudavad \emph{exploit}-ida mingisuguseid Tartu 
Ülikoolis olevaid masinaid\label{sisu!ylikooli_root}, mingeid professoreid seal jälgida ja nii edasi. See 
avaldas mulle nii kohutavalt mulje, et mind hakkas lisaks sellele varasemale 
viiruste teemale huvitama ka arvutiturvalisus.

Ma arvan, et see on esimest korda, kui ma avalikult sellest räägin aga ma 
\emph{sniff}isin loomulikult ka meie  võrku ja \emph{sniff}isin, mida siis 
Valitsusside\index{Valitsusside} insenerid seal tegid. Ega seal vahel midagi 
olnud,  sellesama  kaabli otsas oli kaks ametit: oli Riigi Elektriside 
Inspektsioon\index{Riigi Elektriside Inspektsioon} kõigi oma töötajatega ja 
\emph{Valitsusside}. Et kui Valitsusside insenerid käisid oma ruutereid või 
keskjaamu üle telneti konfimas, siis loomulikult see liiklus levis lahtise 
tekstina võrgus. Seda oli päris huvitav jälgida, mida nad siis seal teevad.  
Loomulikult ma seda kunagi pahatahtlikult ära ei kasutanud, see oli lihtsalt 
selline puhas  noore mehe huvi.

\question{Eks see seik iseloomustab suurepäraselt toonast aega. Ma usun, et kui 
praegu keegi lahtise traadi peal lahtist kanalit kasutaks, korraldataks umbes 
poole tunni jooksul mingi jama}

Jah, ma arvan ka. See mulje jäi niivõrd meelde, et kogu see võrguvärk on 
niivõrd ebaturvaline, et nii kui Soomest keegi härrasmees\sidenote{Tatu 
Ylönen, Helsingi Tehnoloogiaülikooli teadlane.} tegi 
\emph{secure shell}i esimese versiooni, siis ma hakkasin seda praktiliselt kohe 
kasutama, kui ma sellest teada sain. 

Veel Keila Gümnaasiumi\index{Keila Gümnaasium} juurde tagasi tulles. 
Pärast seda, kui ma kooli olin juba ära lõpetanud,  jäin ma edasi 
administreerima serverit, mis sinna maha jäi. Nagu tollel ajal ikka,  pidid 
kõikidel Unixi masinatel  olema ilusad nimed. Kodus rääkisin sellest teemast ja 
isa pakkus välja, et aga \enquote{kratt} oleks jube hea nimi. Ja praegu 
vaatasin nimeserverist järgi, et siiamaani on Keila Gümnaasiumis oleva serveri 
nimi kratt.keila.edu.ee\index{kratt.keila.edu.ee}.

\question{Hakka seda nime siis takkajärgi muutma. Loodetavasti riistvara ei ole 
päris seesama?}

Riistvara kindlasti ei ole seesama, sest seda koolimaja füüsiliselt enam alles 
ei ole. Keilas on nüüd uus koolimaja, kus mu enda lapsed käivad, sest ma elan 
siiamaani Keilas. Aga aadress on olemas.

\question{Seepärast ongi asjade nimetamine oluline, et need nimed võivad pikalt 
kesta}

Just. FidoNeti ajast veel üks huvitav \emph{impact} minu meelest, mis mul  
hiljem on väga kasulikuks osutunud oli see, et modemid töötasid
AT-käsustikuga\sidenote{Hayes käsustik, tuntud ka kui
AT-käsustik, on käsukeel, mille Dennis Hayes lõi 1981. 
aastal omanimelise ettevõtte 300-boodise Smartmodem modemi juhtimise tarbeks.}. 
Too käsustik oli selles mõttes universaalne asi, et seda kasutati hiljem 
erinevates muudes rakendustes. Loomulikult BBSidesse sissehelistamine toimus 
lihtsa terminaliga ehk et sa pidid nagu häkker käsustikku teadma. Enne 
helistamist pidi sisestama  ATDT, telefoninumber ja nii edasi, võib-olla veel 
seadistama protokolli. Loomulikult, tolleaegsed inimesed teavad täpselt, 
missuguse protokolliga vilistab  memcpy intro. See oli ka võib-olla selline 
asi, mis edaspidiselt  mõnes mõttes kaasa aitas.

\question{BBSil oli kliendisoft ka?}

Ei olnud. Helistasid terminaliga, kliendisoft oli ainult FidoNetil. Oli soft 
nimega FrontDoor, mis helistas, ja oli soft, mis pakkis kokku FidoNeti 
\emph{echo}d ja  saatis  selle paki edasi. Aga BBSil kui sellisel ei olnud 
kliendisofti. Läksid lihtsalt \emph{telnet}iga külge ja hakkasid seal edasi 
tegutsema.

\question{See läheb minu mäletamisega kokku küll. Oleks ju olnud loogiline, et 
keegi oleks mingisuguse tarkvara teinud BBSide ette, \emph{cache} jaoks 
näiteks?}

Jah, kui vaadata, mis Ameerikamaal toimus, kus siis olid need nii-öelda 
\emph{Online Service Provider}id  nagu AOL  ja 
CompuServe\index{CompuServe}\sidenote[][-1.7cm]{Interneti-eelsel ajal domineerisid USA 
turul agressiivsete turunduskampaaniatega (ühel hetkel oli pool \emph{kõigist} 
toodetud CDdest AOLi logoga) teenusepakkujad, kes pakkusid kummalist segu 
BBS-laadsetest ja Interneti-teenustest. Neist suurimad olid CompuServe, Prodigy 
ja America Online.} ja nii edasi, siis neil oli tarkvara. Ma mäletan seda, et 
kui ma USAs modemi  olin ostnud, siis loomulikult noorte poistena meil tuli 
seda proovida. Ja, kujuta ette, meil oli julgus kruvikeerajaga lahti keerata 
üks selline suur soliidne arvuti, see oli vist Computer 2000\sidenote{Computer 2000 
oli küll ka siinmail tegutsenud arvutiäri, kuid ilmselt peab Kain silmas Gateway 
2000 nimelist ettevõtmist, mis sama nime all personaalarvuteid tootis.} või mis iganes 
tolleaegne  selline hästi kõva valge PC bränd oli. Sõbra tädimees oli 
arhitekt, tal oli selline väike arhitektibüroo, ning meil oli julgus  
omavoliliselt kruvikeerajaga  lahti keerata üks nende suur \emph{tower} ja 
sinna sisse proovida seda sisemist modemit. Modemiga oli kaasas kas 
CompuServe'i või mingi muu sarnase teenuse CD plaat või flopi, äkki mäletan 
valesti. Ja siis sai helistatud Ameerika BBSi.  

\question{Kui sa nendes BBSides kolasid, kas sulle midagi muud peale tarkvara 
ka silma jäi? Raamatuid ja MODe sa mainisid?}

Raamatud mind eriti  tollel hetkel ei köitnud, BBSidest mina ikkagi laadisin 
peaasjalikult tarkvara ja siis muusika MODe. Aga kogu infovoog 
tuli FidoNetist. FidoNet oli minu jaoks täiesti puhas kulla-auk. Nagu varem 
mainisin,  mul ei ole olnud ligipääsu sellistesse teadusasutustesse või 
ülikoolidesse,  mul ei ole olnud piltlikult öeldes mentorit. Meil oli kamp 
poisse, kes omavahel  infot vahetasid. Meil ei olnud nagu sellist vanemat, kes 
teab, kuidas asjad käivad,  kõik käis katse-eksituse meetodil.

\question{Isegi hästi, et te kuidagi paha peale ei läinud selle kambaga. Noored 
poisid, tont teab, mida hakkavad tegema}

Ju siis me olime piisavalt mõistlikud. Ma arvan, et sellest ajast saadik on mul 
selline ise õppimise  oskus. Võib-olla see sai ka saatuslikuks, miks ma 
Tehnikaülikoolis ei suutnud väga kaua õppida,  ainult ühe aasta nagu tollel 
ajal võib-olla paljudel teistelgi kombeks.

Peale gümnaasiumi ma läksin kohe Tehnikaülikooli informaatikasse\index{Tallinna 
Tehnikaülikool!Informaatika}, aga kuna ma juba tollel hetkel töötasin, siis 
igasuguseid huvipakkuvaid projekte oli  palju kõrval. Mina eeldasin seda, et 
nüüd ma saan hakata programmeerimist ja igasugust muud sellist huvitavat asja 
õppima, aga siis tuli välja, et ei,  sa pead kõigepealt läbima füüsikad ja 
matemaatikad. Matemaatikast mul oli juba nagu natukene \enquote{kopp ees}, kuna 
meie meil oli selline väga püüdlik matemaatikaõpetaja gümnaasiumi ajal. Me 
tegelesime väga põhjaliku matemaatikaga, mingit sisse saamise probleemi 
Tehnikaülikoolis  absoluutselt ei olnud,  matemaatika eksamist lihtsalt 
lendasid läbi.

Ja nii see ülikool järgmisel aastal pooleli jäi.

\question{Kuidas sul kaitseväega on?}

Siis tuligi Kaitsevägi. Kui ülikoolis ei ole, siis varem või 
hiljem leitakse sind üles. Aga  Kaitseväkke ma läksin 1997. aasta suvel, ehk et 
ma olin siis juba aasta otsa Netit teinud. 

Ahjaa, et kuidas ma sinna sattusin. Töö Elektriside Inspektsioonis\index{Riigi 
Elektriside Inspektsioon}  hakkas natuke nagu ära tüütama. Nagu ikka,  tahad 
edasi areneda. Hakkasin otsima, et tahaks kuhugi  huvitavasse kohta tööle 
minna.  Mul mingil hetkel oli soov kindla peale töötada arvutifirmas, sest kuhu 
sa ikka lähed. Arvutifirmasse, seal olen kindel, et saan arvutitele väga 
lähedale.

Vanu \emph{backup}e läbi kammides jäi silma, mingil hetkel kandideerisin isegi 
Helmesesse\index{Helmes}, aga sinna ma ei saanud. Õnneks, tagantjärgi ma 
mõtlen. Keskkooli ja ülikooli vahelisel ajal suvel ma töötasin poolteist kuud 
Tõnu Samueli\index[ppl]{Samuel, Tõnu} IT-firmas nimega Eramees\index{Eramees} 
ja ma istusin samale kohale, kust oli just lahkunud Pronto\index[ppl]{Pronto}.  
 Tõnu ütles mulle, et Pronto müüs siin  neid Gravis 
Ultrasound\sidenote{Üheksakümnendatel väga populaarsed helikaardid, mis 
esimesena omataoliste hulgas suutsid toimetada päris instrumentide 
sämplingutega.} kaarte, et kuule, hakka nüüd sina sellega tegelema. Aga ma olin 
noor koolipoiss, ma ei  ei teadnud kaubandusest mitte essugi. Ma ma ei usu, et 
minust seal ettevõttes erilist kasu oli muidu kui nii-öelda patsiga poisist. 

\question{Ära ütle, päris mitmed inimesed kuni Tarmo Talini\index[ppl]{Tali, 
Tarmo} välja on mingil hetkel tegelenud müügitööga ja seejuures üldse mitte 
halvasti}

Eramehes üks asi, mis mul on veel eredalt meeles on, et Tõnu BBS oli siis 
kontoris. Kontor asus Eesti Talleksi majas, Mustamäe tee 1 vist, kui ma ei 
eksi. Ja BBS oli põhimõtteliselt  laiali laotatud arvutijupid  aknalaual. Seal 
oli siis USR Courier\index{US Robotics!Courier} modem\sidenote{US Roboticsi 
ülemise otsa Courier tooteliin oli oma töökindluse ja suurte kiiruste tõttu 
BBSide ja varaste Internetipakkujate lemmik, ka Eestis.},  emaplaat, toiteplokk  
ja nii edasi, lihtsalt hunnik juppe ja juhtmed, mis oli aknalauale laiali 
laotatud. Ja see oli siis Tõnu BBS või \emph{node}.

Pärast Erameest ma kandideerisin Estpak Datasse\index{Estpak Data}, sest mulle 
tundus, et ISP, et see on tegelikult veel huvitavam asi, sellepärast et nad 
tegelevad ju Internetiga.

\question{Kas Estpak oli tol ajal juba Eesti Telefoni oma või oli veel eraldi?}

Ta oli tollel ajal eraldi. Kui õieti mäletan, siis Estpak Data omanikuks oli 
siis Eesti Telekom\sidenote{Eesti Telekom pika nimega Riigiettevõte Eesti 
Telekommunikatsioonid oli Teede- ja Sideministeeriumi haldusalas töötav 
\emph{holding}-ettevõte, mis valdas Eesti Telefoni, Eesti Mobiiltelefoni, Eesti 
Kaugotsingu, EsData, Estpak Data ja TeleMedia aktsiaid. Hiljem viidi ettevõte 
börsile ja sealtkaudu sai tema ainuomanikuks Telia.}, mitte  Eesti Telefon, ta 
oli täiesti eraldiseisev ettevõte Eesti Telefonist. Huvitaval kombel kellelgi 
oli tulnud selline idee, et meil on kuidagi vaja edendada veebi  
virtuaalhostimist. Keegi oli välja mõeldud neti.ee\index{neti.ee} nimelise 
domeeni ja selle domeeni alt siis üritati müüa sellist traditsioonilist 
veebihostingut. Tollel ajal ta veel traditsiooniline ei olnud, aga ütleme, et 
siis tänapäeva mõistes. Ja  Estpak Data palkas mind kui nii öelda webmasterit, 
kes pidi hoolitsema veebi hostinguserveri ja teenuse eest. Ja siis muu seas oli 
neil selline idee, et kuidas me seda veebi hostingu äri ikka muud moodi 
edendame, kui meil on vaja mingit kataloogi. Inimesed peavad ju leidma üles 
need veebilehed, mida  kliendid sinna panevad.

\question{Kas tol ajal Meediamaa oli juba olemas?}

Meediamaa\index{Meediamaa} startis umbes samal ajal. Enne seda oli olemas  
Eesti veebisaitide nimekiri, mis oli nlibi ehk siis 
Rahvusraamatukogu\index{Rahvusraamatukogu} domeenis, kus Toomas 
Mölder\index[ppl]{Mölder, Toomas} tegutses. Ja Toomas Mölder kolis, ma arvan, 
sellesama nimekirja Meediamaasse ja sealt www.ee\index{www.ee}'sse. Kuna 
Meediamaa üks tegelane oli Tarvi Martens\index[ppl]{Martens, Tarvi}, siis neil 
õnnestus kuidagi EENetilt\index{EENet} välja meelitada domeen nimega 
www.ee\sidenote[][-1.5cm]{Alates oma asutamisest 1993. aastal kuni 2013. aastani oli 
EENet .ee domeeni registrar ja sellisena rakendas mitmeid suhteliselt rangeid 
reegleid. Näiteks oli domeeni registreerimine küll tasuta, kuid ühel 
organisatsioonil tohtis olla vaid üks domeen.}. Ma arvan, et mitte kellelegi 
teisele kui Tarvile ei oleks sellist domeeni elu sees välja antud.

\question{Seda ma kujutan ette küll. Kas sa seda kataloogi tegid siis käsitsi 
alguses?}

Jah, alguses alguses sai seda kataloogi käsitsi tehtud. Ta oligi selline väga 
algeline ja puine. Aga asi hakkas lendama siis, kui kui ma kutsusin kataloogi 
puhul endale appi Jaanus Vainu\index[ppl]{Vainu, Jaanus}, kellega ma olin kokku 
saanud Riigi Elektriside Inspektsioonis\index{Riigi Elektriside Inspektsioon}. 
Jaanus on ka omamoodi huvitav tegelane. Elektriside Inspektsioonis tema mõtles 
välja kogu meie FM 108 sageduse plaani, ehk kõik Eesti raadiojaamade 
sagedusnumbrid on tema tehtud. Nõukogude ajal oli meil teistsugune FM 
sagedusala -- kuidas saab nii, et sa saad poest osta raadio, millega saab 
välismaa raadiojaama kuulata. See ei sobinud kuidagi, Eesti Vabariigi alguses 
koliti Lääne sagedustele üle. Jaanus oli üks nendest, kes käis mööda Eestit  
mõõtmas ja tegi sagedusplaani. Tal oli väga detailselt Corel Draw's\index{Corel 
Draw} joonistatud kõik need nii-öelda sagedusringid Eesti kaardi peale. Eesmärk 
oli  planeerida sagedused niimoodi, et üle Eesti saatjatel oleksid sagedused, 
millel on võimalikult vähe häireid naaberriikidega ja omavahel.  

\question{Kogu seda teadust tehti Corel Draw abil?}

Jah. Jaanus on selline tohutu pedant,  tohutu  töövõimega katalogiseerija. 
Tema enda isiklik huvi on \emph{bluegrass}. Mäletan seda, et tema oli esimene 
inimene, keda mina tean, kes välismaalt e-poest asju tellis.  Tema tellis 
CDNow'st\sidenote{CDNow oli 1994. aastal asutatud Interneti-põhine muusikamüüja, 
kes paraku esimest dot-com-mulli üle ei elanud ja sajandivahetusel uksed 
sulges.}  plaate endale. Mina väga imestasin, et kuidas selline asi üldse 
võimalik on. Et ta tellib kuskilt, Jumal teab kust ja tulebki pakiga kohale CD 
muusikaga.

\question{Jaa, isegi üheksakümnendate lõpus oli Amazonist raamatute tellimine 
suhteliselt eksootiline tegevus. Aga mis hetkel ja kuidas te 
neti.ee\index{neti.ee} ära automatiseerisite?}

See meie tandem Jaanusega töötas selles mõttes ülihästi, et mina olin  
programmeerija ja arendasin tarkvara ja Jaanus oli  katalogiseerija.  Kui Jaanus 
selle projektiga liitus, siis võiks öelda, et projekt hakkas täielikult 
lendama. Ma arvan, et meil läks võib-olla paar kuud aega, kui me olime 
Meediamaast\index{Meediamaa} igatpidi kõikide näitajate poolest mööda läinud. 
Me olime tollel ajal võib-olla isegi natukene liiga ebaviisakad noored mehed. 
Näiteks me reklaamisime netit spämmides, tegime  ühe korra sellise 
masspostituse, saates kõikvõimalikele meiliaadressidele teate, et \enquote{nüüd 
on selline huvitav teenus olemas nagu neti.ee, tulge, külastage}. Midagi 
sarnast. Kusjuures huvitav on see, et kui ma vaatasin enda \emph{backupe}, siis 
ma nimetasin enda \emph{crawlerit} Nuhiks, seda otsingurobotit, kes mööda lehti 
ringi kolab. 

Ja huvitaval kombel ma olin selle Nuhi programmeerimist alustanud juba mitu kuud 
varem ehk nagu nagu miski oleks suunanud mind sellele teele, et seda võib vaja 
minna. Ja  otsingumootoreid ma olin ka natuke varem teinud. Kui ma pärast 
Erameest  ülikooli läksin, siis  üks sealt saadud kontaktidest kutsus mind 
tegema ühte ärikataloogi sarnast teenust,  mille pealkiri oli Bartanet. See 
asus EsData\index{EsData} serveris, oli mingi Suni server Akadeemia tee 21  
teisel korrusel, samas majas, kus me hetkel viibime. Ja selles Suni serveris, 
ma ei tea, mis asjaoludel, aga ma millegipärast sain seal teha FTP-serverite 
otsingut. Ma panin seal püsti otsinguteenuse nimega Filerix, mis töötas umbes 
kolm-neli kuud, mille ainukeseks sisuks oli see, et ta võimaldas väga hõlpsasti 
faile üles leida igasugustest kohalikest FTP \emph{mirror}itest. Tollel ajal  
Marek Tiits\index[ppl]{Tiits, Marek} IBSist\index{Institute of Baltic Studies} 
hostis sellist asja nagu TuCows\sidenote{TuCows (\emph{The Ultimate Collection 
Of Winsock Software}) keskendus oma algusaegadel tasuta tarkvarale. Kuna 
Interneti kiirus sõltus veel väga suurel määral geograafiast, opereeris 
ettevõte skeemi, kus  huvilised võisid jooksutada TuCows.com lehekülje 
lokaalseid peegleid. Ühte sellist Marek pidaski.}. Minu  otsingumootor 
võimaldas hõlpsasti failinimede järgi üles leida tarkvara tolleaegsele Windows 
95'le, vanadele Windowsidele ja nii edasi. Tollest pooleaastasest projektist nii-öelda kõrvalprojektina ma tegin failiotsingut.

\question{Suure hulga failide indekseerimine ei ole enam päris naljaasi ja 
eeldab programmeerimisoskust. Kust sa selle üles oled korjanud?}

Tollel hetkel ma oskasin programmeerida Perli\index{Perl} ja siis kõike 
seda, mis  Unixi \emph{shell}is saada on. See tuligi  sellest ajaperioodist, 
kui ma uurisin, mis on nii-öelda Unixil  kõhus.

\question{Ise korjasidki üles selle algoritmika ja muu sellise?}

Jah, mis puudutab veebi \emph{crawl}imist,  siis jah, selle peale tuli juba 
mõelda.

\question{Puhtalt konteksti pärast, kaua su \emph{crawler}il aega läks, et 
kogu Eesti veeb üle käia?}

Ma arvan, et see oli mingi stiilis ööpäev või midagi sellist, sest veeb oli 
tollel ajal väga väike. Ma täpselt pole vaadanud, aga ma usun, et selle 
kataloogi suurus oli võib-olla paar tuhat linki ja mitte rohkem tollel ajal. Ja 
keskmine koduleht oli ka selline kolm kuni viis lehekülge, et see ei  olnud 
eriline teema. Huvitavamaks läks pärast, siis kui see linkide hulk juba 
miljonitesse läks, siis mingil hetkel oli ikka selline \emph{crawler}, mis 
töötas paralleelselt  paljudes \emph{thread}ides ja nii. Aga noh, see oli kõik 
selline loomulik evolutsioon. 

Aga jah, ma mäletan tegelikult, et miks ma arvan, et miks mind Estpak 
Datasse\index{Estpak Data} tööle võeti. Ühe sellise kõrvalprojektina ma olin 
teinud HTML-i tutvustuse. Ma arvan, et mul oli vist koolis olnud vaja seda 
kellelegi õpetada. Ehk siis gümnaasiumis tollel perioodil, kui ma siis Keila 
seda serverit administreerisin. Ja siis mulle tundus, et ma seda ikka õpetan, 
mingit eestikeelset materjali pole ja siis ma tegin ühe esimese  eestikeelse 
HTML-i tutvustuse, mis võttis läbi kõiki üksikuid elemendid. 

\question{Millegi pärast tuleb see maru tuttav ette, ma arvan, et ma olen sealt 
mingeid asju otsinud}

Kusjuures seesama HTML tutvustus on sellel samal aadressil täna ka üleval ja ma 
olen üsna kindel, et see on üks kõige vanemaid veebilehti, mis leidub täna 
Eesti veebiruumis, mis on originaalkujul originaalaadressil. 

\question{Mis aastast see on?}

1996. Ja siis veel ühe projektina ma olin teinud veebi pokkeri, sellise  
veebipõhise mängu. Selles mõttes,  ei saa öelda, et  mul pole kunagi huvi olnud 
ka mänge teha, aga ma olen  rohkem  oma elus programmeerinud nii-öelda 
veebi-asju, kui \emph{desktop}is või masinas töötavaid rakendusi. 

Nende teadmiste baasil mind sinna Estpaki  siis tööle võeti. Tõenäoliselt ma  
näitasingi seda, et vaadake, ma olen teinud sellise veebi pokkerimängu, ma olen 
teinud HTML-i tutvustuse ja võib-olla ma rääkisin ka seda, et ma olen selle 
\emph{crawler}i teinud. Igal juhul mind võeti sinna tööle ja ma sain jätkata 
sellesama koha pealt, kus ma juba olin.

\question{Kes teil seda toote poolt tegi, või polnud niisugust mõistet, nagu 
tootejuht?}

Ei olnudki. Piltlikult öeldes pandi mind  istuma, et palun istu siia ja tee. 
Tegelikult  see oli ikkagi läbi mõeldud mõnes mõttes. Estpak Data\index{Estpak 
Data} tegi koostööd ühe reklaamiagentuuriga, mis rentis ruume Kullo majas 
Mustamäe teel. Nii et tegelikult minu füüsiline töökoht  asus selles 
reklaamiagentuuris Kullo majas. Minul oli arvuti, millel oli püsiühendus 19.2 
kilobitti sekundis ja sealt ma töötasin. Noh, noore mehena nagu ikka, et sind 
ei huvita, kuidas rahad liiguvad ja nii edasi,  sind huvitab ainult see 
tehniline pool. Idee siis seisnes selles, et reklaamiagentuur aitab 
potentsiaalsetel Estpak Data klientidel teha kodulehti, aitab teha neile 
reklaami ja nii edasi, umbes selline kokkulepe oli. PRC Nord Decor
oli tolle agentuuri nimi, ma ei tea, kas see kellelegi midagi ütleb. Aga, aga 
selles mõttes oli huvitav, et üks kolleeg, kes Nord Decoris töötas, oli kunagise 
OK Jutuka\index{OK jutukas}\sidenote{OK Jutukas oli üks esimesi tõeliselt massidesse läinud 
sotsiaalvõrgustiku laadne rakendus Eestis. Jututube - kohti, kus sai üle 
telneti kaaskodanikega suhelda - oli  veel, aga 1996. aastal käivitatud OK oli 
üks esimesi veebipõhiseid jutukaid ja tõenäoliselt omataolistest siinkandis 
suurim. Üheaegselt lobises omavahel kuni 300 inimest ja jutuka esimese 
aastapäeva pidu kajastas isegi toonane Päevaleht.} üks asutajatest. Mitte  
Kaupo Kalda\index[ppl]{Kalda, Kaupo}, aga Tiit Sermann\index[ppl]{Sermann, 
Tiit}. Kusjuures oligi nagu naljakas, et tema alias oli Ott\sidenote{OK Jutuka nimi tulenes siis asutajate nimedest: Ott ja Kaups.}, aga tegelikult tema 
päris nimi oli Tiit. Lihtsalt selline huvitav asi. Kuidagi tundub, et kogu see 
maailm oli tollel ajal nii pisikene, et kui sa natukene selles maailmas ringi 
käisid, siis sa puutusid paratamatult kuidagi kõikide nende inimestega kokku, 
kes tollel ajal toimetasid.

\question{Räägi korra palun sellest, kuidas te Hoti tegite?}

Ja,see oli tegelikult ka päris huvitav. Kaitseväest tagasi tulles oli Eesti 
Telefon\index{Eesti Telefon}  Estpak Data\index{Estpak Data} ära söönud, Estpak 
Data lakkas olemast. Mingil hetkel ma töötasin siis Lasnamäel Koorti 15, kus 
Estpak Data enne oli, vana Eesti Telefoni maja, aga siis õite pea koliti meid 
sealt siis päris Eesti Telefoni muudesse ruumidesse ära. Ma olin Eesti Telefoni 
sellises allüksuses, mille pealkiri oli Teleteenuste Arendus.  Eesti Telefon 
oli teleteenuseid pakkuv ettevõte ja too üksus oli siis Eesti Telefoni 
arendusüksus, kelle eesmärk oligi välja töötada uusi teenuseid. Ja siis oma 
neti.ee tegemisega me sinna sattusime. 

Kontoriruumi jagasin ma ühe teise noormehega, kes arendas 
sissehelistamisteenust. Ja  huvitaval kombel meil vedeles kapi peal üks pisike 
Ascendi sissehelistamiskeskus, seal väga palju liine ei olnud. Ma küsisin, et 
kas ma võin seda uurida.

\question{See oli siis mingi tükk riistvara? Seal käisid tavalised modemid 
külge või oli ta juba valmis lahendus?}

Ei, ta oligi \emph{dedicated} sissehelistamiskeskus, et sa põhimõtteliselt 
installeerisid ta \emph{rack}i, panid  juhtmed külge ja ta hakkaski numbreid 
kuulama ja teenust osutama. Aga miks ma seda räägin, on see, et tolle keskuse 
uurimise käigus ma avastasin selle, et  sissehelistamiskeskus autendib ennast  
vastu sellist autentimisserverit nagu Radius. Sealt edasi uurisin, et mis asi 
see Radius on, sain teada, et see on \emph{dictionary}-põhine protokoll, 
üldsegi mitte keeruline ja ma programmeerisin siis Radiuse serveri, kes  suutis 
sissehelistamiskeskust juhtida. Avastasin, et selle sissehelistamiskeskusse 
\emph{firmware} võimaldab igasuguseid huvitavaid asju, mis tundusid olevat nagu 
seni kasuta. Näiteks see, et sa võid kohe Radiuse serverist öelda 
sissehelistamiskeskusele, kui kaua see kasutaja võib ühenduses olla. Ja sellest 
teadmisest näiteks sündis selline toode nagu Atlas Surf\index{Atlas Surf}, mida 
Eesti Telefon \emph{prepaid} Internetina \sidenote{Sarnane kontseptsioon nagu mobiili kõnekaardid.} müüs. Ühe sõnaga, see toode sündis 
puhtalt sellest, et mina häkkisin  seda väikest sissehelistamiskeskust, mis oli 
tegelikult üldse mõeldud  mobiilidega sisse helistamiseks. Ta toetas sellist 
huvitavat protokolli nagu V.35. Paljud pole sellest ilmselt mitte kunagi 
kuulnud, aga see oli selline \emph{wideband} protokoll, mis töötas üle GSMi. 
Kui sul oli selline GSM telefon, mida sai arvutiga ühendada, siis ta võimaldas 
sisse helistada selle V.35 protokolliga ja sa said veidi suurema kiiruse kui 
tavalist modemit vilistades üle  mobiili. 


Võib olla korraks hüppan natuke tulevikku. Oli aasta kaks tuhat, kõik mäletavad 
Y2K\sidenote{\enquote{Sinu lapselapsed neavad päeva, mil sa otsustasid oma 
koodi optimeerida}. Kuna pikka aega leiti aastaarvu hoidmiseks kahekohaline 
number piisav olevat, tehti sajandivahetuse paiku üüratus koguses tööd ja raha 
tagamaks, et aasta 2000 ei oleks arvutite arvates võrdne aastaga 1900. Vaata ka 
\enquote{Aasta 2038 probleem}.} probleem, kohutavalt hirmus, sest arvutid 
lähevad katki sest nende kell  lakkab töötamast ja nii. Ja ka Eesti Telefonis 
kardeti seda,  \emph{legacy} süsteeme oli tohutu palju. Kõik insenerid, kes 
olid mingisuguste süsteemidega seotud, pidid jääma  valvesse. Ma ei tea kuidas, 
minul õnnestus sellest ära nihverdada niimoodi, et tol hetkel ma olin Soomes 
suusatamas, sõpradega lumelauaga mäest alla laskmas. Stiilis paar päeva enne 
aastavahetust tuleb mulle  klienditeenindusest kõne, et kuuled, et nüüd sisse 
helistada enam ei saa, et mingi jama on. Läksin siis autoni, mul oli läptop 
kaasas. Olles Soome Vabariigis, panin telefoni läptopile järgi, helistasin 
sellesse meie enda privaatkeskusse sisse sellesama V.35  protokolliga ja 
hakkasin  vaatama, et mille pärast siis Hoti kliendid sisse helistada ei saa. 
Tuli välja, et keegi oli veel viimasel hetkel mingisuguse turvapaiga peale 
laadinud Y2K hirmus ja see muutis natukene seda teadet, mis Radiuse serverile 
saadeti ja siis Radiuse server läks selle peale katki, kuna talle tuli tundmatu 
sisuga \emph{dictionary}.

Surfist edasi tekkis selline olukord, kus Eesti Telefoni kontsessioonileping 
oli juba lõppenud või lõppemas, ja turule tuli Tele2\index{Tele2} Rootsist. 
Tele2  idee oli korrata Eestis täpselt sama, mida ta tegi Rootsis, ehk et ta 
soovis sellelt suurelt \emph{telco}lt palju raha välja imeda. Kuna Eesti 
Telefon üüris ruume, liine ja nii edasi, oli meil teada, et Tele2 paneb oma 
sissehelistamiskeskuseid püsti. Eesti Telefoni juhtkond oli sellest paanikas, 
ma ise ka külastasin mingisugust sellist laiendatud juhatuse koosolekut, kus 
sellest arutati. Ma mäletan, et ma tulin sealt üsna mornilt tagasi. Mulle 
tundus, et vanemad kolleegid ei suuda  midagi otsustada või ära teha. Mina 
sellise noore mehena oleks tahtnud kohe rauh-rauh, et läksime. Ma ei mäleta, 
mis asjaoludel ma olin kodus, aga ma pidasin siis telefonikõne Priit 
Pirsoga\index[ppl]{Pirso, Priit}, kes oli tollel hetkel selle valdkonna juht 
Eesti Telefonis. Ja selle telefonikõne käigus me otsustasime, et me teeme Eesti 
Telefoni osutatavale Atlas Starter teenusele alternatiivse teenuse, sellepärast 
et Atlas Starter absoluutselt ei sobi Tele2'ga konkureerimiseks. Meil on vaja 
sellist teenust, kus kasutajate  registreerimise protseduur ja selline on 
automaatne. \emph{Self-service}, kasutaja saab ennast ise registreerida ja nii 
edasi. Kuna  kuutasu niikuinii pärast seda Tele2 jampsi enam ei ole, siis 
ainukesed, mis maksavad, on kõneminuti hinnad. Tele2  lootis  raha teenida 
sellest, et ta termineerib kõnet ja Eesti Telefon on sunnitud talle 
nii vahendama kliendi käest küsitud kõneminuti hinda. 

Selle telefonikõne käigus me leppisime kokku, kes mida teeb, kuidas teeb ja ma 
olen ka üsna kindel, et selle kõne käigus me leppisime kokku, et toote saab 
nimeks saab Hot\index{hot.ee}. Sest ma muu seas juba arvutist vaatasin, et 
millised huvitavad domeenid on meil vabad. Kusjuures tollel ajal oli veel see 
aeg, kui EENet\index{EENet}  ei nõustunud andma ühele ettevõttele mitut 
domeeni. Mina ei tea, kuidas, aga minu üks tänane kolleeg, Guido 
Kõiv\index[ppl]{Kõiv, Guido}, temal õnnestus kuidagi EENetist saada 
hot.ee domeen meile, ma ei tea, kuidas. Sarnane \emph{inside}, nagu nagu 
Tarvil\index[ppl]{Martens, Tarvi} oli www.ee'le. Igal juhul me saime selle ühe 
või paari telefonikõnega, väga lühikese ajaga kokku lepitud, kes mida teeb. Ja 
kujutad sa ette, kahe nädala pärast me olime \emph{live}'is. See tähendab, et 
meil toimus teenuse \emph{launch} ja meil hakkas kasutajaid registreeruma 
tempoga tuhat tükki päevas.

Sealt sai siis hot.ee alguse. Minu teha jäi  seesama Radiuse pool. 
Hoti\index{hot.ee} omaaegne sisu tegelikult oli järgmine. Meie huvi oli see, et 
inimesed helistaksid meile sisse.  Tollel ajal hakkas ka juba olema kombeks, et 
anname ka kasutajale e-maili. Aga kuna varasemalt küsi e-maili eest raha, siis 
meile tundus, et lihtsalt nii samas neid e-maile jagada ei tahaks. Ja siis sai 
tehtud  selline kriuks, et sa saad küll veebipõhiselt konto luua (kusjuures 
imelik \emph{chicken-and-egg} probleem, et sul on Internetiühenduse konto 
loomiseks Internetti vaja, aga tundus, et see ei olnud takistuseks), aga see  
meilikonto ja ka kodulehekonto ei hakanud enne tööle kui sa olid selle 
registreeritud kontoga vähemalt ühe telefonikõne teinud 
sissehelistamiskeskusesse. Seda loogikat võimaldas siis minu \emph{custom} 
Radius, kes kasutajatel järge pidas. 

\question{Ühel hetkel oli hot.ee-s veebimeil ka, eks ole?}

Veebimailer oli, ma arvan, et suhteliselt algusest kohe juba sellesama esimese 
kujunduse osa juba. Aga see ei olnud minu programmeeritud, see oli Internetist 
leitud vabavara, mida me saime kasutada. Ma arvan, et me isegi seda nii-öelda 
ei \emph{re-brand}inud enda värvidesse, vaid see oli lihtsalt meie lehelt 
lingitud. Me ise hostisime teda.

\question{See seletab, miks meil mõned aastad hiljem veebimeileri tegemine 
Hansapangas\index{Hansapank} nurja läks, meil miskipärast ei tulnud pähe mõtet 
see lahendus Internetist lihtsalt alla laadida}

Mulle ei tulnud selline mõte pähe, et seda ise teha. Küll aga mäletan seda, et 
hiljem kui keegi mäletab, oli selline huvitav protokoll nagu WAP. Ehk siis 
mobiilivariant Internetist\sidenote{\emph{\enquote{WAP - Wireless Application 
Protocol}} oli sajandivahetuse paiku tekkinud ja põgusalt ka kasutusel olnud 
katse luua toona kasinate sidevõimalustega mobiiltelefonide jaoks lihtsam pinu 
internetiprotokolle 4. kuni 7. OSI kihini. Muu hulgas sisaldas standard ka 
erilist \emph{markup}-keelt toonase mobiiltelefoni mõnerealisele ekraanile 
sobivate kasutajaliideste loomiseks}. Vot selle WAP-meili ma küll tegin täiesti 
nullist sellelesamale Hotile.

\question{Mis õnneks ei olnud väga pika elueaga sest WAP ei olnud väga pika 
elueaga}

Mina mäletan ka seda legendaarset väidet Ando Meentalolt\index[ppl]{Meentalo, 
Ando}, kes oli tollel ajal EMT üks arendusjuhte, kes kommenteeris minu WAP 
meili nii, et \enquote{noh, sa võid ju sinna suahiili keele ka panna, aga 
ilmselt pole sellest väga palju kasu}. Aga kogu see WAP sai minul isiklikult 
alguse sellest, et ma olin saanud endale WAPi-võimelise telefoni. Ma arvan, et 
see oli üks ainukesi telefone, mida ma olen iialgi tööandjalt saanud. See oli 
siis Nokia 7110\sidenote{Tegu oli 1999. aastal uskumatult innovatiivse 
telefoniga: mitut tekstirida näitav ekraan, rullikuga kasutajaliides, T9 
ennustav tekstisisestus sõnumite puhul, vedruga uhkelt lahti hüppav klapp, WAP, 
ebamaiselt küütlev korpus jne. Oma isepärase kuju tõttu sai aparaat rahva seas 
hüüdnimeks \enquote{banaan}}, klapiga telefon, millel oli suur ekraan. 

\question{See telefon oli muide suurepärane põhjus Hansapangale WAP-i põhine 
internetipank teha. Sest selle testimiseks pidi ju pank ometigi väljastama ka 
sobiliku seadme}

Mul oli \emph{vice-versa} selles mõttes, et ma sain kõigepealt telefoni, siis 
mul tuli idee, et mul telefon nüüd on, aga mida ma sellega teen  ja et jube äge 
oleks enda postkasti sisse vaadata sellisel mugaval moel.  Ja siis ma tegin 
WAP-meili.

\question{Aga sellega algab juba uus sajand ja sellest me räägime võib olla 
mõni teine kord. Lõpetuseks küsin veel, et mis sa praegu teed?}

Praegu ma olen Bolt serveri infra peal.  Minu üks kauaaegseid 
kolleege sealt samast Eesti Telefonist Tarmo Kople\index[ppl]{Kople, Tarmo} on 
üks nendest inseneridest, kellega me alustasime Bolti kogu seda 
serverimajandust praktiliselt juba algusest peale. Kui meie alustasime Tarmoga 
serverite poole majandamist Boltis, siis meil oli stiilis tuhandeid kliente ja 
tuhandeid sõitusid kuus ja nüüd see on siis asendunud miljonite klientide ja 
miljonite sõitudega.


\chapter{Kersti Kaljulaid}
\index[ppl]{Kaljulaid, Kersti}

\question{Alustaks kohe sealt, kust asjad ikka algavad ja kust me oleme kõiki 
neid jutuajamisi alustanud. Kuidas Teie jõudsite arvutite 
juurde?}

See juhtus üsna ammu, Nõukogude Liidu päevil 
Õpilaste Teaduslikus Ühingus\index{Õpilaste Teaduslik Ühing}. Ma arvan, et 
päris paljud minuvanused inimesed, kes on hiljem töötanud Eesti e-riigi või 
\emph{start-up}-kogukonnas, teavad, mida tähendab 
Küber\index{Küber} või Küberi arvutuskeskus. Tartu 
Ülikoolil\index{Tartu Ülikool} olid samuti olemas arvutuskeskused. Nii põhjas 
kui ka lõunas otsustasid täiskasvanud millegipärast, et lasevad lapsi sinna 
mängima. 

\question{Millised täiskasvanud?}

Õpilaste Teadusliku Ühingu eestvedajad. Näiteks Peeter 
Lorents\index[ppl]{Lorents, Peeter}, kes juhtis matemaatikasektsiooni. Ma ei tea, kes Tartus seda eest vedas, küll aga seda, et ka Tartu 
koolinoortel, näiteks Unineti\index{Uninet} Taavi Talvikul\index[ppl]{Talvik, 
Taavi} oli ligipääs Tartu Ülikooli arvutipargile. Toimus instinktiivne 
õpe, mis viis meid Õpilaste Teaduslikus Ühingus aruteludeni, 
kuidas kirjutada sellist asja, mida keskserveril oleks mõnusam analüüsida. 

Tollal oli nii, et üks asi arvutas ja ümberringi olid terminalid, kus me 
oma koodi kirjutasime. Tolleaegsed masinad olid mitteselektiivsed --- 
need ei otsinud, kes meist efektiivsema rea on kirjutanud, et seda siis 
töödelda. Aga meile tundus, et äkki oleks võimalik üksteisega kunagi 
võistelda, kes kirjutab sellise asja, mida keskserveril 
oleks mõnusam analüüsida. Mäletan laadilisi debatte, ükskord toimus
Õpilaste Teadusliku Ühingu suvelaagris\index{Õpilaste Teaduslik 
Ühing!Suvelaager} vist isegi öine matemaatikadebatt. 

Rõhutan, et mina ei kuulunud matemaatikasektsiooni, aga mind võeti kuidagi 
kampa. Olin üheteistkümnendas klassis ühingu 
teaduslik peasekretär, aga ise tegelikult 
ornitoloogiasektsioonist. Mulle meeldis Linné-aegne 
bioloogia\sidenote{Carl Linnaeus, pärast aadliseisusse tõstmist 1761. aastal 
Carl von Linné (1707--1778), oli Rootsi teadlane, kes formaliseeris organismide 
nimetamise süsteemi ja keda tuntakse moodsa taksonoomia isana.}, mis oli 
koolilapsele kättesaadav. Linnud, loomad, taimed --- kõik see viis mind 
laia maailma, küll kuuendikul planeedist, aga olümpiaadidel käies
sai reisida. 

Igatahes Õpilaste Teaduslikus Ühingus tekkis mul kokkupuude Küberi pundi ja 
arvutusvõimsusega.

\question{Nii et isegi ornitolooge viidi arvuti juurde?}

Ei, ornitolooge ei viidud, mul lihtsalt olid matemaatikasektsioonis sõbrad. Aga 
Lorents\index[ppl]{Lorents, Peeter}, Engelbrecht\index[ppl]{Engelbrecht, Jüri} ja teised ei teinud selles mõttes vahet, et me võisime olla 
neliteist-viisteist-kuusteist, aga saime osaleda täiskasvanute 
akadeemilistes mängudes.

\question{Neljateistaastasel noorel, eriti ornitoloogiahuvilisel, 
on miljon muud asja teha. Mis tõmbas just arvuti poole?}

Esiteks oli see põnev maailm. Teiseks, minu ainuke ornitoloogiaalane 
publitseeritud teadustöö, mis tegeles vainurästa pesitsuskommetega, viis mind statistilise tööni ja ilmselt seeläbi arvutini. See ei 
käinud nii, et vaatasin metsas, kus vainurästas elab --- selle olid 
teised inimesed ära teinud. Eestimaa Ornitoloogiaühingus\index{Eesti Ornitoloogiaühing} (või kuidas iganes Vene ajal seda kogukonda ka ei nimetatud\sidenote{Nimetati \enquote{Eesti Looduseuurijate Seltsi ornitoloogiasektsioon}.}) oli 
kogunenud meeletus koguses pesitsuskaarte, kõik täiesti 
süstematiseerimata materjal. Minu akadeemiline tegevus Õpilaste Teaduslikus 
Ühingus seisneski selles, et otsisin statistiliste meetoditega erinevaid 
korrelatsioone. Näiteks tuli sealt välja, et linnas pesitseb vainurästas 
kõrgemal kui kuskil looduslikus biotoobis. Vainurästast ennast polnud vaja selle töö jaoks isegi mitte metsas ära tunda. Mitte 
et ma ei tunneks, aga arvutamiseks seda vaja ei olnud.

\question{Milles akadeemilised mängud seisnesid? Mis tüüpi ülesandeid te 
arvutiga lahendasite?}

Lihtsaid programmeerimisharjutusi, mida täna teevad paljud lapsed algkoolideski. Mõni täiskasvanu oli meil alati juures või siis tulime ka ise selle peale. Mina kaugemale väga ei jõudnudki, sest ma 
ei olnud matemaatikasektsioonis.

\question{Millised inimesed matemaatikasektsioonis olid? Kas
tüüpilised nohikud?}

Ei, seal oli erinevaid. Näiteks Tarvi Martens\index[ppl]{Martens, Tarvi} ja 
Tarmo Uustalu\index[ppl]{Uustalu, Tarmo} on täitsa erinevate kategooriate inimesed. Oli mitmesuguseid matemaatikahuvilisi noori erinevatest Eesti 
nurkadest. Minu arust ei ole olemas sellist stereotüüpi, mida alati 
otsitakse. Gruppidevahelised erisused on teadupärast väiksemad kui 
grupisisesed.


\question{Kas Õpilaste Teaduslikul Ühingul oli ka võrgustiku loomise 
funktsioon?}

Kindlasti. Olid erinevad sektsioonid: matemaatika, loodusteaduste, 
geograafia ja ajaloo, sealhulgas NSV Liidu ajaloo sektsioon, mis 
pandi ükskord kinni, mille üle mõned punasemad noored olid väheke nördinud. Näiteks Teet 
Jagomägi\index[ppl]{Jagomägi, Teet}, kes on tänaseks selgelt IT-ettevõtja, juhtis 
geograafiasektsiooni. Päris palju praegusest umbes 50aastaste kogukonnast on sealt ühel või teisel viisil läbi käinud. Kõik 
tundsid kõiki nagu ikka tollal.

\question{Mis siis sai, kui teaduslik ühing lõppes?}

Läksin Tartu Ülikooli. Üksiti oli see ka hüvastijätmine 
Linné-aegse bioloogiaga, sest minu juhendaja Raivo Mänd\index[ppl]{Mänd, 
Raivo} ütles, et tuleb õppida uusi asju, mis toovad tulevikus leiva lauale. Sellepärast ongi minu eriala geneetika, täpsemalt plasmiidigeneetika 
või bakterigeneetika. Bioloogias on ju keemia, füüsika ja matemaatika 
kõik koos, nii et stuudiumi jooksul tugevnes kindlasti minu arusaam 
matemaatikast kui kõike kirjeldavast ja kõiges toeks olevast teadusharust. 
Matemaatika on minu jaoks nagu keel. Ma ei ole selles ülearu osav, aga vajaduse 
piires olen suutnud toimetada.

\question{Tartu Ülikooli Arvutuskeskuse\index{Tartu Ülikool!Arvutuskeskus} 
kohta on räägitud, et seal käis koos üsna 
kirju seltskond teoloogidest jumal teab kelleni. Kas Teil oli selle kohaga ka 
kokkupuudet?}

Arvutuskeskuse seltskonna kirjusus tulenes muu seas sellest, et matemaatika 
ja füüsika olid erialad, millele konkurss Tartu Ülikoolis puudus. Seal oli 
alati kohti rohkem kui rahvast ja tihtipeale pugesid nendesse teaduskondadesse 
peitu ka inimesed, kes iga hinna eest tahtsid näiteks vältida Nõukogude 
sõjaväge. Minu aastal ülikooli astunud füüsikutest vist üks lõpetas füüsikuna. 
Küll aga astus sinna sisse näiteks Anzori 
Barkalaja\index[ppl]{Barkalaja, Anzori}, kindlasti teise eriala inimene. 

Seltskond oli kirju, aga arvutiteadus ongi suuresti 
interdistsiplinaarne, mitte spetsiifiline. Olen 
ise kodus märganud, et see on kuidagi pärilik --- minu esimene abikaasa 
ja vanem poeg elavad mõlemad arvutimaailmas, elavad ja hingavad 
bitte ja baite. Kuigi seda on ju võimatu näidata, kuidas pärilikkus saab 
millegi nii tehnogeensega koos käia, tundub mulle, et mingisugune ajutüüp 
peab selleks siiski olema. Ka vanemal tütrel on arvutiinimese aju. 
Tol ajal hakkaski selguma, et osal inimestel on 
arvutiinimeste ajud --- nad tõmbusid arvutuskeskusesse kokku, said 
üksteisest aru ja hakkasid vähehaaval kaotama sidet humanitaarsema poolega 
ühiskonnast.

\question{Mõtlesin, et lause lõpeb sellega, et \enquote{hakkasid kaotama 
sidet reaalsusega}, sest ka seda juhtus seal majas kergesti \ldots}

Ei, seda mitte. Arvan, et osa 
inimesi see maailm ei kõneta ja teisi intuitiivselt kõnetab. 
Olen kogu aeg tundnud, et ma ise arvutimaailmas sees ei ole, aga võib-olla suudan kahe 
maailma vahel natuke tõlkida. See tunnetus on olnud päris varasest 
noorusest peale. Olen telekomisektoris töötanud, telefonijaam on ju nagu
arvuti: seda tuli samamoodi konfigureerida ja programmeerida. Töötasin üheksakümnendatel
palju koos inimestega, kes pidasid oma tööks arvutitega töötamist; ma ise seda ei teinud, küll aga püüdsin luua neile töötingimusi mõnes 
ettevõttes. Ma olen olnud nagu piirpinnal kõndija.

\question{See on väga põnev, sest seestpoolt vaadates tunduvad mõned 
asjad ilmselged ja mõned ebaselged ning kõrvalt vaadates võivad 
asjad paremini paista. Kuidas Te 
bakterigeneetikast ühtäkki telekomi sattusite?}

See oli imelihtne. Kui ma lõpetasin ülikooli, siis täpselt samal päeval tuli 
Eesti kroon\sidenote{20. juunil 1992.} ja ülikoolide pakutud tulutase 
(jäin pärast lõpetamist ülikooli tööle) oli nii väike, et sellega ei olnud 
võimalik lasteaiatasusidki katta. Sain aru, et 
uues Eestis läheb kas väga kaua aega, kuni see valdkond hakkab ära tasuma, või 
siis tuleb minna Eestist ära. Paljud kursusekaaslased lahkusidki Eestist 
ja neil kõigil on olnud väga edukas karjäär. Paljud on tulnud ka
tagasi ning on nüüd professorid Tartu Ülikoolis ja Eesti Maaülikoolis. Nad
ehitasid oma karjääri mujal üles ning kui Euroopa Liit asus 
laienema ja ka meie teadustaristut üles ehitama, siis neil oli super võimalus 
oma kolmekümnendate keskpaigas Eestisse naasta ja oma näo järgi 
laboreid ja uurimiskeskusi kujundada. Selles mõttes tõsine võitjate põlvkond.
 
Mina ei tahtnud Eestist ära minna, sest meil olid väiksed lapsed ja ma 
soovisin, et nad oleksid eestlased. Läksin puhtalt 
raha pärast teadusest erasektorisse. Töötasin pisikeses ettevõttes, mis paigaldas Siemensi 
telekommunikatsioonijaamu. Tööle võeti mind selleks, et tõlgiksin materjale eesti keelde, sest korralik inglise keele 
oskus ei olnud tollal nii levinud. Siis aga selgus, et tõenäoliselt kõlban 
päris hästi ka müüma. Ja kuna väikestel ettevõtetel on üks juht, siis 
oledki nii müügidirektor kui ka lihtsalt direktor. 

\question{Mis aastal see oli?}

Läksin sinna 1994. aasta sügisel. 

\question{1994 oli veel suhteliselt hull aeg. 
Kes tol ajal üldse Siemensi jaama endale paigaldas? Isegi analoogtelefon oli 
mõnes kohas haruldane.}

Metsikult pandi. See oli just see aeg, kus saadi aru, et büroohoonetes, 
ülikoolides, raamatukogudes ja igal pool mujal on tuppa telefoni vaja. 
Mul endal on vastupidine mulje, et Siemens, Ericsson ja 
väiksematest tegijatest näiteks Panasonic 
müüsid terve linna täis alates Rahvusraamatukogust ja lõpetades 
pankadega. Turgu oli kõvasti! 

Siis tuli Eesti esimene riigihangete seadus ja
pidi hakkama selle järgi pakkumisi koostama. Tundus tohutu põnev, aga ka pisut 
hirmutav, sest varem ei olnud niimoodi käinud, et teed pakkumise ja siis 
loetakse kõik ette. Mul on meeles, kuidas istusime vist Tallinna Vanglas sealse  
telefonikeskjaama hankel ja selline tunne oli, et ei teagi, kas välja enam 
saab. Mitte et oleksime midagi valesti teinud, 
aga maailm muutus, süsteemi tekkis selgroogu ja struktuuri 
juurde.

\question{Te jõudsite pärast ka Eesti Telekomi, aga mina mäletan tollest ajast, et 
see oli mingisugune õudne monstrum!}

See polnud siis veel Eesti Telekom, vaid Eesti Telefon\index{Eesti Telefon} ja 
mina töötasin sellises peenes kohas nagu Eesti Telefoni äriklienditalitus. Läksin erasektorist sinna, sest seal tundus 
olevat rohkem karjäärivõimalusi. Olin väikses ettevõttes tipus ja tundsin, et 
tahaksin edasi liikuda, suuremat struktuuri vaadata. 

Eesti Telefonis oli tollal päris keeruline. Ükskord öeldi mulle, et sel kuul ei saa rohkem müüa, sest meil sai 
sisseostuplaan täis. Siis ma olin jube kuri ja tegin üheselt selgeks, et ma ei 
taha mitte kunagi enam sellist väidet kuulda --- kui müüme, siis müüme, ja 
kui te ei taha, siis ma lähen midagi muud tegema. Aga müüsime küll.

\question{Eesti Telefon oli minusuguste nohikute jaoks 
üsna õudne ettevõtte: küll ei suutnud traati pakkuda ja ei müünud mingil hetkel
isegi internetti, väites, et telefoniliini peal ei 
peagi internet töötama, sest liin on helistamiseks.}

Ma ei tea äriklienditalituse loojate kaalutlusi, aga arvan, et asja 
mõte oligi tuua sinna struktuuri üks üksus, mis hakkaks senist organisatsioonikultuuri seestpoolt õõnestama ja muutma. 
Tolle talituse rahvas pidi müüma täitsa tavalistele eraettevõtetele ja 
 neidsamu jaamu, mida Siemens ja Ericsson ka eraldi müüsid. Ma täpselt ei mäleta, kuidas Valdo Kalm\index[ppl]{Kalm, Valdo} selle 
talitusega seotud oli, aga igal juhul oli ja aitas
ettevõtte muutumisele kindlasti väga palju kaasa. 

Algus oli minu jaoks keeruline. Mõni inimene küsis, miks mind kunagi 
oma laua taga ei ole. Minu arust müügijuht ei peagi olema oma laua taga. Aga tasapisi see kultuurimuutus tekkis.

\question{Kultuuri mõttes oligi huvitav aeg, kui sidevaldkonnas kombineerusid
äri, akadeemiline kogukond ja häkkerite-nohikute maailm. Kas see paistis Eesti Telefoni poolelt ka välja?}

Muidugi paistis, sest inimesed olid samad. Võtame või Taavi 
Talviku\index[ppl]{Talvik, Taavi}, minu esimese abikaasa, kes tuli Tartu 
Ülikooli Arvutuskeskusest\index{Tartu Ülikool!Arvutuskeskus} läbi 
Valitsusside\index{Valitsusside}. Seejärel tegid nad Andes 
Baumaniga\index[ppl]{Bauman, Andres} oma ettevõtte Uninet, mis hiljem müüdi ära ja millest sai 
Elisa. 

Selles mõttes oligi üks maailm. Mis seal lõppude lõpuks vahet on, kas helistad või saadad muid 
andmeühikuid? Digitaalne tehnoloogia tol ajal just tuli. Tekkisid 
probleemid, kuidas tagada läbilaskvus ja ühenduste laius --- kogu see maailm 
hakkas vaikselt arenema ja kasvama. Minu meelest 
pole see Eestis kunagi eraldi olnud. Kandja pool ei olnud kindlasti eraldi ja sisu 
poole ettevõtteid tollal ju eriti ei olnudki veel. Esimene internetipank tekkis vist aastal 1994? Aastal 1997 oli juba e-maksuamet.

Sisuteenused hakkasid ka üsna kiiresti tulema, aga siis algas kohe ka see 
võistlus, et toru on küll olemas, aga sisu tahab laiemat, ja kui toru saab laiemaks, siis tahab sisu 
veel laiemat. Ma ise olin selgelt toru, mitte sisu poolel.

\question{Rääkisime, et stereotüüpe ei ole. 
Ometigi on Eesti Vabariigis olnud laialt käibel niisugune mõiste nagu 
\enquote{patsiga poiss}. Mis inimene see on? Mis teda iseloomustab?}

Neid on väga erinevaid. Eesti Telekomi aegadest ma ei mäleta peaaegu kedagi 
peale Valdo Kalmu\index[ppl]{Kalm, Valdo}, kes seal minuga veel koos töötasid --- palun vabandust endiste kolleegide ees! Aga näiteks mäletan Uku 
Kuuti\index[ppl]{Kuut, Uku}, kes oli meil süsadmin, patsiga poiss. Ja 
kui tema tuppa läksid, sest midagi oli paigast ära, siis tal muusika alati käis. 
Tundus küll nagu teine maailm võrreldes paljude teistega, aga kindlasti 
oli ka pöetud habemega rahvast.

Need olidki üsna erinevad seltskonnad. Kui müüsid telefonijaama ja tegelesid 
pankadega, siis oli selgelt näha, et pankade tehnikajuhid vastutasid juba 
tollal suhteliselt suure struktuuri püstihoidmise ja edasiarendamise 
eest, olid hästi makstud ja ei erinenud millegi poolest pankade
raamatupidajatest. Neid ei kutsutud siis veel CTOdeks, aga seda nad sisuliselt 
olid ja ei erinenud muude valdkondade eest vastutajatest. 

Ja siis oli iseõppinud vendi, kes olid kuidagipidi (ega seda ülikoolides ei õpetatud) 
ise arvuteid pidi ringi nuhkinud ja saavutanud oskuse hoida asjad töös. 
Nende hulgas oli jah võib-olla seda stereotüüpi, et nad
suhtlevad parema meelega masinaga. Samas ei olnud masin tol ajal nii huvitav 
suhtluspartner ja ei olnud võimalust internetti päris ära kaduda. Ma arvan, et see on üle võimendatud, 
kuidas seal kastis saab kogu ööpäeva ära sisustada. 

Ühesõnaga, inimesi oli igasuguseid.

\question{Meil on olnud teistega juttu sellest, et teatava peakujuga 
inimesi tõmbas Tartu Ülikooli arvutuskeskusse --- võib-olla see tõmme on 
ühine nimetaja?}

Kindlasti. Need, kes Tallinnas 
Küberis\index{Küber} koos käisid, on kõik selles sektoris tänini leitavad, nad on olnud
püsivamad ja järjepidevamad kui mina. Midagi ju on, mis tõmbab meid
matemaatika või keelte juurde. Need on erinevad 
asjad. Võib-olla praegune 
keeleinstituudi\index{Eesti Keele Instituut} direktor Arvi Tavast\index[ppl]{Tavast, Arvi} on mõlemal poolel 
kõndija: ühtpidi IT-tegija ja teistpidi on teda sügavalt huvitanud 
keeled ning need kaks asja saavad tänases maailmas kokku. Enamik inimesi 
kipub siiski olema paremal või vasakul. Ma ei tea, miks.

\enquote{Kui vaadata kasvõi inimesi, kellega ma selle raamatu raames rääkinud olen, siis 
naisi on vähe. Tol ajal oligi neid selles valdkonnas vähe. Miks?}\label{sisu:tydrukud}

Naised on alalhoidlikumad ja toimetavad valdavalt sektorites, mis on 
sisse töötatud. Paratamatult on nad ka karjääri mõttes 
alalhoidlikumad. Minulgi oli ühel hetkel päris 
palju ideid ja valikuid, mida võiks teha, näiteks kas tekitada oma butiik ja 
hakata seda arendama. Ma ei teinud seda sel lihtsal põhjusel, et pidin ülal pidama kaht alaealist last --- olin tol hetkel üksikema. See oli teadlik valik, sest mul oli tarvis rohkem kindlustunnet ja struktureeritud elu. Ma 
ei saanud endale lubada, et olen võib-olla järgmised viis kuud ilma palgata. 

Kuna see oli tollal uus valdkond, siis naised lihtsalt ei võtnud neid riske. Võib-olla 
seiklusjanu oli ka esialgu väiksem ja ega keegi ei näinud ju 
sellega ka teadlikult vaeva. Me räägime sügavatest 1990ndatest! Toon ühe
eheda näite. Mina müüsin siin Siemensi Hicom 300\sidenote{Siemensi paindlik 
telefonijaamade sari.}, Soomes tegeles müügiga üks Tiina. Kord läksime 
Siemensi suurte telefonikeskjaamade müügimeeste kokkutulekule, kus olid peale meie ainult 
mehed, kes küsisid uskumatul ilmel: 
\enquote{Kas te tõesti müüte neid suuri telefonijaamu?} Me ei saanud aru, miks 
me ei võiks seda teha. Tol ajal ei olnud see naiste maailm. 

\enquote{Kui Eestis oli selline avantürism arusaadav, siis muu maailm oli selles mõttes ikkagi teistsugune?!}

Võtame Kesk-Euroopa, näiteks Saksamaa. Seal on
minu põlvkonnas veel päris palju koduperenaisi, just Lääne-Saksamaal. 
Idas ei ole. Statistiliselt on Ida-Saksamaa naiste 
pensionid kõrgemad, samuti lahutuste arv, sest nad saavad 
seda endale lubada erinevalt läänest. Kui me kujutame ette, et 
üheksakümnendatel oli ärikultuuris tohutu võrdõiguslikkus, siis ma kaldun arvama, 
et see pole sinna päriselt jõudnudki ja selle nimel võideldakse. Meil ei maksa luua endale illusioone, 
et seda tööd ei pea enam tegema. Selles mõttes on Taavi Kotka\index[ppl]{Kotka, 
Taavi} Unicorn Squad, Rakett ja teised sarnased ettevõtmised tüdrukute 
toomiseks tehnoloogia ligi ühiskonnale tohutu väärtusega. Ei 
ole mitte ühtegi põhjust, miks tütarlaps ei võiks toimetada tehnoloogiarikastel 
aladel.

\question{Tütarlapse jaoks võib ehk olla vähem loomulik see, et ta 
magab kontorilaua all, sest uni tuli peale?}

Ära hakka seda \enquote{loomulik või vähem loomulik}! Ei ole niimoodi, kuigi paljud võivad sedasi arvata! Mis seal vahet on? Sul 
on 20aastane vaba inimene, lapsi ega peret ei ole --- ükskõik, kas ta on mees või naine. Kui 
ta seal kontorilaua all magab, siis teksad on nagunii jalas, soengu ja 
seelikuga sinna ju ei lähe. See ongi see alateadlik stereotüüp, isegi kui see ei ole pahatahtlik. Aga see on täiesti olemas, nagu ka sinu 
väites!

\question{Tõsi, ka punkareid oli igasuguseid.}

Jah, samamoodi võib olla insenere ja keda iganes.

\question{Mul lihtsalt tuleb silme ette üks konkreetne habemega punkar, kes 
vedeles niimoodi hommikul laua all. Aga tõepoolest, see on minu stereotüüp ja 
kujutluspilt, et see on habemega punkar --- miks ta ei võiks olla 
teistsugune punkar!}

Mu tütar rääkis ülikooliajast loo, kuidas üks ettevõte otsis tööjõudu. 
Päris paljud käisid ennast pakkumas ja võeti üks poiss, kes oli ülikoolis 
silmnähtavalt teistest laisem ja kehvema õppeedukusega. Mõne aja pärast, kui ta oli kohanenud, küsis poiss
tööandjalt: \enquote{Kuule, meilt kandideerisid veel need ja 
need inimesed, miks nemad ei saanud?}. Mille peale talle öeldi: 
\enquote{Vaata ümberringi, näed sa siin mõnda naist?} Selline tõrjuv kultuur, 
eks ole. See ei ole nüüd side-, telekomi- ega ka IT-ettevõtte, vaid lihtsalt insenerikultuuri näide Eestist. Ja need inimesed, 
kellest ma räägin, on praegu 32---33-, mitte 50aastased.

\question{Mugavam on palgata omasugust. Iseküsimus, kas ka kasulikum.}

Ei ole, sellepärast et statistiliselt tuleb naiste pähe 50 protsenti headest 
ideedest ja meeste pähe 50 protsenti.

\question{Pigem on isegi teistpidi \ldots}

Ma ei lähe sinna kunagi. Me ei peaks ütlema, 
et naised on kuidagi teistmoodi juhid või insenerid. Väidan, et oleme 
ajupotentsiaali mõttes võrdsed. Paraku on ka minu käest 
küsitud: \enquote{Kuidas siis nüüd nii, et naine juhib elektrijaama?} 
Olen siis väga otsekoheselt vastu küsinud, et kuule, räägi nüüd, 
mida sa selle tilliga teed jaama juhtides? 

\question{Kuna meie IT-värk tuleb sellest seltskonnast, mis oli
faktiliselt ühele poole kallutatud, siis kas see on meid kuidagi 
digiühiskonnana tagasi hoidnud või edasi aidanud või üldse mingit mõju avaldanud?}

Tegelikult ei ole. Olen meie digiriigi arengu peale palju mõelnud. Miks me oleme nii 
teistmoodi, kuigi mujal on palju tugevama IT-sektoriga erasektor kui meil 
Eestis? Ühiskonda muudab ikkagi riik, midagi ei ole teha. 
Erasektoris võidakse teha geniaalseid asju, aga \emph{mainstream}'imise 
maailmameistrid on kõik riigid. See on see koht, kus riik peab midagi 
tegema hakates kaasa võtma kõik: vanad, noored, mehed, naised! Ja 
selle muudatuse on Eesti ära teinud.

Priit Alamäe\index[ppl]{Alamäe, 
Priit} vist tõi kasutusele termini \emph{digitally transformed nation}. Siin on 
nüüd see koht, kus me läksime teist teed kui teised maailma riigid. Kui 
riik hakkab tundma huvi mõne sektori võimaluste kasutamise vastu 
riigiteenuste osutamiseks, siis see hakkab päriselt ühiskonda muutma 
ja kujundama. Meil juhtus see, et ühest hetkest olid kaasatud 
mehed, naised, lapsed, vanurid, ja tekkisid ka positiivsed kõrvalefektid. Näiteks koroonapandeemia tekkides meil ju ei olnud häda, et 
70aastased ei saa pangas käidud või 
telekomilepinguid uuendatud, sest nad olid 50aastased, kui ID-kaart tuli. Sellised
positiivsed kõrvalefektid on olnud kogu ühiskonna jaoks hästi suured.

\question{Miks meil juhtus niimoodi?}

Sellepärast, et meil ei olnud midagi. Ajalooliselt oli ju esimene 
e-teenus e-maksuamet. Kas sa kujutad ette, et Eesti inimesed oleksid 
nõustunud seisma tundidepikkustes sabades, et riigile oma maksud ära viia?

\question{Praegu enam ei kujuta.}

Aga siis ka ei kujutanud. Oli tõenäone, et maksulaekumised ei ole ülearu head, 
kui loodad sellisele asjale. Õnneks oli selline aeg, et 
e-pangandus ju oli ja sai teha e-maksuameti. Keegi ei taha ju maksuametnikku näha! 

\question{Ma ei ole kuskil mujal näinud sellist usaldust 
keskmise bürokraadi ja \emph{hardcore} inseneri vahel. Ühest küljest 
insener usaldab, et bürokraat ei keera asja kihva. Teisest küljest 
bürokraat usaldab, et kui tehnik hakkab rääkima XML-sõnumite 
vahetamisest, siis ta päris udu ei aja ja tarnib tulemuse. Kust meil see usaldus
tuli? IT-kogukond isekeskis küll teadis, tundis ja usaldas üksteist, 
aga kuidas laiem ühiskond juurde tuli?}

Ega ei tulnudki. Kui ma läksin 1999. aastal peaminister Laari\index[ppl]{Laar, 
Mart} juurde tööle, siis tema tellimus oli suhteliselt mittespetsiifiline. Ta võttis oma nõunikud kokku ja ütles: \enquote{Nüüd 
on nii, et me kõik saame aru, et Eestis palgad kasvavad ja varsti me ei 
ole enam rikka mehe kuluefektiivsuse lahendus. Vaadake ringi, kuhu edasi minna.} 
Sealt algas see, et kui minu juurde tuli Andres 
Metspalu\index[ppl]{Metspalu, Andres} jutuga, et oleks vaja teha 
Geenivaramu\index{Geenivaramu}, siis sai kõik rattad käima lükatud. Eiki 
Nestor\index[ppl]{Nestor, Eiki} tuli ka appi ja tegime inimgeeniuuringute 
seaduse. 

Teise valdkonna --- ID-kaardi --- pakkus välja Linnar 
Viik\index[ppl]{Viik, Linnar}. Tol ajal 
hakkas juba tekkima ka sisuteenust ning Kaarel Tarand\index[ppl]{Tarand, Kaarel} nägi 
väga hästi seda pilti, kuhu see sisu pool kommunikatsioonis ja mujal
minemas on. ID-kaardi idee vist tekkiski sellest, et 
pangad ei olnud ju pikaajaliselt nõus võtma vastutust, et riik jooksutab oma 
e-teenuseid nende platvormidelt. Seetõttu tuligi teha ID-kaart, aga see 
sündis ühise veenmistöö tulemusel. 

Jube raske oli veenda rahandusministeeriumit, kes tahtis kohe 
teada, kus on ROI\sidenote{\emph{Return on investment} --- investeeringu 
tasuvuse näitaja.}. Täna tundub see naljakas küsimus: \enquote{Mis 
mõttes, vaadake, milline e-riik meil on!} Aga kust Linnar\index[ppl]{Viik, 
Linnar} tollal need arvud oleks võtnud? Põhjendus, 
millega tegelikult ID-kaarti valitsusele müüdi, oli ju absurdne: e-valitsus. See, et valitsuses olid 
arvutid laua peal, oli toonud meile nii palju tasuta artikleid 
välisajakirjanduses, et see süsteem oli ennast kolme kuuga tasa teeninud 
võrreldes sellega, kui oleksime lihtsalt ostnud \emph{Estonia --- Positively 
Transforming}\sidenote{2002. aastal käivitatud suur ja mitmesugust 
meediatähelepanu pälvinud \emph{Brand Estonia} kontseptsioon \enquote{Welcome 
to Estonia} põhiline tunnuslause.} lehepinda. Selle argumendiga 
tehti ID-kaart! 

Nii et see arvamus, et kõik tulid Linnar 
Viigi\index[ppl]{Viik, Linnar} ja teistega kohe kaasa, on vale. Töötasid teised argumendid: kulu ei olnud nii suur ja 
Vabariigi Valitsuse ruum oli tõepoolest väga palju välismaist positiivset tähelepanu 
saanud. Seejuures teenimatut, sest nendesamade Saksa ja Soome inimeste erafirmades oli intranet ju täiesti tavaline. 
Eestis samuti --- Eesti esimene intranet hakkas tööle Postimehe\index{Postimees} toimetuses aastal 1991 või 1992. Aga riikide tasandil seda ei tehtud ja see siis oligi sirgelt ID-kaardi müügiargument: 
võime saada palju välismaist tähelepanu.

\question{Ilmselt oli maitse suus, et korra oleme juba saanud, küllap 
nüüd ka!}

Just. Ja asja geniaalsus seisnes selles, et näiteks Saksamaal pead 
siiamaani omale digi-IDd taotlema, aga meie pistsime selle 
lihtsalt kõikidele kaardi peale. Kasutavad või ei kasuta, aga ega see liiga ka ei 
tee. Tegime ainult ühte tüüpi kiibiga ID-kaardi ja see oli geniaalselt 
õige lahendus.

\question{Minu teada ei ole keegi teine seda ka eriti järele teinud.}

Nüüd vist ikka juba on, aga mitukümmend aastat hiljem \ldots

\question{Kui palju oli arusaamist, et 
tegemist on geniaalse lükkega, kui palju pikka visiooni ja kui palju 
praktilist kaalutlust, et \enquote{paneme lihtsalt käima}?}

Ma mäletan seda arutelu. Äkki ei teeks, äkki teeks. Teeks 
kõigile. Maksab nii palju. Kui teeme osadele, kas maksab vähem? Ei 
maksa vähem, vaid tegelikult rohkem, kui on erinevad süsteemid. 

Oli tõesti hulk inimesi, sealhulgas Infotehnoloogiafirmade 
Liit\index{Infotehnoloogiafirmade Liit} ja Linnar 
Viik\index[ppl]{Viik, Linnar} ning ilmselt teisigi, kes ütlesid, et võtab küll aega, kuni teenused peale 
lähevad, aga anname kõigile. Analoogselt olime ühel hetkel 
otsustanud, et mitte keegi ei hakka enam siin riigis sularahas palka saama, vaid kõik said 
omale pangaarved --- kõik pidid tegema ja raha hakkas minema panka. Kuigi ka siis oli algul palgapäeval ATMi 
juures saba, sest inimesed võtsid kogu raha välja. Funktsionaalsus ei läinud kohe käima. 

Seda analoogi tõime 
palju oma aruteludes ka põhjendusena, miks peaks ID-kaardi tegema 
universaalsena. Mart Laar\index[ppl]{Laar, Mart} lükkas seda hoogsalt tagant ja tihtipeale tuli lükata just nimelt neid, kes 
hästi hoolega raha lugesid --- Reformierakonda.

\question{Kas see lükkamine käis tal põhimõtteliselt sellesama arusaama pealt, 
et see on strateegiliselt oluline asi?}

Jah, tema uskus seda, Linnar\index[ppl]{Viik, Linnar} oli suutnud ta seda 
uskuma panna. Mina pidin selles protsessis olema kaasas sellepärast, et 
olin majandusnõunik (muidu oli peaministri büroos geeniseadus nagu rohkem minu laps). Minule ütleski rahandusminister, et unustage ära, sest te ei 
suuda mulle ROId näidata. Siis me mõtlesimegi välja selle, et \enquote{aga 
see asi tasus ennast ju kolme kuuga ära!}.

\question{Järelikult vastab tõele legend sellest, kuidas 
Linnar\index[ppl]{Viik, Linnar} ja Mart\index[ppl]{Laar, Mart} olla kord istunud laua 
taga ja Mart olla öelnud: \enquote{Linnar, mis me teeme?} ja Linnar olla 
vastanud: \enquote{Teeme interneti.}}

Vastab tõele küll, aga Mardil ei olnud nii kitsas vaade. Ta tahtis lihtsalt 
teada, et öelge kõik midagi, mida me võiksime teha.

\question{Huvitav kombinatsioon praktikast ja visioonist. Tollane 
Eesti Vabariik oli ikka oluliselt teistsugune kui praegu, muresid oli 
miljon!}

Oligi. Jään ka selle juurde, et me ei saa kogu seda au endale võtta. 
1999. aastal saime ju Euroopa Liidu teadus-arendusprogrammi liikmeks. Siis hakkas Euroopa Liit valmistama meid ette 
liitumiseks ja institutsioonide ning igasuguste muude asjade ehitamiseks tulid rahad peale. Väidan, et Euroopa Liidu rolli ei tohi alahinnata --- 
digiriiki on oluliselt lihtsam ehitada, kui keegi teine maksab koolide, teede ja muude asjade 
remondi kinni. Raha hakkas siin riigis liikuma palju rohkem ja tänu sellele oli ka võimalik 
teatud kõõl sellest digiteenuste arengusse suunata.

\question{Marek Tiitsu\index[ppl]{Tiits, Marek} 
IBS\index{Institute of Baltic Studies} ja igasugused muud asjad olid ju kõik 
välisfondide rahaga tehtud.}

Ja kui teised nägid, mida me teeme, siis tekkis üsna kiiresti laboriefekt: 
aitame, toetame ja saame ise ka näpud vahele sellele, mida nad seal teevad. Meist ei oleks
kindlasti saanud sellist e-riiki, kui meil ei oleks avanenud 
võimalust saada suures koguses välisabi. Vähehaaval oma toonasest SKTst (elasime tollal tegelikult ju ikkagi Maailmapanga vaeste riikide 
kriteeriumite järgi) me seda ei oleks teinud. Kuigi, mööngem, et 
täna ei saa sa ilmselt isegi ühte korralikku e-maksuametit selle raha eest, millega me 
esimesed kümme aastat oma e-riiki ehitasime.

\question{Kuidas see muutus operatiivtasandil käis? Minu käest on
näiteks Prantsuse ametnikud küsinud, et arusaadav, tegite e-riigi, aga 
kuidas te selle ametnikele ära seletasite?}

Jaa, seda küsis ka minu käest president 
Macroni esimene peaminister Édouard Philippe, kes oli otsustanud, et nüüd tuleb Prantsusmaal ka teha 
digipööre. Ta uuris, mis avaliku sektori
töökohtadest sai. Ütlesin talle, et vaata, Édouard, 
nüüd on nii, et meil läks maksuametis 60 protsenti töökohti kaotsi, aga meil 
ei olnudki üldse nii korralikku maksuametit nagu teil. Ära selle pärast 
muretse, te ju teete tööturu liberaliseerimise reformid ka ära, eks! Mida 
nad ongi teinud, ehkki muidugi mitte määrani, mida meie peame igal 
juhul normiks. 

Vastab tõele, et Eestis kadus ka töökohti, aga (see on nüüd 
teisest valdkonnast) kui Hoiu- ja Hansapank liitusid\sidenote{See toimus 
jaanuaris 1998.}, said Eesti ettevõtted endale finantsjuhid. Miks? Sest panganduses jäid üle sisulise poole spetsialistid, kes suutsid arvutada, 
ja ettevõtetel oli neid vaja kasvõi 
selleks, et nendesamade pankadega läbi rääkida. Neid polnud aga kuskilt võtta ja kui nüüd
kaks panka ühinesid, siis korraks tekkis selles valdkonnas tööjõu ülejääk 
ja hopsti! Täpselt samamoodi vajab erasektor e-riigi 
ehitamiseks kogu aeg inimesi, kes teavad, kuidas riigis protsessid käivad, ja minu 
arust on nad kogu aeg ise ära söönud sellesama tööjõu, mille nad on avalikus 
sektoris hävitanud.

\question{Hüve hüveks, aga miks ametnikud vastu ei hakanud töötama?}

Sellepärast ei hakanudki, et eestvedajad, kes võtavad 
juhtrolli, ei jää ju kunagi ilma tööta. Kindlasti oli kuskil ka neid, kes 
kannatasid ja kelle töökohad kadusidki. Eestvedajad ei 
muretse selle asja pärast, sa ise õpid selles protsessis nii palju. Ja 
väga paljud hüppasid teise paati koos erasektoriga --- nad osteti üle, et pakkuda riigile seda teenust tagasi.

\question{Meil ei olnud tol ajal see aparaat veel kivistunud, sa ei saanud 
olla olnud 15 aastat ametis, sest vabariiki polnud nii kaua eksisteerinud.}

Igal pool muutusid ju tegelikult noored meie riigi näoks. Ükskord kui olime Laariga Leedus visiidil, ütles üks Leedu erastamisagentuuri juht mulle, et te eestlased teete hästi julgeid asju, sellepärast 
et te olete kõik nii noored ja ei taju üldse, mis hirmsad riskid selle kõigega 
kaasnevad. Selles oli kindlasti oma iva. Sama lugu oli panganduses. Pärast Hoiu-Hansa ühinemist tuli siia üks 
ungarlane ja ma näitasin talle panga
ülemist korrust, mis oli täis oma nappi kolmekümmet eluaastat 
prilliraamide taha varjata püüdvaid tüüpe. Ta küsis mu käest: 
\enquote{Ütle, Kersti, mis te vanemate inimestega tegite?} Aga selle hind on 
see, et meie põlvkond pidaski üleval oma vanemaid ja kasvatas oma lapsi ning
tõenäoliselt jääb meil keskmise eluea osas mingi negatiivne hüpe 
sisse.

\question{Panga seltskond oli tol ajal jah nooruslik. Ja kui vaatame 
Laari, siis ta tänapäeva mõistes ajalooõpetaja hariduse pealt istus maha ja 
tegi maksureformi!}

Mardil oli meeletu usaldus oma nõunike vastu. Ma ei mäleta 1992.---1994. aasta 
perioodi\sidenote{Mart Laar\index[ppl]{Laar, Mart} oli peaminister aastatel 1992--1994 ja 
1999--2002.}, mina siis seal ei töötanud, aga ta lasi Ardo 
Hanssonil\index[ppl]{Hansson, Ardo} ilmselgelt otsustada ja möllata, nii nagu 
ka hiljem Linnar Viigil\index[ppl]{Viik, Linnar}, Kaarel 
Tarandil\index[ppl]{Tarand, Kaarel}, Simmu Tiigil\index[ppl]{Tiik, Simmu} 
või meil teistel. Saime vabad käed ja ta oli meie seljataga. Ütlesime lihtsalt: \enquote{Kuule, me tahaksime nüüd sellise asja ära 
teha.} 

Tol ajal oli igal ministeeriumil oma pank. Täna on meil 
ainult üks EAS ja KredEx --- mu arust on seegi risuks jalus, aga hea küll. 
Ja siseministeeriumis oli umbes kaks fondi, mis andsid raha 
välja. Ütlesin Mardile, et see on jube ebaefektiivne ja 
milleks need pangad üldse välja on mõeldud, kuna see raha 
seal ei kulu just kõige efektiivsemalt. Mart ütles kohe: \enquote{Tee ära, 
koristage need asjad ära.} Pärast hakkas eurorahasid liikuma ja siis oli 
EASi ka vaja. (Muide, minu magistritöö teema oli riigi asutatud 
sihtasutuste juhtimine ja see oli just nimelt seotud koondamise ja muu säärasega.) Mart ütles niisiis, et andke tuld, aga ministreid tuli ikka 
ise veenda, seda ei hakanud ta meie eest ära tegema. Käisime ja veensime. Padarit ei veennud ära ja Maaelu 
Arendamise Sihtasutus jäigi eraldi.\sidenote[][-1.7cm]{Ivari Padar\index[ppl]{Padar, Ivari} oli 
Mart Laari teises valitsuses aastatel 1999--2002 põllumajandusminister. Maaelu Edendamise 
Sihtasutus (MES) on tänaseni maaeluministeeriumi valitsemisalasse kuuluv 
sihtasutus.}

\question{See on täpselt selline kombinatsioon, et sul on 
oma strateegiline vaade, aga toimetama lased suhteliselt 
apoliitilise seltskonna, praktilised inimesed, kes saavad aru, 
mida on vaja teha.}

Meil oli eile Latitude'il\sidenote{Konverents Latitude59 toimus Kultuurikatlas 
19.-20. mail 2022.} arutelu. Minu vestluspartneriks oli prantslane, kes küsis 
kogu aeg, mida peab riik tegema selleks, et ka Euroopas oleksid toredad digifirmad nagu ameeriklastel. Lõpuks
ütlesin talle: \enquote{Me oleme siin nüüd pool tundi rääkinud täpselt sellest, 
kuidas Eesti riik ei sekku sellesse, millised majanduslikud valikud 
erasektoris tehakse ja millised sektorid peavad arenema. 
Dirižiste\sidenote{Dirižism, prantsuskeelsest sõnast \emph{diriger} (suunama) 
on majanduslik doktriin, milles riik mängib tugevat suunavat rolli 
kapitalistliku majanduse suhtes.} meie hulgas ei leidu. Tulemus: kümme
\emph{unicorn}'i ühe miljoni kohta. Kas see ütleb sulle midagi või ei ütle?} 

Umbes selline peabki minu arvates olema poliitikategemise roll: sa 
võimaldad asju teha. Samamoodi meie maksusüsteem --- mida tähendab ettevõtete tulumaksuvabastus? Igas investeeringus (rõhutan \emph{igas}, mitte valdkondlikult valitud 
eelisarendatavas valdkonnas) on riik ju 20 protsendiga sees ning võtab 
riski nagu ettevõtjagi. Võib-olla ei hakka sealt kunagi dividende tulema, mida 
saaks maksustada.

Meil on tegelikult kihvt riik!

\question{Ka mina olen selle raamatu koostamise ajal korduvalt 
läinud mõttes tagasi 1990ndatesse ja iga kord tunnen ennast 
hästi, kui tore riik meil on!}

Oled sa vahel mõelnud, mis oleks saanud, kui oleksime valinud ennast juhtima 
inimesed, kellel riigi ja oma rahakott lähevad segamini? Meil, muide, on ka 
praegu poliitikas selliseid inimesi päris palju, kes jäävad kogu aeg vahele 
sellega, et nad on oma võimupositsiooni kasutanud enda või partei 
hüvanguks. Väga vabalt oleks võinud niimoodi minna. Ja siis 
oleksime täna omadega Ukrainas.

\question{Meil õnnestus valida mingi ime läbi 
mõistlikud inimesed \ldots}

Tegelikult ei õnnestunud. Lennarti\index[ppl]{Meri, Lennart}\sidenote{Lennart 
Meri, Eesti president aastatel 1992--2001.} kõige vastuolulisem tegu 
põhiseaduse kontekstis oli see, et ta lasi Laaril moodustada valitsuse, kui 
Savisaar oli valimised võitnud. Aga kuna kolmikliit\sidenote{Kolme Eesti partei --- Eesti Reformierakonna, Isamaaliidu ja Mõõdukate ---
valimisliit.} oli eelmoodustatud, siis ta läks sellele teele. Kusjuures kolmikliit seisis ka nii habrastel alustel, et poleks üldse 
moodustunud, kui Edgar Savisaar\index[ppl]{Savisaar, Edgar} oleks saanud 
võimaluse mõne nendest potentsiaalsetest partneritest ära rääkida. Oleksime võinud 
minna märksa konservatiivsemat majandusarengut, märksa lõdvemat eelarvepoliitikat ja võib-olla ka märksa oligarhsemat majandusmudelit pidi, kui 
mõelda, mida Keskerakond tollal või ka täna endast kujutab.

\question{Ja tolles keskkonnas ei oleks ilmselt ka IT-kogukond saanud oma ideid realiseerida. Kui palju on Tarvi Martens 
ID-kaardiga, Küberi seltskond küberturbega ja teised saanud minna riigi 
juurde ja öelda: \enquote{Kuulge, see mõistlik asi, teeme!} ja neid on 
kuulatud!\nopagebreak[100000]}

Jah, ka Ukrainaga ei oleks pidanud nii halvast minema, nagu läks, sest
oligarhid võtsid majanduse päriselt vangi. Samas näiteks Sloveenia puhul, kes ühines Euroopa Liiduga ELi keskmise 
tulutasemega umbes 70, ma täna küll ei näe, et nad oleksid meist kuidagi 
paremad --- pigem on meie statistika parem. Nendel läks
majandus rohkem ettevõtete juhtide kätte, samas kui meil oli üksikuid 
sellisid ettevõtteid. Nad ei loonud väga palju uusi sidemeid uute turgudega, 
vaid jooksutasid majandust nii, nagu seda ikka oli jooksutatud, ja majandus 
restruktureerus palju aeglasemalt. 

Meil see Mart Laari tohutu valus 
1992.---1994. aasta periood, kui vana asi istuti katki, oli kohutavalt valus suurele 
osale töötajaskonnast. Paljudel inimestel on siiamaani rusikas taskus ja põhjusega, sest me ei osanud 
neid kõrvalefekte hallata. Meil ei olnud selleks ka raha ja 
uskusime, et tõusulaine tõstab kõiki paate.

Muide, seepärast ma tundsingi 2016. aastal, et nüüd on aeg teadvustada 
endale, et tõusulaine kõiki paate ei tõsta ja nõrgematele tuleb padi alla 
panna, kui me tahame heaoluühiskonda. Nii et kui jõudsin 20 aastat 
hiljem ringiga tagasi riigi tegemiste juurde, olles vahepeal igal pool mujal 
olnud, tekkiski võimalus asuda seda viga parandama ja ma loodan, et 
oleme õigel teel.

\question{Lõpetuseks lähme tagasi päris asjade 
alguse juurde. Miks on Eestis IT-kogukond, mis 
siiamaani toimib koos, aga näiteks lätlastel ei ole?}

Minu jaoks kannab ka meie IT --- tänapäeval isegi mitte ainult IT, sest kõik 
\emph{start-up}'id ei ole ju IT-sektori ettevõtted --- tol ajal tekkinud 
kultuuri, et võtame vastutuse riigi eest enda kätte. Kui võrrelda 
seda meie vana majanduse ettevõtetega ja paradigmaga, siis tähtsad vana majanduse 
ettevõtjad tulid peaministri nõuniku juurde ja ütlesid: \enquote{Me maksame nii palju makse, mida te meie jaoks teete?} 
IT-kogukond on aga alati olnud sellise suhtumisega, et riik ei saa seda 
teha, riik on selle jaoks liiga jäik ja paindumatu, \emph{fine}, me teeme ise! 
Teeme Jõhvi koodikooli või mille iganes! See suhtumine on alles jäänud ja ma 
olen hästi rahul. 

Mingisugune juurikas on kindlasti ka selles, et 
üheksakümnendatel lasti hästi palju teha ja tekkis positiivne tagasiside. 
Meie Pavlovi refleks on see, et saab küll. Paljud 
IT-ettevõtjad ütlevad küll, et alati kui nad on natuke aega ametnikega rääkinud, siis tahaks looteasendisse tõmbuda, aga seda vastu nina saamist ei ole ikkagi 
nii palju olnud, et lõplikult alla anda. Inimestel on 
usk, et saame tegelikult asjad tehtud, ja ka poliitikutel 
on lootus, et nende ametnikud ei ole ainult nagu tennisesein --- pall tuleb ja 
läheb kohe tagasi, vaid kuskilt peab see ka läbi minema. Isegi kui 
läbi läheb üks sajast, on see päris hea tulemus.

\question{Sest meil on kogemus, et on ju saanud ja
toiminud!}

Just. Sellepärast on ka meie \emph{start-up}-kogukond sotsiaalselt väga
vastutustundlik. Kui võtta vahelt ära parlament (mõnes
mõttes ongi võetud, sest meie parlament ei ole täna tegelikult mõttekoda, 
mida ta võiks olla) ja lasta neil ilma keskse organiseerimiseta, difuusselt 
seda riiki ajada, siis väga palju hullem see ei saaks.

\question{Siinkohal ongi ehk mõistlik lõpetada tõdemusega, et päris hea 
riik on saanud!}

On. Aga ärme riku seda ära! Mida kõrgemal tulutasemel oled, seda suuremad on 
riskid midagi teha ja muuta. Peame suutma kogu aeg 
uueneda ja edasi minna. Näen täna, et oleme 
proovinud tekitada lubavat seadusruumi uutele tehnoloogiatele, aga tegelikult 
hästi ei õnnestu. Ja see kõlab nüüd õudselt ebapopulaarselt, aga meie parlamendi palgad 
peaksid olema palju paremad selleks, et parlament töötaks mõttekojana, 
mis viiks ka tehnoloogilist poolt edasi. Ta peab uuesti hakkama tõmbama ligi ka 
akadeemilist ja \emph{start-up}-kogukonda, mida ta täna ei 
kõneta. Midagi ei ole teha --- kahjuks on nii, et mida maksad, seda saad.

\chapter{Andres Kütt}
%!TEX TS-program = arara
% arara: myindex

\textbf{\enquote{Kuidas sa arvutite juurde jõudsid?}}

Sündisin 1975. aastal Võrus\index{Võru}. Millestki midagi aru saama hakkasin mälu järgi kaheksakümnendate teisel poolel. See oli mitmes mõttes üsna kole aeg. Noorukile kõige arusaadavam neist koledustest oli lihtlabane praktiline puudus. Päris nälga ei olnud aga midagi vähegi leivast ja piimast edevamat saada ei olnud. Kui linnakeses levis kuuldus, et olla toodud kast jäätist, oli poes veerand tunniga saba ning poole tunni pärast kõik otsas. Muu hulgas oli kaubandusvõrgus saada kahte tüüpi meeste talvejopesid. Mitte kahtekümmet ja mitte kahtesadat vaid kahte. Ühed olid hallid ja neid said lihtsurelikud osta\sidenote{Huvitaval kombel oli tolle jope põuetasku 5.25" lai, sinna mahtus üks flopi täpselt sisse} ja teised olid punase suure a-tähega ja neid said osta ainult inimesed, kes teadsid kedagi, kes teadis kedagi. Ajad olid sellised. Kõige hämmastamaval moel käisid ka seda viletsust inimesed Pihkvast bussidega uudistamas ja viimastki kaupa ära ostmas. 

Aga kogu selle halluse keskel suutis Nõukogude Liit meie Võru Kreutzwaldi Gümnaasiumile\index{Koolid!Võru Kreutzwaldi Gümnaasium} tarnida arvutiklassitäie arvuteid Agat\index{Arvutid!Agat}\sidenote{Agat oli Nõukogudemaal valmistatud arvuti, mis oli küll Apple II\index{Arvutid!Apple II}'st inspireeritud, kuid siiski mitte täpne kloon}. Kust nad tulid ja kes seda asja ajas, ei tea. Küll aga mäletan, et nende saabumine oli pikalt oodatud ja edasi lükatud. Miks ja mida oodatud sai, ei oska öelda. Tean ainult seda, et kui klass tekkis, läksin ma sinna sisse ja enam välja ei tulnud. 

Ega tolle purgiga palju teha ei olnud. Olid mõned mängud ja programmeerimiseks Basic. Tolles meid programmeerima õpetatigi. Esimese hooga ei õpetatud seejuures mitte kõiki käske, näiteks for-tsükkel oli tükk aega saladus. Kui aga nohikud aru said, et nende eest tarkust varjatakse, kadus igasugune respekt ja läks lahti suuremaks isepusimiseks. Kõik muutus, kui kooli saabus noor, minu meelest värskelt ülikoolist tulnud, arvutiõpetaja Aivar Halapuu\index[ppl]{Halapuu, Aivar}. Temaga tekkis kohe mingisugune pool-kamraadlik side, mis siiski alati suurt kogust meiepoolest lugupidamist sisaldas. Tolleks ajaks oli meil tekkinud väiksem seltskond poisse, kes seal klassis toimetas ja kes kohe end \emph{in corpore} Aivarile sappa haakis. Aivar viitsis meiega tegeleda ja, kuigi ta meile suurt midagi arvutite mõttes ei õpetanud, sai tema käest midagi, mida vist kultuuriks nimetatakse. Meiega üritati bridži mängida, räägiti mänguteooriast ja nii. 

Kuna me seal klassis sisuliselt elasime, siis usaldati meie kätte üsna pea ka arvutiklassi võti. Aga \emph{kooli} võtit meie kätte keegi ei andnud. Seetõttu oli oluline hoida järjepidevust: keegi oli alati klassis olemas ja hõikamise või kivikese viske peale lasi tulija sisse. Mingitel tingimustel oli meie käes siiski ka välisukse võti aga tihti roniti ka aknast. 

Ühel hetkel avanesid kraanid ja saabus humantiaarabi. Võrus oli vist seoses rahvamuusikaga igasugu põnevaid suhteid välismaa asutustega, kes hakkasid meie suunas igasugu põnevat kola saatma. Saabus klassitäis mingeid rootsikeelsete paberite ja tarkvaraga masinaid, millega me mitte midagi teha ei osanud. Mis neist sai, ei tea. Aga tuli ka mingi iidne aparaat, mille külge käis neli-viis terminali ja kaks kokku külmkapisuurust kettaseadet. Seadmete sisse käisid hiigelsuured plastkarbis kettad. Tegu oli industriaalseadmega: kui tuurid sisse võttis, siis oli alla tänavale kuulda, et \enquote{arvuti töötab}. Tolle masina peal midagi tarka teha ei osanud keegi, tarkvara polnud. Sai mingeid mänge mängitud ja see oli ka kõik. Mäletan siiski, et seal puutusin esimest korda kokku Zorki\index{Mängud!Zork} nimelise mänguga\sidenote{Zork on üks varasemaid teksipõhiseid arvutimänge. Mängija sisestas teksti ja talle ka vastati tekstiga vastavalt sellele, mis mängus parasjagu juhtus. Kuna mängu alguses sattuti lagendikule valge maja ette, oli meie puhul ilmselt tegemist Zork I-ga}.

Lõpuks tulid meile Jukud\index{Arvutid!Juku} ja üheksakümnendatel lõpuks ka pc-d. Jukusid oodati väga, sest Agat oli päris jube aparaat\sidenote{Ma ei ole kunagi hiljem kohanud arvutit, mis suudab flopiketta füüsiliselt ära rikkuda}. Ja Jukud olidki väga ägedad, ainsaks nõrgaks kohaks oli minu mälu järgi klaviatuur. Ainus, mis palju ei muutunud, oli tarkvara. Võru ei ole Tartu ega Tallinn. Meie seltskond ei suhelnud õieti kellegagi, ei uut tarkvara ega teadmist ei tulud eriti kuskilt peale. Ajakirjast \enquote{Arvutustehnika \& Andmetöötlus} võis küll lugeda Unicode võludest aga programmeerida tuli ikkagi kas assembleris või basicus. Seejuures sain alles hiljem teada, et eksisteeris ka asi nimega makro-assembler. Tavalises pidi JMP käsule argumendiks andma suhtelise aadressi (mis muidugi kohe valeks osutus, kui kuskile mingi rea vahele panid)\sidenote{See oli probleemi minusugustele surelikele. Inimesed nagu klassivend Vallo Trell\index[ppl]{Trell, Vallo} suutsid ka otse BIOSi prompti peal mällu baite kirjutades masinkoodis programmeerida} aga tolles uuemas sai silte kasutada. Mingitel üritustel sai Tallinnas käidud (mäletan Pedas\index{Pedagoogikaülikool} asunud MSXide\index{Arvutid!Yamaha MSX} klassi) ja sealt ka mingit tarkvara kaasa toodud aga üldiselt olime üsna omaette. Isegi flopisid käisime ostmas Tallinnas, seal oli teada üks komisjonipood, kust selliseid sai. Tavaliselt kasutati ära mõnda käiku teatrisse, reeglina jäi kuhugi paar tundi linnas kolamise aega. 

Olin ka üks õnnelikest, kellele lõpuks arvuti suveks koju usaldati. Esmalt Agat, siis Juku. Kuna ekraanid olid mõlemal nigelad, veetsin kaks või kolm suve ette tõmmatud kardinate taga arvutiga toimetades. Juku peal mäletan kahte suuremat projekti. Esimene oli Norton Commanderi moodi failihaldur ja teine fondiredaktor. Jukul sai tähekujusid suhteliselt lihtsasti ümber teha, mälus olid vist kaheksabaidised bitimaatriksid ning teksti kuvamine käis kiiremini, kui muu graafika. Mõlemat kirjutasin assembleris ja kumbki päris valmis ei saanudki, sest teatud mahust alates muutus kood hoomamatuks. Sel ajal omandasin ka pärast palju vaeva põhjustanud kombe \enquote{tunde järgi} koodi kirjutada. Teed muutuse, kompileerid, proovid, muudad pikalt mõtlemata uuesti. Kood oli nii kole, et selle iga kord uuesti läbi mõtlemine oli liiga keeruline ja mingid \emph{off-by-one} vead olid sagedased, reeglina sai mingi konstandi ühe võrra nihutamise peale koodi käima. Sellest rumalast kombest pole ma siiani lõpuni vabanenud. 

Aga Juku peal sai ka andmebaase teha, täitsa oli olemas dBASE\index{dBASE}. Selle abil õnnestus maik suhu saada kellelegi arvuti abil kasulik olemisest. Koolivend Aini dieedi-teemalise uurimistöö jaoks tegin andmestiku ja kirjutasin ka programmi kassetiümbriste trükkimiseks. Tollal käibis muusika kassettidel, mida ohtralt kopeeriti\sidenote{Eksisteeris ka tänapäeval mõeldamatu täiesti põrandapealne muusikakopeerimise asutus, selline oli ka Tartus. Läksid kohale, valisid kataloogist albumi välja, jätsid tühja kasseti maha ja mõni päev hiljem sai sobiva summa vastu muusikaga kasseti tagasi}. Seetõttu kirjutati lugude nimesid käsitsi ning see oli tüütu. Minu tarkvara võimaldas aga kiiresti eri plaatide jaoks kassetiümbrised trükkida. Selle teenuse eest sai vist ühelt klassivennalt isegi raha küsitud.

Linna peal eri kohtades sai ka PCdega tutvust teha. Mööblivabrikus oli kellelgi tutvusi, seal toimus isegi mõned korrad mingisugune õpe. Istusime ilmselt raamatupidamise masinate taga ja meile näidati, kuidas FoxPros\index{FoxPro} vorme joonistada ja andmeid hoida. 

Keskkoolis õnnestus käia väga murdelistel aastatel 1990-1993. Võrus möllas punkar Saare Ain\index[ppl]{Saar, Ain}\sidenote{Kodanikunimega Ain Saar, asutas Vaba Sõltumatu Noortekolonni number 1 ja tegi muid tükke}, Võru surnuaial taastati Vabadussõja mälestussammas ja miilits ajas koertega üritusi laiali. Ühe sellise intsidendi järel oli koolis näha kummalistes ülikondades seltsimehi, kes pingsalt vanemate klasside õpilaste nägusid jälgisid ilmses lootuses tuttavaid kohata. Aga tekkis ka äri. Leidsime sõpradega mingist ajalehest kuulutuse, milles otsiti meie jaoks ulmeliste palkadega (mahus umbes meie vanemate aasta palk paarinädalase projekti eest) meelitades C programmeerijaid. Kandideerimise tähtaeg oli suurusjärgus kaks nädalat, see tundus täiesti mõistlik aeg, millega omale C selgeks teha. Kuskilt sai hangitud klassikaline Brian Kernighan ja Dennis Ritchie \enquote{The C Programming Language}\sidenote{Paraku läks mu koopia hiljem kaotsi. Kust ma selle raamatu sain, ei oska öelda, aga kindlasti ei tulnud see kuskilt välismaalt. Ilmselt oli tegu mingi kvaliteetse piraat-väljaandega, millel isegi kaanekujundus õige oli. Hiljem järele uurides selgub näiteks, et raamatus puudus igasugune märge trükkimise koha ja väljaandja kohta}. Seda sai siis kampas tudeeritud ja tundus sihuke loogiline. Kuna puudus juurdepääs C kompilaatorile, siis päris koodi kirjutada ei saanud. See meid ei heidutanud ja mingid kirjad me isegi välja saatsime. Vastust muidugi ei tulnud. Hiljem olen mõelnud, kas võis tegu olla tollesama legendaarse lehekuulutusega, mis viis kokku Bluemooni\index{Bluemoon} poisid ja Stefan Obergi\index[ppl]{Oberg, Stefan} aga ajastus vist ei klapi. 

Siiski saavad kõik head asjad otsa, nii ka keskkool. Tol hetkel sai mingites piirides omale lõpueksamit valida ning oleks olnud kummaline, kui meie seltskond ei oleks valinud arvutieksamit. Tolleks hetkeks olime Aivarist kaugel ees, sest meil sõna tõsises mõttes ei olnud mitte midagi muud teha, kui arvutit torkida. Laulsin kül ka kooris\sidenote{Kooriga välisreisile (kas Saksamaale või Soome) minek oli ka põhjuseks, miks ma ei ole kunagi vabariiklikul informaatikaolümpiaadil käinud. Tol ühel kevadel, kui sinna õnnestus välja murda, oli ka reis plaanis. Otsustavaks sai, et ma ei tahtnud koori hätta jätta. Mitte, et ma seal mingit kandvat rolli oleksin mänginud, aga siiski.} aga põhimõtteliselt kogu muu vaba aeg oli arvutite päralt. Isegi õppetöö ei seganud, sest põhikoolis tegin endale kõva põhja alla. Aga see kõik ei vähendanud sugugi eksami pidulikkust. Sisenesime ruumi, võtsime pileti, lahendasime, vastasime komisjonile, kõik oli nii nagu peab. Aivar oleks võinud meile kõigile viied välja kirjutada aga ometi viidi eksam täie tõsidusega läbi. 

Kuna õnnestus kool nibin-nabin kullaga lõpetada, sain Tartu Ülikooli Matemaatikateaduskonda\index{Tartu Ülikool!Matemaatikateaduskond} eksamiteta sisse. Sinna minek tundus loogiline, sest Tallinn oli kaugel ja tundmata ning arvuti-värki tahtsin kindlasti õppida. Sõjaväega probleeme ei olnud. Esiteks olid segased ajad ning Eesti riik polnud veel päriselt välja mõelnud, mis moodi oleks mõistlik väeteenistust korraldada. Teiseks oli mu silmanägemine nii paha, et mulle öeldi Kaitseväe tohtrite poolt: \enquote{Kui venelane peale tuleb, siis paneme su laipu vedama, seniks mine koju}. Nii veetsingi suve Võru ja Tartu vahel hääletades, käisin näiteks ka Steni\index[ppl]{Tamkivi, Sten} juures\sidenote{Tema ema ja minu tädi olid juba ülikooli aegsed sõbrannad, Steni vanaisa elas Võrus ja nii me juba üsna õrnas eas tuttavaks saimegi.} Primexis\index{Primex Data} külas. 

Sügisest algas ülikool ja jäin pidevalt Tartusse. Kuna jäin paberite ajamisega töllerdama, siis teiste matemaatikutega Tiigi ühikasse kohta saada ei õnnestunud. Ühe või kaks talve olin sugulase juures üüriliseks, ühe talve elasime kambaga Tartu Kurtide Ühingus (!)\index{Tartu Kurtide Ühing}, kes tudengitele tuba välja üüris. Küll aga sai külas käidud klassivendadel, kes läksid enamuses Tartusse majandust õppima, ja kelle ühikaks olid Narva Maantee Tornid. Toona Tartu ühikates toimunu on omaette lugu, millesse süvenemine viiks meid teemast kõrvale.

Ülikoolis sain kohe piltlikult öeldes ägeda laksu silmade vahele. Esmalt selgus, et, erinevalt keskkoolist, on ülikoolis vaja päriselt õppida. Aga oskus selleks oli juba kadunud ja tuli uuesti tekitada. Teiseks selgus, et puhtast ropust tööst enam heade hinnete saamiseks ei piisanud, vaja oli ka annet. Aga seda on mul kogu aeg nappinud. Kolmandaks selgus, et teistel seda annet jagus ning see tegi egole haiget. Inimesed nagu Meelis Roos\index[ppl]{Roos, Meelis} ja Rene Prillop\index[ppl]{Prillop, Rene} seilasid igasugu matemaatikast läbi ilma nähtava pingutuseta ja kirjutasid koodi, nagu jumalad. Margus Sutt\index[ppl]{Sutt, Margus} teadis arvutitest nähtavasti kõike ja oli tolleks ajaks juba tegelenud täiesti müstilisena tunduvate asjadega. Asko Seeba\index[ppl]{Seeba, Asko}, oli kõike seda \emph{ja} oli seejuures veel setskondlik ning tüdrukute hulgas popp. Ei jäänud midagi üle, tuli tasapisi inimeseks õppima hakata. 

Igatahes oli vaja tööle minna, sest ema käest ei saanud ju jääda raha küsima. Proovisin saada baarmaniks, vast avatud Atlantise ööklubi valgustajaks ja isegi arvutigraafikuks aga asjata. Lõpuks sattusin kuidagi ettevõttesse Korel IN\index{Korel IN} programmeerijaks, mu esimene tööpäev oli detsembri alguses aastal 1993. Mind ja kamraad Veljot\index[ppl]{Hagu, Veljo} võeti palgale eesmärgiga luua firmale arvetega majandamiseks vajalik tarkvara. Keeleks oli Visual Basic\index{Keeled!Visual Basic} ja ei läinud palju aega, kui meil mingid asjad juba töötasid. \enquote{Programmeerija} kõlab märkimisväärselt galmuursemalt, kui asi tegelikult välja nägi. Tegime kõike alates kauba tassimisest (kontor asus viiendal või kuuendal korrusel, kahekümnetolline CRT-monitor on päris raske), kuni isegi mõningase müügitööni. Toonasele arvutiärile iseloomulikult ei teadnud eales, mis seisus su töökoht kontorisse jõudes oli. Mõnikord oli ära müüdud mälu, mõnikord võrgu kaart või monitor. Mäletan end kirjutamas koodi üheksatollise must-valge kassamonitori ees taburetil istudes. 

Tartu ei ole suur linn ja nii puutusime Korelis töötades kokku suure osaga toonasest arvutiseltskonnast. Tarmo Tali\index[ppl]{Tali, Tarmo} oli meil müügimeheks ja aeg-ajalt käis tal külas Asko Oja\index[ppl]{Oja, Asko}, keda hellitavalt \enquote{Tarmo blondiiniks} kutsuti. Vahel astus Sorose sajalisi tuulutades läbi Marek Tiits\index[ppl]{Tiits, Marek}, kellele mingi ime läbi õnnestus isegi üks Suni tööjaam müüa. Kui ütlen, et puutusime, siis tegelikult mina ei puutunud eriti kellegagi kokku, olin toona ja olen siiani küllalt asotsiaalne. Igasugu toredat rahvast käis poest läbi, enamasti sai lihtsalt silmad punnis peas spetsialistide jutte kuulatud ilma nende nimesidki teadmata. 

Kuidagi tekkis Koreli lähedale aktiivne kodanik nimega Tanel Urbanik\index[ppl]{Urbanik, Tanel}. Ta pandi meile alguses ülemuseks aga üsna varsti vedas ta meid Korelist minema asutades uue ettevõtmise nimega HClub. Nimi tuli sellest, et meie tuba Koreli päris-ärimeeste hulgas veidi põlastavalt häkkeriklubiks kutsuti. Tanel tahtis tarkvaraäri teha, küllap seetõttu tal Koreliga teed lahku läksidki. Meie peamiseks leivanumbriks sai kassasüsteemide ehitamine, peamisteks klientideks erinevad tanklad, näiteks Favora omad. Kirjutasin muu hulgas ka näiteks Ravimiametile\index{Ravimiamet} nende ühe esimestest andmebaasidest. Selguse mõttes olgu üle korratud, et toona mingist klient-server arhitektuurist juttu ei olnud. Kõik lahendused hoidsid andmeid võrguketta peal Microsoft Accessi\index{Microsoft Access} andmebaasis ja selle poole pöördumine käis kliendi juurde paigaldatud \enquote{paksu} kliendi abil. 

Tollele ajale tagasi mõeldes tundub hämmastav, et meie tarkvara töötas. Meid olid ainult mõned inimesed, mingist testimisest või versioneerimisest ei teadnud keegi midagi. Mäletan, et korra pidin Tartust Võrru tanklasse tagasi sõitma, sest värsket versiooni flopi peal kohale viies olin midagi valesti teinud. Vähemalt minu kood püsis kindlasti koos peamiselt tati ja teibiga. Veljo oli märkimisväärselt pädevam programmeerija aga tarkvaratehnikast polnud ilmselt palju aimu temalgi. 

See mind lõpuks HClubist (päris suure tüliga, tuleb tunnistada) ära viiski. Ma ei jaksanud enam kõige selle kokku punutud ja päris kliente teenindava tarkvara eest vastutada. Põlesin läbi ja kõndisin Tanelit pipramaale saates ära. Toonaseid seiku nägin veel aastaid unes ja ärkasin keset ööd. Oma rolli mängis ilmselt ka see, et just tol ajal, kui õigesti mäletan, läksid põhja mu unistused saada arvutialane haridus. Nimelt oli toona matemaatikateaduskonnas esimesed paar aastat kõigile ühised, seejärel tuli valida kas arvutiteaduse, statistika või rakendusmatemaatika vahel. Valik käis seejuures õpitulemuste alusel. Minu õpitulemused võimaldasid napilt ennast arvutiteadlaseks pidada ja nii esitasin vajaliku avalduse ning asusin järgmisest semestrist hoogsalt arvutiteaduse aineid kuulama. Neid loeti enamasti Liivi tänava õppehoones. Dekanaat oma teadetetahvliga asus aga Vanemuise õppehoones. Ja kuna ma ka oma ut.ee meiliaadressi ei jälginud, läks minust täiesti mööda dekanaadi mõte, et peaks ikka veel mingeid pabereid küsima. Kui ma ükskord jaole sain, olid arvutiteaduse õppekohad täis ja minust sai statistikaüliõpilane. See oli päris valus hoop. Kuigi arvutiteaduse ained olid minu jaoks rasked (mäletan end kolm korda kompileerimismeetodite eksamit tegemas), oli mul siiski mingi lootus sealtkaudu kuidagi paremaks programmeerijaks saada ning kamraadidele järgi jõuda. Toonane ülikooliharidus oli tänasest väga erinev ja asus praktilisest elust valgusaastate kaugusel, aga lootus jäi. Statistikast huvitusin ma vähe ja ei näinud mingit võimalust sellest oma töises elus kasu saada (masinõppe-revolutsioonini jäi veel paarkümmend aastat). Seetõttu tegin edaspidi minimaalse, et kuidagi koolist läbi saada ja keskendusin tööl käimisele. 

Kogu BBSindus läks minust üsna suure kaarega mööda. Võrus ei olnud kohalikku BBSi ja kaugekõne ei tulnud kõne allagi. Sten Primexis küll vist näitas kuhugi helistamist, aga tuhka ma aru sain. Korelis oli küll väline modem ja aegajalt sai kuhugi sisse helistatud, aga seda väga sporaadiliselt. Peamiseks selleteemalise info allikaks oli kursavend Mati Muts\index[ppl]{Muts, Mati}, peamiselt sai käidud Lucifer BBSis\index{BBS!Lucifer BBS}. Küll aga oli ülikoolil tol ajal juba täiesti korralik internetiühendus ja palju aega kulus Vanemuise õppehoones\index{Tartu Ülikool!Vanemusise tänava õppehoone} terminali taga FTPd pidi ringi kolades. Mäletan, et tõmbasin kas ftp.funet.fi või ftp.sunet.se serverist tükk aega mingi Metallica albumi kaanepilti ja olin väga rahul, kui see ka päriselt kohale jõudis. 

Selgelt mäletan ka seda, kuidas ma kohtusin HTMLiga. See oli Liivi tänaval\index{Tartu Ülikool!Liivi õppehoone}, seal oli mingi Suni klass\sidenote{Need pidid olema Sunid, sest mäletan ruudulist hiirepatja. Mis muidugi ei olnud mingi padi. Kuna Sun kasutas toona levinud palli asemel hiire liikumise lugemiseks eesrindlikke optilisi sensoreid aga tehnoloogia polnud veel kuigi arenenud, pidi sensoritele teadaolevate vahedega ruudustikku näitama. Seetõttu töötas hiir ainult spetsiaalse metallist mati peal, kuhu oli joonistatud peen ruudustik} ning seal sukeldusin ma veebilehe tegemise võrratusse maailma. Pärast pikka pusimist suutsin omale tekitada kodulehe, kus asju õiges kohas hoidis tabel! Ega sinna kodulehele midagi kirjutada ei olnud aga tabeli ridade ja lahtrite saladuste lahti pusimine oli põnev.

Ja kõik see osutus kasulikuks, sest HClubi järel võttis mu oma juurde tehnikuks klassivend Meelis Mäeots\index[ppl]{Mäeots, Meelis}. Ta tegeles tol ajal igasugu imelike asjadega, kuid muu hulgas asutas ka internetifirma. See koosnes alguses peamiselt minust ja temast. Firma tegeles Unineti\index{Uninet} \emph{dial-up} ühenduste edasi müümisega, tegi kodulehekülgi ja pidas isegi Infomeistri nimelist interneti infokataloogi. See viimane oli täiesti hämmastav äri. Meelis käis ja rääkis mingitele firmadele augu pähe. Mina kirjutasin firma andmed kuskil serveris asunud staatilisse (!) HTMLi. Mis kasu sellest kellelegi ammu enne otsingumootorite laia levikut tõusta võis, on mulle siiani arusaamatu. Ma ka ei mäleta, et seal lehel keegi väga käinud oleks. Ometi sealt mingi kopika sai ja ma väga loodan, et tolle tegevuse käigus antud lubadused ikka enam-vähem täidetud said. 

Kuna teadsin Steni juba varasemast ja Meelis vist ka puutus temaga kokku, lõpetasime ühel hetkel modemitega jantimise ja infokataloogi pidamise ning asusime Halo\index{Halo Interactive DDB} nime all kodulehekülgi tegema. Kampa võeti ka mõned kunstnikud (näiteks väga andekas Oliver Reitalu\index[ppl]{Reitalu, Oliver} ja mitte vähem andekas Alar Koort\index[ppl]{Koort, Alar}, keda ilmselt tema rajude elukommete tõttu Helbekeseks kutsuti) ja projektijuhiks Priit Sasi\index[ppl]{Sasi, Priit}, keda kõik tema joviaalse oleku ja suure habeme tõttu Sasuks kutsusid. Sasu õpetas mind briti punki ja Alar kurjemat sorti hiphoppi kuulama ja elu oli päris tore. Miskipärast mäletan, et minu käe alt tuli Eesti esimene kommertsalustel tehtud (st. ettevõte maksis kellelegi lehe tegemise eest raha) kodulehekülg, see sai tehtud Tartu Raadiole\index{Tartu Raadio}, kui mälu ei peta. Kunstnik joonistas pildid valmis ja lõikas tükkideks, mina kirjutasin Notepadiga HTMLi ja nii see töö käis. 

Mingil hetkel hakkasime lehekülgede tekitamist automatiseerima, kirjutasime Perli skripte. Mõnda aega ei olnud meil ei oma serverit ega üldse kuskil Perli jooksutada. Siis sai programmeeritud nii, et skript läks e-mailiga Unineti\index{Uninet} süsadminile, see kopeeris faili õigesse kohta, meie vajutasime brauseris nuppu, saime veateate, admin saatis e-mailiga konsooli veateated, mina parandasin koodi ja saatsin uue versiooni. Admini kannatus lõppes enne, kui minu oma. 

Siiski jõudsime lõpuks päris kaugele oma tegemistega. Perli skriptid läksid järjest pikemaks ja, kuna andmebaasi pidamiseks ei olnud meil serverites piisavalt õigusi, hoiti andmeid enamasti lihtsalt tekstifailis. Üllataval moel kattis see ära päris suure hulga vajadusi. Perlilt liikusime ühel hetkel PHPle ja ühel hetkel tekkis ka levinud kui seetõttu mitte vähem rumal mõte endale ise oma sisuhaldussüsteem kirjutada. See vist sai isegi valmis aga konkreetsed mälestused tollest elukast puuduvad. 

Ma ei mäleta, et see äri kuidagi tänapäevases mõistes äri moodi välja oleks näinud. Raha oli alati vähe ja nii tuli teha kõike, mille eest maksti. Kuidagi müüs Sten Ühispangale maha mõtte anda nende aastaraamat välja CDl. Mis muud, õppisime selgeks Macromedia Director'i kasutamise ja video redigeerimise ja andsime minna. Ainus asi, millega hakkama ei saanud, oli heli. Õnneks oli Sten hea sõber Lauri Liivakuga\index[ppl]{Liivak, Lauri}, kelle Forwards Studio\index{Forwards Studio} asus meiega tol ajal sama koridori peal. Lauri tegi kenad kõllid ja plõnnid ja aitas selle kõik visuaaliga ära sünkroniseerida. Tulemus sai päris kena. 

Igatahes hakkas meile järjest rohkem Tallinna kliente sigima. Samuti müüs Sten suure tüki ettevõttest Brand Sellers DDBle\index{Brand Sellers DDB}. Too oli minusugusele Tartu nohikule täiesti müstiline kamp inimesi. Intelligentsed, säravad, jõukad (nii mulle tundus) ning andekad. Bruno Lill\index[ppl]{Lill, Bruno} oma terava ütlemisega on siiani meeles.  Nii tehti kampas otsus kolida Tallinna. Olin tegelikult ligi aasta üsna kahepaikne pendeldades Tartu ja Tallinna vahel. Ülikoolis olid veel viimased sabad lõpetada ja Mari\index[ppl]{Kütt, Maria}, kellega toona juba koos elasime, käis samuti veel koolis. Lõpuks sai lõputöö kaitstud ja, kuna selliseks triviaalseks asjaks ei hakanud ju keegi Tartusse sõitma, käis Mari diplomit dekanaadist ära toomas. Prouad nõudsid allkirjastatud volitust, mis ukse taga ka kohe valmis tehtud sai ning nii omandasin ma oma esimese teaduskraadi. Tartu Ülikooli peahoone sammaste vahelt ei ole ma kunagi välja astunud ja, kuigi toonaseid õppejõude hindan siiani kõrgelt, pean oma alma materiks siiski Massachusettsi Tehnoloogiainstituuti. 

Tallinnasse kolimisega sai läbi üks etapp Halo kasvu loost. Senise boheemliku mis-võib-ikka-valesti-minna mentaliteedi asemel tuli hakata käibenumbritest rääkima. Samuti oli meeskond kasvanud. Veel Tartu päevadel olin saanud omale oma elu esimese alluva olles samal ajal ka tema esimeseks ülemuseks. Vist veel keskkooli lõpetav noor nutikas tüüp aitas mul koodi kirjutada ja hängis niisama ringi, ei mina teadnud, kuidas inimesi juhitakse või mida üks ülemus tegema peaks. Nimeks oli tüübil Taavet Hinrikus\index[ppl]{Hinrikus, Taavet}. Inimesi lisandus veelgi ja ma ei saanud enam aru, miks ja kuidas asju tehakse. Nii leidsingi ühel ilusal päeval kuskilt kuulutuse, et Hansapank\index{Hansapank} otsib oma internetipanga meeskonda inimesi. Läksin intervjuule. Mäletan siiani seda tunnet, kui Liivalaia tänava pangahoone tolle aja kohta ülišiki lifti uksed kaheksandal korrusel avanesid ja minu ees avanes hurmav vaade vanalinnale. Olin müüdud mees, õnneks arvas Vilve Vene\index[ppl]{Vene, Vilve}, kes toona arendust vedas, samuti. Nii sai minust veidi enne sajandivahetust Hansapankur. Mul vedas kohutavalt, pank oli praeguses mõistes ulmeliselt dünaamiline asutus. Vägesid juhatas Indrek Neivelt\index[ppl]{Neivelt, Indrek}. Vaata Maailma programm oli just käima minemas ja sellega tegeles Tiit Pekk\index[ppl]{Pekk, Tiit}. Marketsi tiim eesotsas Erkki Raasukesega\index[ppl]{Raasuke, Erkki} pidas ülejäänud panka talumatuteks venivillemiteks ja tootis Erik Jõgi\index[ppl]{Jõgi, Erik} juhtimisel imeilusat koodi. Aga see, nagu öeldakse, on juba üks teine jutt.

\chapter{Jaanus Lillenberg}
\index[ppl]{Lillenberg, Jaanus}
\question{Kuidas sina said arvutite juurde ja arvutid sinu juurde?}

See sai alguse aastal 1983, kui Tartu 
Ülikoolis\index{Tartu Ülikool} tehti Nõukogude Liidu ja Jaapani koostöö 
tulemusena personaalarvutite klass.

\question{Mis arvutid need olid?}

Need olid Yamaha MSXid\index{Yamaha MSX}. Yamaha MSX kuulub samasse põlvkonda, mis Commodore 64, mõned vihasemad Sinclairid ja 
ka Apple II. Äge oli see, et nendes arvutites jooksis 
tegelikult Microsofti tava-kasutajatele mõeldud operatsioonisüsteem.

\question{Kas see oli Microsofti oma?}

MSX nagu Microsoft \emph{Extended} vist\sidenote{Lühendi päritolu kohta liigub mitmeid variante, ka asja juures olnud inimesed ei mäleta enam täpselt.}. Igatahes nägi see äge välja. Arvutiklass paiknes kooli peauksest kümne sammu kaugusel f
keldris, mille aken avanes täpselt kooliukse ette. Ühele üheteistaastasele, kes läks sellest igal hommikul ja õhtul mööda, 
oli see vastupandamatu. Selles mõttes valikuvõimalusi tegelikult 
ei jäetud. 

\question{Sa lihtsalt pidid sealt uksest sisse minema?}

Kõndisin ühel päeval otse aknast sisse, sest 
aken oli tänavaga samal tasapinnal. Küsisin, kas võib tulla vaatama, ja ära mind otseselt ei aetud. Kolmandal päeval 
andis keegi mulle MSX BASICu\index{BASIC!MSX BASIC} 
manuaali koopia. Ma küll ei saanud 
inglise keelest aru, aga mängude tegemine tundus huvitav. 
Arvutiklassis toimetasin kolm-neli aastat ja olin vahepeal ka abiõpetaja. Kirjutasin ise tekstiredaktoreid ja mänge ning loomulikult häkkisin 
lõputus koguses olemasolevaid mänge. Kirjutasin ka oma elu esimese viiruse, mis hävitas flopiketta. 

\question{Mis koolis sa käisid?}

Tartu 10. Keskkoolis\index{Tartu 10. Keskkool}, praegu on see
Tartu Mart Reiniku Kool\index{Mart Reiniku Gümnaasium}.
 Arvutiklass paiknes Vanemuise tänaval teatri vastas 
oleva õppehoone\index{Tartu Ülikool!Vanemuise tänava õppehoone} keldrikorrusel. 
Seal oli isegi kaks arvutiklassi. Teises olid 
Agatid\index{Agat}, mis olid venelaste pihta pandud 
Commodore'i või Apple II koopiad\sidenote{Agat kasutas küll sama 
6502 protsessorit, mis Commodore 64 ja Apple II, ning oli suuresti viimasest 
inspireeritud, kuid erines disaini poolest mõlemast ja otsese koopiaga tegu 
ei olnud.}. Kusjuures mul läks rohkem kui aasta, enne kui sain aru, et see oli tegelikult 
Apple II koopia. Klassis olid ka Apple 
II\index{Apple II} arvutid ja kuigi protsessori tasandil olid need sarnased, 
oli sisu väga erinev. 

\question{See nõukogude variant oli üsna industriaalse väljanägemisega.}

Jah. Kas sa tead näiteks, et kui inglise keeles on klaviatuuril vasakult paremale lugedes 
QWERTY, siis vene klaviatuuril tuleb sama moodi lugedes kokku \enquote{pidev \emph{lag}}? 

\question{Tol ajal ei teadnud veel keegi \emph{lag}'ist midagi.}

Kui sellest ajast kümmekond aastat edasi hüpata, siis olid Tartu Ülikoolis juba arvuti- ja 
terminaliklassid. Tol ajal olid arvutid nii võimsad, et neil oli 
hunnik terminale, mis moodustasid terminaliklassi. Siis oli juba ka
väga palju võrgutegevust. Arvutiklassi kõrval oli 
IBMi koopia või litsentsi alusel tehtud ES\index{ES 
EVM}\sidenote{ES EVM (\begin{russian}ЕС ЭВМ, единая система электронных 
вычислительных машин\end{russian}) oli sari IBM 
System/360\index{System/360} ja System/370\index{System/370} 
kloone. Nende riistvara põhines küll IBMi omal, kuid oli väheste eranditega 
siiski Nõukogude Liidus välja töötatud. Tarkvara seevastu oli IBMi tarkvara 
lokaliseeritud ja väheste muutustega koopia. Neid masinaid nimetati eesti 
keeles hellitavalt jessukesteks.}, Nõukogude arvuti vene klaviatuuriga ja \enquote{pidev \emph{lag}} nii klaviatuuril kui arvutis oli väga ilmne kontseptsioon.

\question{Kõik üheteistaastased, kes arvutiklassist mööda kõndisid, ometi ei roninud 
aknast sisse. Sul pidi järelikult olema tehnika- või elektroonikahuvi.}

Ei olnud, ma käisin hoopis ratsutamistrennis. Aga mõni 
asi on kohe visuaalselt uus ja lahe ning vastandub 
kõigele muule ümbritsevale. Kujuta ette, et lähed mööda näiteks
lendamistrennist, kus inimest õpetatakse lendama. Sa ei hakka ju arutama, et ma pidin minema malet mängima või telekat vaatama. 
Lendamine on universaalselt väga \emph{cool} asi, kõigist 
teistest asjadest kümme korda kõvem.

\question{Ja siis ei oskagi pärast hästi seletada, miks sulle 
lendamistrenn meeldis ja miks sa ei läinud malet mängima.}

Lihtsalt kaldusid teelt kõrvale. Muide, ma ratsutasin neli aastat, see ei seganud.

Võtsin ühe klassivenna ka arvutiklassi kaasa. Mäletan, kuidas me arutasime omavahel, kuidas mänge tehakse. Kuidas Assembler või 
masinkood näeb nii suvaline välja ja äkki on 
võimalik \emph{random} kombinatsioone katsetades saada 
lahedaid mänge. Mõtlesime küll, et see vist ikka ei ole tõsi, 
aga oleks äge, kui nii saaks! Katsetad kümmet tuhandet 
kombinatsiooni, kõikvõimalikke koodivariante ja vaatad, milline läheb käima 
ja milline mitte. Õnneks nädal hiljem olime juba \emph{Hello 
World}\sidenote{Siiamaani peetakse oluliseks, et uut programmeerimiskeelt katsetades luuakse esmalt programm, mis väljastab kuhugi teksti \enquote{Hello World}. Traditsioon pärineb kuulsast The C Programming Language raamatust\index{The C Programming Language}.} kirjutanud ja asjad läksid natuke selgemaks. MSX 
BASICust\index{BASIC!MSX BASIC} kasvas muide välja Visual 
Basic\index{Visual Basic}, nii et Visual Basicu õppimine oli meie jaoks 
\emph{what else is new}.

\question{Kes seda arvutiklassi vedas? Pidi ju olema keegi, kes sind aknast sisse 
lasi ja kohe välja ei visanud.}

Mind visati sealt mitu korda välja, aga nad olid oma väljaviskamises 
tunduvalt vähem veenvad kui mina sisseronimises. Ma ei osanud väljaviskamise peale
kuidagi solvuda või seda pahaks panna. 
Sain ju aru, et see klass ei ole minu jaoks tehtud. Näiteks ronisin 
mitu korda sisse õpetajate täiendkoolitusele, kus tegelikult ei õpetatud 
arvuti kasutamist, vaid seda, et maailm muutub ja et arvutiõpe on 
hea käegakatsutav asi seda muutust kirjeldama. Selle visiooni taga oli üks 
väga vinge inimene, Anne Villems\index[ppl]{Villems, Anne}.

Anne Villems on tohutult kirglik, tema kirg on maailma 
paremaks teha. Õpetame neid inimesi, kes õpetavad teisi! Näitame neile ja nemad
näitavad väikestele inimestele, milline maailm võiks olla! 
Arvan, et tema täienduskoolitustele jõudnud inimesed 
olid mõnes mõttes juba paremad. Nad suutsid endale 
sõnastada, et peaksid sinna minema, sest äkki maailm muutub sellest paremaks. Need, 
kes koolituse läbi tegid ja seal omavahel suhtlesid, 
olid üks suur rest kive selles vundamendis, mille peale meie IT-riik on 
ehitatud.

\question{Õpetajad õpetasid omakorda õpilasi, kellest said abiõpetajad, ja nii see teadmine levis.}

Bingo! Koolitusel oli näiteks
üks Tartu Kunstikooli\index{Tartu Kunstikool} õpetaja, kes hiljem tõi oma lapsed 
arvutiklassi tundi pidama. Kunstikooli õpilastel oli üks
joonistusprogramm --- 64 värvi, maa ja ilm. Ja nad tegid arvutiga päriselt mingeid asju, olgugi et printerit kahjuks ei olnud. 
Igatahes said nad tunnetuse, kuidas kontseptuaalselt täiesti uuel viisil kunsti teha. 

Isegi mina, kes ma mõtlesin primitiivselt, kuidas saaks mänge teha ja 
mängida, jõudsin lõpuks kuhugi välja. Aga nemad võtsid 
graafilise \emph{editor}'i ja tegid sellega võimsaid
asju\ldots Samasugune trikk nagu iPhone'i tulek: me ei teadnud algul, kui kõva asi see oli, aga kindlasti 
sajal ägedal moel. Omal ajal oli sama lugu personaalarvutite ja MSXidega.

\question{Iga uudsus läheb ju lõpuks üle, kas sinu jaoks arvutite puhul ei 
läinud üle?}

Ei läinud, see liikus baastasandilt järgmisele tasandile. Toon ühe näite kaks aastat hilisemast ajast --- Tõravere observatooriumi\index{Tõravere Observatoorium} 
astrofüüsikud, kes olid maailmaga hoopis teistsuguses kontaktis kui 
koolipoisid. Üks selline oli minu alumine naaber Enn 
Kasak\index[ppl]{Kasak, Enn}. Ühest küljest olid kontaktid 
teadusmaailmaga, aga teisest küljest maailm sulas ja oli 
võimalik bisnist teha. Nad tõid endale Amiga 
500\index{Amiga!Amiga 500}\sidenote{Tuntud ka kui A500, oli Amiga 500 
koduarvuti 1987. aastal Commodore'i poolt turule toodud professionaalse Amiga 
2000 vaste. Tegu oli populaarseima Amiga mudeliga, eriti Euroopas.}, mis olid 
järgmine põlvkond Commodore 64st. Sellel oli kümme 
korda võimsam protsessor ja hoopis teisest klassist graafika. 
Kui MSXi Z80 protsessor võimaldas kolmehäälset muusikat teha, siis Amiga 
suutis pakkuda kuutteist kanalit. Tänapäeva mõistes oli 1986. aastal võimalik täielik MIDI-lahendus kodus püsti panna. Oi, kuidas 
ma seal nende Amigadega muusikat tegin! Täiesti häbitult ja ööde kaupa.

\question{Kas sul muusikahuvi oli enne olemas või tekkis koos Amigadega?}

Igal inimesel on mingisugune arvamus, kas talle meeldib muusika või mitte. Osaliselt on see seotud sellega, 
kas pead viisi või ei pea. Kui mind ei võetud esimeses klassis lastekoori, siis sain aru, et mulle meeldib muusika, sest ma olin väga kurb. 

Kui ma Yamahadega tegelesin, siis see ei olnud ainult mängimine. 
Meil oli täiesti mitteametlik arvutiring: 
kutid vajusid iga päev pärast kooli kohale ja enne ära ei läinud, kui välja 
visati. Klassis tegutsesid tegelikult 
üliõpilased, näiteks Ain Sakk\index[ppl]{Sakk, Ain}, Alar Pandis\index[ppl]{Pandis, 
Alar} ja mõned teised kutid, kes jätkasid pärast ülikooli lõppu vist ka pedagoogidena. Nad olid lastesõbralikud ja toetasid meid. Meil oli võimalik seal käia sellepärast, et
arvuteid oli klassis viisteist tükki, aga 
täiendkoolitustel enamasti seitse kuni kümme inimest, nii et 
alati olid mõned arvutid vabad. Asi toimis põhimõttel \enquote{kes ees, see mees}. Kui arvuti said, siis enam seda ära ei 
andnud. Koolituse ajal muud võis teha, aga mängida mitte, nii et ootasime, hambad ristis, mingi \emph{manual} 
kõrval, mille järgi proovisime asju teha. 

MSX BASICuga\index{BASIC!MSX 
BASIC} sai samuti muusikat teha: noote ritta seada, 
rütmi kiiremaks ja aeglasemaks sättida, oktaavi muuta ning vist ka näiteks 
kolm erinevat meloodiat kokku panna. Ühetoonilist muusikat sai 
kindlasti teha --- kuulasin midagi ja proovisin 
järele teha. Amiga oli selle kõrval hoopis teine tera.
Erinevus oli sama suur, nagu panna endale papist 
tiivad külge ja mängida lennukit või minna päris lennuki peale. 
Ühel juhul paned teksti-\emph{editor}'is nooditähti paika, mängid selle ette ja kuulad. Teisel juhul on täisgraafiline muusika-\emph{editor} koos nootide ja digiklaveriga, mida saab arvutiklahvide peal mängida ja salvestada nii nagu tänapäeval. Pluss sadu pille, mille seast valida, mis 
kõlasid küll digipiiksudena, aga mis olid nii ära tuunitud, et viiul ja klaver kostsid kõrvale ikkagi erinevalt.

\question{Selleks et suuta kõrva järgi muusikat järele teha, peab kõrva olema. Kas sul oli muusikaline kuulmine olemas?}

Midagi oli jah. Ega noodid ju kõik õiged pruukinud olla, aga rõõm tegemisest oli suur! Iga kord, kui midagi 
natukenegi välja tuli, viskas see puid alla juurde ja leek läks suurema hooga põlema.

Tartus oli selline võimas 
organisatsioon nagu Tartu 
Tähetorn\index{Tartu Tähetorn}, aga infotehnoloogilise ajaloo prismas oli see ainult väike ripats Eesti 
Biokeskuse\index{Eesti Biokeskus}\sidenote{Eesti Biokeskus moodustati 1986. 
aastal Tartu Ülikooli\index{Tartu Ülikool} ja KBFI\index{KBFI} ühisasutusena.} 
küljes, mis oli tähetorni kõrval väikene kuut, aga kus toimusid 
ülisuured asjad. Tähetorni katusele oli hea panna \enquote{satipann}: sealt paistis kaugele, puid ümber ei olnud ja
signaal oli alati hea. Eesti kahest
esimesest internetiühendusest üks oli Tallinnas KBFIs\index{KBFI} ja teine, Tartu oma, paikneski 
tähetornis, õigemini biokeskuses, mille ruumid olid tähetorni 
lähedal. Biokeskuses tegutses Richard Villems\index[ppl]{Villems, 
Richard}\sidenote{Eesti Biokeskuse juht selle asutamisest 
alates.}, kes koos Lippmaadega üldse selle interneti-maailma Eestile avas.

Igatahes Amigad jõudsid tähetorni ja ühendati internetti, sest kõik, kes tee peal olid, istutasid ennast ka selle traadi peale, mis enne biokeskust katuselt 
alla tuli. 

\question{Kui veel ajas tagasi minna, siis sul pidi tublisti distsipliini olema --- koolituse ajal 
taganurgas istudes tuli ju vagusi olla.}

Ma mõtlesin välja sellise asja nagu võtmeluba: mulle anti klassi võti. Kuna nädalavahetustel koolitusi ei toimunud ja valvur ärkas kell seitse üles, 
siis selleks hetkeks sai ukse taha mindud. Ukse avas väga unine valvur, kes alguses ei uskunud, et mul on mingi võtmeluba, ja ajas mind minema. 
Aga kui olin juba kell seitse hommikul kohale läinud, siis ega ma sealt ära ei 
läinud. Palusin süüdimatult helistada pühapäeva hommikul mingitele inimestele, et need võtmeloa olemasolu kinnitaksid. Üksikud uued valvurid ei lasknud sisse, aga paari-kolme kuuga 
olid nad kõik välja õpetatud.

\question{Sest jama ei tekkinud, keegi ei läbustanud ja midagi ei 
varastatud.}

Läbustamiseks polnud aega. Ainukene jama oli vaba arvuti saamine: inimesed ootasid arvutiruumi ukse taga koolituse lõppu, et äkki järgmisel koolitusel on auk ja pääseb sisse. 

\question{Miks see luba just sulle anti? Kas paistsid kuidagi silma? 
Olid eriti tubli, korralik, pealetükkiv?}

Kõik see kokku. Samas tegin ma tänapäeva 
mõistes vabatahtlikku tööd. Tahtsin nii väga olla arvutite juures, et olin nõus tegema koolituste ajal abiõpetaja tööd, 
oma vabast ajast ja ilma rahata. 
Seal õpetati ülilihtsaid oskusi, mille laps omandab paari-kolme 
päevaga, nagu arvuti käimapanek ja mis tähendab \emph{press any key}. Mul ei olnud probleemi näidata tädidele, kuhu tuleb 
vajutada. Tädidel oli ka hea meel, et lapsed oskavad seda teha. Ja Anne Villems\index[ppl]{Villems, Anne} ei visanud ka mind välja eriti. 

\question{\enquote{Eriti}...}

Ma ei tea, kui palju oli sealpool seda, et nad ei 
jaksa enam võidelda ja ei ole mõtet välja visata. Meid oli vist kolm, kellel oli võtmeluba. 

\question{Seda on ikkagi vähe.}

Kõik ei jaksanud kogu aeg käia ega mahtunud ka. Eks visamad lõpuks jäid. 

\question{Kas sa käisid seal kuni keskkoolini?}

Jah. Keskkooli läksin teise kooli, 
Treffnerisse\index{Hugo Treffneri Gümnaasium}. Seal olid küll
oma arvutiklassid, aga siis oli juba oluline tähetorn\index{Tartu Tähetorn}. Seal olid Amigad, seal lindistasime esimesed lood, mu naaber töötas seal, käisin astronoomiaringis
õppisin C-d\index{C} kirjutama. Seda õpetas mulle Kaur Virunurm\index[ppl]{Virunurm, Kaur}, 
ainuke tüüp, kes suutis, sest see, mida me seal tegime, on maailma kõige halvem 
õppimismeetod. Kujuta ette, et sinu kõrval on inimene, kes tahab mingit asja 
õudselt osata, aga ta ei oska mitte midagi ega viitsi \emph{manual}'i 
lugeda. Põhimõtteliselt sind muudetakse elavaks \emph{manual}'iks ja iga kahe-kolme minuti tagant küsib õpilane, et \emph{are we there yet}.

Tartus oli teine keskus veel, füüsikahoone\index{Tartu 
Ülikool!Füüsikahoone} Tähe tänava alguses. Seal toimetas Taavi 
Talvik\index[ppl]{Talvik, Taavi}, kes andis mulle ühe C 
\emph{manual}'i, mis oli vist fotoaparaadiga üles pildistatud.

\question{Kas see võis olla Richie kuulus sinise C-ga raamat\index{The C 
Programming Language}\label{sisu:richie}?}

Jah, aga see oli ainult must ja valge, sinist ei olnud seal midagi. Lehitsesin kümme-viisteist aastat hiljem neid 
üksikuid fotokoopiaid ja vaatasin, et päris hea kraam. 
Ega ma tollal väga palju asju C-s\index{C} ei kirjutanud, aga
hiljem küll. Igatahes Kaur\index[ppl]{Virunurm, Kaur} jaksas minu tohutut huvi taluda. 

Tol ajal olid ilmunud juba esimesed XTd ja tolle 
Assembler\index{Assembler} oli hoopiski teisest klassist kui Z80 Assembler --- kaheksa-, mitte neljakohaliste koodidega! 

Ühel hetkel lõi aga murdeiga sisse ja arvutid ei võtnud enam sada, vaid seitsekümmend protsenti ajast. 

\question{Tavaliselt tekib inimestel keskkooliajal mingisugune kultuuriline 
kontekst --- muusikat sa mainisid, aga raamatud? Veel midagi?}

Väga hea, et sa selle välja tõid. Me peame aru saama, millisele lavale 
idud kasvama läksid. Tartus oli akadeemiline keskkond ja see tähendas 
enamasti kõrget lugemust ning paremat kirjandusega kursis olekut.

Kasakul\index[ppl]{Kasak, Enn} oli tolle aja kohta täitsa hea raamatukogu, 
samuti mu tädil. Autoritest oli Asimov
kindlasti kõva sõna. 

Mu vanaema oli tõlkija, ta tõlkis kuuekümnendatel näiteks teose \enquote{I, 
robot} eesti keelde. Ta vist tõlkis kellegagi koos terve Asimovi 
kogumiku. Teine väga tugev liin oli sõrmused ja nende 
isandad. 
Võibolla mõtlen natuke üle, aga \enquote{Sõrmuste isand} on tegelikult lugu suurest ja palju tugevamast kurjusest, mille 
vastu ei saa. Mõtle, mis aastad need olid, 1987--1989! See lootus! Need 
raamatud sobisid hästi sellesse aega.

\question{\enquote{Kääbik} oli jah eesti keeles olemas, aga mina sain teada, et see on osa 
suuremast loost, alles üheksakümnendate lõpus. Kas sul olid ingliskeelsed 
raamatud?}

Jah. Need olid erilised, keskmise vene papitrüki kõrval läikivad ja ilusad. Kuidas tunda ära 
inimesi, kes olid tollal tegevad? Neil kõikidel on kodus nähtaval kohal kogu Tolkieni looming.

Ulme oli teine liin, näiteks Asimov ja Bradbury. Osa teoseid avaldati Mirabilia sarjas\sidenote[][-2.3cm]{Mirabilia oli 
kirjastuse Eesti Raamat aastatel 1973--2012 ilmunud raamatusari, mis 
keskendus peamiselt ulme- ja kriminaalromaanidele. Omal ajal oli tegu 
suurepärase võimalusega tutvuda üldjuhul väga hästi 
tõlgitud ulmekirjanduse klassikaga: Simaki, Lemi, Strugatskite, 
Asimovi, Bradbury ja paljude teiste romaanide ja lühilugude kogumikega. Paljuski 
kujundas just see sari terve põlvkonna ulmehuviliste maitse ja lääne klassikute 
hulgas ilmus ka Eesti, Soome ja Läti autorite loomingut.}. 

\question{Aga Strugatskid?}

Loomulikult nemad ka ja Stanisław Lem\sidenote{Stanisław Herman Lem (1921--2006) oli Poola ulmekirjanik, kelle 
teosed olid ühekorraga nii filosoofilised kui ka satiirilised ja 
humoorikad.} ja teised. Oli selline ulmekirjanduse kogumik nagu \enquote{Lilled 
Algernonile}\sidenote{Ilmus aastal 1976 sarjas \enquote{Ajast Aega}.}, see oli ka suhteliselt kohustuslik kirjandus. Kui 
tütarlastel oli võibolla kotis Herman Hesse \enquote{Stepihunt}, siis poisid raamatut 
kaasas ei kandnud, aga kuskil oli neil natuke äranäritud nurkadega \enquote{Lilled 
Algernonile}. USA ulmekirjanduse sissevool lõi toimuvale tõepoolest 
kultuurilise tagapõhja.

Muusikaga oli teistmoodi. Suured inimesed kuulasid suurte inimeste 
muusikat, noored noorte muusikat. Tähetornis\index{Tartu Tähetorn} 
kuulati palju muusikat maki pealt. Seal olid koos naljakad 
kooslused: esiteks tähetorn ise, siis 
füüsikatudengid, kes pühendusid astronoomiasuunale (näiteks Kaur 
Virunurm\index[ppl]{Virunurm, Kaur}), ja teisalt tähetorni 
direktori lapsed, kes käisid Miina Härma 
Gümnaasiumis\index{Miina Härma Gümnaasium} ja vedasid sinna 
omi koolivendi ja -õdesid. 
Tartus oli tol ajal kaks kooli, kes defineerisid, mis on äge: 
Treffner\index{Hugo Treffneri Gümnaasium} ja Miina Härma ning mõlemad 
arvasid, et on parem kui teine. Tartu värk. Tähetornis olidki 
ka mõned Treffneri tüübid. Sedasi tekkis segu keskkoolist, 
ülikoolist ja internetist, millest ei saanudki tulla mitte midagi peale plahvatuse. Seal oli tüüpe, kes on tänapäeval Eestis kõik
väga asjalikud.

\question{Keskkoolinoorena astrofüüsikutega sammu pidada ja originaalkeeles 
Tolkieni lugeda ei ole lihtne. Sa pidid ikka nutikas 
inimene olema.}

Mul oli sõnaraamat kõrval, kuni ühel hetkel polnud seda
enam vaja. Treffneris\index{Hugo Treffneri Gümnaasium} 
oli gümnaasiumis bioloogia-keemia õppesuund, kus õpetati ladina keelt, 
tavalist inglise keelt, aga ka teaduslikku inglise keelt, mille 
kõrval tavaline inglise keel oli \emph{walk in the park}. Seda ainet andis 
muide bioloogiaõpetaja\sidenote{Õpetaja Tago Sarapuu\index[ppl]{Sarapuu, Tago} 
õpetas ka Meelis Roosile\index[ppl]{Roos, Meelis} bioloogiat.}, kes oli tugeva akadeemilise taustaga ja 
tegi hiljem pikalt akadeemilist karjääri. Sa 
mainisid sinise kaanega C-õpikut. Kujuta ette, et sul on näiteks geneetika 
kohta samasugune ning sa võtad ja närid ennast sellest lihtsalt läbi. Paberit lendab 
kahele poole, aga sa tõlgid selle kõik ära. See aitas hiljem väga hästi kaasa.

Meil oli saksa keel ka, ma sain saksa keele lõpueksamil kooli parima hinde. 
Aga miks? Sellepärast, et Kaur Virunurmel\index[ppl]{Virunurm, Kaur} oli samal 
ajal ülikoolis saksa keele eksam. Ta tõmbas sõnaraamatu arvutisse, tegi sellest baasi ja siis oli 
võimalik skoorida: õige vastus andis punkti, vale vastus miinuspunkti. Mina 
valmistusin saksa keele eksamiks niimoodi, et viimasel õhtu enne eksamit mängisin punktide peale saksa keele sõnade tõlkimist, ja 
see aitas mind väga hästi. See oli üks esimesi kordasid, kus 
võin kindlalt väita, et infotehnoloogiline tööriist parandas mu sooritust 
hüppeliselt. Lõpuklassis oli mul küll saksa keel vist ühel veerandil ka kaks, aga 
see oli ealine iseärasus, hinded ja teadmised ei ole alati 
omavahel lineaarses seoses.

Kui nüüd muusika juurde tagasi tulla, siis Miina Härma \index{Miina Härma Gümnaasium} kutid tõid tähetorni
The Smithsi, The Cure'i ja 
ka vene muusikat. Ilmusid välja kitarrid ja 
midagi ka lindistati, ilmselt kassettidele. 

Samas oli tähetornis nõukogudehõnguline teaduskultuur, mille juurde 
käis näiteks konjaki ja kohvi joomine. Keskkooli- ja 
üliõpilased ei saanud seda küll endale lubada, aga tubades oli see hõng üleval. Ma ei 
teagi, mis asjaoludel seal joodi, sest läbusid 
ei toimunud, aga see kõik tekitas erilise atmosfääri.

\question{Seal tehti ju teadust ja mitte halba.}

Just. Ma käisin isegi astronoomiaringis ja mind saadeti 
oma tööga kuskile rahvusvahelisele
õpilaskonverentsile esinema. Kõike sai teha.

\question{Kuhu sa pärast keskkooli õppima läksid?}

Tahtsin minna ülikooli ajakirjandust õppima. Ülikooli mitteametlik \emph{statement} oli see, 
et kui tahad tulla ajakirjandust õppima, siis pidi olema ette näidata portfoolio ehk
pidid olema midagi avaldanud. Ajakirjanduse või üldse meedia õpetamine on 
suhteliselt kallim tegevus kui näiteks keeleõpe ja nad tahtsid olla veendunud, et üliõpilane tõesti tahab ajakirjandust õppida, mitte ei astu 
juhuslikult sisse. Mulle tundus, et kultuuriajakirjanik on äge olla. Ilmselt selles vanuses 
arvab iga mees, et kultuuriajakirjanik on äge olla, sest on olemas oma arvamus maailmast, mille masstiražeerimine tundub 
veidral kombel teiste aitamisena.

Ühesõnaga, tegin ettevalmistusi: käisin 
kontsertidel, kirjutasin intervjuusid ja arvustusi. 
Aga samal ajal käisin ka näiteringis. Lõpuklassis olid veebruaris-märtsis 
Viljandis lavaka sisseastumiskatsete 
eelvoorud. Mõtlesin, et äge oleks minna Viljandisse trallima ja 
seiklema, aga sain eelvoorust edasi ja hiljem lavakasse sisse.

Ma ei pabistanud üldse ja eks see aitas. Teadsin, et mul on
ajakirjandusega plaanid ja head soovituskirjad 
toonastelt tuttavatelt Postimehe\index{Postimees} ajakirjanikelt. Õppisin 
lavakas ligi aasta, ent ühel hetkel sain aru, et see ei ole ikkagi see, mida ma teha tahan. 
Sellele otsusele jõudmist mõjutas kõik eelnevalt kirjeldatu 
väga tugevalt. See \enquote{lendamistrenn} paistis kogu aeg aknast. 

Kooli kõrvalt sattusin sellisesse ägedasse kohta nagu Riigikogu 
Kantselei\index{Riigikogu Kantselei}.

\question{Kas sel ajal olid seal juba võrgud ja BBSid?}

Riigikogu Kantseleis oli täiesti adekvaatne kraam juba aastal 1992. 
Ühtlasi üks äge tekstipõhise kasutajaliidesega võrgumäng, 
põhimõtteliselt tolle aja Fortnite, mille nimi oli 
MUME\sidenote{Üks populaarsemaid MUD-tüüpi mänge, mille nimi tuleneb fraasist 
\emph{Multi-Users in Middle-Earth} ja mis põhines 1991. aastal loodud ja siiani 
aktiivselt arendataval DikuMUDi\index{Muda!DikuMUD} mootoril. Vt ka märkust \ref{sidenote!muda} 
lk \pageref{sidenote!muda}.}, tegu oli Tolkieni ainetel loodud 
Mudaga\index{Muda}. Tuletan meelde, et need ringkonnad olid kõik 
tugevalt Tolkieni usku.

\question{Kas seal oli server, kuhu mängijad külge läksid?}

Jah. Telneti pordi kaudu tõmmati sind külge, kõik istusid oma 
\emph{socket}'is\sidenote{St. omasid iseseisvat ühendust serveriga.}, aga nägid, mida teised teevad. Ja kuna see oli 
tekstipõhine, siis võrguühenduse kiirus ei olnud probleem.

\question{Kuidas adekvaatne kraam Riigikogu Kantseleisse\index{Riigikogu Kantselei} 
sai? 1992. aastal ei olnud Eesti Vabariik veel kuigi 
heal järjel ja oli muudki, mida korrastada.}

Väga hea küsimus. Tarvi 
Martens\index[ppl]{Martens, Tarvi} kindlasti teab seda. Samuti
Toomas Mölder\index[ppl]{Mölder, Toomas}, kes oli nlibi, 
tollase Rahvusraamatukogu\index{Rahvusraamatukogu} IT-juht tänapäeva mõistes. Ja eks KBFI\index{KBFI} rahvas aitas ka.

\question{Kas peale MUME'i mängimise tegite seal kasulikke asju ka?}

Ma käisin tol ajal lavakas ja teadsin küll, et nad teevad 
vingeid asju, aga tavaliselt siis, kui mina sinna saabusin, 
lõppes töö ära, sest tuli \emph{orc}'ideks kehastuda ja minna 
\emph{whiteskin}'e tapma. MUME'i tekitatav adrenaliinitase ei jäänud alla 
tänapäevaste arvutimängude omale. Näiteks olid seal haldjas ja sulle tuli 
teade: \enquote{\emph{An orc enters the room}.} Selle peale lükkab ka täna teatud 
seltskonnal vererõhu kakskümmend protsenti ülespoole. MUDe oli veel, aga 
MUME oli üks esimesi selliseid mänge, mis kestis aastaid. Esimesed 
eestlased, kes seal mängisid, tegid oma tegelased aastal 1991 või 1992 ja mängisid kolm-neli aastat järjest. Pronto\index[ppl]{Pronto} ehk Tanel Raja mängis muide väga kõvasti MUME'i. 

\question{Mäletan, et 1993. aastal sisenesid mingid inimesed Liivi tänaval 
VAXi klassi\index{Tartu Ülikool!Liivi Õppehoone!Vase klass}\sidenote{\label{sidenote!vaks}Sõna \enquote{vask} mitmesuguste 
variatsioonidega kutsutud ja ilmselt klassi toitnud arvuti tüübinime 
VAX\index{VAX} järgi 
nime saanud klass asus matemaatikateaduskonna Liivi tänava 
õppehoone esimesel korrusel ja koosnes vask.ut.ee\index{vask.ut.ee} 
külge ühendatud terminalidest.} ja kui mina ükskord ülikooli lõpetasin, siis nad 
olid ikka veel seal.}

Jah, \enquote{pidev \emph{lag}} oli muide sealsamas VAXi klassi kõrval olevas 
ES-klassis, mis oli alati tühi, kuna need arvutid olid jamad\sidenote{Ilmselt 
peab Jaanus silmas Raua\index{raud.ut.ee} klassi, mis käis päris IBMi 
riistvara, mitte ESi peal. Raul Tölp\index[ppl]{Tölp, Raul} meenutab: \enquote{Sattusin kas 
1996. või 1997. aastal Liivi 2 hoonesse, kus mul paluti IBMi esindajana teha raud.ut.ee serverile masina viimane \emph{power off}. Räägiti, et masin küttis 
vesijahutusega tervet maja.}}. Liivi tänava 
VAXi klass on omaette peatükk, milleni kohe jõuame.\label{sisu:jaanus:vask}

Kui nüüd õpingute juurde tagasi tulla, siis lahkusin lavakast esimese kursuse viimasel veerandil. Üheksateistaastane laseb end
välisest tugevalt mõjutada. Tollal erines
näitlejaamet sada protsenti sellest, mis see täna on. 
Oli nädalaid, kui jõin vähemalt pool pudelit viina päevas. 
Organism oli tugev, vedas ilusti välja, aga nägin kutte, kes olid 
seda kümme aastat teinud. Ühel hetkel küsisin endalt, kas ma jaksaksin 
ja tahaksin niimoodi elada. \emph{Hell no}! See ja \enquote{lendamistrenn} akna taga aitasid äratuleku otsust teha. 

Siis läksin Tartusse\index{Tartu Ülikool} eesti keelt, täpsemalt 
arvutuslingvistikat õppima.

\question{Kes seda õpetas? See oli tollal mujal maailmaski suhteliselt uus ala.}

Nüüd jään vastuse võlgu. Seal oli üks lahe lühike vanamees, kes 
oli tõeline guru. Hästi viisakas, vaikne ja rahulik sell, nii palju kui 
mina temaga suhtlesin. Aga tema juurde jõudsin alles kolmandal aastal pärast 
spetsialiseerumist. Enne olime lihtsalt ühes toredas teaduskonnas, kus 
põhiliselt õppisid tüdrukud.

Selle taustal oli mul ikkagi tunne, et peaksin ka mingit tööd tegema. Samas olin
ainult \enquote{lendamistrennis} käinud. Mängu tuli seesama Vase klass.

Mängisime seal \emph{StackMUD}i\index{Muda}. Stacken.kth.se\index{stacken.kth.se} oli Rootsi 
Kuningliku Tehnikaülikooli\index{Rootsi Kuninglik Tehnikaülikool} 
VAX\index{VAX}\sidenote{Virtual Address Extension --- arvutisari, mille töötas välja DEC
seitsmekümnendate keskel. Siiani üks tuntumaid omalaadseid arhitektuure,
oli see PDP-11\index{PDP-11} edasiarendus, peamiselt mälu virtuaalse
adresseerimise suunas.}, mille peal 
jooksis BSD\index{BSD}, mille peal pandi käima Muda. 
Originaalne DikuMUD on muidu tehtud Taanis.

Mängimise tegi huvitavaks see, et mängijad ei olnud matemaatika üliõpilased, 
nagu oleks võinud arvata, vaid eesti keele ja usuteaduse 
üliõpilased. Näiteks praegune kirjanik ja usuteaduskonna õppejõud Meelis 
Friedenthal\index[ppl]{Friedenthal, Meelis} oli väga originaalne mudamängija. 
Oli teisigi tüüpe, kes käisid tõesti palju mängimas, mina 
sealhulgas. Suhtlus selle seltskonnaga ei piirdunud ainult 
mängimisega, me ka ehitasime seda maailma. Ma olin üks põhilisi ehitajaid.

\question{Kuidas see käis? Kas kirjutasid koodi või skripti?}

See oli väga lihtne: sain koodi koopia ja mul oli andmebaasi struktuur, kus 
täitsin väljad ära. Võtsin andmebaasi \emph{dump}'i 
teksti \emph{editor}'is lahti ja tegin näiteks ridadest sada 
kuni tuhat koopia ning kirjutasin sinna asjad teistmoodi. \emph{Editor} oli loomulikult vi\sidenote{Unixi spetsifikatsiooni osaks 
saanud, 1976. aastal kirjutatud tekstiredaktor, mis on siiani teatud 
ringkondades (ka käesolev tekst sünnib osalt vi abil) väga populaarne. Siin 
kontekstis on oluline, et erinevalt tänapäevastest tekstiredaktoritest ei 
olnud vi ainult tekstipõhine, vaid ka suhtles kasutajaga 
ähmaste käsu- ja klahvikombinatsioonide abil. Näiteks on 
legendaarsed algajate kasutajate tulutud katsed redaktorist väljuda, kuna 
selleks kasutatavad klahvikombinatsioonid \texttt{ZZ}, \texttt{:q!} ja veel kümmekond 
samalaadset ei ole just intuitiivsed.}. Kirjutamine käis tsoonide kaupa: ühes tsoonis oli 
sada ruumi ja iga ruumi kohta kirjutasin ingliskeelse kirjelduse. Mitmesuguseid asju sai 
ruumis olla vist kuni 255, kolle sai ka olla teatud
kogus. Täitsid kõigi ruumide kohta statistika ära, lisasin kirjeldused ja
tegevused ning postitasin. Kutid kompileerisid selle ära ja nii see tuli. 

Me saime Rootsi mängutegijatega üsna hästi läbi ning rääkisime 
vahel ka olmest ja inimlikest teemadest. Näiteks et meil on ainult üks klass ja 
seegi on kogu aeg pooltäis, ja kui mõni ahv FTPga tont teab mida tõmbab, ei 
saa üldse mängida. Nemad ütlesid, 
et neil on üks arvuti üle, kuna said uuema VAXi\index{VAX}, mille peal on 
BSD\index{BSD}. Ma küsisin, kas me vana arvutit kuidagi endale ei 
saaks, ja öeldi, et saate küll. 

\question{See masin oli ju Rootsis.} \label{sisu!jaanus_liivi_tn}

Jah. Anne Villems soovitas mul rääkida Otto Telleriga\index[ppl]{Teller, Otto}, kes oli vist 
arvutiteaduse õppetooli juht\sidenote{Tõenäoliselt toimetas Otto Teller siiski Tartu Ülikooli 
arvutuskeskuses\index{Tartu Ülikool!Arvutuskeskus}, mis oli 
eraldiseisev üksus.}, ja ütlesin, et ilmselt te mind ei usu, aga Rootsis on üks arvuti. Sellega tuleb kaasa 16 terminali, 
me saaksime teha terve klassi ja võin ise seda 
administreerida. (Kui muda sees mängid, siis oled kaelani porine, st 
süsteemide administreerimise oskus tekkis iseenesest.) 
Aga ma olen lihtsalt üliõpilane ega tööta siin, palun aidake. Lõpuks Otto Teller vastas, et see on küll kõik
väga imelik, aga olgu, ja ajas asja korda.

Siis kirjutasin igale poole kirju ja sain vastuseid. Ma ei tea, mida Otto 
Teller tegi, aga ta hüppas pea ees tundmatusse mingi kahekümneaastase 
kuti kätega vehkimise peale. 1993. sügisel sõitsime Rootsi klassi 
üle vaatama ja talvel oli korraga üks furgoon 
Laia tänava ukse taga. Tekkis Laia tänava arvutiklass\index{Tartu Ülikool!Laia tänava 
arvutiklass} ja mina sain selle administraatoriks. See oli minu 
esimene töökoht.

Tollal ei olnud administraator ainult tehniline töötaja, vaid ka 
administratiivne tegelane, kohalik jumal. Oleks mul võimuiha olnud, 
võinuksin seda väga hästi realiseerida, aga ma andsin hunnikule tüüpidele 
võtmed ja palusin, et nad serveriruumi ei läheks. Kuskilt 
veeti eraldi kaablid, Zyxeli\sidenote{1988. aastal 
Taiwanil asutatud Zyxel Communications Corporation tootis 
ülipopulaarseid ja hinnatud modemeid.}\index{Zyxel} modemid said üle püsiliini 
internetiühenduse ja seal me müttasime.

Olgem ausad, tänapäeva mõistes oli see klass totaalne õnnetus. Nii 
madala käideldavusega asja pole ma hiljem näinud. Arvuti oli väga 
vana, läks tihti katki ja ma ei tundnud nii professionaalset raudvara hästi. Mul oli abiks Viljo 
Soo\index[ppl]{Soo, Viljo}, kes oli tänapäeva mõistes \emph{sysadmin} ja kes 
aitas masinat palju kordi käima panna. Läks paar kuud ja 
saime klassi tööle. Klassi nimi oli Cure\index{cure.ut.ee}, 
The Cure'i järgi. 

Kunagi oli selline mäng nagu Nethack\index{Nethack}\label{sisu:nethack} ja sellest oli 
üks naljakas kloon tehtud. Otsustasin selle eesti keelde 
tõlkida. See käis põhimõtteliselt samamoodi: võtsin C koodi lahti ja 
hakkasin esimesest reast lugema.

\question{NetHacki lähtekood on niisamagi hea lugemine, see on üks kahest, 
mida ma olen oma elus lugemise eesmärgil välja trükkinud. Teine on Perl.}

Ühesõnaga ma tõlkisin kõik eesti keelde esimesest reast viimaseni, kaasa arvatud \emph{library}'d ja kõik muu, mis kaasas oli. Põhilise ekraaniteadete osani jõudsin 
kell neli hommikul. Teadupärast tekib väga suure väsimuse korral ühel hetkel veider pooleufooriline meeleolu. Mul saabus see hetk keset tõlget ja mäng kukkus
naljakas välja\sidenote{Mängus tembutanud \enquote{mõõkhambulisi varblasi} meenutatakse siiani hea sõnaga.}. Seda mängiti klassis väga palju, seda enam, et 
võrguühendus alati ei töötanud, aga NetHack oli kohalik. Senikaua, 
kuni keegi Viljo Sood\index[ppl]{Soo, Viljo} otsis, et ta 
modemitele restardi teeks, mängiti NetHacki. Ma ise ei pidanud seda suureks saavutuseks, lihtsalt tegin valmis. Kahjuks läks kood koos Cure'i 
masinaga kaduma.

\question{Nutikal inimesel oli tollal tüüpiliselt kaks suunda, kuhu kiskus: 
kas akadeemilisse maailma teadust tegema või äri suunas. Kas sind ei tõmmanud 
kumbki?}

Mind äri ei tõmmanud, sest olen pärit äärmiselt vaesest perekonnast. Raha ei 
olnud midagi erilist, mul lihtsalt ei olnud seda kunagi. Kuude kaupa elasin saiast ja piimast. Ja kui raha ei ole, siis ei teki sellega ka lähedast 
suhet. Mis puutub akadeemilise maailma, siis olin sel ajal
alles esimesel kursusel.

Cure'i klassi tegemisest mõni kuu hiljem toimus üks tüvikursus. Pildile ilmus taas Anne Villems\index[ppl]{Villems, 
Anne}, kes korraldas 1994. aasta alguses Eesti esimesed \emph{webmaster}'ite kursused. 
Liivi tänaval olid kuulutused üleval.

Mina olin siis oma arust juba kõva käpp. 
Tol ajal kasutati Gopherit\index{Gopher}\sidenote{Gopher oli varajane hüpertekstiprotokoll, WWW 
protokollistiku eellane. Erinevalt suhteliselt lõdvalt struktureeritud veebist 
surus Gopher sisu küllalt rangesse hierarhiasse ja oli navigeeritav 
menüüsüsteemi abil.}, HTML 1.0 standardi eelkäijat, millest oli lihtne aru saada: oli klient, server ja \emph{markup language}, mille 
põhimõttest sai kümne minutiga aru.

Kursusel selgus, et arusaamisega läheb natuke 
rohkem kui kümme minutit. Kursusel käinud seltskond oli 
kirju, sinna sattus igasuguseid karvaseid ja sulelisi erinevatest 
teaduskondadest. Teiste hulgas oli seal näiteks Anto Veldre\index[ppl]{Veldre, Anto}, aga 
ma ei mäleta, kas õpetaja või õpilasena.

\question{Mis seal ikka nii väga vahet on.}

Tollal ei olnud jah vahet. Üks oli asja läbi lugenud ja rääkis teistele 
edasi. Aga Anne Villems\index[ppl]{Villems, Anne} oli kursuse väga hästi ette valmistanud. Päris kohe selle järel otseselt midagi suurt küll veel ei juhtunud, aga osalejate nimekiri jäi alles. 

Kui EENet\index{EENet} tegi endale veebilehte, olid nad kuulnud, et on olemas
\emph{webmaster}'id, kes oskavad veebi teha, tänu millele tekib organisatsioonis 
avaliku infohalduse funktsioon. (Nüüd oskan ma seda niimoodi nimetada, aga 
tollal panime lihtsalt asju internetti, näiteks 
võtsime Eesti kaardi ja sellele vajutades juhtus midagi.) EENet otsustas teha endale täiskohaga \emph{webmaster}'i ametikoha. Võimalik, et seni oli seda tööd teinud mõni ülimalt nutikas 
sekretär või tollal seal töötanud Marek 
Tiits\index[ppl]{Tiits, Marek}, aga 
asi lõppes sellega, et pool aastat pärast kursust kutsus EENet mind
\emph{webmaster}'iks. Palk oli kolm korda suurem kui arvutiklassis, nii et raske oli ei öelda.

Sealt edasi läks elu väga ägedaks. Saime tuttavaks Tarvi 
Martensi\index[ppl]{Martens, Tarvi} ja Toomas 
Mölderiga\index[ppl]{Mölder, Toomas}, tegime EENetile korraliku veebi ning käisime Eestit esindamas. Tollal oli veeb küll korralik, 
aga hiljem tehti veel ägedamaks, kui liitus näiteks Pille 
Pruulmann-Vengerfeldt\index[ppl]{Pruulmann-Vengerfeldt, Pille}, kes on praegu
Rootsis meediaprofessor\sidenote{Intervjuu toimumise ajal, 2019. aasta novembris, oli 
Pille Pruulmann-Vengerfeldt Malmö ülikoolis meedia ja kommunikatsiooni professor.} ja 
ERRi\index{Eesti Rahvusringhääling} nõukogu 
liige. Seal puutusin Marek Tiitsu\index[ppl]{Tiits, Marek} kaudu esimest 
korda kokku europrojektidega. Tollal küll veel ei olnud euro, vaid eküü\sidenote{Selle valuuta tähiseks oli ECU: 
\emph{European Currency Unit}.}. 

\question{Kas Marek oli see võlur, kes valdas unikaalset teadmist, 
kuidas fondidest raha saab?}

Just. Mina tulin lagedale veidrate ja üsna ebareaalsete
ideedega ja tema kasutas väikest osa neist hullustest, millel oli mingi point, projektides ära.

\question{Tol ajal liikus EENeti ja IBSi\index{IBS|see{Institute of Baltic 
Studies}}\index{IBS} kaudu tohutult palju põnevaid ja kasulikke projekte.}

Seal jooksis igasuguseid naljakaid teenuseid, aga see polnud kõik, millega tegeleti. 
Näiteks suutis Marek hankida mulle tööarvutiks Silicon 
Graphicsi\index{Silicon Graphics}\sidenote{1990. aastal asutatud ja 
2009. aastal pankrotistunud Silicon Graphics oli peamiselt 3D-graafikale 
keskendunud riist- ja tarkvaratootja. Mitmed varased arvutiabi kasutanud 
filmid, näiteks 1993. aastal linastunud \enquote{Jurassic Park}, kasutasid just Silicon 
Graphicsi tööriistu. Ettevõttele tegi lõpu odavate laiatarbe x86-arvutite 
võimsuse kiire kasv.} masina. Silicon Graphics oli väga kõva asi, tänapäeval 
võibolla võrreldav Mac Proga. Ulmeline aparaat ja milline disain! 
Korpused olid värvilised, kõvaketas 
käis lahti kangiga. See oli nagu automaailma 
Bugatti või Porsche 911, täiesti \emph{over the 
edge}. Mõni ime, et ma selle töö hea meelega vastu võtsin, olles tulnud 
totaalselt vananenud VAXi klassi administraatori kohalt.

Kõige selle taga oli Richard Villemsi\index[ppl]{Villems, Richard} pikk, aga heledat ja positiivset 
värvi vari. Tema seda kõike püsti hoidis.

\question{Hoidis püsti, aga vist ka natuke nagu joonte sees, et inimesed päris 
hullustega ei tegeleks?}

Jaa. Richard Villems on muide Anne Villemsi\index[ppl]{Villems, Anne} abikaasa. 
Tänu Richardi suurele 
mõjule toimusid EENetis\index{EENet} väga kõvade projektide arutelud, 
mis ei käinud tegelikult üldse selle organisatsiooni põhikirjajärgse tegevuse alla. 
Sealsamas Liivi tänaval oli ka ülikool, arvutiteaduse 
instituut\index{Tartu Ülikool!Matemaatikateaduskond!Arvutiteaduse Instituut}, samuti füüsikamaja. Nii et püssirohtu jagus ja 
tekkis igasuguseid initsiatiive.

\question{Kas selles maailmas BBSid ka kuidagi figureerisid?}

BBSid olid kogu aeg taustal ja mõnes olid mul isegi kasutajad, käisin
isegi seal. Aga BBSi tehnoloogiline \emph{carrier} oli modemiga üle
telefoniliini ühendumine serverisse. Mina sain varakult väga kiire 
interneti juurde, 64 kilobitti sekundis on tuntavalt kiire. 

Edasi tulid väga kiiresti IRCd\sidenote{Internet Relay Chat 
(IRC) on kliendi-serveri arhitektuuril põhinev tekstipõhise kommunikatsiooni protokoll. 
Peamiselt disainitud suhtluseks suuremates gruppides, kuid 
võimaldab ka üks ühele suhtlust.}, mis võtsid BBSide funktsiooni üle. 

EENetis\index{EENet} toimus palju lahedaid asju, mõnes mõttes oli see 
ebaproportsionaalselt nähtav organisatsioon. Näiteks aitasid
nad korraldada koolidesse internetti.

\question{Lisaks sellele, et keegi kuskil poliitilisi otsuseid teeb, peab keegi 
suutma ja viitsima neid otsuseid ka ellu viia. Sõita talveööl 
Põlva kanti kooli modemeid installeerima ei ole palga eest tehtav asi.}

Seesama \emph{case} võtab kokku kogu tolle aja mentaliteedi. 
Seda üldse ei arutatudki, kui palju koolide internetti 
ühendamine maksab, kuna paljudes kohtades raha polnudki. 
Arutati ainult ühte asja: tuleb teha ja Marek\index[ppl]{Tiits, 
Marek} otsib, kust raha saab. Marek ei kantinud raha autodeks ega suvilateks, Marek tegi europrojekti ja tuli. Põmm, kümme 
Suni. Põmm, kakskümmend Zyxelit. Paar-kolm kutti viisid asja ellu ja 
keegi ei pahandanud. Aeg-ajalt käis mõni Antsla kooli mees 
küsimas, kuidas läheb ja kas oktoobris tuleb. Ja tuligi, kuigi vahel 
novembris. Keegi ei arutanud, kas teha. Prooviti vaid vaadata, et teenuse laienedes kvaliteet ei kukuks. 

\question{Mis seda kõike edasi vedas?}

Küllap iga agraarühiskonna tung harida maad, kus midagi ei kasva. Mõnes mõttes oli see lihtne: prioriteedi määras 
see, milline kool kõige rohkem ise huvi tundis. Alguses ei olnudki huvi suur, sest ei saadud aru, millest üldse jutt käis.

\question{See on väga õige lähenemine, et kõige suuremad hädalised, kes 
kõige rohkem oskasid internetiga midagi teha, said selle ka esimesena 
kätte.}

Ma ei tea, võibolla mõnesse kohta jõudiski internet alles 2000. aastal, aga 
vahet ei olnud, sest selleks ajaks oli klõps juba 
ära käinud. Nii naljakas kui see ka pole, aga üheksakümmend protsenti tööst oli 
veel tegemata, kui kümme protsenti internetiühendusega koole oli kaalu juba 
nii alla vajutanud, et ülejäänute puhul oli üksnes aja küsimus, millal juhe nendeni 
viiakse. Aga et seda kõike oli vaja, oli Anne Villemsi\index[ppl]{Villems, 
Anne} ja tema pundi sügav veendumus. 

\question{Mida sa praegu teed?}

Püüan avalik-õiguslikus meedias saada ühele poole transformatsiooniga, mille eraõiguslik meedia on kümmekond aastat tagasi ära teinud.

\question{Kindlasti vääriline töö, kus väljakutseid jagub.}

Kui avalik-õiguslik ringhääling\index{Eesti Rahvusringhääling} ise kaua aega ei 
tunnetanud, et peaks internetikasutajakeskse hüppe tegema, 
siis oli ka teistel institutsioonidel raske seda nende eest ära 
tunnetada. Selle tagajärjel tekkisid mitmed fundamentaalsed küsimused: 
kuidas te ütlete, et teil on sellise asja jaoks raha tarvis? Aga kus te siis 
olite, kui teised organisatsioonid sellega tegelesid? Riigi jaoks on olnud keeruline 
mõista, et kui avalik-õiguslik ringhääling niisuguse hüppe ette võtab, on see
ikkagi ookeani ületamine ja parvega seda ära ei tee.

\question{Ja kui terve riik on läinud teisele poole ookeani, siis on pisut
sandisti, kui ERR teisele poole maha jääb.}

Sellal kui suurem osa audiovisuaalsest meediast toimus kinoringvaate vormis (oli 
filmilint, mis ilmutati ja mida projektori abil näidati), oli 
televisioonil kuuskümmend aastat tagasi juba \emph{live}-signaali halduse kontseptsioon, mis töötas ja oli 
piisavalt lollikindel, et sellega Eestis eetris olla. 
Televisiooni tehnoloogia on arenenud oma kinniste protokollide ja 
signaalihaldusmudelitega ning oli internetist kaua aega signaali 
loogika poolest maas. Teleasi maksab muidugi ulmeliselt palju, aga 
see on olnud kogu aeg terviklik kinnine maailm, mis arenes teist 
evolutsioonipuu haru mööda. Mõni aasta tagasi jõudsid Euroopa 
Ringhäälingute Liit ja teised üleilmsed ringhäälinguorganisatsioonid oma standarditega nii kaugele, et on olemas IP-põhine 
signaalihaldusstandard, mis ei ole veel valmis, aga millest mõned tükid 
töötavad. Aga nad tulid sellega lagedale aastal 2017. Mõtle, kui kaua aega on olnud 
normaalselt töötav internet.

Praegu saame öelda, et tegelikult ei ole mõttekas mitte-IP-põhist 
tehnoloogiat ehitada, aga tollal läks terve tööstusharu teist rada pidi kaugele 
edasi.

\question{Seega ERRi vaatepunktist mitte ainult ei ületata parvega ookeani, vaid parvel on 
känguru ja hobune, keda üritatakse panna kuidagi järglasi saama.}

Sealjuures on veel tugevad kogemused hundiga, kes puhub puust ja õlgedest 
maja ära. Järelikult on parv tehtud igaks juhuks betoonist. 

Meil on nii äraspidised kogemused, et puust paadi kontseptsioon 
tundub algatuseks lihtsalt ohtlik. Televisioonisignaali haldusloogika seisneb selles, et ehitame 
signaali nii, et see ei saaks katkeda. Kui palju see maksab? Nii palju, kui vaja! Teeme 
nii, et ei katke! IP-põhine paketihaldusloogika ütleb, et lükkame paketid läbi ja, kui vaja, parandame. Need on 
fundamentaalselt erinevad mudelid, aga pikas plaanis on parandamine 
odavam kui kohe hästi tegemine.


\chapter{Kaspar Loit}
\index[ppl]{Loit, Kaspar}
\index[ppl]{B'Knows}
\index[ppl]{B'Knows|see{Loit, Kaspar}}

\question{Kes sa oled?}

Mina olen Kaspar. Ja kunagi, kuna me peame tagasi kerima mingisugune miljard 
aastat, siis mu aka oli B'Knows. 

\question{Aga kust sa said sihukese aka?}

Seda ei mäleta enam keegi. Seal on nagu kaks komponenti. Üks on nagu 
\enquote{B} ja siis on nagu \enquote{knows}, ehk siis see B peaks  midagi nagu 
teadma. Pronto\index[ppl]{Pronto} alati kutsus mind Buttknows.

\question{Kuidas sina arvutite juurde said või arvutid said sinu juurde?}

Mul on selge mälupilt, et mu tädi, kes on superkuul ja minust mõnevõrra vanem, 
töötas Tartus
vist Bioloogia Instituudis. Ja  talle oli kuidagi jäänud mulje, et mind võivad 
huvitada sellised asjad. Ma arvan, et ma olin mingi, ma ei tea, 
kaheksa-üheksa-kümme, \emph{something like that}. Kui ma tal ükskord Tartus 
külas käisin, viis ta mind instituuti. Muidugi peale tööd, kõik oli juba pime. 
Mingi kabinet oli lahti ja laua peal seisis masin, mille nimi oli Apple II 
Europlus\index{Arvutid!Apple II}\sidenote{Apple II Europlus oli Apple Euroopa 
turule kohandatud versioon. Muu hulgas erines toiteblokk aga ka video osa tuli 
ümber teha, sest Steve Wozniaki trikid NCTS signaali genereerimisel keerukama 
PAL süsteemi puhul enam ei toiminud.}. See oli \emph{freaking awesome}. Ta oli 
seal mingi laborandi käest küsinud, et kuidas  midagi käib. Sai laadida paar 
üli superägedat mängu, mis tekstiekraanil jooksid. Üks oli vist \emph{Train 
Robbery}\index{Mängud!Train Robbery}, ma mäletan, see pilt on täitsa silme ees. 
Ja sellest hetkest ma arvan, ma olin müüdud ka. Ma ei oska  meenutada, kas ma 
olin arvutitega enne ka kokku puutunud aga tõenäoliselt mitte, see oli ikkagi 
liiga  vara. 

Kuna mulle tohutult meeldisid koolis Nintendo väiksed Game \& 
Watch\index{Nintendo Game \& Watch} mängud, võib-olla mäletad? Tänapäeva 
telefoni suurused umbes, neil oli LCD ekraan, millesse oli ette joonistatud 
mingisugused tegelased ja siis nuppudega said mängu mängida seal 
ekraanil\sidenote{Vaata ka märkust \ref{sidenote!gameandwatch} leheküljel 
\pageref{sidenote!gameandwatch}.}. Ja  ma kuidagi mõtlesin, et \enquote{oh, kui 
lahe oleks, kui saaks ise niisuguseid teha}, aga no ma sain aru, et seal taga 
on mingi tootmine ja see ei ole nagu reaalne. Ja nüüd järsku saada aru, et 
selliseid asju on võimalik  masina sisse programmeerida ilma, et sa pead mingit 
elektroonikaskeemi tootma, et sul ei pea üldse mingit tehast olema. See oli üks 
niisugune \emph{revalation}, onju. See muidugi  tõenäoliselt viis mind kunagi 
ka mängude tegemiseni. 

Aga niisugune päris toimetamine hakkas ilmselt kuskil seoses Jaak 
Loondega\index[ppl]{Loonde, Jaak}. Ta oli kindlasti ülioluline tegelane, sest 
sel ajal oli arvutile ligipääs oluline asi. Ja ma mäletan, et ma tegelikult 
olin kaardistanud omale kõik kohad, kus üldse  tõenäoliselt Eesti Vabariigis 
arvutitele ligi pääses. Nõo oli liiga kaugel selgelt, aga Tallinnas neid kohti 
ikka oli. 

Aga Jaak Loonde\index[ppl]{Loonde, Jaak} oli selles mõttes lahe tegelane, et 
minu meelest vist tema kaudu ma esimest korda sain midagi progeda. 

\question{Kas ta siis andis selle võimaluse või ikka õpetas ka?}

Ta ikka õpetas, loomulikult. Mingil põhjusel, ma ei tea kuidas, ma sattusin  3. 
Keskkooli\index{Koolid!Tallinna 3. Keskkool}, kus oli sihuke suur klass 
mingisuguste masinatega, tõenäoliselt ka need olid MSX'id\index{Arvutid!Yamaha 
MSX}. Seal BASIC\index{Keeled!BASIC} oli ees ja võis midagi hakata klõpsima. 
MSX selle pärast  üldse oli pull masin, et  ta tegelikult vist mõeldi välja, et 
ühtlustada koduarvutite standardit ja BASIC'uid, mida nad jookseksid.  See 
initsiatiiv tuli vist isegi  mingisuguse Jaapani Microsofti \emph{executiv}'i 
poolt.

\question{Ta \emph{boot}'is BASIC'usse otse, eks?}

Jah. Sa põhimõtteliselt hakkasidki peale niimoodi, et esimesel ekraanil sa võid 
 kirjutada \verb|10| ja siis kirjutada ühe rea. Kirjutad \verb|20|, kirjutad 
teise rea, ütled \verb|list|, siis ta  näitab, mis sul on. Kirjutad uuesti 
\verb|20|, kirjutad selle rea üle ja kirjutad \verb|run| ja \emph{that's that}, 
ta sul kohe käib. 

Seal oli kari tegelasi, paari tükki ma tundsin ja nende kaudu ma vist kuidagi 
sain sellest klassist teada. Ma mäletan, et üks mu koolikaaslane nägi sihukest 
masinat esimest korda ja meile öeldigi, et hakake midagi tegema. No ja siis ta 
kirjutaski \verb|please draw me a circle|. Ütleme, et 
NLP\sidenote{\emph{Natural Language Processing - NLP}} ei olnud veel nii kaugel 
ja sealt midagi ei tulnud.

Tolle klassi ümber sagis palju rahvast, Karel Kannel\index[ppl]{Karel Kannel} 
seal kuidagi toimetas ja lisaks veel hulk toonaseid väikseid tattnokki. Aga 
palju huvitavam oli tegelikult see, et Jaak Loondel\index[ppl]{Loonde, Jaak} 
oli ka üks masin, nimi oli 
MIR-2\index{Arvutid!MIR-2}\sidenote{\begin{russian}МИР\end{russian} oli varane 
Nõukogude miniarvutite sari, mille kolm generatsiooni (MIR, MIR-1 ja MIR-2) 
töötati välja aastatel 1965-1969. Sarja nimi oli lühend pikemast nimest 
\begin{russian}Машина для Инженерных Расчётов\end{russian} (Inseneriarvutuste 
Masin).}, mingisugune nõukogude aparaat. See masin oli põhimõtteliselt sihuke 
pikk kapp, ikka pikem kui viis meetrit, poisikesest oli ta kõrgem, Jaagule 
võib-olla ninani. Masin tegi meeletut, sihukese villa kraasimise masina sarnast 
häält, millest põhiline tuli jahutusest. Ilge müra. Aeg-ajalt, kui inimestel 
viskas  kopa ette,  nad lülitasid masina jahutuse välja ja siis 
Jack\index[ppl]{Jack}\index[ppl]{Jack|see{Loonde, Jaak}} tuli muidugi ja 
tohutult röökis, sest too masin oleks ilma jahutuseta kokku küpsenud. 

Selles mõttes oli ta ka nagu geniaalne vana, et kust ta sellise \emph{advanced} 
masina üldse kätte oli saanud? Seal oli ikkagi võimalik klaviatuurist käske 
sisestada, kusjuures klaviatuur oli elektronkirjutusmasin, mis  põhimõtteliselt 
oli nagu klaver ja printer ühekorraga. Nii et \emph{hard copy} tuli ka samast 
masinast. Ja temaga oli ühendatud mustvalge telekas, aga tal oli ka 
valguspliiats. Ehk siis sa said nagu ekraanil tabada mingisuguseid punkte ja  
masin tundis selle ära. Ilmselt ta luges seda  kineskoobi kiirt ja selle järgi 
pani asukoha kokku. Ja ta suutis ka mingisugust rudimentaarset graafikat 
kuvada, ehk et tal ei olnud ainult tekstiekraan, vaid ta suutis ekraanile  
mingit punkti kuvada. See viimane oli arvuti jaoks tohutu ülesanne, punkt ei 
püsinud hästi paigal, õrnalt ujus, aga sai hakkama. 

See programmeerimiskeel, mida MIR-2 peal kasutati, oli vene keeles, vene 
tähtedega, kõik olid mingid lühendid, superluks, onju. Mu esimene programm, ma 
mäletan, oli mingisugune graafiline nelja-tipuline täht. Ma arvan, et ta 
koosnes ütleme siis  umbes kuueteistkümnest punktist võib-olla ja see ikka 
tõmbas selle arvuti täiesti kooma. Kõik see ekraan ujus, aga väga uhke tunne 
oli. 

Aga mida Jack\index[ppl]{Loonde, Jaak} meile veel õpetas, oli näiteks 
perfolindi lugemist. Masin sõi kahte moodi meediat. Üks oli paber, perfolint, 
siuke õhukene, mis lasti vurinal masinast läbi. Teoorias teravamad vennad, ma 
arvan, suutsid nõela või augurauaga  perfolindi peale proge torkida. Mul on 
selline tunne, et äkki ma oskasin seda kunagi. Aga aga siis olid seal veel 
mingisugused vahvad magnetkaardid. 

\question{Magnetkaart? Perfokaarti ma tean aga magnetkaardist ei ole kuulnud.}

Magnetkaart oli selline  tänapäeva  telefonist suurem, ma arvan, et mingi 
kaheksa senti korda mingi viisteist senti sihukene pruun latakas, mis 
põhimõtteliselt meenutas oma materjalilt seda, mis flopi diski sees on. Ja sa 
põhimõtteliselt panid selle mingist \emph{slot}'ist sisse, masin tõmbas selle 
surtsti läbi ja luges sealt midagi. Aga see oli nagu sihuke müstika, seda enam 
niisama lugeda ei saadud. 

\question{Noorel nagamannil peab olema ikka päris änksa tahtmine, et ekraani 
peale tähe joonistamine huvitav oleks?}

See oli  müstiliselt äge. Sa kirjutasid midagi ja see  sulle  sinna ekraanile 
tekkis pilt.

\question{See huvitav asi oli just see, et mina andsin käsu ja masin tegi 
midagi?}

Noh, täpselt, et kui see oleks nii lihtne, et \verb|please draw me a circle|, 
siis  ilmselt oleks huvi kaotanud, aga see oli ikkagi \emph{complicated} värk. 
Selles oli mingi alkeemiline element, see oli ülikõva. Väikestele poistele 
meeldivad salakeeled, koodid, lipukirjad ja igasusugused niisugused asjad. See  
oli nagu kõike seda  ja veel midagi,  see oli ikkagi super, noh!

Selge oli see, et too masin oli  meeletult piiratud, et kaua sa seal ikka 
jändad. Ja lisaks see, et seda MIR'i oli ainult üks, õnneks MSX'i klass oli 
suurem. Aga klassiga olid vist jälle mingisugused piirangud, kuna see oli 
keskkooli all. Ja ilmselt sellepärast Jack\index[ppl]{Loonde, Jaak} sebiski 
Roopa tänavale selle ÕTK\index{Tallinna Oktoobrirajooni 
Õppetootmiskombinaat}\sidenote{Täpsemalt Tallinna Oktoobrirajooni 
Õppetootmiskombinaat. Sellest asutusest on natuke rohkem juttu leheküljel 
\pageref{content!OTK}.}, kus oli siis ka terve klass. ÕTK's oli  üks niisugune 
nagu juhtarvuti, millel oli mingisugune draiv (ma eeldan, et see oli mingi 
flopidraiv) ja terve klassitäis arvuteid, mis said sellest peaarvutist omale 
asju alla laadida. Sa võisid ka lihtsalt oma programme kirjutada, aga kuna kuna 
draive oli ainult üks, siis kui sa tahtsid mida salvestada, pidi selle 
juhtarvutisse saatma. 

Kuna Jackil\index[ppl]{Loonde, Jaak} ei olnud aega seal klassis väga  hängida, 
siis möllas seal kogu aeg mingi poistekari ja tal oli paar nutikamat venda 
pandud seda vedama. Üks legendaarne tegelane oli 
Mukats\index[ppl]{Mukats|see{Edesi, Linnar}} ehk Linnar Edesi\index[ppl]{Edesi, 
Linnar}, kes täna vist kuskil Soomes toimetab. Tema oli selgelt minu esimene 
guru, keda ma nägin. Ta oli, ma arvan, umbes minuvanune, aga ta oli omandanud 
kõik need arvutiasjanduse peenemad alged. Põhiline, mida ta oskas oli  kahest 
programmijupist  ühe terviku  kokku panek ja  selle paketeerimine nii, et 
tulemust sai programmina laadida. Point oli selles, et väga suur osa softi 
levis tavalistel magnetofoni kassettidel. Ja vist oli see kuidagi nii, et 
šeffimad mängud olid 32 kilobaiti umbes pikad, noh see oli ikka \emph{massive}. 
Aga selleks, et nad kasseti peale ära mahuks, olid nad tehtud pooleks: 16+16 
kilo, sa pidid vahepeal kasseti ümber keerama. Ühesõnaga kogu see 
kassetimajandus oli keeruline. Aga kui sul oli juba nii kõva asi nagu 
flopidraiv, siis sa said selle kasseti pealt kuusteist kilo sisse lugeda, tõsta 
see kuskil mälus mujale,  lugeda teise kuusteist kilo ja selle esimese jupiga 
kokku panna. Tekkis  tervik, mida sai kuidagimoodi \emph{launch}'ida. 
Ühesõnaga, see kõik oli täielik supermaagia. 

\question{Järelikult tekkis sul niisugune teadmistel põhinev eeskuju. Keegi 
inimene, kellele sa vaatasid alt üles, sest ta teadis rohkem kui sina?}

Oo jaa, selliseid tegelasi oli veel. Üks vend, kelle nimi oli 
Kont\index[ppl]{Kont} (ma ei mäleta, mis ta eesnimi oli). Tal oli väikene 
metallkohvrikene, mille sees oli kogu MSX'i manual. See oli fotokopeeritud 
sihuke \emph{stack of paper}. Ta käis sellega väga uhkelt ringi, aegajalt tegi 
kohvri lahti ja siis midagi selle alusel kirjutas. Sihukesi tegelasi oli veel 
seal. 

Kuna sa istusid seal ilma otsese \emph{access}'ita sellele draivile, võib-olla  
mängisid või siis kirjutasid oma BASIC'ut\index{Keeled!BASIC}. Sellele 
tegevusele tulid mingid piirid ette. Loomulikult mind huvitas graafika pool, ma 
üritasin mingisuguseid pilte ekraanile manada. MSX'il\index{Arvutid!Yamaha MSX} 
 tegelikult ei olnud graafika ekraani,  seda emuleeriti tekstiekraaniga. Ehk 
siis  põhimõtteliselt iga pilt, iga mäng, mis MSX'il toona jooksis, oli 
tegelikult otse mälus tähegeneraatorei ümber programmeerimine. 

\question{Jukuga\index{Arvutid!Juku} oli sama lugu, et sa said kuskile mällu 
oma nii-öelda fondi laadida. Iga tähe asemele panid \emph{bitmap}'i ja nendest 
sai mida tahes kokku laduda}

Just, põhimõtteliselt sama laks. MSX'i ekraan oli otse aadresseeritav. Kui sa 
teadsid, et selle režiimi ekraan algas aadressil heksas 
\emph{whatever-whatever} onju, siis sa said sinna järjest kirjutada. Iga bait 
oli üks rida ja võis olla kas läbipaistev, taustavärv või esivärv ja mingites 
režiimides sai rida-realt neid värve vahetada, see oli eriti \emph{advanced}. 
Mängude puhul on niisugune mõiste, nagu sprait, ma ei tea, kas sa oled kokku 
puutunud. Need on siis graafikatükid, mis tausta ees liiguvad. MSX'il neid ka 
emuleeriti sellesama tähegeneraatoriga, seda programmeeriti jooksvalt ringi. 
Ekraan kirjutati sümboleid täis, ja siis neid kogu aeg adresseeriti ja 
kirjutati ringi. Ekraan vist oli jagatud kolmeks osaks, igas osas sa said eri 
tähestiku väänata.

Minu jaoks oligi see võlu, kui ma sain selgeks, et on olemas 
assembler\index{Keeled!Assembler}, assembleris saab ühele mälu aadressile ühe 
baidi laadida ja siis ma põhimõtteliselt veetsin suure osa oma ärkveloleku 
ajast mingisuguseid tegelasi  millimeetripaberile joonistades, neid heksaks 
tõlkides ja kuskile mällu laadides. 

\question{Kui ma nüüd tagasi peegeldan, siis ega tänapäevane arvutigraafika ei 
ole ka lihtne aga keerukus tundub olevat teises kohas. Sa pead 3D geomeetriast 
aru saama ja spetsiifilisi API'sid tundma jne.}

No täna ikkagi \emph{layers of stuff} on sinu ja pildi vahel, aga MSXil oli 
just see, et sa põhimõtteliselt sa toorelt toppisid otse ekraanile midagi. 

\question{Sa pidid ikkagi välja mõtlema selle, mida sinna mällu toppida, ja 
hoolitsema, et see värskendatud saaks ja nii edasi}

Ütleme, et seal oli igasugused trikid, et ta töötaks. Aga kuna keelestik ja 
kõik see asi oli nii lihtne, oligi tulemus super elegantne, super  lihtne.  Ma 
arvan, et ega minu progemise aeg jäigi sinna kaheksakümnendate keskele. Pärast 
ma olen võib-olla natuke HTML'i ja võib-olla CSS'i nokkinud, aga see võlu läks 
nagu üle kohe, kui asjad läksid keerulisemaks. Aga õnneks tulid tasemele 
igasugused graafikapaketid ja ja muud asjad. 

\question{Aga millal see oli? Juba keskkooli ajal?}

Olles enda jaoks kaardistanud ära kõik kohad, kus sai midagi arvutitega 
näppida, jõudsin ma läbi

Ma olin enda jaoks ju ära kaardistanud kõik kohad, kus sai midagi arvutitega 
näppida. TPI's\index{Tallinna Tehnikaülikool} oli ka üks klass, kus olid 
MSX'id\index{Arvutid!Yamaha MSX}. Aga seal oli igal masinal juba draiv taga. 
See oli ka juba super \emph{advanced} ja seal see guru staatus oli ka nagu juba 
järgmisele \emph{level}'il. Seal olid  laborandid, Aare Tali\index[ppl]{Tali, 
Aare} nimi tuleb kuidagi ette aga ma ei ole kindel, kas see on õige nimi õige 
näo juures\sidenote{Aare tegutses TTÜ-s küll ja temast on ka varasemalt juttu 
olnud (vt. lk. \pageref{sisu!aare_tali}).}. Aga igal juhul toimetasid seal juba 
üliõpilased või isegi juba \emph{post-graduate},  palju kõvemad vennad ja 
loomulikult see nagade jada seal ukse taga neid selgelt tüütas. Nad kehtestasid 
siis oma  reegleid, olid suured jumalused. Näiteks, kui info levis ja järjest 
rohkem kutte tekkis sinna värava taha,  oli vaja reglementeerida, et kes saab 
ligi ja kes mitte. Siis nad võtsid ühe kõige popima mängu, mis seal parasjagu 
oli, Kings Valley\index{Mängud!King's Valley}, trükkisid välja kogu selle 
\emph{source} koodi. See oli \emph{stack of paper}, ilusti perforeeritud ja 
niimoodi. Nad lugesid seda koodi, punase pastakaga tõmbasid ringe ja progesid 
selle mängu ringi. Minu jaoks oli see nagu jumaluse tase! Nad progesid selle 
niimoodi ringi, et nad said väikeste ise-ehitatud \emph{joystick}-idega juhtida 
selle mängu kolle. Reegel oli põhimõtteliselt see, et kui sa said nende 
jumalate vastu ühe taseme läbi siis sa said ühe päeva klassis käia. Iseenesest 
see mängufaktor  oli põnev aga just see, et nad tõesti võtsid selle mängu, mis 
minu jaoks tundus superkeeruline ja lihtsalt kirjutasid \emph{binary} ringi. 
Nad mitte lihtsalt ei teinud seal mingile tegelasele mütsi pähe,  vaid nad 
lihtsalt nagu tegidki kõik ringi, käitumine muutus. Tänapäeval muidugi tagasi 
vaadates tundub, et see kõik oli tegelikult väga lihtne.

\question{Eks see oli ju \emph{gamification}, mis praegugi populaarne on ja 
üksiti kindlustas veelgi lugupeetavate jumala-staatust poistekamba silmis}

Aga igal juhul lõppkokkuvõttes ma lõpetasin kuskil Kullos\index{Kullo}, kus oli 
ka üks klass, kus olid vist juba natukene kõvemad MSX'id\index{Arvutid!Yamaha 
MSX}. Neil oli juba  mingi graafikarežiim ja igasugused muud asjad, ehkki, kui 
sa midagi kiiresti liigutada tahtsid, pidid ikkagi kasutama tekstiekraani. Toda 
klassi majandas selline legend nagu Räni Meister\index[ppl]{Meister, Räni}, kes 
toona oli selgelt sihuke tore punkar, kes oli tulnud kuskilt Võru 
Gaasianalüsaatorite tehasest umbes ja viitsis poistega jahmerdada. Aga ta 
vaikselt seal hakkas tegelema Commodore 
Amigadega\index{Arvutid!Amiga}\sidenote{Amiga oli Commodore poolt 1985. aastal 
turule toodud personaalarvutite sari. Teistest põlvkonnakaaslastest eristas 
seda perekonda spetsiaalse graafika- ja heliriistvara lisamine ning 
väljatõrjuva mitmetegumilisuse realiseerinud AmigaOS.}, mis oli juba sihukene 
\emph{super advanced} raud. 

Kuidagi mahtus see kõik videotootmise ja selliste asjade tähe alla, tänu 
sellele ta oli ka loomulikult kaardistanud, et kus niisugune asi veel toimub. 
Eesti Televisioon\index{Eesti Rahvusringhääling!Eesti Televisioon} oli selgelt 
üks ja veel oli mingisuguste vene metalliärikate turundusharu. Ilmselt keegi 
vend oli piisavalt palju lobi teinud ning kuskil Kristiines  keldris püsti 
pannud väikse nii-öelda reklaamistuudio kus ta siis tootis värki ja tal oli 
seal ka üks Amiga\index{Arvutid!Amiga}. 

\question{See pidi siis olema üheksakümnendate algus juba, eks?}

Jah, kuskil sealkandis. Kullos me hakkasime ka juba mingeid mänge tegema ja 
nii. Markus Klessman\index[ppl]{Klessman, Margus} toimetas seal näiteks. Raul 
Keller\index[ppl]{Keller, Raul}, kelle aka oli 
\enquote{Killer}\index[ppl]{Killer|see{Keller, Raul}}, üritas MSX-i mänge vist 
kuidagi publitseerida, aga see (vähemalt mulle ja toona) tundus kuidagi väga 
kahtlane ja naiivne tegevus.

Aga siis juba Räni\index[ppl]{Meister, Räni}, kuidagi nähes minus potentsiaali, 
meelitas mu Eesti Televisiooni\index{Eesti Rahvusringhääling!Eesti 
Televisioon}. Põhimõtteliselt ma ei olnud isegi veel keska viimases klassis, 
kui ma töötasin juba Aktuaalses Kaameras. Uudistetoimetuse kõrval oli sihuke 
väike kubrik, kus me siis tegime Aktuaalse Kaamera infonurki, mis olid  diktori 
taga seina peal. Ja kuna Amiga\index{Arvutid!Amiga} oli selline tore masin, et  
sinna sai lasta videosignaali sisse ja sealt tuli videosignaal välja, sai 
temaga juba toona digimiksi teha. 

\question{Ossa, see oli PC peal jõhkralt kallis riistvara toona}

Ongi. Miks need Amigad siinkandis selles vallas levisid, oli just see, et PC 
jaoks selliste võimalustega videokaart oli Hollywoodi hinnaga asi. Ja PC-del 
oli enamasti, mingisugune CGA ja neli värvi onju, samas kui 
Amiga\index{Arvutid!Amiga} oli \emph{full video}. Põhimõtteliselt polnud sul 
vaja isegi arvuti monitori, sa võisid talle teleka järele panna. Ta oli ju  
kodukodutarbimisest arenenud. 

Seal tegime oma ilmakaarte ja panime videopilte sinna ja põhilise osa ajast 
muidugi mängisime arvutimänge, sest  Amigal olid superšefid mängud. 

\question{Aga mis tarkvaraga te tegite seda kõike? Ega te ju nullist ei 
kirjutanud kogu seda kraami?}

Olid olemas täitsa viisakad graafikapaketid, Deluxe Paint\index{Deluxe 
Paint}\sidenote{Deluxe Paint on rastergraafika redaktorite sari, mille lõi 
Electronic Arts'i jaoks Dan Silva. Programm alustas elu majasisese 
graafikaprogrammina, kuid sai pärast avaldamist \emph{de facto} standardiks 
Amiga platvormil.} on üks šefimaid graafikasofte, mis oli igasugustest 
Photoshoppidest ja kurat teab millest ikka kümme aastat ees. 

Me olime nagu sellised \emph{in-the-know} amiga-vennad ja vaatasime  kõikide 
PC-de  ja muude vendade peale ikka väga ülevalt alla, sest et nad ikka ei 
teadnud, milles nad seal sorkisid, onju. Paraku see Amiga bisness oli kehva ja 
läks lõpuks nurja, aga tehnika iseenesest oli superäge. 

Meil tekkis mingi väikese punt tegelastest, kellel kas oli kodus Amiga või kes 
nendega kuidagi tegelesid. Näiteks Martin Rinne\index[ppl]{Rinne, Martin}, kes 
täna teeb Directot\index{Directo} tema juba kuidagi tekkis  sinna telesse ja 
siis Margus Kliimask\index[ppl]{Kliimask, Margus}, kes tegeles Eesti 
Videos\index{Eesti Video} Siilatsi mingite asjadega ja Mati 
Veermets\index[ppl]{Veermets, Mati} kellest pärast sai Tallinna linna disainer. 
 Kõigil oli nagu mingisugune \emph{access} Amiga-te juurde. 

Jällegi loomulikult seal ka võlus mind pigem see, et sa tekitasid mingisuguse 
elava pildi, sul ei  pidanud olema kaamerad ja näitlejad ja nii edasi, vaid sa 
võisid teha mingeid väikseid animatsioone otse arvutis teha.

\question{Isegi animatsiooni ta vedas välja?}

Noh, selles mõttes, et sa said seda teha põhimõtteliselt \emph{stop 
motion}'iga. Ütleme,  reaalajas animatsioon kippus ikka nõgisema juba, kuigi me 
tegelikult ikka telepäid tegime reaalajas ka, sest keegi ei viitsinud 
\emph{stop motion}'iga lasta. Aga minu meelest ikka enamus sellised asju käisid 
reaalajas. Deluxe Paint-is\index{Deluxe Paint} olid sisse ehitatud igasugused 
nutikad asjad. Näiteks  liikumise aeglustamine või kiirendamine. Sa andsid 
talle põhimõtteliset ette, et siin on sulle kast, nüüd see kast peab liikuma 
mingisuguse viiekümne kaadriga siia, siis ta automaatselt täitis need 
viiskümmend kaadrit ära. Ja vajadusel, kui sa ütlesid, et \emph{ease in}, siis 
ta tõmbas lõpus hoo maha ja kõik oli väga \emph{fine}.  

Ma mäletan, kui ma alles läksin sinna telesse (selle järgi võib muidugi aasta 
paika panna), me tegime Öölaulupeole\index{Öölaulupidu}\sidenote{Esimene 
Öölaulupidu toimus 1987. aasta juunis Tallinna Vanalinna päevade ajal, aga 
toona meediakajastus puudus ja üritus toimus spontaanselt. 1988. aastal oli 
Öölaulupidu juba ametlikult Vanalinna päevade programmi lülitatud.}  
mingisuguseid valgusklippe, see oli jällegi \emph{super advanced}.

\question{Võru poisina ma Öölaulupidudest ei tea midagi, aga 
Öötelevisioonil\sidenote{1990. aastal aset leidnud omas ajas mitmes mõttes 
innovatiivne teleprojekt, mille käigus Eesti Televisioon\index{Eesti 
Rahvusringhääling!Eesti Televisioon} öö läbi katkematult otse-eetris oli.} oli 
väga äge graafika}

Jajah, see oli ka meie tehtud. Tegelikult oli kogu tele sihuksesed asjad meie 
rida, sest  alternatiiv oli tiitrimasin, mis oli mingi räme pool-analoog pult 
ja mis  eriti koledat jälge tootis. Meil oli ikka \emph{super-advanced} 
animatsioonidega ja värviline, sai teha mida iganes. Vahest tegime mingit 
haltuurat mingite reklaamide jaoks ja igasugu lollusi sai tehtud.

\question{Kas see oli puhas ise-õppimise värk või hakkas kusagilt mingit 
informatsiooni ka juba tulema?}

Ei, see oli ikka puhas iseõppimise teema. Need vahendid olid suhteliselt 
piiratud ja ega seal midagi väga keerulist ei olnud. Kunagi hiljem tulid ka 
esimesed 3D paketid, nendega sai pusserdatud. Tase oli nendega ikka hoopis 
teine, sa pidid ikkagi punkt punkti haaval mingisuguseid pindu konstrueerima ja 
siis nendega kuidagi opereerima. Tänapäeval vaatad, kuidas väänatakse mingeid 
\emph{bump mapping}-uid\sidenote{Arvutigraafika tehnika, mille abil 
kolmemõõtmelise objekti pinnale simuleeritakse kühme ja kortse. Lihtsalt 
öeldes, oranžist kerast tehakse usutava väljanägemisega apelsin.} ja 
mingisuguseid asju \emph{layer}-ite kaupa ja see kõik annab kuidagi tulemuse,  
see on täiesti müstika. 

\question{Mis sa siis tegid, kui sa Eesti Televisioonis\index{Eesti 
Rahvusringhääling!Eesti Televisioon} enam ei olnud?}

Kuidagi tundus, et see videograafika oli  väga põnev, aga hakkas tekkima 
mingisugune \emph{business}. Sõbrad, kes kuidagi olid rohkem sattunud 
trükigraafika peale, kes  kujundas Eesti Ekspressi ja kes seal tegi mida, see 
tundus kuidagi nagu rohkem \emph{business}. Kuidagi ma sain aru, et, ahah, 
videot teeme Amigaga aga selle selle \emph{business}'i tarvis peaks ennast 
kuidagi PC-de peale  sebima. Sealsamas telemajas kuidagi tekkisid 
potentsiaalsed kliendid ja ma pidin hakkama tootma trüki-kõlbulikku kujundust. 
Ma ei olnud  kunagi näinud sellist programmi nagu Corel Draw\index{Corel Draw} 
aga mul oli see töö vaja  ära teha ja ma istusin öö läbi ja tegin ta endale 
selgeks. Mis oli tohutult frustreeriv, sest ta oli täiesti teine maailm. 
Tänapäeval on nii, et sa joonistad ja su joonistatud pilt on  ekraanil. Siis 
oli niimoodi, et sa  konstrueerisid mustvalgelt mingisuguse \emph{vector 
mesh}'i, panid sinna mingid värvid peale, vaatasid \emph{preview}-d ja siis ta 
joonistas sulle selle pildi aeglaselt ette. Ja alles siis sa läksid uuesti 
selle pildi kallale.

\question{Kuidas sul Corel Draw õppimine välja tuli, sest minu mälestuste järgi 
ta ei olnud kuigi töökindel: aegajalt tegi faile katki ja nii?}

Olles kasvanud nende arvutitega üles, sa arvestasid ju, et nad aeg-ajalt 
jooksid kokku ja aeg-ajalt nad tegid rumalusi. Aga võib-olla siin mängis natuke 
rolli ka see  poisikesepõlves ÕTK's\index{Tallinna Oktoobrirajooni 
Õppetootmiskombinaat} õpitud arvutist üleolek läbi ühe lihtsa fakti. 
MSX-il\index{Arvutid!Yamaha MSX} oli paremas nurgas port, mille sisse käis kas 
siis kettaseade või mingi mälu \emph{cartridge}. See oli sihuke päris suur 
sahtel. Selleks, et mitte \emph{cartridge} sisse lükkamise hetkel midagi tuksi 
keerata, oli sahtli sees üks väike lüliti, mis tegi masinale reseti. Ja 
loomulikult õpiti kiirelt ära, et kui sa oled midagi tuksi keeranud, näiteks 
olid kirjutanud programmi, mis jäi tsüklisse, siis selle asemel, et voolu välja 
võtta, panid kohe näpud sinna auku ja masin oli surnud. Alati sa teadsid, et 
mingi valemiga sa saad temast jagu. See teadmine on olnud minuga siiani, et kui 
ma kuskilt seinast ikka lõpuks juhtme kätte saan, siis on ta surnud. Ma ei pea 
teda pelgama.

\question{Selle koha peal ma pean järgi andma kihule ja ära küsima küsimused, 
mida ma väga tahan küsida. Me jõuame Microlinki\index{Microlink} ja 
.EXE-ni\index{.EXE}. Kuidas sa nende juurde jõudsid?}

Kui ma olin juba selle prindiga alustanud, siis ma vahepeal kuidagi sattusin 
mingisse niisugusesse maailma, kus ma peamiselt sellega tegelesingi. Ma olin 
Margusega\index[ppl]{Kliimask, Margus}\sidenote{Kaspar peab ilmselt silmas 
Margus Kliimaskit} televisioonis varem suhelnud ja tema omakorda suhtles 
sellise tegelasega nagu Lõvi\index[ppl]{Lõvi}. Lõvi on muidugi kõige olulisem 
tegelane üldse, tema juurest ilmselt algab kogu Eesti arvuti \emph{business}. 
Kui Jaak Loondest\index[ppl]{Loonde, Jaak} algab kogu Eesti arvutiteadvus, siis 
ma arvan, et Lõvist algab kogu arvuti-\emph{business} kuigi ta ise pole vist 
bisnest kunagi teinud. No ja Rainer Nõlvak\index[ppl]{Nõlvak, Rainer} ja kõik 
see nagu klikkis kokku. Ma saan aru, et Rainer oli Margusele teinud ettepaneku 
toimetada mingisugust ajakirja. Tema siis võttis mul varrukast kinni ja ütles, 
et \enquote{davai, nüüd on vaja ajakirja teha}. Mina muidugi pigem oleks 
mänginud arvutimänge, nagu ma olin teles harjunud, kus ikkagi üheksakümmend 
protsenti meie tegevusest oli arvutimängude mängimine. Aga noh, ma sain omale 
väga korrektse 486-e, ma arvan, ja selle peal jooksis Ultima 
Underworld\index{Mängud!Ultima Underworld}\sidenote{1992. aastal Blue Sky 
Productions'i poolt üllitatud Ultima Underworld oli väga mitmes mõttes 
(kolmemõõtmeline keskkond, simuleeritud mittelineaarne mängu käik jne.) 
teedrajav rollimäng.} ja oli täitsa tore. 

Toimetustegevusega mitte tuttava inimesena ma mõtlesin, et  alustama peaks 
ikkagi ajakirja esikaanest. Ja siis ma Corel Draw-s seda esikaant  hiirega 
joonistasin. Praegu tagantjärgi mõeldes mulle tundub, et ma joonistasin seda 
kuude kaupa. Tõenäoliselt see nii ei olnud aga sinna läks tohutu aur. 
Pronto\index[ppl]{Pronto} luges kokku, et neid numbreid nii väga palju ei olnud 
ja nendega läks suhteliselt palju aega. Ja kuna tegu polnud nagu otseselt ka 
äriline ettevõtmisega vaid .EXE oligi pigem promo, siis keegi nagu väga ei 
survestanud seda aja poolt ka. Meil ei olnud kohustust, meil ei olnud 
tellijaid, et ta peab nüüd iga kuu ilmuma.

\question{Aga kuskil Võrus istus üks nohik, kes kurvastas, et \enquote{miks ei 
ole tulnud veel .EXE't}!}

No vot, me ei adunud, et meil on \emph{impact}.

\question{Mõju oli kindlasti olemas. Võin omal näitel kinnitada, ja et ta 
Pronto panduna praegu niimoodi Internetis on\sidenote{\url{punktexe.ee}}, on ka 
selgesti märk mõjukusest. Seetõttu ma ka küsin.}

Ta oli oluline igas plaanis. Olles selles asjas sees, siis minu jaoks ei olnud  
küsimus, et kas arvutid tulevad maailma muutma. Ma isegi ei mõelnud sellele, 
nendega oli lihtsalt hea asju teha ja  tõenäoliselt inimesed, kes ei teinud, 
olid ikka täiesti rumalad. Kõrvalt vaadates ma isegi ei saanud aru, kuivõrd 
vähe tegelikult arvuteid toona kasutati, sest me istusime MicroLinki 
peakontoris ja seal käis kogu aeg mingisugune sebilung. Telemajas ja igal pool, 
mul oli \emph{access} arvutitele päris hea. Aga ma mäletan, et .EXE  esimeses 
numbris  oli arhitekt Kalle Rõõmuse\index[ppl]{Rõõmus, Kalle} büroo niisugune 
väikene tutvustus  läbi selle, et nad hakkasid kasutama arvuteid 
projekteerimisel. See oli see midagi täiesti epohhiloovat. Ja ma isegi toona ei 
saanud sellest aru, kuivõrd imelik see  üldse on, et keegi teeb  paberil 
midagi. Ega  ma üldiselt  laksisin artiklid paika ja panin pildid külge ja mind 
võib olla väga ei huvitanudki, mis seal kirjas oli, välja arvatud need, mis ma 
ise kirjutasin. Aga tolles Kalle Rõõmuse büroo artiklis kirjutati, et üks 
arhitekt käis Kanadas stažeerimas. Kanadas tegeleti aga just sellega, et osteti 
personaalarvutid ja töö muutus efektiivsemaks võrreldes sellega, kui arhitektid 
ja konstruktorid päevad läbi kalka peale midagi joonistasid.  Järsku panid 
selle kõik arvutisse ja kõik oli nagu hästi. Tegelikult on huvitav vaadata 
seda, et tänaseks on meil see nii-öelda 
BIM-modelleerimine\sidenote{\emph{Building Information Modeling - BIM.} 
Protsess, mille käigus füüsilisi ruume käsitletakse digitaalsete vahenditega.} 
ja siis sa kuuled, mis väljakutsed sellega seoses on. Mul üks sõber töötab 
startupis, mis tegeleb BIM-mudelite konfliktide analüüsiga. Üritavad aru saada, 
et näiteks ventilatsioonitoru ei tohi läbi akna minna. Ja siis sa mõtled, et 
\enquote{issand jumal, millega need inimesed on tegelenud, miks nad seda 
arvutit pole varem kasutusele võtnud?}. Kui palju on aega raisatud!

.EXE\index{.EXE}, olgugi, et temast jäi mulje, et ta on ikka \emph{super 
advanced} ja mingi häkkerite värk, üritas anda pilti sellest, mis tegelikult 
toimub. Et arvuti ei ole ainult raamatupidaja kalkulaator. 

\question{Kuidas sa joonistamise juurest kirjutamise juurde jõudsid?}

Oli vaja ju \emph{content}'i toota ja ega keegi toona ei olnud 
arvutiajakirjanik. Ja  mulle meeldis arvutimänge mängida ja ma arvan, et 
kirjutamine on iseenesest tore tegevus. 

\question{Kas sul juba kooli ajal lõi kirjutamise ja kirjandi soon kuidagi 
välja?}

Ei, ma olen võimeline kirjutama okeilt. Mulle joonistada meeldib võib-olla 
rohkem, sest kirjutamine on selline raske asi, et sa pead laused läbi mõtlema 
ja siis sulle tundub, et nad ei ole head. On nagu liiga
konkreetne formaat.

\question{Teema jätkuks veel üks oluline küsimus. Mõni aeg tagasi Tõnis Kahu 
(keda tuleb ilmselt uskuda)\index[ppl]{Kahu, Tõnis} lükkas ümber mu arusaama 
sellest, misasi on küberpunk. Aga minu vastav arusaam tuleb ühest konkreetsest 
.EXE artiklist, kus on sinu ja Pronto\index[ppl]{Pronto} nimed 
all\sidenote[][-5cm]{Kaspar Loit. (1994). Kes sa selline oled, küberpunk? .EXE, 
(3), 60-63. Kontrollides Pronto nime artikli juurest ei leia. Küll aga leiab 
sedastuse, et \enquote{\ldots õiged küberpungid lasevad selliste loetelude 
peale suht laias kaares}.}. Räägi nüüd ära, kuidas te tolle sisu 
produtseerisite}

Väga raske öelda tagantjärgi. Aga eks meil oli mingi ettekujutus. Ega küberpunk 
ei ole mingisugune geneetiline  organism, mis on välja arenenud ja siis pärast 
on hea klassifitseerida, et  pool on hüljes ja pool on mingisugune gepard. Ma 
eeldan, et me toona juba teadsime juba Gibsoni 
\enquote{Neuromancer}'it\sidenote[][-5.5cm]{\enquote{Neuromancer} on William 
Gibsoni 1984. aastal ilmunud romaan, esimene tema \emph{sprawl}'i triloogiast. 
Romaani loetakse \v{z}anri üheks mõjukamaks ning on ainus romaan, mis on 
võitnud nii Nebula, Hugo kui ka Philip K. Dick'i auhinna.}. Kui kõik räägivad 
Hichiker Guide'st\sidenote[][-3cm]{Douglas Adams, \enquote{The Hitchhiker's 
Guide to the Galaxy}. 1978. aastal  raadiokuuldemänguna alustanud komöödia, mis 
avaldati viieosalise raamatutriloogiana ning millele kuuenda osa lisas pärast 
autori surma avaldamata materjali põhjal Eoin Colfer. Sarja raamatud levisid 
tekstifailidena laialt BBSide ja interneti vahendusel olles seega ka siinmail 
kergesti kättesaadavad.}, siis see oli väga oluline teos. Aga noh, minu jaoks 
Gibsoni \enquote{Burning Chrome}\sidenote{\enquote{Burning Chrome} on William 
Gibsoni 1982. aastal ilmunud novell, kus tutvustatakse \emph{sprawl}'i 
triloogia maailma ja mille sündmusi ning tegelasi mainitakse triloogias 
korduvalt.} ja \enquote{Neuromancer} lasid ikkagi aju täiesti välja.

\question{Ma siiamaani loen neid asju regulaarselt üle. Härra Gibson kirjutas 
need raamatud trükimasinaga paberi peale ja aastal 2019 täpselt nii ongi!}

Oled sa tema uuemaid raamatuid ka lugenud? Need lähevad veel hirmuäratavamalt 
tõepärasemaks  ja ajahorisont tuleb üha lähemale.

\question{Siit siis loogiline küsimus, et pidi ju olema mingi allikas, te ei 
mõelnud ju küberpungi mõistet (mida Gibson ei maini) ise välja? Olid teil 
välismaa BBS-id, internet?}

Ma arvan, et kõik see nimetatu klikkis kuidagi kokku, tõenäoliselt. Ma ei oska 
Pronto\index[ppl]{Pronto} eest rääkida, aga \enquote{Blade 
Runner}\sidenote{\enquote{Blade Runner} on 1982. aastal linastunud Ridley 
Scott'i film, milles peaosa mängib Harrison Ford ja unustamatu improviseeritud 
lõpumonoloogi esitab Rutger Hauer. Film toetub lõdvalt  Philip K. Dick'i 1968. 
aasta, samuti klassikaks peetavale, novellile \enquote{Do Androids Dream of 
Electric Sheep?}.} on, eksole, eepiline nurgakivi, Syd Mead\sidenote{Sydney Jay Mead oli USA tööstusdisainer ja kunstnik, kes töötas lisaks Blade Runnerile välja ka selliste 
filmide nagu Aliens (1986) ja Tron (1982) visuaalid.} oli  see futuroloog, kes 
joonistas ilusaid düstoopilisi pilte .See kõik kujundas meil välja mingisuguse düstoopilise 
arusaama tehnilisest maailmast, kus kõik on kõige külge ühendatav. Kasvõi 
\enquote{Battle Angel Alita}\sidenote{\enquote{Alita: Battle Angel} on 2019. 
aastal linastunud  Robert Rodriguez'i film, mis tugineb Jaapani mangakunstniku 
Yukito Kishiro 1990. aastate sarjal \enquote{Battle Angel Alita}}, mis täitsa 
juhuslikult praegu kinno jõudis, onju. Ma arvan, et vähesed inimesed Eestis 
teavad seda originaallugu, ma olin selle toonane totaalfänn. Ma käisin 
aeg-ajalt Helsingis Akadeemilises Kirjakauppa's\sidenote{
Akateeminen Kirjakauppa on Helsingi kesklinnas asuv kuulsusrikas raamatupood, 
mis just kõikvõimalike servapealsete huvidega eestlasi pikki aastaid 
raamatutega varustas. Palverännak sellesse poodi oli ka minu Helsingi-käikude 
lahutamatu osa.} ja seal kogu aeg vaatasin, et kas uus osa on tulnud. Üheksa 
raamatut, mul on nad kõik olemas. Kuidas mingid metalltorud lähevad su 
silmamuna sisse ja ajust on  järgi ainult mingisugused \emph{chip}'id ja 
natukene pudru. See kuidagi kujundas meid, me elasime selle asja sees, ma 
arvan. Mängumaailmas tõenäoliselt olid paar mängu jälle, mis kuidagi sinna 
kontributeerisid. 

Pluss on see, et me muidugi üritasime  siis ka mänge teha. Enne veel, kui me 
Bluemoon'iga\index{Bluemoon} \sidenote{Kaspar peab ilmsesti silmas Bluemoon 
Interactive nimelist ettevõtet, millest on juttu leheküljel 
\pageref{sisu!bluemoon}, ja mitte samanimelist Londoni ööklubi (mille ees Ahtit 
ja Jaani korduvalt pildistatud on)} midagi tegime üritasime Amiga maailmas  Ott 
Aaloe\index[ppl]{Aaloe, Ott} ja Juhan Soonetsaga\index[ppl]{Soomets, Juhan} 
midagi korda saata. Tegime Rocketsi-nimelise\index{Mängud!Rockets} mängu, mille 
Bluemoon pärast keeras PC peale palju ägedamana, aga võib-olla vähem ägedanda. 
Selle mängu intro, ma mäletan, oli väga selgelt kantud kõigist neist vee 
kaheksatest ja rakettidest ja nii edasi. Väga tähtis oli kindlasti see, et 
päikseprillid olid õige kujuga. Andrus Aaslaid\index[ppl]{Aaslaid, Andrus} 
sinna kõrvale rääkis või kirjutas lugusid, kuidas plinkiva valgusega saab sul 
aju ümber programmeerida. See kõik nagu absorbeerus ja tekitas mingisuguse 
omaette alternatiivse reaalsuse, ma arvan, et see on see meie arusaam 
küberpungist. 

\question{Kui teie tuleviku ettekujutuses oli ajust järel natuke putru ja palju 
kiipe ja tulevik oli muidu ka düstoopiline, siis miks te sellele vaatamata pika 
sammuga tolle tuleviku suunas astusite?}

No aga seda tagasi hoida on tõenäoliselt mõttetu, ludiidid ka üritasid. Aga 
parem on olla kohal enne teisi. Et sa paned juba õiged \emph{chipid} omale 
õigesse kohta  ära ja võtad selle pudru osakaalu väiksemaks. 

\question{Et üheksakümnendatel selliseid mõtteid mõelda, oli ikka korralikku 
visiooni tarvis. Aga räägime Bluemoonist: kuidas sa Ahti\index[ppl]{Heinla, 
Ahti}, Jaani\index[ppl]{Tallinn, Jaan} ja tolle pundiga kokku sattusid?}

Toona igaüks mõtles, et ta on super-häkker. Nii see, kellel oli kastis see 
MSX-i manual kui see, kes oskas neid faile kokku panna. Ma olin kogu selle 
niiöelda super-häkkerite seltskonnas üks väheseid tegelasi, kes joonistas 
pilte. Ma  tegelikult oskan ka ilma arvutita päris hästi joonistada, arvutis 
tundus see lihtsalt  kuidagi nagu lahedam, seal sai asju salvestada. Ja sai 
\emph{undo} teha, see on nagu põhiline. Kui sa lihtsalt joonistad, siis 
\emph{undo} teha on väga raske, isegi võiks öelda, et peaaegu võimatu. Jällegi, 
see seltskond ei olnud nagu nii suur ja nad kõik nagu \emph{connect}'isid 
kuidagi. 

Teen korra kiire kõrvalehüpe. Tuli meelde, kuidas BBS-ide ja värkidega suheldi, 
et meil sealsamas Telemajas oli täpselt samasugune ambitsioon. Meil olid Amigad 
aga mänge ju ei olnud, siis pidi neid mänge kuskilt pirama. Ma loodan, et 
tagantjärgi mingid piramisasjatundjad ei hakka peale lendama. Aga igal juhul 
üks viis neid mänge saada oligi see, et sa pidid jõudma kuskile mingisugusesse 
BBS-i ja kuidagi sinna sisse pääsema. Ega seal ei olnud niimoodi, et 
\enquote{astu sisse}. Seal tavaliselt istusid mingid vennad, kes monitoorisid 
tegevust. Eesti tundus eksklusiivne, sihukene veider koht, sama hea kui 
eskimod. Mingi hetk meil isegi tekkis mingisugune \emph{trading capacity}, et 
meil juba oli midagi, mida  vastu pakkuda. Aga tavaliselt me ikka mängisime 
sellist vaest sugulast ja siis me isegi nagu 
\emph{bluebox}'isime\sidenote{Ennevanasti liikusid instruktsioonid 
telefonikeskjaamadele konkreetse kõne kohta samas kanalis, kui kõne ise. Seega, 
tõstes toru ja vilistades  õigeid signaale, võis muuta kõne teekonda 
keskjaamade vahel ja, mis kõige olulisem, saada mööda kõnetasudest. Kõvemad 
spetsialistid, nagu Joe \enquote{Joybubbles} Engressia suutsid kaugekõneliini 
lähtestamiseks vajalikku 2600 Hz signaali suuga vilistada. Natuke nõrgemad, 
nagu John \enquote{Captain Crunch} Draper, vajasid tehnilisi vahendeid. 
Lihtsurelikud aga kasutasid elektroonilisi seadmeid. Neist esimene, mille 
ehitas 1960. aastal Robert Barclay, oli pakendatud sinisesse kesta, sealt ka 
mõiste \enquote{blueboxing}.} ennast sinna sisse. 

\question{Oot, räägime nüüd sellest lähemalt. Kas see tähendas seda, et sa 
pidid toonasele telefonikeskjaamale kõrva vilistama midagi, mida too tingimata 
kuulda ei tahtnud?}

Kusjuures nüüd, kui ma hakkan mõtlema, siis Margus\index[ppl]{Kliimask, Margus} 
oli põhiline \emph{bluebox}'i spetsialist, aga kas me ka reaalselt 
\emph{bluebox}-imiseni ka jõudsime, see on nüüd hea küsimus.

\question{Ma mäletan, .EXE's ilmus manuaal selle kohta\sidenote{Mark Tabas. 
(1993). Blueboxing parimates peredes. .EXE, 1-2. Tegu on nähtavasti 
tõlkeartikliga, Mark Tabas oli legendaarse häkkerirühmituse Legion of Doom 
asutajaliige}, mis oli jube huvitav lugeda}

Jaa. Põhimõtteliselt see on ju iseenesest üsna lihtne, kuna toona nood 
keskjaamad olid suhteliselt rumalad.  Võtame kasvõi selle, kui kiired olid 
modemid. Ma mäletan, et oli kolmesajane Hayes. Ja seda, et minu meelest kas 
Mast\index[ppl]{Kaal, Madis} või keegi, oli väidetavasti suuteline selle modemi 
\emph{handshake}-i ära vilistama sellele modemile. Ta  oli  piisavalt aeglane, 
talle sai põhimõtteliselt suusõnaliselt selgeks teha, mis sa tahtsid.

Võib-olla oli see, et igaühel meist oli oma fookus, on ju. Kes tahtis rohkem 
seal mingit \emph{network}'i häkkida, kes tahtis rohkem lihtsalt häkkida, kes 
tahtis progeda. Mind huvitasid selgelt mängud, liikuvad pildid, värvilised 
pildid, kuidas neid ise teha, 3D, kõik niisugune värk. Ma pigem nagu otsisin 
neid võimalusi. Eks see viis meid ka tegelikult kokku siis lõpuks 
Bluemoon'i\index{Bluemoon} pundiga, kellel oli  kindel soov, et nad tahavad 
teha mängu. Minu jaoks oli see lihtsalt natukene niisugune nõme ülesanne, kuna 
mul oli Amiga\index{Arvutid!Amiga}, seal olid miljonid värvid ja neil oli 
mingisugune EGA (alguses vist isegi CGA) ja sinna pidi mingi nelja värviga 
midagi valmis nikerdama. \emph{Why not}, teeme ära. Sellest sündis siis 
Kosmonaut\index{Mängud!Kosmonaut}. 

See, kuidas nad turustasid ja toimetasid, ma nagu lihtsalt vaatasin ja 
imestasin. Jube lahe oli. Mul on meeles kuivõrd \emph{dedicated} need vennad 
toona olid ja on vist siiamaani. Nad olid nagu tõesti  \emph{focused} oma asja 
tegemisele. 

Aga see graafiline pool oli super lihtne. Nokkisin selle valmis, siis nad tegid 
oma selle musa \emph{editor}-i, sinna ma nokkisin ka mingisugused ikoonid, 
mingid kitarred ja  trummid ja väga lahe oli.

\question{Seda ma mäletan küll, et inimesi, kes oskasid arvutiga joonistada, 
oli vähe. Kas sul kõigepealt tuli arvutiga joonistamine ja siis joonistamine 
või oli sul enne ka joonistamise huvi?}

Ma ikka enne joonistasin ka. Kes ikka ennast kiidab, kui mitte ise, eks ole. 
Akadeemilist joonistamist ma valdan  suhteliselt väga hästi. Selles mõttes mul 
ei olnud nagu keeruline neid oskusi omandada. Toona ei joonistatud tabletitega, 
vaid oli seesama munaga hiir, mille muna aeg-ajalt jooksis mingit pahna täis ja 
sa pidid teda küünega puhastama. Ma ikkagi endale muidugi otsisin mingi hiire, 
mis nagu enam-vähem jooksis. Selles mõttes minu jaoks ei ole vahet, kas see on 
pliiats või hiir või tablet või mis iganes. 

\question{Arvuti oli sinu kunsti tegemise nii-öelda laiendus}

Jah, ta oli lihtsalt nagu mingi teistsugune tehnika ja tunduvalt andeksandvam, 
kui näiteks akvarell. Täna sa vaatad, et  kõik kunstnikud kasutavad mingit 
Cintiq'u tabletti, neil on kõik super ägedad toolid. Ma siiamaani aeg-ajalt, 
kui mul on vaja midagi nikerdada kasutan oma läppari \emph{touchpad}'i ja kõik 
vaatavad, et ma olen peast soe. See on mugav ja käe järgi \emph{tool} 
tegelikult, kui sa ta ära omandad ja keerad kursori piisavalt kiireks.

\question{Üks hetk sul tekkis mõte, et võiks hakata veebi tegema?}
 
See oli pigem läbi selle, et ma olin aru saanud, et ma ei ole piisavalt 
järjepidev ja see progemise osa  tundus toona liiga kuiv. Mul olid sõbrad, kes 
sellega tegelesid ja tulemus ei olnud nagu seksikas. Veeb oli alguses ka super 
\emph{boring}, see Mosaic, või mis see esimene brauser oli, oli ikka \emph{ugly 
as hell}. Aga siis kui ma sain aru, et tabelitel saab keerata ported maha ja  
mingite üksikute ühe-piksliste tükkidega hakata mingit \emph{layout}'i tegema,  
ma olin müüdud mees. Ma sain mitte lihtsalt enam selle pildi oma käe seest 
ekraanile, vaid ma sain selle pildi nagu pauh kõigile nina ette, eks ole.  
Töötasin ühes reklaamibüroos ja midagi  seal katsetasin. 
Mindworks\index{Mindworks} oli juba olemas ja selle asutajad siis vaatasid, et 
ma jagan natuke sellest reklaami-bisnesist ka, et paneme seljad kokku. Aasta 
oli 1996 või 1997, me olime umbes kakskümmend viis, me polnud enam niiväga 
poisikesed ja sai juba bisnest teha. Lõikasid neid pikseleid \ldots Ma mäletan, 
meil oli selline klient nagu Reval Hotel Group, mingid nagad tulid ja võtsid 
sihukesed kliendid ja tegid neile ägedaid asju.

\question{Ja jätkuvalt oli sind liigutav faktor see, et sa said oma pildi 
inimestele silma ette?}

Tegelikult mul ei oleks isegi vahet, kas inimestele silma ette, vaid just 
nimelt see, et sul oli mingisugune distsipliin, HTML. Ja sa teadsid, kuidas 
optimeerida GIF-e, sul oli mingisugune \emph{toolset} ja sa valdasid seda 
suhteliselt hästi. Tekitas rõõmu, et  sai teha mingisuguseid asju, mida 
võib-olla teised ei osanud teha. Sihuke \emph{job satisfaction}'i värk. Ma 
kujutan ette, et muru niitmine on ka selles mõttes lahe, et sa näed, kuidas 
niidetud muru jääb su taga maha. Sihuke suhteliselt \emph{instant 
gratification}.

\question{Mis sa praegu teed?}

Ma olen kuidagi liigselt distantseerunud sellest disaineri rollist aga samas 
mitte. Ajapikku ma olen aru saanud, et selle pildi tegemine on mõnes mõttes 
niisugune käsitöölise töö. Tegelikult need lõikelauad, mida minu lapsepõlves 
turul müüdi, kus  põletiga oli tehtud \begin{russian}ну погоди\end{russian} 
peale, on ju nagu veidi sarnane. Palju  šeffim on tegelikult võtta ja aru saada 
mingitest äriprotsessidest või mingisugusest inimeste mõttemallidest ja 
disainida neist midagi. Progemine minu jaoks on see, et kui asi õigesti 
sõnastada, siis mingisugused vennad teevad selle valmis ja see muutub nagu 
päriseks, see  protsessi toetav  asi seal masina sees toimetab täpselt nii nagu 
sa oled talle  öelnud, et toimeta. 

\question{Nüüd mitte ei tule käe seest pilt vaid tuleb sinu pea seest mõte, 
kuidas see kupatus võiks käia, programmeerijad teevad selle valmis ja siis 
käibki nii}

Just. Lapsepõlves kuidagi, ma mäletan selgelt, mind võlus mõte, et tehas on 
tore asi, sest et ta võtab mingisuguse toorme ja detailid ja siis neist 
pannakse kokku mingi asi. Ja jällegi me jõuame sedasama  Nintendo \emph{device} 
juurde, et füüsilisel kujul seda toota on jõle tüütu. Palju lihtsam oleks teha 
seesama asi nii, et oleks bittide jada, mis kõik grupeeruvad, moodustavaid 
mustreid ja sellest peaaegu nagu võluväel tekivad mingisugused asjad, mis 
inimestele tegelikult on tänaseks sama reaalsed tööriistad kui haamer ja höövel.


\chapter{Tarmo Mamers}
\index[ppl]{Mamers, Tarmo}

\question{Kuidas sina arvutite juurde said?}

Mul oli üks klassikaaslane, kelle isa töötas Küberneetika 
Instituudis\index{Küberneetika Instituut!Juhtimissüsteemide osakond}. Millalgi
keskkooliaastate alguses, aastatel 1983–1985 
käisin seal mitu korda tutvumas Apple 
IIga\index{Apple II}. See oli mõnevõrra keeruline, sest Apple oli 
üsna koormatud, kuna seda kasutati
teadus- ja uurimistööks. 
Monitori asemel oli sel tavaline telekas ja mäletan, et kui talvel sinna esimesi kordi sattusin, kasutati seda põhiliselt 
suusahüppe MMi jälgimiseks. Nii et päris iga kord ei saanud 
arvutit näppida või kui sai, siis ilma pildita. 

\question{Mida sa tegid arvutiga?}

Kõigepealt vaatasin, mida sõbrad teevad, põhiliselt mängisid nad igasuguseid 
huvitavaid mänge. Kui hakkasin ise arvutit
näperdama, siis mind huvitas pigem see, et kui arvutimängud on 
teatud hulga elude, relvade ja 
abivahenditega, siis kas neid nii-öelda ära häkkida ei 
saaks, et oleks veel rohkem elusid ja abivahendeid ning 
saaks rohkem punkte. Mind huvitas mängude ümbertegemine. 

Ma ei teadnud midagi programmeerimisest ega ka arvutite 
tööpõhimõttest, see oli esimene ajendav tegur, mis tõi kokkupuute esiteks 
BASICu\index{BASIC} keelega ja järgmiseks Apple 
assembleriga\index{Assembler}. Tundsin tollal huvi ka 
elektroonika, eriti digitaalelektroonika vastu ja juhtumisi jäid mulle 
ette selle Apple'i manuaalid, mis sisaldasid ka 
elektroonikaskeeme. Ajasin näpuga järge, kuidas bitid liiguvad, kui midagi 
printida või nuppu vajutada, ja kuidas ekraani peale tekitatakse 
mälus olevatest bittidest pilt. See oligi algus.

\question{Kas mängud olid väljastpoolt tulnud või liikus ka isetehtud asju?}

Apple'is olevad mängud olid ilmselt tulnud välismaalt sama kanalit pidi, kust 
arvutid isegi Nõukogude Liitu tulid. Osa huvilisi, eeskätt 
Apple'i kasutajad Tartus või Nõos, kellel oli kontakte teiste Apple'i kasutajatega mujal 
maailmas, said vist neilt ka.

\question{Tehti ju ise ka algelisi mänge, näiteks ühes Juku\index{Juku} 
mängus sai mõisa majandada.}

Ma sellest ajast ei mäleta kodukootud mänge, 
küll aga mõnevõrra hilisemast ajast, kui mul oli kasutada Nõukogude 
päritolu arvuti 
Iskra-226\index{Iskra!Iskra-226}\sidenote{\begin{russian}Искра 
226\end{russian} oli Nõukogude Liidus toodetud arvuti Wang 2200 kloon, mis oli 
originaaliga binaarkoodi mõttes sajaprotsendiliselt ühilduv. 
Iskra-226 sisemine struktuur oli siiski oluliselt erinev ja sisaldas mitmeid 
täiendusi, mis muutsid selle sobilikumaks tööstusrakendusteks.}. Seal oli küll 
isetehtud mänge, mille idee oli võetud kuskilt mujalt, 
või siis oli mõni originaalne mänguidee arvutimänguks vormistatud. 

Venelastel oli Apple'i kloon, mille nimi oli Agat\index{Agat}. 
Kui neid hakkas Eestisse tulema, siis nende jaoks oli muidugi vene 
mänge. Osa olid selgelt Apple'i pealt maha lükatud ja osa olid 
„originaalid“. Sellest ajast ei mäleta ma samuti kodumaist päritolu mänge.

\question{Kas need, kes Apple II päriselt kasutasid, lasid sul 
sellele niisama lihtsalt kõhu alla vaadata ja kaant maha kruvida?}

Küberneetika Instituudis\index{Küberneetika Instituut} väga palju 
ei lastud, sest seal oli oluline ikkagi see, et masin oleks töökorras ja igal 
ajal uurimistööks kasutatav.

Hiljem keskkooliajal sattusin ka tollasesse TPIsse ehk 
praegusesse Tehnikaülikooli\index{Tallinna
Tehnikaülikool!Automaatikateaduskond!Raadiotehnika kateeder}, kus raadiotehnika 
kateedris oli ka üks Apple II\index{Apple II}. Seal olid 
raadiotehnikud, kelle igapäevane leib oligi vaadata pigem 
seda, mis kõhus on ja kuidas see käib. Seal sai masina sisse vaadata, 
kõrval jootekolb, millega sai pädevam seltskond teha ise Apple'i 
jaoks perifeeriakaarte\sidenote{St. arvutisse käivaid trükiplaate, mis võimaldasid arvutil suhelda perfeeriaseadmetega.}, et TPI majas midagi juhtida või mõõta. 

\question{Kas tollal oli üsna hästi teada, kus mõni Apple II 
või Iskra saadaval oli?}

Jah, sest neid olid piiratud hulk ja arvutihuvilised 
tundsid ja teadsid üksteist üsna hästi. Seltskond, kellega ma arvutiringides iga nädal kokku puutusin, 
võis olla kolmkümmend-nelikümmend inimest. 
Ma ise käisin kolmes-neljas arvutiringis ja seal liikus info, mis arvuteid kuskil näppida saab.

Näiteks Oktoobrirajooni õppetootmiskombinaadis\index{Tallinna 
Oktoobrirajooni Õppetootmiskombinaat} oli arvutiklass Yamaha 
MSX\index{Yamaha MSX} arvutitega, mis kindlasti on Prontol\index[ppl]{Pronto} 
paremini meeles, seal käis päris suur seltskond noori huvilisi koos. 

Oli ka TPI arvutiring\index{TPI Arvutiring}, mida juhendas 
Vladimir Viies\index[ppl]{Viies, Vladimir} ja hulk teisi TPI õppejõude ning kus 
olid Robotronid\index{Robotron}. TPI arvutiringis 
puutusin esimest korda kokku Iskra-226ga\index{Iskra!Iskra-226}.

Tallinna 3. Keskkoolis\index{Tallinna 3. Keskkool} oli matemaatikaõpetaja Jaak Loonde\index[ppl]{Loonde, 
Jaak}. Tema oli haridussüsteemis omamoodi fanaatik, kes populariseeris 
arvutite jõudmist kooli nii riistvara kui ka arvutiõppe näol. Jaak Loondel oligi üks 
Agat\index{Agat} kasutada, mille nende kool sai ilmselt kuskilt Venemaalt. 

Lisaks arvutiringidele käisin 34. kooli 
tehnikaringis\index{Tallinna 34. Kool! 
Tehnikaring}, mille juhendaja oli Ants Reili\index[ppl]{Reili, Ants}. Seal oli 
päris mitu poissi, kes ei olnud otseselt üldise tehnika- 
või elektroonikahuviga, vaid just arvutihuviga, kuigi arvutiklassi
seal ei olnud.

Need olidki minu põhiseltskonnad, mille liikmeskond mingil määral kattus. Muidugi oli 
veel seltskondi või huviliste ringe.

\question{Kas need olid kõik Tallinnas?}

Mina puutusin kokku Tallinna omadega, Tartus koonduti ülikooli 
juurde. Seal oli Anne Villems\index[ppl]{Villems, Anne} ja hulk
Apple'eid. Tartus oli Apple olemas ka Füüsika 
Instituudis\index{Füüsika Instituut}, kus tegutses Jaan 
Pruulmann\index[ppl]{Pruulmann, Jaan}, kellega mina puutusin esimest korda kokku siis, 
kui käisin sealset Apple II\index{Apple II} vaatamas. Sattusime Tartusse talvisel ajal kooli- või 
ringikaaslastega ja otsustasime Füüsika Instituuti külla minna, 
et sooja saada ja ehk ka midagi
arvutis teha. Seal saime Pruulmaniga tuttavaks.

\question{Kooli ja ülikooli huvi arvutite vastu on arusaadav, aga mis ajendas
Oktoobrirajooni asutust arvutiklassi hankima?}\label{content!OTK}

Õppetootmiskombinaadid olid mitme kooli peale ehk rajooni 
kaupa (Tallinnas oli tol ajal neli rajooni\sidenote{Alates 1974. aastast 
jagunes Tallinn Lenini (endine Keskrajoon), Kalinini, Oktoobri- ja Mererajooniks.}). 
Asutuse nimetus oli õppetootmiskombinaat, mis ilmselt viitas 
sellele, et seal sai praktilisi asju proovida ja 
ametikogemusi. Arvutiring oli puhtalt huviring, kus ei 
olnud mingit tootmisväljundit, nagu õppetoootmiskombinaadi nimetusest võinuks 
eeldada.

\question{Seda minagi imestan, miks nad hankisid arvutid. See oli ju 
keeruline.}

Igal juhul oli neil arvutiklass tosina Yamaha arvutitega. Võimalik, et 
kuna klass oli otsapidi seotud Jaak Loondega\index[ppl]{Loonde, 
Jaak}, sebis tema selle sinna ja õppetootmiskombinaat oli lihtsalt \enquote{katus}. 
Sedasi ei olnud arvutiklass ühes konkreetses koolis, mis võinuks tekitada pingeid teiste koolidega.

\question{Kui sa arvutitega toimetama hakkasid, siis millele sa toetusid? Tühja 
koha pealt inimene ju ei vaata arvutiskeemilt, kust kuhu bitid liiguvad.}

Elektroonikatausta oli mul nii palju, et teadsin, mismoodi bitid 
liiguvad ja loogikatehted toimivad ning kuidas asju tööle panna. 
Näiteks kuidas teha LCD-ekraaniga elektronkella. 

\question{Kuidas sa seda oskasid?}

Ühelt poolt 34. kooli tehnikaringi\index{Tallinna 34. Kool! 
Tehnikaring} teadmiste baasil. Teisalt lunisin vanematelt 
endale küllaltki palju kirjandust, mis oli põhiliselt vene ja saksa 
keeles, sest ingliskeelset kirjandust ei olnud tol ajal saada. Või 
kui oli, siis ilukirjandust, aimet ja ulmet, mitte 
teaduskirjandust. Seda liikus saksa keeles. Ega ma ühtegi
raamatut otsast lõpuni läbi ei lugenud, aga 
natuke siiski sirvisin ja lugesin olulisemaid peatükke. Sealt 
tasapisi see teadmine tekkis.

\question{Kas sellest võib järeldada, et saksa ja vene keeles tehnilise 
teksti lugemine ei olnud keeruline?}

Minu jaoks saksa keel küll oli, sest koolis õppisin inglise keelt 
süvendatult. Inglise keel oli küll peaaegu nii selge nagu teine emakeel, aga saksa keelt ma ei osanud üldse. 
Tänapäeval on küll nii, et võtad raamatu lahti ja mis sest, et saksa 
keelt ei oska, aga paljud sõnad on inglise või muu tuntud keelega nii sarnased, et üldisest mõttest saab aru. 
Konkreetseid juhiseid või faktilist infot ma saksa keeles 
täisväärtuslikult ei loe. 

\question{Mis koolis sa käisid?}

44. keskkoolis\index{Tallinna 44. Keskkool}, mis on tänapäeval Mustamäe 
gümnaasium\index{Tallinna Mustamäe Gümnaasium|see{Tallinna 44. Keskkool}}. Kooliajal ei saanud inglise keelt väga palju rakendada, ka mitte 
IT-maailmas,
peale selle, et Basicus on käsk \verb|print| inglise keeles. Aga 
Basicus ei ole käske palju ja neid ei ole keeruline pähe 
õppida, juhul kui inglise keelt ei valda. 

\question{Kas tehnikakirjanduse juurde käis ka mõni muu 
kirjandus- või ulmehuvi?}

Kooliajal lugesin üsna palju ulmet inglise keeles ja 
mulle sattus kätte ka Douglas Adamsi Hitchhikeri raamat\sidenote{Vt 
märkust lk \pageref{sidenote!adams}.}.
Kuna õppisime inglise keelt süvitsi, siis meil oli ka
inglise keele kodulugemise tund. Pidime kodus lugema ingliskeelset
ilukirjandust ja tunnis õpetajale jutustama. Nägin raamatupoes 
üleval lae all riiulil Hitchhikeri esimest osa, 
vaatasin, et huvitav pealkiri, ostsin raamatu ära ja võtsin selle inglise keele kodulugemiseks. See ei olnud siiski väga mõistlik 
valik, sest kui mõelda väljamõeldud sõnadele, mida seal kasutatakse, siis need on 
eesti keelde üsna raskesti tõlgitavad, selleks peab väga hea fantaasiaga tõlkija 
olema. Aga mina hakkasin entusiastlikult raamatut lugema ja õpetajale jutustama. Ma 
küll ei tea, kui palju ta sellest aru sai, aga vähemasti jäi ta rahule. 

\question{Seal ei ole ju narratiivi, vaid mingisugune keeruline sõlm, mis viienda 
raamatu lõpuks umbsõlme läheb!}

Ega minagi saanud esimesest osast eriti aru. Hiljem lugesin ülejäänud osad ka läbi, siis loksus 
pilt paika.

\question{Kas ingliskeelset ilukirjandust oli tollal saada?}

Jaa, seda oli igal pool, isegi Tallinnas. 
Pärast kooli lõppu, kui töötasin TPIs\index{Tallinna Tehnikaülikool}, 
käisin palju Moskvas ja Leningradis komandeeringus, kus oli väga 
lai valik. Asimovilt ma ei lugenud esimesena tema tuntuimat sarja „Asum“\sidenote{Vt 
ka märkust lk \pageref{sidenote!asum}.}, vaid 
üksiklugusid. Asimov ja Adams olid kaks ulmekirjanikku, 
kellega ma inglise keele vahendusel esimesena kokku puutusin. 

\question{Aga vene klassikud? Strugatskid?}

Jaa, Strugatskeid olin lugenud varem, sest neid oli tõlgitud eesti keelde: 
\enquote{Purpurpunaste pilvede maa}\label{sisu:purpur}\sidenote[][-1cm]{Arkadi ja Boris 
Strugatski \enquote{\begin{russian}Страна багровых туч\end{russian}} (1959), mis eesti 
keeles ilmus 1961. aastal Ralf Tominga tõlkes (värsid Lembe Hiedel) sarjas 
\enquote{Seiklusjutte maalt ja merelt}.}, „Amfiibinimene“ oli vist ka 
Strugatskite oma\sidenote{„\begin{russian}Человек-амфибия\end{russian}“ on siiski 
Aleksandr Beljajevi 1928. aastal ilmunud romaan, mis eesti keeles ilmus 1960. 
aastal sarjas \enquote{Seiklusjutte maalt ja merelt} koos romaaniga 
\enquote{Maailmavalitseja} („\begin{russian}Властелин мира\end{russian}“, 
1926).}. Neelasin võimalust mööda ka populaarteaduslikku ehk 
aimekirjandust, näiteks raamatusarja \enquote{Mosaiik}\sidenote{\enquote{Mosaiik} oli kirjastuses 
Valgus aastastel 1973–1991 välja antud populaarteaduslike raamatute sari, mis 
käsitles äärmiselt laia teemaringi ajaloost ja psühholoogiast kuni topoloogiani.}. Muud
aimekirjandust eriti ei olnudki.

Kriminullid olid teine ilukirjanduse valdkond, mis tol ajal peale ulmekirjanduse 
huvi pakkus. Neid ma ka lausa neelasin.

Üldine lugemistempo oligi kiire. Näiteks Rootsis töötades lugesin ühe „Asumi“ osa 
inglise keeles ühe ööga läbi. Järgmisel päeval ostsin raamatupoest järgmise osa ja nii edasi. Kool oli äsja lõpetatud ning sain töö ja muu 
elu kõrvalt seda lubada, selle asemel et öösel magada ja puhata.

\question{Kui sa keskkooli lõpetasid, kas läksid siis kohe TPIsse tööle või edasi õppima?}

Pärast keskkooli läksin TPIsse\index{Tallinna 
Tehnikaülikool} ikkagi tööle, mitte õppima. Töökoht sattus mulle kätte tänu 
Vladimir Viiese\index[ppl]{Viies, Vladimir} juhendatud arvutiringile. 
Asusin tööle 
elektronarvutite kateedris\index{Tallinna Tehnikaülikool!Elektronarvutite 
kateeder}, kus töötas ka Viies. Aitasin teha arvutihooldustöid ning 
häälestasin ja valmistasin ette arvutilaboreid, mida õppejõududel vaja oli. 
Hiljem lõin kaasa
tarkvaraprojektides, kus oli vaja programmeerida, 
sisend-väljundseadme jaoks draiver kirjutada. 

\question{Kas läksid õppima ka?}

Õppima läksin mõnevõrra hiljem, sest mind ei köitnud
punaste ainete küllus. Tundsin 
nende vastu nii suurt vastumeelsust, et ei tahtnud ülikooli 
õppima minna isegi tehnilist ala, kui seal on punased ained juures, näiteks NLKP ajalugu. Paar aastat hiljem, vist 1990. aasta sügisel läksin siiski TPI õhtusesse osakonda õppima. Olin 
küll üks enamikust, kes ei lõpetanud. Meie kursusele astus sisse vist 
kakskümmend viis inimest, kellest lõpetas kaks, aga õhtuses osakonnas lõpetanute madal protsent oligi üheksakümnendate algul üsna tavapärane. 

\question{Mis eriala see oli?}

Elektronarvutid. Sealt alates ongi tegelikult kõik töökohad ja üsna palju ka vaba aja tegemised olnud seotud programmeerimisega ja arvuti tehnilise poolega.

TPIs töötades asus elektronarvutite kateeder teisel korrusel. 
Samas korpuses neljandal korrusel oli raadiotehnika 
kateeder\index{Tallinna Tehnikaülikool!Automaatikateaduskond!Raadiotehnika 
kateeder}, kus oli see Apple II\index{Apple II}. Meil tekkis 
Mastiga\index[ppl]{Kaal, Madis} (Madis Kaal\sidenote{Madis Kaal toimetas sel ajal
raadiotehnika kateedris. Vt lk \pageref{sisu!mast_raadiotehnikas}.}) 
mõte ühendada teisel korrusel olevad PC arvutid, mis olid minu igapäevased tööriistad, 
Apple IIga, mis oli neljandal korrusel Masti igapäevane tööriist. 
Ehitasime nende vahele \emph{current loop}'i ehk RS-232, 
misjärel sai PCst kopeerida andmeid Apple II sisse ja vastupidi. 
Nagu pilv – kasutad kellegi teise arvutit. 

Umbes 1990. aastal jõudis Eestisse info BBSide olemasolust. TPI majas oligi Mast\index[ppl]{Kaal, Madis} see 
entusiast, kes pani esimese BBSi jooksma. Mina esialgu vaatasin 
kõrvalt, tundmata selle vastu erilist huvi. Seal sai mänge vahetada, 
aga kuna ma ei ole kunagi mängufanaatik olnud, siis selle pärast
BBSindus mind ei tõmmanud. Hiljem küll 
leidsin, et BBSid võivad olla kasulikud – 
neis leidub tekstifaile, mis on nagu
\emph{manual}'id, standardid või
programmeerimisõpikud IBMide või Apple'i jaoks.

\question{Kas need olid \emph{plain text} failid või \LaTeX?}

Need olid vormindatud tekstifailid 
tabulatsiooni ja leheküljevahedega ning neid sai maatriksprinteriga 
ilusti vormindatuna paberile trükkida. Rasvase 
ja kaldkirjaga teksti oli ka võimalik kasutada.
Maatriksprinterid olid ilmselt kättesaadava hinnaga, sest need olid enamiku arvutite taga. 
Suurtel arvutitel (ESid ja SMid, mis olid TPIs ja Küberneetika 
Instituudis) olid laiad ridaprinterid, selle sõna mõtles välja vist Ustus Agur\index[ppl]{Agur, 
Ustus}. Ühesõnaga koledat häält ja värinat tegevad 
printerid.

\question{Kas siis, kui said aru, et BBSidest saab igasuguseid dokumente, hakkasid need sulle huvi pakkuma?}

Jah, see oli see hetk ja ajend, kui leidsin, et sealt sai
peale mängude ja muu tilulilu ka midagi mõistlikku. Millalgi panin oma BBSi 
püsti ja selleks ajaks oli ka Fidonet Eestisse 
jõudnud\sidenote{Esimene Fidoneti Eesti regiooni 2:49 sisaldanud 
\emph{nodelist} on 271 28. septembrist 1990. Regiooni koordinaatorina on 
kirjas Andrus Suitsu\index[ppl]{Suitsu, Andrus} ja \emph{host} on Tarmo 
Ausing\index[ppl]{Ausing, Tarmo}. Vt lk \pageref{sisu:nodelist}.}. Paljud, kes on ajalooliselt tagasi vaadanud ja 
sellest ajast rääkinud, ei pruugi eriti olla vahet teinud BBSindusel ja Fidonetil, mis 
olid kaks eraldi maailma. BBS oli lihtsalt 
süsteem, kuhu sai modemiga sisse helistada ja siis seal ringi toimetada, 
andmeid failide näol tõmmata ja sõnumeid 
vahetada. Kogu info ja sõnumid olid salvestatud ühte 
konkreetsesse BBSi süsteemi.

Fidonet sai alguse BBSidest ja selle eesmärk 
oli BBSide ja muude Fidoneti liikmesüsteemide vahel sõnumeid
edasi-tagasi toimetada. 

\question{Kas see tähendab, et Fidoneti sisse helistamise kohad helistasid ka 
üksteisele sisse ja vahetasid andmeid?}

Jah, see oli siis juba automatiseeritud süsteem, kus olid vahendid 
sõnumite ehk meilide vahetamiseks. Sõnumeid oli kahte liiki: 
privaatmeilid ja konverentsmeilid (tänapäeva mõistes 
meiligrupid või -listid).

\question{Kas \emph{Usenet} tekkis ka sel ajal?}

Usenet oli olemas palju varem. Usenet ja UUCP protokoll 
on Unixi maailma päritolu,  
Unixi arvutite vahelise meilivahetuse protokoll. See Usenet, mis sinna 
ümber tekkis, oli konverentside või vestlusringide 
süsteem Unixiga töötavate arvutite kasutajate vahel. Fidonet ja BBSid töötasid
enamjaolt PCdel.

\question{Kas seda peegeldati Fidosse ka?}

Jah, seal olid lüüsid olemas. Usenetist sai konvertida kirju Fidoneti 
\emph{echo}'desse ehk konverentsidesse. Muu hulgas ka faile, neid vahetati
Usenetis väga palju ja neid oli võimalik konvertida 
PC failideks, mis kuskil BBSis üles pandi.

\question{Kas eestlased toimetasid Usenetis oma gruppides või olemasolevates?}

Usenetis ei olnud tollal Eesti-spetsiifilisi või 
regionaalseid gruppe, erinevalt Fidonetist, kus oli küll viisteist-kakskümmend lokaalset vestlustgruppi ehk \emph{echo}'t. 
Neist kaks-kolm olid liikmeskonna mõttes üsna populaarsed.

\question{Mida see tähendab? 50, 100, 500 liiget?}

Lugejaid võis seal olla palju, sest pidevalt tuleb välja 
inimesi, kellega mina ei ole kunagi kokku puutunud, 
aga kes räägivad, et nad on seal \emph{echo}'des midagi lugenud. Selleks, et \emph{echo}'sid lugeda, ei pidanud ise
omama BBSi ega Fidoneti süsteemi. BBSi sai sisse helistada, seal lugeda ja soovi korral kirjutada. Kui kellelgi oli Fidoneti süsteem 
püsti pandud, siis selle eelis seisnes selles, et kirjad tulid 
automaatselt koju kätte ja lugemiseks-kirjutamiseks ei olnud vaja ise kuskile 
kaugele helistada. See ring inimesi, kes ainult luges, 
võis olla päris suur. Aktiivselt suhtlesid ja ka kirjutasid 
võibolla kakssada inimest.

\question{Seda on päris palju. Kas sa oma Fido \emph{node}'i panidki püsti 
selleks, et asjad tuleksid koju kätte? Mis selle asja nimi oli?}

Eesmärk oli jah saada asjad piisavalt automatiseerituks, et 
ei peaks kulutama aega 
BBSi löögile saamisele, sest kui BBSi küljes oli 
välismaailmaga suhtlemiseks üks modem, siis sai 
seda BBSi kui teenust korraga kasutada üks inimene. 
See tähendas, et helistasid modemiga, telefon oli kinni, helistasid viie minuti 
pärast, ikka kinni. Milleks niimoodi vaeva näha ja pidevalt 
proovida? Tõsi küll, modemi sai panna automaatselt kordusvalima 
ja kui see lõpuks löögile sai, andis signaali. Aga ma leidsin, et parem on lasta teha
seda Fidoneti automaatikal. Siis sain rahumeeli endale sobival hetkel avada meililugemise programmi ja lugeda vahepeal masinasse tõmmatud meile. 

\question{Mis su \emph{node}'i nimi oli?}

Minu \emph{node}'i nimi oli MamBox. Ma ei mäleta, mis hetkest alates hakkasin oma perekonnanimest tulenevat eesliidet kasutama, aga BBSi tehes panin MamBox. 
Oma lõbuks programme kirjutades märkisin kaubamärgiks
\enquote{\emph{Copyright MamSoft}}\sidenote{Tegu oli levinud praktikaga. 
Sellest, \emph{misasi} üks firma on, oli arusaam ähmane, ent sellest, et firma 
\emph{nimi} tuleb kindlasti ära mainida ja kuulsaks teha, oli arusaam väga 
konkreetne.}.

Üsna tüüpiline oli see, et kui kellelgi oli BBS, siis lisas
ta ühel hetkel sinna Fidoneti funktsionaalsuse. Oli ka teistsugune suundumus, et kui kellelgi oli 
tekkinud Fidoneti \emph{node}, siis paljud omanikud 
leidsid, et võiks ka BBSi püsti panna. 
Muidugi oli ka Fidoneti \emph{node}'e, kelle omanike või 
\emph{sysop}'ide eesmärk oligi lugeda-kirjutada ja lasta automaatselt sõnumeid 
vahetada; nende huvi ei olnud BBSi üleval pidada.

\question{Teisisõnu, kui mõni BBS sai populaarseks, siis põhjuseks võis olla see, 
et aktiivne kogukond vahetas seal omavahel faile, kui ka see, et miskipärast 
otsustasid paljud kasutajad just sealtkaudu Fidonetile ligi pääseda.}

Fidoneti sisule pääses ligi kõikidest BBSidest, mis olid Fidoneti liikmed, sest 
kõigis oli ühesugune koopia konverentskirjadest ehk \emph{echo}'dest. 
Iseasi olid privaatkirjad – siis oli vaja Fidoneti \emph{node}'i numbrit teada, 
et konkreetsele inimesele kirja saata. Konverentskirjad 
olid ühtmoodi saadaval igas BBSis.

Ega see muidugi mõnus ei olnud, et täna loed meili siit, homme hoopis 
teisest BBSist. On ju viidad, kui palju sul on loetud meile, kus 
su lugemisjärjekord on, kas oled millelegi vastanud või ei ole. See läheb 
sassi, kui ei ole oma nii-öelda kodu-BBSi. Ja oli ka selge, et 
populaarsed failide tõmbamise BBSid olid üsna hõivatud ja 
tihtipeale liinid kinni. 

\question{Faili tõmbamine võttis ju tükk aega!}

Jah. Algusaegadel, kui BBSid ja Fidoneti \emph{node}'id Eestis tekkisid, 
siis 14 400boodine (ümmarguselt 14 400 
bitti ehk 14 kilobitti sekundis) andmevahetuskiirus oli üsna tüüpiline.

\question{Ma isegi mäletan 9600boodiseid.}

9600 oli jah lihtne ja odav igamehe tehnoloogia, aga kõik püüdlesid ikka 14 400boodiste 
modemite poole. Edasi tulid
19 200, 26 600 ja veelgi suuremad kiirused. Minul oli kasutada isegi
33 600boodise töökiirusega modem, aga selline kiirus tuli 
kätte ainult juhul, kui teisel pool sideliini oli vastas täpselt 
sama tootja modem. Selle nimi oli TrailBlazer\index{Telebit 
TrailBlazer}\sidenote{USA tootja Telebit, kes Trailblazeri sarja tootis, 
kasutas üldlevinud V-seeria protokollide asemel omaenda protokolli Packetized 
Ensemble Protocol (PEP).}. USRoboticsid\index{USRobotics} töötasid BBSide 
põhiajastul kõige kiiremini vist 34,4 kiloboodi juures.

\question{Kas BBSidega majandamine tekitaski sul võrguhuvi? Sa rääkisid, 
kuidas te Mastiga Apple'it ja PCd paaritasite.}

Ilmselt Apple'i ja PC paaritamine tekitaski võrgunduse 
pisiku, sest TPIs ega ka kuskil mujal, kus ma alguses arvutitega kokku 
puutusin, ei olnud kohtvõrgutamise tehnoloogiaid 
kasutusel. Ainuke oli UUCP, mis käis Unixite vahel, aga see oli 
tõsisemate ja suuremate arvutite sidepidamine akadeemilistes ja teadusringkondades. Fidonet seevastu oli 
rohkem asjaarmastajalik. Alles pärast TPId, järgmises töökohas, puutusin esimest 
korda kokku päris kohtvõrgutehnoloogiaga ARCNet\sidenote{ARCNet oli 1980. aastatel levinud esimene laialdast 
kasutust leidnud mikroarvutite võrgusüsteem, mis on siiani kasutusel 
sardsüsteemide puhul.}.

\question{Kus see oli ja mis aastal?}

See oli aastal 1991 ettevõttes Skriining\index{Skriining}, mis 
tegutseb tänapäevalgi. Skriiningus puutusingi kokku ARCNetiga, mis jooksis 
kahe ja poole megabitisel kiirusel. See oli koaksiaalkaabli võrk, 
peaaegu nagu esimesed Etherneti võrgud, aga neli korda aeglasem. ARCNeti koaksiaalkaabel oli ka vist 75oomine versus Etherneti 50oomine kaabel. 

ARCNet oli üsna lühiajaline, sellega puutusin kokku peamiselt 
tänu sellele, et tollal käidi Soomest ja mujalt lähivälismaalt 
seljakotiga kraami toomas. Väga palju Soome kraami oli 
seal maha kantud, seda vist ei tahetud seal ära visata, 
sest utiliseerimine maksis ja nii antigi ära. Ma ei mäleta, et ARCNetiga midagi väga tõsist oleks tehtud, aga 
kokkupuuted sellega ikkagi olid. Selle järel tuli koaksiaalkaabli Ethernet, kümme megabitti sekundis. See hakkas reaalselt ettevõtetesse jõudma ja selle baasil 
hakati üsna palju kohtvõrke ehitama.

\question{Räägi palun pisut Skriiningust\index{Skriining}. Arvutiäri 
jaoks peaks nime järgi olema kaks poolt: arvuti ja äri. Aga et aastal 1991 
oleks kumbagi olnud, tundub natuke uskumatu.}

Skriiningu arvutid tulidki alguses seljakotis piiri tagant. Järgmises faasis tulid need endiselt seljakotiga piiri tagant, aga 
selleks oli vaja piiri taha kõigepealt seljakotiga 
sularaha viia. Keskmise arvuti hind võis olla kakskümmend tuhat rubla, kuid mina hindade ja müügitööga ei tegelenud. Nii et ma täpselt ei kujuta ette, kui palju arvutid tol 
ajal maksid, aga arvutustehnika oli meeletult kallis.

\question{Kuidas selline firma üldse võis tekkida tol ajal? Ei saanud ju panna
internetti kuulutust, et tulge meile tööle.}

IT-maailmas liikusid inimesed ilmselt tutvuste kaudu ühest kohast 
teise tööle. Ka mina jõudsin Skriiningusse\index{Skriining} tuttava 
kaudu, mu varasem kolleeg TPIs sattus 
Skriiningusse tööle ja kutsus paar aastat hiljem mindki sinna. 
Skriiningu nii-öelda vertikaal- või kliendisegment oli ja on ka tänapäeval 
meditsiiniasutused: võrgud, arvutibaas ja infosüsteemid, 
programmeerimine ja hooldamine. Ma arvan, et see on ka üks põhjus, miks 
Skriining on tänapäeval endiselt elus: tal on oma 
üsna kitsas kliendisegment ning kindlad ja pikaajaliselt välja kujunenud kliendisuhted.

\question{Sinu jutu järgi tundub, et esimesed arvutifirmad olid sõprus- 
või vähemalt tutvuskonnapõhised.}

Mitte päris. Too endine kolleeg, kes mind Skriiningusse
kutsus, oli ainuke, keda ma seal tundsin. Aga sellised arvutifirmad ei olnud suured, Skriiningus oli kõige rohkem viis-kuus 
inimest ja kõik tegid enam-vähem kõike. Võibolla 
mõni programmeeris rohkem ja mõni teine, nagu mina näiteks, vedas rohkem 
kaablit, keeras kruvisid ja timmis asju 
arvutikaane all. Teatud eelistused olid kindlasti inimestel olemas, aga 
üldjoontes võib öelda, et kõik käisid mingil määral vähemalt üle kõikidest süsteemidest, mida firmas kasutati või millega see tegeles.

Samas oli muidugi ka mitmeid sõprade seltskondi, kes üheskoos tegid mõne arvutifirma.

\question{Kas sa sel ajal pidasid veel oma BBSi ka?}

Jah. BBS oli mul üleval päris pikka aega, ma olen teda ühest töökohast teise kaasa vedanud, sest kodus ei saanud seda pidada. 
Esiteks ei olnud eriti kellelgi võimalik koju arvutit hankida, see oli kallis. 
Ja kui ka oli võimalus mõni niru arvuti saada, siis selle peale 
BBSi hästi püsti ei pannud. Teiseks ei olnud tol ajal kodus telefoniga 
väljahelistamine just odav lõbu. Pealegi, kui mõelda 
Fidoneti peale ja et see oli ülemaailmne süsteem, siis Fidoneti side 
hõlmas ka teatud hulka rahvusvahelisi kõnesid. Sel ajal ei olnud kodustelt numbritelt 
võimalik otse välismaale helistada, kaugvalimine toimus läbi 
inimoperaatori\sidenote{Jaan Tallinn\index[ppl]{Tallinn, Jaan} on rääkinud, et 
inimoperaatoritele oli täiesti võimalik arvutiside vahendamine selgeks 
õpetada. Tuli öelda, et \enquote{kui vilistama hakkab, ühendage ära, nii peabki 
olema}.}. Ja ega ka kõikidest ettevõtetest ei olnud võimalik välismaale 
otse helistada. Tihtipeale oli ettevõttes selleks kümne 
või saja telefoni peale ainult üks telefoninumber, mida püüti 
endale ära rääkida, et selle taha BBSi ühendada. Tihti olid ka 
BBSi omanikel kokkulepped, et nende BBS töötab ja telefoniliini saab öösiti kasutada 
ning päeval kasutatakse liini kontoritööks. 
See tekkis hiljem, et BBSi jaoks oli mõnes firmas võimalik 
saada 24 tundi ööpäevas eraldi telefoniliin, ja eriti heal juhul sai sealt ka 
välismaale helistada.

Liini oli võimalik jagada ka BBSi modemi ja ettevõtte faksiseadme vahel, siis said
mõlemad sõbralikult töötada 24 tundi ööpäevas.

Kui ma tol ajal ühest töökohast teise liikusin, siis hindasin
uut ettevõtet muu hulgas selle järgi, kas mul on võimalik 
BBS sinna kaasa võtta ja selle jaoks 
kaugvalimisega telefoniliin saada – veel parem, kui liin oleks ööpäev läbi kasutatav.

\question{BBSi kaasavedamine pakkus sulle siis päris olulisi valikuid. Mis selle juures huvitav oli?}

Ikka see, et BBSidest saadav info tuli üsna lihtsalt kätte ja seda oli kerge maailmast üles otsida, kui juba Fidoneti \emph{node} püsti 
oli. Lisaks sai meili- ja 
failivahetust automatiseerida. Kui tahtsin kuskilt kaugelt BBSist mõnd 
faili kätte saada ja teadsin selle nime, siis polnud tarvis jälle endal käsitsi 
sinna BBSi sisse logida, vaid sain seda teha 
Fidoneti automaatika abil.

\question{Toonases Fidoneti maailmas toimetav seltskond oli suhteliselt suur 
ja sinu nimi jookseb nende juttudest päris palju läbi. Miks see nii on?}

Tegelikult on olemas palju nimekamaid 
BBSi pidajaid, kes BBSi maailma Eestis põhimõtteliselt alustasid.

Kui BBSid ja Fidonet olid Eestis 
levima hakanud ja üsna agarasti kasutusele võetud, siis peagi tekkis meil 
ühes Fidoneti seltskonnas äratundmine, et meid on 
küll sada kuni kakssada inimest, kes igapäevaselt Fidoneti 
kaudu suhtlevad, kirju vahetavad, nalja teevad ja vahetevahel üksteist ka sõimavad, aga me teame võibolla kümmet neist nime- ja nägupidi. Jõudsime
järeldusele, et sellele probleemile tuleks lahendus leida. 

1991. aasta suvel kasvas huvi teiste BBSi kasutajatega näost näkku kohtuda nii suureks, 
et enam-vähem seesama kümnene seltskond mõtles teha kokkutuleku. Ühel augusti nädalavahetusel saigi BBSummer Väänas teoks. Osavõtumaks oli ka, 
võibolla viiskümmend rubla, võibolla vähem\sidenote[][-5.6cm]{Sel ajal valitses Eestis 
hüperinflatsioon ja hinnad kerkisid kiiresti, mistõttu 50 rubla tollast väärtust on raske hinnata. Lisaks olid enne 1992. aastat 
teatud kaupade hinnad riikliku kontrolli all ja paljusid kaupu polnud üldse saada, valitses 
ka sularahapuudus ja toimis elav ning väga volatiilsete hindadega 
must turg. 50 rubla eest võis saada näiteks 20 kilo kartulit või ühe 5.25 tollise flopi.}. Plaan oli suhelda ja mängida 
IT-kalduvustega mänge: mitte arvutimänge, aga näiteks flopiheidet ja 
kõvakettaheidet. 

BBSummerist kujunes traditsioon, igasuvine kokkutulek. Mina aitasin 
esimest BBSummerit ette valmistada ja läbi viia ning panin ka hilisematel BBSummeritel õla alla.
Eks seepärast mu nimi BBSi seltskonna mällu on jäänud.

\question{1991. aastal andis ikka kõvaketast heita!}

Kõvaketas koosnes tollal suurtest 19- või 21tollise 
läbimõõduga plaatidest, mis ei olnud 
hermeetilises korpuses nagu tänapäevased pöörlevad kettad. 
Ühe käepidemega varre külge oli pandud kaheksa või kümme plaati ja neid sai 
kettaseadme seest välja tõsta ning vahetada\sidenote[][-6.8cm]{Sellised kettapakid, näiteks IBMi 1316, 
suutsid talletada mõned megabaidid infot ja olid tolleks ajaks 
selgelt iganenud. Eestisse sattusid sedalaadi seadmed tõenäoliselt 
humanitaarabina, mis tõi meie kanti hulganisti kummalist vananenud riistvara. 
Üks selline kettalugeja oli 1992. aastal näiteks Võru I 
Keskkoolis\index{Võru Kreutzwaldi Gümnaasium}. 
Arvutiklass asus teisel korrusel ja kui kettaseade sisse lülitati, oli undamist 
tänavale kosta – selle järgi sai hinnata, kas klassis parajasti oli keegi või 
mitte.}. Sealt lahti lammutatud kettaid me lennutasime küll esimesel 
kokkutulekul. 

Kokkutuleku nimi BBSummer\index{BBSummer} tulenes 
\enquote{BB} lühendist BBS ja suvest. Üks aasta varem 
oli toimunud esimene Rock Summer\sidenote[][-3.2cm]{Rock Summer oli 1980. aastate lõpus 
ja 1990. aastatel Tallinnas lauluväljakul peetud muusikafestival. Tegu oli esimese suurema 
rokifestivaliga siin kandis ja platsil valitsenud atmosfäär avaldas keskmisele 
nõukogude noorele radikaalset mõju. Kuna tegu oli ühega esimestest 
võimalustest piiluda raudse eesriide taha, meelitas festival kohale ka 
nimekaid lääne ansambleid.}, aga nii palju, kui oleme erinevate inimestega 
meenutanud, ei olnud Rock Summer kuidagimoodi \enquote{Summeri} nimeosa 
eeskuju või põhjustaja – meil olid sõltumatud kaubamärgid. 

Kui see 1991. aasta BBSummer toimus\sidenote[][-1.5cm]{Tarmo saatis esimese BBSummeri 
(ametliku nimetusega \enquote{Eesti amatöörarvutivõrgu kasutajate I seminar-laager}) 
kutse 12. juulil 1991 ja üritus toimus 26.-27. augustil Tugamanni tuulikus (ametlikult EPT Tallinna osakonna puhkekompleks).}, 
siis mõni päev varem tehti
Moskvas riigipööre ja Tallinnas tulid tankid tänavale. Mina ütlesin seepeale, et BBSummeri teeme ära igal juhul, kui just ei ole 
liikumiskeeldu. Olukord oli üsna pingeline. Esimesel BBSummeril 
oli vist viiskümmend kuus osalejat. Suur hulk oli seal puhtalt sellepärast, et nad olid Fidoneti 
\emph{sysop}'id, aga kõvasti üle poole 
olid BBSi lihtkasutajad, kes tõmbasid faile ja 
vahetasid meile, ilma et neil endal oleks olnud oma BBS või \emph{node}.

\question{See oli üsna korralik suhe teenuse pakkujate ja tarbijate vahel, 
BBSi pidamise barjäär oli kõrge ja seltskond seega üsna tehniline?}

See oli jah parajalt tehniline. Kes tundis, et tehnika on temast 
üle, tõenäoliselt ei pidanud BBSi. Selle
häälestamine, korralikult tööle panemine ja Fidoneti 
automaatika käivitamine ei olnud triviaalne tegevus. Internetist juhendvideot vaadata ju ka ei saanud, küll 
aga sai lugeda tekstifaile samm-sammult juhistega.

\question{Siis sündis ju FAQ, \emph{Frequently Asked Questions}, mis praegu on 
lihtsalt osa veebilehest. Toona oli tegu konkreetse eraldi leviva 
failiga, kuhu jõudsidki \emph{echo}'des ja uudisgruppides sagedasti küsitud 
küsimused koos pädevate vastustega.}

Jah, olid küsimused-vastused, kuidas asi käima panna ja milliseid sümptomeid vaadata, kui midagi ei tööta.

\question{Kas sedalaadi sisu Eestis ainult tarbiti või panustati ise
ka?}

Jah, panustati ikka, kui toodeti sisu ehk kui keegi kirjutas 
programmi, mis ei olnud mõeldud ainult oma tarbeks ega olnud mäng, 
vaid näiteks funktsioonide või alamprogrammide teek ehk 
\emph{library}. Näiteks Mast\index[ppl]{Kaal, Madis} kirjutas 
tekstiliideste tegemiseks ühe funktsioonide teegi, millega sai teha ekraanimenüüsid ja 
-kaste. Tekstirežiimis sai
hiirega menüüdes ringi klikata. Selliste asjade jaoks olid 
FAQd või lihtsad juhendid olemas igal vähegi mõistlikumal 
autoril. Mängude puhul tuli küll installida flopi, mäng käima panna
ja siis vaadata, kuidas see tööle hakkab ja mida mingi nupp 
teeb. Mängude manuaale ilmselt keegi eriti ei lugenud.

\question{Kas pärast Skriiningut jõudsid Uninetti ka?}

Jah, Uninet on olnud minu tööandja küll.

\question{Kas päris alguses või millalgi hiljem?}

See oli mul vist viies töökoht. Pärast Skriiningut ja aastast maasikakorjamist Soomes
(loe: C++ ja x86 asm programmeerimistööd Rootsis)\sidenote{Tol ajal oli levinud viis korraga palju raha teenida käia Soomes maasikaid korjamas. Töö oli füüsiliselt raske, suhteliselt nüri aga toonase Eesti mõistes väga hästi makstud. Ilmselt olid Tarmo tööl Rootsis samad tunnused.} sattusin Baltic 
Computer Systemsisse\index{Baltic Computer Systems}.
BCSis tegelesin konkreetselt arvutivõrkudega: 
meil oli arvutivõrkude osakond ja me tegelesime ühelt poolt 
kaabeldusega ja teisalt serverite ning mingil määral ka sellise 
tarkvaraga, mis oli vaja võrgus käima panna. Näiteks andmebaasid, mis olid 
mõeldud algselt ühes arvutis kasutamiseks, aga mida sooviti hiljem võrgus kasutada. Seejärel tuli Uninet.

\question{Tol ajal oli enamik andmebaase mõeldud käima ühes 
arvutis. See tähendas, et sinna sisse ei olnud ehitatud transaktsioone ega muud säärast.}

Seda võimalust ei olnud jah tihtipeale olemas, aga oli viise, kuidas 
sellest mööda hiilida, et andmebaasi avamisel arvutis ei oleks see võrgus kasutajate jaoks lukus, vaid et seal saaks midagi 
teha. 

\question{Mida sa praegu teed?}

Hiljem olen teinud erinevaid asju, mis ei ole olnud enam niivõrd seotud
võrgu tehnilise ülesehitusega, vaid võrgus 
töötavate rakenduste ja võrguturbega. Praeguses töökohas olen mõnes mõttes uuesti
sattunud tagasi sellise tegevuse juurde, mis on seotud andmesidevõrgu 
baasprotokollidega: IP, TCP, UDP ja DNS. Sel tööl on küll
endiselt seos rakendusprogrammide ja mobiiliäppidega, 
sest suur osa tööst vajab ka teadmist, kuidas äpid võrgus käituvad: mismoodi 
nende liiklus ja andmevahetus on võrgus üles ehitatud ja kuidas 
andmevahetust filtreerida. Kuidas hakkama saada lahendustega, 
mida Google ja teised suured tegijad välja pakuvad ja mille eesmärgiks 
peetakse üldiselt seda, et kasutajal oleks internetis mugavam toimetada ja turvalisem olla, aga mis samas võivad kaasa tuua
teatud negatiivseid nähte – näiteks võrguliikluse tarbetu kasvu. Minu töö on 
aidata neid negatiivseid nähte teatud kasutuskohtades kõrvaldada.

\question{Siis on ju selles mõttes toredasti, et kui sa alguses rääkisid huvist 
mängus tegelasele kapoti all toimetades teist värvi müts pähe panna, siis 
praegu on see tegelane teistsuguse arvuti sees ja kapotialune on natuke 
keerulisem, aga ülesanne suuresti sama.}

Täpselt nii. Minu jaoks on oluline see, mis on karul kõhus ja kuidas see 
seal töötab. Kui ei tööta hästi, siis tuleb mõelda, kas ja mida paremaks teha. Ja 
kui töötab hästi, siis saab sellest hoolimata midagi teistmoodi teha.



\chapter{Tarvi Martens}
\index[ppl]{Martens, Tarvi}

\question{Kuidas sina said arvutite ja arvutid sinu juurde?\sidenote{Kuna Tarviga rääkisime juttu mitmel 
korral, on jutulõng mõnevõrra hüplik. Katkemiskohad on tekstis markeeritud.}}


Ma olen pärit Pärnust ja seal arvuteid minu meelest tollal ei olnud, aga 
ma käisin olümpiaadidel, nii et matemaatika ei olnud minu jaoks 
mingi teema. Viiendas klassis
võitsin kuuenda klassi matemaatika linnaolümpiaadi, mille üle kõik olid suhteliselt 
jahmunud. Ühe riikliku olümpiaadi käigus viidi meid 
ekskursioonile Nõo Keskkooli\index{Koolid!Nõo Keskkool}, kus oli suur arvuti. See oli teistsugune maailm, aga kui mind sinna õppima 
taheti viia, siis ma ei tahtnud väga minna. Mul oli Pärnus oma bänd.

\question{Sul oli oma bänd?}

Jah. Tegime punki nagu ikka sel ajal. Käisin Pärnus muusikakallakuga koolis ja bänditegemine oli 
elementaarne. Kooliteater tegi ka oma esimesi samme. Kadunud Aare Laanemets\index[ppl]{Laanemets, Aare} ja Elmar 
Trink\index[ppl]{Trink, Elmar} tegid esimese kooliteatri, kus ka mina osalesin. 
Kõik see oli nii tore ja ma mõtlesin, et ei viitsi kuhugi kaugele 
kooli minna. Aga matemaatikaõpetaja käis mu vanemate juures, rääkis nad 
pehmeks ja nii see läks. 

\question{Kas sel ajal Nõo legend alles kujunes või oli see juba tuntud paik?}

Jah, oli kindlasti tuntud. Oli teisigi tugevaid koole, 
Tartus-Tallinnas, aga Nõo kool oli üle kõige. Põhiliselt 
sellepärast, et neile oli oma arvutuskeskus ehitatud, nii et sinna tuldi üle 
vabariigi kokku. Samas enamik olid ümberkaudsed maalapsed, kes ei olnud võibolla väga suured geeniused. 

Nõo Keskkoolis oli Nairi 3-1\index{Arvutid!Nairi!Nairi-3-1}, niisugune 
\emph{mainframe}, millele sai perfolinti sisse sööta ja laiprinterist 
tulemuse välja printida. Aga see ei tundunud väga huvitav. Umbes üheksanda klassi poisina 
avastasin Tartu Ülikooli Vanemuise õppehoone\index{Tartu 
Ülikool!Vanemuise tänava õppehoone} keldrikorruselt kabineti, kus oli 
kaks ja pool Apple II\index{Arvutid!Apple II}. Kaks ja pool sellepärast, 
et üks oli kogu aeg katki ja Andres Peiker\index[ppl]{Peiker, Andres}, kes oli 
selle keldri kunn, remontis seda.

Koolipoisina konkureerisin arvutiaja pärast tõeliste 
üliõpilastega nagu Tanel Tammet\index[ppl]{Tammet, Tanel}, Margus 
Liiv\index[ppl]{Liiv, Margus} ja teised. Sain ennast kuidagi 
vahele pista ja enamiku ajast ei käinud enam väga palju 
koolis, vaid olin rohkem Tartus.

\question{Ometigi oli Nõo kool mõeldud sinusuguste harimiseks süvendatult. Kas sul oli vaja veel rohkem süvitsi minna?}

Mis sa seal Nairi juures perfolindiga harid! Saatuse vingerpussina saabus 
aasta hiljem, kümnendas klassis Nõo kooli hunnik 
Agate\index{Arvutid!Agat}, mis olid Apple II kloonid, 
ainult värvilised. Kõige naljakam oli see, et kohalikud arvutiõpetajaid ei 
teadnud nendest midagi ja siis tuli välja, et on üks Tarvi, kes tunneb Agati
protsessorit läbi ja lõhki. Sel olid küll oma operatsioonisüsteem ja 
venekeelsed programmeerimiskeeled, aga sellest polnud midagi. Nii et ühel hetkel 
oli mul arvutuskeskuses oma kabinet ja arvuti. 

\question{Kas selleks piisas Tartus Apple II uurimisest? Kas said sahibide 
vahel noka piisavalt märjaks, et Nõos kunn olla?}

Täpselt nii, pärast õpetasin õpetajaid. 

\question{Kas Agat oli Apple II kloon kuni riistavara disaini ja arhitektuurini 
välja?}

Vähemalt protsessori mõttes oli see kindlasti sama. Ma ei ole väga suur riistvara 
asjatundja, kuigi assembleris\index{Keeled!Assembler} programmeerisin 
vabalt sel ajal. Küllap see oli üsna täpne kloon, aga 
värviline võrreldes Apple IIga. See tähendab, et pilt virvendas kogu aeg 
silme ees. 

\question{Kui sa omale kabineti said, kas siis oli uhke tunne?}

Mis seal ikka erilist oli. Hea oli see, et sain oma asja ajada ega pidanud enam Tartu vahet 
käima.

\question{Kas see õppimist ei hakanud segama?}

Ei hakanud. Mul ei ole sellega kunagi probleeme olnud. Tuleb 
lihtsalt kontrolltööd ja eksamid ära teha ja siis keegi ei õienda.

Nairi peal olid tõsiste inimeste keeled nagu Algol\index{Keeled!Algol}, aga 
lastele õpetati programmeerimiskeeli ROPS\index{Keeled!ROPS} ja 
KÕPS\index{Keeled!KÕPS}\sidenote{Vt ka märkust 3 lk
\pageref{sidenote:ROPS}.}, mis olid eestikeelsed. KÕPSis 
sai programmeerida joonistamist, näiteks kuidas plotter 
liigub: mine üles, mine alla, mine paremale; jäta joon, ära jäta. ROPS oli 
päris programmeerimiskeel. Ma tegin need keeled ka Agati peale ringi, et 
lapsed ei peaks Nairiga tegelema. 

\question{Matemaatika tuli sul lihtsalt, aga kuidas matemaatikahuvi läks üle nii suureks arvutihuviks, et käisid Nõost Tartus arvutis ja portisid programmeerimiskeeli? Mis 
sind selle puhul tõmbas?}

See on hea küsimus, aga mul ei ole head vastust. Arvuti oli selgelt täiesti 
teistmoodi, nagu praktiline matemaatika – rehkendusmasin, mis on kalkulaatorist intelligentsem. Mõtlesin vist
juba siis, et see on paratamatu tulevik ja teistmoodi ei saagi olla. 

\question{Huvitav, et sul on matemaatika ja arvutite seos algusest peale selge 
olnud. Mõnel tekib see seos palju hiljem kui üldse.}

Matemaatiline loogika on olnud kogu aeg üks minu lemmikdistsipliine, arvutid 
ja muusika on väga loogilised asjad. 

Ühel hetkel lõpetasin kooli ära ja läksin TPIsse\index{Tallinna 
Tehnikaülikool}.

\question{Miks sinna? Tartu Ülikool oli ju sulle juba tuttav.}

Mulle tundus, et TPI oli natukene praktilisema hoiakuga, ja aastal 
1987 räägiti Tartu Ülikooli informaatika kohta, 
et seal rohkem ikka joonistatakse tahvli peale. Ja päris matemaatikuks ma kindlasti 
ei tahtnud saada.

Tegelikult olin Tallinna vahet enne käinud. Seal oli Õpilaste 
Teaduslik Ühing\index{Õpilaste Teaduslik Ühing}, kus Peeter 
Lorents\index[ppl]{Lorents, Peeter} tegi matemaatikasektsiooni. Käisin 
Lorentsi juures aeg-ajalt, ta andis mulle kaelamurdvaid 
ülesandeid. Kahekordsete integraalidega 
elu oli huvitav, nii et TPIsse minek tundus loogiline.

\question{Mida sa õppima läksid?}

Automaatikateaduskonda ja eriala oli
LI\index{Tallinna Tehnikaülikool!Automaatikateaduskond!LI} ehk arvutid ja 
arvutitehnika. Seal juhtus kohe mitu asja. 

Kõigepealt ütlesin esimeses programmeerimistunnis, et siia tundi ma rohkem 
ei tule. Õppejõud ei solvunud, sest kirjutasin sissejuhatavas tunnis salaja
ühe programmi valmis ja näitasin seda talle.

Teiseks oli Teaduste Akadeemia Küberneetika Instituudi 
Erikonstrueerimisbüroo\index{EKTA} juhtimissüsteemide osakonnas\index{Teaduste Akadeemia 
Küberneetika Instituut|see{Küberneetika Instituut}}\index{Küberneetika 
Instituut!Juhtimissüsteemide osakond}\sidenote{Esineb ka nimekuju Arvutustehnika Erikonstrueerimisbüroo ja
Arvutustehnika Arendusbüroo, mis paistavad viitavat samale asutusele.} just
leiutatud kooliarvuti Juku\index{Arvutid!Juku}. Nad asusid sealsamas Küberneetika majas, kus olin juba käinud, ja 
septembri esimesel nädalal sadasin sinna sisse. Mul jäi 
õpilaste keskkondade pärast mure, et kui tuleb kooliarvuti, siis võiks olla ka 
õpilastele mõeldud programmeerimiskeeled, ja ROPSi\index{Keeled!ROPS} portimine 
Jukule oli tegemata. Rääkisin Juku tegijatele, et oleks vaja vastavasuunalist 
arendust. Nad lubasid mul enda juures hängida ja nelja kuu pärast 
olin tööle võetud. 

\question{Kas ülikool jäi kõrvale?}

Ei jäänud, käisin korralikult eksameid tegemas. 
Vahepeal, pärast esimest kursust, käisin Vene kroonus ka. Olin viimane 
lend, kes sai kroonusse minna, ja olen selle üle väga õnnelik. Meid viidi Leningradi lähistele, aga kuna
sain puhkpilliorkestrisse ja tegelikult tegin jälle bändi, siis polnud häda midagi. 
Jälle üks kogemus juures. 

Kroonust tulles paljud langevad ülikoolist välja, sest leiavad, et võiks 
midagi praktilist teha ja ennast targaks ajamine ei tasu ära. Mulgi 
oli teise kursuse poole peal kriis, kui mõtlesin, et mul on kohal 
käimata ja et kui eksameid ära ei tee, siis on kõik. Aga tegin 
eksamid ära ja võtsingi selle elustiili, et pühendasin ülikoolile umbes 
kolm nädalat poole aasta kohta. Imesin materjali sisse, tegin eksamid ära ja 
kõik töötas. 

\question{Minu puhul möödus keskkool mängides ja lauldes, sest 
kõik oli lihtne, kuid ülikooli minnes lõppes lihtsus ära. Kas sinul ei 
lõppenud?}

Lihtsus lõppes tõesti. Õigemini olid keerukad esimesed poolteist või kaks 
aastat, kui taoti pähe fundamentaalset kõrgemat füüsikat ja matemaatikat, mis lööb kaane pealt ära. Aga edasi läks erialasemaks 
ja inimlikumaks, õppimine ei olnud enam nii teoreetiliselt tappev. 

\question{Kas ülejäänud aja tegelesid Jukudega?}

Ei, kui kroonust tulin, oli kontorisse toodud juba esimene 286. Oli huvitav aeg, et käisin küll 
tööl, aga tööd oli vähe. Kui 
leidsid endale haltuuraotsi, oli suhtumine väga soosiv. Kõige suurema haltuuraotsa puhul, 
mida mäletan, tuldi koos arvutiga. Sain personaalse arvuti ja 
tööandja eraldas ka kabineti. 

\question{Kes need haltuurapakkujad olid? Kas oskad mõne näite tuua?}

Igasugused. Arvutiga tuli Soome laevaehitaja. 
Pean seda siiamaani kõige vingemaks programmiks, mille ma olen teinud. Ülesanne 
oli selline, et on kümne tekiga sõjalaev, mis vajab 
elektrivarustust; kuskil on jõuallikad ja kuskil tarbijad. Ja nüüd tuleb
hakata nende asjade vahele erineva jämedusega kaableid vedama. Kaablirennid 
on olemas, aga ühel hetkel saab kaablirenn täis. Mis me teeme? Veame 
teistpidi. Aga kes ütleb, et kaablikulu on sealjuures kõige optimaalsem? 

\question{Kas siis oli veel sügav Nõukogude aeg?}

Ei, siis oli juba sula ja hell aeg. See oli pärast kroonut, 1990 või 1991.

\question{Sel ajal ei tohtinud isegi mitte arvuteid 
Nõukogude Liitu tuua, aga sina arvutasid sõjalaevade kaableid.}

Kes seda ikka teadis. 

\question{Kuidas see haltuurapakkuja oskas sinu juurde tulla?}

See õppejõud, kellele esimeses tunnis ütlesin, et 
ma rohkem ei käi sinu juures, leidis mulle otsi. Inimesed 
teadsid mind ja oskasid soovitada. Just ülikooliajal sai väga 
eripalgelisi asju tehtud. Ma olin siis kõva programmeerija, kirjutasin muu hulgas 
oma andmebaasisüsteemi, mis oli FoxProst kordades kiirem. Vanasti oli 
kõvaketta poole pöördumine ränk tegevus, mis võttis 
aega, mitte nagu praegu SSD puhul. Ma kirjutasin andmebaasisüsteemi, millel olid 
fikseeritud pikkusega väljade asemel sujuva pikkusega väljad. See tähendab, et andmeid oli ketta peal täpselt nii palju, kui oli, mitte ei 
olnud eraldatud kindel hulk megabaite. Tõmbasin
keskmise andmebaasi umbes kaheksa korda kokku ja vastavalt sellele suurenes 
töötlemiskiirus.

\question{Kuidas sa kirjeid pakid ja mis saab siis, kui välja pikkus 
muutub? See ei ole ju lihtne.}

Miks see peaks lihtne olema? Mis see geniaalsele programmeerijale ja 
matemaatikule ära ei ole välja rehkendada? Nagu sõjalaevade kaalutud 
graaf, milline on kõige optimaalsem kaablikulu. 

\question{See tegevus läheb otsapidi teadusse, mujal maailmaski ei olnud
andmebaase teab mis palju. Kas sa teadlaseks ei tahtnud saada?}

Ei, mulle meeldis praktiline pool. Lõpuks läksin pika hambaga 
magistrantuuri ja virelesin seal umbes kuus aastat. Siis kui
ainepunktid hakkasid ära kustuma, tegin jõuga lõputöö. Mulle kuiv teooria ei paku 
eriti midagi, mulle meeldib maailma muuta. 

\question{Kas sa olid kuulus ka?}

Ei olnud. Eks ühe või teise tehtud töö tõttu renomee levis ja ka õppejõud
Peeter Lorents\index[ppl]{Lorents, Peeter} levitas sõna, nii et kõik käis
tutvuste ja sidemete kaudu. See ei olnud massiline, tegin umbes kümmekond projekti, aga need olid päris suured.

Tööasju tegin ka loomulikult, aga tööd oli toona vähe 
ja mentaliteet oli selline, et parem olgu inimene olemas ja valmis. Kui tööd 
tuleb, siis saab seda teha. Too kontor, mis on tänase nimega 
Ektaco\index{Ektaco}, oli fantastiline koht. Seal oli umbes 
viiskümmend inimest, tehti riistvara ja tarkvara, \emph{fifty-fifty}. 
Juku oli muidugi nende tehtud. Muu hulgas tegi Elleri-papi 
ehtekarbist valmis esimese hiire maailmas\sidenote[][-1cm]{Arvo 
Eller\index[ppl]{Eller, Arvo} oli Juku loomise eestvedaja (Ants Vill (2010). 
Meenutusi aegadest, kui arvuteid tehti veel käsitsi. Linnaleht (Tallinn), 
46). Kas tema loodud hiir just maailma esimene oli, aga ehtekarbi lugu kordab 
ka viidatud allikas.}.

Pooled inimesed olid \emph{cum laude} TPI lõpetanud, nii et sealne 
ajupotentsiaal oli nauditav. Näiteks kui ülemusel oli sünnipäev, siis
vennad mõtlesid, et teevad kingiks rääkiva papagoi. Tegidki. Seal oli 
briljantseid ja lahedaid tüüpe. 

\question{Mis see töö sisu seal ikkagi oli? Kas ise mõeldi projekte välja?}

Nii ja naa. Üks põhiline valdkond oli 
tööstuskontrollerid: ise mõtlesid välja, ise tegid, ise programmeerisid. Need olid 
\emph{rack}'i-suurused, täna saab samasuguse asja osta Hiinast 
kiibisuurusena. Kontroller koosneb analoogsisenditest ja 
-väljunditest, digitaalsisenditest ja -väljunditest ning nendevahelisest 
loogikast. 
Tollal oli vaene aeg ja Ektaco\index{Ektaco} tehti ühisettevõttena ühe Soome partneriga. Tänase 
päevani teevad nad kassasüsteeme Compucash, mida võib 
baarides aeg-ajalt siiamaani näha. Toona tuli soomlane ja ütles, et tehke mulle 
proovitöö – selline maatriksklaviatuur, et kui baarmen vajutab \enquote{õlu}, on 
kohe olemas. See tuli välja ja koostöö jätkus. Tollal ei olnud lihtne 
tellimusi leida, seetõttu suur osa
inimesi istuski pool aega jõude. 

\question{Ja sina muudkui programmeerisid?}

Mina muudkui programmeerisin. Ektacos\index{Ektaco} olin kokku viis aastat, enam-vähem kogu
ülikooliaja. Aastal 1992 läksin siiski tagasi
nii-öelda peamajja, Küberneetika Instituuti\index{Küberneetika 
Instituut}. Seal tekkis uus rakuke, mis esialgu alustas krüptograafia alusuuringuid. Seltskonnas 
olid mõned teadlase moodi ülikoolipoisid ka, näiteks Ahto Buldas. Ülo Jaaksoo\index[ppl]{Jaaksoo, Ülo} oli 
toonud välismaalt paksu raamatu krüptograafia aluste kohta ja seda me siis koos 
lugesime. Keegi luges peatüki läbi, proovis aru saada ja seletas 
teistele ka. Krüptograafia kui teadus Eestis puudus arusaadavatel põhjustel. Kui Eesti
iseseisvus, oli plats lage ja kuskilt pidi alustama.

\question{Kuidas mujal maailmas oli krüptoga? Mis tolleks hetkeks juba 
olemas oli?}

RSA oli olemas, aastast 1978. Ma täpselt ei tea, sest ei ole ennast 
kunagi krüptoloogiks pidanud. Minu eriala on rohkem nii-öelda 
rakenduskrüptograafia, mitte süvakrüptograafia.

\question{Miks sa sinna läksid? Sul oli Ektacos ju mõnus oma projekte teha.}

Pooled inimesed olid suurepärased insenerid, lõpetanud \emph{cum laude}, aga firmas ei saadud aru, et nende arenguga peaks tegelema. 
Oli väga selge seisukoht, et igaühe areng on tema enda asi. 
Interneti panek firmasse, ajakirjade ostmine või 
inimeste saatmine konverentsile ei tulnud kõne allagi. Pinge 
kogunes ja mingil hetkel, oma sünnipäeval, saatsin kohalikku võrku essee, mis firmas valesti on, mida tsiteeriti
pärast aastaid. Kümme aastat hiljem võeti see välja ja vaadati, et ikka on sama lugu. 

\question{Kuidas see kamp ülejäänud Eesti kogukonnaga kokku käis? Tol ajal pidas osa inimesi juba BBSe.}

Mu hea sõber ja kolleeg Heiki Kask\index[ppl]{Kask, Heiki} pidas ühte 
BBSi ja ma liitusin sellega. Sealtkaudu sattusin lõpuks fidonautide 
sekka ja hakkasin nendega läbi käima. 

\question{Kas see ei olnud sinu jaoks tähtis asi?}

Fidonet ei olnud minu jaoks tähtis, see oli lahe ja andis 
esialgse maigu suhu, aga nii kui tuli Internet, armusin sellesse.

\question{Mis interneti juures nii armastusväärset oli? Meile ja uudiseid 
sai Fidoneti kaudu ka.}

Meil oli esialgu UUCP ja modemiga helistamine mitu aastat, 1991–1993, kui ma 
ei eksi. Sai meili saata, mis oli väga tore, aga mulle jõudis kohale, et kuskil on 
olemas nii-öelda püsiühendusega internet ja suhelda saab reaalajas\sidenote[][-.8cm]{Mõiste \enquote{püsiühendus} oli tol ajal maagilise 
tähendusega: ei unistatud mitte kiirest, vaid pidevalt ühendatud 
internetist. Võimalus kaugete arvutitega vahetult suhelda tundus imeline.}. 
See oli minu jaoks nii võluv, et 
loomulikult tahtsin seda ühel või teisel moel uurida. Nii et UUCP 
aegadel mäletan ennast pühapäeviti kuskil modemi küljes rippumas ja RFCsid\sidenote{\emph{Request For Comments (RFC)} on juba alates 1969. aastast kasutusel olev standardne viis kõiksugu internetiga seotud standardite avaldamiseks ja kokku leppimiseks, RFCd on nummerdatud ja tuntudki oma numbrite järgi. Need sätestavad sõna tõsises mõttes kõike alates Interneti alusprotokollidest kuni tuvide abil side korraldamise (RFC 1149 — A Standard for the Transmission of IP Datagrams on Avian Carriers, D. Waitzman, 4/1/1990, 2 pp.) ja kohvi keetmiseni (RFC 2324 — Hyper Text Coffee Pot Control Protocol (HTCPCP/1.0), L. Masinter, 4/1/1998, 10 pp.).} 
alla laadimas, et need kõik algusest peale läbi lugeda.

\question{Kas see oli tol ajal võimalik?}

Oli küll. RFCde ülemine ots oli kuskil tuhande kandis alles, nii et see ei olnud 
probleem. Osad olid lühikesed ja osad mõttetud, ja oli ilmselge maniakaalsus 
koguda endale hästi palju materjali, et küll ükspäev loen.

\question{Kas seal uues üksuses oli internet sinu jaoks siis infoallikas?}

Eks jah. Sai meili kirjutada, lahe värk. Enne veebi olid 
põhilised FTP-saidid – ei pidanud mõtlema, mis \emph{node}'ist või kust 
mida saad. Mõnikord sai FTPst ka mõne mängu kätte, seal ikka liikus kraami. 
Seal sai ju samamoodi alla ja üles laadida, nagu Fidonetis. 

\question{Kas sa mängisid arvutimänge ka?}

Suur mängumees ma ei olnud, aga noorest peast midagi ikka õhtuti põristasin 
ja täristasin. See oli lõõgastumisviis, mitte huvi. 

\question{Sinu fookus oli matemaatikal.}

Programmeerimisel, mulle meeldis arvutit oma pilli järgi tantsima panna, mitte 
arvuti pilli järgi tantsida. Kui Windows\index{OS!Windows} tuli, 
siis ma kaotasin usu arvutitesse, sest ma ei suutnud enam igat 
bitti kontrollida. Kuni sinnamaani teadsin opsüsteemi, EEPROMi 
tasemel, mis sünnib, aga nii kui Windows tuli, siis kontroll kadus ja mul läks tuju ära.

\question{Kui tekkis Linux\index{OS!Linux}, kas siis tuli tuju tagasi?}

Linux aitas jah Windowsi aja üle elada, aga hulluks 
Linuxi kasutajaks ma ikkagi ei hakanud. Kui läksin 
Ektacost\index{Ektaco} Küberneetikasse\index{Küberneetika 
Instituut}, siis jätsin programmeerimise maha. Viimane asi, mille 
tegin, oli 1996. aastal mail.ee\index{mail.ee}. 

\question{Miks sa selle tegid?}

UNDP\sidenote{\emph{ÜRO Arenguprogramm}. Üheksakümnendatel läks Eesti veel üsna 
selgesti arengumaana kirja ja sai paljudest kanalitest igasugust abi. 
Tänaseks on humanitaarabi mõiste õnneks suuresti ununenud, kuid toona tuli 
seda kõikvõimalikul kujul päris palju ning oli tõesti abiks.} andis selle tegemiseks väikse grandi.
Kõigepealt tekkis hea mõte, et igal soovijal võiks olla meiliaadress. 

Pean alustama sellest, et 1994. aastal sai tehtud 
firma Teleport\index{Teleport} (mitte ajada segi selle sajandi 
Teleportiga!). Meid oli kaheksa tudengit, kellest kuus õppisid välismaal, sest 
neil oli raha. Eesti tudengitel raha ei olnud. Kaheksakesi panime rahad 
kokku, ostsime Soomest portsu modemeid ja tegime sissehelistamiskeskuse, kus 
sai ilma lepinguta 900-numbri\sidenote{Telefoninumbrid 
algusega 900, millele helistamisel kehtis eritariif. Tariifi 
jagati teenusepakkujaga ja see võimaldas tasulisi teenuseid osutada.} kaudu helistada. Saime 
tänu 900-teenusele kohe oma raha kätte. Kommertsiaalse interneti pakkumine oli sel 
ajal vaat et olematu ja laiadele massidele mõeldes täiesti 
puudulik. 

\question{Mis aastal Uninet\index{Uninet} meile tuli?}

Uninet oli juba olemas, aga selleks tuli leping sõlmida. 
EsData\index{EsData} oli ka olemas, me istusime tegelikult nende võrgu peal. 
Kuu hiljem tuli Microlink Online\index{Micolink Online} ja sõi meid massiga 
ära. Teleportist sai mõnesid partnereid kaasates 
Meediamaa\index{Meediamaa} ehk www.ee\index{www.ee}. See oli Eesti 
esimene veebiäri, kus proovisime inimestele rääkida, et kui sind pole 
internetis, pole sind olemas, ja et tulevikus pole sul oma kaubaauto peal muud vaja kui URLi. Nad vaatasid meid nagu idioote, aga nüüd ainult URLiga 
kaubaautosid näebki. 

\question{Miks teie kui programmeerijad firma tegite?}

Pigem olime ikka tudengid. Tarvi selgitas, et niisugust teenust turul ei ole, ja see 
tundus väga lahe, et inimesed saavad juurdepääsu internetile. 

\question{Kas see oli siis puhas missiooniüritus?}

Eks mõttes lootsime raha ka teenida, sest see tundus olematu bisness, 
kus on võimalik kanda kinnitada. Veebiga oli sama lugu. Samas oli see 
paljuski ka missiooni ja eestvedamise asi. Kirjutasin 1996. aastal internetist ka raamatu, mis oli esimene eestikeelne 
selleteemaline originaalteos\sidenote{Tarvi Martens, Vello Hanson. Internet. Ilo, 
1996.}. See oli interneti propageerimine. Samal ajal 
ehitasin riigile andmesidevõrkusid ja TCP/IP 
tehnoloogia laialdane levik tundus mulle sellel kümnendil väga tähtis.

\question{Miks?}

Saavutamaks seda olukorda, kus me täna oleme. 

\question{Kas sul oli peas olemas teadmine, et selline olukord peab ja hakkab olema ning see on hea?}

Ma teadsin, et see on hea. Ma ei teadnud, kui kiiresti ja kui massiliselt see levib, aga 
hüved olid ilmselged. 

\question{Kas su juttu keegi kuulas ka?}

Arvan, et jah. Me oleme näinud, et igasuguse uue tehnoloogia evitamine 
võtab palju aega. Siis on täitsa loomulik, et räägime kahekümne viie aasta tagusest ajast, mille järelmeid võib näha täna. Samamoodi
ei ole ID-kaardi ja e-hääletamise tulemused tulnud 
päeva, kuu või aastaga. Rääkisin kord 
ühele psühholoogile, mida ma teen, ja ta ütles: \enquote{Tarvi, sa oled 
hull. Need asjad, mida sa teed, on inimeste käitumise muutmine. Ühiskondliku 
käitumise muutumine võtab minimaalselt seitse kuni kaheksa aastat aega. Sa ei 
saa oma tibusid lugeda enne, kui jääd vanaks.}

\question{Vähe sellest, tagantjärele on too algne impulss sisuliselt tuvastamatu 
ja seega keegi aitäh ei ütle.}

Ma ei igatsegi seda, see on väga okei. Lihtsalt vaatan 
ringi ja naeratan. 

\question{Sa mainisid, et tegid riigile 
andmesideühendusi.}

Ojaa, see on üks tore lugu. Tegime sel ajal
riigiga palju koostööd standardite ja andmekogude 
vallas, näiteks disainisime Andmekaitse Inspektsioonile\index{Andmekaitse Inspektsioon}. 
Usun, et oli aasta 1993, kui Eesti toll\index{Tolliamet} ja piirivalve\index{Piirivalveamet} tulid Küberneetika Instituuti\index{Küberneetika Instituut} ja ütlesid, et 
oleks vaja piirivalve ja toll üles ehitada. Neil on 
ühised piiripunktid, kus pole mingit sidet, mõnikord isegi mitte 
telefonisidet, ja kas Küberneetika Instituut saaks aidata. 
Joonistasin projekti, klient tuli paari kuu pärast tagasi ja ütles, et mitte keegi ei 
suuda seda projekti ellu viia ja tehke see ise ära. 
Pidimegi hakkama paberimäärimisest tegudele üle minema. 
Koostöös Eesti Telefoniga\index{Eesti Telefon} 
said esimesed ühendused tehtud ja siis hakkas see tegevus mullina 
paisuma. Järgmisena tuli politsei ja riburada teised järel. Me 
tegutsesime Küberneetika Instituudi katuse all, mis oli väga hea ja
amorfne asutus: tahtsid, tegid teadust; tahtsid, tegid äri.

Raha hakkas liikuma, pidime ruutereid 
ostma (kulud jagasime tellijaga – näiteks ostsime piiripunkti ruuteri ja tegime piirivalvega kulud pooleks) ja seega oli vaja moodustada mingi juriidiline keha. Tegime midagi niisugust, mida 
ei tohtinud tegelikult seaduse järgi teha, põhimõtteliselt MTÜ riigiasutustest. 
See MTÜ oli Andmeside Osakond\index{ASO}\index{Andmeside 
Osakond|see{ASO}}, mida juhtis nõukogu, kus oli iga riigiasutuse esindaja.
Raamatupidamistoimkond nurises iga aasta, et sellist asja ei tohi teha, aga 
ülemused ja ministrid ütlesid, et ärme lõhu 
toimivat asja.

\question{See eeldas, et keegi riigi poolel kuulas sind ja 
mõtles kaasa. Kas need olid tippjuhid või IT-juhid?}

Kõigepealt kuulasid IT-juhid, kes rääkisid oma tippjuhtidele. 
Mäletan selgelt, kuidas 31. detsembril istusid toonase 
piirivalve\index{Piirivalveamet} ülema Kõutsi\index[ppl]{Kõuts, Tarmo} 
kabinetis kõik asjaosalised – politsei, piirivalve, toll ja Küberneetika 
Instituut – laua ümber ja kirjutasid lepingule alla. Kõuts veel ütles: \enquote{Ma saan aru, et meil on siin juhikandidaat ka laua taga.} Ma olin siis alles kahekümne viie aastane naga. 

Edasi läks väga huvitavaks, sest meil oli tegelikult olemas selline asutus nagu 
Valitsusside\index{Valitsusside}, kes tegeles erivõrkudega.

\question{Kas nad su peale kurjaks ei saanud?}

Teatav konflikt tekkis jah erinevatel põhjustel, sealhulgas 
koolkondade vastasseis – Jaaksood \emph{versus} Lippmaad. Vanemad inimesed 
teavad seda väga hästi.

Aga juhtus jah, et piirivalvel oli kagupiir täiesti lage, 
seal polnud mingit sidet. Ja selle asemel et minna 
Valitsussidesse, kes pidanuks seda tegema, tulid nad minu juurde ja ütlesid, 
et näed, Tarvi, siin on kümme miljonit\sidenote{Tegu on Eesti 
kroonidega. Arvestades valuutakursse ja  inflatsiooni, on tänases kontekstis 
tegu umbes 1,2 miljoni euroga. Arvutades protsenti riigieelarve (mis oli tänasega võrreldes väga pisike) 
kuludest, maksis too projekt tänases mõistes suurusjärgus 21 miljonit eurot.}, meil on seal lage 
plats, kaheksa piiripunkti on vaja ühendada, tee midagi. Ma ütlesin, et jaa, väga huvitav. Aasta oli 1994 või 1995.

\question{See oli tol ajal suur raha. Kõike oli ju vaja ehitada, kust tekkis 
idee see raha just sidele kulutada?}

Kui oled keset tühja platsi, kus ei ole 
mobiililevi ega mitte midagi, kuidas sa seda piiri pead? Jutt käib elementaarsest telefonisidest ja sõnumivahetusest, mitte suvalisest veebibrausimisest.

Minust sai projektijuht ja me ehitasime tühjale kohale 2,4gigase raadioside
kaheksa mastiga, taldrikud otsa.

\question{See ei ole raadioside disaini mõttes triviaalne ülesanne – kas
õppisid seda kuskilt raamatust?}

Mõtlesin kasutada 
kõrget sagedust ja seega pidi olema otsenähtavus. Aga kuidas seda 
kindlaks teha? Lõuna-Eesti maastik, mäed ja orud. Leidsin 
Maa-ametist\index{Maa-amet} ühe tuttava, kes oli hakanud 
Vene ohvitserikaarte (kõige täpsemaid, mis tollal oli) digiteerima ja oli 
selle kõige huvitavama osa ehk Võrumaa sisse saanud. Ta suutis mulle 
väljastada profiili: andsin talle otspunktid ja tema mulle arvujada. Kirjutasin ise programmi, keerasin maa kumeraks, panin mastid kasvama ja 
vaatasin, kas on otsenähtavus. Selle järgi sai mastide kõrguse 
rehkendada ning mida kõrgem mast, seda kallim oli. 
EMT\index{EMT} ei 
vaadanud mingit profiili, pani 80 meetrit igale poole. Mul aga oli 
52 ja 54 meetrit, mille puhul pidi 
lennutuled lisama ja jälle oli kallim. Sattusin ühe teadjamehe peale 
Eesti Telefonist\index{Eesti Telefon}, kes vaatas tehtut ja ütles: 
\enquote{Kuule, mees, kas sa tegid nädalaga sellise asja? Trassi projekteerimiseks 
läheb poolteist aastat, tuleb jala kõik läbi käia, puud ära kaardistada!} Aga 
mul olid juba mastid tellitud. Ta rääkis, et on olemas Fresneli tsoon – 
saatja ja vastuvõtja vahele ei teki mitte kiir, vaid vorsti moodi 
asi\sidenote{Fresneli tsoon on ellipsoidne tsoon, mida pidi raadiolained 
saatjast vastuvõtjani levivad. Tsooni võivad sattuda ja seega sidet segada 
ka otsenähtavusest väljapoole jäävad objektid.}. See võttis natuke jahedaks 
küll, kuid mastid olid tellitud ja side läks käima. Järgmisel aastal tegin Peipsi 
äärde sama viguri. 

\question{Ühesõnaga sa ei teadnud, et nii ei saa teha?}

Ei teadnud, mõtlesin inseneri mõistusega, kuidas see käib. 

\question{Miks sa üldse kulude optimeerimisega vaeva nägid, kui nii palju raha anti kätte?}

Vabariigi algusajal ei olnud raha palju. Igas vallas pidi olema optimaalne ja tegema parimat, mis teha 
annab. See ei olnud teab mis üleliia suur raha, kulus kõik ära. 

See oli väga tore aeg, kui sai tõesti käegakatsutavalt riigi arengut 
toetada, pealegi minu lemmiktehnoloogia ehk 
interneti osas. 

\question{Kui ma sind kuulan, siis sa olid programmeerija, kuni saabus internet 
ja leidsid, et tuleb hoopis sinna panustada, sest maailm läheb sellest 
paremaks.}

Jah. Programmeerida oskas sel ajal juba üha rohkem inimesi, ma ei olnud enam 
unikaalne ja kaua sa ikka programmeerid.

\question{Mõni programmeerib eluaeg.}

Arusaadav, aga kõrgemad ja üllamad mõtted tundusid 
järjest paremad. Võibolla see on ka isiksuse arenguga seotud. Ausalt öeldes, 
kui olin programmeerija, siis kartsin telefonihelinat, sest ma ei 
tahtnud inimestega suhelda. Ühel hetkel läks see üle. Linna peal teadsid kõik, et kui Martens tuleb jaurama, siis 
proovib kindlasti Küberneetikasse tööle meelitada. 

\question{Kas sa olid Küberneetika Instituudis\index{Küberneetika Instituut} juhtkonnas, et käisid teisi tööle meelitamas?}

Olin ASO\index{ASO} pealik, see sai üle antud 
Informaatikakeskusele\index{Informaatikakeskus}, mis oli RIA\index{Riigi Infosüsteemi Amet} eelkäija.\sidenote{Eesti Informaatikakeskus koos 
Riigihangete Keskusega liideti aastal 2003 Riigi Infosüsteemi Arenduskeskuseks, 
millest 2011. aastal sai Riigi Infosüsteemi Amet ehk RIA.}. 

Aastal 1997 toimus reformatsioon: instituudid kui eraldiseisvad institutsioonid 
kaotati ja pidid liituma ülikoolidega. Küberneetika Instituut jagunes kolmeks: kõige väiksem osa ehk 
andmesideosakond läks informaatikakeskusele, teisest osast sai aktsiaselts ja kolmas liikus
Tallinna Tehnikaülikooli alla. Kuna Küberneetika Instituudis oli 
praktilist tegevust hästi palju, siis kõigest praktilisest moodustati 
Küberneetika Aktsiaselts\index{Küberneetika AS}, mis on siiamaani alles. See 
asutati riigiettevõttena ja nüüd on vist erastatud. 

Küberneetika AS oli väga 
huvitav kombinatsioon. Oli osakond, kus programmeeriti Tolliameti\index{Tolliamet} 
infosüsteeme. Minu osakond oli keskendunud infoturbele nii teoorias, 
praktikas, konsultatsioonides kui ka analüüsides. Ja seal kõrval oli 
meremärgindus ja -navigatsioon ning valgusfooride tegemine. Lisaks
kinnisvarahaldus, aga seda enam pole. 
 
\question{Sinu jutu sisse sigineb tasapisi juhiroll. Mõned inimesed saavad selle maigu suhu ja siis ainult sellega 
tegelevadki. Kas sul ei olnud nii?}

Pidin jõuga maigu suhu saama, sest tegevust oli vaja laiendada ja 
töö tahtis tegemist. Inimesi oli vaja, neid tuli meelitada. 
Küberneetika ASi\index{Küberneetika AS} moodustamisel sai minust selle
arendusdirektor. 

Mõeldi küll, et vaatan laiemat asja ning tegelen ka meremärkide 
ja poidega, aga selle õnge ma ei läinud. Hakkasin arendama infoturbetooteid. 1996. aastal tegime esimese tulemüüri valmis, siis 
VPNi toote ja SSLi \emph{proxy}'sid. 

\question{Kas see oli pärast Meediamaad?}

Jah, see oli hiljem. Infoturbetoodete arendamine läks esialgu väga
hästi. Tegime Linuxi peale veebipõhise liidese 
jubinatele, millest osav insener saab ise tulemüüri teha. Tegime selle veebiliidese kaudu lihtsamaks ja oligi jämedas plaanis 
toode valmis. Eesmärk oli teha keskmisest viis korda odavam toode – keskmine 
tulemüür maksis tollal kolm tuhat dollarit. Ja tuli välja. 

Ilmselt siin oli seos, sest just 
riigiasutused ostsid meeleldi meie tehtud tooteid. \enquote{Tarvi 
tegi võrgud, nüüd müüb neile turva ka peale.}

\question{Enamasti tekib riikides soov teha omale privaatne turvaline
internet. Kas Eestis seda ei mõeldud või üritati teha ja ei tulnud välja?}

Loomulikult üritati, tegime 
VPNi toote, mis oli võrreldes praegustega unikaalne. Kui kast 
oli võrgul ees, siis ei saanud internetti, see lasi ainult teise omasuguse juurde. 
Näiteks igas maakonnas on kontorid, kus paned rohelise kasti võrgule ette ja kamba peale on üks tulemüür ka, näiteks Tallinnas, ja ainult läbi selle tulemüüri saab 
välja. Muidu on täielikult sisevõrk. 

\question{Sa kirjeldad ju X-teed. Arhitektuuri mõttes tundub 
väga sarnane.}

Ei ole, sellel pole andmete semantikaga mingit pistmist. 

\question{Kas see tähendab, et projektide vahel ei toimunud mingit risttolmlemist?}

Ei, see oli privaattorude ehitamine, X-tee on OSI tasemetes 
natuke kõrgemal.

\question{Kas sa tol ajal tegelesid interneti propageerimisega paralleelselt 
edasi või oli see lihtsalt üks faas?}

Siis oli turul juba piisavalt tegijaid ja ma ei tundnud vajadust 
sellega tegeleda. Pigem oli minu jaoks saabunud järgmine faas teha 
internet turvaliseks. Kolmas elementaarne faas 
oli osapooled internetis identifitseerida, et saaks ka
\emph{business}'it teha. 

\question{Kust tuli mõte, et internet peab turvaline olema?}

Hakkasime teoreetiliselt turvalisusega tegelema juba 
1992. aastal. Kontseptsioon, kuidas ja miks seda 
teha, oli mulle tuttav. Meie roheliste kastide puhul oligi 
eesmärk puhas ja turvaline andmeside, muud midagi. Minu sõnum oli see, 
et ärme teeme eraldi X.25 võrku, sest üle avaliku interneti toimetades on palju 
kuluefektiivsem.

\question{Kuidas sul ikkagi tekkis mõte, et interneti turvalisus on 
probleem, mida tuleb hakata lahendama? Kas keegi luuras või häkkerid kiusasid? Kust 
probleem tekkis?}

Probleem on olnud aegade algusest. Ja olles infoturbega algusest 
peale tegelenud, oli selge, et võrkudes on infoturve teemaks. See on 
elementaarne. 

\question{Kui mina oma ajaloo peale mõtlen, siis minu jaoks ei olnud. Ehitasin pikalt oma asju ja võrke, üldse mõtlemata, et need võiksid ka turvalised 
olla.}

Infoturve oli minu eriala, ükskõik mis 
ametis, ja see sai alguse tolle 
ühe raamatu kooslugemisest.

\question{Lisaks on sul matemaatiku, programmeerija ja antenniehitaja 
taust, nii et saad päris süvitsi minna.}

Jah, ma olen kirjutanud Jukule\index{Arvutid!Juku} püsimälu. 
Minu töö puudutas tähtede joonistamist ekraanile, EEPROMi tasemel 
sai ESC-käskudega aknaid teha. 

\bigskip
\noindent\rule{.3\textwidth}{.7pt}
\bigskip

Mõtlesin, mis lugusid veel võiks rääkida, ja mõned tulid meelde.

Ma ei olnud Tallinna poiss ja Jaak Loondet\index[ppl]{Loonde, Jaak}, keda 
mitmed varasemad rääkijad on maininud, ei tundnud. Küll aga kuulsin temast 
Fidoneti inimestelt. 

Juhtus niisugune lugu, et varajastel üheksakümnendatel, kui 
Eestis ei olnud isegi piisavalt leiba, oli talongide peal\sidenote{1980ndate
lõpust kuni umbes 1993. aastani, kui vaba turg hakkas enam-vähem 
toimima, müüdi elementaarseid toidu- ja tööstuskaupu, sealhulgas 
periooditi leiba, üksnes talongide esitamisel.}, otsustas 
Soome Rotary klubi Eesti koolidele natuke arvuteid kinkida. Ilmselt oli PC-aeg peale tulnud ja ühel tehaseinimesel jäi komptuureid üle. 
Need olid kummalised masinad, aga lahe oli see, et need olid võrgus ja emaarvuti ka. Soome Rotary tegi haridusministeeriumile
ettepaneku kinkida need Eesti koolidele. 
Minu mentor Peeter Lorents\index[ppl]{Lorents, Peeter} oli sel ajal 
ministeeriumis mingi tegelinski ja sattus selle peale. 
Läksimegi kolmekesi – autojuht, Peeter ja mina eksperdina – 
kohapeale vaatama, mis arvutid need on ja kuidas töötavad. Tõime need Eestisse ja siis tekkis küsimus, mida me 
nendega peale hakkame. 

\question{Kui palju neid masinaid oli?}

Kuus-seitse tükki, terve klassitäis. Eesti peale ei olnud palju, aga 
Rotary klubi sai endale linnukese kirja: Eestit aidatud, heategevus tehtud. Ja 
siis meenuski mulle Jaak Loonde\index[ppl]{Loonde, Jaak}. Sain temaga kokku ja Jaak 
oli kohe nõus sellega tegelema, silmad peas põlemas nagu ikka. Mõne aasta pärast saime kokku 
ja küsisin, kas masinatel pruukimist ka oli, ja tuli välja, et need olid väga 
hästi vastu võetud ja nendega igasuguseid vigureid tehtud. 

\question{Nii et Jaak toimetas edasi ka pärast seda, kui 
enamik temast rääkinuid olid koolipoisieast välja kasvanud?}

Jaa, ta oli legendaarne, toimetas arvutitega elu lõpuni. Tema põhiline soov oli, et lapsed saaksid näpud arvuti külge.


\bigskip
\noindent\rule{.3\textwidth}{.7pt}
\bigskip

1993. aastal tegin ma esimese 
jututoa, mille nimi oli Anna\index{Jutukad!Anna}. See oli umbes samasugune asi nagu praegu Messenger: hulk inimesi logib sisse ja hakkab omavahel suhtlema. 

\question{Kas see käis sinu enda tehtud tarkvara peal või said selle kuskilt?}

Sain kuskilt tarkvara ja tõlkisin käsud eesti keelde, käsk algas 
punktiga. Olin tollal Göteborgis neli kuud asumisel ja mul polnud 
seal suurt midagi teha, nii et putitasingi seda jututuba. 

Anna jututoas kaitsti isegi üks Tallinna 
Tehnikaülikooli\index{Tallinna Tehnikaülikool} diplomitöö ära – kaitsja asus 
Uus-Meremaal, õppejõud kogunesid jututuppa.

\question{Mis oli jutukate fenomen? Seal käis igasugust rahvast, mitte ainult tehnikud.}

See oli \emph{community building}, umbes samasugune grupp nagu Fidonet. Edasi 
tekkisid OK \index{Jutukad!OK} ja 
Cafe\index{Jutukad!Cafe}\sidenote{Cafe pärisnimi oli \emph{The Roadkill 
Cafe} ja see asus aadressil \texttt{ns.uninet.ee:5555}. Selle pani 23. 
veebruaril 1996 NUTSi (\emph{Neil's Unix Talk Server}) versiooni 2.3 
lähtekoodist püsti Indrek Siitan\index[ppl]{Siitan, Indrek}.} jutukad. Meil oli 
isegi Anna kasutajate kokkutulek Viljandi lähistel, mida 
Jüri Ruut\index[ppl]{Ruut, Jüri} veab siiamaani, nüüd küll ee.kevade nime all.

Jutukates käis suvaline rahvas, seal ei olnud õnneks üksnes tehnofriigid, vaid ka tütarlapsi. 

\question{See pidi siis olema väga vajalik teenus, sest 
mittetehnofriigile pidi see tehnika olema paras barjäär.}

See oli tegelikult lihtne, kui ainult terminalile ligi said. Panid 
\verb|telnet anna.ioc.ee|\index{Masinad!anna.ioc.ee} ja läks. 

\question{Kas sa hoidsid jutukat Küberneetika Instituudis\index{Küberneetika 
Instituut}?}

Pean tunnistama, et jah. Alustasime Küberis Unixi pruukimist aastal 
1992, kui tõime Soomest flopidega Linuxi\index{OS!Linux}. Teistmoodi ei 
saanud seda kätte. 

\question{Kas otse Linuse käest?}

Enam-vähem. Proovisin tollal Unixi kultuuri aretada. Kord ostsime hirmsa
raha eest ühe Suni. Kui küsiti, mis sellele nimeks panna, siis ütlesin suvaliselt 
\enquote{keeks} ja tekkiski igavesti kuulus FTP-server keeks.ioc.ee\index{Masinad!keeks.ioc.ee}. Pärast pidin \enquote{keeksi} lahti mõtestama ja 
arvasin, et see on Küberi Esimene Eestimeelsete Kasutajate Server.

\question{Tuleme korraks jutukate juurde tagasi. Selleks et sotsiaalvõrk 
lendu läheks, peaks olema algne seltskond. Kes need inimesed olid ja 
kuidas sa selle võrgustiku tekitasid?}

Ma täpselt ei mäleta, aga küllap rääkisin sõpradele, nemad oma 
sõpradele ja nii see vaikselt levis. Ühtegi erilist 
aktsiooni ei mäleta, piisas sõprade ringist, aga lõpuks läks 
ring väga laiaks – üle poole või rohkemgi olid 
täiesti tundmatud inimesed. 

Annaga\index{Jutukad!Anna} juhtus nii, et ühel hetkel vaatasin, et 
teised jutukad hakkavad ka tekkima, ning panin selle pidulikult kinni. 
Anna matused olid eraldi sündmus. Asja peab ära lõpetama, mitte laskma 
sel lihtsalt hääbuda. 

Kui Unixi juurde tagasi tulla, siis oli meil 
Eesti Unixi Pruukijate Selts ehk EUPS\sidenote{Selts asutati 1994. aastal ja sellel oli 62 
asutajaliiget. Asutavasse toimkonda kuulusid lisaks Tarvile Andres 
Bauman\index[ppl]{Bauman, Andres}, Margus Liiv\index[ppl]{Liiv, Margus}, Jaanus 
Pöial\index[ppl]{Pöial, Jaanus} ja Anto Veldre\index[ppl]{Veldre, Anto}.}. Teised tahtsid panna \enquote{Kasutajate Selts}, aga EUKS kõlab 
halvasti ja mina ütlesin, et peab ikka pruukima. Meil oli 
Tõraveres isegi kokkutulek.

\question{Miks te Soomest Linuxi\index{OS!Linux} tõite? Kas te ei tahtnud Sunile 
raha anda?}

Ühelt poolt ei tahtnud raha anda ja teiselt poolt oli see uus värske 
tuul, mis oli vaja ära proovida. Linuxi eelis oli see, et see käis 
PC peal. 

\question{Linux on praeguseni hädas oma kõrge sisenemisbarjääriga, inimestel on 
raske sellega liikuma saada. Kuidas toona oli?}

Me rääkisime Linuxist serveri kontekstis, tööjaama-Linux ei olnud teema. 
Tol hetkel pidi raha eest ostma mingi tarkvara, et failiserverit ringi 
ajada. Ma ütlesin, et ärme tee seda! Panen Linuxi püsti, kasutame 
seda. 

\question{Tol ajal taheti igasuguste asjade eest, nagu 
veebiserver, raha saada ja kommertstarkvara oli väga kallis.}

See oli ropult kallis, kuna kirjutajaid oli vähe ja see oli eksklusiivne asi. Kui hakkasime Küberis\index{Küberneetika 
Instituut} 1996. aastal tegema esimesi tulemüüre nimega 
Barrikaad\index{Barrikaad}, siis tol hetkel maksis keskmine tulemüür maailmas 
kolm tuhat dollarit. See on ju absurdne. Me võtsime Linuxi, tegime näo pähe ja 
müüsime viis korda odavamalt.

Seoses kogukondadega ei saa mainimata jätta 
sellist olulist \emph{community}'t nagu WC Fauna\index{WC Fauna}. 
Raske öelda, mis see täpselt oli või kes sinna kuulusid, see oli rohkem 
mõtte- ja eluviis. Selle liikmed tegid igasuguseid asju, pahatihti käisid 
lihtsalt kõrtsides või tegid niisama nalja ja ehitasid lumelinna.

Vanasti olid kompuutrimessid tähtsad\sidenote{Aastatel 1993–1999 
korraldati Eestis igakevadist arvuti-, side- ja bürootehnika messi 
\enquote{Kompuuter}. Tegu oli olulise kogukondliku ja 
müügiüritusega, mida Päevaleht tituleeris lausa infotehnoloogia laulupeoks.}. Ühel messil pakuti meile oma boksi ja pidime selle 
kuidagi sisustama. Boksis oli üks kompuuter, mis luges sekundeid tuleviku 
alguseni, ja WC Fauna leviala kaart, milleks oli punaste läbipaistvate 
vorstinahkadega kaetud Eesti kaart, politseilindiga ümber tõmmatud. 

\question{Tänapäeval läheks selline asi kunstiprojektina kirja.}

Jah, ilmselt küll. Eks see oli häppening, igasuguseid erinevaid asju sai tehtud. Näiteks oli 
IT-inimeste kokkutulek 
OK-fest\index{OK-fest}\sidenote{1994. aastast Eesti Infotehnoloogia- ja 
Telekommunikatsiooniettevõtjate Liidu\index{Eesti Infotehnoloogia- ja 
Telekommunikatsiooniettevõtjate Liit} korraldatud suvine kokkutulek.}, 
kus \emph{community} kokku sai. WC Fauna nimi sai alguse sellest, et ühel 
OK-festil oli vaja jalgpallimeeskond kokku panna. Mõtlesime, et FC Flora juba 
on, paneme siis WC Fauna. Aga see oli ka vist viimane kord, kui jalgpalli 
mängisime. 

\question{Sinu jutust kumab läbi palju 
ühistegevust, aga tavaliselt ei tegeleta arvutitega sellepärast, et 
meeldib teiste inimestega suhelda. Kuidas sul arvutite ja inimeste suhe 
kokku käib?}

Ma olengi imelik loom, kellest pole kunagi aru saadud. Üks tuttav 
psühholoog ütles: \enquote{On olemas insenerid ja on olemas kunstiteadlased, 
aga kumb sina oled, aru ei saa.} 

Inimene areneb vaikselt. Nagu ma mainisin, siis algusaegadel olin 
introvert, kes istus nurgas ja programmeeris ning kartis, kui telefon 
helises. Hiljem hakkasin inimestega suhtlema, seejärel ühiskonda nägema ja sealt tulid ka riigi- ja 
vaat et maailmalaiused asjad. 

\label{sisu:everyday}Üks lugu, milles maksab kindlasti rääkida, on see, kust Skype\index{Skype} tegelikult alguse sai ja kus see kamp 
kogunes. Ilmselt nii mõnigi mäletab, et umbes 1994. või 
1995. aastal oli lehes kuulutus \enquote{otsime programmeerijat, maksame 
viis tuhat krooni päevas}\sidenote{Teiste allikate alusel oli kuulutus lehes 
1999. aastal, mis on loogilisem – muidu jääb Skype'i asutamise ja 
Bluemooni Tele2-seikluse vahele liiga pikk paus.}. Viis tuhat krooni oli kaks kuupalka. 
Kuulutuse tagamaa oli see, et Tele2\index{Tele2}, kes oli juba Eestis olemas, ja 
Bonnier Media\index{Bonnier Media} sepitsesid Rootsis 
nii-öelda uue põlvkonna portaali
everyday.com\index{everyday.com}. Niipea kui nad uudise välja lasid, et 
niisugune portaal tuleb, tõusis nende turuväärtus poolteist miljardit. 
Absurdne, aga nii see oli. Eestisse tuldi jutuga, et meil on tiimid Itaalias, Rootsis ja Taanis ning kõik on juba tükk aega programmeerinud. Kahte 
programmeerijat Eestist on veel vaja, siis saab kõik korda\sidenote{Eestis 
töötas toona Tele2s Stefan Öberg\index[ppl]{Öberg, Stefan}, kes hiljem täitis Skype'is 
mitmeid juhtivaid rolle. Tema juhataski viimase kahe tegija otsijad 
Eestisse.}. 

Mina sattusin seda otsingut nõustama ja lõpuks projektijuhiks, kes 
pidi need inimesed välja valima ja asjad ära tegema. Valisin välja 
Bluemooni\index{Bluemoon} poisid. Sõitsin kõik need Itaalia, 
Taani ja Rootsi kontorid läbi ning sain aru, et peale Rootsi, kus oli tehtud väike 
andmebaasimootor, olid kõik teised tiimid tootnud täielikku kräppi. Nii ei 
jäänudki projekti päästmiseks muud üle, kui kogu värk ise teha. Bluemooni
poistel ei olnud probleem see käsile võtta ja nädala-paariga 
portaal kokku veeretada, kuigi nad PHPd\index{Keeled!PHP} ei tundnud.

Tulevane miljardär\sidenote{Tarvi peab silmas Niklas 
Zennströmi\index[ppl]{Zennström, Niklas}.} oli Tele2s projektijuht ja talle 
hakkasid need poisid meeldima. 

\question{Sina olid portaalis projektijuht. Kui tegid portaali valmis, kas siis 
ei tekkinud mõtet, et peaks suures Rootsi kontsernis kosmilist karjääri tegema?}

Absoluutselt mitte, see oli kõrvaltegevus – aitamisprojekt ja raha 
maksti ka.

\question{Mis su põhitegevus oli?}

Ehitasin riigivõrku ja juhatasin neid 
vägesid. Sinna kõrvale mahtus veel üks kõrvaltegevus, 
mail.ee\index{mail.ee}, mille omanikuks sai ka lõpuks Tele2. 

\question{Kas mail.ee all oli standardne SMTP-server?}

Täpselt nii. Alustuseks oli ilma näota 
meilboks. See tähendas, et igaüks sai endale aadressi luua, aga pidi enda 
meilerit kasutama. Teine arengufaas oli sellele veebi nägu pähe teha, seal oli veebimeiler ka. See sai täitsa ise kirjutatud, all 
oli loomulikult standardne kompott. 

\question{Nii et sa ei läinud ise sinna maailma midagi leiutama, vaid võtsid 
tükid ja ladusid kokku?}

Jaa, see on mul kogu aeg veres olnud. Ühel hetkel sain aru, et 
programmeerimine on üldse kurjast, sest kõik on juba ära tehtud. 
Tegelikult on kunst tükid üles leida ja oskuslikult kokku panna. 
Tänapäeval on tükkide arv muutunud hoomamatuks ja väga raske on neist midagi kokku panna. Ilmselt on 
tekkinud kildkonnad ja voolud. Kunst on muutunud.

\question{Mis üldse on tänapäeval sinu jaoks programmeerimine?}

See kipub olema järjest igavam asi, sest vanasti oli 
see selgelt loometöö. Nii kui hakkasid tulema igasugused 
mudelid ja RUPid\sidenote{\emph{Rational Unified Process (RUP)}. RUP oli 
1990ndatel suurorganisatsioonides levinud tarkvaraarenduse raamistik, 
mis keskendus arendusprotsessi keerukuse vähendamisele läbi standardiseeritud 
rutiinide. Et samal ajal üritati keerukat tarkvara tarnida harva ja suure 
pauguga, võis RUP küll teha projektid paremini kontrollitavaks, kuid ei vähendanud kuigivõrd arendajate frustratsiooni.}, siis hakkas see
järjest rohkem tunduma kraavikaevamisena. Arhitektid joonistavad asja ette ja sina lihtsalt täidad 
funktsiooni. See ei ole eriti keeruline. 

\question{Ometigi ehitatakse igasuguseid hullusi, nagu tekstiterminalis 
video mahamängimine.}

Loomulikult, nalja pärast saab ikka teha. Ma räägin raha 
eest või tööstuslikust programmeerimisest, kus tuleb konkreetset asja teha. 
Vanasti olid mees nagu orkester ja mõtlesid ise välja, kuidas arhitektuur 
võiks välja näha. Tegid oma äranägemise järgi ja keegi ei kobisenud. Nüüd 
on arhitektid. Loovust on 
programmeerijatele jäänud kindlasti vähemaks. 

\question{Kui me juba selle teema juurde jõudsime, siis küsin ka sinu käest, 
milline on ilus kood?}

Ilus kood on loetav kood, siin ei ole kahtepidi mõtlemist. 

\bigskip
\noindent\rule{.3\textwidth}{.7pt}
\bigskip

\question{Kuidas sündis ID-kaart?}

Küberis\index{Küber} tegutsesin ma kahel rindel. Ühelt poolt ehitasin võrke, 
aga olin ka kogu aeg infoturbe ja krüptograafia keskel. Lisaks 
võrguturbele, mis oli sel ajal väga oluline, tundus avaliku võtme 
krüptograafia huvitav ala ja pakkus oma rakenduste poolel pinget. 
Küberis sai jälgitud, kuidas 1995. aastal vist Rootsi Post alustas oma 
ID-kaardi väljalaskmisega ja avaldas ID-kaardi profiili. Päris vara, 
üheksakümnendatel, toodi mulle Ektacosse\index{Ektaco} Schlumbergeri 
kiipkaardid ja paluti vaadata, mis elukad need on. 
Kirjutasin sinna peale programmi nimega \emph{Clevercard}.

\question{Kas see oli Java kaart?}

Javat polnud veel väljagi mõeldud, 
krüptokaarte ka mitte. Mälukaart see ei olnud, protsessor 
oli sees. Sellele kiipkaardile sai käske anda, näiteks „tee fail“. Kõige all oli 
kaardi operatsioonisüsteem. Baidid ajasid sisse, baidid tulid vastu ja ma kirjutasin 
PC-le programmi, millega seda sai mõnusalt teha. 

Aeg läks vaikselt edasi ja see oli umbes 1996.
aastal, kui tegin Äripäeva lahti ja esimesel leheküljel oli pildil Kaja 
Kuivjõgi\index[ppl]{Kuivjõgi, Kaja}, keda ma tundsin ja kes 
oli siis Kodakondsus- ja Migratsiooniameti\index{Kodakondsus- ja 
Migratsiooniamet} asedirektor. Pildi juures oli kirjas, et riik 
planeerib uut dokumenti ja et esimesed passid, mis võeti kasutusele 1992. aastal, saavad 2002. aastal läbi. 
Sinna on viis aastat aega ja KMAs on moodustatud töörühm, kes 
uurib variante millegi uuega välja tulla. 

Võtsin Kajaga ühendust ja ta 
näitas mulle töörühmas arutatud materjale. Kui olin need 
läbi vaadanud, sain aru, et nende tehniline teadmus on üsna allpool 
nulli. Seal räägiti kiibiga varustatud vöötkootidest. 

\question{Mida nad teha tahtsid? Uut ja paremat passi?}

Nad mõtlesid ikkagi kaardi suunas, aga milline see võiks olla – 
kas kiibiga varustatud vöötkood või mis – ei olnud selge. 

\question{Mina olen kogu aeg arvanud, et kaardi pakkusid
välja tehnikud, mitte ametnikud.}

Soov oli tol hetkel väga hägune ja igasugused 
variandid olid laual. Aga oli selge, et kuna tekib suurem 
passivahetus, siis on võimalik inimesi üllatada millegi uuega ning vaadata, mis 
maailmas tehnoloogia vallas toimub. 

Oli päris selge, et KMA\index{Kodakondsus- ja Migratsiooniamet} 
töörühmal ei ole mõtet jätkata. Tehti ettepanek moodustada 
laiem töörühm ja võtta laua taha ka eksperte: pangad, 
telekomid, riigisektori ja Küberi\index{Küber} inimesed.

\question{Kas tänapäeval tundub veider, et riik võtab pangad ja 
telekomid laua taha sellist dokumenti arutama?} 

Absoluutselt mitte. Ei tundu praegu ega tundunud ka tol ajal. 
Laiapõhjaline koostöö riigi- ja erasektori vahel on meile alati edu 
toonud nii ühes kui ka teises. 

\question{See on haruldane asi, mida mujal sageli ei näe.}

Eesti on nii väike riik, et põhimõtteliselt tead kõiki, kes midagi teavad, ja 
ei ole mõtet kedagi kõrvale jätta sellepärast, et ta on parasjagu erasektoris. Me räägime ikkagi eksperditeadmisest ja 
ekspertide kogumist, mitte institutsionaalsest asjast. 

Tuligi töörühm kokku ja arutas asju. Telliti kaks tööd, 
KMA\index{Kodakondsus- ja Migratsiooniamet} maksis. Ühe töö viis läbi 
aktsiaselts Aprote\index{Aprote}, kes uuris, milleks kõigeks 
võiks seda kaarti kasutada. Nad läksid näiteks tanklaketti ja küsisid, 
mida nemad tahaksid. Tulemus oli muidugi väga ulmeline, aga turuootuste uurimine oli 
vajalik tegevus, vaat et kohustuslik samm. 
Teine töö, mida tegime meie Küberis\index{Küber}, oli 
tehnoloogiline ülevaade, milleks kiipkaardid on suutelised, kaasa
arvatud see, mida on Rootsis ja Soomes tehtud. 
Millised on profiilid ja tehnoloogiad, sealhulgas Microsofti 
PC/SC\sidenote{\emph{Personal Computer/Smart Card} – spetsifikatsioon 
tarkade kaartide integratsiooniks arvutustehnikaga.}. 

1996. aastal joonistasin projektiplaani, et neljateist kuuga 
toome kaardi välja, kaasa arvatud pilootprojekt ja muu säärane. Võttis see siiski viis 
aastat, sest see oli väga oluline samm ühiskonnas 
ja vajas pikemat kaalumist. Peale selle tuli seadusi juurde ja ringi teha. 

\question{Kas kõike seda vedas KMA?}\index{Kodakondsus- ja Migratsiooniamet}

Ei, kindlasti mitte. Digiallkirja seadust näiteks vedas 
Majandusministeerium\index{Majandusministeerium}.

\question{Kuidas nii? Asi ju algas 
dokumendi väljastamise vajadusest ja siis äkki tahtis Majandusministeerium digiallkirja 
teha?}

Kindlasti oli suunanäitajaks Saksamaa, kes võttis esimesena vastu 
digiallkirja seaduse, mille pealt Eesti oma on paljuski maha viksitud. Meie 
digiallkirja seadus või vähemalt selle kavand nägi ilmavalgust enne, kui 
oli olemas Euroopa 1999. aasta direktiiv\sidenote{Euroopa Parlamendi ja nõukogu 
direktiiv 1999/93/EÜ.}. Seetõttu oli meie seadus mõnevõrra erinev. Euroopa 
direktiiv lubas igasuguseid lahjasid allkirju ja sellist koledust nagu 
näpuga ekraanile kirjutamist ning ei läinudki tööle. Seepärast tuli ka 
lõpuks eIDAS\sidenote{\emph{electronic IDentification, Authentication and trust 
Services - eIDAS} – Euroopa Parlamendi ja nõukogu määrus 910/2014 e-identimise ja e-tehingute kohta.}, et direktiiv 
oli väga lahja. Kehitati õlgu ja ei kasutatud, tehti lahjasid allkirju ja 
öeldi, et nüüd ongi kõik hästi. 

Meie seadus ütles algusest peale, et ainult 
kvalifitseeritud allkirjad\sidenote{Lihtsalt öeldes on kvalifitseeritud 
elektrooniline allkiri selline allkiri, mida võib pidada võrdväärseks 
omakäelise allkirjaga. Keerulisemalt on öeldud eelviidatud eIDASi direktiivis ja 
selle rakendusaktides.} on aktsepteeritud, ja mingeid lahjasid allkirju ei 
tunnistatud. Neid seadus ei käsitlenudki. 

\question{Kas seaduse väljatöötamises osalesid ka eksperdid või oli 
see Majandusministeeriumi tehtud?}

Eksperdid olid kaasatud. Oli töörühm, kus osalesid
inimesed krüptoloogist kuni juurateadlasteni. Nad tegid seda tööd ligikaudu
kaks aastat, nii et see ei tekkinud niisama, vaid mõeldi väga põhjalikult 
läbi. Kuskilt mujalt kui Saksamaalt ei olnud šnitti võtta. Nii mõnigi 
seadusepunkt oli inspireeritud nii-öelda krüptograafide mõtlemisest. 

\question{Sinu jutust ei kõla läbi kõikehõlmav õilis visioon 
sellest, kuidas ühel päeval sünnib Eesti digiühiskond ja kõik saab e-teenuste 
abil uueks loodud.}

Eks see võibolla kuskil ajusopis oli, aga mis sellest ikka rääkida, asju 
tuleb teha. 

\question{Seda ma peangi silmas, et liikumine toimus samm-sammult ja tegeldi 
konkreetsete asjadega.}

Jah, kasvõi seesama ID-kaardi väljatoomine. Võib ju digiallkirja seaduse vastu 
võtta (mis aastal 2000 ka vastu võeti), aga kui inimestel ei ole vahendit, 
millega digiallkirja anda, siis pole seadusel suuremat mõtet. 
Euroopas valitses ka selle direktiivi tegemise ajal nägemus, et 
kommertsfirmad hakkavad sertifikaati müüma ja seetõttu on vaja neid 
reguleerida. Kuidas see võiks käia? Teed turule putka ja hakkad sertifikaate 
müüma: suured ja väikesed sertifikaadid, punased ja kollased? 

See ettenägemisvõime oli meil küll, et niisugune visioon, et müüme inimestele 
sertifikaate, neid ostetakse ja kuidagi tekib kasutus, on üdini vale. Selles mõttes oli näiteks Soome, kes tegi ID-kaardi 
mittekohustuslikuks ja pani kohe hinnaks nelikümmend eurot. Siis juhtub see, et teenusepakkujad ei hakka ID-kaarti toetama, sest 
nad teavad, et inimestel ei ole seda (viiel protsendil võibolla on). Ja inimestel ei ole kaarti, sest teenuseid ei 
ole. Siis ongi nokk kinni, saba kinni ja mudel, mis ei toimi. 

Kõigepealt peab elektroonilise identiteedi 
taristu looma ja siis võibolla hakkavad asjad juhtuma. Samamoodi nagu ei 
saa proovida kuskil metsa sees müüa kilomeetrit maanteed kohalikule 
metsaelanikule. 

\question{Su jutust kõlab läbi üsna suur usaldus ekspertide vastu. Poliitika eest vastutava inimese ja krüptoloogi 
vahel pidi olema usalduslik vahekord, et viimasel lasti seadusesse punkte kirjutada.}

Skepsis on väga raske tekkima, kui laua taga on Eesti paremad pead – misasja sa ikka kahtled või kõhkled. Mida targemaks inimesed saavad ja 
mida rohkem eksperte on, seda rohkem tekib diskussiooni. 

\question{Kui suur see ekspertide ring oli, kes töörühmades käis ja seda 
ideed kujundas?}

Viis kuni kümme võtmeinimest.

\question{Ja kogu tarkus tugines tollele salapärasele raamatule, mis 
Küberis oli?}

Oo ei, see raamat oli lihtsalt algus. Me ei räägi ainult 
krüptoloogiast või infoturbest. Näiteks ID-kaart ei puuduta ainult 
infoturvet, vaid väga paljusid rakenduslikke ja isegi sotsiaalseid aspekte. Ei saa rääkida, et krüptograafia päästis maailma. 

\question{Sagedasti inimesed arvavad, et kui asjad saaks 
ära krüptida, siis olekski maailm päästetud.}

Ma jään selle juurde, et ei saa kilomeetrit maanteed müüa 
külaelanikule, sest ta küsib: \enquote{Mis ma teen selle maanteega?} – 
\enquote{Hakkad autoga sõitma.} – \enquote{Aga mis see auto on?} See on 
ufo müümine, täiesti mõttetu tegevus. Sa ehitad teed valmis, lased autod müüki, 
paned sõidukoolid püsti ja siis ühel päeval võibolla inimesed avastavad, et 
transpordist on kasu. Aga kui hakkad sellest pihta, et proovid 
igaühele juppi maanteed müüa, siis see ei toimi. 

\question{Küberneetika Instituut\index{Küberneetika 
Instituut} ja selle järelmid on väga pikalt Eestis olulist rolli mänginud. 
Sina oled olnud seal sees ja, mis veel olulisem, ka sellest väljas. Kas sa oskad 
öelda, mis maagiline asi selle asutuse nii võimsaks teeb?}

Nagu ma mainisin, siis 1997. aastal jagunes Küberneetika Instituut kolmeks. Osa 
läks ülikooli alla, osast moodustus aktsiaselts ja andmeside osa läks riigile. 
Võibolla kõige nähtavam osa IT-inimeste jaoks ongi Küberi instituudi või 
aktsiaseltsi ehk infoturbe ja programmeerimise osa. Seal tehakse meremärke 
ka, aga need on merel ja ei paista välja.

Mul on olnud au omal ajal umbes kolmkümmend inimest sinna tööle võtta ja 
Tartu labor\index{Cybernetica!Andmeturbelabor} asutada, mis on nüüd inimeste arvu mõttes nüüd isegi suurem, kui Tallinn. Need olid väga toredad ajad. Aga fenomen seisneb selles, et Küberit peetakse põhimõtteliselt ainukeseks firmaks 
Eestis, kes oskab turvaliselt programmeerida ja teab midagi infoturbest. Seetõttu 
on neile ka sattunud niisugused tegevused ja projektid alates X-teest ja 
lõpetades Smart-IDga\index{SplitKey}, kus turvalisuse ja krüptograafia komponent on omal 
kohal. 

Lisaks on seal tõsiseid inimesi, kes tegelevad puhtalt teadusega ja koodi ei 
kirjuta. Küberis on oma teadusosakond ja teadusdirektor. Ühtlasi käivad 
teadusega tegelevad inimesed Küberi ja ülikoolide vahet. Sellist sümbioosi 
otsitakse nagu spunki mööda Eestit taga ja ka Teaduste 
Akadeemia president ei väsi rääkimast, et Küber on fenomen ja suur erand. 

\question{Miks ei ole näiteks Helmes võtnud endale teadusdirektorit 
tööle ja hakanud sama tegema?}

Asi on selles, kas teed kõigepealt teadust ja siis hakkad seda 
rakendama ühiskonnas või proovid vastupidi teha: kõigepealt oled kõva
programmeerija ja siis mõtled, et teeks teadust ka kuidagi. Päris 
nii see ei käi, need juured on natuke sügavamal.

Juhtusin hiljuti nägema Küberi töökuulutust: otsitakse 
projektijuhti, nõutav CISA\sidenote{\emph{Certified Information Systems Auditor} – sertifitseeritud infosüsteemide audiitor.} sertifikaat. 
Halleluuja!

Omal ajal kõik teadsid, et Martens tuleb jälle jaurama ja Küberisse tööle 
meelitama. Mul oli väga lihtne äriidee: ajan kõige targemad 
inimesed ühte suurde ruumi ja annan neile teema kätte. Nad asuvad 
plaksti tööle, ise panen jalad seina peal. Töötas! Väga hästi töötas! Pärast 
lugesin kuskilt raamatust, et niimoodi tuleb käituda, ja täpselt nii ma 
olengi käitunud.

%--------------------------


\question{Kas sa oled siis asjade käimalükkaja ja visionäär?}

Kui sa nii ütled.

\question{Ma ei ütle, ma küsin. Sa ütlesid, et programmeerija sa enam ei ole. Kes 
sa niisugune oled?}

Ma ei oska ennast sildistada. Mul on see häda küljes, et mõtlen kogu aeg kuidagi 
laiemalt.

\question{Miks see häda on?}

Võiks ju midagi näpu vahel teha, kaltsuvaiba või midagi. On vähemalt füüsiline tükk taga, 
suurtest sõnadest ei jää midagi\ldots

\question{Mida sa praegu teed?}

Olen endiselt elektroonilise hääletuse juht, juba aastast 2003. 
Hiljuti olid meil kümnendad valimised, kohe on algamas üheteistkümnendad, 
Europarlamendi valimised\sidenote{Jutuajamine Tarviga leidis aset 2019. aasta mai algul, Europarlamendi valimised toimusid 26. mail, edukas 
elektrooniline hääletamine 16.–22. mail.}. Aga valimised võtavad võib-olla 
kaks-kolm kuud tähelepanu. Valimistevahelisel ajal ma palju 
suurt ei teegi. Jõudumööda, nii nagu kutsutakse, käin maailmas ringi ja proovin 
inimesi aidata nende arengus erinevates riikides – nii 
elektroonilise identiteedi teemal kui ka IKT rakendamise alal 
nii-öelda valimismajanduses.



\chapter{Peeter Marvet}
\index[ppl]{Marvet, Peeter}
\index[ppl]{Tehnokratt|see{Marvet, Peeter}}

\question{Kuidas ja umbes millal sa jõudsid arvutite juurde?}

Ma arvan, et see oli umbes täpselt aastal 1985, ma pidin siis olema 15 aastat 
vana. Ilmselt ma olin arvuteid eelnevalt näinud Soome telekast, ma usun, et 
seal reklaamiti ilmselt Commodore 64 ja Spectrum masinaid. 

\question{Sa oled siis Tallinna poiss, ma järeldan?}

Jah. Ma olen sündinud Tartus aga pikalt olnud Tallinnas. 

Aga kui ma veel tagasi krutin, siis ma arvan, et kokkupuude arvutitegaa oli 
veidi varem, mu papsi laboris. Laboriks oli toonase TPI Veekvaliteedi 
Labor\index{Tallinna Tehnikaülikool!Veekvaliteedi labor}, asus selle koha peal, 
kus keset Järvevana teed on praegu Maru ehituse maja. Seal oli olemas üks 
terminal, mis käis Datasaabi\index{Arvutid!Datasaab} nimelise 
arvuti\sidenote{Datasaab oli Rootsi lennukitootja Saab arvutustehnoloogia 
eraldi ettevõtteks kasvanud divisjon. Toodeti nii tsiviil- kui 
militaarkasutuseks mõeldud arvuteid.} külge, mis asus kusagil Mustamäe teel. 
See masin oli ostetud ühelt nii-öelda Rahvamajandussaavutuste Näituselt, kus 
meil vahetevahel käisid väljakad ka kohal. Oli saadud raha ja ostetud valuuta 
eest välismaa arvuti, mille külge käisid modemitega sellised oranžid või 
\emph{amber} värvi terminalid. 

Datasaabi lugu oli selline, et see osteti väidetavasti ilma 
operatsioonisüsteemi ja igasuguse rakendustarkvarata. Selle jaoks nagu rohkem 
rutsi sellel ajal ei olnud. Aga Nõukogude insenerid olid vinged, nõukogude 
insenerid kirjutasid sinna ise mingi operatsioonisüsteemi peale. Või siis pandi 
midagi kusagilt tuuri. Ei, ma arvan, et tuuri ei pandud, sest mulle meenub, et 
vist õnnestus sellest kirjutatud tarkvarast veel mingisugune jupp Datasaabile 
tagasi müüa ja saada selle eest ilmselt siis vastu mingisugust mälu või 
lisakomponente. 

Tollel Datasaabi terminalil, millega ma esimesena kokku sattusin õnnestus 
lihtsalt ilma ühenduseta niimoodi \emph{backspace}-i ja  tühikuga 
\enquote{ronge kokku haakida}. See on minu esimene mälestus sellest, kuidas ma 
olen arvutiga suhtestunud. 

Aga järgmine mälestus on  aastast 1985, kui ma ise olin viieteist aastane, 
ilmselt seitsmendas klassis. Toonases Pedas toimus üks Tallinna koolide 
füüsikavõistlus ja meid viidi lisaks sellele füüsika asjale ka arvutisaali. Ma 
arvan, äkki oli see  Minsk\index{Arvutid!Minsk}. Seal ma sain tuttavaks ühe 
koolivennaga, kes oli minust aasta vanem, ehk siis juba kaheksandas klassis, ja 
kellel oli kaasas isiklik perfolint programmiga. Selleks koolivennaks oli ei 
keegi muu kui Sulo Kallas\index[ppl]{Kallas, Sulo}. Oma perfolindi peal oli tal 
mingisugune selline mängulaadne asi, mis minu arust ise  arendas mingisuguseid 
organisme või midagi sellist\sidenote{Tõenäoliselt oli  lindil Briti 
matemaatiku John Horton Conway poolt välja mõeldud rakuautomaat, mida tuntakse 
nime all \emph{Game of Life}. Tegu on mängijateta mänguga, mis ainsa sisendina 
vajab algseisu määratlemist. Automaat on ühest küljest levinud 
programmeerimisülesannne ja teisalt  põnev uurimisobjekt, seetõttu võis selle 
realiseerimine olla noorele arvutihuvilisele  nii huvitav kui ka jõukohane.}. 
Sulo lugu  oli selles, et tema vend oli Raadiomaja 
Arvutuskeskuses\index{Raadiomaja Arvutuskeskus}, nii et selleks ajaks oli Sulo 
juba mõnda aega arvutitega tegelenud. 

Ja see oligi nagu esimene kord, kus sattusin arvutiga kokku ja mõtlesin 
\enquote{oo, vinge!}.

\question{Mis seal vinget oli, mis konks see oli, mille külge sa jäid?}

Tol hetkel  oli see rohkem selline nagu \enquote{ahaa, vau, et teebki 
mingisugust asja!}. 

\question{Aga Sulo oli kõva mees küll oma perfolindiga?}

Nojaa, sul on nagu kaheksandik ees,  nagu vanem koolivend, kellel on,  kujutad 
sa ette, isiklik perfolint! Vau! Sellised kutid on ümberringi! No siis peab 
ikka ka vaatama, et mis seal tehakse. Ma arvan, et  midagi Soome telekast 
arvutitega seoses nähtut oli tegelikult see, mis tekitas soovi, et olgu või 
nõukogude oma ja perfolindiga, aga ikkagi arvuti. 

Ja järgmine asi tegelikult ei olnud sellest kaugel. See oli, vast mõni kuu või 
natukene hiljem, kui kooli tulid paar djuudi, kes tekid mingit arvutiklubi ja 
kutsusid sina osalema. 

\question{Kas see oli legendaarne arvutiklubi Ahhaa?}

Ei. See on legendaarne arvutiklubi Juta\index{Arvutiklubi!Juta}, mida vedas 
juudi papi nimega Lev Moišeejevitš Šoroht\index[ppl]{Šoroht, Lev}. Klubi 
tegutses Raua ja  Kreutzwaldi tänava nurga peal. Kui mööda Raua tänavat  
kesklinna poolt tulla, siis vasakut kätt on ühel majal keldrikorrusel kaares 
aknad. Vaat sealsamas kaares akende taga vedas ennast Arvutiklubi Juta. Kui 
Ahhaa puhul võiks kujutada ette, kust see nimi tuleb, siis kust tuleb nimi 
\enquote{Juta}? See tuleb vene keelest: \begin{russian}Юный Техник 
Автомат\end{russian}. Ma eeldan, et Lev Moišeejevitš Šoroht ei satu seda  
kuulama, aga kui  kellelgi meenub, et aastal 1985  on ta kas TPI või ka Peda 
üliõpilasena vedanud mingisuguseid noori arvutiklubisse, siis ma suurima hea 
meelega saaks kokku ja teeks väikesed õlled välja. Sest sealt see alguse sai. 

Klubis õpetati meile programmeerimist keeles PL/I\index{Keeled!PL/I}.

\question{Ja klubisse käidi kutsumas, mitte ei joostud ust maha, et arvuti ligi 
saada?}

Ma arvan küll. Ega ma täpselt ei mäleta, aga ma arvan, et see sõnum jõudis 
meieni kuidagi kooli või õpetajate kaudu. Ja ma arvan, et Sulo oli ka seal, 
sest et kui oli võimalik kusagil veel arvutisse saada, siis loomulikult seda 
kasutati.

\question{Seda ma mõtlengi, et tol ajal ju otsiti tikutulega neid kohti, kus 
\enquote{arvutisse saada}?}

Kusagil seitsmenda klassi klassi lõpupoole tuli  arutuskeskusest ühe minu 
programmi väljatrükk, laia aukudega paberi peal, \emph{line} printeri peal 
välja lastud. Keegi mu klassivendadest, kes siis ilmselt oli ka meiega seal 
koos käinud, tõi selle väljatrüki mulle klassi ja siis kõik vaatasid, et 
\enquote{oo, Soome telekavad}. Sest et tollel ajal oli inimestel selline kõige 
üldisem seos arvutitega see, et arvutuskeskustes trükiti välja Soome 
telekavasid. Neis olid näiteks tabuleeritud kujul eraldi väljavõtted, kust 
seriaalid jooksid, parimatel vendadel olid olemas nädalakavad. Aastal 1985 
keskmine teadlikkus arvutitest umbes selline oligi.

\question{Kust need kavad saadi?}

Nad liikusid arvutuskeskuste vahel ja ma tean, et (jällegi nüüd täpselt 
ajaloolisi hetki kokku panemata), vähemasti mingil hetkel oli üheks selliseks 
kohaks Postimaja Arvutuskeskus\index{Postimaja Arvutuskeskus}. Mis on 
iseenesest üks väheseid kohti, kus ma ise pole käinud. Seal oli mäletamist 
mööda SM-4\index{Arvutid!SM-4}, mille külge oli ehitatud mingisugune teksti-TV 
vastuvõtja\sidenote{Mäletatakse ka, et vastav riistvara oli olemas TPI Raadiotehnika 
Kateedris\index{Tallinna Tehnikaülikool!Raadiotehnika kateeder} Apple II küljes.}  
ja sealt see kava tuli. Ma arvan, et see oli mõni aasta hiljem: kui 
ühes kohas oli Minsk ja teises SM-4, äkki ma ajan mingisugused ajajooned 
natukene sassi. Selle SM-4  küljes oli kolmesajaboodine modem, millega siis 
pumbati neid kavasid mööda linna laiali. Ma mäletan mingisuguste järgmist 
etappi sellest, kus  minu päralt oli üks PC, välja arvatud vist kolmapäeviti, 
kui lõuna paiku saabus Postimajast üle modemi Soome telekava ja siis maatriks 
printeri peal enam-vähem nähtamatuks kulunud lindiga trükiti seda välja. Siis 
ma pidin omale  muud tegevust leidma, muudel pärastlõunatel ma sain seda 
arvutit kasutada.

\question{Ma katkestasin sind seal, kus sa PL/I keeles programmeerisid\ldots}

Meile õpetati natukese programmeerimist ja seal oli palju hästi segaseid ja 
täiesti arusaamatuid asju. Oli programmeerimiskeel mingisuguste  muutujatega, 
mis nagu mõnevõrra koitis. Ja  ilmselt esimene programm, mida meile seletati, 
kuidas teha, oli ruutvõrrandi lahendamine. Annad paar muutujat sisse ja siis 
trükitakse sulle tulemus mingisuguse käraka  paberi peal välja. Alguses meid 
vist nagu päriselt arvuti juurde ei lastudki: Keegi  \emph{punch}-is meie 
programmid sisse ja pärast saime väljundi kätte. 

Pärast meile leiti võimalus arvutitega tegeleda veel kahes kohas. Üks oli 
Tihnikus, kus oli ETKVL-i Arvutuskeskus\index{ETKVLi Arvutuskeskus}. Järgmised 
põlvkonnad teavad seda kohta kui esimest Maksimarketit, seal ühes majas oli üks 
vinge ES\index{Arvutid!ES EVM}. Teine koht oli see, kus Endla tänaval asub (või 
vähemalt mõni aeg tagasi asus) Maksuameti üks ots\sidenote{2013. aastani asus 
Maksu- ja Tolliameti teenindussaal aadressil Endla 8.}. Seal  üleval kolmandal 
korrusel olid Ehituskomitee\index{Ehituskomitee} ES-id\index{Arvutid!ES EVM}. 

Minuga läks sealt edasi läks umbes niimoodi, nagu õpetatakse tööõpetuses, et on 
oluline  anda lastele mingisugune selline asi, mille nad saavad valmis voolida, 
tulla koju ja perele näidata. Siis laps saab kiita, tal  läheb edaspidi väga 
hästi ja ta teeb paremaid puulusikaid. Minuga läks niimoodi, et olles teinud 
oma esimese  kolmeteist-realise programmi ruutvõrrandi lahendamiseks, laekusin 
ma  selle väljatrükiga koju ja köögis näitasin siis  lapsevanematele, et 
\enquote{näe sihukse raha eest tegin sihukse asja}. Paps, kes oma 
teadustegevuse poole pealt tegeles elektrokeemia ja hapnikuanduritega ja teise 
poole pealt oli džässpianist vaatas mu tööd ja ütles, et \enquote{mul oli siin 
just mingisugune tudeng, aga kadus ära. Temast jäid ainult mingid listingud 
järgi. Kas sa saad sotti,  mul oleks vaja mingisuguseid  teadusandmeid 
töödelda.} Ülesandeks oli mingisuguste anduri toimekõverate kokkuajamine 
mingite matemaatiliste valemitega, et õnnestuks digitaalseid mõõteriistu teha 
või midagi sellist. Ja niimoodi juhtuski, et olles  programmeerinud oma 
esimesed kolmteist rida esimeses mulle täiesti tundmatus keeles, 
\emph{switch}-isin ma koheselt järgmisele keelele.

Nii ongi mu pea selles mõttes nagu puder ja kapsad, et ma suudan kirjutada 
ainult dokumentatsiooni abil, kaasa arvatud keeli, mida ma igapäevaselt kipun  
kirjutama nagu  PHP\index{Keeled!PHP} ja JavaScript{Keeled!JavaSctipt}. Need 
süntaksid on peas nii segi selles mõttes, et ma kunagi ei mäleta, täpselt mis 
oli PHP-s  \verb|for| tsüklis  parameetrite järjekord. Õnneks on tänapäeval 
olemas kõikvõimalikud IDE-d, mis teevad mõningase töö ära ja aitavad \emph{auto 
complete}-da ja mida iganes. 

\question{Aga siis su puulusikas mitte ainult ei saanud kiita vaid pandi kohe 
ka tööle!}

Puulusikas võeti kohe tööle, sealt edasi ma olen olnud lapstööjõud. Mingil 
hetkel juu papsil hakkas nagu kahju ka, et laps võiks  lisaks 
ekspluateerimisele natukese ka raha saada. Ehk ma olin siis ametlikult kirjas 
mingi veerandkohase laborandina või midagi sellist.  Parem pool oli muidugi 
see, et tänu sellele oli mul ligipääs kõikide papsi sõprade arvutuskeskustele. 
Ja kuna paps tegeles selle oma hapnikuga TPIs, siis loomulikult olid nende 
sõprade hulgas TPI ja ports seltskondi, kes olid seotud mingisuguste 
anduritega. Näiteks see pool, mis tegutses kunagi Pirital Masti tänaval,  kus  
aretati sportlaste mõõtmise lahendusi. Selle mingisugune teine ots asus 
Kiirabihaigla arvutuskeskuses\index{Kiirabihaigla arvutuskeskus}. See oligi see 
koht, kus mul oli üks arvuti pidevalt kasutusel, välja arvatud kolmapäeviti. 
Sanyo PC, väga vinge. Seal oli muuseas ka Apple II\index{Arvutid!Apple II} 
olemas, mille peal sai mängitud Karatekat\index{Mängud!Karateka}, minu arust 
mingeid muid muid toredaid rakendusi tolle masina peal ei olnud. Ja kui ma 
hästi mäletan, oli seal ka mingisugune Labtam-i\sidenote{Labtam oli suhteliselt 
obskuurne Austraalia arvutitootja, kes tegutses aastatel 1972 kuni 1990. 
Miskipärast olid neil oma lõpu-aastatel Nõukogude Liiduga head suhted: aastal 
1984 disainis Novosibirski Riikliku Ülikooli tudengite Kronos Research Group 
neile emaplaadi, URAL-LABTAM OOO tegutseb Venemaal siiani ning nende arvuteid 
leidub lisaks Austraalia arvutimuuseumidele ka Tartu Ülikooli omas. Labtami 
arvuteid osteti naftadollarite eest ka Küberneetika 
Instituuti\index{Küberneetika Instituut}} nimeline \emph{kone}. Lisaks, oli 
seal mingisugune selliste suurte trumlitega andmetöötlus-\emph{kone}, mille 
nime ma ei mäleta. Seda äkki oskab Kalle Lotamõis\index[ppl]{Lotamõis, Kalle} 
või keegi selline  meenutada, et mis see oli. Mina selle suure masinaga sellel 
hetkel ei suhtestunud. 

\question{Kas sulle see andurite maailm ja elektroonika ei  pakkunud huvi?}

Mitte liigselt. Progemise pool oli  huvipakkuvam. 

Siis oli üks keskus, mis oli TPI Santehnika Kateedri\index{Tallinna 
Tehnikaülikool!Santehnika Kateeder} SM-4\index{Arvutid!SM-4}. Ehituse all oli 
selline kateeder, vee kvaliteet ja kõik selline kuulus  sinna alla. Mingil 
hetkel tekkis sinnasamasse Järvevana teele, kus asus ka Läänemere 
Instituut\index{Läänemere Instituut}\sidenote{Ei ole selge, mis asutust Peeter 
silmas peab. Eestis on Läänemere Instituut tegutsenud eelmise sajandi 
kolmekümnendatel ja praegu tegutseb sellenimeline asutus Soomes.}, ka 
SM-4\index{Arvutid!SM-4}, mis oli  enam-vähem niisugune masin, mille ma kohale 
minnes  lülitasin ise sisse ning pärast töö tegemist jälle viisakalt välja. 

\question{Neid arvutid siis ikkagi ju oli?}

Neid oli selles mõttes, et kui sa nagu sattusid õigesse kohta ja ilmselt 
oskasid õigel ajal vait olla ja mitte liiga palju täiskasvanuid  segada nende 
tähtsas töös,  siis üldiselt neid nagu jagus. 

Tulles korraks veel tagasi  alguse, ehk Juta\index{Arvutiklubi!Juta} juurde 
siis selle asutajal oli endal ka paar niisugust  huvitavat projekti, millega ta 
üritas Nõukogude Liidu tasemel kuulsaks saada. Üks neist võiks olla  võrreldav 
Facebookiga. Selline üleliiduline projekt, kuidas inimesed, kirjasõbrad, 
saadaksid oma andmed kõik kokku, need sisestatakse perfokaartide peal 
\emph{mainframe}-i ja see kuidagi teeb mingisugust \emph{match-making}-ut ja 
siis saadetakse kirjad leitud \emph{match}-idele laiali. 

Mina tegelen igasuguste muude projektidega ja siis mingisugusel hetkel ma 
arvan, et me olime selleks ajaks juba jõudnud keskkooli, tekkisid arvutid ka 
meile kooli, ma olen  algusest lõpuni käinud Reaalkoolis.  Nendeks masinateks 
oli klassitäis Yamaha MSX-e\index{Arvutid!Yamaha MSX}. Tolle klassiga seoses on 
meeles, et kuna erinevate koolide vahel oli nende masinate saamiseks 
konkurents,  olla kool saanud ka mingisuguse ähvarduskõne. Mille peale 
loomulikult vaprad raadioruffi ja füüsikaklassi tagaruumi noored organiseerisid 
öö läbi valve koolimajja. Arvutikastid olid kusagil kas direktori kabinetis ja 
kus iganes ja siis me ööbisime koolis ja valvasime neid. Pärast see 
arvutiklassis sai suhteliselt meie selliseks nagu koduks.

\question{Mis seltskond see raadioruffi ja füüsikaklassi tagaruumi oma oli?}

Seal olime mina ja Sulo Kallas\index[ppl]{Kallas, Sulo} ja Heiki 
Savitš\index[ppl]{Savitš, Heiki} ja Vallo Veinthal\index[ppl]{Veinthal, Vallo}  
ja Reimo Mesipuu\index[ppl]{Mesipuu, Reimo} ja no ma kindlasti jätan kedagi 
ebaviisakalt mainimata.  Avo Nappo\index[ppl]{Nappo, Avo} tiirles meie ümber 
rohkem nagu arvutiseltskonna poolest, raadioruumis olime rohkem vist mina, Sulo 
ja Reimo.

\question{Ma küsin pigem seda, et mille alusel too seltskond moodustus. 
Klassivennad? Tehnikahuvi?}

Otseseid klassivendi oli  vähe, me olime kõik sama mingi paariaastane vahemik. 
Sulo oli kõige vanem, Vallo ja Heiki  olid meist aasta nooremad. Ehk et see oli 
selline parasjagu  klass üles, klass alla seltskond, kes siis oli kooli aktiiv 
niisuguse tehnilise poole pealt. Kuidas me täpselt sattusime raadioruumi? Ju me 
siis hängisime füüsika kandis ringi ja  tuli välja, et seal on raadioruum, kus 
on ka mingeid nuppe, mida saab keerata. Kuidagi nagu sealt ümbert tekkis kogu 
seesama punt, kellest nagu väiksem osa oli  alguses raadioruumi ümber ja 
kellest siis suurem seltskond arvutiklassi ümber tekkis. 

\question{Kas programmerimine ja raadioruumitamine koolitööd ei hakanud segama?}

Ei tea. Ma usun, et ega ma mingisugune medaliga lõpetaja oleks niikuinii 
viitsinud olla. See nagu ei olnud kuidagi minu maailmavaates sees. Noh, 
neljade-viitega lõpetasin, selles mõttes probleemi ei olnud.

\question{Eks see ongi tunnetuse küsimus, kumb primaarne oli tollel hetkel?}

Ma arvan, et eks arvuti pool oli põhiline. Keskkool möödus üleüldse enam-vähem 
niimoodi  kuidagi, et vahepeal sai käidud kohvikus ja siis vahepeal sai käidud 
olümpiaadidel. Kui olid olümpiaadid, olid jälle head hinded, sest õpetajad ei 
saanud ometi olümpiaadil esinejatele halbu hindeid panna. Aga kui olümpiaadil 
ei käinud, siis kippusid hinded kehvemaks minema sest et oli ununenud ära 
koolis käimine ja kõik muu selline. Ma siiamaani näen unenägusid sellest, et on 
tulekul eksam ja ma olen unustanud terve veerandi käigus tunnis käia. Aga 
sellest on  tekkinud  mõtteviis, et ma ei pea olema midagi õppinud. Ehk et kui 
TPI-sse läksin ja mata eksam tehti koos raamatutega, siis ma jõudsin eksami 
käigus alati ära õppida selle, mida oli eksamiks vaja. Ma nagu ei pidanud  
eelnevalt liiga palju mingisugusele loengus käimisele pühenduma, ma võisin 
lihtsalt tulla ja eksamid ära teha. Ülejäänud semestri võis lihtsalt arvutitega 
tegeleda. Ma kindlasti ei soovitaks seda kellegile noortest, aga noh, näed, 
niimoodi on see juhtunud. 

Selline lähenemine on tekitanud mingisugune väga, sellise, kuidagi teistsuguse 
arusaamise ümbritsevast tehnikast. Ma nagu  ilmselt ei karda midagi selles 
mõttes, et ma kui on midagi vaja, siis lihtsalt tuleb võtta \emph{manual}  või 
kood ette. Läheb aega selles mõttes, et ma loomulikult läksin eksamile 
esimesena sisse ja tulin viimasena välja. Aga  põhimõtteliselt kolme tunniga 
sain nagu õige asjaga hakkama. Tundus nagu efektiivne lähenemine. Võib-olla ma  
oleksin saanud targemaks, kui ma oleksin süsteemsemalt õppinud. 

\question{Mida sa sinna TPI-sse õppima läksid?}

See oli TI ehk majandusinfo töötlus\index{Tallinna Tehnikaülikool!TI}. Linnar 
Viik\index[ppl]{Viik, Linnar} on lõpetanud sama asja mõned aastad enne mind. 
Aasta sattus 1989 olema, kus päris nagu Pol.Ök.-i ja Kompartei 
Ajalugu\sidenote{Nõukogude ajal kuulus ülikoolihariduse juurde Poliitökonoomia, 
Kommunistliku Partei Ajaloo ja teiste niiöelda \enquote{punaste ainete} 
läbimine.} ei tahaks õppida sinna kõrvale, aga nad ei olnud veel päriselt välja 
mõelnud, et mis see teine asi on, mida õpetada. Oli ka muid asju, millest ma 
päris täpselt toona aru ei saanud, miks seda peaks tegema. Kaasa arvatud see, 
et miks ma peaksin tegema tansistoritest valmis 8080 protsessori paar käsku. 
Eriti, kui normaalsed inimesed kasutavad vähemasti Z80-t ja mitte 8080-t. 
Teismelise värk, ei ole \emph{cool} piisavalt. \enquote{Intel 8008? 
Zilog\index{Zilog}\sidenote{Zilogi toodetud 8-bitine Z80 protsessor oli Inteli 
8080 protsessoriga ühilduv aga märkimisväärselt odavam} on normaalne!} No 
täpselt selline asi, nagu on täna hipsteri habe või mingisugused muud välised 
tundemärgid. Pean takkajärgi tunnistama, et kuigi ma olin selle suhtes 
kriitiline, siis ma hetkel käin koolitusel, kus räägitakse sellest, kuidas 
\emph{fuzz}-imisega\sidenote{\emph{Fuzzing} on tarkvara (turva) testimise 
meetod, kus programmile söödetakse süsteemselt juhuslikku sisendit.} 
mälukorruptsiooni juhtumeid leida. Kui lektor ütles, et see on maru  keeruline, 
räägime hästi aeglaselt ja mitu korda nagu miilitsatele, siis minu arust midagi 
nii rasket seal polnud, \emph{stack} on \emph{stack}. Protsessoril on 
registrid, eks ju, ma põhimõtteliselt olen neid transidest teinud. Et kui sa 
pead nagu protsessori arhitektuuri tasemel läbi mõtlema, kuidas käskude 
töötlemine toimib, kuidas seal inkremenditakse mingisuguseid pointereid ja 
kuidas need asjandused  mäluga seotud on, sa tegelikult saad aru, kuidas arvuti 
masinkoodi tasemel töötab. Mul on  väga tore kuulata, et kui mu vanem poeg on 
Tartu Ülikoolis, siis neid sunnitakse ka aru saama protsessori sise-ehitusest. 
Tõsi küll,  raamatu tasemel, aga progevad ka assemblerit, see on hästi oluline. 

\question{Kas sind akadeemiline maailm ei tõmmanud, kuigi sa seal servapidi 
juba sees olid?}

Ei, sest et ma olin keska ajal sattunud sellisesse seltskonda, nagu seda oli 
Vabariiklik Õpilasstaap\index{Vabariiklik Õpilasstaap}, mis oli selline 
Komsomoli Keskkomitee juures tegutsev mittekommunistlik vastupanuliikumine. 
Tiina Tšatšua\index[ppl]{Tšatšua, Tiina} oli näiteks üks selle eestvedajaid. 
Sellest sai üks selliseid toonaseid orgunn tiime vabariigis, kes korraldas 
suurüritusi, milleks alustuseks olid komsomoli ja EKP kongressid. Aga orgunni 
mõttes on ju suht savi, kas on EKP kongress või Eesti Kongress või Rahvarinde 
oma. Inimesed tulevad kohale, sul on mingisugused tegevused nagu  
registreerimine ja kusagil tuleb neid toita. Kui on dokumentidega üritus (mida 
tänapäeval eriti toimu, aga kõikide Eesti Kongressi ja Rahvarinde kongressid 
olid sellised), siis on sul näiteks mingisugune redaktsioonitoimkond. Me olime 
seal kui arvuti-tiim, kes orgunnis seda, et registratuur toimiks mingite 
listidega ja samuti toetasime redaktsioonitoimkondi  kõikvõimalike 
tekstitöötluste pooltega ja  väljatrükkimiste ja vormistamistega ja millega 
iganes. 

Kui mul  keskkool sai läbi 1989, siis meenub, et suveks oli tööots. Tallinnas 
toimus ÜRO inva-ekspertide mingisugune hüper tipptaseme kokkusaamine. Sellega 
seoses mäletan, et Tallinnas lõigati sel puhul esimesed äärekivid faasi ja minu 
arust oli Jack Lippmaa\index[ppl]{Lippmaa, Jaak}\sidenote{Peeter peab ilmselt 
silmas Jaak Lippmaad} see, kes isiklikult ehitas ümber paar 
Ikarus-bussi\sidenote{Ungari tootja Ikarus bussid olid Eestis laialt kasutusel 
liinibussidena} nii, et neisse  kuidagi  ratastooliga sisse saaks. Kuidas see 
võimalik oli, ma ei kujuta ette. Meie see Reaalkooli tiim toetas ürituse 
redaktsioonitoimkonda,  kes seal ÜRO-le kohaselt kõigis põhikeeltes 
mingisuguseid dokumente vormistas. Mis tähendas, et oli ilge posu tõlke. Aga 
noh, aastal 1989 ükski tõlki ei olnud ilmselt arvutit näinud rohkem kui 
võib-olla Soome telekast reklaamist. Meid oli piisavalt palju, kümmekond tükki, 
 ja hoidsime siis neil tõlkidel kätt ja jalga, kogu aeg oli olemas mehitus  
praktiliselt igaühe jaoks. Kui kellelgil tekkis niisugune kivistunud pilk, siis 
keegi tuli ja \emph{reboot}-is selle tõlgi arvuti taga või arvuti enda, kumb 
igatahes parasjagu oli rohkem kinni jooksnud. 

Minu enda niisuguses hilisemas eluloos on see episood huvitav sellepärast, et 
tolle ürituse jaoks, olles parasjagu keska lõpetanud,  õnnestus mul lihtsalt 
omaenda sõna peale linna pealt toatäis PC-sid kokku laenata. Ma lihtsalt 
läksin, ütlesin et mul oleks nagu vaja, ja siit sai jälle paar tükki, sealt sai 
paar tükki ja niimoodi sai need umbes kaheksa arvutit kokku. Kõige kihvtim 
neist tuli surnukuurist. See oli üks PC, aga tema peal oli selline asi nagu 
Xerox Ventura Publisher koos Xeroxi graafilise kasutajaliidesega, milleks oli 
GEM, nägi põhimõtteliselt välja nagu MacOS\sidenote{GEM (\emph{Graphics 
Environment Manager} oli tõepoolest üks varastest graafilistest 
kasutajaliidestest. Liigne sarnasus Apple tarkvaraga viis ka kohtuasjani.}. GEM 
sai DOS-ist üles \emph{boot}-itud,  läks ilusti siukseks mustvalgeks ja halliks 
kasutajaliideseks ja seal peal jooksis minu esimene küljendusprogramm. Ja me 
saime neilt ka ühe laserprinteri kasutada, mis ei olnud küll PostScript tollel 
hetkel, aga mis oli täiesti  laserprinter. 

\question{See oli ju nõukogude aeg veel?}

Ilmselt siis meditsiin oli ikkagi saanud mingeid asju kusagilt valuuta eest 
osta. Tegelikult Kivilo\index[ppl]{Kivilo, Agu} plaanis oma  
diagnostikakeskust\sidenote{1988. aastal asutatud Diagnostikakeskus oli omal 
ajal märgilise tähendusega. Ühest küljest oli tegemist kõrgtehnoloogiliste 
teenustega, keskuse algusaegadel asus seal Eesti ainus kompuutertomograaf. 
Teisalt oli aga tegemist väga innovatiivse organisatsioonilise 
konstruktsiooniga, milline asjaolu viis hiljem mitmete keskust ümbritsenud 
kõrge profiiliga afäärideni.} kesklinna ja meditsiini poolel olid  tegelikult  
väga kõvad tegijad. Eesti arvutinduse arendusest teatakse rohkem seda 
seltskonda, kes oli nagu Tartu poole pealt ja seotud geeniga ja võib-olla 
Küber\index{Küber}. Mina olen tulnud  nagu meditsiiniliini pidi sisse, seal 
valdkonnas tegeldi päris kõvasti teadus- ja arendustegevusega. 

\question{Ja sealt said sa oma küljendamise-konksu?}

Jah. Otse loomulikult sai hunnik flopikettaid tolle Venturaga ära kopeeritud, 
eks ju. Mis toona oli nagu igati tavapärane \emph{standard operating 
procedure}:  kõigest, mis kätte satub, tehti koopia. Ja niimoodi siis sattuski, 
et kusagil sealsamas 1989-1990  oli minu jaoks ülikoolis käimisest palju 
huvitavam asi see, et arvuti peal on  võimalik küljendust või kujundust või 
disaini teha. 

\question{Kas sul muidu ka niisugune joonistamise soon oli?}

Ei ole. Ma kahtlustanud, et inimesed, kes on pidanud minu küljendatud raamatuid 
tarbima, on kindlasti selle all kannatanud, nii et ma väga vabandan. Kunagi  
Avita\index{Avita kirjastus} kirjastuse  algalgusaegade raamatutest suur hulk 
oli minu tehtud, näiteks.

\question{Aga mis sind selle küljendamise juures niimoodi köitis, kui sul muidu 
sellist visuaalkunstide huvi ei olnud?}

See oli pisut teistmoodi, mingisugune selline hoopis teistmoodi  arvutiga 
tegelemine, kui seda  oli  programmeerimine ja andmetööstus mis olid ka tore. 
Aga see, et sul õnnestub mingeid asju ekraanil teha, see oli see, mis mind 
kuidagi väga sellel hetkel tõmbas. 

Ehk siis 1989. aasta suvel too üritus sai läbi, mina läksin ülikooli. Ja siis 
Mart Siilmann\index[ppl]{Siilman, Mart}, kes oli äsja lõppenud ürituse orgunni 
pealik, ütles, et \enquote{Kuule, mul järgmine suvi on ka mingi asi, et 
arvutiabi vaja, et tule}. Ja see oli aastal, siis 1990 toimunud European 
Nuclear Disarmament Convention, ehk suur rahuvõitlejate ja roheliste üritus. 
Sellega seoses tekkis meil ühte kontorisse, mis asus enam-vähem nüüdseks 
lõpetanud No-teatri ruumides, kusagil teisel korrusel, üks PC, ma arvan, et 
äkki oli Sanyo. Ja selle küljes oli ma arvan, et 1200 boodine või bps-ine 
modem. Mingi koha peal lähevad boodid ja bps-id vist lahku, kui ma hästi 
mäletan\sidenote{Bps (\emph{bits per second}) on bittide hulk, mida sekundis 
edastatakse. \enquote{Boodid} (\emph{baud rate}) näitavad aga, mitu korda 
sekundis signaal muutub. Kuniks kasutatakse tavalist jadaporti, kus signaalil 
on kaks taset, on väärtused võrdsed. Keerulisemate skeemide korral aga võib ühe 
signaalimuutusega edastada rohkem, kui ühe biti ning kiiruseühikud lahknevad.}. 
 Meie ametlik tegevus oli suhtestumine Orgkomiteega. Ja see toimus niimoodi, et 
sai  helistatud kaugekõnega Tallinnast Helsingisse. Eestis oli otsevalimine, 
meil oli selles mõttes väga vinged positsioon, Baltikumis mujal otse välismaa 
numbreid valida ei saanud. Ka Tallinnas igal ei olnud, aga meil oli, sest see 
oli ürituse jaoks oluline. Mart Siilman, kes on endine Fila\sidenote{Peeter 
peab silmas Eesti NSV Riiklikku Filharmooniat\index{Eesti Riiklik 
Filharmoonia}, mille järeltulijaks on alates 1989. aastast Sihtasutus Eesti 
Kontsert. Tegu oli tohutult mõjuka asutusega, mille korraldada oli kogu 
kontserdi-elu Eestis, sealhulgas ka levi- ja jazzmuusika ning estraad. Seega 
oli \enquote{Fila endine direktor} äärmiselt mõjukas inimene, kelle jaoks Soome 
otsevalimise korraldamine kindlasti võimalik oli.} direktor ja muud sihukesed 
asjad, siis üldiselt juba oletan, et ta orgunnis, mida vaja. Kuidas, ei tea. 
Igatahes saime helistada Datapakki X.25 võrku,  X.25 kaudu oli siis võimalik 
suhelda ühe Rootsi serveriga, teine server oli Kanadas. Sealtkaudu me siis nagu 
suhtlesime tolle ürituse orgkomiteega aga  hakkasime ka vaatama, et kuhu veel 
õnnestub helistada.

\question{Kuidas te \emph{bootstrap}-isite seda asja? Mida kliendi poole vaja 
oli, et sinna võrku saada?}

Tavaline modem ja tavaline modemiga suhtlev mingisugunegi terminaliproge. 
Sellega modemiga helistas, siis Datapaki liidestuspunkti, kust edasi läks, asi 
pakettvõrguks või X.25-ks. Ja siis sealsamas terminali peal, nagu terminali 
peal ikka: lehekülg \emph{scrollib} ja siis seal on mingi menüü, valid mingi 
üks, kaks, üksteist, eks. Mingi meilboks oli seal, kus sai kirju vahetada, oli  
jututubade või listide alajaotus. Aga siis,  parafraseerides Heinleini, et 
\enquote{\emph{Have modem, will find BBS-es}}\sidenote{Peeter viitab Robert A. 
Heinlein-i 1958. aasta jutustusele \emph{Have Space Suit - Will Travel}.}. 
Loomulikult leidsime kusagilt üles ka selle, et on  BBS-id. Umbes samas 1989. 
aasta lõpus tekkis Lembit Pirnil\index[ppl]{Pirn, Lembit} esimene 
PirnBoxi\index{BBS!PirnBox} nimeline  BBS, mis asus seal kusagil, kus trammid 
väga kõva kriginaga keerasid toona ehk praeguse SEB vastas, seal oli 
Autotranspordi Arvutuskeskus\sidenote{Asutuse täpne nimi oli Eesti NSV 
Autotranspordi Arvutuskeskus (ATAK)} ilmselt. Ehk tal oli seal arvuti ja me 
kõik alguses helistasime  sinna Pirnile sisse. 

Natukese aja pärast tekkis selline asi nagu HNS ehk \emph{Hacker's Night 
System}\index{BBS!HNS}. Ja siis kolmas oli Goodwin BBS\index{BBS!Goodwin} meil 
Suloga\index[ppl]{Kallas, Sulo}, mis  ilmselt jooksis sellesama 
väljahelistamise liini otsas. Öösel jätsime  arvuti sisse ja kõik said sinna 
sisse helistada. Kui sa tahad kuhugi sisse helistada aga liinid on kogu aeg 
kinni, siis ainuke võimalus olukorda parandada on see, et panna ise ka mingi 
\emph{box} püsti, eks ju. 

Sealt kusagilt tekkis siis ka Fido pool. Jällegi see sissehelistamise küsimus. 
Et kui meil on  võimalik teha see, et e-post ja jututoad oleksid omavahel nagu 
kuidagi  sünkroniseeritud eri masinates, siis pole ju vahet, kuhu me sisse 
helistame. Masinad vahepeal öösel või päeval käivad ja vahetavad omavahel need 
sõnumid ära. Fido oli selles mõttes nagu tõsiselt distributeeritud nett. See, 
mida nüüd räägitakse, kuidas  veeb kolm tuleb äkki nagu sarnane. 

\question{Igasugu võrgustike \emph{bootstrap}-imine on keeruline just inimeste 
mõttes. Selleks, et kuhugi külge minna, peaks seal olema huvitav. Selleks, et 
seal oleks huvitav, peaksid seal olema inimesed. Mis te näiteks seal PirnBoxis 
tegite, et huvitav oli?}

Eks ilmselt sai lihtsalt nagu lämisetud. Ma pean tunnistama, et ma ei mäleta, 
mida me  tegime, aga igal juhul väga huvitav oli. Ma oletan, et kusagilt  
pääses ligi mingisugustele faili kujul sci-fi  raamatutele ja mingitele muudele 
laiematele uudisgruppidele, mingi hetk  kusagilt igal juhul liidestusid need 
Fidonetiga ära. Ehk et seal informatsiooni nagu liikus. Ja lihtsalt selles 
mõttes oli ka huvitav kirjutadagi, et vau, et traadi kaudu see kõik liigubki! 
See oli nii \emph{amazing} selle aja kohta. Sellest ma sain aru, et 
programmeerida saab ja midagi kujundada aga et saab nagu  reaalselt suhelda ka!

\question{Mis tol ajal tegi ühe BBS-i populaarsemaks kui teise? Goodwin ja HNS 
olid ikkagi pikalt populaarsed, kuigi PirnBox oli esimene?}

Ta oli esimene, aga ta vist jooksis mingisugust asja, mis toona oli vist vähem 
levinud. Meil oli, kui ma hästi mäletan, äkki  Maximus. Ja ma ei mäleta, kuidas 
Pirni ja Fidonetiga läks, võib-olla oli tal palju tööd teha? Kuidagi ta nagu 
nende noorte poiste käe alla läks igatahes. Ma ei oska öelda, miks.

Sulo oli muidugi omamoodi nagu arvamusliider ehk selles mõttes, et tal olid  
kõikvõimalike asjade suhtes oma sellised väga toredad ja tugevad seisukohad. 
Mina olin ka niisuguse tutu-lutu taustaga, olles olnud muu hulgas 
Reaalkooli\index{Koolid!Tallinna 2. Keskkool} viimane komsomolisekretär.
Arvestades seda, et enne mind oli komsomolisekretär Karl Martin 
Sinijärv\index[ppl]{Sinijärv, Karl Martin} selles mõttes, me  ei võtnud seda 
asja väga tõsiselt.

Kuidagi me sattusime seda asja vedama  kuna meil oli tänu sellele 
tuuma-üritusele ressurssi käes. Meil mingil hetkel tekkis igal juhul  kaks 
telefoniliini. Võib-olla see oli mingi aastake hiljem, kui  üritus läbi sai ja 
me olime juba Eesti Instituudi\index{Eesti Instituut} ruumides, veidike enne 
seda kui Eesti Instituut osutus tegelikult Eesti välisesinduste ja iseseisvuse 
ettevalmistuslavaks. Näiteks sellel hetkel, kui kuulutati välja iseseisvus, 
tuli järsku välja, et Jüri Luiged\index[ppl]{Luik, Jüri} ja kõik muud, kes seal 
mööda maailma laiali olid, et neil olid juhuslikult kaasas ka pruunid ümbrikud 
esitamiseks kohalikule võimupealikule küsimusega, et kas teie ekstsellents 
lubaks meil asutada suursaatkonda. 

Eesti Instituudis olid meil ka mingid omad arvutid, jällegi ei mäleta täpselt 
kelle arvutid need olid, kust nad pärit olid. Äkki olid instituudi omad, äkki 
olid meie omad. Me olime Suloga\index[ppl]{Kallas, Sulo} need, kes öösiti  
faktse saatsid. Seal mitmed toredad kolleegid, vähemasti niimoodi huumoriga 
pooleks, olid sügavalt veendunud, et faks ongi selline  seade, et kui sinna  
peale panna paber koos kollase postitiga, kus on telefoninumber, siis on see 
hommikuks ennast ära saatnud, eks ju. Sest et tollased liinid olid sellised, et 
nad öösiti toimisid oluliselt paremini, kui päeval. 

Tingituna sellest, et seda välisühendust oli meil läbi modemi helistamise  
suhteliselt piiramatult käes ja liine oli seal ka mitmeid, siis oli meil  kaks 
modemiühendust. Mingil hetkel hakkas meie kaudu Fidoneti kaudu väljapoole 
ühenduma Läti.

\question{Ma teadsin, et Vene Fidonet käis läbi meie aga et Läti ka?}

Venemaa, tekkis ka millaski jah. Oli Läti, mis oli Eesti all  mis iganes see 
Fidoneti järgmine selline tase oli. Leedukad loomulikult ei oleks millegi 
selliseni laskunud, et nad on mingisugune Eesti regioon kusagil 
võrgustruktuuris. Nad selle asemel kord nädalas helistasid ja tõid enam-vähem 
nagu ämbriga e-posti. Välja arvatud, ma arvan, et see oli Kaunase 
Ülikoool\index{Kaunase Ülikool} ja Leedu parlament\index{Leedu Seim}, kes olid 
Goodwin BBS-i pointid. Seal oli hädasti vaja ja siis uhkus jäeti kõrvale. 

Lätlased käisid meil külas ka. Panid raha kokku ja tõid meile selle eest, et on 
ühendus. Saime mingisugune, ma ei tea, kakssada dollarit, mille meie 
investeerisime sellesse, et  ostsime endale kaks modemit. US robots\index{US 
Robotics}-i  HST-d, mis tegid vist kas neliteist kilo või midagi sihukest 
kiirust. Väga väärt aparaadid, niimoodi on lätlased panustanud  Eesti neti  
arengusse.

Samal ajal Internetiga nagu ametlikku postivahetust pidas seltskond 
Küberist\index{Küberneetika Instituut}. Aga neil seal Mustamäel oli ikka 
selline suhteliselt sant jaam,  mis  krabises ilmselt rohkem, kui sidet läbi 
lasi. Pluss veel see, et need akadeemilised tüübid olid millegipärast koledad 
UNIX-i sõbrad ja kasutasid PEP-i TrailBlazer-eid\index{Telebit TrailBlazer}. 
Mis esiteks olid 9.6 kilo, ehk mõttetult aeglased ja teiseks kuidagi nende 
post-sovieti liinidega HST suutis nagu paremini oma sidet vilistada. Me olime 
sügavalt veendunud, et nad olid ka oma reaalselt võimekuselt pikki seansse ja 
kiirust üleval pidada  oluliselt paremad. 

\question{Ja kui sa ütled \enquote{liin}, siis sa mõtled ikka telefoniliin?}

Jah, liinid olid kõik tavaline analoog, kus otsa käis kettaga telefon. Ja 
keskjaamas, ma arvan, et kui sa ikka numbri valisid, siis kusagil  mingisugused 
 releed jooksid kontakte mööda ringi. Kui modem valis, oli kuulda  klõbinat, 
tal oli seal mingisugune relee või herkon või mis iganes, millega ta katkestusi 
tekitas. Kõik oli reaalselt selline elektriline, sellepärast ma ka kujutan 
ette, kuidas andmeside  tegelikult toimub. Aga  kuidas on võimalik, et mingid 
vennad panevad läbi ADSLi kümme mega, me panime enam-vähem samasugust asjast 
läbi neliteist kilobaiti, hästi arusaamatu. Või wifi täpselt samuti. Ma ei 
kujuta ette kuidas see põhimõtteliselt saab üldse toimida. 

\question{Kas kujundamise rida käis sul kõige selle kõrvale?}

See käis kuidagi sinna jah, selles mõttes kõrvale, et ma seal mingis umbes 
samas ajajärgus sattusin seltskonda, kellel oli arvuti ja printer ja vaja  
midagi trükkida. See oli poistekoor\index{RAM-i poistekoor} ja Venno 
Laul\index[ppl]{Laul, Venno}\sidenote{Venno Laul asutas 1971. aastal Riikliku 
Akadeemilise Meeskoori juurde poistekoori ning oli kuni 1990. aastani selle 
kunstiline juht ja peadirigent.}. Neil oli esimene PostScript printer, mille 
ostus ma olen osalenud, kas Tectronicsi või millegi sellise A4 formaadis 
kolmesaja DPI-ga laserprinter. Aga see oli PostScript printer. Kui sina sai 
Ventura külge ühendatud, siis \ldots 

Põhimõtteliselt kõik toonased kujundusprogrammid olidki niimoodi, et sul ikkagi 
enam-vähem \emph{bitmap} fondid olid arvutis, eks ju. Kuidagi need 
\emph{bitmap}-id saadeti juhet pidi printerisse ja kõik see mõtles hästi pikka 
aega. Aga PostScript tegi nii, et sa said selle lehekülje nagu programmi saata 
printerisse ja printer oma tarkusega joonistas. Mis oli kunagi  Adobe ja Apple 
vendade poolt väga mõistlik valik, olles Silicon Valleys kokku saanud ja  
otsustanud, kes mis osa maailmast vallutama hakkab. Tõesti, kui sul on kontor, 
ilmselt igal vennal ei ole printerit ja selleks, et inimesed saaksid printida, 
võiks olla printer ka tark. Väga spetsiifiliselt tark, et ta suudaks 
joonistadanii-öelda lehekülje endal mälus valmis ja siis välja trükkida. 

Ja kusagil samal ajal ma sattusin kokku ka Sirbiga\index{Sirp} (ma ei tea, ka 
see toona oli juba Sirp või veel Sirp ja Vasar), mis oli üks esimesi 
ajakirjandusväljaanded, mis läks digitaalse töövoo peale. On raske öelda, kes  
täpselt see esimene oli, aga igatahes Sirp läks ka. Alguses protsess oli umbes 
selline, et toimus tinaladu. Tinalaoga tehti kas siis üks tõmme paberi peale ja 
see siis vist pildistati üles. Põhimõtteliselt sellel hetkel ofset-trükk toimus 
veel läbi tinalao. Ja nüüd, kui oli võimalik minna üle selle peale, et arvutist 
saaks välja trükkida, siis see oli ikka mega raju.

\question{Põhimõtteliselt PostScripti printerist lasti kile peale trükitavad 
asjad, eksole?}

Põhimõtteliselt jah, ja peegelpildis. Üks asju, mis ma mäletan, mis me 
Suloga\index[ppl]{Kallas, Sulo} koos tegime, või Sulo tegi, kui me Eesti 
Instituudis\index{Eesti Instituut} olime, oli PostScripti \emph{pre-header}. 
PostScripti puhul sageli oligi, et programmiga tuli kaasa mingisugune 
koodijupp, mis siis kirjeldas sellise nagu programmeerimiskeskkonna, defineeris 
täiendavad funktsioonid ja mingisugused muud oma käsud. \emph{Pre-header}? 
Preambul oli vist. Seejärel tuli  kood ise ja lõpus mingisugune 
koristusfunktsioon või midagi, mis välja trükis. PostScript oli tore selles 
mõttes, et ta oli nagu \emph{open source}. Mitte nii-öelda vaba tarkvara, aga 
ta oli nagu nähtava lähtekoodiga. Ehk oligi võimalik võtta sama Ventura ette, 
mis kusagilt laadis selle preambuli, mis oli tekstifail ja mida oli võimalik 
muuta. Ja oli võimalik kirjutada selline transformatsioon sinna ette, mis 
keeras pildi peeglisse. Too Sulo PostScripti preambul õnnestus meil maha müüa 
Avita\index{Avita kirjastus} kirjastusele Tiit Aunastale\index[ppl]{Aunaste, 
Tiit}, kes hakkasid tegema kooliraamatute kirjastamist. Ma ise sattusin ka  
mingi aeg hiljem  Avitasse tööle, mis oli ka mingi a'la 1991, asjad liikusid 
väga kiiresti tol ajal.

\question{Jah, sest umbes viis aastat hiljem, mina mäletan sind Eesti 
vaieldamatu autoriteedina teemal, kuidas arvutist värviline asi trükki saada}

Eks ma olin seda piisavalt praktiseerinud. Me olime teinud ilmselt 
mingisuguseid nii-öelda haltuura otsasid sellesama Ventura peal. Igast muud 
softi oli ka, näiteks Arts \& Letters, millega oli võimalik panna tähti ümber 
ringi käima. Toona, kui kõik asutasid endale aktsiaseltse ja börse, siis kõigil 
neil oli vaja endale pitsatit. See oli meeletu innovatsioon, et oli võimalik 
arvutist ühe matsuga põhimõtteliselt pitsat valmis teha ja ei pidanudki 
kujundajatädi mingisuguseid fotolao tähti välja lõikama ja kleepima. 

Sirbi  toimetus andis välja mingit Jutulehe\sidenote{Ilmus AS Kodamu väljaandel 
aasatel 1990-1992.} nimelist asja, mille  \emph{layout}-i vist mina tegingi 
tegelikult. 

Jah, ja nii edasi, kuidagi ma sattusin selle ala peale. Sai käidud vaatamas, 
kuidas Helsingis Helsingin Sanomat\index{Helsingin Sanomat} tehakse, kus olid 
mingisugused miniarvutid ja   rohelised terminalid. Seal oli ka 
Linotronic\sidenote{Mergenthaler Linotype Company poolt toodetud 
kõrgekvaliteediline printer. Tegu oli kalli seadmega, kuid võimaldas trükkida 
resolutsioonis kuni 2540 dpi.}, millega trükiti põhimõtteliselt veergu välja 
fotopaberi peale. Fotopaber oli kolmkümmend senti lai ja sinna lasti välja  üks 
ajaleheveerg. Ajaleheveerud lõigati kääriga sealt välja ja pandi sellise suure 
maketi peale, mis oli mingisuguse vaha või millegagi koos. Rastreeritud fotod 
pandi veergude vahele, niimoodi pandi leht kokku ja tulemus saadeti faksiga 
trükikotta. Faks ei olnud loomulikult see tavaline faks, vaid mingisugune 
selline \emph{industrial-grade} ajalehe formaadis kõrge-resolutsiooniline,  mis 
siis  skännis ühelt poolt sisse ja teisel pool siis  ilmselt trükis välja 
filmi, millega valgustati trükiplaadid. 

\question{Räägi, mis see \enquote{pull} oli? Kas tehnoloogiline keerukus või 
see, et protsessil oli palju samme või veel midagi?}

Kõige huvitavam on tegeleda asjadega, millega teised parasjagu ei tegele. Või 
siis, teistpidi, mingi asi, mis toimib nagu kuidagi teistmoodi, kui sa oled 
siiamaani arvanud, et asjad toimivad. \enquote{Aa, ongi niimoodi, et ma saan 
seda teha, okei?}. Ja niisama Pascalis programmeerida, ma ei tea, seda õpetati 
koolis, et see ei olnud nagu midagi väga huvitavat. Aga kuna mul oli ilmselt  
parasjagu olemas taust, mul oli arusaamine sellest, kuidas need asjad töötavad 
ja mis seal masinates on,  siis ma suutsin asju efektiivsemalt tööle panna. Ehk 
siis, kuidas siduda kirjastuse kontekstis see, et sul on  küljendusprogramm ja 
sul on tekstitöötlusprogramm. Tekstitöötluseks oli sul WordPerfect (Perfect? 
Prefect? Ford Prefect ja Word Perfect!\sidenote{Peeter viitab tõenäoliselt 
Douglas Adamsi loodud tegelaskujule, mitte omaaegsele populaarsele 
automargile.}),  me küll kasutasime rohkem Volkswriteri-nimelist 
asja\sidenote{Volkswriter oli üks esimese PC-platvormi tekstitöötlusprogramme, 
mida arendati peamiselt eelmise sajandi kaheksakümnendatel aastatel. 
Volkswriter oli saadaval ka eespool mainitud GEM platvormile, sellest ilmselt 
tema kasutamine kirjastustöös.}. Just see, et sa said lasta tekstitoimetajal 
võtta ette selle WordPerfecti faili, tema toimetab seal mingid asjad ära, seal 
kuidagi on juba sees see märgendus, mis ütleb ära, mis on  stiilid. Siis sa 
loed selle oma küljendusprogrammi tagasi ja sul on põhimõtteliselt võimalik 
teha teksti korrektuuri ilma et,  keegi oleks nagu kallima arvuti või 
keerulisem programmi taga. See oli see, mis meil õnnestus väga efektiivselt 
kuidagi juurutada. 

Kusagil sealsamas enne rubla-aja lõppu, 1992. aasta alguses, tekkis Prisma 
Printi\index{Prisma Print} esimene Linotronic ehk filmiprinter. Sinnamaani 
peeti kuuesaja DPI-st laserit juba väga heaks, nüüd tekkis järsku 1200 DPI-ne 
filmiprinter. Prismas  alumisel korrusel olid Crosfieldi suured 
trummelskännerid\sidenote{Crosfield Electronics oli Briti firma, mille toodetud 
skännereid peetakse siiani ühtedeks parimateks, mis iial tehtud.} millega sai 
teha värvilahutusi, filmi peale, juba rastrisse. Ja endiselt kogu montaaž 
toimus niimoodi, et sul olid teksti kile ja pildi kile või film ja need siis 
valgustati või füüsiliselt lõigati kuidagi kokku. 

Kuna mul oli  keskmisest parem ettekujutus sellest, kuidas need süsteemid 
töötavad, siis enamasti,  kui mina jõudsin oma failidega kohale, siis kõik seal 
nagu nägid vaeva, kuidas QuarkExpressist midagi välja printida ja muud 
sihukest. Minul olid kaasas oma flopid ja võib-olla hiljem mingid 
magnet-optilised või  \emph{syquest}-i kettad\sidenote{SyQuesti 88 megabaiti 
mahutavad eemaldatavad kõvakettad olid üheksakümnendatel \emph{de facto} 
standard suurte failide liigutamiseks Apple maailmas.} (vist mitte syquest, see 
on üks väheseid asju, mida mul pole kunagi olnud) ja ma tulin sinna vahele, et 
\enquote{laske minu omad vahepeal välja, ei viitsi oodata teiste järel}. Tehti 
ära. Põhjus oli selles,  et minu asjad käisid tõesti kiiresti läbi. Sest ma 
kujutasin kujundust tehes ette, kuidas see PostScriptiks läheb. Minu jaoks nii 
küljendusprogramm, kui, ütleme, seesama Ventura või graafikaprogrammid nagu 
Illustrator või Freehand oli tegelikult programm, mis aitas mul visuaalselt 
valmistada ette PostScripti. Ma põhimõtteliselt teadsin, kuidas see asi koodis 
välja näeb, ma võin võtta faili ette ja näha, kuskohas miski asi on. Ja tänu 
sellele ma teadsin ka seda, mis on printeri jaoks keerulised asjad ja ma 
oskasin neid asju lihtsustada ja mitte liiga keerulisi asju kasutada. Sest et 
see prose, mis seal taga oli, oli ikka suhteliselt vaene. Kui sa suudad nagu 
tekitada olukorra, kus programmil on tsükkel tsüklis tsüklis (tänapäeval tuleb 
sinna otsa veel SQL-i päring), siis üldiselt on see asi suhteliselt ebapädev.  

Arvuti taust ja siis kuidagi kokku sattumine selle kujundusega on tekitanud 
selle, et kuidagi ka sõprade hulgas on suur hulk igasuguseid disainereid ja, 
ütleme, kõiki kes tollest ajast on tegevad olnud, ma kuidagi nagu tunnen 
sellest samast Prisma Prindi väljatrükijärjekorrast. 

Sealt ma sattusin edasi Uniprinti. Alustuseks töötades andetu disainerina aga 
siis leides tasapisi võimalusi selliseid, noh, nagu vähem disainimaid asju 
teha, kus mängis rolli just see, et ma suutsin võtta, ette mingi Eesti Näituste 
näituse kataloogi andmebaasi (äkki oli Microsoft Accessis?) ja sealt 
genereerida väljundiks tekstifaili, mis oli juba märgendatud stiilidega ja mis 
oli võimalik lihtsalt küljendusse sisse lugeda. Jällegi asi, mida väga palju 
sellises \emph{desktop publishing}-us ei olnud tavaks kasutada: sa  valmistad 
stiilid ette ja siis tekst lihtsalt kasutab neid stiile, kohapeal midagi tegema 
ei pea.

\question{Põhimõtteliselt CSS?}

\emph{Right}, täpselt. Põhipõhimõtteliselt nagu CSS, ainult et tekst ja paber 
ja vanasti. Need kõik töötavad siiamaani niimoodi, aga see oli selline meetod, 
kuidas nagu mina tegin. See  tekitas võimaluse teha selliseid huvitavaid 
töövoogusid.  Minul oli põhimõtteliselt  see andmebaas käes, tüdrukud näitustel 
müüsid seal järgmisi bokse maha ja tegid ürituste korraldust ja tõstsid asju 
ümber ja täiendasid firmade andmeid  ja parandasid telefoninumbreid ja mida 
iganes. Mina mingisugusel hetkel trükkisin \emph{layout}-i välja, viisin neile 
ja nemad tegid andmebaasi korrektuuri, mina nii kaua joonistasin logosid 
puhtaks, ja siis niimoodi ma õppisin. Ma pean tunnistama, et ma olen andetu 
disainer. Aga nende tehniliste protsesside ja töökorralduste poole pealt 
ilmselt teadsin oluliselt rohkem, kui keegi teine jah, tollel hetkel. 

\question{Aga see tuleb ju kenasti selle juurde, mida sa praegu tundud tegevat? 
Millega sa praegu üldse tegeled?}

Jah, on igasuguseid seoseid. Kui ma veel trüki alal tegutsesin, oli mul mingil 
hetkel ilmselt liiga palju vaba aega tänu tänu sellele, et ma olin suutnud oma 
tööd ära optimeerida. Ja tänu sellele, et Uniprindi pealikud Sirje ja Andrus 
Reinsoo\index[ppl]{Reinsoo, Sirje}\index[ppl]{Reinsoo, Andrus}, kes on just 
mõlemad lahkunud\sidenote{Intervjuu Peetriga leidis aset märtsis 2019.} jätsid 
mulle piisavalt vabadust tegeleda. Ja nii ma käisin ja kolasin Ameerikas 
konverentsidel. See oli ajal, kui enamasti oli suhtumine, et \enquote{Mis 
mõttes väljamaa ja konverentsid? Me oleme Eestist ja teame väga hästi}. Mina 
käisin, olid mingisugused \emph{cyber publishing} seminarid, mis just olid 
seotud selle trükipoole alaga, mis mulle huvi pakkus: plaaditrükkida ja kõik 
muu. Ja oli lugu selles, et kuna ma olin põhimõtteliselt nagu ajakirjanik, siis 
mul oli võimalik möllida ennast igale poole konverentsidele, mis muidu maksid 
mingi paar tuhat taala (röögatult kallis tolle aja mõttes) põhimõtteliselt 
ajakirjaniku passiga sisse. 
Vist aastast 1994 olen ma sattunud  kirjutama.

See sai alguse niimoodi, et koolivend ja paralleelklassivend Peeter-Eerik 
Ots\index[ppl]{Ots, Peeter-Eerik} oli Äripäevas ajakirjanik ja kirjutas 
mingisugused tehnoloogiateemalisi lugusid. Aga minul reaalikana hakkas nagu 
mõnevõrra piinlik, kirjutatu ei tundunud olema piisavalt pädev. No see oli ka 
kõik sellest post-BBS-i ajast, ma olin kindlasti ka võrdlemisi 
\emph{opinionated} noor inimene. Eelarvamuskindel,  omade kindlate 
eelarvamustega. Ma kirjutasin teisele Peetrile paar lugu ette, et avalda parem 
neid, vähemasti  kirjutatud kellegi poolt, kes enam-vähem saab aru,  millega 
tegemist on. Peeter ütles, et kuidagi nagu väga imelik, et võiks ikka minu nime 
alt ka hakata avaldama ja saaks honorari ka maksta. Nii ma sattusin Äripäeva 
kirjutama. Sealt sattusin jälle igale poole mujale kirjutama, 
Arvutimaailma\index{Ajakiri!Arvutimaailm} ja kuhu iganes. 

\question{Tehnoloogia tehnoloogiaks, aga mis sind kirjutamise juurde tõmbas?}

Kirjandite kirjutamisega sain suhteliselt nagu hakkama juba kooliajal. Minu 
esimene avalikustatud töö oli Pikri\index{Ajakiri!Pikker} mingisugune noorte 
huumorivõistluse võidutöö. Ilmselt ma olin midagi lugenud ka,  selline sõprus 
sõnaga oli nagu olemas, ma olin juba teinud  kooli omavalitsust ja muud muud 
sellist. Ilmselt olin niisugune parasjagu jutukas ka, eks ju. Kirjutamine ei 
olnud keeruline. Võib-olla meeldis mulle ka õpetada,  läbi kirjutamises on 
võimalik teisi inimesi õpetada ja panna midagi teisiti tegema. Pluss, 
klassikaline küsimus, miks ma olen väga tänulik Peeter-Eerik Otsale on see, et 
ta tegi midagi valesti. See tüüpiline küsimus interneti puhul, et 
\enquote{\emph{wait, somebody is just wrong on the Internet!}}. Kirjutamine 
ilmselt sai alguse sellest, et \emph{somebody was wrong} ja mul oli vaja 
kaitsta oma  seisukohta, ja noh, Reaalkooli au loomulikult ka, eks. 

Sealt sai asi alguse ja edasi inimesed ütlesid, et ma võiksin neile kirjutada. 
Noh, ja  olles õppinud, palju asju, ma siiamaani ei oska \enquote{ei} öelda. 
Ilmselt mingi edevus ka, et \enquote{oh, keegi tahab, et ma midagi teeksin!}. 
Eks ma siis kirjutasin ja hetkel, kus selle valdkonna kohta suuresti nagu ei 
olnud kirjutatud, siis see kõik nagu kuidagi hakkas toimima. Mingil hetkel seal 
kusagil üheksakümmend kas mingisugune viis-kuus, kui Avo Raup\index[ppl]{Raup, 
Avo} tegi Raadio 2-s saadet \enquote{Võrgutaja}, kutsus ta mind kui juba 
kirjutanud ja tuntud inimest, saatesse külaliseks. See asi hakkas meil kuidagi 
niimoodi klappima, et minust sai resident-saatekülaline. Esimene inimene, keda 
ma sattusin üldse esimesena üksinda tegema  (Avo oli haigeks jäänud või 
midagi), oli Abobase Systems-ist\index{Abobase Systems} Kaido 
Saarma\index[ppl]{Saarma, Kaido}. 

Kusagil sealsamas 1999 tuli mingil hetkel minu juurde 
Sarvik\index[ppl]{Sarvik|see{Sarv, Henn}}\sidenote{Legendaarne IT-mees Henn 
Sarv\index[ppl]{Sarv, Henn}.} ja ütles, et Kukust kas siis Lang või Tiido või 
keegi oli öelnud et on vaja teha arvutisaadet. Istusime sealsamas Uniprindi 
lähedal Pärnu maanteel Westmani poe vastas keldris ühes mingisugune Hollandi 
õlletoas ja mõtlesime välja, et võiks olla selline asi nagu Tehnokratt. Ja 
hakkasime tootma raadiosaadet. Juba esimesel hooajal sattusime Kukus kokku 
tegelastega, kellel oli mõte ETV-sse ka midagi teha\index{Eesti 
Rahvusringhääling!Eesti Televisioon}. \emph{Whatever}, toodame! Nii ma sattusin 
telesaatesse olema korraga enam-vähem toimetaja, saatejuht ja (mis muidugi 
mõnevõrra üllatuseks tuli) pidin panema kokku ka montaažiriba.

\question{Ja nüüd sa oled tagasi ringiga\ldots} 

Kas nüüd tagasi või edasi. Praegu ma olen Zone-s\index{Zone}, mis on täiesti 
juhuslikult ajaloos esimene kord, kus ma olen töötanud mingit otsa pidi 
IT-firmas. Ma olen vahepeal olnud reklaamiagentuuris, küll digi-tiimi juht, 
võiks öelda, et ka natuke IT poole, aga ta oli ikka nagu reklaamiagentuur, 
eksju. Trüki poole peal ja kus iganes, ma olen koolitanud ja kõike muud teinud 
aga see on esimene kord, kus mingid IT-tüübid mõtlesid, et palkaks Marveti siia 
tuututama. Ametlikult mu  müts on seotud turunduse ja kommunikatsiooniga. Aga 
ma tegelen ka selle poolega, et kui on  keegi ütleb, et midagi ei tööta ja kõik 
ütlevad, et ei noh, aga töötab, siis kuidas saada aru, et mida inimene 
tegelikult tahab. Äkki tal on õigus, et tal ei tööta. Äkki on olemas võimalus, 
et see asi, mida meie oleme nunnutanud ja silunud ja teinud maailma kõige-kõige 
paremaks, et see tema kontekstis ei tööta. Ja täiesti üllatavalt tuleb välja, 
et kui sul on piisavalt keerulised süsteemid, siis  neid olukordi, kus on mingi 
asi, mis vaatamata kõige suurematele ja parematele püüdlustele vajaks ikkagi 
teisiti toimima panemist või siis võib-olla seletamist,  et seda on uskumatult 
palju. 

\question{Küll sa selle turunduse asja ka ära optimiseerid, nagu sa kõik asjad 
ära oled optimiseerinud!}

Jah, ma üritan. Mul see lootus on natuke teistpidine. 

Kunagi Andres Kulli\index[ppl]{Kull, Andres} ja Kroonpressi\index{Kroonpress} 
seltskond tuli küsima, et kuidas panna reklaami ajalehte. Mina rääkisin, et on 
olemas PDF. Teeme parem nii, et kõik  teeksid korraliku PDF-i, leheküljendaja 
tõstab selle küljendus-softi sisse ja kõik töötab. Kull selle peale, ikkagi 
suure trükikoja juht, ütles \enquote{Väga hea, siis  me otsustame niimoodi. 
Kõik peavad saatma PDF-ina asju Postimehesse}. Ja üllatus-üllatus, nii läkski. 
Mu enda roll selle kõige juures oli, et ma olin olnud pikka aega Prismas ja 
muudes reprodes selline majasõber, kes sageli tolkneski seal ja üritas endale 
tegevust leida ja saada aru, kuidas need asjad käivad. Kuni, kaasa arvatud see, 
et Eesti esimese Linotronicu me oleme pulkadeks lahti võtnud ja midagi seal 
jootnud, sest ta  otsustas lõpetades parasjagu töö kui oli vaja midagi välja 
lasta. 

Ma olin näinud, millist roppu vaeva näevad kõik minu  sõbrad, kes on sellised 
repropealiku või  sellise repro tehniku rollis,  kehvasti ette valmistatud 
originaalidega. Ja kui see PDF-i asi hakkas meile endale majja tulema, ma 
mõtlesin, et no mina selle ussipurgi avamist küll enda peale võtma ei hakka 
(tänapäeval räägitakse rohkem surströmmingust, kui Pandora laekast). Et ainuke 
asi, mis ma saan teha, on õeptada kliendid paremini originaale saatma. Mis 
loomulikult tundus äärmiselt lihtne. See on ju nii lihtne teha:  ma lihtsalt 
ütlen teile, et seal on vaja nagu mõned linnukesed panna ja siis see kõik 
lähebki nii, nagu vaja. Aga tuleb välja, et ei. Ma olen õppinud, et on päris 
kõva pingutus aru saada,  mida teised inimesed teavad, mis on nende taust. Ja 
siis viia nad selleni, et nad saaksid aru millestki, millest sina aru saad 
seejuures võimalust mööda ise mitte liigselt kas siis masendumata või siis 
nende peale kurjaks saamata. Nii ma sattusingi õpetama  kõiki neid Pagemakereid 
ja InDesigne ja Photoshopoe ja kõiki muid asju just sellise töökorralduse poole 
pealt. Ja hetkel ma lihtsalt Zones näen, et kui vaadata kogu seda veebiga 
seonduvat, siis ilmselt tuleb selle kõigega proovida rohkem edasi minna. 

\chapter{Andres Peiker}
\index[ppl]{Peiker, Andres}

\question{Kuidas sina arvutite juurde said?}

Oli mingi 1984. aasta, äkki? Suht juhuslikult tegelikult selles mõttes, et ma 
õppisin siis keskkoolis ja käisin mingisugustel füüsika loengutel Tartu 
Ülikoolis\index{Tartu Ülikool}. Ühe tolle loengu lõpus mees nimega Otto 
Teller\index[ppl]{Teller, Otto} astus auditooriumi ette ja ütles, et kes on 
arvutitest huvitatud, võivad natukeseks veel  siia jääda. Ja noh, mingisugune 
seltskond jäi ja Otto Teller viis meid Tähe 4  
õppehoonesse\index{Tartu Ülikool!Füüsikahoone}, kus oli kaks Nairi arvutit, 
Nairi-K\index{Arvutid!Nairi!Nairi-K} ja Nairi-2\index{Arvutid!Nairi!Nairi-2} vist. 
Otto näitas noid ja ütles, et põhimõtteliselt siin nagu mingitel õhtustel 
aegadel on võimalik käia, programmeerida ja proovida asju.

\question{Millest ma kohe järeldan, et sa oled Tartu poiss?}

Jah, absoluutselt. Esimesed kakskümmend viis eluaastat ma elasin Tartus. 

\question{Ja mis ma veel järeldan on, et kui sa keskkooli ajal kuskil Tartu 
Ülikoolis loengutes käisid, sul pidi olema mingi reaalainete huvi?}

No ma õppisin Tartu 1. Keskkoolis\index{Koolid!Tartu 1. Keskkool}, meil oli 
matemaatika-füüsika eriklass. Käisin olümpiaadidel, ei mäleta täpselt, 
kust too ülikooli loengute teema üldse tuli. Füüsika tundus mulle nagu kõige 
põnevam asi üldse, et siis  saigi seal käidud.

\question{Selle loengu lõpus, mind paneb imestama, et keegi üldse ära läks. 
Kõik sedalaadi rahvas tundus arvutite vastu huvi tundvat, või siis ei olnud 
nii?}

Ei, ikka ei olnud. Ma arvan, et ikkagi pooled läksid ära. Esimesel korral 
käisime neid arvuteid vaatamas, siis öeldi, et järgmine kord saaks nagu sel 
päeval tulla, siis tuli juba vähem inimesi ja lõpuks jäi mingisugune, ma arvan, 
kolm-neli inimest võib-olla alles, kes seal nagu rohkem käima hakkasid.

\question{Oli see mingisugune ring või lihtsalt Otto isetegevus?}

Ma enam nii täpselt ei mäleta. Ma arvan, et Otto Teller\index[ppl]{Teller, 
Otto} ikkagi seal natukene juhendas ka alguses, et mis ja kuidas. Kuidagi me 
tolle AP 
programmeerimiskeelega\index{Keeled!AP}\sidenote[][-2cm]{\begin{russian}АП 
(Автоматическое Программирование)\end{russian} oli Nairil kasutusel olnud 
programmeerimiskeel. Kuna Nairi-2 oli väga levinud, muutus AP keel ka 
universaalseks algoritmide kirjeldamise keeleks venekeelses teaduskirjanduses. Keel oli muu 
hulgas kuulus ka oma mõnevõrra ebasündsa konnotatsiooni omandanud võtmesõnade 
poolest.\label{sisu:apkeel}}, mis seal Nairide  peal oli, tuttavaks saime. Ma arvan, et läbi 
Telleri, tema rääkis.

\question{Aga mingeid raamatuid või muud sellist?}
Ei, seda ma küll ei mäleta, et oleks olnud.

\question{See on huvitav asi. Läbivalt inimesed ei suuda meenutada, kuidas nad 
programmeerima õppisid, see lihtsalt tuli. Aga mida te tegite nonde Nairidega?}

Seal sai ikkagi teha väga lihtsaid  arvutusprogramme. Tollel 
arvutil ju kuvarit ei olnud, ta oli elektroonilise kirjutusmasinaga. Sa 
kirjutasid programmi, too tuli paberi peale ja  oli ainukene eksemplar  
programmist, seda sa pidid siis alles hoidma. Sest kui sa programmi parandada 
tahtsid, sa pidid vaatama toda prinditud paberit. Kõige kõvem asi, mille ma 
seal valmis tegin oli biorütmide arvutamise programm, tollal olid need 
tähtsad asjad. Noid Arvutuskeskuses\index{Tartu Ülikool!Arvutuskeskus} tehti 
ja  ma tegin Nairi-K\index{Arvutid!Nairi!Nairi-K} 
peale ka  biorütmide programmi. Ma saan aru, et too programm osutus seal 
populaarseks, et tollest 
minu perfolindist tegi keegi koopia ja siis lasti toda seal Tähe 4 
töötajatele usinasti välja, ilma et ma midagi teadnud oleksin.

\question{Need biorütmide algoritmid, mis liikusid, olid vist mõeldud käsitsi 
arvutamiseks? Kuidagi numbriliste meetoditega arvutati siinust vist.}

Ta oligi lihtne siinus,  lihtsalt sa pidid  sünniaja ütlema ja  nendel emotsionaalsel, füüsilisel,  seksuaalsel ja mis neljas oli, ma ei mäletagi, rütmil oli  siinuse 
lainepikkus lihtsalt erinev. Arvutas elatud päevade arvu ja tolle pealt 
joonistas nood neli siinust sisuliselt välja. Täiesti 
triviaalne asi iseenesest tegelikult, rohkem oligi too, et kuidas sa paberi 
peale toda siinust joonistad elektroonilise trükimasinaga. 

\question{Huvitav asi, mis oli toona oluline aga täna hakata inimestele 
biorütme joonistama\ldots}

Too siis mingisugune väga popp asi ja tundus tolle arvuti jaoks nagu niisugune 
jõukohane ülesanne. Ma arvan, et kas neli kilobaiti oli mälu tollele arvutil 
või? Ja ta oli sama suur kui mul, ma ei tea, kodus köögimööbel.

\question{Aga milleks füüsikud teda kasutasid?}

Samamoodi, ikka arvutamiseks.

\question{Aga mida nad arvutasid?}

Ei tea. Sellest nagu ei olnud juttu. Too Nairi-K\index{Arvutid!Nairi!Nairi-K} oli 
nagu väiksem masin, teises toas oli Nairi-2\index{Arvutid!Nairi!Nairi-2}. Põhiliselt, 
ma saan aru, kasuti toda. Meie sinna masinale nagu eriti ligi ei saanud, too 
oli nagu rohkem hõivatud ja rohkem \emph{advanced} ka selles mõttes, et 
tal olid lindiseadmed. Need suured lindikapid, kus siis seda magnetlinti 
keerutati ja tollel ei olnud mitte see tavaline elektrooniline kirjutusmasin 
vaid sihuke trumliga printer. Mis suutis ikka päris kiiresti paberit  
välja lasta.

\question{Kas see arvutitega möllamine oli puhtalt selline põnev pusimine või 
tundus seal mingi sügavam asi ka taga olema, et see on see, mida sa teha 
tahadki?}

Siis oli see kõik puhtalt seotud tegelikult sellega, et kuna ma olin 
matemaatika-füüsika eriklassis, siis  Andres Jaeger\index[ppl]{Jaeger, 
Andres} ülikoolist andis meile programmeerimist ka kolm aastat. Aga too 
programmeerimine oli sisuliselt ainult mingite blokk-skeemide joonistamine 
paberi peale, me arvuti ligi ei saanud. Ja seal Tähe tänaval oli võimalus nagu 
ise järele proovida  seda, mida sa olid  paberi peal teinud. 
Toda  kooli õppeprogramm ei võimaldanud.

\question{Paberi peale skeemide joonistamine võib ju lihtsasti huvi ära tappa 
aga sul ei tapnud?}

Ei, ta ei tapnud kindlasti, ka too blokk-skeemide joonistamine 
oli huvitav. Kui sa tolle ülesande said, siis Andres ütles ka, et umbes, et kes 
suudab  kolme \emph{if}-iga teha on hea, kahe \emph{if}-iga on 
väga hea ja ühega juba kahtlane. Ja siis sul oli nagu eesmärk olemas, et sa 
pidid ühega ära tegema, too asi kõnetas mind. Ja siis sa said õpetaja käest 
kiita ka, et \enquote{Tõesti, ma mäletan siin võib-olla viis aastat tagasi oli 
meil ka üks õpilane, kes suutis selle algoritmi selliselt ära teha,  
väga hea!}.

\question{Ühesõnaga sinu jaoks oli too blokk-skeemide joonistamine niisugune 
ülesanne või pusle.}

Jah, absoluutselt. Loomulikult jõuga sa suudad tolle ülesande lihtsalt ära teha, aga et 
nüüd kuidas ta nagu kõige optimaalsem saaks, kõige parem. Too oli huvitav.

\question{Huvitav, et koolil tollal ühtegi arvutit ei olnud. Tartu linna peal 
arvuteid ju oli, miks koolist mõne juures ei käidud?}

Ei olnud, siis ei olnud ikka mingisuguseid arvuteid kuskil. Tähe tänaval olid 
nood kaks Nairit. Loomulikult seal ülikooli arvutuskeskuses\index{Tartu 
Ülikool!Arvutuskeskus} oli ES\index{Arvutid!ES EVM}, kus üldse nagu ise arvuti 
ligi ei saanud,  operaatorid lõid programmi sisse. Ja siis oli Füüsika 
Instituudis\index{Füüsika Instituut} Riia maantee lõpus, seal oli ka 
mingisugune PDP-11\index{Arvutid!PDP-11} äkki?

Ma arvan, et umbes tollel ajal kuskil Anne Villems\index[ppl]{Villems, Anne} 
sebis need Apple II-d\index{Arvutid!Apple II} ka tegelikult siis Vanemuise 
tänavale\index{Tartu Ülikool!Vanemuise tänava õppehoone}. Too oli koht, kuhu ma 
järgmisena jõudsin peale  Nairisid.  

Ma arvan, et see oli ka Otto Teller, kes meid sinna viis. Ma isegi tundsin nagu 
pärast natukene piinlikkust, et tema näitas noid masinaid ja siis ma tegelikult 
hülgasin tolle Tähe tänava ja ei käinud enam tema juures. Vahtisin ainult seal 
Vanemuise tänaval, Apple-d olid palju  ägedamad. Ikkagi monitor ja nelikümmend 
kaheksa kilobaiti mälu, oli ikkagi nagu ulmeliselt kiire.

\question{Kas Apple peal mängimine ka teemaks tuli?}

Jaa,  absoluutselt. Too oli päris hull selles mõttes, see oli kindlasti minu 
elu kõige suurem arvutimängude periood. Ma oleks peaaegu kahe klassivennaga 
keemia eksamile hiljaks jäänud. Sellepärast, et siis seisu salvestada  ei 
saanud, sa pidid lihtsalt nii kaugele mängima, kui said. Juhuslikult juhtus 
nii hästi minema, et oleks pidanud juba eksamile minema, aga tuli järgmine 
level ja pidid edasi mängima.

\question{Mis mängu te mängisite?}

Apple-i peal sihuke standardne mäng on nagu Pacman, mis on mänguautomaatides ja 
igal pool. Apple'i peal nimetati teda Super Puckman-iks\index{Mängud!Super 
Puckman}. Ma siiamaani pean teda kõige lahedamaks mänguks, mida ma olen kunagi 
mänginud. Toda Pacman-i oli kõigi teiste arvutite peal ka. Aga 
  oli  katastrofaalne  erinevust tolles algoritmis, kuidas nood 
neli kolli liikusid: kõigi ülejäänud arvutite peal, nii palju kui mina olen 
mänginud, liikusid nad \emph{random}-iga. Aga Apple-i peal oli neil oma kindel 
algoritm. Ja tulemuseks oli see, et kui sa ise tegid täpselt ühtemoodi siis 
situatsioon kordus mängust mängu. Meil olid esimese kuue 
leveli jaoks tegelikult sammud sisuliselt algusest lõpuni välja töötatud. Sa 
teadsid täpselt, 
kuidas sa terve ekraani puhtaks mängisid ja järgmisele levelile said. 
Sealt edasi oli sisuliselt mingisugused paar avangut, mida sai erinevatel 
levelitel kasutada.

\question{Põhimõtteliselt Pacman kui male?}

Natukene. Ja selle tõttu polnud võimalik teiste versioonide peal mängida, sest 
seal kollid liikusid lihtsalt  \emph{random}-iga. Arvutimängud jah, 
absoluutselt. No seal oli teisi teisi veel, aga Super Pucman kindlasti oli 
kõige olulisem.

\question{Ja programmeerimine ka kindlasti?}

Nojah, muidugi. Seal oli Basic\index{Keeled!Basic}. Ütleme, et alguses ma 
kirjutasin ikkagi Basicus, aga pärast sai ikkagi valdavalt 
assembleris\index{Keeled!Assembler} kirjutatud tolle pärast, et programm töötas 
oluliselt kiiremini kui sa ta assembleris tegid. 

\question{Kuidas Basicust assemblerisse hüppamine käis? Basicu võib tõesti 
suhtliselt lihtsasti üles korjata aga assembleris sa pead ikka täpselt teadma, 
mida sa teed?}

Ka Basicu puhul sa pidid ikkagi  arvuti arhitektuurist aru saama. Et kus 
tolles neljakümne kaheksas kilobaidis nüüd paiknes tekstiekraan, kus  
graafiline ekraan, kus  su programm, kus oli opsüsteem. Tegelikult 
arvuti arhitektuurist aru saamine tekkis Basicu kõrvalt suhteliselt kiiresti. Aga 
assembler tuli ikkagi tänu sellele, et osad asjad olid väga aeglased. 

Üks asi, 
mida ma seal tegin, oli orienteerumisneljapäevakute protokollid. Tollega 
alustas tegelikult Peep Abel\index[ppl]{Abel, Peep}, kes ülikoolis 
rakendusmatemaatikat õppis, aga ta lõpetas ülikooli ja andis mulle kogu 
tolle programmi komplekti üle. Aga see oli minu jaoks liiga aeglane. Andmemaht 
oli tolle  neljakümne kaheksa kilobaidi jaoks natuke liiga suur, seal oli ka 
mitme flopiga mängimist, et need andmebaasid ära mahuksid. Ja ma mõtlesin, 
et ma kirjutan \emph{from scratch} kõik assembleris. Kirjutasingi, kõik oli 
kohe palju kiirem. 

Seejärel sai kogu too Apple II opsüsteem tegelikult disassembleeritud, kood ära 
kommenteeritud ja  imestatud, et päris mitmes kohas Steve Wozniak oli 
hämmastavaid trikke teinud. Tolleks ajaks, 
kui hakkasin diassembleerima, siis ise ka arvasid et tead assemblerit juba 
väga hästi. Aga siis ikkagi oli paar sellist asja, mida avastasid, et 
\enquote{vau, kuidas teha saab!}, sihukene nagu  pisut \emph{hidden trick} 
tegelikult. Ühesõnaga, assembleris olid ühebaidised, kahebaidised ja 
kolmebaidised käsud. Ja trikk oli see, et ühte kolmebaidist käsku oli võimalik 
kasutada selliselt, et kui sul programm jooksis otse läbi, siis too 
kolmebaidine käsk ei teinud midagi. Aga sa said tolle kolme baidi viimast 
kahte baiti kasutada selliselt, et sa kuskilt eespoolt hüppasid tolle teise 
baidi peale, mis oli siis teine \emph{command}. Sinna kolmebaidise käsu 
viimasesse kahte baiti sa paigutasid tegelikult teise assembleri käsu. Sihukesi 
elegantseid trikke oli tehtud. Pärast püüdsid ise ka nagu mõnes kohas mõelda, 
et kas ma saaks toda nippi efektiivselt kasutada.

\question{Aga kust see teadmine tekkis, et nii saab teha, mingi teadmine oli ju 
selle kõige aluseks?}

Ei, no kui me tolle koodi disassembeerisime, siis sa pidid kogu tollest 
algoritmist aru saama, et mismoodi see asi töötab. Tegelikult oli asi selles, 
et kettaga suhtlemine oli suhteliselt aeglane, ma tahtsin seda kiiremaks saada. 
\emph{Seek time} oli seal üks asi, mis palju mängis ja asi lõppes sellega, et 
ma tegelikult kirjutasin assembleris ise ketaste kopeerimise programmi, mis 
töötas  opsüsteemist mingi kümme korda kiiremini. 

\question{Kümme korda!}

Sa pidid  optimeerima lihtsalt tolle pea liikumise. Kui sa tahtsid kogu ketta 
ära kopeerida, siis sa pidid, ma ei mäleta, kas seestpoolt väljapoole või 
väljaspoolt sissepoole sõitma. Tegid ühe liikumisega kirjutamise ära, 
mitte ei käinud edasi-tagasi. Standardselt käis ketta pea alati edasi-tagasi 
\sidenote{Flopiseadmed, nagu elektromehaanilised kõvakettad siiani, 
kasutavad andmete lugemiseks ja kirjutamiseks pöörleva ketta pinna lähedal 
liikuvat tundlikku pead. Ja üheks kõige suuremaks pudelikaelaks andmete 
liigutamisel flopilt oligi seadme võimekus toda pead ühest ketta servast teise 
viia.}.

\question{Jällegi, kust üldse tekkis niisugune arusaam, et kettaseadmega saab 
niisugusi trikke teha, et võiks ette võtta sihukese asja? See vajab ju 
teadmist, julgust ja natuke arrogantsi ka, et \enquote{mis see Woz ikka 
ketastest teab}?}

Too opsüsteem oli ikkagi  tehtud universaalne, selles mõttes see ketta 
kopeerimise programm sai tehtud nagu \emph{dedicated} ehk optimeerituna 
mingisuguse konkreetse masina jaoks. Tol ajal oli oluline, et kui kuskilt kas või 
Moskvast mingisugune tüüp tuli, tal oli kettaid kaasas, neid pidid 
kiiresti suutma kopeerida. Mitte, et sa ei jokuta seal nagu tund aega 
kopeerida, vaid et sa saad  nood programmid endale kiirelt ära tõmmata. 
Praktiline vajadus. Kust sa neid programme saad, internetti ei olnud. Ma käisin 
ise tegelikult koos ühe klassivennaga korra Moskvas puhtalt selle pärast, et 
mingisuguseid arvutimänge saada.

\question{Moskva on suur linn\ldots}

Lihtsalt too vend käis ise Tartu Ülikoolisja me saime temaga kokku, tal olid 
mingid programmid, me kopeerisime need ära. Nii me saime tema kontakti ja ta 
ütles umbes, et \enquote{kui Moskvasse satute, siis alati \emph{very welcome}}. 
Ja siis me ükskord lihtsalt läksimegi.

\question{Rongiga?}

Rongiga. 

\question{Huvitav, mis see andmeside kiirus tuleb, kui arvestada, et sõidad 
rongiga edasi-tagasi ja kopeerid flopisid?}

Ei julge öelda, kolmsada kuuskümmend kilobaiti oli üks ketas või? Eks see oli 
vast kõige kiirem viis ikkagi andmeid transportida. E-mail tuli ikkagi 
mingisugused aastad hiljem ja too käis  kord päevas, helistasid modemiga 
sisse ja tõmbasid meilid ära.

\question{Kui keskkool otsa sai, mida sa õppima läksid?}

Tartu Ülikooli rakendusmatemaatikat\index{Tartu 
Ülikool!Matemaatikateaduskond!Rakendusmatemaatika}. 

\question{Sõjaväkke ei võetud vahepeal?}

Õnnestus ära viilida. Aga õppimisest tegelikult palju välja ei tulnud 
tolle pärast, et ma istusin ikkagi seal Apple-te juures edasi, nii nagu 
keskkooli ajal. Esimese kursuse ma tegelikult tegin ära, kõik matemaatika 
eksamid olid viied, aga inglise keelega kukkusin välja. Kuna ma hõbemedaliga 
lõpetasin  siis uuesti sisse astumine oli väga lihtne, pidi ainult matemaatika 
eksami tegema, mis oli triviaalne. Aga siis ma enam ei viitsinud üldse 
loengutesse minna, sest  kõik matemaatika eksamid olid tehtud, me oleks pidanud 
ainult inglise keele pärast seal esimese kursusel käima.

Nii ma istusin seal Apple-te taga. 

Loomakasvatuse ja Veterinaaria instituudi direktor Olkonen\sidenote{Instituudi 
täpne nimetus oli aastani 1994 (mil ta liideti Põllumajandusülikooliga) Eesti 
Loomakasvatuse ja Veterinaaria Teadusliku Uurimise Instituut. Ja 
piimandusteadlane Arvi Olkonen\index[ppl]{Olkonen, Arvi} ei olnud mitte terve 
instituudi direktor, vaid juhatas instituudi piimanduse laborit.} tegi 
doktoritööd ja tal oli terve bussitäis tädisid, kes olid valmis andmeid 
sisestama. Aga tal ei olnud, kuhu neid andmeid sisestada ja seda, mis arvutab. 
Ma tegin talle  programmi, mis võimaldas andmeid sisestada. Seal oli siis 
oluline teha selline \emph{user inteface}, et tädid kuidagi eksida ei 
saaks, too oli kindlasti kõige keerulisem. Arvutuse osa oli  tegelikult 
lihtne.

\question{Mida seal arvutada tuli? Mingit statistikat?}

Need olid mingid piimaproovid, kus olid siis laktoosi protsent, valgu protsent ja hulk 
igasuguseid muid karakteristikud.  Ma ei mäleta täpselt, Olkonen ütles ikkagi 
nood algoritmid ette, mida tuleb teha. Ma võisin matemaatiliselt nõu anda, aga  
üldiselt ta ikkagi teadis ise, mida ta tegi. 

\question{Mis tähendab siis seda, et sa olid kusagil palgal?}

Ma olin seal arvutiklassis Tartu Ülikoolis poole kohaga palgal,  insenerina. 
Kuna too Loomakasvatuse ja Veterinaaria Instituut mulle kuidagi nagu 
ühekordselt maksta ei saanud, siis mind võeti sinna ka tööle. Aga ma ei käinud 
seal kunagi, ma olin seal mingi aasta või kaks tööl lihtsalt selleks, et tolle 
programmi eest nii-öelda tasu saada. Ma ei viitsinud palka ka minna välja 
võtma,  siis pangakontosid ei olnud. Mispeale ülemus tuli mulle autoga järgi ja 
viis palka välja võtma, sest ta ei jõudnud enam  kassapidaja kisa ära 
kuulata.

\question{Tuldi autoga järgi ja viidi raha saama. Programmeerija magus 
elu\ldots}

Jah, direktor Olkonen oli väga lõbus sell. Nende oma inimestega, ta oli hirmus 
kuri. Alati kui me sinna läksime, siis ta kõigepealt sõimas kõigil näo täis. 
Aga väga ettevõtlik tüüp. Ma mäletan, et kunagi ma olin kodus, isaga saunas. Ja 
ema tuli sauna, et \enquote{kuule, mingi mees tuli}. See oli siis Olkonen, tal 
oli midagi kiirelt vaja. Ja mul ema ütles, et ta oli nagu enam-vähem ilma 
tutvustamata uksest sisse astunud, läinud kohe elutuppa ja maha istunud. Et ta 
võib oodata, ei ole probleemi. Väga sihikindel. 

\question{Kui sa loengutesse ei jõudnud, siis mingi asi pidi sind seal arvutite 
juures kinni hoidma? Assembleri ja pusimise huvi või midagi muud?}

Nojah, ikka. Assembleris ma kirjutasin tekstiredaktori, kuhu sai ikka päris 
ohtralt igasuguseid \emph{feature}-sid tehtud. Too oli kindlasti kõige 
keerulisem asi, mul peaks vist isegi kood paberi peal väljatrükituna  
alles olema. Kas viis või kuus tuhat rida assemblerit. 

\question{Seda ei ole üldse nii palju}

Ikka üksjagu. 

\question{Assembleri koodi mõttes on seda palju, aga tekstiredaktor viie 
tuhande reaga pole paha!}

No ma aitasin seal ühel tütarlapsel, kes mulle väga meeldis,  kursusetöid teha 
ja tolleks oli seda tekstiredaktorit nagu vaja. Muidu oleks pidanud 
kirjutusmasinal trükkima. Aga arvutis ühtegi korralikku tekstiredaktorit polnud. No 
oleks kah saanud ühte või teistviisi teha,  mingisugused hädised asjad olid. 
Aga et kõik suured-väiksed tähed, sellised asjad, nende ei olnud lahendust ja siis 
ma kirjutasin.

\question{Jaan Tallinn kirjutas ka\sidenote{Vt. lk. 
\pageref{sisu!jaani_tekstiredaktor}} esimese asjana omale tekstiredaktori. Kas 
siis Internet on teinud hoopis karuteene? Vanasti, kui tahtsid 
tektsiredaktorit, pidid ise kirjutama. Nüüd võtad millise iganes.}

Eks ta siis oli ka natukene lihtsalt see, et sul ei olnud neid programme 
kuskilt saada. Ameerikas olid Apple'i jaoks ilmselt kõik programmid olemas, aga nad ei jõudnud lihtsalt Eestisse ja siis tegid ise. Aega ka oli ja\ldots

\question{Mis sul tol ajal ettekujutus oli, et kuhu see viib? Et istud järgmised 
20 aastat Vanemuise tänava klassis?}

Mul ausalt öeldes ei olnud küll mingisugust väga konkreetset plaani, kuhu see 
viib.

Kuskil üheksakümnendal aastal tuli meilinduse käimapanek seal Vanemuise tänavas, 
umbes sel ajal Taavi Talvik\index[ppl]{Talvik, Taavi} 
kutsus mind Postimehe\index{Postimees} toimetusse.

Sinna ta oli mingisuguses SCO UNIX'i\index{OS!SCO UNIX} valmis pannud ja 
mingisuguse hulga terminale, mille kaudu ajakirjanikud oma artikleid 
sisestasid. Emacs oli vist tekstiredaktor. Ja eesmärk oli siis sinna peale eesti keele 
õigekirjakontrolli teha. Tollega ma siis seal tegelesin. 

\question{Ühesõnaga sa läksid sealt Vanemuise tänavalt Postimehesse?}

Jah. Ma käisin seal Vanemuise tänaval ka ikkagi sest,  nood 
tädikesed, kes seda arvutiklassi seal haldasid, tehniliselt liiga võimekad ei 
olnud. Ja nii oli ikkagi kasulik, kui ma sel õhtul läbi käisin ja neile mingite 
asjadega nõu andsin. Aga jah, Postimehes tuli ikkagi \emph{full time job}.

\question{Õigekirjakontrolli tegemine ei ole triviaalne asi, seal peab ju 
keelest ka aru saama?}

Ei ole triviaalne asi. Siis ma avastasingi, kui neetult keeruline see eesti 
keel on. Miks kurat peab iga teine sõna mingi erand olema? Väga tüütu oli. Ega 
me toda valmis ei saanud, tegelikult.

\question{Umbes 93. aasta paiku hakkasid tekkima Filosoftid\index{Filosoft} ja 
niisugused asjad, nad tegid Wordile eesti keele spelleri. Aga selleks ajaks 
olid teadus ja arvutusvõimsus edasi läinud}

Jah, absoluutselt, aga  1993 oli juba see aeg, kui ma tulin ära Tallinnasse 
Hansapanka\index{Pangad!Hansapank}.

\question{Sa tulidki otse Postimehest?}

Mingisugune lühikene periood oli Postimehe ja Hansapanga vahel ka tegelikult, 
kus ma olin  hulgifirmas\sidenote{Nõukogude ajal olid poed 
prestii\v{z}sed asutused, sest nende kaudu jagati valitutele defitsiitset 
nõukogude kaupa. Kujutage nüüd ette, kuidas vaadati vastses vabariigis hulgifirmadele, 
kelle kaudu ühtäkki jagati \emph{lääne} kaupa.}. Neile ma  kirjutasin ka mingit programmi. 
Aga too oli väga selline kaootiline koht selles mõttes, et  bisnis läks 
hulgifirmal nagu hirmus hästi ja iga kuu nood kolm kutti, kes ta omanikud olid, 
ostsid igaüks endale uue BMW ja tolmutasid nendega ümber tolle maja sõita. Et 
ei olnud liiga motiveeriv keskkond, tegelikult.

\question{Kust Taavi sind üles leidis?}

Ega ma nüüd ei julge öelda, ausalt öeldes peast, kus ma Taaviga tuttavaks sain. 
Sel ajal, kui mina Apple'te taga istusin, istus Taavi tegelikult sealsamas Tähe 
4, kus ma esimest korda Nairidega kokku puutusin, keldris, kus oli mingisugune 
IBM PC.

Ja kas Taavi tegi midagi äkki Tartu Ülikooli Raamatukogule\index{Tartu 
Ülikool!Raamatukogu} ja mina olin ka tollega kuidagi seotud ja kas me äkki seal 
Tartu Ülikooli Raamatukogus PC taga saime kuidagi kokku? Ma näitasin Apple'id 
Taavile ja tema näitas mulle toda PC-d. Oli sihukene mäng nagu King's 
Quest\index{Mängud!King's Quest}, mida me Taaviga mängisime seal Tähe tänaval. 
Ja noh, sealt me tuttavaks saime, et me tolle King's Questi\sidenote{King's 
Quest on Sierra Entertainmenti seiklusmängude sari, mida peetakse omas 
valdkonnas klassikaks. Mängud ilmusid aastatel 1980 kuni 1998.} läbi mängisime.

\question{King's Quest oli ju seiklusmäng?}

Just. Tollega läks ikka aega, et lõpuni mängida,  istusime ikka palju õhtuid 
seal.

\question{Eks Taavil oli ühel hetkel Postimehes abi vaja ja siis ta kutsus 
sind?}

Täpselt. 

\question{Aga kuidas sa Hansapanka\index{Pangad!Hansapank} sattusid, see on huvitav 
lugu!}

Hansapanka ma sattusin samuti tänu Taavi Talvikule\index[ppl]{Talvik, Taavi}. 
Taavi töötas Valitsussides\index{Valitsusside} tol ajal ja 
Microlinkist\index{Microlink} Rainer Nõlvak\index[ppl]{Nõlvak, Rainer}, ma 
arvan, oli see, kes küsis Taavi käest,  et Tõnis Sildmäe\index[ppl]{Sildmäe, 
Tõnis} otsib kedagi, kes Unixit tunneks. Taavi ütles, et tema küll ei taha 
minna ja küsis minu käest. Mõtlesin, et ah, suva, et ma võin ju rääkida ja 
kuulata, et mis seal siis teema on. Tulin Tallinnasse Tõnis Sildmäega rääkima, 
Sildmäe küll jättis mulje, et tal on terve bussitäis Unixi-mehi ukse taga 
järjekorras, keda ta kõiki intervjueerib, aga vist tegelikult peale minu ühtegi 
ei olnudki. Igatahes ma sain panka tööle.

Too SCO UNIX\index{OS!SCO UNIX} oli sinna juba ära installitud ja Tarmo 
Pajumets\index[ppl]{Pajumets, Tarmo} püüdis  sinna peale Oracle't\index{Oracle} 
installida. Aga ega nad tollest SCO UNIX'ist midagi ei teadnud, nii et esimese 
päeva lõunaks nad mõlemad läksid sealt konsooli tagant ära ja rohkem  
tagasi ei tulnud, kui nad vaatasid, et ma vist tean natuke rohkem.

\question{Inimesed said oma teadmiste piiridest aru. Tollel hetkel pank kui 
selline oli juba olemas, mis infosüsteemi peal ta käis?}

Pank käis Paradox'i\index{Paradox} peal. Aga eks too Oracle'i andmebaasi majja 
toomine oli, ma arvan,  üks väga paljudest Hansapanga edu aluseks olevatest 
strateegilistest otsusest. Paradox 
töötas tol hetkel täiesti normaalselt, ei olnud häda midagi. Aga juba oli Tõnis 
Sildmäe\index[ppl]{Sildmäe, Tõnis} välja raalinud, et tegelikult me 
peaksime  mingisuguse tõsisema andmebaasi mootori sinna alla panema. 
Esialgu läks Oracle Novelli\index{Novell} peale  aga siis me saime tolle SCO ka 
nii kaugele, et migreerisime andmebaasi sinna.

\question{Räägi sellest palun korra lähemalt. Kellegi, ilmselt siis Tõnise, 
peas oli arusaam, et arhitektuursed otsused võivad olla ärilise edu aluseks. 
Üheksakümnendate alguses see ei olnud väga levinud asi, kust tal see tuli?}

Ma usun,  et panga seltskonnal oli ikkagi ka selge arusaamine Paradoxi 
tehnoloogilistest piirangutest ja samal ajal oli ka arusaamine, kuhu poole see 
pank liigub. Ma arvan, et tolleks hetkeks, kui mina sinna tulin, äkki 
oli seal minu kõrvallaua peal juba tegelikult esimene sularahaautomaat.  Oli 
sihuke suhteliselt pisikene,  mahtus laua peale, IBMi oma. Kaart ei käinud 
sisse, vaid tuli magnetriba lihtsalt läbi tõmmata. Ma arvan, et ATM-ide asi oli 
üks, mis  tolle Paradoxi andmebaasi piirangud välja tõi. Samamoodi, kuna 
klientide arv kasvas plahvatuslikult, ilmselt ka tolle pealt nähti, et too 
Paradox ei suuda tegelikult, kui selline kasv jätkub, ära teenindada.

\question{Teine asi, mis mind ikka on huvitanud on, et samal ajal toimetati 
päris mitmes pangas valmis tarkvaraga. Osteti Britimaalt pangasoft ja tehti 
sellega panka. Miks Hansapank teistmoodi tegi?}

Seda oskavad need öelda, kes päris alguses pangas olid. Ma ei 
tea, kuidas too Spin Development\index{Spin Development} sinna Hansapanka tuli 
Et noh, nimigi selline,  ilmselt oli seal siis mingeid \emph{developer}-e. Ja 
ilmselt esimene ülesanne, mida teha tuli, oligi mingisugune väike tükikene ja 
kui too läks hästi, siis sealt hakkas asi arenema. Ma ei oska öelda.

\question{Spin Development on siis Crebiti\index{Crebit} algus?}

Jah. Kui mina tööle läksin, siis esimese palga maksja oli tegelikult Spin 
Development, mis minu meelest nimetati lihtsalt 
Crebit-iks ringi. Ja mingi aeg hiljem siis, ma saan aru, Londoni 
kindlustusfirma ütles pangale, et kuulge, et te ei tea IT-st mitte midagi, kogu 
asi on väljas mingisuguses täiesti iseseisvas ettevõttes, et kuidas te oma 
riske juhite. Mis siis lõppes sellega, et Hansapank ostis Tõnise käest 
Crebiti aktsiad ära ja kõik me tulime  Hansapanka tööle. Crebiti 
juriidiline keha jäi alles ja on kuni tänase päevani Swedbank Support OÜ nime 
all olemas.

\question{Huvitav, et kultuur oli jätkuvalt Crebiti oma. Sest kui mina pangast 
ära tulin aastal 2002, siis viimane särk, mis pank mulle väljastas, oli Crebiti 
logoga. Jube elujõuline asi!}

No kindlasti. Mitte ainult Crebit, vaid toob pank ise tervikuna oli tegelikult 
äärmiselt elujõuline. Ütleme, vähemalt kuni tolle hetkeni, kui
Hoiupangaga\index{Pangad!Hoiupank} liituti. Siis toimus ikkagi suur kultuuriline 
muutus, tuli lihtsalt väga palju teisi inimesi juurde.

\question{Kust see kultuur tuli?}

Ma olen toda mõelnud. Ilmselt ühelt poolt ilmselt oli kõigil inimestel, 
ikkagi väga selge saavutusvajadus oma asja väga hästi teha. Seal 
isegi ei pidanud minu meelest neid, kes võib-olla ei \emph{perform}-inud 
piisavalt hästi, lahti laskma, vaid nad läksid ise ära. Seesama lugu, kui  
Pajumets\index[ppl]{Pajumets, Tarmo} seda Oracle't installis. 
Tegelikult oli üks mees seal veel kõrval, kes tolle SCO UNIXi installis. 
Kui mina liitusin, siis ma saan aru, et too mees ise läks paari nädala pärast 
ära, tegelikult. Teda ei lasknud keegi lahti, ise kirjutas lahkumisavalduse, 
sest et ta sai aru, et tal ei ole seal enam midagi teha. Ja noh, too kultuur 
oli absoluutselt kõigile ühesugune. Sa ei pidanud kaks korda kellelegi ütlema, 
sa teadsid, et asi on tehtud.

\question{Kui sa alustasid, siis sa kirjutasid assembleris koodi, puhas 
arenduse värk. Aga pangas sa läksid kohe asjade opereerimise peale, kuidas ja 
miks see nihe toimus?}

Ega seal mingisugust  teadlikku valikut väga ei olnud. Töö tundus huvitav, ega 
ma Oracle baasi polnud varem näinud ju. Ma kuidagi ei mõelnud, et ma olen 
programmeerija. Tegelikult ju ka seal arvutiklassi Apple'te juures  tööülesanne 
oli tegelikult kõigi nende inimeste assisteerimine, ülalhoid, et nood arvutid 
töötaksid, et probleemid nendega oleksid lahendatud. Programmeerimine oli 
puhtalt hobi muu töö kõrval. Kuigi kogu asi algas loomulikult 
programmeerimisest Nairide peal. Ja ma arvan, et tollal ka ei olnud liiga palju 
konteksti, et \enquote{need on arendajad ja need on ülalhoidjad}. Ma arvan, et 
too tuli hiljem, siis kõik lihtsalt tegutsesid.

\question{Õige, see on nii pikalt olnud aga ei ole pidanud alati nii olema. SCO 
jäi sulle külge Postimehest, aga Oracle?}

No see saigi külge sealt Hansapangast. 

\question{Hakkasid lihtsalt tegema? Mitte väga palju aastaid hiljem oli 
tegemist ühe maailma suurima Oracle koodi baase, Oracle konsultandid käisid 
majas ja rääkisid, et nad ei ole kuskil maailmas midagi sellist näinud}

Võis olla küll. Vilve Vene\index[ppl]{Vene, Vilve} ja Juta 
Joost\index[ppl]{Joost, Juta} ja kes seal kirjutasid, kirjutasid PL/SQL-i ikka 
usinasti. Toda oli tõesti väga palju, väga väga palju. Oracle Forms 
oli tehnoloogiaks, et ilmselt oli nii teda kõige efektiivsem teha. Eks 
 baasi protseduurid käisid kõik kiiremini kui mingisugune, noh, 
klient-server asi.

\question{Sul ei tekkinud tunnet, et las see Oracle käib siin, et 
programmeeriks parem?}

Eks programmeerima pidi ikka natukene selles mõttes, et skripte tuli kirjutada, 
mis  siis kogu toda asja üleval hoidsid. Kas või seesama, et kui sa SCO 
UNIX-iga masina üles \emph{bood}-id, et kuidas andmebaas käima pannakse. Ega 
 Oracle installi juures mingeid skripte ei olnud, kirjutasid ise 
 skriptid, mis tolle baasi ja \emph{listener}-i käima panid. Kogu 
\emph{backup}-i tegemine, tolleks pidi skriptid kirjutama, lisaks sellele kõik 
need \emph{batch} protsessid, mis olid C-s kirjutatud. 
Noid skripte  sai tegelikult kirjutatud ikkagi päris päris palju.

\question{See, mida sa kirjeldad, on päris keeruline asi, mis kuidagi nendest 
eri tehnoloogiatest peaks terviku moodustama. Kuidas see tervik tekkis, kes 
seda juhtis? Kes arhitekt oli?}

Ega siis kedagi arhitektiks ei nimetatud. Ma julgeks ise arvata, et vast 
tarkvara kontekstis arhitekt oli ikkagi Vilve Vene\index[ppl]{Vene, Vilve}. 
Vähemalt  mulle on selline mulje jäänud, keegi sellist terminit ei kasutanud. 

See kontseptsioon, kuidas  kõik tarkvaraliselt kokku töötab, ma arvan, tuli 
ikkagi ennekõike Vilvelt. \emph{Non-functional requirements}-id tekitasin mina. 
See sama skriptide asi, et iga  C programm, mis mingit \emph{patch}-i tegi, ei 
oleks erinev. See tuli kuidagi ära standardiseerida,  ma pidin mingisugused  
mittefunktsionaalsed nõuded esitama, et nad kõik oleksid ühetaolised, et ma 
saaksin kasutada mingit ühte skripti paljude asjade käivitamiseks.

\question{Kui Postimees käib ka siis, kui ajakirjanikud trükimasinaid 
kasutavad, siis pank enam trükimasina peal ei käi. \enquote{Pusime ja vaatame, 
kuidas Woz on teinud} pidi struktuursemaks muutuma, kuidas see juhtus?}

Kuna  kasv oli nii kiire, siis igaüks pidi tegelikult vaatama  mitte ainult 
seda, kuidas see asi täna ära \emph{run}-ib, vaid ka seda, milline see asi nagu 
aasta pärast välja näeks. Ja kindlasti Tõnis Sildmäe\index[ppl]{Sildmäe, Tõnis} 
fasiliteeris ka seda, et tuleksid igasugused erinevad kontaktid, kes 
mingisuguseid uusi lahendusi pakuksid. Nii need 
asjad arenesid edasi ka. SCO UNIX-ile ju samamoodi  tulid tehnilised piirangud 
ette. 1996 või 1997 sai see HP-UX'i\index{OS!HP-UX} vastu välja vahetatud, enne 
vahetust olid mul nii HP kui SUN-i server laua peal ja sai võrreldud, kumb  
kiirem on. Tolleks ajaks oli pank ka piisavalt suur ja oli selge, et me  ei 
pane ühte masinat, vaid me paneme klastri. Sai nende \emph{vendor}-itega  
klastri lahendused läbi räägitud\ldots

Mingisugust hüpet ei toimunud, et enne oli anarhia ja siis  tehti kõik asjad 
korda. Kõik arenes evolutsiooniliselt, igal aastal vahetati lahendusi uute 
vastu välja. Teisti ei oleks lihtsalt üle elanud toda kümme aastat kestnud olukorda, 
kus iga, ma ei tea, üheksa kuu tagant  kahekordistusid klientide arv, käive 
kasum, mis iganes. Kõik numbrid kahekordistusid üheksa kuuga kümme aastat 
järjest.

\question{Sellist kasvu ei kujuta tänapäeval nagu väga ette enam, kui sa just 
kuskil Skype moodi kohas ei tööta}

Nojah, ega neid ettevõtted ongi maailmas väga vähe vist, kes nii kiiresti nii 
pikalt suudavad kasvada. Oli mingisugune \emph{success story}, jah.

\question{Ma mäletan, et sajandivahetuseks oli panka tekkinud üsna 
spetsialiseerunud tiim, kes kogu kupatust opereeris. Kuidas see kolmik, 
mille peal  kogu panga maailm püsti seisis, tekkis?}

Aja jooksul selles mõttes, et Madis Ollisaar\index[ppl]{Ollisaar, Madis} oli 
enne mind olemas. Ma ei teagi päris täpselt, mis tema roll päris alguses oli. 
Sel ajal, kui mina seal tolle Oracle baasiga toimetama hakkasin, siis minu 
asi oli tehniline pool, et andmebaasi \emph{engine} töötaks ja Madise 
asi oli luua uusi tabeleid ja teha indekseid ja vaadata, et päringud hästi 
käivad ja nii edasi-tagasi. Ja too roll jätkus tal edasi. Toomas 
Suurmets\index[ppl]{Suurmets, Toomas} tuli\ldots
Ma pidin peaaegu ütlema, et ta tuli koos Hoiupangaga liitumisega aga tegelikult 
ei tulnud. Ta töötas Hoiupangas, aga 
tegelikult ta tuli Hansapanka kaks aastat enne seda, kui Hoiupank ära osteti. 
Tolleks ajaks istus 
tema juba õigel pool lauda. Ja Toomas  nagu täiendas seda seltskonda. Kui 
Madis oli nagu kõige ülemine, nii-öelda \emph{data layer}, mina teadsin 
andmebaasi \emph{engine} osa, siis Toomas oli see mees, kes \emph{netowrkist} 
ja \emph{storage}-st hästi jagas.  Too kokku andiski kogu tehnoloogilise 
\emph{stack}-i, et põhiasi töötaks

\question{See tiim töötas jube hästi!}

No me istusime ühes toas. Ühes infoväljas kogu aeg, alati on võimalik öelda, 
mis toimub.

\question{Ma tean, sul on Wagneri huvi, kas see oli juba tol ajal? Ma mäletan, 
et teie toas kapi otsas oli makk, kust aegajalt tuli eepilist klassikalist 
muusikat?}

Ausalt öeldes ma isegi päris aastaarvu jälle julge öelda. Amazoni veel ei olnud, 
esimesed CD-d, mis ma Internetist ostsin, ostsin kohast nimega cdnow.com. 
Toda klassikalist muusikat sai mängitud jah. Mitte küll 
Wagnerit, põhiliselt tegelikult, ma julgeks arvata, Mozartit tol ajal. Päris 
alguses ma ostsin mingisugused Enrico Caruso plaadid. Aga pärast ma ostsin 
Mozartit ka ja mängisime seal neid, teisi see kuidagi ei seganud. Eks me tegime 
erinevaid asju. Mingi periood oli, kus, meile öeldi, et teatud lõhnad on teie 
toas igapäevaselt tunda. Mingi periood oli tõesti, kus meil oli alati 
konjakipudel kapis  ja päeva sai alustatudki pitsi konjakiga. Loomulikult 
mingit joomist ei olnud, aga eks lõhnaks piisas tollest ühest pitsist juba, sul oli ka klaas 
laua peal võib olla kuni lõunani. Ega keegi ei joonud, aga lihtsalt natukene. 
Ja WRC ralli oli ka, mille Toomas vist püsti pani, seda me 
mängisime ka  mingisugune periood. Too tahtis väga palju \emph{network}-i 
aga kuna Toomas oli \emph{network}-i põhjaga vend, siis kui toda 
\emph{bandwidth}-i kellelgi oli, siis meie toas ennekõike.

\question{Kust sul see klassika huvi tuleb?}

Klassika huvi tuli sealt Enrico Caruso-st tegelikult. Mul oli vanematel kodus 
Vittorio Tortorelli raamat\sidenote{Enrico Caruso. Vittorio Tortorelli. Eesti 
Raanat 1968, tõlkija Õ. Karask.}, Tortorelli  on itaallane ja kuna ta oli vist 
Carusole kaugelt sugulane ka, siis raamat 
on ülimalt ülistav, aga ta oli huvitav lugeda ja jättis väga sügava mulje. 
Ja kui ükskord Internetist  oli võimalik tellida, siis ma tellisin huvi pärast sealt 
CDNow-st neid Caruso plaate. Teine asi, mis draivis oli ikkagi 1984. aasta 
Miloš Forman'i Amadeus, mida ma kindlasti soovitan kõigil vaadata, suurepärane 
film. Sealt tuli Mozarti huvi, kuidagi selliselt ta läks,  tellisin 
mingisuguseid raamatuid Mozarti eluloost, mingi neli-viis, mõni on üle tuhande 
lehekülje paks. Sealt edasi on juba  lihtne. Beethoven, Schubert, 
Schumann, Tšaikovski \ldots

\question{Mis sa praegu teed?}

Töötan G4S-is\index{G4S}, turvalisem Eesti. Baasteenuste arendusjuht,  
sisuliselt  vastutan ülalhoiu eest, et kõik asjad oleksid püsti ja valvatud. 
Jah, mitte siis ainult IT vaid ka  tehniline valve, kuhu puutub, siis ka see 
raadiosidevõrk on meil üle Eesti on. Et kõik need signaalid jõuaksid  keskele 
kokku.


\chapter{Jaan Priisalu}
\index[ppl]{Priisalu, Jaan}

\ldots ütles eestlaste kohta väga hästi. Et eestlased otsused saunas 
teevad, eks ole, on \emph{no-brainer}. Aga kuidas seda tehakse? 
Inimesed käivad saunas, on paljad, räägivad midagi. Ja kui ära 
lähevad, kõik nagu teavad, mis otsus on. Seda ei hääldatud mitte ühtegi korda 
välja 
ja ei ole aru saada, kes on liider, kes selle \emph{move}-ga välja tuli. 
Sul on pikka aega olnud mingid võõrad sellid peal, 
kelle eesmärk on mingi muu, kui kohaliku rahva eesmärk. Ja sa tead, et võimu 
peale loota ei saa. Aga kui sa teed otsuseid sellise \emph{mode}-ga, et sa oma 
liidreid välja ei näita, on see tegelikult liidrite kaitsmise süsteem. Mis 
muuhulgas tähendab, et me oleme projektirahvas. Kui sa Vabadussõda vaatad, 
siis see oli ka selgelt projekt.

\ldots Mulle on seda küsimust mitu korda esitatud. Ameeriklased tulevad ja 
küsivad, et kui me ringi vaatame, siis need \emph{challenge}'d, mis te 
välja käite, on nagu pooltel maailma riikidel. Aga miks siin välja tuleb?

\bigskip
\noindent\rule{.3\textwidth}{.7pt}
\bigskip

\question{Kuidas sina sattusid arvutite juurde?}

Arvutite juurde sattusin neljandas või viiendas klassis. Meil oli pioneerisalk, kellega tegime igasuguseid asju, ja mind pandi seda juhtima. Salgas oli ka klassivend Kermo 
Jaaksoo\index[ppl]{Jaaksoo, Kermo}, kes pakkus välja, et võiksime 
minna tema isa töö juurde. Ülo\sidenote{
Kermo isa, akadeemik Ülo Jaaksoo\index[ppl]{Jaaksoo, Ülo}} töötas sel ajal 
Estonia puiesteel ja tal oli seal ES-1010\index{Arvutid!ES 
EVM!ES-1010}\sidenote{1010 oli ES EVMi esimese alaseeria esimene mudel.}. Ega me seal arvutiga midagi muud teha ei 
osanud kui Kuule maandumise mängu mängida. 

Ma käisin 1. keskkoolis\index{Koolid!Tallinna 1. Keskkool}. Kui lastelt küsitakse, kelleks nad saada tahavad, siis tavaliselt 
öeldakse, et tuletõrjujaks, politseinikuks, autojuhiks. Mu vanemad 
väidavad, et kui minu käest seda esimest korda küsiti, siis mina tahtsin saada
inseneriks. Seepeale otsustasid
vanemad, et poiss tuleks panna matemaatikat ja füüsikat õppima. Isa leidis, et 1. keskkool on õige koht, see 
oli elukohajärgne kool ka – me elasime vanglahoovis, Suur-Patarei 29. 
Läksime kooli katsetele, tegin katsed ära ja siis direktor küsis isalt, 
miks nad tahavad mind sinna panna. Isa rääkis inseneriloo ära ja ütles, et poisil on vaja 
matemaatikat ja füüsikat teada. Direktor mõtles natuke ja ütles, et see kõik on ju väga tore ja tõepoolest, poiss tegi katsed 
edukalt, aga meil on see häda, et matemaatika-füüsika klass hakkab üheksandast 
klassist, mis ta seni teeb? Seepeale pandi mind prantsuse keelt õppima.

\question{See on ka ilmselt väga tarviline olnud!}

Prantsuse keel on paarikümne aasta pärast
kõige kõneldum keel maailmas. Kui mõelda, kui levinud see on Aafrikas, siis on sellel seal sama funktsioon, mis Indias inglise keelel. Ja 
kui 
vaadata, missugune rahvastikuplahvatus neil on ja palju neil maad käes on, siis 
Aafrikal ei ole India inimeste tiheduse probleemi, vaid paisumisruumi on
palju.

\question{Kas pioneerirühmas kohtusite esimest korda essukesega\sidenote{ES EVM seeria masinate levinud hellitusnimi.}, maandusite 
Kuule ja mis veel?}

Põnev oli vaadata, kuidas lindid käivad ringi. Sel ajal olid juba 
vahetatavad kõvakettad. Nad näitasid meile ka trummelsalvestit, kuigi see 
ei olnud käigus, vaid oli lahti ühendatud.

\question{Mis asutus see oli Estonia puiesteel?}

Üks Teaduste Akadeemia asutus, ma täpselt ei mäleta. Ilmselt 
praegune Küber\index{Küber}, sest Ülo oli kunagi Küberi 
direktor.

\question{Kas tollest ajast jäidki seal oma rühmaga käima?}

Ei, mitte päris. Järgmine kord oli siis, kui meil olid arvutiõppe tunnid 
ja õpetaja Loonde\index[ppl]{Loonde, Jaak} viis meid Pedasse\index{Tallinna Pedagoogikaülikool}. 
Teine arvuti mu elus oli sealne Nairi-2\index{Arvutid!Nairi!Nairi-2}. Nairi on 
transistorite peal arvuti, perfolint on viierealine. Kui 
matemaatika-füüsika klassi läksime, siis Jako Bergson\index[ppl]{Bergson, Jako} tõi 
Kirovi kalurikolhoosist\index{Kirovi Kalurikolhoos} MIR-2\index{Arvutid!MIR-2} 
ära. Mir-2 on tegelikult maailma esimene personaalarvuti, mis oli tehtud inimese aitamiseks. Selle protsessor kaalus pool 
tonni, aga ideoloogia oli selles, et inimene saaks 
oma rehkendused tehtud. Muu hulgas olid integreerimine ja diferentseerimine 
rauas\sidenote{St. riistvaraliselt.} realiseeritud.

\question{Sest inimesel oli ju vaja integreerida ja diferentseerida, mille 
jaoks talle muidu üldse arvuti!}

See oli Ukrainas tehtud arvuti, seal arvutati gaasiturbiine, 
rakette ja muud säärast. Programmil oli 
Algoliga sarnane keel ja see algas käsuga \verb|RAZR|, mis tähendas 
\begin{russian}разрядность\end{russian} ehk kui pikad arvud on. 

\question{Kui pikaks arve sai keerata?}

Me keerasime kas 300 või 400 peale, kümnendkohtades. Panime programmi piid arvutama ja arve ritta ajama ning see lõppes sellega, et 
kuigi oli talveaeg ja me tegime aknad lahti, kuumenes arvuti ikkagi üle. 
Pooleteist tundi saigi sellega tegutseda.

\question{Kust teil selline mõte, et võiks piid arvutada?}

See lihtsalt tundus lahe.

\question{Ja kust te matemaatika saite?}

Mul oli üks paks venekeelne matemaatika õpik või 
entsüklopeedia. Seal olid igasugused read ja pii rida oli 
üks nendest.

\question{Järelikult oli teil koolis tolleks hetkeks matemaatika juba pihta 
hakanud?}

Jah. Meie koolis oli nii, et kui käisid 
matemaatika-füüsika eriklassis, siis esimese asjana jagas õpetaja 
Uudelepp\index[ppl]{Uudelepp, Helgi}\sidenote{Gustav Adolfi 
Gümnaasiumi legendaarne matemaatikaõpetaja Helgi Uudelepp.} klassi pooleks. Pool klassi hakkas õppima tavalise 
keskkooliprogrammi järgi ja ülejäänud kambale anti olümpiaadi ülesandeid. 
Kamp oli väga kõva (näiteks Mati 
Pentus näiteks\index[ppl]{Pentus, Mati}\sidenote{Mati Pentus on Eesti 
matemaatik, alates 2003. aastast Moskva Riikliku Ülikooli professor.}), probleem oli koolist üldse välja jõuda. Rajoonist sai 
niikuinii vabariiklikule edasi.\sidenote{Toonased olümpiaadid olid organiseeritud kooli-, rajooni- ja 
vabariiklikeks olümpiaadideks. Vabariiklikult olümpiaadilt oli võimalik pääseda ka 
üleliidulisele olümpiaadile.}. 

\question{Kas teil oli koolis arvutitund ka ja kasutasite MIR-2?}

Jah. See asus küll kooli spordihoones. 

Selge see, et 1. keskkool seisis direktor Viikholmi\sidenote{Helmi 
Viikholm\index[ppl]{Viikholm, Helmi} oli kooli direktor aastatel 1962–1982.} najal ja kui 
Viikholm läks pensionile, juhtus nagu ikka organisatsioonidega juhtub, et 
hakatakse rootsi keelt või muud sellist õpetama.

\question{Kas selleks ajaks olid sina sealt koolist juba läinud?}

Ma käisin siis veel koolis, kui Viikholm ära läks. Organisatsioon toimis vana energiaga 
veel natuke aega edasi, tavaliselt võtab lagunemine 
kaks-kolm aastat aega.

\question{Kas keskkooli ajal pusisite Jaaguga MIRi peal või oli juba muid 
võimalusi ka?}

Mina sain oma esimese palga progemise eest aastal 1984, 
üks suvi enne keskkooli lõpetamist. Paldiski maantee 1 asus termo- ja 
elektrofüüsika instituut\sidenote{Eesti NSV Teaduste 
Akadeemia Termofüüsika ja Elektrofüüsika Instituut 
(TEFI).\index{Teaduste Akadeemia!Termofüüsika ja Elektrofüüsika Instituut}}, nende käes oli näiteks Arnold Veimeri nimeline laev.
Mina olin akadeemik Krummi\sidenote{Akadeemik Lembit Krumm\index[ppl]{Krumm, Lembit} (1928–2016).} juures, 
kes arvutas elektrivõrkude staatilisi režiime. Neil olid ka 
arvutid ja mina tegin oma esimese töö arvutil 
Iskra-226\index{Arvutid!Iskra!Iskra-226}, mis on sisu poolest Wang 2200 koopia ja millel on muu hulgas videoprotsessor
8080. Üks vend tegeles sellega, et 
pani selle videoprotsessori peale käima CP/Mi. 

\question{Kust nad su leidsid?}

Ma ei mäleta, aga ilmselt tutvuste 
kaudu või siis lihtsalt läksin ise sinna.

\question{Ja seal arvutati elektrivõrke?}

Esimese asjana pidin tegema rehkenduse ühe
ministeeriumi aruandluse jaoks – tabelarvutuse Basicus\index{Keeled!BASIC}. Tegin programmi valmis ja nägin esimest korda 
päris \emph{user}'i probleemi ka. Korraldasime piduliku üleandmise, 
komisjon tuli 
kokku ja naisterahvale, kes pidi seda 
programmi kasutama hakkama, öeldi, et istu arvuti taha ja proovi 
midagi toksida. Naine keeldus. Keegi ei saanud aru, 
mis juhtus – kas programm ei meeldi või milles asi. \enquote{Ei, ma kardan 
elektrit!} vastas tema. Ma ei mäletagi, kuidas see tsirkus lõppes. 
Siis tuli juba järgmine töö. Neil oli programm, millega nad suuri jakobiaane 
arvutasid, ja see käis essukese\index{Arvutid!ES EVM} peal, mis oli vist 1055. Neid 
oli Küberis kaks tükki: üks oli 360 ja teine 370 koopia\sidenote{ES EVMi 
esimene alaseeria oli IBM System/360 ning teine ja kolmas System/370 koopiad. 
Mudel 1055 kuulus teise ja 1066 kolmandasse seeriasse.}. Nende terminal ei olnud 
mitte VT100 nagu mujal maailmas, vaid IBMi VT52, mis näeb üsna 
hüperteksti moodi välja. Kirjeldad ära sisendi ja väljundi väljad ning terve 
ekraan saadetakse korraga, töötab nagu veebileht. Sinna peale 
ma tegin ühe programmi, mis võimaldas sisendandmeid mõistlikult sisestada ja tulemusi vaadata. Kuna essuke oli \emph{patch}-arvuti, siis pidi 
vahele tegema programmi, mis \emph{patch}'i tulemused 
sisse söödaks või välja võtaks ning 
terminaliga suhtleks. Tolle vahejupi tegi vist Tarmo Mere\index[ppl]{Mere, 
Tarmo}. See kõik oli üsna aeglane, ringitõstmine võttis
aega ja käis läbi ketta.

Teine Küberi essuke oli 1066. Selle peal nad andsid mulle ühe
assembleri makro, et paneksin asja käima, aga protsessor oli juba 32bitine. Olin Inteli assemblerit vaadanud, aga no üldse ei olnud 
sarnane. Kõiki registreid sai kõikides funktsioonides kasutada ja kõige 
hullem, millest ma ei saanud aru, oli see, et kõiki aadresse võeti 
baasregistri suhtes. Sealjuures eeldati, et sa lihtsalt 
tead seda. Aga kuidas teha kindlaks, milline on baasregister ja kust see 
algab? Leidsin baasregistri valimise käsu üles, kuid 
midagi oli ikka puudu. Tuli välja, et seal, kust baasregister hakkas aadresse 
lugema, oli vaja lihtsalt kõikide käsuridade ette panna üks 
tärn.

\question{Programmeerimisoskus oli sul järelikult olemas. Kas õppisid seda Jaagult, 
pusisid ise või kust see tuli?}

Jaak\index[ppl]{Loonde, Jaak} õpetas jah, MIR-2 peal me mingit nalja 
tegime. Kui juba piid arvutad, siis peaksid ka paar rida oskama kirjutada. Järgmisena tuli Basic TEFIs. VT52 puhul ma ei mäleta, 
milles ma programmi tegin, võibolla oli Fortran. 

\question{Kuidas VT52 puhul progemine käis, kui ekraanil olid lihtsalt 
mingisugused väljad?}

Ei mäleta, interaktsiooni kirjeldus on selline, et saadad 
terve 
andmepaketi korraga minema ja saad terve andmepaketi korraga tagasi. Andmete 
töötlus ja esitus on eraldatud. 

\question{Naljakaid aparaate on olemas!}

Mäletan, kuidas vennad presenteerisid modemit, millega me Küberisse 
helistasime. 1200boodine modem oli külmkapisuurune. Kui tuli 2400boodine modem, siis see oli tükk maad väiksem, pool külmkappi.

Ja millega vennikesed veel tegelesid?! 1984. aastal olid olümpiamängud LAs. Mängude 
ajal nad jälgisid elektrivõrgu parameetreid, peamiselt sagedust, ja selle põhjal 
ütlesid, palju tootmist seisab ja mitu inimest vaatab olümpiamänge. 

\question{Kas nad ütlesid seda ka ametlikult kellelegi?}

Ei, nad vaatasid oma lõbuks. Neil oli kihlveokontor – vedasid kihla, palju järgmisel päeval 
vaatajaid on.

\question{Sa olid matemaatikas ja füüsikas tugev, aga arvutiasja 
pidid suuresti ise pusima. Mis sind selle juures tõmbas?}

Äge on see, kui saab oma kätega midagi teha. Üks liik inimesi armastab teooriaid 
välja mõelda, teised asju kokku 
ja käima panna. Täna ma olen 
mõtlemise ja teooria poole peal.

\question{Eks asjad ole maailmas tasakaalus. Nii et tol ajal meeldis sulle 
vajutada arvutiklahve ja arvuti muudkui tegi?}

Automatiseerimine – see, et asjad ise juhtusid – oli väga äge. Näiteks et
auto sõidab ise. Sel ajal oli natukenegi targem 
juhtimisalgoritm haruldus.

\question{Ja see hoidiski sind nii palju arvuti taga, et õppisid programmeerima 
ja õigetesse kohtadesse tärne panema?}

Jah.

\question{Kas selle juurde käis ka mõni laiem valdkondlik huvi? Mõni on 
rääkinud, et raamatud, muusika ja muu selline suunas arvutite poole.}

See oli sügav Vene aeg, mis mõttes \enquote{raamatud ja muusika}? 
Loomulikult lugesin ma Asimovit, robotivärgist räägiti seal päris palju. Lugesin kõiki 
raamatuid, mida kätte sain. Neid, mida ei saanud, lugesin 
juba Prantsusmaal, kui läksin Toulouse'i õppima. Ma panin 
kooliminekuga natuke puusse – kui ma poole septembri pealt kohale läksin, 
polnud koolis veel kedagi. Nii ma siis 
istusin raamatukogus ja lugesin matemaatikat ning Asimovi jutte, et keeleoskust parandada.

\question{Enne kui Toulouse'i juurde jõuame, küsin, kas 
töölkäimine keskkoolis õppimist ei seganud?}

Ei seganud, see oli eluviisi osa, ma olen alati kooli kõrvalt tööl 
käinud.

\question{Mida sa ülikooli õppima läksid?}

Automaatikat, automaatjuhtimissüsteeme tehnikaülikoolis\index{Tallinna 
Tehnikaülikool!Automaatikateaduskond}.

\question{Kuidas see otsus sündis? Oli see loomulik valik?}

Oli küll loomulik valik. Kuulsin oma sugulaselt Jaan 
Võrgult\index[ppl]{Võrk, Jaan}, mis 
see automaatika üldse on. Meie rühm oli väga äge, klassivendi ja -õdesid
oli umbes kümme. 

Gibbs\index[ppl]{Kübbar, Heiki} ütles mulle hiljem, et seda oli 
vastik vaadata – ise higistad matemaatika- ja füüsikaloengutes, aga 
need vennad tulevad kuskilt, teevad pulli, lähevad eksamile ja saavad 
kõik viied. 

\question{Kas ta oli su kursavend?}

Me oleme kindlasti koos loengus käinud. Ta on aasta noorem, aga ilmselt läksid meil loengud minu sõjaväest tagasitulekuga kuidagi sünki – ma istusin oma kaks aastat sõjaväes ära.

\question{Kas sind võeti enne ülikooli kroonusse?}

Ülikooli esimeselt kursuselt. Arvasin, et ma ei pea minema, aga tolle aasta viimane võtmine oli detsembris ning mind pandi rongi peale ja läksin.

\question{Kus sa need kaks aastat veetsid?}

Põhikoht oli Rostov Doni ääres sisevägedes ehk siis vangivalvurid. 
Alguses olin mingisuguses isolaatoris, kus oli kolm varianti, mida saab teha 
veest ja hapukapsastest. Esimene oli hapukapsasupp ehk vesi hapukapsastega. 
Teine oli praad ehk hapukapsad ilma veeta. Kolmas oli kissell ehk 
vesi ilma hapukapsasteta. Ma vaatasin, et suren sinna ära, kui pean seal väga pikalt 
olema, ja munsterdasin ennast \emph{utšebka}'sse\sidenote{Õppeväeosa.}, 
natuke luuletasin 
ka. Seepeale saadeti mind Galatši valveseadmete inseneriks õppima. Galatš on see koht, kus Stalingradi kott kokku murti. Nii et ma olen
Volgogradi Venemaa Ema Mamajevi kurgaani peal päriselt lähedalt näinud. 
Hirmus roostes kolakas oli – kaugelt vaadates ilus, aga lähedale minnes roostes.

\question{Mina küll Vene kroonusse napilt ei jõudnud aga, nohik nagu ma olin, 
kartsin, kuidas ma seal füüsilise koormuse ja keelega hakkama saaksin. Kas sul seda 
hirmu ei olnud?}

See, et üritad ellu jääda, oli igal juhul. Ja selge see, 
et vene keelt koolis ära ei õppinud. Seal aga ei olnud valikut.

\question{Klassikalise haridusega vene proua ju kolme- ja neljatähelisi sõnu
ei õpeta.}

Jah, need on sõnad, mille abil õpid tõepoolest kõiki asju ära ütlema. Seal 
tehti naftast viina. Läksin kord brigaadi töökotta, et juhendada 
järgmist venda, üht Leedu poolakat, kes teadis elektroonikast 
tegelikult rohkem kui mina. Prapporid\sidenote{Praportšik ehk lepinguline 
allohvitser Nõukogude armees.} tulid oma viinapudeliga sinna ja tahtsid, et me selle 
treipingis ära tsentrifuugiksime. Panin pudelile rätiku ümber, treipinki ja 
pöörded peale. Seesama major, kes mind \emph{utšebka}'sse vajas, vaatas kõrvalt 
ja ütles: \enquote{\begin{russian}ты уважай русский язык, ты хот \ldots\ 
скажи!\end{russian}}. See oli päris hull keel, kindlasti mitte tavaline.

\question{Mõned inimesed on rääkinud, et neil oli kolmetäheliste maailmast
keeruline tagasi teadusmaailma tulla. Kas sul seda probleemi ei olnud?}

Kindlasti oli. Sõjaväest tagasi 
tulnud loobiti kõik eraldi kursusele, neid ei lastud puutumatute 
inimestega kokkugi.

\question{Kas sa läksid Tehnikaülikooli tagasi sama asja õppima?}

Jah, sain isegi sama töökoha tagasi, aga siis ühel hetkel läksin
EKTAsse\index{EKTA}\sidenote{Arvutustehnika Erikonstrueerimisbüroo oli 
Eesti NSV Teaduste Akadeemia Küberneetika Instituudi autonoomne osakond}. 
Ektaco\index{Ektaco} on EKTA \emph{spin-off} ja EKTA direktor oli 
Märtin\sidenote{Kaarel Märtin\index[ppl]{Märtin, Kaarel} oli siiski EKTA 
tarkvaraosakonna pealik, tema alluvuses Jaan ilmselt töötaski. EKTA direktor 
oli Kalju Leppik\index[ppl]{Leppik, Kalju}, Ektaco oma Rein 
Haavel\index[ppl]{Haavel, Rein}.}. Ma hakkasin seal FoxPros andmebaase kirjutama.

Üks huvitav kogemus oli käia putši ajal Moskvas. Tegime sealsele 
juveelitehasele 
väärismetallide arvestusprogrammi, aga sel ajal pidi softile autor kaasa 
minema, sest need asjad ei olnud väga töökindlad. Tehase 
osakonna juhataja oli tiba juudi verd. Kui ta kuulis, et 
erakorraline komitee on võimu üle võtnud, ütles ta mulle kohe, et see kõlab 
halvasti, \enquote{vedur teise otsa ja kohe koju tagasi!}. Mina aga olen eluaeg lollustega 
maha saanud või hirmus otse öelnud. Arvasin vastu: 
\enquote{Mis sa jamad, vaata kui hästi teevad
Levitani\sidenote{Juri Borissovitš Levitan oli 
Nõukogude diktor, kelle kanda oli peamiste oluliste uudiste edastamine Teise 
maailmasõja ajal, tema iseloomulikku häält tunti hästi.} järele, 
nagu oleks sõjaajast pärit.} Läksime tänavatele ja need olidki 
BTRe\sidenote{Nõukogude Liidus valmistatud soomustransportöör.} täis ning madin käis. Seal olid 
suured seitsmerealised tänavad, mis olid kõik autodest tühjad. 
Inimesed korjasid sillutisekive ja ehitasid nendest barrikaade. Vennikesed 
hüüdsid mulle veel uhkelt, et vaata, kui kõvad mehed me oleme, ehitame nii kõrgeid barrikaade. Barrikaadid olid aga põlvekõrgused.
Ütlesin neile, et tank T-72 tehnilises 
spetsifikatsioonis on kirjas, et see sõidab 70 kilomeetrit tunnis, kui maapinna 
ebatasasus ei ületa meetrit. 
Juveelimessil rääkis üks korralik proua, kuidas see kõik on nii 
kohutav, ja küsis, mis mina sellest 
arvan. Ma ütlesin, et kõik on ju hästi. Proua imestas: \enquote{Mis mõttes hästi?} 
Ma siis seletasin: \enquote{Vaadake, seni on venelased tapnud kõiki teisi rahvaid, nüüd 
tapavad venelased venelasi.} Populaarsust ma sellega muidugi ei võitnud.

\question{Kas sul igav ei olnud andmebaase treida, tulles matemaatiliselt keeruliste 
asjade juurest? See on ju rutiinne töö.}

Ei olnud, seal oli tegelikult sisendit ja väljundit palju ning pusimist piisavalt, et erinevatele 
inimestele vaated teha. Ja mul olid väga lahedad 
töökaaslased Jüri Freiberg\index[ppl]{Freiberg, Jüri} ja Ülle 
Heinla\index[ppl]{Heinla, Ülle} – Ahti\index[ppl]{Heinla, Ahti} ema, kes näitas uhkusega poja
tehtud mängu. 

\question{Kaua sa neid andmebaase tegid?}

Ma ei mäleta. Kui ma läksin Ektacosse\index{Ektaco}, tegin seal 
baase edasi. Olin siis just Prantsusmaalt tagasi tulnud ning tegin juba 
niisuguseid baase, kus olid füüsilised asjad ka taga, nagu lukud ja kassad.

\question{Kuidas sa Prantsusmaale sattusid?}

Nõukogude Liidule oli eraldatud 300 stipendiumit ja kui liit 
lagunes, 
siis kuus stipendiumit tuli Eestisse. Kuna sel ajal oli prantsuse keele 
oskajaid suhteliselt vähe, korraldati 
avalik konkurss. TPIst korjati ka inimesi ja sealne prantsuse keele õpetaja 
pani mu naise (kes oli ka 1. keskkoolist) kirja. Aga dekanaat tõmbas ta 
maha, et naine on rase ja kuidas ta sinna läheb. Naine oli suhteliselt kõva 
iseloomuga, et mis see dekanaadi asi on, kas ta on rase või mitte. Otsustasime
saatkonda minna, aga tee peal jäi tal samm järsku 
aeglasemaks ja ta ütles, et kuule, ma olen tõesti rase, mine sina.

Saatkond teatas, et stipendiumi saamiseks 
peab paar tingimust täitma. Esiteks, võimalikult kõrgel õppima. Kuna mul oli 
neli aastat ülikooli seljataga, siis soovitati kohe magistrisse minna. Teiseks soovitati mitte 
Pariisi minna. Kuna Leo Mõtusel\index[ppl]{Mõtus, Leo} oli Toulouse'is 
üks tehisintellektiga tegelev tuttavtegelane, siis läksin 
sinna.

\question{Üheksakümnendate algus oli huvitav aeg tehisintellektiga tegelda, 
see oli ju enne riistvara läbimurret.}

Selle aja peale oli juba igasuguseid asju tehtud: 
produktsioonisüsteemid\sidenote{Produktsioonisüsteem on tüüpiliselt tehisintellekti pakkumiseks rakendatud arvutiprogramm, 
mis koosneb formaalsetest reeglitest, mehhanismidest nende reeglite järgimiseks 
ning süsteemi olekut säilitavast andmebaasist.}, esimesed teoreemitõestajad, otsustuspuud ja muud asjad. Ma ei mäleta, millal 
Rete algoritm\sidenote{Charles L. Forgy poolt 1974. aastal maailmale tutvustatud 
algoritm efektiivseks formaalsete reeglite rakendamiseks.},
produktsioonisüsteemide indeks tehti. Tolleks ajaks oli 
selliseid põhialuseid laotud juba päris palju. Masinõpet vist väga 
ei tehtud ega osatud, nii palju jõudu käes ei olnud.

\question{Kui kaua sa olid Prantsusmaal?}

Aasta. 

\question{Kas sealt hakkas su võrguhuvi tekkima?}

Jah, see oli esimene koht, kus ma internetti nägin. Naine oli veel Tallinnas, 
tema käis Küberis\index{Küber} interneti küljes. Oli niisugune programm nagu talk: 
ühel pool Unixi masinas kirjutad sina ja teisel pool teine. Kuna sel ajal pidi
kaugekõnesid tellima ja see oli keeruline protseduur, siis
talk võimaldas paremini suhelda.

\question{Kas sul sellist mõtet ei olnud, et hakkaks teadust tegema?}

Oli. Aga naine käis mul Prantsusmaal külas ja siis sündis meil teine 
laps ka ning tulin koju tagasi.

\question{Eestis saab ju ka teadust teha.}

Sel ajal ei saanud. Oli üheksakümnendate algus ja lihtsalt raha ei olnud. Pere jaoks oli vaja raha teenida ja kuidagi korter saada. 
Korterihinnad olid naeruväärselt madalad. Sain 
Prantsusmaal kõrget magistristippi ja pool sellest hoidsin kokku 
ning ostsime korteri.

\question{Mida sa Prantsusmaalt tagasi tulles tegema hakkasid? 
Kas programmeerisid jätkuvalt?}

Jah, ikka. Läksin Ektacosse\index{Ektaco} ja tegin lukkude juhtimist. 
Muu hulgas tuli seda teha ühes pullis kohas, Viimsi 
Talveaias\sidenote{Viimsi Talveaed asub Pringi külas ja valmis 1973. aastal 
Kirovi-nimelisele näidiskalurikolhoosile. See kolhoos (ja 
sealsed kolhoosnikud) olid tolle aja mõistes põhjatult rikkad ning Talveaiast 
kujunes Tallinna 
peenema rahva peokoht. Hulludel üheksakümnendatel oli tegu populaarseima 
paigaga, kus 
kiiresti ja kõikvõimalikel viisidel rikastunud inimesed käisid oma rikkust 
demonstreerimas.}. Seal garderoobis oli püramiid, kuhu korjati numbri vastu 
relvad ära. Avamispeo ajal oli lukkudega mingi jama, need ei töötanud. Läksin sinna ja saunapõrandal oli kiht paljaid purjus 
naisi. Väga imelik koht.
 
\question{Mis lukuprogrammeerimises huvitavat oli?}
 
Seal on pusimist, et kõik asjad paika saada. Mul oli ka see 
probleem, et kui Prantsusmaal õpitu kokku võtta, siis oli põhimõtteliselt tegu diskreetse matemaatikaga. Õppisin, kuidas kompilaatoreid tehakse, 
kategooriate teooriat, eri liiki semantikat (loogiline, denotatsiooniline ja 
operatsiooniline). Aga kus seda vaja 
läheb? Tuled tagasi ja raha saad ikka selle eest, kui kellelgi mõne päris 
probleemi lahendad. See lukuprojekt läks hulluks kätte. Kõigepealt 
pidime tegema kassasüsteemi, mille külge tulid lukud, ja nii see pintsaku 
nööbi ümber õmblemine käis. Tellija tahtis hästi palju muutusi, aga ma suutsin andmemudeli kohe niimoodi paika panna, et ei pidanud seda 
pärast enam muutma, ainult juurde tuli panna. Seepeale sain järsku 
aru, et olen midagi õppinud ka.

\question{See vajab päris head rakendusvõimet, et nii abstraktset 
teemat kohe baasi mudelis kasutada. Kategooriate teooria ju ei ütle sulle, 
millised tabelid olema peavad.}

Jah ja ei. Kategooriate teooria õpetab seda, kuidas maailmas 
asjad on korrastatud. Matemaatika point on selles, et see korrastab 
mõtlemist.

\question{Üheksakümnendate lõpus tegelesid juba infoturbe ja -riskidega, 
kuidas sa lukkude juurest selleni jõudsid?}

Mul hakkas igav. Enn Lakspere\index[ppl]{Lakspere, Enn}
läks Küberisse\index{Küber} tööle. Monika\sidenote{Monika 
Oit\index[ppl]{Oit, Monika}} ja Ülo Jaaksoo\index[ppl]{Jaaksoo, Ülo} olid 
teinud turvaseltskonna enne, kui Eesti Vabariigi iseseisvus paistma 
hakkas, sest nad arvasid, et see on strateegiline oskus, mida on igal riigil 
vaja. Ja neil on selles suhtes õigus.

\question{Kas neil oli selline visioon juba tol ajal?}

Jah, neil oli enne iseseisvumist visioon olemas, et iseseisvus 
ühel hetkel tuleb ja selleks ajaks peab kompetentsi olema. Riigi 
infoturve on riigi jaoks strateegiline asi ja see tuleb korda saada.

\question{Mõnes kohas ei ole sellest siiamaani aru saadud!}

Ilmselt ei saadagi. Need olid kindlasti väga suure visiooniga inimesed. 

\question{Ja sa läksid nende juurde tööle?}

Enn Lakspere viis mind sinna. Kokkulepe oli, et mina teen uurimusi ja tema otsib 
tööd. Minu eriala olid kiipkaardid ja teda kaardid huvitasid, kuna ta tuli 
Ektacost, kus kassasüsteemide külge käisid ka kaardid. Mina pidin kiipkaarte 
uurima. Kirjutasingi raha eest uurimusi, kolm lehekülge puhast teksti 
päevas, seda on päris palju.

Vello Hanson\index[ppl]{Hanson, Vello} õpetas mind kirjutama. Osa tema 
õpetustest olen tänaseks küll ära unustanud, aga Vello Hanson on tõsiselt kõva 
vend. 

Kirjutasin näiteks Pankade Kaardikeskuse\index{Pankade 
Kaardikeskus} arhitektuuri. Keskpank tahtis sellele asjale litsentsi anda ja 
menetleda. Aga ma kirjutasin sinna ühe asja, mis oli 
Sildmäe\index[ppl]{Sildmäe, Tõnis} jaoks uudis. Ütlesin, et ärge \emph{settlement}'i ja 
raha liigutamist üldse sinna 
keskusse pange. Võtke lihtsalt info, kes kellele kui palju võlgu on, ja tehke 
bilateraalne \emph{settlement}. Ta käis üle küsimas, 
kas nii saab. Nad hoidsid sedasi paar aastat 
puhast regulatsiooni kokku. Grupivend Margus Aun\index[ppl]{Aun, 
Margus} 
läks seda värki juhendama. 

Ühel hetkel küsis Ülo Jaaksoo minult, kas
kaartidega mässamisest ühiskonnale ka midagi kasulikku teha saab. Kõrval 
oli Ahto Buldas\index[ppl]{Buldas, Ahto}, kes rääkis mulle asümmeetrilisest 
krüptost ja et sellega saab digiallkirja teha. Lugesin selle kohta veel 
kuskilt juurde ja ühel siseseminaril 1994. aastal pakkusin välja, et 
kiipkaardid võiks inimestele kätte anda ja nendega digiallkirja teha. 
Esimene avalik esinemine sel teemal oli Küberis 1995. aastal. Lõpuks müüs
Tarvi\index[ppl]{Martens, Tarvi} selle idee
riigile maha ja nii see läkski. Tarvi oli tegelikult selle asja juurutaja 
ja innovaator.

\question{Kust tuleb legend, et tegu on soomlaste tehnoloogiaga? Või on tehnoloogia
soomlastelt ja idee teilt?}

See ei ole tõsi, soomlaste tehnoloogia ei ole ka originaalne. Kui 
digiallkirja seadust hakati tegema, tellis Tarvi minu käest profiili, missugune 
see kaart peaks olema. Vaatasin ringi, mis kuskil tehtud on, ja rootslastel oli 
kaardi profiil kirjeldatud, nad tegid kolm võtmepaari. Soomlased kopeerisid 
rootslasi ja panid kaks võtmepaari kokku. Eri võtmepaare on vaja 
seetõttu, et neil on täiesti erinev poliitika. Autentimise ja 
krüpteerimise võtmeid ei tohiks kokku panna, sest krüpteerimise võtmel peaks 
olema taaste, kui tahad seda pikaajaliseks säilitamiseks kasutada. Allkirja 
võtmel aga ei tohi olla taastet. 
Autentimisvõtme jaoks ei ole mingit põhjust taastet tahta. Need on erinevad 
poliitikad, mis tegelikult ei sobi hästi kokku, aga nii on lihtsam inimesi 
õpetada. Turbe põhimõte on see, et ainult lihtsad asjad töötavad. 
Jaapanlased võib-olla saavad keerulise asjaga ka hakkama, aga meie ei saa. Ja 
kuna ükski inimene krüpteerida ei oska, siis talle tuleb anda arvuti, mis teeb
seda tema eest. Kiipkaardist lihtsamat arvutit ei ole olemas.

\question{Ja nii jõudsidki infoturbeni?}

Jah.

\question{Infoturve tundub olevat sinu juttu kuulates ideaalne kombinatsioon: 
sai asju ära teha, oli palju matemaatikat ja arvuteid, kõik kenasti koos.}

Küber oli üsna selge teadusasutus: väga palju 
tehti teooriat ja natuke kirjutati ka programmi. Arne\sidenote{Arne 
Ansper\index[ppl]{Ansper, Arne}} oli juba sel ajal kodeerimises kibe käsi.

Ma läksin sealt ära sellepärast, et tekkis tunne, et kirjutan 
igasuguseid plaane ja siis teised mehed plaanide põhjal ehitavad. 
Ühispank\index{Pangad!Ühispank} oli viimane pank, kellel ei olnud oma 
kaardiserverit, nii et ma läksin seda 
tegema. Ja kuna turvainimesi ka ei olnud, siis pidin olema turbe ja 
maksekaartide peal. Edasi läksin Hansasse\index{Pangad!Hansapank}. 

Tore oli pangasüsteemi ringitõstmine, kui tuli ühendada kõik 
maapangad\sidenote{Ühispanga asutasid 15. detsembril 1992 kaheksa 
maapanka, Viljandi kommertspank ja Nordpank}, ning selleks oli vaja süsteemi. 
IT-direktor oli 
Novelli-mees. Ma rehkendasin talle \emph{roundtrip} aegadega, mitu 
transaktsiooni ta tänu lukustamistele jõuab üldse päevas teha, see oli kas 30 
000 või 40 000. See tähendas, et tal Tallinnas oleva 
panga jaoks jätkus, aga terve Eesti peale oli vähe. Siis sai Unix sinna alla 
valitud, et lukustamine käiks ühes masinas ära. Tema valis millegipärast HP, aga
tegelikult oli HP-UX\index{OS!HP-UX} väga äge Unix. Inimesed arvavad, mis 
nad arvavad, aga väga robustne riist. Solaris, millest tavaliselt 
räägitakse, oli tükk maad hellem. 

Toona oli Windowsiga selline õnnetus, et TCP \emph{stack}'i eest 
pidi eraldi raha maksma, Linuxi käimapanek oli kaks korda odavam. Seepärast ühendatigi
kõik maapangad niimoodi ära, et pandi Linux \emph{front}'i ja 
ühe ööga keerati kontor ringi, süsteemi vahetus ja \emph{front}'i vahetus. 

\question{Mida sa praegu teed?}

Praegu uurin kriitilisi sõltuvusi, see teema on pärit 
RIA\index{Riigi Infosüsteemi Amet}\sidenote{Jaan oli aastatel 2011–2015 Riigi 
Infosüsteemi Ameti peadirektor} ajast. 
Kui pead vastutama niisuguse asja eest nagu massiivne küberrünnak, siis 
selle juhtimiseks lükkad kokku staabi. Ja esimene küsimus on muidugi, kas see, mida sa näed, on õige. Niisugust infosüsteemist 
sisse tulevat müra, kus pead süsteemi enda käitumist rünnakuks, on 
päris palju. Teine küsimus on, mis edasi 
juhtub ja kust rünnak peale hakkas. Tavaliselt ei oska inimesed kummalegi 
vastata. Minu algne mõte oli, et äkki nendel sõltuvustel on mingi 
võrestruktuur, võre moodi osaline järjestus. Kui leiad
selles osalises järjestuses miinimumi, siis võib see olla algpõhjus. Ja kui võtad transitiivse 
sulundi, saad kõik tuleviku asjad kätte. Nii tekkis mõte hakata pakkuma 
planeerimisabi. Selleks aga on vaja 
need sõltuvused kuidagi kirja panna. Nüüd olen saanud nii palju targemaks, 
et mu arvamus, et seal ei ole tsükleid, ei ole õige. Tsüklid on ja neid on väga 
palju ning väga lühikesi. Kogu majandus on tegelikult tsükleid ja tagasisidet 
täis ning ma ise olen dünaamilisi süsteeme õppinud ja näinud, mida sellised asjad teevad. Ühesõnaga, nüüd üritan neist asjust aru saada.

\question{Kõlab, nagu oleksid asjad jätkuvalt huvitavad ja see on üks väheseid 
olulisi asju. Või eelistad sa igavaid?}

Ei, seda kindlasti mitte, aga mõnikord võivad need liiga huvitavaks minna. 
Keerukus kipub kasvama ja seda peab jõudma
jõuga maha võtta – \emph{refactoring} on alati töö.


\chapter{Tanel Raja}
\index[ppl]{Raja, Tanel}
\label{sisu:pronto}

\question{Miks sind Prontoks kutsutakse?}

Jäi lihtsalt külge, ma täpselt ei mäleta, mis asjaoludel. Tol 
ajal pidi olema igaühel oma hüüdnimi ja minu nimi oli lõpuks see.

\question{Kuidas sa arvutite juurde sattusid?}

Arvutite juurde sattusin ma enne, kui minust sai Pronto.

Kas sa oled lugenud sellist raamatut nagu \enquote{Professor Lillepooli 
kroonika}\sidenote{Herta Laipaiga ulmelugu, mis ilmus kirjastuse Eesti Raamat väljaandes 1982. 
aastal. Raamatu peategelased kohtuvad muu hulgas arvutiga nimega Kunigunde.}? 
Seal toimus kohutavalt põnev tegevus: tüübid tegid oma Musta Kassi 
ordu ja selle käigus käisid vist TPIs ning tegid mingi skeemi 
valmis. See on tagantjärele mõeldes naeruväärne, aga väikse poisina tundus 
tohutult põnev. Sealt tekkiski arvutihuvi. 

Edasi oli kaks liini. Mu onu oli IT valdkonnas tegev 
tükk maad varem kui mina, ta oli juba sügaval 
nõukaajal arvutite kallal. Käisin Tartus tema juures ja see kõik oli kohutavalt põnev.

\question{Mis selle põnevaks tegi?}

Põnevad olid igasugused nupud ja see, et asjad toimusid. 
See oli väikese poisi jaoks unistus, et mingit masinat saab täielikult kontrollida ja et jõud võiks sellest üle käia.

Teine liin oli Tallinnas, Jaak 
Loonde\index[ppl]{Loonde, Jaak} Luise tänava\index{Tallinna 
Oktoobrirajooni Õppetootmiskombinaat} klass, kus olid Yamaha 
MSXid\index{Yamaha MSX}.\sidenote{Seal asus Tallinna Oktoobrirajooni Õppetootmiskombinaat, mida on vahel paigutatud ka Roopa tänavale ja mille kohta öeldakse ka lihtsalt Luise tänava klass.} Sama tüüpi aparaat, mis mul siin külje all 
seisab.

\question{Mis klass see Jaagul oli? Kas see asus mõne kooli juures?}

See oli pigem huvialamaja, ma täpselt ei mäleta. Igatahes sai seal klassis käia arvuti taga istumas ja nikerdamas. Noored nagad tahtsid 
loomulikult hullupööra mängida, aga kurjajuur Jaak Loonde
ütles, et ei-ei, tuleb hirmsal kombel ikka programmeerida. Nii oligi 
tasakaal nende kahe asja vahel üsna hästi paigas, sest niipea kui 
Jack kõrvale vaatas, olid poisid kohe mingid asjad käima tõmmanud. Kui 
keegi lasi võrku mängu, siis said kõik seda endale laadida. 

\question{Kust mängud tulid? Poest neid osta ju ei saanud.}

Klassis oli õpetaja arvuti ja õpilaste töökohad. Nende vahel oli võrk, mis 
oli naljakal kombel üles ehitatud MIDI kaabli otsa. MIDI on küll rohkem mõeldud muusikariistade juhtimiseks -- Yamahad olid tegelikult algselt 
muusikaarvutid ja olid läbi muusikasüsteemi kohapeal võrku pandud.

Mängud liikusid kassettidel ja ketastel. Aeg-ajalt käidi välismaal. Mäletan, kuidas TTÜs tuli keegi välismaalt 
mingi teadustööga seotud ettevõtmiselt tagasi ja pani lauale kolm kolmetollist flopit. Kõik seisid kõrval ja ootasid, mis 
nende peal on. See oli kohutavalt harras hetk.

\question{Tol ajal ju osta ega alla laadida kuskilt midagi ei olnud, kõik 
käis käest kätte.}

See oli keeruline jah, sest oli sügav nõukaaeg, 
kaheksakümnendate keskpaik. Servast hakkasid vabaduse kiired juba 
terendama, aga need ei olnud kuskilt otsast veel materialiseerunud. Põhiline 
värk oli see, et inimestel lasti teha asju, mille eest varem 
oleks türmi pistetud.

\question{Kas Luise tänaval käisid keskkooli ajal?}

See oli enne keskkooli, olin umbes 13-14aastane -- kael juba kandis, aga mitte väga. 
Täiskasvanuks veel ei peetud, selline ebamäärane aeg, kui veel ei
tea, mis sust saab.

\question{Kas ühel hetkel hakkasid tulema BBSid ja FidoNet?}

Need olid tükk aega hiljem. Me kasvasime suuremaks ja  täiskasvanud hakkasid ka meiesuguseid nolke 
tõsisemalt võtma. Enne rääkisime põgusalt, et kui nõukaaeg lõppes 
ja Eesti aeg algas, siis esimene palk oli 300 
krooni, mis praegusel ajal on 20 eurot. See oli kuupalk ja sellest elas kenasti ära. Mitte küll nii, et oleks midagi hullupööra huvitavat selle eest 
saanud osta, aga elas ära. See võtab tegelikult nõukaaja 
elatustaseme üsna hästi kokku: kuna asjad, mis meil siin ringi liikusid, olid 
valmistatud oma Normas, Salvos või Kommunaaris, siis olid nii hinnad kui ka sissetulek väiksed. Asjad 
olid tasakaalus. 

\question{Arvutit 20 euro eest ei osta.}

Arvutit tõesti ei osta, aga need vahendid olid olemas asutustel. 
Teine asi oli see, et Nõukogude Liitu oli keelatud eksportida kõvemat 
arvutustehnikat. Et meile tekkisid siia näiteks MSXid, oli osaliselt 
tingitud sellest, et tegu oli suhteliselt alumise otsa masinatega, rohkem 
mänguasjade kui päris tööriistadega. Näiteks kui 
Soomest veeti Eestisse üks 386, kui need oli just äsja välja tulnud, siis sündis
sellest kohutav rahvusvaheline skandaal. Ühendriigid võtsid soomlastel kõri 
pihku ja ütlesid, et mis mõttes te veate Nõukogude Liitu niisugust 
tehnoloogiat, millega on võimalik rakette arvutada ja mida iganes teha! Meie 
mõistes oli see masin tol hetkel väga kõva sõna. Praegu on see muidugi 
naeruväärne, suvaline kell on ka võimsam.

\question{Ehk et arvutile ligi saada, pidi mõnele asutusele külje 
alla pugema.}

Jah. Asutustel olid arvutid, millega nad üritasid oma asjatoimetusi läbi viia.
Arvuteid oli mitmesuguseid, näiteks klassikalisi nõukaaegseid Uralitel 
põhinevaid süsteeme, kus olid terminalid ja suured kastid.
Ajapikku tekkisid ka muud, välismaise päritoluga masinad, peaasjalikult 
286d ning koos nendega võrgud. Seega oli vaja 
inimesi, kes seda kõike haldaksid. Aga inimestest oli põud, kuna 
keegi ei teadnud, mida nende kastidega peale hakata. Ja kui seal kõrval 
nikerdamas käisid, siis muutusid päris kähku kasulikuks. Noore poisina oli 
aega, ei mingeid perekondlikke kohustusi ega muud säärast, huvi oli suur ning 
saidki seal eksperimenteerida. Läksid õhtul pärast kooli sinna, nemad läksid 
töölt ära ja said seal istuda kuni üheksa-kümneni. Ja kokkulepe oli see, et üritad kuidagimoodi kasulik olla, ning 
lõpuks sai arvuti taga istumiskohast töökoht, kui kool läbi sai.

\question{Mis asutuses sina käisid?}

Minul oli alguses linnavalitsus\index{Tallinna Linnavalitsus} ja hiljem 
riigikantselei\index{Riigikantselei} -- kogu selle 
huvitava perioodi, kui Eesti Vabariik välja kuulutati, töötasin ma
riigikogu majas. Seda nimetati 
vist peaministri kantseleiks, Stenbocki maja siis veel ei olnud. Seal olid ka 
Uralid\index{Ural}\sidenote{Nõukogude Liidus Pensas aastatel 1956--1964 toodetud arvutite sari.}, 
üüratud kapid, mille sees olid viiemegabaidised trummelkettad, mis tuli 
hommikul käima lükata.

\question{Kas sul akadeemiline haridus jäi pooleli?}

Mul on jah lõpetamata kõrgharidus. 
Üritan seda seniajani lõpetada ja loodetavasti paari aasta 
jooksul seda ka teen.

Tol ajal oli valida, kas tegeled arvutitega või õpid. Suuresti oli mu valik 
väga selge: ma õppisin arvuti taga oluliselt rohkem.

\question{Millal FidoNet Eestis jalad alla võttis?}

Ma täpset aastaarvu ei oska öelda, aga sellega tegutsesid 
Tõnis Reimo\index[ppl]{Reimo, Tõnis}, Tarmo 
Ausing\index[ppl]{Ausing, Tarmo} ja Virko Püss\index[ppl]{Püss, Virko}. Lisaks
jõlkusime seal mina ja Miko Raud\index[ppl]{Raud, Miko}.

\question{Kus te tegutsesite?}

Erinevates kohtades, näiteks Narva maanteel. Eks grupi tuumik 
teab nüansse paremini kui mina. Vahepeal tekkis seal 
võimalusi peaasjalikult soomlastega asju ajada, kui avastati enda jaoks BBSid ja aduti, et tarkvara peab ka kusagilt tulema. Sel ajal
hakkas lisaks flopidele tekkima võimalus modemiga asju alla laadida ja tulid 9600sed modemid.

Asjad hakkasid jumet võtma. Tol ajal olid tarkvarapaketid maksimaalselt paari 
mega baidised ja olid tõmmatavad umbes päevaga.

\question{Seega helistati Soome BBSidesse sisse?}

Jah. 

\question{Kuidas see käis? Jaan Tallinn\index[ppl]{Tallinn, Jaan} on 
rääkinud läbi inimoperaatori arvuti külge helistamisest.}

Igasuguseid imeasju tehti. Näiteks selgus, et lifti 
telefoniühendusest oli võimalik välismaale helistada, sest keegi polnud taibanud 
seda sealt välja lülitada. See tähendab, et liftist sai helistada ja öelda, et 
appi-appi, olen siia kinni jäänud, aga sellesama ühendusega sai helistada ka Soome. Keegi ei olnud nõukaajal kindel, kes 
selle kinni peab maksma, ja seetõttu jäigi see kuidagi ripakile. 
Loomulikult olid ligipääsud erinevatele keskjaamadele ja raha 
tekkis kusagil süsteemides ning kadus kuhugi, nii et 
tegelikult kasutati seda ühte- või teistpidi kurjasti ära. See oli üks viis Nõukogude süsteemi õõnestada.

Sellega seoses tekkisid kontaktid. Näiteks BBSi sisse logides vaatas
\emph{sysop}, et ohoo, Eestist 
mingid tüübid, ja tahtis paar 
sõna juttu ajada. See oli üsna tavaline, et BBSi operaator rääkis külalistega.

BBS ei olnud väga erinev tänapäeva 
sotsiaalmeediast. BBS pandi püsti kahel põhjusel: esiteks, et kontakte luua ja
\emph{networking}'ut teha, olla nii-öelda elu pulsil. Teiseks millegi
propageerimiseks, näiteks oli BBS mõne firma juures või oli 
mingisuguse demo grupp enda oma püsti pannud. See BBS, kust me esimese 
kontakti saime, oli Poison Door\index{Poison Door}.

\question{BBS võis ka mingi demo grupi juures olla, soomlaste 
\emph{demoscene}\sidenote{Demo on arvutikunsti teos, mis kujutab endast 
terviklikku, sageli väga väikest arvutiprogrammi, mis esitab 
audiovisuaalset vaatemängu. Demo eesmärk on demonstreerida (nagu nimigi 
ütleb) autorite programmeerimise, visuaalkunsti ja arvutimuusika oskusi. Demode 
ümber tekkis kogukond, \emph{demoscene}, mis sai kokku demopidudeks kutsutud 
festivalidel. Üks kuulsamaid on siiamaani regulaarselt Helsingis 
toimuv Assembly.} oli tol ajal väga kõva.}

Mul endal oli kontakt Future Crew\index{Future 
Crew}\sidenote{Soome demogrupp, mis peamiselt tegutses aastatel
1987--1994. Nende tehtud oli tõenäoliselt kõigi aegade mõjukaim demo 
\enquote{Second Reality} (avaldati Assembly demopeol 1993. aastal). See tegi 
tänapäeva mõistes olematu riistvara peal reaalajas asju, mis tundusid täiesti 
võimatud, nägi üliäge välja ja sisaldas muusikat, mis siiani kananahka tekitab. 
1999. aastal hääletasid Slashdoti lugejad selle demo kõigi aegade kümne 
vingeima häki hulka.} tüüpidega. Ma laadisin nende BBSist alla niisuguse 
toreda mängu nagu \enquote{Wing Commander}\index{Wing Commander}. 
Omadele anti asju, mis tegelikult ei olnud
päris ametlikult väljas. Peaaegu kõikidel BBSidel olid tagatoad, 
kus hoiti nodi, mida kasutati vahetuskaubana. Tarkvara oli sel 
ajal kõva valuuta. Me panime püsti kahepoolse ühenduse: mina 
laadisin üles mingi muu asja, mille olin kusagilt saanud, ja 
sealtpoolt tõmbasin vastu \enquote{Wing Commanderit} ning samal ajal sai rääkida ka. 
See tarkvara võimaldas kahepoolset sidet ja samas ka 
\emph{chat}'ida, mis ei võtnud väga palju ühenduse mahtu.

\question{Iga klahvivajutus oli üks sümbol, fondi või värvide 
informatsioon kaasa ei liikunud.}

Just, see oli tavaline tekst. Kogu mängu allatõmbamine võttis 
aega tunde. Selleks ajaks olin endale ise ühenduse sebinud, riigikantseleil oli
selline võimalus nõukaaja lõpus. 

Ühesõnaga, tutvuti ja info liikus. Ja oli aja küsimus, millal lõpuks siingi 
oma BBS püsti pandi ja FidoNeti kontakt saadi. Ma just hiljuti uurisin selle 
kohta ja paistab, et Fido on nüüdseks lõplikult hinge heitnud.

\question{Üsna kaua võttis aega!}

Võttis küll, aga võibolla on mõttekas see uuesti üles tõmmata. See eksisteerib endiselt ja 
tänapäeval on retroasjad moes, nii et ehk ärkab see kunagi uuesti ellu.

\question{Mis oli Eesti üks esimesi suuri BBSe, kus rahvas hulgakaupa sees 
käis?}

Esimene tõsiseltvõetav BBS, just nimelt FidoNeti mõistes, oli Hackers Night 
System\index{HNS}\index{Hackers Night System|see{HNS}}. Nagu 
nimigi ütleb, oli tegu häkkerite öösüsteemiga. Päeval olid telefoniliinid muuks 
otstarbeks, öösel käis nende peal BBSidesse helistamine. 

\question{Miks sel ingliskeelne nimi oli?}

Et oleks rahvusvaheline ja äge.

\question{Kes HNSi käigus hoidis?}
Seesama kamp: Reimo\index[ppl]{Reimo, Tõnis}, Ausing 
\index[ppl]{Ausing, Tarmo} ja Virk\index[ppl]{Püss, Virko}.
 
\question{BBSi jaoks oli ju mingit riistvara ja modemeid vaja?}

Oli jah, sinna juurde käis paras sebimine. Tol ajal olid vahendid suuresti riigi rahakotis. Selle küljes siis 
istuti ja kui oldi juba kasulikud, siis sai alati ka neid ressursse juhtida õiges suunas. 

\question{Mis aastal see oli?}

Kaheksakümnendate lõpus, mitte 1989, vaid varem. Ma täpselt ei mäleta, 
vanus oli selline, et keegi ei olnud veel täiskasvanu, aga ka mitte enam laps. Aeg omas siis teist tähendust ja nüüd hiljem on
raske mõõtkava peale panna.

\question{Kas tol ajal oli BBSil üks modem ja üks liin?}

Ojaa. Tegelikult oli muid ka, paralleelse side katseid, näiteks 
PirnBox\index{PirnBox}. FidoNeti mõistes klassikalistest BBSidest oli HNS esimene ja sealt läks 
asi krõbinal laiali.\sidenote[][-4mm]{Pronto ise pidas BBSi New Age 
System\index{New Age System} FidoNeti aadressiga 2:490/12.}

\question{Kui palju neid BBSe tipphetkel 
oli?}

Tipphetkel oli 20--30. Süsteem nägi ette \emph{point}'e, mis olid nii-öelda pool-BBSid. Täpsemalt olid \emph{point}'id ja \emph{full 
node}'id. \emph{Node}'il olid kohustused: meile tõmmata, hoida ja 
jagada. \emph{Point}'iga sai lihtsalt tõmmata. Paljud 
BBSid otsustasid \emph{point}'iks olemise kasuks puhtalt sellepärast, et need 
ei saanud ennast kogu aeg käimas hoida. \emph{Node}'idel olid \emph{point}'id, keda nad varustasid 
informatsiooniga, ja \emph{node}'i käimas hoidmine eeldas ühte- või teistpidi 
võimekust olla teatud hetkedel üleval.

\question{Seega oli Eestis tol ajal 
paarkümmend inimest, kellel oli võimekus sebida liin ja riistvara ning ka 
süsteemi käigus hoida.}

Tipphetkel küll jah. Vahepeal sai nõukaaeg otsa ja tuli Eesti Vabariik ning ühel 
hetkel hakkas asi selles mõttes käest ära minema, et raha hakkas omama 
tähendust. Enam ei saanud lihtsalt kusagil ettevõtte küljes istuda ja oma
asju teha. BBS koos telefonikõnedega tekitas kulusid ja peod
hakkasid vaikselt kinni minema. Inimesed vahetasid töökohti ja uutes 
kohtades ei vaadatud selle peale enam lahke pilguga.

\question{Kuidas sellest ürgsupist Eesti arvutifirmad tekkisid? Kas BBSide 
seltskond läks sujuvalt üle teenuste pakkumisele?}

Osaliselt küll. Need inimesed olid ühte- või teistpidi 
arvutifirmadega seotud, aga tihtipeale ei olnud need päris samad 
inimesed. Teatavasti on sogases vees kõige parem kala püüda, seal on kõige 
suuremad purikad. Sogasel ajal leiti erinevaid viise, kuidas endale 
raha teha. Näiteks Peterburist veeti autoga Tallinnasse igasugust IT-tehnikat. Peterburis olid punktid, kust sai asju 
osta ja Eestisse tuua. Nii see elu vaikselt edenes.

\question{Kas sina olid sel ajal veel riigikantseleis\index{Riigikantselei}?}

Jah, aga oli näha, kuidas hakkasid tekkima esimesed firmad, mõned edukad, 
mõned vähem edukad. Ühel hetkel läksin riigikantseleist 
minema, sest ka seal toimusid struktuurimuudatused.

\question{Mida sa tol ajal peamiselt arvutiga tegid? Kas kirjutasid 
koodi?}

Nüüd tundub see ehk naljakas, aga siis oli see nagu eluviis. Ega see väga ei erinenudki praegusest eluviisist, vahe on ainult selles, 
et nüüd ei pea näiteks Facebookile ligipääsu saamiseks kulmulihastel ringi roomama. Tollal ei olnud see kõikidele kättesaadav. 

Arvuti kasutamisel oli siis küllaltki kõrge lävi, mis eeldas teatud ülevaadet tehnikast ja võimalustest. Praegu on internet ise ennast 
sõlme tõmmanud, aga varem pidi täpselt teadma aadresse, 
kuhu minna, sest polnud otsinguid. Siis alles hakkasid tekkima esimesed 
otsingumootorid: WebCrawler, AltaVista ja lõpuks Google. Need tõmbasid läve madalaks. 

\question{Mida BBSiga teha sai?}

Sai faile jagada ja kirju vahetada. FidoNet oli tänapäeva mõistes suuresti
interneti meilisüsteemi sarnane. Olid ka uudisgrupid ja useneti grupid, mis on asendunud näiteks Facebooki ja Redditiga, kus käib 
info vahetamine.

\question{Useneti grupid olid tollal hierarhilised, aga praeguseks on see struktuur laiali vajunud.}

Jah, olid hierarhiad ja etiketid, mida võhikul oli väga raske 
aduda. Tihtipeale inimesed tundsid küll üksteist üsna lähedalt, aga 
teinekord mõnd jutuajamist jälgides tekkis täieliku \emph{outsider}'i tunne, kui ei saanud aru, millest jutt käib. Kõikidel oli oma taust.

\question{Kus inimesed tuttavaks said? Kas nendes gruppides?}

Oli kaks varianti. Keegi tutvustas ja aitas ree peale või siis kiibitsesid mõnda 
aega ja ühel hetkel hakkasid aru saama, mis toimub. Kui üldse hakkasid, see ei olnud lihtne.

\question{Kui tihedalt Eesti FidoNeti seltskond omavahel läbi käis?}

Seltskond pidi paratamatult läbi käima, sest FidoNeti tekkides moodustusid ka grupid, kus tuli sisu 
tekitada. Ja kuna esialgu oli inimesi vähe, siis paratamatult ei olnud ka
kommunikatsioon meeletult tihe. FidoNetiga tegeles 
paar-kolmkümmend inimest ja isiklikult tuttavaks saamine ei olnud keeruline.

\question{Kes need inimesed olid?}

Enamasti samasugused IT valdkonna inimesed nagu mina, kellel olid
sarnased huvid -- meil oli, millest rääkida. Olid ka 
teemad, mis siis olid parajasti \emph{zeitgeist}. 
Näiteks \enquote{King's Quest 
IV}\index{King's Quest} mängides ei olnud võimalust minna 
veebi ja otsida \emph{walkthrough}'d. Inimesed üritasid omal jõul 
läbi närida ja aeg-ajalt vahetati kogemusi. Muide, sellest ajast pärineb 
ka Habichti raamat \enquote{Selles mängus ei hüpata}\sidenote[][]{Juhan Habichti novellikogumik, mis ilmus 1993. aastal kirjastuse Katherine väljaandel.}. 
See mäng oli 
\enquote{Larry}\index{Larry (mängusari)}\sidenote[][]{\enquote{Leisure Suit Larry} oli Al Lowe'i
loodud seiklusmängude sari, mis ilmus aastatel 1987--2009 ning oli 
tuntud omapärase huumori ja alaealistele sobimatu sisu poolest. 
Näiteks katsus mängija riiulil seisvat kopratopist ja Larry teatas: \enquote{\emph{I've always 
liked the feeling of a good beaver}}.}.

Samuti räägiti võimalustest ja nende 
vahetamisest. Ühel oli üks asi, teisel teine ja pandi seljad kokku. Kuna 
inimesi oli vähe ja üksteist teati, siis ei olnud ka väga 
suurt kanakitkumist.

\question{Kas trollimist või muud säärast ka toimus?}

Kui auditooriumi ei ole, siis inimesed jäävad tavalisteks inimesteks. Kui 
annad normaalsele inimesele anonüümsuse ja publiku, siis saab tast igavene 
tõpranahk.

\question{Isegi kui sul oli \emph{handle}, siis sa ju tegelikult ei olnud anonüümne.}

\emph{Handle} oli lihtsalt nimi, tegelikult kõik teadsid, kes on kes. 
Isegi kui olid anonüümne, siis ei 
kasutatud laest võetud nimesid. Kui olid oma nimele
feimi tekitanud, siis sa ju ei tahtnud sellega 
uisapäisa ringi käia.

\question{Sa oled siiamaani Pronto ja teistele tähendab see senini midagi. 
Kui keegi hakkas sigatsema, kas ta visati siis välja?}

Jah, juhe tõmmati seinast ja olid kohemaid \emph{persona 
non grata}. Kuna see oli seotud sinu enda huvidega ning
mineviku, oleviku ja tulevikuga, siis ei saanud seda omale lubada.

\question{Seega käitusid kõik viisakalt?}

Kõik olid seal paadis võrdsed. Kui keegi hakkas paati 
kõigutama, siis ta kõigepealt kõigutas seda enda all ja kui ta seda jätkas, siis ta lihtsalt eemaldati paadist ja pidi ise vaatama, kuidas 
veekogus hakkama saab.

\question{Kas seda juhtus ka?}

Otseselt mitte või kui juhtus, siis juba hilisemal ajal. 
Alguses oli ikkagi tihe seltskond ja kuigi kõik ei saanud omavahel 
ideaalselt läbi, mõisteti, et selles paadis ollakse koos. Seetõttu tüli põhjustada võivaid teemasid
lihtsalt välditi.

\question{Seega saadi aru, et teatud asjadest ei tasu rääkida.}

Jah. Trollimine ju ongi rääkimine asjadest, mis teisele 
inimesele peavalu valmistavad.

\question{Tuleme sinu juurde tagasi. Kui sa 
riigikantseleist\index{Riigikantselei} ära tulid, mida sa siis tegid?}

Töötasin sellises kohas nagu Marvin-Ekspert, sain seal ostmise ja müümisega käe valgeks. Tegelesin selliste toodetega nagu Gravis Ultrasound ja 
IOMega.\sidenote{Gravis Ultrasound oli toona PC-maailmas helikaartide tipptootja ja IOMega tegeles väga innovatiivsete andmesalvetuslahendustega.}

See oli selles mõttes huvitav aeg, et Gravis Ultrasound maksis väikse 
varanduse, aga samas oli see tükk maad parem kui mõni teine toode. Müüsin neid umbes sama palju kui kõiki 
ülejäänud asju kokku müüdi, kuigi see oli kallis. Mõnes mõttes oli sellega sama lugu nagu 
Apple'iga: kallimat asja on alati lihtsam müüa, sest kalliduse taga on tavaliselt 
väärtus, toode ei ole kallis niisama.

\question{Mis aastal see oli?}

Ilmselt 1994. või 1992. aasta kanti.

\question{Kas sel ajal hakkas tasapisi MicroLink tekkima?}

MicroLink tekkis tegelikult üsna aegade alguses. See oli üks nendest firmadest, kes 
alustas sellest, et hakati kotiga Peterburist asju tooma. Esialgu müüdi 
arvuteid firmadele, sest nendel oli raha, kuigi 
omandisuhted polnud veel päris paigas.

\question{Kapitalismi oli veel vähe?}

Kapitalismi oli jah vähe, olid veel nõukaaja jäägid -- keegi oli kuskil käpa peale 
pannud. Oli niisugune aeg, kui ma paljusid asju ei teadnud ja 
paljusid teadsin, aga ei tahtnud teada. Asju, mille kohta võib öelda, et mis juhtus Vegases, las jääb Vegasesse. Sel ajal tehti 
igasuguseid asju, mis praegu võivad näida küsitava 
eetilise ja moraalse taustaga, kuid siis olid 
tegelikult õiged ja vajalikud.

\question{Tol ajal ju kujuneski välja, mis on õige ja mis mitte.}

Jah. Loomulikult tehti sel ajal igasugust erastamist ja ärastamist, aga ka see oli 
hädavajalik puhtalt sellepärast, et tookord tehtud otsused eristavadki meid 
tänapäeva Moldovast, kus omal ajal tehti teistsuguseid otsuseid. Isegi 
need, kes meil siin ärastasid, tegid seda teataval määral 
\enquote{eesmärk pühendab abinõu} kaalutlustel.

\question{Räägi pisut ka ajakirjast .EXE\index{.EXE}.}

See ajakiri oli osaliselt MicroLinki püüd ennast nähtavaks 
teha. Eestis oli tollal kaks arvutiajakirja: 
Arvutimaailm\index{Arvutimaailm} ja .EXE. Arvutustehnika \& 
Andmetöötlus\index{Arvutustehnika \& Andmetöötlus} ei olnud klassikalises mõistes ajakiri, vaid rohkem 
vihik. Nõukaaja lõpus ja Eesti aja alguses anti välja vihikuformaadis 
erialaväljaandeid, mis ei olnud mõeldud laiaks tarbeks.

.EXE tekkis umbes samal ajal kui Arvutimaailm, MicroLink 
püüdis tekitada endale laiatarbeväljundi.

\question{Selles ei olnud palju laiatarbeasju, 
vaid stiilipuhas \emph{hard core} küberpungijutt!}

.EXE oli selles mõttes \enquote{laiatarbeväljund}, et sel ajal ei olnud inimestel 
raha arvuti soetamiseks. Pidi olema ikkagi väga suur tahtmine ja vastavalt sellele kujunes ka ajakirja sisu. 

Sel ajakirjal oli ajastu hõng juures. Mida inimesed arvutiga parasjagu tegid, 
see sealt ka läbi kumas. 

\question{Kuidas sa selle juurde sattusid? Kas kirjutasid juba enne .EXEt?}

Tol ajal kirjutati näiteks naljaviluks mängudest, sisu toodeti vabatahtlikult. 
Gruppidesse postitati dokke, häkiti ja nii edasi. Kuna ma olin mängudest kirjutamisega silma 
paistnud ja ka kirjaoskus enam-vähem olemas, siis nii ma ajakirja sattusin. 

See oli päris 
naljakas aeg -- ajakirja koostamine oli omamoodi 
häkkimine. Tavalist kogunes kolleegium (seltskond, kes sisu kokku pani) 
kokku, lükati ette kaks kasti õlut ja enne toast välja ei 
lastud, kui ajakiri oli kokku pandud. Igaüks võttis endale mingid kohustused 
ja kadus nendega tegelema.

\question{Kaua .EXE üldse ilmus?}

See ilmus umbes poolteist aastat.

\question{Nii vähe?}

Jah, ma ühel hetkel korjasin kõik numbrid kokku\sidenote{Aadressil 
\url{punktexe.ee} on kõik ilmunud numbrid täies mahus olemas.}. Esimene number ilmus aprillis 1993 ja viimane 
1995. aastal. Nii et kaks aastat, vahepeal läks ilmumine eklektiliseks.

\question{Kes neid ilusaid kaanepilte joonistas?}

Kaspar Loit\index[ppl]{Loit, Kaspar} \emph{alias} BKnows.

\question{Arvestades, milline mõju ajakirjal oli, palju seda loeti ja kuidas fännati, 
siis oli ilmumise lühidusest hoolimata tegu väga mõjuka asjaga.}

Jah, numbreid oli kokku vist kaheksa. Igaüks oli omaette šedööver, kuna see oli südamega tehtud, eriala inimestelt eriala inimestele. .EXEt anti välja selleks, et skenet juurutada, mitte et selle pealt 
üüratut kasumit teenida. 

\question{Millist skenet? Arvutiinimeste oma?}

Jah. Inimesele tänavalt
oli see ajakiri ehk pisut raskevõitu. Tol ajal oli 
arvutiajakirjandus teistsugune kui praegu, mil
igaühel on arvuti ja loetakse, kuidas oma mobiiliga 
ühte, teist või kolmandat teha. Arvuti oli siis suur asi, 
seda polnud kaugeltki mitte kõigil. Praeguses mõistes üks-kaks protsenti inimestest tabas tegelikult reaalselt arvutit ja oskas seda 
igapäevaelus kasutada.

\question{Seevastu inimesi, kes tahtsid kasutada, oli rohkem. Ja nii nad lugesidki hardalt, kuidas Pronto seikleb \enquote{Day of the 
Tentacle'is}\index{Day of the Tentacle}\sidenote{Legendaarne mäng, mis ilmus 1993. aastal LucasArtsi väljalaskel ja uuendatud graafikaga 
2016. aastal ning mille \emph{walkthrough} avaldati .EXE teises numbris 
novembris 1993, autoriteks BKnows\index[ppl]{BKnows} ja 
Pronto\index[ppl]{Pronto}.}}

Eks see kõik hakkaski pisitasa tuult tiibadesse võtma. Sel ajal toimus 
jõhker inflatsioon ehk räägitud kahekümnest eurost said päris kiiresti 
sajad eurod. Arvutid muutusid jõukohaseks ka teistele ja Rootsist veeti siia humanitaarabi korras pruugitud tehnikat.

\question{Kas kogu selle ajal jooksul müüsid sina muudkui Gravist?}

Gravist ja Iomega Bernouilli draive\sidenote{1992. aastal turule tulnud, oma aja kohta suure mahutavuse ja 
eemaldatava kettaga salvestussüsteem Bernouilli Box 
oli Iomega esimene laialt kasutust leidnud toode.}, QIC-80 
teipe ja muud säärast.

\question{Huvitav, et mitmed inimesed on teatud faasis tegelnud just 
arvutustehnika müügiga.}

Kuskilt tuleb raha teenida. Kätte jõudis aeg, kui varad said laiali 
jagatud ja sa pidid oma tegevust põhjendama, näiteks miks sul on 
BBS. Ainuke võimalus seda asja edasi edendada oligi müügi 
egiidi all.

\question{Kas koodikirjutamisega ei saanud elatist teenida?}

Tol ajal ei olnud eriti mingeid koode, mida kirjutada. Väikseid asju loomulikult oli, aga valdavalt käis koodikirjutamine 
andmebaaside ümber, näiteks olid FoxBase ja DBase, kus tehti 
ettevõtete raamatupidamist ja inventuuri. 

\question{Kas iga ettevõte pusis endale ise tolle rakenduse kokku?}

Kas ise või osteti firmadelt, aga süsteem koosnes tavaliselt 
mõnest andmebaasilahendusest. Oli ka muid asju, 
näiteks meditsiiniga seotud lahendusi, millel olid juba infosüsteemid, aga 
need olid väga spetsiifilised ja neid arendati enamasti väikses mahus.

\question{Eestlane üldiselt ei ole suurem asi müügiinimene, 
aga IT-asja on meil õnnestunud rahvusvaheliselt päris hästi müüa. Kas ehk
seetõttu, et kriitilisel hulgal inimestel on olnud müügikogemus?}

Kindlasti. Tol ajal oli see paratamatu, sest kui tahtsid 
saada ligipääsu, pidid juba siis ennast müüma. See on üks asi, mis on muutnud 
vana kooli IT-vennad teistsuguseks -- sa pidid paratamatult suutma müüa. Kui ei suutnud, siis polnud sul IT valdkonda asja. Kõige 
tähtsam kaup olid sa ise.

\question{Sest muud sul ei olnud?}

Muud ei olnud, isegi mitte kogemusi, sest kogemused tulevad töö käigus. Sa pidid suutma endast teha väga vajaliku tegelase.

\question{Nii et kui enesemüügi oskus on olemas, siis võib igasuguseid 
asju juhtuda.}

Kui tähelepanelikult vaadata, siis IT valdkonna müügis ongi 
läbimurrete taga tihtipeale ühed ja samad inimesed ning just 
vana kooli kaader, kes enamasti on oma läbimurde ehk müügi saavutanud mitte 
tänu avalikkusele, vaid vaatamata sellele. Teatavasti tunneb avalikkus 
kohemaid muret, kui keegi teenib paremini või tunneb ennast kuidagi paremini. 
Hari läheb kohe kadedusest punaseks.

\question{Tihti öeldakse, et meil on vedanud, sest õiged inimesed on sattunud 
õigetesse kohtadesse. Sinu jutust tuleb välja, et tollest seltskonnast tulidki 
inimesed, kes sattusid õigetele kohtadele.}

Täpselt nii. Need inimesed on siiani 
alles, osa neist üle viiekümne, osa alla selle, aga üks või teine on suuremate 
läbimurrete taga.

\question{Oskad sa öelda, mis 
hetkel kaotas see maailm oma süütuse? Kui romantilisest õllekasti abil 
toimetamisest sai raha teenimine.}

Ma ei oska seda niimoodi paika panna, sest tegelikult on see 
ikkagi suuresti väljaspool loodud kuvand. Kui on mingisugune grupp, 
siis paratamatult tekivad autsaiderid, kes tunnevad pahatahtlikku 
kadedust. 

\question{Ja nimetavad inimesi häkkeriteks?}

\enquote{Häkker} hakkas omandama lihtsalt teistsugust tähendust.

\question{Viidates ühele .EXE loole, mis on 
küberpunk?\sidenote[][]{Allkirjastamata, kuid BKnowsi\index[ppl]{BKnows} piltidega 
lugu \enquote{Kes sa selline oled, küberpunk?} ilmus .EXE kolmandas 
numbris 1994. aasta aprillis. Seejuures tuleb tunnustada artikli asjakohasust: nii ilmumise (eba)regulaarsuse ja lühiduse kui ka kultusliku staatuse poolest .EXEga sarnane, kuid suurema levikuga ajakiri MONDO 2000 (aastatel 1984--1998 ilmus USAs 17 numbrit) avaldas oma samateemalise satiirilise artikli \enquote{R.U. A CYBERPUNK?} oma 10. väljaandes 1993. aastal.}}

Kõik asjad, mis on punk, nagu aurupunk, küberpunk või diiselpunk, on lihtsalt 
žanr, mis läbib mitut asja; valdavalt seda, kuidas siduda teadvus 
tehnikaga. Mõnes mõttes on meie ühiskond praegu nii-öelda küberpungi jaoks 
esimesel tasemel, sest see, kui inimesed istuvad ninapidi telefonis, on 
lihtsalt liidestamise küsimus. Inimesed on ennast tegelikult arvutiga juba väga 
intiimselt liidestanud.

\question{Nagu sa mainisid, siis algas see juba kaheksakümnendate lõpus, kui 
kogu sinu elu oli arvutis. Lihtsalt liides oli kandilisem.}

Liides oli kandilisem ja olemas vähestel inimestel; seetõttu polnud see elu, vaid 
mu \emph{alter ego}. See ongi üks põhjus, millepärast valiti omale sellised
tunnused, nagu mul on Pronto -- et teha vahet sellel, mis toimub arvutis ja mis 
niisama. Põhimõtteliselt loodi endale identiteet. 

\question{Just nimelt loodi, mitte ei valitud!}

Ja sellega elati osaliselt tulevikus, aga ka muu elu jäi alles. Pere, sõbrad ja see õlu, mida joodi, jäi kõik teise ellu.

\question{BBSi rahvas käis ju koos ka.}

Käis küll. Kõigepealt olid \emph{sysop}'ide saunad ja muud üritused, kust kasvas välja FidoNet. Hiljem tekkisid
BBSummerid\index{BBSummer}.

\question{Kui palju neid toimus?}

Need said alguse nõukaaja lõpus ja neid toimus üksjagu. Üks 
BBSummeritest, vist teine või kolmas, lükati edasi sellepärast, et tankid sõitsid Eestisse 
sisse.

\question{Olen näinud BBSummeri pilte, mille peal on kõik MicroLinki, Skype'i, Unineti ja 
teiste hilisemate suurte asjade alustajad. Kas tol ajal, asja sees 
olles, ei olnud niisugust tunnet, et oi, küll me oleme ägedad?}

Muidugi oli! Me olimegi hullult ägedad! See oli ka üks põhjus, miks me sellega tegelesime.

\question{Tulles meie jutu alguse juurde tagasi, kas selle ägeduse tuum oli 
jätkuvalt see, et sai masina mõne näpuliigutusega oma tahtele allutada?}

Kindlasti. Teiseks ei piirdunud elu enam oma õuega, vaid koos FidoNetiga tekkis ka ülejäänud maailm sinna otsa. See ei 
olnud väga erinev tänapäeva Redditist, Facebookist või Twitterist, kus ei
saa suhelda mitte ainult paari lähema tuttavaga, vaid kogu ülejäänud 
maailmaga. See andis 
näiteks võimaluse keeli omandada ja suhelda erinevates keeltes, mis omakorda aitas edasi.

\question{Nii et see tekitas maailma avardumise tunde?}

Maailm avardus kindlasti. See oli mõneti samasugune tunne nagu 
kosmonaudil, kui ta atmosfäärist väljub. Eriti kui see pind, millelt üles 
tõusti, oli tükk maad madalamal kui enamiku maailma jaoks -- me tegime
otse nõukaajast sammu tulevikku.

\question{Ühel hetkel olid Nõukogude pioneer ja pisut hiljem 
vestlesid California kuttidega keskjaamadest.}

Jah, absoluutselt. Tekkisid võimalused ja kogemused. Näiteks mõnes mõttes 
positiivne nähtus oli see, et Eestis puudusid \emph{legacy} süsteemid, meil 
polnud IT valdkonnas mineviku taaka, vaid asi oli lihtsalt poolik. 
Mineviku taaga puudumine võimaldas Eestil kihutada päris kiiresti 
päris kaugele võrreldes ülejäänud maailmaga, kes pidi oma asju käimas hoidma. 
Me oleme nüüd jõudnud sinnamaani, kus meil on oma taak tekkinud ja peame 
sellega tegelema.

\question{Lõpuks ikka saab inerts otsa, aga seni on see meid päris kaugele 
vedanud.}

Seda sai üsna hästi ära kasutatud just nimelt sellepärast, et õigel hetkel sattusid õiged inimesed pumba juurde ja saagi tõmmati 
käima nii kaua, kui jõuti, enne kui ärimehed jaole jõudsid. 

Kuna oli hulk inimesi, kes tegid midagi, mis oli 
müstiline, keeruline, käsitamatu ja ilmselt ka veidi elitaarne, siis loomulikult hakkasid 
tekkima needki, kes hakkasid kaikaid kodaratesse pilduma. Inimesed, 
kes tahtsid ka löögile pääseda ja tundsid ennast halvasti, et neid ei 
võetud jutule puhtalt sellepärast, et nad ei saanud aru paadi mittekõigutamise 
mentaliteedist. See oligi mõnes mõttes ajastu lõpp, kui igaühel 
tekkis ligipääs, lävi läks palju madalamaks ja ka lühemate pükstega mehed said 
paati astuda.

Tekkisid inimesed, keda keegi ei teadnud, kes olid anonüümsed ja kellel olid 
ambitsioonid, aga puudusid võimekus ja soov panustada. 

BBSummerid hakkasid samuti kasvama ja kihistuma. Ürituste lõppu tähistas see, kui hakkasid toimuma BB-üritused BB-ürituste sees. 

\question{Mida sa praegu teed? Kuhu see tee sind on toonud?}

Praegu olen juba viimased kümme aastat tegelenud veebipoodidega. Minu eriala on 
veebiarendused, täpsemalt veebipoed ehk e-kaubandus. 

Olen nüüd rohkem programmeerimise peal, sest tänapäeval on 
peaaegu kõik ühte- või teistmoodi seotud tarkvaraarendusega. Tol ajal ei 
olnud firmadel internetilehte nagu praegu. Tol ajal ei pakutud teenuseid 
interneti kaudu, aga nüüd pakutakse. Ja seega on tekkinud vajadus tehnilise võimekusega inimeste järele. Üks võimalus on värvata nad 
endale või siis palgata firma, kes sellega tegeleb.

\chapter{Priit Raspel}
\index[ppl]{Raspel, Priit}


Minul on selline Jaapani ajaarvamine, mille järgi ma asju paika panen. Tean
mingeid sündmusi, mille suhtes ma teisi määratlen. Näiteks 
1993. aastal käisin maailmameistrivõistlustel. 

\question{Mille maailmameistrivõistlustel?}

4GL\sidenote{Neljanda põlvkonna programmeerimiskeel.} 
programmeerimise maailmameistrivõistlustel. Käisime seal kolmekesi: mina, Veiko 
Herne\index[ppl]{Herne, Veiko} ja Tiiu Lumberg\index[ppl]{Lumberg, Tiiu}. 
Tulime viiendaks ja saime veel eripreemia kõige elegantsema lahenduse eest. 
Mõtlesime välja Amazoni, mida tol hetkel ei olnud veel olemas. 

Ja selle järgi teangi, et samal ajal hakkasin ära tulema 
Innovatsioonipangast\index{Innovatsioonipank}. 

See jaapani \emph{native} ajaarvamine käib suurte sündmuste vahel, näiteks 
keisri võimule tulemisest kolmas aasta. Ja kui tuli suur maavärin, siis 
see keiser unustati ära. Teati küll, et maavärin oli keisri võimuletulekust nii mitu aastat hiljem, aga edaspidi arvutati aega maavärisemise järgi. 
Sedasi on jube hea, sest muidu ei pane asju enam liini peale. 

\question{Kuidas sina arvutite juurde said?}

Mina olen juhuslik inimene. Keskkoolis olin 
täiesti veendunud, et see on viimane roppus, mida ma õppima lähen. 

\question{Mis aastal see oli?}

Keskkooli lõpetasin 1979. aastal. Ma käisin Kusti-koolis ehk \mbox{1.
keskkoolis}\index{Tallinna 1. Keskkool}, praeguses Gustav Adolfi Gümnaasiumis. Tol 
ajal oli seal matemaatika-füüsika eriklass ning arvutid fakultatiivselt õppekavas sees. 

\question{Kas juba 1970ndate lõpus?!}

Jah. Meile õpetati programmeerimist Fortranis\index{Fortran} ja pagan 
teab milles. Arvutis käisime Teaduste Akadeemias\index{Teaduste Akadeemia} 
Lenini puiesteel (praegu Rävala puiestee), seal nurga peal, kus on ka
raamatukogu. Arvuti oli Minsk-32\index{Minsk!Minsk-32}, mis 
koliti siis just välja ja toodi asemele vist ES-1022\index{ES EVM!ES-1022}. Mäletan, kuidas seda Minsk-32 välja koliti -- terve alumine klaasfuajee oli 
tükke täis. Ühe klassivenna onu, kes oli seal programmeerija, ütles, et 
arvuti visatakse prügimäele. Me käisime sealt plaate välja koukimas, sest 
seal oli P13 transistor, mis oli kõige defkam 
ja igasuguste asjade tegemisel kasulik.

\question{Nii et sul oli juba siis elektroonikahuvi?}

Jaa, esimese raadio panin kokku vist kuueaastaselt. Isa oli ju raadiotehnik. Kodus vedeles jube palju
juppe ja raamaturiiulis oli raamat \enquote{Noor 
raadioamatöör}\sidenote[][-2mm]{Noor raadioamatöör, 
Viktor Borissov, tõlkinud Arnold Isotamm, Tallinn: Eesti Riiklik Kirjastus, 
1953.}, kus oli kirjas, kuidas detektorvastuvõtjat teha. Eks ma siis 
nihverdasin isa sahtlist mõned dioodid, tegin pooli, panin kõrvaklapi külge 
ja sain mingi Majaki\sidenote{1964. aastal käivitatud üleliiduline raadiojaam, mis
tegutseb siiani.} kätte. Majak oli muidugi nii võimsa signaaliga, et see oleks tulnud ka 
pliidiraua pealt. Kui pliit oleks pilti näidanud, oleks pilti ka tulnud. 

Õppimisega oli selline värk, et meile õpetati jube kehvasti. Õpetajaid ei 
olnud saada ja igal poolaastal luges erinev õpetaja mingisugust täiesti erinevat 
asja, mida ta parasjagu ise just oskas. Metoodika puudus, aga vähemalt saime käia 
Teaduste Akadeemia arvutis. Seal olid perfokaardi 
\enquote{kivipurustajad}, mis lõid auke läbi. Saime oma pakid ühte kappi panna ja nii palju ma ikka tegin, et see tundus päris huvitav. 

Olin täitsa kindel, et lähen elektroonikat õppima. Tegin poistele 
raadiosaatjaid ja ükskord isa küsis: \enquote{Poiss, kas see on sinu töö, et
peilingaator sõidab akna all?} Muidugi oli minu töö. \enquote{Näita, mis sa 
tegid? Kurat, sul on sihukesed lõpptransid, me peame võimsuse tagasi tõmbama!} 
Elasime Tõnismäel ja segajad olid sealsamas\sidenote{Nõukogude Liidus oli 
komme välismaiseid raadiojaamu sihipäraselt segada. Üks selleotstarbelistest 
raadiojaamadest (need allusid sideministeeriumile), nr 602, asus Tõnismäe 
ligidal Luha tänaval.}. Vennad püüdsid signaali kinni ja tulid otsima, kus see on. 
Lõpuks rehkendasime 300 meetri peale, kust lähemale ei tohtinud minna ja 
kui läks, siis tuli ruttu ära kaduda. 

Nii et ma tinutasin igasuguseid asju kokku. Siis juhtus selline jama, et mul tuli 
üks üsna raske haigus, mis viskas mul korra aastas 
teadvuse ära. Ravi 
oli keeruline: pool aastat üpris kangeid tablette täpse 
režiimi järgi ja kui ei mõjunud, siis pidi pool aastat pausi pidama. Haiguse äraminek võttis kolm aastat aega, nii et veel keskkooli lõpus olin haige 
ja elektroonika õppimaminek oli vastunäidustatud, kuigi ma tahtsin just seda teha, sest olin lapsest peale aparaate 
ehitanud. 

Olin noor vihane inimene ja mõtlesin, et ei lähe kuskile. Ei taha, 
mul pole vaja, ma lähen tööle elektroonikuks! Kolb püsib käes, skeemist saan 
aru, oskan isegi skeemi koostada, montaažplaate teha, telekat 
käsikaudu parandada (selles mõttes, et ilma igasuguste mõõteriistadeta leian 
vea üles ja parandan ära), raadioga saan hakkama. Aga isa rääkis augu pähe. 

\question{Mida su ema tegi?}

Ema oli Eesti Raadio\index{Eesti Rahvusringhääling!Eesti Raadio} 
majandusjuhataja, tema viis mind sealse tehnokeskuse meestega kokku. Poes 
polnud ju suurt midagi. Koostasin listi kõigest, mida ma poest ei saanud, 
läksin tehnokeskusse ja sealt leiti mulle. Ükshaaval küsisin juppe ja kuskilt 
karbist need mulle ka leiti. Ja kui nad teinekord midagi maha kandsid, siis 
andsid need tükid ka mulle. Mul on praegugi üks karbitäis asju alles.

\question{Kas sa kokkuvõttes leidsid, et pead \emph{midagi} ikkagi õppima minema?}

Jah, isa arvas, et võiksin ikkagi õppima minna. Ütles, et ole nüüd kaval, esimesel 
aastal on üldained -- mine õpi midagi sellist, kus eksamid on samad. 
Läksin ITsse, mis oli siis majandusliku informatsiooni 
mehhaniseeritud töötlemise organiseerimine -- Eesti kõige pikema nimega eriala 
üldse. Kusjuures selles suhtes vahva nimega, et sõjavägi ei saanud 
aru, keda me koolitame. Terve selle eriala jooksul ei läinud mitte ükski poiss 
ohvitserina pärast kooli sõjaväkke, sest nad ei taibanud, et seal koolitatakse
puhtaverelisi programmeerijaid.

\question{Nii et sa pole Vene kroonus käinud?}

Ei ole. 

Meil oli lahe grupp, kus oli ka kuus kooliõde-venda, ja seltsielu läks kohe käima. Kokku oli meid 25, poisse ainult kuus, sest meil oli 
majandusteaduskonna grupp ja sinna teaduskonda tuli palju ilusaid ja tarku tüdrukuid, kes programmeerisid ka päris kõvasti. 

Läks poolteist kuud ja ma olin müüdud mees.

\question{Mille peale see sul juhtus?}

Ma nägin selle maailma ilu -- neid võimalusi, 
kuidas kõik, mis kõrvade vahel olemas, on võimalik ka päriselt. Ega see 
mul lihtsalt ei läinud, sest Gustav Adolfi Gümnaasiumis õpetati 
neid asju väga hüplikult. Mulle on eluaeg meeldinud süsteemne lähenemine, 
aga seal räägitut ei pannud keegi minu jaoks süsteemi ja 
tol ajal ju ei olnud kohta ka, kust lugeda. Internetti polnud ja ka
raamatuid ei olnud eriti võimalik saada. 

Üks sõber, Lembit Sammel\index[ppl]{Sammel, Lembit} vedas mu Leo 
Võhandu\index[ppl]{Võhandu, Leo} jutu peale kolmanda ühika alla, kus olid Nairi-2\index{Nairi!Nairi-2}, AP keel\index{AP keel} ja 
elektriline kirjutusmasin Konsul. Hakkasime APs kirjutama: tegime biorütme ja 
silusime neid ning tegin ka oma esimese mängu, mis oli tiku äravõtmine -- masin mängis vastu selle peale, kes võtab 
viimase tiku, ja sain sellega päris ilusti hakkama. Vaat 
sellesama mängu kirjutamise ajal see asi haaraski mind, sest ma ei 
saanud sellest päris lõpuni aru. Mulle on jäänud see eluks ajaks meelde, kuidas istusin ühel vihmasel oktoobriõhtul laua 
taga, paber ees, lamp põlemas, ja kirjutasin blokkskeemi. 
Pusisin ja pusisin, aga ei tulnud, ja ühel hetkel käis täiesti kuuldav nips ja sain 
aru, mida ma pean tegema. Kirjutasin üsna kiiresti algoritmi valmis ja 
pärast seda ei ole mul algoritmide kirjutamisega mitte mingisugust probleemi 
olnud. 

\question{Kas sa sattusid üsna varsti tööle ka?}

Teisel kursusel, sest esimesel kursusel kammisin ikkagi ühika vahet. Meil oli esmaspäev vaba ja hommikul 
läksin nii vara kohale, kui ühikaalune klass lahti tehti, ning tulin alles siis tulema, kui mind jõuga 
välja visati. Lembit Sammel, hüüdnimega Sass, tegi täpselt samamoodi. 
Panime nagu paaris härjad. Ei olnud päeva, kus me mõnda masinat nässu ei 
keeranud, sest kui neid pidevalt piinata, siis need põlesid läbi. Arvutiklassis sain tuttavaks ka Lindre Reinuga\index[ppl]{Lindre, Rein}, kes oli 
seal inseneriks ja kellega me hiljem tegime koos ühe vahva asja. 

Teisel kursusel läksin tööle EPTsse ehk Eesti 
Põllumajandustehnikasse\index{Eesti Põllumajandustehnika}. Keskkontor asus Salve 
tänaval, kus olid Minsk-32\index{Minsk!Minsk-32} ja 
ES-1022\index{ES EVM!ES-1022}. ESi numbriga ma võin eksida, aga 
Minsk-32 oli küll. Suured saalid olid mürisevaid seadmeid täis. 

Sain seal operaatoritega hästi läbi ega pidanud enam TPI arvutuskeskuses päev otsa perfokaardiga jamama, et see 
järgmisel päeval kuskilt kapist kätte saada. Läksin tüdrukute 
juurde ja ütlesin, et kuulge, laske mu pakk läbi. 

\question{Kas sind võeti sinna mõnd konkreetset asja programmeerima?}

Ei, lihtsalt otsiti inimest, kes oleks noor ja avatud ning keda nad saaksid ise
oma käe järgi välja õpetada. Oleg Kase\index[ppl]{Kase, Oleg} oli selle tiimi 
juht, kus tegutses näiteks väga geniaalne programmeerija Tõnu Toomus\index[ppl]{Toomus, Tõnu}, kes läks kahjuks 
Estoniaga minema. Nad 
hakkasid mind õpetama ja alustasin lihtsamatest asjadest. Esialgu tegin
alltöid, mingeid funktsioone ja värke, mida neil vaja oli. Aga üsna pea 
jõudsin selleni, et tahaksin ise midagi teha. Selle peale öeldi, et vali 
ise. Kuna seal oli parasjagu igasuguste 
kasutajaliideste tegemine, laoarvestuse ja kõige muu viimine suurest ESist suurde 
SMi\index{SM EVM!SM-4}, siis ma tegin vormi 
generaatori: joonistasin ekraani, sidusin andmebaasiga ära ja siis
MUMPSi\sidenote[][]{Massachusetts General Hospital Utility 
Multi-Programming System -- transaktsiooniline võtme-väärtuse andmebaas, 
millega on integreeritud ka programmeerimiskeel. Selle süsteemi puhul oli 
fookus jõudlusel (selle kaudu käib tänini rohkem kui poolte USA patsientide 
terviseinfo), mitte loetavusel; kõiki käske võis lühendada ja reavahetus ei 
olnud oluline. Tulemuseks oli sageli raskesti loetav kood.} süsteemipuuga ära ja valmis.

\question{Ma olen MUMPSist lugusid kuulnud. Kui tänapäeval oled õpetatud 
programmeerija, siis MUMPS on hoopis teisest maailmast!}

Ja, see on endiselt täitsa olemas, jäi mulle ükspäev kogemata 
internetis jalgu. 

MUMPSis ei ole indekseid, need peab ise tegema üle inverteeritud 
immituste. Kuna seal on puu, siis peab puu võtme, \emph{path}'i kirja 
panema ja registreerima, et nüüd on selle kohta võti olemas. Võtmega
võib \emph{path}'i järgi objektile otse peale minna. 

MUMPSis on näiteks niimoodi, et kui tahad mõnd seadet kasutusele võtta, 
siis pead teadma seadme numbrit. Näiteks printer oli 
vist 80. Kirjutad \verb|U:80|, mis tähendab, et nüüd läheb kõik ülejäänud jama, mida 
väljundisse paned, printeri peale. Kui tahtsid kuvarit saada, siis 
igal kuvaril oli oma number. Kuna need olid füüsilised masinad, siis tegid 
andmebaasi kõigepealt loendi olemasolevatest kuvaritest ehk nimetasid need
ära. Seejärel öeldes näiteks \verb|U:1|, sattusid 
esimese kuvari, konsooli peale. 

Lisaks oli võimalik anda \emph{wait}-aegu ehk \enquote{oota nii kaua ja siis mine edasi}, aga ma ei mäleta, mis sümbol 
seal vahel oli. Üldiselt oli selle programmeerimiskeel enam-vähem loogiline. Kuna sel oli 
hierarhiline, puukujuline andmebaas all, siis see pani muidugi omaette 
pitseri, sest \emph{pre-} ja \emph{post-order} ning muu säärane 
pidi hästi käpas olema. 

\question{Kas EPT-l oli igal pool kontoreid?}

Jah, näiteks Sauel, Paides, Tartus, igal pool 
tugevad inimesed eesotsas. Tiimid olid väikesed, neli inimest. Muidugi tehti siis kontoris suitsu 
ja joodi kohvi. Kuvarid olid eraldi ruumis, arvuti teises ruumis, 
igaühel oli ühel pool kuvarit tuhatoos ja teisel pool kohvitass. Kohv oli täiesti 
must ja suhkruta, sest piim läheb ju hapuks, seda ei saa kuskil 
kapis hoida, ja suhkurgi saab otsa, mis sest ikka osta. Nii et seal ma õppisingi musta 
kohvi jooma ja suitsu tegema, mis on mind tükati mitmeid aastaid 
saatnud. Sõltuvust mul ei ole, võin jätta suitsetamise 
maha niimoodi, et panen paki lauaservale, tikutopsi peale ja seal see 
seisab, mind see ei häiri. 

\question{Kuidas kontorite vahel side käis?}

Side käis kahtemoodi: kas ümbrikuga\sidenote{Tõenäoliselt peab Priit silmas tavalist postiteenust.} või läbi teletaibi kanali. Igas EPT 
kontoris oli teletaibi aparaat. Kirjutusmasinaga teletaip oli teksti edastamiseks ja selle küljes oli ka perfolindi lugeja. Meie 
masinast lasti perfolint välja ja söödeti teletaipi ning teisel pool, näiteks 
Paides, lasti lint välja ja söödeti sealsesse masinasse. Aga ega see nii 
lihtsalt ei käinud, seal oli protokoll ka, sest side ei olnud püsiv 
ja kippus ära kukkuma. Siis tegi teletaip piiksu. Selle 
peale tõstis inimene telefoni, helistas teisele osapoolele ja ütles: 
\enquote{Kuule, tõstan kümme kirjet tagasi.} Teisel pool tõstis operaator õla 
üles ja sättis perfolindi tagasi, vastuvõtja tõmbas lindile poole pastakaga 
joone. Seda võis mitu korda juhtuda. Ja kui lint oli lõpuni jõudnud, siis 
vastuvõtja lõikas lindid märgitud kohast katki, võrdles, kus asi kokku langeb, 
liimis otsad kokku (spetsiaalne rakis oli, millega augud läbi torgati, et need 
puhtad oleksid) ja söötis lindi masinasse. 

\question{Ühesõnaga elektrooniline seade muutis andmed kõigepealt 
pabermeediasse, siis elektroonilisse meediasse, siis uuesti pabermeediasse ja 
lõpuks tagasi elektroonilise meediasse. Ja veaparandus oli manuaalne!}

Jah. Tekkis neli koopiat linte: üks, mis lasti siitpoolt välja ja võeti sealpool 
vastu, ning teine, mis lasti seal sisse ja võeti siin vastu. 
Töötas suurepäraselt. Aga siis õpetas Leo Võhandu\index[ppl]{Võhandu, 
Leo} mulle andmeedastust ja protokolle. Läksin meie keskuse peainseneri Rolandi 
juurde ja ütlesin: \enquote{Roland, kas saad mulle 
sellise seadme teha, mis paneb selle arvuti ja selle teletaibi kokku, ja 
teisele poole samasuguse vastuvõtmiseks?} Ma ei teadnud, et see on modem, siis 
ei olnud sellist asja olemas. Tema ütles: \enquote{Oot, ma vaatan, ma just Radio ajakirjas (tolleaegne tehniliste nikerdajate ajakiri) nägin ühte skeemi.} Ja tegigi valmis. Montaažplaat oli 50 x 50 
cm, kuna see oli SMi sisemine plaat, läks nagu riiul 
sisse. Aga skeem oli nurga peal 10 x 10 cm. Ta pani selle 
käima ja mina otsisin vahepeal mööda opsüsteemi, mis võimalused on. Leidsin 
ühe struktureerimata ala -- eraldad lihtsalt mälu ja 
struktureerid ära. Ehitasin sinna peale kataloogisüsteemi ja kirjutasin linte väljastavad
programmid niimoodi ringi, et need kirjutasid 
kataloogisüsteemi, mitte ei saatnud, ja ka seda, kellele saata. 
Esialgu ei olnud mujale saata kui Paidesse, kuigi perfektsionist kirjutab ikkagi adressaadi ka juurde, 
sest mine tea, kellele on veel vaja saata. 
Üks programm vaatas aeg-ajalt, kas on linte tekkinud, ja helistas teise poole välja. Kui side 
kukkus, siis tõstis 10 kirjet tagasi, teine teadis seda ja hakkas uuesti saatma. Probleem oli selles, et sa ei teadnud, millisel 
hetkel side kukkus ehk ei olnud võimalik määrata, mis kirjed olid ära läinud. 
Üks saatis minema, teine võttis vastu ja võrdles, kus hakkasid samasugused kirjed 
tulema, ning loksutas paika. Asjad läksid täitsa ilusti üle. Teinekord kui side ei taastunud, siis võis seanss olla terve päev katki.

\question{Mis tempos see andmeside toimus?}

See võis olla 100--300 bitti sekundis, kiirem küll olla ei saanud. 

Side kiirenes siis, kui läksime elektroonseks. Kõik läks hästi niikaua, 
kuni Tartu ütles, et nemad tahavad ka. Ka nüüd läks kõik esialgu 
hästi, kuni juhtus niimoodi, et Paide helistas mulle peale, side kukkus ja 
siis helistas Tartu peale. Aga ma ei teadnud, et see on Tartu. Üritasin vastu 
võtma hakata, aga sealt ei tulnud midagi tuttavat. Siis mõtlesime välja sessioonivõtme ehk ühe \emph{hash}'i. Enne seansi
tekitamist saatis programm \emph{hash}'i ette ja nüüd oli teada, et 
kui see uuesti tuleb, siis sellesama \emph{hash}'iga, nii et võis ka segamini 
vastu võtta. 

\question{TCP\sidenote{Üks interneti alusprotokolle.} käib põhimõtteliselt samamoodi.}

See on jah vahva, et tunned pidevalt asju ära. 

Ühesõnaga, ma töötasin seal kuni TPI lõpuni, kokku neli aastat ja täitsa huvitav oli.

\question{Kas sa lõpetasid ülikooli töö kõrvalt nominaalajaga? Töö koolis käimist ei seganud?}

Ei! Tööst oli palju kasu, ma olin teistest kogu aeg peajagu üle, sest olin 
saanud kõike seda, mis meile õpetati, elus katsetada. EPT seltskond lubas 
mul väga vabalt toimetada ja ütles, et kasuta kõike, mida saad, peaasi 
et on hea. 

Me tegime Eesti esimese \emph{online}-messi 
arvutitega ja vedasime ise pool kilomeetrit kaablit posti otsas Saue mõisa. Tol 
ajal müüdi hooaja lõpus kõik varuosad maha ja see käis tavaliselt 
niimoodi, et kaubatundjad istusid suurte paberite taga ja tõmbasid maha, mis 
müüdud sai, aga meie vedasime Saue mõisa side ja panime viis kuvarit üles. Peainsenerid ja kolhoosiesimehed said ise 
arvutist valida ja asju selekteerida. Töötas suurepäraselt. Meil oli väga innovatiivne 
kamp.

EPTs oli üks tuntud mees, Tõnu Lume\index[ppl]{Lume, Tõnu}, kes mängis filmis 
Lurichit\sidenote{Tallinnfilmis 1984. aastal valminud film \enquote{Lurich}.} ja 
oli EPT arvutuskeskuse juhataja asetäitja. Juhataja oli Jaak Raja\index[ppl]{Raja, Jaak}, karm mees, aga minusse suhtus hästi. 

\question{Miks sa EPTsse pikemaks ei jäänud?}

Tuli kooli lõpp ja suunamine. Tänapäeval keegi ei teagi, mis on suunamine. 
Tol ajal tehti hinnete põhjal pingejärjekord ja said valida, kuhu lähed. Aga nimekirjas oli kaks kohta, mis 
olid spetsiaalselt mulle -- kui mina neid ei valinud, siis keegi teine neid 
ka valida ei saanud. Üks oli EPT ja teine TTÜ, mis oli sel 
ajal muutunud põnevaks kohaks, kuna sinna hakkas välismaist
tehnikalt tulema, sealhulgas personaalarvuteid. 

Käisime koos Sven 
Jürgensoniga\index[ppl]{Jürgenson, Sven} esimesi personaalarvuteid rongiga Moskvast toomas. Need olid kaheksabitised Yamahad, 
Z80 prosega.

\question{Mis ametikohta sulle pakuti?}

Juhtivinseneri kohta, see oli 
teadusliku uurimise sektori ametikoht infotehnoloogia kateedri 
all\index{Tallinna Tehnikaülikool!Infotehnoloogia kateeder}. Lisaks pakuti head 
palka, kuigi ega EPTs ka kehv palk ei olnud. Tollal oli hea kuupalk 
120 rubla, mina teenisin 155 rubla. Tegelikult aga läks elu 
kehvemaks, sest õppimise ajal elasin ma ikka väga priskelt: sain lisaks
60 rubla kõrgendatud stippi, EPTst poole koha eest 60 rubla ja teadusliku uurimise sektorile\index{Tallinna 
Tehnikaülikool!Teadusliku Uurimise Sektor} tehtud tööde eest 40 rubla. Lisaks maksis EPT 
60, vahel isegi 100 rubla kvartalipreemiat. Kokku mingi kakssada rubla kuus! Aga nüüd oli mu palk 155 
rubla. Mul läks tükk aega, enne kui kõik liinid tööle sain ja teadusliku uurimise 
sektor hakkas mulle lisa maksma. 

\question{Mida see teadusliku uurimise sektor endast kujutas?}

TPIs tehti kõik lepingulised tööd teadusliku uurimise 
sektori alt, kes sõlmis lepinguid ja võttis vahelt oma obroki. 

\question{Mida sealt telliti?}

Igasuguseid asju, näiteks kriminalistika infosüsteeme. 

Ma valisin TTÜ ja läksin Raja Jaagule\index[ppl]{Raja, Jaak} 
lahkumisavaldust viima. Tema ütles: \enquote{Priit, jää ikka
poole kohaga tööle. Sa ei pea kogu aeg käima, astu vahel läbi ja 
ütle, mis arvad.} Nii ma töötasingi kaks-kolm 
aastat veel seal. Käisin ikkagi kohal, sest niisama raha vastu võtta
südametunnistus ei lubanud. 
Täitsa huvitavaid asju sai veel tehtud. Aastaid hiljem, kui ma seal enam ei töötanud, olid SMi 
matused viina ja kartulisalatiga. Masin ise oli maha müüdud, aga 
protsessorikast maeti kuskile Sauele maha. 

\question{Sa oled ainus inimene, keda ma tean, kes on päriselt laulu sisse pandud.\sidenote[][]{Ansambli Folkmill 1996. aasta albumi \enquote{Paksult 
rahul} populaarses avaloos \enquote{Madis Mäekalle valss} on salm: \\
Üks talv oli see, jube libe oli tee,\\
Madis mütaki istuli kukkus.\\
Aga igav oli maas, seltsiks vaid kaevukaas,\\
Madis ohkas ja tudile tukkus.\\
Siis ühmatas Raspeli Priidu,\\
kes kunagi ei kiskund riidu:\\
\enquote{Sa aja end, Madis, nüüd püsti\\
ja tunne end pagana hästi.}} Kuidas sa sinna sattusid?}

Lauri Saatpalu\index[ppl]{Saatpalu, Lauri}\sidenote{Folkmilli 
laulja ja käilakuju.} on minu hea sõber ja tal on niisugune komme, et kui tal 
millestki muust enam laule pole kirjutada, siis ta hakkab sõpradest kirjutama. 

\question{Kust sa teda tead?}

Käisime Lauriga EÜEs\index{Eesti Üliõpilaste 
Ehitusmalev}\sidenote[][]{Tagantjärele vaadates nõukogude aega oma vaimsuse, 
suhtumise ja ärimudeliga hämmastavalt halvasti sobitunud, tudengite jaoks 
organiseeritud suvise töö tegemise vorm. EÜE organisaatoritest, legendaarsetest 
trubaduuridest, sõpruskondadest ja suhetest on hiljem nii mitmeski 
valdkonnas suuri asju võrsunud.} ja oleme koos mitmeid laule teinud. 
Üldiselt tulevad tal sõnad hästi, aga on ka juhtunud, et ei tule, ja siis ma olen 
katalüsaatorina töötanud. Olen ise ka maleva jaoks laulusõnu teinud. 

Lauriga kohtusime esimesel Tiirimetsa suvel, aastat ei mäleta. Hakkasime kohe hästi läbi saama, 
ta on vaimukas inimene. Serbati Tom ja 
Mäekalle\sidenote{Tegelased viidatud laulus.} on kõik reaalsed 
inimesed.

\question{Nii et sa oled muusikamees ka?}

Jah. Ma olen õppinud muusikat päris palju, alustasin kuueaastaselt ja 
õppisin suisa neli aastat muusikakeskkoolis\index{Tallinna Muusikakeskkool}, 
aga siis sain aru, et ma ei peaks seal olema. Seal midagi muud ei õpetatud, aga mul olid muud asjad ka tähtsad. 
Pealegi olen ma natuke rutiinitalumatu nagu infotehnoloogid ikka -- kogu aeg peab \emph{action} käima, sama pala kaheksakümnendat 
korda mängida oli piinav. Tegelikult ma ei tahtnud seda katki jätta, aga seal oli üks solfiõpetaja, kes mind terroriseeris. Ta on 
kõiki terroriseerinud, aga mina olin tal eriline lemmik ja see lõi mu lukku, ma 
ei saanud solfiga hakkama. Ja tulingi ära. Ütlesin emale, et lähen hoopis
laste muusikakooli\index{Tallinna Lastemuusikakool} klarnetit 
õppima. Seal sattusin Aleksander Rjabovi\index[ppl]{Rjabov, Aleksander} 
juurde, kes on Eesti džässi suurkuju ja väga hea õpetaja. Solfiõpetaja
Porrason oli ka kuldne inimene. 
Selgus, et mul on kõik oskused olemas, ainult et need olid lukus. Mul 
on absoluutne kuulmine, mitte küll kõige kõrgemal, aga täiesti arvestataval 
tasemel. See on elus ka veidi piinarikas -- nii kui midagi 
valesti kõlab, siis kohe kratsib. 

Õppisin ka laulmist, nii et poistekoor pluss eraldi ansamblitunnid andsid kõva laulmiskooli. Hiljem õppisin ise saksi ja kitarri 
juurde, klaverit ka natuke. Ma ei ole ammu mänginud, aga klarnet ja saks on 
nii käes, et need tuleb ainult kastist välja võtta. Mul on 
kapis üks siinkandi paremaid klarneteid. Iseenesest on kahju, et see minu käes on, aga kuna see on kingitud pill, 
siis ei saa seda ära anda. Selle kinkis üks 
Eesti välishelilooja sünninimega Elmar Rossman\index[ppl]{Rossman, 
Elmar}, kes on Priit Ardna\index[ppl]{Ardna, 
Priit|see{Rossman, Elmar}} nime all kirjutanud \enquote{Kuldrannakese}. Käisime nädalavahetusel just Ugalas, kus ajaloomuuseumis
on tema ooperi reklaam seina peal. Väärt inimesed on elust läbi 
käinud.

\question{Tuleme korraks tagasi tehnikaülikooli juurde.}
 
Tehnikaülikoolis sattusin 
Toomas Mikli\index[ppl]{Mikli, Toomas} juurde, kellega saime väga hästi 
läbi. Ta oli väga keeruline tüüp, temaga oli raske rääkida. 
Seda suutsid suhteliselt vähesed inimesed, sest ta jättis umbes kolm loogilist 
taset vahele ja alustas neljandalt ning sa pidid ise puuduvad kihid vahele 
ehitama ja mina suutsin seda. 
Tema suutis panna mind andmebaasidest innustuma ja oli 
mu diplomitöö juhendaja. Diplomitöö oli meil muide 300 lehekülge. 

Natuke uhkustan ka, et aastal 1984, kui keegi selle peale veel ei mõelnud, oli mul üks osa
diplomitööst pühendatud konsultatiivinfole ehk \emph{help}-tekstidele. 
Tegelesin tööl metoodilise palgaarvestusega, kus muu 
hulgas õpetasin ja juhendasin kasutajat, mismoodi süsteem töötab. 

\question{Täitsa innovatiivne mõte tol ajal!}

Tollal jah keegi sellest veel suurt ei rääkinud. Korjasin selle teema
Tomiga vestluse käigus üles ja tegin ära. 
Diplomitöö kirjutamise käigus sain veel ühe asjaga hakkama. TTÜs kasutati 
SETORi\sidenote{Varastel kaheksakümnendatel liikvele läinud TOTALi 
andmebaasisüsteemi kloon ESide jaoks.} ehk andmebaasi, mida ülejäänud 
maailm tunneb nimega Total\sidenote{Ka TOTAL. 1968. aastal asutatud Cincom 
Systems Inci andmebaasimootor, mis oli esimene omasuguste seas.}. Arvutitel 
oli mälu vähe, 256 KB, millest 16 kilo jäi puhvrisse, kui kuvarid taga 
olid, ja nüüd ei mahtunud enam kompilaatorid ja linkurid mällu ära. Diplomitöö 
käigus kirjutasin skripti, mis vaatas programmis järele, milliseid teeke 
vaja on, ja linkis ainult need teegi osad sinna külge, mida tõepoolest vaja 
oli. Sedasi oli võimalik 16 KBga hakkama saada. Veel mõtlesin välja puhverdamissüsteemi, kuidas läbi puhvri erinevaid mooduleid siduda, sest 
suur tükk ei mahtunud korraga mällu. 

Tom pani kokku grupi, kuhu kuulusin mina, Mart Roost\index[ppl]{Roost, Mart} 
(praegu tunnustatud õppejõud), Lea 
Elmik\index[ppl]{Elmik, Lea} ja Tiiu Lumberg\index[ppl]{Lumberg, Tiiu}. Meid hakati kutsuma \enquote{Mikli noorteks 
ekstremistideks}. Me kõik kirjutasime oma teadustööd, aga me ei teinud 
kunagi midagi nii nagu teised. 

Moskvas oli üks kaval juut Tjomov, kes istus kuskil instituudis 
Iskra-226\index{Iskra!Iskra-226} peal, mis oli laetava BASICuga arvuti, ja
kirjutas opsüsteemi Skoropis ehk kiirkiri. See oli esimene viitadega keel, 
mida ma nägin. Tal oli \emph{time sharing} ilusti sisse ehitatud. Programmi täitmine 
käis nii, et tõmbasid programmi stringi, panid viida peale ja ütlesid, et 
selle viida järgi hakkad nüüd täitma. Mälu oli jälle vähe, 64 KB, aga meil 
tekkis Lindre Reinuga\index[ppl]{Lindre, Rein}, kellega me 
arvutisaalis tuttavaks saime, mõte panna Iskrale veel kaks kuvarit külge. Mina 
kirjutasin opsüsteemi ringi, tema tegi kaks videokaarti, panime Videotoni kuvarid 
taha ja vaatasime, kas hakkab tööle. Selleks ma tegin 
\emph{overlapping}'u -- kui tundsin ära, et programm on 
juba mälus, siis lisasin teise viida ja panin selle veel kord tööle. Programm visati 
välja alles siis, kui viitasid enam ei olnud. 

\question{Kas mälukaitse või turve ei olnud probleem?}

Ei. Kogu infoturve seisnes selles, et masinat ei saanud käimagi, flopi oli välja 
võetud ja tuba käis lukku. Kuigi oli olemas ka kahemegane ketas, mis nägi välja nagu
suur valge \emph{baraban} -- mul on praegugi kapi otsas kaks tükki, üks 
Iskra ja teine SMi oma.

Mul on seal kapi otsas terve muuseum, näiteks lint ja kolmesajane 
modem (nimega Nightingale, laksutab nagu ööbik) ja üks 
esimesi läpakaid, mis Eestisse tuli ja mis oli Siim Kallase\index[ppl]{Kallas, Siim} oma, 
kui ta oli Eesti Panga president. Lisaks arvelaud, lükati, kaheksa-, viie- ja kolmetollised 
flopid, magnetoptilised kettad -- 
ühesõnaga kõik, mis elus ette on jäänud. Kõige vanem eksemplar, taskukalkulaator, on pärit 
aastast 1936.

\question{Kas Feliks?\sidenote{Nõukogude Liidus aastatel 1920--1970 toodetud 
mehaaniliste kalkulaatorite sari, mille tootmise algatas Nõukogude 
julgeolekuteenistuse asutaja Feliks Edmundovit{\v s} Dzer{\v z}inski, mistõttu laienes 
tema hüüdnimi Raudne Feliks ka kalkulaatoritele.}}

Feliks on ka. Aga see kõige vanem on sakslaste tehtud mehaaniline numbrinäiduga taskukalkulaator, mis liidab ja 
lahutab ning on umbes 3 mm paks ja 6 x 10 cm suur. Vanaisa kinkis selle isale kuuendaks 
sünnipäevaks ja isa kinkis mulle. 

\question{Kas sa tehnikaülikoolis teadust ka tegid?}

Jaa, ma hakkasin tegelema andmeedastusega, aga tulid segased ajad, raha sai 
otsa ja see jäi seisma. Tegelesin sünkronisatsioonimudelitega, millega olen 
elus hiljemgi väga palju tegelenud, ja praegu võiks nendest kirjutada sellise
töö, mida keegi pole kunagi välja mõelnud. Aga nüüd on mul muud huvid tekkinud\ldots 

\question{Mis on sünkronisatsioonimudelid?}

Nende põhimõte seisneb selles, et kui on kaks infosüsteemi, siis millist mudelit 
kasutada, et kõige odavamalt välja tulla, ja mismoodi see automaatselt käima 
saada, et nad süngis oleksid. Tollal ma mõtlesin välja ühe termini, mida on hakatud minu suureks rõõmuks tänapäeval kasutama -- 
\enquote{automaagiline}. Kasutasin seda kunagi ühel konverentsil ja Jaak Tepandi\index[ppl]{Tepandi, Jaak} küsis, mida ma selle all mõtlen. See on asi, mis muutub automaatselt, aga ma ei 
tea täpselt, mismoodi. Ja nüüd kasutatakse seda reklaamideski. 

Meil tekkis 
Reinuga\index[ppl]{Lindre, Rein} oma rühm, sest Žiguli autotehas AutoVAZ soovis meilt süsteemi väikejaamade jaoks -- 
ladu, remont ja muu säärane. Ütlesime, et teeme küll, aga omamoodi, 
meil peab teadus sees olema. Kirjutasimegi 
neljakesi nullist süsteemi, mille loogikat poleks tänagi häbi näidata. Mart\index[ppl]{Roost, Mart} kirjutas 
andmebaasi mootori, Tiiu\index[ppl]{Lumberg, Tiiu} vormi generaatori, Lea\index[ppl]{Elmik, Lea} raporti generaatori ja 
mina süsteemi arhitektuuri kirjelduse ning mõtlesin välja ka
tolle aja mõistes XMLi. Suurem-väiksem märgi asemel olid kandilised sulud ja 
\emph{slash}'i asemel sõna \enquote{END}, aga keel oli sama. 
Tõestus on olemas ühes TPI kogumikus, kuhu ma kirjutasin selle kohta artikli. 

Mul oli väga lihtne tõsta asjad seal keeles ringi ja süsteem hakkaski 
teistmoodi menüüsid ehitama ning igasuguseid küsimusi küsima.

\question{Kas nad võtsid süsteemi kasutusele ka?}

Jah, me kasutasime seda AutoVAZi jaoks ja hiljem mujalgi.

\question{Kas ühel hetkel sukeldusid pangandusse?}

Selleni läks aega, enne toimus see
maailmameistrivõistlus. 

Ühel hetkel sai raha otsa ja palka sai 
TPIst\index{Tallinna Tehnikaülikool} nii palju, et kui auto oli olemas, 
jaksasid autoga tööl käimiseks bensiini osta. Sain tuttavaks niisuguse huvitava mehega nagu Veiko 
Herne\index[ppl]{Herne, Veiko}, kes praegu elab Euroopas nii-öelda kodutuna. Ta tahabki seda ja see 
ei tähenda, et ta halvasti elaks, vaid ta rändab ringi. Tema eluunistus oli olla vaba. Ja nüüd ta kirjutabki mobiiliäppe pargis. Kui kuskil on 
põllumajandusperiood, siis läheb põllumajandusse tööle ja aeg-ajalt paneb 
feissarisse, kus ta käinud on. Välimuselt on ta minu täielik vastand: 
pisike, kõhetu ja ümmarguste prillidega. Geniaalne programmeerija ja orgunnimeister. Ta kutsus mind tarkvara tegema oma loodud firmasse
OÜ Tarkvara ja andis kolmandiku osakuid mulle.

\question{Kui OÜ, siis pidi ajahetk olema 1991.}

Millalgi siis jah. Ühel päeval ütles ta: \enquote{Kuule, hakkame tõsiselt 
tegema -- ma leidsin Microsoft Magazini sabast ühe süsteemi nimega 
Gupta\index{Gupta} SQLBase\sidenote{Tegu oli esimese relatsioonilise kliendi-serveri 
andmebaasiga, mis jooksis PC platvormil, mitte mini{\-}arvutitel.}. 
Nad pakuvad, et hakkaksime nende esindajaks, ma käin korra Inglismaal.} Mina 
olin TPIst selleks ajaks otsad juba lahti võtnud. Üks asi, millega me raha 
teenisime, oli Robotroni nõelprinterite ümberprogrammeerimine eesti 
tähestiku peale. Sellest tekkis natukene 
algkapitali. Inglismaa-sõidu ja lansseerimise 
peale läks ilge raha, 100 000 rubla, aga kuidagi me selle kokku 
kraapisime. Igal juhul Veiks tuli Inglismaalt tagasi ja olimegi esimese kliendi-serveri süsteemi ametlikud esindajad Eestis. Siis ei 
olnud veel Oracle'it, Cybase'i ega kedagi. Korraldasime seminari ja tuli ainult 
vilistada -- terve Küberi amfiteater oli inimesti puupüsti täis. 

Tegime lepingu Põlva Piimaga ja Võrus oli 
eksperimentaalne õmblustootmiskoondis, kellega sõlmisime süsteemide 
arenduslepingud. Uurisime süsteemid välja ja panime andmebaasid käima. Põlva 
Piim oli väga suur projekt, seda me ei hallanud enam kolmekesi ära, nii et võtsime 
Andres Lombi\index[ppl]{Lomp, Andres} ja IE-tarkvara\index{IE-Tarkvara} appi 
programmeerima. 

Mul on õudselt hea nina igasuguste vigade peale. Leidsin SQLBase'ist ühe laheda vea, et kui tingimused
\verb|IN| ja \verb|NOT IN| olid järjest, siis täitusid suvalised 
tingimused. Ja kui panid sinna vahele \verb|1=1 AND|, siis hakkas tööle. 
Vennad ei uskunud seda ja kaks tükki tulid suisa kohale. Korraldasime ruttu seminari ja panime nad esinema. Näitasin neile enda tehtud asju ja kuidas me oleme nende 
süsteemi kasutanud. Saime nendega täitsa \enquote{kuuma liini}. Ükspäev teatas Veiko\index[ppl]{Herne, Veiko}, et 
Gupta\index{Gupta} otsib endale esindajat 
maailmameistrivõistlustele 4GL programmeerimises ja kas lähme. Guptast öeldi, et te 
olete nii kõvad vennad küll, minge, aga ise peate oma arvutitega Rootsi jõudma. 

Oli selline väga tark soome poiss nagu Pauli Visuri\index[ppl]{Visuri, 
Pauli}, kes müüs Olivettisid. Tänu talle tõi
üks Rootsi Olivetti esindaja meile messiboksi tuttuued masinad, meie 
lihtsalt sõitsime lennukiga kohale. Tahtsime ööbida mingis tagasihoidlikus 
kohakeses, aga Gupta ütles, et ei, meie meeskond ööbib ainult Kung 
Carlis, maksame selle teile kinni. 

Läksime sinna ja nägime esimest korda elus 66 MHz Suprema masinaid, millel oli peal Plug-n-Play, Windows 3.11. Hakkasime installima, aga ei õnnestunud, hiir ei läinud külge. Arvasin, 
et seal on Microsofti meeskond, panin käed puusa ja teatasin: 
\enquote{See teie opsüsteem on igavene pask! \emph{Plug and play} küll, aga hiired külge ei 
lähe!} Kaks venda istusid meie masina taha ja hea oli vaadata, kuidas 
ini-failid\sidenote{.INI laiendiga failides hoiti Windowsi platvormil 
tavakohaselt programmide konfiguratsiooni.} lendasid näppude alt 
välja. Lasid-lasid ja üks masin läks käima. Ajasin nad minema, 
kopeerisin ini-failid kõikidesse masinatesse ja oligi korras. Veiko hakkas 
proovima häältuvastust, mis oli just välja tulnud, aga kuna messihalli helifoon oli 
väga kõva, siis ta karjus oma arvuti peale: \enquote{Õupen, õupen, õupen, 
klõus, klõus, klõus, ran!} Järsku kostis teiselt poolt seina: 
\enquote{\emph{Clear all!}} Küllap ta käis kellelegi närvidele. 

\question{Mis ülesannet te lahendasite?}

Ülesanne oli vahva, umbes selline, et kass ärkas, sirutas, hüppas ja sattus klaviatuurile. Arvuti tegi 
piiks, kass tegi näu ja selle peale ärkas üles tema perenaine Celia, kes 
mõtles, et täna on kolmapäev -- mida ma olen tellinud, mis kaubad peaksid täna 
tulema ja mis mul veel tellida oleks vaja? Siis räägiti Peterist, kes istub 
kesklaos ja paneb kaupu liini peale, ning Larryst, kes sõidab 
\emph{lorry}'ga ringi ja veab kaupu laiali. Klassikaline 
veebikaubanduse logistika, mida tollal ajal veel ei olnud. Meile anti 
ette kaart ja GPS-signaal ning pidime programmeerime auto armatuurlaua 
koos GPSi liigutamisega ja tsentrumi. Meie lahendus erines teiste omast, sest ma ei 
viitsinud seda igavat ladu programmeerida ja tegin keskele 
logistikakeskuse, kus laod olid eraldi. Ladusid imiteerisime omaette failidest 
ja Peter võttis lihtsalt tellimusi vastu ja jagas laiali. 
Pärast messi lõpetamisel istusime žüriiga ühes lauas ja nad ütlesid, et kurat, mingid postsotsialistlikud vennad tulevad meile kapitalismi 
õpetama! 

Võistlus kestis 24 tundi: algas ühel päeval kell kolm ja lõppes teisel päeval 
kell kolm, seejärel hakkasid järjest esitlused tulema. Meie saime esitlema 
alles kell kümme õhtul, kui olime 24 tundi üleval olnud. Mina kirjutasin kogu koodi ja 
projekteerisin peas asjad. Veiko, hea suhtleja, süstematiseeris mu küsimused ja tõi 
žürii käest vastused. Tiiu joonistas vorme ja tegeles kogu kasutajaliidesega. 

Alustasin 15:20 ja kell viis öösel läksid näpud krampi -- umbes pool tundi 
ei liikunud, siis läks uuesti lahti. Korraks tekkis psühholoogiline tõrge, aga siis tegime edasi, kaks tundi enne tähtaega saime 
valmis. Hommikul kell kaheksa tekkis uuesti jama tunne, kui üks
Maci meeskond juba lõpetas. Mõtlesin, et olen ikka jube sant mees. Lõpuks 
selgus, et nad katkestasid.

\question{Kes sellist üritust korraldas?}

Täpselt ei mäleta, aga üks rootslaste softiliit. 

Ma teadsin sellest tänu ühele sõbrale, Tartu EPT juhile Kalle Kullmanile\index[ppl]{Kullman, Kalle}, kes oli seal kunagi ülesande tegijana osalenud. Temaga saime 
kokku marksistliku-leninliku kommunismi kandidaadimiinimumi täiendusloengus, kus me istusime kõrvuti. Loengut 
luges Otto Stein\index[ppl]{Stein, Otto}, keda kutsuti Otto 
von Steiniks. Ta oli saadetud Tartust Tallinnasse kommunistlike filosoofide 
kaadri tugevdamiseks, mispeale nii kaader kui seltsimees Stein tugevnesid. Hull vanamees oli.

Mäletan siiamaani, et loeng oli \enquote{Kommunismi on kolm allikat, kolm 
komponenti}. Need on inglise poliitökonoomia, saksa utopism ja \ldots\sidenote[][-.7cm]{Kommunismi kolm allikat 
toonase õppe järgi olid saksa klassikaline filosoofia (peamiselt Georg Wilhelm 
Friedrich Hegeli ja Ludwig Feuerbachi järgi), inglise poliitiline ökonoomia 
(Adam Smith, David Ricardo) ja prantsuse utopistlik sotsialism (Claude Henri de 
Saint-Simon ja Charles Fourier). Neid \enquote{arendasid edasi} 
marksismi-leninismi kolm komponenti: dialektiline ja ajalooline materialism, 
poliitiline ökonoomia ja teaduslik kommunism.}. Stein läks esimese tudengi juurde: 
\enquote{Öelge esimene, nii, õige.} Siis teise juurde: \enquote{Öelge teine.} 
Kolmanda juurde: \enquote{Öelge kolmas.} Ja nii edasi ühe inimese juurest teise juurde: 
\enquote{Teine, esimene, kolmas, teine.} Kui tiir minuni jõudis, tõusin püsti 
ja ütlesin: \enquote{Teate, mina selles tsirkuses ei osale!} ja jalutasin välja. 
Kui Stein püüdis asja leevendada ja ütles Kallele, et no öelge siis 
teie, teatas Kalle: \enquote{Mina ka mitte!} ja tõusis samuti püsti. Läksime 
välja, istusime Tuljaku baari maha, ajasime juttu ja oleme siiamaani suured 
sõbrad.

\question{Kuidas sa ikkagi panka sattusid?}

Olin nii-öelda vabakutseline häkker. 1991. aastal oli 
elutempo selline, et päevarütm oli täiesti sassis võrreldes teistega ja 
tööpäevad olid 72tunnised, pärast mida sõitsin autoga Valka ja Põlvasse asju 
üle andma. Mul oli siis juba kaks last, Anna oli just sündinud, ja selgus, et 
selline töörütm ei klapi enam. Seesama Tiiu\index[ppl]{Lumberg, Tiiu}, 
kellega käisime maailmameistrivõistlustel, rääkis, et 
Innovatsioonipank\index{Innovatsioonipank}\sidenote{18. 
septembril 1989. aastal ENSV Ministrite Nõukogu presiidiumi otsusega number 21 
Eesti NSV riigieelarve \enquote{üle plaani laekunud tulude} arvel asutatud 
pank.} otsib IT-juhti. Seda panka juhtis Peep 
Sillandi\index[ppl]{Sillandi, Peep}, pärastine mikro- ja makroökonoomika 
õppejõud EBSis, kes õpetas tudengeid softi peal mudeleid koostama. 
Peep oli lahe kuju, meil jutt klappis kohe ja lõpuks ta küsis: \enquote{Homme siis tuled 
või? Näe, tool on siin.} Mõtlesin, et olgu peale. 

Ja nii saigi minust IT-juht. Hommikul tuli minna poole üheksaks tööle, harjuda ära
sellega, et ei saa kella kolmeni öösel üleval olla. Ma ei ole sellega siiamaani 
harjunud, lähen üsna tihti praegugi öösel kell kaks magama ja ärkan kell 
seitse. Viis tundi on minu jaoks \emph{enough}. 

\question{Mida kujutas endast 1991. aastal väikepanga IT-juhi töö?}

Igasuguseid asju. Kui esimene päev uksest sisse tulin, istusin maha ja mõtlesin, kuidas mind on võimalik 
vangi panna (ma olen muuseas seda 
pärast igas ettevõttes teinud). 

Hakkasin seda kohta otsima ja leidsingi. Tollal käis keskpangaga 
infovahetus programmiga, mille kirjutas väidetavalt üks
armeenlane, kes oli kõik juuksed peast ajanud, et kammimise peale aega ei 
läheks, ja kuna suhkur on ajutoit, sõi ainult suhkrut -- geniaalne vend! Ta oli 
teinud nii käsuliidese kui ka dialoogiga käiva suhteliselt pisikese programmi, mis natuke 
krüptis ka -- küll väga vähe, aga tolle aja kohta ilmselt kõvasti -- ja saatis maksed 
panka ära. Iga päev tehti pangas kaks faili, üks hommikul ja teine õhtul. 
Õhtuses failis olid hommikuni käibed, mis tuli keskpanka saata, ja teises oli 
teistpidi. Igal pangal oli oma aeg, millal ta pidi failid ära saatma, ja samal 
ajal sai teistest pankadest tulnud asjad vastu. Kogu see asi seisis vabalt. 

Sain kohe aru, et panka saab röövida niimoodi, et ma ei võta kellegi kontolt 
raha ära, vaid tekitan sellesse kanalisse raha juurde. Siis ei hakka keegi 
selle järele igatsema, natuke aega tuleb ainult nostro- ja vostro-kontode sisu 
varjata, et ei oleks näha, et seal on mingi jama tekkinud, aga sellega 
saab hakkama. 

Esimene asjana tegime nii, et sellesse kohta sai faile panna 
ainult üks konkreetne programm, teine sai faile võtta ja kui keegi sellesse 
piirkonda sisse logis, katkestati kõik ära. 

Teine jama oli 
pangakontorite vaheliste ühendustega, näiteks kuidas saada 
Mustamäele kontor püsti nii, et see meie süsteemiga kokku saaks. Ehitasime ja 
testisime mingeid seadmeid. Telefonikanal ju ei 
püsinud.

Juhtus ka triviaalsemaid asju. Ühel päeval tuli teller minu juurde ja ütles: 
\enquote{Kuule, Priit, klient küsib oma käivet sellest ajast, aga seda ei ole.} 
Uurima hakates selgus, et süsteem oli üles ehitatud niimoodi, et kaks aastat vanad 
käibed hävitati ilma küsimata ja pikemat aega panna ei saanudki. 

\question{Mis too panga tuum oli, mis niimoodi tegi?}

See oli Midas Kapiti süsteem Kapiti, mis istus AS/400\index{AS/400} otsas. 
Aga kuna meil AS/400 ei olnud, siis meil käis OS/2 Warp\index{OS/2!OS/2 
Warp}\sidenote{OS/2 oli IBMi arendatud personaalarvutite 
operatsioonisüsteem. OS/2 Warp oli selle kolmas versioon, mis tuli turule 1994. 
aastal.}, mille peal istus AS/400 emulaator ja mille sees käis panga tuum. 

\question{Kas see oli kuskilt ostetud?}

Jaa, Kapiti käest, Midaseks muutus see hiljem. Ja mida 
Priit tegi? Istus maha, poisid hakkasid sortima \emph{backup}'e (neid me 
tegime hoolega) ja kirjutasime sellise softi, mis lappas \emph{backup}'idest SQLBase'i 
peale kokku andmelao. Me siis ei teadnud, et see asi on andmeladu. Kui ma panka läksin, lasin SQLBase'i osta, sest see oli hea 
kliendi-serveri lahendus, lihtsasti kättesaadav ja ei olnud väga kallis võrreldes 
teistega. 

Lisaks olin panga nõukogu sekretär. Peep arvas, et ma oskan 
piisavalt loetavalt kirjutada, pärast puhtaks lüüa ja saan asjast aru 
ka. Ütles veel, et ega sa liige ole, aga arvamuse saad ikka sekka 
öelda. Nii et olin nõukogus hääleõiguseta arvamusliider.

\question{See oli ju IT-juhile väga praktiline koht, said info kätte!}

Just nimelt, see oligi Peebu mõte ja jube hea mõte. Nõukogus 
olid väga vahvad liikmed, kellega ma tuttavaks sain. Näiteks Arvo 
Kallion\index[ppl]{Kallion, Arvo}, omaaegne
parteiboss ja valitsuses keegi, aga väga tark mees. 

Innovatsioonipank oli taskupank. Genin\index[ppl]{Genin, Alex}\sidenote{Alex 
Genin, Innovatsioonipanga nõukogu esimees.} oli niisugune juut Ameerikast, kes 
elas sellest, et tegi erinevates riikides panku, ajas nad riigi süül 
pankrotti ja siis hakkas kahjutasu nõudma. 
Sotsiaalpank\index{Sotsiaalpank} läks pankrotti, sealt pankrotipesast 
ostis ta kõige vingema kontori ning lasi panga põhja. Ma nägin ette, et see läheb nii. Selleks ajaks Peepu enam ei olnud 
sest Genin oli oma Miša (ma ei mäleta Mihhaili perekonnanime) panga etteotsa 
pannud. Tore poiss muidu, aga tegi täpselt, mis Genin ütles. Läksin Miša juurde, 
panin avalduse lauale ja ütlesin, et Miša, ma lähen nüüd ära. 
\enquote{Aga miks sa lähed?}  \enquote{See pank läheb varsti pankrotti.} 
\enquote{Sa eksid!}  \enquote{Ei eksi, Miša, ma olen majandusharidusega ja ma 
näen igal õhtul bilanssi, see pank läheb varsti pankrotti.} 

Mul oli on uus koht olemas, Tööstuspank\index{Tööstuspank}. Üks 
laenujuht läks sinna ja kutsus mind arendustiimi juhiks, kuna neil oli vaja uut 
infosüsteemi. Ma ei saanud aru, mis toimub: sain seal kõik, mida küsisin. Oma kontor ehitati Koplisse koos magamisruumi, köögi ja kõige muuga. Tirisin Innovatsioonipangast Eriku\index[ppl]{Matt, 
Erik}\sidenote{Erik Matt.} ja Raivo\index[ppl]{Tali, Raivo}\sidenote{Raivo 
Tali.} kaasa ja kuskilt tõin ära Ville Remmeri\index[ppl]{Remmer, 
Ville}.

Kuus kuud uurisime ja puurisime, ja kui meil oli kõik valmis, siis öeldi mulle: 
\enquote{Tead, me valetasime sulle. Me ei kutsunudki sind siia uut süsteemi 
tegema. Meil on siin IT-osakond, aga me ei usalda neid.} Seal IT-osakonnas 
oli igasugu tegelasi ja juht oli Poldzadze, kes nägi välja nagu Kirgiisi bai. Aga 
panga juhtkonnal oli vaja inimest, kes teaks, mis panga ITs toimub, sest nad 
hakkasid just Hoiupangaga kokku minema. 

Mulle öeldi: \enquote{Sa võtad nüüd IT juhtimise üle.} Sain endale kõige uhkema ametinimetuse, 
mis mul kunagi olnud on: panga esimehe 
volitatud eriesindaja IT küsimustes. Mul oli õigus käskida, puua ja lasta. 
IT-osakond oli sotsialistlik kamp, kellel oli vaja nimega ülemust. Mul oli õigus 
teha ükskõik mida, peaasi, et pank püsti püsiks ja midagi ära ei läheks. Ja 
ühel päeval läkski kuus ja pool miljonit krooni minema. Täpselt sedasama teed pidi, nagu 
ma arvasin. 

Sellel päeval, kui ma võimu võitsin, ei olnud mul veel võimalik midagi teha. 
Tõnu Liik\index[ppl]{Liik, Tõnu}, kes oli Hoiupanga IT-juht, tuli sinna, Ants 
Leitmäe\index[ppl]{Leitmäe, Ants} kaasas. Ants istus aknalauale ja kuulas 
tuima näoga pealt, tema pidi mind oma tiimiga tehniliselt toetama hakkama, sest ma 
ei teadnud kedagi usaldada. Siis kutsuti kõik kokku ja Tõnu teatas, et 
uus juht on nüüd siin. Ja kõik läks ilusti. 

Esimese asjana sai kõigil 
kasutajaõigused ära võetud, panin oma poisid masinate taha ja hakkasime uuesti 
õigusi õigetesse kohtadesse tagasi andma. Õhtul istusin pearaamatupidaja Irina juures (perekonnanime ei mäleta), kes toodi ka enne liitumist majja, ja ajasime 
juttu. Irina võttis lahti tagasi tulnud hommikuse kontrollsummade faili 
ja tegi istmelt meetrise hüppe üles. Kuus ja pool miljonit oli jagunenud pankade 
vahel teisiti, kui oli hommikul välja läinud -- raha oli läinud Rakvere Maapanka\index{Rakvere Maapank}. Egas midagi, 
joostes minema, kogu IT-osakond puhtaks ja arvutid tuli lahti jätta. Panime oma poisid 
peale ja hakkasime kontonumbri järgi \emph{search}'i tegema. Ja leidsimegi ühest 
masinast, kusjuures avastasime tänu sellele, et võtsin kõigil hommikul 
õigused ära. Tüübil oli kontrollsumma muutus ka tehtud, aga ta ei saanud 
õhtul enam vajalikule kohale ligi. Irina helistas kohe Rakverre, blokkis summa 
ära ja järgmisel päeval saime tagasi, sest õhtul olid pangad juba kella neljast kinni ja pangaautomaati 
ei olnud. Nad olid plaaninud tegutseda järgmisel hommikul. 

Vend võeti kinni ja viidi raudus minema. Masina panime raha transportimise kotti, pitseerisime 
kinni ja lukustasime seifi. Järgmisel päeval tegime manukate juuresolekul lahti, võtsime 
ketta välja ja tegime sellest kolm koopiat. Üks läks TTÜsse analüüsi, teine Hoiupanga 
tiimile ja kolmas minu tiimile. Seal masinas oli peale kontonumbri veel üks 
klimp, mis oli parooliga zipitud. Küsime venna käest: \enquote{Mis \emph{password} on?} -- \enquote{Ma ei 
tea, kui te mulle ütleksite, oleks mul endalgi huvitavaid asju seal sees.} 
Läksime kontorisse, et installida murdmisklaster. Siis läksime 
Raivoga\index[ppl]{Tali, Raivo} suitsu tegema ja tagasi tulles ütles 
Erik\index[ppl]{Matt, Erik}: \enquote{Poisid, vabandust, te oleksite ka 
kindlasti seda näha tahtnud. Ma lasin prooviks klastri käima, leidis, sõnastiku alguses, 
\emph{konjak} väikse tähega\ldots}. Raivo ütles selle peale: \enquote{Jube madal 
profiil, ma oleksin vähemalt Mercedes-Benz pannud.} 

Seal failis oligi kogu värk sees. Süsteem oli Foxis, andmebaas oli DBF, siis 
oli tehtud üks \emph{browse}, mida klikkisid ja mis jäeti meelde, ning lõpuks F10 
vajutades kanti kõik ühe konto peale kokku ja tehti fail valmis. Samuti
kontrollsumma fail, mis oleks õhtul lihtsalt õigesse kohta 
tõstetud. 

\question{Nii et sulle köögi ehitamine tasus kohe esimesel päeval ära?}

Ma ütlen sulle, et siin ei ole midagi oodata. Kõik juhtub kohe.

Hoiupank ostis Eesti Kindlustuse ära ja ma läksin sinna IT-juhiks. Seal oli ITs 
kaks ja pool meest, kellest ühte ma ei näinudki. See oli osakonna juhataja, kellel 
olid mingid probleemid, ja kui ta kuulis, et on uus IT-juht, siis ta ei 
tulnudki enam. Iseenesest geniaalne mees, mitmeid matemaatikaõpikuid 
kirjutanud. Võtsin oma poisid Ville 
Remmeri\index[ppl]{Remmer, Ville}, Erik Matti\index[ppl]{Matt, Erik} ja Raivo 
Tali\index[ppl]{Tali, Raivo} kaasa ja kolistasime sinna. 

Leidsime eest mingi õuduste maa. Katastroofi kuubis. Või okse kolme x-iga -- ma ei tea, kuidas seda kõige paremini
iseloomustada. Näiteks elukindlustus käis niimoodi, et paberi peal korjati dokumendid 
kokku ja need läksid kümne fakiirsisestaja kätte. Neil oli ekraani peal triip, 
kus olid postid vahel, ja triibu vahele sisestasid nad andmed, mille põhjal 
tehti reserviarvutusi ja kõike muud. Neil oli käigus postivõrk -- venelased kutsuvad seda
pastlavõrguks. Kõik raportid pandi posti, 
saadeti Tallinna, siin sisestati ära, trükiti raportid välja, pandi postikotti 
ja saadeti tagasi. Ma panin Ville kiiresti kirjutama lokaalset kasutajaliidest. 
Raivo ja Eerik hakkasid otsima võimalusi, kuidas teha ära side kõikide meie 
kontoritega. Ise kappasin esimestel nädalatel mööda kõiki esindusi, et uurida, mis 
probleemid seal on. Ühe tehnikutest võtsin kaasa, et ta vaataks tehnilist 
poolt. 

See kõik juhtus augustis. Mulle lubati kahe töökoha vahel viis päeva puhkust, 
nii kiire oli asjaga. Uue süsteemi tõmbasime käima jaanuaris. Selleks 
ajaks olid meil kõik ühendused tehtud, uus server olemas, 
soft töötas ja inimesed koolitatud. Iga agent hakkas ise oma asju sisestama.

\question{Mis aastal see oli?}

Umbes 1996. 

Nüüd tuli hakata ka ülejäänud süsteemi uuendama, aga võrk oli kohutav. 
Tellisime võrguehitustööd, ehitasime korraliku serveriruumi, panime seina
tulekindlad materjalid ja väljapoole jahutuse. Serveriruum asus maja keskel kõige paksemate 
müüride vahel, lasin sealt WC ja duširuumid koos torudega välja 
lõhkuda. Ja siis ühel päeval, vist märtsis, juhtus niisugune lugu, et kui 
hakkasime üle viima oma viimast serverit, siis
selgus, et \emph{backup} ei loe ja ketas läks nässu. \emph{Backup} oli meil 
korralikult tehtud, aga majas käis remont ja tolm oli selle ära rikkunud. 
Lindiseadmed olid tol ajal nii lollid, et ei näidanud seda. Ja kes see tol ajal ikka
\emph{backup}'e kontrollis -- kui kahes eksemplaris teha, siis oli ju piisav. Aga nüüd olid mõlemad tuksis. Seal peal olid raamatupidamisandmed ja me olime ju 
börsiettevõte. 

Egas midagi, Priit võttis ketta kaenlasse ja sõitis järgmisel päeval Inglismaale OnTracki, kes taastab kettaid. Väga kihvt firma, kõik maailma 
suured on nende kliendid, kaasa arvatud CIA ja KGB, aga vaevalt, et nad oma 
kettaid taastavad. Jõudsin kohale reedel ja tagasi tulin teisipäeva hommikul 
kahe kassetiga, kus olid kõik andmed peal. Kutid olid selle ajaga kõik 
valmis pannud ja nii kui ma lennukist maha astusin, võeti lindid, taastati 
ära ja asi läks käima.

Siis hakkas Hansapank Hoiupanka ära sööma. Tegelikult alguses oli ühinemine, 
pärast ülevõtmine. Mul ei ole selle kohta paberit, aga seest 
vaadates käis minu arvates asi niimoodi, et kõigepealt kuulutati välja ühinemine ja kui kõik 
oli teada, siis võeti lihtsalt üle. 

Tõnu Liik\index[ppl]{Liik, Tõnu} viis suurema osa ITst 
Hanschmidti\sidenote{Toonane legendaarne Ühispanga juhatuse esimees Ain Hanschmidt.} juurde ehk tollasesse Ühispanka\index{Ühispank}. Me tegime uut vinget 
infosüsteemi ka, aga see ei saanud valmis, sest kindlustus lõpetati ära. 

Elasin Mähel suvilas ja Tõnu helistas mulle ühel laupäeva hommikul. Tundsin Tõnu 
juba Tööstuspanga aegadest, kui tegime koos esimesi kaardiprojekte. 
Sulo Muldia\index[ppl]{Muldia, Sulo} Raepangast\index{Raepank} ja kes meil seal kõik olid, omaaegsed 
karismaatilised kujud erinevatest pankadest. Niisiis, Tõnu helistas mulle: \enquote{Kus sa oled? Ma tulen kohe sinu juurde.} Jõin parajasti aias kohvi ja ei jõudnud hommikumantlitki ära võtta, kui
Tõnu astus juba uksest: \enquote{Nüüd on sihuke värk, et 
ütle jah ja ma lähen kohe ära. Ega ma enne ei lähe ka. Tule, ma annan 
sulle uue kindlustuse, tee see valmis, mis tegemata jäi.} Pakkusin talle kohvi, 
jõime selle ära ja oligi kokku lepitud. Läksin SEBsse\index{SEB} kindlustuse 
arendusjuhiks. Tegime hea mudeliga elukindlustuse, mis oli 
selles mõttes märgiline süsteem, et see on \emph{proof of concept}. 
Mul on nimelt oma andmete modelleerimise teooria ja too süsteem on mudel, mis 
näitab, et see teooria töötab, sest need süsteemi osad, mis me siis tegime, on 
aastast 1999 samad. 

\question{See on kõige parem kvaliteedinäitaja, et asi peab kõigile muutustele 
vastu!}

Tegelikult on üks veel vahvam näide, aga see on varasemast ja ma ei teinud seda 
teadlikult. Aastal 1986 kirjutasin ma ühe palgasüsteemi ja müüsin seda ka 
mõnele, aga siis lõpetasin ära, sest see oli FoxPro ja musta ekraaniga. Ja aastal 1994 
helistati mulle ja öeldi, et \enquote{kuulge, see teie palgasüsteem \ldots} 
\enquote{Mul ei ole ühtegi palgasüsteemi.} -- \enquote{Mäletate, te müüsite selle meile aastal 1986, aga meil nüüd firma nimi muutus ja me ei oska seda ära vahetada. Proovisime
teisi süsteemi ka, aga seal on vead sees, oleme kõik muu suutnud selle järgi 
häälestada.} Sotsialismist tuli süsteem kapitalismi ja elas selle asja üle! 

\question{Mida sa praegu teed?}

SEBs\index{SEB} olin ma 17 aastat. Käisin küll 
vahepeal ära, kui mul oli üks kümnekuune huvitav periood. Tänapäeval nimetatakse 
neid \mbox{startup}'ideks, aga tol ajal me lihtsalt arvasime, et oleks tarvis teha üks 
produkt, mis õnnetuseks sattus IT-mulli lõhkemisega samale ajale ja me ei 
saanud enam riskirahasid peale. Me lõime sellist süsteemi, mis teeb kirjelduste 
pealt suvalisi dialooge, ühesõnaga salvestab andmed andmebaasi. 
Kirjeldused on olemas ja samast kirjeldusest võib teha \emph{voice}'i või 
mobiilirakenduse või ükskõik mida. Oma aja kohta 
oli see kõvasti ajast ees.

Sealt ma tulin tagasi SEBsse ja siis läksin RIAsse.\index{Riigi Infosüsteemi 
Amet}\sidenote{Riigi Infosüsteemi Amet.} RIAs juhtus niisugune lugu, et Katrin 
Reinhold\index[ppl]{Reinhold, Katrin}\sidenote{Priidu kolleeg Riigi 
Infosüsteemi Ametis.} tuli TEHIKusse\sidenote{Tervise ja Heaolu Infosüsteemide Keskus, sotsiaalministeeriumi 
IT-maja.} direktoriks. Katrin hakkas mind aeg-ajalt kutsuma arvamust avaldama 
ja kord, kui me siia alla kööki läksime, küsisin ta käest, et 
Katrin, sa vist lubasid Taimarile\index[ppl]{Peterkop, 
Taimar}\sidenote{Riigi Infosüsteemi Ameti peadirektor ajal, kui Katrin ja Priit 
seal töötasid.}, et sa ei võta kedagi kaasa. \enquote{Jah, ma lubasin.} -- 
\enquote{Aga kui ma ise küsin?} Ja ta vastas nagu Teele: \enquote{Ma 
mõtlesin, et sa ei küsigi!} See oli elu kõige lühem tööle värbamise vestlus. 

Tööle asudes oli meil analüüsi osakond, aga nüüd on selle 
nimi andmekorralduse ja andmeanalüüsi osakond. Selle all on kaks talitust. Üks 
on andmekorralduse talitus, kes teeb HL7\sidenote{Meditsiinis laialt 
kasutatav andmestandard.} standardi peale andmevorminguid, millega kogu 
tervishoiu infovahetus Eestis käib. Meil on väga hea ja detailse mõtlemisega arhitekt Andrus 
Tamboom\index[ppl]{Tamboom, Andrus}, kellele mina 
käin algul mõne asja välja ja tema mõtleb edasi ning joonistab lõpuni. Kahe analüütikuga ehitan üles 
analüüsikeskkonda ja palkan ka parasjagu andmelao talituse juhti, et kogu 
andmelaondus korralikult üles ehitada. 

\question{Järeldan kõige selle põhjal, et sul 
läheb hästi.}

Ei mul pole häda, väga huvitav on! Tegelen 
selliste asjadega, millega ma varem pole otseselt kokku puutunud, aga see valdkond muutub ka 
sellise kiirusega, et teinekord on raske kannul püsida. Kogu aeg 
lappan kohutavat kogust materjale läbi, kas on midagi uut. 

\question{Sul on vist kogu aeg nii olnud?}

Just nimelt, ma ei ole kunagi töötanud sellise ameti peal, kus mulle ei ole 
meeldinud. See on kõige olulisem. 

\chapter{Tõnis Reimo}
\index[ppl]{Reimo, Tõnis}

                 
\ldots see minu alguse lugu, nagu ma arvan paljudel, oli ammu enne 
BBS-indust. Ta on ikkagi otseselt  seotud sellega, et isa töötas tollal arvutite 
teemal ja vist oli Rahvusraamatukogu\index{Rahvusraamatukogu} (tollal  
Kreutzwaldi nimelise raamatukogu) mingi arvutus- või arenduskeskusega, mingi 
sellise imeasjaga, seotud. Sealt pääsesin arvutite ligi. 

\question{Mis vanuses see umbes oli?}

Ma arvan, et see võis olla mingi viies või kuues klass. Tollal mindi aasta 
hiljem kooli, nii et siis tänapäevase mõistes kuues-seitsmes klass. 

Sealt tekkis selline võimalus arvutitele üldse ligi pääseda ja loomulikult 
mängima saada, sest see oli tollal  ainuke asi, mis huvitas.

\question{Päriselt või?}

Ütleks niimoodi, et progemisest oli asi veel kaugel.

Mänge ju ei olnud. Kas olid malmist põranda külge needitud mänguasjad või siis 
need esimesed arvutimängud, mis  umbes nagu jooksid pigem trükimasinal kui 
arvutil. 

\question{Mis arvutid need olid?}

Eks sain natukene näpitud mingeid Jessukesi\index{Arvutid!ES EVM} ja sellist 
Vene toodangut, aga tekkisid mingid esmased personaalarvutid nagu 
Schneider\sidenote{Schneider oli kunagise arvutitootja Amstradi esindaja 
Saksamaal, Austrias ja \v{S}veitsis, kelle müügivõrku viimane kasutas ning kes 
Amstradi arvuteid ka mõnel määral oma turule kohaldas. 1988. aastast alates 
läksid ettevõtete teed lahku ja Schneider hakkas tootma oma PC arvuteid.}, 
Lääne-Saksamaa importtoodang.

\question{See oli siis mingi XT analoog?}

Isegi XT eelne. Aga tema peal oli juba võimalik  mingeid primitiivseid mänge 
mängida ja sealt vaikselt  see huvi arenes. 

\question{Mida nende arvutitega päris tööks tehti? Sina käisid mängimas, 
aga\ldots}

Milleks täiskasvanud neid rakendasid, sellest  ma alguses aru ei saanud ja 
väga ei huvitanud ka. Minu jaoks olid täiskasvanud lihtsalt tüliks, sest nad 
takistasid arvutisse saamist. See arusaamine, mida täiskasvanud arvutiga 
teevad,  tuli alles  aastaid hiljem, kui tulid ka esimesed katsetused 
progeda.

Sealt kasvas ka selline mingi laiem huvi välja, et sai liitutud  Jaak 
Loonde\index[ppl]{Loonde, Jaak} veetud arvutiklubiga 
Ahhaa\index{Arvutiklubi!Ahhaa}. Ahhaa, loodi minu arust kas 1985 või 1986, Jaak 
Loonde oli siis 3. Keskkooli\index{Koolid!Tallinna 3. Keskkool} legendaarne 
matemaatikaõpetaja. 

Ta vedas seda arvutiklubi alguses mingi nõukaaegse toodangu peal, mingi 
ES-i laadne masin, millel meie mõistes ekraani ei olnud, aga oli teletaip. 
Tekst trükiti paberile ja arvuti oma tulemuse trükkis ka paberile. 
Põhimõtteliselt oli trükimasin, kuhu lõid käsud, käsud trükiti sinnasamasse 
paberile, sinna sinu käskude alla trükkis masin oma vastused. Loomulikult olid ka
 mingid lindi pealt sisse lugemise võimalused.

Sealt läksime juba kiiresti üle esimestele Yamaha 
MSX-idele\index{Arvutid!Yamaha MSX} mis tulid. Sealt edasi juba Jukud ja 
Iskrad ja kogu see kloonindus.

\question{Kuidas sa sinna arvutiklubisse sattusid? Kas huvitas või mõni sõber 
kutsus?}

Ma isegi ei mäleta, kuidas. Võib-olla see asi, et ma esimesed kolm aastat oma 
elust käisin ise 3. Keskkoolis, seetõttu oli nagu mingi seos olemas. Võib-olla 
kooli kaudu teadsin Jaaku. 
                 
\question{Konteksti mõttes, aruvti-inimesed on tavaliselt \emph{sci-fi} sõbrad 
ja muud sellist, kas sul sihukest asja ka oli?}

Ikka, Soome televisioonist sai Battlestar Galactica-t vaadata.

\question{Originaalseeria, eks, kandiliste plekist robotitega!}

Jaa, Cylonid, tuli käis edasi-tagasi. Ja loomulikult kogu nõukaaegne 
\emph{sci-fi} kirjandus oli läbi töötatud, niipalju kui kätte ulatus.
      
\question{Mingit konkreetset eredamat asja oskad meenutada?}           

No ikka alustatud sai Maailm ja Mõnda\sidenote{Maailm ja Mõnda oli Eestis 
ilmunud raamatusari, mida algselt andis välja Eesti Riiklik Kirjastus, hiljem 
Eesti Raamat ja teised. Sari keskendus  peamiselt reisikirjadele ja 
loodusraamatutele.} sarjast, keerukam ja elegantsem osa kõik tuli kõik hiljem. 

Praegu ma panen puusalt, igasugused \enquote{Purpurpunaste pilvede 
maad}\sidenote{Seda raamatut meenutab ka Tarmo Mamers leheküljel 
\pageref{sisu:purpur}.} ja kõik sellised asjad. See oli nõukogude vaimustuses 
kantud \emph{sci-fi}, jutustas kuidas kommunistliku ühiskonna liikmed 
kangelaslikult kosmost vallutavad.

\question{Ega neid raamatuid ei olnud Eesti keeles palju saada, seega mingi 
seltskond luges väga paljus samu raamatuid ja saadi üksteisest paremini aru.}   

Jah, see oli üks asi, aga seltskond, kellega mina tollal kokku puutusin oli 
ikkagi suhteliselt piiratud. Omavanustest olid need ikkagi selliste inimeste 
lapsed, kes töötasid arvutitega. See tähendas seda, et nende vanemad olid kas 
KBFI-s, Tallinna Tehnikaülikoolis või mingites asutustes, kus oli arvuteid. 
Tehniline intelligents. Päris juhuslikku rahvast väga palju seltskonnas ei 
olnud.

Nii palju, kui mina nägin, siis enamus, nii 80\%, oli keskendunud arvutitega 
mängimisele. Sealt edasi tulid sellised praktilised probleemid, et kuidas mänge 
kopeerida ja kuidas mänge avada ja kuidas need üles otsida. Sealt tegid 
opsüsteemiga tutvust ja lõpuks siis tekkisid nagu esimesed huvid, et 
\enquote{aga kuidas neid ise teha?}
                 
\question{Kaua sa seal Ahhaa klubis käisid?}

Ahhaa klubil olid rohkem nagu üritused. Nii palju kui mina mäletan, et mingitel 
kindlatel päevadel oli sul kusagil ligipääs arvutitele. See \enquote{kusagil} 
oli mingi Tööjõureservide Õppekeskus, Tehnikaülikool, mingid sellised veidrad 
kohad.
Ja eks aja jooksul tuli neid kohti juurde, kellelgi jälle vanemad või sõbrad 
jälle sokutasid ja nii me rändtirtsudena lendasime peale vabale 
arvutusressursile.

\question{Kas need täiskasvanud inimesed, kes loodetavasti tões ja väes tööd 
üritasid teha, ei pahandanud?}
                 
Enamasti oli see tegevus ikkagi  töökeskkonnast nagu eraldatud. Ma väljaspool 
isa töökohta ei mäleta väga palju, et me oleksime otseselt tööruumides 
olnud. Mingit õppeklassid ja mingid sellised kohad. Loomulikult, hiljem 
vaadates, siis kindlasti sai kõvasti närvidele käidud isa kolleegidele. Sel 
moel, et nende arvutiterminale hõivatud siis, kui nad tahtsid tööd teha. Aga 
see oli nagu selline põnev mäng, et nad peitsid mängud ära ja siis meil oli 
jälle põhjust vaeva näha nende üles otsimiseks.

\question{Millest ma järeldan, et täiskasvanutel olid ka mängud kuskil seal 
masinates!}

Olid olid, ega ma ise neid sinna ei pannud. No mõningatesse panime. Aga  kui 
tollase Tallinna Linna Täitevkomitee (mis oli tollal üks isa töökohtadest) 
keldrisse tekkis UNIXi masin, siis sinna ikka ise midagi  ei kopeerinud. Masin 
oli  varustatud sellega, mis seal oli ja siis tuli seal kähku orienteeruda, 
\verb|su| käsud selgeks saada ja ruudud.

\question{Huvitav on see, et kui nad Eestisse jõudsid, olid UNIXi purgid pigem 
suletud ja kõik muud pigem sellised, kuhu sai ise asju sokutada.}

No ega sa Jessukesse ka kindlasti said sokutada, aga lihtsalt  programmeerimine 
ja programmi sokutamine olid väga tülikad, sest see käis perfokaartide kaudu.
                 
\question{Sa rääkisid, et su ja root said selgeks, kuidas nad selgeks said, 
kust see info tuli?}

Ma praegu loomulikult ei mäleta, tõenäoliselt ta pidi tekkima seeläbi, et 
vingusid oma mängu senikaua, kuni keegi sulle midagi ette näitas. Vaata, kui 
vanalt tänapäeval nutitelefon selgeks saadakse, enam-vähem samal ajal koos 
rääkimisega, et asi siis nüüd selle \emph{super user}-i kasutamine selgeks saada 
on. Oluliselt vanemana, kusjuures. Seda enam, et tollal  infoturbe teema oli 
suhteliselt olematu,  kõikide login oli eesnimi ja kõikide parool oli tema 
perekonnanimi.

\question{See võis konduktiivne küll olla ringi pusimisele. Ehk, kui tol hetkel 
oleks infoturve olnud paremini paigas, sisi terve põlvkond inimesi oleks 
sisuliselt ilma arvutita jäänud?}

Ma arvan, et mitte, sest tegelikult saadi masina ligi läbi vanemate ja eks sa 
ikka naaksud oma isa ja ema kallal senikaua, kuni ta selle mängu käima paneb. 
Eks siis vahepeal vaatad, jälgid, paned tähele, mida tehti, kuidas sai. Aga ma 
arvan, see tase oli erinev, sest mõned tundsid programmeerimise vastu nagu 
võib-olla põhjalikumalt huvi, mina  alguses vähem. Ausalt öeldes 
programmeerimine ei ole läbi elu mu tugevam külg olnud. Ma olen sellise müügi, 
turunduse, juhtimise, projektijuhtimise, tootejuhtimise kallakuga olnud.

\question{Teades, mis tooteid sa oled juhtinud, seda ju ei saa teha saamata 
väga hästi aru, mis kapoti all toimub?}

Jah. See huvi on ka loomulikult alati olnud, et kuidas asi töötab. Aga selleks 
ei pea alati ise tegema.
                 
\question{Kui sa said tonks vanemaks, siis mingil hetkel see külakorda käimine 
pidi ju muutuma?}

Ühiskond läks edasi, eks ole. Alguses tegutsesime  isa loodud Eesti-Rootsi 
ühisfirma tiiva all, kus olid personaalarvutid. 286, 386, hiljem 486. Ja sealt 
edasi siis lõime ka oma firma. Aga alguses sai selle ühisfirma ruumides ja tiiva 
all tegutsetud ja tegeletud arvutite maale toomisega. Sama eesmärk, et enamus 
aega läks mängimisele, aga oli ju vaja kusagilt saada seda riistvara, millega 
mängida. See oligi üks \emph{driver}. Ostad, mängid mõnda aega, müüd maha ja 
nii see äri käima läks. Hiljem võttis äri mängimise üle, sest et siis ei olnud 
enam aega mängimisega tegeleda, kogu aeg läks äri peale. 

\question{Mis võib olla nii positiivne, kui negatiivne. See oli pärast 
keskkooli?}

Jah, see on kusagil pärast tehnikumi. Ma lõpetasin Tallinna 
Polütehnikumi\index{Tallinna Polütehnikum} aastal 1990, 1991. aastal vist 
tegime siis oma HNS-i\index{HNS} nimelise firma mis omakorda tulenes juba enne 
seda loodud BBS-ist, mis kandis \emph{Hackers Night System}-i\index{BBS!Hackers 
Night System} nime. Mis omakorda ei tekkinud tühja koha peale, vaid tegelikult 
oli enne seda olemas Lembit Pirni\index[ppl]{Pirn, Lembit} Eesti BBS 
\#1\index{BBS!Eesti BBS \#1}. Lembit Pirn tegutses täna Tornimäel asuvas 
sellises madalas valges hoones. Seal oli tollal mingi transpordi informaatika 
keskus või midagi sellist. Ja temal oli esimene modemiga töötav BBS püsti 
pandud, seda sai kõvasti  külastatud. 

\question{Miks ta selle tegi ja miks seal oli vaja käia?}

Miks Lembit seda tegi, peab tema käest küsima. Seal oli väljas ühelt poolt 
mänge, teiselt poolt  oli mingi suhtluskeskkond, hakkas tekkima juba nagu 
selline \emph{bulletin board}. Ta läks käima tarkvaravahetuse pealt, mis tollal 
oli täiesti tavaline, tänapäevases mõistes räigelt illegaalne tegevus. Aga noh, 
nõukogude ajal ja üleminekuajal see mõiste oli võõras. 

Kuna ta vist oli ka Fidoneti liige, siis sealt sai infot ka teiste BBS-ide 
kohta maailmas. Sai hakatud Soomes BBS-e külastama, näiteks Jouni Salo 
BBS\index{BBS!Jouni Salo} ja Ron Dwightiga\index[ppl]{Dwight, Ron} suhtlema, kes  oli 
 Fidoneti Euroopa \emph{tsooni} pidaja. Eks nemad aitasid-juhendasid edasi 
ja sealt tekkis loomulikult mõte oma BBS püsti panna. Tänu sellele, et me saime 
isa firma ruumides tegutseda, oli meil unikaalne võimalus teha otse välismaale 
kaugekõnesid. See, mis on tänapäeval suhteliselt elementaarne, et sa kusagile 
otse helistad, ei olnud tollal isegi Eesti piires elementaarne. Kõik kõned 
tehti läbi keskjaama. Keskjaam oli  selline kindel number, kuhu sa helistasid, 
kus võttis keegi naisterahvas vastu,  kellele  teatasid, kuhu sa tahad 
helistada, lugesid oma numbri ette. Kui liin vabanes, siis ta  helistas sulle 
tagasi ja teatas, et nüüd on siis kõne. Aga kuna tegemist oli Eesti-Rootsi 
ühisfirmaga, siis oli seal unikaalne võimalus  helistada otse automaatvalimisega 
 maailmas igale poole. Ja see võimaldas ka modemiga enam-vähem 
üle kogu maailma helistada.

\question{Kust need modemid tulid?}

Esimene modem oli  mingi arvutiga kaasas, mingi 2400 bitti sekundis läbi laskev 
asi. Hiljem meil õnnestus Ron Dwighti\index[ppl]{Dwight, Ron} ja nende kaudu 
saada esimene US Robotics\index{US Robotics} mis vist oli kas 9600 või midagi 
sellist, ehk oluline edasihüpe. Kuna tollal nendes BBS-ides liikus palju 
erinevat tarkvara (ma nüüd ei ütleks, et  legaalset), siis selles ringi 
surfamine, sobramine, andis ühelt poolt vahendid ja teisalt  teadmise ja oskuse 
kuidas  erinevad tarkvarad töötasid, mida nendega teha sai ja nii edasi. 
Sisuliselt kõik on  ise õpitud,  puhtalt  katsetades, eksitusmeetodil. Isegi 
ma mäletan nõukaaja lõpus, kui veel täiesti vabalt lennata sai Nõukogude Liidu 
piires, siis Vladivostokist, Moskvast ja Leningradist oli teisi Fidoneti 
kasutajaid, kes lendasid külla viietolliste flopidega.
                 
\question{Inimesed tulid Vladivostokist viietolliste flopidega Fidoneti?!}

Jah,  kuna modemiga imeda võttis  rohkem aega ja raha, kui lihtsalt võtta  
sadu viietolliseid flopisid kohvriga kaasa ja  lennata Vladivostokist 
Tallinnasse ja lihtsalt kopeerida neid paar ööd-päeva läbi.

\question{Ahaa, nii et Vene suunal oli ka Fido side olemas?}

Jaa, nad samamoodi helistasid peale Eesti saitidele ja sealt liikus info nende 
suunas. 

\question{Sa korra mainisid, et Eesti BBS \#1\index{BBS!Eesti BBS \#1} ümber 
toimus mingi suhtlemine ja tekkis kogukond. Kes need inimesed olid?}
                 
Vot ma seda väga palju ei mäleta. Ma mäletan, et sealt me liikusime kiiresti 
edasi. Ta vist oli suhteliselt staatiline, vaikne ja rahulik pärast  
loomist, et seal vist väga sellist kommuuni ei tekkinud või siis vähemalt ma ei 
mäleta sellest.  Kommuun tekkis siis, kui BBS-i pidajaid tuli juurde ja mingil 
hetkel sai kokku kutsutud  esimene süsoppide nõupidamine, mis toimus Viru 
hotelli kas  20. või 22. korrusel asunud väikeses äärepealses toas. Seal olid 
Lõvi\index[ppl]{Lõvi}, mina Tarmo Ausing\index[ppl]{Ausing, Tarmo}, Tarmo 
Mamers\index[ppl]{Mamers, Tarmo}, Virko Püss\index[ppl]{Püss, Virko} vist. 
Tegelikult on Mamersil vist isegi mingid memuaarid kirjas, nad olid mõnda aega 
veel isegi internetis üleval. 

\question{Kas too kokkusaamine oli rohkem  sotsiaalne üritus või oli seal 
mingisugust probleemi ka lahendada?}

Ta oli mõlemat. Natukene oli minu arust juttu  sellest, et kuidas Fidonetti  
korraldada, organiseerida, kuidas  meililiiklust vist teha. Kuna meil olid 
tollal väliskõned nii-öelda tasuta käes, siis meie saime olla Eesti  see 
esimene \emph{node}, kelle kaudu meil  liikus Eestist välja. 

\question{Ja see meie on Hackers Night System?\index{BBS!Hackers Night System}}

Jah. Teised saatsid oma kirjad meile, meie saatsime need öösel üle 
modemi järgmisele Euroopa \emph{node}-le, kes nad siis  omakorda laiali jagas.
                 
\question{BBS-i püstipanekuks on ju mingit tarkvara ka vaja?}

See oli  suhteliselt sellise paketina saadaval, samamoodi BBS-ist tõmbasid 
alla. Mingi Maximus BBS või midagi sellist. Selline vahva platvorm. 
Tõmbasid \emph{boot}-imisel BAT failiga üles, see jäi modemist ootama sisse 
tulevat kõnet ja kogu lugu. Seal all olid siis põhimõtteliselt failikataloogid 
ja meilisuhtlus ja kasutajate haldus.

\question{BBS-i sisse helistamiseks ei olnud mingit eraldi softi vaja?}

Tavaline terminali soft oli, see  oli vist tollal isegi, ma pakun, enamusele 
opsüsteemidele olemas. Ma praegu muidugi oletan, aga  tollal oli ju suur 
osa ju \emph{mainframe}-del ja nendega suhtlemine käis üle telneti.

\question{Ma miskipärast mäletan Norton Commanderi sees mingit 
helistamisvõimalust}

Võib olla. Seda mina ei taibanud kasutada. Mis kellelgi käepärast oli. Sellega 
sai siis juhtida nii kasutatava modemi režiimi, terminali režiime, sealt sai 
alla tirida tarkvara. See oli meie tegevuste põhiskoop lisaks siis mängimisele 
ja selle tarkvara uurimisele, mis me saime.
                 
\question{Ehk, kogukond tuli põhimõtteliselt BBS-i adminide hulgast. Aga tuumiku 
ümber pidi ju olema ju ka kasutajaid. Oskad sa suurusjärgus hinnata, palju neid 
võis Eesti peale olla?}

Ma arvan, et alguses, üheksakümnenda aasta paiku, võis olla mingi paarkümmend 
kasutajat per \emph{node} ja neid \emph{node}-sid oli ka nii viie kuni kümne 
ringis. Hiljem, interneti tuleku eelsel ajal, läks asi juba päris suureks ja 
massiliseks aga selleks ajaks olime meie juba sellest kogukonnast eemaldunud. 
Peamiselt seetõttu, et äri võttis kogu aja üle. 

\question{Arvutame siis. 20 inimeste per \emph{node}, 5-10 \emph{node} 
teeb\ldots} 

Vast paarsada inimest üle Eesti, neid võiks ka  rohkem olla. Tollal oli ka 
see asi, et igalühel ei olnud oma arvutit ja oma modemit vaid see oli selline 
\emph{round-robin}, et 10 inimest sama arvuti ja modemiga. 

\question{Nojah, klassitäis kaake helistab sisse, mine võta kinni, palju neid 
seal on}                 

Ei, vaata, neid \enquote{klassitäis kaake}  väga palju ei olnud, sest et 
ikkagi enamasti olid meievanused tegelased, oli ka natuke vanemaid. 
Päris sellist akadeemilist seltskonda, 
teadlasi ja uurijaid, ma ei mäleta. Neil olid ilmselt omad vahendid ja 
võimalused. Ka Usenet oli kusagil olemas.

\question{Panite oma HNS-i püsti ja hakkasite Nõukogude Liidu avarustest juppe 
ja neid Eestis maha müüma?}

Pigem vastupidi. Me hakkasime tooma Saksamaalt arvutijuppe, neid siin kokku 
panema ja siis Eestis maha müüma. Võib-olla tänu sellele, et isa kõrvalt 
tekkis varakult selline impordi-ekspordi kogemus, mis tollal oli üllatavalt 
haruldane kompetents. Et kuidas väljamaalt midagi osta ja üle tollipiiride 
Eestisse tuua. See oli (20 aastat enne e-poode) siiski  suhteliselt 
haruldane teadmine. Esiteks, kuidas leida üldse see kontakt, kellele helistada, 
Kuidas talt hinnapakkumist saada. Sul ju ei ole aimugi, kes müüb. Sa ei tea, 
kellele helistada, et talt üldse hinnapakkumist küsida. 

\question{Aga kust sina teada said?}

Tänu sellele, et isa oli selline aktiivne tegelane. Kuna tema  inglise keelt 
väga ei osanud, siis ma pidin jooksvalt ka tema asju ajama ja siis tema 
eksimustest ja õnnestumistest, eks siis sai õpitud. 

\question{Eks me kõik seisame hiiglaste õlgadel. BBS-induses alguses mingit 
ärilist aspekti ju ei olnud?}

Ei. Puhas selline fänlus. Paljuski seisis ta kahel jalal. Esiteks suhtlemine, 
ehk  vestlustoad, inimesed omavahel suhtlesid erinevatel teemadel, selline 
\emph{community} värk. Ja teine oli siis softi vahetamine.

\question{Tollel ajal hakkasid ju esimesed jutukad ka tekkima?}                 

Jah, ma arvan, et  jutukad saidki paljuski alguses nendest samadest BBS-ide 
tubadest.

\question{Mis need esimesed olid? Ma isegi ei mäleta nimesid, ei ole kunagi 
neis käinud?}

Ma jutukates ei ole väga palju olnud. Kas OK jutukas\index{Jutukad!OK} ei olnud 
üks ja Cafe\index{Jutukad!Cafe} vist oli teine. Need hakkasid tekkima siis, kui 
ma nagu enam väga selle sotsiaalse tegevusega ei tegelenud, kogu aeg ja energia 
läks oma firma arendamisele.
                 
\question{Kuidas see äri tol ajal toimus? Üheksakümnendate keskel oli 
suhteliselt palju veel kogukondlikku toimetamist, kasvõi .EXE tegemine.}

Jah. Ma arvan, et too äri seisis sellesama kogukonna õlgadel, mis Fidonetist  
alguse sai. Eks lõpuks pidid ju kõik kusagil tööd tegema. Ja pigem lihtsalt 
needsamad suhted ja asjad liikusid ärisse edasi. Paljuski needsamad nimed, kes 
ka seda ajakirja .EXE\index{.EXE} tegid olid Fidonetist tuttavad. 

\question{Sina keskendusid siis täitsa ärile?}

90.-te algusest või keskelt nagu läks enam-vähem selline elu käima, üheteist-tunnised 
tööpäevad ja. 

\question{Sealt kõrvalt vist väga palju enam programmeerimiseks aega ei jäänud?}

Ei. Me tegelesime eelkõige riistvara vahendamisega. Riistvara, võrgud, selline 
suhteliselt primi tegevus. HNS\index{HNS},  hilisem Zebra 
Infosüsteemid\index{Zebra Infosüsteemid}, nagu tarkvaraarenduseni ei jõudnudki.

\question{Isegi tavalise võrgu ehitamine oli ju koaksiaalkaablil?}

Koaksiaalkaabli vedamine oli mul väga selge. Mitmete tänaste suurfirmade 
laudade alt sai roomatud ja tolmu pühitud ladudest ja seintelt ja lagedelt.

\question{Jaa. Mäletan Ühispanga kontorit, kus kontoriarvutid olid seljaga 
kliendi poole ja kaabel jooksis masina juurest masina juurde. Järjekorras 
seistes oleks võinud lihtsasti termika maha keerata ja terve kontor oleks 
seisma jäänud.}

Siis, kui me saime kunagi Seesamile\index{Seesam Kindlustus} IBM Token Ring 
nimelist võrku vedada. Token Ring oli 
 esimene selline tähtkujuline topoloogia, millega me kokku puutusime. Sellel  
olid sellised rusikasuurused stepslid, siis see oli täielik müstika.
 
\question{Kust teil tuli mõte sihukese müstikaga tegelda, see hakkab ju 
tasapisi üle piiri kasvama, mille endale lihtsalt katsetades selgeks teeb?} 

Selles mõttes, et ega nagu äris ikka. Sa võtad mingeid riske ja ütled, et 
\enquote{loomulikult me saame selle asjaga hakkama}. Ja mis see siis nüüd ära 
ei ole. Hakkad tegema ja  selgub, et  kõik töötabki niimoodi, nagu ette nähtud.
Ja ega  nende võrkude ja asjade üles panek ei olnud  progemisega võrreldes nagu 
mingi eriline tegevused. Lihtsalt konfigureerid süsteemi ära ja jalutad koju.

\question{Tol ajal koolist niisugust teadmist vist üldse saada ei olnud 
võimalik}

Absoluutselt. Mina lõpetasin polütehnikumis\index{Koolid!Tallinna Polütehnikum} 
raadioside ja levi eriala. See valdavalt põhines lamp-elektroonikal, kuigi 
meile õpetati ka pooljuhte ja nii edasi, aga loogilisi skeeme, loogilisi 
ahelaid, kõike seda me saime vist pool aastat. Ja sedagi  
teoreetiliselt. Enamus  meie  elektroonika õppimisest polütehnikumi ajal oli 
väga selline analoog-elektroonika: lambid, elektromehaanika, samm-valijatel 
põhinevad telefonijaamad. Noh, mis oli selline paras küberpunk tänapäeva 
mõistes juba. 

\question{Kui ma kuulan su juttu, siis sealt kumab välja soov asjadest aru 
saada, mõista, mis on karbi sees?}

Absoluutselt. Et ma läksin polütehnikumi, oli pigem nagu selline perekonna 
traditsiooni järgimine. Ma olen soodumuselt pigem nagu humanitaar olnud, kuna 
sinnamaani, ehk tehnikumi minekuni, olid mul neljad-viied kõik humanitaarained: 
keeled, kirjandus. Kõik reaalained olid kahed-kolmed.

Ilmselt siis tasub õppida seda, mida sa ei tea, ülejäänud tuleb lihtsalt niigi. 

Aga noh, see on andnud  selle tugeva külje või plussi või aluse, et sa oskad 
asju, millest sa aru ei saa, üldistada või teisendada sellisteks mustadeks 
kastideks, millel on mingid defineeritud sisendid ja väljundid, mille põhjal 
saad sa omakorda teha mingeid järeldusi selle musta kasti sisu kohta. 
Selline paradigma oli seal koolituses pidevalt olemas, ma arvan, et see on 
siiski selle põhja ja vundamendi andnud.
                 
\question{Nojah, mis seal kasti sees on, ei jõua sulle keegi õpetada, sest 
homme on seal teine asi. Aga lähenemine on kasulik.}

Vaata, tollal muutused olid palju aeglasemad. XT-d olid ikkagi  aastaid 
ja 286-t oli ka ikka aastaid, enne kui 386 ja 486 tulid. Lihtsalt nüüd on see 
muutuste tempo  radikaliseerunud.
                 
\question{Huvitaval kombel see alati tundub nii, et just praegu on palju 
kiirem, kui vanasti.}

Võib-olla jah, on see, et vanemaks saades elu tempo ise muutub.

\question{Mis hetkel see kogukond hakkas selgelt alla jääma sellele, et igaühel 
oli laen ja liising ja pere?}

BBS-i kogukond tegelikult oli ju mitte ainult need \emph{bulletin board}-id 
vaid toimusid ühisüritused. BBWinterid\index{BBWinter}, 
BBSummerid\index{BBSummer}. Iseenesest, need said alguse just sellestsamas 
süsoppide esimesest kokkutulekust. Hiljem hakati juba laiemaid üritusi  
tegema, kuhu olid kutsutud ka BBS-ide kasutajad. Ja seda tegelikult jätkus  
Interneti tulekust veel edasigi, et siin vist äsja veel arutati, et kellel veel 
BBS töötab. Ma üldse ei imestaks, kui veel leitaks Eestist mõni töötav BBS 
kusagil virtukas tiksumas. 

Ma arvan, et minu jaoks ta loomulikult vajus laiali rohkem sellega, et ise ei 
jaksanud sinna enam panustada. Aga seal elu ja tegevus toimus veel 
aastaid-aastaid hiljemgi. Ta hakkas  laiali vajuma sellest, kui 
internetipõhised keskkonnad hakkasid  kasutajaid lihtsalt üle võtma. 
Suhtluskeskkond sulas igapäevase töö  ja muude tegemistega seotud keskkonnaga 
üheks, modemiga kuhugi helistamine tundus nagu pisut arusaamatu.

\question{Et kui BBS-is vahetati faile ja aeti juttu, siis Internetis sai tööd 
ka teha?}

Absoluutselt, Internetis olid need Cafe ja OK jutukad  olemas ja ega enne 
Facebook-e ja Rate-sid olid ju ka olemas Geocities ja mis iganes need keskkonnad 
olid. Ega need ei ole mingid uued asjad. Nad on vahetanud kesta ja vormi ja 
platvormi ja värvi ja natuke funktsionaalsust, aga inimesed on juba 30-40 
aastat liikunud sellesama tegevusega  ühest keskkonnast teise.
                 
\question{Kes BBSummereid ja BBWintereid korraldas ja kes seal käisid?}

Ma mäletan ühte korraldajat, kelle kasutajanimi oli vist Kristrap aga Piret 
Part\index[ppl]{Part, Piret} oli vist pärisnimi.

\question{See oli puhas kogukonna värk, mingit sponsorit taga ei olnud?}

Hiljem, kui üritused suuremaks läksid, siis me HNS-iga firma poolt loomulikult 
sponsoreerisime. Aga alguses need üritused olid piisavalt väikesed. Et 10 
inimest omavahel kokku tuleksid, natuke asja arutaksid ja selle juures mõned 
õlled teeks, ei vaja sponsorit. Hiljem  arvutifirmad toetasid, samad inimesed 
töötasid ju paljuski arvutifirmades, olid olulised tegelased kui mitte 
omanikud, ja panustasid.

\question{Ehk, Eesti Fido kogukond läks sujuvalt üle Eesti IT-tööstuseks?}

Jah, absoluutselt. Ega paljud tuttavad nimed on ju sealtsamast pärit. Tõnu 
Samuel\index[ppl]{Samuel, Tõnu} näiteks on ju samamoodi sealt keskkonnast pärit. Ma arvan, et lihtsam on 
võtta ja üles otsida tollased nimekirjad Fidonetist, mingi jututubade 
\emph{printout}-id, mida vist Mamers\index[ppl]{Mamers, Tarmo} peab ja vaadata, 
kes need nimed on, kes seal eksisteerivad. Ma arvan, et sa leiad nad praeguste 
it-meeste gruppidest üles.

\question{Kes neil üritustel käisid? Süsopid, ma saan aru, aga lisaks?}
                 
Ega ma ei oska sulle öelda. Süsopid loomulikult, aga eks oli neid, kes ei 
tahtnud, ei viitsinud või ei saanud ise BBS-i pidada. Aga kellele see oli 
lihtsalt üks selline asi, mida arvutiga teha. Külastada BBS-e.

\question{Nojah, kuna progemiseks olid barjäärid kõrged ja kogu aeg mängida ei 
jaksa, siis tehti midagi muud ka.}

Eks mõned tulidki läbi progemise huvi. Minule endale tuli see läbi mänguhuvi, 
eks ole. See oli võimalus suhtlemiseks ja tarkvara vahetamiseks,  ka tehniliste 
teadmiste vahetamiseks.

Tollal kusjuures ju suhtlemine oli oluliselt raskem, kõigil ei olnud 
isegi mitte kodus telefoni. Tänapäeval tundub  kõik nagu käeulatuses, sa 
pead lihtsalt taskust telefoni võtma, Google'isse otsingu panema. Tollal sa 
pidid otsima telefoni, mõtlema, kellele helistada, et küsida, kas ta teab 
kedagi, kes teab midagi. Fidonet andis ju ka  selle võimaluse, et sa said 
avalikku \emph{board}-i panna mingi küsimuse, et kas keegi teab, kuidas 
lahendada ühte või teist probleemi.

\question{Tuli vastuseid ka?}

Kindlasti. Kus nad pääsesid. Eks see sõltus  küsimusest. Eks seal oli asjalike 
jutte, olid oma läbu kohad, kus niisama jaurati, tehnilised vestlusringid.
                 
\question{Jah, aeg paneb asjad teise konteksti. Kui sul isegi kodus telefoni ei 
ole, siis võimalus rahvusvaheliselt inimestega suhelda on märkimisväärselt 
teise väärtusega, kui siis, kui sul on taskus mobiiltelefon ja Internetti 
ühendatud arvuti alati käepärast.}

No just. Mõtle sellele, et Google'it ei olnud tollal.

Google oli sul üle laua su kolleeg, naaber ja sõber ja sul endal pidi olema 
selline kontaktibaas ja varamu piisavalt suur ja lai, et teaksid kedagi, kes 
teaks kedagi, kes teab kedagi, kes oskab sulle öelda midagi.
                 
\question{Kas sa oled nõus Pronto ütlemisega, et väga paljus seesama 
kontaktibaas võimaldas väga loomulikult kogukonnast ärisse üle minna sest ka 
äris sõltus palju kontaktidest?}

Seda võiks ka nimetada \emph{street reputation}-iks, eks ju. Sul oli 
\emph{credibility} mingil määral olemas. Sind juba teati ja tunti selles 
keskkonnas. Kui tänapäeval räägitakse \emph{Estonian Mafia}-st, siis tollal see 
oli Fidoneti seltskond.

Loomulikult, eks paljud meie kliendid tulid läbi Fidoeneti või vähemalt teadsid 
meid sealt kaudu.

Paljud klientidest töötasid hiljem, eks ole, mingis pangas või firmas, firmad 
laienesid, tahtsid saada arvuteid. Kusagil oli neil palgatud mingi itimees, 
kes pidi selle probleemi lahendama ja ega tal ka ju ei olnud Google'it või 
e-poodi, kust  parimat pakkumist küsima minna. Tal oli endal ka inimesed, keda 
ta teadis ja usaldas, kelle käest siis seda pakkumist minna küsima.
                 
\question{Ei olnud nii, et lähed poodi: \enquote{Palun mulle 16 arvutit}.}

Ega see arvutiäri alguses oli ka selline, et kuna valuutat 
ju väga palju kellelgi ei ringelnud, siis laoseisud olid olematud. 
Põhimõtteliselt võeti ettemaks, ettemaks maksti välismaale, selle eest oodati, 
kuni arvuti kohale laekus, siis pandi see kokku ja tarniti kliendile. Kui hästi 
läks, sai alla kuu aja kätte. Ka see eeldas tegelikult ju usaldust, et sa 
annad kellegile 1000 raha. Arvutikategooria mõttes 
hinnad ei ole eriti muutunud: hea arvuti oli üle 1000 ja tavaline 1000.

See, et sa annad mingile matsile selle raha ära, ta ütleb, et \enquote{ära 
muretse, kuu aja pärast saad kätte}, see eeldab üksjagu usaldust. Ega valuutat 
ei olnud, enamasti tehingud alguses tehti rublades aga rubla ei olnud 
konverteeritav millekski muuks kui rublaks. Ja oli veel see aeg, kui paljudel 
asjadel oli kaks hinda: ülekande rubla hind ja sularaha rubla hind. Odavam oli 
sularahas, sellepärast et sa ei saanud alati pangast sularaha kätte. Olid 
mingid kindlad hetked, millal panka toodi sularaha ja siis sa pidid teadma, kas 
õigeid inimesi õiget aega, et saada sularaha.

\question{See muidugi seletab kõiki neid legende, kuidas arvutifirmas ja pangas 
hoiti sularahapakke kuskil kapis ja vetsus ja kus iganes.}

Kui taheti midagi ülekandega osta, siis see oli nüüd meie enda valik, et kas me 
müüsime midagi valuuta või rublade eest. Kui  ülekande eest otsustasime müüa, 
siis oli hind kallim lihtsalt puhtalt seetõttu, et pärast selle raha kätte 
saamine pangast oli nagu oluliselt keerulisem.

\question{Ja sellest ajast sa oledki jäänud niimoodi arvutitega tegelevaid 
inimesi juhtima?}

Nojah, mida aeg edasi, seda enam on mind 
huvitanud rohkem sisulised asjad, kuidas asjad töötavad. Ja vähem huvitanud 
inimeste juhtimine. Ütleme niimoodi, et inimesed on keerulised, arvutid on 
lihtsad.

\question{Kuidas sa infoturbe ja selle maailma juurde jõudsid?}

Esimest korda me jõudsime sinna juurde läbi väga praktiliste sammude. Me nimelt häkkisime 
Täitevkomitee arvuteid. Selleks, et sinna ligi  ja mängima saada, me 
avastasime, et tollal Õnnepaleeks kutsutud majas on olemas üks modemite peal 
töötav teenus, mille 
 kaudu sai abielusid, sünde ja surmasid registreerida. Ja muu hulgas sellesama 
modemi otsas sai eraldada kortereid. Nimelt kortereid ei saanud tollal osta, 
vaid neid eraldati sulle riigi poolt.

\question{Ja teie avastasite koha, kuhu sai sisse helistada ja eraldada 
kortereid?}

Me avastasime, et sinna sai sisse helistada, aga teenus oli parooliga kaitstud 
ja parool oli jällegi eesnimi ja perekonnanimi. Me alustasime nii-öelda 
tagumisest otsast sellel infoturbel.

Mina ise sattusin Privadori\index{Privador}, siis juba  10-15 aastat hiljem 
aastal 1990, kui Tarvi Martens oli tollasest Küberneetikast\index{Küber} teinud 
investorite kaasabil sellise asja, mida tol ajal nimetati \emph{spin-off} ja 
mida tänapäeval kutsutakse \emph{startup}. Privadori  eesmärgiks oli  
digitaalselt signeeritud dokumentide pikaajalise tõestusväärtuse loomise 
süsteem nimega TruSign. Seda muidugi hakati tegema kaks aastat enne ID-kaardi 
projekti algust, enne kui ühtegi signeeritud dokumenti keegi näinudki polnud.

\question{Väga huvitav. Krüptot enam paljakäsi ei tee, selleks on haridust 
vaja. Ehk, juba üheksakümnendate lõpus pidid omavahel kokku saama inimesed, kes 
olid iseõppijad ja minigid teistsugused inimesed, kes kindlasti ei olnud 
iseõppijad. Kuidas see käis, seal mingisugust hõõrumist ei tekkinud?}
                 
Enamus tollastest kolleegidest olid vanad tuttavad Fido ajast. Mina Privadori 
liitusin  küll turunduse ja müügi funktsioonis. Alles hiljem hakkasin 
tootejuhtimise ja nii-öelda juhatajana seal tööle. Aga algne funktsioon oli mul 
seal pigem müük. 

\question{See Fido seltskond ei olnud väga suur ju}

Ei olnud. Tegelikult kogu see seltskond, kellel üldse oli üheksakümnendatel 
ligipääs arvutitele, ei olnud väga suur. See oli veel see aeg, kui arvutid olid 
peamiselt firmades aga mitte kodudes. Üheksakümnendate lõpus alles hakkas 
tekkima see trend, kus firmad olid enam-vähem arvutitega varustatud, hakati 
ostma rohkem rakendustarkvara (või noh, ütleme turu trendid liikusid rohkem 
rakendustarkvara poole) ja eraisikud hakkasid endale koju arvutit ostma.

Ega tegelikult  sovhoosidest ja kolhoosidest parematel olid omad 
arvutuskeskused olemas juba nõukogude ajal. Arvuti kui selline Eestis ei 
tekkinud päris üheksakümnendatest, vaid see ikka oli ammu enne minu sündi 
olemas. 


\chapter{Meelis Roos}
%!TEX TS-program = arara
% arara: myindex

\index[ppl]{Roos, Meelis}
\textbf{\enquote{Kuidas sina arvutite juurde said?}}

Kõige esimene mälestus millestki arvutitega seoses on koolieelsest ajast, kui mõnikord läksime emaga lasteaiast koju mööda Liivi tänavat. Paremat kätt mäe otsas oli üks neljakordne maja. Ema ütles, et see on arvutuskeskus, ja see kõlas aukartustäratavalt ja põnevalt.
Päris arvutite juurde sattusin isa töö juures kaheksakümnendate lõpus. Füüsikud ostsid omale mõned arvutid elektromeetria laborisse, eksperimendi juhtimiseks. Arvutid olid CAMAC\sidenote{\emph{Computer-Aided Measurement And Control (CAMAC)} (elektroonikastandard andmete kogumiseks ja seadmete kontrolliks; kasutusel (osakeste) füüsikas aga ka tööstuses}) kontrolleriga vene DVK-d\index{Arvutid!DVK}\sidenote{\begin{russian}ДВК, Диалоговый вычислительный комплекс\end{russian}. Nõukogude personaalarvuti, ühilduv DECi PDP-11\index{PDP-11} perekonnaga. Varasemad mudelid on tuntud ka kui Elektronika MS-0501\index{Arvutid!Elektronika} ja Elektronika MS-0502}.

\textbf{\enquote{Kus see kõik sündis?}}

See juhtus Tartus\index{Tartu}, Tartu Ülikooli\index{Tartu Ülikool} juures. Isa oli Tartu Ülikooli Füüsika Osakonnas\index{Tartu Ülikool!Füüsika Osakond}\sidenote{Täpsemalt oli tegu Füüsika-Keemiateaduskonna Füüsika osakonnaga} füüsik. Nad tegelesid elektroonika mõõteseadmete välja töötamisega ja said isegi mingisuguse auhinna elektromeetri eest, mis eriti väikesi laenguid registreeris. Näiteks visati pastaka kuul, millel oli mingi laeng, kusagilt läbi ja mõõdeti see laeng mööda minnes ära. Neil oli seal elektromeetria sektoris lahe töögrupp, noored ülikoolist tulnud mehed tegid koos lahedaid asju. Nende katsete juures oli vaja andmeid töödelda ja katseid juhtida, selleks käis arvuti külge spetsiaalne lisaplokk. Ploki sees oli analoog-digitaaalmuundur (võibolla vastupidi ka aga igatahes niipidi neid kasutati). Füüsikud õppisid programmeerima, et suuta oma eksperimendi andmeid reaalajas kätte saada. 

\textbf{\enquote{Aga mis arvuti see selline oli, mis suutis andmeid niimoodi reaalajas kätte saada?}}

DVK-2M. Vene LSI-11\sidenote{DECi PDP-11 perekonna liige, tuntud ka kui PDP-11/03. Masinat tutvustati 1975. aastal ja ta oli oma sarjas esimene, mille CPU oli integreeritud. Mitte küll ühele, vaid neljale Western Digitali poolt toodetud \emph{Large Scale Integraton (LSI)} kiibile). Meelise sõnul: \enquote{PDP-11 oli legendaarne DECi masin iidsel ajal enne meie aega}} analoogid. Peaaegu täpne kloon aga natuke kohapeal ka täiendatud. Programmide poolt ühilduv aga mitte identne. DVK peal jooksis näiteks DECi originaal-opsüsteem RT-11\index{RT-11}. RT-11SJ oli igapäevane opsüsteem, see oli \emph{single job} ja RT-11FB'l oli \emph{foreground} ja \emph{background}, millega sai taustal jooksutada mingisugust teist tegevust. 

\textbf{\enquote{Kui vana sa olid, kui su isa need arvutid omale hankis?}}

Põhikooli teises pooles. Ega mul ei olnud põhjalikku teadmist, mida selle arvutiga teha saab. Kui ma tegin isale tekstisisestustööd, näiteks sugupuu andmete sisestamiseks, siis sain ma pärast seda kuni õhtuni mängida. Lemmikmäng oli Wall\index{Mängud!Wall}, seina pommitamine reketiga. Isa pani mind arvuti taga kohe tööle, et mu huvist miskit miskit kasu oleks, mis ma niisama aega raiskan. Programmeerima õpetati ka, eks nad ise ka õppisid. Isa rühmas programmeeriti BASICus\index{Keeled!BASIC}, FORTRANis\index{Keeled!FORTRAN} ja CASICus\index{Keeled!CASIC}. See viimane oli CAMACi kontrollerite programmeerimiseks mõeldud BASICu ja Pascali\index{Keeled!Pascal} vaheline keel\sidenote{Ilmselt peetakse silmas keelt formaalse nimega \emph{ANSI Standard Real-Time BASIC}, mille spetsifitseerib IEEE standard \enquote{726-1982 - IEEE Standard Real-Time BASIC for CAMAC}}. Selles viimases mina ei sattunud programmeerima, küll aga BASICus. Minu parim programm oli programm, mis ajas inimesega eesti keeles juttu. Programm ütles ühe lause, kasutaja ütles lause ja programm valis juhuslikult vastuse sisseprogrammeeritud lausete hulgast. Ta suutis mõnikord teemas ka püsida. Näiteks kui programm ütles \enquote{Osta elevant ära}, siis järgmised kaks lauset olid, et \enquote{Kõik ütlevad nii, aga osta elevant ära}. Enne ta ei läinud järgmist lauset valima kui ta oli kaks vastust saanud. Seda mängu teiste töötajate lapsed mängisid ja neil oli lõbus. See oli lahe emotsioon, et ma tegin midagi, mis teistele lahe oli. 

\textbf{\enquote{Huvitav, et sa kohe hakkasid mängu tegema ja seejuures kohe midagi AI-sarnast}}

See tundus kõige lahedam asi mida teha! 

\textbf{\enquote{Need füüsikud pidid ju kähku õppima, sest reaalajas riistvarast andmeid lugeda on ju keeruline?}}

Neid oli seal rühmas vähemasti kolm meest, kes programmeerimist õppisid. Neil oli üks natuke noorem pundis, kes oli nende põhiline arvuti-mees ja kes seda vist paremini jagas kui teised. Tema juures oli see CAMAC kontroller, millest enne juttu oli. Arvuteid oli selle labori peale vähemasti kolm tükki, aga üks oli see põhiline eksperimendi juhtimise oma. Mina kasutasin arvutit, mis oli niisama masinakirjutaja toas ja mida kasutati programmide sisestamiseks ja muidu andmetöötluse jaoks. Näiteks isa tegi selle abil sugupuu üles joonistamist, neid puid sai rullpaberile\sidenote{Toonaste printerite puhul oli tavaline, et paberi jooksis printerisse perforeeritud servadega rullist, nii sai paberit kiiremini liigutada} välja trükkida. Kui hiljem koolis tulid mingid tudengid ja andsid igaühele paberi, et joonistage oma sugupuu üles, siis mina palusin isal lihtsalt ühe koopia välja trükkida.

\textbf{\enquote{Aga miks sa lasid ennast sellesse suhteliselt igavasse andmesisestaja rolli suruda? Lihtsalt, et saaks mängida?}}

Algul selleks, et saaks mängida. Aga kui selgus, et ise programmeerida saab ka ja see on täitsa lahe, siis ma pigem mängimise asemel keskendusin rohkem sellele. Ma ei jätnud mängimist päris maha, mängisin ikka ka vahel. 

Mind köitis programmeerimise juures, et programm võis vähendada käsitööd. Näiteks ESC koodidega printerile õigeid asju saates\sidenote{\emph{Epson Standard Code for Printers, ESC/P\index{ESC/P}} on Epsoni poolt maatriksprinterite jaoks välja töötatud (ja termoprinteritel siiani kasutusel olev) keel, mis võimaldab juhtida rastrivõimekuseta printerit. Keel sai oma nime sellest, et tema käsud algasid sümboliga ESC (ASCII 27). Näiteks ESC E lülitas sisse ja ESC F välja rasvase trüki} trükkisin oma õpiku silte, kus oli rasvases ja suuremas või väiksemas kirjas kõik vajalik erinevatel ridadel kirjas. Üks ema tuttav tahtis oma firma logo visiitkaartidele, see logo tuli siis teisendada Epsoni printeri graafika ESC-jadadeks. Ma joonistasin selle \emph{bitmap}ina üles aga siis leidsime, et ei tasu vaeva ja seda logo ma ei teinud. See oli näiteks koht, kus ma leidsin, et programmist võiks oluliselt kasu olla. Ja üheksandas klassis oli seik, kus ma jäin füüsika tunnis programmeerimisega vahele -- kirjutasin oma vihikusse mingit BASIC-programmi ja õpetaja läks mööda ja ütles midagi stiilis, et siin tunnis tegeleme füüsikaga, mitte programmeerimisega. Ja keskkoolis tegin programmi, mis otsis lähendusi kaheteistkümnendale juurele kahest, nii et saaks isaga süntesaatori ehitamisel sagedusjagaja täpse teha -- oli esimene kasulik programm, mida ma mäletan.

\textbf{\enquote{Kuidas õppimine käis?}}

Ma sain mingisuguseid venekeelseid raamatuid. Osalt raamatukogust isa tõi, osalt oli ehk mõni raamat tal töö juures olemas.  Need olid enamasti kusagilt laenatud. Näiteks mul oli segadus ASCII koodi ja \emph{Escape} koodidega, mida sai printerile ja terminalile saata. Siis ma mäletan, et küsisin isalt nõu, et mis neil vahet on et kas see on seesama asi. Ja siis oli erinevaid raamatuid . Näiteks oli üks raamat BASICu kohta, kus mingisuguse käsu kohta on mul siia maani meeles kirjeldus, mis minu meelest ei sobinud niisugusesse raamatusse: \begin{russian}\enquote{эта команда работает хорошо}\end{russian}. See käsk töötab hästi. Minu meelest oli see lati liiga madalale laskmine. Minu meelest peaks kõik hästi töötama, asjad tuleks nii teha. 

\textbf{\enquote{Sul oli ju siis päris korralik vene keele oskus?}}

Jah, ma olin üheksandas klassis umbes kui ma programmeerimist õppisin ja kannatas venekeelset raamatut lugeda küll. Meil oli põhikoolis selline vene keele õpetaja, kellega pidi õppima, mul tõenäoliselt oli üsna normaalne vene keele oskus selle vanuse kohta. Ma käisin Tartu 12. Keskkoolis\index{Koolid!Tartu 12. Keskkool}. Meil oli üks ukrainlanna, Zinaida Tovkatš vene keele õpetajaks. Tema kohta meie kirusime, et ta on väga range ja isegi haige ei ole kunagi. Muudkui peab õppima ja muidu ei pääse. 

\textbf{\enquote{Kas keegi sind õpetas ka või käis ainult raamatu järgi see asi?}}

Isa õpetas mulle neid asju, mida tema teadis. Näiteks õpetas ta mulle plokkskeeme, sest ta ise õppis nende abil. See kestis kuni keskkooli ajani välja, et kui mina tegin programmi ja see ei töötanud, siis oli kaks viisi silumiseks. Üks oli see, et ma trükin ta rullpaberil välja ja loen õhtul kodus. Teine võimalus on see, et ma joonistan selle asja plokkskeemiks ja lähen näitan isale. Sealt pealt tema oskas vigu leida küll. Ja plokkskeemiks joonistamisel leidsin ma tihti vead ise ka üles. Ja isegi kui ma Pascal-keeles kirjutasin, mida isa ei osanud, ma sain temalt ikkagi plokkskeemide tasemel abi. Sest isal oli hea loogiline mõtlemine ja ta seletas mulle minu vead ära küll. 

Minu ülesanne oli kodus keskkütte katla alla tuli teha. Selle süütamiseks oli füüsikaosakonnast toodud vanapaberit, mille hulgas oli teinekord mingeid arvuti väljatrükke, mida ma lugesin. Panin need kõrvale samal ajal kui ajalehed ja muud läksid katla süütamiseks. Näiteks ma leidsin Minsk 32\index{Arvutid!Minsk-32}-e\sidenote{Minsk-32 loodi kuuekümnendatel, nagu nimigi ütleb, Minskis. Tegu oli mitmest mudelist koosneva Minsk suurarvutite sarja kõige võimekama esindajaga. Oli laialdasel kasutusel, kuni asendati seitsmekümnendatel IBM 360 kloonidega} mingisugused 32-bitised krahhi- või muidu mälutõmmised. Ma olin üllatunud, et minul on 16-bitised PCd (see oli tol hetkel hiljem vist kui ma juba PC taga olin) aga nendel oli juba siis 32-bitine arvuti. Ja seal olid FORTRAN-programmid, mida ma huviga lugesin. Isa kõrvalt ütles, et ah, need ei ole suurt midagi väärt, et see mees, kelle programmid need on, ei oska veel eriti programmeerida, tema programmide pealt pole eriti mõtet eeskuju võtta. Aga põnev oli neid lugeda sellegi poolest. FORTRANit õppisin keldris katla kütmise juures!

\textbf{\enquote{Miski pani sind tulehakatust lugema, mis see oli?}}

Seal olid uued põnevad asjad!

\textbf{\enquote{Kas sa peale tulehakatuse midagi muud ka lugesid? Või oli näiteks muusika huvi?}}

Ulme huvi natuke oli. Mul õnnestus saada venekeelsed Asumi\sidenote{Isaac Asimovi poolt kirjutatud sari. Ilmus esmakordselt triloogiana 1951. aastal, tunnustati 1966. aastal Hugo auhinnaga \enquote{\emph{Best All-Time Series}}. Alates 1981. aastast lisandus triloogiale veel köiteid} seeria raamatud, neid oli rohkem kui kaks esimest\sidenote{Eesti keeles ilmusid kaks esimest Asumi raamatut \enquote{Asum} ning \enquote{Asum ja impeerium} vastavalt 1985. ja 1989. aastal Linda Ariva tõlkes}. Asumid mulle meeldisid ja ühe isa sõbra käest laenasime venekeelsed ülejäänud Asumi raamatud. Mul õnnestus vene keeles raamatut lugeda, ma olin selle üle sügavalt üllatunud. Isa luges neid algul ise, hiljem mina. Nii et ulme huvi oli küll, aga see ei olnud väga sügav. Seda oli valdavalt nii palju, kui kodus sattus Mirabilia sarja ulmekaid olema. Need said kõik läbi loetud. See ei olnud esialgu eriti seotud arvutitega, arvutid olid asi, mis tuli reaalsest maailmast. Näiteks sõitsin bussiga koju ja ükskord Pärmivabriku peatusest mööda sõites parajasti ema seletas mulle arvutiviiruste kohta, mida ta oli kuskilt Horisondist või mõnest niisugusest kohast lugenud. Väga põnev oli. Parajasti sõitsime Pärmivabriku peatusest mööda, kui ma esimest korda arvutiviirustest kuulsin. Seda ma mäletan. 

\textbf{\enquote{Kas sa olümpiaadidel ka käisid?}}

Jaa, käisin. Matemaatikaolümpiaadil käisin neljandast klassist saadik. Oli naljakas korrelatsioon: lastest, kellega ma olin koos käinud ülikooli töötajate lasteaias, neist nii mõndagi sai seal olümpiaadidel kohatud. Järgmine laine olümpiaadidega oli keskkooli minnes. 

Miks ma vanast koolist ära läksin? Vanas koolis oli nii, et keskkoolis pidi tulema kaks klassi. Reaalkallakuga ja humanitaarkallakuga. Ja humanitaarkallakuga pidi see \enquote{A} ja eliitklass tulema, kuhu paremad õpilased lähevad ja ülejäänud võinuksid minna sinna reaalkallakuga klassi. Ma leidsin, et see on lati alla laskmine, et ma tahaksin ikka paremat. Mind kutsuti Nõkku\index{Koolid!Nõo Keskkool}. Hilisem ülemus Cyberneticast\index{Cybernetica}, toonane Nõo kooli direktor Uuno Puus\index[ppl]{Puus, Uuno} saatis laiali kõikidele olümpiaadikutele Nõo kooli kutseid. Sain ka. Kaalusin. Oli kaugel. Raske. Siis selgus, et esimene keskkool Tartus\index{Koolid!Tartu 1. Keskkool} on ka täitsa kõva tasemega. Helistasin kooli ja küsisin, et kas teil arvutiklass on. Direktor võttis vastu ja reklaamis, et neil on väga hea arvutiklass. Selle peale ma otsustasin sinna minna. Viisin paberid Esimesse Keskkooli, kui 1990. aastal sisse astusin, oli see juba Hugo Treffneri Gümnaasium\index{Koolid!Hugo Treffneri Gümnaasium|see{Tartu 1. Keskkool}}. Olid tõesti väga head arvutid (Yamaha MSX-II), lisaks arvutiklassile ka Juku\index{Arvutid!Juku}-klass. 

\textbf{\enquote{Sul oli siis selge arusaam, et sa just sinna kooli tahad minna?}}

Jah, ma läksin nimelt sinna. Selle kohta tegi ajaloo õpetaja meil kunagi pisikese kiire küsitluse üheksanda klassi kevadel. Et paljud teist siia jäävad ja paljud lähevad kuhugi mujale. Ja siis ta küsis kolme tema nina all oleva tegelase käest. Esimeses pingis sattusin mina istuma ja minu tagant kahe tüdruku käest, kes olid ka kätt tõstnud, et lähevad mujale, küsiti, mis nad teevad. Need oli täpselt need kolm, kes läksid Esimesse Keskkooli. Nii et kõik, mis ta küsis, sai vastuseks, et lähme ära esimesse keskkooli. Tüdrukud läksid teise paralleeli, bioloogia-keemia harusse. See tundus olevat umbes see vanus, kus mõned hakkasid ise mõtlema oma tulevikule ning seda planeerima ja mõned lasid asjadel isevoolu teed minna. Mina olin nende hulgas, kes leidis, et ma tahan ise oma tulevikku kujundada.

\textbf{\enquote{See oli see aeg, kui ühiskonnas hakkas juba muutus tulema, eks ole}}

Natuke oli juba varem selles mõttes, et kooperatiivid\sidenote{Nõukogude Liidu lõpuaastatel lubatud spetsiifiline ettevõtlusvorm, neid kasutati esimesel võimalusel massiliselt väike-ettevõtluse alustamiseks} olid juba varem olemas ja asjadest tohtis rääkida. Selle sama üheksanda klassi jooksul ma jõudsin kaks korda kirjutada ühele õpetajale referaate, millest võib olla aasta varem oleks vanematel pahandus tulnud. Aga siis juba tohtis. Selle õpetaja kohta oli teada, et ta on üks paras punane. Aga sain nende referaatide eest isegi kiita, mis oli üllatav. Ma mõtlesin, et tuleb kuidagi oma seisukohti kaitsta, sain hoopis kiita. 

\textbf{\enquote{Kas sind keskkooli ajal tööle ei tõmmatud kuhugi?}}

Ainult natukene. Tiražeerisin isa töö juures elektromeetrite trükkplaate. Joonistasin ahjulakiga ja risti ära lõigatud otsaga süstlaga rajad, söövitasin plaadi ära, tinatasin ära ja jootsin sinna peale kõik elemendid vastavalt skeemile. 

\textbf{\enquote{Aga see tahab ju käelist oskust ja elektroonikahuvi, kust sul see?}}

Seitsmeaastaselt oli mulle vist isa töö juures jootekolb esimest korda kätte sattunud, kui ma suvalisi tükke kokku jootsin. Eks ma oskasin kolbi hoida ja elektroonikahuvi mul oli. Aga elektroonikat ma ei osanud, analoogelektroonikat ei ole ma kunagi ära õppinud. Üldisi põhimõtteid tean aga ise midagi teha ei ole osanud.

Digielektroonika oli seal kõrval. Kui keskkool hakkas läbi saama ja oli vaja ülikooli minna, siis mina olin neljandast klassist peale kindel olnud, et ma lähen füüsikat ja nimelt elektroonikat õppima. Aga siis tulid arvutid, kah põnev elektroonika värk, neid sai matemaatikateaduskonnas ka õppida. Mul oli kuhugi ilma eksamiteta sisse saamised, äkki matemaatikasse ja füüsikasse olümpiaadi tulemuste pärast või midagi. Otsustasin matemaatika kasuks, sest füüsikaosakonnas ma olin kogu aeg kohal ja mulle ei meeldinud see. Tundus, et kui midagi ära tahta teha, siis peab ainult endale lootma. Oli nihukesi saarekesi, kes tegelesid oma kitsa erialaga, aga laiemat kandepinda ma ei märganud. Oli töögruppe, kes olid vingel tasemel ja tegelesid oma asjaga. Võib olla, et ma ei sattunud õigete inimestega kokku, aga tundus, et pigem on füüsika nihukene seisev konnatiik. Igaüks on seal kinni, kus on, ja nii on. 

Ega seal oli huvitavaid ja põnevaid asju ka. Näiteks olid füüsikapäevad, kus mu isa käis kuulamas Undo Uus\index[ppl]{Uus, Undo}i, kes rääkis materialismi ümber lükkamisest filosoofiliselt. Isa tuli koju, jutustas. Mina panin kõrva taha. Selliseid asju oli sealt ikka päris mitmeid. Füüsikalist maailmapilti tuli vanemate kõrvalt üksjagu, see oli mul olemas. 

\textbf{\enquote{Kuidas sa siis ikkagi matemaatikat sattusid õppima? Lihtsalt seepärast, et sai eksamiteta sisse?}}

Füüsikasse ma oleks vist ka saanud ilma eksamiteta, need ei oleks probleem ka olnud, ma arvan. Olin lihtsalt laisk, laisad me olime kõik. Keskkoolis klassijuhatajal tuli kaheteistkümnendas klassis üritada meile ikka auku pähe rääkida, et poisid olge tublid ja võtke tehke need eksamid ikka ära, siis saab medalile pretendeerida, muidu ei saa. Aga medaleid oleks ju vaja. Siis me tegime vist kolm medalit klassi peale või midagi. Mina sain hõbeda. Ma täpselt ei mäletanudki, kunagi hiljem kooli koduleheküljelt lugesin. Seda ma mäletasin, et medal oli, aga mis medal, seda ei mäletanud. Polnud oluline, see tuli iseenesest. 

\textbf{\enquote{Ühesõnaga, matemaatikasse sa läksid seepärast, et füüsika tundus natuke seisev vesi olevat?}}

Jah. Ja ma olin kuu aega enne paberite sisse andmist kindel, et matemaatikasse ma küll ei lähe. Me käisime koolist tiimiga Moskva\index{Moskva} lahtisel olümpiaadil matemaatikas. Seal olid mingid doktorandid, kes meiega tegelesid. Ühtlasi toimus seal ka \begin{russian}Международная конференция старшикласников "Наука, природа, человек"\end{russian}\sidenote{\enquote{Rahvusvaheline vanemate klasside konverents \enquote{Teadus, loodus, inimene}}} kus keskkooliõpilased said ise tehtud asju esitada. Keegi oli teinud kiiret vektorgraafikat, et voldime siin kuubikut kiiremini kui AutoCAD, või mis iganes. Ägedaid asju oli tehtud. Seal oli mingit Hollandi rahvast ka, oli rahvusvaheline küll. Seal need doktorandid, kes meiega tegelesid, olid nihukesed parajad uhuud. Näiteks tuleb tegelane hommikul tahvli ette, triiksärk on lükatud alukate sisse, alukad ulatuvad kümme sentimeetrit pikkade pükste pealt välja ja tuleb niimoodi tahvli ette. Ma leidsin, et vot matemaatikuks mina küll ei lähe. Aga siis ma mõtlesin ikkagi ümber. Matemaatikuks ma ei tahtnudki, ma läksin neid arvuteid õppima matemaatikateaduskonna\index{Tartu Ülikool!matemaatikateaduskond} poolt. Mitte elektroonika poolt aga programmeerimise poolt. 

\textbf{\enquote{Kuidas sulle ülikooli üleminek tundus? Sa ütlesid, et olla laisk olnud. Minu mälu järgi pidi ülikoolis kohe hakkama tööd tegema?}}

Jaa. Keskkoolis ma sain endale lubada laisk olemist isegi seal eliitkoolis, no vähemalt mingil tasemel. Ja ma sain keskkoolis arvutimängude mängimise isu täis mängida. Ostsin omale üheksanda klassi lõpus ZX Spectrum-i\index{Arvutid!ZX Spectrum}\sidenote{ZX Spectrum oli Sinclair Research'i poolt 1982. aastal Ühendkuningriigi turule lastud 8-bitine personaalarvuti, mõeldud peamiselt koduseks kasutamiseks. Selle kloone liikus Nõukogude Liidus hulganisti, skeemid olid koguni hobiajakirjades avaldatud} Leningradi turu klooni 1500 rubla eest, kui rubla juba kukkus. Siis oli suur rahanumber, aga ma sain oma isu täis mängida. Joystick\sidenote{Eesti keeles \enquote{juhtkang}. Eelmise sajandi algul Ameerika Ühendriikides patenteeritud, Teises Maailmasõjas Saksa vägede poolt laialt kasutatud ja kuuekümnendate lõpus arvutimängude külge jõudnud kaheteljeline juhtimisvahend. 21. sajandil kaotas ta mängude juhtimisel kiiresti populaarsust hiirtele ja on praegu peamiselt kasutusel lennunduses} sai peeneks mängitud, plastmassi paikasin alumiiniumiga. Tuttav treial tegi talle uue varre, pärast kippusid kontaktid läbi põhja tulema. Aga Spectrum oli nii hea arvuti, sellest sai aru igat pidi! Sai programmeerida BASICus ja Z80 Assembleris\index{Keeled!Assembler}. Sellest arvutist võis lõpuni aru saada. Elektroonikast peaaegu ka, välja arvatud videopildi genereerimise osa. Originaalis kasutati ULA kivi, vene variandis realiseeriti see laus-elektroonikana\sidenote{Originaalne ZX Spectrum sisaldas kahte suurt 40-jalaga mikroskeemi - Z80 protsessor ja üks eelprogrammeeritud loogikamassiiv (ULA - Uncommitted Logic Array). N-liidus tehtud Sinclairi koopiad kasutasid viimase asemel tervet trükkplaaditäit lihtloogikaelemente.}, sest seda kivi ei olnud kloonina võtta. Nii et ma sain sõbra Sinclairi diagnoosimisega hakkama. Näiteks, et sul on ROMi see ja see jalg lahti ja ei anna kontakti, seetõttu on tähtedel vertikaalsed kriipsud läbi, nagu dollarimärgid. Tähtede tabel oli ROMis ja kui seal bitt oli maas, siis joonistati selle biti koha peale alati täpp ja tekkis püstkriips. Järelikult pidi sellel ROMi kivil selle biti jalg mitte kontaktis olema.

\textbf{\enquote{See tähendab, seda, et sa pidid neid asju põhjalikumalt uurima?}}

Skeeme ma ikka kuskilt raamatutest ja mujalt nägin. Keskkooli lõpus, kui Venemaal käisin, ostsin metroost raamatu \begin{russian}Введение в схемотехники IBM PC / AT\end{russian}\sidenote{Eesti keeles \enquote{Sissejuhatus IBM PC/AT skeemitehnikasse}. Ilmselt peab Meelis silmas kodanike \begin{russian} Г. Н. Левкин\end{russian} ja \begin{russian}В. Е. Левкин\end{russian} 1991. aastal ilmutatud raamatut}. Venelased olid 286 skeemid välja ajanud arvuti järgi ja üles joonistanud. Neil oli seal viga, minu mälu järgi. Mingi reset signaali puhul oli aktiivne null ja aktiivne üks kusagil segamini, niisugust asja trükitud raamatus avastada oli igatahes lõbus. See Venemaal käik oli seesama kord, kui me olümpiaadil ja konverentsil käisime. Konverentsi osast ei teadnud me enne midagi, kui me sinna kohale sattusime. Meil ei olnud mingeid ettekandeid, kuulasime niisama, mis räägitakse. Ja vaatasime, mihukesed on kenamad tüdrukud. Üks vene Maša oli kõige kenam. 

Olümpiaadil me eriti hiilgavaid tulemusi keegi ei saanud. Mina sain meie pundist kõige parema tulemuse, sest ma ei joonud eelmisel õhtul alkoholi. Seda oli seal saada ja siis järgmisel hommikul pohmakaga inimesed ei esinenud oma võimete tasemel. Nii tuligi välja, et mina olin meie omadest parim, kuigi vähemasti üks kaasas olnud meestest oli parema peaga. Minu jaoks oli õppetund, mida rõõmsalt teistele edasi jagada: et näe, olümpiaadi tulemus sõltus selgelt sellest, kes ja mida eelmisel õhtul jõi. 

\textbf{\enquote{Räägi palun ülikoolist, me sattusime seal 1993. aastal kokku. Kuidas sulle see matemaatika tundus, mida me kohe esimese semestri alguses saama hakkasime?}}

See oli üks suur kukkumine. Ma näiteks mõtlesin ülikooli tulles, et ma tean, mis on reaalarv. Siis tuli matemaatilise analüüsi esimene loeng, kus hakati neid defineerima. Kõike hakati algusest peale defineerima, kõik muu ehitati ainult nende definitsioonide otsa. See kõik tahtis palju harjumist ja palju tööd aga mina ei olnud harjunud tööd tegema. 

Ma mõtlesin, et ma oskan programmeerida, kui ma ülikooli tulin. Aga Rein Pranki\index[ppl]{Prank, Rein} matemaatilise loogika õppeprogrammid näitasid, et on veel palju asju, millest ma aru ei saa. Seal joonistati näiteks ekraanile tõestuspuu ja ma mõtlesin, et \enquote{Vau, puud ma niimoodi joonistada ei oska}. Me õppisime seda küll hiljem umbes kolmandal kursusel Varmo Vene\index[ppl]{Vene, Varmo} Funktsionaalses Programmeerimises, kus me mingi \emph{minimax}i\sidenote{\emph{Minimax} on algselt nullsummamängude analüüsiks formuleeritud otsustusalgoritm, kuid mida on hiljem oluliselt laiendatud ning mis leiab laiemalt kasutust tehisintellekti puhul, statistikas, filosoofias ja mujal. Algoritm minimeerib võimalikku kahju halvimal, maksimaalse kahjuga, juhul andes optimaalse mängustrateegia eeldades, et ka oponent mängib optimaalselt} ülesande tüübi näiteülesandeks puu paigutust tegime. Esimese kursuse järel oleks seda ehk rekursiooniga ka kuidagi teha saanud, aga see oli jah näide sellest, et kõik ei ole ikka triviaalne. Ei saa igale asjale jõuga peale minna. 

\textbf{\enquote{matemaatiline analüüs, eriti matemaatiline analüüs II, võttis meil kursuse peal palju rahvast hõredamaks, see tahtis harjumist saada}}

Algebra tahtis ka. Kogu see matemaatiline lähenemine, et me ehitame asju üles mingite definitsioonide ja aksioomide otsa. Kogu see asi tahtis kõvasti tööd. Lisaks kukkusin ma esimesel kursusel haiglasse. Eksamisessiooni ajal ei jõudnud ma mõnesid eksameid tehtudki, tegin neid alles järgmise semestri sees. Käisin dekaanilt küsimas sessi pikendust, sest vanemad õpetasid, et nii tuleb teha. Siis dekaan ütles, et meie ajal enam niisugust asja pole, lihtsalt tehke need eksamid ära, kuidas saate. 

\textbf{\enquote{Mis hetkel oli võimalik minna arvutiteadust õppima?}} 

Mingid põhimoodulid oli vaja ära teha ja siis vist esimese aasta järel sai spetsialiseeruda. Kuna ma need moodulid sain kokku, siis kaldusin üldisest õppekavast kõrvale sellega, et läksin võtsin koos aasta vanematega põnevaid arvutiteaduse aineid. Käisin aasta vanema rahvaga koos lahedaid asju kuulamas. Ja siis järgmine aasta tuli võtta need ained ka, mis õppekavast tegemata olid. Minu oma kursus oli need ära teinud, mina tegin neid siis koos aasta noorematega. Mingeid tõenäosusteooriaid ja mingisuguseid matemaatikaaineid.

Juhtus ka seda, et ma kodutöö programme teiste pealt maha kirjutasin. Meil oli Algebra ja Analüüsi Numbrilised Meetodid, kus me arvutusmeetoditega numbriliselt tegelesime. Ma sain algoritmidest aru, nad ei pakkunud mulle algoritmi tasemel pinget ja ma ei viitsinud neid teha. Piisas, kui ma olin aru saanud, mis seal tehakse. Leidus üks lahke kaastudeng Jane, kelle programme ma esitamiseks kasutasin. Muutsin vist natuke treppimist ja muutujate nimesid. Mäletan, ma kirjutasin ühele kommentaaridesse üles \enquote{Viimati modifitseerinud Meelis Roos}\sidenote{Enne, kui vabavaralised tsentraliseeritud ja hajutatud koodirepositooriumid laialt levima hakkasid, hoiti koodi enamasti lihtsalt kettal. Seetõttu oli levinud praktikaks faili päisesse lisada kommentaar faili autori, viimase muutmise kuupäeva ja muu tarvilikuga}. Eks see praktikumi juhendaja teadis, et neid programme üksteise pealt üksjagu maha võetakse. Seepärast lasi ta endale ette seletada, mida see programm täpselt teeb, sellega polnud probleemi ja nii sain kõik asjad ilusti tehtud. Kirjutasin programme tüdrukute pealt maha, sest ma ei viitsinud programmeerida. 

\textbf{\enquote{Kas see ülikooli arvutuskeskus seal Liivi tänaval ei neelanud sind kuidagi endasse, nagu ta nii mõnedki neelas?}}\index{Tartu Ülikool!matemaatikateaduskond!Liivi õppehoone} 

Neelas ka mind aga natuke teistel viisidel. Mina ei kadunud ära Muda\index{Mängud!Muda}\sidenote{Originaalis \enquote{Multi User Dungeon (MUD)}. Paljude osapooltega reaalajaline tekstipõhine seiklusmäng. Täpsemalt siiski mängude alaliik, sest leidus mitmeid eri rõhuasetusega eri koodibaase kasutavaid versioone, mida jooksutati mitmetes eri serverites. Kuna Muda pakkus toona ainulaadset koos mängimise ja suhtlemise viisi, tekkis paljudel kiiresti sõltuvus ja liigne Mudas veedetud aeg oli sagedane ülikoolist välja langemise põhjus.} mängima. Muda oli küll tore: kui ma oma telneti klienti kirjutasin, sai seda Muda serveri vastu testida näiteks. Selleks oli Muda tore. 

\textbf{\enquote{Miks sa kirjutasid oma telneti kliendi?}} 

Võrguprogrammeerimise harjutamiseks. Tahtsin osata igasuguseid sokliühendusi teha. Ma kirjutasin oma netcati laadset asja, mis ei teinud mingisugust telneti \emph{handshake}'i  ja ei osanud \verb|echo off|i ja selliseid keerulisemaid asju, vaid lihtsalt sokli kuhugi ühendas. Sellise asja kirjutasin endale igasuguste asjade torkimiseks. Seal olid mingid mured stiilis et kui pikkade pakettidega asju saata ja vastu võtta võis. TCP võis andmed ju suvalise koha pealt ära hakkida. Ei saanud eeldada, et kui teiselt poolt rida sisse kirjutatakse, et sa selle täpselt rea suuruste tükkidena kätte saad. See oli põnev.

Aga mind neelas see arvutuskeskus natuke teistmoodi. Teisel korrusel Ülo Kaasiku\index[ppl]{Kaasik, Ülo} kabineti kõrval oli magistrantide arvutiklass, kus olid värvilised Sun'id. See oli ette nähtud magistrantidele, aga kellelgi ei olnud eriti probleeme, kui mina ka sinna imbusin. Aegajalt seal ei olnud kohti ja tuli ette, et ma kellelegi oma koha pidin loovutama, aga enamasti ei pidanud. Aasta vanema Raul Tölbiga\index[ppl]{Tölp, Raul} istusime seal koos ja seal sai õpitud ära Unix. 

Kuidas ma üldse sinna Unixit kasutama sattusin, oli omakorda lõbus. Seda ma võin lausa rääkida, kust on pärit minu kasutajanimi \enquote{mroos}. Minu esimene online konto oli masinas vask.ut.ee\index{Masinad!vask.ut.ee}. See oli VAX\index{Arvutid!VAX}\sidenote{Arvutisari, mille töötas DEC välja seitsmekümnendate keskel. Siiani üks kõige tuntumaid omalaadseid arhitektuure, oli ta PDP-11\index{PDP-11} edasiarendus, peamiselt mälu virtuaalse adresseerimise suunas. \emph{VAX - Virtual Address Extension}} tüüpi arvuti VMS\sidenote{VAX arvutite \enquote{kohalik} operatsioonisüsteem} opsüsteemiga. Selline umbes kuupmeetrine kast pluss kettad seal kõrval. Teine VAX oli rubiin.physic.ut.ee\index{Masinad!rubiin.physic.ut.ee} füüsikamajas. See oli MicroVAX, ainult sahtlitumba suurune masin. Vot need olid VMSid. Esimesel kursusel, selle asemel, et sessi ajal õppida, olin mina raamatukogust võtnud omale VAX/VMSi raamatu ja õppisin VMSi. Seal oli huvitavaid asju! Näiteks olid struktuursed failid. Sa võisid tekitada tühja faili, millel on ette antud kirjestruktuur. Opsüsteemi tasemel oli \emph{Record Management System}, millega mingis keeles kirjeldati struktuur ära ja tekitati selle kirjelduse järgi fail. Fail võis olla ka tühi, aga tal oli struktuur olemas. 

Kogu õiguste süsteem selles operatsioonisüsteemis oli keeruline. Windows NT\index{OS!Windows NT} on selle sisemiselt pärinud või umbes niimoodi. Nii keerukas ei ole minu meelest kui VMSis aga kui ma nägin Windows \emph{syscalli} \verb|CreateProcess| koos portsu argumentidega, siis tuli tuttav ette, sest VMSi SYS\$CREATEPROCESS oli umbes samasuguste argumentidega. SYS\$ käis syscallide funktsioonide nimede ette lihtsalt. 

Sealt ma käisin näiteks Lynxiga veebis surfamas. Tõmbasin FTP-ga mingeid faile, mida kuskilt kolmandat teed mööda kuidagi flopi peale sain. Käisin Internetis ka igasugu asju lugemas. Ma eriti ei programmeerinud VMSis. Kui vaja oli kursaõele Pascalis programmeerimist õpetada, aga ainult VAXu klass vaba oli, siis ma näitasin talle Pascalis programmeerimist VAXu peal. Ta oli väga üllatunud, et seda arvutit saab ka programmeerida. Aga sai. 

Seal oli lahe programm nimega SWIM, mis lasi ühe terminali peale multipleksida mitu akent, sai lausa akende suurusi muuta. Sellega ma kasutasin kolme rakendust korraga. Aga SWIM kippus ajama terminali hanguma, kõditas vist mingit VMSi terminali draiveri bugi või mida iganes. Siis tuli leida administraator, keda tihti majas ei olnud, või siis keegi sõber tudeng logis üle võrgu rubiini\index{Masinad!rubiin.physic.ut.ee} ja talk-is Ville Hallikuga\index[ppl]{Hallik, Ville}, kes oli sealne VMSi admin. Villel oli juurdepääs vaske olema ja ta sai tulla ja terminali päästa - hangunud terminali tagant ei saanud keegi enam midagi kasutada. Tappis SWIMi ja mingid asjad ära seal, nii et terminal sai jälle vabaks. Nii et SWIM oli tülikas. Keegi rääkis, et arvutiteaduse instituudi Sun'ides on Unixis programm nimega screen, millega sedasama teha saab. Ja siis tekkis mõte seda kasutada. Ma olin Unixit seni juba korra kasutanud. Math.ut.ee-s\index{Masinad!math.ut.ee}, kui tekkis online võrk, tuli 386BSD\index{OS!386BSD}. Ja see uuendati 93. aasta lõpus mingile uuele tundmatule opsüsteemile. Sinna osteti 486 arvuti asemele, suure kahe-gigase\sidenote{Meelis peab silmas kahte gigabaiti. Konteksti mõttes on oluline märkida, et tol ajal piisas keskmist sorti arvutifirma failiserveri kõvakettaks ühest gigabaidist üsna pikaks ajaks. Aastal 2020 täidab keskmine koduinternetiühendus selle mahu umbes minutiga} SCSI vindiga. Selle SCSI kaardi jaoks 386BSD enam ei sobinud ja pandi asemele mingi uus tundmatu asi nimega Linux\index{OS!Linux}. Versioon 0.99pl3 või midagi, kui õigesti mäletan. 

\textbf{\enquote{Kust selline asi sattus Tartu linna?}} 

No aga kust 386BSD sai? Internet oli ju olemas. Kasutajad koliti 386BSDst Linuxisse siuhti üle ja mul oli mingis Linuxis kasutaja. Jaanuaris umbes uuendati see Linux ära kerneli versioonile 1.0.2. Ma olin natukene nuusutanud Linuxit. Kui ma tahtsin seal Liivi tänaval Unixi screeni, siis math.ut.ee ühendus oli päris aeglane\sidenote{math.ut.ee asus füüsiliselt matemaatikateaduskonna hoones Vanemuise tänaval. Seega peetakse järgnevas silmas internetiühendust kahe, linnulennult 550 meetrise vahega, hoone vahel Tartu linnas}. 9600ne ühendus jagatud paljude kasutajate ja meilide ja muude vahel. Siis ma küsisin omale cs3-e (hilisem romulus.cs.ut.ee\index{Masinad!romulus.cs.ut.ee}) konto ja põhjendasin seda, et tahaksin näppida mõnda mitte-Linux Unixit. Seal oli Solaris\index{OS!Solaris}. Ja see tundus Toomas Soomele\index[ppl]{Soome, Toomas} piisavalt hea põhjendus. Toomas Soome kasutajanimi oli \enquote{tsoome}, ma mõtlesin, et ahaa, et eks Unixis käib see niimoodi. Küsisin siis omale tema süsteemi sama skeemi järgi kasutajanimeks \enquote{mroos}. Antigi. Seda ma olen sellest ajast edaspidi kasutanud igal pool. Isegi kui mul on kodus testarvuti, seal olen ma ka seal harjumusest mroos. Et tsoome mulle kasutajanime teeks, tuli öelda, et ma tahan Solarist kasutada ja kasutajanimi peaks ka samas formaadis olema, et võimalikult vähe küsimusi oleks. 

Mul möödunud aastal \sidenote{Intervjuu Meelisega toimus 2020. aasta kevadel} oli väga sürr kogemus, kui kevadel võttis minuga ühendust Toomas Soome, kellel oli siiamaani magistrikraad tegemata. Ta tahtis, et ma juhendaksin tema magistritööd. Ma mõtlesin, et muna õpetab kana, et mida mina siin teen. Aga tal oli korralik tehniline töö olemas ja mina teadsin, mismoodi üks magistritöö peab enam-vähem välja nägema. Sellest teadmisest oli kasu, see töö sai tal vormistatud magistritööks ja ta kaitses selle edukalt. Aga algul lihtsalt oli väga sürr reaktsioon. Arvutiteaduste Instituudis\index{Tartu Ülikool!matemaatikateaduskond!Arvutiteaduste Instituut} oli terve hulk rahvast, kes tegid oma magistrikraadi hiljem.

\textbf{\enquote{Kas sind teadust ei tõmmanud tegema?}} 

Ei, vot teadust tegema ei ole mind kunagi eriti tõmmanud ja keegi ei suutnud mulle ka auku pähe rääkida sel teemal. Väga ei proovitud ka. Meelitati erinevate viisidega, mingeid materjale ette söötes. Materjalid olid nii teadusega kui mitte-teadusega seotud. Näiteks Jaanus Pöial\index[ppl]{Pöial, Jaanus} jagas mulle omal algatusel kunagi \emph{Java Language Specification}i, et näe üks uus moodne asi. Selliseid asju ülikoolist ikka sattus. 

Ma mäletan, ma olin rebane ning ei olnud veel spetsialiseerunud Arvutiteaduse Instituuti informaatika erialale. Aga mul oli vaja kusagil välja trükkida viietollise flopi pealt mingit tekstifaili, raamatukogust mingisuguse kataloogi otsingu tulemus mingite raamatute otsimiseks. Äkitselt tekkis vajadus laupäevasel päeval trükkida. Ma lihtsalt vajusin kohale Liivi tänavale ja käisin mööda uksi koputamas. Oli vist laupäev ka või muidu õhtune aeg ja seal ei olnud palju rahvast. Sattusin Mati Tombaku\index[ppl]{Tombak, Mati} ukse taha, kes lahkelt lasi trükkida. Ja sellest tekkis nihukene tänutunne kogu selle instituudi vastu, et siin on lahked inimesed. See oli minu esimene isiklikul tasemel kontakt instituudi inimestega.

\textbf{\enquote{Millal sa tööle läksid?}} 	

Minu esimene ametlik töökoht oli Tartu Ülikooli Täppisteaduste Koolis\index{Tartu Ülikool!Täppisteaduste Kool} metoodik. See oli tegelikult postmasteri töö. Aga postmasteri nimelist ametinimetust ei olnud, oli metoodik. Korraldati programmeerimise kursust e-mailitsi koolides. Mina olin see, kes pidas arvet selle üle, kellel olid mis ülesanded lahendatud, ja saatis neile järgmisi. Arvutiõpetajad, kellele vastused saadeti ja kes neid parandasid, saatsid minule seisu ja mina siis selle järgi saatsin järgmisi ülesandeid. Mina olen laisk inimene. Esimesel tööpäeval võtsin nägin pool päeva vaeva ja kirjutasin skripti. Panin kuhugi tekstifaili valmis nimed. Programm võttis sealt järjest nimesid ja saatis neile ülesande ja pidas arvestust, et kellele on juba saadetud, et kellelegi topelt ei saaks. Ja kui ma selle skripti käima panin, siis rubiin.physic.ut.ee\index{Masinad!rubiin.physic.ut.ee}, tollane füüsikamaja Unixi server, kõristas umbes pool tundi. Pärastpoole ma õppisin \verb|nice| käsu\sidenote{Võimaldab Unixi keskkonnas kontrollida, kas kogu programm kasutab ära kogu saadaoleva arvutusressursi või jätab midagi ka teistele arvutikasutajatele} ka ära. Aga see tähendas, et kogu minu edasine töö pärast selle skripti kirjutamist oli copy-paste meili seest sinna sisendfaili ja skript tööle lükata. Automatiseerisin oma töö lihtsalt ära. 

\textbf{\enquote{Aga kuidas sa sinna sattusid?}}

Ma arvan, et Indrek Jentson\index[ppl]{Jentson, Indrek} Täppisteaduste koolist kutsus mind. Indrek oli matemaatikateaduskonnas vanem tegelane ja olümpiaadidega tegelenud. Ma läksin Täppisteaduste Kooli ukse taha, tuli Viire Sepp\index[ppl]{Sepp, Viire} vastu, kes juhataja oli, ütlesin, et tere, tulin töölepingut tegema. \enquote{Mis töölepingut?}, küsis tema. Ma siis seletasin, et Indrek Jentson saatis mind siia postmasteri töölepingut tegema. Kuskil 95. või 96. aasta algul, täpselt ei mäleta. 

\textbf{\enquote{See oli üsna vara ju? Tuleb häbiga tunnistada, ma läksin 93. aastal tööle juba}}

Te olite Veljo Haguga\index[ppl]{Hagu, Veljo} Korelis\index{Korel IN}, eks? Ma käisin Veljo töö juures vahel. Seal olid mingid mängud. Dune'i\index{Mängud!Dune} mängis Veljo näiteks õhtul näiteks millalgi kui ma sinna sattusin, vaatasin, kuidas see käib. Mängimisega ei olnud mul erilist suhet. Ma sain keskkooli ajal oma mängimise isu täis mängida Sinclairi peal ja lülituda juba programmeerimisele sellega, et ma tean, et see on palju põnevam asi. Ma kirjutasin näiteks oma \emph{binary editor}i, millega mängudest järgmiste levelite paroole välja nuuskida ja muid nihukesi asju. See oli juba keskkoolis, et sai igasugustel arvutiturva teemadel nuusitud ja huvi tuntud. 

Arvutiturva teema on mul keskkoolist saadik sees tõesti. Meil olid keskkoolis väga põnevad võidujooksud arvutiõpetajaga. Väga harivad. Näiteks oli õpetaja arvuti klaviatuur parooli all. Aegajalt tehti sellega meilivahetust, nii et masinal klaviatuur oli lukus aga muidu masin töötas edasi. IBM PS/2\index{Arvutid!IBM PS/2}\sidenote{PS/2 oli IBMi kolmas personaalarvutite põlvkond, mida tutvustati 1987. aastal. Paljud tolle masina innovatsioonid nagu näiteks VGA video muutusid \emph{de facto} standardiks pikkadeks aastateks}tedel oli mingi selline klaviatuuriluku võimalus. Küll ma üritasin leida meetodeid sellest mööda hiilimaks. Kui ma sain mingeid skeeme kuskilt näha, siis mul tekkis idee, kuidas i8042 klaviatuurikontrolleri kaudu teha masinale sobivat \emph{warm booti}, et sealt mööda hiilida, aga klaviatuurikontroller oli lukus edasi. Kirusin, et IBMi omad on kavalad olnud. See oli algul. 

Lõpuks selle arvuti parool saadi teada lihtsal viisil. Vaadati üle selle arvutiõpetaja õla, kes aeglasemalt tippis. Kui see oli teada saadud, ega me sellega midagi ei teinud, see ei olnud eesmärk. Aga minul oli edasi põnevam see, kui keskkoolis viimasel aastal oli 386d kohale jõudnud ja nende C: ketas, kõvaketas, pandi kirjutuskaitse alla nii, et mingi spetsiaalne draiver laaditi \verb|config.sys|-ist, mis tegi virtuaalse D: ketta ja keeras kogu C: \emph{read-only}'ks. Ja ma avastasin selle niimoodi, et mul oli mingi enda softi katsetamiseks see asi autoexec.bati või \verb|config.sys|i panna või sealt midagi välja kommenteerida, et minu asi ära mahuks või täpselt ei mäleta mis. Igatahes oli mul vaja sinna sekkuda. Kui ma sekkutud sain, siis ma pärast alati taastasin endise olukorra. 

\textbf{\enquote{Ka tol ajal mingit võrgu häkkimist ei toimunud?}}

Anto Veldre\index[ppl]{Veldre, Anto} rääkis jah\sidenote{Meelis peab ilmselt silmas varem eetrisse läinud memcpy episoodi Anto Veldrega}, kuidas tema poisid ülikooli adminidel ruutusid käest ära võtsid\sidenote{Unixi-laadsetel süsteemidel on root (mis eesti kõnekeeles mungandub tihtipeale sõnaks \enquote{ruut}) ees süsteemi täielike õigustega peakasutaja. Seega tähendab termin \enquote{ruutu võtma} arvutisüsteemi üle täieliku kontrolli saavutamist, tihti algset peakasutajat virtuaalse ukse taha jättes}. Tema jagas oma poistele modemeid ja terminale, mis tulid kuskilt humanitaarabina. Meil oli üks modem õpetaja arvuti küljes. Ühel poisil oli oma modem korra koolis kaasas, mida ta näitas, aga me ei osanud nendega midagi teha ja kohalikku võrku meil ei olnud. LAN\sidenote{\emph{LAN - Local Area Network}, kohtvõrk} tekkis meile alles 12. klassi kevadel, kui ma enam väga ei tegelenud sellega. OK, ma häkkisin LANtasticu\sidenote{LANtastic oli \emph{peer-to-peer} LANi operatsioonisüsteem, mida arendas Artisoft ja mis jäi hiljem Novelli ja Microsofti toodete varju} lahti \emph{social engineering}u meetodil. Sügisel pärast minu ära minekut oli kellelgi vaja saada LANtasticule juurdepääsu. Servermasinas oli nihuke koht nagu \emph{network control directory}. Seal olid andmebaasid binaarsena. Ja vot minu programm oskas käia ja binaarselt andmebaasi modifitseerida ja tekitada ühe administraatori juurde või panna kellelegi õigusi juurde või midagi. Ehk siis tuli meelitada noorem arvutiõpetaja flopi pealt ühte programmi käivitama seal masinas, viisakalt tänada ja puha. Tema poolt oli ka kõik OK. 

Aga varem oli see C:-ketta kirjutuskaitse. Algul me käisime Nortoni \emph{Disk Editor}iga kuskil seal \verb|config.sys| algust ära sodimas, et seda ei loetaks. Järgmisel tarkvara versioonil oli see koht paremini kaitstud ja siis oli vaja ikka flopi pealt bootida. Aga BIOS oli parooli all. A: ja C: vs C: ja A:. Noh, siis järelikult muugime BIOSi paroolid lahti. Need on obfuskeeritud kujul kirjutatud kuhugi CMOS-mälusse ja masina ROM oli välja loetav. Ma võtsin ja disassembleerisin selle Sourcereri-nimelise disassembleriga ja matemaatika tunni ajal kirjutasin omale matemaatika vihikusse kõrvallehe peale programmi, mis seda obfuskeeritud asja lahti võtab. Järgmine tund oli ajaloo tund. Läksin ajaloo tunnist ära arvutiklassi, realiseerisin selle programmi ära ja muukisin BIOSi paroolid lahti. Mul tuli suur pahandus, sest see oli ajaloo tund, kust väga paljud olid puudunud, õpetaja oli väga kuri ja keeras käkki. Mul oli pärast vaja see tund järgi teha ja õnnestus ikkagi. Põhjendasime ikka kui väga hea programmi me tegime spetsifitseerimata, mis see oli. Et väga hea idee oli ja tuli lihtsalt minna arvutiklassi ja kohe ära teha. Parool oli obfuskeeritud jadašifrina või baithaaval võibolla isegi, et otsast proovides järjest tähthaaval sai selle ära arvata. Ma kunagi arvutiõpetajalt küsisin, et miks teil nii imelik parool on. Siis ta lahendas selle turvaprobleemi niimoodi, et delegeeris osa vastutust arvutiklassi haldamises ja võttis appi arvutiklassi haldama. Väga hea pedagoogiline meetod, töötas. Ei häkitud enam, ei olnud enam huvi edasi jagada paroole, mida ma kätte saan. 

\textbf{\enquote{Aga kust sul see krüpto huvi?}}

Seda läks sealsamas kandis ka vaja. Näiteks meie õpetaja ässitas Norton Diskreet'i\sidenote{Diskreet oli tarkvarapaketi Norton Utilities 6.0 osa ning sisaldas paljuski kurikuulsat (Kevin Mitnicku\index[ppl]{Mitnick, Kevin} andmetel kasutati väidetud 56 biti asemel 30 bitist võtit, ka teised uurijad on osundanud mitmetele olulistele nõrkustele) DESi implementatsiooni} DESi\sidenote{\emph{DES - Data Encryption Standard} on sümmeetriline algoritm andmete krüpteerimiseks. Algoritm on oma väikese võtmeruumi tõttu tänapäeval kasutamiseks sobimatu (murti avalikult jaanuaris 1999), kuid oli siiski alates 1977. aastast USA föderaalse andmetöötlusstandardi (FIPS) osa.} kallale. DESist ma ei saanud jagu, ma ei saanud DESist arugi tol hetkel. Aga tema suunas. Ta oli üldse sedasorti kaval mees, et kui ta näiteks kuulis kunagi, kui meil pinginaaber Veljo Haguga\index[ppl]{Hagu, Veljo} oli plaan kirjutada viirus, siis ta suutis meid sellest eemal hoida. Me olime mingeid olemasolevaid viirusi disassembleerinud ja vaadanud, kuidas need käivad. Õpetaja sattus pealt kuulma, kui me rääkisime viiruse tegemisest ja ütles, et kui teha, siis teha kohe selline \enquote{stealth}-viirus. Me olime väga nõus, aga seda me ei viitsinud teha, ja nii jäi viirus tegemata. 

Ta leidis meile muidu ka rakendust. Keskkoolis üldine taustaülesanne oli midagi arvutada. Minu arvutusülesanne oli arvutada arvu $e$ kahe tuhande komakohaga 30 sekundiga 10 MHz 286 peal. Üks klassivend arvutas $\pi$-d tuhande komakohaga 60 sekundiga, sest see koondus aeglasemalt. Ja kust tulid ajapiirangud? Õpetaja oli vaadanud, kui kiiresti temal vastus tuleb selle arvuti peal. Ma sain 35-sekundilise programmiga juba viie kätte, sest vastus oli õigem kui õpetajal. Kuna need erinesid, siis ta võttis targa raamatu ja siis selgus, et minul oli õige. Mul oli selleks hetkeks 21 sekundiline programm, mis käigu pealt suurendas mingi hetk arvutüübi pikkust. Algul tegi lühema tüübiga ja hiljem pikemaga, et kiiremini saaks. Aga see oli veel bugine ja ei töötanud õigesti. Ma kontrollisin oma enda programmi vastu. Ma olin minut aega töötava programmiga algul tulemuse välja arvutanud tulemuse faili kirjutanud. Siis oli mul ka näiteks variant programmist, mis küsis, et kui mitme sekundiga oli vaja arvutada ja siis ütles \emph{hard-coded} vastuse. Aga see ei sobinud õpetajale. Aga 35-sekundiline juba sobis, kui vastus oli õigem tema oma. Minu 21-sekundiline ei läinud tööle, aga õpetaja seepeale võttis ja kirjutas ise asja haljas assembleris\index{Keeled!Assembler} ja sai kolme sekundiga. Muidu me kirjutasime Pascalis\index{Keeled!Pascal}. 

Teine asi, mida me tegime, millega oli keskkooli ajal hulga nuputamist, oli interferentsi simuleerimine arvuti ekraanil. On kaks punktlaineallikat ringlainetega ja tuleb arvutada, kuidas lained liituvad, et tekiks interferentspilt. Seal ma nägin ka vaeva, arvutasin ruutjuurt assembleris\index{Keeled!Assembler} Newtoni meetodil. Ma arvutasin iga ekraani punkti kohta pimesi selle faasi välja nii et ühtegi punkti näha ei olnud aga ma sättisin pikslite väärtused nii, et palett oli seatud üleni mustaks. Arvutasin kõik väärtused ära assembleris optimeeritud arvutusvalemiga ja õpetaja õpetas Newtoni meetodit sinna juurde. Oli abiks. Assembleris sai Newtoni meetodit tehtud! Oli väga hariv. 

Ja lõpuks ma siis ketrasin VGA paletti. Tehnilise dokumentatsiooni failid liikusid. Seal oli kirjas, kuidas VGA paletti muuta ja ma seadsin siis paletti niimoodi, et need värvid, mis mul on, liikusid sujuvalt heleduse järgi. Ja siis tulemus oli see, nagu oleks liikunud lained ekraanil. Ja see oli minu meelest tippsaavutus, see oli väga ilus sujuv liikumine selle kümne megahertsi juures, punkte üle arvutada poleks kuidagi jõudnud. Pärast viis õpetaja mind ühe teise õpetaja tehtud programmi vaatama. See teine õpetaja ütles, et tema ideest see alguse saigi, et interferentsi simuleerida. Tema tegi Juku peal \verb|circle| käsuga valgeid rõngaid üksteise ümber viiemillimeetrise vahega. Need läksid mida edasi seda aeglasemaks ja minu reaalajalise sujuva pildi vastu ei olnud see midagi. Mul oli tükk tegu, et mitte naerma hakata. Aga kiitsin siis takka. Meie õpetaja suutis anda sellise ülesande, mille peale mul kulus ikka kaua ja sain palju targemaks. Õpetaja oli Tarmo Ainsaar\index[ppl]{Ainsaar, Tarmo}. Seesama, kes suunas meid viiruse kirjutamiselt ära ja kes lahendas selle BIOSi paroolide haldusteema probleemi meiega nii, et probleemi ei tulnud. Väga hea õpetaja. Ta suutis meid suunata tegema õigeid asju nii, et me seejuures õpime ja paha peale ei lähe. 

\textbf{\enquote{Kuidas sa Cyberisse sattusid?}}

Ma töötasin HClubis\index{HClub} (sattusin HClubi tööle seoses sellega, et ma installisin sinna Linuxi serveri, \emph{gateway} veebi ja meili jaoks) ja mõtlesin, et mida võiks magistriks teha. Seal tegeldi hajusate andmebaasidega. Me saatsime SQL-käsk-haaval andmebaaside \emph{diff}e üle võrgu mitmes suunas. See oli põnev, me saime selle tehniliselt lahendatud. Algul käis see mul üle UUCP, hiljem üle PPP ja POP3 ja SMTP. Mina ehitasin Internetti sinna alla, oli ka põnev. Ja neid \emph{diff}e siis saatsime ja tekkis küsimus andmebaaside konsistentsusest: mis tingimustel jääb ja mis tingimustel ei jää andmestik konsistentseks. Et kas me saame mingi \emph{eventually consistent} mudeli sealt või mitte. Ma mõtlesin, et ma hakkan sel teemal magistrit tegema.

Aga HClubis Interneti teemal, mis mind huvitas tol hetkel, ei olnud mul eriti kuhugi areneda. Seal ei olnud kellegi teise käest sedasorti asju õppida. Kui siis, ise õppida ja ehitada. Asju, mida oleks võinud --- pisi ISP-na veel ehitada sinna ISDN sissehelistamiskeskus, kui oleks leitud raha ja et see rentaabel on. Nihukesi asju oleks ehk saanud. 

Samal ajal ma käisin mõnes koolis abiks Linuxit installimas ja käisin laenamas RedHati installplaati 1997. a. suvel Elmer Joandi\index[ppl]{Joandi, Elmer} käest Tartu lähedalt maalt. Tal oli see plaadina kohe olemas ja ei pidanud flopidega mässama. Elmer ütles, et Tarvil\index[ppl]{Martens, Tarvi} olla plaan Tartusse meiesuguste jaoks pesa teha. Ja siis mina käisin juunikuus umbes Tallinnas Cyberneticas\index{Cybernetica} Helger Lipmaa \index[ppl]{Lipmaa, Helger} juures, et tuleks magistrit tegema hoopis krüpto teemal. Ma mõtlesin, et näiteks pordiks OpenSSLi\sidenote{Teek arvutite omavaheliseks turvaliseks suhtluseks krüptograafia abil. Tuntuim ja levinuim omataoline} Windowsile, sest mul oli Windowsi all krüptot vaja olnud aga seda polnud kuskilt võtta. Sellest konkreetsest ideest arvati kehvasti, et keegi vist juba on portinud ka midagi. Aga Tarvi kutsus niisama mitte-krüptot progema, mitte-krüptot. Oleks peaaegu Küberisse tulemata jäänud. Helger kutsus mu ikka turva asju tegema. Ja siis kutsuti mind 1997. aasta sügisel Arula motelli Küberi\sidenote{Kõnekeeles on \enquote{Küber}, \enquote{Küberneetika AS}, \enquote{Küberneetika Instituut} ja \enquote{AS Cybernetica} sisuliselt sünonüümid. Aastal 1960 asutati Eesti Teaduste Akadeemia Küberneetika Instituut. 1997. aastal reorganiseeriti see Küberneetika AS-iks ja hiljem nimetati ümber Cybernetica AS-iks. Aga sisu jäi suuresti samaks} väljasõiduistungile ehk \enquote{kvartalnajale}\sidenote{Nii kutsutakse Cybernetica töötajate regulaarseid (algselt kvartaalseid, sellest nimi) ja legendaarsed meeskonnaüritusi.}. Seal oli kutsutud kogu tulevane Küberi Tartu Andmeturbelabor\index{Cybernetica!Andmeturbelabor}. Viljar Tulit\index[ppl]{Tulit, Viljar} seal oma habemese diktsiooniga ütles, et seda sa pead ikka ise suutma ära otsustada millegi järgi, kas sa tahad siia tulla või ei taha, kui ma ütlesin, et segane on veel, kas ma tulen või ei tule.

Aga läksin ma Küberisse tarkade inimeste juurde. Seal olid Arne Ansper\index[ppl]{Ansper, Arne} ja Viljar Tulit, kes oli kogenud süsadmin (kogenum, kui mina). Kui mina tegin näiteks tükk aega FTP otsingumootorit Nuuskur\index{Nuuskur} koos teiste tudengitega, siis Arne oli selle stiilis nädala otsa õhtutega ära teinud. Arnel oli Vosa\index{Vosa} nimeline FTP otsingumootor Eesti FTP serverite kohta. Vosa nagu \enquote{Vanaisa Oli Sulle Archie\sidenote{Archie oli üks esimesi Interneti otsingumootoreid, mis võimaldas otsingut üle FTP arhiivide}}. Tal oli ainult veebiliides, meil oli muid liideseid ka. Meil oli telneti liides ja archie prospero protokolli\sidenote{Archie kataloogides navigeerimiseks loodud protokoll, mida võib pidada tänapäevase www protokolli eellaseks. Prosperot kasutades võis terve Internet välja näha nagu üks suur ühine kataloogipuu} liides, millega vana archie klient töötaks, ja meililiides ka. Meil oli võimas vinge süsteem tehtud kamba peale. Kõike ei teinud mina, teised tegid ka. Ma olin lihtsalt üks vedajaid ning lõpuks see, kes tegi kõige rohkem tükke. Ja sellega selgus, et Arne on tark. Seal oli veel asju, millega see selgus. Näiteks tal oli Fido ja Interneti vaheline gateway. Ma olin selle kaudu Fido\index{FidoNet}\sidenote{FidoNet oli ülemaailmne arvutivõrk, mida kasutati BBSide omavaheliseks suhtluseks} lugeja. Ma pole päris Fidonetti kunagi näinudki. Minu jaoks Fido oli lihtsalt järjekordne NNTP\index{NNTP}\sidenote{\emph{Network News Transfer Protocol (NNTP). Useneti} uudiste vahetamiseks kasutatud protokoll} server stiilis keeks.ioc.ee\index{Masinad!keeks.ioc.ee}. Sinna tuli kasutajanime ja parooliga läheneda ja sai tavalise \emph{newsreader}iga lugeda ja kirjutada. Minu jaoks oli Fido teenus üle Interneti, mida vahendas Arne tehtud süsteem. 

\textbf{\enquote{Mis sa praegu teed?}}
Praegu ma olen Küberis\index{Cybernetica} turvainsener. Praktikas ka tarneinsener, kes pakendab lahendusi ja ehitab nende jaoks automatiseeritult mingeid keskkondi. Õpetan ülikoolis\sidenote{Tartu Ülikool}, olen ülikoolis hajussüsteemide külalislektor, õpetan operatsioonisüsteeme baaskursusena, andmeturvet baaskursusena ja magistrantidele õpetan turvalist programmeerimist. See viimane tegeleb sellega, kuidas teha nii, et koodis poleks auke. Mõni auk ikka kuskil leidub aga eks neid ole aja jooksul endale piisavalt vastu tulnud. Andmeturbe kursus sai tehtud siis, kui ma olin magistrant Helger Lipmaa\index[ppl]{Lipmaa, Helger} juhendamisel. Helger ütles, et kuule, et sa võiks ülikooli  sellise andmeturbe kursuse teha. Mõeldud tehtud, tegingi. Kellegagi eriti nõu ei pidanud. Küberi turvaraamatu\sidenote{Hanson, V., Lipmaa H., Buldas A., Ansper A., Tulit V., Martens V. \enquote{Infosüsteemide turve 1. osa}, 1997. Hanson, V., Lipmaa H., Buldas A., Ansper A., Tulit V., Martens V. \enquote{Infosüsteemide turve 2. osa}, 1998.} võtsin vihjete jaoks aluseks, vist esimene valdavalt esimese köite. 


\textbf{\enquote{See tundub olevat nii sinu moodi. Võtad, teed ja saab väga hea?}}
Parim kiitus, mis ma andmeturbe ainele kuulnud olen oli siis, kui hakati küberkaitse magistrikava tegema. Tallinnas oli sel teemal koosolek. Ja oli häda, et kui me tahame neile kõike seda õpetada, mida tahaks, ei mahu see meil ainetesse ära. Selle peale oli vist Enn Tõugu\index[ppl]{Tõugu, Enn}, kes ütles, et \enquote{Kuidas, Meelis jõuab andmeturbe kursuses neist kõigist asjadest rääkida, mahutame ikka magistrikavasse ka ära}. Mis sest, et põhjalikumalt, aga küll me mahutame. See oli hea kompliment kursusele, et Meelis räägib sellest kõigest. 




\chapter{Henn Ruukel}
\index[ppl]{Ruukel, Henn}

                 
\question{Nagu ikka, hakkame pihta päris algusest. Kuidas, lillekene, arvutid 
sinu juurde jõudsid? }

Ma mäletan, kuidagi kohe seda, et jõudsid kooli arvutiklassi kaudu.

\question{Mis kool see oli?}

Tallinna 10. keskkool\index{Koolid!Tallinna 10. keskkool}, tänane Nõmme 
Gümnaasium\index{Koolid!Nõmme Gümnaasium|see{Tallinna 10. keskkool}}. Ja kui ma 
õieti mäletan, oli see niimoodi, et  pärast mingit suvevaheaega, ma pean nüüd 
hoolega mõtlema, mis klass võis olla. Aga pärast suvevaheaega tulime kooli ja 
meie mata klassi oli ehitatud arvutiklass. Klassis olid sellised lauad, mis 
nägid välja nagu koolilauad aga sai lauaplaadi üles tõsta ja sealt seest tuli 
välja omakorda mingi  pööratava metallkonstruktsiooniga  
Elektronika\index{Arvutid!Elektronika} arvutid. Ja nad olid omavahel võrgus 
niimoodi, et  õpetaja laua peal oli siis flopidraiviga 
Iskra\index{Arvutid!Iskra}. Nendes Elektronikates jooksis ainult Basic. 
Lükkasid käima, Basic\index{Keeled!BASIC} kohe jooksis, said koodi 
kirjutada ja olid Basic-u \emph{command}-id, millega sai oma programmi \emph{store}-ida oma 
sinna Iskra flopikettale. Ehk sa viisid õpetaja kätte oma 
flopi ketta, ta pani selle masinasse, tegi midagi seal Iskras ilmselt ka. Ja 
siis said salvestada oma proge maha või pärast järgmine päev laadida. Ja nii 
kui see juhtus ära, siis muidugi me hakkasime manguma kohe mata õpetajat kes 
oli Ramil Izamentinov\index[ppl]{Izamentinov, Ramil}, kes on selles mõttes äge 
mees, et \emph{never} oma elus ma pole enam näinud inimest (ta oli astmahaige), 
kes nii ägedalt naeraks oma naljade üle. Ta andis meile matat, eksole, ja kui 
ta tegi mingi nalja, siis et kõik saaksid aru, et see oli nali, tegi ta sihukest 
spetsiifilist kähinat. 

No, \emph{anyway}, ta lasi meil pärast tunde olla seal klassis ja me 
hakkasime kirjutama igast suvalisi progesid. 

\question{Mis klassis see oli?}

See pidi olema enne keskat kõvasti. Ma arvan, et see võis olla mingi seitsmes 
klass või? Järelikult, kui seitsmes klass, siis 1986? Aga see oli kindlasti 
enne keskat. Ja minu arust see kooli arvutiklassi aeg, kui ma seal käisin, 
kestis suurusjärk ühe õppeaasta. Põhiliselt sai kirjutatud mingeid progesid. 
Mänge seal üldse ei olnud, selles mõttes äge \emph{envrionment}, et  
ei olnud lihtsalt ühtegi arvutimängu, mida mängida. Nii, et kõik mängud, mis me 
tahtsime, et saaks mängida, tuli ise Basic-us teha, 
 kirjutada flopi peale ja sealt pärast laadida. Mingeid 
lihtsaid mänge me tegime, trips-traps-trullid ja asjad. Ma mäletan kindlasti, 
et äge \emph{challenge} oli\ldots Elektronikal  oli graafiline liides, nii et 
sa said ekraanile joonistada. Näiteks, ma mäletan, kindlasti üks programm oli 
selline, et sa andsid ruutvõrrandi parameetrid, ta arvutab lahendid välja pluss joonistab 
graafiku. Selle järgi kusjuures saaks õppeaasta \emph{detect}-ida. Selliseid asju 
tegime. 

\question{Oot, aga miks te hakkasite manguma, et nondele arvutitele ligi saaks?}

Seda ma ei mäleta. 

Aga igatahes see kuidagi tundus kohe sihuke asi, et tahaks ja oleks äge, et kas 
võib käia ja mis õhtuti. Ikkagi algul oli vist nii, et õpetaja oli ise kohal, 
aga vist  juba lõpuks oli meil mata klassi võti. Aga see on kõik nii ammu. Paar huvilist 
oli veel minu klassist, kellega me käisime seal, ja eraldi \emph{arrangement} 
tuli õppealajuhatajaga teha selleks, et saada algklasside pikapäeva rühma söögi 
peale. Et ei peaks koolist ära käima, kui tunnid lõpevad, et me saaksime 
väikeste lastega koos sööklas süüa ja siis minna arvutiklassi õhtuni. 

Eks ta oli ikka nii põnev asi, see oli ikka hoopis teine aeg ju, kodus polnud 
videomakkigi arvutitest rääkimata või et kellegi töö juures oleks arvuti, 
sellist asja polnud. 

Kui nüüd sealt edasi liikuda, kuidas järgmine faas tuli, oli see, et ma kuulsin 
ühelt oma kaugelt sugulaselt mingil pere sünnipäeval, et tema käib TPI-s 
arvutiringis\index{Arvutiklubi!TPI arvutiring}. Ja see on see kuulus 
arvutiring, mida pidasid Julius\index[ppl]{Raimla, Tõnu}\index[ppl]{Julius|see{Railma, Tõnu}}\sidenote{Henn peab ilmselt silmas Tõnu Raimlat toonase hüüdnimega \enquote{Julius}} ja Aare Tali\index[ppl]{Tali, Aare}. 

Nad pidasid TPI-s nihukest \emph{need to know 
basis} arvutiring, et sa pidid teadma, et sa lähed teisipäeval Raadiotehnika 
kateedri\index{Tallinna Tehnikaülikool!Raadiotehnika Kateeder} taha otsa ja 
ootad ühe ukse taga. Seal oli sinusuguseid jõnglasi nagu murdu. Ja mingil 
kindlal kellaajal Julius tegi ukse lahti. Klassis oli  
Yamaha-d\index{Arvutid!Yamaha MSX}, MicroBee-d\index{Arvutid!MicroBee} ja 
Robotron 1715-d\index{Arvutid!Robotron!Robotron 1715}. Kolmes reas. Keskel 
MicroBeed,  ääres olid Yamahad ja akna ääres Robotronid. Yamahade peale käis 
tegelikult see \emph{run},  kõik tahtsid Yamahadesse saada, sest Yamahades olid 
mängud ja kihvtid mängud, onju. MicroBeed eriti ei huvitanud kedagi ja taga 
Robotronid oli ka nagu põnevad. Mina põhiliselt olin Robotronide peal. Seal sai 
progeda seal sai, ma mäletan, mingeid kooli referaate teha. Yamahadesse mina 
kunagi löögile ei saanud. 

Yamahad olid 3.5 tolliste diskettidega juba. 

See oli nagu järgmine level. See pidi olema 1988. aasta.

\question{Ja oligi nii, et uks tehti lahti ja kes istuma sai, sai istuma?} 

Jaa. Kui midagi ära lõhkusid, siis oli selleks päevaks \emph{ban} aga järgmine 
päev võisid tagasi tulla. Ma mäletan seda selle pärast, et ma ükskord istusin 
mingi Robotroni taha, kus oli mingi probleem klaviatuuriga ja ma võtsin teise 
masina klaveri. Sellel on mingi imelik pistik, millega klaviatuur masina külge 
käis ja ma panin selle kuidagi niimoodi kehvasti sisse, et rikkusin pistiku 
pin-id  kõveraks. Ja siis tuli minna taha ruumi  
ja  öelda, et näed,  niisugune jama. Ma ei mäleta, kumb neist tuli, kas Julius 
või Tali, vaatas, ütles, et \enquote{okei, tänaseks kõik,  tule järgmine 
teisipäev tagasi}. Nad nagu hoidsid korda.  Ürituse  
 lõpp sõltus sellest, millal nemad enam ei viitsinud olla ja tahtsid 
koju minna. Sest tegelikult nad olid mingid töötajad seal kateedris ja ma ei 
tea, mida nad seal tagaruumis ise tegid, ilmselt progesid. Aga, ühesõnaga, meie 
olime kõik jõnglased seal, \emph{average} oli mingi 13. Ma praegu mõtlen välja, 
et see pidi olema 1988.  aasta, sest 89 läks mu ema tööle 
Diagnostikakeskusse\index{Diagnostikakeskus}, millest ma pärast poole räägin ja 
sinna ma juba liikusin edasi. 

Ja kuidas nad siis korda hoidsid, oli toitekas. Jõnglased, arvutiruum, kõigil jube 
põnev, kes mida teeb , kes progeb, kes mängib mänge, kes häkib mänge, et teha 
sinna mingisugused oma tegelased sisse. Saad teisipäeva õhtul umbes õhtul kell 
kuus uksest sisse. Ametlik lõpp oli ka, aga tegelikult keegi ei pidanud 
sellest kinni. Selles mõttes see arvutiring lihtsalt oli, mind ei pandud mitte 
kunagi kuskile kirja, keegi ei õpetanud mitte midagi, ise teed. Ühel hetkel 
viskas kuttidel alati üle ja siis nad tegid tagant ruumi ukse lahti ja ütlesid 
\enquote{viie mintsa pärast toitekas}. See oli kõik, mis nad ütlesid. Viie 
mintsa pärast toitekas. See tähendas seda, et sul on viis mintsa aega mis 
iganes sul pooleli on ära salvestada, sest viie mintsa pärast emb kumb neist  
tuli,  ei hakanud nende jõnglastega midagi vaidlema, jalutasid kilbi juurde, 
tõmbasid, laks, peakilbi välja. Pimedaks. Ja see oli signaal, et nüüd tuleb koju 
minna. 

\question{Võru 1. Keskkoolis oli sama skeem ja kedagi ei huvitanud, mis selle 
arvutiga juhutub. Minu meelest ka ei juhtunud suurt midagi.}

Täiesti \emph{ruthless}. Väge ei juhtunud aga eks ilmselt oli oht, et kui 
kirjutamine jäi pooleli, siis ilmselt mingi \emph{disc corruption} võis 
juhtuda. 

See oli tegelikult äge periood. Sealt ma mäletan paari tüüpi uduselt, aga seal 
\emph{collaboration}-it või suhtlust oli nagu vähe. Oli justkui mingi 
generatsioon Yamahade juures, kes tundsid teineteist ennem ja kes hoidsid 
kokku, aga ülejäänud  igaüks  nokitses omasoodu. 

Aga see kõik oli ikkagi alles algus. Aga niisugune \emph{breaktrhough}, kus 
mina jõudsin paradiisi ja õndsusesse  toimus  tänu Mihhail Gorbatšovile. Sest 
Mihhail Gorbatšov, kes oli Nõukogude Liidu  kompartei esimees, otsustas 
perestroika käigus aastal 1986 või midagi sellist, et meditsiiniga üle Nõukogude 
Liidu on üldiselt probleem. Ja et kõige suurem probleem Nõukogude meditsiinis 
on see, et pannakse valesid diagnoose või diagnoosi jaoks info või analüüsid on 
ebatäpsed, valed või liiga aeglased. Seega tuleb investeerida infrastruktuuri, 
millega meditsiinipersonal saaks kiiremini teha ära olulised mõõtmised ja 
analüüsid. On see mingi südame EKG, on see vereanalüüs, on see siis mingi 
magnetresonantstomograafia. Nihukesed asjad, mis on täna põhimõtteliselt  
külahaiglas olemas. Selliseid asju polnud ja vereanalüüsi tegemine võttis mega 
kaua aega, \emph{result}-id tulid tagasi ebatäpsed, kõik sihuke kraam.  Ja see 
oli globaalselt Nõukogude Liidu suur probleem. Gorbatšov otsustas, et ta teeb 
sellise liigutuse, et kuna tal ei ole piisavalt valuutafondi\sidenote{Nõukogude 
elu valitsesid, laias laastus, kaks terminit: fond ja limiit. Fond ütles, kui 
palju midagi kellegi jaoks olemas oli ja limiit seda, kui palju midagi kas 
toota või tarbida tohtis. Ehk, \enquote{pole piisavalt valuutafondi} tähendab 
tänases mõistes \enquote{ei olnud eraldatud piisavalt valuutat}.}, et teha neid 
võimalusi igasse jumala haiglasse, siis igasse Nõukogude Liidu osariiki, 
näiteks Eestisse, pidi pealinna moodustatama selline asi nagu 
diagnostikakeskus. Kuhu  investeeritakse valuutafondi, ostetakse välismaalt 
\emph{top notch} tehnika sisse ja terve vastava osariigi analüüsid või uuringud 
tehakse ühes kohas. A la, kui sul on mingi kopsuhaigus, siis sind ravib, jah, 
sinu arst näiteks Mustamäe haiglas, aga Mustamäe haiglasse ei ole ressurssi 
osta mingeid kopsuanalüsaatoreid, vaid need ostetakse ühte kohta,  tehakse 
eraldi \emph{facility}, eraldi asutus.

No vot, ja Eestisse pidi ka tulema sihuke. Mis seal oli oluline oli see, et 
 tol ajal kehtis Nõukogude Liidule välismajandusembargo, ehk Nõukogude 
Liitu ei saanud sisse tuua lääneriikide moodsat tehnoloogiat, \emph{except} 
meditsiinitehnoloogiat, see oli okei. Mille pärast meil ei olnud PC-sid siin 
oligi see, et  embargo oli peal. Isegi kui mingid välkarid oleks tahtnud müüa, 
isegi kui siin oleks kellelgi valuutat  olnud tol ajal, ei saanud  
selliseid asju osta. 

Hakati siis Tallinnasse moodustama seda 
diagnostikakeskust\index{Diagnostikakeskus}. Ago Kivilo\index[ppl]{Kivilo, Ago} 
nimeline mees seda vedas, tänaseks on ta siit ilmast kadunud. Keskus pidi tulema 
sinna, kus praegu on vana Tallinna Panga maja Tallinna linnaosavalitsuse 
kõrval. Muidugi läks eht eestlaslikult kemplemiseks selle ümber, et kes 
saab keskuse juhiks, millise haigla kõrvale see üldse teha, et asukoht ei ole 
ikka hea, haiglatest kaugel ja nii edasi. Et äkki teha Mustamäe haigla kõrvale. 
Kivilo ütles, ma teen selle ise, ma teen eraldi asutusena. Mustamäe haigla 
ilmselt tahtis keskust umbes enda allasutuseks. Ma neid detaile ei tea, sest ma 
olin umbes 14 tol ajal, aga natuke tean selle pärast, et mina olin see mees, 
kes aitas Agu Kivilol teha kõiki powerpointe, mis tal oli vaja teha. Tol ajal 
see oli \emph{skill}. 

\question{Kas PowerPoint oli olemas tol ajal?} 

Kohe jõuan sinna. Miks meil oli PowerPoint olemas, miks oli niisugune maagiline 
kohta olemas nagu Diagnostikakeskus, oli siis see, et Eesti riigis selliseid 
asju nagu laserprinterid, PC-d, värvilised kuvarid, arvutivõrgud polnud olemas, polnud 
nähtud. Okei arvutivõrke oli, aga need olid pigem 
CP/M-ide arvutivõrgud ja sellised, Tehnikaülikoolis või niimoodi. Aga kuna 
Diagnostikakeskus oli meditsiiniasutus ja  nad ostsid Soomest 
meditsiinitehnikat, siis oli Soomes olemas  mingisugune OY, minu arust see oli 
Pekka OY. See oli reaalselt mingi omaniku eesnimi,  võib-olla oli mingi sõna 
veel. Ja tema, ma saan aru, oli põhimõtteliselt see vahendusfirma, kes vahendas 
Soomest igasugu kopsuanalüsaatoreid, vereanalüsaatoreid ja sihukest kama. Ma kujutan 
ette, et juriidiline skeem pidi ka suht keeruline olema, kuna see kõik liikus 
mingite Moskva valuutafondide kaudu. \emph{Anyway}, aga kuna see kaup liikus, 
siis selle kauba raames tuua Eestisse \emph{top notch} PC-sid polnud mingi probleem. 
Sest \emph{on paper} paisits see kõik nagu meditsiinitehnika. 

\question{Ta vist on võrreldes muu tehnikaga ka odav?}

Esimene magnetresonantstomograaf tuli sealt kaudu ju. Millega omakorda 
\emph{back} või \emph{side story} on see, et Gorbatšov eraldas valuuta, valuuta 
oli reaalselt olemas aga seda ei suudetud ära kulutada, sest kulus lõputu aeg, 
\emph{calendar time}, selle peale, et vaieldi, kuhu  keskuss ehitada. Aga 
selle tomograafi jaoks peab maja vundament juba spetsiifiline olema, et ei oleks 
värinaid ja vibratsioone. Sinna kulus aeg ja Nõukogude Liit hakkas juba lagunema 
aga valuuta oli ikka veel kontol. Ja oli nagu reaalne probleem, kuidas see 
raha kuhugi ära kulutada, enne kui ta kaduma läheb,  
Kivilo\index[ppl]{Kivilo, Ago} tegeles sellega. See tomograaf näiteks saadigi 
Eestisse kätte, see on tänaseni olemas. Kokkuvõttes ehitati keskus  Magdaleena 
haigla kõrvale. Kivilo sai selle tomogoraafi niimoodi kätte, et kas   
tarnijafirma või mingi vahefirma (neid detaile ma ei tea) sai sisuliselt 
\emph{prepaymenti} ja kunagi aastaid hiljem reaalselt tarnis selle seadme. 

\emph{Anyway}. Selle kõige käigus tuli ka arvuteid. Kuidas mina sinna 
sattusin, oli see, et mu ema läks sinna vereanalüsaatori peale tööle, ta on med 
taustaga inimene, apteeker. Ja küsis Kivilolt, et kas tema poeg võib siin 
arvutis käia, umbes pärast kooli. Kivilo ütles, et las poiss käib. Seal ma sain 
kokku selle mehega, kelle nimi on Mart Palmas\index[ppl]{Palmas, Mart} ja kes 
põhimõtteliselt on Eesti IT-sse toonud  mind ja Madis Kaalu\index[ppl]{Kaal, 
Madis}. Madis Kaalu ta püüdis kuskilt Tipi-kooli pealt kinni, kes oli sealt 
välja kukkumas, ütles, et \enquote{kuule, tule nüüd, ma panen su arvutite 
peale}. Nii et Palmas hakkas õpetama mind  progema, ennem olin oma käe 
peal harjutanud. Ja siis ma käisin seal niimoodi, pärast tunde. 

\question{Mida Mart Diagnostikakeskuses tegi?}

Diagnostikakeskuse arvutite hooldamiseks oli loodud väikeettevõte 
Skriining\index{Skriining}, mis on siiamaani olemas, Kalle 
Lotamõis\index[ppl]{Lotamõis, Kalle} asutas. Mart töötas seal programmeerijana, 
ma arvan. Aga noh, põhimõtteliselt, mida ta tegi oli, et hoidis neid arvuteid 
korras või kogu Skriining hoidis neid arvuteid korras. Ja mina sain seal 
hängida nagu \emph{for free}, ma ei olnud kuidagi juriidiliselt seotud. 
\emph{Except}, minu esimene üldse töö, välja arvatud TPL-id\index{TPL - Töö ja 
Puhke Laager. Nõukogude koolilastele pakutud võimalus suviti organiseeritult 
tööd teha ja elu nautida.} ja rohimised, oligi üks suvi seal, Mu esimene 
\emph{task} oli vedada maja peale laiali koaksiaalvõrk, mille otsa me panime käima 
Novell Network-i. Tuli paigaldada  arvutitesse võrgukaardid, häälestada, 
õiged IRQ-d, mingi soft tuli peale panna, kõik sihukesed asjad. See oli mu 
esimene suvetöö, ma mäletan.

\question{Kes sulle sellise ülesande andis ja miks oli tal alust arvata, et sa 
seda teha oskad?}

Eks ma olin esiteks sellel kevade läbi hänginud juba vähemalt. See oli 
tegelikult Kalle\index[ppl]{Lotamõis, Kalle}, Skriiningu juhi, otsus. 
\enquote{Kuule, et kas oleks mingit suvetööd?} \enquote{No tule, pane neid 
arvuteid siin kokku ja võta kastist välja, pane üles ja kui kellelgi on 
mingi mure, siis aita}. Me olime kõik selles Suur-Ameerika 18 majas. 

Ma tahtsin rääkida lihtsalt sellest, et see kontekst, kus me viibisime\ldots Ma 
rääkisin varem, et meil olid PowerPoint-id ja  laserprinterid. Aga, kujutad sa 
ette, et (ma täpselt ei tea muidugi, aga ma arvan) terves Tipi-koolis oli heal 
juhul suurusjärk 10 8086-t, ehk siis XT-d, ja ilmselt mustvalge 
\emph{display}-ga. Meil oli reaalselt üks ruum, mis oli triiki täis 
avamata arvuteid, mis olid juba kõik 80286-d, kõigil 40 mega IDE vinti ja VGA 
graafika. Lihtalt \emph{unpacked}. 

Nii et noh, kui näiteks Mamers\index[ppl]{Mamers, Tarmo} meile tööle tuli, 
siis ta \emph{unpack}-is omale kohe kaks tükki! Ühe peal jooksis BBS, teise 
peal tegid tööd, mis iganes see töö siis oli. Ega paljuski, kui ta panna 
tagantjärele konteksti, mida tänapäeval tööks nimetatakse, siis seal tööd 
tegelikult ei olnud. Selles mõttes, et meditsiinipersonali arvuteid oli 
suurusjärk kümme, neid hooldas suurusjärk kümme inimest, kes põhiliselt 
lihtsalt kaifisid seda, mis tehnika keskele nad sattusid, on ju. Ega me tegime 
muidugi kõik asjad ära, mis  teha oli vaja. Aga see oli see, kus ma 
arvan, mina sattusin reaalselt sellisesse niivõrd viljastavasse keskkonda. Mul 
olidki seal Mamers\index[ppl]{Mamers, Tarmo}, Palmas\index[ppl]{Palmas, Mart}, 
Hannu Krosing\index[ppl]{Krosing, Hannu} astus korra nädalas läbi. See oli ka 
koht, kus ma sain aru, et minust ei saa kunagi progejat. Sest ma mäletan nii 
hästi, kuidas ma nädal aega pusin millegi kallal ja siis tuleb Hannu, ma näitan 
talle selle nädalaga kirjutatud koodi ja ta võtab paberilehe ja sisuliselt 
kirjutab kahe reaga sellesama koodi. Turbo C-s.\index{Keeled!Turbo C}

\question{Leidsid ka kellega ennast võrrelda!}

Ma sain aru, milline on delta, onju. Võib-olla see ei olnud ainus põhjus, aga 
see oli üks koht, kus ma sain aru, et talendi või \emph{skill}-i vahe on ikka 
hüpersuur.

\question{Hannuga võrreldes on kellega iganes talendivahe väga suur!} 

Tõsi, tõsi.

Tol ajal põhiline projekt, mille kallal ma töötasin ja millel oli ka üks 
\emph{user}, oli selline programm, mille nimi oli clabel. Sellele ma ostsin 
ametlikult 25 krooniga juba Madis Kaalult\index[ppl]{Kaal, Madis} graafilise 
liidese \emph{library}, millega sai juba menüüd ja pop-uppe ja selliseid asju 
teha. Nimi oli SLACK, \emph{Slim} \ldots midagi sellist. 25, toona juba, 
krooni maksin selle eest, et mul oleks ametlik \emph{lifetime} litsents. 
Kirjutasin siis sihukese programmi nagu clabel. Programm tegi  sellist 
asja, et sul oli muusika (meil olid tol ajal kõigil kopeeritud muusika 
kassetid\sidenote{Vaata ka märkus lk \pageref{sisu!kassetid},}), sul oli kassette palju ja sa tahtsid omale normaalset andmebaasi, et 
mis kasseti peal sul mis lood on, mis bändid sul on. Pluss, sa tahtsid need 
välja trükkida sellisena, et sa saad nad \emph{fold}-ida ilusti ümber kasseti 
ja panna kasseti karbi kaane alla. Ta trükis kohe sellise paberi välja, et üleval serva 
peal oli näha, mis on A- ja mis on B-poolel ja suure külje peale trükkis 
A-poole kõik laulud ja B-poole omad. Pluss oli siis graafiline liides, kus sa 
said brausida ja edida ja printida. 

\question{Minu esimene arvutiga teenitud raha tuli täpselt samasuguse softi 
abil.}

No vot. Mul oli üks \emph{user}, kelleks oli Toivo Annus\index[ppl]{Annus, 
Toivo}. Väga kasulik \emph{user}, sest ta reaalselt kasutas seda ja tegi  mulle 
kogu aeg bugireporte.

\question{Kuidas sa Toivoga kokku said?}

Fido kaudu. Ma ilmselt \emph{upload}-esin selle softi kuskile BBS-i ja Fidos 
promosin, et mul on niisugune asi. Toivo hakkas kasutama ja hakkas mul külas 
käima, rääkima, mis ei tööta, mis töötab. Ma arvan, et tema oli 16, mina olin 
umbes 14. 

\question{Järelikult juba Diagnostikakeskuses sukeldusid sa Fido maailma?}

Ma ei mäleta, millal Fido tekkis, ma isegi ei mäleta, kumb oli enne, kas minu
 Diagnostikakeskusse minek või Fido tulek. Ilmselt 
oli nii, et ma algul läksin ja Fidot polnud ja siis millalgi tekkis. Täpselt 
sündmuste ahelat enam ei mäleta. Aga Fido ja BBS-id oli meil seal mingi hetk 
täitsa nagu \emph{bread and butter}. Mardil\index[ppl]{Palmas, Mart} oli oma 
\emph{node}, muidugi Mamersil\index[ppl]{Mamers, Tarmo} Mambox. 
BBSummer\index{BBSummer} tuli ka juba, esimene toimus kas äkki esimesel või  
järgmisel suvel. Diagnostikakeskusel lisaks 
kõigele muule oli ka humanitaarabina Rootsist saadud täiesti töökorras Volvo 
põhjale ehitatud kiirabiauto. Kuna diagnostikakeskusel polnud sellega 
midagi teha, sõitis sellega ringi Mart. Aga sellel autol olid kõik  
operatiivauto load olemas, vilkurid peal, täismäng, taga oli kanderaam, 
põhimõtteliselt pane inimene sisse. Sellega, ma mäletan, käisime mina, 
Mamers\index[ppl]{Mamers, Tarmo}, äkki ka Kaido Kärner\index[ppl]{Kärner, 
Kaido}, toomas Saku Õlletehasest BBSummeri jaoks õlut. Sellega oli hea vedada, 
hea suur. Esimene summer toimus Väänas Tugamanni Veskis, mina veel läksin 
lõpuks sinna mopeediga kohale. See oli, vaata, juba niisugune ülemineku aeg, 
kus poes polnud midagi saada. Ma ei mäleta, kelle tutvuste kaudu kuidagi saadi 
Sakuga kokkuleppele, et sealt sai otse õlut osta. 

\question{Kas see oli see kord, kui, nagu Mast\index[ppl]{Kaal, Madis} rääkis, 
õlut sai villitud Fanta kankudesse ja õllel oli apelsini mekk man?}

Täitsa võimalik. Mina olin selles vanuses, et  mina õlut ei  joonud, 14 või mis 
iganes, ja ei oska kommenteerida. Aga mäletan, et \emph{event} oli kihvt. Ja ma 
mäletan, et see esimene BBSummer oli ka koht, kus ma esimest korda kuulsin 
asjast nimega Internet ja asjast nimega emaili aadress. Tartu Füüsika 
Instituudis\index{Tartu Ülikool!Füüsika Instituut} oli ka mingi kamp, lausa 
mingi perekond itikaid, aga ma ei mäleta, mis nende nimed olid\sidenote{Tarmo Mamers\index[ppl]{Mamers, Tamo} arvab, et tõenäoliselt oli tegu perekond Pruulmannidega ehk muudestki juttudest läbi käiva Jaan Pruulmanni\index[ppl]{Pruulmann, Jaan} ja tema abikaasaga.}. Igatahes keegi 
nendest pidas ettekande ja ta oli tolleks ajaks juba käima pannud Interneti ja 
Fidoneti vahelise \emph{gateway} meili jaoks. Nii et põhimõtteliselt kõigil 
Fido inimestel oli sel ajal, ja sisuliselt \emph{effectively} kohe ka mul,   
emaili aadress. Ma ei mäleta, mis see tagumine ots oli, domeenid ja asjad, aga 
see ülejäänud ots moodustus sellest, mis \emph{node} küljes olid ja kes sa 
olid. Nii et teoreetiliselt sai saata mulle emaili ja mina sain saata välja. 
Aga ma ei mäleta, et ma seda praktikas kasutanud oleks. Mul oli 
too emaili aadress isegi kuskile märkmikusse üles kirjutatud, aga sellega 
polnud kellelegi saata. 

\question{Esimese faksiomaniku probleem. Aga järelikult siis BBSummeril oli  
hariduslik või sisuline sisu ka?} 

Jah. Oli ikka väga palju loenguid, või \emph{knowledge-sharing}-ut. Minu, 
14-aastase, arust oli väga äge. Mäletan Martiini\index[ppl]{Martiini|see{Rinne, 
Martin}}, Martin Rinne\index[ppl]{Rinne, Martin}, demos seda, mida saab teha ja 
mida nad teevad Amigaga. Ta oli Eesti Televisiooni\index{Eesti Rahvusringhääling!Eesti Televisioon} 
juures, tegi kõiki neid 
saatetiitreid ja asju ja demos seda poolt. Mulle täiesti uus maailm, ma polnud 
seda osa üldse näinud. Siis ma mäletan seda interneti-teemalist loengut, aga 
eks seal oli väga palju vaba suhtlemist ja jutuajamist ka. See oli see koht, 
kus ma esimest korda nägin paljusid inimesi, kellega ma olin umbes aasta juba 
suhelnud. 

\question{Suhelnud siis sõnumitega Fidos?}

Jah, olid jututoad, teemade kaupa, põhimõtteliselt nagu tänapäeval foorumid. 
Ta ei olnud kaugeltki \emph{real time} seepärast, et sa sisuliselt tõmbasid 
alla värsked sõnumid, lugesid need läbi, kirjutasid nendele valmis 
\emph{reply}-d ja siis  helistasid sisse ja \emph{upload}-isid oma valmis 
kirjutatud asjad. 

\question{Järelikult pidi sul olema mingi kliendisoft?}

Jah, nendega oli lihtne. Kõige keerulisem asi oli see, et pidi olema modem ja 
telefoniliin. Need olid nagu pigem tol ajal asjad, mida oli raske hankida. 
Niipea, kui me saime selle Novelli võrgu püsti, siis meil majas sees liikusid 
sõnumid sealtkaudu. Selleks, et mina lugeda ja kirjutada saaksin, ei pidanud ma 
oma masinas modemit omama, sõnumid läksid otse Mambox-i. Täpseid samme, mis 
seal vaheetapid olid, ma ei mäleta. Pigem ma mäletan, et kui ma sealt liikusin 
Salva Kindlustusse, kus me Toivoga moodustasime Salva Kindlustuse IT-osakonna, 
siis oli meil lihtsalt ühes masinas \emph{point}. 

\question{Miks te seda kõike tegite? Diagnostikakeskuse jaoks ei olnud seda ju 
vaja, kas teil lihtsalt oli aega ja tahtmist mängida või keegi visiooniga 
inimene andis teile ülesande võrk ehitada?}

Novelli võrk oli väga praktiline asi tol ajal, tänapäeval on sama praktiline 
asi teha igasse kontorisse internetiühendus. Ta andis ettevõttele väärtuse 
mõttes kaks asja: võrguketta (mina salvestan maha ja sina saad teisest 
arvutist kohe kätte) ja võrguprinteri. See on ju \emph{pre-Windows} aeg. Sa 
said omale võrguketta külge, said faile jagada ja  programmidele ligi, 
said printida. Laserprinterid olid kallid ja see oli hästi suur efekt, et sa 
said maja peale ühe laseri osta ja ükskõik mis arvutist sinna trükkida, see oli 
\emph{magic} tol ajal. Kõiki asju tuleb ju konteksti panna ja ma pean ka nagu 
mõtlema, kuidas esile tuua seda, mis oli tol ajal tolle aja kohta eriline. 

\question{Diagnostikakeskuses olid siis kuni keskkooli lõpuni?}

Tekkis tegelikult see asi, et Diagnostikakeskus\index{Diagnostikakeskus} oli 
ikkagi meditsiiniettevõte, Skriining\index{Skriining} oli selle küljes 
tütarfirma. Algul need väikeettevõtted pidigi moodustama mingisuguse 
riikliku ettevõtte juurde. Nii et, ma ei tea, mis aastal see täpselt oli, aga 
millalgi sai Skriining ennast Diagnostikakeskuse küljest lahti aktsiaseltsiks. 
Kõigepealt hakkasin  Skriiningus suvetööl käima, tegin  võrguhaldust. Üsna pea 
õnnestus mul ennast sebida  kooli kõrvalt palgale. Nii et kogu keskkooli aja 
kindlasti ma käisin sellises režiimis, et  pärast kooli kohe linna, 
Skriiningusse. Siis meil oli juba rohkem kliente, mitte ainult 
Diagnostikakeskus, vaid erinevaid meditsiiniettevõtteid. Põhiäri oli 
 arvutivõrk, on see siis mingi haigla või midagi muud. Näiteks 
Haigekassa\index{Haigekassa} arvutivõrgu vedamine, mäletan, oli üks  minu 
töid. Ta asus tol ajal seal, kus praegu on Prantsuse Lütseumi 
algkool\sidenote{Hariduse 8, Tallinn.}. Ma mäletan seda sellepärast hästi, et, 
esiteks, tänapäeval keegi ei laseks sedasi vedada, kaabel veeti lihtsalt pinna 
peale. Tõmbasin kaablit ja lõin klambreid seina peale, 
tänapäeval tunduks robustne. Teiseks, töövahenditeks oli haamer, klambrid, 
kaabel ja midagi, mis meenutab kaugelt vaadates trelli. Aga tänapäeval keegi 
ilmselt ei kavatseks sellega ühtegi auku puurida. Mina pidin sellega kõik augud 
puurima. Puure trelli otsas oli ka üks, mis meenutas puuri, ja sellega tuli 
suvalisest materjalist läbi minna. See toimus sellisel nühkimis-meetodil. 

\question{Kas puuri diameeter oli suurem, kui kaabli diameeter?}

Jah, vähemalt see oli hea. Põhimõtteliselt ainuke lootustandvad kohad, kust 
õnnestus läbi minna, kuna puuri pikkus ei olnud ka väga pikk, olid ikkagi 
uksepiidad. Koaksiaalkaabel on ju niisugune, et see peab läbi kogu maja 
põhimõtteliselt moodustama pideva \emph{loop}-i. Ta saab kuskil alata ja 
kuskil lõppeda, aga ta ei saa katkeda ja teda ei saa olla mitu. Nii et sa 
pidid nagu mõtlema, et nendesse tubadesse kõigisse on arvutivõrku vaja, et 
kuidas see ahel nagu teha. Pidid terve toa läbi jalutama ja uuesti kaabliga välja 
minema. Ja, loomulikult, kui ta kuskilt katki läks, oli kogu võrk maas, 
seepärast, et ta ei ole nagu \emph{twisted pair}-i võrk, kus mingist ruuterist 
või \emph{switch}-ist läheb kaabel seadmini ja kogu moos. 

Ühesõnaga, Skriiningus ma käisin palgatööl. Ühel hetkel kolisime Suur Ameerikast 
ära, Skriining sai nii-öelda oma ruumid ja me olime pikalt, minu arust ikkagi pea 
kogu mu keska aja, Estonia puiesteel. Seal, kus 
praegu on Mati Mobiiliäri\sidenote{Estonia puiestee 5, Tallinn.}, 
põhimõtteliselt kohe Estonia teatri vastas. See oli äge aeg. 

Aaslaid\sidenote{Henn peab silmas Andrus Aaslaidu\index[ppl]{Aaslaid, Andrus}} elas kontoris. Tal oli mingi korter ka kuskil, aga ma ei tea mis iganes 
põhjusel ta ei viitsinud seal käia. Mina ka, tulin koolist ja tiksusin viimase 
bussini seal kas  mingeid oma programme kirjutades või siis mingeid töö 
\emph{task}-e tehes. Töö \emph{task}-id olidki siis võrgu laiali vedamised,  
võrkude hooldused, serverite paigaldamised. Need muidugi käisid maha vahepeal, 
tuli joosta kuskile teise linna otsa, asjad uuesti käima ajada. 

Kõva \emph{bread and butter} Skriininugs oli tol ajal see, et arvuteid pandi  
komponentidest kokku. Klient ütles, et \enquote{andke mulle üks arvuti}. Lepiti 
mingis enam-vähem konfis kokku, aga ma ei ole kindel, et see teine osapool, kes 
tellis, teadis, mis need parameetrid on. Otsiti linna pealt komponendid 
mingitelt partner arvutifirmadelt kokku, umbes kust mäluplaat, kust kõvaketas, 
kust korpus. Mina keerasin selle kõik kokku, panin sinna ööseks mingid 
testid peale jooksma, hommikuks, kui kõik toimis, läks arvuti karpi, Skriiningu 
kleepekas peale ja  kliendi juurde. Siis panid ta üles, näitasid inimesele (kelle 
jaoks see enamasti oli elu esimene arvuti), kust sisse 
lülitatakse,  turbo nupp\sidenote{Vanematel PC tüüpi arvutitel oli küljes nupp 
sildiga \enquote{turbo}. Vastupidiselt ootusele, sellele vajutamine vähendas, 
ja mitte ei suurendanud arvuti töökiirust. Asi oli selles, et 8088 
protsessoriga arvuti jaoks loodud tarkvara (eriti mängud) olid mõnikord 
sõltuvad masina 4.77 Mhz taktsagedusest ja seetõttu uuematel kiirematel 
masinatel korralikult ei käinud. Võimaldamaks tagurpidi ühilduvust, lisati 
riistvaraline võimalus arvutit aeglasemaks teha.}, et seda ära vajuta, seda 
pole vaja, kogu aeg las olla sees. Ja mida sa siis arvutiga teha saad, kuidas 
võrku saad logida, enamasti oli mingi Novelli võrk. Kõige tavalisemalt  oli 
kasutajal vaja saada ligi kas mingile raamatupidamisprogrammile, mida ka 
Skriining ise kirjutas, või siis olid Skriiningu sihukesed enda \emph{custom} 
mõnes mõttes nagu haiglate infosüsteemid. Oli see siis mingi kaardiregistrite 
süsteem või midagi sellist. Tegid mingi koolituse ja\ldots 

Sedasi läks keskkooli osa. 

\question{Kas see kõik õppimist ei hakanud segama?} 

Ei hakanud. Õppimine ei seganud seda pigem. Ega ma nüüd mingi viieline ei 
olnud, tol ajal oli mu elu ikkagi, tagantjärele mõeldes, väga IT poole kaldu. 
Mind see kooli asi  absoluutselt ei huvitanud. Mitte, et ma ei saanud 
aru aga ma tegin \emph{bare minimum}-i, et saaks kähku arvuti taha. Tundus, et 
kõik põnev asi toimub seal. 

Ja mitte ainult arvuti taha, vaid, kuna me olime nii tsentraalselt keskkohas, 
siis Skriiningu kontor oli nagu läbikäiguhoov.  Kogu aeg keegi astus uksest 
sisse, oli see siis Tanel Raja\index[ppl]{Raja, Tanel} või Hannu 
Krosing\index[ppl]{Krosing, Hannu} või keegi teine. Ajas juttu, said jälle 
midagi teada, mida tema oli kuskil näinud või kuulnud. Mast\index[ppl]{Kaal, 
Madis} oli ju Forekspangas kohe seal üle hoovi, tema käis külas. Väga 
sotsiaalne oli see elu tol ajal. Tänapäeval see on kõik kuidagi onlainis, tol 
ajal oli see kõik ka IT-meeste vahel üsna \emph{offline} mingis mõttes. 

\question{Aga BBS-id?}

BBS-id olid olemas, aga minu arust sel ajal Fido vunk hakkas juba hääbuma. 
Sellel oli ilmselt mitu draiverit. Üks oli see, et kapitalism jõudis kohale, 
tööd oli vaja teha. Ega ei saanud päevad läbi lihtsalt istuda ja häkkida 
mingeid arvuteid, et umbes kui palju ma suudan mälu siin efektiivsemaks panna 
või kui ilusti ma oskan oma faile pakkide ketta peale niimoodi, et nad 
võimalikult kiiresti \emph{access}-itavad oleksid. Selliste asjadega ühel 
hetkel enam ei olnud aega tegeleda, vaid tuli teha reaalset tööd. Mis iganes, 
telliti siis sult mingeid progemist, mingit arvuti kokkupanemist. Ma arvan, et  
elu läks  tõsisemaks ja selle tõttu läks  suhtlemist nagu vähemaks. 

\question{Aga füüsiliselt ikka üksteisel külas käisite?}

Käisime, aga ka see hakkas tegelikult ühel hetkel hääbuma. Kui ma nüüd 
tagantjärele mõtlen siis Skriiningus\index{Skriining} see veel nii oli. Kui me 
Toivoga\index[ppl]{Annus, Toivo} juba Salva Kindlustus IT-d tegime, siis ma ei 
mäleta, et seal siukest nagu hängimist või kohvitamist oleks olnud, et keegi 
oleks külla tulnud ja oleks olnud aega nagu väga pläkutada. 

\question{Korraks BBS-ide ja Fido juurde tagasi minnes, oskad sa mõnda näidet 
tuua, mis kohtades, mis teemalistes tubades sa juttu rääkisid?}

Ma nüüd väga kaudselt mäletan, aga minu arust oli olemas mingid siuksed ruumid 
nagu umbes EW.NALJAD, kus lihtsalt keegi jagas mingeid anekdoote, sihuke väga 
\emph{social}. EW.JUTUTUBA  oli sihukene \emph{general topic}. Ikkagi väga kõva 
diskussioon käis igasugustes tehnilistes 
\emph{channel}-ites. Ma kahjuks enam ei mäleta, mille kaupa need olid. 
Kusjuures eal ei olnud tegelikult ainult ju eesti kanalid, sa said 
\emph{subscribe}-da tegelikult ju globaalsetes kanalitesse ka. Näiteks mäletan 
kindlasti, et  ma mingil hetkel lugesin  Novelli halduse ja adminimise kanaleid 
ja need olid  globaalsed,  tänapäeva foorumi laadsed asjad. Need olid 
teemapõhised, aga kus ma kõige aktiivsemalt\ldots 

Päris algul see Eesti Fido ringkond oli suurusjärk 40-50 inimest, täitsa 
\emph{manageable size}. Muidugi ta läkski ühel hetkel  läbuks kätte, kui 
maht kasvas ja lisandus väga erineva taustaga inimesi. Näiteks Salvas, kui meil 
oli \emph{point} niikuinii püsti ja Novellis \emph{access}-itav, istusid seal sees, ma 
ei tea, sekretärid ja kes iganes, kellel oli aega. Varieeruvus läks väga 
suureks, inimeste taust ja huvid läksid väga erinevaks. 

\question{Ei olnud enam nii elitaarne klubi?}

Elitaarne\ldots Fido ei olnud minu arust kunagi kuidagi suletud klubi,  ma ei ole 
kunagi tajunud seda, et see on mingisugune nagu salajane või mingisuguse 
erilise müstilise ligipääsuga, vaid pigem lihtsalt neid inimesi oli algul vähe. 

Ivar Zarans\index[ppl]{Zarans, Ivar} sai ju Ehinaga\index[ppl]{Zarans, Ehin} 
Fido kaudu tuttavaks, sellest ajast koos elanud, ühised lapsed suureks 
kasvatanud. Zarans oli, ma võib-olla teen talle nüüd liiga, aga tuntud 
poissmees Tallinnas. 

Ühesõnaga Fido hakkas ühel hetkel minu elus küll kuidagi kõrvale jääma või 
hääbuma ja mul on sihuke tunne, et sinna tekkis nagu rohkem \emph{traffic}-ut, 
rohkem inimesi. Tegelikult ka  üritused hakkasid ära jääma. BBSummerid, 
mis algul toimusid regulaarselt, olid BBWinterid, onju. Ja kuna  seltskonna 
taust  läks väga varieeruvaks, siis nähtavasti enam ei oldud nagu nii ühiste 
huvidega võib olla. 

\question{Tulles tagasi sinu isiku juurde. Tolleks hetkeks olid sa juba 
leidnud, et kuna maailmas on olemas Hannu Krosing, siis sina enam 
programmeerida ei taha?}

Jah, ma ei mäleta isegi, kuidas see nii selgelt läks. Aga kuidagi ma tööalaselt 
läksin pigem nagu mujale. Kõigepealt võrkude adminimine ja haldus, see oli 
kuidagi mulle väga jõukohane ja ma sain aru, kus ma väärtust lisan. Samas 
progemises\ldots. Tol ajal ka progemises võib-olla nõudlust, vähemalt ses osas, 
kuhu mina nägin, et ma saaks nagu pakkumist esitada, oli vähe. See  buum, kus 
oli vaja vett kõrbe viia, oli ikkagi see, et kuidas arvutid võrku saada, 
arvutid tööle saada, kuidas üldse arvuteid kokku panna. Paljud inimesed, kes 
oskasid arvuteid progeda, ikkagi oma leiva teenisid sellist tüüpi asjadega. 
Arvuteid kokku panime ja konfisime me Skriiningus  minu arust küll kõik. 
Ega see arvuti kokkupanek ei olnud ka nii lihtne, nagu ta tänapäeval on. No 
nüüd just poistele tegin, või aitasin oma poistel, mänguarvutid  kokku panna. 
Neil olid läpparid, ütlesid, et  on lahjad, tahavad \emph{desktop}-e ja 
siis ma huvi pärast tellisin komponendid ja tegin nendega koos. See on 
tänapäeval super lihtne, põhimõtteliselt sul ei ole pistikut võimalik valesti 
panna, ta läheb ainult ühte kohta kogu süsteemis. Tol ajal sa pidid ikka väga 
täpselt teadma, mis \emph{jumper}-id panna, kuidas, kuhu panna pistik ja kui sa 
tegid seda valesti siis masin kärssas lihtsalt läbi. Ja ei olnud olemas 
Google'it. Sul oli kaasas dokument või dokumendilaadne asi, kust nagu leiutada, 
et mis \emph{jumper}-itega see kõvaketas töötada võiks, mida sa ilmselt nägid 
esimest ja viimast korda, sest  ka  juppide tarned Eestisse käisid suhteliselt 
naljakal viisil. 

\question{Kes tõi kohvris kuskilt Singapurist või jumal teab kust.}

Sellega seoses on mul muideks üks hea lugu. Mul olid ikka juba autojuhiload, ma 
pidin järgult üle kaheksateistkümne olema. Ja oli niimoodi, et oli üks habemega 
mees, kes pidas siin Mustamäel arvutifirmat, mille nimi mul pole enam meeles, 
aga ta läks Estoniaga põhja, seda ma mäletan. Sihuke suur mees nagu karu oli. 
Ja siis oli Kalle Lotamõis oma Skriininguga\index{Skriining}. Emb-kumb neist, 
info liikus tollal faksidega, said Hiinast faksi, et on soodsalt pakkuda mälu 
SIMM-e, mälumooduleid. Hind oli röögatult hea. Nad kahe peale, oli mingi 
miinimum \emph{order},  ja ilmselt kuskilt veel  laenates ajasid raha kokku, 
tegid  ülekande ära. Pappkast jõudis kohale, jõudis tolli. Ma mäletan seda 
sellepärast hästi, et meie Palmasega käisime kastil tollilaos järel. Ja kurb 
oli siis pilt, kui selle pappkasti tegid lahti ja seal sees oli ainult 
vahtplast. Seda raha muidugi enam mitte kunagi ei nähtud. Sinna pandi suur raha 
alla, Skriining ikka lakkus neid haavu, peab Kalle käest küsima, kui kaua 
täpselt, aga kindlasti rohkem kui kuu ja pigem aasta. Ja samas, kui seal oleks 
olnud \emph{legit} asi, siis  oleks olnud muidugi vinge marginaal kohe. Väga 
sihuke \emph{cowboy times}. 

\question{Kuidagi mulle jäi kõrva, kuidas sa ütlesid, et sulle tundus, et sa 
lisad niimoodi rohkem väärtust. Kas sa tõesti mõtlesid juba tol ajal sellest, 
kuidas sa saad kasulik olla?} 

Võib-olla lihtsalt see, et mis sul nagu välja kukub. Mulle tol ajal juba 
tundus, et mul kukub hästi välja tehnoloogia ja inimeste vahel  liimiks 
olemine. Et ma lähen selle inimese juurde, kes kunagi pole ühtegi arvutit 
näinud, ma pakin talle selle lahti, panen tööle ja näitan, kuidas käib. Ma 
olin kõrvust tõstetud sellest, et ma sain talle kasulik olla. Tema sai oma tööd 
hakata nüüd tegema hoopis teistmoodi kui varem. 

\question{Ahjaa, sul ei olnud mitte abstraktne klient  vaid konkreetne 
inimene, kellel läks nägu särama!}

Jah. Sealt hakkas tulema ka  esimene niisugune 24/7 kogemus. Servereid, okei, 
öösiti enamasti õnneks küll ei kasutatud, aga need olid tegelikult ju 
ärikriitilised ja kui nad läksid maha, siis tuli väga kiiresti kohale jõuda ja 
nad väga kiiresti tööle saada. Pluss siis selle väga vihase kliendiga tegeleda. 
Talle seletada, mis juhtus. Nende jaoks olid need mingid maagilised kastid 
nurgas. Kuidas sa siis inimarusaadavas keeles seletad, mis juhtus ja miks sa 
arvad, et seda uuesti ei juhtu.

\question{Ja miks see sinu süü ei ole!} 

Või siis, et miks ma arvan, et tõenäosus, et see kohe nüüd uuesti juhtuks, on 
väiksem. Eks seal oli tihti mingeid raua probleeme, voolukõikumisi, miljon 
asja, ma täpselt isegi ei mäleta, mis seal oli. 

\question{Kõik käis ju tol ajal tati ja teibiga kokku, kellel oli serveriruum?}

Ei, ei, sellist asja polnud olemas. See oli liiga moodne sõna selle aja kohta. 
Need muidugi tekkisid millalgi aga tol ajal ei olnud. Valiti, et kus oleks 
niisugune puhas ruum, et otseselt vett ei tilguks, et  veeavarii tõenäosus 
oleks väiksem. Tihti mingi raamatupidaja kabinet või mingi sihuke, kus ta kõige 
loogilisem panna oli. 

\question{Ja ühel hetkel sa läksid Salvasse?}

Jah, Skriiningust ma sattusin juba Salvasse\index{Salva Kindlustus}. See oli 
see hetk, kus Mast\index[ppl]{Kaal, Madis} oli juba 
Foreksis\index{Pangad!Forekspank}. Ma veel mäletan hästi seda, et 
Toivo\index[ppl]{Annus, Toivo} kutsus mind Salvasse ja Madisega oli ka juttu, 
et äkki minna Forekspanka. Ma ei mäleta isegi, mis põhjusel see oli nii, et sai 
Toivo kasuks otsustatud. Salvas oli lihtne. Toivo ise käis ülikoolis, temal 
polnud aega sellega \emph{full time} tegeleda ja ta otsis kedagi, kes oleks 
päeval kontoris kohal. Midagi arvutivõrgust oli juba olemas, aga  
ettevõte kasvas  kiiresti, nii et  oli vaja esiteks tööjaamasid ja võrku 
hooldada ja teiseks inimesi \emph{support}-ida. 

Ajad läksid kogu aeg kiiresti moodsamaks. Meil oli mitte enam 
Novell 3.11, vaid 3.12. Laserprinterid olid mitmel korrusel. Ma 
mäletan, minu projekt oli vedada maja peal laiali kaablikanalid nii, et kaablid 
polnud enam lihtsalt naelaga seina peale löödud, vaid käisid ilusasti 
plastikkanali sees. Suur ja äge asi oli internetipanga eellane telefonipank. 
See oli siis niimoodi, et minu arvutis oli modem ja telefoniliin. Ja kuna me 
olime  Novelli võrgus, siis raamatupidaja sai oma arvutis maksed ette 
valmistada, sisetelefoniga mulle helistada, et nüüd on kõik valmis ja mina 
tegin  sessiooni oma modemiga Hansapanka\index{Pangad!Hansapank}. Ülekanded 
läksid üle ja samal ajal tõmbasime ära panga väljavõtte. 

\question{Hansapangal Telehansa oli siis vist tõesti juba olemas.}

Ma ei mäleta, mis selle toote täpne nimi oli, aga see käis modemi teel ja mingi 
\emph{fat client} tuli omale installida, millega sai tõmmata panga väljavõtteid 
ja teha ülekandeid. Arvuti ekraanil oli ikkagi ülekandevorm, mille täitsid ära. 
Ja \emph{roles and rights} oli juba olemas. Näiteks mina sain teha  
pangasessioone, aga ei saanud teha ülekandeid. 

\question{Aga tõenäoliselt see fail ei olnud kuidagimoodi krüptitud, nii et 
mõningase vaevaga oleks saanud raha kanda ka kusagile mujale?}

Seda ma kusjuures peast isegi ei mäleta, kui kõva see \emph{security} seal oli. 
Mul on nagu see \emph{awareness} olemas, et seal ikkagi esimesed jäljed 
\emph{roles and rights}-ist olid juba olemas, selle peale juba mõeldi. 

Aga nagu töö või protsessi mõttes, see oli \emph{huge step forwards}. Vanasti 
ju raamatupidajad tiksusid panga vahet kogu aeg. Nüüd ei pidanudki 
põhimõtteliselt pangas käima. Ainult  sularaha oli vaja ära viia. Aga see, et 
palju meil kontol laekumisi on, palju kontojääk on, sellised asjad 
\emph{magically} olid tema arvutis olemas põhimõtteliselt iga kell kui ta  
tahtis. 

See oli ka see hetk, kus Salva tegi selle otsuse, et enam ei maksta sularahas 
palka, mis oli tol ajal ju tavaline, vaid Salva Kindlustus avas kõigile 
töötajatele Hansapangas kontod ja deebetkaardid. Konto avamine ja deebetkaart 
olid nii kallid asjad, et ilmselt kui oleks jäetud  töötajate teha, siis enamus 
poleks selle projektiga kaasa tulnud. Palju lihtsam oleks olnud palk sulas iga 
kuu välja võtta, kõik olid sellega harjunud. Aga siis hakkas palk reaalselt panka tulema, 
ja pidi leidma ATM-i, kust palk korraga või jupikaupa välja välja võtta. 

\question{Miks see kõik Salvale kasulik oli?}

Et saaks sulahraga mässamisest lahti, et kõik oleks \emph{clean}-im. Ega 
kellelegi, ilmselt eriti mitte raamatupidajale, ei meeldinud sellega tegeleda. 

\question{Mille peal Salva peamine äriprotsess jooksis?}

Hea küsimus. Udune mälestus on, et minu arust meil oli mingi naljakas 
Inglismaalt ostetud programm, nime ei mäleta. Küll aga ma mäletan seda, et me 
käisime  Tõnu Laagiga\index[ppl]{Laak, Tõnu} kahekesi mingis Lõuna-Soome 
kindlustusseltsis vaatamas nende infosüsteemi plaaniga see osta. Ost 
jäi kokkuvõttes katki, aga seda käiku ma mäletan. Ajastu näitena oli see, 
et Toivo\index[ppl]{Annus, Toivo} ei saanud sinna kaasa tulla sellepärast, et 
temal ei olnud välispassi aga minul oli.

\question{Miks sul välispass oli?}

Miks mul välispass oli, seda ma ei mäleta. Olin kuskil äkki spordi pärast 
võistlemas käinud? See pidi olema vist ikkagi veel 
vene passiga. Või sai see olla Eesti pass ja mul oli viisa aga Toivol polnud? 
Mingite paberite tõttu Toivo ei saanud tulla igatahes. Aga Soomes, oli neil 
ikka sihukene \emph{proper} infosüsteemi moodi asi. 

\question{Ärme siit nüüd kiiresti üle lähme. Kas sa tegid sporti ka?}

Pigem ma sel ajal just enam ei teinud. Ma olin kooli ajal teinud 
orienteerumist ja kui ma tulin IT-sse, siis jäi see katki. Nii et sisuliselt ma 
vahetasin  spordi IT vastu natuke Nõukogude Liidu lagunemise tõttu, natuke, 
sest arvutid tulid. Orienteerumine oli meil väga tugevalt finantseeritud Saue 
sovhoosi\index{Saue sovhoos}\sidenote{Täpsemalt V. I. Lenini nimeline 
köögiviljakasvatuse näidissovhoos.} poolt ja kui sovhoos ära kadus \ldots Minu 
jaoks pigem oli see, et IT tuli varem sisse aga tegelikult treening-grupp vajus 
ka laiali.

\question{Aga sa teed ju praegu ka?}

Jah, pärast sõjaväge hakkasin uuesti tegema. Sõjaväkke ma sattusin aastaid 
hiljem, kui ma olin juba väga vana.  Ma algul ikkagi olin ülikoolis, tol ajal 
ülikool vabastas sõjaväest või pikendas. Aga oli vaja  nii palju tööl käia, 
et  ei jõudnud ülikoolis käia ja  lõpuks ma kukkusin  ülikoolist välja. Siis 
jõudsin veel olla mõnda aega nii, et ei saadetud sõjaväkke aga lõpuks ikkagi 
läks asi tõsiseks  ja tuli ära käia. 

\question{Mis sa ülikoolis õppisid?}

Infosüsteemid, ma arvan, oli TPI-s\index{Tallinna Tehnikaülikool!Informaatika}. 
Käisin koolis Salva kõrvalt. Oligi nii, et keska lõppedes mul oli koolist ja 
õppimisest töö kõrvalt nii suur tüdimus peal, et ma olin endale lubanud, et ma 
kõigepealt teen aasta aega nüüd ainult rahus tööd. Ma jätsin aastase nagu 
vaheaasta ja siis läksin TPI-sse. Aga see oli tegelikult viga, sest siis ma 
olin juba niivõrd \emph{vested} sellesse töö-\emph{mode}-i, et koolis 
käimisest ei tulnud eriti midagi välja. See oli nagu kõige madalama 
prioriteediga asi ja pigem käis kummi venitamine, kuni lõpuks mingi kahe või 
kolme aastaga eksmatt tuli. 

Pärast sõjaväge ma tegin muidugi EBS-is baka ära. EBS tegi omale IT-juhtimise 
eriala, see oli muidugi juba kahetuhandendatel juba. Ma olin esimeses lennus 
koos Alek Kozlovi\index[ppl]{Kozlov, Alek} ja  kambaga. Tegime omale täitsa  
\emph{proper} viieaastase baka veel, kusjuures, nüüd on kolmene baka. See oli 
ka tegelikult väga kihvt, aga see on juba teine ajalugu. 

\question{Mis sa pärast Salvat tegid?}

Salvast ma sattusin sinna kohta, kus ma nüüd saan uhkustada, et  töötasin seal 
koos Eesti Vabariigi presidendiga. Jälle ühe sugulase tõttu. See sugulane 
rääkis mulle, et \enquote{üks firma, kes tegeleb sidesüsteemidega, otsib 
inimest, et kas tahaksid rääkida}. Ja miks ka mitte. Ma olin Salvas ma ei 
mäleta, mitu aastat olnud juba. Sain kokku mehega, kelle nimi oli Rene 
Maksimovski\index[ppl]{Maksimovski, Georgi-Rene}. Kes on siis tänane presidendi 
abikaasa ja kes oli  sellise ettevõtte omanik, kes pani Eestis üles Siemensi 
telefoni keskjaamasid. Sihtgrupp oli põhimõtteliselt suured ettevõtted: pangad, 
riigiasutused. See oli nagu \emph{next phase} selles mõttes, et kui 
personaalarvutid olid raamatupidamisosakondades olemas, siis aastal umbes 95 
oli see aeg, kus oli sidet vaja. Oli lihtsalt vaja võimalust helistada nii 
majas sees kui välismaale nii, et ei krõbiseks ja saaks kaugekõnesid teha. See, 
kuidas me tol ajal Siemensi telefoni keskjaamasid paigaldasime ja hooldasime 
oli samamoodi sihukene vee viimine kõrbesse. Täna polegi lauatelefone nagu 
õieti kellelegi tarvis, mobiiltelefonid teevad asja ära. Aga see oli veel enne 
mobiiltelefonide aegne aeg ja lihtsalt see, et sul oleks igal inimesel laual 
telefon, millega saab helistada maja piires või majast välja ja majast väljast 
saab talle otse laua peale\ldots 

\question{Jagada mingit väikest hulka telefoniliine!}

See oli vinge etapp, aga ta oli IT-st kaugemal, telekomi maailm. Haaberstist\index{Haaberst} 
edasi ma sattusingi tegelikult juba Uninetti\index{Uninet}. Seal ma tegin kaasa selle aja, 
kui Uninet tuli telefonivõrgu turule, tolle telefoni keskjaama paigaldus siia 
Eestisse ja sealt omakorda Elisasse\index{Elisa}, Elisast Skype'i\index{Skype}. 

\question{Kui ma su juttu kuulan, jääb mulje, et sul on olnud alati selline 
kaabli tõmbamise töö. Aga nii kaua, kui mina sind tean, on su tegevuse 
subjektiks eestvedaja või juhina pigem inimene kui kaabel. Mis hetkel sul see 
vahetus toimus ja kas sa üldse näed siin vahet?}

Selline kaabli vedamine oli see jah, kust mu karjäär nagu alguse sai, 
kaablitööd ma ei oska tegelikult üldse hästi. Tegelikult periood kuni Haaberstini oli 
sügavalt ikkagi arvutite ja arvutivõrkudega  seotud ja kogu see Haabersti ja 
Unineti periood oli pigem ikkagi keskjaamade progemine. Ja, noh, kui keskjaamad 
olid valmis progetud ja töötasid, siis nende laiendamine või siis mingid 
veasituatsioonid. Ka hästi niisugune 24/7. Kui Jõhvi piirkonna politsei või 
misiganes side maha läheb ja seal piirkonnas 112 ei tööta, siis on päris suur 
probleem. 

\question{Aga inimesed?}

See algas Haaberstis\index{Haaberst}. Seal  organisatsiooni kasvades ühel hetkel oli vaja 
formaliseerida ja, ma isegi ei mäleta, kuidas, aga keegi ilmselt tegi mulle 
siis selles osas ettepaneku. Et ma võiksin hakata teisi insenere juhtima. Ma 
kasvasin sisuliselt tiimist välja tiimi juhiks. Meil oli tehnikaosakond, sain 
tehnikaosakonna juhiks. Põhimõtteliselt on kindlasti murdepunkt  see, kus sa 
võtad vastutuse mitte ainult enda töö eest, vaid sa võtad vastutuse teiste 
inimeste töö eest. Sealt edasi on  karjäär edasi paraku peaaegu kogu aeg 
olnud kas siis tootejuhtimine, projektijuhtimine või inimeste juhtimine või 
nende segu. 

\question{Miks \enquote{paraku}?} 

Alati oleks ilus minna tagasi spetsialisti liistude juurde, kus ma saaks 
vastutada ainult oma töö eest. 

\question{Oleks või?}

Kõvasti lihtsam oleks. 

\question{Aga miks sa pole läinud?}

Pole julgenud vist, ma pean mõtlema. 

Okei, üks põhjus on kindlasti see, et ma arvan, et ma olen oma kompetentsi 
kõvasti kaotanud. Jah, see on võib-olla kõige õigem vastus. Nüüd, juhtides 
tehnilisi tiime on see siis Skype'is või Elisas või siin Fleep'is\index{Fleep}, sa näed 
seda, kui andekaid tehnilisi talente on tegelikult olemas. Ja siis sa näed, et  
nende tehnilistes oskustes ei ole sa nagu üldse konkurentsivõimeline. See on 
üks.

\question{See on väga tuttav tunne, jah.} 

Ja teine kindlasti on nagu illusioon sellest, et  inimestevaheline suhtlus ja 
koordinatsioon on asi, milles ma olen mingid \emph{skill}-id omandanud ja ma 
parem siis nagu sõidan nende peal. Ma arvan, et see vist on see vastus. 

\question{Sul tuli see pärast esimest otsust nagu loomulikult?}

Jah. Kuigi tegelikult Uninetti  ma läksin ka ikkagi seda suurt telefoni 
keskjaama paigaldama ja haldama. Haaberstis olid  ettevõtete keskjaamad, 
väikesed. Uninetiga me panime üles selle, mis on nii-öelda telekomi keskjaam. 
See, mille külge  väiksed klientide keskjaamad käivad käivad ja mis siis 
ühendub luba SS7-ga ülejäänud telefonivõrku. Meil olid ühendused nii Soome 
Finneti  kui Eesti kõigi operaatoritega. See oli ka väga põnev aeg. Aga seal ma 
paraku jälle kasvasin  ikkagi tehnikatiimi või siis selle võrguoperaatorüksuse 
juhiks. Ja pärast Elisas juhtisin mobiilivõrku, kus oli ka raadiopool sees. 
Ikkagi sisuliselt juba inimeste ja tehnoloogia koordineerimine. Enamasti selles 
rollis on ikkagi ka mingi tehnoloogia strateegia pool sees. 

\question{See kõik kuidagi seletab väga hästi seda, miks ma sind hästi 
praktilise inimesena tean.}

Ei tea jah, minu arust mul sihukest teadlikku karjääri planeerimist pole kunagi 
olnud. Pigem on see, et oled sattunud mingitesse väga huvitavatesse tiimidesse 
või kollektiividesse, need kõik arendavad sind kuhugi suunas ja annavad  
mingeid kogemusi. Ja alati avavad uksi kuskile järgmises suunas. Ma ei ole ise 
kunagi  väga teadlikult oma karjääri arendanud või mingeid samme ette mõtelnud, 
et kuhu ma tahaksin jõuda. Pigem olen alati otsinud selliseid meeskondi, kus ma 
sisse minnes tajun, et ma olen kõige rumalam inimene ruumis. 

\question{Vot, ma just tahtsin küsida, milline on äge tiim aga just vastasid 
sellele.}

Muud teemad ka. Et oleks võimalikult mitmekesine, et oleks teineteist toetav, 
igasugused niisugused asjad on mulle ka tähtsad. Aga ennekõike see, et ma 
tunnen, et mul on niisugune arendav valdkond või ala. Võib olla kõige parem 
näide oli Salvast Haaberstisse minek. Ma arvan, et Novelli adminimises ja 
arvutivõrkudes ma tundsin ennast juba suhteliselt kindlalt. Aga see, mis asjad 
on telefoni keskjaam, ISDN, mis on kahe megabitised ühendused? Õrna aimugi 
polnud! Veelgi enam, mind saadeti Rahvusraamatukokku, kus oli just keskjaam 
üles pandud, et mine tee koolitus. Ma polnud seda telefoni mitte kunagi elus 
näinud, mille koolitust ma tegema pidin. Hüppa vette, uju ja õpi selle käigus, 
 kuidas ujumine käib. 

Väga kihvtid inimesed on olnud ja see on minu arust kõige tänuväärsem asi. Ja 
samas on veel kihvt, kuidas nagu mingid inimesed käivad ringiga. See, kuidas ma 
jõuan lõpuks Masti\index[ppl]{Kaal, Madis} või Toivoga\index[ppl]{Annus, Toivo} 
Skype'is\index{Skype} kokku tagasi. Mõnes mõttes on see maailm väga suur, aga  teistpidi 
jälle nagu kummaliselt mingid inimesed tulevad su juurde  ringiga tagasi. Või 
ka mingeid hooned, näiteks see hoone, kus me praegu oleme\sidenote{Ajasime juttu Tehnopoli hoones 
Akadeemia teel.} või siis see Suur-Ameerika 18 hoone, kuhu ma sattusin  
Haaberstis uuesti ja töötasin kõrvaltoas sellest, kus ma esimest korda  
286 taga istusin. Vahepeal käid suure tiiru ära kuskil mingites teistes ettevõtetes ja 
teistes kantides ja aastaid läheb mööda ja siis järsku leiad ennast kõrvaltoas 
sellest, kus sa kunagi oled töötanud. 

\question{Ja siinsamas majas oli ju Skype!}

Täpselt selles tiivas, mina tulin tööle siiasamma tiiba, esimesele korrusele. 
Siin vastas üle koridori oli Paananen\index[ppl]{Paananen, Tiit} oma 
\emph{certification} tiimiga. Kuigi, jah, nad vist alustasid kuskilt mingist 
teisest ruumist, ka siin maja sees koliti. Aga mul oli väga nagu \emph{coming 
back home} mingis mõttes, kui see osa siin valmis sai ja meile siia ruumi 
pakuti. 

\chapter{Tõnu Samuel}
\index[ppl]{Samuel, Tõnu}
\label{sisu:tonu}

\question{Kuidas jõudsid arvutid sinu juurde ja sina arvutite juurde?}

Arvutite juurde sain ma kolmeteistaastaselt, aastal 1985, kui mul tekkis 
esimest korda ligipääs ühele programmeeritavale 
taskuarvutile. Aga see oli ikka mäekõrguselt rohkem kui 
MK-51\sidenote{Elektronika MK-51 oli alates 1982. aastast 
Zelenogradis, Billuris ja Rodonis toodetud Nõukogude 
taskuarvuti, mis oli modelleeritud Casio FX-2500 alusel.}, mis oli tollal 
tavalisel koolijütsil unistuste arvuti. 

\question{MK-51 oli ju Nõukogude kalkulaator?}

MK-51 oli jah Nõukogude kalkulaator, mis oskas trigonomeetriat, aga mina sain 
ligi välismaa Casio PB-100-le\sidenote{Casio üks esimesi ja 
lihtsamaid samme taskuarvuti juurest päris arvuti poole. See toodi PB-100 nime
all turule 1982. aastal ja ka 1983. aastal kui TRS-80 PC-4 (Tandy Radio 
Shack) ja OP-544 (Olympia). Tegu oli QWERTY klaviatuuriga päriselt 
programmeeritava arvutiga, kuigi üherealine ekraan tegi Casio 
BASICus\index{BASIC!Casio BASIC} programmeerimise küllaltki vaevarikkaks.}. 

\question{See on ju klassika!}

Mõnes mõttes jah. See on taskuarvuti, mälu on pool kilobaiti, mis 
tänapäeval tundub ulmeliselt vähe, ja oskab ainult BASICut. Aga tollal võttis mul 
silme eest kirjuks, sest klaviatuuril oli terve tähestik peal. 
Tundus ikka täielik kosmos. 

\question{Kust niisugune asi õndsasse Nõukogude vabariiki sattus?}

Nägin seda kõigepealt kuskil komisjonipoes 
ja kuna sellel kalkulaatoril olid tähed, siis jättis see mulle
nii sügava mulje, et rääkisin sellest igale sõbrale. Paar 
päeva hiljem tuli üks tuttav, kes juhtus olema suhteliselt rahakast perest, sellesama asjaga mulle nina alla liputama.
Siis sain seda natukene veel näppida. Läks mööda nädal või kuu ja juhtus selline ime, mis juhtub ainult filmides. Igatahes NSV Liidu 
ajal oli see täpselt sihukest laadi sündmus. Tuli info, et mul on välismaal, Šveitsis 
rikas vanatädi. Selle \enquote{rikas} mõtlesin sinna juurde vist ise, sest tollal 
tundus kõik rikas, mis oli välismaal. Juhuslikult saatis ta meile veel kirja ka, 
et äkki tahate midagi siit. Ja mina teadsin täpselt, et tahan 
samasugust asja, ja kirjutasin sellest talle kohe.

Ma olin kolmteist ja tagantjärele mõeldes tundsin ennast vist
väga täiskasvanuna, aga kui praegu vaatan kolmeteistaastaseid, siis saan 
aru küll, miks ma sellise kirja kirjutasin. Väga lakooniliselt, et kui sa juba 
küsisid, siis üht arvutit oleks mulle vaja küll. Ja tuligi, küll 
pool aastat hiljem. Tollal käis niimoodi, et kiri läks siit Šveitsi kaks kuud ja 
kaks kuud hiljem tuli vastus, et vabanda, ma oleksin sulle hea meelega saatnud, aga sa unustasid tüübi kirjutada. Siis 
ma kirjutasin jälle, ootasin veel mitu kuud ja ühel hetkel tuli 
tollist kiri, et saatke viiskümmend rubla tollimaksuks. See oli tollal 
kosmiliselt suur raha\sidenote[][]{Pudel viina maksis 10 rubla, pudel {\v Z}iguli õlut aga 33 kopikat.}, aga
ma kuskilt laenasin selle. 

\question{Tollimaks tol ajal?}

Ma olin väga vaesest perest, viiskümmend rubla oli minu jaoks 
ulmeliselt suur raha. Ja kui lõpuks nägin tolli hinnakirja, tuli välja, et 
välismaine kalkulaator on viiskümmend rubla. Aga selle karbile oli kirjutatud 
\emph{personal computer} ja selle puhul oleks maks olnud vist 
kolmsada või kolm tuhat, mida ma kohe 
kindlasti ei oleks suutnud kuskilt leida. 

Sellest ajast alates hakkas asi minema. Kui olin viisteist, õnnestus mul esimest 
korda Normasse tööle saada. Ja kui olin kuusteist, siis selleks ajaks 
olin täiesti veendunud, et tahan arvutitega tööd teha. Haridust mul ei 
olnud, teadsin, et keegi mind tööle ei võta, aga oli unistus, et 
lähen kuhugi arvutuskeskusse, kus on tohutu suur arvuti, koristajaks.

\question{Mida sa selle personaalse kompuutriga tegid?}

Näiteks lahendasin ruutvõrrandit koolis. Kolm numbrit sisse, kaks välja, 
väga kiiresti. Teised kõik tagusid oma kalkulaatorit, aga vigaselt ja said tihti 
valesid vastuseid. Seitsmes klass oli selles suhtes imelik, et
tegime matemaatikas vist pool aastat ruutvõrrandeid. See viskas kõigil
väga ära ja mina olin ainuke, kes kunagi ei eksinud, sest mul oli 
kolm numbrit sisse, kaks välja. Kõige naljakam oli see, et 
matemaatikaõpetaja lõpetas kodus asjade lahendamise, sest tunnis käis 
nii, et keegi pidi püsti tõusma ja oma vastused ette lugema ning 
pärast vaatas õpetaja küsivalt minu poole: kui noogutasin, 
siis järelikult oli õige. Suutsin enam-vähem reaalajas neid numbreid 
toksida, sel ajal kui teine vastuseid luges. 

\question{Õpetaja tegi ju ka käsitsi, temal ei olnud sellist 
arvutit!}

Ei olnud jah, ta pidi selle nimel tööd tegema ja eksis samuti. Ükskord sain õpetaja vea kätte ja pärast seda ta vist 
loobuski. 

Lisaks trollisin inglise keele õpetajat. Küsisin talt, kas võin eksamil arvutit kasutada. 
Vanemapoolne daam oli tõsiselt hämmastunud ja ütles, et kui 
sul sest kasu on, eks kasuta. Ja siis käis laua alt klaviatuur mürtsuga laua 
peale, terve sõnastik sees. Seejärel oli tal väga piinlik öelda, et ikkagi ei tohi. 

\question{Mis sind arvuti juures paelus? Kas
klaviatuur või et sai programmeerida?}

Midagi seal oli. Olin
selline laps, kes absoluutselt kõik asjad ära lõhkus. Näiteks kodus oli raadio 
ja minule ei mahtunud pähe, kuidas selle kasti sees saab olla inimene. 
Selge see, et inimest seal ei ole, aga sa ju kuuled, et midagi on. Lammutasin kõlari ära ja sain selle eest 
peaaegu peksa. Meil oli suur lampraadio ja ma uuristasin 
ennast seal ees olevast võrgust läbi, et teada saada. Raadio oli tollal asi, 
mis osteti üks kord elus. Kui selle ära lõhkusid, siis oli pahandust
palju. Kõik äratuskellad võtsin ka lahti, aga kokku panna enam ei osanud.

Oma taskuarvuti lõhkusin ka lõpuks ära. Üritasin seda lahti võtta, aga 
see oli kleepsuga kokku pandud. Venitasin seda lahti, 
kleeps muudkui venis ja venis ning ma kasutasin kääre, et natukene 
kaasa aidata, aga seal oli üks lintkaabel sees, mida ma ei märganud, ja suutsin 
selle ka läbi hammustada.

Nii et mul oli jah meeletu huvi, kuidas asjad töötavad. Mul on üks väga oluline mentor ka elus olnud, naabrimees, kes viitsis 
mulle seletada. Lapsest saati olen igasugu asju ehitanud, 
väikseid raadiosaatjaid ja muid vidinaid. Ühelt poolt olid sel naabrimehel 
närvid läbi, aga teiselt poolt viitsis ta aeg-ajalt õpetada. 

\question{Kas sa oled Tallinna poiss?}

Jah, ma olen kogu aeg Tallinnas olnud. 

\question{Kas juppe ja pudinaid, millest raadiosaatjaid teha, siis ikka 
liikus?}

Tegelikult oli väga raske. Tavaliselt käis asi nii, et sain kuskilt 
mingi skeemi, aga 
kui jooksin sellega poodi, et nüüd hakkan ehitama, siis selgus, et 
kõige põhilisemat asja ei ole. Näiteks käis tollal 
raadioajakirjast\sidenote{Tõenäoliselt peab Tõnu silmas ka mujal jutuks olnud 
ajakirja \begin{russian}Радио\end{russian}.\index{Radio}} läbi arvutiskeem ja mõtlesin, et nüüd teen ise arvuti. Jooksin ringi, et asju saada, ja selgus, et mikroskeeme lihtsalt ei ole. 
Videokontroller, CPU ja muu oli defitsiit, ma ei omanud ühtegi kanalit, kust 
midagi saab. Tagantjärele tean, et tuttavad, kelle vanemad töötasid 
sõjatööstuses, suutsid kõike hankida. Ühe tuttava vanemad olid näiteks 
keemikud ja kui tahtsime poisikestena plahvatavaid asju teha, 
siis käisime nende juures ükshaaval aineid pinnimas, 
varjates muidugi nende tegelikku otstarvet. Tagantjärele mõeldes 
pidid nad muidugi aru saama, 
milleks me näiteks salpeetrit vajame. Või siis oli tegu nii tarkade 
inimestega, et teadsid veel sadat otstarvet ega suutnud seal vahel enam 
ohtu näha. 

\question{Või mõtlesid, et kui poiss oskab juba salpeetrit küsida, siis vast 
näppe päris küljest ei lase?}

Vot ei tea. Meil keemiaõpetaja läks ükskord väga närvi sellepärast, et 
hakkasime nitroglütseriini valmistama.\sidenote{1956. 
aastal ilmus Eesti Riikliku Kirjastuse sarjas \enquote{Seiklusjutte maalt ja 
merelt} Jules Verne'i \enquote{Saladuslik saar} ja seal oli juttu nii 
nitroglütseriini plahvatusjõust kui ka selle valmistamisprotsessi 
detailidest.} Keemiaõpetaja veel seletas meile 
innustunult, et see kodustes tingimustes ei õnnestu. Tagantjärele
saan aru, et seda ta täpselt üritaski öelda, et ärge tehke, 
sest teisiti ei õnnestunud meid veenda. 

\question{Nüüd on ju teistpidi. Kui keegi läheb eetrisse ja teatab, et nende 
süsteemid ei ole häkitavad, siis kohe palju inimesi proovib!}

Tavaliselt käib see veel sellise ülbusenoodiga, et meie oleme 
paremad kui nemad. Ja kui väidad end kellestki parem olevat, siis 
enamasti see keegi solvub. Igatahes tavaliselt on mingi motivaator, miks 
selline väide toob sellise tulemuse. 

\question{Kust sul see arvutuskeskuse mõte tuli?}

Ma ei mäleta, kust ma selle info sain, et selline asi üldse olemas on. Vist 
üks tuttav käis kuskil arvutuskeskuses mingi onu juures, mul õnnestus kord ennast 
sinna kaasa nihverdada ja see tundus nii põnev. Tuba oli arvutit täis. Igatahes 
teadsin, et ma tahan arvutuskeskusesse, ja kuna ma midagi ei osanud, siis 
käisin mööda uksetaguseid. Tagantjärele saan aru, et ääretult naiivne 
oli käia viieteistaastaselt, ilma igasuguse 
hariduseta, uksi kulutamas ja mõelda, et lähen sinna tööle. Aga võeti tööle. Olin siis kuusteist.

\question{Kellena sa seal tööd said?}

Arvutioperaatorina. 

\question{Kas keskkool jäi sul tegemata?}

Üheksanda, tollal kaheksanda klassi tegin lõpuks hiljem ära, aga edasi ei 
teinud midagi. 

\question{Mida arvutioperaatori töö endast kujutas?}

Oi, see oli väga ülbe töö -- mul oli lubatud isiklikult 
arvuti käima panna ja klahve vajutada. See ületas mu ootusi tugevalt, olin sel 
hetkel endaga tõsiselt rahul. 

Algne ülesanne oli mingit teksti sisse panna. Vene keelt oskasin ma hästi ja 
üks osa keskuse tööst oli tarkvara tegemine Venemaa 
asutustele. Meil oli asutusetäis tädisid -- tollal oli mul tunne, et täiesti 
surmalähedased vanamammid, sellised neljakümnesed -- ja 
nemad programmeerisid midagi, millest me (paar 
poissi tuli veel) väga aru ei saanud. Kui nad kirjutasid kasutajajuhendi käsitsi paberile, siis meie pidime selle 
arvutisse sisestama ja välja printima. See tõi paratamatult õiguse ka
printerit näppida.

\question{Mis süsteemi too arvutuskeskus kuulus?}

Sideministeeriumi alla. Asutusel oli edev nimi 
sideministeeriumi info- ja arvutuskeskus\index{Sideministeeriumi info- ja 
arvutuskeskus}, mis oli siseministeeriumi endaga äravahetamiseni sarnane. 
Töötõendil olid punased kaaned, mis tegid nii mõnegi vajaliku 
töö ära, kui oli vaja midagi kuskilt läbi suruda. Lõid oma kaaned lauale (ma ei mäleta, kas Lenin oli ka peal)
ja jäi mulje, nagu töötaksid siseministeeriumis. 

\question{Sina sisestasid tekste, aga mida need programmid tegid?}

Palgaarvestust ja põhivara 
arvestust, lõpupoole pidin neid ise tegema. Mul oli tollal poisikesena väga raske aru saada, mis asi on 
põhivara ja mis väikevara, ja siis üks mammi seletas, et sina 
istud väikevara peal, aga mina istun põhivara peal. Kuskil viiskümmend 
rubla oli see piir ja temal oli viiekümne viie rublane tool. 

Mäletan, et kirjutasin ka postitoodangu arvestuse programmi. Näiteks löödi sisse, et kirja kohaletassimine kellelegi koju on 0,01 kopikat, ja 
postkontor arvutas niimoodi oma toodangumahtu. 

\question{Nii et põhimõtteliselt jooksis seal arvutuskeskuses kellegi ERP?}

Jah, midagi niisugust, kuigi tollal olid kõik need sõnad võõrad. Mõnes mõttes oli see muidugi väga nõme, mis tollal sai tehtud -- tänapäeva mõistes väga lihtsaid asju. 

\question{Tekstide toksimine võis olla üsna nüri tegevus. Kas see huvi ära 
ei võtnud või said nende masinatega omi asju ka teha?}

Mul oli huvi nii suur, et ma istusin seal nii kaua, kui üldse kannatas olla. Sain venekeelse tekstiga hästi hakkama, harjusin vene keele klaviatuuriga ära. Eesmärk oli saada töö kaelast ära, et teha enda jaoks huvitavaid asju. Kui ma sinna tööle läksin, siis ei osanud ma midagi. Pool aastat hiljem teadsin kõikidest tädidest rohkem. Ja kuna poisse oli peale minu veel neli, siis hakkasime omavahel infot jagama. Tollal internetti ei olnud, mitte midagi ei olnud kuskilt võtta, nii et enamik asju käis kuulujuttude põhjal ja katsetamise teel. 

Me õppisime näiteks residentseid programme kirjutama. Tänapäeval võiks seda 
isegi viiruseks nimetada. See oli tollal väga \emph{high tech}\ldots

\question{Kuidas te sellise asja välja uurisite? Teil olid ju Nõukogude arvutid.}

Meil oli erinevaid. Arvutuskeskuses oli kaks suurt arvutit: ES-1022\index{ES EVM!ES-1022} ja ES-1045\index{ES EVM!ES-1045}. Nende jaoks oli terve korrus. Mälu oli kuus megabaiti -- ferriitmälu, bitid olid ükshaaval 
silmaga näha. Aeg-ajalt läks mõni bitt tuksi ja insenerid parandasid neid. 
Kokku oli 24 inseneri, kes seda kõike lappima pidid. Ja kuna mälu 
pidi kord nädalas piiritusega puhastama, siis firmapeod möödusid ilma muu 
alkoholita.

Aga meil oli teine osakond, kõigil personaalarvutid -- 
kaheksabitised Robotron 1715d\index{Robotron!Robotron 1715}. Siis tulid 
Iskra 1030d\index{Iskra!Iskra 1030}, mis olid PC kloonid, kohutavalt halvad 
arvutid. Kuskilt tuli üks DVK-2\index{DVK!DVK-2}, mis oli IBMi kloon. 

Igatahes asi muudkui arenes ja näiteks DVK-2 on naljaka arvutina 
meelde jäänud. Nimelt oli arvutuskeskus suhteliselt külm maja -- Endla 16, 
tänapäeval Eesti Telefoni maja\sidenote{Selles majas asus tõesti kunagi Eesti 
Telekom, praeguseks on maja põhjalikult renoveeritud.} ---, mis ei pidanud sooja. Hommikul oli tubades viisteist kraadi või 
vähemgi ja arvuti ei läinud selle külmaga käima. Flopidraivi rihmad olid 
jäigad, mootor käis rihma sees ringi, aga flopi ringi ei läinud. 
DVK-l oli ees flopi jaoks nii suur auk, et sinna mahtus käsi sisse. 
Pistsid käe sisse ja tõmbasid kogu selle asja ringi käima, et see saaks piisava hoo, 
ja lükkasid flopi ruttu järele ning luugi kinni. Kui seda piisavalt kärmelt 
teha, sai arvuti käima. Hiljem, päeva peale, ei olnud enam probleemi. 

\question{Kust te ikkagi infot saite? Manuaalidest?}

Manuaalid olid täiesti kasutud. Kui Iskrad\index{Iskra!Iskra 1030} 
tulid, siis me veel eraldi mõnitasime neid, sest peale oli kirjutatud 
\begin{russian}электронная персональная вычислительная профессиональная 
машина\end{russian} -- personaalne professionaalne elektrooniline arvutuslik 
masin. 

\question{Kuidas te siis oskasite? Ei ole ju nii, et paned aga suvalisi käske 
ja äkki jääb programm residentseks!}

Tollal, kui mina arvutuskeskuses olin, seal võrku ei 
olnud. Enamik asju toimus nii, et keegi käis teises 
arvutuskeskuses külas, kus keegi oli kuskilt midagi välja nuuskinud, kas siis välismaalt kuulnud või mujal käinud, aga tavaliselt liikus info koos 
inimesega. Keegi käis kuskil ja õppis seal, kuidas teha 
mingit asja, mida me olime juba pool aastat mõelnud. Kusjuures probleem oli 
tavaliselt väike. Näiteks meil ei olnud printeril täpitähti ja siis tuli keegi tervishoiuministeeriumi arvutuskeskusest 
väga kavala trikiga, kuidas \emph{map}'ida klaviatuuril kandilised sulud 
täpitähtedeks.

Teine häda oli see, et klaviatuurid olid aeglased. Meil oli tihti vaja teha
mingeid jooni niimoodi, et pidime hästi palju miinuseid panema. Hoidsid seda miinust 
nagu ma ei tea mida, aga see ei jooksnud. Keegi õppis seda 
kiirendama, selleks tuli kuhugi porti kirjutada mingi number. Aga selline info 
liikus ainult suust suhu. Assemblerit\index{Assembler} ja muid sääraseid asju oskasime juba 
kõik ise teha, aga muu oli müstika. Mingid pordid ja mida sinna 
kirjutada oli dokumenteerimata. Kuskilt aga tuli info, et kui kirjutad porti, läheb see klaviatuurile ja klaver on kiirem. 

\question{Kas te seda folkloori kuidagi üles ka kirjutasite või jäi see lihtsalt 
inimeste pähe?}

Ei, see oli täiesti suuline. Seda ei olnud tollal nii palju, et oleks tulnud üldse pähe üles 
kirjutada. Pealegi oli see kõik selline info, mida sa niisama ära ei 
andnud, sest see tõstis sind teistest kõrgemale. Ega me 
tädidele ei rääkinud, kuidas residentseid programme kirjutada. Esiteks
pidasime neid madalamaks klassiks, kes nagunii aru ei saaks. Teiseks andis see 
meile võimaluse teha arvutis mida iganes, ilma et nad oleksid aru
saanud. Oleks tollal tahtnud mingit kräkki jooksutada, küll me oleks seda siis
jooksutanud \emph{background}'is.

\question{Kas te seda infot teiste arvutuskeskuste tüüpidega 
jagasite?}

Tavaliselt käis see jah vorst vorsti vastu. Sul pidi olema mingi mõju 
inimese üle. Tavaliselt professionaalid hindavad üksteist ja jagavad sellist 
infot, aga kui tuleb mingi jobu küsima, siis ega sa talle ei ütle. Sina oled palju 
vaeva näinud, et asi ära lahendada -- ringi jooksnud, küsinud ja mõelnud ---, ja 
sa ei anna seda infot niisama ära. Vahest antakse, aga see oli osa 
käitumismustrist. 

FidoNet oli tõenäoliselt üks esimesi arvutivõrke, millest ma kuulsin ja kuskil 
nägin. Sellest võis rääkida Tarmo Mamers\index[ppl]{Mamers, Tarmo}, aga ma ei ole kindel. Igatahes hakkas see selliseks asjaks muutuma, et 
küsisin töö juures modemit ja võrku. Meie 
arvutuskeskus ei pidanud seda kuidagi vajalikuks ja ma ei osanud kuhugi 
õigesse kohta vajutada ka. Kord vist insenerid kuskilt tagatoast 
pakkusid üht modemit, mis oli kingakarbisuurune ja mida ei saanud telefoniliini otsa panna. Oli mingisugune 
\emph{leased line} modem, mis ei osanud helistada. Igatahes 
minu jaoks oli see täiesti tarbetu, sest mul oli vaja modemit, mis käib 
telefoniliini külge. Telefoniliin oli ka tol ajal väga suur ressurss. Meil oli 
asutuse peale piiratud arv telefone, kusjuures me olime sideministeeriumi 
arvutuskeskus! Jaam oli meie enda majas, aga meie osakonnas oli kakskümmend 
inimest ja kaks numbrit. 

\question{Küsin inseneride kohta. Kui mõtlen, kas lasta noor inimene 
tarkvara või riistvara juurde, siis mina julgeksin teda pigem riistvara kallale lubada. 
Miks sind tarkvara juurde lasti?}

Riistvara ei olnud meil tollal midagi erilist.

\question{Näiteks oleks pandud piiritusega ferriitrõngaid 
nühkima?}

Vastus, miks ma sinna tööle sain, on mul hiljem 
tulnud. Tollal uskusin, et jätsin tõsise mulje, nagu oleksin oma 
taskuarvutiga mingeid programme teinud. Tagantjärele saan aru, et tegelikult oli tollal probleem selles, et IT eriala ei olnud üldse 
populaarne. Keegi ei tahtnud seal töötada, palgadki olid vist madalad. Tõsine 
mees tegi kuskil haltuurat, näiteks kui töötasid viinapoes, siis said 
sealt midagi müüa. Arvutuskeskuses sai ametlikku 
palka ja midagi varastada ei olnud. Reputatsiooniga oli häda -- sinna läksid 
matemaatikaharidusega naisterahvad, samal ajal kui kõik teravamad 
tüübid läksid traktoristiks, sest seal sai kütust varastada. 
Igatahes sinna tööle ei mindud. 

\question{Aga siis tuli üks, kes tahtis!}

Jah. Kuna osakonnas oli kakskümmend naist ja osakonna 
juhataja oli mees, siis olen enda jaoks selle dekodeerinud niimoodi, 
et ta oli andnud kaadriosakonnale käsu, et kui tuleb ükskõik mis meesterahvas, 
tuleb ta kinni võtta ja temale anda. Nende paari aasta jooksul, mis 
ma seal olin, nägin, millised kismad seal käisid -- naised kraapisid vaat et
üksteise silmad verele. Midagi füüsilist muidugi ei olnud, aga ussitamist oli 
korralikult. Hiljemgi olen tajunud, et kollektiivis peavad 
mehed-naised tasakaalus olema, vastasel juhul tekib mõlemas suunas
jama. Ja mul on tunne, et mina olin esimene meesterahvas, kes talle ukse taha 
tuli; teda ei huvitanud ükski muu asi peale selle, et olin mees. 

\question{BBSi aeg sattus arvutuskeskuses olemise aja tagumisse 
otsa. Sa rääkisid, et tahtsid sinna modemit saada.}

Tahtsin, aga ei saanud. Tollal, vist 1989. aastal, hakkasid tekkima
kooperatiivid ja kooperatiivinduse tüüpidel oli raha 
paksult käes. Kui midagi väga tahtsid, siis nad sulle ostsid. Mul 
õnnestus saada modemi ligi niimoodi, et see ja arvuti olid ainult minu kasutada ning suutsin ennast FidoNetti ajada.

FidoNet oli 
kullaauk, täpselt nagu tänapäeval internet. Kus seal info jooksis! 
Kui kellelgi midagi oli, siis sellest räägiti, ja see seltskond tundus 
kohutavalt suur võrreldes paari arvutuskeskuse inimesega, kellega ma 
tavaliselt lävisin ja keda vääristasin endaga võrdseks. FidoNetis aga oli 
kümnete viisi inimesi. Tänapäeva internetis ei kujuta seda enam ette, aga FidoNetis oli Eestis 
50---100 aktiivset inimest, mitte rohkem, ja need inimesed 
olid targad. Jooksin hommikul arvuti juurde, et panna see käima, tõmmata 
kõik viimased kirjad ära ja vaadata, mida nad räägivad. Oli terve hulk 
legendaarseid mehi, kelle iga kiri oli kulda täis.

\question{Näiteks?}

Mulle jättis mulje näiteks Sulo Kallas\index[ppl]{Kallas, Sulo}. Tollal teadsime, et CD-plaat
on olemas, aga oma silmaga polnud näinud. Seda üksnes reklaamiti, et nüüd on lõpuks 
ometi puhas heli. Sulo Kallas oli audiofriik, kes sai endale Sony CD-mängija sel ajal, kui teised igatsesid endale Vene oma. Ta tunnistas helikvaliteedi ebakõlblikuks, pildus selle kasti sisust tühjaks ja tegi sinna uue 
elektroonika. Minule jättis see kustumatu mulje, mäletan seda mitukümmend 
aastat hiljemgi. Ja nüüd, kus ma olen selle härraga ise koostööd teinud, austan teda endiselt.

\question{Kui sa alguses olid Fidos klient, siis ühel hetkel hakkasid sa ka oma 
\emph{node}'i pidama. Millal see tuli?}

See tuli üsna ruttu. Ma olin alguses kellelegi \emph{point} ja ühel 
hetkel \emph{node} numbriga 25. Minu jaoks on Tarmo Mamers\index[ppl]{Mamers, 
Tarmo} alati olnud see vaimne isa, kelle käest olen väga palju vastuseid ja abi
saanud, ja tõenäoliselt tõmbasin tema juurest alguses ka kogu 
oma meili. Hiljem muutusin ise nii suureks, et vahendasin näiteks kogu 
Venemaa meili Eesti vahel. 

\question{Kas Venemaal olid BBSid ja Fido vähem levinud?}

Millegipärast jah. Eestis käis päris 
algusaegadel (mina olen sellest ilma jäänud) kõik Soome küljes. 
Eestit ei tunnustatud välismaal üldse, meil puudus oma aadressruum, kõik olid 
Soome \emph{node}'id. Millalgi aga hakkas see asi Eestis kasvama nagu seen pärmi 
peal ja siis Vene omad olid kõik Eesti \emph{node}'id. Venemaa võrk oli minu 
meelest umbes sama suur kui Eesti oma. 

\question{Olen kuulnud legende sellest, kuidas kaugelt Venemaa 
avarustest käidi lennukiga Eestisse Fidosse, kohver flopisid kaasas, ja muudkui 
kopeeriti öö läbi.}

Tollal oli see vist kiirem jah, sest üle telefoni läksid asjad nii 
aeglaselt, et oli odavam kohale lennata. Minu meelest maksis
Moskva lend umbes üksteist rubla, mis oli üsna väike 
raha.

\question{Kuidas Eesti oma tsooni saamine käis? Siis oli ju veel Nõukogude Liit
või ei olnud enam?}

Vaat ei oska öelda, mina jäin sellest otsast ilma. Äkki Sulo 
Kallas\index[ppl]{Kallas, Sulo} või keegi teine juurguru vanematest aegadest 
oskab rääkida. 

\question{Fido \emph{node}'i sa panid püsti, kas sul BBS on ka olnud?}

Mul ei ole kunagi otseselt BBSi olnud. Kindlasti olen midagi mänginud ja paar tükki äriinimestele kommertsasjade jaoks püsti pannud. Neid 
\emph{impress}'is kohutavalt see, et mingid tüübid olid neilt paar 
aastat raha küsinud, et midagi programmeerida, ja nad ei olnud kunagi näinud, 
mis tulu sellest sai. Siis tuli Tõnu, küsis mingi mõttetu 
rahasumma ja kaks tundi hiljem asi töötas. Tollal jättis see
kustumatu mulje, et lõpuks ometi keegi, kes aitas. Aga sealt edasi 
ei ole ma midagi teinud. 

\question{Kas kuskil olid inimesed, kellel oli äriline põhjus BBS püsti 
panna?}

Tollal oli lootus, et äkki nüüd hakkab äri minema, sest tulid inimesed, kes 
ütlesid, et nüüd läheb kogu äri internetti. Toon samast ajast 
võrdluseks sellise pisiasja, et tol ajal kõige populaarsemal tarkvaral, 
Maximusel, mida kõik BBSid kasutasid, oli \emph{user ID} ühebaidine. Tänapäeval ei kujuta ettegi sellise tarkvara tegemist. 
Keegi hoidis ruumi kokku ja tegi ühebaidise \emph{user ID}! Enamik BBSidel ei 
olnud nii palju kasutajaid, et neil oleks sellest puudu jäänud. 

\question{Järelikult tehti disainis õige otsus!}

Eestis oli üks või kaks BBSi, keda see hakkas ühel hetkel tõsiselt häirima. Ma 
tahan öelda, et see kogukond ei olnud üldse nii suur. Mina tunnetan seda siiamaani 
kui väga elitaarset seltskonda, sest kõik, kes sinna 
suutsid tulla, olid targad inimesed. Näiteks FidoNetti saamiseks oli vaja kolm erinevat tarkvara koos tööle 
panna, et sinna üldse ligi saada, see ei olnud üldse nii lihtne. 

\question{Kas sul hakkas siis Fidos tekkima seltskond, kellega 
juttu rääkida?}

Jah, ja need suhted kestavad üle mitmekümne aasta. 
Näiteks Venemaaga hakkasime äri tegema nendesamade Fido \emph{node}'idega, 
kes sealpool olid. Hüperinflatsioon oli selline kummaline asi, et kuna 
Nõukogude Liit oli suur, siis ühes otsas liikusid hinnad kiiremini kui teises, 
Eesti oli Moskvast kuni nädal aega hindadega maas. 

\question{Ja sul oli infot ning said seda vahendada!}

Enamik inimesi toimetas ajalehekuulutuste kaudu, aga mina rääkisin tuttavaga Peterburis või Moskvas, et 
tahan printereid saada. Tema ütles, et hind on selline. 
Ostsin selsamal õhtul pileti, hommikuks olin juba seal, ladusin asjad 
peale, ülejärgmisel ööl tulin Eestisse ja müüsin need 
Kinexisse\index{Kinex}\sidenote{Üks varaseid Eesti arvutifirmasid, mis hiljem 
tegeles äritarkvara ja sellega seotud konsultatsioonidega.} maha. Kinex oli 
nii õnnelik -- nad ostsid mult kõik kakskümmend printerit korraga ära. Raha oli tollal
päris palju ja vaheltkasuga sai Venemaalt järgmise kuhja tuua.

\question{Kas siis tulidki arvutuskeskusest ära ja hakkasid äri tegema?}

Arvutuskeskuses ei olnud enam mõtet olla, sain seal 110 rubla miinus maksud. 
Eraäris sain juba päris alguses 800 rubla päevas. Neil lihtsalt ei 
olnud mind enam mitte millegagi motiveerida, ainult 
sellega, et \enquote{Tõnu, sinu programmid näevad paremad välja kui minu omad} -- 
kirjutasin neile jubinaid, mis käivitasid nende programme. 
Nad tegid oma moodulid igaüks eraldi binaarina ja minul oli üks
akendega asi, mis neid käivitas. Akendel olid varjud taga, ega seda ka igaüks 
teha ei osanud. 

\question{Kas see oli tekstipõhine värviline terminal?}

Puhas tekstivärk. Meie arvutuskeskus oli selle poolest teistest halvem, et kõigil teistel olid mingid graafilised asjad. Meil oli neid vähe 
või ei olnud üldse. Ma tundsin ennast maru halvasti, sest teised mängisid 
värvilisi mänge ja mina ei saanud. 

\question{Mis mänge sa mängisid?}

Näiteks \enquote{Diggerit}\index{Digger}\sidenote{1983. aastast pärit, 
suhteliselt lihtsa graafikaga arvutimäng, mis oli omal ajal väga levinud.}, mis tollal 
oli täisvärviline. Aga meil olid Iskrad\index{Iskra}, millel olid rohelised 
ekraanid, ja kõik veel null ja üks, polnud pooltoonegi. 
Häkkisime jootekolviga, et saaks vähemalt pooltoonid kätte. 

\question{Järelikult oli sul tol ajal jootekolvi- ja insenerihuvi olemas?}

Oli, aga oskused olid vähesed. See info tuli jälle mõnest teisest 
arvutuskeskusest, et vot sinna kohta tuleb panna kaks takistit. Praegu 
suudaksin ka ise viie minuti jooksul välja 
mõelda, et teeks monitori sellise \emph{fix}'i, aga tollal istusime
pool aastat monokroomsete üks-null monitoride taga. 
Alles siis tuli keegi ülimalt hea infoga, et kui monitor lahti teha, seal 
kaks juhet lahti võtta ja takistid vahele panna, siis tekivad pooltoonid. Meil käed 
värisesid, kui me seda tegime. 

\question{Muidugi, monitor oli ju kallis!}

Sellel ei olnudki hinda. Kui tegid katki, siis lihtsalt rohkem ei saanud kasutada.

\question{Aga oli piisavalt julgust, et kaas maha võtta?}

Kuidagi oli. Kui viis poissi koos on, küll see julgus tekib. Ei ole ilus 
öelda, aga me aeg-ajalt jõime seal ka koos. Tollal tekkis 
arusaam, et kui natuke peale võtta, siis edeneb programmeerimine 
kiiremini. 

Huvitav, et kui hommikul iseenda kirjutatud 
tarkvara vaatasid, siis oli tunne, et üks väga tark inimene on kirjutanud. Aru ei 
saanud, töötas, aga kui puutusid, läks katki. See oli kõrvalefekt. 

\question{Sa ütlesid, et Fidost hakkas kohe infot tulema. Mis infot? Manuaale?}

Digitaalseid manuaale kui selliseid tollal ei olnudki, kõik liikus prinditud info peal 
ja nende digiteerimiseks ei olnud mingeid lahendusi. Kui keegi midagi teada sai, siis ta seda levitas. Teine asi oli igasugune 
ostan-müün-vahetan. Nõukogude Liidus oli kõigest puudu ja sellepärast oli väga oluline 
teada, et keegi midagi müüb. Kasvõi oma vana tooli, üks jalg 
alt ära, aga see oli NSV Liidus väärt info. Ostsin oma 
esimese auto Indrek Sauli\index[ppl]{Saul, Indrek} käest, kes oli tollal 
aktiivne Fidokas. Tal oli Žiguli eksportvariant. Eksportvariant oli tavaliselt 
1500se mootoriga, aga temal oli 1600ne! Enam kõvemat autot ei andnud 
ette kujutada ja ma ostsin selle ära. Ja kuna ma olin FidoNetis, kus ta seda reklaamis 
ja ainult paarkümmend inimest nägi, siis oli mul eelis.

\question{Kas mänge, muusikat, graafikat ja muud sellist kraami ka liikus?}

BBSides liikus väga palju, aga tollal oli arvutivõrk niivõrd aeglane, et ei
tulnud isegi mõtet saata binaare meiliga. Tollal vaatasid, et 
oi, siin on kümme kilobaiti suur asi, ja panid modemi ööseks tõmbama. 
Poolel rahval olid 1200boodised 
modemid. Peter Marvet\index[ppl]{Marvet, Peeter} kirjutas kord suhteliselt ülbes 
toonis meili, et alles 9600 bps modemiga tunned, et tegemist on 
\emph{communication}'iga. Nii et 9600 bps-i üle võis uhke 
olla. Mina ei jõudnud seda osta, aga temal oli selline kuskilt 
saadud, ta oligi minust kõrgemal. 

Kui midagi väga otsisid, siis keegi teadis öelda, et 
vot seal BBSis ma nägin seda, sest enamik asju taandus ikkagi 
piraattarkvarale. Muusika tuli veel hiljem. Mäletan siiamaani, 
et MP3 korralikuks mängimiseks oli vaja 100 MHz 486, mis oli täpselt selle 
piiri peal, et kui hiirt liigutasid, oli muusika kinni. Ja kui tulid 120 MHz 
486d, siis võisid hiirt ka liigutada.

\question{Demoskene asjad ju liikusid.}

Oi, demod liikusid, see oli ilus! Igatsen siiamaani neid vidinaid, mis olid 
imeväikesed, aga kui käima tõmbasid, siis oli tuba muusikat ja ekraan 
graafikat täis. Lausa kolmemõõtmelist ja see tundus mulle 
kosmiliselt ilus -- CGA graafika\sidenote{Võimaldas 320 x 200 
ekraanilahutusega kuvada nelja ja 640 x 200 lahutusega kahte värvi.}, mida 
tänapäeval keegi ei vaata. 

Mäletan seda hetke, kui nägin esimest korda \enquote{Diggerit}\index{Digger}. Meil polnud arvutuskeskuses ühtegi helikaarti ja ma käisin 
tervishoiuministeeriumi arvutuskeskuses\index{Tervishoiuministeeriumi 
arvutuskeskus} asju ajamas ja 
keegi mängis seal \enquote{Diggerit} Olivetti arvuti peal. See heli ja värvid tundusid nii 
võimsad! 

Muide, tervishoiuministeeriumi arvutuskeskus asus 
surnukuuri kõrval, vist Tervise tänaval. 

\question{Mõnikord käib arvutihuviga kaasas ulmehuvi, kas sinul ka?}

Ma ei mäleta. Olen kõik seiklusjutte vee alt ja kuu pealt sarjad läbi 
lugenud, hakkasin väga vara lugema ja lugesin ulmeliselt palju. Aga 
selleks ajaks huvitas mind meeletult reaalne, mitte väljamõeldud maailm, sest see
on alati välja mõeldud. Ma olen täiskasvanueas kogu aeg fakte 
otsinud, mind huvitavad faktipõhised asjad, näiteks ajalugu. 

\question{Ajaloo kohta ütleb mõni, et see ei ole ju fakt, vaid puhas 
arvamus.}

NSV Liidust tulnud inimesel on see hea omadus, et suudad päris palju 
filtreerida. Ma kuulan Vene propagandat hea meelega, kuna saan üsna hästi aru, mida nad üritavad näidata ja mis on 
tegelikkus. Aeg-ajalt just see, 
kuhu nad propagandat suunavad, annab vastuseid. 

\question{Kas pärast arvutuskeskust toimetasid iseseisvalt või oli sul 
mõni kooperatiiv?}

Ma sattusin eraärisse niimoodi, et hakkasin tooma Venemaalt arvuteid ja 
teenisin selle eest tolle aja kohta meeletut raha. Kuigi see oli imelik aeg, sest
see raha oli ikkagi väiksem kui kellegi teise meeletu raha. Kuna 
praktiliselt kõiki Eesti arvutipoed olid mu kliendid, 
siis üks, kellele ma kogu aeg asju vedasin, ütles, et kuule, hakkame 
parem koos tegema. Tal oli
idee, et müüb kogu mu kolu maha, aga mina toon ainult talle. Mind see huvitas, sest muidu otsisin mööda linna, kellele oma 
kolu lükata. 

\question{Järelikult pidi sul olema päris korralik suhtevõrgustik nii 
Venemaal kui ka Eestis.}

Arv oli väike, aga võrgustik korralik ning suhted kestavad 
siiamaani. Näiteks Venemaal olen püüdnud igasuguseid asju ajada ja olen alati 
lõpuks petta saanud, aga seal on üks inimene, keda usaldan siiamaani 
siiralt. 

\question{Kas see võrgustik tekkis ainult tänu Fidole?}

Jah. Vanad võrgustikud ongi erakordselt usaldusväärsed. Inimesed, kes on üksteist
kolmkümmend aastat tundnud, ei keera üksteisele jama kokku. Kui oled
kommuunis endale pleki külge saanud, siis ei ole enam kuhugi taganeda. 

\question{Ma ei kujuta hästi ette, et Fido seltskond oleks ideaalsetest 
inimestest koosnenud. Kindlasti visati keegi välja ka?}

Mind visati ka välja. Ma ei mäleta, mida ma halvasti ütlesin, aga Tarmo 
Mamers\index[ppl]{Mamers, Tarmo} viskas mu väga kiirelt välja. See oli väga 
hea õppetund, et Fidos ei ole demokraatiat. Fidos on igaühel oma kuningriik ja 
sa oled alati kellegi kuningriigis, pead tema reeglite järgi mängima ja 
kõik. Kui tahad, võid oma kuningriigi luua ja hakata sinna rahvast 
meelitama, aga tavaliselt istud seal üksi. 

\question{See seab selle esimese adminnide saunaõhtu ju hoopis teise 
valgusse!}

Neil inimestel oli reaalne võim sind informatsioonist ära lõigata, aga seda 
ei kuritarvitatud. Kes sai kinga, sai asja eest. Ja mina 
sain ka asja eest. Aga sain ka väga kiirelt aru, kus need 
piirid on. Kakskümmend neli tundi hiljem olin tagasi, sest käisin kenasti 
õige inimese juures vabandamas ja rohkem ei teinud. 

Kloune oli seal üht- ja teistsuguseid, ja võimuga inimesi oli erinevaid. Näiteks 
oli teada, et üks mees oskab karatet, ja kui oli vaja kellelegi peksa anda, siis 
räägiti pigem temaga. Kui oli vaja elektroonikat teha, 
teadsid teise inimesega juttu rääkida, näiteks Madis Kaaluga\index[ppl]{Kaal, 
Madis}, kes hiljem töötas Skype'is. Tema oli see vend, kes julges nii 
kallist asja nagu arvuti häkkida. Enamik meist ei julgenud, sest arvuti oli 
sul üks elu jooksul. Aga tema kraapis vaibanoaga mingid rajad lahti ja 
panin relee vahele, et modemit lahti ühendada, kui see lolliks läks. See tundus 
nii riskantne tegevus, et isegi kui teadsin, mida teha, siis mina ei julgenud. Nii et
kui oli riistvara probleem, räägiti temaga. 

Mingid tegelased kogu aeg müüsid midagi ja oli teada, kelle käest 
mida saab. Näiteks igaüks teadis kedagi MicroLinkist või mujalt ja kui tahtsid allahindlusega asja 
saada, siis tuli temaga suhelda. Fidokad tegid tavaliselt omavahel 
allahindlusi. 

\question{Mis selle võrgustiku nii tihedaks tegi? Kas vastastikune respekt kõrge sisenemisbarjääri tõttu või veel midagi?}

Respekt kindlasti, sest alternatiivi ei olnud. See oli oma tsunft: sa kas 
olid seal või ei olnud. Need olid targad inimesed. Tänapäeva internetiga võrreldes on 
tohutu vahe -- kuskil aastast 2000 edasi on internetti tulnud lollid. Ma ei 
mõtle muidugi kõiki. Kui varem lugesid midagi võrgust, siis see oli 
kuld. Üheksakümnendate lõpus, kui Eestis pandi 
püsti Delfi ja igaüks sai endale koju
neti, siis hakkasid horoskoobid ja muu jama nii hullusti 
levima, et enam ei teadnud, mis internetis on tõsi. 

\question{Targad inimesed oleksid ju võinud oma kogukonna kolida teise, 
kõrgema barjääri taha.}

Fido on eksisteerinud tükk aega ja eksisteerib mingil kujul vist siiani. 
Seesama barjäär on ka väga tarkadel inimestel lihtsalt jalus, nad ei viitsi 
seda teha. Tollased piirangud olid ka tülikad. Näiteks asjad ei 
käinud reaalajas, vaid pidid kuhugi helistama, saatma oma kirjad ära, panema toru 
hargile. Keegi teine pidi helistama sama numbri peale, kui sina olid toru 
hargile pannud (muidu ta ei saanud helistada), ja tõmbama kirjad ära. Aga ta ei 
teadnud, et peab just sel päeval helistama, kuna sina oled kirja saatnud. Kui ta otsustas 
sulle vastata, siis tavaliselt käis see kahekümne nelja tunnise 
tsükliga. Ma kirjutasin oma mure ära ja sain sama päeva jooksul kuidagi oma 
vastused kätte. 

\question{Mina tean sind rohkem infoturbe inimesena. Mis hetkel sa 
hakkasid arvutite toomise asemel arvutitega seotud probleeme lahendama?}

Ma ei tea, aga mul on 
alati olnud sügav huvi asjade vastu ja millegipärast mõte töötab ka alati 
tagurpidi, et mida selle asjaga veel teha saab. Sel ajal müüsin automaatvastajaid palju, 
enamik Eestis olnud Panasonicu automaatvastajatest tulid minu käest, 
samuti Citizeni ja Casio kalkulaatorid. Eesti 
Pank\index{Eesti Pank} kasutas valuutakursside teatamiseks Panasonicu 
automaatvastajat, see oli ainuke ametlik kanal, kust sai valuutakursse teada, 
ja seda muudeti vist kord päevas või kord nädalas. Igatahes oli 
see väga tõsine infokanal: helistasid numbrile ja sealt loeti maha, 
et näiteks Ameerika dollar on nii palju. Neil 
oli Panasonicu automaatvastaja vaikeparool muutmata jäänud, seal oli 
kolmekohaline number, vist 555. Ja kui 
valisid rääkimise ajal 555, siis tegi automaatvastaja piiksu, ja kui vajutasid 7, võisid sinna uue teate peale lugeda. Nägin kohe, 
et põhimõtteliselt saaks nii teha. 

\question{Kas tegid?}

Ei. Lihtsalt ütlen, et see oli võimalik. 

\question{Kas sa neile ütlesid, et vahetage oma kood ära?}

Ei, sest tollal ei olnud selleks kanalit, maailm töötas teistmoodi. Siis ei olnud ju isegi 
telefoni, vaid pidi minema telefoniputkasse ja otsima telefoniraamatust numbri. 
Asjad ei käinud nii nagu praegu. Kui tahtsid sõbrale helistada, siis mõtlesid, 
et homme helistan, kuna homme lähen inimese juurde, kellel on 
telefon. 

\question{Ometi ei saanud sinust riistvaraärimeest, vaid huvi asjade 
toimimise vastu sai nii tugevaks, et hakkasid hoopis sellega tegelema.}

Olen vist kogu aeg jooksnud huvi ja raha kombinatsiooni järgi. Näiteks 
sattusin haltuura tegemise ja spekuleerimise järel Kinexi\index{Kinex} 
direktoriks, mis oli tollal Eesti tuntuim ja küllaltki 
tõsiseltvõetav arvutifirma. See tegeles kõigega ja tarkvara osa oli päris oluline. 

\question{Sa pidid siis ju hakkama inimesi juhtima!}

Jah, aga ma olin seda tegelikult kogu aeg teinud, näiteks enda 
erafirmas, mida me kahekesi tegime. Alguses seisime kordamööda letis: kui tema 
jooksis kauba järele, seisin mina leti taga, ja kui mina olin Venemaal, siis seisis 
tema leti taga. Ühel hetkel oli raha nii palju, et mõtlesime, mida me siin 
seisame, võtame kellegi tööle. Võtsimegi. Siis istusime kodus ja 
vaatasime, kuidas see keegi müüb. Ja mina kirjutasin poele nullist tarkvara. Pikapeale läks
seltskond päris suureks, meil oli Tallinnas palju poode ja 
päris arvestatav hulgiäri. Enamik Eesti kontoritehnikast tuli meilt. 

\question{Kas inimeste juhtimine tuli sul loomulikult välja?}

Jah, see oli sõpruskond, kes usaldas üksteist ilma suuremate probleemideta. Ma ei ole kunagi konfliktidesse 
sattunud. Seda olen küll täheldanud, et kui ma ära lähen, siis on aeg-ajalt 
hakatud üksteisele jalga taha panema ja konfliktid
eskaleerunud.

\question{Mida sa praegu teed?}

Praegu on mul selline firma nagu Tochimo Lab\index{Tochimo Lab}. See on nii uus, et keegi ei ole sellest veel kuulnudki, aga mõte on teha
Planet Way Corporationi all Skunkworksi moodi moodustis, kus 
teeme uusi projekte. Planet Way on ise muutunud selliseks, et meil 
on väga tõsised kliendid ja tootmine peab igapäevaselt jooksma, seal ei tohi 
mitte midagi katsetada -- kõik peab käima nagu kellavärk. 

\question{Aga katsetada sulle meeldib!}

Jah, mul on vaja teha just asju, mida veel ei ole olemas. 

\question{Miks?}

Uute asjade tegemine on ääretult põnev ja olen viimasel ajal saanud
aru, et mida võimatum ülesanne, seda rohkem see mulle meeldib, ja see on osaliselt muutunud mu tugevuseks. Pen-testide\sidenote[][]{Penetratsioonitest -- autoriseeritud küberrünnak, 
mille käigus testija üritab süsteemi siseneda nii, nagu päris häkker seda 
teeks.} tegemine on andnud sellise mõttemaailma, et lammutad süsteeme, mis 
on ehitatud kindlaks. Ma ei taha pen-teste enam teha, sest see on 
tõsiselt depressiivne töö, aga see on andnud lihtsa eelise, et kui 
kaks nädalat ei tule ühtegi ideed ja tekib 
totaalne depressioon, siis aeg-ajalt tuleb pärast seda läbimurre. 
Inseneriteadustel on üldiselt see hea omadus, et kui paned piisavalt 
ressursse alla, siis hakkab iga asi juhtuma. 

\question{Kas nüüd tegeledki uute asjade leiutamisega?}

On asju, mille järele on olemas ärilised vajadused (päris udu ei tee), aga 
millele ei ole päris selgeid vastuseid. 

Pooltel inseneridel on see häda, et kui anda neile väga udune 
ülesanne, siis nad ei suuda seda teha.

\question{Just, sest kuidas arvutada midagi, mille kohta ei tea, mida see 
tegema hakkab!}

See ongi Skunkworks. Kui 
SR-71\sidenote{Lockheed SR-71 Blackbird on strateegiline pikamaa
luurelennuk, mille arendas välja Lockheedi Skunk Worksi osakond. 
Lennuki loomisest Clarence \enquote{Kelly} Johnsoni käe all on tema toonane 
alluv Ben R. Rich kirjutanud inseneride hulgas populaarse raamatu, mida 
loetakse nii innovatsiooniõpikuna kui ka inspiratsiooni saamiseks. } 
tehti, siis eesmärk ei olnud teha mitte kolme-machine-lennuk\sidenote{Lennuk, mis suudaks kolmekordselt helikiiruse ületada.}, nagu lõpuks välja 
kukkus, vaid Venemaa kohal luurata. Keegi ei 
öelnud, mida tegema pidi. Tollal ei 
tundunud ükski asi reaalne, sest kõike, mis lendab, saab raketiga alla lasta. 

Peab olema sellise peakujuga inimene, kes mõtleb nii kaua, kuni saab selle 
vastuse. Tehti nii kiire lennuk, et selleks ajaks, kui rakett õhku tõuseb, on 
lennuk taevast kadunud. Praegu tundub see vastus elementaarne, aga 
tollal oli see võimatu. Selleks peavad olema 
inimesed, kes ei mõtle esimese asjana, et ma ei võta ette, see ei ole 
tehtav, vaid tarbivad kuu aega raha ja muid ressursse ning 
tulevad välja kõige hullumeelsemate mõtetega, millele pannakse hind külge. 
Siis on juba kliendi asi, kas ta tahab seda või ei taha. Näiteks SR-71 
ehitamisel oli see hind, et läks vaja suuremas koguses titaani, kui terve 
läänemaailm toota suutis. See tähendas ärioperatsiooni kuskil Venemaa kõhus, 
kust sai titaani osta. CIA tegi erioperatsiooni, et varjata, milleks ostetavat titaani vaja läheb. Kusjuures titaan on 
nii haruldane materjal, et selle otstarvet on raske varjata. 

\question{Titaanist kelli ju sellises koguses ei tee!}

Just, see oli väga tõsine ressursiprobleem. Ja olla siis see hull, kes ütleb, et 
kuulge, teeme ülikalli asja, me suudame teha! Kusjuures need 
vennad, kes lennukit ehitasid, ei teadnud, kas nad suudavad. 

\question{Aga nad ütlesid ja uskusid, et suudavad.}

Jah, ja kui loed nende inimeste kirjutatud raamatuid, 
siis\ldots{ }Lennuk oli pooleldi ehitatud ja siis ilmnesid mingid hädad, näiteks 
et lennuk venib kuumenedes kolmkümmend sentimeetrit. Selgus, et kütusevoolikud, mis mootorisse jooksevad, 
peavad ka venima, aga kuna temperatuur tõuseb kuuesaja kraadini, 
siis ei saa voolikuid teha ühestki mittemetallist, aga metall ju ei veni. Tekkisid 
probleemid, mida ei olnud võimalik lahendada. Nad tegid torud 
üksteise sisse. Kui lennuk on maa peal, siis lekib kohutavalt. Lennuk 
tangitakse minimaalselt täis, lendab üles, teeb paar ülehelikiiruselist 
tõmmet, kuumeneb mõnisada kraadi ülespoole ja siis pannakse 
tankurlennuki pealt paagid täis. 

\question{Kas sinul on see usk olemas, et mõtled välja ja ongi võimalik?}

No vot, selleks peab täiesti hull olema, et mitte tagasi 
põrkuda. Pen-testide tegemine on andnud selle, et ma enam ei pelga hullusti 
probleeme. 

\question{Kas seda on ka juhtunud, et ei tule välja?}

Oi, kindlasti. See oli üks 
põhjus, miks ma pen-testimise maha jätsin: iial ei tea, millal ja kas jõuad
tulemuseni, ja see on meeletult depressiivne. Teine häda pen-testidega on see, et igal juhul saad peksa. Kui sa seda ära ei lõhu, mida 
ette antakse, siis oled nõrk, ja kui lõhud ära, siis on kõik su peale 
solvunud. Tavaliselt on lõhkumine nii lihtne ka ja siis
öeldakse, et nojah, nii me oleks isegi osanud. See
on tõsiselt ebameeldiv töö, ei soovita kellelegi. 

Uute asjade ehitamine on selles mõttes lahe, et kui olen sinna aega 
investeerinud, siis olen tavaliselt sealt ise midagi saanud ja 
kommuun samuti. See on see koht, kus mulle meeldib 
inseneride hulgas näidata, mida ma tegin. 

\question{Ja sa ju ei ehita triviaalseid asju! Kuidas sa oskad? Lihtsalt 
kogemusest?}

Vist jah, ma isegi ei tea, mida sa silmas pead. 

\question{Näiteks see sõrmus, mille kallal sa töötasid.\sidenote{Tõnu on 
ehitanud sõrmuse, mis toimib {\v z}estikontrolleri, võtme, NFC maksevahendi 
ja märguandjana, olemata palju suurem tavalisest gümnaasiumi lõpusõrmusest.}}

Sõrmusega oli lihtne: sellel on Bluetoothi saatja, 
patarei, väike mikroprotsessor ja andurid. Sellist asja suudab 
igaüks ehitada. Ainuke asi, et ei tea, kui suurt. Esimese asjana mõtlesingi 
välja, et kui tahaksin midagi sellist ehitada, siis kui suur see umbes 
tuleks. Vaadates, mida poes müüakse, Arduino näiteks, siis tegelikult suudab igaüks 
selle väga lihtsalt ehitada. 

Seejärel mõtlesin, kas suudan selle 
väiksemaks teha. Kuna ma elan Jaapanis, 
siis võtsin esimeseks reegliks, et peksame igast suurfirmast välja lahenduse, mis on väiksem kui 
turult saadav. Kui tean, et mingi asi on näiteks 
kümnemillimeetrine, siis lähen nende ukse taha ja ütlen, et enne ma ära ei 
lähe, kui saan üheksamillimeetrise. Ma tahtsin konkurentsieelist. 

Ja tuli lihtsalt ehitama hakata, sest siis tulid ka avastused. Esimene avastus oli 
see, et teatud asjad, mida pidasin oluliseks, polnud üldse olulised. Näiteks kui
palju Bluetooth voolu tarbib. Selgus, et see ei loe üldse, vaid hoopis 
see, kui palju Bluetooth magades voolu sööb, sest selle saatmise hetk on 
niivõrd lühike, et see võib tarbida, palju tahab, mind see ei sega. Aga kui see
paari päevaga magades tühjaks jookseb, siis see häirib mind. Me suutsime 
sõrmuse ajada nii kaugele, et see suudab viis aastat karbis voolu sees hoida. 
Kui klient saab karbi kätte ja teeb lahti, siis sõrmus ärkab ellu, kui 
see on toodetud viimase viie aasta sees. 

\question{Tundub, et sul on teatud printsiibid, millest lähtudes on võimalik ehitada mida 
iganes.}

Tavaliselt on jah mõni väga lihtne läbiv idee. Tuleb teha endale lihtne rusikareegel. Näiteks auto 
on mootor, rool ja pidurid ning sa pead hakkama seda selle põhjal kuidagi tükeldama, minimaalse asja valmis tegema. Siis tekib arusaam, milline on 
probleem, mida sa tegelikult lahendad. 

\question{Praktiline käegakatsutav lahendus?}

Jah, ma tegelikult ei saa vist üldse 
keerulistest asjadest aru. Minu esimene samm on asja lihtsustada. 

\question{Keegi ei saa keerulistest asjadest aru, seepärast need ongi 
keerulised!}

Jah, aga mul on tunne, et see on vist see tee, kuidas ma asju teen. 

Toon teise näite. Mind hakkas millalgi huvitama arvutiga 
nägemine: kuidas arvuti näha saab? Võtsin raamatu ja hakkasin otsast lugema, 
mis on see OpenCV teek, mis on \emph{computer vision}'i ehk arvutiga 
nägemise puhul kõige levinum teek. Kui olin poole raamatu peal, siis
mul juba näpud sügelesid kohutavalt, sest olin kõik 
ideed kätte saanud, mida see teeb, ja printsiibid olid lihtsad. 

Kõik teavad, et arvutiga saab nägusid otsida -- tänapäeval teeb seda juba iga 
telefon. Aga mind hakkas huvitama, kas suudan kokku panna, et kui 
inimene on pooleks lõigatud, siis milline on alumine ja milline ülemine ots. 
Nägusid me leiame, aga kas ka jalgu või midagi 
muud? Kas seesama printsiip on rakendatav? Üritasin midagi kokku käkerdada ja 
sain tulemuseks, et võid frankensteine ehitada nii palju, kui tahad, 
arvuti leiab inimesele täiesti sobiva alakeha. See näeb maru naljakas välja, 
aga on ülimalt loogiline, selline inimene võiks isegi 
olemas olla. Ainult et ta ei ole õige. 

Tähtis on proovida. Tollal jõudsin selle projektiga 
nii kaugele, et sain aru, et tegelikult on taust palju olulisem kui inimene. 
Kas sa oled kuulnud sellest projektist, kus ma tõmbasin terve 
rate.ee\index{rate.ee}\sidenote{rate.ee oli Eesti esimene tõeliselt 
populaarseks saanud sotsiaal{\-}võrgulaadne teenus, mille sisu seisnes peamiselt 
üksteise piltidele hinnangute andmises.} alla? See oli hästi lihtsasti 
kopeeritav ja seal oli praktiliselt terve elanikkond sees, igaühest 
hulk pilte. Rate.ee sai alla tõmmatud ja avastasin, et Eestis on
selline sait nagu sexinestonia.com -- väike kommuun, umbes tuhat kasutajat. Kui 
Eestist on tuhat kasutajat, kes kasutab pornosaiti iseenda reklaamiks -- üks on su õpetaja, teine kolleeg, kolmas ülemus --, 
siis on see väga sensitiivne asi. Üritasin leida paralleele, millised 
profiilid kattuvad rate.ee omadega. 

Mind huvitas tehniliselt, kas olen võimeline neid kokku viima, ja 
avastasin, et neid ei ole väga palju, kes kasutab mõlemas teenuses sama 
telefoninumbrit. Selle järgi kokku viia oli elementaarne, aga oli 
muidki asju. Ma vaatasin \emph{computer vision}'iga taustamustreid. Tuli välja, et tapeedimuster on üsna unikaalne asi. Kui 
pildil on kolm erinevat mustrit, näiteks tapeedi-, vaiba- ja mööblimuster, ja kui nende kombinatsioon on unikaalne, 
siis see ongi unikaalne ruum. Ja kui leiad juba ruumil samasuse, siis 
tavaliselt on ka profiil kohe arusaadav. Saad täpselt aru, et see 
rate.ee-s olev tore väikeste lastega inimene on sama, kes sexinestonias 
kõiki neid muid asju teeb. 

\question{Need on ju küsimused, millele inimesed ei taha üldjuhul vastust 
saada. Miks sina tahad?}

Mind huvitas, kas see on võimalik, ja see oli võimalik. See oli väga 
vapustav avastus, et nii saabki. See tuli ka IT-inimestele üllatuseks, 
isegi turvaala omadele. Kui käisin seda pankuritele 
näitamas, siis neile tuli see selles mõttes ebameeldiva üllatusena, et 
pangas on tuhandeid tellereid ja kui nad hoiavad endast kuskil alasti pilti, siis
muutuvad nad santažeeritavaks. Panga jaoks on see probleem, kui häkker 
teab seda, aga pank ei tea. 

Tekkis mõtlemine, et tuleb iseenda käitumist korrigeerida. 
Keegi teab sust alati rohkem, kui sa ise arvad. 
Tavaliselt on see lihtsalt vandenõuteooria, aga see on see koht, kus saad ise 
tunda, et mina tegingi selle süsteemi, mina tean. 

Ma kasutasin seda spämmerite püüdmisel ära. Suutsin nende 
kohta nii mõnegi pildi leida.

\question{Läksime küll BBSide juurest kaugele, aga sellest ei ole 
lugu, sest su jutt on väga huvitav! Aitäh!}

Sa küsid küsimusi, mida ma ei ole iseendaltki korralikult küsinud. Ma ei tea, miks ma 
midagi teen või kuidas ma teen! 

\question{Ega ei peagi teadma!}

Põhiline reegel on see, et üritan olla ise ja iseenda vastu aus.


\chapter{Asko Seeba}
\label{sisu:asko}
\index[ppl]{Seeba, Asko}

\question{Kuidas jõudsid arvutid sinu juurde ja sina arvutite juurde?}

Mina elasin oma teadliku lapsepõlve 
Viljandimaal, Viljandist Riia maanteed pidi linnapiirist viis-kuus kilomeetrit välja sõita, ja käisin linnaservas Carl 
Robert Jakobsoni nimelises Viljandi 1. Keskkoolis\index{Viljandi 
1. Keskkool}. Hiljem oli see Jakobsoni 
gümnaasium ja nüüd peale riigigümnaasiumite tegemist gümnaasiumi enam ei ole, 
Jakobsoni kooli nime all on ainult põhikool. 

Arvutiteni jõudsin sealsamas koolis. Meil oli tore arvutiõpetaja 
Heiki Pettai\index[ppl]{Pettai, Heiki}, kes tegutseb vist praegugi 
IT-vallas, küll enam ammu mitte õpetajana, aga rohkem 
spetsialistina. Käisin neljandas või viiendas 
klassis, kui sain teada, et naabripoiss Toomas Aas\index[ppl]{Aas, Toomas} (praegu üks kõvemaid tarkvaraarendajaid) käib arvutiringis. Ta oli sellest 
rääkinud, aga minu teadmised arvutist olid hästi lapselikud. Olin arvutit näinud telekast lastesaates ja sellega
tehti midagi naljakat, aga ma ei osanud sellest tol hetkel 
midagi arvata. 

See hetk, kui klõps käis, oli hästi ootamatu ja 
lühike, enam-vähem sekundiga. Sattusin ükskord nägema koolikoridori 
peal, kuidas seesama naabripoiss läks arvutiklassi pisikesest uksest sisse ja kui uks paotus, paistsid sealt arvutid!
Seal oli küll ainult kolm arvutit, Vene DVK-2d\index{DVK!DVK-2}, aga see oli minu jaoks
maagiline moment, et märkasin midagi, mida olin telekast näinud ja
väga kaugeks pidanud, ning nüüd oli see järsku kahe-kolme meetri 
kaugusel. Klõps käis ära ja ma küsisin täiesti spontaanselt, kas 
tohin ka sisse tulla. Arvutiõpetaja Heiki Pettai 
lubas ja ma olin hetkega müüdud ning tahtsin seal käima 
hakata. Sellest hetkest peale teadsin, et mu ülejäänud elu peab olema 
arvutitega seotud. 

\question{Oskad sa öelda, mis täpselt see maagiline asi oli?}

Mis see lapsel võis olla? Et näed mingisugust lahedat asja, millest aru 
ei saa. Tolleaegsed arvutid olid rohkem sellised --- inglise keeles 
on tore väljend \emph{exposed} --- füüsiliselt avatud, 
igasuguseid keerulisi asju sai silmaga näha: juhtmeid, trükiplaate ja muud värki ja möllu. Tolleaegse 
poisina köitsid mind mehhanismid, tehnika ja asjad oma 
koledas ilus. 

\question{Mis aastal see oli? Kaheksakümnendate keskel?}

1982. aastal läksin esimesse klassi, 
sealt loeme neli-viis aastat edasi, nii et 1986 või 1987. 

\question{Võrus selliseid arvuteid ei olnud, need tekkisid märksa hiljem. 
Teil pidi olema millegi poolest eriline kool, et suudeti arvutid 
välja rääkida. Või oli õpetaja eriline?}

Heiki Pettai\index[ppl]{Pettai, Heiki} oli suhteliselt noor õpetaja, kui ta 
meie kooli tuli, vist otse Tartu Ülikoolist. Ma täpselt ei tea, lihtsalt 
spekuleerin, et tal olid jätkuvalt aktiivsed suhted ülikooliaegse 
kontaktvõrgustikuga, näiteks Viljo Sooga\index[ppl]{Soo, Viljo}. Sealtkaudu võis infot liikuda ja ta 
võis olla õigel ajal õiges kohas, et sai Viljandi kooli midagi hankida. 

\question{Kas arvuteid kasutati ka õppetööks või käis seal ainult 
arvutiring? Kuidas nende kolme masina peal õpetada sai?}

Tagantjärele mõeldes nägi see hästi improviseeritud värk välja küll. 
Ta üritas ka keskkooliõpilastega väikestes rühmades tunde
läbi viia, sest ruumi ei mahtunud 
palju inimesi. DVK-2\index{DVK!DVK-2} masinatel oli olemas graafikakaardi \emph{slot}, 
aga graafikakaarti ühelgi sees ei olnud, jooksis ainult 
tekstipõhine režiim. Lapsed harjutasid, kuidas 
ASCII graafikas tekstiredaktoriga pilti joonistada. Nooleklahvidega ringi
sõites ja sümboleid vajutades sai joonistada 
ning õpetaja pani selle eest hindeid. Lisaks
lihtsamate asjadega tegelejatele oli seal
paar ägedamat, häkkerimat last. Kas Ivar Smolin\index[ppl]{Smolin, 
Ivar} oli juba seal olemas või tuli ta hiljem, kui see klass kolis
kutsekasse? Sealt kerkis järgnevate aastate jooksul teisigi tänapäeval tuntud inimesi, 
nagu Janek Hiis\index[ppl]{Hiis, Janek} ja Kaido 
Kärner\index[ppl]{Kärner, Kaido}.

\question{Kas sina joonistasid ka pilte või tegid midagi muud?} 

Alguses ei osanud ma muud teha, lihtsalt põnev oli arvutit 
katsuda. Jube äge oli klaviatuuri klõbistada ja vaadata, kuidas ekraanil toimub selle 
peale midagi. See oli püha emotsioon, mille nimel 
tasus istuda ja kannatlikult järjekorras oodata. Aeg-ajalt, kui rahvast oli vähem ja kas õpetaja ise või mõni 
edumeelsem õpilane teadis, millise flopiketta peal mängud asusid, 
sai Rottide\index{Rotid}-nimelist mängu mängida, mis oli sisuliselt 
Pacmani\index{Pacman} imitatsioon. Ka Snake\index{Snake} jooksis 
kusagilt flopi pealt. Vahepeal sai niisiis mängida, aga õpetaja üritas 
mängimise fooni loomulikult natuke alla suruda --- see oli rohkem nagu 
preemia, kui midagi asjalikku ära tegid. 

\question{Mis oli \enquote{asjalik}? Kas koodi kirjutasite?}

Mõned ägedamad vennad juba kirjutasid koodi ka. Ma ise DVK-2\index{DVK!DVK-2} peal veel koodi 
kirjutamiseni ei jõudnud. Selleks oli vaja 
rohkem vaba aega, kui lapsed parasjagu ei rüselenud 
liiga palju ja sai süveneda. Need, kes käisid 
lähemalt kooli ja said seal hilisematel õhtutundidel istuda, olid 
eelistatud seisus, sest mina elasin linnast väljas 
ja pidin bussigraafikuga arvestama. Niisama lihtsalt seal hilja õhtuni hängida 
ei õnnestunud. 

Koodimiseni jõudsin aastake või paar hiljem, kui see klass 
liikus suuremasse ruumi ja tulid pisikesed 
BKd\index{Elektronika!BK}\sidenote{Nõukogude kuueteistbitiste 
koduarvutite sari, mida huvitaval kombel (sest Viljandisse sattusid need teises järjekorras) 
peetakse varem mainitud DVKde 
eelkäijaks.}. Nagu tol ajal ikka, tekkisid need kusagilt Vene 
arvutitööstusest\sidenote{Selle arvuti töötas 1983. aastal välja Zelenogradis 
asunud asutus \begin{russian}НПО \enquote{Научный Центр}\end{russian}, 
toonase Nõukogude Liidu juhtiv mikroprotsessorite disaini 
keskus.}, neil olid hästi väikesed monitorid ja väike 
must kandiline aju või plokk, mis oskas kas makilindi pealt 
või siis võrgukaabliga ühendatult emaarvutist andmeid lugeda. Emaarvutiks oli pandud DVK-2\index{DVK!DVK-2}, mille 
ketaste pealt BKd said lugeda mingisuguse protokolliga, 
millest ma ei teadnud tol ajal ega ka tagantjärele midagi. Teadsin ainult, 
mis käske tuleb sisestada, et asjad toimiksid.

BK-l oli kuhugi püsimälusse sisse keevitatud 
BASICu\index{BASIC} interpretaator. Kui selle sisse lülitasid, siis 
esimese asjana tuli ette \emph{line number 10} --- hakka kirjutama. 
Nii et sisuliselt sai interaktiivselt BASICu käske kirjutada ja siis
hakkasingi esimest korda koodi kirjutama. 

\question{Mida need esimesed programmid tegid?}

Mis see teismelise koolipoisi kõige esimene programm ikka põnevat teeb? 
Kõigepealt oli BASICus käsk number 10, mis printis ekraanile midagi toredat, näiteks 
\enquote{loll}. Järgmine rida 
oli käsk number 20, mille peale oli \verb|goto 10|. Selline tore lõpmatu 
tsükli harjutus, aga sellele järgnesid kiiresti igasugused muud näpuharjutused. 
Seal oli juba graafika olemas, sai ekraanile jooni kuvada, ja siis esimene 
tsükli harjutus oli see, et sai joon kuidagi liikuma pandud ekraani ühest 
servast teise. 

\question{Ega ei saanud ju lihtsalt joont liigutada, eelmine positsioon tuli 
mustaga üle joonistada \ldots}

Jah, just, sealt hakkas vaikselt algoritmika, mis
sundis lapse aju algoritmiliselt mõtlema. Kõik vead paistsid kohe 
välja, kui olid midagi valesti mõelnud. 

\question{Kust see programmeerimisõpetus tuli? Kas õpetajalt või raamatutest?}

See tuli pigem kellegi käest, me lastena ei viitsinud väga
manuaale lugeda. Aeg-ajalt näidati küll, et näe, loe sealt. Need tekstid olid üldjuhul venekeelsed. Vaatad natukene tuima näoga nagu ahv kirjutusmasinat ja siis 
küsid ikka naabripoisi käest, et kuule, kuidas sa seda tegid. 
Mõningaid asju näitas õpetaja, teisi asju mõni teine targem laps --- niimoodi killuke siit ja sealt muudkui korjasid. 

\question{Ja esimene tunne ei läinud üle?}

Ei, üle see ei läinud. Psühholoogiline sõltuvus või 
vajadus arvuti taha istuda ja seal midagi teha oli kogu aeg olemas. 
Eks lastel mängib rolli mängudega jändamise võimalus. Mul oli motivatsioon kohale minna, et äkki saab mängida. Aga 
kuna vaikselt tekkis ka programmeerimise kihk, siis oli see 
piisavalt põnev, et kutsus sinna asju tegema. 

Põhiline oli see, et vahel lubas õpetaja lastel, keda ta rohkem tundis või 
usaldas, pisikest BKd\index{Elektronika!BK} kas koolivaheajaks võiks suveks koju viia. See
oli piisavalt väike, mahtus kotti. Aga sellega oli üks probleem: 
kuna salvestusseadet ei olnud, siis oli 
kaks varianti. Kas tõmbasid makilindilt programmi sisse ja selleks pidid olema
kõik vajalikud kaablid, juhtmed ja oskused, et sellega õigesti 
ümber käia. Või teine variant, et sul oli programmi \emph{printout} ja iga 
kord, kui tahtsid mängida, pidid kõigepealt kogu mängukoodi 
vigadeta sisse toksima. Ühesõnaga, tund-poolteist nägid vaeva ja 
järgmised tund-poolteist said mängida --- see oli päris huvitav kogemus. 
Tagantjärele mõeldes pidid need mängukoodid hästi ökonoomsed olema, et neid sai lühikese ajaga 
sisse toksida. 

\question{Kas teil 
häkkimist ei esinenud? Siin on räägitud\sidenote{Vt lk 
\pageref{sisu!ylikooli_root}.}, kuidas inimesed veel enne keskkooli ülikooli 
adminnide käest root-õigused ära võtsid.}

Mul otsest spetsialiseerumist või 
spinni ei tekkinud, et oleks kursi mõne konkreetse 
asja peale võtnud. Olen eluaeg olnud tarkvaraarendaja, ma ei ole 
läinud kuhugi riistvara häkkima ega muud säärast tegema. 

\question{Kas see ei ole huvi pakkunud?} 

Eks vahel on olnud uudishimusähvatusi, aga minu jaoks on kogu aeg 
olnud piisavalt atraktiivne tegeleda mõne uue laheda 
tarkvarakeskkonnaga. 

Ma olin matemaatika-füüsika
süvaklassis ja meil oli keskkoolis eraldi arvutitund. Tegime Jukudes\index{Juku}
dBase'is programmeerimisülesandeid, kus oli vaja programmeerida
andmebaasi- või tabelarvutuselaadseid asju. dBase on FoxPro 
sugulane ja päris äge. Oluline oli see, et 
kogu aeg oli midagi uut avastada, ja midagi muud pole mul motivatsiooniks vaja olnud. Pidevalt on uue 
asja avastamise rõõm. Andi Hektor\index[ppl]{Hektor, Andi} oli mu 
klassivend --- paljud teavad teda kui Eesti üht tuntumat füüsikut. 
Arvutitunnis olime sisuliselt kaks ärksamat pead, istusime ja 
õpetasime vastastikku üksteist ning tegime keerulisemaid asju. 

\question{Kas arvuteid muude ainete, näiteks matemaatikaga ka seoti?}

Mainisin just Andi Hektorit\index[ppl]{Hektor, Andi}, tema 
jaoks olid ilmselt füüsika ja keemia väga köitvad ained. Ta oli 
väga terav ja käis olümpiaadidel, pani neid järjest kinni. Tartu Ülikooli 
füüsikasse sai ta ilma eksamiteta sisse tänu sellele, et oli vahetult enne 
keskkooli lõppu vabariikliku füüsikaolümpiaadi võitnud. Tema puhul 
kindlasti see pool toimis. Minul kukkusid füüsika ja matemaatika kuidagi
loomulikult välja, sain vajalikud hinded kätte ja osalesin isegi 
Tartu Ülikooli matemaatikakoolis, mida keskkooli õpilastele 
kirja teel korraldati. Üritasin valmistuda Tartu 
Ülikooli sissesaamiseks, aga ikkagi rohkem informaatika motiiviga. Mul ei olnud
otsest huvi füüsikavalemitesse kaevuda, tugev emotsioon oli ikkagi 
arvutite vastu. 

\question{Kas sul muusika- või kirjandushuvi oli ka? Sa muusikamees oled ju olnud?}

Hobi korras mängisin jah kitarri. Nokkisin selle üles 
teismelisena isa kõrvalt, aga klassikalist muusikakooliharidust mul ei 
ole. Kõik, mida ma muusikast tean, olen ise üles korjanud. 

\question{Kas sellist mõtet, et üritaks Jukuga midagi lindistada või muusikat 
teha, ei tulnud?}

Nii kaugele ma tollal ei jõudnud. Mingisugused tüübid tulid ükskord 
arvutiklassi ja lasid päris äratuntava 
kvaliteediga Roxette'i muusikat läbi Juku. Aga see oli ka ainus selline 
moment. 

Üks moment meenub veel, olin siis keskkoolis. 
Lõpetasin keskkooli 1993. aastal, nii et see võis olla 1990ndate algul. Millal 
Bluemooni\index{Bluemoon} tüübid SoundClubi\index{SoundClub}\sidenote{SoundClubi hakkasid Ahti\index[ppl]{Heinla, Ahti} ja 
Jaan\index[ppl]{Tallinn, Jaan} kirjutama 1991. aastal Tartus füüsikat õppides 
ja see avaldati \emph{shareware}-litsentsiga 1993. aastal. Samas võisid 
selle versioonid ka varem ringelda.} arendasid? Keskkooli lõpuklassides 
tulid meil 286d --- ma ei mäleta, kas me 386 nägime. 
Igal juhul oli värviline graafiline keskkond juba mingil määral olemas ja 
SoundClub meile sinna kooliarvutitesse jõudis. Sellega sai küll mingit 
tehnomuusikat kokku tõstetud. Mitte et oleksin midagi hullult programmeerinud, oli lihtsalt lahe ja arusaadav kasutajaliides, kus sai
rütmiriffe ja asju kokku pandud, nii palju kui tol hetkel 
muusikalist arusaamist oli.\sidenote{Asko ei olnud ainus. Vennaskonna 
omaaegne hittlugu \enquote{Disko} on loodud SoundClubii abil ja, nagu tegijad on meenutanud, 
umbes samal meetodil.}

\question{Kas arvutiklassis hängiv seltskond oli muus mõttes ka sõpruskond 
või puutusite kokku ainult seal?}

Nii ja naa. Kujunes küll jah välja tuumik, kes sai omavahel ka väga 
hästi läbi. Keskkooli lõpus olid seal peale 
minu ja Andi Hektori\index[ppl]{Hektor, Andi} meist paar aastat 
nooremad Janek Hiis\index[ppl]{Hiis, Janek} ja Janek 
Palõnski\index[ppl]{Palõnski, Janek}. Keegi Kristjan, kelle teine 
nimi praegu ei meenu, oli ka arvutite peal päris kõva tegija. Samuti Raivo 
Kotov\index[ppl]{Kotov, Raivo}, kel on praegu Andrus 
Kõresaarega\index[ppl]{Kõresaar, Andrus} arhitektuuri- ja 
disainibüroo. Andrus oli ka mu klassivend.

\question{Kas sind keskkooliajal tööle ei võetud?}

Kahjuks või õnneks ei tulnud ette. 

\question{Kas Viljandis oleks tollal mõni koht olnud, kes 
oleks võinud programmeerija tööle võtta?}

Ma ei tea, et seal oleks tollal otseselt programmeerija väljavaateid 
olnud. Samas mingid arvutispetsialistid hakkasid küll juba ringi 
toimetama, sest järjest rohkem ettevõtteid võtsid arvuteid kasutusele. Tolleaegse nimega 
Eesti Telefon ja postiasutus panid juba IT-võrke püsti. Valdavalt vist Heiki 
Pettai koordineeriski neid asju. Võib-olla kusagil andmeid sisestada oleks 
heal juhul saanud, mis oleks olnud keskkooliõpilase jaoks okei. 

\question{Kas pärast keskkooli lõppu läksid otsejoones Tartu Ülikooli matemaatikat 
õppima?\index{Tartu Ülikool!Matemaatikateaduskond}}

Jah, kuna olin kooliajal õppeedukuse
poolest suhteliselt lohh, siis sain nibin-nabin ülikooli sisse. 
Õnneks oli siis üks madalama konkurentsiga aastaid. Olin esialgses pingereas joone peal täpselt viimane, kes sisse sai. 
Siis olid ju veel sisseastumiseksamid, kombinatsioon 
eksamihinnetest ja lõpuhinnetest. Sain sisse, tekkis jess!-emotsioon ja edasi hakkas ülikoolielu. 

\question{Seal vahepeal oli kummaline periood, kui Nõukogude sõjaväeteenistus 
läks Eesti kaitseväeteenistuseks üle --- kas sa sõjaväes ei käinud?}

Jah, see aken oli hästi lühike ja ma sattusin täpselt sellesse aknasse: 
nõukaaegne armeekord oli 
lagunenud ja sinna ei võetud juba mitu aastat, aga Eestis kehtestati üldine 
sõjaväekohustus 1993. aasta sügisel. Ma olin suvel ülikooli sisse 
saanud ja kõik enne sügist sissesaanud olid justkui vabad, eeldusel et nad ülikooli ära lõpetavad. 

\question{Naljakas aken oli jah! Mina pääsesin tervisega, aga meie kursuselt 
ei käinud keegi sõjaväes.}

Kusjuures mul oli endal selline suhtumine, et 
oleks täitsa okei olnud minna. Tegelesin sel ajal maskuliinsemate spordialadega, näiteks harrastasin karated, ja 
arvasin, et mis see sõjavägi siis ära ei ole --- kui vaja, siis 
teen ära. Aga ära see minu jaoks jäi ja hiljem olin selle üle õnnelik. 
Tollal oli sõjaväesüsteem 
väga lapsekingades ja ei oleks tõenäoliselt midagi 
väga meenutamisväärset olnud. Minu 
klassivendadest kaotas ajateenistuse tõttu oma elu kaks inimest. See 
näitab tollast taset, õnnetusi ja korralagedust oli veel päris palju. 

\question{Kui ma su juttu kuulan, koorub välja suhteliselt haruldane 
kombinatsioon: teeks sporti ja olümpiaadidel pigem ei käiks, aga samas 
programmeeriks isuga.}

Neid asju, millega ma paralleelselt tegelesin, oli tegelikult mitu. Võib-olla oli seetõttu raske otsustada, mille juurde jääda. 
Laulmise ja muusika mõttes mul kuulmist oli, aga määravaks sai see, et 
kuna mul muusikakooli haridust ei olnud, siis olin juba 
rongist maas. Kaalusin isegi 
kultuurikolledžisse minekut, aga olin ikkagi suhteliselt lahja 
vend. Spordiga sai tegeldud ja käisin ka kunstiringis, muuhulgas koos 
Kotovi\index[ppl]{Kotov, Raivo} ja Kõresaarega\index[ppl]{Kõresaar, Andrus}. Seda vedas Jakobsoni gümnaasiumis Grünbach\sidenote{Asko peab ilmselt silmas õpetaja 
Rein Grünbachi\index[ppl]{Grünbach, Rein}.}. Aga
lõpuks jäin ikkagi arvutite juurde, kuna tundus, et sellega läheb kõige paremini. Motivatsiooni mõttes need teised asjad 
ilmselt ei kinnistunud nii tugevalt kui arvutid. 

\question{Ja nii astusimegi sinuga koos 1993. aastal matemaatikateaduskonda. 
Esimesed kaks aastat tambiti meile haljast matemaatikat, kuidas see tundus?}

Ülikool oli minu jaoks omaette saaga. Tüütu oli see, et kõik pidid 
esimesed kaks aastat sama programmi õppima. Alguses ei saanud veel 
otsustada, kas minna informaatika, matemaatika või statistika suuna 
peale. Valikuvõimalus anti alles teise aasta keskel ja siis vaadati ka õunte 
pealt, kui hea sa oled ühes või teises asjas. Astusin Tartu Ülikooli 
matemaatikateaduskonda selle tõe pähe, et informaatikasuund on seal 
olemas, aga kas sinna saab, seda alguses ei teadnud. Selles 
mõttes oli ülikool minu jaoks paras \emph{challenge}, sest olin lohh edasi, vähemasti esimestel aastatel. Mul ei läinud
matemaatikaained just kõige paremini, kuna need ei olnud minu jaoks motivatsiooni 
põhipõhjus. Alguses oli päris palju keerulisi asju, näiteks matemaatiline analüüs I ja II.

\question{Matemaatiline analüüs I võttis ju lausa kolmandiku kursusest!}

Jah, see niitis rahvast korralikult, aga see ei olnud veel kõige hullem. Kõige 
hullem oli võibolla isegi algebra Mati Kilbi\index[ppl]{Kilp, 
Mati} väga karmi käe all, kõik asjad tuli korrektselt selgeks saada. Minu jaoks oli
algebra oma abstraktsuses kõige raskemini omandatav. 
Matemaatiline loogika seevastu, mida õpetas tollal levinud folkloori järgi 
üks karmimaid õppejõude Rein Prank\index[ppl]{Prank, Rein}, tuli 
lihtsasti, kuna oma mõttemudeli poolest haakus see programmeerimisega palju 
paremini. 

\question{See oli naljakas aine jah, otseselt keeruline ei olnud, aga ometigi
peeti raskeks.}

Ilmselt mingite inimeste jaoks oli see keeruline, aga meie, programmeerijate 
jaoks tuli see kuidagi loomulikult. 

\question{Vanemuise tänava õppehoones olid laiad aknalauad, mille peal istuti, 
sest kuskil mujal ei olnud istuda. Ja seetõttu värviti neid
regulaarselt üle. Sellele vaatamata oli alati kuskile 
sisse kratsitud \enquote{Prank on loll}.}

Meil oli rebaste vandes, kui sa mäletad, palju lauseid, mida me kõike 
tõotasime, ja üks neist oli \enquote{tõotan Prangile kõik eksamid ära teha 
hiljemalt seitsmendal katsel}. 

\question{Paljudel ilmselt nii läkski. Kas sinul oli 
programmeerimisunistus nii tugevalt silme ees, et ronisid ikkagi matemaatikast läbi?}

Lohistasin ennast läbi, aga kriisimoment oli olemas 
küll, olin tegelikult matemaatika tõttu väljakukkumise äärel. Kuna puhas matemaatika mind väga ei motiveerinud, siis veetsin suurema osa 
ülikooliajast 
arvutuskeskuses\index{Tartu Ülikool!Arvutuskeskus}, 
nõndanimetatud Väksu klassis\index{Tartu Ülikool!Liivi Õppehoone!Vase klass}\label{sisu:vase_klass}. Tolleaegse 
vask.ut.ee\index{vask.ut.ee} serveri VT100-terminalid olid ühendatud ühe 
VAX VMSi süsteemi taha. Sealt sain oma esimesed suuremad 
programmeerimise tuleristsed. 

Avastasin seal enda jaoks 
nii-öelda \emph{on the dark side} maailma ehk
mudamängud\index{Muda}. Need olid üheksakümnendate 
esimese poole internetipõhised arvutimängud, virtuaalsed maailmad, kus ei olnud 
midagi graafilist, kõik oli tekstipõhine. Kusagil jooksis server, 
kuhu võeti telnetiga ühendust, ja seal maailmas käis möll ja 
tagaajamine ja \emph{quest}'ide lahendamine. Sattusin üsna kiiresti ise ühe sellise mänguserveri 
programmeerimise meeskonda. Mäng on oma 
olemuselt päris keeruline elukas. Mängumootor peab 
seesmiselt maailma mudeldama, seal on tuhandete viisi 
ruume, liste ja muid asju. Algoritmikat, kuidas seda kõike 
struktureerida, on üksjagu ja see oli minu jaoks üks esimesi tõsisemaid 
C\index{C} programmeerimise kogemusi. 

\question{Kas sa selleks hetkeks juba oskasid Cd?}

Loengutes õpetati programmeerimist Pascali\index{Pascal} baasil, 
nagu sa mäletad. Selle korjasin suhteliselt kergesti üles, kuna see oli 
hea tüpiseeritud keel ja õpetati ka enam-vähem okeilt. Aga kusagil kripeldas, et mingid vennad panevad Cd. Mind
häiris, et ise ei saa. Ostsin eestikeelse C-õpiku, mis oli 
täiesti arusaamatu, kuna oli nii halvasti koostatud. Pealkirja ei mäleta, aga kaanekujundus oli kollane-punane. Selle asemel et näidata esimeses peatükis, kuidas Hello 
Worldi teha, hakati kohe baidi \emph{alignment}'i 
arutama. Kamoon, mis otsast te pihta hakkate! 

Ühel hetkel aga 
jõudis minuni info, et Kernighani ja Richie \enquote{The C Programming 
Language}\index{The C Programming Language}\sidenote{Dennis M. Ritchie, Brian 
W. Kernighan ja Michael E. Lesk. The C programming language. Englewood 
Cliffs: Prentice Hall, 1988. Samast raamatust on juttu ka
lk. \pageref{sisu:richie}.} on hea raamat, otse C-keele autoritelt. 
Igatahes kuskilt ma selle endale hankisin. 

\question{Neid liikus ilmselt Venemaal piraadituna ka.\sidenote{Vt lk.
\pageref{sisu:richie_vene}.}}

Mina ostsin täiesti legitiimse raamatu, mitte 
piraaditud väljatrüki. Mul on see vist siiani riiulis alles, kuigi kapsaks 
muutunud ja natuke teibitud. See oli metoodiliselt hea raamat: 
hakkas lihtsatest asjadest pihta ja läks lõpuks \emph{hard core}'ini 
välja. Neelasin selle läbi, tegin kõik harjutused ära ja 
sain C-keele selgeks. Kui üritada näppu peale panna 
raamatule, mis on mu karjääri kõige rohkem mõjutanud, siis see on 
see. 

Ühesõnaga, olin selle teose läbi protsessinud, enne kui jõudsin mängu 
progemiseni.

\question{Mutta vajus meil kursa pealt ka mitu inimest, kes enam 
Vaxu klassist ei väljunudki.}

Mõned jah, rohkem vajus sinna aasta vanemaid. Muda\index{Muda} 
oli tulnud aasta enne seda, kui meie sisse astusime, internetiga umbes 
ühel ajal. Internet tuligi koos paari 
pahega ja see oli üks peamisi. Nii et osa inimesi sattus mängu haardesse.

\question{Kas sina ei sattunud?}

Mul kestis see periood ainult paar kuud, see konverteerus suhteliselt ruttu 
programmeerimise entusiasmiks. 

\question{Kas toda serverit kirjutanud tiim oli Eestis või välismaal?}

Mudast\index{Muda} oli arendatud palju erinevaid versioone, kõik 
olid tollaste vabavaraliste litsentsidega, avatud koodiga, mida sai 
FTP-saitidelt tõmmata. Nendest arenes siin ja seal erinevate tiimide 
käes igasuguseid \emph{fork}'e\sidenote{\emph{Fork} on tarkvara-maailmas koopia tarkvara
lähtekoodist, mida arendatakse edasi sõltumatult algsest versioonist}. 
Kui \emph{fork}'ide hierarhiat 
joonistada, siis üks kuulsamaid ja levinumaid juur-fork'e oli 
DikuMUD\index{Muda!DikuMUD}, millest oli tehtud haru Merc, 
millest omakorda oli tehtud \emph{fork} nimega ROM. Raul Tölp\index[ppl]{Tölp, Raul} 
oli see, kes võttis ühe ROMi-põhise versiooni ja hakkas sellest Eesti 
oma Estonia-nimelist asja arendama. Tolle versiooniga mina liitusingi. Sai seda maailma edasi arendatud ja kohendatud, vigu parandatud ja muid asju tehtud. 

\question{Too oli ju huvitav kogemus, sest erinevalt ülikoolis õpetatavast 
tavalisest programmeerimispraktikast oli tegu meeskonnatööga!}

Jah, spontaanne meeskonnaelement tuli sisse. Laiem meeskond kirjeldas
mängumaailma, maailmafaile oli hästi palju: 
ruumid, kollid, mobprogid\sidenote{Lühikesed programmijupid, mis käivituvad mängija teatud tegevuste peale ja
võimaldavad kollidel neile reageerida. Samuti on olemas oprogid, mis võimaldavad 
sedasama objektide jaoks.}, propsid ja mis seal sees kõik elasid. 
Ägedamad koodivennad läksid järjest rohkem koodi 
sisse. Ühel hetkel üritasime mingi bandega hakata täiesti nullist, 
hoopis uutel alustel MUDi tegema. 

ROMi-põhine oli C-keele baasil ja võrguga suhtlus 
käis \verb|select| \emph{loop}'iga, deskriptorid olid \verb|select| 
listis, kus on oma piirangud --- maksimum tuhatkond 
\emph{connection}'it saab korraga püsti olla ja kõike protsessiti ühes 
\emph{single-threaded} tsüklis. Hakkasime Toomas 
Soomega\index[ppl]{Soome, Toomas}, kes oli tollal arvutuskeskuse süsadmin, 
ja Peeter Lauaga\index[ppl]{Laud, Peeter}, kes oli minust aasta hiljem ülikooliga 
liitunud, tegema täiesti uut arhitektuuri, mis oli C++\index{C++}-põhine 
ja \emph{multi-threaded}, et saada paralleelsus paremaks. 

Üritasime 
\verb|select| deskriptorite listist ja piirangutest lahti saada ning
tegime igasugust ägedat \emph{hardcore} värki, kus asendasime tekstiga 
maailmafailide sisse parsimise mingi kiirema ja efektiivsem asjaga. Kuna 
maailm koosnes tuhandetest ruumidest, siis serveri \emph{boot} võttis 
mitu minutit aega ja kui server \emph{chrash}'is, närisid mängijad küüsi, et 
kaua läheb ja kas saab uuesti sisse tagasi. Püüdsime selle asendada 
mingisuguse asjaga, kus maailmafailid olid eelkompileeritud 
mälu \emph{dump}'ideks, ja need \verb|mmap|'iga faili sisse 
lugeda, et saaks kohe hoobilt, murdosa sekundiga kõik püsti. 
Moproge hakkasime kirjutama dünaamiliselt lingitud libradeks kompileeritud 
failidena, mida sai jooksev server \verb|dlsym|'iga käigu pealt sisse linkida ja 
käivitada. Ühesõnaga, ajasime kontseptsiooni päris keeruliseks ning omandasime Unixi keskkonnas
korraliku \emph{hardcore} C ja C++ häkkimise oskuse.

\question{Kuidas see kõik sulle külge jäi? Lihtsalt õhust?} 

Korjasime vastastikku järjest üles ja toetasime üksteist. Toomas 
Soome\index[ppl]{Soome, Toomas} tegi otsa lahti ja
viskas palju ideid lauale just arhitektuuri osas. Meie 
Peeter Lauaga\index[ppl]{Laud, Peeter} korjasime ideed suhteliselt kiiresti 
üles ja hakkasime üksteist täiendama. 

\question{Teil pidi siis palju vaba aega olema, kas sa tööl ei käinud?}

Sellepärast mul oligi 
ülikoolis püsimisega raskusi. Veetsin suurema osa ajast 
arvutuskeskuses, vahel varajaste hommikutundideni välja, ja tihtipeale 
loengutesse ei jõudnud. Vahepeal oli mahajäämus vajalikes ainepunktides nii 
suur, et oleksingi võibolla kolmanda aasta keskel välja kukkunud, kui ma 
ei oleks ennast kätte võtnud. Mul käis mingisugune klõps, lapsest sai täiskasvanu. 

Ülikoolistress läks hästi tugevaks ja ühel hetkel jätsin 
mängud ja programmeerimise kõrvale ning hakkasin 
järjest laduma ülikooliõpinguid. Panin ühe 
semestriga 40 ainepunkti\sidenote[][-9mm]{See oli topelt 
tavapärasest semestri õpikoormusest ja nõudis ilmselt tõesti suurt tööd, sest 
kolmandal aastal väga palju lihtsaid aineid enam järel ei olnud.} jutti, 
et saada ree peale. Kui ma varem ei suutnud 
ennast kokku võtta, siis peale seda olen praktiliselt kogu aeg 
suutnud. Baka sai 1998. aastal lõpetatud. Hinded ei olnud head, sest käis sõna otseses mõttes toores tootmine, et kõik ainepunktid 
kätte saada. 

Kui magistrisse läksin (pidin minema tasulise kohale, sest 
hinnete tõttu ma riigieelarvelisele kohale ei pääsenud), siis seal tegin
hindelised ained maksimumi peale. 

\question{Miks sa magistrisse läksid? See ei olnud tol ajal 
vaikimisi valik.\sidenote{Toonane bakalaureusekraad kestis nominaalselt neli 
aastat ja võrdsustati hiljem haridustaseme mõttes praeguse magistrikraadiga.}}

Ma ei mäleta, kuidagi tekkis tahtmine. Magistris sain keskenduda 
teemadele, mis mind ennast huvitasid. Kui bakatasemel oli hästi palju 
sunniviisilist programmi, siis magistris valisin ise, mida teha. Tööle 
läksin 1995. aastal ja esimese palga häkkimise eest sain Tartu 
Ülikooli arvutuskeskusest\index{Tartu Ülikool!Arvutuskeskus}. 

Viljo Soo\index[ppl]{Soo, Viljo} andis operatsioonisüsteemide loengut\sidenote{Viljo 
loetud operatsioonisüsteemide ainel oli väga hea maine, seda peeti keeruliseks 
ning huvitavaks ja seal silma paista oli väga kõva sõna.} ja kord 
ühe loengu vahepausi ajal tuli minu juurde ja küsis, kas tahaksin natukene tööd ka teha. Ta oli ilmselt 
tähele pannud, et korjasin ehk
mingeid asju ladnamalt üles. Arvutuskeskuses oli süsadminnimise 
kõrval ja käigus vaja aeg-ajalt erinevaid asju arendada ja nii mind võetigi 
sinna programmeerijaks. 

\question{Tol hetkel oli võimalik juba ka arvutitega äri teha, kas see ei tõmmanud 
sind?}

Jah, neid tegijaid ümberringi toimetas, aga 
õnneks või kahjuks suutsin niisugustest ahvatlustest kõrvale hoida. Näiteks Alo Toom\index[ppl]{Toom, Alo} ja Ülo Säre\index[ppl]{Säre, 
Ülo} tegid alguses A ja Ü\index{A ja Ü|see{PC Expert}}, millest arenes välja PC 
Expert\index{PC Expert}. Ja mõnda aega eksisteeris niisugune tarkvarafirma 
nagu Codewiser\index{Codewiser}, mis on ka samast ettevõtete perest. Minu lapsepõlvesõber Alo üritas mind A ja Üsse tööle ahvatleda, 
aga tollal oli see rohkem tehnika \emph{support}'i ja konsultatsiooni 
firma, nad aitasid näiteks klientidel katkiläinud kõvakettaid taastada. Mind see eriti ei köitnud, mul oli emotsioon rohkem 
programmikoodi suunas. 

Samuti üritati mind meelitada meditsiinivallas tegutsevasse 
AtFuti\index{AtFut}, mille asutas Jaan 
Pruulmann\index[ppl]{Pruulmann, Jaan}. Teda kutsutakse ka
papa Pruulmaniks, sest Pruulmannide dünastia on suur ja 
lai, neil on seitsmevennaline perekond, kellest suurem osa on täna 
teada-tuntud IT-tegijad. Üks tuttav nimega Elvar Vask\index[ppl]{Vask, Elvar} \emph{alias} 
Cuprum\index[ppl]{Cuprum|see{Vask, Elvar}} sebis mind sinna 
töövestlusele ja ma vist küsisin liiga kõrget 
palka ning sel hetkel lõppes meie diskussioon ära. Muidu tundus, et klikk oli. 

\question{Kas sind akadeemiasse teadust tegema ei tõmmanud?}

Seda, et oleksin hullult tahtnud teadlaseks saada, ei olnud. Pigem tõmbas mind ikka
praktiline programmeerimine ja häkkimine. 

\question{Sa pidid kiusatusele kõvasti vastu seisma, nii mõnigi libastus ja jäi koolist kõrvale. 
Aga sina suutsid keskenduda?}

Selle mänguga seoses tekkis aeg-ajalt ikka unistusi. Arutasime tiimiga, et võib-olla õnnestub sellest midagi kommertsiaalsemat 
välja arendada, 
aga ilmselt ei saanud me tervikpilti piisavalt hästi 
kokku, et see tegevus kuhugi välja 
viiks. 

Oma karjääri alustasin ikkagi palgatöölisena. Kõigepealt töötasin Tartu Ülikooli arvutuskeskuses ja
vahetult enne baka lõppu, 1998. aasta talvel läksin
ProMedi\index{ProMed}, mis just samal kevadel liitus Magnum 
Medicaliga\index{Magnum Medical}, millest hiljem sai Magnum ProMed. Seega
olen kaudselt olnud seotud ka viimasel ajal palju kõneainet pakkuva 
Margus Linnamäe\index[ppl]{Linnamäe, Margus} tegutsemise algusaegadega. 

Selle ühinemise käigus kogu ProMedi seltskond koondati kohe. 
Sain jube mugava paketi: tegin paar kuud sisulist tööd ja siis tuli 
koondamisteade, mis tolle aja reeglite järgi tähendas seda, et kui koondatakse 
rohkem kui 20 inimest, siis makstakse neile nelja kuu raha. Sain ilusti 
oma baka ära lõpetada niimoodi, et ei olnud vaja 
muretseda lauale leiva saamise pärast. Suvel läksin tööle
Medisofti\index{Medisoft}, mis tegeleb tänaseni raviasutuste 
infosüsteemide arendusega ja kus tol hetkel oli valdavaks arenduskeskkonnaks 
Borlandi Delphi\index{Borland Delphi}. See oli ülikooli algusaegadest 
tuttava Pascali keele põhjal, aga täiesti uus graafiline keskkond, kus oli hästi mõnus 
\emph{desktop}-rakendusi teha. Seal tegelesin niisiis
\emph{desktop}-rakenduste ja andmebaaside kokkupanekuga. 

Mul oli veel paar alternatiivi, kuhu tööle 
minna. Tagantjärele mõtlen, et tegin huvitava otsuse --- 
teistpidi otsustades oleks mu elu võibolla praegu teistsugune. Üks variant oli Medisoft ja teine see punt, kus toimetasid kursusekaaslased Rene Prillop\index[ppl]{Prillop, Rene} 
ja Mati Muts\index[ppl]{Muts, Mati}; seesama 
tuumik, kes hiljem Eesti-poolset PlayTechi\index{PlayTech} asutasid ja 
tegid. Nad pakkusid kõrgemat palka, aga ma kaldusin tollal
alalhoidlikkusele ja Medisofti pakkumine tundus parem, kuna seal olid kindlama meelega
vanemad tegijad ja stabiilsemad asjad. Läksin 
selle peale välja. Alguses oli hästi huvitav, sest õppisin
iga päev midagi uut --- mul on kogu aeg millegi uue saamise 
motivatsioon. Olin Medisoftis 1998. aasta suvest 1999. aasta suveni, 
kui mind kutsuti tööle Küberisse\index{Küber}. 

Küberis hakkasid juba naljakamad ja põnevamad 
lood, mis võibolla haakuvad tänapäevaga rohkem. Sattusin 
sinna täpselt sellel kuumal momendil, kui põhituumik --- Tarvi 
Martens\index[ppl]{Martens, Tarvi}, Aarne Ansper\index[ppl]{Ansper, Arne}, 
Viljar Tulit\index[ppl]{Tulit, Viljar} ja Monika Oit\index[ppl]{Oit, Monika} 
--- oli koos ja parasjagu visioneeriti DigiDoci ja muud põnevat. Krüptoteadlaste kambast olid seal Ahto 
Buldas\index[ppl]{Buldas, Ahto}, Helger Lipmaa\index[ppl]{Lipmaa, Helger} ja Jan 
Villemson\index[ppl]{Villemson, Jan}, samuti
Meelis Roos\index[ppl]{Roos, Meelis}, Peeter Laud\index[ppl]{Laud, 
Peeter} ja Ville Hallik\index[ppl]{Hallik, Ville}. Niisugune ajutrust, kõik tuntud korüfeed! 
Esimese hooga sattusin kohe C++-s\index{C++} programmeerima, 
Visual Studios DigiDoci kliendi prototüüpi. 

Teadlased leiutasid ajatemplite linkimise skeeme 
räsiahelate otsas ja samuti arendati digitaalse notari kontseptsiooni. Tollal me veel ei teadnud, et kümme aastat hiljem 
hakatakse seda nimetama \emph{blockchain}'iks. Iga natukese aja tagant tuleb keegi 
välja järjekordse teooriaga, kes on Satoshi Nakamoto\sidenote[][-3mm]{Bitcoini leiutaja või leiutajate poolt kasutatav pseudonüüm.}. Ka \emph{blockchain}'i puhul otsitakse ikka veel selle leiutajat. Paar aastat 
tagasi\sidenote{Jutuajamine Askoga toimus 2020. aasta jaanuaris.} tuli üks USA jurist 
lagedale teooriaga, et see oli Helger Lippmaa\index[ppl]{Lipmaa, 
Helger}\sidenote{Keegi Justin Sobaje 2018. aastal sellise teooriaga, 
mida Helger visalt eitas, tõesti välja tuli.}. Nii et meil on oma 
Satoshi Nakamoto olemas. Ahto Buldaselt\index[ppl]{Buldas, Ahto} võeti ka 
selle jutu peale intervjuu ja tema muigas ning ütles, et põhimõtteliselt oleksime me kõik 
võinud Satoshid olla. 

\question{Lähme korraks üheksakümnendate juurde tagasi. Kuidas sul arvuti ja muu 
maailma tasakaalustamine käis? Matemaatikateaduskonna ja arvutuskeskuse 
ümber käis ju äge seltsielu.}

Arvutuskeskuses istus ööde viisi seltskond koos 
ja toimetas oma asju. Üks püsiv ja stabiilne kaader olid 
Muda\index{Muda} mängijad. Osa inimesi \emph{chat}'is 
tolleaegsetes varajastes telnetipõhistes jutukates --- 
tänapäeval on meil nende asemel Facebook ja muud kohad. Ja mingi seltskond tegi asjalikumaid asju, näiteks õpingutega seotult. 

\question{Tehnikaülikooli arvutuskeskuse kohta on 
öeldud, et see oli nagu pooleldi klubi, kus käisid ka need, kes enam ammu ülikoolis 
ei õppinud.}

Jah, ja Tartu Ülikooli arvutuskeskuse kohta võis üheksakümnendatel 
enam-vähem sama öelda, sest seal ei olnud ainult üliõpilased. Sealt 
käis läbi igasugust rahvast ka väljastpoolt ülikooli ning oli tekkinud 
kamp, kus kõik tundsid enam-vähem kõiki, kes seal stabiilselt käisid. 

\question{Tartus oli vähemalt üks selline kogunemiskoht ka Tähetorn.}

Neid kohti oli isegi rohkem ja tuumikud puutusid omavahel 
kokku ka, aga kokkuvõttes oli mitu gravitatsioonikeskust. Tähetorn oli 
võibolla isegi üks suletumaid ja väiksemaid, ma ise sinna
eriti ei sattunud. Aga lisaks Tartu Ülikooli 
arvutuskeskusele\index{Tartu Ülikool!Arvutuskeskus} oli 
füüsikahoone\index{Tartu Ülikool!Füüsikahoone}, kus Ville 
Hallik\index[ppl]{Hallik, Ville} ja Otto Teller\index[ppl]{Teller, Otto} 
vägesid juhatasid ja asju kontrolli all hoidsid. Seal käisid
füüsikud ja ka sotsiaalteaduskonna 
tudengid. Veel üks
gravitatsioonikeskus oli ülikooli peahoone kõrval kunagine Marksu 
maja\index{Tartu Ülikool!Marksu maja}, mille all oli üks punkt, kus ma ka
ise mingil perioodil üpris tihti viibisin. Keemiahoone\index{Tartu 
Ülikool!Keemiahoone} all oli ka midagi, aga see toimis vist lühemat aega ja
oli natuke ebamäärasem. 

\question{Kui paljud arvutuskeskuse seltskonnast oli matemaatikud? Kui Anne Villems\index[ppl]{Villems, Anne} oma esimesed 
veebmasterite kursused tegi, käis seal kuuldavasti igasugust rahvast psühholoogidest 
usuteadlasteni.}

Teistest teaduskondadest tean ma üksikuid inimesi, nii et ma
statistilist pilti anda ei oska. Kuna arvutuskeskuse 
palgaline kaader oli valdavalt ikkagi 
matemaatika-informaatikateaduskonnaga tihedalt seotud, siis paratamatult oli 
selle teaduskonna tudengkond seal ka kõige rohkem esindatud. 
Praegu tuntud nimedest oli näiteks Jaanus Lillenberg\index[ppl]{Lillenberg, 
Jaanus}, kes praegu juhib ERRis IT-vägesid, mingil perioodil seal hästi 
aktiivne külaline\sidenote{Jaanuse seiklustest Liivi tänaval loe lähemalt 
lk\pageref{sisu!jaanus_liivi_tn}.}. Tema teaduskondlik taust oli vist
midagi muud, ta ei olnud IT-vallast. 

\question{Mida sa praegu teed?}
Praegu olen ettevõtja. 

\question{Vist juba üle kümne aasta?}

Selgelt ettevõtjaks sain ma pärast Skype'i\index{Skype}. Enne 
seda olid aeg-ajalt mingisugused unistused ja visioonid ning
projektilaadsed eksperimendid, aga esimese OÜ, Mooncascade'i\index{Mooncascade} asutasin oma väriseva 
käega paar kuud enne Skype'ist lahkumist, 2007. aasta septembris. 
Sellega olengi kõige rohkem seotud olnud. 

Mooncascade'i visioon tekkis mul juba Skypest lahkudes, aga 
eestlastest Skype'i asutajate punt, Ambient Sound 
Investmenti seltskond, kutsus mind kohe jutule, sest neil oli 
käsil üks inkubaatorilaadne eksperiment, kus jooksis neli 
erinevat projekti. Nad kutsusid mind ühte nendest projektidest vedama 
ja kuna pundis oli ka Ahti Heinla\index[ppl]{Heinla, Ahti}, kes 
täna veab Starshipi, siis minu jaoks oli piisavalt motiveeriv panna Mooncascade seniks riiulile. Nii me tegimegi paar aastat 
ühte suhteliselt jõhkrat andmekaevelaadset projekti, mis äriliselt 
lõpuks ikkagi lendu ei läinud ja sai ära konserveeritud. Seejärel sai Mooncascade 
reaktiveeritud. Mooncascade alustas aktiivset tegutsemist 2009. aasta lõpus või 2010. aasta 
alguses ja tegutseb tänaseni. 


\chapter{Margus Sutt}
\index[ppl]{Sutt, Margus}

\question{Kust sa pärit oled?}
Ma olen Tallinnast.

\question{Kuidas arvutid 
sinu juurde said ja sina arvutite juurde?}

Kui mul oli aeg kooli minna, siis 
meie pere parajasti kolis sellise kooli nagu Tallinna 3. 
Keskkool\index{Tallinna 3. Keskkool}, mis praegu kannab nime
Lilleküla gümnaasium\index{Lilleküla gümnaasium|see{Tallinna 3. 
Keskkool}}, piirkonda. Ja selles koolis oli 
selline tüüp nagu Jaak Loonde\index[ppl]{Loonde, Jaak}.

\question{Jälle Jaak!}

Jah. Põhikooli esimese kaheksa aasta jooksul ei 
juhtunud veel midagi, aga keskkooli minnes toimusid 
katsed ja konkursid ning meil tehti juurde kolmas klass, 
nii-öelda informaatika-matemaatika eriklass. Aasta oli siis 1985 või 1986.

\question{Kas seda klassi vedas 
Jack\index[ppl]{Jack}?}

Jah, tema organiseeris selle klassi.

\question{Kas ta oli klassijuhataja ka?}

Ei, klassijuhataja ta ei olnud, ta ajas asju haridusministeeriumiga või 
mis iganes asutus see siis veel Vene ajal oli, ja igatahes sai 
selle klassi. Kaua see klass vist ei püsinud, kaks 
komplekti oli seda kindlasti, aga kolmandas ma kindel ei ole, kuna 
lõpetasin juba kooli. Jack ei pidanud seal koolis kaua vastu ja peale 
teda see asi hääbus. 

Jackil oli koolis selline tore külmkastide rivi nagu 
MIR-2\index{MIR-2}, kus olid perfolindid ja -kaardid.

\question{Kas see oli tal lausa koolis?}

See oli tema klassiruumis, ruume oli tal muidugi mitu. Lisaks 
klassiruumile ka algeline raadioruum, millest sai hiljem päris raadioruum. 
Meie klassi tegelased ehitasid selle, mitte küll mina, aga põhiaktivist oli Andrus Tamboom\index[ppl]{Tamboom, Andrus}. Tema ehitas kooli 
raadiovõrgu või pigem taastas, sest juhtmed olid vanast 
ajast seinas, aga ta kohendas seda ja ajas kuidagi käima.

\question{Kuidas perfolindi tingimustes arvutiõpe praktiliselt välja nägi?}

Ega õpet väga palju ei olnud. Võibolla pool aastat tegelesime
MIR-2ga\index{MIR-2} ja edasi juba natuke 
keerulisemate asjadega. Ekraani peal sai mingisugust parabooli
joonistatud. Perfolindiga oli nii, et oli lugeja, mille sisse tuli lint pista
ja kuskil mingit käsku vajutada ning see hakkas linti hästi 
kiiresti läbi vedama.

\question{Kas kooliõpilased perforeerisid linti või see usaldati kellegi teise 
kätte?}

Arvuti perforeeris ise ka, sellel olid nii sisend kui ka väljund olemas.

\question{Nii et nagu trükimasin --- toksid sisse ja arvuti teeb perfolindi?}

Ja pärast on võimeline seda lugema. 

\question{Kuidas sul see mõte sündis, et võiks just sellesse klassi
minna, kus informaatikat õpetati?}

Ma olen selle peale mõelnud, aga ei mäleta, kuidas. Võibolla vanemad 
torkisid tagant või oli endal mingil määral huvi. Matemaatikaga mul probleeme 
ei olnud ja käisin isegi olümpiaadidel.

\question{Tavaliselt on arvutiõppes kaks osa: see, mis tunnis räägitakse, 
ja programmiväline tegevus, mille käigus ise pusitakse. Kuidas teil 
see vahekord oli?}

Selles mõttes tuleb jälle Jacki\index[ppl]{Jack} tänada. Tema õpetamismeetod 
oli klassikalisest kardinaalselt erinev. Matemaatikat 
ta raamatust ei õpetanud, vaid tal olid enda tehtud töölehed, kus oli 
muu hulgas üritatud tekitada ka siirdeid teistesse ainetesse. Füüsika oli 
kõige lihtsam, keemiat ma ei mäleta, aga eesti keelt oli ka
kuskil mainitud. 

\question{Nii et ta õpetas matemaatikat ka?}

Jah, meie klassile ta õpetas matemaatikat ka, järgmisele aastakäigule enam mitte. 

\question{Kas töölehed ei sobinud?}

Seal oli igasuguseid probleeme. Järgmist aastakäiku mainin sellepärast, et 
seal õppisid Priit Kasesalu\index[ppl]{Kasesalu, Priit}, Mikk 
Orglaan\index[ppl]{Orglaan, Mikk} ja Janno 
Ossaar\index[ppl]{Ossaar, Janno}. 

Jaagu\index[ppl]{Loonde, Jaak} õpetamismeetod oli suures osas vetteviskamine, vähemalt 
arvuti poolel. Et näete, siin on arvuti, tegelege. Muu hulgas õnnestus tal 
näiteks koos klassijuhataja Tiiu Neemega\index[ppl]{Neeme, Tiiu} viimases 
klassis minule ja veel paarile klassikaaslasele välja rääkida selline eriprogramm, et me osas 
tundides ei pidanudki käima, tegime ainult semestrite või trimeistrite 
lõpus töid ja arvestusi.

\question{Et sellist eriprogrammi võimaldada, pidi akadeemiline 
jõudlus tasemel olema?}

Sellega ilmselt ei olnud probleemi jah. 

Arvutitest puutusin kõigepealt kokku MSXidega\index{Yamaha 
MSX} Luise tänava ÕTKs\index{Tallinna Oktoobrirajooni 
Õppetootmiskombinaat}\sidenote{Tallinna Oktoobrirajooni Õppetootmiskombinaat.}. 
Algul käisime seal klassiga ja pärast käisin juba ise. 
MSXidega oli tore see, et Jackil\index[ppl]{Jack} õnnestus need 
vaheaegadeks 3. keskkooli laenata. Tänu sellele oli mul hulga parem ligipääs arvutitele, sest mul oli lisaks Jacki ruumi võtmele ka kooli välisvõti. Nii et
keskkooliajal tekkis võimalus suvevaheajal ööpäev läbi arvutis olla.

\question{Mida sa arvutiga tegid? Teiste juttu kuulates tundub olema kahte 
liiki inimesi: need, keda huvitas rohkem programmeerimine, ja need, keda tõmbas 
mängude poole. Kumb sul rohkem domineeris?}

Esialgu kindlasti mängimine. Yamahal\index{Yamaha MSX} olid head ilusad mängud, eks need meelitasid 
oma graafikaga ja sellega, et võrreldes MIRiga oli see ikka hoopis teine maailm. 
Isegi võrreldes Agatiga\index{Agat}. Keskkooli 
esimesel või teisel aastal tekkis meile üks Agat oma kooli\index{Tallinna 3. 
Keskkool} ka. 

Progemise mõttes oli mul Agatist kindlasti rohkem kasu, sest seal käis 
Tarmo Mamers\index[ppl]{Mamers, Tarmo} vahel asja 
uurimas ja üle tema õla kiibitsedes õnnestus mul ka üht-teist-kolmandat 
omandada.

Agati peal oli Basic\index{BASIC}, aga muu hulgas sai ka juba assemblerit 
vaadatud.

\question{Kas Agat oli Apple II kloon?}

See oli Apple II\index{Apple II} kloon, mille sees oli originaal-Apple'i \emph{chip}, kust oli info maha kraabitud. Vähemalt sellel konkreetsel Agatil oli 
nii.

\question{Mina mäletan Agatil veidrusi, mida 
Apple'il vaevalt et oli. Ilmselt see OS oli Nõukogude oma?}

Ma ei mäleta nii täpselt, sest ma pole päris Apple'it puutunud ega oska võrrelda. Tean, et Tarmol\index[ppl]{Mamers, Tarmo} olid Apple'ist 
väljatrükid BIOSi ja selle \emph{call}'ide kohta, ja ta üritas seda 
Agati peal rakendada.

\question{See on juba oluline teadmine, et Apple'i \emph{call}'id võiksid 
töötada!}

Jah, muu hulgas oli mul näiteks selline huviprojekt, et üritasin
Apple'i peale Norton Commanderit\index{Norton Commander} kirjutada. 
Assembleris, loomulikult.

\question{Huvitav, minul oli samasugune projekt Juku peal.}

See tee on vist jah aastate jooksul kõigil läbi käidud. 

\question{Kust teile ülesanded tulid? 
Liiga lihtne ülesanne ei ole huvitav, aga liiga keerulise ülesande puhul jooksed end algajana sodiks, enne kui lootusekiir paistma hakkab. 
Kas teid juhendas Jack või mõtlesid ise välja?}

Ei, Jack\index[ppl]{Jack} selles mõttes eriti ei juhendanud. Tema antud 
programmid olid pigem matemaatikateemalised. Ja ega Jack ilmselt ei 
küündinudki sellise juhendamiseni või kui küündis, siis
ei pidanud seda vajalikuks. Tuli ise vaadata, mida teed. 

Mängude puhul püüdsin spraite või muid selliseid asju 
liigutada. Mäletan veel ka, et võtsin 
sinusoidi ja nii-öelda nihutasin selle ruumi. Kui tekitad mõlema koordinaadi 
suhtes väikse nihke, siis tekib ruumiline efekt. Ja kui 
paned programmi tagant kustutama ja eest uuesti joonistama, on tulemuseks 
tänapäevane \emph{screensaver}. Tol ajal oli see huvitav.

\question{Kas arvutihuvi juurde käis ka 
spetsiifiline muusika- või kirjandushuvi? Päris mitmed on rääkinud, 
kuidas nad on Asimovi ja Gibsoni peal üles kasvanud.}

Ma olen lugenud küll fantastikat ja \emph{sci-fi}'d, aga ma ei 
tea, kui palju see nüüd arvutiga seotud on. Pigem on need
konfliktis, sest võistlevad aja pärast. Enne arvutiaega lugesin 
väga palju ja kiiresti, näiteks \enquote{Seiklusjutte maalt ja merelt}. Lugesin kõike, mida kätte 
sai. Aga kui arvuti tuli, siis ma enam nii palju ei 
lugenud. Umbes kümme aastat tagasi võtsin kätte ja lugesin 
läbi enamiku McCaffreyst\sidenote{Margus peab ilmselt silmas ameerika-iiri 
ulmekirjanikku Anne McCaffreyt. Ainuüksi tema loheratsurite sarjas on 23 
romaani ja see on vaid üks paljudest selle kirjaniku loodud 
maailmadest.}, alustades lohelugudest, neist kaks on isegi eesti keeles olemas.

\question{Mõni on rääkinud, et juba keskkooliajal kippus töötegemiseks 
minema. Kas sul ei olnud nii?}

Ei, töötegemiseks ei läinud, kuigi sidemed esimese töökohaga juba tekkisid.

\question{Mis su esimene töökoht oli?}

Seda töökohta iseloomustab kõige paremini Raivo Rebase\index[ppl]{Rebane, Raivo} 
nimi. Ma ei tea, kuidas ta Jackiga oli seotud, aga kuidagi igatahes oli. 
Ühel hetkel oli ta Jacki juures ja põhimõtteliselt otsis jüngreid ehk tegeles 
\emph{head-hunting}'uga.

\question{Vaat kus! Tänapäeval keskkoolist vist enam ei käida otsimas!}

Ta vist plaanis oma arvutifirmat teha, tol 
hetkel tal seda veel ei olnud. Ta tegutses Küberis\index{Küber} sellise 
härrasmehe nagu Raul-Roman Tavasti\index[ppl]{Tavast, Raul-Roman} juures. Neil 
oli vist mingi firma Küberi kõrval. Mäletan seda sellepärast, et seal sain tõenäoliselt esimest korda PCd katsuda. Pärast värbamist saime koolist paari-kolmekesi hakata käima 
kuskil Küberi majas, kus olid PCd. Vist 
lausa 386d, mille mälu oli neli mega. Kui tegin 
\emph{boot floppy}, kus oli mäludraiv peal, siis sain väga palju 
kõvakettaruumi, kus jooksutada Turbo C \emph{editor}'i.

\question{Turbo C on ikka juba meeste vahend, kuidas sa 
selleni jõudsid?}

Ma ei mäleta, kuidas ma C\index{C} juurde 
jõudsin, sest Pascalit ei ole ma kunagi õppinud, aga kuidagi mul tekkis 
Turbo C. Agati peal seda ei saanud olla, ilmselt tuli Rebase kaudu. 

Rebane pistis mulle pihku Kernighan-Ritchie\sidenote{Kuulus valgete kaante ja 
sinise Cga raamat, vt lk \pageref{sisu:richie}.} koopia, mille ma neelasin 
mõne nädalaga läbi.

\question{Kas MSXilt ja Agatilt C peale üle minek keeruline ei olnud?}

Ma ei mäleta, kuidas see täpselt oli. Seal Rebase juures ma käisin ja ühel
hetkel lõi ta Küberist lahku. Järgmine koht oli Liivalaia tänaval praeguses
Swedbanki majas.\sidenote{Liivalaia 8, Tallinn.} Seal oli 12. korrusel, kus 
praegu on nii-öelda ülemuste korrus, arvutuskeskus.\sidenote{Margus peab ilmselt silmas sel aadressil asunud EKE 
Projekti nimelist asutust.}. Seda vaadet ma nautisin üheksakümnendatel. Pärast õnnestus peaaegu samasse kohta tagasi kolida, Rebane sai sinna üheksandele 
korrusele ruumid. 

Arvutuskeskuses oli ka üks suur kastarvuti, millega meie õnneks kokku ei puutunud. Meil olid oma 
PCd, kus hakkasime muu hulgas tegelema ka Unixiga\index{Unix}.

\question{Kuidas te PC peal Unixit tegite?}

SCO\index{SCO UNIX} ja BSD\index{BSD} olid sel ajal olemas.

\question{Kuidas need tol ajal Eesti Vabariiki jõudsid?}

Ma ei tea, võibolla olid kuskilt ostetud. SCO-l olid originaalkirjadega plaadid. Või siis ikka flopid, 
plaadid tulid hiljem. 

\question{Millega te seal tegelesite? Firma pidi ju äri tegema?}

Äriga oli alguses kehvasti, näiteks ma ei mäleta, millal ma palka 
hakkasin saama. Ühel hetkel aga läks väga huvitav radariäri käima. Tegime koostööd Vene firmadega, 
ühe seltskonnaga Peterburist. Nemad tegid radarile riistvara ja kaardi, 
mis läks PC sisse, ja meie ehitasime sinna peale softi.

\question{Radarid on ju sõjaväega seotud ja salajane värk?}

Ei, sõjaväega meil kokkupuudet ei olnud, tegime tsiviilsuuna jaoks. Pulkovo 
radarijaamas sai näiteks korduvalt käidud.

\question{Kas teie soft võttis radarilt signaali ja 
joonistas kaardile mummud?}

Jah. See käis BSD peal, 
võibolla alguses isegi SCO peal, ja 
lõpuks läksime Linuxi\index{Linux} peale, kui see tekkis ja oli juba niivõrd 
kobe, et seda sai kasutada.

\question{Alles joonistasid ekraanile siinust ja siis juba lugesid radarilt signaale --- siin tundub lünk olema. Ülesande keerukus on ju 
palju suurem!}

Ega mina seda kõike ei teinud. Selle taga oli terve grupp inimesi.

\question{Kui suur see grupp oli?}

Ei olnud väga suur, alla kümne. Meie firma ei läinud väga suureks. 
Seal olime mina, Raivo Rebane\index[ppl]{Rebane, Raivo} ja Mart 
Rüütel\index[ppl]{Rüütel, Mart}, kes on vist praegu ka veel seal. Firma nimi 
on nüüd R-Süsteemid\index{R-Süsteemid}, mõnda aeg oli Virumaa 
Tiivad\index{Virumaa Tiivad|see{R-Süsteemid}}, sest tegeles natuke ka 
lennundusega. 

Vahepeal oli meil tööl üks lennundusfänn ja muu hulgas sai 
korraldatud väikelennukite ülelennu üritus, mille raames mul 
õnnestus näiteks sõita mingi Piperiga Kuressaare lennujaamast Viljandi lennujaama. Kuna need, kelle lennukiga ma lendasin, olid Saksamaalt, siis 
ega nemad ei teadnud, kus lennujaam täpselt on, ja Viljandi 
lennujaama otsimisega oli natukene tegemist. Et kuhu me nüüd siis 
maandume?

\question{Kes klient oli? Needsamad venelased?}

See soft oli vist mõnda aega ka Tallinnas kasutusel. Softis olid mõlemad, nii 
primaar- kui ka sekundaarradar. Primaarradar on see nii-öelda loll radar, 
mille signaal põrkab lihtsalt kuskilt tagasi või siis ei põrka. Ja 
sekundaarradar on see info, mille lennuk ise välja saadab oma 
transponderiga.

\question{Kuidas te radari sofit töökindluse tagasite?}

Sellel oli mitu kihti loogikat peal. Radaritel sai isegi 
vahvaid maatriks-algoritme kasutatud. Üks tegelane, kelle nime ma 
paraku ei mäleta (keegi Kaido?), tegi nendest lausa TPI 
lõputöö. Nagu lõputöödega ikka vahel on, pannakse kuskile mõni
\emph{catch} sisse lootuses, et niikuinii keegi ei loe. Nende algoritmidega 
juhtus nii, et see koht koodis, kus seda kõike pidi 
välja kutsutama, oli pikka aega välja kommenteeritud. Pärast tema järeltulijad 
avastasid, et oleks hulga parem, kui seda funktsionaalsust ka
kasutataks.

\question{Radari riistvaraga suhtlemine oli ilmselt keeruline?}

Riistvara suhtlemisel oli mängus ajakriitilisus. Tänu sellele tuli paratamatult kerneli alasse ronida, kuna signaal liigub kiiresti ja sul on vaja täpselt 
teada, millal signaal sisse tuli. Sealt omatehtud kaardilt tuli 
puhvrid (mis ei olnud küll tol ajal suured) kähku ära lugeda.

\question{Kas see oli SCO ajal?}

See algas SCO\index{SCO UNIX} ajal ja meil olid selle jaoks
ametlikud raamatud ja ametlik \emph{dev kit} ostetud.

\question{Neid inimesi, kes seda oskasid, ei saanud Eestis palju olla.}

Jah, ega kellegi käest küsida väga ei olnud. Esimene inimene, kellele see asi 
huvi pakkus ja kes midagi teemast teadis ning suutis kaasa rääkida, oli 
aastaid hiljem (kui ma 1993. aastal Tartusse jõudsin) Meelis 
Roos\index[ppl]{Roos, Meelis}. Ta on minust muidugi kõvasti ette jõudnud, 
sest ta on praeguseks päris palju kerneli \emph{patch}'e \emph{post}'inud\sidenote{Linuxi tuuma, ehk kerneli, arendajad on oma tegevuse suure mõju ning kõrge keerukuse tõttu programmeerijate hulgas kõrgelt hinnatud. \emph{Patch}'i ehk paiga postitamine tähendab, et paiga autor ei ole nende töö tulemuses mitte ainult probleemi leidnud vaid suutnud ka probleemi lahendada.}. 

\question{SCO kohta olid teil raamatud, aga kuidas oli muu infoga? Räägi BBSidest, 
Tarmo Mamers juba jooksis korraks jutust läbi.}

Jah, Tarmo kõrval sai kogemust koguda ja koos 
pitsat süüa, Peetri Pizzas, sest tol ajal Tallinnas väga valikut ei 
olnud, see oli enam-vähem ainus pitsakoht. 

\question{Kas see kohtusite Skriiningu kontoris?}

Jah, Skriiningus\index{Skriining} olin ma sage külaline. Teine BBS, kus ma tihti 
külas käisin, oli Dark Corner\index{Dark Corner}.

\question{Tol ajal oli arvutifirma vist natuke klubi moodi asi. Sõltub 
firmast muidugi, aga kogu aeg käisid inimesed läbi, jõid kohvi ja vahetasid 
uudiseid.}

Tundus vist olevat küll jah. Vähemalt osas kohtades, ka Skriiningus, oli 
niimoodi.

\question{Millised \emph{hot-spot}'id peale Skriiningu veel olid?}

Mina väga palju mujal ei käinud. Microlink oli vist ka millalgi selline koht. 
Dark Corner BBSis --- Priit Kasesalu\index[ppl]{Kasesalu, Priit}, Ahti 
Heinla\index[ppl]{Heinla, Ahti} ja kes seal teised olid --- sai 
ainult õhtuti ja öösiti käidud, sest neil oli päevatöö ka.
Tarmo juures oli sotsiaalne osa vist pigem ikka ka
õhtupoolikutel, päevasel ajal ei olnud nii palju 
läbikäimist.

\question{See oli see aeg, kui pidi hakkama tööd tegema ja raha teenima. 
Ometi panid sa oma BBSi püsti. Ühel hetkel olid sa MamBoxi \emph{point} 
ja siis tegid enda oma?}

Jah, MamBoxi\index{MamBox} point olin mõnda aega ja siis sai 
firma abiga enda oma tekitatud. Isiklik see ei olnud, ikkagi firma 
riistvara peal ja firma kontoris. Kiirust väga ei olnud, alguses vist 
1200, hiljem 2400. 

\question{Mis selle nimi oli?}

Boksi nimi oli Flying Discs BBS. 

\question{Väga lennukas! Aga miks sa selle tegid?}

Tundus põnev. Eks sealt sai mänge ja muusikat tõmbasin ka, minu MP3ndus 
sai sel ajal alguse.

\question{1200 modemiga MP3 allatõmbamine võtab ju kaua aega!}

Selles mõttes oligi parem pitsa kaasa võtta ja külla minna, efektiivsem kui helistada. Millalgi saime sellise modemi nagu 
Zyxel\index{Zyxel}, mis suutis rohkem välja vilistada, aga vist 
ainult teise Zyxeliga. Siis tuli otsida lähikonnast Skandinaaviast 
kohti, kuhu sai niimoodi helistada.

\question{Nii et siis sai juba kaugekõnet teha?}

Sai jah, mina ei pidanud elama üle seda aega, kus pidi tädile 
telefonis kõigepealt ütlema, et ühendage mind sinna ja tänna. Minul oli EKE 
Projekti arvutuskeskuses välisliin algusest peale olemas. 

\question{Kuidas sa 1993. aastal Tartusse sattusid?}

Sellega oli nii ja naa. Mõnes mõttes oli see mugavustsoonist väljaminek. Kui ma 1989. aastal keskkooli lõpetasin, käisin aasta 
TPIs. Eriala oli vist LI\index{Tallinna Tehnikaülikool!Automaatikateaduskond!LI} ehk arvutid ja 
arvutivõrgud. Pidasin vastu aasta, sest see ei andnud mulle mitte 
midagi. Arvutiaine eksam 
või arvestus tuli teha Pascalis. Kuigi Pascalit\index{Pascal} polnud ma
kunagi õppinud, tegin selle töö esimese kuu lõpuks ära, esitasin õppejõule ja 
rohkem kohal ei käinud. Partei 
ajalugu ei pidanud küll enam õppima, aga see-eest oli füüsika, kus olid 
Rusalepad\sidenote[][-5.4cm]{Ilmselt peab Margus silmas Ervin ja Maret 
Rusaleppa\index[ppl]{Rusalep, Maret}\index[ppl]{Rusalep, Ervin}.} kahekesi 
vastas, ja sellest ma vist kukkusingi läbi.

\question{Misjärel jõudsid Tartusse?}

Olin kolm aastat nii-öelda tööl: tegelesin igasuguste põnevate või 
vähem põnevate asjadega või mängisin arvutiga. UNIXi peal avastasin enda 
jaoks mängu, millest ma ei ole siiamaani lahti saanud --- 
Rogue\index{Rogue}\index{Nethack}\sidenote[][-7cm]{Esimene mäng, kus tuli tekstipõhisel ekraanil programmi poolt genereeritud 
koobastikest koosnevas fantaasiamaailmas seiklusi otsida. Peategelase surm 
oli seejuures permanentne ja mängija võis komistada ka oma varasemate 
tegelaste laipadele. Hiljem nimetatigi sedalaadi mänge (polulaarsemad Hack, 
Nethack, Moria, Angband) ühise nimetajaga \emph{roguelike} ehk 
\emph{rogue}'i-sarnased.} või Nethack\sidenote[][-3.6cm]{Vt ka lk 
\pageref{sisu:nethack}.}.

\question{Mis versiooni\sidenote[][-3.5cm]{Nethack on pigem kultuuriline 
fenomen kui arvutimäng, ka selle lähtekood ja andmefailid on mõnuga loetavad, 
sisaldades viiteid algmüütidele, tsitaate, luulet jne. Ühest küljest tähendab 
see pidevat arengut, kuid teisalt ka seda, et mängijad peavad vaid üht 
konkreetset versiooni selleks \enquote{õigeks}, täpselt nii, nagu suhtutakse 
vahel skepsisega uuematesse \enquote{Star Warsi} filmidesse.} sa mängid?}

Pean tunnistama, et mängin seda ka viimasel ajal päris palju. Võtsin 
eesmärgiks kõikide rollidega lõpuni mängida. Üks roll on veel jäänud, 
\emph{priest}\sidenote[][-1cm]{Nethackis on 13 rolli, neist igaühel unikaalsed 
võimed ja vaenlased. Juba ühe rolliga mängu läbimine on keeruline ettevõtmine, 
teha mäng läbi kõigi rollidega peale preestri (mis tähendab, et Margus on 
kõikvõimalikele ohtudele vastu astunud ka näiteks turisti rollis, kelle peamine 
võimekus on vastupanu mürkidele) on Nethacki austajate hulgas üsna eepilistes 
mõõtmetes saavutus.}.

Tegelikult on temaga ju lihtne: \emph{priest} näeb kohe ära, kas asja 
saab selga panna või mitte ehk kas see on \emph{blessed} või \emph{cursed}. 
Alguses tundub lihtne, aga võibolla see lihtsus maksabki kätte. Mul on 
temaga olnud ka väga pikki mänge, aga lõpuni ei ole veel jõudnud. Mängin 
viimast versiooni, samas ma kõiki neid kahe käe relvi ja muid 
uuendusi ei kasuta.

\question{Miks Tartu ja matemaatika?}

Seal olime ju koos sinuga ja igasuguste teiste huvitavate tegelastega, nagu Meelis 
Roos\index[ppl]{Roos, Meelis}\sidenote{Meelise lugu algab lk
\pageref{sisu:mroos}.} ja Asko Seeba\index[ppl]{Seeba, 
Asko}\sidenote{Asko lugu algab lk \pageref{sisu:asko}.}, ja meie 
gasell\sidenote{Margus viitab Asko juhitava firma Mooncascade'i saavutustele 
Äripäeva koostatavas gasellettevõtete pingereas.}. Ülo 
Kaasik\index[ppl]{Kaasik, Ülo} on ka tuntud nimi tänapäeval, kuigi mitte küll 
arvutimaailmas.

\question{Põnev seltskond oli tõesti. Aga ikkagi miks just Tartu ja 
matemaatika?}

Üks koolivend keskkoolist, kellega me hästi klappisime ja tänapäevalgi läbi 
käime, oli kohe pärast kooli läinud Tartusse rakendusmatemaatikat õppima. Kui ta oleks 
kaasa kutsunud, võibolla oleksin läinud. Pärast oleme sellest 
rääkinud, et ehk oleks parem olnud, kui oleksin kohe läinud. 
Aga selliseid asju ei tea ette. Valisin informaatika\sidenote{Matemaatikateaduskonnas sai baasteadmiste omandamise järel spetsialiseeruda informaatikale.}, kuna mulle on arvutiasi 
südamelähedane, ja mõtlesin, et äkki sealt koolist saab midagi rohkemat. 

\question{Kas sai?}

Aeg oli edasi läinud ja sealt ikka üht-teist juba sai, kuigi selle 
kooliga ma ka lõpuni ei jõudnud. See oli pikk protsess. Esimesed kolm aastat 
läksid ilusti, olin kõigi asjadega graafikus, aga siis sai raha otsa. Võtsin 
aasta akadeemilist ja tegin tööd, millest märgatava osa ajast olin 
Venemaal Peterburis. Tegime sealsete inimestega koostööd sellesama radari 
teemal. R-Süsteemidel oli teine suund merelokaatorid. Need on selles 
mõttes sarnased, et erilist vahet ei ole, kas kajalokatsioon on õhus või vees, kuigi merepõhja läheb üldiselt keerulisem signaal, kanaleid 
on rohkem.

\question{Kas sa Tartus pidasid BBSi edasi või oli see pausi peal?}

BBSindus jäi Tartus katki. Tartusse minek oli igatpidi
mugavustsoonist väljaminek. Helistamine jäi ära, kuigi järele 
mõeldes oli internetiühendus meil ka alguses ikkagi niimoodi, et tuli 
helistada. Pikka aega oli ka üks telefon kogu aeg modemi 
taga kinni, alguses BBSi taga ja pärast internetis. Püsiühendused tulid 
hiljem.

\question{Kas Tartu Ülikool\index{Tartu Ülikool!Matemaatikateaduskond} sind 
akadeemilisse maailma ei tõmmanud? Neid näiteid oli meie kursuselt ka.}

Ei, väga ei tõmmanud. Kursatöö juhendajaga käisin küll korra ühel 
välisreisil kaasas, eks ta üritas mind sellega meelitada. Norras 
Bergenis oli konverents, kus ta pidi oma pabereid esitama. Ta võttis mu kaasa, sain selle 
eest ilmselt ka mõned akadeemilise maailma \emph{exp}'i punktid, et tema rääkis ja mina vajutasin 
arvutiklahve. Tol ajal ei olnud pulte ega muid vahvaid asju. 

\question{Aga see ei tõmmanud sind?}

Ei tõmmanud. Igasugust vahvat riistvara, 
nagu SUN, oli seal küll \ldots 

\question{Tartu Ülikool oli tollal vist päris hästi varustatud?}

Jah. Ühikas\sidenote{Tartu Ülikooli Tiigi tänava ühiselamu\index{Tartu 
Ülikool!Tiigi ühikas}.} me ju alustasime ise internetiga. Asko 
Tiiduma\index[ppl]{Tiidumaa, Asko} oli nii-öelda ühika sysop\sidenote{Asko mäletab, 
et ta oli küll idee tekkimise juures ja võttis hiljem sysopi rolli üle, 
kuid ühika interneti ehitas siiski Aldo Mett\index[ppl]{Mett, Aldo}.}, sest tema 
toas asus sisendpurk. Katusel oli antenn ja tema elas seal antenni 
all. Ega ühikas kaabli vedamine ei olnud lihtne! Kui vales kohas puurid, tuleb terve telliskivi välja!

\question{Lõpetuseks, kuhu see tee on sind tänaseks toonud?}

Tänaseks, või pigem viimaseks peaaegu kahekümneks aastaks, olen maandunud 
pangandusse ja muidugi IT-valdkonda. Kunagi oli see Hansapank, 
nüüd Swedbank. Hansasse tulin 2000. või 2001. aastal ja siin ma nüüd olen.

\question{Panganduses on ju ülesanded hoopis teist 
masti kui radarisignaali lugemine. Millest selline muutus?}

Oli muutus küll ja ega see ei tulnud kergelt. R-Süsteemides oli tore 
seltskond. Meil oli näiteks suviti kombeks arvutitega 
mere äärde minna. Võrk pandi kuskil mujal püsti, internet võeti näiteks kohaliku kooli juurest ja siis sai vaheldumisi tööd teha ja rannas käia . Aga palgaga oli kehvasti, nii et olid majanduslikud probleemid. Töö oli üldiselt 
huvitav, aga kui on väga väike firma, siis tellimusi ei ole kogu aeg. Ma ei 
ole ise müügiinimene, et läheksin ja otsiksin tänavalt tööd. Ja kuna seal 
kippus see pool natuke lonkama, siis ma mõnes mõttes läksin 
lihtsama vastupanu teed. 

Olen vahepeal päris palju baasi 
kirjutanud. Kui panka tööle tulin, siis vestlusel rääkisin, 
et baasi ma ei ole kirjutanud ja väga ei taha ka. Vahepeal hakkas 
see mulle täitsa meeldima, aga nüüd vaatan, et tuul on jälle sinnapoole, et \emph{micro-service}'i maailmas on baas jälle väga \emph{evil}. 

\question{See on vist pikalt arvutitega toimetamise hüve, et näed neid 
tsükleid ja ringe.}

Jah, eks vahepeal ikka kaldutakse äärmustesse ja tõde on seal kuskil 
keskel.

\chapter{Jaan Tallinn}
\index[ppl]{Tallinn, Jaan}

\question{Kuidas ja umbes millal sa jõudsid arvutite juurde?}

Ma mäletan  seda aega, kuskohas mul isa hakkas kaheksakümnendatel Soome vahet 
käima, seal mingisuguseid
filmi ja videorežiitöid tegemas. Ja ta tõi mulle erinevaid ajakirju, neid oli 
hea, odav ja võib-olla isegi tasuta tuua. Neist päris mitmed olid 
arvutiajakirjad ja see tundus kohe olema väga põnev. Armumine esimesest 
pilgust. Algkooli kas  viimases või eelviimases klassis juhtus selline asi, et 
üks kooli lapsevanem valis mind ja mõningaid mu klassivendi (sealhulgas näiteks 
Priit Kasesalu\index[ppl]{Kasesalu, Priit}) eksperimentaal-katsejänesteks, et 
viia  meid õhtuti  kuskil Kopli servas asuvasse Sideministeeriumi 
Arvutuskeskusesse\index{Sideministeeriumi Info- ja Arvutuskeskus} ja lasta seal 
suurte \emph{mainframe}-de peal lahti ja vaadata, mis juhtub. Nii et inimkatse 
tulemus. 

\question{Mis kool see oli?}

See oli Lasnamäel kuuekümnes keskkool\index{Tallinna 60. Keskkool}.

\question{Kust selline mõte tuli, et peaks inimestega niimoodi tegema?}

Seda ma ei tea, aga sa võid ta enda käest küsida. Ta nimi on Jüri 
Malsub\index[ppl]{Malsub, Jüri}, talle meeldib sellest väga pikalt rääkida. 
Seal seltskonnas olin mina, Priit Kasesalu ja veel kaks klassivenda, kelles 
ühest (Mikk Orglaan\index[ppl]{Orglaan, Mikk}) sai ka arvuti-ettevõtja. Neljas 
oli Martin Kruusvall\index[ppl]{Kruusvall, Martin}, kellele sai selgeks, et 
numbrid teda väga ei paelu, et ta on rohkem nagu luuletaja tüüp.

Keskkoolis liitus selle seltskonnaga Ahti Heinla\index[ppl]{Heinla, Ahti}. Siis 
ma olin juba läinud Tallinnas Gustav Adolfi Gümnaasiumi, toona esimesse 
keskkooli\index{Tallinna 1. Keskkool} ja hakanud tõsiselt tegema 
olümpiaadidega. Meie füüsikaõpetaja, kadunud Vilma Kukrus\index[ppl]{Kukrus, 
Vilma} ühel hetkel (peale seda, kui Ahti oli vabariikliku füüsika olümpiaadi 
kinni pannud) rääkis Ahti pehmeks, et mis sa seal Õismäel passid, tule parem 
Gustav Adolfisse. Nii, et ta tuli meile mitte esimese keskkooli klassi, vaid  
teise  ja me saime suhteliselt kiiresti headeks sõpradeks. Ilmselt mina, kes 
see muu võis olla, kutsusin teda sellesse seltskonna, kellega me olime juba  
mõned aastad seal Kopli piiril tegutsenud. 

\question{Ja kogu selle aja te käisite \emph{mainframe}-i näppimas? Mis te 
tegite nendega?}

No \emph{mainframe}-d said muidugi kiire lõpu, kuna arvutustehnika arenes. 
Esimene mitte-\emph{mainframe} platvorm, kuhu me kolisime, oli sealsamas 
keskuses õhtuti meisterdatud riistvaraplatvorm nimega 
Entel\index{Entel}, mis oli selline CP/M masin. Ta kasutas mingisugust 
Intel 8088 protsessorit, või mingit Vene klooni sellest kuulsast 
kaheksabitilisest protsessorist. CP/M tarkvara oli, aga midagi sellist 
spetsiifilist tema jaoks kirjutatud ei olnud ja  siis oligi nagu koht, kus sai 
hakata mitte-\emph{mainframe}-de peal kätt proovima. 

\question{Aga mis te nende arvutitega siis tegite? Noorel inimesel on ju see 
probleem, et kui valid liiga raske ülesande, ei saa hakkama ja on halb ja kui 
liiga kerge, siis on igav ja ka halb?}

See on üks väga relevantne küsimus, sellepärast et mõnes mõttes meie 
generatsioonil on arvutitega vedanud. Sel hetkel, kui arvutite juurde 
sattusime, olid nad sellised, et nagu midagi väga huvitavat ei toimunud. 
Arvutite peamine köitlus oli potentsiaal, mis neis selgelt sees tuksus. Versus 
see, et sul on Youtube ja Minecraft ühe kliki kaugusel. Ükskõik, kui palju sa  
pingutataksid, midagi  ligilähedastki sa võimeline tegema ei ole. Ja teine asi, 
et arvutid olid toona aeglased,  umbes  miljon korda aeglasemad kui praegu. 
Mistõttu, kui tahtsin midagi ägedat teha, siis pidin kohe kiiresti selle 
hingeelu endale põhjalikult selgeks tegema, et pigistada välja viimane 
efektiivsusepiisk.

\question{Sa jooksid kohe mingitesse riistvara piirangutesse sisse ja isegi 
mingi lihtsa asja ekraanil liigutamiseks pidi hoolega mõtlema, et kuidas see 
ikka täpselt käib!}

Täpselt. Mistõttu suhteliselt kiiresti läksime 
assembleri\index{Assembler} peale. Kõigepealt siis kodukootud 
Entel-arvutite peal ja siis aasta-paari pärast tekkisid Eestis esimesed IBM PC 
kloonid. 

\question{Assembleri peale kolimine eeldab siiski, et programmeerimisest on 
mingi aimdus olemas. Kust see tekkis?}

See tekkiski nende \emph{mainframe}-de peal. Robotron või mis ta oli.

\question{Aga kuidas? Lugesite raamatuid või\ldots?}

Lugesin läbi, mis selle nimi oligi, Programmeerimine 
Pascalis\sidenote{Tõenäoliselt R. Jürgenson Programmeerimine Pascal-keeles.} 
või midagi sellist. Mul on seal siiamaani mingisugused esimeste programmide 
väljatrükid  vahel, raamat on raamaturiiulis. Kirjutan 
BASIC'us\index{BASIC} programmi ja kirusin, et keel on ikkagi erinev kui 
Pascal. BASIC'ut ma ei osanud aga Pascalit natuke siis teoreetiliselt oskasin ja 
kahe peale siis hakkasin avastama. Esimene programm vist oli ruutvõrrandi 
lahendaja.

\question{See on klassika, ilmselt seetõttu, et teda on praktiliselt vaja. Aga 
ikkagi, sealt Assemblerisse minna on pikk samm, juba arusaam, et kuskil on 
Assembler, on küsimus. Kust te infot saite? Keegi õpetas? Raamatud? Ajakirjad?}

Jah, seal \emph{mainframe}-de peal ma isegi jäin BASIC'usse, tegin seal isegi 
oma esimese mängu. Ja kui me kolisime  \emph{mainframe}-de pealt ära nende 
kodukootud kaheksabitiste arvutite peale, oli näha, et seal on lihtsam  
riistvarale ligi saada, eks. Ja üks asi, mis kohe ahvatlema ja paistma hakkas 
oli C programmeerimiskeel\index{C}.  Mäletan, et samas grupis aeg-ajalt 
näitas oma nägu selline sell nagu Hannu Krosing\index[ppl]{Krosing, Hannu}, 
endine Skype kolleeg, kes  otseselt samas seltskonnas ei olnud. Ja tema oli 
selleks hetkeks  kirjutanud Assembleri õpiku, oli mingi selline pisikene 
brošüür põhimõtteliselt.  Ja ta kas pistis selle mulle pihku või, ma ei tea, 
igal juhul ma lihtsalt lugesin selle läbi, et \enquote{ohoo, mingi päris 
huvitav asi}. 

\question{Oot, mis aastal see võis olla?}

See võis olla 1987 äkki? 1988?

\question{87. aastaks oli Hannu kirjutanud Assembleri õpiku!?}

Jah, mingi sellise brošüüri vormis, 
samizdat\sidenote{\begin{russian}Cамиздат\end{russian}, tõlkes umbes 
\enquote{iseavaldamine} oli Nõukogude Liidus levinud keelatud või põrandaaluse 
kirjanduse levitamise viis. Teksti trükiti läbi mitmete kopeerpaberite 
õhukesele paberile ümber, tulemused levisid käest kätte ning neid paljundati 
omakorda. Mäletan, et ka minu vanaema tegeles sellise toksimisega, ning lapsena 
ei mõistnud, miks sellest väga rääkida ei tohi. Kuna kõik klahvidega asjad mind 
väga huvitasid, nuiasin välja võimaluse ka ise tekste ümber lüüa, miskipärast 
olen veendunud, et olen aidanud paljundada mingit budistlikku teksti.} umbes. 
Oli ikka Hannu õpik? Temaga ma mäletan, ma sellel teemal arutasin,  üsna kindel 
et tema oli selle autor.

\question{Ja mis te tegite selle Assembleriga?}

Mäletan, üks nagu selliseid korralikumaid projekte, võib-olla tegin mingeid 
väiksemaid asju ka, oli tekstiredaktor\label{sisu!jaani_tekstiredaktor}. 
Sattusime Priit Kasesaluga\index[ppl]{Kasesalu, Priit} sellisesse 
võistlusrežiimi. Mõtlesime, et mida oleks hea sellele uuele kodukootud 
platvormile kirjutada ja leidsime, et siin ei ole korralikku tekstiredaktorit. 
Hakkasime mõlemad tegema, kõigepealt BASIC'us\index{BASIC}. Ja üritasime 
üksteist üle trumbata, et kellel tuleb parem. Mäletan, et 
Hannuga\index[ppl]{Krosing, Hannu} rääkisime  mingist tehnikast, kuidas 
tekstiredaktoris teha  mingisugust \emph{split buffer} arhitektuuri, et 
liikumine ja \emph{insert}-imine kiired oleksid.

Tuli koolivaheaeg ja meil jäi võistlus pooleli. Aga mina panin nagu edasi, suvi 
otsa kirjutasin paberi peal tekstiredaktorit, assembleris. Ja kui tagasi tulin, 
polnud  Priit muidugi suvega midagi viitsinud teha ja sellega oli võistlus 
läbi. Siis kirjutasin selle Assembleri paberi pealt arvutisse.

\question{Töötas ka?}

Esimene kord muidugi ei töötanud, eks ole. Aga tööle ma ta igal juhul sain, asi 
toimimis ja  vaatasin, et \enquote{oo, see on ikka päris äge}. Kiire, mugav ja 
palju parem, kui ükskõik milline tekstiredaktor sellel arvutil. See andis mulle 
väga positiivse tagasiside. Peale seda hakkasin mänge kirjutama.

\question{Reflekteerides siit tundub, et sul pidi olema oskus päris suuri ja 
keerulisi abstraktseid struktuure peas ette kujutada, et sa suudaksid selle 
koodi kõik paberil asmi valada. Kust see oskus tuli või on see sul kogu aeg 
olnud või oskad sa natuke selle juuri avada?}

Ma ei tea, mulle tundus see suhteliselt loomulik. Sellised instruktsioonid 
lihtsalt, nagu sammude kirjeldus. Et mida sa tahad, et arvuti teeks, eks. On 
vaja täpselt üles kirjutada, mida sa tahad. Alguses BASIC'us sain esimese 
tagasiside, et kuidas tsükkel käib, ja ühel hetkel assembleris nägin, et see on 
lihtsalt natuke tülikam, aga teisalt jälle rohkem positiivset tagasisidet 
pakkuv, kui sa ta käima saad. Käib nagu väga muljetavaldavalt võrreldes 
BASIC'uga. 

\question{Kas see tekstiredaktor kuskile jõudis ka või sai lihtsalt oma lõbuks 
tehtud?}

See seltskond, kes siis seda Entel\index{Entel} arvutit tegi, 
vormistasid niipea, kui eraettevõtlus seaduslikuks muutus, kooperatiivi ja 
hakkasid  neid arvuteid tootma ja müüma. Muu hulgas käis selle arvuti juurde 
tema jaoks toodetud tarkvara: CP/M ja  lisaks see minu tekstiredaktor.

\question{Ehk esimene suurem projekt, mis sa kirjutasid, läks kohe kliendile 
müüki?}

Jah, ma ei tea, palju seda kasutati, aga kui sa endale kaheksakümnendate lõpus 
selle Eestis toodetud arvuti ostsid, siis oli seal minu tekstiredaktor kaasas. 
Selle tõttu me saime esimest palka ja tekkisid esimesed mingisugused 
sissetulekud

\question{See ju tahab tarkvaraarenduse mõttes küpsust, et sa mõtled kõik 
nurgatagused juhtumid läbi ja võib-olla kirjutad abiteksti?}

Mõnes mõttes mitte väga optimaalne, aga ma olen märganud, et mul on selline 
OCD, \emph{obsessive compulsive disorder}. Et kui midagi alustan, tahan selle 
kindlasti lõpule viia, panna i-dele punktid peale. Seetõttu paljudele 
projektidele, kus ei ole ajasurvet taga, läheb kole palju aega. Alates sellest 
tekstiredaktorist. Ma tahtsin, et kõik oleks väga ilus, kõik funktsionaalsus 
oleks olemas ja pusin senikaua, kuni oli. 

\question{Ehk siis kombinatsioon täiuslikkuse soovist ja võimekusest see ka ära 
täita. Mul võib olla soov täiuslik teemant lihvida aga ma lihtsalt ei oska seda 
teha.}

Ja, jällegi, millega mul vedas, oli see, ma astusin arvutite juurde sellisel 
hetkel kuskohas kogu tarkvara, mis seal arvutites juba oli, oli väga lihtne. 
Mistõttu see ei olnud selline nii-öelda hingemattev kogemus, et ma olen nii 
pisikene selle tarkvara kõrval vaid \enquote{ahah, okei, ma saan enam-vähem 
aru, kas tehtud on, ma teeksin paremini}.

\question{Seda on mitmed öelnud, et oma esimest arvutit nad tundsid põhjani.}

Ka auto-entusiastidel, uunikumide austajatel, on samasugune lugu, neil on 
ikkagi väga lihtsad riistapuud.

\question{Aga see annab sulle kontrolli tunde, eksole.}

No mul täna läks Tesla katki. Ja midagi ei ole teha, tuleb Soome saata.

\question{Ma korraks võtaks kinni noist alguses räägitud arvutiajakirjadest. 
Oskad sa takkajärgi öelda, oled sa mõelnud, mis sind nende juures paelus?}

Asjad, mis seal kohe väga prominentselt silma paistsid, olid mingisuguste 
arvutimängude reklaamid, mingid Atari reklaamid ja sellised asjad. Noh, nagu 
ikka, reklaamidel muidugi joonistati natuke ilusamaks, kui päris maailm, aga 
nad andsid vaate mingisugusesse oma seaduste järgi toimivasse fantaasiamaailma, 
mis tohutult paelus. Mingisugused ekraanitõmmised, kus mingid tegelased on peal 
ja ma vaatasin, et \enquote{ahaa, see on vist väga äge asi!}.

\question{Seda on ka räägitud, et see oli nagu täitsa teine maailm, kuhu sai 
sisse minna.}

Ja veelgi enam, sa said neid maailmu ise luua, see teadmine tekkis mõne aja 
pärast. Et sa ei ole passiivne tarbija vaid aktiivne looja.

\question{Ja selle aktiivsusega sa kirjutasid selle tekstiredaktori valmis ja 
sind võeti palgale?}

Ma ei mäleta, kuidas see järjekord täpselt oli, võimalik, et meid võeti palgale 
seal alguses, kui ma seal niisama katsetasime. Aga võimalik, et see tõesti oli 
pärast seda, kui me esimesed asjad ära tegime.

\question{See oli keskkooli ajal veel?}

Jah, see oli vist keskkooli alguses, ma arvan. Kaheksakümmend seitse oli 
keskkooli algus. Kaheksakümmend kuus võis olla see aasta, kus ma üldse sinna 
sattusin ja siis 87-88 oli see, kus palka hakkasin saama. 

\question{Kas arvutis käimine olümpiaade ei hakanud segama või käis see 
õppimisega kuidagi lihtsasti kokku?}

Üldse ei seganud. Arvuti-värk on mul ikka suhteliselt kogu aeg põhiline asi 
elus olnud, ülikooli lõpetasin ka nii-öelda kõrvalhobina ära. Aga juba siis 
olid kool ja olümpiaadid arvuti taustal.

\question{Kas juba siis hakkas moodustuma seltskond, mis pärast sai 
Bluemooniks\index{Bluemoon}?}

Just. Bluemooni süda tegelikult oligi see seltskond, mõned klassikaaslased. 

\question{Kas tollest arvutikooperatiivist eraldusite kohe eraldi ettevõtteks 
või oli seal vahepeal mingi faas veel?}

See oli niimoodi, et ühel hetkel meil Ahtiga\index[ppl]{Heinla, Ahti} tekkis 
mõte, et teeks ühe arvutimängu. Korraliku mängu, mis jookseb PC peale, mitte 
ainult seal kodukootud arvutite peal. Meil oli mingi eeskuju ka, mille järgi  
mängu teha, mis oli Yamaha MSX-ide\index{Yamaha MSX} peal, mis oli 
palju vähempopulaarsem platvorm kui, PC. Oli näha, et PC-d hakkavad juba jõudma 
sinnamaani, kuskohas saab juba midagi huvitavat teha. Ja väga sellise sügava 
mulje jätsid toona Ahti matemaatikuvõimed. Kuidas ta jagas ära, et 
\enquote{siin tuleb tangensit kasutada, et  perspektiivi luua}. Mingisuguseid 
esimesi eksperimente tegime tema vanemate juures Küberis\index{Küber}, kus tal 
oli arvutitele ligipääs. Ahti hakkas palju varem programmeerima, kui mina. 

Üks tõuge selle mängu tegemiseks oli see, et meil keskkooli viimases klassis 
(oli vist ikka viimane klass?) tekkis võimalus minna klassiga Rootsi. Esimene 
välisreis üldse, aastal 1989 suhteliselt unikaalne võimalus. Läksime sinna läbi 
Leningradi, siis oli vaja teha mingisuguseid imelikke trikke väljamaale 
saamiseks. Seal onutütre mees ütles, et \enquote{Väljamaal nad õudselt tahavad 
softi, kirjutage mingisugune lahe soft. Lähed sinna, müüd maha ja mingit 
probleemi pole!}. Mõtlesin, et \enquote{aga teeme} ja hakkasime tegema. 
Leidsime, et teeme mängu, teeme korraliku mängu. Hakkasime tegema ja muidugi ei 
jõudnud valmis. Softiga, nagu ma nüüd hiljem tean, tuleb kõik ennustused  umbes 
piiga läbim korrutada, kulub umbes kolm korda rohkem aega kui alguses arvad. 

Muidugi me seda valmis ei saanud, aga samas oli juba piisavalt suur hoog sees. 
Et ühel hetkel võtsime sinna juurde korraliku kunstniku, Kaspar 
Loit\index[ppl]{Loit, Kaspar} ehk BKnows. Ja muusika ning heli-inimese Ott 
Aloe\index[ppl]{Aaloe, Ott}. Ja tegime mitte ainult ühe mängu vaid täitsa 
sellise mängude seeria. Millega meil vedas, oli see, et see esimene mäng 
õnnestus Rootsi müüa hoolimata sellest, et meil Rootsis käisime ajal seda mängu 
kuhugi kellelegi pakkuda ei olnud isegi, kui ta oleks valmis olnud.

\question{Kuidas? Meil oli ju veel Nõukogude Eesti?}

Selle Sideministeeriumi arvutuskeskuse\index{Sideministeeriumi Info- ja 
Arvutuskeskus} juhataja Jüri Malsubi\index[ppl]{Malsub, Jüri}  tuttav oli üks 
sell nimega Tiit Vasli\index[ppl]{Vasli, Tiit}, kellel oli väljamaal suhteid, 
ta vahendas metalli, mingeid sihukesi asju. Ma isegi ei teadnud, mida ta 
vahendas. Ta oli selline mees, keda oli kaugelt näha, sellepärast et tal oli 
üks Eesti esimesi mobiiltelefone, mille antenn oli mingi kolm meetrit kõrge. 
Oli kaugelt näha, et tema läheb seal kuskil tänaval. Tal oli Rootsis sidemeid 
ja ta mõtles, et \enquote{noh, ma vaatan, räägin} ja müüski. Ta äripartnerid 
olid huvitatud sellisest eksootilisest asjast nagu raudse eesriide taga 
toodetud mäng. 

Selle mängumüümise tulemusena me teenisime rohkem, kui mu vanemad kunagi oma 
elu jooksul teeninud olid. Mis oli vist mingi viis tuhat dollarit. Arvestades 
muidugi inflatsiooni, mitte reaalväärtuses, vaid nominaalväärtuses. 

Kui see mäng nii-öelda müüki läks, tekkis meil tõsine küsimus, see oli siis 
keskkooli lõpp, ülikooli algus, et kuidas me nüüd seda administratiivselt 
korraldame. Oleme selles kooperatiivis  ametlikult tööl, eks, aga on tegelikult 
näha, et meie plaanid võivad suuremaks kujuneda, kui see kooperatiiv. Rääkisime 
läbi. Mäletan pingelist läbirääkimist Jüri Malsubiga\index[ppl]{Malsub, Jüri} 
kuidas seda mängu tulu jagada. Nemad on ühelt poolt panustanud ja meie oleme 
teiselt poolt panustanud, tahaks nagu oma asja teha. Lõpuks saime meie poolt 
vaadates väga mõistliku kokkuleppe ja leidsime, et nüüd on aeg vormistada asi 
mingiks oma ettevõtteks. Mida me siis ka tegime, aastal 1990, ma arvan. 

\question{Kas te mõtlesite nullist välja, et teil on vaja kunstnikku ja 
muusikut ja kuidas nende töö programmeerimisega siduda või oli teil eeskujusid 
ka?}

No me olime teisi mänge näinud ja nägime, et nad näevad paremad välja kui see 
meie katsetus ilma kunstniketa. Ma ei mäleta, kes meid  
BKnowsiga\index[ppl]{Loit, Kaspar} tutvustas, see võis olla isegi Tanel 
Hiir\index[ppl]{Hiir, Tanel}, ei mäleta. Kaspari kunstniku-võimed toona jätsid 
mulle väga sügava mulje. Teda oli raske tööle saada, ma mäletan, tihtipeale 
pidi selja taga istuma, et \enquote{tee nüüd}, aga kui ta tööle sai, oli väga 
äge. 

\question{Ma just mõtlengi seda, et kindlad viisid graafikat kasutada, 
töödelda, laadida on ju tänaseks välja kujunenud, kas teie mõtlesite need ise 
välja?}

Üsna, jah, sest, jällegi, need platvormid olid miljon korda aeglasemad, kui 
praegu. Mistõttu tööriistad olid Turbo Pascal\index{Turbo Pascal} ja 
Borland C\index{Turbo C}. Kaspar tegi asju Amigal, seal olid tal oma 
tööriistad.

\question{Mind on painanud see küsimus, et te ju tegite muusikaprogrammi. 
Kuidas te sinna valdkonda sattusite, te pole ükski muusikainimene ju, nii 
palju, kui ma tean?}

Ükskord ülikoolis oli sihuke lahe hetk, kus olin arvutiklassis ja mingid tüübid 
istusid arvuti taga ning komponeerisid  SoundClubis\index{SoundClub} muusikat. 
Kiibitsesin natuke ja ütlesin, et see on minu programm. Nad ei uskunud. 

Ma ei mäleta, kuidas see algtõuge sattus. Tänu sellele, et me olime juba mänge 
teinud, oli meil kindlasti kokkupuude sellega, kuidas teha taustamuusikat. Ja 
toona, üheksakümnendate alguses, oli väga suur trend trackerid, ehk mingite 
sämplite baasil muusika kirjutamise väga sellised platoonilised riistapuud. Ja 
sealt tuli mõte, et  heli on väga hea,  aga kasutajakogemus tundus  vähemalt 
harjumatule silmale väga-väga ebamugav. Mõtlesime, et kuidas kasutada sedasama 
tehnilist võimekust, aga teha  kasutajaliides, mis oleks äge eriti inimestele, 
kes ei ole pidevalt muusika kirjutamise juures.

Teema hakkas järjest huvitama, kuna seal on väga mitmeid nüansse, nagu UI 
disain, muusika pool asjas (kuigi ükski nendest autoritest ei olnud muusikud), 
kuidas tehniliselt teha aeglastel arvutitel head heli. Seal ma puutusin esimest 
korda kokku mingisuguste matemaatiliste teoreemidega, mida ma siis 
Ahti\index[ppl]{Heinla, Ahti} abil üritasin lahendada. Üks huvitav asi oli see, 
et kuna me need instrumendid korjasime endale kuskilt BBS-idest kokku, olid 
need õudse kvaliteediga. Mäletan, et Ahti kirjutas mingisuguse tarkvara, kus ta 
tegi Fourier analüüsi, et nad häälde viia. Ükskord Tartus istusin ja 
häälestasin pille niimoodi, et endal väga suurt muusikaharidust ei olnud, 
natuke olin pilli õppinud. Aga Fourier analüüsiga sai ikka väga hea häälestuse. 

\question{Selle tarkvaraga on ju tehtud igasugu asju Vennaskona Diskost alates 
ja ta käib ka siin mõnest loost läbi. Küll aga ma ei mäleta, et keegi oleks 
rääkinud selle tarkvara ostmisest?}

Jaa!  

Siiamaani võib-olla vähem kui kord nädalas, aga kord kuus vähemalt saame mingi 
fännikirja, et näete ma olen on SkyRoads-i\index{SkyRoads} peal üles kasvanud. On isegi mõned 
kloonid tehtud,  teda saab tänapäeval veebis mängida. Ja SoundClub oli teine 
suurem projekt. Meil oli siis juba Bluemoon firmana ja meil oli kaks toodet 
SkyRoads (mis tegelikult oli järg tollele esimesele Rootsi müüdud mängule, 
mille nimi oli Kosmonaut\index{Kosmonaut}) ja SoundClub. 

Ja nüüd oli küsimus, kuidas neid müüa. Mäletan, et see oli mingisuguste 
telefonide ja faktidega ja tšekkidega jamamine. Mõlemad olid \emph{shareware}, 
osalt saadeti lihtsalt ümbrikus sularaha aga tavaliselt saadeti tšekke, mida ma 
käisin Eesti Maapangas või Rahvapangas lunastamas, selline kogemus. 

Teine asi, mis oli tegelikult väga äge kogemus, oli läbirääkimiste pidamine 
olukorras, kuskohas teisel poolel ei ole mingit juriidilist motivatsiooni 
lepinguid järgida. Mistõttu tuli tihtilugu tekitada selline olukord, kuskohas 
sa nagu lood sellise helge tuleviku, et partnerlusel oleks jumet. Mõnes mõttes 
selline \emph{iterated prisoner's dilemma}\sidenote{Mänguteoreetiline 
konstruktsioon, mille abil uuritakse osapoolte koostööstrateegiaid. Selle 
valdkonna üks teadustulemusi on, et (eriti mängu iteratiivses, korduvalt 
mängitavas ja eelmisi tulemusi \enquote{mäletavas} versioonis) indiviidile 
annavad pikas perspektiivis parema tulemuse altruistlikud, mitte egoistlikud 
strateegiad.}, sa pead looma olukorra, kus teisel poolel, hoolimata sellest, et 
mingit sundmehhanismi ei ole, on lihtsalt huvi olla osa sinu tulevikust ja 
seeläbi lepinguid järgida.

Alguses oli meil \emph{shareware} aga inimesed hakkasid kirjutama, et tahaks 
seda mingisugusesse ajakirja panna või tahaks seda kuskil levitada. Mingi väga 
lahe sell tekiks meil Saksamaale, kes hakkas mitmeid meie asju levitama, aastal 
1996, käisin tal lõpuks isegi külas. Samuti üks väga lahe omaette kogemus oli 
müük Taiwani telefoni ja faksi abil, kuskil Tartu Estiko\index{Estiko} 
kontoris. 

\question{Kuidas sa Tartusse sattusid?}

Ülikooli läksin.

\question{Mida sa õppima läksid?}

Füüsikat. Nii mina kui Ahti läksime füüsikat õppima aga Ahti kukkus sealt juba 
teisel aastal välja. Mina punnitasin lõpuni. 

\question{Miks just füüsika? Ahti rääkis, et see ala tundus talle mõnes mõttes 
kõige puhtam?}

Ma arvan, et ta on vähem puhtam kui arvutiteadus või matemaatika, eks. Kaks 
põhjust oli füüsika valikuks, Ahti põhjused olid ilmselt korelleeritud. Üks oli 
see, et ma tundsin, et  arvutites ja matemaatikas olen ma juba piisavalt sees, 
et füüsika oleks nagu silmaringi laiendav. Ja teine oli see, et meie füüsika 
õpetaja, Vilma Kukrus\index[ppl]{Kukrus, Vilma}, oli ikka väga väga äge õpetaja 
ja tekitas füüsika vastu sügava huvi. Või vähemalt süvaga austuse. Mul on väga 
hea meel, et ma füüsika lõpetasin.

\question{See ilmselt mõjutas päris olulisel määral noore inimese maailmapilti 
ka?}

Absoluutselt. Füüsika on selles mõttes optimaalne teadus, et sa suhtestud 
reaalse maailmaga niimoodi, et kui matemaatikud võivad minna niivõrd 
abstraktseks, et nad kaugenevad reaalse maailma piirangutest, siis füüsikas 
reaalne maailm tõmbab su alati maa peale tagasi. Sõna otseses mõttes, 
tihtilugu. Ja seetõttu sul tekib intuitiivne arusaam sellest, misasi on teadus. 

\question{Ahtiga\index[ppl]{Heinla, Ahti} oli ka nii, sinu puhul on samasugune 
muster, seepärast küsin. See, mis ma kuulen ei kõla nagu keskmine 
\emph{teenager}. See kõlab nagu üsna küpse inimese jutt?}

No praegu ma enam \emph{teenager} ei ole!

\question{Nüüd jah, aga need otsused ja see viis, kuidas toona asju aeti on 
üsna kaine, arutlev lähenemine. Kust see pärit on?}

Üks oluline asi oli ikkagi, ma arvan, et Ahtile ma võlgnen väga palju tänu. 
Meil oli super hea koostöö. Priit ka, eks, aga praktiliselt kõiki selliseid 
probleeme lahendasime tiimiga. Minu  ja Ahti vahel tekkis väga tihti selline 
asi, et Ahti on nupukas ja ta mõtleb väga erinevalt, kui mina. Mistõttu 
koostöös temaga sündinud otsused olid just nimelt ägedad, kuna nendes oli kaks 
väga erinevat vaatepunkti, mida see otsus pidi rahuldama.

\question{Ma olen alati tahtnud küsida. Võib olla ruttame natuke ette, aga kui 
me vaatame kasvõi Bluemooni kodulehekülge, on seal loetletud üksjagu edukaid 
asju aga ka päris mitu asja, mis ei ole ühel või teisel põhjusel välja tulnud. 
Inimesed ei suuda mõnikord isegi läbi suure edu tiimina toimima jääda aga teie 
olete koos läbi nii suure edu kui mitmete ebaõnnestumiste. Kuidas te seda 
teete?}

Vahemärkusena, mäletan, mõni aasta  tagasi sain Sean Parkeriga\sidenote{Sean 
Parker on Napsteri kaasasutaja ja, muu hulgas, 
Facebooki esimene president. Ta esineb ka tegelaskujuna Facebookist rääkivas 
filmis, kus kujutatud intriigidest ja tülist tõukub ka eelnev küsimus.} kokku 
ja meenutasime  Napsteri ja Kazaa aegu, tema tegi Napsterit. Selline lahe 
kogemus.

Ma arvan ikkagi, et sellised kohatised eduelamused olid piisavad, et  läbi 
suruda ka sellistest mitte õnnestunud projektidest. Ja mõned hetked olid ikkagi 
jube rasked. Konkreetselt mäletan  sellist hetke, kus kogu mänguarendus läks 
üles-suunas ja siis ühel hetkel lõpetas meie kirjastaja Ameerikas, Interactive 
Magic, \emph{milestone}-de maksmise. Raha jaoks oli meil Exceli tabel, kus 
\emph{runway} oli kogu aeg kirjas, mitmeks kuuks  meil raha on põhimõtteliselt. 
See \emph{runway} hakkas siis kahanema ja ühel hetkel oli selge, et nad on 
pankrotis, sealt enam midagi ei tule. Oli tõsine küsimus, et mis nüüd edasi 
saab. Ja Ahti\index[ppl]{Heinla, Ahti} oli just see, kes ütles, et \enquote{ah, 
küll me välja ujume!}. Ujusimegi.

\question{Ühel hetkel te läksite mängu kirjutamise juurest ära ja kirjutasite 
Everyday. Kas see legend, et see käis kuidagi lehekuulutuse kaudu vastab tõele?}

Vastab tõesti, jah. See oligi just see raske hetk, kuskohas ma mõtlesin, et mis 
me nüüd teeme.

\question{Mis aastal see oli?}

See oli aastal 1999. Ühel reede hommikul vaatasin, ise olin Tartus, oli 
lehekuulutus, et pakutakse inimestele mingisugust ulmelist palka\sidenote{Teist 
perspektiivi sellele loole vaata Tarvi jutust leheküljelt 
\pageref{sisu:everyday}.}. Everyday portaal oli arendushädas ja Tele2 oli 
börsile lubanud, et kohe tuleb sihuke asi välja. Nad olid juba mingi aasta või 
paar arendanud ja olid välja tulekust kaugel. Kuulutuses oli  pikk nimekiri  
nõuetest, mida arendajad peaksid olema osanud. Pikk nimekiri asjadest, millest 
ma elu sees kuulnud ei olnud. IMAP ja POP3 ja PHP ja SQL ja mingid niisugused 
asjad. Tähtaeg oli esmaspäev, oli reede, mina olin Tartus ja teised olid 
Tallinnas. Mäletan, et helistasin Ahtile\index[ppl]{Heinla, Ahti},  rääkisime 
läbi, mõtlesime, et proovime, vaatame, mis juhtub. Tegime kohe  nädalavahetuse 
plaani ja põhimõtteliselt esmaspäevaks oli valmis prototüüp sellest portaalist, 
mida nad tahtsid. Kui me esmaspäeval intervjuule läksime, oli meil dilemma, et 
kas me ütleme, et see oli ühe nädalavahetusega kirjutatud või mitte. Ta nägi 
väga hea välja, kuna meil oli palgal arvutimängudega karastunud kunstnik, 
näiteks. Ma arvan, et prototüüp nägi parem välja, kui lõpptoode. Ja toimis 
täitsa, võisid  sisse logida, erinevaid paneele ringi lohistada, võisid emaili 
kirjutada, uudiseid lugeda, ilmateateid, mida iganes. 

\question{See tähendab ju, et tolle nädalavahetusega pidi sigima päris hea 
arusaam sellest, kuidas HTML ja brauseri renderdus ja muu selline töötab?}

Andmebaasid. Mäletan, et Priidule\index[ppl]{Kasesalu, Priit} jäi 
andmebaasidega tegelemine. Ta sattus hätta, ei saanud  loogikast aru. Ja ma 
mäletan, et ta võttis telefoni, helistas mingile andmebaasieksperdile, kahjuks 
ei mäleta, kes see võis olla. Oli laupäeva hommik. Kuulsin seda kõnet kõrvalt, 
et \enquote{kuule, mul on üks niisugune kogemus, on sul nagu hetk aega? Aa 
okei, okei.} ja pani toru ära. Ei olnud aega. Hea küll, tagasi uurima, kuidas 
SQL  käib uuris välja. Sai tehtud.

\question{Kõlab üsna ulmelisena, seal peab ju olema mingi meetod taga, kuidas 
seda teadmist omandada?}

See oli väga äge kogemus, jah. Ega tegelikult tehnoloogiad toona ei olnud super 
keerulised, nad olid meile lihtsalt võõrad. Ja meil oli tiim tõesti äge tooma 
ning saime tööjaotuse tehtud: igaüks pidi mingi kindla aspekti välja uurima. 
Magasime natuke, mitte eriti. 48 tundi tundi tööd.

\question{See tähendab, et te pidite kuidagimoodi oma tööd ka koordineerima, 
kes seda kampa teil juhtis?}

No mina olin nii-öelda ametlik juht. Samas see tiim töötas ise ka päris hästi. 
Välja arvatud, jah, võib-olla kunsti pool, mis põhjustas võib olla kõige rohkem 
meelehärmi, et kuidas Kasparilt\index[ppl]{Loit, Kaspar}  lubatud asjad kätte 
saada. Kunstniku asi, rohkem boheemlane, kui teised.

\question{Arvutades leiame, et kui kuulutus oli 1999 ja keskkool 
üheksakümnendate alguses, siis te kogu kümnendi kirjutasite mänge?}

Jah, päris mitmeid mänge tegime, kutsusime ennast Eesti mängutööstuseks. 

\question{Kui suur see tiim oli?}

1999. aastaks ega ta väga palju suuremaks ei läinud. \emph{Core} tiim oli siis 
mina, Priit\index[ppl]{Kasesalu, Priit}, Ahti\index[ppl]{Heinla, Ahti}. Artur 
Vill\index[ppl]{Vill, Artur}, kes oli 3D-kunstnik ja kes muide on teinud 
sellise filmi nagu Happy Feet mingid \emph{landscape}-d ja maastikud. Ta kolis 
pärast Bluemooni kokku kukkumist Austraaliasse ja seal tõusis tähelennuna, väga 
kõva vend 3D modelleerimises ja kunstis.

Ja kõrvalt Kaspar\index[ppl]{Loit, Kaspar} tegi kunsti, Ott\index[ppl]{Aaloe, 
Ott} ja Glen Pilvre\index[ppl]{Pilvre, Glen} tegid muusikat. Juhan 
Soomets\index[ppl]{Soomets, Juhan} tegi ka nagu poole kohaga 3D-graafikat ja 
vist oligi kõik. Kui bluemoon.ee lehele minna, siis see tiim on seal siiamaani 
üleval.

\question{Isegi toonase tehnoloogia lihtsuse juures pidi teil siis ju selle 
väikse tiimi peale tööd palju olema?}

Tööd oli päris kõvasti jah. Põnev oli ka muidugi.

\question{Mis see põnevus oli? Kui ühe mängu valmis olite teinud, kas siis 
igavaks ei läinud?}

Mängude tegemine ongi selles mõttes äge, et see on  niivõrd palju rahuldust 
pakkuv, kui mingi asi tööle läheb. Kirjutad mingisugust andmeanalüüsi. Kui asi 
tööle läheb, tuleb ekraanile õige number. Aga kui mängus asi tööle läheb, tuleb 
vägev plahvatus, näiteks. Või tulevad mingid väga, sellist, rahuldust pakkuvad 
stseenid,  efektid või lood või midagi sellist. 

\question{Nojah, vaade mingisse teise maailma, millest sa oled nüüd järgmise 
tüki loonud.}

Just, jah. Ja nüüd sa saad seal testimise käigus  ringi käia ja mingisugused 
kohati väga vapustavaid vaateid, sündmuseid, mis on toimunud\ldots

\question{Ühte asja ma tahtsin veel küsida. Kui sa rääkisid, et sa tegid üksi 
tekstiredaktori ja seal pidi kõik asjad ilusti reas olema, sellest ma saan aru. 
Aga kui meeskonnana softi kirjutada, siis see vajab ju \emph{software 
engineering}-u protsesse ka, kust teil need tulid?}

Üheksakümnendatel olid lihtsalt zip-failid ja \emph{backup directory}-d. 
Versioneerimist või selliseid  asju me üldse ei teinud. 

\question{Aga kuidas te siis tagasite, et see kupatus teil kokku ei kukkunud?}

Me olime väga ettevaatlikud! Üks põnts, mis meil juhtus, oli see, et meil murti 
kontorisse sisse ja varastati arvutid ära. Sealt läkski mingisuguse SkyRoadsi\index{SkyRoads} 
või mingi asja mingi versioon. Meil olid diskide peal \emph{backup}-id ja 
midagi jäi alles, aga mingisugused asjad Bluemooni ajaloost läksid lõplikult 
kaduma. 

\question{Siiski, kas te oma töökorralduse mõtlesite lihtsalt jooksu pealt 
välja?}

Istusime telefoni otsas, põhimõtteliselt\sidenote{Huvitav, kust see Skype idee 
küll sündida võis?}. Mina olin Tartus, Ahti\index[ppl]{Heinla, Ahti} kolis ühel 
hetkel Tallinna  tagasi. Istusime telefoni otsas, koordineerimine käis ka meili 
teel. Eks meil tekkis spetsialiseerumine ka. Mina manageerisin tiimi, 
kunstnikke, kirjutasin mingisuguseid tarkvaralõike. Priit\index[ppl]{Kasesalu, 
Priit} spetsialiseerus operatsioonisüsteemi asjadele, graafikale,  Windowsi API 
ja sihukesed asjad. Ahti\index[ppl]{Heinla, Ahti} tegi sellist rohkem 
teadusmahukat asja, kuskohas oli vaja midagi AI-laadset või siis mingisugust 
matemaatikat.  Kui vaja, tal oli võtta. Et \enquote{ahaa, ma tean, selle jaoks 
on siin sellel leheküljel Knuthi Art of Computer Programming-us\sidenote{Knuth, 
Donald E. Art of computer programming (TAOCP). Tuntud ka kui Knuthi Piibel. 
Tegu on monumentaalse teosega, mille seitse köidet pidid katma kogu teadaoleva 
arvutiteaduse. Praeguseks on ilmunud kolm köidet ja esimene osa neljandast. 
Kuna teise köite küljenduse kvaliteet lugupeetud autorit ei rahuldanud, lõi ta 
oma raamatu ilusaks tegemiseks süsteemi TeX, mille derivaatide abil kirjutab 
praegu oma artikleid kogu teadusmaailm ja mille abil on kujundatud ka käesolev 
tekst.} on õige algoritm, teeme selle!}. 

\question{Tundub siis, et kuna meeskond töötas tiimina hästi, siis see lahendas 
ka üksiti ära \emph{software engineering}-u probleemid. Ei tekkinud mingeid 
merge konflikte ega probleeme, sest te töötasite inimlikult nii hästi koos.} 

Tiim oli väike ka, räägime kolmest programmeerijast. 

\question{Kui sa vaatad tagasi enda kui programmeerija peale üheksakümnendatel, 
oskad sa kuidagi kirjeldada enda arengut?}

Põhiasi, mis, ma arvan, domineeris seda arengut, oli see, et arvutid läksid iga 
kahe aasta tagant kaks korda kiiremaks. Mistõttu oli vaja kogu aeg hoolitseda 
selle eest, et sa ajale jalgu ei jää. Lõpuks me ikkagi jäime, mängutööstuse aeg 
sai läbi. Kümme aastat, arvutid läksid selle aja jooksul mingi, mis see siis 
on,  kolmkümmend korda kiiremaks. Ja  võimalused: heli läks rikkalikumaks,  
mälu läks suuremaks, graafika ägedamaks, võrgundus tuli juurde. Kogu aeg tuli 
ennast hoida aja tasemel. 

\question{Aga programmeerimise kunsti mõttes? Mitte see, et kas ma tean üht või 
teist API-t vaid kas ma olen programmeerijana täna parem, kui eile?}

See on huvitav küsimus, ma väga palju ei ole selle peale mõelnud. Kindlasti  
kogemus õpetas. Ma ei oska praegu  tagantjärgi seda kuidagi kompresseerida. Ma 
tean, et programmeerijana arengud on mul pigem nagu hiljem olnud, võib-olla ma 
aga mäletan hilisemaid arenguid paremini. See, kus ma kolisin rohkem 
funktsionaalse programmeerimise peale,  juhtus peale Skype'i. Kuni Skype'i 
lõpuni ma ikka kirjutasin oma vanade tööriistadega.

\question{Pärast Everyday intervjuud, kui kiiresti te tolle \emph{production} 
versiooni välja lasite?}

Ma hästi ei mäleta, aga see võis olla nii, et  suvi otsa kirjutasime ja kuskil 
sügisel või umbes nii tuli välja. Mingi esimese versiooni jaoks võis kolm-neli 
kuud minna.

\question{Sellesama väikse tiimiga?}

Põhimõtteliselt küll. Kuigi nüüd oli nii, et seal oli juures mingeid rootslaste 
tehtud asju ja see väike tiim oli osa  palju suuremast organisatsioonist. 
Mistõttu läks asi ka oluliselt aeglasemaks. Mingeid asju oli vaja rootsi keele 
tõlkida. Ma mäletan ükskord öösel sain Niklaselt\index[ppl]{Zennström, 
Niklas}\sidenote{Niklas Zennström, hilisem Skype asutaja.}, meili 
rootsikeelsete vastetega inglisekeelsetele fraasidele ja all oli 
\enquote{midnight translation services by Niklas Zennström}.

\question{Sellised teenused siis. Mis Niklas tegi seal projektis?}

Niklas oli everyday.com-i CEO. 

\question{Niklas oli siis see mees, kes ei suutnud kogu oma rootslaste tiimiga 
tarnida?}

Seda ma ei tea täpselt, kuidas see atributsioon seal täpselt  oli, aga Niklas 
oli põhimõttelist see, kelle lõplik otsus oli see, et Eestist arendaja otsida. 
Linnar Viik\index[ppl]{Viik, Linnar} vist oli pakkunud, et võtaks Eestis 
programmeerijaid ja Niklas oli see, kes otsuse langetas, et Bluemooni 
seltskonda kaasata.

\question{Kuidas tiimi skaleerumine tundus? Kui te olite kõik see aeg 
kirjutanud kompaktses kõgproffide tiimis keerulist softi, siis veebiarenduses 
on rõhk ju mujal?}

Ma hästi ei mäleta, et seal mingisuguseid olulisi probleeme oleks olnud peale 
selle, et kohe tuli  kommunikatsiooni ülesanne juurde. Nagu ikka, kui on kaks 
programmeerijate tiimi, siis esimene reaktsioon kõigil on, et \enquote{see on 
teise tiimi bugi}. Neid asju  tekkis kohe kõvasti. Aga ma ei mäleta, et oleks 
mingi tohutu külma vee kaela saamine olnud. Saime hakkama küll. 

\question{See kõik toob meid 1999. aastasse ja seega ka otsapidi väljapoole 
meie ajahorisonti, milleks on kaheksa- ja üheksakümnendad. Mitte, et pärast ei 
oleks igasugu põnevaid asju veel juhtunud.}

Enamus asju juhtus hiljem!

\question{Aga inimeseks said sa ju varem. Mis sa praegu teed?}

Peamine ja kõige olulisem tegevus on hoolitsemine selle eest, et juhul, kui 
inimajastu peaks mõne aastakümne (loodetavasti mitte mõne aasta) jooksul 
lõppema,  inimesed siia planeedile alles jääksid.

\question{Ära pane pahaks, aga ma hästi ei näe mõtteliini ekraani peale 
plahvatuse joonistamisest selle teemani. Palun selgita!}

Sinna mahub üks kuni kaks aastakümmet veel, ehk see, millest me ei ole 
rääkinud. 

\question{Sa oled lihtsalt jõudnud selleni, et see on sinu jaoks oluline 
probleem?}

Jah, selleni jõudsin ma aastal 2008 või midagi sihukest, kui Skype'is juba hoog 
hakkas raugema, sealt enam väga väljundit ei olnud. Sattusin rääkima 
inimestega, kellega me viimased kümme aastat olen ehitanud sellist 
\emph{community}-t, kes üritavad teha ära AI-uurijate kodutöö. Ehk siis teha 
asju, mida on vaja selle jaoks, et AI-ga tulevik oleks inimestele soodne, aga 
millega AI-arendajad ise ei ole näidanud mingit motivatsiooni tegeleda peale 
selle abstraktse motivatsiooni, et nad on ka inimesed.

Üks võimalus probleemi kirjeldada on see, et meil on fundamentaalne 
\emph{trade-off}, ma isegi tea, kuidas seda eesti keeles öelda. Et sa ei saa 
nagu kahte asja korraga. Super kompetentset süsteemi ja sellist süsteem, mille 
üle sul on täielik kontroll. See ei ole isegi arvutite spetsiifiline probleem, 
inim-juhtidega on sama probleem: mida rohkem ta delegeerib, seda 
kompetentsemaks muutub süsteem või suuremaks kasvab organisatsiooni võimekus. 
Aga tema isiklik kontroll selle üle, mis toimub, väheneb. See on fundamentaalne 
printsiip. Ja mida inimkond praegu teeb, iga päevaga järjest rohkem, ta 
delegeerib oma otsuseid masinatele. Mistõttu sellise delegatsiooniga tegelikult 
väheneb inimeste kontroll tuleviku üle. Võib juba öelda, et praegu on inimeste 
kontroll tuleviku üle väiksem, kui see oli näiteks viiskümmend aastat tagasi. 
Ja see tendents tõenäoliselt jätkub. Nüüd on küsimus see, et kuidas me siiski 
säilitaksime kontrolli mingisuguste oluliste aspektide üle. Näiteks atmosfääri 
koostis, mis on meile oluline. Temperatuur, mis tundub juba praegu keeruline. 
Räägime siin veebruarikuus, väljas on kolm kraadi sooja, sajab vihma. Juba 
inimestel on raske keskkonna üle kontrolli säilitada. Lisame siia entusiastliku 
delegatsiooni arvutitele, kellel on keskkonnast täiesti ükskõik! Sellepärast me 
saadamegi roboteid radioaktiivsetesse aladesse või kosmosesse, et neid keskkond 
ei huvita. Probleem on selles, et AI arendajatel on motivatsioon aretada just 
nimelt delegatsiooni poolt, et delegatsioon oleks võimalikult lai ja tulemus 
oleks mingi meetrika järgi võimalikult kompetentne. Ja palju vähem on 
motivatsiooni selle jaoks, et mõelda selle peale, et kuidas kogemata mitte 
delegeerida selliseid asju, mida meie elus olekuks on vaja.

\question{Ma südamest loodan, et sul tuleb välja, sest muidu on pahasti!}

Ma tihtilugu ütlen inimestele, et \enquote{wish me luck, you are going to need 
it!}


\chapter{Taavi Talvik}
\index[ppl]{Talvik, Taavi}

\question{Kuidas sina arvutite juurde jõudsid?}

Arvutite juurde jõudmine oli iseenesest väga lihtne. Kodus sattus olema paar 
põnevat raamatut, vist Ustus 
Aguri \enquote{Abakusest raalini}\sidenote{Sarja \enquote{Mosaiik} 1980. aastal 
ilmunud 28. teose autorid olid siiski Rafail Guter ja Juri Poljunov, tõlkijaks 
Madis Järv.} ja Norbert Wieneri \enquote{Küberneetika}.\sidenote{Norbert 
Wiener. Küberneetika ehk juhtimine ja side loomas ning masinas. Eesti Riiklik 
Kirjastus, 1961.} Igal juhul tundusid need jube põnevad. Ja kuna esivanemad olid Tartu Ülikoolis\index{Tartu Ülikool} keemikutena 
ametis ja käisid kuulujutud, et ülikoolis mõni arvuti ikka on, siis hakkasin 
neile kohe pinda käima, et kuulge, ma tahaksin näha, missugune 
arvuti päriselt välja näeb.

\question{Sa oled järelikult Tartust pärit?}

Jah, ma olen Tartust. Tartu on väga okei, väike 
linnakene Elva lähedal, ja lapsepõlves mulle seal väga meeldis. 

\question{Kui vana sa olid, kui hakkasid vanematele arvuti asjus pinda käima?}

Arvan, et olin
üheksandas klassis, aastal 1985. Ja tõepoolest neil seal ülikoolis arvutid olid, isegi välismaa 
omad. Sel ajal oli välismaa arvuti haruldus, 
aga kuna nad tegid mingisuguseid imelikke elliptiliste kilede mõõtmisi, siis 
oli kilede mõõtmise masinaga kogemata kaasa ostetud arvuti, mille nimi 
oli Hewlett-Packard 85\index{HP-85}. 

\question{Kas see oli lauaarvuti?}

See oli lauaarvuti, millel oli pisikene, vist viietolline ekraan,\sidenote{Eri 
allikate andmetel tuli HP-85 kas viie- või kuuetollise CRT ekraaniga.} 
klaviatuur, kassettmakk ja termoprinter ning taga hunnik juhtmeid, mis 
ühendasid arvuti mõõtmisseadmetega.

\question{Niisugune mudel ei ole küll kellegi jutust läbi käinud, kõlab täitsa 
eksootiliselt!}

See oligi väga eksootiline mudel. Sellel oli oma Hewlett 
Packardi protsessor,\sidenote{Protsessori koodnimeks oli  Capricorn ja see toimis 
taktsagedusel 0,6 MHz (!).} mis oli oma aja kohta täitsa innovaatiline ja 
tore. Väljapaistvaks tegi selle arvuti tavaline 
BASIC\index{BASIC} ja see, et ekraanil sai jutte joonistada. Ja kui 
ise jutte joonistada ei osanud, siis sai pingpongi või kosmonautide maandumist 
mängida. 

\question{Kas sind lasti kohe seda niisama näppima?}

Jah. Eks vanemate kolleegid (kui ma õigesti mäletan, siis ühe
nimi oli Zirk) õpetasid ka, et kaua sa siin mängid, parem proovi kokku liita arve 
ühest kümneni või midagi sellist, ja nii see asi pihta hakkas.

\question{Kas teil koolis ei olnud arvuteid? Mis koolis sa käisid?}

Tartu 10. Keskkoolis\index{Tartu 10. Keskkool}, tänapäeval Mart 
Reiniku Gümnaasium\index{Mart Reiniku Gümnaasium|see{Tartu 10. 
Keskkool}}. Koolis ei olnud sel ajal veel mitte midagi, täitsa tühi maa. 
Tõenäoliselt Nõos midagi oli, aga Nõo on Tartust nii kaugel ja 
selleks pidi tutvusi olema, et sinna keegi oleks kutsunud.

\question{Üheksandas-kümnendas klassis 
kipuvad ju igasugused muud põnevad hobid olema Wieneri lugemise 
asemel. Miks sa neid lugesid?}

Tore ja huvitav oli, ja võibolla vanemad sokutasid ka lugemist, et 
poiss saaks targemaks. Tagantjärele enam ei mäleta, mis see täpne ajend oli.

\question{Kas sul oligi populaarteaduslike asjade 
või ulmehuvi?}

Oli nii populaarteaduslike asjade kui ka ulmehuvi. Kuna mu nupp 
reaalteadusi jagas -- käisin ka olümpiaadidel --, siis see 
tundus loomulik. 

\question{Kuidas sul reaalteaduste jagamine esile kerkis? Kas kohe 
esimesest klassist alates tundsid ennast sellel alal mugavalt või tegeles keegi sinu 
arendamisega spetsiifiliselt?} 

Tundsin end suhteliselt mugavalt tänu sellele, et isa-ema olid ülikoolis õppejõud, 
aeg-ajalt nad keemikutega midagi rääkisid ja nende käest oli alati 
võimalik küsida, kui füüsikas, keemias või matemaatikas hätta jäin. 
Ja kui keegi hädast üle aitab, siis tekib endal ka mugav 
tunne ja ei saa vastu näppe.

\question{Miks sa keemiku teed ei 
läinud?}

Arvutid olid põnevamad. Kui kord näpp oli antud, siis põnevus järjest kasvas.

\question{Too HP-85 ei saanud ju väga kauaks põnevaks jääda?}

Sellel oli oma võlu. Sai natuke trips-traps-trulli-laadseid 
mänge kirjutada ja esimese hea edukogemuse kätte. Edasi sai 
järgmistesse kohtadesse, kus olid veidi ägedamad ja võimsamad arvutid.

Lisaks keemiahoonele\index{Tartu Ülikool!Keemiahoone} oli Tartus olemas 
füüsikahoone\index{Tartu Ülikool!Füüsikahoone}, mille keldrikorrusel tegutses 
Alo Raidaru\index[ppl]{Raidaru, Alo}, kellel olid aadressil Tähe 4 PC-laadsed arvutid. Nendega sai juba teha palju rohkemat kui selle väikse 
õnnetu HP-85ga\index{HP-85}. Lõpptulemusena võeti mind umbes 
kümnendas klassis füüsikahoonesse nii-öelda laborandina tööle. 
Tööülesandeks oli üht, teist, kolmandat või neljandat programmeerida. 

\question{Kust sul programmeerimisoskus tuli?}

Ma ei tea, tuli järjest. Kõik aitasid ja õpetasid kõrvalt ning see kasvas kuidagi 
naturaalselt.

Loodan, et tegin seal midagi kasulikku. Laborandi palk oli viiskümmend 
rubla kuus, mis tekitas kümnenda klassi poisis kröösuse tunde. Eks 
palk natuke toetas mu huvi ja 
enamasti läksin pärast koolitunde kodu asemel füüsikahoonesse.

\question{Mis selle valdkonna sinu jaoks põnevaks tegi?}

Põnev oli see, et kui tegin arvutile mõne programmilaadse asja 
selgeks, siis arvuti tegigi midagi, mida ma ootasin. 
Mõnikord ei teinud ka, aga väga tihti tegi ja see oli jube kihvt, kui midagi 
juhtus. Mingi asi allus minu korraldusele -- täiesti unikaalne situatsioon 
maailmas!

\question{Oskad sa tuua mõnd näidet, mida sa laborandina progesid?}

Üks asi, mis meelde tuleb, on antiviirus. 

\question{Antiviirus 1980ndatel?!}

Umbes aastal 1986. Maailm hakkas vaikselt lahti minema ja 
tasapisi ilmusid Eestisse viirused. Tekkis see jama, et viirus jõudis ka
meie juurde ja sellest oli vaja kuidagi lahti saada, kuna arvutid hakkasid 
imelikult käituma. Viiruse nimi oli, kui ma õigesti mäletan, Yankee 
Doodle,\sidenote{Internet ütleb küll, et Yankee Doodle avastati 1989. 
aastal ja tähti kukutav Cascade 1988. aastal.} mis tegi piikse ja ekraanil 
hakkasid vist tähed kukkuma. Sai uuritud, kuidas see käitub, ja tehtud pisike 
programmikene, et sellest lahti saada.

Põhiliselt tegime Alo Raidaru\index[ppl]{Raidaru, Alo} laboris füüsikutele
elektroonikat: lisasid mõõteseadmetele, 
katseeksperimentidele ja nii edasi. Seoses sellega tegid nad ise ka trükkplaate. 
Tükkplaatide tegemiseks olid esimesed SmartCADi-laadsed programmid, millega 
õnnestus joonistada elektroonikaskeem ja trükkplaat ning see ka välja 
printida. Alo ehitas arvuti külge freespingi juhtimise \emph{interface}'i, mis 
freesis trükkplaadi välja. Lisaks väljafreesitud trükkplaadi 
radadele oli vaja puurida läbiviigu augud ja see 
puurimisprogramm usaldati mulle.

\question{Moodsas terminoloogias tegelesid sa järelikult kohe IoT-ga!}

Seda võib tänapäeval IoT-ks nimetada, aga tegelikult oli see trükkplaatidesse 
aukude puurimine. 

\question{Nimetame siis robootikaks!}

Nimetame asja ikkagi õigete nimedega. Tegelesin puurpingi puuri õigele kohale 
viimisega ja siis käsu andmisega, et mine alla ja tule üles tagasi.

\question{Kuna tõenäoliselt mingisuguseid teeke või draivereid ei olnud, siis kas
sinu programm käis riistvarani välja?}

Põhimõtteliselt küll, CADi programmist sai aukude koordinaadid 
ja nende koordinaatide peale tuli augud puurida. Vahepeale sai 
tehtud puuride liigutamise keel: astu sada sammu siiapoole, 
mine alla, tule üles, astu sada sammu sinnapoole.

\question{Kas selle keele mõtlesid ka sina välja?}

Vanemad inimesed kõrval aitasid -- nii kui kinni jooksin, tuli 
keegi appi. Hästi turvaline kasvukeskkond.

\question{Teised on rääkinud, et said üsna varakult teiste omasugustega 
ninapidi kokku, vahetati infot ja tekkis kogukonna moodi asi. Kas sul seda ei 
olnud?}

Kooliajal ei olnud, aga ülikooli astudes tekkis kogukonna tunne 
üsna kohe, esimesel kursusel.

\question{Kas laboranditöö õppimist segama ei hakanud?}

Ei, otseselt ei hakanud. Kuna nupp natukene lõikas, siis võis mõne koha pealt 
üle nurga lasta. Koolis lõputult pingutada ei olnud vaja, võibolla eesti 
keele kontrolltööd läksid kehvemaks, aga üldtase jäi nelja juurde. Kooli lõputunnistusel olid kõik neljad, välja 
arvatud üks viis ja üks kolm. 

\question{Mis aine kolm oli?}

Enam ei mäleta. Võimalik, et vene keel.

\question{Mis ülikooli sa läksid?}

Tartu Ülikooli, füüsikateaduskonda.

\question{Kas sellepärast, et sul oli seal laborandina juba käsi sees?}

Laborandina oli käsi sees ja tegelikult füüsika kui nähtus huvitas ka, 
oluliselt rohkem kui matemaatika. Füüsikas oli keerukusaste 
väiksem selles mõttes, et matemaatikud läksid teise või seitsmenda 
tuletiseni välja, samas kui füüsikud ütlesid: \enquote{Teine tuletis on nii 
ebaoluline, et seda efekti sellel kursusel ei aruta, jääb kolmanda kursuse 
materjaliks,} ja see sobis mulle.

\question{Ma just mõtleksin vastupidi, et füüsikas on päris 
maailm kogu oma keerukuse ja ebakorrapärasusega?}

Ei, vastupidi. On suhteliselt lihtsad rusikareeglid ja kui nendest 
aru saad, tuleb peenhäälestamine selle
peale ja, nagu ma ütlesin, siis alles järgmisel kursusel, nii et
esialgu võib selle kõrvale jätta.

\question{Kas sul tekkis spetsialiseerumine ka?}

Spetsialiseerumisega läks natuke sandisti, sest 
Vene sõjavägi tuli peale. Hiljem küll jätkasin füüsikas kaks 
pool aastat, aga siis tuli ka muu elu kõrvale ja õppimisvaimustus vaikselt hajus.

\question{Kus sa teenisid?}

Valgevenes, sellises superkohas nagu Borissov 13. 

Mõnes mõttes oli see ajaraiskamine, aga teisalt nägin maailma ja seda, kui 
palju erinevaid inimesi on tegelikult olemas.

\question{Arvutiinimesele oli see ilmselt üsna silmiavav!}

Ilmselt jah -- kui paljud meist 1988. aastal, kui ma sõjaväkke läksin, 
tegelikult reisinud olid? Võibolla Nõukogude liidu piires siin-seal 
käinud. Nii et tagantjärele mõeldes oli uute inimeste nägemine sõjaväes tegelikult päris kasulik kogemus.

\question{Millal sa sõjaväest tagasi tulid?}

Tagasi tulin aastal 1989. Mul õnnestus Vene sõjaväest pääseda ühe aastaga, kuna 
Gorbatšov ütles, et üliõpilased on meie sotsialistliku riigi tulevik ja mingu 
parem õppigu ülikoolis edasi, mitte ärgu jooksku püssiga ringi. 

\question{Mispeale tulevik läks Tartu Ülikooli füüsikat edasi õppima?}

Tulevik läks jah Tartu Ülikooli füüsikat edasi õppima ja proovis 
spetsialiseeruda astronoomiale.

\question{Miks just astronoomiale?}

See oli juhus. Tõenäoliselt sa tead sellist ulmekirjanikku nagu 
Isaac Asimov ja suure tõenäosusega oled kuulnud, et lisaks oli 
ta hea teaduse populariseerija. Koduses 
raamaturiiulis oli mul raamat 
\enquote{Universum}.\sidenote{Isaac Asimov. The Universe: From Flat Earth to 
Quasar, 1966. Venekeelne tõlge 1969.} See oli küll venekeelne tõlge pealkirjaga \enquote{\begin{russian}Вселенная\end{russian}}, 
aga kirjeldas niivõrd fantastiliselt seda, kuidas universum toimib, et see 
jäi kuklas kripeldama. Mõtlesin, et järsku peaks seda teemat edasi uurima, ja proovisin astronoomiale spetsialiseeruda.

\question{Mis tähendab \enquote{proovisid}? Kas ei tulnud välja?}

Otseselt ei tulnud välja, kuna koolieelsest tööelust kasvas välja järgmine 
tööelu, mis hakkas natuke õppimist segama. Seesama Alo 
Raidaru\index[ppl]{Raidaru, Alo} sokutas mind tööle ajalehte 
Edasi\index{Edasi|see{Postimees}}.

\question{Järelikult edasi ei olnud vaja auke puurida?}

Auke puurida ei olnud vaja, aga 1989. aastal tekkisid 
reaalselt ka ettevõtetesse esimesed arvutid ja 
\emph{desktop publishing}. Kuna üks Alo hobidest oli ülikooli teatmiku väljaandmine, siis tema käest küsiti nõu, kas Edasis saaks arvutit 
kasutada kuidagi ajalehe väljaandmiseks. Mind sokutati neid
sinna aitama ja siis aitasingi neli-viis aastat.

\question{Kas abi seisnes ainult \emph{publishing}-programmi 
käimaajamises?}

Ei, mitte ainult programmi käimaajamises. Reaalsuses on teatud töörutiinid ja kui need lähevad lihtsamaks, siis käivad kiiremini. Sulle nüüd 
omakorda küsimus, mis võis olla esimene asi Edasis\index{Edasi}, 
mida aastal 1989 arvutiga automatiseeriti?

\question{Eesti keele spellerit ju veel ei olnud \ldots}

Eesti keele speller oli juba olemas, aga see selleks. Mis võiks olla see 
teema, mida automatiseeriti?

\question{Ei oska öelda!}

Väga lihtne, see oli see valdkond, kust ajalehte raha tuli -- surmakuulutused. 
Tänaselgi päeval on Postimehes populaarsed paar eelviimast lehekülge, kus on 
surmakuulutused. Neil on see hea omadus, et on suhteliselt standardses 
formaadis: neil on kümmekond erinevat 
kujundust ja kui need kuidagi mallideks teha, siis 
surmakuulutuse publitseerimise aeg vähenes drastiliselt. Kuna see oli 
sisuliselt ainuke allikas, kust lisaks tellimustele raha tuli, siis selle vastu 
oli lehe juhtkonnas päris korralik huvi.

\question{Mis väljundisse need mallid läksid?}

Väljundiks oli kilele trükitud lehekülg, mis läks siis ofsettrükki.

\question{Kas sinu tarkvara optimeeris selle otse kilele?}

Jah, laserprinteriga lased paberi abil läbi kile ja selle, mis sealt välja tuleb, 
saad trükkalitele anda, et kleepige õigesse kohta.

\question{Teisisõnu produtseerisid sa PostScripti?}

Jah.

\question{See on ju päris keeruline!}

Ventura Publisher, see \emph{publishing} tarkvara, tegi põhitöö ära ja mõned üksikud asjad vajasid otseselt 
PostScripti tasemele minekut. 

\question{See kõik eeldab jälle teadmisi, kust sa neid juurde hankisid?}

Istusin ja nokitsesin. Küll ta lõpuks tuleb, kus ta pääseb!

\question{Mainisid enne, et sul tekkis midagi kogukonnalaadset.}

Üliõpilasena käid ju ikka seltskonnaga ringi. Proovid ühes ja
proovid teises arvutiklassis. Selleks ajaks olid juba Tartu Ülikooli 
matemaatikateaduskonda\index{Tartu Ülikool!Matemaatikateaduskond} ka 
arvutiklassid tekkinud ja sealsete inimestega suheldes see kogukond vaikselt 
tekkis. Samamoodi tekkis kogukond füüsika 
kursusekaaslastest.

\question{Kas tol ajal arvutisidet ei olnud?}

Tol ajal veel ei olnud, aga eks see tuli ka suhteliselt kiiresti. 
Ühtedel meestel oli ühtelaadi arvuti ja teistel teistlaadi arvuti, mis 
omavahel flopikettaid ei lugenud. Siis pandi kaks-kolm traati kokku ja 
prooviti kuskilt saadud programme teisele mehele ka üle kanda.

\question{Kuidas Edasis töövoog välja nägi? Millega ajakirjanik teksti kirjutas?}

Edasi\index{Edasi} aegadel kirjutas ajakirjanik ikkagi kirjutusmasinaga ja oli 
tinaladu. Aga kui tekkis rohkem arvuteid, mindi tinalaolt üle kilele 
trükitud väljundile. Seal vahel oli veel terve hunnik etappe, enne kui ajakirjanikud 
arvutid said. Arvuti oli tol ajal suhteliselt kallis, terminalid 
natuke odavamad. Postimehes\index{Postimees}, mis oli siis juba erastatud ja 
Postimeheks muutumas, sai pandud üles üks Unixi server, mille küljes oli 
kuusteist terminali, mis sai ajakirjanikele maja peale laiali veetud. 
Terminalide ühenduseks vajalikud kaardid sai Tõraverest, seal oli 
Urania\index{Urania|see{Astrodata}}-nimeline firma, millest kasvas välja 
Astrodata\index{Astrodata}.

Teksti sisestamiseks oli terminal ajakirjanikule piisavalt lihtne, seda ei 
olnud vaja ilusaks ajada, peaasi, et tekst oli olemas. Kui tekst oli valmis, 
pandi see \emph{publishing}-tarkvarasse ja lasti kile peale välja. Kiled 
kleebiti kleeplindiga kokku küljeks, mis läks öösel trükikotta. 

\question{Mis Unix seal serveris jooksis?}

BSDi Unix\index{Unix!BSDi Unix}, mis sai täiesti ausalt ostetud, \emph{source} 
koodi ja kõige muu värgiga. 

\question{Tol ajal oli ju lausa embargo, kuidas te selle serveri hankisite?}

Jah, embargo oli, aga need 386-laadsed arvutid embargo alla ei 
kuulunud. Ülemine ots, nagu PDP\index{PDP},\sidenote{\emph{Programmed Data Processor} (PDP). Üldnimetus Digital Equipment Corporation'i poolt 1957--1990 toodetud mitmetele miniarvutite sarjadele.} oli embargo all. 

Need kuusteist terminali jaksas rahulikult välja vedada. Arvutite hankimine oli
tollal keerukas. Kui Postimees sai tellimuste 
rublad kätte, veeti need kohvriga oskuslike ärimeeste juurde, kes said 
Moskvast mingi arvuti. Siis oli niisugune vorstikauba aeg. 
Igatahes lõpuks oli võimalik vajalikud arvutid välja ajada.

\question{Rääkisime ka Veiko Tammega\index[ppl]{Tamm, Veiko} sellest 
pikalt.\sidenote{Vt lk \pageref{sisu!veiko_moskvas}.}}

Just. Mõned kohvrid jõudsid tema juurde ka, ta oli põhiline Postimehele 
arvutite hankija.

\question{Mind paneb jällegi imestama, kui sujuvalt sinu skoop laienes. Kui 
programmeerimisest ma saan aru, siis nüüd on teemaks Unixi serverid, töövood, 
võrgud ja nii edasi. Mis hoidis sind seda ringi laiendamas? Oleks ju olnud 
lihtne programmeerimisele või millelegi muule keskenduda!}

Arvan, et see ümbritsev seltskond. Edasi või Postimehe ajakirjanikel läks silm särama, kui tekkis niisugune võimalus
oma tööd paremini teha, ja see tekitas soovi neid kuidagi aidata. 
Kui teisel inimesel silm särab midagi tehes, läheb endal ka silm särama 
teda aidates. Nii lihtne see ongi.

\question{See eeldab ka huvi inimeste vastu.}

Absoluutselt. Kui sa igapäevaselt kellegi kõrval istud, tekib see 
huvi tahes-tahtmata. Ei ole võimalik, et ei tekiks. Ja kui istud veel 
intrigeerivate inimeste kõrval, kes hoiavad kätt elu pulsil ja 
räägivad sulle: \enquote{Oh, Tallinnas Toompeal räägiti seda ja toda, ja kas 
paneme selle lehte või ei pane?}, hakkab kõrv liikuma küll ja tahad selle melu 
sees olla.

\question{See võis äge aeg olla küll, igasuguseid väljaandeid hakkas ilmuma!}

Tartlasena ma kõiki Tallinna asju ei tea, aga ilmuma hakkasid Eesti Ekspress ja 
Liivimaa Kuller Kalle Mülleri\index[ppl]{Müller, Kalle} ja Väino 
Koorbergi\index[ppl]{Koorberg, Väino} vedamisel. Samuti Kroonika, algul Kalle 
Mülleri, siis Ingrid Veidenbergi\index[ppl]{Veidenberg, Ingrid} vedamisel. 
Kõigi nende juures olid megakihvtid ja huvitavad momendid. 

Alguses oli väljaanne mustvalge, siis tekkis värviline logo ja nii edasi.

\question{Värviline logo oli suur asi! Kõigi väljaannete juures oli ilmselt mõni
\emph{publishing}'u või trükiinimene ametis. Kas Peeter 
Marvet\index[ppl]{Marvet, Peeter} tembutas juba ka kuskil?}

Tõenäoliselt Tallinnas tembutas, aga Tallinn ja Tartu on täiesti erinevad asjad.

\question{Ma seepärast küsingi, et kas sedalaadi inimestel oma kogukonda 
ei tekkinud, et näiteks programme vahetada?}

Kindlasti oli, aga seda teemat ma enam ei mäleta, kuna pärast tuli igasuguseid muid asju nii palju peale.

\question{Ühel hetkel sai Edasi asi otsa, mida sa edasi tegid?}

Enne kui Edasi otsa sai, oli üks huvitav moment, millest 
tahaksin rääkida. 

Edasiga seoses sain ma umbes aastal 1989 või 1990 endale meili, mille aadress oli 
\verb|taavi@pm.ew.su|.

\question{Kas EW nagu \enquote{Eesti Wabariik} ja SU nagu \enquote{Soviet Union}!? 
Kas selline domeen oli olemas?}

Jah, niisugune domeen oli olemas ja domeen SU on endiselt alles.

\question{Kust sa selle aadressi said?}

Mõnes mõttes tänu ülikoolile, kuna psühholoogia 
teaduskonnast\index{Tartu Ülikool!Psühholoogia teaduskond} Tiit 
Mogom\index[ppl]{Mogom, Tiit} oli Tallinnas kas Küberis\index{Küber} või 
KBFIs\index{KBFI} käima ajanud modemiga UUCP 
meiliside. Sealt see meiliaadress tuli. 

Sellest ajast mäletan veel esimest suuremahulist meili umbe 
ajamise intsidenti. Olid meililistid, mida sai tellida, ja 
ma tellisin kogemata ühe aktiivsema kirjavahetusega meililisti. Meile 
muudkui tuli ja tuli ja modem ei pannud toru hargile. Läks tund, teine ja kolmas, \enquote{no täitsa pekkis, kogu see 
maailm on umbes ja läheb katki!} Võtsin jalad selga ja kõndisin üle Toome 
Tiidu\index[ppl]{Mogom, Tiit} juurde: \enquote{Kuule, aita mul see meilivoog 
ära katkestada!} Umbes oli ju modemiga helistamine minu juurest tema juurde, 
tema juurest kuhugi Tallinnasse ja Tallinnast veel kuhugi Soome.

\question{Kusjuures tänapäeval on täitsa unustatud, kui oluline asi oli 
meil -- kõik asjad käisid üle selle. Oli olemas FTP üle meili, kus 
failid keerati sobiva pikkusega juppideks, lasti \mbox{BASE64ga} kokku ja saadeti 
meili peale!}

Täpselt. Sellisel kujul oli võimalik list- või 
arhiiviserveritest endale tellida vaba tarkvara lähtekoodi. Levis muidki asju, 
aga meid huvitas just vaba tarkvara lähtetekstide kättesaamine.

\question{Miks see huvitas?}

Siis sai jälle mingit uut ja võimsamat asja teha!

Edasis tuli järjest uusi automatiseerimisi peale. Ühel hetkel 
tehti oma kojukanne. Selle puhul oli oluline, et postiljonidele 
jagataks pakid sihtrajoonide järgi, õiged kleepsud oleksid peal, õigesse 
hunnikusse saaks õige kogus ajalehti ja et neil oleksid nimekirjad, mille järgi 
viia. Sai tehtud kojukande infosüsteem. 

\question{Jällegi, tänapäeval sellele ei mõelda!}

Eesti Postis või Omnivas on kojukande infosüsteem raudselt olemas!

\question{Just! Aga tänapäeval ei juhtu just sagedasti, et astud uksest 
sisse ja hakkad nullist sellist infosüsteemi programmeerima. Tüüpiliselt on 
midagi juba olemas.}

Sel ajal ei olnud võimalik midagi aluseks võtta, sest mitte midagi lihtsalt ei 
olnud. Äriliselt oli Postimehel ainuke võimalus ise kojukandesüsteem teha, 
kuna riiklik kojukanne ei toiminud. Ajaleht viidi kätte lõunaks, aga Postimees 
tahtis, et hommikuse kohvijoomise ajaks oleks leht olemas, ja ehitas nullist 
üles oma kojukandesüsteemi, mis pärast vist liitus Express Posti omaga. 
Praegugi on see vist alternatiivse kojukandesüsteemina teatud ulatuses 
toimiv.

\question{Ma mäletan seda küll, sest see, et Postimees oli hommikul vara 
postkastis, oli nagu \enquote{läänes}, nagu \enquote{päris}! Mida sa 
meiliga peale listide lugemise veel tegid?}

Kaks asja on eredalt meeles. 1991. aastal oli 
Moskvas putš: vahetati valitsust, tankid olid
teletornide ees ja mis kõik veel. Oli suur infoauk, mis toimub ja kus toimub. 
Siis sai Moskva arvutihäkkeritega kokku lepitud, et teeme 
otseliini, paneme infolistid käima. Jube põnev oli saada toimiv ja 
ajakirjanikele kasulik info meili teel kätte natuke enne, kui see teab 
kust tekkis. 

\question{Ja sul olid need kontaktid olemas?}

Jah, ühel Unixi kasutajate konverentsil käimisest Moskva taga Vladimiris. Muide, seal olid kohal esinejad Berkeley ülikoolist, 
kaks Berkeley Unixi loojat või guru. Kui õigesti mäletan, oli üks neist 
Keith Bostic.\sidenote{Berkeley Software Distributioni (BSD) 
ajaloo üks võtmetegijaid, kelle kõikvõimalikku panust vaba tarkvara ajalukku on 
raske üle hinnata.} Selles mõttes ei tasu naerda -- venelased suutsid need mehed 
enda juurde meelitada, tõenäoliselt oli neil ka huvitav ja konverents oli 
megakihvt.

\question{See võis tõesti olla tol ajal väga kõva sõna!}

Jah, oli küll. Tagantjärele mõeldes tekitas see jälle tunde, et arvutiteema on hea 
teema: hoiad näppu pulsil ja oled suhteliselt lähedal sellele, mis 
maailmas ägedat toimub.

Ülikoolis tekkis ka esimest korda moment, kui nähti, et meilindus on päris 
kihvt asi ja et lisaks sellele on olemas püsiühendus ja muu säärane. 1992. 
aastal pani Jaak Lippmaa\index[ppl]{Lippmaa, Jaak} 
Tallinnas püsti taldriku Rootsi Tele-Xiga ja Tartu tähetorni\index{Tartu 
tähetorn} sai satelliidi abil püsiühenduse, mis oli 64 kilobitti. Selleks et 
ühendus tähetornist Toomel kuhugi mujale ka leviks, sai Postimehe 
eestvedamisel tähetornist kõigepealt keemiahoonesse\index{Tartu 
Ülikool!Keemiahoone}, sealt ülikooli peahoonesse ja edasi Postimehe majja 
üle katuste veetud Tartu esimene püsiühendus. See oli ehitatud vene 
sõjaväelaste käest 500 rubla eest ostetud ülejäänud kaabli rullist.
Selle kaabli ümber tekkis ka nii-öelda internetikommuun. 

\question{Sest need, kes tee peale jäid, said ka endale interneti?}

Absoluutselt. Ja see oli täiesti \emph{online}!

\question{Aga latents pidi äge olema?}

Kuskil kuussada millisekundit.

\question{Nii hull ei olnudki!}

64 kilobitti ei ole tänapäeva mõistes mingi superkiirus, aga 
tollal, kui internet oli veel tühi igasugustest kassipiltidest, oli see 
täitsa okei.

\question{Kes see kogukond oli, kes kaabli ümber kogunes?}

EENeti\index{EENet} inimesed, Enok Sein\index[ppl]{Sein, Enok}, 
Anne Villems\index[ppl]{Villems, Anne}, Richard Villems\index[ppl]{Villems, Richard}, 
Tiit Mogom\index[ppl]{Mogom, Tiit}, Marek Tiits\index[ppl]{Tiits, 
Marek} Balti Uuringute Instituudist ja tõenäoliselt oli neid inimesi 
veel, kes praegu ei meenu.

\question{Kas sel ajal oli EENet juba olemas?}

EENeti veel ei olnud, aga tuumik oli seesama, kes kogukonna 
moodustas ja kes nägi vaeva selle nimel, et interneti püsiühendus oleks 
olemas. Lisaks kogunes modemiga sissehelistajaid ja kasutajate hulk hakkas vaikselt kasvama.

\question{Kas Tartu ja Tallinna side käis
üle satelliidi?}

Jah, alguses käis üle satelliidi, aastapäevad hiljem tuli ka maapealne 
püsiühendus Küberi\index{Küber} eestvedamisel. See on see koht, kus 
Eesti internetimaailmas tekkis ilmselt kaks nii-öelda rististe suuskadega 
kommuuni: Küberi ja KBFI\index{KBFI} oma.

\question{Kas siis ei olnud veel AS Cybernetica, vaid Küberneetika Instituut?}

Jah, aktsiaselts tekkis hiljem.

\question{Miks suusad risti läksid?}

Seda mina ei tea. Tartu inimesena ei saanud ma sellistest Tallinna 
probleemidest lihtsalt aru. Kui ressursiga on kitsas, nagu Eesti Vabariigi 
alguses oli, siis võibolla olid seal rahade jagamise 
või teaduse finantseerimise mured, et kes sai oskuslikumalt finantseerimisele 
ligi. See oli olelusvõitlus, mis jättis kõigile jälje.

\question{Mis edasi sai?}

Postimehe periood sai läbi ja umbes aastakese töötasin Tartu 
Ülikooli raamatukogus\index{Tartu Ülikool!Raamatukogu}, kuhu sai Rootsi kunni 
abiga muretsetud esimene serveri moodi asi ja otsast hakatud kirjutama 
raamatukogu infosüsteemi.

\question{Kas see tähendab, et sinu ajast on seal need 
kuulsad kalanimega serverid?\sidenote{Kui kõigil serveritel oli veel oma nimi, oli 
kombeks anda ühe asutuse serveritele samalaadsed nimed. Näiteks Halo kunagised masinad 
olid nimetatud kuulsate häkkerite järgi: woz, mitnick jne. Kuuldus
sellest kombest levis just läbi Tartu Ülikooli kalanimega serverite.}}

Enam ei mäleta, kilu.nlib.ee\index{kilu.nlib.ee} oli vist 
SPARCStation 2\index{SPARC!SPARCStation 2}. Kusjuures selle serveri põhifunktsioon oli ikkagi esimene ülikooli 
raamatukogu elektrooniline kataloog, mida kohapealsed inimesed ise 
programmeerisid.

\question{Selle süsteemi kasutamiseks olid raamatukogus isegi terminalid. 
Oma aja kohta oli see funktsionaalne ja tore lahendus!}

Jaa, tööd oli sellega kõvasti, andmesisestust on nullist väga raske teha.

Kahjuks või õnneks jäi raamatukoguperiood 
lühikeseks, kuna Jaak Lippmaa\index[ppl]{Lippmaa, Jaak} kutsus mind 
Tallinnasse ja see tundus veel põnevam. 

Ta kutsus mind sellisesse riigiasutusse nagu Valitsusside\index{Valitsusside}, 
umbes sellise mõttega: \enquote{Sina oled nüüd internetiga natuke 
kokku puutunud, oskad ühest arvutist teise sümboleid saata. Eesti riigil tuleb mingisugune 
KGB sidekeskus üle võtta, tule aita!} Ja tulingi.

KGBst võib rääkida mida iganes, aga tehnoloogilise poole 
pealt nägi asi välja suhteliselt õnnetu. Ädala tänava 
majas\sidenote{Ädala 4d. Selles endises KGB sidekeskuses paikneb tänaseni hulk eri organisatsioonide serveriruume.} olid suured saalid täis mittetöötavaid 
telefonijaamu. Võin nüüd natukene valetada, aga seal oli suurusjärgus 200 
töötavat telefoni, mis olid ette nähtud riigiorganite jaoks. Samas kakssada töötavat telefoni kogu valitsuse peale 
on suhteliselt nadi number.

Järgmine projekt oli Siemensi telefonijaamadega sisetelefonijaamade 
võrgu käimapanek Toompeal, Kadriorus, välisministeeriumis ja kes nad seal 
kõik Pikal tänaval olid -- kokku paar tuhat numbrit. Õnneks tuli see täitsa 
edukalt välja ja oli ka reaalne kasu olemas.

\question{Kas see oli veel analoogjaam?}

Võrgustatav digijaam Siemens TopCom. See oli päris sidevõrk.

\question{Kust tuli visioon, et nii kallist jaama ehitada? Odavam oleks olnud analoogjaamadega hakkama saada, aga 
tehti investeering.}

Investeeringu lüke tuli Jaagu isiklikul initsiatiivil, 
tehnoloogilise poole pealt konsulteeris ta Aavo Pikofiga\index[ppl]{Pikof, Aavo}, 
kes oli Tallinna Telefonivõrgus. Sinna taha tulid tänu Jaagu ja Endel 
Lippmaa\index[ppl]{Lippmaa, Endel} tutvustele ka riigi funktsioonid. Seda oli 
tõepoolest vaja, visioon osteti ja finantseeriti ära. Tänu sellele saadi reaalselt 
töötav sisetelefonivõrk, aga eks see jäi mõne aja pärast ajale jalgu.

\question{Kas füüsiline kaabeldus oli Eesti Telefoni oma?}

Osa oli Eesti Telefoni käest ja osa oli vana KGB kaablivõrk, mida oli 
Tallinnas mõnisada kilomeetrit.

\question{Kuigi keskjaam ei toiminud, olid kaablid ikkagi olemas?}

Olid korraliku tinakestaga kaablid, mis oli umbes nagu tuumasõja üleelamiseks 
ette nähtud ja meenutasid rohkem tanki kui kaablit.

\question{See kõlab projektina, mille käigus kohtub huvitavate inimestega.}

Absoluutselt. Telefonijaama installeerimine ja käimapanek Kadrioru lossis, 
kus Lennart Meri\index[ppl]{Meri, Lennart} vaatas üle õla ja õpetas, kuidas 
telefonipistikuid ühendada, võib tagantjärele muigama panna, aga nii see 
oli. 

\question{Kõlab väga Lennarti moodi!}

Absoluutselt, ta oskas igas asjas nõu anda.

\question{Kui kaua sa valitsuse kaableid vedasid?}

Kokku kolm aastat. Peale esimest edukogemust telefonijaamadega tahtsime loomulikult 
järgmisi edukogemusi. See internetinimeline asi ronis 
kogu aeg uksest ja aknast sisse. Kirjutasime siia-sinna järgmiseid pabereid, et 
nüüd oleks mõttekas sellesse kuidagi investeerida. Sel ajal nõuti juba ükskõik kelle käest, kes natukenegi internetiga tegeles, et \enquote{mina tahan ka}, soovijaid tuli järjest juurde. 

Aga paberite kirjutamine ei olnud edukas ja niimoodi see asi ei toiminud. Siis hakkasin kõiki tuttavaid 
läbi sõitma, et võtaks pundi kokku ja hakkaks 
normaalselt ise tegema. Käisin läbi vist kõik inimesed, kes vähegi 
internetiga tegelesid. Tartlased vedu ei võtnud, seal on nii mugav ülikooli ja Pirogovi juures olla, kuigi Marek Tiits\index[ppl]{Tiits, 
Marek} aitas päris palju ideed formuleerida. Kõige rohkem võttis 
vedu Andres Bauman\index[ppl]{Bauman, Andres} KBFIst\index{KBFI} ja temaga 
koos tegimegi internetiettevõtte. 

Käisin poes, ostsin riiulifirma, mille nimi oli 
Nösper\index{Nösper|see{Uninet}} ja mis praegu on Elisa -- sellesama 
riiulilt ostetud firma ja Elisa\index{Elisa} registreerimiskood on sama. 

\question{Milline juriidiline järjepidevus! Mis aastal see oli?}

See oli aastal 1995.

\question{See oli ju üsna karm aeg -- mitte midagi polnud saada ja kõik asjad 
tuli nullist ehitada!}

Oli küll, aga teisalt inimesed olid leidlikud. Näiteks Data Telekom\index{Data 
Telekom} Neeme Takise\index[ppl]{Takis, Neeme} eestvedamisel ehitas ise RS422 
elektrilise ühenduse peal modemeid. Kui ei olnud, siis tehti ise või kohandati. Midagi oli võimalik igal juhul teha.

Mingisugused lisaboonused olid ka. Tänapäeval inimesed arvavad, et kümne 
euro eest kuus peaks tulema nii jäme internet, kui maailmas üldse olemas on. 
Mis see siis teeb, kolm kohvitassi kuus? Tollal aga oli internet nii seksikas 
ja uus asi, et oldi nõus selle eest maksma. Oli niisugune helge aeg, et ettevõte ostis ise seadmed välja, maksis kinni paigalduse ja tasus veel 
kuutasu ka. Tänu sellele oli 
võimalik teha esimesed seadmeinvesteeringud. Sellise mudeliga nagu täna, et kõik on kümme või viis kohvitassi kuus, ei oleks internet kunagi Eestisse 
jõudnud.

\question{Kui ma õigesti mäletan, siis hakkasin teie edasimüüjana toimetama vist
aastal 1996, mis tähendab, et teil oli üsna varakult lai 
võrk teenuseid ja partnereid.}

Põhiline oli sissehelistamine. Ma ei mäleta, mis süsteemiga sai 
püsti pandud füüsiline modemite \emph{pool} telefoninumbritega, aga sellest see 
tegevus tegelikult pihta hakkas. Tõenäoliselt aitas telefoninumbritega, kuna need 
olid defka, Jaak Lippmaa\index[ppl]{Lippmaa, Jaak} isa Endel 
Lippmaa\index[ppl]{Lippmaa, Endel}, et saaks kuidagi sobiva 
koha sobivas järjekorras.

Sealt hakkas internetiäri vaikselt kasvama, kuna inimesed tahtsid ja huvi 
suurenes. 1997. aasta alguseks oli 
üldkasutatav interneti välisühendus igasuguste puukide poolt, nagu meie 
puukfirma, nii täis aetud, et EENet\index{EENet} tegi otsuse, et
ärikasutajad peavad muretsema endale oma välisühenduse.

\question{Lausa nii hilja? Siis saite üsna kaua akadeemilise traadi peal toimetada.}

Jaa. Internet ei olnud veel päris akadeemilisest maailmast välja pääsenud ja 
eks kõik toimetasid akadeemilise traadi peal. Kuni EENet jalga 
maha ei pannud, oli see täiesti loomulik nähtus. 

\question{Kuidas te siis ühenduse saite? Kas kuhugi Soome?}

Mul oli mingist ajast jäänud kontakt Helsingi telefonivõrguga ja nende 
inimestega koos sai kanal tellitud ning ühendus ja seadmed hangitud. Meil oli 
juba sedavõrd palju käivet, et suutsime enam-vähem isegi väliskanali soliidse
kuutasu kinni maksta, mis oli pea 100 000 krooni kuus. 

\question{See oli tollal jõhker summa!}

See oli väga jõhker summa,\sidenote{Selle raha eest võis endale Kadriorgu 
ühetoalise korteri osta!} kogu see internetitehnoloogia maksis hingehinda. Kuue 
pordiga Cisco ruuter, marki enam ei mäleta, maksis 250 000 krooni, S-klassi Mersu hinna.

\question{Järelikult oli füüsiline kaabel Eesti ja 
Soome vahel olemas?}

Füüsiline kaabel oli olemas. Peale seda, kui Telekom\index{Eesti Telekom} sai 
Eesti riigilt kontsessioonilepingu, vedas ta suhteliselt kiiresti Eesti-Soome 
vahele ka kaks valguskaablit. See võis olla 1996. või 1997. aastal. 
Telial\index{Telia} oli see kogemus olemas -- investeerimisvajadus kogu
analoogvõrgu väljavahetamiseks oli väga selge ja Telial tegelikult 
investeerimisjõudu oli. 

\question{Kas tegite \emph{hosting}'ut ka?}

Jaa, pidasime servereid, et ei peaks traadi raha maksma.

\question{Kuidas see käis? Kui praegu ostan endale virtuaalmasina, siis tol 
ajal sain konto ja parooli.}

Said Unixi \emph{account}'i ja oligi kõik.

\question{See oli legendaarne ettevõtmine ja legendaarne aeg, aga mida sa 
praegu teed?}

Praegu olen sellises huvitavas ettevõttes nagu RebelRoam. 
Tegeleme transpordiettevõtetele wifi-teenuste pakkumisega ja nende 
optimeerimisega. Kliendibaasiks on mööda Euroopat ja Ameerikat 
sõitvad bussid, kus on wifi, mida bussireisijad saavad kasutada. 
Või jõe- ja kruiisilaevad, kus pensionärid ostavad 3000--4000dollarilise nädalase reisipaketi ja kui nad Pariisis Eiffeli torni 
pildistavad, siis peavad õhtul saama saata oma pildi lastelastele -- muidu on 
reisikogemus natukene nõrk. 

See võib tunduda eestlasele imelik, sest meil on internet kogu aeg igal pool 
vabalt kättesaadav, aga näiteks Prantsusmaal, Inglismaal või Ameerikas ei 
ole 4G ega 5G nii levinud, et selle kvaliteet oleks kõigile piisav. Seal 
on teenuse optimeerimise vajadus täitsa olemas.

\question{Järelikult kõik need inimesed, kes Ameerika kiirteel Greyhoundi 
bussi järel sõidavad, et wifit kasutada, on teie kliendid?}

Greyhound on jah meie klient. Ma ei tea küll, kas kõikidel bussiliinidel või rohkem 
lääneranniku pool.

\question{Paistab, et ring on täis saanud. Kui sa Edasi 
puhul rääkisid, kuidas sulle tegi rõõmu sära ajakirjaniku silmas, siis nüüd 
tõid ka kohe esimese näite Eiffeli torniga ja et lõpuks tegeled ikka inimeste 
rõõmustamisega.}

Kui sa teda rõõmsaks ei tee ja ta ostab sinu teenust vihaga, siis ega ta sulle raha
ka ei maksa. Aga kui teed rõõmsaks, siis tõenäoliselt laekub ka raha 
pangakontole. Ja on rõõm endal ka midagi paremaks teha.

\question{Kas koodi veel kirjutad?}

Jah.

\question{Kas võrku veel konfid ise?}

Võrku ise ei konfi, ma vist ei oskaks seda enam korralikult teha. Võrk on 
tänapäeval niivõrd ära virtualiseeritud, et enam ei oskaks konfigureerida.

\question{Mina küll märkasin, kuidas keegi küsis selle kohta Facebookis ja sa mitte ei 
vastanud, kuidas saab teha, vaid kirjutasid ifconfigi käsurea, mis võtmest 
töötas!}

See on oma arvuti, mitte võrgu konfimine. Ma pidasin silmas
ikkagi päris võrku, kus on jämedad ruuterid ja vilkuvad sinised LEDid ja kust käib 
läbi pool Eesti interneti \emph{traffic}'ust. See on võrgu 
konfimine! 

\chapter{Sten Tamkivi}
\index[ppl]{Tamkivi, Sten}

\question{Kas sa oled Tartu poiss?}

Olen Tartus sündinud ja kasvanud. 

\question{Sa oled natuke noorema põlvkonna inimene, kui teised, kellega rääkinud 
olen. Mis aastal sa sündisid?}

1978. Olen noorem jah, sest kui olen memcpy saateid kuulanud, siis minu jaoks 
enamik saates esinenuid olid siis juba legendaarsed ja \emph{established} nimed, kui 
mina 1980ndate lõpus arvutite ja interneti juurde jõudsin. Mõnega neist olen hiljem tuttavaks saanud ja avastanud näiteks, 
et ohoo, Madis Kaal\index[ppl]{Kaal, Madis} on päriselt ka olemas 
ja ei olegi nii palju vanem, kui ma arvasin.

\question{Palun räägi oma kujunemisloost. Mõned on olnud hirmsad 
olümpiaadihundid, teisi on huvitanud raamatud. Kuidas sina 
arvutivärgi juurde jõudsid?}

Oli paar erinevat suunda. Esiteks olen pärit teadlaste 
suguvõsast, isa ja vanaisa olid mõlemad füüsikud ja ma kasvasin üles 
Tartus füüsika instituudi järgi nime saanud FI rajoonis. Mis 
tähendab seda, et kõik lasteaiakaaslased ja naabripoisid, kellega õues 
mängisin, olid kuidagi füüsika instituudiga seotud. Oma osa mängis ehk ka see, et 1980ndate lõpus, 1990ndate algul pani instituudi elektroonik majadesse piraatkaabeltelevisiooni ja 
instituudis sai ligi esimestele arvutitele. 

Teiseks õppisin ma Miina Härma Gümnaasiumis\index{Miina Härma 
Gümnaasium}, 1985. aastal esimesse klassi minnes oli see Tartu 2. 
Keskkool\index{Tartu 2. Keskkool|see{Miina Härma Gümnaasium}}. See kool 
oli mõnes mõttes Tartu tollastest esikoolidest humanitaarsem. 
Treffner\index{Hugo Treffneri Gümnaasium}, toonane 1. keskkool, oli 
selgelt reaalainete kallakuga, kuigi ka Miina Härmas oli 
reaalainete suund täiesti olemas. Käisin isegi olümpiaadidel. 

Üks varasemaid pilte kuskil vanemate või vanavanemate fotoalbumis 
on mustvalge foto sellest, kuidas keegi tõi 2. keskkooli 
algklassilastele näha kooliarvuti Juku\index{Juku}. Ka mustvalgel pildil 
on juba näha, et silmad läigivad arvutit lähedalt nähes.

\question{Ühesõnaga sinu esimene arvutikogemus oli FIs?\index{Tartu 
Ülikool!Füüsika Instituut}}

Ma arvan küll. Tekkis selline \enquote{mina ja arvuti} aeg, kui isa 
lasi mind ühel õhtul kellegi kabinetti. Neid kohti oli veel, 
näiteks pinginaaber Amish Mody\index[ppl]{Mody, Amish} isa töötas Tartu Ülikooli raamatukogus\index{Tartu 
Ülikool!Raamatukogu}, kus saime arvuti taga käia, ja hakkas 1990ndate alguses 
isegi arvutikaupu müüma või kooperatiivipoodi pidama. 

Amishil oli kodus ka arvuti. Esimene koht, kus ma Amigat nägin, oli aasta või kaks noorema 
koolivenna Lemmit Kaplinski\index[ppl]{Kaplinski, Lemmit} juures, kelle isa oli
kirjanikuna ilmselt kuskilt maailma pealt Amiga kätte saanud. Ka 
tähetornis\index{Tartu tähetorn} olid mingisugused kuulsad arvutid. 
Noorte Tehnikute Majas\index{Tartu Noorte Tehnikute Maja} olid 
Yamahad. Kui niimoodi loendama hakata, siis võibolla 
iseloomustabki toda perioodi see, et kellelgi ei olnud püsivat kohta, vaid 
otsiti erinevaid võimalusi arvuti taha saada.

\question{Neid kohti oli Tartus palju!}

Tartu on nagu Eesti Cambridge või Berkeley, kus 
suhteliselt väikeses asulas on nii domineeriv ülikool. Kogu internet 
hakkas ju akadeemilistest võrkudest peale. Mäletan, et Tartu püsiühendused 
olid ka satelliidiga tähetornist Rootsi. 
Ehk enne, kui tekkis Tallinna-Tartu link, tekkis Tartu link mingisse Rootsi 
ülikooli. 

\question{Kui said \enquote{sina ja arvuti} aega, siis kui 
palju ja mis õpetust sulle anti, mida arvutiga teha?}

Küllap see muster on enamikul inimestel täpselt sama: alguses 
tahad mängida, siis aru saada, kuidas mänge tehakse, ja seejärel hakkad natuke 
programmeerima. Mina hakkasin 9. klassis 15aastasena pärast kooli 
programmeerijana tööl käima. Tolleks hetkeks olin 
kaks-kolm aastat omal käel progenud. Isa oli Tartu 
Teaduspargi\index{Tartu Teaduspark} asutaja ja seal tegutses 
mitmeid tehnoloogiafirmasid. 

Ilmselt ta küsis, kas keegi leiaks poisile mingit kasulikku 
tegevust, ja leidus sihuke hulljulge mees nagu Valentin 
Abramov\index[ppl]{Abramov, Valentin}. Teatavasti toimus 1990ndate alguses 
Eestis ohjeldamatu metalliäri. Tartus oli metallikonglomeraat 
Primex ja sellel tütarfirma Primex Data\index{Primex Data}, kus 
tehti igasuguseid asju, põhiliselt pandi PC kloone kokku. Mõned inimesed progesid projektijuhtimistarkvara, teised
raamatupidamistarkvara, näiteks Tarmo 
Tali\index[ppl]{Tali, Tarmo}. Valja\index[ppl]{Valja|see{Abramov, 
Valentin}}\sidenote{Valentin Abramov.} palkas mind nii-öelda programmeerijaks, 
aga tegelikult oli see selline koolipoisi pärastlõunane ajaviide. Midagi 
vist progesin ka, aga ma ei usu, et sealt midagi \emph{production}'isse 
jõudis. Küll aga haldasin kohalikku arvutivõrku ja aitasin arvuteid kokku 
panna, lisaks hakkasin BBSi pidama -- oli selline tee-mida-tahad maailm.

\question{Järelikult pidi sul olema piisavalt alust väita 
end programmeerija olevat. Kas sa õppisid raamatute järgi? Internetti ju 
veel ei olnud.}

Ega raamatuid ka ei olnud päris alguses saada. Käisin hooti
Noorte Tehnikute Majas\index{Tartu Noorte Tehnikute Maja} 
arvutiringis, aga korralikku progemise algharidust sealt siiski ei saanud. Mäletan, et 
olen kirjutanud ka paberi peal koodi, kui parajasti 
ei olnud ühelegi arvutile ligipääsu, aga üritasin pabermaterjalide pealt 
midagi tuletada või teha, enne kui jälle arvuti taha sain. Mäletan sellist asja nagu
Arvutustehnika \& Andmetöötlus\index{Arvutustehnika \& Andmetöötlus}\sidenote{Vt lk \pageref{sisu:aa}.}. 

\question{Kui praegu on küsimus, 
kuidas lapsi programmeerima õpetada, siis nendest lugudest tuleb läbivalt
välja, et keegi ei oska öelda, kuidas tollal programmeerima õpiti. See kuidagi imbus 
õhust või läbi naha. Kuidas see nii on?}

Olen mõelnud, et mis puudutab 
nii-öelda kooliarvuteid, nõukogudeaegseid Agate\index{Agat}, 
Yamahasid\index{Yamaha} ja Jukusid\index{Juku}, siis seal oli 
ikkagi arenduskeskkond esimene asi, kuhu ennast alguses sisse 
\emph{boot}'isid. Suhteliselt raske oli arvutit kasutada ilma
arendusvahendite otsa komistamata. iPhone'i puhul pead kurja 
vaeva nägema, et saada üles keskkond, millega saaksid midagi teha. See 
on kindlasti niisugune muutus. 

Kooliarvutite ajastu oli nii 
palju põnev, et kui mu onu ostis endale DOSiga \emph{laptop}'i, ilmselt Compaqi, siis käisin tal külas seda kasutamas. Aga kuna sellel ei olnud ühtegi arendusvahendit, siis no mida sa 
teed seal? Kaua sa seal DOSi \emph{directory} puus ringi surfad, see ei ole väga huvitav. See oli selline äriarvuti, tekstiredaktori ja muude asjadega. Ja see oli
esimene kord, kui 
sattusin kasutama arvutit, mis ei olnud ennekõike arendamiseks mõeldud. 

Kui me sinuga kunagi tuttavaks saime, sattusin sulle Võrru 
külla, kui sa olid laenanud kooli arvutiklassist suveks koju ühe 
Agati\index{Agat}. Agat oli ekstreemne juhtum -- selleks et üldse midagi teha, pidid kõigepealt heksis või BASICus 
sisestama koodi, et saada \emph{prompt}, kuhu saaks hakata midagi 
kirjutama. Kui koolipoiss istub suvel arvuti taga ja toksib sisse 
kuueteistkümnendsüsteemis koodi, et arvutiga saaks midagi 
mõistlikku teha, siis on tema suhe arvutiga selgelt teistsugune kui 
lihtsalt meedia tarbimine. 

Lisaks teeb arvutite lihtsus või piiratus (kui 
visuaalne mängumaa on 25 rida korda 80 tähemärki või hiljem EGA- või 
VGA-graafika) selle kättesaadavamaks. Nii palju rohkem jäetakse 
fantaasia hooleks, et laps suudab tegelikult ka midagi progeda. Ehk kui keegi 
teeb tekstirežiimis mängu, siis see ongi nii-öelda selle arvuti tipptase. 
Kui täna keegi võtab koduse mängu-PC ja teeb seal midagi 
tekstirežiimis, siis\ldots{ }Ühesõnaga, kõik, mis ei ole tohutult videokaardi 
võimalusi kasutav 3D-renderdus reaalajas, tundub nüüd naeruväärne, aga tollal 
kõik see, mida suutsid ise oma kätega teha, ei olnud naeruväärne. 

\question{Kas sul oli ka raamatute või muusika vastu huvi? 
Selles seltskonnas, kus sa liikusid, pidi ju liikuma ägedat ingliskeelset 
kirjandust.}

Liikus ikka. Miina Härma\index{Miina Härma 
Gümnaasium} oli selles suhtes 
äge kool, et enamik asju, mis seal toimusid ja jälje jätsid, olid pigem 
seotud 
teiste õpilastega. Koolibände oli kõvasti, ma ise üheski küll ei olnud. 
Näiteks 1990ndate lõpus tekkinud Bizarre\index{Bizarre}, mille liikmetega ma väga palju hängisin 
ja osadega 
siiamaani läbi käin, oli Miina Härma koolibändist välja kasvanud. Otsapidi ka väga elektrooniline ning avas minu jaoks arvuti 
ja muusika seoste maailma. 

Raamatuid loeti, aga küberpungi ja \emph{science fiction}'i 
juurde jõudsin mina 1990ndate teises pooles, kui Ameerikasse 
sattusin. Enne seda lugesin pigem Tolkieni \enquote{Kääbikut} kui 
\emph{science fiction}'it.

\question{Kas Primexis programmeerijana töötasid enne, kui Ameerikasse 
läksid?}

Jah, see oli aastal 1993.

\question{Miks
Primexis seda softi progeti? Kas enda tarbeks või taheti sellega äri 
teha?}

Eesti IT-tööstuse ajalugu on käinud paari lainena. 
1980ndate lõpul, 1990ndate alguses, kui alustati 
täiesti tühjalt lehelt ja kõigil oli arvuteid vaja, siis kõik tõid juppe ja panid 
arvuteid kokku. Oli Primex Data\index{Primex Data}, kuskil kõrvalmajas
Ordi\index{Ordi}, samuti Astrodata\index{Astrodata} ja Tallinnas MicroLink\index{MicroLink} -- kõik tegid sama asja. Siis liiguti 
tasapisi tarkvara kihti. Riigil pigem raha ei olnud, pankadel oli, aga 
võibolla huvi ei olnud, ja tekkisid firmad, kes arendasid tarkvara teenusena. Kogu 
see Helmeste ja Webmediate laine on selle kõige tugevam näide. 
Tänaseks on \emph{mainstream}'iks muutunud toodete ehitamine. 

Primex Data oli naljakas hübriid. Ühest küljest oli seal arvutiäri, 
mida tegid ka kõik teised ja kust tuli põhiline käive. Teisest 
küljest hakati tarkvara tegema ikkagi tootena, tehti asju, mida loodeti ilmselt flopi peal müüa. Tekkis mingisugune turg ja sellised raamatupidamise tarkvara firmad nagu
Merit Tarkvara\index{Merit Tarkvara}\sidenote{Ehitab raamatupidamis- ja 
personalitarkvara, tegutseb siiani.} ja 
Eetasoft\index{Eetasoft}\sidenote{Ehitab Eeva-nimelist 
raamatupidamistarkvara, tegutseb siiani.}, 
osa neist tegutseb siiamaani. Projektijuhtimise tarkvara tegid kaks 
progejat nimega Urmas ja Jürgen, kes minu teada kirjutasid sellest samal ajal Tartu 
Ülikoolis oma magistritööd. Projektijuhtimine, Gantti graafikud ja selle kohta 
eestikeelne tarkvara -- ilmselt selline akadeemiline asi, mida nad 
lootsid ka müüa. Samas ma ei mäleta, et sellest oleks suurt äri 
tekkinud. 

Sellise hübriidi pidamine oli suhteliselt jabur. Ma ei mäleta, kas see juhtus minu 
või Tarmo Tali\index[ppl]{Tali, Tarmo} arvutiga või mõlemaga (me 
istusime kõrvuti), et tuled pärastlõunal tööle ja hakkad
progema, aga selgub, et keegi on su arvutist mälu ära müünud. Ja siis 
tegeled alustuseks sellega, et enne oli kaheksa megabaiti mälu, aga äkki leiab kuskilt
laost neljamegabaidise mooduli.\sidenote{Nii oli ka Korelis. Vt lk \pageref{sisu:jupimyyk}.}

Üks näide tarkvaramaailma läbipõimumisest muu maailmaga: ühel päeval istusin ja 
nokitsesin midagi teha ja sisse astus Jaan Tallinn\index[ppl]{Tallinn, Jaan}, 
kellest ma olin kuulnud. \enquote{Kosmonaut} ja nii, legendaarne mängutiim! Ja Jaan tuli 
monitori ostma! Kuigi olin riistvara müügiga tegelenud, siis põhikooli- või keskkoolipoisina käed värisesid, 
jube põnev oli!

\question{Seda on mitmest loost kosta, et tol ajal oli arvutiäri nii võimsa
marginaaliga, et selle kõrval sai näiteks pidada tudengeid 
projektijuhtimise tarkvara kirjutamas. Ilmselt kuulub siia alla ka sinu amet?}

Täpselt. Tänu sellele võtan ma kindlasti täna märksa
parema meelega endale praktikante, interne ja töövarjusid. Kui palju mind see 
võimalus mõjutas, mille Valja \index[ppl]{Valja} mulle 
andis! Nad maksid mulle isegi palka, aga see oli selgelt olukord, kus poleks 
mingit vahet olnud, kui nad ei oleks mulle palka maksnud.

Teiseks oli seal huvitav, et sattusin esimest korda võrkude 
juurde. Enne seda oli arvuti nagu iseseisev, eraldiolev asi. Täpselt umbes 1993. aastal ilmus
Tartusse internet ja Primex Data 
või vähemasti teaduspark\index{Tartu Teaduspark} võis olla esimene koht, kus 
mitteülikooli asjad sattusid võrku. See oli jadaühendusega võrk, mille otsas oli 
terminaator ja said ilge siraka, kui arvuti oli maandamata. Välja nägi see võrk
nii, et kuskilt läbi seina tuli üks ots ja sa ei teadnud, mis 
masinad selles jadas veel on. Kõik olid ühes võrgus maja peal laiali. 

Meil oli 
1993. aastal püsiühendus internetti ja Lynxi-põhine veeb enne, kui Mosaic\index{Mosaic} 
välja ilmus! Ühest otsast pidasin Primex Data nime all BBSi ja teistpidi 
oli meil olemas püsiühendus, kust oli võimalik 
\enquote{Doomi}\index{Doom} demo või \enquote{Wolfenstein}\index{Wolfenstein 3D} 
FTPga kätte saada ja BBSi üles panna. Paljude teiste 
BBSide jaoks oli failide levitamine modemiga 
\emph{peer-to-peer} loksutamine, aga meie olime nii-öelda pumba juures. 

Teise näitena sellest, kuidas arvutifirma hobina tehes hoopis 
teistsugune välja nägi, meenub meie sisevõrk. Selles oli Novelli server, 
millel oli 300-megabaidine kõvaketas\sidenote{300 megabaiti oli ulmeliselt suur 
andmemaht. Tol ajal oli normiks 3,5" flopiketas mahutavusega 
1,44 megabaiti. Laial kasutusel olid ka 5,25" kettad mahutavusega mõnisada 
kilobaiti. Neile lõigati kääridega(!) lisasälk, et oleks võimalik 
salvestuseks tarvitada ketta mõlemat poolt nii mahtu kahekordistada. Kettaid kanti spetsiaalses  
karbis endaga kaasas, sellisesse karpi mahtus kellegi kogu digitaalne elu.}, millest 
tööasjadeks (kood, mida seal 
kirjutati, ja hinnakirja Wordi fail) oli kulutatud umbes paarkümmend megabaiti. Ülejäänu laadisid 
mingisugused Hollandi tüübid öösiti tarkvara täis, sest 
kuskil Ida-Euroopas, kus ei olnud ka intellektuaalse 
omandi kaitset, oli uks lahti tehtud \ldots

\question{Tollal polnud Euroopaski isegi ligilähedaselt nii rangeid 
intellektuaalse omandi reegleid, rääkimata Eestist!}

Jah. Hommikul tuulasin tolle ketta läbi ja vaatasin, mis asju BBSis ülejäänud 
Eestiga jagada.

\question{Päris nii ikka ei olnud, et avad FTP pordi või telefoninumbri ja 
muudkui hakatakse väärt kraami laadima. Sul pidi järelikult mingi võrgustik olema. Kuidas 
see tekkis?}

Usenet ja selle uudisgrupid võis olla esimene rahvusvaheline
kogukond, kuhu ma sattusin. Eesti FidoNeti grupid ka. \emph{Overlap} oli ilmselt täitsa olemas, kuna Eesti FidoNeti 
gruppides juba arutati, kus internetis käia ja kus keegi istub ning tarkvarale 
ligi pääseb. Minu jaoks tulid reaalajas 
\emph{chat-room}'id, Random\index{Random} ja teised 
jututoad, hiljem. Istusin ka IRC kanalites, aga ma ei mäleta, et sealt 
oleks tohutu side või sõpruskond tekkinud. See oli rohkem 
\emph{ad hoc}. 

\question{Kuidas sul üldse tekkis mõte hakata BBSi pidama\index{Primex 
Data}? Kas sulle anti tööülesanne?}

See on hea küsimus! Mäletan, et idee müüsin küll Primexile maha väitega, 
et jube kasulik on tärkavas võrgus nähtaval olla. Pagan teab, võibolla oli 
laos olemas modem ja tekkis küsimus, mida sellega teha saab. 
Kodus ei olnud mul arvutit 1990ndate lõpuni. Ei olnud nii, et 
oleksin olnud BBSide kasutaja ja nüüd tekkis võimalus üks ise püsti panna. 
Ju see oli ikka sedapidi, et laos oli modem ja sellega sai helistada sisse 
teistesse BBSidesse, mis juba Tallinnas olid.

Enne kui sa külla tulid, hakkasin mõtlema, et mäletan jätkuvalt FidoNeti 
\emph{node}'i numbrit peast: 2491/2.2. Tartu tsoon oli vist 
kaks, kõik Tallinna asjad olid ühe all. Ja siis selguski, et Tartu tsoon on veel 
suhteliselt hõre ja on võimalik olla teine BBS Tartus. Esimene oli vist Jaan 
Pruulmann\index[ppl]{Pruulmann, Jaan}.

\question{Järelikult oli Primexi BBS olemas enne 
Luciferi\index{Lucifer BBS} oma?}

Mis tema number oli? Veikot\index[ppl]{Tamm, 
Veiko}\sidenote{Veiko Tamm, Lucifer BBSi pidaja. Vt lk 
\pageref{chptr:lucifer}.} tundes võis see olla 666, kui ta endale
sellise nime pani. Võibolla need numbrid ei olnud puhtalt lineaarsed. 

\question{Kas sinu BBSil oli üks liin ja üks modem?}

Jaa. Alguses oli 2400boodine modem, hiljem kiirem.

\question{Kui olen küsinud, mida inimesed BBSis hoidsid, siis tavaliselt on 
öeldud, et seal hoiti endale huvitavana tundunud asju. Mis sorti kola sul seal 
oli?}

See värk oli suhteliselt kaootiline. Huvitav, kas see \emph{file 
list} oleks kuskil alles? Ilmselt olid mängud ja
\emph{utility}'d nagu kõigil teistelgi. Ressursikitsikusest tingitud
asjad, mis olid huvitavad või vajalikud, ja kiiresti arenev tehnoloogia, näiteks 
pakkimisalgoritmid, nagu zip ja arj\sidenote{ARJ (\emph{Archived by Robert Jung}) 
oli 1990ndatel väga levinud efektiivne pakkimisformaat, mis praeguseks 
on enamasti unustatud. Selle eelis oli võimekus suuri faile sujuvalt mitme 
flopi peale laiali jagada.}. Pluss nendega 
oli see rõõm, et need olid väiksed ja neid sai kiiresti liigutada, kui midagi uut 
tuli.

\question{See oli ju osa rutiinist, et esimesena asjana pidi arvuti juures 
käepärast olema mõni pakkija ja vahendid mälu laiendamiseks!}

Ruum ja modemi aeg oli ju kallis. Ja kui 
öine meili sünkimine venis tunniajaliseks, siis vaatasid, et 
võibolla kõiki neid gruppe pole vaja, mida ise ei loe. 

\question{Kui palju sul sellest aimu oli, kes sul BBSi küljes käisid? Kas
lihtsalt mingid numbrid helistasid?}

Mäletan seda minimaalselt. 
Pigem mäletan seda tunnet, et kui ise juhtumisi seal olin (ma öösiti 
käisin ikkagi kodus magamas) ja modem kõne vastu võttis, siis vaatasin
põnevusega, mida see inimene teeb. Ma ei mäleta, kas see oli 
Windows või OS/2\index{OS/2}, millega tollal katsetasin. Aga 
vist OS/2 peal sai kuskil aknas jooksutada DOSi virtuaalmasinaid või programme. Mul multitaskis see asi taustal isegi siis, kui tööd tegin.

Teine asi oli meil väljahelistamine. Minu arust istus Tartu 
teaduspark\index{Tartu Teaduspark} linna kõige vanema telefonijaama taga, mis 
oli legendi järgi 1950ndatel püsti 
pandud analoogjaam. Üritasime teda ühte- ja teistpidi ka lahti 
häkkida. Üks asi, mis meil vist korra töötas, oli see, et kettaga telefonil on ühest kuni üheksani klõpsude arv 
ja kümme klõpsu on null. Lugesime kuskilt, et kui teed üksteist klõpsu, siis 
saad kätte kaugekõne \emph{trunk}'i. Tegime üksteist klõpsu, saime teise tooniga 
heli, helistasime mingisse Hongkongi BBSi ja minu arust ei tulnud selle 
eest kunagi arvet. Aga see lõbu kestis jube vähe, sest jaam oli tõesti 
muldvana ja kasutajaid oli ilmselt palju taga ning see vahetati esimeste seas 
digikeskjaama vastu välja. 

Üheksakümnendate lõpus, kui nägin esimest korda 
häkkeriajakirju, näiteks 2600\sidenote{2600: The Hacker 
Quarterly. Enamasti lugejate enda poolt sisuga täidetud kultuslik perioodiline 
ajakiri, mis käsitles kõikvõimalikke arvuti-, interneti- ja 
telefonisüsteemidega seotud tehnilisi teemasid ning üldisi 
\emph{underground} arvutiuudiseid. Nagu eespool (vt lk 
\pageref{sisu:2600}) jutuks tuli, oli 2600 Hz toonaste telefonijaamade 
jaoks oluline sagedus, sealt ka publikatsiooni nimi.}, ja lugesin 
kaheksakümnendate lõpu ja üheksakümnendate alguse \emph{phone phreaking}'u laine 
kohta USAs, siis see oli korraks relevantne ka Eestis. 

Teine asi, mis kindlalt töötas, oli see, et kui telefonikaardil (selle 
pooleaastase perioodi jooksul, kui Eesti Telefoni juurutas kiipkaardiga 
telefoniautomaadid\sidenote{Tõenäoliselt andis 
Eesti Telefon kiibiga kaarte välja aastatel 1995--2010. Iseküsimus on, kui 
palju sajandivahetuseks kaarti aktsepteerivaid taksofone järel oli.}) üks klemm 
kinni teipida, sai tasuta helistada. See oli ka tehnoloogiahuviliste 
noorte rõõm.

\question{Kust te sellest kõigest teada saite ja kuidas 2600d Eestisse 
sattusid?}

Kui said FidoNetis esimese ringi Eesti gruppidele peale ja 
lisaks tellisid mõne USA grupi või Usenetis häkkerigrupi, siis 
seal oli osa asju ASCII tekstina olemas. Paberkoopiat nägin 
USAsse minnes ja see oli šokeeriv kogemus. Ilmselt usklikel 
on mõne piibelliku teksti originaali 
juurde sattudes samasugune tunne. 

\question{Kas Ameerikasse sattusid vahetusõpilasena keskkooli ajal?}

Jah. Ma tegin avalduse Rotary klubi vahetusprogrammi, mille 
ankeet keskendus sellele, kes see konkreetne kooliõpilane on. Rotary 
vahetus oli kahesuunaline: klubi saatis kellegi kuskile välja ja 
samal ajal võttis mujalt vahetusõpilasi vastu. 
Ma ei tea, kas sellepärast, et mu ankeet oli nii arvutiasjade 
keskne, aga igatahes sattusin Eesti mõttes 11. klassi Silicon Valley keskele. Elasin
sellises õrnas eas aasta aega Cupertinos, kus on muu hulgas 
Apple'i peakontor, ja käisin Monta Vista High nimelises keskkoolis. 

Juhtumisi oli USAs just \emph{information 
superhighway} hullus puhkenud ja Al Gore oli aasta enne 
minu USAsse minekut kuulutanud viis kooli interneti 
pilootkoolideks. Monta Vista High oli üks nendest. Kool asus
Apple'i peakorterist paari kilomeetri kaugusel ja meil oli 1400 õpilast ja 800 arvutit. Selgus, et arvutilaboris 
assistendiks olemise eest saab ainepunkte ja ühe tunni asemel võis iga 
päev istuda arvutilaboris, kus olid Macid, Sunid ja Silicon Graphicsi 
asjad. Yahood kasutasin aadressil yahoo.cs.stanford.edu, sest see ei olnud siis veel 
firmaks muutunud.

\question{Ühesõnaga, sattusid paradiisi!}

Põhimõtteliselt küll jah. See mõjutas kindlasti tohutult seda, mis edasi 
sai. 

\question{Kuidas sa sellega toime tulid? Nõukogude 
liiduvabariigist sellisesse kohta sattumine võttis ilmselt jalust nõrgaks?}

Hea küsimus. See oli mu esimene lend Eestist välja. Üksi. Ilmselt vanemad 
pidid lennupileti jaoks raha laenama ja see kõik oli ka 
nende jaoks üsna hullumeelne. Aga eks vanemate 
asi ongi muretseda. Ise sellises vanuses lihtsalt teed ja lähed, 
oled nagu käsn ja võtad kõike seda sisse, mis tuleb. 

Koolikogemuse mõttes oli lihtne
see, et Eestist tulnuna ja matemaatika-füüsikaolümpiaadil käinuna olid sealses
11. klassis kõik mu reaalained 12. klassi \emph{honours}-taseme ained. Olin kõik selle Eestis juba läbinud -- koolisüsteemid olid nii palju erinevad. 
Teistmoodi oli see, et esimest korda elus pidin 
mitte üksiküritajana testi ära lahendama, vaid moodustama grupi kolme inimesega, 
kellega ma ei olnud enne koos töötanud -- midagi koos välja mõtlema ja klassi ees 
ette kandma. Eestis reaalainete tugevus oli olemas, aga selliseid asju pidin esimest korda tegema. Õppeviis oli seal 
ebamäärasem. 

Ühiskondade kontrast oli küll. Kui ma ise olin enne seda ju lausa tööl käinud, 
siis sinna jõudes küsisid vahel ka väga heasoovlikud klassivennad, kas meil 
Eestis telekaid on. Nende arusaam raudse eesriide taga 
toimuvast oli üsna hägune. 

\question{Kas käisid seal ka tööl?}

Ma ei tohtinud, vahetusõpilase värk. Alguses see kurvastas mind 
väga, aga siis sain aru, et seda defineeritakse läbi palga. Leidsin sellise lahenduse, et käisin 
ühes arvutipoes pärast kooli abiks, aga ma ei tohtinud palka saada, ja kui 
Eestisse tagasi kolisin, kinkis pood mulle arvuti. Aga mind käsitleti tõesti pigem nagu 
koolipoissi, kellel lubati arvutit kokku panna, kuna sealne IT-äri oli ju palju 
reglementeeritum võrreldes sellega, mis Eestis samal ajal toimus. 

Enne Ameerikasse minekut, 9. või 10. klassis juhtus veel selline asi, et tekkis rühmitus nimega Interactive 
Aspelungs\index{Interactive Aspelungs}, kuhu kuulusime mina, Mark Tehver\index[ppl]{Tehver, Mark}, 
Kristjan Jansen\index[ppl]{Jansen, Kristjan} ja natuke hiljem Alari 
Aho\index[ppl]{Aho, Alari}. Mark ja Kika\sidenote{Kristjan 
Jansen.}\index[ppl]{Kika|see{Jansen, Kristjan}} olid 
Treffneris\index{Hugo Treffneri Gümnaasium}, mina ja Alari Miina 
Härmas\index{Miina Härma Gümnaasium}. Miks ma 
ei ole tänapäeval programmeerija, oli see, et mul õnnestus väga õrnas eas näha 
lähedalt inimesi, kes tegelikult oskavad programmeerida. Mark oli juba koolipoisina selline inimene, kes hommikul hakkas tekstifaili assemblerit 
kirjutama ja õhtul pani selle käima ning see töötas. Näiteks 
graafikamootor. Meil tekkis selline sümbioos, et 
Mark proges, Kika disainis, Alari tegi muusikat ja mina korraldasin igasuguseid asju, näiteks laenasin Primexist SoundBlasteri kaardi, et Mark 
saaks ka sellele audiodraiveri kirjutada. Toode, 
mida me ehitasime, oli arvutimäng \enquote{Drunkard}\index{Drunkard}.

Mina hakkasin seda mängu maha 
müüma. Olin tihedas kirjavahetuses selliste ettevõtetega nagu Epic Megagames\sidenote{Praegu tuntud kui 
lihtsalt Epic.} ja Apogee\sidenote{Praegu tuntud kui 3D Realms. Mõlemad mainitud ettevõtted olid omal ajal 
mängutööstuse absoluutsed gigandid.}, kes olid valmis meiega rääkima. 
Kui ma USAsse läksin, siis tekkis veider olukord, et sain USA 
postiaadressilt saata flopiga demosid ja jätta mulje, nagu meil oleks 
päris firma. 

Paraku ei teinud me seda mängu kooli kõrvalt lõpuni valmis, ainult demod olid 
olemas. Panime ka paari aastaga selles suhtes mööda, et kirjutasime 2D 
platvormikat, mis 1991. aastal oleks olnud selle taseme juures, mis 
Mark\index[ppl]{Tehver, Mark} ja Kika\index[ppl]{Jansen, Kristjan} välja
töötasid, ilmselt täiesti vabalt müüdav, nii nagu 
Bluemooni\index{Bluemoon} kutidki oma mänge maailma viisid. Aga meie 
komistasime täpselt sinna hetke, et kui saatsime demosid, siis ID 
Software \enquote{Wolfenstein}\index{Wolfenstein 3D} oli juba 
väljas\sidenote{Avaldati 5. mail 1992.} ja \enquote{Doom}\index{Doom} 
tulemas\sidenote{Avaldati 1993. aastal.}. Ehk graafika tase jõudis sinna, kus Tartu koolipoiste 
platvormikas ei paistnud enam säravalt silma. Aga põhimõtteliselt see asi isegi töötas.

\question{Ise mängu kirjutamine tundus toona täiesti hoomamatu ettevõtmine: 
muusika mängib taustal, kuidas sa seda teed?!}

Tundub isegi tänapäeval, aga tollal alustas iga mängukirjutaja sellest, et 
proges endale töövahendid. Nelja või 
kaheksa kanaliga \emph{tracker}'id taustmuusika tegemiseks olid olemas, mida jällegi 
Bluemoon\index{Bluemoon} tootis, ent \emph{level}'i disainimiseks või isegi 
animeerimiseks tööriistu ei olnud. Pildi-\emph{editor}'iga tegid spraidi\sidenote{Ingl 
\emph{sprite} -- kahemõõtmeline ühik rastergraafikat, mis integreeritakse 
suuremasse stseeni.} valmis, aga selle animeerimiseks pidid 
jälle oma töövahendi tegema. Meil oli kogu see \emph{stack} olemas.

\question{Miks te seda tegite? Kas tundus äge või tahtsite rikkaks saada?}

See tundus lihtsalt äge. Mark\index[ppl]{Tehver, Mark} ja 
Kika\index[ppl]{Jansen, Kristjan} tegutsesid juba enne, neil oli hoog 
sees. Mitu aastat tagasi kogus Kika kokku kõik meie tollase 
kirjavahetuse ja pildifailid ning pani avalikult internetti -- see oli hea
nostalgiarännak. 

Mängu peategelasel Drunkardil oli oluline, et alkoholitase 
veres ei langeks. Selleks pidi ta korjama pudeleid ja siis ta sai 
tühjade pudelitega loopida. Olid ka teatud olukorrad, kus ta pidi ajutiselt 
saama natuke kainemaks. Markil\index[ppl]{Tehver, Mark} ja 
Kikal\index[ppl]{Jansen, Kristjan} oli Treffneris üks klassivend, kes, kui ta 
koolipeol liiga palju õlut jõi, hakkas kükke tegema, et kainemaks saada. Ja 
Drunkardiga oli ka nii: allanoolt all hoides tegi ta kükke ja alkoholitase veres langes.

\question{Mingi loovuse element oli sees, sest ei tulnud lihtsalt joosta ja 
kirvestega loopida?}

Täpselt, meil oli idee teha selgelt teistsugune mäng, kus ei käi 
tulistamine ja tapmine. Ma ei mäleta, kui 
teadlikult me seda mõtlesime, et sihtida täitmata turunišši ja teha nii-öelda 
täiskasvanute mäng. Muidugi alaealiste idaeurooplaste arusaam sellest, mis 
on \emph{adult entertainment}, oli mõnevõrra teistsugune kui päris 
\emph{adult}'idel, aga vastas selgelt üheksakümnendate Tartule.

\question{Kas tegid Californias veel midagi huvitavat?}

Seal ma progesin ka. Ma ei mäleta, kas ma BBSi pidasin, aga 
eraisikuna koduliini peal BBSi pidamine ei ole nagu see, 
pluss satud USA telefoninumbrite ruumi. Tegelikult olin ikkagi pigem 
BBSide kasutaja. Ka graafiline veebibrauser ilmus orbiidile ja pilt hakkas 
muutuma. 

Kirjutasin hobiprojektina ka ühte BBSi softi, kuna tundus, et see 
võiks olla vajalik. Kui parajasti mängu ei proge, siis ju ikka mõtled, mida endal vaja on. Selle käigus 
uurisin, mis veel turu peal saada on, ja leidsin ühe ägeda BBSi softi, mis oli \emph{shareware} ja kinni keeratud. Ma murdsin selle lahti -- 
noorele häkkerile ei midagi komplitseeritut. Jooksutasin seda 
\emph{debugger}'is ja vaatasin, et ootamatute kohtade peal hüpatakse mälus 
ühele aadressile ja tehakse seal mingi väga lihtne tehe. Muutsin masinkoodi 
tasemel ära, et sinna enam ei hüpataks, ja ootamatult selgus, et oligi 
koopiakaitse maas. Raporteerisin sellest autorile ja ta andis mulle eluaegse tasuta litsentsi. See võttis mul oluliselt 
motti maha oma BBSi softi kirjutada, sest mul oli see nüüd olemas. 

Sain ka õudselt hea õppetunni enda tasemel progejana. 
Ma kohutavalt abstraheerisin selle asja üle: kirjutasin C++-s\index{C++} 
BBSi, kus katsusin hoida väga puhtalt eri kihtidena näiteks seda, kuidas 
käib modemi ja terminali händlimine, olles valmis selleks,
et tulevikus võib olla asju, millega liidestuda, mitmeid. 
Ühesõnaga ma olin kogu aeg jube kaugel sellest, et asi töötaks 
minimaalses skoobis. Hilisema elu tarkvara{\-}arendusprojektideks oli see väga hea 
õppetund. Täna tegelen rohkem \emph{start-up}'idega, MVP\sidenote{\emph{Minimum viable product} -- vähim elujõuline toode.} on kuidagi 
armsam. Pigem teha vähem, aga varem.

\question{Ja ühel päeval tulid Ameerikast tagasi, kaasas arvuti ja raamatud?}

Jah, paar sellist kasti oligi. 

\question{Mida sa siis tegid?}

Läksin keskkooli 12. klassi. Kuna mul oli juba natuke hoog sees, 
siis asusin ka tööle sellisesse Tartu firmasse nagu Triip\index{Triip}, mis oli 
algselt trükikoda ja disainibüroo. Mulle on eluaeg, isegi siis, kui 
progesin, meeldinud see, kuidas tehniline ja visuaalne osa kokku saavad. 
Olen alati töötanud koos progejate 
ja disaineritega, ka hiljem.

USA perioodist meenub veel üks asi, mille me tegime. Mäletad, olid kunstirühmitused 
kes tegid ASCII \emph{art}'i ja hiljem VGA \emph{art}'i, ning koos ühe koolivennaga me komistasime
varjunimede all paari neist sisse. Minu ASCII ja ANSI \emph{art} on isegi olnud mõnedes distributsioonipakkides. USAs oli üks kooliaine \emph{commercial art} --
tootedisain ja pakend. Ilmusin nende näidistega Triipu ja ütlesin, et 
tahaksin pärast kooli natuke arvuti taga istuda, mis tähendas 
disainimist, ja siis sattusingi sinna tööle.
 
Teine mind hästi palju mõjutanud inimene tol ajal oli Marek 
Tiits\index[ppl]{Tiits, Marek}, kes pidas Balti Uuringute 
Instituuti\index{Institute of Baltic Studies}, mille alt ta tõstis\sidenote{Raha \enquote{tõstmise} all peetakse \emph{startup}-kogukonnas silmas mingile ettevõtmisele rahastuse hankimist.} edukalt 
eurorahasid ja tegi ägedaid projekte, näiteks Eesti seaduste 
otsingumootori.

Marek andis ka mulle kui lihtsalt ringi hängivale 
koolipoisile võimaluse tulla ja aeg-ajalt midagi teha. See tähendas 
ligipääsu tähetorni\index{Tartu tähetorn} arvutitele. Seal oli üks Silicon Graphicsi\index{Silicon Graphics} 
arvuti, millel oli veebikaamera. Üheksakümnendate keskel! Arvutil oli oluline 
funktsioon: sinna oli võimalik sisse logida ja vaadata veebikaamerast, kas 
kohvimasin on täis jooksnud, et ei peaks tühja tassiga alumiselt korruselt 
teisele tulema. Samuti oli seal üks Zyxeli\index{Zyxel} modem, 
millel oli ka faksi funktsionaalsus, ja internet. Ma 
kirjutasin Perlis\index{Perl} veebipõhise faktide saatmise ja vastuvõtmise programmi: kui keegi saatis sellele numbrile faksi, siis võttis 
modem vastu, kirjutas failid Suni serverisse maha ja üle veebi oli 
võimalik neid näha. Ma ei ole kindel, kas mõtlesin ise, et võiks selle teha, ja Marek ütles, et tee muidugi, või oli see 
mõne projekti jaoks vajalik.

\question{Veel aastaid hiljem räägiti ühes suures ettevõttes, mis 
kindlasti ei olnud telekommunikatsiooniettevõte, et äge oleks 
üle interneti faksi saata!}

Minu arust kaks aastat tagasi\sidenote{Ajasime Steniga juttu novembris 2019.} 
tegid Twilio\index{Twilio} progejad aprillinaljana Twilio faksi API ja 
nüüd on see oluliselt kasvav ärisuund, sest võrgus on 
miljoneid inimesi, kes tahavad kogu aeg faksi saata. Mõtle, mis kõik oleks 
võinud olla! 

\question{Räägi palun Triibust.}

Tundsin sealt Jussi, Juhan Peedimaad\index[ppl]{Peedimaa, Juhan}, Evat, 
kes on nüüd ka Peedimaa\index[ppl]{Peedimaa, Eva}, ja Priit 
Jagomäge\index[ppl]{Jagomägi, Priit}. Jagomägede perekond on kuulus Regio ja 
kartograafia poolest, aga Priit oli selline \emph{rebel} vend, kes tegi oma firmat,
mitte ei töötanud Regios. 

Päris mitmed asjad Eestis juhtusid sellepärast, et 
meil oli täiesti tühi maa. Kui ma keskkooli 
lõpetasin, oli Eestis umbes 40 panka ja keskmine panga CEO vanus oli umbes 28. 
Kaheksateistaastaselt tekkis tunne, et oled mingist rongist maha jäänud. Tol hetkel me veel ei teadnud, et 
maandume tehnoloogia- ja internetiettevõtluse lainele. Aga oli selge, 
et see rong ei olnud veel läinud ja et saame endale ise rongi ehitada. 

Triip\index{Triip} oli täpselt selline ettevõte. Nad alustasid sellest, et kuna Eestis oli 
täiesti tühi maa, iga päev tehti kümneid firmasid ning igale firmale oli vaja logo 
ja visiitkaarti, siis hakkasidki neid tegema. Disainist tuli nii palju raha peale, 
et osteti oma trükikoda ja veel mingisuguseid asju kokku, ning siis selle 
sisse komistasime internetiasjaga. Kui algul kujundati trükiseid ja 
reklaame, siis mina hakkasin samadele klientidele ka veebilehti tegema. Ja 
kui ma keskkooli lõpetasin, siis läksin Triibust 
ära ja tegin oma esimese ettevõtte 
Halo\index{Halo Interactive DDB}, mis tegeles veebiga \emph{full time}. 

Halo aegadest oli ilusamaid mälestusi näiteks see, et ühel hommikul läksin 
kontorisse ja nägin, et üks disainer poeb valgete linade vahelt välja --
selgus, et ta tõsteti juba kolm kuud tagasi ühikast välja. Lihtsalt mina 
arvasin, et ta on kogu aeg esimesena ööl ja lahkub viimasena. Tegelikult ta 
elas seal.\sidenote{Lugu vastab tõele ja see 
ei olnud ainus juhtum, kus Halo kontor kellelegi ajutist peavarju 
pakkus. Üheksakümnendate Tartu oli mõnevõrra teistsugune \emph{start-up}'i-keskkond kui 
selle sajandi kahekümnendate Tallinn.}

\question{Sa rääkisid, et oli võimalus nii-öelda oma rong ehitada. 
Kas sul oligi selge visioon lainest ja endast selle peal või lihtsalt 
tegelesid asjadega, mis tundusid mõnusad?}

Ettemõtlemist oli selgelt liiga vähe. Halo\index{Halo Interactive DDB} läks nelja 
aastaga pankrotti, sest teatud piirini on võimalik asju ehitada intuitsiooni 
pealt, aga ühel hetkel peaks läbi arvutama ja mõtlema, mida sa 
teed. 

See oli mingil määral avastus, et veebilehed on asi, mida vajatakse. 
Juba Triibus hakkasid kliendiks saama ja Halos olid 
klientideks kõik suured pangad, Eestisse jõudnud rahvusvahelised 
brändid, nagu Audi, ja ESS, mis tol hetkel tohutult 
kasvas. Sellised kaubamärgid, mida kõik teadsid. Ja internetivärk 
oli nende jaoks arusaamatu. Meie jaoks oli see intuitiivne ja lihtne -- 
milles probleem, teeme ära! -- ja suured kaubamärgid olid 
nõus alla neelama selle, et ostavad mingite kaheksa- ja 
üheksateistaastaste Tartu kuttide käest teenust. Ei osanud me seda hinnastada 
ega midagi, iga kord mõtlesime, et see number kõlab liiga 
suur. Alguses kõik toimis liiga hästi või liiga lihtsalt. 

Alustasime Haloga\index{Halo Interactive DDB} 1996. aastal, 1997. aasta kevadel ostis üks suur reklaamikett, 
DDB, kontrolli Halos ära ja me sattusime ootamatult päris 
ärikeskkonda, kus olid päris inimesed, mingid Soome juhid, kes olid nõukogus ja 
tahtsid eelarveid näha. Ühest küljest oli 
see kõik nagu varases nooruses kokkupuude päris asjadega. Teistpidi läksid ka 
need reklaamiinimesed ehk investorid
parajasti internetibuumi sisse. Ka nende kliendid loopisid raha vasakule ja 
paremale. Näiteks selline agentuur nagu Razorfish võttis  esimese 
kohtumise eest 100 000 dollarit. See tähendas, et nemadki ei 
suutnud kõike ette näha. 

Me näiteks palkasime selgelt liiga palju inimesi, 
sest kohe-kohe pidid lääne kliendid tulema Eestisse asju arendama -- meil olid 
tõsised jutud DDB keti sees. Ja kui see mull 2000. aastal 
lõhkes\sidenote{NASDAQ Composite indeks tõusis 1995. ja 2000. aasta märtsi 
(mida loetakse mulli lõhkemise hetkeks) vahel 400\% ning kukkus 2002. aasta 
oktoobriks 78\%, kaotades kogu mulli jooksul kogutud kasvu.}, siis 
kõigi jaoks korraga: lõhkes meil Eestis ja lõhkes seal. Ja seda ei suutnud
me täiesti roheliste tüüpidena ette näha. Et majanduses on 
tõusud ja langused, oli täielik müstika. 

\question{Sest sa ei olnud ühtegi varem näinud!}

Ei olnud. Ja mitte ühelgi hetkel ei olnud seal tunnet, et oleme 
ettevõtjad või \emph{start-up}. Tegime asju, mida oskasime ja 
mille jaoks oli tõmme turult ja nõudlus. Homme 
küsitakse multimeedia CD-ROMi -- \enquote{oh, teeme neid ka!} 

\question{See multimeedia CD-ROM\sidenote[][-.9cm]{Tänapäeval küllalt raskesti seletatav 
asi. Kujutage ette, et teil on interaktiivne veebileht, mis sisaldab videoid, 
muusikat ja teksti, aga ei ole kättesaadav mitte läbi brauseri ja interneti, 
vaid levib CD-plaadi kujul ning on realiseeritud täiesti teistsuguse 
tehnoloogia abil.} oli täiesti müstiline asi. Kas tõesti tuli Ühispank\sidenote{Üks 
Halo suuremaid projekte oli Ühispanga\index{Ühispank} aastaraamatu väljaandmine CD-l.} seda meie käest küsima? Andres Aarma\index[ppl]{Aarma, 
Andres}\sidenote{Andres tegeles tol ajal Ühispanga avalike suhetega.} samas 
võis tulla küll \ldots}

Kaks tükki me neid suuri tegime: ühe Ühispangale 
ja selle lainel müüsime teise Eesti Telekomile\index{Eesti Telekom}. 
Idee seisnes selles, et kuulge, varsti on aasta 2000, miks te oma 
aastaraamatut paberile trükite! Mäletan, et üks neist 
maksis selle eest 200 000 Eesti krooni. Täna ei saa 15 
000 euro eest ühtegi progejat liigutama.

\question{Tagantjärele mõeldes hämmastab
see, et me ei teadnud midagi versioneerimisest, testimisest ega üldse 
mingist süsteemsest tarkvaraarendusest ja siiski suutsime punuda 
softi, mis enam-vähem töötas. Ja meil oli häbematust see kliendile tarnida ja 
klient maksis arved ära!}

Olen sellele paar korda mõelnud. Üheksakümnendate internetilainel räägiti kogu aeg uuest majandusest ja hästi palju oli neid kohti, 
kus arvati, et uus majandus ei allu vana majanduse reeglitele. Mõnes asjas ei 
allu ka, nagu turu suuruse mõõtmine või füüsiline kaugus, aga põhiasjades, näiteks et tulud võiksid ületada kulusid, allub. Samamoodi 
vastandati e-kommertsi ja päris kaubandust ning kõike käsitleti 
eraldi. Me tegime mitu aastat seda äri selles suhtes 
valesti, et mõtlesime, et veebiehitamine ei ole tarkvaraarendus. 
Tegelikult ei olnud meil tiimis kedagi, kes oleks mõelnud veebisaidi 
ehitamisest nagu tarkvaraarendusprojektist või lugenud läbi mõne raamatu, kuidas suuri projekte teha. Veebivärk tundus 
nii lihtne. Kui kogemata ehitad sinna taha ka sisuhaldussüsteemi, 
mida me üritasime tootestada, siis sellel hetkel oleks pidanud hakanud teadlikumalt tarkvara arendama. 

Tagantjärele arvan, et see, mida Taavi Kotka\index[ppl]{Kotka, Taavi} 
Webmedias\index{Webmedia}\sidenote{Praegu Nortal.} tegi -- nad hakkasid 
mõtlema oma tarkvaraarenduse protsessi peale --, hoidis neid elus. 
Nad alustasid meist natuke hiljem, aga hammustasid kiiremini läbi, et kui 
lähevad suurt maksuameti infosüsteemi tegema, siis ei peaks seda veebilehena 
käsitlema. 

\question{Jah, me ehitasime ikkagi veebilehti \ldots}

\ldots millel oli kogemata mingeid skripte taga.\sidenote{Kuna andmebaas ja 
selle haldus tundus keeruline, siis hoiti kõike tekstifailides, 
mida muudeti Perli abil.}

\question{Kas sa olid tol ajal ka Tiigrihüppega\index{Tiigrihüpe} seotud?}

See oli ka üks kooliaegne asi. Tartus oli füüsika 
instituudi\index{Tartu Ülikool!Füüsika Instituut} füüsikute seas Jaak 
Aaviksoo\index[ppl]{Aaviksoo, Jaak}, kellega mu isa käis samal ajal 
ülikoolis ja kes oli ka Miina Härma\index{Miina Härma Gümnaasium} või 
2. keskkooli vilistlane. Kui Tiigrihüpet tegema hakati, siis Toomas Hendrik 
Ilves\index[ppl]{Ilves, Toomas Hendrik} ja Jaak Aaviksoo\index[ppl]{Aaviksoo, 
Jaak} mõtlesid selle välja. Nagu nad ise räägivad, kolmekesi: Ilves, Aaviksoo 
ja Johnny Walker. Mõtlesid selle oma rollides välis- ja 
haridusministrina välja ning moodustasid selle ümber Tiigrihüppe peakomitee. 
1996. aastal, kui ma olin keskkooli viimases klassis, kutsusid nad mind sellesse 
komiteesse õpilaste esindajaks -- iseenesest ilus žest, 
haridusprojekti tehes võiks mõni õpilane ka sellega seotud olla! Sõitsingi Tartust bussiga Tallinnasse haridusministeeriumisse 
koosolekule, kus olid Peeter Marvet\index[ppl]{Marvet, Peeter}, Marju 
Lauristin\index[ppl]{Lauristin, Marju}, Ants Sild\index[ppl]{Sild, Ants} ja 
teised korüfeed laua ümber. Istusin seal 
lihtsalt vait ja kuulasin, see tundus täiesti müstiline. 

Minu suurim panus Tiigrihüpesse oli see, kui käisime otsesaates, kus toimus Tiigrihüppe 
teemal väitlus: Marju Lauristin\index[ppl]{Lauristin, Marju} \emph{versus} 
Lauri Leesi\index[ppl]{Leesi, Lauri} ja mina. Olin keskkoolipoisina otse-eetris, esimest korda 
televisioonis, ja minu vastas oli kaks oma 
valdkonnas tunnustatud inimest. Kuulasin, kuidas Lauri Leesi 
jahus, et arvutit pole kooli vaja ja et krihvel ja tahvel 
on aastasadu vastu pidanud, ning sain aru, kui läinud rong tema
jaoks on, sest ma ise juba tunnetasin, kuhu maailm liigub ja mis juhtub. Umbes sellest hetkest tekkis mul ka tunne, 
et ma ei lähe ülikooli arvutiteadusi õppima, kuna ma juba progesin, pidasin
BBSi ja hängisin internetis, ja et tegelik lünk on see, miks need 
asjad toimivad. Võrgustikud ja sotsiaalne pool. Saatest lahkudes ütles Lauristin mulle: \enquote{Kuule, me teeme 
Tartusse uut eriala, kus hakkame kommunikatsiooniteooriat ja asju õpetama -- 
tule sinna.} Nii ma sattusingi sotsiaalteadusi õppima, mis 
oli paljude ja ka minu enda jaoks suhteliselt üllatav.

\question{Üheksakümnendatel juhtus igasuguseid üllatavaid asju!}

Jah, võrreldes kõigi nende progejatega, kes olid teoloogid, läks mul isegi 
hästi.

\question{Millega sa praegu tegeled? Kuhu see tee sind tänaseks on toonud?}

Põhiliselt ehitan ettevõtteid, pigem nende algfaasis. 
Mõnes mõttes võib öelda, et teen täpselt sama, mida me üheksakümnendatel 
tegime, aga nüüd juba iga kord uue kogemusega ja
natuke tean ka, mida ma räägin. 

Hakkasin viis 
aastat tagasi tegema sellist ettevõtet nagu Teleport ja kaks aastat tagasi ostis Topia
meid ära ja nüüd tegutsen seal edasi. Asutajana tekkis see hetk, et kas tahan suurt tükki 
väikesest pirukast või väikest tükki suurest pirukast, ja Teleporti müük 
Topiale andis võimaluse paar aastat vahele jätta ning hüpata trepil 
kõrgemale. Kaheteist inimesega \emph{start-up}'ist sai 170 inimesega tiim, mul on tootearenduses 70 
inimest, kellega saab sama visiooni kiiremini ehitada. 

Ja kui mul aega üle on, siis ka investeerin \emph{start-up}'idesse ja annan mõne
nõu. 

Sarnasus üheksakümnendatega on see, et ma tean täpselt, mida ma suudan üksi 
ehitada -- on see siis BBSi soft või mitte --, ning mis väärtus ja võlu 
on sellel, kui progeja, disainer ja äriinimene koos midagi valmis teevad, alates mängust \enquote{Drunkard} ja 
Halo veebiprojektidest. Tean ka, et ma ei taha kunagi elus müüa oma aega 
tunnipõhiselt, sest tunde on lõplik hulk. Järelikult tuleb ehitada tooteid. 

See läheb küll nüüd meie ajaraamist välja, aga hiljem 
Skype'i\index{Skype} puhul oli näha, kuidas paar inimest ehitavad mõne kuuga asja, 
mida kuu aega hiljem kasutab miljon inimest. Üheksakümnendate 
rahmeldamine seda võibolla õpetaski ja nüüd otsin alati kohta, kus 
sissepandud töötundide hulk konverteeruks võimalikult suureks väärtuseks. Tol 
hetkel kippusin väga palju õhinapõhiselt tegutsema ja 
oleksin kõiki neidsamu asju teinud ka siis, kui ei oleks palka saanud. Kui 
müüd oma töötunde, siis on sul lihtsalt väga pikad päevad ja väga lühikesed 
ööd. Üheksakümnendatel põles osa tüüpe ka kohe läbi, osa põles hiljem. 

Austan sellest ajast väga näiteks
Taavi Talvikut\index[ppl]{Talvik, Taavi}, kes ehitas sama tausta ja huvide 
pealt Unineti\index{Uninet} ning võib ka magada, sest bitid müüvad 
ennast ise. See oli küll teenuseinfrastruktuuri äri, aga seal oli see alge 
olemas, kuidas ehitada midagi, mis saab hakkama ka siis, kui ei ole ise kogu 
aeg näppupidi juures.

\chapter{Veiko Tamm}
\label{chptr:lucifer}
\index[ppl]{Tamm, Veiko}

\question{Hakkame peale sealt, kust asjad ikka pihta hakkavad. Kuidas arvutit 
sinu juurde said?}
                 
See on komplitseeritud küsimus. Ma ise olen ülikoolis keemia eriala\index{Tartu 
Ülikool!Keemia Instituut} lõpetanud. Aga ei oma neljanda kursuse kursusetööd ja 
diplomitööd ei ole ma ühegi kolvi ega katseklaasiga solberdanud, sest  mind 
kutsuti tollal üsna põnevasse asja. Nimelt juhendajaks oli mul Mati 
Karelson\index[ppl]{Karelson, Mati}, kes alustas arvuti ja kompuuterkeemiaga,  
ühesõnaga kvantkeemiaga.  Ja sealt siis olid minu esimesed kokkupuuted. Meie 
tööväljaks olid alguses perfolindid alguses, viieaugulised.

\question{Juba tol ajal ta üritas teha kvantkeemiat? Nonde arvutitega?}   

Jah, see asi hakkas pihta juba kaheksakümnendatel.
                 
\question{Ja sul enne seda üldse mingit kokkupuudet arvutitega ei olnud?}

No meil matemaatika kursuse käigus väga lühidalt näidati selliseid arvuteid 
nagu Nairi-2\index{Nairi!Nairi-2}, ja nende PA ja AP keel\sidenote{Vt lk 
\pageref{sisu:apkeel}.}. Sai endale \enquote{Hello World!} trükkida ja 
\emph{that's it}. 

Aga no, ütleme, need masinad, millega me hiljem töötasime\ldots Alguses oli 
Minsk-32\index{Minsk!Minsk-32} ja hiljem  Jessukesteks\index{Jessuke|see{ES 
EVM}} kutsutud ES-1022 arvutid\index{ES EVM!ES-1022}, mis olid Ülikooli 
Arvutuskeskuses. Arvutamine käiski nii et  perforeerisid oma programmi sisse, 
viisid sinna ja tema lahendas. 

\question{Mida need programmid tegid?}

Oli mitmesuguseid kvantkeemia meetodeid. Aatomorbitaalid, kuidas need 
keemilised sidemed moodustuvad, kuidas elektronpilved suhtuvad üksteisega ja 
kas see aitaks seletada neid keemia asju, kas  lähedaks arvutuslikke tulemusi 
reaalsetele. Ja need arvutused oli ikka nii, et võtsime lihtsad kahe aatomiga 
molekulid ja nendega oli nii,  nagu on. Aga keerukamate, näiteks seal metaan 
\ce{CH4},  molekuli välja arvutamine nõudis  sellelt suurelt arvutilt umbes 10 
korda rohkem tööd kui terve lihakombinaadi aastaaruande välja arvutamine. See 
oli päris kõva ja kallis arvutiaeg, mis sinna alla läks.
           
\question{Kuidas see käis? Juhendaja ütles \enquote{Veiko, hakka 
programmeerima} ja sa hakkasid programmeerima?}      

Ei, programmeerimisest jäi asi ikka kaugele seal. Tuli natukene keeltega 
pusserdada küll,  põhimõtteliselt tuli mõningaid asju Fortranis\index{Fortran} 
kirjutada ka. Et midagi ma nagu aru sain, aga seda programmeerimise asja ma ei 
ole selgeks saanudki. 

Siis katkes see asi tükiks ajaks ära. Taaskohtumine arvutitega oli, nüüd võiks 
öelda juba üle 31 aasta tagasi, kui Tartu Tähetornis\index{Tartu Tähetorn} ajas 
tol ajal suur arvuti-fänn Enn Kasak kokku arvutihuviliste ringi ja hankis sinna 
arvuteid.

\question{Mis aastal see oli?}

See oli 1988. aasta suvel. Sel ajal algas ju suur kooperatiivide ajastu. Sai 
tehtud kooperatiiv Tähetark\index{Tähetark}, mille liige ma olin ja mille 
eesmärk oli hankida planetaarium. See planetaarium sai isegi ostetud ja  
Füüsika Instituudi\index{Tartu Ülikool!Füüsikahoone} fuajeesse kuidagi üles 
säetud, aga ega temaga mingit tööd tegema ei hakatud. 

Aga samal oli see selline teadlik ja teaduslik keha,  mille varjus sai arvuteid 
osta-müüa. Arvutite müügiga oli üldse see asi, et hiljem keelati igasugustel 
väikestel asjadel see ära. Kui sa oled kunagi näinud vene seriaali 
\begin{russian}Бригада\end{russian}, segaste aegade maffia-elust, kus äritseti 
kõigega. Juhtuski niiviisi, et ma sattusin  olukorda, et mine ja too mingeid 
arvuteid. Käisin Peterburis, Moskvas, sain kätte otsekontaktid ja sidemed ja 
nii see arvutiärisse sattumine nagu järsult puhkeski. Sest arvutid olid tol 
ajal ikkagi hirmsalt kallid asjad. Minu esimene arvuti näiteks, Amiga 
500\index{Amiga!Amiga 500},  maksis sama palju, kui tutikas 08\sidenote{Nii 
kutsuti autosid VAZ-2108, tuntud ka kui Lada Samara. Tegu oli oma aja ja 
Nõukogude Liidu kohta innovatiivse autoga, Samara oli teine (esimene oli siiani 
populaarne Niva) ise arendatud AvtoVAZ-i mudel ja esimene, mis ei tuginenud 
Fiat 124 mehaanikale.}. No ikka mingi 40 000 rubla.
                 
\question{Miks sa sinna Tähetorni juurde läksid, arvutivärk ikka nagu tõmbas 
või?}

Sõber kutsus, et tule kaasa, põnev värk selline.

Ja siis ma mäletan jah, kuidas tuli algusest peale endale kõik see värk selgeks 
teha. Kuidas vaadata, et palju tal op mälu on ja testiprogramme kasutada. 
Alguses kuiva trennina, aga järjest hakkas see asi nagu liikuma. Ja kui endale 
sai arvuti koju ostetud,  oli jube põnev. Alguses ma rändasin  üle Amigate.  
500\index{Amiga!Amiga 500}, siis 500 Plus\index{Amiga!Amiga 500 Plus}, siis 
Amiga 1000\index{Amiga!Amiga 1000}, Amiga 2000\index{Amiga!Amiga 2000}. Ja siis 
sai sinna kõrvale esimene PC, sest tuli üks suurem tellimus. Arvuteid saada oli 
kuradi raske, sellepärast et igal pool olid embargod nende sisse vedamiseks ja  
põhimõtteliselt tõime arvuteid Moskvast ja 
Peterburist.\label{sisu!veiko_moskvas}

\question{Kuidas need arvutid Moskvasse ja Peterburi said?}

No näiteks see Moskva kanal, mis mul oli ja millega ma embargo-arvuteid tõin ja 
äritsesin, tuli läbi saatkondade. Singapuri saatkonnas toodi paljundusmadina 
kastis arvuti saatkonda ja siis Lumumba\sidenote[][-3cm]{Moskvas asutati 1960. aastal 
Vene Rahvaste Sõpruse Ülikool (\begin{russian}Российский университет дружбы 
народов\end{russian}), mis 1961. aastal nimetati Kongo poliitiku ja 
vabadusvõitleja Patrice Émery Lumumba auks ümber Patrice Lumumba Ülikooliks. 
Kooli eesmärk oli toetada vastselt koloniaalsõltuvusest vabanenud riike 
koolitades sealset tulevast teadus-tehnilist eliiti. Praktikas oli tegu Moskva 
ühe vähese ülikooliga, kus õppis suurel määral välisüliõpilasi.} üliõpilaste 
kaudu läks see kohalike ärikate kätte ja sealt sain siis mina osta. 

\question{Sihuke tarneahel!}

Jaa, tarneahel oli päris võimas, kusjuures hinnad tarneahelas liikusid väga 
põnevalt. Iga ots pani omale ikka julgelt kuskilt seal 30 kuni 50 protsenti 
otsa, aga väga sageli ei viitsinud hakata liigutamagi, kui 100 protsenti 
kasumit ei olnud. 

\question{See oli ju riskantne äri?}

Oli jah, tean ikka väga palju tuttavaid, kes said kuuli ja nuga. Olen isegi 
kihutanud öises Peterburis punase tule alt läbi, sabad järel ja kõik. Kui sa 
ikka sõidad sinna kolm miljonit rubla sularahas seljakottidega kaasas, siis ta 
on riskantne.

Võtame ülekande rubla, vene ajal nimetati \begin{russian}песналик\end{russian}. 
Arvuti hind ülekande rublas oli kaks miljonit. Kui sa tõid sularahas selle 
raha, said arvuti kätte, ütleme, 1.4 miljoniga. Aga kui sa maksid valuutas, 
võisid üldse  ühe miljoniga kätte saada ümberarvutatult väärtustes. Ja teine 
lisaväärtus, mis tuligi  Moskvaga tuuritades oli see, et Venemaal, Moskvas, 
süva-venemaal, hinnati saksa marka ja  dollarit. Meil siin jälle, vastupidi, 
olid hinnas Rootsi kroonid, Soome margad, kellega ärikad äri ajasid. Ja siis 
oli nii, et ostsid siit kokku saksa margad, ostsid kokku dollarid, läksid 
Moskvasse, maksid nendega ära. Ja kõik need saksa margad ja dollarid, mis üle 
jäid,  vahetasid seal Soome markadeks ja Rootsi kroonideks. Ja tulid siia ja 
vahetasid ära ja üksinda selle ülejäänud nii-öelda valuutavahetuse eest võtsid 
ka rahulikult iga raksu pealt seal mingisugune sada tuhat vahelt. 

See oli hirmus aeg. Siis eriti enam ei kütitud, kuigi nõuka aja lõpuni kehtis 
ju  valuutaseadus. Et suurtes hulkades valuuta äritsemise eest võisid saada  
seitse aastat. Suureks hulgaks loeti juba seda, kui sul oli rohkem kui 100 
dollarit. Aga seal sai tuhandetega arvestatud.

\question{Kes neid arvuteid ostis?}

Oh jumal, kõik ajasid taga. Riiklikud ettevõtted, instituudid\ldots  Kui ma oma 
esimese PC arvuti ostsin, siis see läks Tartu Ülikooli 
Füüsika-Keemiateaduskonna pea serveriks. Ja see oli 286, 20 MhZ. Võimas masin, 
kahekümne megahertsine!. Tavalisel masinal oli ju ainult 12 MhZ. Tal oli kaks 
megabaiti op mälu ja 120-megane kõvaketas. Ja teise sellise arvuti ma ostsin 
endale. Muidugi käisid paljud tuttavad, kes ka arvutitega tegelesid, vaatamas, 
et \enquote{Mida sa, loll, tast endale ostsid, mis sa teed selle arvutiga, kuna 
sa selle 120 mega arvad täis saavat?}. Vot sellised ajad olid. Mulle siiamaani 
meeldib, Enn Kasaku\index[ppl]{Kasak, Enn}  üks selline paralleelne näide, mida 
ma olen pidevalt kasutanud. Et kui auto-teadus oleks samamoodi arenenud, nagu 
arvutiteadus, siis sõidaks Mercedes praegu valguse kiirusega, võtaks  10 000 
kilomeetri peale tilga bensiini ja maksaks pool senti.

\question{Tundub tõepärane. Mis tolle 286 serveri peal jooksis? Novell?}

Tead, ma isegi ei mäleta, mis nad sinna panid. Põhiliselt, mida jooksutati, 
olid ikka Unixid. V5\index{Unix!System V} näiteks ja kõik sellised asjad. Siin 
ikka käis rahvast meie teadlastel välismaalt külas ja kõik väga imestasid seda, 
et nii palju tehakse Unixitega. Aga kõigi Eesti Unixite seerianumber oli üks. 
Ega peale piraatluse muud võimalust ei olnud. Tarkvara hindade juures, kes 
andis mingi tarkvara jaoks sellist raha!

\question{Ühesõnaga, kuna ei ühte ega viit ei jaksanud osta, piraaditi viis ja 
oligi edusamm!}

Et nagunii ei ühte ega teist ei olnud aga saadi vähemalt midagi teha!

\question{Mis aastal see PC-lugu oli?}

Mingi 1989? 
                 
\question{Sa ketrasid ennast Amigatest siis ikka väga ruttu läbi PC peale?}

Jah, kogu aeg vahetasin. 

Üheksakümnendate alguses, kui juba iseseisvus hakkas, oli mul kõige esimene 
kodu-386. Jälle imestati, milleks seda vaja on, kes sellise asjaga tegeleb, mis 
sa sellega teed. See oli tüüpiline.

\question{Sul pidi ikka siis mingi huvi olema, et sa neid arvuteid nii sageli 
vahetasid ja kooperatiivi ka sisse jäid?} 

Kooperatiivis meil väga kaua see asi ei kestnud,  sellepärast et väga kiiresti 
tuli peale see seadus, mis keelas mitteriiklikele ettevõtetele ja 
kooperatiividele arvutitega äritsemise. Ja kuna mul olid sidemed olemas, polnud 
kooperatiivi enam vaja. Me paari tuttavaga kliente leidsime ja nii edasi, 
tekkis küsimus, et kus kohta ja mida me teeme. Mida meil vaja on? Tootmisruume 
ei ole vaja! Meil ei ole vaja mingeid ladusid, mingeid tooraineid, midagi. Mis 
meil vaja on? Raha!

Mõtlesime, et Tartu Kommertspank\index{Tartu Kommertspank}\sidenote{1988. 
aastal tegevusloa saanud Tartu Kommertspank oli esimene aktsiaseltsina tegutsev 
ning ka välisvaluutatehingute litsentsi saanud kommertspank NSV Liidus. Panga 
tegevus lõppes pankrotiga 1994. aastal. See pank oli mingis mõttes oma aja 
tõeline sümbol põledes heledalt ja kiiresti. Ka Hansapank\index{Hansapank} 
alustas tegevust Tartu Kommertspanga filiaalina!}! Tore koht! Läksime 
Veetõusme\index[ppl]{Veetõusme, Ants}\sidenote{Ants Veetõusme, kes kuni 1990. 
aastani oli Tartu Kommertspanga juhatuse esimees.} jutule. Poisid ütlesid ka, 
et kui lähed, räägi, mis vaja on, et võiks olla nagu oma raha ka raha 
loksutada, küsi kuskil sada tuhat. Hinnad olid sellised, et sellega sai juba 
enam-vähem masina osta! Muidu äri käis ju kogu aeg ettemaksetega. Raha tuli ära 
ja sa võisid teda tükk aega pööritada, siis ostsid  masina ja andsid kliendile 
kätte. See, et kuu aega tuli oodata, oli tavaline nähtus. Tulid jälle suuremad 
summad, siis meil oli käsi üsna hästi sees Novgorodi elektroonikatehases, mis 
tegi  Panasonicu MV-25 pealt maha viksitud vene videomakke VM-12. Laadisid 
furgooni neid täis! Kui sinna läksid, et oleks hea suhe, kast vana Tallinnat 
paar kasti suitsusinki, meie oma suitsukana. Sellega, kõmm, Novgorodi, auto 
videomakke täis ja neid me ei viitsinud üksikult müüa, müüsime hulgi 
koperativšikutele maha, need siis oma kooperatiivipoodides parseldasid edasi. 

Sellist rahakeerutust sai tehtud kogu aja. Aga sai, jah, mõeldud, et võiks olla 
käibevahendeid. Algkapitali, nagu öeldakse. Tavaliselt tead, et kui  midagi 
küsid, siis nii kui nii tõmmatakse maha. Rääkisin 
Veetõusmele\index[ppl]{Veetõusme, Ants} ära, et vot selline arvuti-äri. Ta oli 
väga huvitatud, kõrvad liikusid, et kas neile ka saaks. Ikka saab! \enquote{Aga 
mis te meile pakute?} Noh, ütlesin, et 11\%. Meie näiteks teenime miljoni, teie 
saate 110 000. \enquote{Täitsa hea mõte! Ja palju te meie käest tahate?}. 
Mõtlesin, et küsin rohkem, niikuinii kaubeldakse alla. Ütlesin miljon. 
\enquote{Ahah. Avage arve, pange miljon peale!}.

Kusjuures meie firma oli selline, et kui see kommertspanga  pankrot pihta 
hakkas, siis meie olime üks väheseid, kes selle raha tagasi maksis. Oleks 
võinud põhimõtteliselt ka teha mingid varifirmad ja asjad ära kantida ja külma 
teha. Aga meie tasusime kogu selle raha ja kuskile võlgu ei jäänud.

\question{Kui sa neid amigasid ja PC-side keerutasid, sul pidi mingi huvi 
olema, mis sa tegid nendega?}

Oh, jumal! See oli ka omaette nuhtlus! Kui Tähetornis\index{Tartu Tähetorn} 
need esimesed MSX\index{Yamaha MSX} arvutid tulid, ma veel töötasin 
keemiainsenerina. Pärast tööd sõidad bussiga alla linna, lähed Tähetorni ja 
siis istud ja mängid seal täpselt nii kaua, et on aeg bussi peale minna ja 
tagasi tööle sõita. Vaatad hommikused ringid üle, keerad kabineti lukku, keerad 
magama. Mängud olid naiivsed, aga tead, ta  oli nii põnev aeg! Ja kui endale 
arvuti tuli, see oli košmaar! Järjekord oli pidevalt ukse taga, kõik tulid 
tasuta mängima. 

\question{Sa siis mängisid?}
Jah. Sai muidugi igasugu asju uuritud ja kui tuli internet, siis\ldots 
Tegelikult hakkas see maailmaga ringi käimine juba enne seda, BBS-i ajal.
                 
\question{Vot sinna ma tahtsin jõuda! Kust sul tuli mõte, et paneks endale 
BBS-i püsti? Ja millal see oli?}

See oli kuskil väga varastel üheksakümnendatel. Päris internet jõudis Eestisse 
kahe satelliiditaldrikuga, üks oli seal KBFI peal Tallinnas ja teine oli Tartus 
Tähetornis. Ja sealt siis üle Rootsi Kuningliku 
Tehnoloogiainstituudi\index{Rootsi Kuninglik Tehnikaülikool}, KTH,  käis meil 
side. Siis hakkas BBS-indus vaikselt juba ära vajuma, kuigi ta  töötas veel 
edasi, eks ole. Mäletan, et internet jõudis Tartusse,  ma elasin tollal  seal, 
märtsikuus 1992. Sain üle  EBC, Biokeskuse\index{Eesti Biokeskus}, endale oma 
isikliku \emph{account}-i juba aprillis, kuu aega hiljem.
                 
Tol ajal oli meie kontor Rüütli tänavas, kohe Treffneri kooli\index{Hugo 
Treffneri Gümnaasium} vastas. Muidugi ägedad trefneristid  käisid seal kõik 
hoolega arvuteid näppimas. Üks põhimehi, kes seda asja suunas ja üles pani ja 
majandas tarkvara poole pealt oli Einar Entsik\index[ppl]{Entsik, Einar}, 
praegu kõva kinnisvaraärimees. Tema oli nagu meie peamine \emph{sysop} ja mina 
olin siis \emph{co-sysop}. Hiljem, kui me kontori likvideerisime, siis Lucifer 
BBS\index{Lucifer BBS} tegutses mul kodus edasi, kuni peaaegu lõpuni, kui see 
BBS-i maailm ära hääbus. Siiamaani mäletan veel oma aadressi: 2:491.666.
           
\question{Millest selline nimi, Lucifer?}      

Ta tõi valgust maailma! 

Muidugi, meil oli väga palju igasugust sellist maagiat ja värki, kuna ma 
loomult olen anti-kristlane olnud eluaeg, nüüd olen ma veel suurem 
anti-islamist. Seda ma ei suuda üldse taluda,  selle kõrval kristlased on 
väikesed voonakesed. Vaimupimedust, mis siin on, keskaega tagasi pürgimist! 

Mäletan selgelt, et oli suur jama, kui KAPO käis meie neid materjale uurimas, 
kui mingisugused nõndanimetatud satanistid pussitasid Tartus Hando 
Runnelit\index[ppl]{Runnel, Hando}. Neid uuriti, et kust saadud ja kellegi 
kaudu tuli  välja, et meie BBS-is oli väga palju neid materjale, Lavey Saatana 
Piibel\sidenote{La Vey, Anton Szandor. The satanic bible. New York: Avon Books, 
1969.}. Käidi, uuriti ja vaadati. Ma mäletan üks mehike tutkis nii põhjalikult, 
et pööras täitsa ära, hakkas ise ka satanistiks!

Kusjuures kui sa küsid, kas ma olen satanist, ma ütlen, et ma ei ole. Kui ei 
usu kristlust, kuidas ma saan siis tema peegelpilti kummardada?

\question{Miks te BBS-i tegite? Äri sai ju muud moodi ka teha?}                 

See oli lihtsalt hobi, poisid tahtsid teha. Igasugused sidemed, materjalid üle 
maailma\ldots Tol ajal kaugekõned olid ju kõik tasulised aga no selle äri 
juures telefoni hinnad! See polnud tähtis, ma võisin tundide kaupa rippuda 
Ameerika või Iisraeli või kuskil\ldots Euroopa polnud üldse küsimus! Sai 
helistatud Jaapanisse, sealt igasugusi materjale tõmmatud, sai sealse skeenega 
suheldud ja.

\question{Kust sa numbrid said, kuhu helistada? Üheksakümnendate alguse Tartut 
meenutades, kust ma võisin saada Jaapani telefoninumbri, kus taga BBS vastas?}

BBS-idel olid ju kõigil suured \emph{listing}-ud olemas, kus oli maailma 
olulised BBS-id loetletud. See on nagu aadressiraamat,  telefoniraamat, seal on 
kõik riikide BBS-id sees! 

\question{Ja sa käisid seal infot lugemas?}

Jah, loed uudiseid, infot\ldots Nüüd  kõik istuvad Facebookis, aga siis seda ju 
ei olnud. Siis olidki BBS-id, mille kaudu käis info vahetamine, meili saatmine 
ja kõik. No hiljem, kui internet tuli, olid juba teised ajad. Et seal 
tegutseda, tuli endale UNIX-i alused selgeks teha ja käsureal töötada. Siis ei 
olnud ju veel Linux-itki olemas ega midagi. Põhiline oli just Santa Cruz 
Operation\sidenote{Siit tuleb lühend SCO.} V5 UNIX\index{Unix!System V}, millega 
me kõik siin tegutsesime.

\question{Ohoh, too BBS käis teil UNIXi all?}

Ei BBS-il on oma tarkvara, UNIX tuli hiljem, kui hakkasime juba internetis 
käima. Kliendiga, lehvik kuskile terminali otsa\ldots Ega siis kellelgi kodus 
interneti polnud, ei olnud võimalik saadagi. Pidid teadma, kus istusid  sisse 
helistamise modemid, millega sa said ennast kaugelt kuskile interneti arvutisse 
sisse logida. Näiteks Toomemäele Tähetorni\index{Tartu Tähetorn} ja sealt siis 
juba edasi liikusid, siuh-säuh, internetiavarustes.
                 
\question{Ja kõik käis käsureal!}
                 
Jah. Hiljem hakkasid tulema Gopher ruumid ja muud sellised algelised 
otsingusüsteemid. Infopangad, kus oli erialaseid raamatuid. Siis tekkisid 
interneti BBS-id. Printa oli näiteks Euroopa üks suurimaid ja 
Iska\sidenote{Iowa Student Corporation Association.} BBS\index{Iska BBS} oli 
maailma kõige suurem interneti BBS veel sellisel kujul. Nagu BBS, 
teadetetahvliga, lihtsalt sinna ligipääs oli Interneti kaudu.

\question{Kas sa teiste Eesti sysopidega ka suhtlesid?}

Ja, ikka, meil olid ju igasugused üritused, BBSummerid\index{BBSummer} ja 
BBWinterid\index{BBWinter}. Seal käisid nii sysopid kui ka  kasutajad, oli 
niisugune päris tihe seltskond, kes seal käis ja omavahel niisama suhtles, no 
nagu praegu Facebookis käivad suhted.  See seltskond ei olnud nii suur, et ka  
ei oleks võimalik hallata. Katsu sa teha näiteks Eesti Facebooki liikmete 
kokkutulekut, võib-olla  jääb tulemata  viis protsenti inimesi!
        
\question{Kui palju teil Luciferis\index{Lucifer BBS} liine oli ja, anna palun 
suurusjärku, kui palju kasutajaid küljes käis?}         

Üksainuke telefoni liin. Sellega oligi see, et kui kasutaja tuli  liini külge, 
siis ta pidi seal rippuma. Kõneaeg jooksis kogu aeg. Et kui sa näiteks 
sikutasid mingit tarkvara kuskilt Ameerikast, siis sa rippusid kogu aeg 
kaugekõnega liini peal, päris soolane kopikas tiksus! Eraldi üüriliinid tulid 
alles ISDN-i ajastul, kui tulid 64 ja 128 kilobitised asjad. Algselt, kõige 
esimene modem, mille ma sain, oli 2400 boodi.
                 
\question{See oli isegi juba kiire, sest 1200-sed olid ka levinud}

Isegi 600-sed! Finlandia  BBS\index{Finlandia BBS} oli 2400,  ülejäänud kõik 
olid aeglasemate peal. Ega see modem maksis ka kaks korda rohkem kui sõiduauto 
Žiguli, niisugune tavaline.

\question{Hobi jaoks tundub Žiguli nagu kallis investeerida?}

No võtame niiviisi, mõni rikas mees korjab hobi jaoks vanemaid autosid, 
uunikume, mis maksavad ka meie praeguses rahas seal sada ja kakassada tuhat. 
Teevad oma automuuseumi. See on samuti hobi, ega te sellega ka midagi muud ei 
teeni, piletit ka ei küsi!
       
\question{Ma ikka ei jäta. Mis sind just selle hobi juures paelus, mis hoidis 
sind arvutite juures?}          
Seesamane kübermaailm. 

Kui vaadata praegu Ghost in the Shell-i\sidenote{Masamune Shirow samanimelisel 
mangal põhinev frantsiis, millesse kuulub nii animesid kui ka 2017. aastal 
Hollywoodis linale tulnud film. Frantsiisi tegevus leiab aset post-küberpunk 
maailmas ja selle peategelane, Major Motoko Kusanagi, on küborg, kelle 
mehaanilises kehas (\emph{shell}) toimib inimese teadvus (\emph{ghost}). Tegu 
on kunagistes küberpunk ringkondades kultusliku teosega, mille mõju on 
võrreldav Willigam Gibsoni loominug omaga.} või midagi sellist, et oleks 
võimalik oma teadvus enne surma Internetti üle kanda. Läheks küll sinna 
virtuaalmaailma tondiks!  

\question{Kas sa tol ajal Gibsonit ka juba lugesid?}
Ja, ikka.

\question{Siis on selge!}
                 
Kõik see küberpunk ja see värk oli sisuliselt \emph{must be} kõigile, kes olid 
toll ajal arvuti-friigid. Siis muidugi virtuaalmaailmades\ldots Kuhu ma eriti 
sisse ei jõudnudki, olid mudad\index{Muda}. Mina sattusin virtuaalmängumaailma 
siis, kui tuli selline asi nagu EverQuest I\index{EverQuest}. Seal sai 
järk-järgult läbi gildide mindud, mängisin seda mängu neli pool aastat jutti. 
Mängus sees oli \emph{online counter}, mis luges, kui palju sa mänginud oled, 
mitu päeva, mitu tundi ja nii edasi. Summeeris kokku. Ja kui ma pärast sealt 
vaatasin, siis selle nelja poole aasta jooksul, ma oleksin pidanud iga jumala 
päev mängima neli ja pool tundi. Aga mõnikord oled välismaal kuskilt ära, ei 
mängi. No, polnud sagedased, aga polnud ka väga haruldased juhtumid, kui ikka 
kakskümmend tundi jutti näiteks suuri \emph{raid}-e peetud.
                 
\question{Kas Gibson ja muu küberpungi kraam levis BBS-ides või olid füüsilised 
raamatud ka?}

Olid füüsilised raamatud, suuremad fännid siin, Jack\index[ppl]{Lippmaa, 
Jaak}\sidenote{Ilmselt peab Veiko silmas Jaak Lippmaad ja mitte Jaak Loondet, 
keda sama nime all tunti.} ja nii edasi, tõlkisid neid Eesti keelde,  isegi 
avaldati. Ja eks olid ingliskeelsed suured raamatuarhiivid. Ei olnud ju midagi 
eriti saada, just ulmet ja ilukirjandust. Teadusraamatukogud  ostsid rohkem 
teaduskirjandust,  ilukirjandust oli ikka väga vähe, ja need hakkasid liikuma 
juba digitaalsel kujul. Digitaalsetest arhiividest sai raamatuid tõmmatud, mul 
endalgi BBS-is kõik, mis kätte tuli, läks sinna üles ja rahvas käis ja sai neid 
sikutada ja hoolega lugeda. 

\question{Ja lugeja vaim sai valgemaks! Mis rahvas sul seal BBS-is käis? Mingi 
aimdus sul ju oli, tudengid või\ldots?}

Olid õpilastest ja tudengitest kuni paremate arvuti-inimesteni välja. Sest meil 
oli seal väga palju igasugust põnevat tarkvara, põnevat arvutialast kirjandust 
ja materjale, mida kuskil ka eriti ei liikunud. 

Ja kuna Printas juhtisin ka üht tuba, olin moderaatoriks,  siis saime nende 
\emph{underground}-iga  tuttavaks, IC piraadigrupi\sidenote{Siin peab Veiko ise 
seletama, mis tolle \emph{warez} grupi nimi täpselt oli. Suuremate gruppide 
nimekirjast sarnase nimega seltskonda leida ei õnnestunud.} liige sai oldud, 
seal liikus vahvat materjali. Oli täitsa kurioosseid olukordasid. Tollel ajal 
ei olnud ju mingeid päris pira FTP-sid. Tehti seda nii, et kui mingi asi algas 
punktiga, siis see oli nähtamatu. Ja kuskil, kus oli firmal FTP server püsti, 
siis mingisugusesse huina-muina kataloogi, kus on mingid süsteemsed asjad ja 
kuhu tavaline inimene ei lähe, tehti punktiga algavaid katalooge. Kui seda 
hakati avastama, tulid kasutusele igasugused muud asjad, näiteks mingid 
kontrollsümbolid. Sümbol, mis tegi näiteks piiksu või mis tegi reavahetuse. 
Seesama \emph{enter}-i vajutamine, et sa sisestad asja, oli ka võimalik 
kontrollsümbolina kirja panna. Ja kui seal see sümbol oli ees, siis sa pidid 
teadma, et sa sinna ette panid just selle kontrollsümboli, vist oli CTRL+L, mis 
käskis lugeda järgnevaid asju kui lihtsalt tekstistringe.

\question{Ehk te munesite kellelegi FTP serverisse  oma piravara?}
                 
Jah, niiviisi oli terve maailm täis! Katoloogide kaupa oli kõikvõimalikke asju.
                 
\question{Suurde rahvusvahelisse piragruppi ligi saamine oli ju seotud, ütleme, 
raskustega. Päris avasüli ei võetud vastu?}

Suurte raskustega! Praeguses torrenti-ajastus\ldots

\question{Aga kuidas see sul õnnestus?}

Selle Printa BBS-i kaudu tulid, kutsusid.

Igasugu põnevaid asju sai uurida ja vaadata. Vaata, ega sa ei oska ju midagi 
soovitada, kui sa ei ole ise seda näppida saanud.

Vene aja lõpus müüdi igal pool mustal turul ja laatadel piraat-kogumikke. 
Igasugused Gamez ja nii edasi. Aga et sa piraat-tarkvaraga raha teenid, loeti 
tõsisemates gruppides väga halvaks märgiks, selle eest said kohe kangiga vasta 
pead. See oli patt. 

Kui tuli välja Windows 95, oli see alguses kohutavalt suur saladus. No nüüd on 
Microsoft ennast täiesti teisipidi pööranud. Tahad, võtke uusi versioone, 
uurige, tutvuge, vaadake! Nad on lõpuks aru saanud, et see, et sa midagi püüad 
kinni hoida, see ei takista. Aga kui sa tahad endale miljonit kuju, kes uut 
toorest tarkvara katsetab ja kakub oma juukseid, mis lähevad tänu sellele 
halliks, et kõik hunnikusse lendab? Sealt tuleb tagasiside! Kõik sõimavad 
\enquote{Parandage see ära, see on valesti!} Sa ei jõua endale nii palju 
töötajaid otsida, kes  kõik selle debugimise töö nii põhjalikult ära teevaed, 
kui see vabatahtlik jõuk. Aga siis oli see, jah, nii keelatud, et kui tuli 
Windows 95, tollal koodnimega \enquote{Chicago}, siis ta oli muidugi olemas, ja 
poisid tegid igavese pulli, nad panid selle Microsofti peamisse FTP serverisse. 
Tegid seal punktidega kataloogid, ja järgmine päev panid sinna Chicago üles. Ja 
kui see üle  maailma kulutulena levis, \enquote{Microsoftist saab pira Windowsi 
tõmmata, 95t!}, oi kuidas siis Microsoft marutas!. Ja kõik, kes seal käinud 
olid (mina käisin lihtsalt vaatamas ja irvitamas,  et \enquote{vaadake, mis 
seal seisab}), nad olid ära loginud. Muu hulgas ka ühe  aadressi, mille kaudu 
mina käisin. Mul oli neid palju, üle maailma, sest üks modem oli kinni, teine 
kinni, eks ikka leidus mingi auk. Tuligi teade, et \enquote{karistage, võtke 
\emph{account} ära, see on igavene vastik piraat, käis piilus meie juures 
Windowsi!}. Kuigi ma ei tõmmanud teda, ammu olemas! Sysadmin tuli minu juurde, 
et \enquote{Windowsi vaatasid, sellistele asjadele saate ligi? Kuule, aga mul 
on üks küsimus. Meil Soome kolleegid näitasid SPSS\sidenote{Algselt IBM-i poolt 
välja töötatud tarkvarapakett, mille nimi on lühend väljendist 
\emph{Statistical Package for the Social Sciences}.}  5.0-i, jube kallis, neil 
on ainult pooled moodulid ostetud. Kas seda oleks kuskilt saada?} Küsisin ICE 
põhimeeste käest. Sealt tuleb vastu, et \enquote{Äh, miks sa viit tahad? 
Poolteist kuud tagasi tuli kuus välja!} Ja oi kuidas siis meie onud-teadlased 
olid rõõmsad, et kui Soomlased tulid vaatama, et Eestis on täis pakett SPSS 
6.0-i, mida Soomes ei ole mitte kellelgi! 

\question{Ühel hetkel toimus ikkagi nihe, mingi hetk hakati tarkvara eest ju 
maksma?}
Oma riik tuli, kõik asju sai hakata ostma ja hinnad ka normaliseerusid. Vene 
aja lõpul olid need hinnad ju\ldots No kujuta ette, kahekümnemegase kõvaketta 
eest maksad sa 45 000 rubla! Mäletan, Eesti kõige esimene 486 läks Punase 
RET-i\sidenote[][-4mm]{Asutatud 1935. aastal OÜ Raadio-Elektrotehnika Tehas nime all. 
Tehas tegutses 1993. aastani ja tootis raadioid ning mitmesugust 
audiotehnikat.} spets konstrueerimisbüroole. See toodi mingisuguse, ma ei tea, 
mis kuradi värgiga (tol ajal Ukraina, Valgevened ja teised hakkasid ka 
eralduma)  peidetud transpordiga Minskisse. 486-d olid ju totaalse embargo all, 
CoCom\sidenote{\emph{Coordinating Committee for Multilateral Export Controls 
(CoCom)} oli mitteformaalne multilateraalne organisatsioon, mille abil USA ja 
tema liitlased üritasid koordineerida erinevaid kommunistlikke riikide suhtes 
strateegilistele kaupadele rakendatud piiranguid.} ei lubanud neid sotsmaadesse 
viia. Ainult 386 kõige lahjemad versioonid olid need, mida juba võis ametlikult 
tuua. Ja selle masina hind oli neli pool miljonit rubla.

\question{Hoomamatu number toona rublades, isegi täna Eurodes!}

Tehti seesamanegi piir, majanduspiir, Narva jõe peale. Sõidame sinna, vastik 
ilm oli. Vahepeal oli see reegel, et ainult juht tohtis läbi sõita, teised 
pidid minema jalgsi läbi putka. Ei viitsinud minna. Olime kahekesi, 
mikrobussiga. Tuleb siis sõdurpoiss, lööb kulpi, \enquote{miks te kahekesi 
olete?} Ütlesin, et, saadan kaupa, seda ei tohi üksi viia. \enquote{Mis kaupa? 
Tehke lahti!} Kolm suurt matkajate seljakotti, näha, et mingid nurgelised 
klotsid on sees. \enquote{Mis te veate?} \enquote{Raha!} \enquote{Mida?} Teeb 
koti lahti, seal on sajaste klotsid, sada rahatähte oli pakk, mis panderolliga 
ümber tõmmati. Kümme sellist pakki oli üks \enquote{tellis}. \enquote{Palju 
siin on?} \enquote{Kuskil ligi viis miljonit\ldots} \enquote{Vabandust!} Edasi 
ei huvitanud, tema jaoks oli see ka hoomamatu. 

Aga meil olid niivõrd head sidemed vene poistega. Venemaaga äri ajades sa pead 
teadma, kuidas ja mismoodi on. Meil oli usaldus nii suur, et teinekord läksid,  
täpselt ei teagi, palju sul on. Võtad kaasa, valid selle ja teise arvuti ja 
jääb näiteks ütleme poolteist miljonit üle. Jätsime rahulikult tema juurde 
seifi. See edasi-tagasi sõidutamine oli kõige riskantsem, kui sul võis saba 
peale lennata, sind maha võtta, ära tulistada auto ja kõik, eks. Hiljem 
helistab \enquote{Kuule, meil tuli selline väga huvitav asi väga hea hinnaga. 
Huvitab?} \enquote{Huvitab!} \enquote{Okei, ma tõstan selle raha siis endale} 
Ja samamoodi, teinekord lähed sinna kahte masinat tooma ja ütleb \enquote{Tead, 
mul õnnestus kolm tükki saada. Tahad? No võta kaasa, järgmine kord tood raha 
ära!} 

\question{Kui ma sind kuulan, siis see ei ole mitte kümned ja sajad ja 
konteinerid, vaid kaks-kolm masinat?}

Jah. Suuremaid tehinguid oli vähe. Meil olid kontaktid arvutite juurde, mõnedel 
teistel meestel olid kontaktid raha üle. Olid meil näiteks niisugused 
sõbralikud suhted EVEA Panga\index{EVEA Pank} mitmete tegelinskitega, kes 
ajasid meile ikka tõsiseid ärisid välja. Tõime suure-pika reisi-Ikarus 
bussitäie arvuteid Poolast, näiteks. Sai need kõik viidud Peterburi, seal 
laaditud sõjaväe transport kopterite peale. Kõik on kuulipildujate ja värkide 
all, relvastatud eriväelased ümber. No aga see tehing oli ka seal, ma ei tea, 
kui palju miljoneid seal kokku läks. Kopterid sõitsid põrr-põrr-põrr, kogu 
Kuibõševi linnavalitsuse arvutipark tuli siitkaudu. Igaüks sai oma sellest! 
Tulevad poisid, istuvad, ajavad juttu, väga head konjakid kaasas, saab joodud, 
hakkavad ära minema. \enquote{Oot, kohvri unustasid maha} Ah, jah, miljon 
sularahas kohvriga näpu otsas kaasas\ldots See oli väga, ütleme, selline 
kauboikapitalismi aeg. 

\question{Kas õnneks või kahjuks see aeg ei kestnud väga kaua}

Jah, kuidas kellelegi. Mõned lõpetasid kuuliga kuklas, teised lõpetasid 
praeguste tippmiljardäride hulgas. Kes kuda mida jõudis kinni võtta.

\question{Tuleme tagasi Luciferi\index{Lucifer BBS} juurde. Ta oli Eestis 
ikkagi üks hetk kõige populaarsem koht, kus käidi. Vähemalt nii on räägitud ja 
endalgi on meeles. Mis ta nii populaarseks tegi?}

Info hulk, mis seal oli. Sest kui BBS-indus veel õitses, siis internetile, kust 
sai neid materjale sikutada, oli ligipääsu väga vähestel. Enamik interneti 
kasutajaid olid tõsised töötegijaid, kes tegid tõsist tööd, eksole. Oma 
kolleegidega seal vahetasid emaile ja \emph{that's it}. Nemad ju ei kaevanud 
ringi mingisuguseid suuri raamatute ja igasugu failide ladusid pidi. Nad ei 
toppinud oma nina igale poole ja tänu sellele kuskile mujale ei jõudnudki. Ja 
kellel olid BBS-id, ei olnud jälle sellist finantsvõimekust, et sikutada kogu 
seda materjali lihtsalt BBS-i kaudu ühest teise, see oli väga kallis.
                 
\question{Ja sinu juures said kokku huvi ja aru saam nende asjade väärtusest ja 
finantsvõimekus?}

Jah.

\question{Ja seetõttu sinu \emph{stash} oli populaarne, sinna oli popp külge 
tulla?}

No sealt igaüks leidis midagi põnevat! Seal oli kõike, alates igasugu maagiast 
ja okultismist ja satanimist üle arvutikirjanduse, ulme ja \emph{science 
fictioni} kokaraamatute ja retseptikogumikeni välja. Kõike.
                 
\question{Räägime korra nendest online-mängudest, mida sa mainisid. Mis aastal 
sul esimene mäng tuli?}
                 
Mina sattusin sinna kuskil kahetuhandendal. Enne ma suur mängur ei olnud, 
mõningaid üksikuid mänge  sai toksitud, aga põhimõtteliselt oli internet see 
maailm, kus ma ringi kolasin. Ja huvitaval kombel muda\index{Muda}, see 
mitte-graafiline \emph{dungeon}, jäi kõrvale. Muidugi sai kõvasti mängitud 
\emph{Dungeons \& Dragons}-it\sidenote{Dungeons \& Dragons, tihti lühendatud ka 
DnD või D\&D, on 1974. aastal esmakordselt ilmavalgust näinud rolli-lauamäng. 
Mäng oli esimene omataoline võimaldades suhteliselt vaba vooluga kuid siiski 
kindla struktuuriga mängu-karakterite ja lugude arendust. Pikad mängukampaaniad 
võivad kesta aastaid.}. Sinna vedas mind Jaanus 
Lillenberg\index[ppl]{Lillenberg, Jaanus} ta on nüüd ERR-is. Tema oli mul 
esimene DM\sidenote{\emph{Dungeon Master} on Dungeons \& Dragons mängu kohtunik 
ja jutustaja, kes täidab ka loo mitte-mängijatest tegelaste rolle. DM 
kontrollib ja organiseerib kogu mängu, temast sõltub mängukogemuse kvaliteet.}, 
see oli vist 1994 kui ma sinna mängu sattusin. Siiamaani saab seda mängitud. 
Käime siin vaikselt ja toksima korra nädalas täringuid ringi.

\question{Väga põnev, sest mina olin ka sel ajal Tartus aga minu jalg tolle 
maailma peale küll ei sattunud?}

No seda oli vähe. Põhimõtteliselt  tõid ta Tartusse sellised mehed nagu Arlis 
Narusberg\index[ppl]{Narusberg, Arlis} ja Uuk\sidenote{Ei ole selge, keda Veiko 
silmas peab.}. Võiks öelda, et DnD üheks kõige esimeseks maaletoojaks on, vana 
hea tuttav Vormsi Enn\index[ppl]{Vormsi Enn|see{Mikker, 
Enn}}\index[ppl]{Mikker, Enn}. Ta sai neil segastel lõpu-aastatel Soome sõita. 
Tema käest telliti, et too ikka mõni mäng, arvutimäng. Tema tuttavad, vanemad 
inimesed, ega nemad ka täpselt ei teadnud, läksid poodi ja ostsid talle teise 
\emph{edition}-i Dungeons \& Dragonsit, \emph{dungeon master}-i raamatud, 
\emph{players handbook}-id ja kõik. Algul oli pettumus, polnudki nagu kuskile 
arvutisse panna seda asja, aga kui süvenesid, oli väga kõva. See oli 
Kunstiinstituudi\index{Eesti NSV Riiklik Kunstiinstituut} punt, kes seda 
mängis. Meelis Mikker\index[ppl]{Mikker, Meelis} näiteks oli väga kõva DM. Ja 
kui sealt lõpetanutest üks ports Tartusse kolis,  tuli nendega koos ka DnD ja 
seda sai ikka  mängitud. Vahepeal üheksakümnendate keskpaik oligi mul selline 
hullumeelne aeg, kus ise mängisid näiteks kahes mängus ja tegid ise kolme-nelja 
mängu. Nii et terve nädal otsa iga päev oli mingisugune seltskond.

\question{Kui sa \enquote{tegid mängu}, siis sa olid DM?}

Jah.
                 
\question{See tahab ju fantaasiat saada, ei ole niisama!}

Aga selleks interneti maailm ja ulmekirjandus ongi, et fantaasiat arendada!

\question{Ja fantaasiat sul on?}

Võiks öelda, et jagub. Siiamaani käivad ja painavad. Üks seltskond, 
filmi-inimesed, tahavad, et ma ingliskeelset mängu hakkaks tegema. On kõik 
põnevil, aga küsimus ongi DM-ide vähesus, kes viitsiks ja oskaks teha. Ma olen 
ka mõned korrad sattunud sellisesse mängu mängima, kus käib asi niiviisi, et 
\enquote{Lähete nädal aega, midagi ei juhtu. Nüüd tuleb kari lendavaid lõvisid, 
hakake lööma!}. Kõik veeretavad täringut, kolm tundi täring klõbiseb, kõik 
kaklevad, lõvid on surnud, \enquote{nüüd lähete veel kuu aega, midagi juhtu}.  
Sisulist mängu nagu ei olegi. 

Kui sa võtad kätte ja lased rahval mängida, tõmbad neile konkse ja igasugu asju 
üles\ldots Ühes mängus oli, kus kõik mängisid nii hästi oma osa. Loomulikult 
kõiki aeti asju nurga taga, et ülejäänud rahvas ei kuule. Tegelikult sellest 
moodulist, mida pidi me  liikusime, ei liigutud ühtegi sammu, kogu asi käis 
omavahel. Keegi ei tea, kõik kahtlustasid, et see või teine on mingisugune kuri 
koll. Mäletan, kuidas Mario Pizzolanti\index[ppl]{Pizzolanti, 
Mario}\sidenote{Tartus tuntud kuju, pidas legendaarset baari 
Zavood\index{Zavood}.}  kellegagi  igavesi lahingud lõid. Ja kui pärast 
sessiooni kokku tõmbasime ja asjad avalikuks tulid, said teise keretäie veel: 
\enquote{Mina arvasin seda!} \enquote{Aga mina arvasin nii!} Ja vahel kõik 
naersid nii, et püksid märjad. 

Siis tuligi seesamunegi EverQuest\index{EverQuest}, seesama \emph{dungeon}, aga 
arvutimaailmas. Et sa ei pea ise olema DM vaid arvuti teeb selle sinu eest ära. 
Ei olnud enam aega DnD-d teha ega midagi, aeg läks kõik sinna. Parimatel 
aegadel ma olin  EverQuest I-s maailma seitsmes \emph{warrior}. Eestist tuli 
neid veelgi, üks sõber on \emph{ranger}-ina veelgi kõrgemale tõusnud. Minul oli 
keskmine mänguaeg ööpäeva kohta neli ja pool tundi, temal oli kuus ja pool. 

\question{Oh jumal, see on ju investeering!}

Jah, ta sel ajal oligi.
                 
\question{Kui ma sind kuulan, siis kerkivad inimesed kuidagi esile. Sa tundud 
nendega hakkama saavat, neist aru saavat?}

Jah. Praegugi juhin gilde. 

\question{Selles mõttes ka, et ega Venemaal nende tõsiste inimestega jutu peale 
saada ei ole lihtne. See tahab pealehakkamist ja suhtlemise oskust?}

Nagu öeldakse, sa pead teadma \begin{russian}русская душа\end{russian}-d, vene 
hinge. 
See on hoopis teine, kui sa seda ei mõista\ldots Jätame poliitika kui sellise 
kõrvale. Aga  praeguse aja noortel ja lääne inimestel äri ajamisel ongi see, et 
nad ei saa aru. Et kui tema ütleb hinna ja selle peale öeldakse \enquote{Ahah, 
et selle tehingu väärtus on meil 20 miljonit? Aga teeme nii, et on 15 
miljonit!} Ei saa aru! Kuidas? Mis? Miks? \enquote{No me anname ühe Šveitsi 
panga arve, kuhu kolm miljonit panna.}
                 
\question{Sa saad ju sellest  eestlase hingest ka aru, sa saad aru, mida 
inimesed vajavad ja mis neid huvitab, kasvõi Luciferi püsti panekuga}

Jah, kindlasti.

\question{Kust see sul tuleb, oled lihtsalt sündinud sellega? Oled sa mõelnud?}

Võib-olla on see oskus. Ma ei oska öelda, ei ole nii palju analüüsinud. Aga 
nii-öelda juutimise asjaga ma olen tegelenud palju. Kui muu rahvas rüüpas EÜE-s 
oma elu, siis mul kõik suved ja talved olid talvel suusamatkadel ja suvel 
mägimatkadel alpilaagrites. Ma olen palju igasugu matkagruppe juhtinud.

Seal ongi see, et kuidas seda asja teha. Arvutifirmasid, erinevad, on saanud 
juhitud ja\ldots Ma mäletan, kui  sattus kätte raamat \enquote{Kuidas võita 
sõpru ja mõjutada inimesi}, Carnegie oma\sidenote{Carnegie, Dale. How to Win 
Friends and Influence People. Simon \& Schuster, 1936.}, siis paljusid asju 
sealt ma olen kuidagi instinktiivselt teinud. \enquote{Ma tahan sind midagi 
tegema panna}, isegi kui see on kasulik, tekitab trotsi. Vastumeelsust. Et kes 
sa selline oled? Kui sa tahad, et inimene midagi teeks, kujunda selline 
olukord, et inimene ise tahab niiviisi mõelda. Vot see on meie poliitikute üks 
suur puudus ka, et kõik tahavad kedagi juhtida, sundida. Öelda, et sa oled väga 
loll, sa mõtled valesti. Anna parem talle võimalus, et mina olen kuidagi rumal. 
Lase tal minu arvamust ümber pöörata selle arvamuse peale, mida ma tahan, et ta 
tegelikult teeks ja mõtleks. Ja hoopis rohkem tuleks  saavutusi. 

Nagu öeldakse, palju on inimesi, kes tahavad, et  keegi midagi ära otsustaks, 
keegi midagi ära teeks. Nad ei ole huvitatud sellest, et nad peavad oma peaga 
mõtlema ja, mis veel hullem, selle mõtlemise tagajärjel tehtud tegude eest 
vastutama. Jõle hea on näidata, et valitsus on loll, minister on loll, euroliit 
on loll, onu trump on loll, jumal taevas on ka loll. Ainult mitte mina!
                 
\question{Sellest järeldame, et sina oled ka loll?}

Loomulikult! See võtab päris kaua aega, enne kui võiks hakata vana kreeklase 
kombel ütlema, et ma tean, et ma midagi ei tea. 

Kasvõi kõik needsamad jumala teemad, alguse ja lõpu teemad. Teadus on jube 
võimsalt edasi. Kõik need kvandid ja mustad augud ja. Aga mis edasi, kuidas 
edasi? Kust see kõik tuli? Suur pauk? Kes paugu tegi? Mis enne suurt pauku oli? 
Ütleme niiviisi, et kui enne ei olnud midagi ja  nüüd korraga tuli maailm, siis 
see ongi nagu maailma loomine. Selles mõttes jumala mõiste, kui me ei hakka 
siin  mõtlema mingit halli habemega taati, kes karjasekepp käes pilve peal 
jalgu kõigutab, võib võtta loodus seaduste, loodusteaduste, looduse enda 
kompleksina. 

Kui sa oled juhuslikult lugenud Ijon Tichy kosmoselendude 
päevikuid\sidenote{Lem, Stanisław. Loomingu Raamatukogu 1962 Nr. 22. Ijon Tichy 
kosmoserändude päevikud. Ajalehtede-Ajakirjade Kirjastus, 1962.}? Mäletad seda, 
kui ta  äikselise ilmaga maja ukse taga koputas, sisse ei tahetud lasta ja kui 
lõpuks lasti, siis hullunud teadlane näitas talle oma ülakorrusel neid 
plaadikaste, kus loeti, et \enquote{see on noor neiu ja see on keegi teine}. 
Aga äkki me olen ise ka plaadimängijad kellegi tolmunud pööningul? Ei tea! Need 
deja vu efektid ja  parapsühholoogia. Minu arust Ijon Tichy väga ilusti 
illustreeris selle ära. Aga ma ei tea , me ei saa seda kontrollida! 

\question{Sind huvitavad sihukesed asjad?}

Aga loomulikult! Kõik räägivad, et kes tõestab jumala olemasolu, kes tõestab 
selle mitte-olemasolu. Mina olen võrrelnud seda sellega, kui meil varbaküüne 
üks rakk hakkaks tõestama inimese olemasolu või mitte-olemasolu. Kuidas ta seda 
teeb? Peremees võib käärid kätte võtta ja küüned lühemaks lõigata\ldots

Meie orgaanilise keemia professor Viktor Palm\index[ppl]{Palm, Viktor} , kui ta 
luges  teadusliku maailmavaatele aluseid, siis ta ütles tol ajal väga julgelt,  
1981. aastal ikkagi, et tema arust on näiteks teaduslik kommunism ja teaduslik 
ateism täpselt samasugused pseudoteadused nagu teaduslik teism või teaduslik 
jumala-õpetus. Nendel asjadel pole teadusega midagi pistmist!
                 
\question{81. aastal öeldi auditooriumi ees selline asi välja?}

Jah. Nad olla ikka selle eest vasu pead ka saanud, sest usinad tegelased käisid 
ikka, käsi kõrva ääres, raporteerimas. Aga miks nimetada mingit asja 
teaduslikuks, kui seal ei ole teadusega mitte mingit pistmist?
                 
\question{No oli ju vaja kuidagi nimetada, et uhkem oleks!}

No eks  praegu on ka, kui me vaatame, igasugused majandusteadused ja nii edasi. 
Jube palju on seal soolapuhumist! Üks mees võtab need meetodid, tõestab ühe 
asja ära, ütleb \enquote{must}, teine ütleb \enquote{ei, ei, ei} ja tõestab 
ära, et kõik on valge. Kolmas räägib pallist ja neljas räägib üldse kokku 
sulanud spektrist. No võta siis kinni, mis on! Täpselt see, kellele mida vaja.
                 
\question{Meil hakkab tasapisi aeg otsa saama, sellepärast küsin selle kohta, 
mis sa praegu teed. Sa teed palju kirjatööd, kuidas sa selle juurde jõudsid? 
Üks asi on palju lugeda, teine asi on  palju kirjutada.}

No miski aeg tagasi, kahe tuhandete alguses, töötas mu naine sellise ajakirja 
nagu Arvutimaailm\index{Arvutimaailm} peatoimetajana. Oli selline tore aeg, kui 
ka IT-ajakirjanikke  mööda maailma lohistati mitte nagu nüüd, kus  keegi meie 
vastu huvi ei tunne ja veetakse ainult autoajakirjanikke, see teeb kohe suisa 
kadedaks. HP-l oli parasjagu mingisugune järjekordne suur konverents tulemas ja 
Arvutimaailma kaastööline, kes pidi sinna minema, tema pass oli ära aegunud, ta 
ei saanud üle piiri. Ja abikaasa küsis, et kas sul on pass korras, sa tunned 
seda värki, teed ära? No mis seal ikka! Läksin sinna, tegin ära. HP ütles, et 
nii põhjaliku ülevaate, nii sisukat, nad pole näinud. Ma olin HP masinatega ka 
juba aastaid aastaid kokku puutunud. 

Ma  ei ole kunagi arvutiteadust õppinud. Aga vaata, kuidas praegu nooremad 
põlvkonnad on hädas, kasvõi DOS-i käsureaga. Kui ikka hiirega lohistatavat 
värvilist ekraani ees ei ole, on kaks käppa püsti. Aga ma olen selle kõigega 
üles kasvanud, nende arvutitega, mis mul endale läbi on käinud, järjest 
arenenud. Ja loomulikult, kui sa nendega tegeled, siis sul tekib huvi. Vastasel 
juhul ei ole vahet, kas sa müüd kartuleid, arvuteid või kaalikad, eks ole. 

Tegin selle HP loo ära, pärast tuldi veel, \enquote{oi kuule, tead, siin on 
jälle üks asi teha} ja tegelikult ongi nii, et väga palju nüüd ütleme nooremast 
põlvest, on neid, kes ei tunne seda raudvara. Raudvara-ajakirjandusega on see, 
et sa pead tegema, sa pead teadma, oskama küsida ja eks ma selle pärast jäin 
neile nagu silma ka. Kukuti igale poole saatma. Teine asi on see, et mul ei 
ole, nagu mainisid, inimestega läbi saamine probleem. Mind ei  kohuta  näiteks, 
kui me konverentsil saame kokku näiteks Inteli viitsepresidentidega. Või kui 
Otellini\sidenote{Paul S. Otellini oli Inteli CEO 2005--2013.}  oli veel see 
kõige suurem füürer, lähed juurde, pistad viis pihku ja ajad  juttu, küsid ta 
käest kõike. Ja kuna mitmetel üritustel sai käidud, siis paljud mehed, näiteks 
põhimine tehnikaohvitser, viitsepresident, juba eemalt tundsid ära,  tuli kohe 
juttu rääkima. Ütlesid, et \enquote{sa oled ainuke, kes asjast aru saab!}

No eks see oli ka üks asi, miks kirjutama kutsuti, miks hoiti orbiidis. Praegu 
on see asi ära vaibunud. Ma olen vist neli korda USAs käinud Singapuris ja ma 
ei tea, mitu korda Koreas, Hiinas, igal pool. Euroopast ei räägigi, seal 
vahepeal oli pidevalt üks konverents teise järel. Aga nüüd on  IT-firmad 
kuidagi nii maha vaibunud. 
                 
\question{Ehk vist saabunud  ka selles vallas nii-öelda pudukaupmeeste ajastu?}

Tihti oli ka see, et nad tulid turgu, sõid ennast sisse, kes saab vedu, kes 
teeb, kellest kirjutatakse, kõik olid väga põnevil sellest asjast. Aga eks nüüd 
on turu stabiliseerumine  käes, turg enam ei kasva. Ütleme, kasvõi 
lauaarvutitega. Ära nad ei kao, nagu paljud ennustasid, sellepärast, mängurid 
tahavad ikka suure 4K ekraani taga korralikult mängida. Siin on küll läpakas 
päris kõvad asjas sees ja kõik, aga ikkagi nii võimas ta ei ole, ta on alla 
\emph{clockitud} võrreldes sellega, mis lauaarvutis \emph{power}-it on. Neil on 
oma nišš olemas ja aga samal ajal niisugust huvi ei ole konkurentsi mõttes, 
nagu on autofirmadel.
                 
\question{Viimane küsimus. Mis oli viimane mäng, mis tõstis heas mõttes karvad 
püsti, et \enquote{see on äge asi!}?}

Kui sa mõtled seda mingit nime, mis on tulemas, siis selleks on 
Pantheon\sidenote{Pantheon: Rise of the Fallen on MMORPG, mille ilmumist on 
mitu korda edasi lükatud ning 2019. aasta lõpus suri juba varem mainitud 
EverQuesti kaasautor ning Pantheoni tootjafirma loovdirektor Brad McQuaid. 
Juunis 2022 on Pantheon eel-alfa staatuses, ajasime Veikoga juttu 2019. aasta 
algul.}. Pantheoni teeb praegu selline mees nagu Brad McQuaid, kes oli ka 
esimese EverQuesti\index{EverQuest} taga peamiseks ajuks. Oma paljude 
kaaslastega, kes dragonistid olid kunagi, kutsume seda mängurite kuldajastuks, 
mida paljud moodsad mängurid sõimavad. Aga meie ei mängi jälle neid kiireid 
piu-pau mänge, ma nimetan neid selja-aju-mängudeks. Ega seal muud pole vaja, 
võta ahv, õpeta kiiresti punast nuppu vajutama, mängib paremini. Ootan ikka 
sellist mängu, kus on tõesti strateegiat, kombinatoorikat, gruppide juhtimist, 
suured AI-d. Selliseid mänge tehti mingil teisel ajastul, seda aega me igatseme 
ja seda lubab Brad McQuaid. 

Mõned moodsad mängud, mis on tulnud, osasid on saadetud, osasid olen ostnud, 
mõned on \emph{free-to-play}, need on kõige jubedamad, kus raha eest võid 
endale elu osta. Hiljuti mängisin seda Fallout 76-te, parasjagu huumoriga, 
jälle  levelid on kõrgel, \emph{quest}-id kõik viimseni tehtud, midagi teha ei 
ole. Käi ringi, kogu sodi, et see sodi maha müüa. Kuu aega mängitud ja kõik. 
Aga kui meenutada viimast tõsiselt head mängu viimasest ajast, jätame 
EverQuest-id sinna kaugustesse, siis üks hea nimi oli Fallen Earth. Ka 
niisugune hea  tuumasõja ja kataklüsmide järgne maailm, millel oli, ma 
ütleksin, kõige parem graafiline ja mängijate vahelise äri süsteem. Muidugi 
vana klassika LOTRO, Lord of the Rings Online, Tolkieni fännidele, kelle hulka 
ma ka ennast loen. Ja, kindlasti Secret World. Väga kõva ja paljulubav nimi, 
aga jälle, nagu on, raha sai otsa ja teiselt poolt oli kahjuks suure fännibaasi 
taga ajamine. See on nagu \emph{modern horror}-i tüüpi,  kõik need Lovecraft'id 
ja Poe'd. Selline maailm, kus mingisugune must kaasaja maagia üritab  maailma 
tungida ja salaühingud võitlevad selle vastu ja ka omavahel. Templirüütlid, 
illuminaadid, dragon'id Hiinas\ldots  Aga ma ei ole näinud nii mõttekaid ja 
põhjalikke ja keerulisi \emph{quest}-e ühelgi mängul. Enamikel on 
\emph{quest}-ide, või noh ülesannete, värk muutunud nii labaseks: mine 
keldrisse, tapa 10 rotti. Nüüd, suur kangelane, maailma päästja, mine uuesti 
keldrisse, tapa 10 rotti ja too sabad ka ära! Või vii pakikene kõrvalkülla onu 
Juliusele. Andke andeks, kas veel lollim saab olla, aga nii ta on.

EverQuestis oli kohti, kus sa pidid näiteks võtma midagi kiilkirjas, sa pidid 
teadma  hieroglüüfe ja muidugi on selge, et keegi neid asju nii täpselt ei tea. 
Selleks oli mängu sisse ehitatud Google'i brauser, sa ei pidanud mängust välja 
minema, said sealt abi otsida. Näiteks oli üks koht, kus oli mingi vihje 
nimega. Leidsid laiba, mille juures oli saatmata postkaart sellele nimele. Ja 
kui hakkasid otsima tuli välja, et see oli üks Saksa kõvemaid krüptograafia 
alusepanijaid, keda väga vähe teatakse. Otsid siis välja: tal on olemas 
spetsiaalne algoritm. Kes tahtsid, võisid seda algoritmi käsitsi kasutada. Aga 
sa võisid  programmi alla tõmmata ja read sinna sisse kopeerida. Ja kui laisk 
olid, pildistasid ekraanilt ära, OCR-isid tekstiks, lasid teksti sinna 
programmi ja tagasi tuli juba mõtestatud tekst, kuhu sa minema pidid.

\chapter{Anto Veldre}
\index[ppl]{Veldre, Anto}

\question{Kuidas sina arvutite juurde 
said?}

Minu ema töötas ülikooli arvutuskeskuses\index{Tartu Ülikool!Arvutuskeskus}.

\question{Millise ülikooli?}

Tartu Ülikooli. Eestis ei ole rohkem ülikoole, ülejäänud on 
\enquote{tech}'id. Ema õppis ülikoolis matemaatikat ja 1959. aastast läks 
arvutuskeskusse. Ma ei tea, kas ta töötas seal juba põhikohaga või 
katsetas niisama, aga igatahes tegeles ta Ural-1\index{Ural!Ural-1} juures 
programmeerimisega. 

\question{Kas arvutuskeskus asus Liivi tänaval?}
Ei, ma arvan, et see oli ülikooli kõrval majas, kus praegu on biofüüsika ja 
kohvik. Ma usun, et see oli 1959. aastal seal. Ma ei tea, mida nad mu 
isaga vahepeal tegid, abiellusid ja midagi veel, aga 1961. aasta augustis 
sündisin mina. 

Minust on esimene pilt, mida näinud olen, arvuti taustal 1962. aasta 
jõuludest. Õigemini nääridest. Usun, et pildil oli Ural-1, mitte Ural-2. 
Ühesõnaga mingi imelik aparaat, mille taustal mind näidati ja mida ma 
tegelikult ise ei mäleta. 

Arvutuskeskus kolis mitu korda: küll Gagarinisse\sidenote{Praegu Jaan Tõnissoni tänav.}, siis Burdenkosse, mis 
on praegu Aia tänav.\sidenote{Siiski Veski.} Erinevad osad olid laiali. Kui hakkasin 
juba teadlikult masinatest aru saama, oli arvutuskeskus seal, kus praegu on 
punane korporatsioonihoone\sidenote{Eesti Üliõpilaste Seltsi maja 
Jaan Tõnissoni 1.}. Seal majas oli Ural-4\index{Ural!Ural-4}, aasta pidi olema umbes 1973\sidenote{Ürikute järgi kolis Tartu Ülikooli arvutuskeskus Liivi tänava 
hoonesse 1972. aastal.}. 

\question{Kas su vanemad olid programmeerijad? Mida nad tegid Uraliga?}

Ema oli arvutikeskuses põhikohaga matemaatik. Kogu arvutuskeskus oli täis
kõrgema haridusega naisterahvaid, kes puistasid varrukast 
korrelatsioonimaatrikseid ning arvutasid lehmade boniteeti ja teab mis 
jubedusi veel. Isa oli bioloog, 
ta muidu töötas zooloogiamuuseumis, aga käis hobiprogrammeerijana haltuurat tegemas. Profid olla teda vihanud, sest ta leidis alati 
mingisuguse lokaalse optimumi. Näiteks kõvaketta \emph{interleaving}'ut siis 
veel ei tuntud, aga oli magnettrummel, mille peal Ural-4 oma mälu pidas. 
Vanamees arvutas välja trumli pöörlemiskiiruse ja hakkas oma 
progesid tegema niimoodi, et seni, kuni tema muid asja tegi, jõudis trummel sama koha peale tagasi. Ja kuigi tema programm nii-öelda 
struktuurilt ei kõlvanud kodulooma istmikku ka, siis töökiiruselt oli vist 
13 korda kiirem kui profiprogejate oma.

Nii et minu lapsepõlv oli huvitav. Isal oli kapis kaks ülikonda. 
Üks, natuke kehvem, oli Vanemuises käimiseks -- teatris käidi tollal ülikonnas, 
mitte nagu praegu. Teine, natuke parem, oli öösel \enquote{Masinasse 
minekuks}, kusjuures Masin kirjutati suure tähega ja see oligi 
Ural-4\index{Ural!Ural-4}. Mind sinna öösiti ei lastud, aga päeval ma seal 
ikka kooserdasin. 

\question{Millal sai hakkasid seal teadlikumalt käima? 
Kas juba põhikooli ajal?} 

Iseasi, kui teadlik see oli. Vene ajal oli ju oluline 
päritolu. Kui oled see neetud intelligent ja sellest kihist pärit, 
siis pead tegelema mille kõigega. Ma käisin muusikakoolis, noorte trummarite 
ringis ja oi jumal, kus kõik veel! Ma ei enam väga täpselt ei mäleta, aga 
ilmselt 1973. aasta lõpus, kui arvutuskeskust juhatas vist Tapfer\sidenote{Jüri 
Tapfer\index[ppl]{Tapfer, Jüri} oli Tartu Ülikooli Arvutuskeskuse juhataja 
aastatel 1971--1995.}, kutsuti mind juhataja kabinetti ja anti pidulikult kätte kasutajatunnus, vist
viiekohaline number. Masinaga polnud sellel midagi pistmist, 
see oli aruandluse jaoks. Pidin paksu žurnaali allkirja 
andma nagu nõukaajal ikka. Ja see tähendas, et kui masinas 
vaba hetk oli, siis tehniliselt oli lubatud ka minu programmi sealt läbi 
jooksutada. Ega ma väga edukas ei olnud, paar programmi hakkasid
enam-vähem tööle. Eks isa aitas natukene siluda.

Urali\index{Ural} küljes oli niisugune 
ese nagu laitrükkal, millel oli minu arust 128 märki reas 
ja sealt tuli paberit päris koleda kiirusega. Ühesõnaga, kiirkirjutusmasin. Umbes nelja-viieaastaselt ema töö
juures käies istusin selle laiprinteri peal, sest see kurinahk oli soe! 
Istusin seal, kõlgutasin jalgu ja vaatasin, mida operaator 
eespool teeb. Nad kodeerisid seal tähtede 
trükitihedusega Mona Lisasid ja muid pilte, kusjuures Mona Lisad 
olid suhteliselt alasti. Mind see ei häirinud, aga mäletan, et
neid hoiti nurga taga. Tehti ka muidugi Leninist ja millest kõigest veel. Üritasin 
ka mingit pilti teha, millel olid loomulikult igasugused vead sees, aga ma ei 
mäleta, kas sain selle lõpuni tehtud või mitte. 

\question{See ju tähendab, et pidid kuskilt programmeerimist õppima. 
Või korjasid selle lihtsalt õhust üles?}

Asjalikke õpikuid ei olnud. Urali\index{Ural} kohta midagi oli, aga 
see ei olnud isegi assembler, puhas masinkood. Käsukoodid: 
null-üks oli liitmine, null-kaks lahutamine ja siis 
sinna midagi järele. Aga seal olid muidugi trikid nagu assembleriski ja nende 
selgekssaamine oli ainult läbi vaeva. 

Asi algas sellest, et tegid lihtsama joonise valmis, mis ei olnud 
muidugi õige. See tuli saada perfokaardi või perfolindi peale, kaks 
võimalikku sisendit oli ja millegipärast oli lihtsam perfokaardiga. Siis tuli 
tagaruumis mõnd telegrafistitädi painata, kes käisid Kesktelegraafist 
lisatööd tegemas. Nad tagusid sisse ma ei tea mitu 
märki sekundis. See üks või poolteist perfokaarti perforeeriti ära, seda ei 
olnud palju. Programm tuli alguses ju kirjutada 
rohelise värviga trükitud plankidele, kus olid operandid ja kommentaarid, 
panna sinna oma kasutajatunnus ja mida kõike veel. 

Kui programm sai 
perfokaardile, siis tuli see viia
masinasaali ukse juurde lahterdatud kasti, nagu on pioneerilaagris 
hambaharjade jaoks. Perfokaart käis tühja lahtrisse ja kui masinal kas mõni 
perifeeriaseade ei töötanud, tähtsamaid programme ei saanud teha või 
operaatoril oli öösel igav, võttis ta need naljaasjad ja lasi läbi kuni 
esimese veani. Siis kirjutas perfokaardile jõleda käekirjaga mingi jõleda 
kommentaari ja võibolla pani väljatrükitud paberi ka sinna. 

\question{Kogu selle vaeva läbimiseks pidi sul ju mingisugune põhjus olema.}

Mina ei tea. Miks mõned poisid käivad näiteks jalgpallis käivad? Ma ei usu, 
et sellele on ratsionaalset selgitust, ma lihtsalt kasvasin selle asutuse seinte vahel 
üles. Muidugi sai seal ka lollusi tehtud. Aia tänaval oli 
vahepeal üks ruum, kus oli paarkümmend tädi Robotroni ja Rheinmetalli 
mehaaniliste arvutitega. Mulle meeldis neid jagamistehtega kinni lasta, aga 
pärast tuli mehaanik kutsuda ja siis sain sõimata. 

\question{Kuidas mehaanilist asja kinni jooksutada?} 

Jagamistehe ei lõpe kunagi. See üritab kelguga kogu aeg edasi jagada, kuni 
kelk jookseb ühele poole kinni ja midagi läheb nässu.

\question{Jagamistehe lõpeb ju millalgi ära!?}

Ei lõpe, nulliga jagamine näiteks ei lõpe kunagi. 

\question{Miks sa panid mehaanilise arvuti nulliga jagama?}

Põnev oli lolli masinat kinni jooksutada. Kes piinab kasse, 
kes paneb bensiinitünni põlema ja kes laseb Rheinmetalli kokku.

\question{Rheinmentall muidugi ei kiunu \ldots} 

Ei, see ragises ja logises, kuna see ei olnud ettenähtud olukord. Minu arvates oli 
nalja kõvasti. Tädide arvates mitte. 

\question{Kas mehaanik hammustas läbi, mis juhtunud oli?}

Muidugi, ja ma sain sõimata. Ega ma ainuke olnud, lapsi oli seal
teisigi.

Ühte programmi tegin veel. Üks insener 
tekitas Urali\index{Ural} külge heli{\-}generaatori. Hiljem kuulsin, et 
see oli praktiliselt igal tolleaegsel arvutil -- Covox Speech 
Thingi\sidenote{Lihtne väline audioseade, mis võimaldas arvutil läbi paralleelpordi 
heli väljastada, koosnedes hulgast takistitest, mis 
moodustasid primitiivse digitaal-analoogmuunduri. Koolipoisid jootsid neid
üheksakümnendatel ise kokku ja pusisid neile ka sobilikud draiverid. Näiteks üks suursaavutus oli \enquote{NetHacki}\index{NetHack} paaritamine \enquote{Warcraft II} 
heliklippidega. Tekstipõhise mängu ekraanipuhvrist loeti kindlast kohast tekst, 
sõnadega olid vastavusse pandud helifailid ja need mängiti Covoxi abil maha.} 
eellane. Andsin arvutile lolle käske 
mõttetute argumentidega, käsukood loeti välja ja kui see oli näiteks 01, tehti 
madalat häält, 02 oli juba natuke kõrgem hääl ja niimoodi sai laulukesi 
teha. Kuna ma käisin muusikakoolis ka, siis üritasingi programmi teha. 
Seegi ei saanud kunagi valmis, alati oli viga sees. 

\question{Midagi see ju ometigi tegi, vähemalt piiksus?}

Muidugi, lihtsalt mõni noot oli vale. Ega see ei olnud 
Sibelius, Cubase või Fruity Loops, millega kuuled kohe! Oi ei -- operaatoriga pidi kokku leppima, et lähed ja kuulad. 
Selline raske elu oli arvutuskeskusse sündinud lapsel. 

\question{Kui vana sa olid, kui neid programme tegid?}

Ilmselt 12--13, maksimum 
14. Kui olin 13, sai Liivi\index{Tartu Ülikool!Liivi 
õppehoone} tänava arvutuskeskus lõpuks valmis. Uued masinad koliti üle 
ja Ural\index{Ural} visati üldse välja. Igatahes 1975. 
aastaks oli see kõik pidulikult läbi. 

\question{Mis Uralist sai? Kas läks lihtsalt utiili?}

Paraku jah. 

Algul läks Ural-1\index{Ural!Ural-1} 
Nõosse\sidenote{1965. aastal, sellest sai alguse Nõo Keskkooli\index{Nõo 
Keskkool} arvutiõpe.} ja siis ka Ural-2\index{Ural!Ural-2}. Selle
plokke vedeles veel Tartu vahel, kui Ural-4\index{Ural!Ural-4} töötas. Jõe 
ääres, keskmise silla juures, oli füüsikamaja, kust sai onude 
käest plokke kaubelda. Nii et olen ka trigeriplokke 
näinud, 6N9S või 6N8S lambi peal. 

\question{Mida sa edasi tegid? Sebisid end Liivi tänavale?}

Ei. Ma olin juba siis kuulus isemõtleja, aga Nõukogude Liidus 
isemõtlemist ei sallitud, nii et selle etapi võib arvutite 
koha pealt vahele jätta. 

Nõukaajal elasin, nagu suutsin. Lihtne ei 
olnud, ülikooli ei lastud ja mõned muud probleemid olid veel. Raha
teenisin aparaatide parandamisega. Tulin 16aastaselt 
Tallinnasse ja läksin polütehnikumi\index{Tallinna Polütehnikum} 
raadiotehnikat õppima. Sealt tuli kirg tinutuskolvi vastu, nii et vahepeal ei tegelenud ma arvutitega väga 
pikalt ja lõbustasin ennast 
elektroonikaga. Segastel aegadel tagas see
äraelamise. Kõik meenutavad, kui raske oli murdehetkedel, kui 
poes ei olnud midagi. Mina seda ei mäleta selles mõttes, et 
mul olid härrased vorstiga ukse taga, sest hommikul kell 
kuus algas jalgpalli-MM ja telekas oli katki. 

\question{See tähendab, et sa pidid kuskilt teadmisi üles korjama? Polütehnikumist?} 

See oli veel üks hobi. Ural-4\index{Ural!Ural-4} 
taga oli ju ka toatäis insenere ja ostsillograaf 
oli kogu aeg arvutil ligi. Ma ei oska öelda, kust ma selle täpselt üles 
korjasin. Kusagilt sealt. 

\question{Paljud räägivad, et süsteemset õpet 
on vähe olnud, aga kuskilt teadmine tuli.}

Nõukaajal algas süsteemne ju sellest, et pidid olema kodumaale lojaalne ja 
igatpidi väga standardne ja siilisoenguga. Siis sind võibolla lasti kuhugi 
õppima ja lõpetasid kuskil salajase töö instituudis. See oli ametlik \emph{track}. 

Kui ma poisikesena Tallinnasse tulin, siis sai just Küberneetika Instituudi\index{Küberneetika Instituut} maja valmis. Arvutuskeskus valmis esimesena, see oli kõige kallim. Käisin seal
mikroskeemide käsiraamatuid nuiamas. Läksin tagauksest, \emph{social 
engineer}'isin ennast Kevin Mitnicku moodi sisse\sidenote{Kevin Mitnick oli aegade alguses tuntud häkker ja tegutseb praegu turvakonsultandina. Tema ja tema juhitud seltskonna oluliseks häkkimisvahendiks oli seadmetele ligipääsu hankimine lihtsalt inimestega suhtlemise või siis ka näiteks prügikastidest ära visatud manuaalide otsimise teel.} ja 
seletasin, kuidas mul on tähtis konstruktsioon pooleli, aga ainult 
kahe mikroskeemi parameetrid on veel puudu. Sain salajase 
käsiraamatu nii-öelda kohapeal kasutamiseks kätte. Nii see asi käis. 

\question{Kas Küberneetika Instituudis olid need raamatud olemas?} 

Jah, see oli üks koht, kust neid sai. Selliseid kohti oli Tallinnas veel, 
näiteks sõjatehased. 

\question{Nii et sa olid põhimõtteliselt vabakutseline, vaba mees?} 

Ei, käisin siis tehnikumis ja õppisin raadiotehnikat. 

\question{Millal moodsad arvutid sinu juurde jõudsid?}

Sinna vahepeale jääb veel segane aeg, kui üritasin 
Vene arvuteid parandada, näiteks Iskra-555\index{Iskra!Iskra-555}. Pärast 
tehti ka Iskraid Intel 8086 kloonide peale, aga Iskra-555 oli ise 
leiutatud magnetkaardi pealt töötav raamatupidamisarvuti. Olen remontinud 
ka suuri mehaanilisi Robotroni raamatupidamisarvuteid, mis on 
Rheinmetalli moodi. Aga kui sa nii-öelda ametlik mehaanik ei ole ja 
sul ei ole kogu dokumentatsiooni, on see õnnemäng. Kuna mehaanikuid telliti 
Moskvast ja ma ei tea kust, siis aeg-ajalt lasti mind ligi. Kogemuse sain, aga 
head mälestust ei ole. 

Vist 1989. aastal sattus vend CeBITile. 

\question{1989 oli ju veel nõukogude aeg!}

Oli jah nõukaaja lõpp, aga ma aastaarve täpselt ei mäleta. Igatahes pani ta kõik oma elusäästud kokku 
ja tõi sealt valge portatiivse Taiwani läpaka Bondwell, millel oli kaks 730 flopit. Vend tegi sellega tööd, aga aeg-ajalt sain seda näppida ja imelikke asju teha. Sellega sai isegi Turbo 
Pascalit\index{Turbo Pascal} käivitada, aga selleks oli vaja 
kahte flopit. Ühe peale ei mahtunud ära. 

Ega must väga Pascali\index{Pascal} progejat polnud, tegin paar näidet 
ja keegi teine silus need ära ning andis tulemuse. Idee oli 
õige, aga näpud lühikesed, sest ma ei olnud seda õppinud. Ma ei ole kunagi
progemises kõva olnud.

1993. aasta alguses, kui oli selge, et nüüd on Eesti riik, ja vahepealne segane periood sai otsa, sattusin tööle 
õpetajaks Tallinna 
43. kooli\index{Tallinna 43. Keskkool}, praegusesse 
tehnikagümnaasiumi\index{Tallinna Tehnikagümnaasium}\phantomsection\label{sisu:43kool}. 
Selles majas oli kaks juriidilist asutust: kool ja 
neljandal korrusel kadunud Ants Reili\index[ppl]{Reili, Ants} 
tehtud ETEK ehk Eesti teaduslik-tehniline 
ettevalmistuskeskus, nagu see nõukaajal käis. 

Reili Ants oli 
kihvt vanamees, õpetas tööõpetust ja tal olid telekas 
tööõpetuse saated. Ta tagus kuskilt välja mingi eksperimentaalse raha 
ja tegi kooli neljandale korrusele keskuse, võttis TPIst vanad elektroonikud tööle ning saavutas selle, et 
kooliõpilastele hakati seal tehnilisi aineid andma. Seal olid arvutid, näiteks vana 
Elektronika\index{Elektronika}, ja isegi mehed, kes neid parandada oskasid, mis oli tollal 
täiesti kriitilise tähtsusega. Aga need vanad mehed ei saanud lastega suurt hakkama ja 
mina olin nii-öelda päästerõngas, kes pidi hakkama tunde andma. 

\question{Kuidas sa sinna sattusid? Kas tutvuste kaudu?}

Ema töötas seal kunagi psühholoogina, see on keeruline lugu. Selle taga on 
tegelikult Keevalliku\sidenote{Andres 
Keevallik\index[ppl]{Keevallik, Andres}, Tallinna Tehnikaülikooli rektor 
aastatel 2000--2005 ja 2010--2015.} pärastine stiil, miks TPIst nii palju välja 
langetakse. Poisid lähevad kooli, et oma eriala kätte saada, aga elama ei õpeta neid keegi -- kuidas õlut korralikult 
juua ja õhtul klubis käia. Lisaks tuleb tööandja, kes võtab esimeselt 
kursuselt juba inimesi ära, ja kui kogu seda asja kokku miksida, siis kukutaksegi välja. 
Enne Keevallikut oli Ants Reili\index[ppl]{Reili, Ants} üks selline, kes 
sai sellest põhimõttest aru ja saavutas selle, et 43. kooli\index{Tallinna 
43. Keskkool} lõpueksamid olid põhimõtteliselt ühtlasi 
tehnikaülikooli\index{Tallinna Tehnikaülikool} sisseastumiseksamid. 

Need, kes tehnika eriala valivad, ei vali seda 
mitte sellepärast, et nad on lollid ja jobud. Tark inimene läheb ju arstiks ja 
advokaadiks. Tegelik põhjus on verbaalne võimekus -- 
kui sa ei suuda seletada kiiresti ja korralikult, mida sa tahad, siis 
arvatakse, et ah, mingi pagana tehnikanohik. Reili aga
ajas sinna kooli kokku inimesi, kes nende poistega kolme aasta vältel 
tegelesid ja õpetasid neid oma mõtteid inimese moodi väljendama. Ja minu ema sattus ühel hetkel sinna psühholoogiks. 

\question{Ta oli ju programmeerija?}

Ta tegeles vahepeal kutsehariduse testidega. Aga 
see kukkus kuidagi nii välja, et mina soovitasin alguses 
Reilile\index[ppl]{Reili, Ants} oma ema ja ema soovitas pärast mind -- 
ühesõnaga see juhtus kuidagi rekursiivselt tutvuste kaudu. 

1993. aasta jaanuaris olin ma seal igatahes paigas ja mulle öeldi, et neljandast 
veerandist pean hakkama juba kellelegi midagi 
õpetama. See oli teisejärguline, kellele ja mida,
projekti eesmärgid pidid olema täidetud.

Põhimõtteliselt mind pandi olukorda, kus oli mingisugune Unix. Mu käest küsiti: \enquote{Tead, mis Unix on?} Ma ütlesin: \enquote{Jaa, ma olen 
vähemalt ühte raamatut lugenud ja umbes saan aru.} Kaks 
kuud läks selleks, et ise aru saada, mis asi see ikkagi on. Seejärel aeti lapsed 
ette ja tuli neile õpetada. Tegu oli SCO 3.2.2\index{SCO UNIX} masinaga, kuhu oli 
Tõraverest kaks Urania\index{Urania} muxi kaarti ostetud, nii et 
sellele sai kaks korda kaheksa terminali taha võtta, 
pluss oma konsool. 

\question{Mis masin see oli?}

386 -- 8 mega mälu, 40 mega ketast ja SCO Unix 3.2.2\index{SCO UNIX}. 

\question{Huvitav kombinatsioon! Kas selle organiseeris seesama koolidirektor?}

Ants Reili\index[ppl]{Reili, Ants} ja tema sõbrad ja sugulased. Kooli direktor 
oli proua Errit\index[ppl]{Errit, Anneli}, tema õpetas vene keelt. 

Eks see oli nii-öelda mentaalne ülekanne vanast \emph{mainframe}'i ajastust. Need mehed mõtlesid lihtsalt niimoodi, sest nad olid sellega üles 
kasvanud. 

\question{Kas 386 vedas tõesti kuutteist terminali?}

Oi, väga hästi! Ega siis ei progetud nagu praegu, et \emph{include}'itakse 
kogu eelnev maailm. Siis kirjutati asju asmis ja korralikult. 

Mulle anti kaks seltskonda: kaheteistkümnendikud ja viiendikud. 
Kaheteistkümnendikega oli veel nii ja naa, aga mida sa nendele viiendikele seletad 
aastal 1993? Aga, nagu öeldud, 
seal olid vanakooli mehed ja süsteemiülem
Sven Turnau\index[ppl]{Turnau, Sven} proges ANSI Cs 
nagu issand jumal ning tegi mulle paar abivahendit. Üks oli mäng, millega 
sai terminali ekraanile tärne joonistada, nagu ma isegi kunagi arvutuskeskuses 
tegin, nii et see oli tuttav asi. Ja see proge töötas, ei kiilunud kinni, lapsed said aru ja pärast sai printerist välja lasta. 

TPI laost saime kilode kaupa vana murdekohtadega paberit, selle 
eest maksma ei pidanud. Printeri lindi eest küll pidi, aga Reili eelarve elas 
selle kuidagi üle, nii et lastel oli praktiline väljund. 
Joonistasid oma jubeduse ekraani peal valmis ja trükkis välja -- ühe tunniga 
tehtud. Teine põnev asi oli see, et SCO Unixil\index{SCO UNIX} oli meil
sees ja üksteise masinate piires sai kirju saata. Mari sai Jürile 
teatada, mida ta temast arvab ja tema emast ja nii edasi, ja seda nad ka väga 
aktiivselt tegid. 

Järgmisel õppeaastal see kõik jätkus. Olukord läks huvitavaks 
aprillis, kui mind veeti Nõkku\index{Nõo Keskkool} Unixi 
koolitusele -- juhuks kui ma veel millestki aru ei saanud, siis et nüüd ikka 
tõesti ise ka aduksin, mis see on. Koolituse korraldas observatooriumi\index{Tõravere 
Observatoorium} all tegutsev Urania\index{Urania}-nimeline firma, Margus Liiv\index[ppl]{Liiv, Margus}, Kaiti 
Kattai\index[ppl]{Kattai, Kaiti} ja kes nad seal olid. Muide, sellest päevast, kas 2. või 5. aprillist, 
hakkab mu digiarhiiv peale.  

Pärast Urania koolitust käisime Nõost\index{Nõo Keskkool} 
ka läbi, et vaadata, mis seal koolis tehakse. Nõos oli 
selline härrasmees nagu Kill Kask\index[ppl]{Kask, Kalju}\index[ppl]{Kask, 
Kill|see{Kask, Kalju}}, kes rääkis: \enquote{Ah, me ripume interneti küljes, vahetame kirju ja 
trillallaa-trullallaa.} Neil oli laual mingisugune karbike ja ma küsisin, mis see on. 
\enquote{See on modem!} \enquote{Ahah.} Rohkem ma ei julgenud küsida. Kui 
modem, siis modem. 

Netti ei olnud ja ma ei tea, kuidas ma selle 
selgeks tegin, aga mõne päevaga oli kontekst selge, mis aparaat see modem on 
ja mida sellega saab teha. Kirjutasin Avatud Eesti Fondi\index{Avatud Eesti Fond} 
taotluse, et tahan ka seda saada. Kool 
saigi raha, nii et 1993. aasta sügisel saime modemi. 
SCO Unix\index{SCO UNIX} võttis selle ilusasti taha ja selgus, et meil 
on relv! Kuidas normaalsetes koolides tollal meili saatmine käis?
Näiteks Tartu Treffner\index{Hugo Treffneri Gümnaasium}, kus sai ka külas 
käidud, oli väga hästi varustatud kool. Neil oli palju arvuteid, üle kümne, aga modem 
oli taga ainult ühel ja selle arvuti taga oli järjekord. Üks õpilane toksis
kahe näpuga oma kirja, saatis minema ja teised ootasid järjekorras.

\question{Sul olid ju terminalid, see on arhitektuurselt palju parem!} 

Absoluutselt! Ja see oli relv. 

Anne Villems\index[ppl]{Villems, Anne} korraldas Tartus
õpilastele mänge, et nad internetiga ära harjuksid. Näiteks üks oli \enquote{Gaia}\sidenote{Gaia käivitus 1994. õppeaasta alguses ja sellest 
võttis osa 20 gruppi 17 keskkoolist või gümnaasiumist.}, kus olid väljamõeldud riigid. Meie kooli nimetati Barbariaks, 
väga õige nimetus!

Nagu öeldud, meil ei pidanud keegi järjekorras seisma, 17 inimest (tegelikult 16, aga 
kui Turnaul\index[ppl]{Turnau, Sven} oli hea tuju, lubati keegi konsooli taha 
ka) võisid korraga oma kirju trükkida. Masin korjas need kokku ja saatis ära
siis, kui oli aega. Aga sellega asi ei piirdunud -- SCO 
Unixis\index{SCO UNIX} sai modemi sissehelistamise peale häälestada. Saime Eesti Energiast\index{Eesti 
Energia} paar vana 1200boodist modemit (ilma veakorrektsioonita, 
ürgaegne värk), Soome tähtedega terminale ja muid
koledaid asju. Soome tähed ju muudavad ära ASCII lõpu, kus on nurksulud, sinna 
tulevad nende ö-d ja ü-d. Aga selle tulemusena õnnestus aktiivile, 
kolmele-neljale kõige aktiivsemalt arvutiklassis käivale poisile saada koju 
modemid ja terminalid, sest kes see kannatas siis endale arvutit osta. Siis
läks asi hulluks kätte, sest poiss logis ennast öösel kooli SCO 
Unixisse ja trükkis kirja valmis. Seepeale ütles Anne Villems\index[ppl]{Villems, 
Anne}, et oleme tehnoloogiliselt teistest nii palju üle, et 
see ei ole enam aus. 

\question{Huvitav on see, et sa ei olnud üle mitte tehnoloogiliselt, ägedama arvutiga, vaid just \emph{setup} oli äge!} 

Jah, organisatoorne pool, sest need vana kooli mehed teadsid, kuidas 
seda püsti panna, ja see sissehelistamine oli väärtuslik. Näiteks Microsofti 
masinatega ei olnud kellelgi mingit sissehelistamist. Ja kui poisil öösel 
kell kolm und ei olnud, aga tuli inspiratsioon peale ja ta tahtis 
\enquote{Gaia} mängus kaasvõitlejad teise planeedi peal saata, siis tal oli selleks 
täielik tehniline võimalus. 

\question{Legend räägib, et umbes sel ajal tehti Eestis esimesed veebmasterite 
kursused. Sina olla seal ka tembutanud.} 

Ei mäleta, aga sealt hakkas üks teine rida. Nimelt sain kiiresti aru, et see 2400ne 
modem on nabanöör. Selle asemel et öösel koju magama minna (mul 
ei olnud kodus siis veel terminali), istusin 1993. aastal ja 1994. aasta alguses 
öösel koolis ja rippusin horos.kbfi.ee\index{horos.kbfi.ee} küljes, kuna sinna sai 
sisse logida. Tallinna välisühendus oli tollal 64 kilobitti sekundis. 
Kuna normaalsed inimesed öösel magasid, siis viis ja pool 
kilobaiti sekundis oli maksimumkiirus, mille kätte sain. See masin 
tõmbas muidu uudiseid, \emph{news group}'e, aga mina sain mööda netti ringi 
kolistada ja terminali ekraaniga asju tõmmata. Öö jooksul suutsin tavaliselt 
umbes viis flopitäit kohale tõmmata. Kui järgmine päev tunde ei olnud, magasin 
välja ja läksin KBFIs\index{KBFI} Andres Baumani\index[ppl]{Bauman, Andres} juurde 
ning anusin, kas saaks tema masinast asjad flopile ära kopeerida. Teine variant oli järgmisel ööl oma 
modemiga tõmmata. 

\question{Mida sa tõmbasid?}

Tollal oli väga palju häid materjale liikvel. Ülikoolidel olid Gopheri ja FTP-saidid ning veeb just 
hakkas tulema. Minu lemmikmeetod oli see, et läksin 
mõnda uudisgruppi, nagu näiteks alt.sex, kus inimesed ikka 
käisid, ja siis mingi progejupiga, mida 
Turnau\index[ppl]{Turnau, Sven} mulle õpetas, otsisin välja ülikoolide 
aadressid. \emph{Strip}'isin nimed eest ära ja läksin käsitsi ülikooli 
FTP-serverisse kolama. Kuna tol ajal andmekaitset ei 
olnud, siis olid absoluutselt kõik asjad ripakil. 

\question{Ütlesid \enquote{anonymous} ja 
\enquote{ftp}\sidenote{Levinud ja kasutajate hulgas hästi teada viis 
avalikke FTP-teenuseid pakkuda oli kasutaja \enquote{anonymous} puhul aktsepteerida kas 
parooli \enquote{ftp} või ka mis iganes sisendit.} ning saidki sisse?}

Ja, aga tollal USAkad ei osanud seda veel karta, nii et 
harvesteerisin vastuoluliste nimedega gruppidest, sest need olid kõige 
suuremad. Sealt sain maksimumkoguse ülikoolide nimesid ja nende järgi tõin FTP-saidist ära kõik, mis tundus lugemisväärne.

\question{Just lugemismaterjali, mitte programme?}

Nii palju kui mul üldse mingisugust krüptoteadmist on, sellest tugevam 
osa on sealt pärit. Seal oli ripakil küll õppematerjale, küll 
igasuguseid teadustöid. Ja tol ajal näiteks arvutiõpetus koolis polnud 
teadus, vaid šamanism. Leidsin esimesed teadustööd, mis 
seda USAs käsitlesid -- neetult huvitav oli lugeda! 

\question{Kas need olid PDFid?}

Enamasti olid tekstifailid, PDFid hakkasid hiljem tulema. 
ASCII printeriga trükkisin välja ja nii see elu käis. 

\question{Kas \LaTeX-i kraami ka leidus?}

Mina ei olnud tollal selle ala inimene, hetkel juba kirjutan selles. Ilmselt leidus. 
Emacsit ei ole ma ka ära õppinud ja ei õpigi. Vi-ga kirjutan küll. 

Pisut hiljem sain tuttavaks mehega, kes rippus samamoodi öösiti Tartu 
satelliiditaldriku taga, tema nimi on Marek Tiits\index[ppl]{Tiits, Marek}. 
Tuli välja, et mina ekspluateerisin siin Tallinna oma ja tema Tartu oma. 

Millalgi sain aru, et ühest 
modemist ei jätku ja et \enquote{otseinternet} on ka olemas. 
Jätan praegu vahele selle, kuidas Andres Bauman\index[ppl]{Bauman, Andres} 
KBFIst kirjutas grandiraha eest UUCPd ja Pegasuse meili siduva 
proge, mida koolid ja teisedki kasutasid, see on eraldi teema. 

Mis veel tehnoloogilise üleoleku salarelva puutub, siis 
loomulikult olid Treffneri omad mõnes mõttes haritumad kui 
meie, vähemalt maailmavaate poolest, aga kahur oli meil jämedam. 

Ühel hetkel hakkasin suhtlema nendega, kes õpetajaid 
koolitasid ja koolides internetti levitasid -- Anne Villemsi\index[ppl]{Villems, 
Anne} ja tema koolkonnaga. Sealt tuli arusaam, et kuidagi on vaja 
püsiühendus saada. 1994. aastal oli see \emph{mission impossible} alates 
rahast ja lõpetades kõige muuga. EENeti poisid aitasid kirjutada 
haridusprojekti, et paneme 117 kooli internetti ja tulevikus otse ka ning seega on kusagil vaja nimeserverit. 

Andres Bauman\index[ppl]{Bauman, Andres} tegi viltuse näo pähe, et koole on nii palju 
ja et tema väike vaene MicroVAXist nimeserver ei jaksa neid enam pidada. 
Arvata võib, mis selle väite tõeväärtus on.\sidenote{Nimeserver on üks 
väiksema ressursivajadusega interneti tuumtehnoloogiaid, jutuks olnud riistvara 
oli selle teenuse pakkumiseks koolidele vähemalt suurusjärgu võrra 
üledimensioneeritud.} Aga niisiis oli vaja teha eraldi koolide nimeserver. Haridusministeeriumi teadusrahadest eraldati selleks
mingisugune Sun, vist SPARCStation\index{SPARC!SPARCStation} 10. 

Ühelt koolikonkursilt õnnestus ka kolm arvutit saada. Rohkem ei antud, 
öeldi, et teistel on ka vaja. Arvuti all mõtlen ma eraldiseisvat 
Microsofti masinat. Kolmest masinast üks läkski nimeserveri alla ja 
EENeti SPARCStationist sai hariduse veebiserver. 

See oli vist 1993. aasta lõpp, kui see Sun oli juba olemas, 
peak.edu.ee\index{peak.edu.ee}, tuksus Baumani juures riiulil ja 
sain sellele kaugelt ligi. Olime kord Liivi tänava\index{Tartu Ülikool!Liivi 
õppehoone} keldris saunas ja Toomas Soome\index[ppl]{Soome, Toomas} 
rääkis, et on olemas mingi jubetumalt kihvt asi, mida nimetatakse veebiks. 
\enquote{Ahaa, rõõm kuulda!} \enquote{Tartu Ülikoolil olla ka!} 

Pärast istusime mingi terminali taha ja Toomas Soome nõidus seal 
natuke, kompileeris. Siis ma ei saanud sellest tegevusest absoluutselt aru, hiljem
poisid koolis õpetasid. Igatahes üks proge sai 
kokku: faili kirjutati rida \enquote{killadi, kolladi, 
\emph{coming soon}, siia varsti tuleb midagi}. Vaadata sai sellise
programmiga nagu Mosaic\index{Mosaic} ja oli isegi kiri, mis 
ütles, et midagi tuleb\ldots{ }Põhimõtteliselt kompileeris Toomas Soome selle laiendatud 
sauna vältel mulle veebiserveri ja tegi sinna esimese faili ühe 
reaga, ja minul lasus nüüd kohustus. Ei jäänud muud üle, kui pidin selle 
selgeks õppima. Väidetavalt oli see Eestis seitsmes veebiserver ja 
niimoodi see www.edu.ee\index{www.edu.ee} sündis. Sisu tekkis alles palju 
hiljem.

\question{Selle pidi ju keegi kirjutama!} 

Mis seal kirjutada, nuiasin haridusministeeriumi 
administraatorilt nende andmebaasi välja ja panin kõik avalikult netti. Praegu ei tohi sellist asja üldse teha, kõik isikuandmed on 
saladus.

\question{Mis andmebaas see oli?}

Tol ajal inimesed ei teadnud, kui palju Eestis koole on. Selles andmebaasis 
oli isegi koolidirektori nimi olemas! Aadress ja üheksa-kümme rida 
infot iga kooli kohta. Koole oli kokku ligi 600, aga sealt tuli valik 
teha, erikoole ei hakanud panema, ja see oligi veebiserveri esimene 
versioon. 

EENetiga\index{EENet} sai
tehtud rahastamise kokkulepe, milleks oli järgmine projekt: tegime
43. kooli\index{Tallinna 43. Keskkool} juurde koolide 
sissehelistamiskeskuse. Poisid olid teenusega hästi rahul, sest nad said 
aru, et modemeid saab muuks ka pruukida -- ise öösel sisse helistada, 
kuhu vaja. Aga trass, vaskkaabel tuli ise välja ajada, 
keldris juhtmeid kokku mässida ja takistusi mõõta. 

Ja siis oli küsimus, millise tehnoloogia saame. Vendomar\index{Vendomar}, kodanik 
Kingissepp\index[ppl]{Kingissepp, Meelis}, üritas meile 
RADi\sidenote[][-4mm]{Tõenäoliselt peab Anto silmas Iisraeli samanimelise firma 
modemeid, mis toona uudse kontseptsioonina ei vajanud toimimiseks eraldi 
toiteallikat.} müüa. See oleks olnud 64kilobitise kiirusega, aga RAD ei hakanud 
meie liini peal tööle. Meile sobis US Robotics\index{US Robotics} ja 
ka mitte päris 33,6 peal, vaid kiirus oli vist 28 või 29. Millalgi juulis saime lõpuks püsiliini kooli ja KBFI\index{KBFI} vahel tööle: 
mõlemas otsas olid modemid ja võttiski \emph{carrier}'i\sidenote[][-4mm]{Kui liinil oli olemas \emph{carrier} signaal, oli ühendus loodud.} üles! 

Siis tekkis õnnetu moment, kui olime otse interneti küljes 
kiirusega 28 kilobitti sekundis ja nurgast astus välja Antti 
Andreimann\index[ppl]{Andreimann, Antti}, kes ütles, et ta on seda hetke 
oodanud sajandeid ja et meil koolis on kõik asjad tuksis, sest meil on ainult 
vana räpane ANSI kompilaator, aga vaja oleks GNU C kompilaatorit, millega saaks 
maailma päästa. Viis ja pool tundi kõik ülejäänud ootasid, internetti kasutada 
ei saanud, sest Antti tõmbas GNU kompilaatorit. Kõik, kes olid pühitsemiseks 
kokku tulnud ja arvasid, et nüüd saab midagi tõmmata\ldots{ } See oli GNU 
kompilaator, mis sealt tuli! Aga usun, et Anttil oli õigus 
ja elu läks oluliselt paremaks, sest nüüdsest õnnestus kompileerida asju, mida 
varem ei õnnestunud. See asi ei käinud nii, et tõid 
funet.fi-st\index{funet.fi} või mõnest muust pirasaidist, Unixi puhul oli 
teisiti. 

See kuu aega, mis koolini jäi, poisid ainult koolis elasid. Augustis 
hakkasid ka esimesed nähtused tulema. Sellise internetikiirusega satuti paratamatult mõne kõrgkooli bugistesse 
veebidesse. SCO Unixil\index{SCO UNIX} oli üks 
knihv, kuidas modemit konfida, ja kui seda knihvi ei teadnud, siis 
\emph{carrier detect} jäi püsti, kui meie kehva telefoniliini pealt maha 
kukkusid. Näiteks Peda ja TTÜ adminnid kukkusid oma ruudu 
\emph{prompt}'i otsast päris sageli maha. Kuna meie poisid teadsid seda 
nippi, siis mõnikord sai niiviisi toimiva terminali otsa kätte. 

Noil aastatel juhtus igasuguseid artefakte. Seadusi hakati tegema alles hiljem, vist 1995 
ja 1996 kuulutati välja\sidenote{Ei ole selge, millist seadust Anto silmas peab, tõenäoliselt on tegu mõne arvutitega toimuvat reguleeriva seadusmuudatusega.}, ning oli loodud ka töögrupp, kus ma käisin 
Tartus Riigikohtu\index{Riigikohus} omadega midagi arutamas. No mida ma oskasin 
neile rääkida?! Nii palju kui olin netist lugenud, kes läänes mille ja kust 
ära varastas, rohkem ei osanud. Oma õpilaste käest sain ka üht-teist teada.

\question{Kas siis puutusidki esimest korda kokku infoturbega?}

Jah. Vist 1995. aastal\sidenote{Vastav lugu Äripäevas, kus ka Antot 
tsiteeritakse, on pärit 1996. aasta aprillist.} ilmus see kuulus \emph{no name} 
lugu. Eestis oli selline pisipank nagu EVEA\index{EVEA Pank}, kus töötas süsadminnina Igor 
Švets\index[ppl]{Švets, Igor}. Tehniliselt väga 
kvaliteetne häkker, kes kolistas kogu Eestit pidi ringi, kuhu aga sisse sai. Ma 
ei tea, kellele ta töötas. Ega minagi kõva käsi ei 
olnud, aga vähemalt olin võimeline aru saama, et keegi elab mul 
masinas. Eks poisid natuke aitasid ja ega Sven Turnau\index[ppl]{Turnau, Sven}, 
ametlik masinaülem, polnud ka loll mees. Sai aru, mis toimub, ja siis sai see 
skandaaliks keeratud ja jõudis ajalehte. Postimees tegi mister \enquote{\emph{no name}'ist} veel mingi noaga pildi. Press on press. 
Politseisse polnud üldse mõtet helistada, need ei saanud sellest aru ja tollal 
polnud selle kohta paragrahve ega midagi. KAPOsse helistasin, nemad ei saanud ka aru. Ühel hetkel suutsime ennast ise ära kaitsta ja aduda, et see maailm 
on tegelikult olemas. 

Keerulisem kui enda kaitsmine oli tegelikult see, 
kuidas poisse seiklustest eemale hoida. Poisid tulid ja
ütlesid: \enquote{Meil juhtus õnnetus}. Ma küsisin, mis 
õnnetus teil täna juhtus. \enquote{Saime kogemata Pedas ruuduks!} 
Pedas\index{Pedagoogika Instituut} oli Cadmus\index{Cadmus}\sidenote{Saksa arvutitootja 
Periphere Computer Systeme (PCS) toodetud tööjaam, mis käis toona kehtinud 
võimeka riistvara kolmandatesse riikidesse eksportimise embargo alla. Kuidagi 
aga õnnestus ühte Teaduste Akadeemia allasutusse selline masin hankida, kus 
selle nimeks sai konspiratsiooni huvides \enquote{Muscad}\index{Muscad}.}, mingi 
tumba. Eks nad läksid netti ja küsisid, mis bugid sel on, ja 
loomulikult sellel olid bugid ja nad said ruuduks. \enquote{Mis me nüüd teeme?} Ma 
ütlesin, et mis siis ikka, parandage bugid ära ja kinkige masin 
tagasi. Parandasidki bugi ära, kompileerisid sellele uue kerneli ja mida kõike veel (ma üldse imestan, et nad seda masinat õhku ei lasknud) ning
saatsid süsadminnile lõpuks kirja, kus nad seletasid, et on bugid ära 
paiganud ja et ta võib nüüd oma masina tagasi saada. See oli hästi 
tüüpiline juhtum.

\question{Kas need olid 12. klassi poisid, kes seal möllasid?}

Ei olnud. See aktiiv, kes üldse arvutitunde ei saanud, olid üheksandikud või 
kümnendikud. See oli lihtsalt niisugune tore aastakäik. 

\question{Nii et üheksakümnendate keskel üheksandikud võtsid ruutu?!}

Jaa! Mõnest asjast ei julge tänaseni rääkida, inimesed on ju skeenel 
alles. Õnnetusi juhtub ikka. Näiteks käis meil ühe teise kooli poiss (ta on 
ka siin\sidenote{Vestlus toimus Cybernetica\index{Cybernetica} kontoris.} 
töötanud), kelle ema oli teadustöötaja ja peres oli raha rohkem või saadi aru, mis on 
oluline, ja temal oli tollal 486. Nähes SCO Unixit\index{SCO UNIX} ja seda igavest 
muret seerianumbritega, kirjutas ta programmi, mis neid 
seerianumbreid nagu küllusesarvest väljastas. Ta ei olnud veel keskkooligi ära 
lõpetanud. Tema oli ka esimene, kes õppis Jack the Ripperiga\sidenote{John (mitte 
Jack) the Ripper on tuntud paroolide murdmise töövahend.} ümber käima. 

Tallinnas Tehnikaülikoolis\index{Tallinna Tehnikaülikool} oli Suni klass, kus olid 
mingid vanad hädaabi masinad ja arvutiklassi pealikuks Rebane\sidenote{Tõenäoliselt peab Anto silmas Enn 
Rebast\index[ppl]{Rebane, Enn}.}. Tol ajal oli
tsentraalne lahendus NISi peal -- tänapäeval ei ütle see sõna mitte midagi --, 
aga tolles lahenduses oli paroolifail kättesaadav. See mainitud
kodanik, kellel oli natuke parem masin (tal oli kodus ka SCO 
Unix\index{SCO UNIX}, kujutad ette!), lasi Jack the Ripperi käima ja sai vist kolme päevaga 70 protsenti paroole lahti. Olid ajad!

Ja siis oli tollal ju veel läbi telefonide helistamine. Modemeid oli maru palju, aga 
turvat mitte mingisugust. Oli olemas selline proge nagu Tone Loc -- kui praegu pingid 
läbi IP-aadresside plokke, siis selle progega sai telefoninumbreid läbi pingida. Kohalik kõne ei maksnud ju midagi. Nii et mina 
olen ka korra elus näinud, kuidas ühest autost, kus on kolm \emph{laptop}'i, 
lähevad krokodillidega juhtmed öösel telefonikappi ja lastakse kümnetuhandeseid 
plokke läbi. Sinna etappi jäävad Eesti Telefoni digikeskjaamade esimesed 
häkid. Mingid transpordikooli poisid said aru, kuidas digijaam töötab ja et paroolid on nirud, ning võtsid detsembris, kui kõik olid jõulupühal või 
välismaal komandeeringutes, mitu jaama üle. Midagi hirmsat küll ei tehtud. 

\question{Kuidas neid poisse siis kantseldati? Neil pidi ju olema mingisugune 
eetiline arusaam, et \enquote{pätti ei tehta ja puruks ei lasta}?}

Ma ei tea, kes need transpordikooli poisid olid. Ma mõnda õpetajat 
tundsin, aga see oli pikantne teema, sest kui saad liiga 
targaks ja täpselt teada, kes need olid, siis 
tekib endal moraalne kohustus. Mina lahendasin selle nii, nagu mulle õige 
tundus: püüdsin neid suurest jamast eemale hoida ja ega see tavatarkus, 
et mis sa nüüd tegid, ei aidanud. Pidid ta ära kuulama ja 
siis väga ettevaatlikult õige tee peale tagasi patsutama. 

\question{Sest tol hetkel oli murdeealisel poisil võimekus 
märksa suurem kui mõistus või arusaam elust!} 

Võimekus on tal suur sellepärast, et ta veel õlut ei joo, selle peale aega, 
energiat ja raha ei kulu, ja tütarlastega ka veel ei semmi. Järelikult, kui parasjagu pole koolipäev, siis on kuusteist tundi vaba aega, 
sest kaheksa tuleb magada. Ja kui sul on inimene, kellel on 16 tundi päevas 
aega istuda ja murda, siis see on väga toores jõud. Rääkimata sellest, kui neid on mitu ja 
nad suhtlevad omavahel.

\question{Kui kaua sa koolis töötasid?}

Täpselt ei mäleta, aga kolm aastat olin seal 
kindlasti. Ühel hetkel, kui hakkasin EENeti\index{EENet} keskust 
tegema, kujunes EENet teiseks tööandjaks. See võis olla 1994. aasta lõpus või 1995. aasta algul. 

Minu ülekandumine 
EENeti toimus kuidagi väga sujuvalt. Ametikoha nimi oli \enquote{insener}, 
põhjaregioonis, tipphetkedel vastutasin ligi 150 kooli 
UUCP side eest. Öösel kell 11 helistas mõni tädi, kooliõpetaja, ja rääkis, 
kuidas tal mitte miski ei tööta. Pidin peas looma mentaalse mudeli ja 
nagu \emph{helpdesk} tema katkendite järgi tegutsema. \enquote{See sinine lätakas} 
-- ahah, selge, Norton Commander. 

Tegin ka mõned koolitused uute metoodikate teemal, näiteks vist 1995. aastal Anne Villemsi\index[ppl]{Villems, Anne} juures 
Tartu Ülikoolis\index{Tartu Ülikool}. Sissehelistamiskeskuseks oli meil minu läpakas. Telefoni keskjaamast tõime 
pikad otsad (kaks päeva oli ettevalmistamist), mida sai klassis iga arvuti 
juurde viia. Ja kooliõpetajatega koolitust alustasime niimoodi, et kõigepealt 
käskisime neil arvutit lüüa, et saaks hirmust üle ja tekiks kohe õige tunne, kes on peremees. Tegelikult oli klassi valdajaga kokku 
lepitud, mis juhtub, kui mõni arvuti \enquote{ära lüüakse}. Aga nad nii kõvasti 
ei julgenud lüüa. 

Teine asi oli hoida tädisid ülearuse info eest. Näiteks modemi konfigureerimisel 
ütlesin, et ära süvene sellesse, mis see on, siin on täpselt ainult 
nii palju valikuid, vali neist üks ära! Tulemus oli see, et kõik koolitusel olnud said oma modemid konfitud, läksid oma koolidesse laiali 
ja said seal ka hakkama. Aga selle ettevalmistamine oli viiskümmend ühele: ühe tunni 
hoolitsuse kohta, mis nad said, läks 50 tundi ettevalmistust. See on üks kõige 
jubedamaid asju olnud, aga metoodika mõttes toimis! 

Kui Ants Reili\index[ppl]{Reili, Ants} suri, siis lagunes ETEK ära. 
Kauplesime endale tehnikaülikoolist\index{Tallinna Tehnikaülikool} Jüri 
Kaljundi\index[ppl]{Kaljundi, Jüri} käest vana Väksu\index{VAX}, ainult et 
transa käigus sai midagi pihta ja see ei läinud enam tööle. Koolidirektor tundis suurt
huvi, kust ma masina toiteks elektrit kavatsen võtta, nii et käima see 
ei läinudki, vaid viidi lõpuks kuhugi utiili. Ühesõnaga, see \emph{business} vajus 
sellisel kujul koost ära. Tipphetk oligi natuke aega enne mind ja minu 
ajal ning mina kandusin rohkem EENeti. 

Nii et enne, kui 1997. aastal järgmise pöörde tegin, istusin 
Tatari tänaval MicroLinki kõrval üleval (seal oli ühes toas EENeti kontor), kus 
üks Fixi kunagine helitehnik, Antti Andreimann\index[ppl]{Andreimann, 
Antti} ja mina tegime kolme peale kokku, mida suutsime. Antti 
sai aru, mis tegelikult toimus, ja suutis kernelit kompileerida ning meie tegime 
muid asju. Sellega lõppes koolide saaga ära ja tekkisid 
Tiigrihüpped\index{Tiigrihüpe} ja muud asjad, sellisel kujul lähenemist ei 
olnud enam vaja. 

\question{Ühesõnaga sa jõudsid infoturbe juurde väga praktiliselt ja 
samas väga inimesekeskselt -- pidid tegelema kaakide 
ohjeldamisega, kes 16 tundi päevas igale poole auke torkisid.} 

Nad ei olnud kaagid, vaid lugupeetud inimesed, kes kõik tegutsevad siin skeenel. 

\question{Mida sa praegu teed?}

Nagu näed, istume Cybernetica-nimelises aktsiaseltsis\index{Cybernetica} ja 
ma olen hetkel kirjaneitsi. See on maru raske küsimus, mida sa oskad. 
Tööle võttes küsitakse alati, mida sa oskad. Mina oskan mõnikord (mitte 
alati) asju suhteliselt selgelt kirja panna. Siin 
majas on segaseid asju, mida on vaja pisut selgemini kirja panna, suurtes 
kogustes. 

\question{Seda sa oskasid küll väga selgesti öelda, ma sain kohe aru, mida 
sa teed!} 

Jah, sõltumata sellest, mis töökoha nimetus või milline järjekordne 
käimasolev projekt on. Tekib natuke teine vaade, 
parem sõnastus ja järgmine seltskond saab sellele tuginedes juba järgmise 
lati ära võtta -- näiteks mõne asja kuhugi riiki maha müüa. 

\chapter{Vilve Vene}
\index[ppl]{Vene, Vilve}

\question{Kuidas sina said arvutite juurde ja arvutid 
sinu juurde?} 

See algas juba koolis. Käisin 1. keskkoolis\index{Tallinna 1. 
Keskkool}, mida täna tuntakse kui GAGi\index{Gustav Adolfi Gümnaasium}, 
matemaatika-füüsikaklassis. Meile õpetati ka programmeerimist. Mul oli väga 
hea matemaatikaõpetaja ja füüsikaõpetaja, mulle õudselt meeldis, aga ma 
ilmselt ei teadnud, kas see meeldib mulle piisavalt palju või meeldib mulle midagi 
muud rohkem. Mulle meeldis ka kirjutada ja tegelikult arvasin 
keskkooli lõpuni, et lähen pigem eesti keelt ja kirjandust 
õppima. Aga kuidagipidi arenes mõte sinnamaani, et õpiks 
füüsikat. 

Läksin Tartusse, vaatasin, et seal oli üldse vist neli tüdrukut, 
mõtlesin, et jube veider, ja andsin avalduse rakendusmatemaatikasse\index{Tartu 
Ülikool!Matemaatikateaduskond!Rakendusmatemaatika}. 
Kui avaldust sisse viisin, vaatas tädi laua taga mulle otsa ja 
ütles: \enquote{Teie oma tunnistuse ja kuldmedaliga saaksite ju 
arstiteaduskonda sisse!} Selgitasin talle, et ma ei taha, ja nii see 
läks. 

\question{Mille peal teid GAGis programmeerima 
õpetati?}

Jessukestel\index{Jessuke} ja perfokaartidel. 

\question{Kas koolis oli kõik see olemas?}

Ei, me käisime Teaduste Akadeemia Arvutuskeskuses\index{Teaduste 
Akadeemia!Arvutuskeskus}.

\question{Kooli poolt väga edasipüüdlik ettevõtmine! Kes seda asja koolis 
ajas?}

Ma arvan, et tollal oli direktoriks Helmi Viikholm.\index[ppl]{Viikholm, 
Helmi}\sidenote{Helmi Viikholm oli 1. keskkooli direktor aastatel 1962--1982.} 
Selline inimene, kes suutis ka nõukogude aja \emph{setup}'is 
organiseerida. Meil olid jube head õpetajad, avatud mõtlemine ja 
nii see tuligi.

\question{Kas Teaduste Akadeemia lasi õpilased oma arvutitele ligi?}

Jah! Aga eks kood läks perfokaardi peale -- ise toksisime 
perfokaardid, need läksid masinasse ja tulemus tuli välja. 

\question{Mis programme te kirjutasite?} 

Tegime lihtsaid asju. Ma sisu ja detaile ei mäleta, aga igatahes oli 
see Fortranis\index{Fortran}. 

\question{Seni on umbes igas teises loos läbi 
jooksnud ruutvõrrandi lahendamine \ldots} 

Vaat seda ma ei mäleta, minu meelest olid meil ikkagi natuke 
ärilisemad teemad.

\question{Nii et pärast keskkooli läksid Tartusse Vanemuise tänavale\index{Tartu 
Ülikool!Vanemuise tänava õppehoone} rakendusmatemaatikat õppima?}

Jah. Meid alustas 20. Minul, kellel 
koolis oli kõik jube lihtne, oli algus hästi raske. Ja alustati 
assemblerist\index{Assembler}. Pärast olen mõelnud, et see oli väga hea -- 
assembleris pead aru saama, mis seal 
sees toimub, ei saa lihtsalt kirjutada mingeid koodiridu, mõistmata, mis sügaval sees toimub. 

\question{Selle raskuse ületamise taga pidi olema mingisugune kihk. Miks sa ei 
läinud arstiks õppima?} 

Mind ei huvitanud!

\question{Ja arvutid huvitasid?} 

Mõtlen siiamaani, et kui edasi tulid intelligentsed 
programmeerimiskeeled, kus ei pea aru saama, mis seal sees toimub, siis 
tänapäeval on programmeerimine minu meelest 
käsitöö. Selles mõttes, et nii kampsunit kududes kui ka 
programme tehes tuleb osata võtteid ja kasutada tööriistu. Tollal
ei olnud sedasi. Kihvt oli alustuseks läbi mängida, mis juhtub: 
\enquote{Okei, see rida teeb nüüd seda, paneb selle sinna ja selle sinna. Kui 
mul on nüüd vaja võtta see sealt ja teha sellega midagi, siis mida ma selleks tegema 
pean?} Said sügavuti aru ja see oli hästi vahva. 

Viis, kuidas nad õpetasid, oli karm, aga tegelikult see 
selekteeris välja inimesed, kes tõesti tahtsid seda teha ja olid selleks ka
võimelised. Alguses tuli loogikat kõvasti, samuti mat-analüüsi. Ikka 
\emph{hard core}, mitte nagu tänapäeval õpetatakse. Rääkisin just
sõbrannaga, kes õppis teoreetilist matti ja kelle kursuselt on ülikoolis 
hästi palju õppejõude. Nad ütlevad, et ei saa enam sellel tasemel õpetada, vaid peavad oma 
programme lihtsustama. 

\question{Mulle on räägitud, et mõni oli keskkooli ajal juba kuskil tööl, sest 
kangesti oli vaja arvuti juurde pääseda. Kas sul ei olnud niisugust kihu?}

Ei, seda ei olnud. Ma ei ole kunagi arvutihull olnud, see on 
võib-olla rohkem poiste teema. 

\question{Õige, mu valim on seni olnud väga kallutatud.} 

Aga Tartus oli küll nii, et asi hakkas tõsiselt meeldima ja sain aru, et see 
on õige, mida ma teen. 

\question{Millest sa aru said?}

Mulle lihtsalt meeldis! Ja see läks minu jaoks ajaga 
lihtsaks, kuigi ma ei saanud aru, kuidas ma saan 
aru. Kõlab väga filosoofiliselt ... Kuna see oli ikkagi rakendusmatemaatika, siis 
lisaks programmeerimisele oli palju teoreetilist matti. Vahel läheb matemaatiline analüüs nii abstraktseks, et sa 
tõesti enam ei saa aru, kuidas sa saad aru. Miskipärast tuli mul see jube hästi välja, kusjuures meid alustas 20 
ja pärast esimest aastat oli järel 14. 

\question{Väike grupp, aga väljalangevus ei olnudki nii suur, meie kursuselt 
läks rohkem. Aga mida sa mõtlesid
tegema hakata, kui kool läbi saab?}

Aeg oli ju selline, et valikuid loomulikult oli, aga enamik neist kuskil arvutuskeskuses. Tollal sai
kohti valida vastavalt sellele, kuidas lõpetasid ja milline oli 
pingerida. Hästi popp koht oli näiteks Raadiomaja 
arvutuskeskus\index{Raadiomaja Arvutuskeskus}, see oli number üks. 
Statistikaametisse ei tahtnud keegi minna.

\question{Miks?}

Ei tea, tagantjärele mõeldes on ju
statistikaametis töö palju huvitavam kui raadiomajas. Aga nii see 
kahjuks oli.

Minul oli selline õnnelik juhus, et sattusin Küberneetika 
Instituudi\index{Küberneetika Instituut} matemaatikaosakonda praktikale. Viiendal kursusel tegin 
diplomitööd, juhendaja oli Otto Vaarman\index[ppl]{Vaarman, 
Otto}, kes oli väga tunnustatud matemaatik. Uurisime Küberis Newtoni tüüpi meetodeid, mis kõige paremini 
lahendavad erinevaid võrrandeid. Otto oli vana meesteadlane, kes ise 
väga palju ei viitsinud teha, aga tal oli paar 
aastat varem lõpetanud noor jünger Maarika Lomp\index[ppl]{Lomp, Maarika}, kes oli 
mul sisuline juhendaja. Kollektiiv oli hästi tore ja töö 
huvitav. Toona tulid juba esimesed variandid, kus enam ei 
pidanud perfokaartide pealt toksima, vaid olid ikkagi 
Jessukese\index{Jessuke} taga ning said juba ise intelligentsel viisil oma 
koodi sisse viia.

Ja siis sattus mingi hetk täiesti ootamatult Otule\index[ppl]{Vaarman, Otto} ja 
Maarikale\index[ppl]{Lomp, Maarika} külla üks ääretult sümpaatne härrasmees 
sellisest huvitavast organisatsioonist nagu Algoritm. Tegelikult 
oli sellel asutusel hästi pikk nimi.\sidenote[][]{\phantomsection\label{sisu:algoritm} 1976. aastal asutatud Tallinna 
Teadus-Tootmiskeskus (TTTK), mis kuulus Üleliidulise Teadus-Tootmiserikoondise 
\enquote{Algoritm} koosseisu, Eestis tuntud lühendnime all Algoritm. Kuna 
asutus allus NSVLi tasemel kaitsetööstuse ministeeriumide gruppi, oli sellele
omistatud koodnimi Postkast A-3433 ja kehtestatud ka vastav tööre\v{z}iim. 
Üleliiduline alluvus seletab ka töökeelt. Asutus tegeles ES EVMi\index{ES EVM} 
hoolduse ja remondiga, IT-koolituse ja selle metoodikaga ning mitmesuguse 
tarkvara (nt matemaatika, arvutidiagnostika ja automatiseeritud 
projekteerimine) arendamise ja tira\v{z}eerimisega. Peale Tallinna olid 
asutusel filiaalid Tartus ja Kohtla-Järvel(!). 1980ndate keskel töötas 
Algoritmis ligi 1000 inimest, lisaks kaasati allhankijatena inimesi 
TPIst\index{Tallinna Tehnikaülikool} ja Tartust\index{Tartu Ülikool}. Algoritm 
lõpetas töö 1992. aastal.}. Küber oli Mustamäel. Kui sealt sõita edasi, kus on 
täna ARK, siis ühe risti peal olid koledad baraki tüüpi 
majad\sidenote{Toonase aadressiga Kadaka puiestee 165.}, kus asuski 
teadusuurimiskeskus Algoritm, millel oli eraldi matemaatikaosakond. 

Algoritm oli täielikult
venelaste sõjaline organisatsioon, kus tehti uurimistööd 
väga kummalistel aladel. Matemaatikaosakond oli ainuke eestlaste osakond ja Ants\sidenote{Ants Roose\index[ppl]{Roose, Ants}, kes hiljem töötas ka Algoritmi teadusala asedirektorina. 
Tegelikult oli osakonna nimi \enquote{matemaatika tarkvara 
osakond}. Rooselt on 
pärit kogu Algoritmi puudutav info.} selle
juhataja. Ühesõnaga tema rääkis mu ära ja läksin sinna tööle, ääretult tore kollektiiv oli. 

\question{Järelikult oli sul ikka akadeemiline siht silme ees?}

Jah, pidin tegelikult minema doktorantuuri Gennadi 
Vainiko\index[ppl]{Vainikko, Gennadi} juurde, aga kuidagipidi hakkasid asjad 
arenema ja huvitavaid teemasid tuli palju. 

Muide, Algoritmis hangiti tarkvara niimoodi, et 
keegi tundis näiteks Minskis kedagi, kes oli kuidagipidi saanud mingi softipaketi, ja meil oli seda vaja. Ülemus saatis 
minu, noore tüdruku, kes vene keelt väga hästi ei rääkinud, 
Minskisse ja ütles, et pean sellega tagasi tulema! Ja tulingi. 

\question{\emph{Millega} täpsemalt sa tagasi tulid?}

Suure lindikettaga! Nii me seda tarkvara hankisime ja testisime. Tegime tööd ja 
oli väga huvitav aeg.

Kui mul sai seal majas kolm aastat täis, hakkasid ajad muutuma. Ain 
Rasva\index[ppl]{Rasva, Ain} oli ka tollal seal ja tema soovitas mind 
ühele inimesele, kes juba toona tegi koostööd soomlastega. Läksingi aastal 1988 Tööstusprojekti\index{Tööstusprojekt}, kus olid juba tol ajal miniarvutid ja koostöö soomlastega. 

\question{Need on ju projekteerijad, mis seal programmeerida oli?}

Oi, palju! Ehitusprojekt koosneb paljudest asjadest, sealhulgas 
tugevusarvutustest.

\question{Sul oli võimalik minna akadeemilisse maailma või sealt ära. Miks sa just sellise valiku tegid?} 

Pakkumine oli tohutult ahvatlev -- võimalus teha tööd ka kuskil mujal, 
Soome projekte koos 
soomlastega. Ja ma sain õppida. Mulle anti C õpikud ja 
esimene proovitöö oli automaatsed konverterid, programmid, mis tõlkisid
Fortrani\index{Fortran} koodi C\index{C} koodiks. See kood, mis 
välja tuli, oli muidugi kohutav, aga töötas. 

1990. aastal olin natuke aega lapsega kodus. Tegin Soome väikseid töid, teenisin 
saja marga kaupa väga head raha. 

\question{Jõhker raha, sada marka!}

Jah! 1991. aasta talvel avaldati ajalehes 
kuulutus, et Rootsi-Eesti ühisfirma otsib programmeerijaid ja just naisterahvaid. Ma polnud elus niisugust asja näinud, aga tundus väga 
huvitav. Avaldusi oli üle kuuesaja. 

\question{Kuuesaja!? See tähendab, et 1991. aastal oli Eesti Vabariigis aktiivsel 
tööturul 600 naist, kes võisid enda kohta öelda \enquote{programmeerija}?}

Jah. Esimesel intervjuul selgus ka väga lihtne loogika, miks nad naisi 
otsisid. Firma taga oli üks rootslane, kes töötas Stockholmi linna ja 
lääni valitsuses (minu arust oli ta lausa IT osakonna juhataja), ja üks Eestist Rootsi läinud mees, keda mäletasin ülikooli ajast, 
Kalle Kullman\index[ppl]{Kullman, Kalle}. Neil tekkis idee, et kui Eesti saab 
vabaks, saab sealt odavalt head tööjõudu. Nad tegid firma, mis pidi hakkama 
Stockholmi lääni valitsusele teenust osutama, ja naisi otsiti sellepärast, et 
naised on korralikumad, leplikumad ja küsivad vähem raha. Meid see 
ei häirinud, sest, kujuta ette, saada tööle firmasse, mis maksab palka 
valuutas, ja teha Rootsi tööd! Fantastiline!

Välja valiti kolm naist: üks oli Maarika Lomp\index[ppl]{Lomp, 
Maarika}, teine mina ja keegi kolmas oli veel. Meile 
üüriti kontoriruumid vana ajakirjandusmaja taga. Pidime ise panema 
püsti kohtvõrgud ja kõik muu! 

\question{Kas tollal oli selline asi nagu kohtvõrk?}

Jah! Ise panime püsti. Suhtlesime üle modemite Rootsiga ja kõik toimis. Muidugi kasutasime tuttavate meesterahvaste abi, kes olid juba võib-olla rohkem võrgu ja selle poole peal, aga saime hakkama. 

Programmeerisime sellises huvitavas keeles nagu Magic\index{Magic}. Oled 
kuulnud? 

\question{MUMPSist olen, aga Magicust mitte.}

See oli juba toona 4GL, Iisraeli päritolu.\sidenote{Platvormi tootnud 
Magic Software Enterprises oli tõesti asutatud 1983. aastal Iisraelis.} Üldse 
oli kasutada vist kaheksa erinevat käsku, neile panid 
parameetrid taha ja nendest koodi kokku. Koodi sai täita vastavalt 
vajadusele kas eest tahapoole või tagant ettepoole. Täiesti müstiline asi! Ja 
sellega sai teha kõike. Esimese projektina tegime nende varade ehk 
autopargi haldusprogrammi, kus olid kõik asjad alates muruniitjatest ja lõpetades 
autodega. 

\question{Kui sind kuulan, kerkib esile huvitav kontrast. See vahend, 
millega te tegite, kõlab palju keerulisem kui tänapäeval 
kasutatavad, aga ülesanne, mida lahendasite, palju 
lihtsam, kui tänapäeval tavaliselt lahendatakse. Kas teile ei tundunud see tegevus kahuriga 
kärbse tapmisena?}

Ma ei oska sulle öelda, miks nad valisid sellise. Igatahes ei kaldunud asi mitte mingil juhul FoxPro, 
Paradoksi või mõne muu sellise asja poole, mis toona juba olemas olid. 
Võib-olla see oli see natuke päritoluga seotud, kuna Kalle on juut ja tal 
olid Iisraeliga tihedad sidemed. Ja kuna ta ise oli hästi kõva matemaatik ja 
keeruliste ülesannete lahendaja, siis talle ilmselt see 
süsteem sümpatiseeris. 

See oli tore aeg. Kasutajaliidesed olid rootsikeelsed ja mäletan siiamaani mingit 
sõnavara. Saime palka 800 Rootsi krooni kuus, mis oli 
tollal siin suur raha. Kui meid viidi restorani kliendiga 
kohtuma, siis meie igaühe restoraniarve oli suurem kui kuupalk. 
Muidugi tegime pikki päevi ja töö oli päris karm. 

\question{Huvitav, et selline programmeerijaamet oli olemas. Näiteks 
Henn Ruukel\index[ppl]{Ruukel, Henn} rääkis, et tema mäletamist mööda enamasti 
inimesed ei tegelenud igapäevatööna programmeerimisega. Leiva tõi lauale 
ikka kaabli vedamine või arvutite kokkupanek.}

No vot, meie tegelesime! Tegime algusest lõpuni: mõtlesime välja, disainisime 
mudelid, programmeerisime ja ka juurutasime Rootsis kliendi juures. 

\question{Kui teil olid arvuti ja modem\sidenote{Ja pidi olema ka võimalus 
välismaa numbritele helistada!}, kas teil ei tekkinud huvi, mida nendega 
veel teha saab?}

Mäletan päevi, mil alustasin 
kell 8 ja lõpetasin kell 12 öösel. Huvisid võis hästi palju 
olla, aga lihtsalt ei jõudnud nendeni. Tegelikult tol ajal tärkas mul huvi 
andmebaaside vastu, sest meile anti üks raamat (\enquote{Lugege läbi, tüdrukud!}), mis minu jaoks esimest korda kirjeldas relatsiooniliste andmebaaside 
teooriat. 

Rootslaste juures olin umbes aasta, siis nägin 
Hansapanga\index{Hansapank} kuulutust. Nad ei otsinud üldse IT inimesi, ma isegi täpselt ei mäleta, keda. 
Lõpetasin parasjagu kaheteisttunnist tööpäeva ja ütlesin Maarikale, et okei, 
ma kirjutan. Saatsin CV, see oli 1992. aasta novembris. Järgmisel päeval 
helistas mulle Tõnis Sildmäe\index[ppl]{Sildmäe, Tõnis} ja kutsus intervjuule. 

Mäletan, et kui ma seal vestlesin (Liivi\index[ppl]{Kompus, 
Liivi}\sidenote{Liivi Kompus, üks Hansapanga legendaarseid IT inimesi.} oli 
ka), rääkisin väga uhkelt oma Rootsi kogemusest ja teadsin juba 
relatsioonilistest andmebaasidest ning jätsin ikka tohutult muljet. Jüri 
Mõis\index[ppl]{Mõis, Jüri} läks vahepeal mööda ja ütles: \enquote{Tõnis, kui sina 
ei taha, ma võtan ise selle tüdruku!} Ja nii mind tööle võeti. 

\question{Kui suur Hansapank\index{Hansapank} toona oli?}

Umbes 40 inimest. ITs olin mina kolmeteistkümnes. See oli tol ajal veel
Crebit\index{Crebit}, tegelikult Spin Development\index{Spin 
Development|see{Crebit}}, mis oli pangast eraldi. 

\question{Kui organisatsioonis on kokku 40 inimest ja neist 13 on IT inimesed, 
siis see on ju päris suur protsent!} 

Pank oli tõesti väga väike, sest toona valdasid inimesed 
hästi laia teemaderingi -- alates klienditeenindusest kuni raamatupidamiseni 
olid samad inimesed. Sealt kasvas välja näiteks Agve 
Aasmaa\index[ppl]{Aasmaa, Agve}, kes tuli tööle telleriks, samuti Tea 
Trahov\index[ppl]{Trahov, Tea}.\sidenote{Mõlemad legendaarsed 
hansapankurid.} Kokku oli meid aga jah vähe. Kui 
olid tähistamised, siis mahtusime väiksesse ruumi ära. 

\question{Kust see suhteliselt suur IT osakaal ikkagi tuli? Praegu ei 
ole ju veerand Swedbanka IT.} 

See oli imetlusväärne visioon, millega omal ajal Hansapanka\index{Hansapank} 
tehti! Taheti teha midagi tõeliselt \emph{cool}'i. Tegelikult need mehed tegid täielikku \emph{start-up}'i. 
Nad tahtsid teha täiesti teistsugust panka, kus paberkataloogide asemel oli 
arvuti. 

\question{Kust neil tekkis arusaam, et sedasi on üldse võimalik?} 

Ma ei ole kunagi küsinud, aga arvan, et see tuli koostöös ja nad olid sõpruskond. 
Hästi palju oli ilmselt Tõnis Sildmäe\index[ppl]{Sildmäe, Tõnis} panust, kes
müüs seda ideed, et nii saab teha. Näiteks kust tuli mõte panna kõik kontorid juba sel ajal \emph{online}'i? 
Mõned aastad hiljem, kui käisime Inglismaal ja Iirimaal kohtumas nende 
suurte ja vanade pankadega, siis need imestasid: \enquote{Issand, lapsed, mis 
te räägite! Teil on mingi Paradoxi lahendus ja see töötab \emph{online}'is?}

See oli julgete mõtete maailm, samas ei tehtud 
lollusi. Keegi ei teadnud, keegi ei osanud, aga kogu aeg õpiti. Mõeldi, 
kuidas me nüüd sellest üle saame? Kuidas me hakkame oskama? Keegi saadeti
kuskile midagi õppima \dots 

Minu esimene ülesanne oli see, et 
Tõnis\index[ppl]{Sildmäe, Tõnis} ütles: \enquote{Näed, mul on siin neli 
programmeerijat kirjutanud. Kes on teinud laenu, kes kontosid. Vaata 
ja ütle, mis võiks olla teistmoodi.} Vaatasin ja stiili 
järgi oli kohe näha, et see on Liivi Kompus, see on Kadri Trahov, see Tõnis 
Argus -- igaühel oli täiesti oma stiil. Andmebaas oli selline, nagu oli, aga kõik
töötas. Kirjutasin ettepanekud ja loomulikult ei olnud need 
selle tehnoloogia peal teostatavad, aga sealt sai alguse mõte, et viime oma 
süsteemi Oracle peale -- teeme uue pangasüsteemi ja viime asja järgmisele 
tasemele.

Kogu eduloo \emph{point} oli ääretult avatud 
suhtlus, kõik soovisid midagi ära teha. 

\question{Mis seda takistas nurjumast? Kui hakata niisama nullist 
kõrgtehnoloogilist panka tegema, siis see võib ju vussi minna.}

Tead, mul on alati olnud ja on siiamaani usk, et inimesed on \emph{key}. Inimesed, kellega sa 
mingit asja teed, isegi kui asi võib minna ka totaalselt 
tuksi. Üks pool on muidugi oskused, aga isegi rohkem 
määrab ära suhtumine ja ambitsioon. 

\question{Milline suhtumine olema peaks?}

See, et tahan midagi ära teha, aga samas tean, et ma ei tee seda üksi, 
vaid me teeme koos. Oluline on ka arusaam, miks ma midagi teen. Palju on selliseid initsiatiive, et on 
hästi huvitav tehnoloogia, aga kas see lahendab mõnd probleemi, selle peale liiga palju ei mõelda. Tollast aega iseloomustabki koos ärategemine. Mida 
kindlasti ei olnud, oli see, et \enquote{kes on kõvem}.

Inimestel, kes seal toona töötasid, oli hästi suur ambitsioon ---
kindlasti mitte isiklik karjäär, vaid meeskondlik ambitsioon ja 
saavutusvajadus. Tänapäeval räägitakse palju ja räägiti ka Swedbankis 
aastal 2000, kui rootslased tulid, protsessidest ja kvaliteedijuhtimisest. Peab 
olema tohutu hulk dokumente ja siis ma teen plaane, raporteerin ja 
kogu jõud lähebki selle peale. Hansapangas seda ei olnud, me ei mõelnud niimoodi. Näiteks kui pank tahtis 
välja tulla eraisiku pangakontoga, kutsus Jüri Mõis\index[ppl]{Mõis, Jüri} 
ühte ruumi kokku kõik, kellel võiks sellega mingit pistmist või 
arvamust olla. Tema rääkis, miks seda teha, ja meie mõtlesime, mis selleks tegema 
peab. Edasi hakkaski igaüks oma osa tegema. Tehti koos ja 
hästi ruttu. 

\question{Magicu moodi asjadega tegelemine annab 
ilmselt päris hea immuunsuse tehnoloogia järel jooksmise vastu. Kui oled korra 
pidanud tagurpidi käivat programmi kirjutama, siis ei ole miski enam väga 
uudne!}

Oluline on kasutada õiget asja õiges kohas. Populistlik jooksmine 
mingi asja järel\ldots{ }See on põhjus, miks ma ei poolda näiteks kunagist riiklikku initsiatiivi, 
et kõik programmid tuleb iga 13 aasta järel ümber 
kirjutada. \emph{Sorry}, aga võib-olla ei peaks neid 13 aasta tagant ümber 
kirjutama, kui kogu aeg teeks nendega midagi? Aitaks järele ja muudaks? See 
oli see, mida me pangas tegime. Vaatasime täpselt seda, kust meil tulevikus 
pigistama hakkab, ja muutsime ning vahetasime välja. See oli pidev protsess. 

\question{Ometi pidid ka pangas uute tehnoloogiate lained tulema. Kuidas te otsustasite, mida üles korjata?}

Me tegime ka valeotsuseid, keegi ei ole ju selle suhtes immuunne. 
Kui tehnoloogia poolelt rääkida, siis esimene laine oli see, et meil oli 
Paradox\index{Paradox}, mis oli failipõhine süsteem ja millest me kindlasti 
nägime, et see hakkab meil takistuseks saama. Kas või näiteks see, et me ei 
suutnud olla 24h kättesaadavad, või päeva vahetuse teema.\sidenote[][]{Päeva vahetus on 
pangas oluline ja keeruline ning seetõttu arvutuslikus mõttes kaua aega 
võttev toiming, mille käigus tehakse konkreetse päeva seisuga raamatupidamiskanded, 
toimetatakse arveldused, esitatakse aruanded ja, lühidalt öeldes, üks pangapäev 
asendub teisega.} 

Minul oli relatsiooniliste baaside teoreetiline 
teadmine ja Tarmo Pajumets\index[ppl]{Pajumets, Tarmo} oli töötanud pool 
aastat Soomes ning arendanud Oracle-põhiseid süsteeme. Oracle\index{Oracle} oli 
sel ajal Eestis ikkagi number üks, seda kasutasid näiteks ka Elion ja EMT. Üks põhjus, miks 
Oracle nii tugevalt Eesti turule tuli, oli see, et meie tugi oli Soomes 
ja Soome Oracle oli väga tugev organisatsioon. Kui alustasin uue süsteemi 
arendamist, siis mul oli otseliin Soome ja tugi telefoni otsas -- võisin helistada Soome ja küsida, miks see või teine asi ei tööta. Võrdle sellega, mis on praegu!

Ja nii me konvertisimegi oma Paradoxi 
Oracle peale. Oli olemas automaatse konverteerimise võimalus, 
et kasutajaliides jäi endiselt Paradoxi ja baas taga oli Oracle. See andis 
meile vabaduse teha oma päevavahetuse protsessid kõik 
niimoodi, et see oli rohkem 24h. Saime hakata sinna külge kaarte panema ja 
tulevikus ka näiteks Telehansa\index{Telehansa}. Ja hakkasime Paradoxi rakendust ennast ümber 
kirjutama. 

\question{Kas Telehansa tuli enne kui Forexi modemipank või hiljem?}

Minu meelest umbes samal ajal.

\question{Mast, kes Forexi asja kirjutas, rääkis, et tolle 
käivitamis{\-}üritusel istunud Hansapanga tütarlapsed esireas ja teinud ohtralt 
märkmeid!}

Võis nii olla, see tuli enam-vähem sinna otsa, palju 
vahet ei olnud. 

\question{Miks te Telehansa\index{Telehansa} tegite?}

See algas sellest, et olid küll kontorid, aga firmade raamatupidajad ei 
tahtnud oma maksetega kontorisse tulla. Meil 
oli palju firmasid. Kui vaadata Hansapanga arengut, siis ta 
kõigepealt võttis ju sellised eesrindlikumad firmad ja seejärel tulid eraisikud 
pangakontodega järele. 

Ühelt poolt tahtsime elu mugavamaks teha ja see oli ka
rahaliselt kasulik. Kontori koormust vähendades hoiad kokku. Teine oluline asi oli
innovatsioon -- oleme teistmoodi pank, 
teeme asju teistmoodi. 

\question{Kui palju Telehansa tuumsüsteemi muutust eeldas? See ju vajab hoopis 
teistsugust interaktsiooni tuumaga.} 

Tegelikult ei olnud Telehansa tegemine kuigi keeruline, selle võimaluse lõi
andmebaasi vahetus. 

\question{Kas ajaliselt tekkis see umbes samal ajal?}

Jah. Andmebaasi vahetus oli vist 1994. aastal ja Telehansa tuli ka siis. See
oli väga kõva asi.

\question{Telehansa on siiamaani väga kõva asi!}

Kolm meest -- Toomas Lassmann\index[ppl]{Lassmann, 
Toomas}, Madis Tapupere\index[ppl]{Tapupere, Madis} ja Riho-Rene 
Ellermaa\index[ppl]{Ellermaa, Riho-Rene} -- selle tegid ja kirjutasid. 

\question{Kas tolleks hetkeks oli IT inimesi juba rohkem kui neliteist?} 

Jah. Toomas Rand ja mina kirjutasime töötlust, et kui 
maksed sisse tulid, siis mis nendega sai. Tiim kasvas väga kiiresti. Neliteist oli 
1992. aasta alguses ja see arv kahekordistus umbes 
aasta või pooleteisega. 

\question{Kuidas te kasvu kontrolli all hoidsite? Kui nii kähku kasvad, 
siis on ka suur tõenäosus mõni loll kogemata palgale võtta.}

Mäletan selgelt, et kui olime kasvanud kuskil 50 inimeseni, siis tegime 
oma esimesed kompromissid. Enne ikka väga valisime inimesi. Hansal kui Eesti majanduse lipulaeval oli võimalik valida! 
Esiteks, et nad oleksid professionaalsed tipptegijad. Teiseks, et nad inimestena sobiksid väga 
hästi tiimi. Aga siis tegime jah esimesed kompromissid\ldots

Hakkasid tekkima esimesed probleemid ja küsimused, kuidas asja hallata. Tekkis \emph{learning by doing} kogemus. Tänapäeval on ilmselt väga vähestel olnud 
võimalus kasvada koos organisatsiooniga väikesest suureks ja sealjuures väga kiiresti. 

Näiteks üks teema oli see, et kui keegi oli arendanud 
laenu- või väärtpaberisüsteemi, siis uue 
funktsionaalsuse tegemiseks tal enam aega ei jätkunud, kuna tal tuli teiselt poolt kogu aeg nii palju 
probleeme peale, mida pidi lahendama. Kes vajas 
raportit, kellel oli mõni \emph{case}, mis ei mahtunud sisse, 
ja seesama inimene tegeles mõlemaga. Siis tegimegi halduse poolel rakenduste halduse osakonna, kus olid 
inimesed, kes oskasid lihtsamaid probleeme lahendada, raporteid genereerida 
ja tundsid andmebaasi andmeid -- ühesõnaga olid arendaja 
kõrval, et arendajal oleks rohkem aega. 

\question{Nii et see otsus sündis praktilisest vajadusest, mitte te ei olnud kuulnud, et peaks olema 
rakendusadministraatorid? Teil oli vaja konkreetset asja teha, leidsite
inimesed, koolitasite neid ja panitegi tegema.}

Seal organisatsioonis sündis kõik praktilisest vajadusest. 

Mul on tohutu austus kadunud Tõnis Sildmäe\index[ppl]{Sildmäe, Tõnis} kui juhi 
vastu. Kuidas ta seda tiimi juhtis! Juhtkonna moodustasid inimesed, kes suutsid 
vedada teatud teemasid või olid juhipotentsiaaliga spetsialistid. Ta 
usaldas meid täielikult. Ta ei tulnud kunagi ütlema, mida keegi peab tegema. 
Ainuke negatiivne asi oli võib-olla see, et ta kaitses liiga palju: kui 
keegi kallale tuli, siis läks ta kohe võitlusse ja teda pidi tagasi 
hoidma. Aga see lõigi kultuuri, et probleemi tekkides istusime koos
ja arutasime, kuidas seda oleks mõistlik lahendada. 

\question{Sinu jutust koorub põhiliselt välja 
inimeste ja juhtimise olulisus. Väga vähe on juttu sellest, et Oracle indekseid peaks 
tegema nii- või naamoodi.} 

Indeksite tegemine on lihtsalt tehnika. Loomulikult peavad olema 
oskused ja teadmised, aga see on õpitav. See, kui 
hästi ma indekseid teen või kui ilusat koodi kirjutan (kood peab ilus 
olema!), ei ole see, mis toob edu. 

\question{Milline on ilus kood?}

See kõlab nüüd hästi populistlikult, aga ilus on kood, millest saab aru. Kui ma isegi ei valda täielikult programmeerimiskeelt või 
tehnoloogiat, milles see on kirjutatud, siis vaatan koodile peale ja saan aru. 
Loomulikult, kui ei ole üldse kogemust sellel alal, siis 
võib-olla ei saa aru. Aga kui oskan Cd kirjutada, siis vaatan Java koodile 
peale ja suudan seda lugeda. 

\question{Järelikult sõltub ilus kood sellest, keda sa 
arvad seda hiljem lugevat -- esimese kursuse tudengit või 20 aastat 
progenud inimest?} 

Ka esimese kursuse tudeng võiks aru saada!

Mul on just andmebaaside poolelt -- PL/SQL või PostgreSQL \mbox{baasi} protseduurid -- palju kogemust. Omal ajal vaatasin nelja inimese 
koodile peale siinsamas majas\sidenote{Meie jutuajamine toimus Tallinnas toonases Icefire 
kontoris aadressil Kauba 2a.} ja võin öelda, et kaks neist kirjutas väga ilusat koodi, kaks mitte. 
Kood töötas perfektselt ning kõik neli on tehniliselt väga head ja tugevad tegijad. 

\question{Aga mõnel on ilus kood ja mõnel ei ole. Kunst?}

Jah, see ongi kunst. See hakkab peale sellest, kuidas ma mõtlen ja 
oskan maailma abstraktselt kujutada. 

\question{See haakub Ahti jutuga, kuidas tema 
õppiski programmeerima just nimelt paberil ja programmist mõeldes. Temagi rõhutas võimekust asja oma peas ette kujutada.} 

Jah. Teine pool on see, et vahel kui lahendus liiga ära abstraheerida, tulevad sellised 
maailmamudelid, mis üldiselt eriti ei toeta \ldots 

\question{Kuidas seda tasakaalu hoida, et oleks piisavalt üldine ja ei paneks arendust kinni, 
aga ei üritaks ka maailma mudeldada?} 

Ilmselt see on 
mõtteviisis kinni ja tuleb loomulikult, kogemusega. Igaühel ei tulegi. 
Võib-olla on asi analüütilises mõtteviisis \ldots 

\question{Sinu jutust jääb kõlama, et kui on 
hästi kokku pandud tiim, siis see tiim jõuab tasakaaluni loomulikult 
oma kogemuste, oskuste ja parasjagu käsil oleva ülesande pealt.} 

Just. Muidugi tiimitöö. Inimeste koosmõtlemist ei ole võimalik üle väärtustada, sest koos välja mõeldud väärtus on tükk maad 
suurem. Meil on väga hea näide -- Jan\sidenote{Vilve peab silmas legendaarset hansapankurit Jan 
Laksperet\index[ppl]{Lakspere, Jan}, kellega nad on 
intervjuu ajaks koos töötanud üle kahekümne aasta.} \emph{versus} mina. Jan on 
perfektsete lahenduste mees: tema lahendused katavad üldiselt ära kõik asjad, sealhulgas
ääre-\emph{case}'id, mis lõpptulemusena võib tekitada olukorra, et lahendus 
läheb liiga keeruliseks. Mina jällegi suudan läheneda selle poole pealt, kuidas 
oleks mõistlik -- me töötame koos jube hästi. 

\question{Tuleme korraks tagasi Hansapanga aja juurde. Kui Tõnis\index[ppl]{Sildmäe, 
Tõnis} juhtkonda kokku pani, oli see ju ka sinu jaoks valikukoht, kas 
hakata inimesi juhtima või jääda koodi kirjutama. Tihti öeldakse, et kui 
heast programmeerijast juht teha, saad omale kehva juhi ja heast 
programmeerijast lahti. Kas sul seda hirmu ei olnud?}

See oli balansseeritud protsess, see ei juhtunud ühe päevaga. 
Kirjutasin ju Hansas peaaegu lõpuni ka koodi. 

\question{Kuidas sa seda tasakaalu hoidsid? Inimesi juhtides on lihtne
sinna sisse ära uppuda, nii et ühel hetkel ei kirjuta 
enam üldse koodi ja minetad selle oskuse.} 

Praegu ma näiteks ei kirjuta.

\question{Aga tahaksid?}

Nojah, vahel on kurb ka, et ei saa. See on 
olnud nii viimased kolm aastat, lihtsalt juhtus sedasi. Aga ma annan endale aru, et keegi pidi 
selle ülesande võtma. Ma ei ütle, et oleksin selle pärast õnnetu, 
kuid vahel ikka kriibib natuke.

\question{Miks sa ikkagi otsustasid juhi rolli ka juurde võtta?} 

Võib-olla oli mul iga asja kohta midagi öelda? Ja siis sind karistatakse selle eest. 

\question{Mäletades seda, kuidas Hansapank oli seesmiselt üles ehitatud, ja 
keskset Oracle baasi, kus kõik maailmaasjad sees olid, siis ühel hetkel läks
see süsteem ikkagi tehnoloogiliselt laiaks. Kuidas seda kontrolli 
all hoiti?} 

Alguses oli kõik väga lihtne ja koodi baasi kirjutamine oli õige otsus. Kood oli ilusti ja loogiliselt
struktureeritud, midagi ei tohtinud segamini ajada. Siis aga tulid igasugused tehnoloogilised \emph{switch}'id. Javat veel ei olnud 
nii palju, osaliselt kirjutati Cs. 

Kui tuli esimene internetipank, siis 
uurisime, kuidas seda teha, sest tehnoloogia ei olnud veel päris sealmaal. Ostsime sisse BroadVisioni platvormi.\sidenote{Varajane internetitehnoloogia, mis lubas igale kasutajale täielikult 
personaliseeritud kogemust. Paraku selgus, et kõigi nende kogemuste pakkumise 
vältimatuks eelduseks on nende väljamõtlemine. Küll aga võimaldas BroadVision 
üsna mõistlikult kombineerida HTMLi ja Cs\index{C} kirjutatud komponente 
(hiljem ka serveripoolset JavaScripti\index{JavaScript}.) ja sellest 
internetipanga, tellerirakenduse jms ehitamiseks juba piisas.} Kui oled kuskil liiga vara, siis teed otsuseid, mis 
ilmselt ei ole jätkusuutlikud, sest uus teletehnoloogia tuleb peale. 

\question{Kui mõelda, kuidas Sergei 
Anikin\index[ppl]{Anikin, Sergei} kirjutas Light Tellerit\sidenote{Lähemalt loe 
Sergei loost lk \pageref{sisu:teller}.}, siis see ongi see, kuidas praegu 
tehakse. Tehnoloogia võis ju jätkusuutmatu olla, aga lahendus oli vägagi 
jätkusuutlik!} 

Tegelikult meil juba tol ajal andmebaas pakkus teenuseid, loogika ei olnud 
segamini kliendis ja baasis. 

\question{Kes tegi otsuse niimoodi teha?} 

Meie tegime. Asi algas sellest, et mina ja Ruta Joost\index[ppl]{Joost, Ruta} 
panime esimesena need mustrid paika, kuna see tundus ainuvõimalik viis. 

\question{See on päris hea viis arhitektuuri teha, et võtad selle, mis tundub 
ainukesena võimalik!} 

Teatud asju ei saa põhjendada, kuidas need peas sünnivad. Ilmselt on see analüütiline mõtlemine, et vaatad, mida see tähendab, 
kui teen nii või naa. Mul oli kohe see mõte, et kui kirjutan 
mingi loogika klienti, siis ma ei saa seda ju korduvalt kasutada. Järelikult ma 
ei tee seda. Kirjutan nii vähe kui võimalik. Ja nii oligi. 
Sergei\index[ppl]{Anikin, Sergei} sai kiiresti teha Light Telleri, sest tal 
olid kõik teenused olemas. 

\question{Hansapank läks ju ka horisontaalselt suureks -- Markets,
kindlustused ja muu. Kuidas te seda pusa kontrolli all 
hoidsite, et keegi lollusi ei teeks?} 

See oligi hästi keeruline ja ega me ei kontrollinudki kõike lõpuni. 
Marketsil\index{Hansapank!Markets} oli väga suur eripära ja neil oli oma IT. 
Seal oli palju asju Excelis, lisaks analüütika, mis nad sealt pealt tegid. Neil olid sisseostetud lahendused, näiteks Condor, sest keegi ei hakanud 
\emph{trading}-lahendust ise tegema. Ja eks tehti
ka valesid otsuseid, näiteks kui osteti Marketsi platvorm, mis 
osutus liiga tooreks ja keeruliseks. Marketsit me tõesti keskselt palju 
ei hallanud, ainult nii palju, kui see haakus meie pakutavate 
süsteemidega. 

\question{Keegi pidi ju tegema otsuse, et \enquote{las nad toimetavad 
omaette}?} 

Just sel ajal, kui organisatsioon laienes, tekkisid meil ITs kindla
valdkonnaga tegelevad inimesed, kes meie poolelt olid sellel konkreetsel 
valdkonnal vastas. Paljuski oli tegu valdkonnajuhi ja Marketsi IT 
vahelise kompromissi, kokkuleppe ja ühise otsusega. 

\question{Kas põhimõtteliselt tõmmati neile organisatoorne kast ümber ja 
lasti neil seal sees vabalt toimetada?}

Jah, aga väljapoole kasti nad ei saanud, välismaailmaga suhtlemiseks olid väga 
selged liidesed. 

Teine näide oli Hansa Capital\index{Hansapank!Hansa Capital}, mis 
kasvas väga kiiresti, oli väga isepäine ja efektiivne ning äriliselt tegi superhead tulemust. Kuni üks hetk tuldi meie juurde ja öeldi: \enquote{Kuulge, 
meil läks Excel katki, tehke midagi!} Me siis vaatasime sellele loomaaiale 
peale, see oli peaaegu sama suur kui pank oma äridega! Panime projektitiimi kokku ja hakkasime tegema. 

\question{See haakub sellega, et üks asi on vedada inimesi, kes on 
\emph{hand picked}, aga teine asi on olla 
organisatsioonis kõrge taseme juht. Järelikult pidid sa aru saama, 
kuidas inimesed töötavad, nendega suhtlema ja panema neid vajalikke asju tegema. Kuidas sul see oskus tuli?}

Õppisin! Loomulikult oli ka palju koolitusi, mis tulid kindlasti kasuks ja panid 
teatud asjade peale mõtlema, aga tegelikult õpitakse läbi tegemise. 

Juhtimine on keeruline ja sellega toimetulemise määravad paljuski isikuomadused. Teine asi on kindlasti kogemus -- arvan, et 
olen praegu palju parem, kui ma olin siis, aga kui ma ei oleks seda 
protsessi läbi teinud, siis ma ei oleks seal, kus ma olen. Kõige keerulisem on 
hakkama saada inimestega, kes kõik sinult midagi tahavad, ja sa tead, et ei saa kõigile jah öelda. Kuidas 
teha seda niimoodi, et sünniksid õiged valikud, mitte anda
tähelepanu sellele, kes kõige kõvemini karjub. 

Ja vahel ei olegi keegi sinuga rahul. Sa pead sellega 
hakkama saama, et kõik sind ei armasta. Inimestega tuleb kindlasti rääkida, ega muu ei aita!

\question{Nii et kuskil sügaval peab lisaks arvutihuvile olema huvi inimese vastu?} 

Jah, muudmoodi ei saa! Huvi inimese vastu peab olema 
võib-olla isegi natuke suurem kui huvi arvuti vastu! 

\question{Ometi läksid sa õppima rakendusmatemaatikat, mitte 
psühholoogiat. Kust sul huvi inimese vastu tuli?}

Mul on see huvi olnud vist kogu aeg, kui nüüd mõelda. Nagu ma rääkisin, siis valisin tegelikult kirjanduse ja 
matemaatika vahel. 

\question{Kui palju te suhtlesite tol ajal pangavälise kogukonnaga? 
Kuskil pulbitses BBSide kamp, toimis mingi võrgustik. Kui palju te selles osalesite?}

Meil oli inimesi, kes seal suhtlesid, aga need olid kindlasti tehnilisemad inimesed,
nagu Toomas Lasmann\index[ppl]{Lassmann, Toomas}. Meie mitte nii väga. Pigem suhtlesime natuke 
kõrgemal tasemel: kes kuhu liigub ja kes milliseid tehnoloogilisi valikuid teeb.

\question{Üks asi, mida mina tollest ajast igatsen, on see, kuidas sündis 
iPizza\sidenote{Hiljem tuntud kui Pangalink. Minu mäletamist mööda sai see kohapeal 
välja mõeldud, teised ütlevad, et tegu oli Soome lahenduse ülevõtmisega. 
Igatahes võimaldas see (lisaks algselt ideeks olnud maksete lahendusele 
eesmärgiga internetis pitsat tellida -- sealt ka nimi) anda panka sisse loginud 
inimese identiteeti edasi teistele osapooltele. ID-kaart ei olnud veel levinud, 
keegi maksuameti paroolikaarti endale ei võtnud, aga maksuametil oli vaja saada 
inimesed internetis tulu deklareerima. Nii sündiski koostöö, sest pangal oli 
vaja anda inimestele hea põhjus nende internetipanka kasutada.}. Kogu protsess 
hetkest, kui astuti uksest sisse, et nüüd hakkame tegema, kuni 
hetkeni, kui maksuametisse sai sisse logida, võttis aega kolm 
nädalat. Praegu võtaks isegi lepingu läbirääkimine tõenäoliselt kauem.} 

Me oleme jõudnud sinnasamasse, kus on kogu vana maailm. Tegelikult on 
see kurb, aga kõik algab inimestest. Miks on lepingu 
läbirääkimine nii pikk? Me oleme ise teinud endale kõik need protseduurid ja 
poliitikad. Meie ise, mitte keegi teine. 

On olemas selline organisatsioon nagu Nordic Finance Innovation Forum, mille eesmärk 
on panna Põhjamaade pankades, kus miski ei liigu (täpselt sama asi, neli 
kuud räägitakse lepingut läbi), mingilgi viisil liikuma innovatsioon. Panna nad 
omavahel koostööd tegema. Ka meie osaleme selles, sest see on huvitav. Nad korraldavad 
päevaseid seminare mõnes Põhjamaade pealinnas, kus 
erinevad inimesed räägivad erinevatest asjadest, mis on tehtud. 

Ükskord rääkis üks kutt väikesest 
Soome firmast, 15 inimest, kuidas nad tegid AliPay integratsiooni Soome 
maksesüsteemiga. Hiinlasena saad praegu minna Helsingis igasse poodi ja 
maksta oma AliPayga. Kujutad ette! Projekt sündis sellest, et inimesed 
ütlesid: \enquote{Nelja kuu pärast, novembri lõpus, lendab siia 
Rovaniemisse mitu suurt lennukitäit, palju-palju 
hiinlasi ja neil kõigil on vaja sellega maksta.} Ideest 
\emph{live}'ini kulus neli kuud, nad tegid selle ära! See kutt ütles: \enquote{Üldiselt on 
meil Soomes nüüd niimoodi, et selleks kulub neli kuud, et saada esimene kohtumine mõnes Soome pangas, et oma 
ideest rääkida.} Aga see projekt tehti nelja kuuga ära!

\question{Ometigi ei olnud Hansa tol ajal enam tilluke organisatsioon?!}

Aga toimis ikkagi! 

\question{Kuidas sai niimoodi, et läbirääkimised ei võtnud neli kuud?}

Selle asja nimi on kultuur! Kultuuri loovad inimesed, see ei sünni mitte 
millestki muust. Ja kui nüüd on organisatsioon, kelle juhil on ainult üks omaenda isiklik ambitsioon 
olla suur juht (ambitsioon rääkida ümber kõike, mida ta on raamatutest lugenud, 
hoolimata sellest, mis tegelikult tehtud saab), siis sünnibki tema ümber 
samasugune kultuur. Eesti häda täna ongi paljuski see. Mitte ainult 
riigis, vaid ka erasektoris. 

\question{Ja Hansas oli juhtkond, kellelt tuli teistsugune kultuur!}

Kui mõelda, mis seal hiljem toimus, siis kõik need inimesed, kes selle 
kultuuri olid ehitanud, ju lahkusid. Mingi põhjus pidi olema, palk ju kehv ei olnud. See on väga kurb.

Olen siin umbes aasta ühele pangale rääkinud, et 
kallikesed, kui tahate, et asjad hakkaksid liikuma, siis peate seda 
vana kultuuri kandvad inimesed lihtsalt ära saatma. Te ei saa enne üle ega 
ümber, kui te ei julge teha seda otsust. Inimesed ei julge tihtipeale 
otsustada, sest need on rasked otsused. 

\question{Kogu see teekond on sind kuhugi toonud. 
Millega sa praegu tegeled?} 

Viimased 16 aastat oleme ehitanud Icefire't. Me kogu aeg muutume ja areneme. Mõtleme, kuidas 
maailm muutub ja kes me võiksime olla ning kuidas sinna jõuda. Siin tuleb mängu jällegi seesama kultuuriküsimus -- me hoiame oma 
kultuuri, mida meie inimesed kannavad. Ja see võib olla ka põhjus, miks 
me ei ole näiteks läinud seda teed, et ostame ettevõtteid kokku, 
lihtsalt kasvatame väärtust, ajame asja suureks ja lõpuks saame väga rikkaks. Oleme 
pigem hoidnud organisatsiooni hoomatavana. Ja täna läheme platvormiärisse, mis tegelikult muudab täiesti seda paradigmat, kus me tegutseme. 

\question{Ja sina oled selle asja juht ja koodi ei kirjuta?} 

Praegu juba kolm aastat ei kirjuta. Mul on fantastiline tiim. Oleme viimase kolme aastaga teinud suurepärase tulemuse pärast seda, kui vahepeal augukeses 
käisime! See, kuidas oleme suutnud ennast kasvatada samal ajal efektiivsust säilitades, näitab, et teeme midagi väga õigesti. Kummalisel kombel olen viimasel aastal märganud, et see külvab 
ümberringi kadedust mingites inimestes, kellega olen kunagi koos 
asju teinud. Miks me tunneme kadedust? Mina tunnen küll 
rõõmu, kui teisel hästi läheb! Aga eks me peame sellega elama. 

\chapter{Anne Villems}
\index[ppl]{Villems, Anne}

\question{Kuidas sina arvutite juurde said?}

Mina käisin Mari Ülikoolis\index{Mari Ülikool|see{Tartu 10. Algkool}}. Seda
nime kandis rahva seas tolleaegne Tartu 10. Algkool\index{Tartu 10. 
Algkool}, mis asub Vanemuise\index{Tartu Ülikool!Vanemuise tänava õppehoone} 
kõrval ja vana Vanemuise vastas. Nimi tulenes karismaatilisest 
matemaatikaõpetajast Marvetist,\sidenote{Vladimir Marvet\index[ppl]{Marvet, 
Vladimir} (1903--1994).} kes oli seal ka õppealajuhatajaks.

Kui me pinginaabriga selle kooli 1960. aasta kevadel lõpetasime, siis 
meie vanemad olid väga huvitatud panemast meid 5. kooli\index{Tartu 5. 
Keskkool}. See asus tol ajal Rostovtsevi ülikoolis, mis oli esimene 
ülikool, mis naisterahvaid vastu võttis. 

Meie ei tahtnud 5. kooli minna, ma ei tea, miks. Meie otsustame! 
Ikkagi 14 aastat isiklikku vanust ja loomulikult tuleb ise otsustada, kuhu kooli 
lähed. Nii otsustasimegi, et läheme hoopis 
Treffneri Gümnaasiumisse\index{Hugo Treffneri Gümnaasium}, mis siis kandis 
nimetust Anton Hansen Tammsaare nimeline Keskkool. 

Minu meelest tol ajal ei 
olnud üldse seda probleemi, et pidi katseid tegema, aga 
meil oleks ka katsetega hästi läinud, sest õppisime mõlemad üsna 
korralikult. Kui olime aasta Treffneris olnud, siis avati seal 
matemaatikaklass. Minu jaoks oli see kohe sirge otsetee, sest
matemaatika oli mu lemmikaine. Kõiges on süüdi muidugi Mari, tähendab õpetaja 
Marvet\index[ppl]{Marvet, Vladimir} -- nii kihvte matemaatikatunde, 
nagu Marvet meile viiendast seitsmenda klassini tegi, tean pärastisest ainult siis, 
kui Olaf Prinits\index[ppl]{Prinits, Olaf} tuli keskkoolis meid õpetama. Aga 
see oli juba matemaatikaklassis. Nii et kui läksime üheksandasse klassi, sest 
tolleaegne algkool lõppes seitsmenda klassiga, läksin mina kõigepealt matemaatikaklassi ja kahe kuu pärast tuli pinginaaber ka järele, sest 
keskkoolis algas mingi praktiline õpetus ja talle ei pakkunud see, mida nende 
klassile anti, erilist pinget. Tuligi ära meile, 
kuigi ta oli humanitaarsete huvidega. 

Matemaatikaklassis oli minu suureks rõõmuks viis tundi matemaatikat nädalas 
ja muude ainete seas ka sellised toredad ained nagu elektrotehnika, ligikaudne 
arvutamine ja programmeerimine. 

\question{Mille peal programmeerimist õpetati?}

Sel ajal oli Eestis olemas üks elektronarvuti, mis kandis 
nimetust Ural\index{Ural}. Sellele ei olnud veel numbrit \enquote{1} külge 
pandud ja see asus Tartu Ülikooli arvutuskeskuses\index{Tartu 
Ülikool!Arvutuskeskus}, mis moodustati 1959. aastal. Masin ise asus peahoone 
kõrval majas, Treffnerist üldse mitte kaugel, üks kvartal. 
Õpetasid meid seal alguses Ülo Kaasik\index[ppl]{Kaasik, Ülo} ja pärast Mati 
Krull\index[ppl]{Krull, Mati}, kes oli sinna tööle läinud sellepärast, 
et tema oli esimeselt kursuselt ülikoolis, kellele Ülo Kaasik\index[ppl]{Kaasik, 
Ülo} programmeerimist õpetas. 

\question{Nii et Ülo Kaasik hakkas ilma arvutita programmeerimist 
õpetama?}

Ei, arvutiga. Me olime üheksandas klassis 1962. aastal ja 
seesamune arvuti oli juba kaks aastat kohal. 

\question{Kellelgi pidi keskkoolis olema parasjagu visiooni ja arusaama, 
et elektrotehnika, ligikaudne arvutamine ja arvutid on olulised. 
Kes see inimene oli?}

Visioon võibolla ei olnud isegi keskkoolil, kuigi meil 
oli ka väga karismaatiline direktor Allan Liim\index[ppl]{Liim, 
Allan}, aga tema oli ajaloolane. Arvan, et initsiatiiv tuli tegelikult 
Olaf Prinitsalt\index[ppl]{Prinits, Olaf} ja Ülo Kaasikult\index[ppl]{Kaasik, Ülo}. See ei oleks olnud Nõukogude Liidus erakordne. Kui nad arvuti 
Nõkku\index{Nõo Keskkool} panid ja Nõos oli esimene kooli arvutuskeskus, siis 
see oli kogu liidus esmakordne juhus. Aga matemaatikaklassid olid tol ajal juba 
olemas ja ka Moskvas. Nii et nad võisid öelda, et järgivad 
Moskva malli, ja probleeme ei tulnud. Treffneri kool sai neil valitud võibolla sellepärast, et see oli südalinnas ja 
ei olnud vaja kaugele minna, sest oli selge, et vähemasti esimestel 
aastatel hakkavad õpetama ülikooli õppejõud. 

Meile õpetas Olaf Prinits 
matemaatikat. Mul ongi olnud täiesti fantastilised 
matemaatikaõpetajad: algkoolis õpetaja Marvet ja keskkoolis Olaf 
Prinits\index[ppl]{Prinits, Olaf}. 

\question{Kuidas matemaatikahuvi arvutihuviks üle läheb? Matemaatika 
on abstraktne ja kaunis kunst, aga arvutid tolmavad ja flogistonid\sidenote{Teadupärast
töötavad arvutid flogistonide abil ja 
kui arvuti katki läheb, siis lahkuvad flogistonid sellest sinise suitsuna ning 
arvuti enam käima ei lähe.} lendavad aeg-ajalt välja \ldots} 

Tolm tuleb pärast. Tol ajal tolmu ei olnud. 

Need on muidugi omavahel seotud oluliselt kõrgemal tasemel. Pead ikka natuke matemaatikat oskama, enne kui saad aru, kuidas matemaatika informaatikale või programmeerimisele kasulik on. Ja 
enne pead natuke programmeerida oskama, kui adud, et mõnikord on ka
matemaatikat vaja. Näiteks alustades sellest, et algoritmiliselt 
mittelahenduvaid ülesandeid eristada algoritmiliselt lahenduvatest ja neid 
lahenduvaid jagada ka sellistesse klassidesse, mille lahendid ei tule mitte 50 
või 50 000 aasta pärast, vaid arvutist kui mitte homme 
hommikuks, siis vähemasti ülehomme lõunaks.

\question{Seda enam tekib küsimus, miks matemaatikahuvilisel inimesel ei ole 
tingimata arvutihuvi. Miks sul oli?}

Alguses oli lihtsalt jube põnev. Kogu Eestis oli ainult üks arvuti ja meile 
õpetati! Tuled vilguvad ja perfolindilt (mis on filmilint, mitte 
telegraafilint) loetakse andmeid! 

Mõnikümmend aastat hiljem on meil igaühel arvuti taskus või laua peal ja meie $\epsilon$-ümbruses\sidenote{Olgu{\ } $(X 
; \rho)$ meetriline ruum, $p \in X$ ja $\epsilon > 0$. 
Punkti $p{\ } \epsilon$-ümbruseks nimetatakse hulka $\{x \in X | 
\rho(x,p)<\epsilon\}$. Teisisõnu on $\epsilon$-ümbrus meie otsene 
ümbrus.} tuhandeid kordi võimsamad arvutid kui see, mis meil tookord 1962. või 1963. aastal (1964 lõpetasin ma keskkooli) võttis suures 
saalis enda alla mitukümmend ruutmeetrit. 

Õppima läksin muidugi puhtalt matemaatikat, sest ega kuskil (võibolla TPIs) ei õpetatud niisuguseid 
tehnikaaineid, kus ka arvutid otsapidi sisse tulid. Ehk isegi õpetati, aga ma 
ei tahtnud, sest tehnika pool mind väga ei tõmmanud. 

Programmeerimine on maagiline tegevus. Enne 
seda arvuti midagi ei oska, siis kirjutad kihvti programmi ja 
siis järsku oskab ning näeb peaaegu välja nii, nagu saaks 
millestki aru. 

\question{Meelis Roosi\index[ppl]{Roos, Meelis} esimene 
programm oli ka vestlusprogramm, mis jätab
mulje, et arvuti saab millestki aru. See on läbiv joon.}

Jah. Üheksandas või kümnendas klassis tegime elu esimese 
programmi. Minu programm ei pakkunud mulle küll nii palju pinget, aga elu lõpuni 
jääb meelde ühe koolikaaslase oma, kus ülesanne seisnes selles, 
et tuli anda kuupäev ja siis pidi välja trükkima, mis nädalapäev see on. Aga 
Ural-1\index{Ural!Ural-1} ainuke väljundseade oli 
kitsas printer, kus sai trükkida ainult arve. Kui õppejõud proovis 
meie programme, siis andiski kõigepealt mingi mõistliku kuupäeva ette ja sai 
vastuse ning seejärel andis ette 30. veebruari, mille peale programm hakkas 
printeril ülalt alla nullide joru trükkima. Seepeale ütles õppejõud: 
\enquote{Nojah, ilmselt trükib mingit jama,} ja katkestas ära. 
Õpilane sõnas väga vaikselt ja tagasihoidlikult: \enquote{Las natuke trükib veel.} Ja trükkiski välja ühe nullide joru ülalt alla, siis ühe 
nullidega täidetud rea, ühe tühja rea, natuke nulle, siis natuke 
nulle äärtes ja keskel, seejärel veel kord nullide joru 
ülalt alla ja ühe terve nullirea ning lõpuks veel kord sama. Kui me selle 
paberi kätte saime, siis oli kõigile näha \enquote{LOLL}. 

\question{Kas tegitegi selliseid nuputamisülesandeid või mida te veel 
kirjutasite?} 

See oli üks huvitavamaid, aga tegime igasuguseid asju, näiteks leidsime mingite jadade keskmisi. Ma kõikide ülesandeid ei teagi, need olid individuaalselt antud tavalised ülesanded, mida ka praegused algajad 
programmeerijad teevad.

Muuseas, selle jaoks, kuidas kuupäevast nädalapäeva teha, on olemas valem. 

\question{Kas sel ajal oli Nõukogude Liidu peale juba ka mõnd arvutiasjade kogukonda, olümpiaade või muud sellist?} 

Ma ei tea, kuna programmeerimist õpetati ikka äärmiselt vähestes koolides. Ma ei 
tea ka, kas näiteks Moskvas matemaatikaklassis arvutit õpetati. Kui ma kunagi
Ülo Kaasiku\index[ppl]{Kaasik, Ülo} käest ühe intervjuu käigus küsisin, 
kust ta võttis metoodika meile programmeerimise õpetamiseks, siis ta vaatas 
mulle suurte silmadega otsa ja ütles: \enquote{Metoodikat ei olnud mitte 
mingisugust!} Vähe sellest, et ei olnud metoodikat, ei olnud ka kirjandust. 
Nii et ta hakkas ise raamatuid kirjutama ja metoodika oli tal nii-öelda 
käigu pealt välja töötatud.

\question{Ühesõnaga, tegelikult käis kogu Eesti asi Ülo Kaasiku peal?} 

Jah, see asi arenes kindlasti tänu talle ja ta õpilastele. Pärast 
seda hakkas matemaatikaklasse tekkima hulgi. Mäletan, et omaaegne Nõo direktor\index{Nõo Keskkool} Ove 
Karu\index[ppl]{Karu, Ove} istus meie matemaatika lõpueksamitel, et vaadata, kuidas me 
oleme matemaatikas arenenud. Mäletan isegi omaenese vastust. Kuna mul oli 
keeruline joonis teha, siis selle seletamisel läksin puntrasse. Astusin kaks 
sammu tahvlist eemale ja alustasin otsast peale: \enquote{See on see tõestus, et 
sirge on risti tasapinnaga, kui see on risti kahe sellega lõikuva sirgega.} See 
oli keskkooliprogramm. 

Meil on arvutiteaduse instituudis\index{Tartu Ülikool!Arvutiteaduse instituut} 
projektikoor -- koorilaul tuleb meelde poolteist aastat enne laulupidu ja siis 
võtame oma koori kokku ja valmistume nii hästi, kui suudame, ettelaulmiseks ning 
käime laulupidudel. Ja kui laulupeol küsiti, kuidas me oma koori 
iseloomustame, siis pakkus igaüks midagi välja. Mina ütlesin, et iga meie koori 
liige teab Pythagorase teoreemile erinevat tõestust. Neid on umbes 200, 
kooriliikmeid ligikaudu 40. Seepeale tuletasid matemaatika 
õpetamisega tegelevad õppejõud meie koorist mulle meelde, et 
Pythagorase teoreemi enam ei tõestata koolis. Nii et ma nüüd ei tea enam mitte 
midagi.

\question{Ühel hetkel sai keskkool otsa ja tuli niisiis Tartu Ülikool ja 
matemaatika?} 

Jah, aasta oli siis 1964 ehk sügav nõukogude aeg, mis välistas minu jaoks arusaadavatel 
põhjustel absoluutselt kõik humanitaaralad. Järele jäi suhteliselt vähe, sealhulgas meditsiin. Isa oli kirurg ja ju ta 
vaikselt lootis, et äkki lähen seda õppima. Mul on loogiliselt mõtlev mälu ja 
igasugused tõestused jäävad kergesti meelde, aga kui pean 
pähe õppima 2000 kontide nimetust ning lihaste, veresoonte, 
ajusagarate ja jumal teab veel mille ladinakeelsed nimed, siis ma ei 
arva, et tunneksin ennast väga hästi. Peale selle ei tahtnud me juba 5. keskkooli 
minna selle pärast, et vanematel olid seal liiga head suhted. 
Tahtsime ikka ise olla. Ka ülikoolis tahtsin olla rohkem ise kui keegi muu. 
Nii jäigi järele matemaatika ja seda ma armastasin koledasti. 

Mul olid viiendast seitsmenda klassini head õpetajad ja ma ei saa midagi 
paha öelda ka oma kaheksanda klassi õpetaja kohta, kes soovitas mind matemaatikaklassi. Lisaks matemaatika õpetamise korüfee Olaf 
Prinits\index[ppl]{Prinits, Olaf}, kelle tunde (näiteks funktsionaalse seose selgitust) mäletan siiamaani, ligi 50 aastat hiljem. 

\question{Kas Tartu Ülikoolis õpetati toona matemaatikutele ikka ka 
programmeerimist?}

Jah, meil oli kaks kursust programmeerimist. Üks oli masinkoodis 
programmeerimine Ural-4\index{Ural!Ural-4} peal, mille tegin 
kohe septembrikuu jooksul ette ära. Õppejõud võttis mu 
vastuse ja ütles: \enquote{Lõhnab natuke Ural-1 järele, aga programmeerida te 
oskate} ning pani mulle arvestuse. Teine oli Ülo Kaasiku\index[ppl]{Kaasik, 
Ülo} Algol 60\index{Algol!Algol 60} õpetus ja vot seda õpetas ta küll pliiatsi ja 
paberiga, sest ühtki translaatorit tol ajal Algoli jaoks ei olnud. 

\question{Miks just Algol?}

Mis keeled aastatel 1966--1967 üldse olemas olid? 

Jumalale tänu, et keegi ei hakanud meile Cobolit\index{Cobol} õpetama. See 
oleks mind küll programmeerimisest viie kilomeetri kaugusele peletanud! Olete 
proovinud kunagi Cobolit lugeda? Ei ole? Ärge proovige ka! Programmeerimine 
on kontsentreeritud väljendus, kus saab valemeid kirjutada valemitena, ja 
siis on mõningad kenad koodsõnad \verb|for|, \verb|do|, 
\verb|if-then-else| ja nii edasi. Nüüd pange sinna mingisugune filoloogide 
soust peale, kus \enquote{palk pluss preemia} tuleb välja kirjutada kolme 
sõnana! Vot niisugune programmeerimiskeel ja see oli vahepeal väga elujõuline.

\question{See kõik kõlab ikkagi 
eksplitsiitse programmeerimisõpetusena. Keegi õpetas metoodiliselt programmi 
kirjutama ja see on üsna haruldane. Üldjuhul ütlevad inimesed, et 
programmeerimine jäi kuidagi külge. Nad ei oska öelda, kes ja kus õpetas.}

Kui tahtmine on väga suur, siis kindlasti on võimalik programmeerimist 
iseseisvalt õppida, aga motivatsioon peab ikka seinakõrgune olema. 

\question{Mis toona arvutite perspektiiv oli, milleks neid kasutati? 
Programmeeriti küll, aga kas see oli teaduslik töövahend, aitas 
rahvamajandusele kaasa või veel midagi?} 

Rahvamajanduses oli kasutus juba Ural-1\index{Ural!Ural-1} ajal, kuigi mälu oli 
väga vähe -- kaks kilo, mida täna keegi ei usu. Õnneks mitte 
baiti, sest baiti tol ajal ei tuntud, vaid sõnu -- kaks kilo sõnu. Võtke oma 
telefon ja vaadake nende mega- või gigabaitide arvu, mis teil taskus on. 
Siis oli väga suur probleem, kuidas neid andmeid, mille pealt midagi pidi 
arvutatama, arvutisse ära mahutada. Aga need pakiti kokku. Välisseadmetega olid ka veel 
oma probleemid.  

Mäletan, kuidas ma ise masinkoodis programmeerimist õpetasin. Nimelt läksin miskipärast viimase kursuse viimasel semestril 1970. 
aastal mehele ja siis järsku langes ära minu Tallinnasse tööleminek, sest 
mees teatas surmkindlalt, et tema küll Tallinnasse ei kavatse minna. Siiamaani 
ei ole läinud. Pidin ruttu Tartus töökoha otsima ja Ülo Kaasik\index[ppl]{Kaasik, Ülo} ütles: 
\enquote{Jah, ma usun küll, et tema võib tudengite ette saata, tal jalad ei 
värise!} Lõpetasin ülikooli juunis ja 
septembris läksin tudengitele programmeerimist õpetama. 

\question{Kas mõnd teadustööd ei sirgunud sealt, oli puhas õppejõu töö?}

Teadusega hakkasin tegelema palju hiljem. Algajate õppejõudude nii-öelda jalul 
ehk auditooriumi ees seismise koormus (vähemasti siis, kui mina õpetasin) oli 24 
või 28 tundi nädalas. Selle kõrvalt ei jõudnud midagi muud teha, eriti esimestel 
õpetamisaastatel. Pealegi tuli õpetada kahes keeles,
matemaatika poolel oli vene keelt rääkivaid inimesi väga vähe. 

\question{Kas oligi kaks eraldi gruppi? Eesti ja vene?} 

Jah. Õpetasin elu esimese loengu majandusteaduskonnas vene 
keeles, vist kaugõppijatele. Vene keelt 
oskasin väga nirult. Esiteks ei olnud seda vaja. Teiseks olin küll lugenud 
väga palju matemaatikaõpikuid, aga võin neid tõestusi lugeda ka 
prantsuse keeles, mida ma ei oska, sest vahesõnu on seal
suhteliselt vähe. Õpetamiseks aga on vaja palju sõnu. 

Õnneks jätkus mul taipu: meie venekeelsetes rühmades oli alati 
kakskeelseid tudengeid ja ma tegin kokkuleppe, et kui ütlen midagi sellist, 
millest ei ole võimalik ka hea tahtmise juures aru saada, siis esimene rida 
annab mulle teada ja püüan ümber sõnastada. Ja kui sõna meelde ei tule, ütlen eesti 
keeles ja kakskeelsed ütlevad mulle venekeelse vaste.

Minu vene keele mitteoskust iseloomustab näiteks see, et ma ei 
teadnud, kuidas on \enquote{lahutama}. Üritasin kasutada sõna
\begin{russian}отнять\end{russian}, mis ei ole ka väga vale, aga sõna 
\begin{russian}вычесть\end{russian} ma ei teadnud.

\question{See on ju pedagoogilise metoodika mõttes õudselt hea kool!} 

Jah! Igal juhul umbes seitse-kaheksa aastat pärast seda 
sattusin kuskil Nõukogude 
kodumaal konverentsile, kus ütlesin midagi seltskonnas vene keeles, mille peale 
küsiti: \enquote{\begin{russian}А вы из Москвы?\end{russian}}. 
Vastasin seepeale, et kui ma teist korda veel suu lahti teen, siis 
saate kohe aru, et ma ei ole ei Moskvast ega ka Leningradist. 

\question{Ja õppejõu töö Tartu Ülikoolis läkski edasi, kuni saabus 
aasta 1980?} 

Veel kauem, siiamaani. 

\question{Aastal 1980 algas see aeg, kui asjad läksid lihtsamaks ja 
arvutid väiksemaks ning neid tuli juurde.}

Arvutid läksid jah väiksemaks. Kaheksakümnendatel oli arvutite saamine omaette tsirkus. 

\question{Jaan Tallinn\index[ppl]{Tallinn, Jaan} on rääkinud,\sidenote{Mitte küll nende kaante vahel olevas jutus.} et tema tõi oma 
esimese arvuti käsipagasis laevaga Rootsist.} 

Minul see ei õnnestunud, sest ma ei olnud nii rikas. Abikaasa oli Uppsalas postdoktorantuuris ja ma käisin seal küll, aga
arvutit ei toonud. Kuigi ma tean, et kui oleksin sealt ostnud arvuti ja siin 
maha müünud, oleks selle eest väga palju muid asju saanud. Paraku mul 
sellist ärivaimu ei olnud. 

\question{Millal hakkas Tartu Ülikoolis\index{Tartu 
Ülikool!Matemaatikateaduskond} arvuti tavaliseks asjaks muutuma ja sai 
igapäevaseks osaks näiteks matemaatikateaduskonna elust?} 

Võibolla üks murdepunkt oli aastal 1982. Tallinna 
sõbrad organiseerisid Tallinna näituseväljakul välisnäituse, kuhu tuli sadakond firmat. Kuigi seal oli näitusi varem ka olnud, köitis see minu tähelepanu, kuna
kohale pidi tulema kellegi sõber 
Suurbritanniast, kes oli hakanud Apple'i diileriks. Ta pidi ehitama ühe seadme ülikõrgete rõhkude jaoks, 
aga seadet oli vaja juhtida ja selleks valis ta välja Apple'i. 
Odavaim viis seda saada oli hakata Apple'i diileriks. Ja nüüd pidi ta siia
näitusele tulema. Ma ei tea, mida muud ta veel tõi, aga igal juhul valmistusid
Teaduste Akadeemia\index{Teaduste Akadeemia} instituudid ja pooled head 
tuttavad tema käest Apple'i arvuteid ostma. 

Mul läks hammas 
koledasti verele: sõbrad saavad miskipärast arvuteid, aga tegelikult on neid 
hoopis mulle vaja. Milleks näiteks kadunud professor Lippmaale\index[ppl]{Lippmaa, 
Endel} arvuti? Minul oli arvutit vaja! Rohkem
kui tal! Tal olid magnetresonantsid, mida ta pidi juhtima, 
aga meie õpetasime tulevasi programmeerijaid välja ja pidime kasutama 
\emph{input-output} kappi, kui keegi teab, mis see on. Tudeng paneb sinna
perforeerimiseks oma blanketile kirjutatud programmi ja saab kolme-nelja päeva 
pärast tagasi süntaksivigadega väljatrüki. 

Selleks, et üldse aru saada, millest 
Apple koosneb, istusin pühapäeviti Tõraveres\index{Tõravere Observatoorium}, 
sest seal käis Byte,\sidenote{Aastatel 1975--1998 välja antud USA 
arvutiajakiri, mis oli kaheksakümnendatel vägagi mõjukas.} kust sain teada, 
mis on arvuti sees ja mis selle külge käib. Panin oma konfiguratsiooni kokku ja sain aru, et mul ongi vaja nii-öelda
alasti arvutit. Pidasin aru Tallinna sõpradega Lippmaa 
instituudist\index{KBFI},\sidenote{Selle nime all tunti KBFid.} mis on 
mõistlik summa, mida plaanikomiteest küsida. Leidsin, et 10 000 kuldrubla (välisturul rubla 
ei toiminud, ainult kuldrubla) oleks piir, mida võiks küsida. See on 
nii pisike summa, et keegi äkki annabki. 

Saime selle eest kolm arvutit, isegi 
monitore ei ostnud, ainult ühe igaks juhuks, sest mine sa isahane tea. Võibolla meie 
nõukogude televiisoritega ei töötagi. 

\question{Mis arvutid need olid?}

Apple II\index{Apple II}. Plaanikomitees käisid minu eest Lippmaa 
instituudi sõbrad Tõnu Karu\index[ppl]{Karu, Tõnu} ja Riivo 
Sinijärv\index[ppl]{Sinijärv, Riivo}. Mina ajasin kõik paberid korda, korjasin 
ülikoolist allkirjad peale kuni rektorini välja ja nemad käisid kohal. Antigi 10 000 ning me saime oma kolm arvutit.

Tegime nendest arvutiklassi ja panime käima 
programmeerimisõppe individuaalgraafikus praktikumidega. See oli väga 
tore aeg, sest inimesed olid harjunud \emph{input-output} kapist saadud paberirulliga oma süntaksivigadega, mida tuli siis parandada. 
Selle pisikese, elu esimese või teise programmi silumine võttis mitu 
nädalat aega. 

\question{Järelikult tarkvaratehniline pool muutus radikaalselt!} 

Jah. Arvutid saime kätte jaanuaris ja veebruari alguses panime 
programmeerimise algõpetuse käima. Kuna meile BASICu\index{BASIC} keel ei 
meeldinud, aga Apple II\index{Apple II} on sündinud BASICuga, mõtlesime, et 
suudame ehk kompenseerida BASICu hädad oma hea õpetamisega. Esimese 
programmi jaoks kõlbas küll. 

Kevadel paistis sinna Liivi tänava\index{Tartu 
Ülikool!Liivi õppehoone} klassi päike. Seisin arvutiklassis tudengite 
selja taga, et neid aidata, ja tegelikult ma ei näinud, mida nad 
ekraani peale kirjutasid, vaid nende enda peegeldust. Ja see, kuidas tudeng 
sisestas oma programmi, pani käima, sai sealt ise oma süntaksivead kohe kätte ... 
See miimika, eriti tütarlaste oma, tasus vaatamist, mõistmaks, et oleme kuhugigi suunas õige sammu astunud! 

\question{Õpetamise metoodika pidi ka ju muutuma?} 

Jah, koos keelega muutub see alati. Põhikisma, mida algõpetuses arutatakse, on 
see, missugust programmeerimiskeelt õpetada esimesena, sest sealt jäävad 
asjad külge. BASICu\index{BASIC} häda on see, et kui olin selles
valmis kirjutanud (küll juba Yamaha arvutil, mida me kasutasime laialdasemalt 
arvutite tutvustamiseks) programmi, mille väljatrükk oli umbes minu enda 
pikkune, siis ma vandusin, et see on minu viimane programm BASICus. Nimelt ei ole selles keeles 
funktsioone ega alamprogramme. Olen seda vannet siiamaani pidanud. 

Mulle õpetati kõigepealt masinkoodi, sest Urali 
peal mingit kõrge taseme keelt ei olnud, samuti mitte assemblerit. Ja nii 
mulle tundus, et mängin otse registritega ja 
saangi aru, mis masinas toimub ja mida see aritmeetiline plokk seal teeb, ning edasi on kõik väga lihtne. 

Kunagi kui võitlesin ühel üleliidulisel 
seminaril selle eest, et alustada tuleb ikkagi masinkoodist, rääkis üks mu nüüdseks 
juba kadunud sõber Novosibirskist minu seisukoha 
ilmestamiseks anekdoodi. Vaene mees läheb 
kirikuõpetaja juurde ja ütleb, et elu on kole raske: peab naise, laste ja ämmaga
elama oma onni ühes toas. Kirikuõpetaja 
küsib: \enquote{Kas sul kits on?} \enquote{Jaa, kits mul on!} \enquote{Ole hea, võta kits ka tuppa.} Talumees imestab, milleks veel kits, aga kuna kirikuõpetaja käsib, siis võtabki 
kitse tuppa. Siis ütleb kirikuõpetaja: \enquote{Nädala pärast vii kits välja 
ja tule räägi minuga.} Nädala pärast viibki mees kitse välja ja kiidab kirikuõpetajale:
\enquote{Nüüd on küll väga hästi, kitse ei ole ja saan oma naise, ämma ja lastega palju paremini hakkama.} Masinkoodist alustamine on 
nagu kitse toomine tuppa, et kui lõpuks saab hakata programmi kirjutama 
\verb|for|, \verb|if| ja \verb|else| abil, siis tekib suur lõõgastus. Enam ei pea 
tõstma midagi registrisse, kontrollima, andma suunamist ja nii edasi. 

\question{Kui 
alustasime sellest, et sulle meeldis programmeerimine, siis ühel hetkel andis 
positiivse emotsiooni näoilme muutus ekraanipeegeldusel. Mis hetkest muutus
programmeerimise õpetamine huvitavamaks kui programmeerimine, kui 
üldse niisugune hetk on olnud?}

Jah, ma usun küll. Selleks et midagi väga tähelepanuväärset programmeerimises ära teha, on vaja 
head meeskonda. Parim suur asi, mida ma
programmeerimises olen teinud, on omal ajal koos Ain Isotammega\index[ppl]{Isotamm, Ain} kirjutatud süsteem Villis, mis oli aruannete 
generaator ja millel olid väga tähelepanuväärsed omadused. See oli ühtlasi elu 
keerulisim programm, mis tegeles magnetlindi ja printeri juhtimisega 
aruannete väljatrükkimise ajal, kui aruanded on pandud segamini magnetlindile (küll tekitamise järjekorras, aga 15 aruannet korraga, jupid vaheldumisi) 
ja printerist pidid tulema kõik aruanded õiges järjekorras. Mul oli kaks 
katkestuste allikat ja tasakaalu hoidmine printeri ja magnetlindi 
puhvrite vahel. Lisaks teisendamistöö seal vahel, et andmetest teksti 
tekitada, aga see oli köömes. 

Selle programmi tegemise käigus õnnestus mul 
muuseas avastada arvuti viga -- see on iga programmeerija unistus. Sulle kogu 
aeg tundub, et arvuti teeb valesti, sest teed ise kõik õigesti, 
aga arvuti eksib. Ja siis lähed seda insenerile rääkima. Kuna 
insener ei tea konteksti, siis hakkad asja seletama
algusest peale. Kuskil poole peal saad aru, millal oled ise vea teinud, haarad oma väljatrükid ja 
ütled insenerile, kes sinnamaani ei ole veel millestki aru saanud, aitäh 
posti mängimise eest ning lähed oma viga parandama. Aga selles programmis oli tõesti
arvuti viga, sest sünkro impulss traatprinterile oli 
halvasti joodetud. Kõik inimesed trükkisid sümbolhaaval, äärmisel 
juhul terve rea kaupa. Mina tahtsin terve lehekülje valmis teha ja
korraga trükkida. Printer trükkis mulle veerand lehekülge, pool 
lehekülge ja edasi lihtsalt ei trükkinud. Kontrollisin printeri juhtkäsku -- 
kõik on õige, aga ei trüki. Siis insenerid avastasid, 
milles asi. See on ainus kord minu programmeerijakarjääris, kui mul on õnnestunud 
arvuti viga avastada. 

\question{Kusjuures mitte arvuti kui tüübi viga, vaid et konkreetne tükk on 
spetsiifilisel viisil katki!}

Jah, konkreetne printer. Aga see on ikkagi riistvara! Inseneri, mitte programmeerija pärusmaa.

\question{Kuidas ikkagi õpetamine läks huvitamaks kui programmeerimine?}

Õpetamise värk on huvitav olnud kogu aeg, muidu ma ei oleks selle 
koha peal nii kaua\ldots{ }Tallinnas oli palju arvutuskeskusi, Tartus mitte eriti, aga ega
tolleaegne automatiseeritud juhtimissüsteemide 
tegemine mind väga ei tõmmanud ka. Nõukogude tehnika töökindlus oli niivõrd vilets. Isegi 
ameeriklastelt üle kavaldatud esimesed Nõukogude personaalarvutid ei kippunud 
hästi töötama. 

Õpetamine oli palju toredam! Meile räägitakse siiamaani, kuidas 
Eestis on ikka veel 8000 IT-spetsialisti puudu. 

\question{Need maagilised Schrödingeri 8000 spetsialisti\ldots \sidenote{Scrödingeri kass on kvantsuperpositsiooni illustreeriv mõtteline eksperiment, mille käigus kass on korraga nii elus kui surnud. IT-spetsialistid on meil korraga nii puudu (sest nii öeldakse) kui ka olemas (sest tööhõive sektoris kogu aeg kasvab).}}

Jah, see on enam-vähem konstantne suurus juba 
palju aastaid. Põhiline ei ole number 8000 ise, vaid
et neid ja neid on puudu -- näeme seda kogu aeg oma teise kursuse pealt. 
Nimelt omandavad inimesed esimese kursuse programmeerimise algõpetuse, teevad 
oma esimese projekti objektorienteeritud programmeerimises ja siis on nad 
firmale kasulikud ja ülikool lükatakse tagaplaanile. Kes veab välja 
ja kes mitte. Enamus ei vea ja jätavad ülikooli sinnapaika. 

\question{Kustkohast tuli idee õpetada välja arvutiõpetajaid? See eeldab 
ägedat visiooni, et keskkoolis või üldse koolis peaks arvutiõpetust 
õpetama.} 

Sellel oli kaks juurikat. Esiteks hakkasid arvutid 
jõudma ka koolidesse. Mitte ainult matemaatikaklassidesse, vaid ka mujale, sest 
tekkisid lihtsalt sellised arvutid, mis said jõuda. Selge see, et Ural-1 ei jaksanud ükski 
kool osta; füüsiliselt oleks see võimlemissaali mahtunud, 
aga seda ei saanud arvuti pärast ära kaotada. Nii et tekkis vajadus õpetada välja õpetajaid, kes 
läksid kooli arvutit õpetama. 

Teine juurikas oli arvutiside, mis hiilis koolidesse tagauksest. Riiklikud programmid, näiteks
Tiigrihüpe, tulid märksa hiljem. Esimesed ligi sada kooli said arvutisidega ühendatud nii, et paljud ülemused ei 
teadnudki üldse, et see olemas on. 

\question{Mille külge need läksid?}

Osa läks modemiga, telefoniside kaudu. 

\question{Tartusse kuskile?} 

Alustame algusest, 1980ndatest. Kõigepealt 
olid Fido vennad, kelle tegevusest ei teadnud ka mina 
suurt midagi, sest amatööride ridadesse ma ei kuulunud. Kuulsin 
alles hiljem, et Fido vendadel oli modemside juba enne, kui arvutid jõudsid UUCPga 
võrku. 

Kui siis arvutivõrgust midagi kuulda oli (ja kaheksakümnendatel juba
oli), siis otsustasid meie Soome sõbrad Nõukogude Eestit järele 
aidata. Tallinn on Helsingile palju lähemal kui Tartu, nii et Tallinna 
Tehnikaülikool\index{Tallinna Tehnikaülikool} sai soomlastelt modemi. Ja 
kuigi meil oli uuendatud telefoniside seoses 1980. aasta olümpiaga, ei pidanud 
selle liinid modemi kiirusele vastu ja soomlased pidid kinkima natuke 
aeglasema modemi, mis võttis ka madalamaid kiiruseid. Sedasi saadi Tallinnast 
UUCP-protokolliga side. 

Meie Tartus olime ka uljad: muretsesime 
endalegi modemi ja üritasime Tartust Helsingisse helistada. Aga Tartus ei 
olnud isegi mitte 1980. aasta olümpiamängude purjeregatti ja meie telefoniliinid 
ei pidanud aeglasele modemile vastu. Siis üritasime 
Tallinnasse helistada. 

Kuna me ei jõudnud kiireid modemeid osta, hankisime
mingi odava sisemise modemi. Aitäh Mati Kilbile\index[ppl]{Kilp, 
Mati}, kes oli tol ajal matemaatikateaduskonna\index{Tartu 
Ülikool!Matemaatikateaduskond} dekaan ja need pisikesed valuutasummad 
leidis. Sest kust sa hing kaheksakümnendatel ikka midagi ostad -- valuutapoest! 

Ühesõnaga, lõpuks käiski niimoodi, et meie helistasime Tallinnasse ja Tallinn helistas UUCPga 
Helsingisse. Alguses kaks, siis kolm korda nädalas, seejärel juba 
iga päev ja mitu korda päevas, kuna maht läks järjest 
suuremaks. Helsingist alates oli päris korralik internet olemas. 

\question{Mille külge keskkoolid tulid?} 

Vahepeal pandi meil tänu Rootsi Kuninglikule Akadeemiale ja 
Sorose Fondile\index{Sorose Fond} üles satelliitside. Nad said nii palju raha 
kokku, et hankida otsaseadmed -- kahes eksemplaris, sest Tartu-Tallinna vahet ei 
jõua ju keegi ära kakelda. Nii et Tartus tähetorni\index{Tartu tähetorn} otsas oli 
Tele-X satelliidi vastuvõtujaam ja teine satelliitseade oli Tallinnas Teaduste Akadeemia katusel (kuna Lippmaa 
instituut\index{Lippmaa Instituut|see{KBFI}} oli endiselt veel 
akadeemia raamatukogu all -- raamatukogu teisel ja nemad 
esimesel korrusel). 

Alguses orienteeriti vist valele satelliidile, meie seal kaasa ei mänginud, see 
oli Rootsi-poolse otsa tegevus. Tekkis küsimus, kes maksab kinni
järgmise inseneri, kes tuleb ja seab õige peale. Aga siis toimus
Rootsi kuninga visiit Tartusse\sidenote{Nende Majesteetide Rootsi kuninga Carl 
XVI Gustafi ja kuninganna Silvia riiklik visiit Eesti Vabariiki toimus 22.--24. 
aprillil 1992.} ja teenusepakkuja (ma ei mäleta, mis firma 
see oli) tahtis näidata, et nemad on kõikjal ja et Rootsi kuningas saab võtta 
telefonitoru Tartu Ülikooli rektori kabinetis ning ühenduda kohe oma koduga 
Rootsis. Seetõttu nad saatsidki mehe, kes pani side õige satelliidi peale. 
Nii et Rootsi kuninga visiidist alates oli meil 64kilone side üle satelliidi. 

Tallinna-Tartu side oli väga huvitav. Nagu me teame, on Tallinnast Tartusse 285 kilomeetrit, teistpidi on alati rohkem olnud. Aga kui 
nüüd vaadata, missuguse tee pidi läbima elektronkiri, et jõuda Tartust 
Tallinnasse, siis kõigepealt pidi see minema Tartusse tähetorni\index{Tartu tähetorn}, siis tähetorni 
satelliidi pealt Kuninglikku Rootsi Tehnikaülikooli Stockholmis ja sealt, olles 
avastanud ee-domeeni -- kas meil tol ajal oli .ee või tuli see natuke hiljem? 
Esimesed aadressid tulid muuseas .su lõpuga --, läks kiri üle satelliidi 
Tallinna Tehnikaülikooli ja siis kulges natuke veel Tallinnas. Arvutasin kunagi, et see tegi
kokku üle 70 000 kilomeetri\phantomsection\label{sisu!70k}. Kaugemal ei ole Tartu 
Tallinnast kunagi asetsenud! Õnneks läbiti see tee valguse kiirusel. 

\question{Kust tuli visioon, et internetti on üldse vaja?}

Olime tolleks ajaks UUCP-sidega juba imeasju teinud. 
Õpetasin internetti ka tudengitele, sest e-kirja teel oli 
võimalik igasuguseid asju saada, näiteks RFCsid, interneti 
alusdokumente. Tudengite 
arvestusülesanne oli mõni RFC kohale meelitada ja seega oli meil kettal 
peaaegu kogu interneti dokumentatsioon, üksikuid üritasime isegi 
välja trükkida, näiteks RFC 822, mis oli elektronkirjade aluseks. Neid oli väga huvitav uurida. Peale selle olime juba suure hulga ülemaailmsete 
listide liikmed. Informatsioon levis. Ja üleüldse oli internet tore asi. 

\question{Nii et saite maigu suhu?}

Jah, edasi said kõik aru, et meil on päris internetti ka vaja, kuigi 
veeb ei olnud veel sündinud. Meil oli Gopher.

Veeb mõeldi välja CERNis, kus
toodetakse palju artikleid ja need artiklid viitavad üksteisele. 
Artiklist arusaamiseks pead viidatavaid artikleid ka lugema ja 
igavene tüütus on käia neid kuskilt otsimas. Siis mõeldigi välja 
World Wide Web, kus on artikkel ja kohe ka pildid sees (see nõudis 
juba graafilist brauserit) ning viited niimoodi, et saad nendel klõpsida 
ja siis tuleb järgmine artikkel kohale. 

\question{Kuidas jõudis veeb Tartusse ja tuli 
mõte, et võiks hakata seda ka õpetama?} 

Need mõtted tulid peaaegu korraga. Kuidas veeb jõudis Eestisse, 
küsige Marek Tiitsu\index[ppl]{Tiits, Marek} käest. Tema töötas tol ajal Tartu 
Ülikooli raamatukogus\index{Tartu Ülikool!Raamatukogu}, kus oli kuskilt 
päranduseks või kingitusena saadud arvuti. Ma ei mäleta, milline, aga sellele oli võimalik veebiserver peale panna, sest, 
vabandage mind väga, veebiserverid töötavad kõiki Unixis. 

Eesti esimese veebiserveri pani püsti Marek Tiits\index[ppl]{Tiits, Marek} 
ja kursusel, mille nimetus oli vist juba tol ajal informaatika 
didaktika, üritasin ma lugeda igasuguseid uusi asju, kaasa arvatud internetti, sest ega mul internetikursust ei olnud. 
Kutsusin Marek Tiitsu tudengitele rääkima nii Gopherist kui 
ka veebist. 

Marek Tiits\index[ppl]{Tiits, Marek}, kes on praegu kindlasti üks paremaid 
lektoreid üldse, oli tollal teise kursuse tudeng. Kui ma pärast küsisin oma 
tudengitelt, kas kõik said aru, mida ta 
rääkis, siis tudengid vastasid: \enquote{Kui enne ei oleks midagi 
teadnud, siis vist ei oleks aru saanud, aga kuna teadsime enne ka midagi, 
siis saime teda üsna hästi jälgida.} Nii et esimese veebiserveri au on tõepoolest
Marek Tiitsul. 

Eesti Biokeskuses\index{Eesti Biokeskus} oli Sun SPARCStation\index{SPARC!SPARCStation}. 
Saatsin oma tudengid (mulle üldse meeldib tudengitega igasuguseid 
lollusi teha) hankima ülikooli pealt (igaüks sai ise teaduskonna) 
igasugust informatsiooni, mida õnnestub kätte saada: teaduskonna 
allüksusi, loetavaid aineid, mida iganes. Sellest nõelusime kokku
toreda ülikooliveebi. Seejärel jätkus meil nahaalsust kutsuda kohale 
rektor ja kaks prorektorit -- palusime neil istuda arvuti taha ja vaadata, kuidas 
Tartu Ülikool\index{Tartu Ülikool} veebis välja näeb. 

\question{Ja nägi välja küll!}

Nägi välja niisugune nagu Eesti metsad praegu: 
noorendik ja lageraie, siis mõni vana tükk ja nii edasi. See oli väga 
lapiline, tulenevalt sellest, kes kust mida kätte sai, ja tükati oli info kindlasti väga halvasti kajastatud. Kujunduse peale me muidugi 
väga palju auru ei raisanud (ikkagi matemaatikateaduskond!). Igal 
juhul sai rektor teada, mis asi on veeb, kuigi tol hetkel sai Eesti veebiserverid
näppudel üles lugeda. Nad võtsid asja üle ja hakkasid päris ülikooliveebi tegema. Rektorid on meil 
alati olnud suhteliselt taibukad, nii palju kui mina neid näinud olen. 

\question{Enne kui jutule joone alla tõmbame, on mul üks 
abstraktne küsimus. Ma ei tea, kas sellele ongi head vastust, aga kui keegi 
teab, siis tõenäoselt sina. Kui suur osa kõigest sellest, mis algas professor Kaasiku\index[ppl]{Kaasik, Ülo} pusimisest 1970ndatel ja on meid Liivi tänavalt
siia Deltasse toonud, on mäetipul kaugusse vaadates püsti pandud visioon ja 
kui palju sellest on \enquote{teeme järgmised kaks nädalat ägedaid 
asju}?} 

Osakaaludeks ma seda jagada ei jõua, aga näiteks internetikoolitus 
õpetajatele oli küll see \enquote{teeme kahe nädalaga kihvte asju}. Kuna Eestis 
arenes internet pärast siia jõudmist
penikoormasaabastega, siis otsustasime, et hoiame ka õpetajaid 
kursis. Tegime igasuguste koolituste käigus suures ringauditooriumis (tol 
ajal ei küsinud ülikool auditooriumi eest tasu, raha meil ei oleks 
olnud) infopäevi: rääkisime, mis on internet ja kuidas see on arenenud. 

Ühel hetkel tekkis Marek Tiitsul\index[ppl]{Tiits, Marek} 
europrojekt, mille käigus ta sai 100 modemit -- 50 oli tal projekti jaoks vaja, 
aga 50 võisime koolidele jagada. Me ei hakanud neid niisama loopima, vaid 
korraldasime süsadminnide või postmasterite kursused. Tegime 
kombineeritud kursusi, kus oli viis õpetajat rühmas: osadele
õpetasime hiire liigutamist, teistele programmeerimist, 
kolmandatele, kuidas modemit paika panna ja sinna teenuseid peale 
tõmmata, ja neljandatele, mis on veeb. Lisaks õpetasime 
süsadminne. 

Kursuseid saime teha kaks korda nii, et kõik 
kohaletulnud koolid, kes läbisid kahepäevase kursuse (nädalavahetusel, õpetajad koolitusid entusiasmist ja omal kulul), 
said kaasa modemi. Meil puudus kontroll, mis nendest pärast sai. 1995. 
aastal ei mahtunud õpetajad kuskile ära, isegi mitte Vanemuise 
suurde auditooriumisse\index{Tartu Ülikool!Vanemuise tänava 
õppehoone!Ringauditoorium}, ja meie võhm hakkas otsa 
saama. 

Käisin PTUIs, Pedagoogika Teadusliku Uurimise Instituudis, kes 
organiseeris õpetajate koolitusi, küsimas, kas nad meie kursustele ei tahaks 
raha anda. Rääkisin, et telekommunikatsioon tuleb kohe, ja 
ülemus, kelle nime ma kahjuks ei mäleta, vastas: \enquote{Misasi? Tele? 
Kommunikatsioon? See asi ei tule Eesti kooli mitte kunagi!} Panin suu kinni 
ja jätsin ütlemata, et 50 kooli on juba modemiga ühendatud. Sain aru, et sealt august raha ei tule, ja keerasin otsa 
ringi. (Hiljem andis meile natuke raha Sorose 
fond\index{Sorose Fond} ehk Avatud Eesti Fond\index{Avatud Eesti Fond}.)

Siis tegimegi ilma rahata. Keegi EENetist, Enok Sein\index[ppl]{Sein, 
Enok} vist, alguses ka Marek Tiits\index[ppl]{Tiits, Marek}, lisaks mu 
tudengid. Nagu ma ütlesin, siis Tartu Ülikool ei küsinud 
auditooriumide ja arvutiklasside kasutamise eest tasu, sest raha meil nagunii ei olnud. 
Kõik me tegime seda puhtast entusiasmist, samuti esimesed e-kursused, kui ei jõudnud enam suuri kahepäevaseid kursusi 
korraldada. Istusime Terje Tuisuga\index[ppl]{Tuisk, Terje} kahekesi 
koos ja mõtlesime, et teeks nüüd õige teisiti. Modemid on ju olemas, 
e-kirju nad saavad, üks inimene on koolis, kes oskab modemi käima panna. 
Korjame tema ümber viis õpetajat, teeme neile koolituse. 

Esimesel koolitusel unustasime piirarvu panemata. Andsime õpetajatele teada, et niisugune koolitus 
tuleb, ja just see inimene, kes modemiga hakkama saab, 
registreerigu oma kool. Arvasime, et kui viis kooli tuleb, on jube 
hästi. Panime fiktiivse meiliaadressi, kuhu nad pidid registreeruma. Meil oli nii kiire, et ei käinud seda vaatamas, ja ühel hetkel 
oli juba üle 20 kooli kirjas. Siis mõtlesime, et 
mis vahet seal on. Mõningad asjad tuli ära muuta: kui on viis 
kooli, igaühest viis inimest, siis nad võivad kõik meile oma elu esimese 
e-kirja saata ja vastame kõigile individuaalselt. Kui aga pärast oli kokku 50 kooli 
ja 400 osavõtjat, saime aru, et me ei jõua isiklikult igaühele 
kirjutada. Lasime neil omavahel suhelda -- panime nad 
paari niimoodi, et neil pidi vähemalt 50 kilomeetrit vahet olema ja nad pidid olema sama aine 
õpetajad.

\question{Täpselt programmeerija lähenemine 
ülesandele! Paneme paari, ei tohi olla üksteise lähedal ja peab olema sama 
aine!}

Tuleb loogiliselt mõelda, see tuleb elus õige mitmes 
kohas kasuks! 

\question{Nende kuldsete sõnadega võikski ehk lõpetada, aga mul on 
üks küsimus veel: millega professor Anne Villems praegu oma aega täidab?}

Esiteks tulin just USAst. Kahjuks minu tuttavad, 
kellega vanasti tihedalt läbi käisin ja kes olid Moskva erinevates 
instituutides ja ülikoolides, ei ole enam Moskvas, vaid Californias. Avastasin, et California külastamiseks on kõige meeldivam aeg 
jaanuari lõpp ja veebruari algus, mis sobib mulle, sest just siis on ülikoolis 
vaheaeg. Nüüd olengi vist neli aastat järjest käinud oma sõpradel Californias külas. 

Veetsin seal just kaks nädalat 
ja mind ära saates ütles sõbranna: \enquote{Nüüd sa oled aru saanud, et kaks 
nädalat on õige aeg, mitte üks nädal. Nii et tule järgmine kord ka kaheks 
nädalaks!} Seega talvel võib mind alati leida 
Californiast kevadisest Palo Altost, kus nii kohalikud kui ka 
sissetoodud taimed, nagu näiteks eukalüptipuud, õitsevad. 

\question{Mis on kõik hurmav lisaks sellele, et tegemist on Palo Altoga!}

Mis on Stanfordi kodulinn ja kus Palo Alto ja ookeani vahel on toredad 
mäed. Mägedes on tore käia. Üks sõber 
viis mu San Jose lähedal mäe otsa, kust oleksime pidanud nägema ühel pool 
San Franciscot ja teisel pool San Josed. Oli piimjas udu ja vihmapilved, ja siis korraks tuli tuul 
ning lõuna pool nägimegi seda vaadet, mida pidi nägema. Nii et jah, väga tore 
on reisida! Aga siin ma veel töötan, tunnitasu alusel, ja loen oma 
armastatud andmebaaside kursust. Kolmes versioonis.


\chapter{Kokkuvõte}
Kuna tegu on inimeste isiklike lugudega, on neid raske kuidagi üheselt kokku võtta: 
kõik lood on unikaalsed, need põimuvad, segunevad kummalistel ja vahel 
ebaloogilistel viisidel, viivad kuskilt kuhugi ja igasugune katse midagi 
üldistada teeb lugudele ja rääkijatele ülekohut. 

Mõnda torkab siiski silma. Kõigepealt see kummaline tõmme, mis inimestel arvutite suhtes oli. Seejuures 
ei ole tegu lihtsalt tehnikahuviliste noorte huviga tehnika vastu. 
Pigem vastupidi: mitmel juhul öeldakse, et arvutid üldiselt ja programmeerimine 
spetsiifiliselt olid pigem vahendid millegi muu saavutamiseks kui eesmärk 
iseeneses. Ka täiesti teiste huvidega (Jaanuse ja Tarvi puhul näitlemine) 
inimesi tõmbas miskipärast tugevalt arvuti poole. Jaanuse kasutatud metafoor 
\enquote{lendamise trennist}, millest on võimatu niisama mööda minna, kajab 
igalt poolt vastu. 

Seejuures tundub see miski, mida arvutite abil saavutada, tugevalt humanistlik, 
üldinimlik, ja ehk seletab mõnel määral meie toonase arvutikogukonna teket. 
Ahti sõnastab seda kui sarnaselt mõtlevate noorte inimeste püüet koos, 
üksteisele toetudes inimeseks saada. Priit ja Jaan ütlevad, et neid võlus asjaolu, et kõik, mis on võimalik inimese peas, on võimalik ka 
arvutis. Eks iga teismeline on kogenud frustratsiooni võimetuse üle viia 
ellu oma suurepäraseid ideid. Ühtäkki aga asendus kontrolli puudumine 
ümbritseva üle täiusliku kontrolliga arvuti üle koos piiramatu vabadusega 
suhelda teiste omasugustega. Ja mis võiks olla veel paeluvam, kui võimalus oma 
unistusi koos teistega ellu viia?

Just koos. Võiks ju arvata, et arvutiinimesed tegelevad pigem arvutite kui 
inimestega, kuid koos tegutsemine ja võimalus suhelda teiste omasugustega on 
oluline teema pea kõigis lugudes. Üksteiselt õpitakse, saadakse abi. Koos 
tehakse suuri asju ja ühel või teisel moel jookseb rõõm headest kaasteelistest 
läbi enamikust lugudest. Kindlasti on ka rivaalitsemist ja tülisid. Tallinna ja 
Tartu asetsesid mingil hetkel hea põhjusega üksteisest 70 000 kilomeetri 
kaugusel.\sidenote{Vt lk \pageref{sisu!70k}.} Ometigi domineerib arusaam, et 
tähtis on olla osa kogukonnast ja et kogukond toimib vaid kõigi osapoolte 
heast tahtest. Skype'i lugugi on ju vaadeldav kui lugu sõprusest.

Tugev tõmme arvuti poole võib aga ilma sobiva keskkonnata lihtsasti vaid 
platooniliseks igatsuseks jääda. Lugude alusel võis see keskkond võtta 
mitmeid vorme, näiteks mitmel puhul maagilise paigana mainitud kooli raadioruumi 
või vanemate arvutitega seotud töökoha näol, ning vahel oli kodus olemas 
elektroonikahuvi. Samas on ka näiteid, kuidas inimene ületab teel arvutini 
hoomamatuid takistusi ning jõuab kaugele. Ehk keskkond kahtlemata toetab 
arvutihuvi, kuid ei ole kuuldud lugude põhjal ilmtingimata vajalik.

Lugedes arvuti juurde jõudmise ja jäämise kogemustest, torkab 
silma tänasega võrreldes drastiliselt erinev ja eriline suhe arvutiga. Juba ammu ei ole arvuti ja 
internet asjad, millele ligipääs on probleemiks. Kuid toona olid arvuti ja 
sellel toimiv tarkvara erinevalt praegusest väga lihtsad; tänapäeval on nii arvuti kui ka
tarkvara ühele inimesele terviklikuks mõistmiseks selgelt liiga keerulised. Näiteks Arne ja Meelis ütlevad, et nad said oma arvutist lõpuni aru: alates BASICu 
detailidest kuni riistvarani välja. Ühelt poolt andis see põhimõtteline 
erinevus kogemuse kontrollist ja teisalt saavutuselamuse. Tihti oli koolipoisil 
puhtpraktiliselt vaja luua olemasolevaga samaväärset või isegi paremat 
tarkvara. Nii Jaan kui ka Andres kirjutasid toimiva ja kasuliku 
tekstiredaktori\sidenote{Sama lugu on olnud mujalgi 
(\url{https://corecursive.com/058-brian-kernighan-unix-bell-labs/}), arvutite 
algusaegadel kulus väga palju auru, võimaldamaks arvutisse teksti sisestada. 
Donald Knuthi \LaTeX, tänu millele ka see raamat sündis, lahendab samuti 
tekstiga seotud probleeme.}, sest seda oli vaja. Tänapäeval ei ole sellisteks 
ettevõtmisteks sageli ei praktilist vajadust ega ka sisulist võimalust.

Need lihtsamad masinad paigutusid mõnes mõttes märksa lihtsamasse sotsiaalsesse 
konteksti, kus segavaid faktoreid oli vähe ning keskendumisvõimalusi palju. 
Jah, kindlasti on teatud vanuses noor inimene juba piisavalt nutikas 
huvitavateks programmeerimisülesanneteks, kuid ei ole veel takerdunud 
täiskasvanu ellu. Samas on teadlik keskendumine nii mitmeski loos 
läbiv teema. Ja on selge, et tänases kommunikatsioonile vaikimisi avatud 
keskkonnas nõuab keskendumine teistsuguseid ja kindlamaid oskusi kui toonases 
suletud kontekstis.

Arvutite ja tarkvara lihtsus võimaldas luua väga kiiresti väga 
kasulikku tarkvara. Jaani tekstiredaktorit sai juba mainitud, aga Masti ja 
Marguse kahe kuuga kirjutatud modemipank oleks samuti tänapäeval küllaltki 
ennekuulmatu asi. Teisalt jookseb juttudest läbi terviku tajumise teema, 
mis tänaste arvutite puhul on raskem. Tõnis ja Tõnu mainivad, kuidas nad ei saa 
keerulistest asjadest aru ning kui oluline on võime taandada keeruline 
probleem lihtsamale kujule. Hoomatavat tervikpilti arvuti ja 
arvutivõrgu toimimisest on lihtsamate arvutite puhul kindlasti suhteliselt 
kergem luua. Samas jääb loodud mudel adekvaatseks ka keerukamate 
süsteemide puhul: Tõnul ei ole probleem tegelda mikroelektroonikaga ja Vilve 
ehitab ülikeerulisi finantssüsteeme, sest neil on olemas lihtsatest toimivatest 
süsteemidest pärinev mõttemudel.

Kujutage ette, et teil on töö juures ülemuse kabinetis umbes pool 
miljonit eurot maksev aparaat ja teie varateismeline laps avaldab soovi sellega 
veidi mängida. Kõlab hullumeelselt? Ometi toimiti kaheksakümnendatel 
täpselt nii kõikvõimalikes asutustes üle Eesti, lubades igas vanuses jõnglasi 
toonases mõistes hirmkalleid arvuteid näppima. Veelgi enam, sagedasti võeti 
rüblik lausa palgale, kuna selgus, et ta jõud käib arvutist üle (sest need 
olid suhteliselt lihtsad!) ning temast on kasu. Mõnda sellist motiivi sisaldab 
peaaegu iga lugu, joonides alla vastastikust usaldust tehnoloogide ja 
mittetehnoloogide vahel. Veelgi enam, kuna arvutiteemalist praktilist haridust 
nappis (väga vähesed meenutavad ülikooli kui olulist arvutiteadmise 
allikat), eksisteeris vahe tehnoloogide ja mittetehnoloogide vahel vaid 
nominaalselt, sõltudes pigem isiklikust huvist kui institutsionaalsest 
määratlusest. Teisisõnu võis arvutiga sina peale saada kes iganes, tehnoloogid olid 
inimesed meie endi keskelt ja seega olid neid lihtsam usaldada.

Ma usun, et usaldus inimeste, kes saavad aru probleemidest, ja 
nende vahel, kes saavad aru lahendustest, on Eesti IT eduloos põhimõttelise 
tähtsusega. Mõlemad osapooled ju mõistavad, et nende huvides ei ole usaldust 
kuritarvitada: kui IT-kutti liialt nöökida, läheb ta mujale, ning kui öise 
mängusessiooni tagajärjed päevatööd häirivad, võetakse võtmed käest. Selsamal 
vastastikusel usaldusel ja sellest tuleneval koostööl põhinevad nii ID-kaart, X-Tee, Hansapank kui ka kõik teised meie eduloo peatükid. Võib ju olla 
visioon teistmoodi pangast, aga tuleb uskuda, et IT-inimesed selle ka valmis 
ehitavad. Ükski riigiametnik ei ärka ühel hommikul mõttega XML-sõnumite 
liikumisest asutuste vahel. See on inseneri mõte ja vajab realiseerumiseks usku 
sedalaadi mõtete kasulikkusse. Omavahelised usalduslikud suhted olid kindlasti 
olulised ka kogukonna sees, kus suhteliselt väikesearvuline seltskond teadis üksteist 
vähemalt nime pidi ning \enquote{letihinnast ikka allahindlust tegi}. 

Usaldusel on kindlasti ka teine pool. Enamik siin raamatus toodud lugudest 
oleksid märksa lühemad, kui toona oleks rakendatud tänapäevases mõistes 
infoturvet. Kindlasti oleksid suured tükid meie IT edulugu olemata, kui 
tarkvarapiraatlusele oleks vaadatud samamoodi kui praegu. Ometi ei kosta 
lugudest usalduse kuritarvitamist, pigem räägitakse üle võetud masinate 
paikamisest ja omanikule tagastamisest. Samamoodi tekib ilmselt küsimus, kui 
legaalne oleks toonane suhteliselt kinnise seltskonna \enquote{käsi peseb kätt} 
lähenemine riigi- ja erasektori piiril tänase hankeregulatsiooni kontekstis. 
Kuid ka siin kostab pigem lugusid riigi raha eest võimalikult hea tulemuse 
toomisest (näiteks Tarvi ja sidemastid) kui seitsme naha koorimisest. 

Hea küll. Maagiline kast tõmbab maagilise jõuga noort inimest enda juurde. Kuid 
mida kohale jõudnuna selle kastiga ette võtta? Kust tulevad vajalikud 
oskused? Läbiv joon on siin selgelt ise õppimine. Seejuures on 
tähelepanuväärne, et institutsionaliseeritud õppimist meenutatakse sisu mõttes 
kasulikuna pigem harva, kuid vaimsuse, seltskonna ja kultuuri mõttes 
valgustavana pigem sageli. On üksikuid erandeid, nagu Ahti ja Vilve, kuid 
üldjuhul inimesed ei oska vastata, kuidas nad programmeerima või elektroonikaga 
tegelema õppisid. Vastupidiselt tänapäevale, kui suund tundub olevat võimalikult 
paljude inimeste programmeerima õpetamisele, võtavad toonase suhtumise 
kenasti kokku Tõnise ütlus: \enquote{Õppida tuleb raskeid asju, lihtsad 
tulevad iseenesest} ning Andruse oma: \enquote{Programmeerimine sünnib 
vajadusest.}

Õppimismeetodina räägitakse palju kas plokkskeemide abil või niisama paberil 
programmeerimisest ning ega perfokaartide abil programmi loomine sellest palju 
ei erinenud. Võib arvata, et ülimalt kõrge barjäär (arvutil kas puudus üldse 
interaktiivne konsool või oli ligipääs sellele väga piiratud) programmi 
sisestamisel sundis rohkem süvenema ja oma koodi läbi mõtlema, viies 
programmeerimiskunsti metoodilisema ja sügavama mõistmiseni, kui internetist 
koodijuppide kopeerimine anda saab.

Samas on tolles ebamäärases ja seletuseta õppeprotsessis väga selge ja suur 
roll kogukonnal. Enamjaolt puudus arvutite kohta ametlik kirjandus, teadmine 
levis folkloorina suust suhu, seda kasutati väärtusliku kaubana, jagati 
vaid valitutega ja kirjutati märkmikesse. Kogukonnaks võis olla 
arvutiklassis kogunev poistekamp, mõnd arvutifirmat ümbritsev seltskond, aga ka 
kooliklass, konkreetne institutsioon (KBFI) või lihtsalt füüsiline koht (Tartu 
tähetorn). Anto ütleb mitmel puhul, et õppis üht või teist asja oma kooli 
poistelt. Siit koorub ehk ka võti, mõistmaks, miks kujunes enamasti tugevalt 
introvertsest arvutirahvast Eestile hoo andnud tugev kogukond. Kuna suurem osa 
teadmisest tuli kellegi teise käest, muutus suur suhtevõrgustik isikliku arengu 
mõttes hädavajalikuks. Tippudel pidi olema väga hea suhtevõrgustik ja kuna 
kõigil suhetel on vähemalt kaks osalist, aitasid nad arendada ka teiste kogukonna 
liikmete võrgustikku ning oskusi. Üllatavalt sageli näeme inimesi tegutsemas 
mingit sorti müügifunktsioonis, mis jällegi rõhutab sotsiaalsete oskuste 
olulisust. 

Kogukonnad võivad olla isetekkelised, kuid üldjuhul mainitakse mõnda 
konkreetset inimest, kelle ümber koonduti. Keegi ei mäleta, et nad oleksid 
Jaak Loonde\index[ppl]{Loonde, Jaak} käest midagi konkreetselt õppinud. Küll 
aga meenutatakse tema hindamatut rolli arvutiklasside tekitamisel ning, mis 
veelgi olulisem, sinna kogunenud seltskonna jaoks katalüsaatorina toimimisel. 
Lõvi\index[ppl]{Lõvi}, Antot\index[ppl]{Veldre, Anto}, Annet\index[ppl]{Villems, 
Anne}, Tarmot\index[ppl]{Mamers, Tarmo} ja teisi meenutatakse soojalt lisaks 
nende teadmistele ka kogukonna loojatena. 

Nende kaante vahele kogutud lugudes torkab silma tugev kallutatus 
eestlastest meesterahvaste poole. Kindlasti tuleneb see osalt ka autorist, kuid 
ka lugudes tegutsevad tavaliselt eesti keelt rääkivad mehed. Seejuures, kui pildile ilmub mõni 
naine, teeb ta seda võimsalt, mõjutades paljusid ja 
liigutades metafoorseid mägesid (Vilve\index[ppl]{Vene, Vilve}, 
Anne\index[ppl]{Villems, Anne} ja kindlasti Kersti\index[ppl]{Kaljulaid, 
Kersti}) või olles peategelase oluliseks suunajaks (Anto ja Ahti emad). Eesti 
ja vene kogukondade omavaheline suhe on aga keerulisem. Ainsana loob nende 
vahele tõsisema silla Sergei\index[ppl]{Anikin, Sergei}, kelle jutust avaneb 
tõeline paralleelmaailm oma seltskondade ja õpetajatega, nagu Jaak Loonde. 
Vilve jutust läbi jooksev keerulise nimega Moskvale allunud asutus annab aimu, 
et eksisteeris ka terve eraldiseisev, enamasti vene töökeelega arvutitega 
tegelevate organisatsioonide võrgustik. Mõlemal puhul tundub, et ühel või 
teisel põhjusel oleme jätnud suure hulga tarku inimesi tähelepanuta, ja sellest 
on kahju.

Lisaks juba mainitud müügitööle on mõnevõrra üllatav meedia, sealhulgas 
trükimeedia, oluline roll inimeste lugudes. Pangandus kui Eesti tehnoloogia 
taimelava on teada-tuntud fenomen, meediast on selles kontekstis vähem 
räägitud. Ometi olid Kaspar, Peeter, Sten ja Taavi ning teised üht- või teistpidi 
seotud pabermeediaga ning Kaspar toimetas teles. Ilmselt oli meedia valdkond, 
kuhu esimesel võimalusel liikus raha ja kus tehnoloogia abil oli võimalik 
saavutada suur kvaliteedihüpe. Tehnoloogia aga tõmbas ligi teatud liiki 
inimesi.

Teiseks mõningaseks üllatuseks oli lugude tugev rahvusvaheline mõõde. Eesti NSV 
oli juba Nõukogude Liidus teistest erinevas rollis Soome füüsilise läheduse 
ja telekomi infra suhtelise kvaliteedi tõttu. Meilt oli teatud tingimustel 
võimalik \enquote{päris} välismaale helistada ning too side oli tänu lühikesele 
distantsile isegi arvutisideks kasutatav! See võimaldas toetada side osas 
näiteks Leedut ja toimida väravana kogu Nõukogude Liidu arvutirahva 
jaoks. Lugu Vladivostokist flopidega Tallinna tarkvara järele lennanud 
inimestest kõlab uskumatuna, kuid on ilmselt tõsi. Seejuures saime ka 
meie suurt abi Soomest ja Rootsist. Rootsis loodi meie esimesed 
satelliitühendused, Soome aitas Tallinna Tehnikaülikoolil modemeid hankida ja nii 
Soome kui ka Rootsi tehti tööd. Kindlasti tuleb ära märkida Ron 
Dwight\index[ppl]{Dwight, Ron}, kelle rolli Eesti Fido kogukonna arengus ei saa 
kuidagi üle hinnata. 

Kuidas siis võtta kokku \verb|print(memcpy[])|? 

Kuigi lugudest saab aimu, kuidas ja miks toonane arvutikogukond kujunes, ei saa 
me täit vastust küsimusele \enquote{miks just Eesti IT edulugu?}. Kahtlemata
mängisid oma rolli suure visiooniga inimesed, kuid palju oli pragmaatilist 
asjade ärategemist ja ka lihtsat lustimist. Õpetajad olid olulised, ent
enamasti mitte teadmiste edastajatena. Akadeemilised asutused olid olulised, 
kuid pigem üksikute kogukonnakollete võimaldajate kui institutsioonidena. 
Eraettevõtted olid olulised, aga olles lugenud toonase kauboikapitalismi 
kohta, valdab aknast Eesti elu vaadates kergendustunne. Kõikidel vastustel 
tundub olevat oma \enquote{aga}.

Niisiis, head vastust algsele küsimusele ei ole me leidnud. Küll aga koondab 
see raamat 29 suurepärase inimese lood. Ja ehk on sellest praeguseks küllalt.



\chapter{Fido ja BBS 1990-1991}

Ajatõmmis Eesti Fidonetist, regiooni 2:49 \emph{nodelist} seisuga 28. september 1990:

\begin{table}
\label{sisu:nodelist}
\centering
\begin{tabular}{lrllrl}
Region & 49  & Estonia              & Andrus Suitsu\index[ppl]{Suitsu, Andrus} & 2400  & MNP  \\
Host   & 490 & NET Estonia           & Tarmo Ausing\index[ppl]{Ausing, Tarmo}  & 9600  & HST  \\
       & 1   & Hackers Night System  & Tarmo Ausing  & 9600  & HST  \\
       & 10  & P.O.Box Maximus       & Andrus Suitsu & 2400  & MNP  \\
       & 20  & Goodwin BBS           & Sulo Kallas\index[ppl]{Kallas, Sulo}   & 2400  & MNP  \\
       & 30  & Mail Shark            & Madis Kaal\index[ppl]{Kaal, Madis}    & 1200  & XX   \\
       & 40  & MamBox                & Tarmo Mamers\index[ppl]{Mamers, Tarmo}  & 19200 & PEP 
\end{tabular}
\end{table}

Ajatõmmis Eesti BBS-i maastikust seisuga 22. veebruar 1991. See konkreetne versioon (kuigi kindlasti leidub teisigi), tuli märkega:

\begin{verbatim}
Compiled   by   Serge   A.   Terekhov   --    with
participation of Yuri PQ  (2:5010/2),  Maxim Nikitin  &
Vitaly Klochko (2:5000/30)
\end{verbatim}

\begin{table}[ht]
\centering
\begin{tabular}{llp{2cm}p{3cm}p{4cm}}
%\toprule
BBS                            & Fido & Modem              & Saadavus (nädala sees/nädala lõpp või üldine) & SysOp                       \\
\midrule
Eesti \#1                      &      & 9600/MNP           & 24                                            & Lembit Pirn\index[ppl]{Pirn, Lembit}                 \\
Flying Disks BBS               & +    & 2400/MNP           &                                               & Margus Sutt\index[ppl]{Sutt, Margus}                 \\
Goodwin BBS                    & +    & 2400/MNP           & 24                                            & Sulo Kallas\index[ppl]{Kallas, Sulo}\index[ppl]{Marvet, Peeter}, Peeter Marvet  \\
Great White of Kopli           & +    & 2400               &                                               & Urmet Jänes\index[ppl]{Jänes, Urmet}                 \\
Hacker's Inn                   &      &                    &                                               &                             \\
Hacker's Night System          & +    & 9600/USR, 2400/MNP & 18-08/24                                      & Tarmo Ausing\index[ppl]{Ausing, Tarmo}, Tõnis Reimo\index[ppl]{Reimo, Tõnis}   \\
Kroon                          &      &                    &                                               &                             \\
Lion's Cave                    & +    & 9600/HST           &                                               & Andres Lepp\index[ppl]{Lepp, Andres}                 \\
Mailbox for citizens of galaxy & +    & 1200               & 21-0230                                       & Madis Kaal\index[ppl]{Kaal, Madis}                  \\
MamBox                         & +    & 19200/PEP          & 20-08/24                                      & Tarmo Mamers\index[ppl]{Mamers, Tarmo}                \\
Micro                          &      & 2400               & 20-08/24                                      & Jan Kuman\index[ppl]{Kuman, Jan}                   \\
New Age System                 & +    & 2400               & 18-09/24                                      & Tanel Raja\index[ppl]{Raja, Tanel}                  \\
New Barbarian                  &      & 2400               & 23-11/24                                      & Yura Zaitsev\index[ppl]{Zaitsev, Yura}                \\
P.O. Box Maximus               & +    & 2400/MNP           & 21-11/20-11                                   & Andres Suitsu\index[ppl]{Suitsu, Andrus},Tarmo Soodla\index[ppl]{Soodla, Tarmo}  \\
The MESO                       & +    & 2400/MNP           & 19-08/24                                      & Viljo Allik\index[ppl]{Allik, Viljo}                 \\
PaPer                          & +    & 1200               & 20-09/24                                      & Taavi Talvik\index[ppl]{Talvik, Taavi}               \\
\bottomrule
\end{tabular}
\end{table}

%%
% The back matter contains appendices, bibliographies, indices, glossaries, etc.



\backmatter

\addcontentsline{toc}{chapter}{Nimeloend}
\printindex[ppl]
\addcontentsline{toc}{chapter}{Indeks}
\printindex


\end{document}

