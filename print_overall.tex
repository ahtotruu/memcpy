%!TEX TS-program = arara
% arara: latexmk: { clean: partial }
% arara: xelatex: { shell: true, synctex: true} 
% arara: makeindex
% arara: xelatex: { shell: true, synctex: true} 
% arara: xelatex: { shell: true, synctex: true} 
% arara: latexmk: { clean: partial }

\documentclass{tufte-book}
\usepackage[
    type={CC},
    modifier={by-nc-nd},
    version={4.0},
]{doclicense}


\ifxetex
  \newcommand{\textls}[2][5]{%
    \begingroup\addfontfeatures{LetterSpace=#1}#2\endgroup
  }
  \renewcommand{\allcapsspacing}[1]{\textls[15]{#1}}
  \renewcommand{\smallcapsspacing}[1]{\textls[10]{#1}}
  \renewcommand{\allcaps}[1]{\textls[15]{\MakeTextUppercase{#1}}}
  \renewcommand{\smallcaps}[1]{\smallcapsspacing{\scshape\MakeTextLowercase{#1}}}
  \renewcommand{\textsc}[1]{\smallcapsspacing{\textsmallcaps{#1}}}
\fi


\usepackage[T1]{fontenc}
%\usepackage[utf8]{inputenc}
\usepackage{polyglossia}
\setmainlanguage{estonian} 
\setotherlanguage{russian}
\newfontfamily\russianfont[Script=Cyrillic]{Linux Libertine}

\hypersetup{colorlinks}% uncomment this line if you prefer colored hyperlinks (e.g., for onscreen viewing)


%%
% Book metadata
%\title{print(memcpy[])\thanks{Thanks to Edward R.~Tufte for his inspiration.}}
\title{memcpy.print()}
\author[Andres Kütt]{Andres Kütt}
\publisher{TeamConsulting}

%%
% If they're installed, use Bergamo and Chantilly from www.fontsite.com.
% They're clones of Bembo and Gill Sans, respectively.
%\IfFileExists{bergamo.sty}{\usepackage[osf]{bergamo}}{}% Bembo
%\IfFileExists{chantill.sty}{\usepackage{chantill}}{}% Gill Sans

%\usepackage{microtype}

%%
% Just some sample text
\usepackage{lipsum}

%%
% For nicely typeset tabular material
\usepackage{booktabs}

%%
% For graphics / images
\usepackage{graphicx}
\setkeys{Gin}{width=\linewidth,totalheight=\textheight,keepaspectratio}
\graphicspath{{graphics/}}

% The fancyvrb package lets us customize the formatting of verbatim
% environments.  We use a slightly smaller font.
\usepackage{fancyvrb}
\fvset{fontsize=\normalsize}

%%
% Prints argument within hanging parentheses (i.e., parentheses that take
% up no horizontal space).  Useful in tabular environments.
\newcommand{\hangp}[1]{\makebox[0pt][r]{(}#1\makebox[0pt][l]{)}}

%%
% Prints an asterisk that takes up no horizontal space.
% Useful in tabular environments.
\newcommand{\hangstar}{\makebox[0pt][l]{*}}

%%
% Prints a trailing space in a smart way.
\usepackage{xspace}

%%
% Some shortcuts for Tufte's book titles.  The lowercase commands will
% produce the initials of the book title in italics.  The all-caps commands
% will print out the full title of the book in italics.
\newcommand{\vdqi}{\textit{VDQI}\xspace}
\newcommand{\ei}{\textit{EI}\xspace}
\newcommand{\ve}{\textit{VE}\xspace}
\newcommand{\be}{\textit{BE}\xspace}
\newcommand{\VDQI}{\textit{The Visual Display of Quantitative Information}\xspace}
\newcommand{\EI}{\textit{Envisioning Information}\xspace}
\newcommand{\VE}{\textit{Visual Explanations}\xspace}
\newcommand{\BE}{\textit{Beautiful Evidence}\xspace}

\newcommand{\TL}{Tufte-\LaTeX\xspace}

% Prints the month name (e.g., January) and the year (e.g., 2008)
\newcommand{\monthyear}{%
  \ifcase\month\or January\or February\or March\or April\or May\or June\or
  July\or August\or September\or October\or November\or
  December\fi\space\number\year
}


% Prints an epigraph and speaker in sans serif, all-caps type.
\newcommand{\openepigraph}[2]{%
  %\sffamily\fontsize{14}{16}\selectfont
  \begin{fullwidth}
  \sffamily\large
  \begin{doublespace}
  \noindent\allcaps{#1}\\% epigraph
  \noindent\allcaps{#2}% author
  \end{doublespace}
  \end{fullwidth}
}

% Inserts a blank page
\newcommand{\blankpage}{\newpage\hbox{}\thispagestyle{empty}\newpage}

\usepackage{units}

% Typesets the font size, leading, and measure in the form of 10/12x26 pc.
\newcommand{\measure}[3]{#1/#2$\times$\unit[#3]{pc}}

% Macros for typesetting the documentation
\newcommand{\hlred}[1]{\textcolor{Maroon}{#1}}% prints in red
\newcommand{\hangleft}[1]{\makebox[0pt][r]{#1}}
\newcommand{\hairsp}{\hspace{1pt}}% hair space
\newcommand{\hquad}{\hskip0.5em\relax}% half quad space
\newcommand{\TODO}{\textcolor{red}{\bf TODO!}\xspace}
\newcommand{\ie}{\textit{i.\hairsp{}e.}\xspace}
\newcommand{\eg}{\textit{e.\hairsp{}g.}\xspace}
\newcommand{\na}{\quad--}% used in tables for N/A cells
\providecommand{\XeLaTeX}{X\lower.5ex\hbox{\kern-0.15em\reflectbox{E}}\kern-0.1em\LaTeX}
\newcommand{\tXeLaTeX}{\XeLaTeX\index{XeLaTeX@\protect\XeLaTeX}}
% \index{\texttt{\textbackslash xyz}@\hangleft{\texttt{\textbackslash}}\texttt{xyz}}
\newcommand{\tuftebs}{\symbol{'134}}% a backslash in tt type in OT1/T1
\newcommand{\doccmdnoindex}[2][]{\texttt{\tuftebs#2}}% command name -- adds backslash automatically (and doesn't add cmd to the index)
\newcommand{\doccmddef}[2][]{%
  \hlred{\texttt{\tuftebs#2}}\label{cmd:#2}%
  \ifthenelse{\isempty{#1}}%
    {% add the command to the index
      \index{#2 command@\protect\hangleft{\texttt{\tuftebs}}\texttt{#2}}% command name
    }%
    {% add the command and package to the index
      \index{#2 command@\protect\hangleft{\texttt{\tuftebs}}\texttt{#2} (\texttt{#1} package)}% command name
      \index{#1 package@\texttt{#1} package}\index{packages!#1@\texttt{#1}}% package name
    }%
}% command name -- adds backslash automatically
\newcommand{\doccmd}[2][]{%
  \texttt{\tuftebs#2}%
  \ifthenelse{\isempty{#1}}%
    {% add the command to the index
      \index{#2 command@\protect\hangleft{\texttt{\tuftebs}}\texttt{#2}}% command name
    }%
    {% add the command and package to the index
      \index{#2 command@\protect\hangleft{\texttt{\tuftebs}}\texttt{#2} (\texttt{#1} package)}% command name
      \index{#1 package@\texttt{#1} package}\index{packages!#1@\texttt{#1}}% package name
    }%
}% command name -- adds backslash automatically
\newcommand{\docopt}[1]{\ensuremath{\langle}\textrm{\textit{#1}}\ensuremath{\rangle}}% optional command argument
\newcommand{\docarg}[1]{\textrm{\textit{#1}}}% (required) command argument
\newenvironment{docspec}{\begin{quotation}\ttfamily\parskip0pt\parindent0pt\ignorespaces}{\end{quotation}}% command specification environment
\newcommand{\docenv}[1]{\texttt{#1}\index{#1 environment@\texttt{#1} environment}\index{environments!#1@\texttt{#1}}}% environment name
\newcommand{\docenvdef}[1]{\hlred{\texttt{#1}}\label{env:#1}\index{#1 environment@\texttt{#1} environment}\index{environments!#1@\texttt{#1}}}% environment name
\newcommand{\docpkg}[1]{\texttt{#1}\index{#1 package@\texttt{#1} package}\index{packages!#1@\texttt{#1}}}% package name
\newcommand{\doccls}[1]{\texttt{#1}}% document class name
\newcommand{\docclsopt}[1]{\texttt{#1}\index{#1 class option@\texttt{#1} class option}\index{class options!#1@\texttt{#1}}}% document class option name
\newcommand{\docclsoptdef}[1]{\hlred{\texttt{#1}}\label{clsopt:#1}\index{#1 class option@\texttt{#1} class option}\index{class options!#1@\texttt{#1}}}% document class option name defined
\newcommand{\docmsg}[2]{\bigskip\begin{fullwidth}\noindent\ttfamily#1\end{fullwidth}\medskip\par\noindent#2}
\newcommand{\docfilehook}[2]{\texttt{#1}\index{file hooks!#2}\index{#1@\texttt{#1}}}
\newcommand{\doccounter}[1]{\texttt{#1}\index{#1 counter@\texttt{#1} counter}}

% Generates the index
\usepackage{imakeidx}
\makeindex[name=ppl, title={Nimede register}]
\makeindex[title={Indeks}]

% See also
\makeatletter
\newcommand{\gobblecomma}[1]{\@gobble{#1}\ignorespaces}
\makeatother

\usepackage{csquotes}


%% Versioneerimine

\newcounter{run}
\InputIfFileExists{\jobname.runs}{}{}
\stepcounter{run}

\usepackage{atveryend}
\usepackage{newfile}
\AtVeryEndDocument{%
  \newoutputstream{runs}%
  \openoutputfile{\jobname.runs}{runs}%
  \addtostream{runs}{\string\setcounter{run}{\number\value{run}}}%
  \closeoutputstream{runs}%
}

%% Küsimuse vormistus
\newcommand{\question}[1]{\textbf{\enquote{#1}}}

\begin{document}

% Front matter
\frontmatter

% r.1 blank page
\blankpage

% v.2 epigraphs
\newpage\thispagestyle{empty}
\openepigraph{%
Design and programming are human activities; forget that and all is lost.
}{Bjarne Stroustrup%, {\itshape Design, Form, and Chaos}
}
\vfill
\begin{fullwidth}
\sffamily\large
\begin{doublespace}
%\noindent\allcaps{Ärge valetage isad }\\ % The quote
%\noindent\allcaps{ära hoia kinni ema mind}\\ % The quote
%\noindent\allcaps{Need ei ole halvad sõbrad}\\ % The quote
\noindent\allcaps{\ldots}\\ % The quote
\noindent\allcaps{see on minu Vennaskond ja ring}\\ % The quote
\noindent\allcaps{Vennaskond. \enquote{Jumal kaitse vennaskonda}} % The author
\end{doublespace}
\end{fullwidth}
%\vfill
%\openepigraph{% 
%Ärge valetage isad ära hoia kinni ema mind Need ei ole halvad sõbrad see on minu Vennaskond ja ring}{Vennaskond. \enquote{Jumal kaitse vennaskonda}}
%\vfill
%\openepigraph{%
%\ldots the designer of a new system must not only be the implementor and the first 
%large-scale user; the designer should also write the first user manual\ldots 
%If I had not participated fully in all these activities, 
%literally hundreds of improvements would never have been made, 
%because I would never have thought of them or perceived 
%why they were important.
%}{Donald E. Knuth}


% r.3 full title page
\maketitle


% v.4 copyright page
\newpage
\begin{fullwidth}
~\vfill
\thispagestyle{empty}
\setlength{\parindent}{0pt}
\setlength{\parskip}{\baselineskip}
Copyright \copyright\ \the\year\ \thanklessauthor

\par\smallcaps{Published by \thanklesspublisher}

%\par\smallcaps{tufte-latex.googlecode.com}

\par \doclicenseThis 

\par\textit{\monthyear. Version V0.\therun}
\end{fullwidth}

% r.5 contents
\tableofcontents

%\listoffigures

%\listoftables

% r.7 dedication
\cleardoublepage
~\vfill
\begin{doublespace}
\noindent\fontsize{18}{22}\selectfont\itshape
\nohyphenation
Pühendatud kõigile Eesti arvuti-inimestele, patsiga ja ilma.
\end{doublespace}
\vfill
\vfill


% r.9 introduction
\cleardoublepage
\chapter*{Sissejuhatus}
Juhatame sisse. 
\begin{itemize}
	\item Miks ma seda teen
	\begin{itemize}
		\item \enquote{Tahan kord saada selliseks, nagu on Villu või Freddy või Rott või Striit.}\sidenote{Villu Tamme, "Paneme punki"}
	\end{itemize}
	\item Eesmärk: kujutada inimesi ja nende suhteid (mitte näiteks kurioosseid hetki või ettevõtteid)
	\item Lühike ajalugu: idee, otsing, podcast, siis analüüs ja raamat
	\item Sisu kohta
	\begin{itemize}
		\item Kõik ei mahtunud raamatusse, kõik ei soovinud rääkida ja kõik ei tulnud pähe. Andestust!
		\begin{itemize}
			\item Välja on jäänud näiteks Mainor ja natuke vanema põlvkonna (näiteks kadunud Ahto Kalja) tegemised
		\end{itemize}
		\item Lood lähevad omavahel vastuollu, see on OK
		\item Kõik on isiksused. Mõned kergemad, mõned raskemad. Olen üritanud suhte-taagast üle olla
		\item Nii \enquote{läbipaistev} vaade, kui võimalik. Sealhulgas näiteks ka ehk liigselt anglitsismirohke keelekasutus
		\item Mõeldud olema ka mitte-arvuti inimesele üldjoontes arusaadav: konteksti mõistmiseks olulised terminid on lahti seletatud kuid detailid otsib huviline ise välja. Samas ei ole eesmärk anda struktureeritud ülevaadet arvutustehnika ajaloost või vanade tehnoloogiate toimimisest
		\item Inimesed tähestikulises järjekorras
		\item Oma jutt on ka, sest muidu jääks juttudesse kummaline auk, lisaks tuleks ju anda aimu, mis prisma läbi ülejäänud asjad on kirjutatud. Intervjueerisin ennast ise
		\item \enquote{Patsiga poisid} kui üldnimetus. Enamasti siiski poisid. Kahju küll, aga nii oli. Raamat on läbilõige toonasest seltskonnast ja oleks vale toda seltskonda kuidagi teistsugusena kujutada
	\end{itemize}
	\item Kuidas lugeda
	\begin{itemize}
		\item On indeks, eraldi inimeste oma
		\item On lühikesed selgitused mainitud riistvara ja arvutite osas
		\item Detailsema jutu leiab igaüks ise internetist
	\end{itemize}
	\item Tänuavaldused
	\begin{itemize}
		\item Rein Rüüsak, A\&A ajaloo välja uurimine
		\item Ott Köstner, memcpy kaanepilt
		\item Vootele Voit, info ZX Spectrumi kiibistiku kohta
		\item Kõik intervjueeritud
		\item Veebipõhine transkriptsioon (Alumäe, Tanel; Tilk, Ottokar; Asadullah. "Advanced Rich Transcription System for Estonian Speech" Baltic HLT 2018)
		\item Mroos toimetamine, kaasamõtlemine ja tehniline tugi
	\end{itemize}
	\item Toimetajavariandid
	\begin{itemize}
		\item Kadri Kustmann
		\item Marja Vaba
	\end{itemize}
 \end{itemize}

%%
% Start the main matter (normal chapters)
\mainmatter


\chapter{Andrus Aaslaid}
%!TEX TS-program = arara
% arara: myindex

\index[ppl]{Aaslaid, Andrus}
\question{Kuidas sa arvutite juurde jõudsid?}

Tihti on nii, et me ei mäleta, kuidas me oma elu muutvad otsused  
tegime. Aga seda juhust ma mäletan täpselt. Mul oli juba toona 
raadiohobi. Olin põhikooli juntsu ja mulle meeldis hirmsasti mööda 
lühilainet ringi kammida. Meil oli kodus Melodija 101 stereo, Riia 
raadiotehase\sidenote{A. S. Popovi nimeline Riia Raadiotehas, alates 1951 Rigas 
Radio Rupnica.} toodang. Sellega ma siis seiklesin suviti, kui midagi targemat 
teha ei olnud, mööda eetrit. Tegelikult oli mul kaks raadiot: lisaks Melodijale 
detektorvastuvõtja, mille mu poolvend 
oli mulle ehitanud. Sellega ma istusin pööningul. Vanemad tegelesid 
põllumajandusega ja neil oli 
üks konkreetne põllumajandusnipp: raamatukogudest toodi vanu ajakirju, 
need rebiti lehtedeks, keerati ümber õõnsa 
põhjaga pudeli väikesteks pottideks, mille sisse istutati taimed. 
Paber lagunes mulla sees ära, taim pääses põllul vabaks. Neid ajakirju oli 
pööningul tohutu hunnik, muu hulgas mitu aastakäiku 
\begin{russian}Техника - молодёжи\end{russian}'t\sidenote{Aastast 1933 ilmuv 
algselt Nõukogude ja nüüd Vene populaarteaduslik ajakiri.}. Lappasin siis 
pööningul neid ajakirju, detektoriklapid peas. 

Igatahes ükskord astusin ma tuppa, lülitasin Melodija sisse ja sealt öeldi, et 
Tallinna 43. Keskkool\index{Tallinna 43. Keskkool}\sidenote{Praegune 
tehnikagümnaasium.\index{Tehnikagümnaasium|see{Tallinna 43. Keskkool}}} 
on otsustanud hakata 
eksperimentaalseks tehnikaülikooli\index{Tallinna 
Tehnikaülikool}\sidenote{Tallinna Polütehniline Instituut, praegune Tallinna 
Tehnikaülikool.} ettevalmistuskooliks ja 
nad võtavad kümnendasse klassi vastu õpilasi, kes tahaksid TPIsse edasi õppima 
minna. Kuulasin uudise ära, lülitasin raadio välja, läksin vanemate juurde ja 
teatasin, et lähen Tallinnasse kooli. Ma olin siis 14.


\question{Kust sa pärit oled, et tahtsid Tallinna kooli 
minna?}

Pärit olen ma tegelikult kahesaja meetri kauguselt sealt, kus ma täna elan, 
ehk siis Tallinnast. Aga kuna mu perekond otsustas evakueeruda 
Muhusse, kui ma olin kahe- või kolmeaastane, siis mind 
deporteeriti sinna. Nii et oma põrsapõlve veetsin Muhus ja siis ühel 
hetkel panin sealt tagasi tehnoloogia juurde putku. 

\question{Mõni ime, et te Mastiga\index[ppl]{Kaal, 
Madis}\index[ppl]{Mast}\sidenote{Vt lk \pageref{cptr:mast}.} 
hästi läbi saate!}

Me oleme Mastiga ühe kooli poisid, Mast oli keskkoolis, kui mina olin 
põhikoolis. Me oleme mõnda aega isegi sama 
raadiosõlme väisanud. Aga ega tollel ajal nooremad ja vanemad väga läbi käinud, 
eriti veel 
maakohtades. Mast oli hea 
poiss, ei peksnud nooremaid ega midagi. 

\question{Mis sealt lühilaine pealt kostis? Muusikat?}

Ei, muusikat kuulati Radio Luxembourgist. Lühilaine pealt tuli erinevaid 
hääli: morset, huvitavaid kahinaid ja sahinaid, keegi 
luges numbreid. Lühilaine on tegelikult siiamaani päris hea tervise 
juures, eeter on maast laeni sodi täis ja olemus 
ei ole väga palju muutunud. Võibolla propagandasaateid on vähemaks 
jäänud ja Hiina raadiojaamu vaikselt kinni pandud  
interneti pealetulekuga. Üldiselt on lühilaine ilmselt ikka samasugune nagu 
nelikümmend aastat tagasi.

\question{Kas nende ajakirjade hulgas oli arvutiajakirju ka?}

Esimest arvutit nägin tänu poolvennale. Ta tundis Guido 
Tammissaart\index[ppl]{Tammissaar, Guido} Eesti Energia 
arvutuskeskusest\index{Eesti Energia!Arvutuskeskus}. Ühel 
päeval tuli poolvend maale ja ütles: \enquote{Tule kaasa paariks päevaks, näed, 
mis asi 
see arvuti on. Sind see tehnikaasi huvitab.} Ja lubatigi mind paariks päevaks 
maalt 
linna. Estonia puiestee arvutuskeskuses olid tollal veel põhiliselt 
SMid\index{SM EVM}\sidenote{\begin{russian}Система Малых ЭВМ (СМ 
ЭВМ)\end{russian} oli mitut tüüpi Nõukogude Liidus toodetud, enamasti lääne 
analoogidel põhinevate arvutite üldnimetus.}. Ja 
üks CP/M\sidenote{CP/M oli 1974. 
aastal Inteli 8080/85 protsessorisarja tarvis turule toodud 
operatsioonisüsteem, mille 1980ndatel asendas mitmes mõttes sarnane MS-DOS.} 
masin, mis tagantjärele tundub oma sotsmaa disaini poolest täiesti kosmiline. 
Küllap Bulgaarias toodetud. Olen mõelnud, et 
peaks üles otsima, mis masin see selline võis olla. 

Sellel CP/M masinal ma klõbistasin niisama, aga 
SM-4\index{SM EVM!SM-4}\sidenote{SM-4 oli PDP-11/40\index{PDP-11} 
ühilduv 
Nõukogude päritolu ja terves idablokis toodetud arvutisüsteem.} peal 
kirjutasin selsamal päeval oma esimese BASICu\index{BASIC} programmi. 
See oli derivaat mingist asjast, mida mulle näidati, et näed, umbes nii 
käib. Ja edasi ma olin \emph{hooked}. Sellest ühest päevast piisas, et sõltlane 
tekitada. 

\question{See oli enne seda, kui otsustasid, et nüüd oled 
neliteist ja lähed Tallinnasse kooli?}

Ma ei oskagi öelda, ma ei ole sada protsenti kindel, kumb oli enne, kumb 
pärast, ja kas huvi tulla Tallinnasse mängis rolli. Ega nad ju 
arvutikallakut tegelikult ei propageerinud, suurem rõhk oli elektroonikal. 
Tarkvara osa nad väga ei reklaaminud. Minust pidi tegelikult elektroonik saama 
ja see minust ka sai, aga tollal tundusid ikkagi arvutid 
see päris asi. 

\question{Kas 43. keskkoolis valmistati päriselt ka ette 
ülikooliks? Oli sellest kasu?}

See oli selline kahe teraga mõõk -- valmistati ette ja 
väga hästi. Keskkooliprogrammi olid kokku pannud 
tollaste inseneride õpetajad, kes teadsid suhteliselt hästi, mida tuleks 
õpetada, et põhi alla saada. Saime 
läbisegi tavalisi keskkooliaineid ja siis ühel hetkel tuli härra 
Tiidemann\index[ppl]{Tiidemann, Tiit} meile rääkima võllide 
epüüridest\sidenote{Epüür (pr \emph{épure}) on teatava suuruse asukohast 
olenevate väärtuste graafiline esitus.}. Sisuliselt tegime käsitsi võllidele 
rakendavate jõudude arvutusi, näiteks kust läheb võll katki, kui see on siit
sellise ja sealt säärase jämedusega. Vahelduseks loeti meile 
teise kursuse elektrotehnikat ja 
inseneripsühholoogiat, mida andis Toomsalu\index[ppl]{Toomsalu, Arvo} ja mis ei 
olnud vist üldse TPI õppekavas. Meie õppekava lühinimetusega ETEK\index{ETEK}, 
mille koostasid Ants Reili\index[ppl]{Reili, Ants} ja 
Peeter Grossberg\index[ppl]{Grossberg, Peeter}, oli kõikide jaoks äge 
eksperiment ja täielik \emph{greenfield}, eriti kuna 
olime esimene lend.\sidenote{vt ka lk \pageref{sisu:43kool}.} 

Lahe oli ka see, et enne meid oli keskkool tühjaks löödud ja me olime kolm 
aastat keskkooli kõige 
vanem klass. Olime koolis nagu jumalad ja 
tänu sellele jäid olemata mitmed probleemid, mida tavalistes 
keskkoolides tol ajal veel eksisteeris. Keegi kedagi ei toginud ega 
nüginud ja samal ajal tekkis kõigil mingisugune väärikus. 

Kahe teraga mõõk oli see aga sellepärast, et nii kõva põhja pealt läksid paljud 
otse tööle. Me saime ju keskkooli lõpetades kõik  
automaatselt TPIsse sisse, sisseastumiseksamit ei olnud vaja teha. Nii et kõik 
meie vist kaheksateist õpilast marssis otse TPIsse. Nendest 
nominaalajaga lõpetas kooli vist paar inimest. Paljud läksid tööle, kuna aeg 
oli 
selline, et see, mida TPIs tollal arvutiteadusena õpetati, ei jõudnud 
päris elule veel järele. See pidi olema aasta 1991 või 1992, kui 
see \enquote{kambriumiplahvatus} siin Eestis toimus.

Mina istusin ööd-päevad arvuti taga ja kirjutasin 
ihuüksi tarkvara, mis pidi 
üleval hoidma tervet suurt autoparki. Samal ajal üritasin ennast kuidagi nügida 
läbi SuperCalci\index{SuperCalc}\sidenote{Varajane tabelarvutussüsteem, 
algselt loodud CP\textbackslash M operatsioonisüsteemile.} arvestusest TPIs, 
kus 
aeg-ajalt tuli õppejõule näidata, et \enquote{ära nii tee, nii see asi päris ei 
käi}. 
Mitte et nad oleksid rumalad olnud, nad õpetasid seda, mida olid kogu 
aeg õpetanud. Nüüd aga tekkis selline seis, kus reaalne elu liikus edasi palju 
kiiremini kui õppekava.

\question{Kuidas sa ikkagi programmeerimise juurde jõudsid? Sa 
pidid seda ju saama kuskil harjutada?}

Tänu 43. keskkoolile see eksperiment kestis ja kestab mõnes mõttes tänaseni. 
Seal oli 
põhimõtteliselt esimest korda selline päris arvutiinimese elu. Kuna 
IT-spetsialiste liiga palju ei olnud, siis juhtus selline hämar lugu, et meile 
Eero Tohvriga\index[ppl]{Tohver, Eero} ulatati kümnendas klassis arvutiklassi 
võtmed ja hakati 
koolist palka maksma. Tegelikult oli see vist seotud 
kerge koolipoolse kaastundega. Peale 
kaheksandat klassi tööstuskooli tulemise traditsiooni ei olnud enam juba 
mõnikümmend aastat ja kõigile tundus see kangesti hirmus, et laps tuleb üksi 
Tallinnasse. Ma arvan, et see oli pigem koolipoolne stipendium. 
Kahe peale maksti meile täisõppejõu palka, mis 
ei olnud ilmselt palju väiksem kui õpetajad 
ise said. Nii hästi kui keskkooli ajal ei ole ma kunagi ei varem ega 
hiljem elanud. 

\question{Mida te selle rahaga tegite?}

Käisime restoranis söömas ja mida ikka lapsed rahaga teevad. Aga kool sai 
selle, et nad ei pidanud rohkem arvutiklassiga tegelema. Klassis oli kolm-neli  
Iskrat\index{Iskra}\sidenote{\begin{russian}Искра\end{russian} oli 
mitmel pool Nõukogude Liidus eri modifikatsioonides toodetud arvutiseeria, mis
 omakorda jagunes erinevaid lääne süsteeme kopeerivateks mudeliperekondadeks.}, 
mida me püsti hoidsime. Meie asi 
oli hoolitseda, et masinad töötaksid ja nendel saaks midagi õpetada. Ühel
hetkel, kui olime ise juba natuke vanemad, tekkis arvutiklassi 
kamp nooremaid huvilisi, kes seal pidevalt hängisid. Arenes
tüüpiline arvutiklassi ökosüsteem. Ühel suvel ka remontisime 
klassi: värvisime ja panime uued põrandakatted. Ühesõnaga käitusime 
loodetavasti heaperemehelikult. 

\question{Tollal ilmselt ei olnud sarnastes situatsioonides hea{\-}peremeheliku 
käitumisega eriti 
probleeme?}

Aeg oli selline, inimeste usaldus oli suur. Arvuti oli müstiline ja teistmoodi 
asi, vanem 
generatsioon justkui kartis seda. Kunagi asus Rävala puiesteel, seal, kus 
praegu on Sakala 3 teatrimaja, turismibüroo Sarved ja Sõrad\index{Sarved ja 
Sõrad} (ma ei tea siiamaani, kellele see kuulus). Juhtusin nende akna alt mööda 
minema ja nägin, et neil on 
seal arvuti. See oli vist aastal 1991, igatahes ma veel ei töötanud 
Skriiningus\index{Skriining}. Keskkool 
oli läbi, sinna mind enam sisse ei lastud arvutit kasutama. Eks sõltlane käis 
mööda linna ja järsku nägi arvutit. Tundmatu värske keskkoolilõpetaja 
marssis tundmatusse firmasse hooga sisse, et 
\enquote{teil on siin arvuti, ma tahaksin seda kasutada}. Ja ilma mingisuguse 
tänapäeval heaks kiidetud taustauuringuta ja töövestluseta ütles firma omanik 
oma kirjutuslaua tagant: \enquote{Jah, loomulikult, me tahaksime seda ise ka 
kasutada.}  Pikema jututa anti mulle kontorivõtmed ja öeldi: \enquote{Tee 
see korda, et meie saaksime ka arvutit kasutada}. Ja avastasingi end
arvuti tagant, ilma et keegi oleks isegi dokumenti vaadanud või mõelnud, kas 
tegu on
vargaga, kes tahab terve firma ära varastada või 
ainult arvuti. Usaldus, mis tollal valitses inimeste vastu, kes 
oskasid arvuti sisse lülitada ja sellega midagi teha, oli 
\emph{enormous}.\phantomsection\label{sisu:andrus_usaldus} 
Tänapäeval ei ole võimalik seda ette kujutada. Värskel keskkoolilõpetajal oli 
põhimõtteliselt võimalik küsida ükskõik millise firma ükskõik millise  
arvuti \enquote{võtmed}. 

Noh, see lõppes muidugi sellega, et lõpuks tuli Imre Perli\index[ppl]{Perli, 
Imre}\sidenote{Imre Perli oli pehmelt öeldes raju elulooga Eesti 
arvutispetsialist, kes sai kuulsaks \enquote{Perli andmebaasi} koostajana. 
Kasutades ära ligipääsu mitmele andmebaasile, lõi ta üheksakümnendate keskel 
althõlma levinud \enquote{superandmebaasi}, mis sisaldas isikustatud andmeid autode, (toona üsna 
haruldaste) mobiiltelefonide, aadresside jms kohta.
 Perli hukkus segastel asjaoludel 15. aprillil 2000 
politseioperatsiooni käigus.} ja kopeeris kellelegi andmebaasid. Eks iga 
aeg saab lõpuks otsa. 

\question{Kuidas see programmeerima õppimise protsess ikkagi käis?}

See on eelmisel sajandil tekkinud paradigma, et 
programmeerimine on midagi, mida peab õppima ja millega tuleb 
spetsiaalselt vaeva näha. Programmeerimine juhtub. Vajadusest ja tahtmisest. 
Keegi ei ole mulle mitte kunagi õpetanud 
ridagi C-d ega assemblerit. 

\question{Ometi said ju kuskilt teada, kuidas \texttt{malloc} käib.}

See sündis tahtmisest teha. Mina hakkasin  
Pascalit\index{Pascal} õppima seepärast, et 
mulle sattus kätte Jürgensoni pruunide kaantega Pascali  
raamat\sidenote{Rein Jürgensoni 
\enquote{Programmeerimine Pascal-keeles} (1985), mis 
huviliste hulgas laialt levis.}, mis on tagantjärele mõeldes 
päris õudne algus programmeerimisele. Kui 
Turbo Pascal hakkas ära tüütama (selles 
keeles midagi normaalset teha oli väga keeruline), siis ühel hetkel 
leidsin, et assembler\index{assembler} on see päris asi. Kuna tol 
ajal oli popp kirjutada igasuguseid demosid ja häkkida kõiki tarkvarasid, mis 
kätte sattus, siis\ldots{ }Kuidas õppida x86 
assemblerit? Võtad raamatu ühte kätte ja AT86 teise kätte ning hakkad tegema.

\question{Kust sa selle raamatu said? Neid ju ei liikunud.}

Liikus küll. Selle eest tuleb tõenäoliselt varem või hiljem anda
presidendi auraha Tarmo Mamersile\index[ppl]{Mamers, 
Tarmo}\index[ppl]{Mamers, Tarmo}, kes oli tollal 
TTÜ-s\index{Tallinna Tehnikaülikool} üks arvutiasjanduse püstihoidjatest. 
Tarmo kaudu materjalid liikusidki, käest kätte. Tema oli raudselt minu varane 
mentor ja veel pikka aega ka siis, kui ma 
juba tööl käisin. Hiljem tuli 
FidoNet. Kui ma oma esimese FidoNeti \emph{point}'i 
püsti panin, siis oli kõik juba palju lihtsam, sest aken maailma oli
olemas. \emph{Point}'i püstipanemine käis ka loomulikult läbi TPI. Seal käis 
põhiline elu ja \emph{action}   
Aare Tali\index[ppl]{Tali, Aare}\phantomsection\label{sisu!aare_tali} ja Tõnu 
Raimla\index[ppl]{Raimla, Tõnu} toas. Tarmo juures teises ruumis oli natuke 
rahulikum 
õhkkond. 

Ühel  
hetkel (töötasin siis Skriiningus\index{Skriining}) tekkis mul kinnisidee teha 
endale FidoNeti \emph{point}, et 
lõpuks olla osa maailmast. Läksin Aare juurde: \enquote{Noh, Aare, 
sa oled siin \emph{sysop} ja värk} ning Aare ütles talle omase abivalmidusega: 
\enquote{Jah, masin on seal.} Leidsingi ennast seepeale BBSi masina tagant ja 
asusin
valmistama FidoNeti \emph{node}'i. Ilmselt Tõnu või keegi 
lõpuks halastas mu peale ja näitas, kuidas seda päriselt teha. 

Edaspidi oli materjal palju kättesaadavam, sai 
igasuguseid dokumente risti-rästi alla laadida. 

\question{Mida sa TPIsse õppima läksid?}

Ma läksin LIsse. Tollal nimetati seda vist informaatikaks. 
Kuna sain suhteliselt ruttu aru, et ma ei ole võimeline hommikul loengutes 
käima, siis läksime pundi inimestega, kes olid ka otsustanud, et nemad 
peavad õhtuõppes käima, dekanaati ja nõudsime õhtust vahetust. Kateedris öeldi, 
et jaa, väga tore mõte, aga 
meie kogemus näitab, et kui te juba sihukese jutuga tulete, siis vaevalt keegi 
teist seal õhtuses ka käima hakkab. Me ei hakka teie jaoks  
eraldi rühma püsti panema, käite ehitajatega esimese aasta koos koolis. Ja kui 
teisel aastal veel siin olete, siis vaatame seda asja. Kas nüüd osalt selle 
pärast või et dekanaadil oli õigus, nii või teisiti kukkusime 
sealt kõik robinal kolmanda kuu lõpuks välja ja läksime tööle. Nii 
et TPI on mul siiamaani lõpetamata. 

\question{Sa mainisid, et kirjutasid autobaasi softi. Kuidas 
sa seda tegema sattusid?}

Tol ajal \emph{start-up}-kultuuri ja ettevõtluse ehitamist veel ei 
eksisteerinud. Me lõpetasime kooli ajal, kui esimesi 
arvutikooperatiive oli väike käputäis. Minu esimene ametlik töökoht pidi 
olema tegelikult Noorsooteatri valgustaja. Kuna mulle juba 
tollel ajal meeldis audioga tegeleda, siis tahtsin sinna helimeheks minna, 
aga helimees oli värskelt tööle võetud ja valgustaja koht oli vaba. Paar päeva 
enne seda, kui pidin lepingu alla kirjutama, 
küsis Tarmo Mamers\index[ppl]{Mamers, Tarmo}, kas ma ei tahaks ikkagi päris 
tööd teha, kuna Skriining\index{Skriining} otsis programmeerijat. 

Nii sattusingi Skriiningusse Kalle Lotamõisa\index[ppl]{Lotamõis, Kalle} juurde 
tööle. Seal öeldi mulle esimese ülesandena, et \enquote{autopark on sellel 
aadressil}. 
Neil oli mingi eriti eksootilise asja peal jooksev andmebaasisüsteem, 
isegi mitte \emph{mainframe}, vaid mingi mini. Ja see tuli 
moodsale vahendile ümber kirjutada. Moodne vahend tähendas tol ajal Novell 
Netware'i\index{Novell} ja Paul Leis\index[ppl]{Leis, Paul} oli värskelt toonud 
Eestisse sellise asja nagu DataFlex\index{DataFlex}. Tegu oli päris 
 korraliku objektorienteeritud kõrgkeelega. Hakkasin ühest otsast õppima, 
kuidas DataFlexis programmeeritakse, ja 
teisest otsast, kuidas autopark töötab. 

\question{Ahaa, läksid kohe äriprotsessi ka sisse!}

Äriprotsessid olid seal paljuski olemas, st töötav tarkvara 
oli olemas. Pigem oli seal äriprotsesside seisukohast hea lastetuba, et ära 
kunagi eelda midagi. Näiteks mina oma IT-inimese mõistusega tegin oma 
arust mõned asjad paremaks ja siis selgus, et päris nii ei sobinud, nagu 
mina olin mõelnud. Raamatupidaja vaatas mind nagu idiooti ja küsis: 
\enquote{Kas sa ikka saad aru, kui palju ma pean numbreid siia päevas sisestama 
ja seda \emph{enter}'it, mille sa siia vahele toppisid, vajutama? Need arvud 
on neljakohalised. Ma sisestan neli numbrit ära ja 
need lähevad ise järgmisele väljale, mitte ma ei pea vajutama. Ma ei saa
vajutada \emph{tab}'i, mis on teises klaviatuuri otsas. Saad aru? Mul on ühes 
käes
paberid ja teise käega vajutan klaviatuuri. Kuidas ma sinna \emph{tab}'i juurde 
sinu 
meelest saan, kui mul on teine käsi kinni?} 

Nad olid väga innovatiivsed tegelikult selles mõttes, et nad olid 
sedasama andmetöötlust selleks ajaks juba aastat kuus-seitse kasutanud. See oli 
 meditsiinitehnika autobaas, Termak\index{Termak}, siiamaani elu ja tervise 
juures. 


\question{Kas nad olid juba nõukogude ajal arvutiasjandusega alustanud?}

Nad olid jah juba sügaval nõukogude ajal end täiesti ära automatiseerinud. 
Selleks 
ajaks, kui mina aastal 1992 sinna jõudsin, oli nende esimene IT-süsteem 
jõudnud moraalselt nii ära vananeda, et see tuli PCde peale ümber 
kirjutada. Neil oli siis juba \emph{legacy}, nad olid nii palju ajast 
ees.

\question{Kuidas Skriining jõudis selleni, et neil on programmeerijat 
vaja? Lihtsalt kasti sai ju ka edukalt müüa?}

Kalle\index[ppl]{Lotamõis, Kalle} hammustaski selle läbi, et kuna nad olid kogu 
aeg meditsiinitehnika ümber sebinud ja proovinud meditsiinisüsteemi arvuteid 
müüa, oli seal ka arendusvõimalusi. Nii saigi Skriiningust\index{Skriining} 
üheksakümnendate alguses arendusfirma. Arvutimüük käis ka, aga mina  
noore inimesena ei süüvinud sellesse, kust raha tuleb. Ilmselt päris palju tuli 
arendusest.


\question{Kas sa tehnikaülikoolis ka veel ringi hängisid?}

Ma hängisin seal pikalt, kuigi ma ei õppinud seal. Seal oli 
elu epitsenter, kuna seal töötasid kõik olulised inimesed: 
Mast\index[ppl]{Kaal, Madis} ülemisel korrusel, Tõnu\index[ppl]{Raimla, Tõnu}, 
Aare\index[ppl]{Tali, Aare} ja Tarmo\index[ppl]{Mamers, Tarmo} alumisel 
korrusel. Lisaks veel 
Martin Rinne\index[ppl]{Rinne, Martin}, Merle Alliksoo\index[ppl]{Alliksoo, 
Merle} ja kõik teised, kes hiljem MicroLinkis\index{MicroLink} lõpetasid. Tegu
oli sotsiaalse elu keskusega. 

\question{Mulle tundub see variant, et sa ei õpi, aga hängid, palju 
mõnusam, kui et õpid, aga ei hängi.}

Eks ma ise ikka soovitan teistele kool kohe 
ära lõpetada, sest pärast osutub see palju raskemaks. Mina ja mu sõbrad oleme 
hakanud 
neljakümnendates oma haridusega lõpuks tegelema. On 
tekkinud natuke rohkem vaba aega ja ka moraalne vajadus --
kuidas sa oled kõige väiksemate pagunitega mees ruumis \ldots

\question{Tol ajal ülikool kuigi palju praktiliselt 
kasulikku ei andnud. Tänapäeval on teistmoodi.}

Paljud ütlevad, et diplom ei olnud mitte tempel selle 
kohta, et tuled koolist välja targemana, vaid tõestus, et 
oled võimeline järjepidevalt, mitu aastat asjaga tegelema. See on pigem 
vastupidavuse ja hoolsuse proov kui koolitus.

\question{Räägi palun BBSidest. Kuidas sa selle \emph{node}'i ikkagi püsti 
said? 
Selleks tuli ju ennast kuskil registreerida?}

BBS oli varane arvutivõrk, mille mõte oli selles, et helistad 
kuhugi oma modemiga ja teises otsas on modem, kes vastab. Modemid saavad 
omavahel andmeühenduse ja siis saab teises arvutis, mille 
küljes teine modem on, ringi sobrada. Kusjuures tollal tõepoolest
sobrati, arvutiturvalisus oli pigem kokkuleppe 
küsimus. Üks suvaline BBSi omanik oleks võinud teise 
omaniku BBSi ilma mingi 
probleemita kaks korda tunnis neljaks tükiks lasta, aga seda lihtsalt ei 
tehtud. See oli nagu 
saarlase ukselukk: kui oled luku ukse ette paika pannud, siis kõik 
teavad, et sind ei ole kodus ja nii on. Ei ole vaja katsuda, kas uks 
on lahti või kinni, kedagi ei ole kodus. BBSidega turvalisusega oli sama lugu. 

BBSi teine ja palju kasulikum omadus oli see, et kui 
oli olemas modem ja arvuti, siis sai ennast FidoNeti 
\emph{node}'iks registreerida. BBS iseenesest ei eeldanud midagi sellist, vaja 
oli vaid
modemi ja vastava tarkvara olemasolu. Mingeid hämaraid teid pidi levisid 
telefoninumbrid, kuhu helistada ja end kohapeal ära registreerida.

FidoNet oli esimene üleilmne arvutivõrk selles 
mõttes, et modemid helistasid üksteisele automaatselt. See oli ka kaunikesti 
hästi toimiv elektronpostiteenus, mille üks eriline omadus 
oli veel see, et see liikus väljaspool KGB huviala. Eks küll 
kahtlustati, et seda kuulatakse pealt, ja aeg-ajalt mingid imelikud modemid 
üritasid sinu modemiga poole jutu pealt rääkida, aga üldiselt seda vist väga ei 
jälgitud. Ma vähemalt ei tea, et kellelgi oleks 
kaheksakümnendatel olnud modem-modemiga sidepidamisega probleeme, ei Eestis ega 
välismaaga. Mis on selles mõttes eriti huvitav, et kui kaugekõneliinid läksid 
nii palju lahti, et oli võimalik kuhugi automaatvalida, siis me ju helistasime 
igale poole välja.  
FidoNeti \emph{mail}, mis tuli Eestisse umbes aastal 1988 või 1989, oli esimene 
vaba ja 
demokraatlik sidekanal väljapoole.

Mina olin siis keskkoolis, esimese \emph{node}'i panin püsti umbes 1991. 
aastal. Ma olingi vist Aare \emph{point}. 
Omaenda \emph{point}'i numbrit ma enam ei mäleta, võibolla oli 
kaksteist-kakstest. \emph{Node}'i number oli
kolmkümmend viis. Eesti oli sel ajal ülemineku vabariik. 
Registreeritud postiaadress andis võimaluse foorumites 
kaasa rääkida. Eestis oli kümmekond gruppi, kus käis jutt erinevatel 
teemadel. Mõnes mõttes oli elu selline, nagu oleme täna 
harjunud, kuigi natuke teistsuguste tehniliste vahenditega. Post oli aeglasem 
ja 
saabus paar korda 
päevas, mitte reaalajas. Ei olnud nii, et kirjutan kirja ja see läheb kohe 
kõigile laiali. Samas täitis see kõik need ülesanded, millega täna tegeleme, 
ära. Nii et kaheksakümnendate lõpus, üheksakümnendate alguses oli see 
\enquote{ökosüsteem}, millega täna oleme harjunud, täiesti olemas ning 
väike käputäis inimesi Eestis omasid selle kasutamise privileegi. 

\question{Kas see väike käputäis olid pigem entusiastid, akadeemiline 
seltskond või kes?}

FidoNeti ökosüsteem koosnes sada 
protsenti entusiastidest. Akadeemilised inimesed läksid ärisse, panid püsti 
esimesed arvutifirmad ja üritasid raha teha. 

\question{Kas eksisteeris ka mõningane spetsialiseerumine, et siit saab 
tarkvara ja seal on huvitavaid jutte-raamatuid?}

BBSidel väike spetsialiseerumine oli, aga mitte eriti suur. Eks 
enam-vähem kõik proovisid endale kõhu alla korjata, mida vähegi said. 
See oli aeg, kus tekkisid esimesed suuremad kõvakettad. 
Lühikest aega valitses olukord, kus tarkvara 
oli vähem kui ruumi. Ruumi mõiste oli ka muidugi tollal huvitav. Kõige 
rohkem ruumi võtsid Sierra\sidenote{1979. aastal 
asutatud Sierra Entertainment (varem On-Line Systems ja Sierra On-Line) 
disainis paljud toonased hittmängud. Eriti populaarsed olid 
seiklusmängude sarjad \enquote{King's Quest}, \enquote{Space Quest} ja \enquote{Leisure 
Suit Larry.}\index{Larry (mängusari)}} mängud, mis olid flopiketaste peal. Neist suuremad, Space 
Questid\index{Space Quest} ja muud, tulid viie-kuue flopi 
kaupa. Mäletan, kuidas arutasime Eeroga\index[ppl]{Tohver, Eero}, et 
kui oleks võimalik panna kokku oma unelmate masin, siis kui suur kõvaketas sel 
peaks olema. Jõudsime järeldusele, et kui oleks umbes kaheksakümmend megabaiti, 
siis ilmselt jätkuks eluajaks, sinna saaks kõik mängud ja
tööasjad peale panna ning umbes pool jääks veel üle.

\question{Sierra oli omaette fenomen, seda mängiti palju. Kas keegi
seda müüs ka?}

Küsime laiemalt, kas Eestis üldse keegi tol ajal tarkvara müüs. 
Äritarkvara, nagu Novell, oli võimalik osta. Teoreetiliselt oli 
Windowsi või DESQview'd\index{DESQview}\sidenote{DESQview oli kaheksakümnendate 
lõpus ja üheksakümnendate algul populaarne tekstipõhine mitme{\-}tegumiline 
keskkond, mis toimis DOSi peal ja võimaldas korraga mitut programmi eri akendes 
käimas hoida.} kindlasti kuskilt võimalik osta. Aga peale Novelli serveri ja 
DataFlexi 
litsentside ei mäleta ma, et oleks üheksakümnendatel kellelgi 
legaalset tarkvara näinud. 

\question{Tuleme tagasi BBSinduse juurde. Kas selle sisu hulk, 
mida enda kõhu alla õnnestus kokku kuhjata, oli ka staatuse 
sümbol?}

Ma ei oska öelda, oskan ainult enda BBSide kohta rääkida. \mbox{Mina} 
korjasin kokku kõik, mida kätte sain, ja pakendasin ringi. See 
oli selline kultuuriküsimus, et tarkvara skaneeriti viiruste vastu 
kõige värskema skanneriga, mis parasjagu käeulatuses oli, ja see käis 
muidugi automaatselt. Siis lisasid arhiivi väikese faili, 
mis sisaldas sinu \emph{header}'it -- väikest 
failijuppi, kus oli graafiliselt (või tollal pseudo{\-}graafiliselt) sinu 
logo sisse punnitatud. Ja siis panid selle välja ja oma faililisti nupukese, 
millega tegu. 

See oli nagu \emph{basic housekeeping}. Kui sinu fail läks 
järgmisse BBSi, siis see viskas sinu logo välja ja pani enda oma 
asemele, \emph{tag}'iti ära nagu grafitiga, et see on 
minu käest tulnud asi. Vähemalt mul oli küll tunne, et välja läks 
kõik, mida olid ise endale mingil põhjusel hankinud. Mitte küll nii, et 
tõmbasid öösel HNSi\index{HNS} tühjaks ja 
panid enda lehekülje peale välja, küll aga mõned asjad, mille olid kätte 
saanud. Duplikaate ei olnud väga palju üllataval kombel.

\question{Tahtsingi küsida, et sedasi oleks pidanud ühel hetkel ju kõigil 
kõik olemas olema, aga seda siis ei tekkinud?}

Seda ei tekkinud. Kuna BBSid olid väga stabiilselt üleval, siis enda jaoks 
vajalikud asjad tõmmati
ära ja pandi omakorda enda juurde üles. Mõttetut \emph{leach}'imist ja püüet 
iga hinna eest oma failiandmebaas kõige suuremaks saada 
väga ei olnud. 

\question{Too mõni näide, mis laadi asjad sulle toona huvi pakkusid.}

Olin siis juba vihane \emph{nerd}, minu spetsialiteet oli 
programmeerimismaterjalid ja -vahendid, käsiraamatud ja
tööriistakesed. 
Kahjuks mul ei ole seda vana faililisti alles, sest kui ma Skriiningust ära 
läksin, lendas see vana SCSI-ketas, 
mille peal BBS jooksis, õhku. \emph{Backup}'i sellest ei olnud ja 
kogu FidoNeti \emph{node} koos failibaasiga läks hingusele.

Ma ise seda järgmisse kohta kaasa ei võtnud, sest läksin Skriiningust panka, 
kus 
olid ees sellised kõvad mehed nagu Mast\index[ppl]{Mast} ja 
Marx\index[ppl]{Marx|see{Kliimask, Margus}}\index[ppl]{Kliimask, Margus}, kes 
olid oma ökosüsteemi püsti pannud. Ühele BBSile seal rohkem ruumi ei olnud. 

\question{Mis panka sa läksid?}

Mina läksin sellesse panka, mille lõpupidu kohe 
kätte jõuab\sidenote{Intervjuu Andrusega toimus 2019. aasta novembri algul.} -- 
praegune Danske\index{Danske Pank}\index{Danske 
Pank|see{Forekspank}}, toona Forekspank\index{Forekspank|see{Eesti 
Forekspank}}. 

\question{Miks sa sinna läksid? 
Skriiningus said ju programmi kirjutada ja BBSi pidada.}

Nagu ma paljudesse kohtadesse olen läinud -- sellepärast et kutsuti. Ja 
kuna parasjagu jooksis Eestis teleseriaal Capital City, mis 
näitas panganduselu väga glamuurse \emph{highroller}'ina, siis mulle tundus, 
et mina tahan ka nii elada. Tuleb tunnistada, et üheksakümnendate panganduses  
ei pidanud väga pettuma, elu oli täitsa lill. Päris nii nagu 
teleseriaalis \enquote{Pank} elu meie majas küll ei käinud. 
Päris hulle pidusid sai peetud, aga et keegi oleks kokaiinise  
ninaga ringi käinud, seda mina ei tea. Meie kandis oli kokaiin täiesti 
tundmatu või ehk tehti seda salaja, mina küll
narkootikumidega pidusid ei näinud.

\question{Kas mäletan õigesti, et tollal tõmbasite panka 
püsiühenduse\sidenote{Enamik varasest internetiühendusest Eestis toimis kuhugi 
sisse helistades. See tähendas, et pidev side puudus ja side 
kvaliteet sõltus suuresti analoogtehnoloogial põhinevatest 
telefonikeskjaamadest. 
Püsiühenduseks kutsuti seda, kui asutusest jooksis füüsiline kaabel interneti 
külge 
ja kaabli olemasolu oli IT-inimeste unelmates kesksel kohal.} sisse?}

Püsiühenduse tõmbasime sisse väga konkreetsel päeval. 
Modemitega oli n-ö poolpüsiühendus juba pikemat aega olemas.  
Forekspank asus Rävala puiesteel, nagu 
juhtumisi ka KBFI\index{KBFI}\sidenote{Keemilise ja Bioloogilise 
Füüsika Instituut\index{Keemilise ja Bioloogilise Füüsika Instituut|see{KBFI}}
 (KBFI). 1979. aastal Endel Lippmaa\index[ppl]{Lippmaa, Endel} 
loodud teadusasutus, tuntud ka kui \enquote{Lippmaa instituut}. Just 
Lippmaade perekonna aktiivse ja laiahaardelise tegutsemise tõttu mängis 
instituut rolli paljudes toonastes olulistes protsessides (sh kohaliku 
interneti arengus).}. Baumaniga\index[ppl]{Bauman, Andres} 
oli läbi räägitud, kuidas internetti saab, ja meil oli suhteliselt 
rivitu ligipääs. Samas tundus ühel hetkel, et see võiks ikka päriselt 
permanentne olla. Võtsime Mastiga\index[ppl]{Mast} kaablirulli ja 
hakkasime üle Rävala puiestee katuste KBFI poole liikuma. Tähelepanuväärne oli, 
et see juhtus päeval, mil Eestit väisas esimest korda paavst.
\sidenote{Paavst Johannes Paulus II külastas Tallinna 10. septembril 1993.} 
Kõik katused olid snaipreid täis, kehtestati tohutu 
\emph{lockdown}, et keegi paavsti käigu pealt ära ei tapaks.  
Seletasime kõigile, et meil on vaja kaablit vedada ja paneme interneti 
püsiühendust. See oli maagiline valem, mis võimaldas ligipääsu 
kõikidele kesklinna katustele, ilma et keegi oleks midagi küsinud. Me küll 
otseselt snaiperitega samale katusele ei sattunud. Natukene tuli häkkida ka, 
et ühest koodlukust läbi minna, aga see ei olnud suur takistus. 

\question{Toona oli maailm järelikult teistsugune. Internet ei 
olnud veel kommertsiaalne, vaid pigem kogukondlik nähtus.}

Selle eest vististi keegi maksis ka kellelegi midagi, aga kui palju, 
seda jällegi ei mäleta. Eks see käis paljuski inimsuhete baasil. 
Kuna me tundsime Andres Baumani\index[ppl]{Bauman, Andres}, siis kuidas raha
seal tegelikult liikus, seda ma ei tea. Mast\index[ppl]{Mast} ajas seda asja. 
Millegipärast ma arvan, et maksime KBFI-le midagi. 
Tegelikult oli meil alates
üheksakümne viiendast aastast
Forekspangas\index{Forekspank} infotehnoloogiliselt selline elu nagu 
tänapäeval. 
Suhteliselt samal ajal tuli Mosaici\index{Mosaic}\sidenote{NCSA Mosaic oli üks 
esimesi internetibrausereid ja mängis WWW populariseerimisel olulist rolli. 
Sama meeskond lõi hiljem Netscape\index{Netscape} Navigatori, mis oli  
Firefoxi eelkäija.} brauser, hakkas veeb arenema ja 
tekkisid meile kõigile e-posti aadressid (need olid 
küll juba pisut varem KBFI kaudu korraks olnud, aga siis tekkisid need 
meie oma foreks.ee domeeni külge). Kogu see ökosüsteem, miinus Facebook, oli 
meil siis juba olemas. 

Tollal me ka täitsa tõsimeeli arutasime, 
et KBFI ühendus on ikkagi nii aeglane, et ehk peaks kogu 
veebi kohalikku serverisse kopeerima. Ja 
kuna see mahuks tõenäoliselt ühele DVD-le ära, siis ehk peaks tegema 
äri ja hakkama müüma internetiga DVDd. 

\question{Ka teistest intervjuudest käib läbi, et toonane maailm põhines 
suuresti 
inimsuhetel. Ometigi ei hakka inimesed arvutitega tegelema, kuna neile 
meeldib tegeleda inimestega. Samas tunduvad Eesti arvutiinimesed küllaltki 
suhtealtid ja -osavad. Miks see nii on?}

Kui inimesel on arvutihuvi, siis on ta
terve keskkooli ja pool ülikooliaega olnud sotsiopaat ning tal ei ole eriti 
olnud kellegagi millestki rääkida. Ja ühel hetkel leiab ta üles omasugused, 
samasuguste huvidega. Puhas \emph{nerd}'i ja nohiku käitumine, eks ole! 
Kui panna nohikud kõik ühte tuppa kinni, siis nad leiavad 
üksteist ja kõigil on järsku lõbus, sest kõik lõpuks ometi naeravad samade 
naljade üle. Pidudega on sama lugu. Kõige karmimad peod, kus ma 
olen osalenud, on ikkagi olnud inimestel, kelle igapäevatöö on kaunikesti 
\emph{boring}. Ma ei taha anda hinnangut teatud inimgruppidele, aga kui näiteks
 raamatupidajad ja andmesisestajad käima lähevad, siis see on 
ikka täiesti teine tase. Keskmised lõbusad inimesed on lõbusad 
kogu aeg. Aga kui nohkarid lõpuks lõbusaks muutuvad, siis juhtub asju.

Nii et see ökosüsteem toimis tänu sellele, et inimestel oli hea meel üksteist 
leida. 
Algul oli neid alla saja, 
võibolla isegi alla viie{\-}kümne inimese. Tegu oli uue 
laine arvutitegelastega, kelle seast suur osa meie tänasest 
\emph{start-up}-ettevõtlusest 
ongi välja kasvanud. Tänu tihedale suhtlusele hakkasid 
toimuma ka legendaarsed BBSummeri\index{BBSummer}-nimelised üritused. 

\question{Räägi lõpetuseks, mida sa praegu teed.}

See on võibolla masendav tõdemus, aga elu pole mind sellelt kursilt
kaugemale ega kuhugi mujale viinud. Laias laastus 
tegelen täna täpselt sama asjaga, millega kakskümmend viis aastat 
tagasi. Olen pendeldanud elektroonika ja tarkvara vahel, 
olnud mitme firma CTO, asutanud firmasid ja neid kihva keeranud, töötanud 
teiste juures ja endale. Ja kui keegi küsib, millega ma tegelen, 
siis tavaliselt ütlen, et annan masinatele hinge. 

\question{See on ilus ütlemine ja läheb kokku küsimusega, mis jäi enne 
küsimata. Tavaliselt inimesed tegelevad kas riist- või tarkvaraga, aga sinul 
tundub olevat üks jalg ühes ja teine teises?}

Mõeldes oma elu peale, siis ma muidugi tahaksin, et tarkvara oleks mu tõmmanud 
endasse. See on mõnes mõttes nii palju lihtsam ala. \mbox{Vigu} on palju 
lihtsam parandada ja katkiseid asju ei tule peaaegu üldse ära visata. 
Kettaruum ei maksa täna eriti palju erinevalt elektroonika valmistamisest ja
utiliseerimisest.

Mul on kuidagi juhtunud niimoodi, et kui panen 
tule vilkuma ja näen, kuidas minu tehtu manifesteerub päris asjades, 
siis mul läheb tuju paremaks. Mul tuleb elektroonika disain välja ka. Kuna ma 
olen ikkagi ka
programmeerija, siis olen sattunud sinna omamoodi side{\-}meheks. Ma suudan tõlkida 
riistvara tarkvara jaoks ja vastupidi. Selle konkreetne töönimetus on 
\emph{embedded engineering}. Vaadates, mis meil täna koolidest 
saabub, siis on see täiesti väljasurev kunst. Neid tegelasi, kes suudavad nii 
riistvara valmistada kui ka sellele tarkvara peale kirjutada, 
nimetatakse mehhatroonikuteks või kelleks iganes, aga fakt on see, et nende 
juurdekasv on järsult pidurdunud ja varem või hiljem hakkab see 
probleemiks muutuma. Tõsi küll, ka töömeetodid muutuvad. Me kasutame täna
töövahendeid, mis annavad näiteks tarkvaratiimile parema 
ettekujutuse riistvarast kui vanasti. Kirjeldused ja 
\emph{markup language}'id, millega seda tehakse, on paremad. Masinale hinge 
andmine tähendab seda, et kui sa näiteks lülitad oma pesumasina sisse, siis on 
oluline, mida see oskab või ei oska sinu 
heaks teha. Hea kasutajakogemus tuleb sellest, kui hästi raua ja tarkvara 
kooslus 
on välja mõeldud. 

\question{Sa ütlesid enne, et sa oled ka CTOna toimetanud. Järelikult tuleb
kolmas element juurde -- sa pead suutma selle kõik ka äriks tõlkida.}

CTO ametit on kaht sorti. Tavaliselt väikestes firmades tähendab 
CTO olek seda, et koosolekule on vaja kedagi kaasa võtta, ja kuidas sa 
ütled, et ta on mul programmeerija, eks. Sa pead talle andma 
visiitkaardi, millega ta näeb presentaabel välja. Väikefirma CTO 
teeb kõike, millel on tehnika maitse 
küljes. Suurema firma CTO tähendab, et ta ongi CTO. Tänases 
\emph{start-up}-maailmas on \emph{customer fit} ja \emph{market fit} kõva 
teema. 
Vanasti sellega väga ei tegeletud, aga nüüd, kus on tohutu kuhi 
investorite raha põlema pandud, ilma et sellest oleks isegi sooja saadud, on 
hakatud rääkima sellest, et toodetut peaks kellelegi päriselt ka
tarvis olema. See paistab olevat uus asi, viimase paari aasta 
paradigma. Kaks-kolm aastat tagasi hakkas Silicon Valleys 
pihta see kultuur, et laste kätte ei taheta raha enam hästi anda. Ehk 
nende kaheksateistaastaste imeettevõtjate aeg, kes suudavad väga suure kuhja 
raha 
korraga põlema panna, nii et sooja ei saa, on läbi saanud. Nüüd on selgunud 
innovatiivne lähenemine, et toodet peab 
kellelegi tarvis olema. See tähendab, et projektidele on erakordselt raske raha 
saada, sest kõik 
on järsku pirtsakas muutunud ja nõudnud, kust raha tagasi tuleb. 

\question{See läheb ju kokku sinu kunagise ettevõtte uksest sisse minekuga: 
seal pidid ka kohe kasulik olema ja ei tohtinud asju tuksi 
keerata.}

Kasumlikkus on tegelikult õudselt valus teema. Riistvaraga on  
asi selles mõttes selgem, et riistvara ei skaleeru, kui keegi seda ei osta. Sa 
ei 
saa valmistada sedasama \emph{recorder}'it, millega me siin praegu salvestame, 
miljon tükki, kui keegi ei osta. Sa lähed 
pankrotti. Tarkvara tiražeerimine ei maksa aga midagi. Ja täpselt samamoodi 
võib  
juhtuda, et tarkvara, millest mitte kellelegi mitte pennigi ei teki, on 
tegelikult väga kasulik. Seega kasulikkus ja ärimudel ei tähenda veel mitte 
midagi. Dotcomi- ja igasugu tarkusemullidega kipub tavaliselt juhtuma, et väga 
raske on tõmmata piiri selle vahel, kus asi ei teeni 
raha sellepärast, et väga head mõtet ei ole veel õpitud rahaks  
tegema, ja nende asjade vahel, mis ongi täiesti mõttetud. 
Seetõttu on väga palju tegelasi, kes suudavad maha müüa täiesti kasutu idee, 
öeldes, et tegu ongi monetariseerimiseelse faasiga ja see ei peagi midagi 
tootma. 
Unustades ära, et tegu on ühtlasi täieliku kräpiga. 
Viimasel ajal on tekkinud paar niisugust suuremat skandaali, näiteks
õnnetu Theranose \emph{case}, kus suudetakse endale nii veenvalt 
valetada. Terve ökosüsteem on üles ehitatud väga kasulikest 
asjadest, mille ainus viga on see, et fundamentaalne eeldus, millele süsteem 
rajati, oli täiesti vale. 

\question{Nii et selle kahekümne viie aastaga ei ole maailm väga muutunud,
aga toimib siiski natuke teisti?}

Üks asi on oluliselt erinev. Tollal valmistati tarkvara kahel põhjusel. 
Esiteks oli seda tarvis, mis tähendas tugevat kliendipoolset 
tõmmet. Teiseks taheti, et midagi sellist eksisteeriks 
maailmas, mis tähendab, et võeti lihtsalt kätte ja kirjutati tarkvara kas enda 
või teiste rõõmuks ning lasti maailma. Hästi palju väikesi ja 
kasulikke 
utiliite oli ju tegelikult kirjutatud kellelgi 
enda jaoks, siis pakendatud ja laiali saadetud. Eestis seda kontseptsiooni 
polnud, et teha tarkvaraga 
raha: kirjutada mõni vidin ja küsida selle eest tasu. \emph{Corporate} maailmas 
tollal 
küll juba osteti-müüdi igasuguseid raamatupidamissüsteeme väga 
edukalt ja see kõik töötas. Mujal maailmas tegeleti utiliitide 
pealt raha teenimisega ka väikest viisi. Eestis üldse mitte. Tänapäeval on 
tarkvara tootmine läinud niimoodi, et kellelgi tuleb mõni väga
hea idee ja ta tahab sellest teha raha tootmise masina. Asi on vastupidine: 
mitte 
vajadus-, vaid unistuspõhine. Nagu me 
aeg-ajalt Ivar Zaransiga\index[ppl]{Zarans, Ivar} naerame, et kui vanasti 
otsiti probleemidele lahendust, siis tänapäeva 
maailmas otsitakse probleeme neid vajavatele lahendustele. See on viimase 
kahekümne viie aasta jooksul kõige suurem paradigma muutus.

\chapter{Sergei Anikin}
%!TEX TS-program = arara
% arara: myindex

\index[ppl]{Anikin, Sergei}
\textbf{\enquote{Kuidas sina arvutite juurde said?}}

Ja see oli päris huvitav lugu. Ma olin, võib öelda, \emph{entitled} mu isa oli
elektroonikainsener töötas Kalinini tehases\index{Kalinini
tehas}\sidenote{Algselt Balti Raudtee Peatehased, mis ehitati 1870. aastal ja
mis aastatel 1902 kuni 1903 seal töötanud Nõukogude riigitegelase järgi 1940.
aastast alates M.I. Kalinini nime kandis. 2007. aastast alates asub samal
territooriumil ja osalt samades hoonetes Telliskivi Loomelinnak restoranide,
kohvikute, kontorite ja loominguliste ruumidega}, see koht, kus meil nüüd on
see kõige popim koht noorte seas. Seesama Kalamaja ja see Lendav Taldrik.
Tegelikult ma olen seal lapsena käinud koos isaga, seal oli valvur, valvurist
pidi läbi minema, et sinna territooriumile saada. Nad tegid rongidele
elektrimootoreid ja jõuelektroonikat, minu isa projekteeris neid. Aga
hobi korras ta on teinud igasugust raadiotehnikat ja mina ise olen proovinud
mingit väikest raadiot kokku panna. Kuigi mina olin täiesti võhik, selles osas kuigi
käisin raadiotehnika mingisuguses ringis.

Aga minu esimene arvuti sai siis minu isa poolt kokku pandud.

\textbf{\enquote{Aga kust ta jupid sai?}}

Isal  oli selline ajakiri Vene ajakiri nagu 
\begin{russian}Радио\end{russian}\sidenote{Igakuine populaarteaduslik 
raadiotehnika ajakiri, mida andsid välja Nõukogude Liidu Siseministeerium ja 
DOSAAF (\begin{russian}Добровольное общество содействия армии, авиации и флоту 
России\end{russian} - Vabatahtlik Venemaa armee, lennunduse ja mereväe 
abistamise selts). Ilmus eri nimede all alates 1925. aastast, 1975. aastal oli 
ajakirja tiraažiks 850 000 eksemplari}. Ja siis aastal 1986 avaldati seal 
kõigepealt arvuti skeemid ja siis, kuidas seda kokku panna. See oli 
Nõukogudemaal välja töötatud arvuti, aga skeemid nad võtsid selle ZX Spectrumi 
pealt\sidenote{Tegemist on arvutiga 
\begin{russian}Радио-86РК\end{russian}\index{Arvutid!Radio-86RK}, mis oli üks 
edukamaid  koduseks kasutamiseks mõeldud Nõukogude arvuteid. Kuigi Nõukogudemaal 
kopeeriti ZX Spectrumit usinasti, oli selle arvuti puhul siiski väidetavasti 
tegu originaalse disainiga, autoriteks Dmitri Gorshkov, Yuri Ozerov, Gennady 
Zelenko ja Sergey Popov (Stachniak, Zbigniew. "Red clones: The soviet computer 
hobby movement of the 1980s." IEEE Annals of the History of Computing 37, no. 1 
(2015): 12-23.)}. Isa siis kõigepealt korjas need komponendid kokku, ise 
joonistas plaadi ja kuna tal oli juurdepääs siis tehases tegi plaadi valmis ja 
pani selle kokku. Ma mäletan, et tal läks ikka paar kuud, enne kui ta kõik need 
vigased kohad seal ostsilloskoobiga välja juuris. Siis pani käima. See käis 
teleka  külge,  telekas oli monitori asemel. See oli mustvalge telekas, 
värvitelekat  meil peres ei olnud.  Aga ega ma sellega  midagi väga teha ei 
saanud, sest tal ei olnud isegi opsüsteemi. Tollel arvutil oli \emph{interface} 
kassettmakiga, aga meil ei olnud ka kassette, mille pealt laadida seda 
opsüsteemi. Selles samas ajakirjas oli baitkoodis opsüsteemi kood trükitud. 
Kakskümmend lehekülge bait baidi haaval. See oli talvel. Pimedad õhtud, ma 
põhimõtteliselt istusin kaks nädalat ja trükkisin need kõik koodid sisse.

\textbf{\enquote{Miks sa tegid seda? Normaalne laps ju ei toksi niimoodi 
pimedatel õhtutel baitkoodi?}}

Kas sellele on eellugu. Isa sõber tõi mulle umbes aasta enne seda lasteraamatu  
programmeerimisest. Seal  mingisugused  tegelased siis õppisid programmeerima 
BASICus\index{Keeled!BASIC}. Lugesin selle raamatu läbi, sain aru, kuidas 
programmi kirjutada, kirjutasin BASICus umbes kümnerealise programmi, mis 
midagi arvutas. Ja siis kompileerimise osas, me ei saanud ju paberi peal
kompileerida, ma näitasin sellele isa sõbrale ja siis ta vaatas, kontrollis,  
ja ütles, et see töötab küll.

Aga noh, see programm oli olemas, aga ma ei saanud proovida seda, et mul oli 
vaja arvuti tööle panna. Siis ma trükkisin need baitkoodid sisse ja  lõpuks 
sain oma programmi umbes aasta pärast sisse trükkida.

See sissetoksimine käis plokkide kaupa. Seal oli, ma ei mäleta, kui suur, aga 
umbes poole leheküljeline plokk, millel oli kontrollkood. Ma sain seda 
kontrollkoodi valideerida, kui see klappis, siis ma salvestasin selle makile. 
Kui ei klappinud, siis ma pidin viga otsima. Põhimõtteliselt ma pidin algusest 
peale selle ploki sisse toksima, sest selle vea leidmine oli väga-väga 
keeruline. Aga, aga ma arvan, et juba sellest ajast, mul tekkis esiteks 
kannatus ja teiseks tähelepanu detailidele. Selle baitkoodide sissetoksimisega 
ma lõpuks sain aru, et mul on hästi oluline kõik need õigesti ja õiges 
järjekorras sisse toksida sest ümber tegemine oli nii piinlik.


\textbf{\enquote{Sulle tehti väiksest peale selgeks, et sa võid küll üle jala 
lasta, aga siis sa ise toksid neid samu asju kolm korda}}

Ja, aga loomulikult, enamik aega mis arvutiga sai veedetud, olid  mängud. Tol 
ajal alguses arvutis olid need tavapärased madu ja mingisugune tennis. Sai neid 
mängitud. Siis isale meeldis arvuteid kokku panna ja ta on pannud ka sellesama 
ZX Spectrumi\index{Arvutid!ZX Spectrum}, isegi mitu tükki, kokku. Tegelesime ka 
selle väliskorpusega. Tol ajal vaata, Eestis on kuiv õhk ju talveti ja siis 
meil olid plastmassist õhuniisutajad, mis käisid radika peale. Sellest sai väga 
hea korpusse sellele arvutile. Ta oli õige kujuga, sinna sai sisse lõigata 
selle klaveri, toiteplokk, plaat kõik, mida vaja. Makk oli eraldi.

\textbf{\enquote{Miks sulle see elektroonika osa huvi ei pakkunud?}}

Ega mul ei olnud arvutite vastu mingisugust suurt kirge, siiamaani ei ole
tegelikult. Minu arust see on ikkagi vahend. Tänapäeval on ju
teada, et need, kes arvutitega tegelevate teenivad päris korralikult raha,
onju. Tol ajal see oli ka mingis mõttes staatuse küsimus, et sul peres oli
arvuti. Kui paljudel peredes
oli arvutid? Alles mitmed aastad hiljem tekkisid  need arvutiklubid või
arvutiga  mängida kohad. Aga mul oli kodus selline. Me ei olnud
jõukas pere, meile polnud raha et osta niisugust asja. 

Arvuti on jah, pigem vahend. Ta on meeldiv hobi ka, aga ma ei ole selline, et
see oleks ainuke hobi. Mingi aeg ma üldse ei tegelenud arvutitega, mulle see
mängimine enam kirge ei pakkunud ja programmeerida lihtsalt enda jaoks ei
tundunud väga huvitav. Aga mul oli üks sõber, me mängisime koos. Ja siis ta
mainis, et \enquote{hoo, et, et ma nüüd käin arvutiklubis. Ja seal õpime programmeerima,
aga mina muidugi enamasti käin seal mängimas}. Ja siis ma mõtlesin, et tema ju
tegelikult ei oskagi midagi. Et mina ju oskan ja peaks koos temaga minema. Sa
ilmselt oled rääkinud paljude inimeste Eesti kogukonnast aga, aga mina sattusin
siis Vene kogukonda. Selle arvutiklubi nimi oli Interface\index{Arvutiklubi!Interface}.

\textbf{\enquote{Aga kes seda klubi pidas ja kuskohas?}}

Selle vedaja oli, ma mäletan, naisterahvas. Ta töötas vist Bioloogia
Instituudis siin Mustamäe teel ja vedas laste arvutiklubi. Ta nii nimi oli
Nina Botina\index[ppl]{Botina, Nina}. Me käisime seal
Reaalkoolis\index{Koolid!Tallinna 2. Keskkool}\index{Koolid!Reaalkool|see{Tallinna 2. Keskkool}},
seal olid arvutiklassid, tundides.

\textbf{\enquote{Mis koolis sa ise käisid?}}

See on kool number kakskümmend kuus\index{Koolid!Tallinna 26. Keskkool}. 
Viimases klassis ma läksin Tõnismäe Reaalkooli\index{Koolid!Tõnismäe Reaalkool} 
kus oli väga tugev matemaatika. Ja tegelikult seesama Nina Botina surus mind ja 
veel ühte klassiõde, et me läheksime teise kooli, et lõpetaksime selle 
matemaatika klassi. Tema pärast me läksime sinna kool ja seal oli hästi palju  
tuttavaid sellest samast arvutiklubist.

Ja hiljem sellest arvutiklubist on kasvanud venekeelne tehnikakool või
arvutitehnikakool, mis oli Erika tänaval. 

\textbf{\enquote{Ma teadsin, et Tartu ja Tallinna vahel on erinevus. Aga 
selgub, et ka Tallinna sees on kaks täiesti isesugust Tallinna?}}

See on huvitav jah. Ja seal oli ka niimoodi, et minu huvi arvutite vastu 
vaheldus. Üks aasta ma olin seal klubis aga siis, kui ma läksin uude kooli, mul 
ei olnud aega, et  sellega tegeleda. Aga siis Nina kutsus, et kuule,  mul ei 
jätku instruktoreid. Tule, mul on uued grupid, tule ja aita 
arvutiklassis. Siis kuidagi  tekitas uuesti huvi. Kui ma kooli lõpetasin ja  
ülikooli läksin majandust õppima\index{Tallinna 
Tehnikaülikool!Majandusteaduskond}. Aga seal jälle  esimese aasta lõpus tekkis 
võimalus spetsialiseeruda majanduslikku andmetöötlusesse. See oli hästi pisike 
grupp, mingi seitse inimest. Kui kõik, kes olid majanduses, õppisid majanduse 
aineid, siis meie enamik meie tunde olid  arvutitehnika gruppidega.

Ja see meie grupp oli eestikeelne. Ma läksin venekeelsesse majandusteaduskonda, 
aga see grupp oli eestikeelne. Aga
see oli huvitav jällegi, et me ei pidanud õppima arvutitehnika baasaineid. 
Esimese aasta arvutitehnikas nad õppisid füüsikat-keemiat, kõiki neid üsna 
keerulisi ained. Ma olen kuulnud õudseid lugusid, kuidas inimesed ülikooli 
lõpuni ei saanud neid tehtud. Aga meie õppisime mikro ja makroökonoomikat, 
inglise keelt. Ja alates teises aastast hakkasime koos arvutitega õppima. Ja ei 
olnud erilist jõudluse vahet, tundus.

Aga ma jällegi mõtlen, et see, kus ma praegu olen, ilmselt on  ka  
põhjustatud sellest, et ma ei läinud väga süvitsi  arvutitehnikasse, vaid pigem 
alati oli arvuti mul  vahend mingi probleemi lahendamiseks.

\textbf{\enquote{Sa mainisid, et sul matemaatika tuli välja. Kas sa kuskil 
olümpiaadidel ka käisid?}}

Käisin, aga ma olin niisugune keskmine. See nii palju sõltub õpetajast, onju. 
Ma mäletan, meil oli kas viies või seitsmes\sidenote{Selle põlvkonna inimestel, 
nii vene- kui eesti koolides, jäi üks klassi vahele, sest koolid läksid 
kaheksakümnendate teisel poolel üle aasta võrra pikemale õppele}, minu meelest 
seitsmes klass, kus hakkas juba geomeetria ja muud sellised asjad. Ja siis mul 
kuidagi klikkis, et iga teoreemi kohta, mida meile räägiti, mul tekkis teine 
viis, kuidas seda tõestada. Ma kuidagi sain nagu aru, et ei ole alati ainult 
ühtemoodi, saab teistmoodi ka. Ja siis jällegi see klikib õpetajaga. Kui 
õpetaja näeb, et õpilane mõtleb, siis ta pöörab rohkem võib-olla tähelepanu 
inimesele. Aga noh, siis tema läks ära ja järgmised õpetajad ei olnud väga head.

Siis meil oli üks väga hea füüsikaõpetaja, tal oli hästi palju kontrolltöid. 
Tema juures ma õppisin seda, et üldse ei pea neid valemeid meelde jätma. Piisab 
sellest, kui sa oskad neid rakendada. Loomulikult spikerdamine ei olnud 
lubatud, aga mul ikkagi need valemid olid spikrina vihiku tagakaanel. Sa pead 
aru saama probleemist, pead aru saama, mis vahendeid kasutada selle probleemi 
lahendamiseks. Ja see õpetaja vaatas läbi sõrmede nende valemite peale, sest 
kui sa  probleemist aru ei saa, siis füüsikas lihtsalt valemid ei aita. 

Ja kui me läksime sinna Tõnismäe Reaalkooli, oli seal legendaarne 
matemaatikaõpetaja Mihhail Vassiljevitš\index[ppl]{Vassiljevitš, Mihhail}, 
siiamaani õpetab. See, kuidas inimene, õpetaja on ju autoriteet, kohtleb 
inimesi! Selles mata klassis, seal selgelt olid kolm või neli  tippõpilast, kes 
võitsid kõik riiklikud olümpiaadid  käisid maailmaolümpiaadidel. Loomulikult ta 
tegeles nendega, aga ta tegeles ka kogu ülejäänud rahvaga. Seal olid ka need, 
kes ei saanud väga aru aga tema juures need nende tase tõusis. Ta oskas 
selgitada ka keerulisi asju nii lihtsasti asju, et kogu klass  põhimõtteliselt 
oli paar taset teistest koolidest üle. Lihtsalt see, et sa olid seal  
keskkonnas juba tõstis sinul taset nii kõvasti.


\textbf{\enquote{Jällegi tuleb välja, et matemaatika tunnis õpiti lisaks 
matemaatikale ka suhtumist ja just see viimane on aastate järel meeles}}

No see olümpiaadide küsimus jällegi, Mina ei saanud seal mingeid kohti. Aga 
klassis, mis oli meist aasta vanem,
 oli selline lugu, et umbes kümme inimest läksid keemiaolümpiaadile, kümme 
inimest läksid matemaatika ja kümme läks füüsika olümpiaadile. Kõik need 
riiklikud olümpiaadid olid ju Tartus. Põhimõtteliselt terve klass läks 
olümpiaadile, aga erinevatel aladel. Ja kuna nad olid juba seal kohal, siis 
neil oli lubatud  ka teiste ainete olümpiaadides osaleda. Mille tulemusena nad 
kõikidel aladel, isegi need, kes ei kvalifitseerunud alguses, said enam-vähem 
kõik esikümnesse kõikidel aladel. Saad aru, see oli selline nii võimas klass, 
täiesti hämmastav.

\textbf{\enquote{Miks sa majandust läksid õppima?}}

Sest mu vanemad ütlesid, et meil on peres juba kaks inseneri olemas, ema oli 
soojustehnik. Eks ma mõtlesin minna kuskile mujale ka õppima, aga  kodu juures 
on palju lihtsam. 

\textbf{\enquote{Kas sul oli mingi ettekujutus sellest ka, mis sa pärast oma 
haridusega ette võtta tahad?}}

Ega ega mul väga ei olnud mingit ettekujutust. Ma arvan, et mul lihtsalt ei 
olnudki mingit plaani. Ma tahtsin lihtsalt näha, et mis  see majandus siis 
õigupoolest on. Suvel ma korra proovisin töötamist müügiinimesena. Selgus, et 
see ei sobi mulle absoluutselt. Sest mulle ei sobinud see, et müügitöös 
üheksakümmend kaheksa protsenti inimestest ütleb \enquote{ei}. Aga mulle see ei 
sobinud, mulle ei meeldi feilida ja minu jaoks \enquote{ei} tol ajal oli feil. 
Tegelikult nüüd kui ma olen siin Pipedrive'is\index{Pipedrive} juba seitse 
aastat töötanud, ma saan aru, et see on osa protsessist, on statistika, see ei 
ole feil. Et feil on see, kus sa ei tee seda üheksakümne üheksandat korda
müüki, mis võib õnnestuda. Müük on see, et sa tead oma neid statistilisi 
numbreid ja plaanid vastavalt nendele. Mitte ees see, et kui esimene juhuslik 
inimene ütleb sulle, et mul ei ole seda teenust vaja, siis sa oled feilinud. 
Tegelikult ei ole.

\textbf{\enquote{Mida sa müüsid?}}

See oli tänavamüük, tegelikult. Tänava peal müüsime erinevaid tooteid a la 
tööriistakaste (mis läksid tegelikult päris hästi), mingisuguseid 
elektroonilisi hambaharju ja nii edasi.

\textbf{\enquote{See on ju igavene raske töö!}}

See on väga raske töö ja  see süsteem oli niimoodi, et igal hommikul me tulime 
sinna, kus oli ladu ja saime päeva kvoodi. Et pead näiteks müüma viisteist 
tööriistakasti. Ja kui sa kvoodi täitsid mingisuguse kahe nädala jooksul siis 
sa said järgmisse tiitli ja selle tiitliga sa said endale õpilasi. Ja kui viis 
õpilast said kvoodi nii-öelda täidetud mingi aja jooksul, siis said oma 
nii-öelda äri. Aga, jällegi, õppetund oli see, et see töö ei ole kindlasti minu 
jaoks. Ja ma teadsin, et kui ma lähen programmeerijaks, siis ma saan oluliselt 
rahulikuma töö eest oluliselt suuremat tasu. See sundis mind mingi hetk, umbes 
pool aastat hiljem, ütlema, et \enquote{Okei, ma lähen nüüd}. See oli ülikooli 
teise aasta poole peal umbes, siis kui ma sellesse informaatika gruppi läksin.

\textbf{\enquote{Kas sa siis juba programmeerisid mingeid tõsisemaid asju ka 
või lihtsalt loengus puutusid kokku?}}

Olen teinud kahte projekti, mis nii palju kui ma mäletan, tõi natuke raha.

Üks oli selline. Tol ajal olid hästi populaarsed need 
sat-tv\sidenote{Kaheksakümnendate lõpus ja üheksakümnendatel oli suhteliselt 
lühike periood, mille jooksul isiklik satelliidivastuvõtja oli ületamatult 
kallis, piraatlusele vaadati läbi sõrmede (õigupoolest keegi ka sealtkaudu 
eriti ei vaadanud), suuri teenusepakkujaid veel polnud aga väikestel oli juba 
võimalus tegutseda. Siis pandigi mõne kortermaja katusele satelliiditaldrik, 
hangiti piraat-kaart tasuliste kanalite jaoks, veeti üle katuste 
ümberkaudsetesse majadesse kaablid ja asuti teenust müüma} firmad. Mõnes 
väikeses rajoonis oli mingis oma kunn, kes pakkus seda sat-TV-d kuutasu eest. 
Siis oli üks tuttav, kes palus teha infosüsteemi, kus oleks kirjas, kes on 
liitunud, kes ei ole ja kui palju nad maksavad ja mis teenust nad kasutavad. 
Emal oli jällegi tööl arvuti, mille peal ma sain teha Accessi\index{Microsoft 
Access} andmebaasi ja selle peale väikese liidese. Ma ei tea, kas ta kasutas 
seda hiljem või mitte.

Teine oli veel huvitavam. Kui ma sain teada, kui palju raha ma selle töö eest 
saan, ma olin väga imestunud. Isa sõbrad, tegelesid valvesüsteemidega. Neil oli 
projekt, vanglaprojekt. Nad panid valvesüsteemi vanglasse ja neil oli 
põhimõtteliselt vaja joonistada selle vangla projekti järgi mingisugune skeem, 
kus oleks näha, kus on alarmid tööle läinud. Ta ei olnud otseselt 
programmeerimine, ta oli rohkem disain või midagi sellist. Ma pidin neid pilte 
joonistama ja siis ma sain mingisuguse kolme nädalaga selle tehtud ja see 
summa, mis ma sain, oli mu isa umbes poole aasta palk. Siis ma sain aru, et, et 
arvutitega tasub toimetada.

\textbf{\enquote{Kust sa infot said? Ega Accessis programmeerimine ei ole ka 
niisama lihtne, et muudkui otsast hakkad tegema?}}

Kusjuures Accessi koht ma ei mäletagi, eks ma vist lugesin dokumentatsiooni. 
Programmeerimist õppisin 
raamatutest. Mul oli üks üks Pascali raamat, mis õpetas objektorienteeritud 
programmeerimist, venekeelne raamat. See aitas mul mõista, just 
objektorienteeritust. Ja ülikoolis tegelikult mõned ained olid väga-väga 
kasulikud. Näiteks andmebaaside projekteerimine. Tänapäeval väga paljud 
inimesed ei oska relatsioonilist andmebaasi projekteerida ja see ja see on üks 
vajalikumaid oskusi, tegelikult, kui sa tahad isegi lihtsat süsteemi kokku 
panna. Tänapäeval lahendatakse selliseid asju tihti lihtsalt jõuga.

\textbf{\enquote{Kas sinu arvuti või reaalainete huvi juurde käis ka 
mingisugune spetsiifiline, näiteks ulme, raamatu-huvi? Vene keeles oli ju palju 
rohkem asju kättesaadavad, mina ei olnud suuteline tol ajal Strugatskeid 
originaalis lugema}}

Ega ma ei mäleta, väga oleks olnud. Raamatuid mulle meeldis lugeda, mulle 
meeldis ka ulme nii-öelda või fantastika. Aga mul ei tekkinud nagu arvutitega 
seost. Minu jaoks arvuti on nii praktiline asi kui olla saab. Eks 
Bulõtšovi\sidenote{Kir Bulõtšov (1934 --- 2003). Nõukogude 
ulmekirjanik}\index{Kir Bulõtšov} ja Strugatskeid\sidenote{Arkadi Strugatski 
(1925 --- 1991) ja Boris Strugatski (1933 --- 2012), Nõukogude ulmekirjanikud. 
Kirjutasid enamasti koos, seega tuntud kui \begin{russian}братья 
Стругацкие\end{russian} või lihtsalt Strugatskid}\index{Strugatskid} aga ka 
välismaa asju. Aga ma olen ka kõik need Barbar Conani\sidenote{Robert E. 
Howard'i poolt 1932. aastal loodud tegelane, kes on sellest ajast tembutanud 
kõikvõimalikes meediumides ajakirjadest ja raamatutest filmide ja 
videomängudeni } ja Tarzani\sidenote{Edgar Rice Burroughs' poolt 1912. aastal 
loodud tegelane, kes Nõukogude Liidus sai tuntuks kinodes näidatud 
trofeefilmide (Johnny Weissmuller'i kehastatud tegelane erines küll oluliselt 
raamatukangelasest) kaudu} raamatud läbi lugenud.

\textbf{\enquote{Mis su esimene päris programmeerija töö oli ja millal see 
oli?}}

Mul olid sõbrad seal juba ees, veebruaris 1996 ma läksin tööle Aeteci
Finantsvara ASi\index{Aeteci Finantsvara AS|see{Profit Software}} mis nüüdseks 
on Profit Software\index{Profit Software}. Nad tegid soomlastele igasugu 
finantskindlustussüsteeme. Ma mäletan, et esimese oma tööülesandega ma 
feilisin, sest mulle anti mingisuguse
valemi programmeerimine. See pidi Cs\index{Keeled!C} olema ja sellest pidi 
\emph{library} saama. No mul ei olnud teadmisi. Ma ei teadnud, kuidas Cs 
kirjutada, ma ei saanud sellest valemist aru (see oli kõrgem matemaatika). 
Ühesõnaga, sellega ma feilisin. Aga milles ma olin väga hea, oli meil Lotus 
Notes'i\index{Lotus Notes Domino} tarkvara, mida kasutati suhtlemiseks omavahel 
ja soomlastega. See oli dokumendianadmebaas tegelikult. Tal oli oma 
skriptimiskeel ja sellega siis ma kirjutasin reisikindlustuse süsteemi 
kindlustusagentidele, et nad saaksid välja arvutada, palju see reisimine maksab 
ja saaksid poliisi teha. Ja see oli internetipõhine aastal 1997. Selle Dominoga 
oli võimalik, need samad dokumendid, mida sa muidu Lotus Notes'i enda kliendiga 
nägid, oli võimalik ka veebiserveri kaudu, HTML dokumentidena näidata.

Aga see kogemus aitas mul saada Hansapanga\index{Hansapank} internetipanga 
tiimi.

\textbf{\enquote{Kuidas sa sinna sattusid?}}

See oli ka naljakas. Tegelikult Hansapanga ITs või üldse pankades ilmselgelt 
oli rohkem raha kui mingis IT-firmas. Ja siis kaks aastat töötasin Aeteci
Finantsvaras ja tundsin, et võiks väikse nii-öelda karjääri teha. Ja tegelikult 
kõikidesse pankadesse proovisin tööle saada, et seal olid vabad kohad. 
SEBs\index{SEB|see{Ühispank}} võis toona Ühispangas\index{Ühispank} ma ei 
saanud isegi vist jutule, aga ma rääkisin Hoiupangas\index{Hoiupank} Aleksei 
Bljahhiniga\index[ppl]{Bljahhin, Aleksei}. Hansas oli ka tööintervjuu, läksin 
Vilve Vene\index[ppl]{Vene, Vilve} ja Heiki Kübbariga\index[ppl]{Kübbar, 
Heiki}. Ja siis ma mõlemast pangast sain tööpakkumise umbes sama summa peale. 
Otsustasin Hansapanga kasuks, sest arvasin, et seal on võib olla natuke rohkem 
karjäärivõimalusi nii-öelda. Minu esimene tööpäev Hansapangas oli 
üheksateistkümnes jaanuar 1998. Kui ma läksin sinna fuajeesse, seal oli värske 
Äripäev, kus oli kirjas, et Hoiupank ja Hansapank ühinevad. See minu ise 
esimene tööpäev oli sama päev, kus teatati ühinemisest. Ja see määras kogu minu 
järgmist nii-öelda karjääri.


\textbf{\enquote{See tähendab, et sa pidid suhteliselt ruttu hakkama 
internetipanga asemel tegelema hoopis Light Telleri nimelise telleri 
töökohasüsteemiga?}}

Sinna läks veel natuke aega. Ma arvan, et see otsus hakata seda tegema sündis 
umbes viis-kuus kuud peale seda kui ühinemine pihta hakkas. Sest alguses ju  
ei olnud veel selge, et kumba süsteemi üldse hakatakse kasutama ja kuidas see 
otsus tehakse. Ja sellel ajal mina õppisin siis kuidas internetipanka teha.

\textbf{\enquote{See kõik on mulle üllatus. Mina sisenesin sinnasamma panka 
1999. aasta lõpus. Light Teller oli selleks hetkeks olemas ja laua taga oli 
vana kala nimega Sergei, kes selle omakäeliselt valmis oli teinud. Kui nüüd 
näppude peal arvutada, siis see tähendas, et sa tegid nullist 
täisfunktsionaalse veebipõhise telleri töökoha umbes kolme kuuga?}}

No ega ma üksi ei olnud. Aga astume sammu tagasi. See Hansapanga esimene 
internetipank oli jällegi ehitatud tehnoloogia peale, mis oli ajast ees. See 
oli Oracle\index{Oracle} mingisugune veebi veebikomponent või veebiserver kus 
sa said PL/SQLiga\index{Keeled!PL/SQL} tekitada HTML mida siis kliendid 
vaatasid. See oli omal ajal hästi lihtne, mitte mingit disaini ei olnud, sest 
vist disaineritest keegi ei teadnud tol ajal, et on selline amet nagu disainer. 
Trükidisainerid kindlasti olid aga kasutajaliidese disaineritest mitte keegi ei 
teadnud tol ajal. Ja siis kui ma tulin siis vaatasin, et, \enquote{oo milline 
ebavõrdsus!}. Et et see internetipank on ainult eesti keeles. Kohe ütlesin, et 
noh, mis ta siin teha on, ma võin näiteks teha niimoodi, et ta on 
mitmes keeles. Öeldi, et tee. Ja siis ma tegingi. Kaks nädalat tegelesin 
sellega, et võtsin kõik tekstid välja, asendasin \verb|lang| funktsiooniga, mis 
arvestas ka kasutajaprofiiliga. Ma veel õppisin sama lõik ülikoolis, ja see 
päev, kui Madis Ollissaar\index[ppl]{Ollissaar, Madis} olin ma ülikoolis. 
Logisin siis sisse, et vaadata, kas töötab. Eesti keel töötab, inglise keel 
töötab, vene keel näitab küsimärke. Ilmselt siiamaani inimesed mässavad nende 
\emph{encoding}utega, aga see oli minu esimene kokkupuude sellega, kui minu 
arvutis töötab aga serveris ei tööta.

Aga aga samal ajal hakkas juhtuma ju mitu asja nii et, toimus ühinemine, 
Aleksei Bljahhin\index[ppl]{Bljahhin, Aleksei} tegeles \emph{data} migraga. 
Tekkis probleem, et telleri programm oli kirjutatud Oracle Formsi. Igas 
kontoris oli Formsi server. Ja kõik tellerid siis kasutasid Formsi klienti mida 
serveeriti sealt serverist ja nad võtsid peaserveriga Oracle 
andmebaasiühenduse. Oracle'i litsentsid, teatavasti, maksid ühenduste arvu 
pealt. Siis kujuta ette, et Hansapangal on, no ma ei tea, mingi nelikümmend 
kontorit äkki? Nüüdseks see on juba suur number, aga Hoiupangal oli nelisada 
kontorit Eestis. Ja paljudes maakohtades ei olnud isegi nii head ühendust, et 
hoida seda pidevat ühendust baasi otsa. Ja kui nad arvutasid, kui palju need 
Oracle litsentsid maksnud oleksid, siis nad ütlesid, et võib-olla anname selle
Hoiupanga tagasi. 

Siis tegelikult oli see hästi julge otsus. Ma ei tea, kes selle nüüd võtsid 
vastu, ilmselt needsamad Vilve\index[ppl]{Vene, Vilve} ja 
Gibbs\index[ppl]{Gibbs|see{Kübbar, Heiki}}. Aga otsus oli, et teeme siis 
interneti telleri programmi ja samal ajal meile müüdi uut tehnoloogiat 
internetipanga tegemiseks, BroadVisioni\index{BroadVision} nimeline platvorm. 
BroadVisioni müügiargumendiks oli, et me saame põhimõtteliselt e-kommertsi 
platvorm,  seal sai igale kasutajale näidata personaalselt välja nägevat 
rakendust.

Aga jällegi iga kasutaja maksis. Mis tähendas, et me kunagi ei kasutanud neid 
võimalusi, kõik oli anonüümne selle süsteemi mõttes. Aga ta pakkus 
\emph{template}'mise võimalust, mis oli väga suur samm võrreldes selle Oracle 
PL/SQLiga, kus sa pidid oma HTMLi ise kokku panema. Selle peale me ehitasime. 
Aga ma arvan, et see telleri arhitektuur, kui sa praegu mõtled sellele tagasi, 
ta oli võimas, aga ta oli hästi lihtne. See võimaldas tegelikult 
funktsionaalsust hästi kiiresti hästi suures koguses toota.

\textbf{\enquote{Selle arhitektuuri kohta ma tahaks küsida. Ta ju oma olemuselt 
oli toonaseid vahendeid kasutades täpselt selline, nagu täna \emph{de facto} 
veebirakendused on. JavaScript\index{Keeled!JavaScript} jooksis brauseris ja 
tegi \emph{backend}i poole päringuid. See lahendus oli oma 20 aastat ajast ees, 
kuidas ta sündis?}}

Seal oli seesama piirang, et maakontorite ühendus oli hästi aeglane. Ehk, me
pidime optimeerima, kui palju me \emph{data}t liigutame kliendi ja serveri 
vahel. See oli üks nõudmistest, mis sundis mõtlema, et me peame palju tööd juba 
kliendis ära tegema. Aga kliendiks oli brauser ja JavaScripti versioon oli 
selline, et parimal juhul sai mingit validatsiooni teha. Midagi joonistada väga
ei saanud või mingid midagi dünaamiliseks teha. Aga siis tulli samal ajal 
Internet Explorer 4.0\index{Internet Explorer}, kus olid \emph{custom} 
JavaScripti võimalusel brauseris, mis võimaldasid väga palju dünaamilisemat 
lehte ehitada. Ei olnud mingisuguseid JavaScripti \emph{library}sid, nagu 
Reactid\index{React} ja muud, mis võimaldavad kõike teha. Sa kirjutasid puhast 
JavaScripti, isegi Github'i ega Stack Overflow'd. Oli Internet Exploreri 
dokumentatsioon.

Ja siis veel üks nõue oli, et kõik need Hoiupanga töötajad on harjunud 
terminaliga, kus  hiirt ei olnud. Hiire kasutamine aeglustab tegelikult tööd. 
Ja siis nõue oli see, et sa pead saama navigeerida rakenduses ilma hiireta. 
Brauseris. 

\textbf{\enquote{Põhimõtteliselt ju tehtav, aga kasutajaliidese disaini mõttes 
päris keeruline ülesanne}}

Võttes kõiki neid piiranguid, ma pidin välja tulema mingisuguse kliendipoolse 
raamistikuga. Ja noh, eks ma siis tulin. Seal tekkis päris palju koodi ja tol 
ajal tüüpilises brauseri rakenduses vajutad \emph{submit} nuppu ja siis sul 
terve leht laetakse uuesti. Ja meil ei olnud seda \emph{bandwidth}i kontorite 
vahel. Kujutan ette, et sul on viis tellerit ja  nad istuvad selle 28K 
modemi\sidenote{Sidet üle telefoniliinide (ja just seda peetakse siin silmas) 
reguleerisid \emph{International Telecommunication Union}'i V-seeria 
soovitused. V.34 kirjeldas sidet kuni 33.6 kbit/s, kuigi levinuim oli just siin 
mainitud 28.8 kbit/s kiirus.} peal ja igaüks iga nupuvajutusega sulle hakkab 
tulema mingisuguse sadades kilobaitides lehte. Tol ajal jällegi tulid 
\emph{frame}d ja \emph{frameset}id\sidenote{HTML 4.0, mis avaldati 1997. aastal 
W3C soovitusena, sisaldas eraldi variatsiooni \enquote{raamide} (ingl. 
\emph{frame}) toega. Raamid võimaldasid HTML lehe jagada eri aadressidelt 
laetavateks alamosadeks. HTML 5.0 enam raame ei toeta}, nende vahel sai andmeid 
vahetada brauseri sees. No ja oligi üks \enquote{menu} \emph{frame}, kus oli 
enamik JavaScripti loogikat, mida kunagi uuesti ei laetud, ja siis oli see 
\enquote{main} \emph{frame}, mille sees siis laeti iga konkreetne tegevus.

\textbf{\enquote{Seal tehti veel mingeid huvitavaid asju, olid peidetud raamid, 
mis käitusid nagu praegune brauserist algatatud REST päring}}

Eks see arenes. Rakenduses oli \enquote{main} \emph{frame},  siis 
kliendiandmete \emph{frame} sest tavaline \emph{workflow}  oli selline, et 
klient tuli, sa leidsid tema konto ja siis sa said selle kontoga teha makseid, 
teha hoiuseid, mis iganes. Alati oli see, et otsid klienti, siis me laeme 
kliendi andmed eraldi kliendi raami, kus on nähtavad kliendi nimi, konto nimi, 
kontonumber. Aga seal all oli veel brauseri poole peal kliendiandmed. Ja siis 
meil oli \enquote{foori} \emph{frame}. Selle kaudu me \emph{submit}isime vormi 
andmeid, sest jällegi valideerimine pidi toimuma koha peal. Nupp käivitas 
valideerimismeetodi  ja valideerimismeetodi lõpus andmed saadeti teise vormi 
kaudu serverisse. Ma ei mäleta, miks me nii tegime, ju see oli vajalik. Aga see 
oli nagu raam, mille sees kõik pangafunktsioonid said tehtud. Põhimõtteliselt 
mul läks mingi kuu aega, et see kõik niimoodi püsti panna ja esimene 
eestisisese makse vorm ära teha. Kui see oli valmis, siis põhimõtteliselt kõik 
ülejäänud funktsioonid tulid mingi kahe kuuga. Põhimõtteliselt 
\emph{copy-paste}, seal ei olnud enam midagi keerukat. Eks seal pärast vigade 
parandamist ja optimeerimist muidugi oli ka, aga ei midagi keerukat.

\textbf{\enquote{Ehk, põhiline arhitektuur sai õigesti paika ja see töötas. Kui 
sa nüüd tagasi mõtled, mis sulle andis põhja, et selline asi teha? Oli see 
ülikool või lihtsalt häkkerimentaliteet või veel midagi?}}

Ma arvan, et ei olnud mitte mitte midagi peale probleemi, mida oli vaja 
lahendada. Mingid muud nõudmised, mis olid nagu \emph{hard} nõudmised, me ei 
saanud neist üle ega ümber. Tol ajal me tegime hanza.net'i\index{hanza.net} 
juba ja see oli värviline ja disaini mõiste oli juba olemas. Aga telleri 
rakenduse kohta oli spetsiifiline nõue et ta ei tohi väsitada inimest, et sa ei 
tohi kasutada erksaid värve, sest inimene teeb selle programmiga kaheksa tundi 
tööd. Ta oligi selline hall.

\textbf{\enquote{Sihukese asja peale isegi tänapäeval sageli ei mõelda, kust 
selline nõue tuli?}}

Meil oli ju tubli pangatehnoloogia osakond, kes mõtlesid, kuidas tellerid saavad 
hästi efektiivselt oma tööd teha. Ja jällegi ma ütlen, et mina olin ainult 
teostaja, seal oli terve tiim seal taga. Meil oli Toomas Rand\index[ppl]{Rand, 
Toomas}, kes tegelikult kirjutas kogu selle panga loogika, mina ju tegelesin 
ainult kasutajaliidesega ja andsin talle andmed. See, mis seal panga süsteemis 
toimus, oli tema teha et tema istus täpselt samamoodi kaksteist tundi päevas ja 
tegi. Aga tänu sellele projektile ka pangasüsteemi arhitektuuris tekkis 
korrastatus. Orcale Formsiga sa said kutsuda suvalisi funktsioone otse vormist. 
Aga meie arhitektuuri ütles, et üks nupuvajutus ja kogu tehing tehtud. Et see 
jällegi oli selline nõue pangasüsteemile. See õpetas, et liides ennekõike. 
Lepid liidese kokku ja siis osapooled saavad oma osaga  edasi tegeleda. See 
võimaldab sul testimist, testimise automatiseerimist, töö paralleliseerimist. 

Kitsendused tegelikult sunnivad inimesi tegema õigeid otsuseid. Ja me näeme 
järjest, et väga paljud inimesed ei oma kogemust sellistes piiratud  
ressurssidega olukorras toimetamisest. Eriti on seda näha välismaa inimeste 
puhul. Näiteks Silicon Valleyst inimene tuleb ja ta ei saa aru, et mis mõttes 
me ei palka juurde inimesi. Et ma ei saa ju kõiki oma ideid realiseerida, mis 
mõttes ma pean prioritiseerima? See on probleem nendele inimestele, nad ei saa 
aru, mis tähendab, et mul ei ole raha. Ma näen, et Eestis on tekkinud selline 
olukord, kus on palju ära tehtud hästi vähese ressursiga täpselt sellepärast, 
et inimesed oskavad teha õigeid valikuid, oskavaid prioritiseerima. Ja see on 
see oskus tuleb omakorda sellest, et sa pead alati prioritiseerime, sest sul ei 
ole ressurssi.

\textbf{\enquote{Ilusast arhitektuurist edasi minnes, milline on ilus kood?}}

Ilus kood on see, kus inimene ei pea küsima, mida see kood teeb. Väga 
paljud inimesed, kes oskavad programmeerida, millegipärast arvavad, et mida 
optimeeritum või lakoonilisem kood on, seda parem. Kuid see teeb halba. On piir, kust
edasi enam teine inimene ei saa aru, mida see kood teeb. Selline kood ei ole hea 
kood, isegi kui ta teeb õiget asja. See on üks asi. Aga teine asi on, et ma 
pean sulle suur aitäh ütlema selle eest, et sa tõid Eestisse Joshua 
Kerievsky\index[ppl]{Kerievsky, Joshua} omal ajal\sidenote{Joshua on USA firma 
Industrial Logic asutaja ja üks pikema kogemusega agiilse tarkvaraarenduse 
praktikuid ja koolitajaid maailmas. Tema Eestisse toomise Hansapanga arendajate 
koolitamiseks kas 2000. aasta lõpus või 2001. aasta algul algatas siiski Erik 
Jõgi\index[ppl]{Jõgi, Erik}}. Sul tekivad elus mingid hetked, kus sa saad aru, 
et see on nüüd \emph{step function}. Ja see koolitus (ta ei olnud pikk, nädal 
vist ja mitte täis päevad), mis me saime, lõi 
tegelikult väga paljud asjad oma kohtadele. Joshua on ju tegelenud koodi \emph{refactor}iga, 
 kuidas teha kehvast koodist ilusat. Ja me rääkisime temaga \emph{unit} 
testimisest \ldots.

See ongi, see, mis tegelikult aitab ilusat koodi kirjutada: sa 
pead seda mitu korda ümber kirjutama, enne kui ta loogiline välja näeb ja 
kood peab loogiline välja nägema.

\textbf{\enquote{Kui ma takkajärgi mõtlen, siis tolleks hetkeks kogu see 
agiilse arenduse liikumine, kogu see mõtteviis, oli veel väga noor}}

Me olime Skype'is\index{Skype}, ja siis ma tulin Pipedrive'i ja siin on meil 
igasugu \emph{agile coach}id. Ma mingi hetk ma mõtlesin, teeme eksperimendi. 
Oli mingisugune grupp. Seal olid \emph{agile coachid}, arendajad. Siis ma 
küsisin, et teeme eksperimenti. Rivistame grupi niimoodi, et kes on kõige kauem 
\emph{agile} liikumisega tegelenud  või vähemalt teadlik olnud. Ja enamasti, 
isegi \emph{coach}idel, oli see aeg mingi 7-8 aastat. Inimesed, ma olen 20 
aastat sellega tegelenud! \emph{Agile Manifesto} minu meelest oli 2001 või 
2002. Tegelikult me kõik saime seda maitsta enne, kui ta popiks muutus.

\textbf{\enquote{Mis sa praegu teed?}}

Ma isegi ma isegi ei saa öelda, et ma juhin \emph{engineering}u 
organisatsiooni, sest ma juhin ka muid organisatsioone nüüd 
Pipedrive'is\index{Pipedrive}. Ma olen siin juba seitse aastat olnud. Aastal 
2013 liitusin, see oli väike firma, ambitsioonikas. Tööintervjuul minult 
küsiti, et kas ma usun, et me saame Salesforce'iga võistelda. Ütlesin, et päris 
Salesforce'iks ei kasva, aga sihuke võib-olla veerand sellest on võimalik. Siis 
oli kakskümmend inimest, oli kümme inseneri. Ja nüüdseks, kuus ja natuke peale 
aastat hiljem, on meil on kuussada inimest.

Kõik need aastad olen tegelenud skaleerimisega. Nii infosüsteemi kui ka 
organisatsiooni skaleerimisega. Ja, jällegi, ei olnud kunagi sellist mõtet, et 
äkki meil ei õnnestu, äkki me ei kasva. Niipea, kui sa niimoodi hakkad mõtlema, 
sa ei kasva. Ma  siiamaani tegelikult ei ole kindel, et kumb on \emph{cause}, 
kumb on \emph{effect}. Et kas see, et me oleme skaleerinud \emph{engineering}ut 
aitas Pipedrive'il kasvada või see, et ta kasvas, aitas meil skaleerida 
\emph{engineering}ut.

Kui vaadata teisi osakondi, ütleme \emph{marketing} ei skaleerunud. 
\emph{Product} pidi skaleeruma koos \emph{engineering}uga, muidu inseneridel 
poleks midagi teha. \emph{Sales} ei skaleerunud, \emph{support} skaleerus 
nii-öelda natuke tagantjärgi. Et tegelikult \emph{engineering}u kasvatamine 
kasvatas firmat. Ma loodan. Aga samas, kui te ei kasvaks, siis me ei saaks  
inimesi juurde palgata, need kasvud on omavahel seotud. Aga küsimus ongi see, 
et mis tõukas seda. Ja ma väidan, et me nagu väga ei vaadanud. Me olime 
kindlad, et me peame skaleeruma, sest minu kõige suurem hirm on olnud, et  kui 
me jääme \emph{bottleneck}iks. Et kui \emph{engineering}u peale hakatakse 
näpuga näitama, et näed, me tahame või peame tegema seda ja toda 
\emph{engineering}ul ei ole ressurssi, või nad ei jõua või süsteemid hakkavad 
kokku kukkuma, kui kui kliente on liiga palju või me palkame inimesi juurde ja 
need inimesed ei saa tööd teha, sest kuskil protsessis on \emph{bottleneck}. 
Või me ei saagi inimesi palgata, sest inimesed ei taha meile tööle tulla. Neid 
pudelikaelu on nii palju, et ma pidi korraga kõikide nendega tegelemine. 

Kui kuidagi ei saa, siis kuidagi ikka saab.


\chapter{Arne Ansper}
\index[ppl]{Ansper, Arne}
\question{Nagu ikka alustame sellest, kuidas asjad alguse said. Kuidas nad siis 
said sinu jaoks alguse?}

No minu jaoks need asjad alguse sellest, et kui ma põhikooli lõpetasin siis 
minu matemaatikaõpetaja arvas, et ma peaksin minema Nõkku\index{Koolid!Nõo 
Keskkool} edasi õppima. Ja suutis mu vanemaid ära veenda, et see on suurepärane 
mõte, siis ma sinna läksingi.

\question{Aga kus sa põhikooli lõpetasid?}

Jõõpres\index{Koolid!Jõõpre kool}\index{Jõõpre}, selline pisike koht Pärnu 
lähedal. Sada õpilast oli see põhikool meil vanas  mõisas, mitte mõisamajas 
endas aga koolimaja oli mõisa keskel. Niisugune väga mõnus koht oli. Ja siis 
mul matemaatika nagu sobis ja õpetaja oli väga usin, andis mulle lisaülesandeid 
ja lõpuks saatis olümpiaadile ja seal läks ka suht hästi.


\question{Sa siis tulid puhtalt matemaatika ja mitte arvutite nurga alt sinna 
Nõkku?}

Ei, mul oli null kokkupuudet arvutiga enne. Vanemad seejuures pigem nagu 
tahtsid, et ma läheks. Ma ise olin väga  kahtleval seisukohal, et kas kodust 
nii kaugele minek, et see on äkki kuidagi raske ja paha ja nii edasi. 

\question{Mis aastal see oli?}

1985. 

\question{Sel ajal oli juba logistiliselt ju keeruline Pärnu lähedalt Nõkku 
saada?}

See oli lihtne ja tüütu, selles mõttes, et olid bussid, mis sõitsid neli tundi 
ja olid tavaliselt maast laeni rahvast täis ja siis veel Pärnust koju kus buss 
käis kahe tunni tagant. Seal ikkagi võttis aega, ütleme nii.  

\question{Ja Nõos pandi kohe arvuti ette?}

Ei, Nõos see oli tavaline keskkoolielu selle väikse vahega, et tuli ühikas 
elada. Mina olin viimane aasta, kes elas poiste ühikas, mis on selline 
suhteliselt raju ja legendaarne koht. Ehitatud kuskil tsaariaja lõpus, Eesti 
aja alguses. Talvel oli niimoodi, et tulid  kodust, tõid sihukesed suured 
märjad puunotid, läksid oma tuppa, mis oli  null kraadi lähedal kütsid ta siis 
üles selleks, et magada saaks. Hommikul lõid ikkagi pesukausi pealt jää katki, 
kui hakkasid hambaid pesema, niisugune koht oli. Esimene aasta oli hästi lahe. 

Alguses oli tavaline keskkond ja siis tuli programmeerimise õpetamise lihtsalt 
ühe regulaarse ainena sisse ja hakati õpetama. See oli ikkagi matemaatika ja 
füüsika kallakuga kool aga programmeerimise õpetamine seal oli lihtsalt nagu 
aine nagu mida iganes muugi. Mahud, loomulikult, olid suuremad nii 
matemaatikal, füüsikal kui ka sellel, programmeerimisel, millel mujal oli null, 
et seal oli siis nagu mingi muu number.

\question{Räägi palun Nõo kooli taustast, kuidas sinna üldse sai?}

Tead, ma ei tea. Mina olin tollal niisugune inimene, et emaga koos me sinna 
läksime. Ma arvan, et me käisime direktori juures rääkimas. Et kuna mul oli 
tegelikult olümpiaadilt mingisugune koht ette näidata siis kuidagi ma sinna 
igatahes sisse sain. Kuidas täpselt, kas seal oli mingi konkurss või mingi muu 
süsteem, ei tea. 

\question{Kes Nõo kooli direktor tol ajal oli? See kool tundus kellegi 
entusiasmi peal käivat?}

Enn Liba\index[ppl]{Liba, Enn} oli minu meelest tol ajal direktor\sidenote{Nõo 
kooli arendas selliseks reaalteaduste ja programmeerimise õppe keskuseks, nagu 
me teda praegu tunneme, Kalju Aigro\index[ppl]{Aigro, Kalju}. Ta oli kooli 
direktoriks aastatel 1951---1982, talle järgneski selles ametis Enn Liba}. Aga 
seda entusiasmi aspekti ja ajalugu, ma pean tunnistama,  ma ei oska 
kommenteerida tollal huvitusin  oluliselt muudest asjadest.

\question{Aga mis asjad need olid, millest sa huvitusid?}

Tegelikult mulle meeldis põhikoolis elektroonika. Aga see oli selline 
platooniline huvi, kuna juppe oli hullult raske kätte saada. Ja mulle meeldisid 
mudellennukid, mis oli ka suhteliselt platooniline. Aga Nõos tuli 
programmeerimine hästi kiiresti peale, kui hakkasime seal õppima. Seal oli suur 
Vene \emph{mainframe} Nairi-3-1\index{Arvutid!Nairi-3}\sidenote{1964. aastal 
Jerevanis välja töötatud Nõukogude arvutiperekonna Nairi kõige võimekam liige}. 
Kõps\index{Keeled!Kõps} ja Rops\index{Keeled!Rops}, eesti keeles sai 
programmeerida, need olid  vahvad. Siis olid seal Agatid\index{Arvutid!Agat}, 
mille ligi suht ruttu sai, mis olid teistmoodi vahvad, kus sai mingit 
valmistarkvaraga ka kasutada. Ikkagi mingite mängude mängimine oli oluline ja  
siis ise mingite asjade proovimine. See nagu hakkas väga kiiresti meeldima.

a \question{Oskad sa takkajärgi kuidagi reflekteerida, mis sulle seal meeldima 
hakkas?}

Väga ei oska, ausalt öeldes. Ma üritasin mõelda, et mis ma siis tegin nende 
arvutitega toona. Mul on umbes kaks asja meeles mida ma Agatiga tegin. Esimene 
programm oli umbes see, et oli \verb|for| tsükkel: muutis värvi, trükis mingi 
teksti nagu, ütleme, \enquote{tere}. Kõigis keeltes ja siis veel vilkuva 
taustaga ka. Sellega sai vähemalt üks õhtu kui mitte kauem möllatud ja timmitud 
neid efekte, tekste ja asju. Ja siis teine asi, mis mul on meeles, ma püüdsin 
ühte Nintendo mängu (need pisikesed puldi mängud, mis olid\sidenote{Arne peab 
ilmselt silmas Nintendo Game \& Watch seeria käes hoitavaid mänge. 
Originaalidest oluliselt rohkem oli liikvel nende Nõukogude kloone, mida müüdi 
Elektronika kaubamärgi all. Tegu polnud siiski alati täpsete koopiatega: 
Nintendo EG-26 kloonis IM-02 püüdis mune Miki Hiire asemel hunt tuntud 
Nõukogude multifilmist \begin{russian}Ну, погоди!\end{russian}}) taasluua, ma 
lõingi. Seal oli, nagu ta on, mingi fikseeritud arv positsioone, mingi tegelane 
liikus, mingid teised tegelased liikusid ja siis olid mingid surmasaamised ja 
mingid boonuste saamised. Probleem oli selles, et ma ei teadnud tollal, mis asi 
on massiiv. Põhimõtteliselt oli niimoodi, et iga objekti jaoks oli mul muutuja, 
mis ütles, et kas objekt on või ei ole. Ja kui seal mingid asjad liikusid, siis 
mul oli lehekülgede kaupa \verb|if| lauseid, et kui see muutuja omab seda 
väärtust, siis järgmisel sammul ta omab teist väärtust. Ja muidugi 
refaktoreerimis-tööriistu ei olnud. Kui ma kuskil vea tegin, siis ma nägin 
päevade kaupa vaeva, et ma nimetasin neid oma muutujaid ja \verb|if| lauseid 
ümber.


\question{Väga huvitav. Tol ajal tundus asjadest mitte rääkimine olevat 
õpetamise metoodika osa. Meile näiteks ei räägitud \texttt{for} tsüklist tükk 
aega}

Ütleme nii, et seda Agati\index{Arvutid!Agat} ei õpetanud meile keegi. Õpetati 
Kõpsi ja Ropsi. Kõik, mis Agati peal sai tehtud, see oli puhas enda välja 
võidetud ja võideldud  arvutiaeg, enda entusiasm. Ma isegi ei mäleta, \emph{by 
example} käis see asi vist, et vaatasid, mida keegi teine oli teinud. Mina küll 
ei mäleta, et oleksin ühtegi, Agati või BASICu\index{Keeled!BASIC}  kohta 
käivat raamatud lugenud kunagi. Kõik see oli lihtsalt nagu folkloor, 
katsetamise ja kõlakate tasemel. Et oleks keegi lekitanud selle info, et 
massiivid on olemas, oleks selle Nintendo mänguga palju rutem valmis saanud. 

\question{See oli suur töö ju, pidi ikka kihu olema?}

No aega oli palju, segavaid faktoreid oli vähe, eks ole. Ja ilmselt siis see 
arvuti alistamine meeldis, nagu välja tuleb. Agatiga\index{Arvutid!Agat} ma 
mäletan seda kindlasti, et ma hankisin endale selle 
assembleri\index{Keeled!Assembler} nii-öelda manuaali. Mis oli põhimõtteliselt 
paar-kolm ruudulist lehte, kuhu ma siis kirjutasin tähtsamad käsud ja registrid 
ja värgid üles ja siis studeerisin seda. Ja ma tean, et ma ikkagi nagu 
tuuseldasin seal Agati assembleri poole peal ringi. Aga mida ma tegin, seda ma 
kindlasti ei mäleta. Mäletan olulisimaid registreid, mida näppides käis piiks 
ja kust sai lugeda mingit vist klaviatuuri sümboleid või midagi sellist, aga 
\emph{that's it}.

\question{Kuidas Nõos tase oli, seal olid kõik sinusugused koos?}

Seal oli  selliseid inimesi, kes olid üle vabariigi kokku tulnud, kellel olid  
mingid huvid ja eeldused  reaalainetega tegelemiseks. Aga seal oli ka noh 
lähikonna inimesi. Et see on nagu päris, selline geto kuskil, see oli ikkagi 
nagu natukene spetsialiseeritud kohalik kool, et seal oli igasuguseid inimesi

\question{Kas sealkandis mingit äri tegemist ka juba käis, keegi raha eest 
programmi ei kirjutanud? Kaheksakümnendate lõpp ikkagi?}

Võib-olla keegi tegi, aga  ma julgeks öelda, et ma isegi ei huvitunud sellest 
ja ma ei tea sellest midagi. 

\question{Tartu vahet ka käisite?}

Jaa. Mingil hetkel, ma ei mäleta enam mis klassis, aga siis ma sain teada, et 
Tartu Ülikooli Raamatukogus\index{Tartu Ülikool!Raamatukogu} on mingisugune 
XTde\index{Arvutid!XT} klass. Kaheksa kuni kümme arvutit oli seal. Kuidagi ma 
sain sinna juurde, ma ei mäleta, mis alustel sinna seda aega sai reserveerida. 
Igatahes ma tean, et ma seal ikkagi jõlkusin päris mitu õhtut nädalas. Sa 
said seal mingisuguse tunni või kahese \emph{slot}i, mul oli umbes kaks flopit, 
millest ühe peal oli Turbo C\index{Keeled!Turbo C} ja teise peal oli siis tüüpi 
opsüsteemi oma asjad ja siis  midagi ma seal programmeerisin. 

\question{Aga kust sa said tolle Turbo C?}

Ma ei kujuta ette, kus ma selle saada võisin. Seal ma käisin päris tükk 
aega aga seal ma põhiliselt tegelesin ka sellega, et mängisin selle Turbo Cga. 
Aga kas mul ka mingi eesmärk oli, seda ma ei mäleta. Aga Turbo C see oli 
igatahes.


\question{On ikka paras hüpe Kõpsust ja Ropsust C ja mälu ja pointeriteni? 
Mille pealt too hüpe tuli sul?}

Jällegi nii kauge aeg, et ma kardan, et meile koolis ma isegi mäletan seda, mis 
meile üheksandas klassis  programmeerimist õpetati, aga ma ei mäleta, mis edasi 
sai, ausalt öeldes. Mida meil seal üldse räägiti. Ilmselt ise liikusin 
kiiremini edasi. Pärast TPIs\index{TPI|see{Tallinna Tehnikaülikool}} 
\index{TPI} ka see asi esimestel kursustel, et need  programmeerimise loengud 
olid  sellised, et sealt ei olnud midagi uut saada. Seal  mingid teised asjad 
olid pigem  need, mis olid uued, aga mitte see programmeerimise pool. 


\question{Kuidas sa sealt Nõost TPIsse\index{TPI} sattusid? Oleks ju loogiline, 
et sa lähed sealt Tartusse matemaatikasse?}

See oli ka suht \emph{random}iga selles mõttes, et ma mõtlesin, et võib sinna 
minna või tänna minna. Need argumendid, miks  Tallinnasse proovida, need olid 
niisugused väga otsitud ja õrnad, et miks ma just sinna Tallinnasse läksin 
proovima,  seda ma tegin. 

\question{Mida õppima?}

LI\index{Tallinna Tehnikaülikool!LI}. Ma täpselt ei mäleta, kas oli arvutid ja 
arvutisüsteemid, tõenäoliselt võis olla.

\question{See LI lühend jookseb mitmelt poolt läbi aga keegi ei tundu teadvat, 
mida see tähendas}

Kas ta üldse midagi tähendas? Et \enquote{L} on tõenäoliselt mingi 
automaatikateaduskonna kood, eks ole, ja \enquote{I} on mingi muu asja kood. 
Seal oli LA, mis oli äkki rohkem automaatika teisi tähti ei mäleta, äkki on LS 
ka olemas olnud. LI  oli jah see, kus mina oma aega veetsin.

\question{Sa ütlesid, et programmeerimise õpe sind väga edasi ei aidanud, kas 
seal üldse midagi õpetati, mis sulle midagi juurde andis?}

Tagantjärgi  vaadates tundub, et  seal LI-s räägiti nagu laiuti alates sellest, 
kuidas transistori teha, kuidas transistoridest saaks teha mingeid 
mikrolülitusi, kuidas saaks kõik see, mis sorti registrid meil on, kuidas 
registritest mingit automaatikat ehitada. Kuidas protsessorit teha, kui sul on 
neid registreid hulgi käes. Ja  teisele  poole minnes ka, kõik sellised asjad 
nagu siduteooria. Need asjad andsid, tagasivaates, need teadmised, et kui sa 
vaatad tänapäeval enda ümber, siis maagilisi asju, mille kohta ma ei tea, et 
kuidas seda saaks teha või ma pean uskuma midagi või ma vajaduse korral ei 
saaks sinna lõpuni välja kaevuda, neid on väga vähe. Ja see, ma arvan, on üks 
asi, mis mina olen leidnud, hästi kasulik. Tänapäeval on neid kihte sinna nii 
palju juurde tulnud, et vanasti oli ikkagi väga lihtne. See oli umbes nagu 
renessansiajastul, kui üks tüüp suutis  kõike, mida oli mõtet teada, teada. 
Natukene, kui  mina seal TPIs käisin, see aeg hakkas läbi saama. Ütleme 
niimoodi, et tänapäeval ilmselt ei ole võimalik, et sa tead kõike, mida oleks 
kasulik teada arvutiasjandusest. Ma mõtlen just tänapäeval seda, mis riistvara 
poole peal on juhtunud. Sinna on laotud neid kihte ja neid virtualiseerimise 
tasemeid ja mida iganes veel juurde. Ja siis \emph{soft}i poolel on ka vastu 
tuldud, sinna neli kihti virtualiseerimist vahele laotud ja nii edasi. See on 
nagu see, kus kipub nagu raskeks minema see järje pidamine.

\question{Kas TPIsse minek oli asjade loomulik käik või oli sul mingi plaan ka, 
mida tegema hakata?}

Mul niisugused pikaajalisi plaane ausalt öeldes ei olnud. Mulle meeldis teha, 
mulle meeldis nende arvutitega mässata, kas ma mässan Tallinnas või mässan 
Tartus, vahet pole. Ja siis ma mässasin nendega Tallinnas. Üks huvitav nüanss 
on veel see, et et umbes seal keskkooli lõpus ma sain isikliku arvuti ka. See 
oli midagi teistsugust, see oli Atari 520 STf\index{Arvutid!Atari 520 STf}. Mis 
oli siis Atari Motorola 68000 prosega tükk. 512kB oli tal mälu, selle ma 
\emph{upgradesin}  ühe megani mingil hetkel. Selle peal ma siis elasin ja 
siis selle peal ma püüdsin nagu süvitsi minna kogu sellega, mis seal nagu teada 
oli. 


\question{Kust sa sihukese aparaadi said kaheksakümnendate lõpus?}

Mul olid vanaonud, kes elasid Rootsis. Ema ja isa ükskord käisid seal ja siis 
sealtkaudu ma selle siis sain. 

\question{See pidi Agati kõrval ikka ulmeline aparaat olema}

Tegelikult oli niimoodi, et teised olid PCde peal. Kui ma nüüd vaatan, siis 
need inimesed, kellega me siis igal pool nagu koos ringi käisin, siis noh 
üheksakümnenda aasta paiku umbes, normaalsed inimesed said PCdele ligi ja siis 
toimetasid nendega.  Ja siis minul oli kodus Atari  ja tegelesin sellega 
põhiliselt. 

\question{Ataril on kihte vähem, sai lihtsamini sügavale välja minna}

Jaa, see oli nagu hoomatav täiesti,  mis seal toimus, midagi väga ulmelist 
polnud. Natuke mängisin ka, aga mitte liiga palju. Mul ikka see 
programmeerimine meeldis kõige rohkem selle asja juures. Selle Atari peal ma 
tegin igasuguseid imelikke asju.

Ma üritasin CAD programmi teha, joonistamisprogrammi. See isegi lõpuks selles 
mõttes töötas, et seal sai teha ringe ja jooni, igast värke, salvestada ja 
laadida. Ja siis mul oli, tagasi vaadates jälle hullumeelsus, et mulle nagu 
kohutav tegi muret see, et mälu saab otsa. Et kui sa teed dünaamilist 
mäluhaldust, eks ole, et siis saab mälu otsa. Üritasin seda siis minimeerida. 
Näiteks mulle tundus, et nagu lokaalsed muutujad, mis on \emph{stack}is, on 
kuradi ebaefektiivsed. Ja sisuliselt see CAD programm oli kirjutatud 
sajaprotsendiliselt globaalsete muutujate otsa. See oli täiesti hullumeelsus 
nagu tagasi mõeldes, seal tuli ikka kõvasti refaktoreerida, sest ma ikkagi panin 
täitsa puusse alguses. Seal seda loll ümberkirjutamist oli nii palju, sealt ma 
sain selgeks, et okei, nii ma mitte kunagi rohkem ja mitte ühtegi asja ei tee. 
Väga-väga palju vigu sai igatahes tehtud.

\question{Eks see on ju õppeprotsess, mõnda asja teoreetiliselt selgeks ei saa}

Jah, absoluutselt nõus. Ütleme, et nii võimekaid inimesi, kes kogu aeg teiste 
vigadest õpivad, et neid väga palju ei ole. Ikka enamus kipub oma vigadest 
õppima. 

\question{Kui sa TPIsse\index{TPI} jõudsid, kas sa seal teisi omasuguseid ka 
kohtasid?}

Meil oli hästi lahe kursus. Aga tegelikult oli niimoodi, et seal TPI ja alguses 
ma ikkagi õppisin, eks ole. Mis sest, et seal programmeerimise vallas mul ei 
olnud väga huvitav, aga neid muid ained ma ikka õppisin korralikult. Ma olen 
ikka väga usin õppur olnud. Ja mul juhtus niisugune asi, et mind 
Tarvi\index[ppl]{Martens, Tarvi} kutsus ühel hetkel Ektaco-sse\index{Ektaco}, 
ma arvan, et see oli üheksakümmend üks aasta. Ja see oli siis see 
\emph{community}, kus ma siis hakkasin nagu inimestega koos olema ja oli siis 
ka töise karjääri algus. Ma arvan, et see võis olla, see võis olla 1991, aga  
sada protsenti kindel ei ole. Mingi kolmas kursus äkki umbes.

\question{Kolmas kursus on üsna hilja ju?}

Tegelikult ongi see, et programmeerimise õppimine, üldse arvutiasjanduse 
õppimine võtab ikkagi aega. Ma tagasi vaadates mõtlen, et mis ma siis tookord 
oskasin või kuidas ma mõtlesin või  kuivõrd hästi ma siis programmeerisin.  
Ütleksin, et palju varem ei ole mõistlik seda tööd üritada teha. See võib  
frustratsiooni tekitada. Mis mul oli, ma olin ikka viis aastat nüüd innustunult 
selle asjaga tegelenud. Ma arvan, et kui ma  tööle sain, siis ma olin ka noh, 
enam-vähem miinimumtasemel, kus oleks  mõistlik, et keegi annab sulle 
ülesandeid, millele on ka mingi tähtsus ja tähendus ja sa teed nad  ära.

\question{Kas sul midagi sellist ei olnud, nagu inimesed on rääkinud, et 
lihtsalt arvutiaja saamiseks tekkis mingi arvutiklassi admini koht?}

Ei, mul ei ole ju midagi taolist. Ütleme tõesti mälu võib olla natuke petab, et 
mis aastal mul see Atari sinna täpselt tekkis, aga mul kuidagi oli alati 
mingisugune võimalus olemas, nii palju, kui mul seda tarvis oli ja sellest 
piisas. 

\question{Oskad sa mõnda näidet tuua, mida sa seal Ektacos alguses 
programmeerisid?}

Ektaco oli niisugune  firma, kus tehti riistvara ja tarkvara. Ta tegi 
tööstuskontrollereid, automatiseeris tehaseid, eks ole. Ja olid need 
sardsüsteemid, seal on väiksed mikroprotsessorid neid oli vaja programmeerida 
ja need programmaatorid olid kallid. Ja siis Ektaco hakkaks tegema oma 
programmaatorit. Põhimõtteliselt mingisugune lisaseade PCle, millega sa saad 
neid kivisid kõrvetada. Üks teine tüüp, kes oli nagu riistvara poole peal (ma 
ei tea, aga ma arvan, et ta oli umbes nagu mina, värskelt laekunud staatuses) 
ja mina tegin siis softi. See oli selles mõttes nagu päris huvitav, et meil oli 
PC/AT platvorm, seal oli ISA siin ja selle arvuti me süstemaatiliselt kogu aeg 
ajasime ikka täiesti lukku. Ja selleks, et saaks mingit sotti, siis meil oli 
seal siuke äge asi nagu loogikaanalüsaator. See on niisugune aparaat, et kui 
Ostsilloskoobiga saab visualiseerida mingit analoogsignaali, siis 
loogikaanalüsaatoril on palju-palju pisikesi klemme, mis sa paned kuskile prose 
või mingite digitaalsignaalide külge. Siis sul on teine arvuti mis  
visualiseerib, et kuidas need signaali mustrid on ja siis sa saad panna 
\emph{triggereid}, et kui mul tekib selline muster,  siis salvesta ja 
taasesita. Ehk et kui me ajasime selle selle PC täiesti hulluks, siis me saime 
sealt loogikaanalüsaatori pealt pärast vaadata, et mis siis juhtus, et mis me 
valesti tegime. Ühesõnaga tema siis tegi riista ja kirjutas siis sinna 
kontrolleri peale programmi ja mina kirjutasin PC peale siis põhimõtteliselt 
draiverite programmi vastu, mis omavahel suhtlesid. Ja siis tegin sellele ka 
kasutajaliidest.

Meil olid igasugused Inteli ja IBM-i \emph{manual}id laua peal, neid me siis 
seal sobrasime ja dekodeerisime, et mis me peame nüüd tegema, et siit 
midagigi läbi läheks. 


\question{See kõlab kuidagi hästi süsteemse ja korraldatud ettevõtmisena?}

Ei, see oli hull häkkimine. Nojah, Ektacos seda kraami, mille abil nagu häkkida, 
seda oli ja meil meil oli võimalus seda kasutada. Ja tegelikult ma tõesti selle 
teise tüübi  tausta ei tea, et võib-olla tema oli  kuidagi kogenum, tema tuli 
ju loogikaanalüsaatoriga sinna laua taha. Aga see oli suhteliselt niisugune 
kasulik ja kergesti omandatav seade, et noh kuidas sa seda pruugid. 

\question{Jah, aga võrreldes sellega, kui (nagu on räägitud) inimesed 
vaibanoaga emaplaadi pealt radu maha kratsisid, et modem tööle saada on tegu 
ikka \emph{high-tech} häkkimisega}

No me tegime ikka sinna radu juurde selleks et see kuidagi tööle saada, me ei 
kratsinud midagi maha! Mina ise seda riista-poolt tol ajal ei puutunud. Ehkki 
meil Ektacos programmeerija töövahendite hulgas oli kindlasti tinutus kolb, et 
nii raua lähedal oli seal see enamus sellest elust. 


\question{Kas te saite tööle ka selle kupatuse?}

Ja, loomulikult. Ja siis sellega seoses muidugi, kuna  see oli veel see aeg, et 
 see Borlandi\index{Borland}\sidenote{Borland Software Corporation oli 1983. 
aastal asutatud ja eri nimede all siiani toimetav tarkvaraettevõte, tuntud 
eelkõige arendajate töövahendite poolest. Neist kuulsaimad olid 
\enquote{Turbo-} eesliitega keeled Assembler, BASIC, C, C++, Pascal ning hiljem ka Delphi}
toodang, igasugused Turbo-blaahid, mis neil olid, need olid nagu  standard, eks 
ole. Siis loomulikult sai kirjutatud oma akendussüsteem, mis nägi välja nagu 
see Borlandi Turbo Vision\index{Turbo Vision}\sidenote{Borlandi poolt 
üheksakümnendate alul arendatud tekstipõhine kasutajaliidese raamistik Pascali 
ja C++ jaoks}, aga oli hoopis parem ja teistmoodi tehtud ja seega töötas väga 
kenasti. 

\question{Milles see väljendus, et ta parem oli?}

Ta oli nagu ägedamini struktureeritud. Siis mul hakkas juba 
C++\index{Keeled!C++}  meeldima, ta oli hullult objektorienteeritud. Tal olid 
mingid oma kontseptsioonid, et kuidas sa neid aknaid ja asju  esitad, kuidas sa 
sündmusi käsitled  selles mõttes, et sul on klaviatuur ja hiir. Mingi asi on 
fookuses, kuidas need sündmused jõuavad õige objektini, ja see on  klaviatuuri 
ja hiire puhul väga erinev loogika. Ja kõik see oli selliseks loogiliseks 
kompotiks keeratud, et sinna oli lihtne rakendusi teha. Sellel tükil oligi 
umbes üks programm, mis  seda ägedat raamistiku kasutas, see oli see sama 
programmaatori kasutajaliidese. Aga noh, selles mõttes oli Ektaco väga tore, et 
need tööülesanded ei olnud väga piiravad. Sa võisid ikkagi, ma ei tea, 
kuude või isegi aastate kaupa rahulikult häkkida ja sealt lõpuks tuli mingi asi 
välja. 

\question{Ja teistpidi, ega sul ei olnud neid akende joonistamise asju võtta 
riiulist kümneid?}

Ei, ikka oli. Sedasama Turbo Visionit oleks võinud pruukida ja seal oli 
igasuguseid teeke. Aga kuidagi, mis see siis on, nagu ametiuhkus ei lubanud 
teise mehe akna teeki kasutada. Tuleks ikka enda oma teha, sest et no mis 
mõttes, ma ei oska nüüd parimat akendusteeki teha. 

\question{Sellist suhtumist pannakse tänapäeval pahaks? Või ei panda?}

Seda tehakse teisel tasemel, eks ole. Tasemeid on juurde tulnud, seal 
nokitsetakse hoopis mingisuguste muude asjade juures, aga mina arvan, et see on 
nagu suht paratamatu, et see on hädavajalik, et inimesed heas mõttes 
leiutatakse jalgratast. Teeks asju, mis on juba tehtud, aga teeks teistmoodi, 
teeks paremini. Põhimõtteliselt olid ju opsüsteemid olemas, et mis mõte oli 
seda Linuxit hakata tegema, PC-Unix oli olemas. See oli olemas, et no mis siis 
häda oli sellel SCOl või millel iganes. 


\question{Jah, põhimõtteliselt oleks ju võinud olla, et siiamaani kõik 
kasutaksid sinu aknategijat}

Kindlasti need inimesed, kes on armunud kaheksakümmend korda kakskümmend viis 
teksti ekraanisse, need oleks olnud siiamaani selle andunud kasutajad. 

\question{Mäletan, FoxPro\index{FoxPro} joonistas lausa mingeid varjusid 
akende taha}

Ja, see on loomulik, varjud akendel pidid olema.

\question{Kas seda teie kiibikõrvetajat kasutati väljaspool 
Ektacot\index{Ektaco} ka?}

Need asjaolud muutusid nii kiiresti, et see, mis oli kallis ja kättesaamatu 
kaks aastat tagasi,  kaks aastat hiljem ei olnud enam seda. Ja ma arvan, et 
seda võib-olla tehti mingi üks või kaks eksemplari ja seda pruugiti Ektaco 
siseselt, aga sellest mingit edulugu ei tulnud. Ja see ei olnudki põhitegevus. 
Mina jälle ei tea, eks ole, et miks seda üldse tegema hakati, kas tõesti oli 
siis nii kättesaamatu või lihtsalt oli äge seda teha.  

\question{Jah, kui ma sind kuulan, see ei kõla suurepärase ärina}

Ektaco tegi ju  äri ka. Ja ma pean tunnistama ausalt, et  mind huvitas tollal 
programmeerimine. See, et mida  kolleegid nagu tegid, ma teadsin, aga ma väga 
ei süvenenud sellesse. See oli hästi selline fokusseeritud toimetamine.

\question{Kas tol ajal tekkis mingi kokkupuude arvutisidega ka juba?}

Seal Ektacos oli mul terve hulk toredaid kolleege. Olid 
Tarvi\index[ppl]{Martens, Tarvi}, Heiki Kask\index[ppl]{Kask, Heiki}, Jaak 
Niit\index[ppl]{Niit, Jaak}, Gunnar Valge\index[ppl]{Valge, Gunnar} oli seal 
minuga samas toas, kindlasti oli veel paar-kolm inimest. Ja siis meil oli 
Fido\index{FidoNet} \emph{point}, mis siis tekkis jälle seal Tarvi ja Heiki 
initsiatiivil, minu meelest ennekõike. Me olime alguses Lõvi point. 
Lõvi\index[ppl]{Lõvi|see{Lepp, Andres}}\sidenote{Lõvi, pärisnimega Andres 
Lepp\index[ppl]{Lepp, Andres}, on legendaarne TPI arvuti-mees, paljude meie 
põlvkonna inimeste sõber, teejuht ja eeskuju} oli siis TPI 
Arvutuskeskuses\index{Tallinna Tehnikaülikool!Arvutuskeskus}. Minu jaoks oli ta 
kunn, ma ei tea, mis ta seal tegelikult oli ja siis olime seal Lõvi 
\emph{point}. Jooksutasime seal FrontDoori\index{FrontDoor}\sidenote{FrontDoor 
oli üks populaarsemaid FidoNeti mailereid} ja mida iganes me jooksutasime. 

Ma arvan, et mingil hetkel me \emph{point}i staatusest \emph{upgrade}sime 
ennast \emph{node}ks. 71 oli meie number, julgeks arvata. Ja me helistasime 
kuhugi sisse ka, sest ma mäletan, et ma olen mingisuguse \emph{prompt}i otsas 
rippunud. Ja vaat seda jälle ei tea, et kust ma sain teada, mis käskudega seal 
Unixis\index{Unix} midagi teha. Ja kuidas mingi binaarne fail ära 
\emph{uuencode}da, selleks et ma saaks seda üle terminali endale 
\emph{dump}ida, selle \emph{dump}i salvestada, oma masinast \emph{decode}da ja 
mingit zipi sealt seest kätte saada. Kuidagi ma teadsin seda, kuidagi ma 
mingisuguseid asju imesin. Aga see on jälle niimoodi, et mingid asjad olid nagu 
õhus nagu mingisugused hallitusseene eosed laiali. Nii, kui kusagil pinnase 
sai, kohe läks kasvama. 

\question{Nii mitu sammu selleks, et midagi kätte saada, barjäärid olid jube 
kõrged toona.}

Info ikkagi liikus, see, ma arvan, ei olnud probleem. Küsimus oli ikkagi 
ennekõike riistvaras ja \emph{access}is ja  telefoniliinides ja niisuguses 
kraamis. Modemid olid ju roppkallid asjad, eks ole. Arvutid,kõik oli roppkallis 
välja arvatud aeg. Töö juures õnneks meil mingeid modemid olid, mitte küll 
kõige härjemad. Meil oli mingi 2400 ja MNP5\sidenote{\emph{Microcom Network 
Protocols (MNP)} on perekond (tähistatud numbritega ühest kümneni) 
veaparandusprotokolle, mida sageli kasutati varastes kiiretes (2400 bit/s ja 
rohkem) modemites} oli see meie lagi, millega me seal alguses toimetasime siis. 
Aga kõik olulised asjad liikusid ikka flopide peal, seda ei viitsinud keegi 
ära tõmmata, tõmmati mingeid pisikesi nublakaid. Tollal oli flopiga bussi peale 
minek reaalselt kiirem kui modemiga toimetamine.

\question{Mis sorti materjali te oma nodes hoidsite?}

Point oli meil puhas Fido point. Meil minu meelest küll BBSi ega midagi olnud. 
Meil oli ikkagi sõnumivahetus, \emph{Echomail} ja \emph{Netmail}, ehk siis 
privaatkirjad ja niisugused avalikud foorumid. See oli see, miks me nii-öelda 
suures pildis seda \emph{node}i pidasime. Kui keegi midagi tõmbas, siis ta 
tõmbas enda jaoks ja võib olla jagas  kolleegidega kuidagi midagi aga meil 
mingit sihukest varamut või niisugust ei olnud.

\question{Kellega te neid meile vahetasite, mis uudisgruppe lugesite? Kogukond 
ei olnud ju suur? Lõviga sai ju niisama ka juttu rääkida, ei pidanud kirja 
saatma?}

Mina lugesin põhiliselt \emph{Echomail}i, mul mingisuguseid kirjasõpru, kellega 
mingeid asju seal väga oleks olnud ajada, et tegelikult väga ei ei olnud. Minu 
jaoks oli see lihtsalt nagu foorum, kus sa saad huvitavat ja enamasti ka väga 
humoorikat  sisu. See väljendustase, see, kuidas inimesed, ükskõik mis teemal, 
viitsisid oma mõtteid sõnastada, need iroonia, sarkasm, huumor, kõik need 
tasemed, see oli niivõrd hea tekst valdavas osas, et seda oli  alati lust 
lugeda. Ükskõik mis oli, mingid autofoorumid, mul polnud  sooja ega külma 
nendest autodest. Aga lihtsalt need naljad, need vihjed, see oli lihtsalt hea 
meelelahutus, enamuses. Muidugi seal on ikka programmeerimised ja riistvara ja 
kõik muud teemad ka. See oli kasulik ja naljakas.

\question{No aga skaalal Tolkienist üle autode C++-ni?}

No kõike, absoluutselt. Kogu elu oli seal minu meelest. Seda jaksas tervikuna 
läbi lugeda sest inimesi oli vähe, palju sa ikka seda head kvaliteetset sisu 
suudad toota. Seda  oli vähe tegelikult, mis seal liikus minu meelest.

\question{Ühesõnaga, praeguses mõistes oli võimalik kogu sisuloomel silm peal?}

No sellel, mis Fido \emph{Echomaili} kaudu tuli, jah. Seal kuskil paralleelselt 
hakkasid arenema mingeid \emph{newsgroupid}, ka Eesti omad, millega mina 
alguses eriti ei puutunud  kokku. See oli natukene teine seltskond minu 
meelest, kes seal nii-öelda internetimaailmas hakkas toimetama. 

\question{Need olid kaks eri maailma, nende vahel mingit silda ei olnud?}

Nii ja naa, kontseptsiooni mõttes olid interneti uudisgrupid ja Fido omad 
samad, aga seal olid mingid ebamugavad erisused. Kunagi  hiljem, kui ma 
Ektacost Küberneetika Instituuti läksin\index{Küber|see{Küberneetika 
Instituut}}\index{Küberneetika instituut|see{Cybernetica}} siis ma tegin oma 
\emph{node} Solarise\index{OS!Solaris} peale. Meil oli seal üks 
SPARC\index{Arvutid!SPARC}\sidenote{\emph{Scalable Processor Architecture 
(SPARC)} on Sun Microsystems'i poolt arendatud RISC-arhitektuur. Sun müüs 
sellele arhitektuurile tuginevaid, siinmail populaarseid, servereid ja 
tööjaamu} server ja siis ma ajasin seal peal käima kogu selle Fido softi. Üks 
venelane oli selle kirjutanud. Ja siis ma tegin \emph{news}i \emph{gateway}, 
mis nagu Fido Echomaili \emph{newsgroup}ideks köitis kahesuunaliselt ja siis 
ühtlasi ka Netmaili siis tavaliseks meiliks köitis. See oli päris popp, ma 
isegi ei mäleta, millal see maha sai võetud. Ma arvan, et seal juhtus see asi, 
et sellele Solarisele oli lõpuks vaja  korralik \emph{upgrade} teha ja siis ma 
ei viitsinud vist enam. Fido oli ära surnud selleks hetkeks ja siis ma tõmbasin 
ta maha. Aga mingil ajal  oli ta hästi popp, mul oli seal, ma arvan, ikkagi 
sadu sadu kliente oli oma personaalse \emph{account}iga seal minu \emph{news}i 
serveri küljes, kellel oli siis nii-öelda kirjutamisõigus Fido gruppidesse. 
Fidos oli see korrapidamine nagu olulisem, seal ei olnud sellist anonüümset 
kasutust, keegi vastutas alati kellegi eest. Keegi  kuskilt kaudu sai 
\emph{access}i ja kui see keegi oli nõme, siis see \emph{access} võeti talt 
ära. Kui ma hakkasin seda asja Newsi \emph{gate}ma, siis ma lubasin sedasama 
teha, eks ole. Ma ei andnud kellelegi Fido gruppidele  kirjutamisõigust, kui ma 
ei teadnud, kes ta on ja ma ei saanud seda \emph{access}i talt ära võtta. 

\question{Aga see on ju, ütleks, autoritaarne?}

See toimis. See oli nagu  endale olulise keskkonna  normaalsena hoidmise 
eeldus. Teistmoodi ei saa. 

\question{Aga mis on \enquote{nõme}?}

No, solvad teisi inimesi, trollid, ütled puhasti, eks ole. See ongi põhiline, 
et kui sa lähed isiklikuks, teed teisele haiget, halba emotsiooni, sihukest 
asja ei ole vaja. See kui sa vaidled, see on okei, seda peab olema, see ongi 
tähtis, eks ole. Aga sa ei tohi  teistele haiget teha. 

\question{Kõlab nagu lihtne, eluterve ja samas fundamentaalne definitsioon. Aga 
kui sa Ektacost ära tulid, kas sa veel õppisid?}

Ei, mu õppimised olid selleks hetkeks õpitud või noh, mitte päris lõpuni 
õpitud, aga ma olin oma inseneridiplomi kätte saanud, vist 1993 või 1992. Sain 
oma kraadiselt kätte. Magistrikraadiks \emph{upgrade}sin ma ta siin natuke 
hiljem. Mina õppisin, viis aastat, sain süsteemiinseneri diplomi, aga pärast 
hakati  kogu seda kraadi värki järjest lahjendama, eks ole. Kui nüüd õppeaastad 
järjest lühenesid, siis \emph{by default} oli mul bakalaureus, aga siis ma 
pidin veel natuke juurde õppima ja tegema magistritöö, ma saaks magistriks. See 
oli kunagi seal 2001. aastal umbes, kui ma selle  ette võtsin. 

\question{Aga tol hetkel sul ei olnud sellist tunnet, nutikas ja usin õppur 
nagu sa olid, et peaks teadusmaailma sukelduma?}

Ega ma seal TPIs ise teadusmaailma suurt kokku ei puutunud. Kuna ma sealt poole 
pealt hakkasin programmeerijana tööd tegema, eks ole, siis see  haaras  
enam-vähem täielikult. Ma arvan, et lõpus läks kas see õppimine natukene 
nigelamaks, sest töö juures oli palju huvitavam ja palju nagu väljakutseid 
pakkuvam. Viimased asjad mis seal Ektacos\index{Ektaco} sai tehtud, oli 
kontrollerite uue sideprotokolli disainimine. Ma olin hullult vaimustatud 
TCP/IPst ja  siis ma trükkisin välja kõik standardid, mis ma sain: TCP, IP, 
Etherneti. Aga kontrollerid on mingi 8051 peal, mis on umbes nagu, nagu väga 
väike. Aga siis ma lugesin need RFCd kõik läbi ja siis ma tegin mingi oma 
sideprotokolli, mis inspireerus siis kõigest: Ethernetist, IPst ja TCPst. Ehkki 
ta ei olnud nagu päris \emph{flow}le orienteeritud, aga pigem  selline 
\emph{datagram}i-põhine protokoll. Sihukesed vanad riistvaraässad Ektacos  olid 
väga nördinud ja solvunud, et mis mõttes ma kirjutan protokolli, mis ei ole 
deterministlik. Mitte \emph{master-slave}, vaid igaüks võib traadi peal 
lobiseda, kui mõte pähe tuleb, ja siis lahendatakse konfliktid ära ja tehakse 
re-transmissioon. Nad olid väga pahased minu katsetuste peale, aga ma arvan, et 
programmeerisin selle lõpuks sinna ära ja ta mingil määral töötas ka. See oli 
päris äge.

\question{Aga mille vahel see protokoll siis käis?}

Põhimõtteliselt oli see, et PC, mis siis juhtis neid tööstusarvuteid. Neil oli 
sihuke karp, mille nimi oli satelliit, mis oli siis tööstuskontroller, millel 
olid igasugused digi- ja analoogsisendid-väljundid, mis kuskil tehases keerasid 
mingit nuppu, et betooni teha või midagi. Ja siis sellel olid mingid
juht-programmid ja neid tuli konfida. Tüüpiline värk, eks ole. Sa pead teadma, mis 
sul tehases toimub. Sa pead käske andma, selleks on mingit võrku vaja. Ja neid 
satelliidikontrollereid võis seal korralikus tehases ikka palju olla. Ja siis 
ta tuli PCsse kokku tõmmata ja ma usun, et keegi kirjutas siis mingit softi 
sinna PC poolele, mis siis neid satelliite siis jälgis ja juhtis.

\question{See kupatus oli päriselt \emph{production}is ja Eesti Vabariigis 
tehti betooni niisuguste seadmetega?}

Jaa. Ma arvan, Palivere Ehitusmaterjalide Tehas\index{Palivere 
Ehitusmaterjalide Tehas} vist oli see, mis oli ära automatiseeritud Ektaco 
poolt ja ma millegipärast arvan, et midagi oli Tallinna 
Veepuhastusjaamas\index{Tallinna Veepuhastusjaam}.Aga seda ma väga kindlalt ei 
tea, aga seal oli neid veel. Neid objekte ikka oli.

Mida mina tegin, see oli järgmine generatsioon, need objektid juba töötasid 
mingisuguse muu protokolli ja mingi muu  tehnika peale, aga kõik see kasvas, 
eks ole. Ja siis algatati uue generatsiooni satelliidi väljatöötamise projekt, 
kus mina siis  protokolli kontributeerisin ja realiseerisin. 

\question{Kui sa võrgundusest juba nii palju teadsid, sind kuhugi interneti 
varasesse maailma ei tõmmatud kaableid vedama või midagi?}

Ei, mulle meeldis programmeerida. Nende muude asjadega ma tegelesin nii palju, 
kui nad olid kasulikud ja vajalikud selleks, et saaks midagi ägedat 
programmeerida. 


\question{Ja Küberis\index{Küber} sai ägedamalt programmeerida?}

Lõpuks jah. Jälle Tarvi\index[ppl]{Martens, Tarvi} kutsus mind sinna. 
Küberneetikasse oli tehtud infotehnoloogia osakond, mis peitis seda infot, et 
tegelikult tegeldi seal infoturbega ja siis oli seal mingi riiklik programm, 
mille eesmärk oli Eesti riigi infoturbe ja krüptograafia vajadusi rahuldada. 

\question{See oli juba enne, kui tekkis AS Cybernetica?}

Jaa, see oli enne seda. Mina läksin sinna 1994, aga see töögrupp tehti 1993, ma 
arvan. Ja siis seal oli terve hulk nutikaid inimesi koos, kes siis  selle 
missiooni elluviimisega tegelesid, et  kompetentsikeskust ehitada.

\question{Kes selle taga oli? Keegi pidi ju selle tellimuse formuleerima, et 
riiklikult on tarvis tegeleda krüpto ja infoturbega?}
Ülo Jaaksoo\index[ppl]{Jaaksoo, Ülo} oli siis Küberneetika 
Instituudi\index{Küberneetika Instituut} direktor. Minu vaates oli tema see, 
kes seda kõike lõi ja korraldas. Kuidas ja  kellega tema läbi rääkis või kust 
see mandaat tuli, seda mina ei oska küll öelda. Aga tema oli jah, kellel see 
visioon  oli, et seda on tarvis. 

\question{Arvestades, kui vähe vajas Eesti riik krüptot ja infoturvet praegu ja 
kui strateegiliselt oluline teema see praegu on, siis sellise visiooni jaoks on 
ju tarvis väga ägedat ettenägemisvõimet?}

No aga kaugemale vaatamine ongi teadlaste ja akadeemikute ülesanne. Kust mujalt 
see tulla saab? 

\question{Visioon visiooniks, mida see töö toona praktiliselt tähendas?}

Esiteks, ise õppida. Teiseks, teisi õpetada. Eestikeelne terminoloogia, 
standardid, profiilid, seminarid, koolitused mida iganes.  Ja  teistpidi 
hakkasid niisugused praktilised asjad tulema. Vaata, tollel ajal maailm oli 
nagu väiksem, ka krüpto ja infoturbemaailm oli väiksem ja mingil hetkel on 
ikkagi veel võimalik hoomata  kõike, mis oli oluline. Mitte küll päris üksi, 
aga sihukese väikese töögrupi sees nagu meil oli. Ma arvan, et mis meil  väga 
hästi läks, oli see, et meil olid inimesed, kes  tegelikult  huvitusid just  
sellest infoturbe süsteemsest poolest. Et mitte see, et mis on nagu see 
tehnika. Aga mis on see organisatsioon, need inimesed, need reeglid, eks ole, 
seadusandlus seal ümber. Ühesõnaga süsteemne lähenemine valdkonnale kui 
tervikule, mis on  väga tähtis ja  mis sellest meie grupist välja kasvas. 

Teiselt poolt oli see, et meil on seal sihukesed \emph{hardcore} häkkerid ja 
\emph{hardcore} krüptograafid, kes nagu olid valmis mida iganes tegema. See 
sümbioos oli minu meelest hästi lahe. Ma arvan, et minu esimene töö 
Küberneetika Instituudis oli see, et ma pidingi riigiasutustele kirjutama 
juhendi, kuidas KA9Q\index{KA9Q} otsas ehitada endale internetti ruuter. 

\question{Mille otsas?}

KA9Q on üks soft. \enquote{KA9Q} on mingi radistide kutsung, mis vastab mingile 
inimesele, kes selle softi kirjutas, on minu arusaamine. Ja see oli DOSi peal 
jooksev \emph{all singing all dancing} asi, mis realiseeris TCP, kõikvõimalikud 
sideprotokollid, võrgukaartide toed, SLIP, PPP, ruuterid, mida iganes. FTP 
deemonid. Täiesti müstilisi asju on tehtud maailmas.  Et kui sul oli üks  
üleliigne PC, modem, võrgukaart ja see soft, siis sa said teha endale ruuteri, 
millega oma organisatsioon kuhugi ära ühendada. Ja siis mina peksin selle käima 
ja kirjutasin eestikeelse lühijuhendi, kuidas seda asja  pruukida, hooldada ja 
nii-öelda käimas hoida. See oli mu esimene nii-öelda, ma ei tea, praktikandi 
töö või mis iganes töö seal Küberis. Aga siis hakkasid igasugused muud asjad
tulema.

Me olime mingis hästi varajases europrojektis, ma mäletan, see võis olla 1995. 
aastal. Ma tean, et ma käisin Darmstadtis\index{Darmstadt}. Sakslased olid 
kirjutanud sellise tarkvara nagu secu-d, mis oli,  ma ei kujuta ette, et ma 
pakun mingi kümme mega haljast C koodi väga halvasti kirjutatud, mis  
realiseeris kogu krüpto, mis tolleks hetkeks oli teada. Kõik sertide töötlus, 
särk-värk. Ja siis me üritasime seda secu-d'd kuidagi rakendada ja kuidagi 
käima peksta. Ütleme niimoodi, et selline \emph{cross-platform} arendus tollal, 
et sul on kood, mida sa kompileerid mingi UNIXi jaoks ja mingi PC jaoks ja siis 
tulid Windowsid, eks ole. Ja teha nii, et see kuidagi enam-vähem  töötab ja 
piisavalt vähe mälu lekib ja piisavalt harva sama mäluplokki kaks korda 
vabastab on  raske ülesanne. Ja siis ma selle secu-d najal ehitasin 
mingisuguseid asju. Turvalist meiliklienti näiteks ja sertifitseerimiskeskust. 
Sertifitseerimiskeskused olid lahedad,  seal mingisugusel  ajaperioodil oli 
see, et me seal Küberneetika Instituudis iga aasta programmeerime vähemalt ühe 
sertifitseerimiskeskus valmis softi mõttes.

\question{Miks?}

See oli mingisugune \emph{blend} sellistest praktilistest vajadustest ja 
teadustöö eesmärkidest. Et üks  sertifitseerimiskeskus, mille me näiteks 
programmeerimine oli näiteks selline. Tollal ei olnud ju mingeid kiipkaarte ja 
riistvaralisi turvamooduleid kätte saada. Ja see oht, et kui sul 
sertifitseerimiskeskuse võti ära 
 kompromiteerub, et siis keegi annab võltssertifikaate välja, see oli suur. Või 
et keegi annab sellele operaatorile altkäemaksu, et annaks võltssertifikaadi 
välja. Sul oleks vaja mitmesilma printsiipi ja sihukest  topeltkaitset. Ja siis 
me realiseerisime selle, et me võitsime selle RSA võtme tükkideks. See on 
seesama, mida praegu SplitKey\index{SplitKey} ja SmartID\index{SmartID} teevad. 
Meil ei olnud küll seda turvalist mitmes osas võtme genereerimist, me lihtsalt 
RSA võtme, jagasime ta osakuteks ja siis meil oli sihuke m-n-ist skeem. 
Selleks, et sertifikaati välja anda, siis viiest operaatorist kolm pidid  
allkirja andma ja siis me kombineeris neist korrektse sertifikaadi kokku. Selle 
nii-öelda initsialiseerimisprotsessi käigus tekitati viis flopit,  millega need 
 operaatorid ringi oleks pidanud käima. Selles mõttes oli ta praktiline, et ta 
töötas,  tegi täitsa korrektseid X.509  sertifikaate ja oli kasutajajuhendiga 
varustatud.  
 
\question{Tundub, et kui sa enne seal ISA siini peal tegelesid väga madala 
taseme asjade katsetamise ja läbi mängimisega, siis nüüd sa tegid sedasama 
krüpto jaoks põhiolemuses olulisi primitiive ja protsesse läbi realiseerides?} 

Jah, et seda võib öelda küll, et mingis mõttes me tegelesime selliste hästi 
\emph{basic} asjadega. Me jõudsime ka rakendusteni välja. Meil oli ka 
hästi-hästi praktilisi asju, aga me kontrollisime tegelikult kogu seda pinu 
ülevalt alla välja. Et sellel ühel hetkel me tegime tulemüüre, mis oli väga 
hästi müüv toode Eesti turul, Barrikaad\index{Barrikaad} oli selle nimi, mul 
siiamaani barrikaadi T-särk alles. Siis me tegime VPN toote, mis oli veel 
ägedam. Selle VPNi teine versioon oli igasugustes Eesti riigiasutustes 
väga-väga pikalt kasutusel ka peale seda, kui selle tugi ametlikult õnnetuseks 
ära lõppes. Ja selle põhieelis oli see, et ta oli projekteeritud hästi 
turvaliseks, keskelt administreeritavaks, eriti töökindlaks. Ehk et see, et sul 
on  harukontorid, kust sa ei taha üldse interneti väljapääsu, vaid tahad läbi 
keskse tulemüüri (mis oli kallis) neid välja juhtida, see oli meil sinna sisse 
ehitatud. Igasugused paralleelsed ruutingud üle erinevate kanalite, eks ole. 
Seal tekivad probleemid, kui sul on VPN tunnel, sul on  sisemised aadressid, 
välimised aadressid, kuidas sa neid majandad niimoodi, et see ruutingu info ka 
seal sisevõrgus korrektselt leviks ja tegelikult ka töötaks. Et kasutajad 
ei peaks  ootama, kuni nende seanss katkisest kanalist tervesse kolib, eks ole, 
et see lihtsalt töötakski. Ja kogu see administreerimine. Meil oli tehtud see 
tükk, mis võimaldas süsteemi konfiguratsiooni muuta, see oli eraldi, see võis 
offlainis olla, see suhtles  muu maailmaga floppide kaudu, see ei olnud võrgus. 
Ja siis oli meil võrgus olev tükk, mis ainult monitooris, kogu sealt infot ja 
täitis neid käske, mis võrgust väljas olev tükk talle  ette pani. Niisugune 
eriti kõrgete turvanõuete jaoks tehtud haldussüsteem. Ja, ja seal me muuhulgas 
siis, kuna tollal ikkagi see PC krüpteerimisvõime oli nõrk, siis me 
realiseerisime ise  šifreid. Tollal just MMXi laiendused tulid prosele välja, 
mis võimalused sul näiteks IDEAt\sidenote{\emph{International Data Encryption 
Algorithm (IDEA)} on esmakordselt 1991. aastal kirjeldatud sümmeetriliste 
võtmetega plokkšiffer} paralleelselt arvutada,  mitu blokki korraga. Ja siis 
Helger Lipmaa\index[ppl]{Lipmaa, Helger} oli veel Küberis tööl, kes 
programmeeris siis Linuxi tuuma jaoks MMXi \emph{extension}eid  kasutava 
AESi\sidenote{\emph{Advanced Encryption Standard (AES)} on Belgia 
krüptograafide poolt välja töötatud Rijndael plokkšifri alamhulk. 1997. aastal 
teatas NIST (\emph{National Institute of Standards and Technology of the United 
States (NIST)}) plaanist asendada avaliku protsessi abil tolleks ajaks 
ohtlikult nõrgenenud DES algoritm. Vincent Rijmen ja Joan Daemen esitasid oma 
ettepaneku valikuprotsessi ja see standardiseeriti NISTi poolt 2001. aastal}  
realisatsiooni. Meil seal Linuxi\index{OS!Linux} tuumas olid oma draiverid, mis 
seda VPNi asja haldasid, seal peal olid  oma deemonid võtmete vahetuseks, konfi 
levituseks, kõigeks muuks  ja siis niimoodi hierarhiliselt üles välja.

\question{See, mis sa räägid, et see ei kõla enam nagu programmeerimine, see 
kõlab nagu arhitekti töö. Kas sa liikusid programmeerija rollist arhitekti 
rolli või mõtlesite te neid asju kambakesi välja, kuidas see käis teil?}

Selles mõttes, et välja mõtlesin kogu aeg lihtsalt enamasti oli see teine tüüp, 
kes asju realiseeris,  sellesama peakolu sees. Lihtsalt seal tulid inimesed 
nagu appi. Meil ei olnud  väga selgelt nagu defineeritud rolle, eriti alguses, 
eks ole. Arhitekt, projektijuht, projektijuhid olid üldse väga haruldased 
nähtused, Me ei teadnud isegi, mis projekt on, me lihtsalt programmeerisime 
mingi hetkeni. Meil oli seal, jah, ikkagi terve hulk inimesi, kes arutasid 
intensiivselt praktiliselt kõigil teemadel. Kui asjad olid selged ja siis 
igaüks natukene läks oma  valdkonnas  süvitsi sellega.

\question{Nutikatel inimestel on vahel oma nutikusele vastav ego ka, keegi nina 
püsti ei ajanud ja ennast arhitektiks ei kuulutanud?}

Ei, päris nii ei olnud. Aga ma ise kardan tagantjärgi võib-olla mina ise 
kippusingi see tüüp olema, kes oma  arvamust teistele peale surus. Aga ma tol 
hetkel ei tajunud seda kindlasti niimoodi. 

\question{Ma arvan, et ega teised ka ei tajunud ja soft ju lõpuks ikkagi 
töötas ju}

Absoluutselt. Nii see tulemüür kui ka see VPN, olid meil ikkagi lõpuks ikkagi 
ääretult stabiilsed ja, ma ütleks, kvaliteetset tükid. 

\question{Privador kasvas ka ju sealt välja?}
Jah, Privador\index{Privador} oli siis Küberneetika Aktsiaseltsi spin-off 
firma, mis siis sai need nii-öelda infoturbetooteid, eesmärgiga need laia 
maailma viia, aga see kahjuks ei õnnestunud. Seal oli  kindlasti ports 
probleeme ja üks probleem, mida mina nägin oli see, et tollal hakkasid tekkima 
standardid, et mis asi on VPN, mis asi on standardne VPN. Ja IPSec oli 
enam-vähem ära standardiseeritud, IKE oli ära standardiseeritud ja see oli 
tegelikult see, mida oleks tahetud osta. \emph{Vendor lockin}i juba päris 
mõõdukalt kuni palju kardeti. Ja ehkki meie olime oma asja ehitanud, eriti need 
alumised kihid, need olid  standardite põhjal ehitatud aga mudel,  kuidas me 
nägime seda võrgu tervikut ette ja mida me pidime tegema, selleks, et neid häid 
omadusi saada,  seal tekkisid konfliktid IKE või ütleme, IPSeci, ideoloogiaga 
natukene. Meil  tegelikult oli töölaua peal  versioon kolm VPNist, mis oleks 
siis olnud täiesti standarditega ühilduv, mis loodetavasti selle  firmapärasuse 
probleemi oleks ära kõrvaldanud, aga see kahjuks ei läinud realiseerimisele. 
Selle asemel me tegime digiallkirja tarkvara ja ajatembeldustarkvara ja 
Notariseerimistarkvara ja kõike muud. Me nagu natuke ennustasime valesti, et 
mis on see \emph{killer} rakendus krüptomaailmas järgmise kümne aasta jooksul. 
Olime nagu natuke ajast seest selles mõttes.

\question{See lähenemine, et võtame alumise kihi standardid ja paneme nad 
kuidagi täitsa uut moodi ülemise kihi standarditeks kokku on ju seesama, mis 
sai digiallkirja konteineriga tehtud ja X-Teega ka}

Absoluutselt. Aga vaat seal ongi see, et standardid on ja peavadki olema 
tegelikult geneerilised, eks ole. Nad peavad olema sellised, et nad lahendavad 
paljude inimeste paljusid probleeme, siis nad on elujõulised. Nii. Aga aga kui 
sa võtad ühe konkreetse riigiasutuse, kellel on konkreetsed vajadused, mis ta 
peab ära lahendama efektiivsel viisil, siis sa ei pääse lihtsalt sellega, et sa 
võtad standarditele vastavat tüki ja evitad selle. See ei ole efektiivne. Ja 
see oli siis see, mida meie tegime. Aga seal oligi vaata natukene see, et me 
võib-olla ei tajunud seda, et kui suur see maailm on ja kui võimas ta on ja kui 
suure massiga ja kui kiiresti ta liigub. Me mõtlesime, et me teeme ikka rajult 
ägeda asja. Ja noh, see on nagu \emph{way}  parem ja praktilisem väga suure 
hulga klientide jaoks. Aga see teadmine, et miski asi on hea ja praktiline,  
seda on väga raske efektiivselt ja kiiresti ühest peast teise viia.  

\question{Arvestades, et samast pundist tulid ju ka X-Tee\index{X-Tee} ja 
ID-kaardi kontseptsioon, siis kahest kolm ei ole üldse mitte paha edu protsent}

X-Teega on muidugi see, et X-Tee omab selles meie VPNi tootes väga selgeid 
juuri. Tegelikult, kui me seda X-Teed tegime, see oli 2001.  Mais või juunis 
hakkas asi pihta või isegi natuke hiljem ja detsembris läks tootesse. Eks ole. 
See oli võimalik ainult tänu sellele, et me võtsime oma selle VPN toote kui 
substraadi. Meil oli kõik see olemas, et kuidas me teeme ühe Linuxi purgi 
turvaliseks, kuidas me sellele  Linuxi purgile paneme peale oma tarkvara 
\emph{patch}id, särgid-värgid, kuidas me seda Linuxit konfime, kuidas me hoiame 
konfi niimoodi, et see on efektiivne, kuidas konfi jagamine käib, see kõik oli 
olemas. Me lihtsalt selle asja peale ehitasime ühe natukene teistsuguse 
protokolli vahenduse tüki, eks ole. 

\question{Aga see kõik on natuke hilisem lugu. Kui mina sinuga esimest korda 
kliendina kohtusin, siis sa ikkagi juba juhtisid vägesid. Mina rääkisin oma 
mure ära ja sina tegid nii, et asjad sündisid. Kuidas sul inimeste juhtimine 
rollina esile kerkis ja kas sa üldse mõtestad seda tegevust niimoodi?}

See tekkis Barrikaadi\index{Barrikaad} või VPNi või Privadori\index{Privador} 
programmeerimise käigus, kui meeskond läks suuremaks. Eriti selle VPNi juures, 
ma arvan,  koordineeriv funktsioon oli ikkagi minu peale, et kes nüüd mida 
programmeerib, eks ole, mis ajaks. Ja kes neid asju evitamas käis, ikka meie 
ise, sealt tuli ka see klientidega suhtlus, eks ole. \emph{Helpdesk}, 
projektijuht, arhitekt, programmeerija, testija, tarneinsener, et mu roll oli 
natukene nagu kõik koos. 

\question{Aga ometi kuidagi jäi see koordineeriv roll just sinu peale?}

No ju siis selles pundis see  kõige paremini  mulle sobis, ei oska muud midagi 
arvata. Keegi pidi selle ära tegema, eks ole. Kui see olin mina, siis olin see 
mina, nii see läks.

\question{Ma selle pärast küsin, et ega sul mingisugust kihu ei olnud inimesi 
juhtida?}

Ei. See pigem oligi sedapidi, et, ma nägin seda, mis see asi võiks olla, mida 
me teeme,  päris detailselt päris paljudes aspektides. Ja siis ma nagu tahtsin, 
et see nii läheks, siis ma olin sunnitud  inimestele  ülesandeid või siis 
eesmärke püstitama. See tuli pigem sedapidi, et üksinda ei jaksa kõik ära 
progeda.

\question{Aga see on jällegi arhitekti vaatenurk. Minu peas on olemas täiuslik 
mudel süsteemist ja siis ma teen niimoodi, et see saaks teoks tehtud. Mis sa 
praegu teed?}

Mis sa praegu teed? Väga paljusid erinevaid asju. Ma suhtlen hästi palju 
klientidega ja potentsiaalsete klientidega, et aru saada, mis on  nende  mured 
ja vajadused, kuidas me saame   neid aidata. See on alates müügitööst, projekti 
juhtimiseni. Teistpidi ikkagi see, ütleme, arhitektuurne töö. Kui probleem on  
arusaadav, et mis oleks see lahendus. Ja need probleemid on keerulisemaks ja 
mastaapsemaks läinud. Mõnes mõttes ka vastutusrikkamaks selles mõttes, et me  
ikkagi tegutseme suuresti turvavaldkonnas. Ja see keskkond on nii palju 
vaenulikum ja nii palju keerulisem ja need panused on nii palju suuremad, et sa 
pead lihtsalt palju palju paremaid asju tegema kui me kunagi tegime. Sedasorti 
arhitektuurne  mõtlemine ja siis inimestele nende ideede jagamine. Nõustamine, 
mõnes mõttes ka võiks öelda isegi natukene koolitamise moodi asjad. 

\question{Sa oled kogenud arhitekt ja tead, mida on vaja selleks, et projekt 
välja tuleks. Kuidas sa viid entusiastlikult pihta hakanud meeskonnale kohale 
selle, et sinu arvates projekt ei saa välja tulla? Ja seda nii, et sind pärast 
tuppa tagasi ka lastakse?}

Samm üks on see, et sa pead aru saama. See võtab tegelikult päris kaua aega ja 
see on nagu see koht, kus tihti suhtled väga vähe. Ega seda, et vaatad peale, 
saad kohe aru,  mis valesti on, kuidas peaks olema, seda ei ole. Kõigepealt 
pead probleemist aru saama. Ja võib olla, et  sellepärast see see olukord ongi 
võib-olla keeruline või halb,  et see ongi olemuslikult keeruline probleem. 
Seal on mingisugused mingisugused põhjused, keegi on teinud mingeid otsuseid, 
mingeid probleeme on lahendatud ja selle käigus on tekkinud niisugune asi. Sa 
pead sellest aru saama. Sa ei saa lihtsalt minna, et \emph{sorry}, vanad, et 
siin on jama. Sa pead kõigepealt aru saama, mida on tehtud ja miks on tehtud, 
need probleemid endale selgeks tegema. Ja siis sa tõenäoliselt marineerid nende 
otsas päris kaua ja see ei tule niimoodi, et hops, homme hommikuks on valmis, 
eks ole. Sa mõtled ja kirjutad ja räägid. Ja ehkki tihti on niimoodi, et  sulle 
endale võib tunduda, et lõpuks kui sa mingeid asju hakkad tegema, et selline 
lahendus oli algusest peale selge. Aga kui sa lähed kontrollima fakte, et mida 
sa tegelikult rääkisid, mida sa oled ise kirjutanud, mis sa arvasid, siis 
selgub, et tegelikult see lõplik lahendus on sinu juurde väga suure kaarega 
tulnud. Sa pead selle lihtsalt välja kannatama ja selle ära tegema. Aga, aga 
point on lõpuks see, et kui sa oled jõudnud mingisuguse asjani, millest sa 
näed, et see ongi okei ja lahendab ära  selle probleemi ja selle probleemi ja 
selle probleemi. Võib-olla see lahendus on keeruline ja on kulukas nagu 
realiseerida ja on isegi riskantne aga ta on õige, ta on juba olemuslikult 
õige. Sa saad aru, et mis see probleem olemuslikult on, kuidas seda asja  
tükeldada, kuidas seda keerukust peita, kuidas seda asja üldistada. Ja siis sa 
pead väga kannatlikult väga paljudele inimestele seletama, miks me võiks teha 
just nii. Seda jõuga ei saa teha. Sa pead neid julgustama ja sa pead olema 
valmis nende eest viskuma džotile, juhul kui on vaja. Aga ma ise muidugi usun, 
et ei lähe vaja, või siis sealt džotist ei tule midagi surmavat välja, eks ole.


\chapter{Ahti Heinla}
\index[ppl]{Heinla, Ahti}
\question{Kuidas sa sattusid arvutite juurde?}
Ma tulen sellisest perekonnast, et minu ema ja isa olid mõlemad 
programmeerijad. Ja nemad olid siis ülikooli lõpetanud ja  said tööl kokku ka, 
see oli kuskil kuuekümnendate lõpp. Ja see oli see aeg, kus Eestisse tekkisid 
esimesed arvutid, mis sel ajal olid muidugi kapi suurused, aga ikkagi.

\question{Kuuekümnendate lõpus ei saanud neid programmeerijaid ju palju olla?}

Jah, kindlasti neid ei olnud palju, kuigi neid siiski ikkagi täiesti 
mingisugusel määral oli. Minul muide muide oli hiljuti selline asi, et kui emal 
oli selline suur juubel, üle seitsmekümne ja niimoodi, ja ta kutsus enda 
kursusekaaslased külla. Ja kes need kursusekaaslased siis on, need on 
rakendusmatemaatikud, praegu siis sellised üle seitsmekümne aastaseid  
inimesed, nii mehed kui naised. Põhimõtteliselt kõik programmeerijad, 
professionaalsed programmeerijad olnud, enam-vähem kõik, mehi ja naisi 
võrdselt. Ja, näed, meil on ikkagi asi juba nii kaugel, et meil on juba nagu 
suhteliselt kaugeid põlvkondi, kes on üles kasvanud programmeerijatena. Ja mina 
sündisin kahe sellise inimese järeltulijana.

\question{Kas see on pigem vedamine või vastupidi? Oleks võinud ju ka ära 
hirmutada?}

Mind see kindlasti ära ei hirmutanud, ma kasvasin üles perfolintide vahel. Ja 
niimoodi, et vahetevahel, kuna arvutiaeg oli ju piltlikult öeldes talongidega 
jagatav, arvuti pidi ikka õhtuti töös olema ja siis mõnikord  ema ja isa käisid 
õhtuti tööl, kui nad said arvuti aja kella kaheksaks õhtul, siis nad said 
arvuti aja kella kaheksaks õhtul. Ja siis nad võtsid vahepeal minu ja mu õe 
siis kaasa, mina jooksin ka arvutikappide vahel ringi ja vaatasin, kuidas seal 
magnetlindid vaikselt käisin nii ja naa ja see oli kindlasti hästi põnev. 
Hoopis teistsugune keskkond ja isegi helid on teistsugused ja vaatad, kuidas 
need masinad seal toimetavad, mingid magnettrumlid vaikselt vihisevad ja 
sahisevad ja kindlasti oli.

\question{Legendid räägivad, et selle põlvkonna rahvas korraldas Ameerikamaal 
lindikappide võidujookse ja muud sellist, tolles sinu arvutiruumis midagi 
sellist ka toimus?}

Mina selliseid asju ei näinud. Ma saan aru, et Eestis ka tehti selle sel ajal 
sellist  pulli, aga võib-olla  seda tegid natukene nooremad inimesed, kellel 
lapsi ei olnud. Minu isa ja ema olid ikka natuke siukse ontlikumad. Nad 
üritasid mingisuguseid konstruktiivseid asju arvutiga teha,  panna neid just 
ühel või teisel moel  käima, aga nad ei olnud sellised, kuidas öelda, häkkerid 
tänapäeva mõistes. Et  mismoodi arvutiga  pead pesta, näiteks, et selliseid 
asju nad ei mõelnud.

\question{Sind ju esialgu ei lastud linte perforeerima? Mis esimene asi oli, 
millele sa ise käed külge said?}

No mu vanemad olid programmeerijad aga mina ei olnud programmeerija, tavaline 
laps nagu ikka. Ma vist olin nagu natuke  matemaatiliselt  andekas, aga 
arvutitega otseselt minu tegelikult esimene kokkupuude oli ikkagi sellest, kui 
ma olin kümne aastane. Lihtsalt järsku päevapealt  ühel õhtul tuli ema  koju ja 
ütles, et kuule, Ahti,  ma õpetan sulle midagi, istume maha. Istusime maha ja 
ta õpetas mind programmeerima. Kolm õhtut niimoodi õpetas. Ja ma sain selle 
kolme õhtuga tegelikult sellest oast aru, et mismoodi see asi käib. Sealt 
alates  siis hakkasin juba ise edasi mõtlema, proovima, katsetama, lugema, 
natukene lolle küsimusi küsima. Kolm päeva ma olen sellist süstemaatilist 
programmeerimise õpetust saanud.

\question{Aga mis ta siis rääkis et see kümneaastasele huvitav oli?}

Eks mind huvitasid sellised asjad kindlasti. Sel ajal oli ju ka niimoodi,  
aasta oli 1992, et  lapsel on tohutult palju mingisuguseid ahvatlusi 
ümberringi, et Facebookid ja Instagrammid hüppavad siia-sinna ja kõikvõimalikud 
muud asjad käivad. Sel ajal oli ikkagi niimoodi, et ega meil kodus ju telefoni 
näiteks ei olnud. Arvutit ka ei olnud, sest mina kirjutasin programmi ikkagi 
alguses niimoodi, et kirjutasin paberi peale. Need kolm päeva õpet käis paberi 
peal. Ja kui  ema tuli õhtul koju ja sellist asja ütles, siis me ikkagi mitu 
tundi istusime maas, eks ole. Ei ole niimoodi, et mul oleks kogu aeg telefoni 
helisenud ja hüpanud, mingisugune asi, et kuule, Ahti, tule nüüd sinna, teeme 
seda. Selles mõttes võimalik, et ei olnudki nii väga vaja, et see oleks nagu 
hullult kuidagi põnev olnud kümne aastasele lapsele. Mul pigem oligi lõpuks  
põnev see, kui ma sain aru, kuidas see asi töötab.

\question{See peab olema päris korralik ettekujutusvõime, et sa paberi peale 
kirjutades saad aru, kuidas miski asi töötab. Sest paberil ei tööta sul midagi, 
seal on lihtsalt tekst}

Nojah, samas aga eks programmeerimise üks selline  võtmeoskus ongi tegelikult 
ju oskus ette kujutada, et mismoodi see masin  töötab. Lõppkokkuvõttes ju 
programmeerija ehitab ju masinat. Ja noh, piltlikult öeldes, ikka samasugust 
masinat nagu,  mingisugused hammasrattad kuskil käiksid. Üks koodirida on 
nii-öelda piltlikult öeldes üks hammasratas, teine koodirida on mingi kangikene 
kuskil seal, eks ole. Ja kui sa ehitad sihukest füüsilist või mehaanilist 
masinat, siis sa näed kõike seda, kuidas see töötab, et siin mingi ratas keerab 
ja siis kang liigub ja kuidas siis teine asi kuskilt midagi lükkab ja mingi 
lint või tross kuskilt midagi tõmbab. Sa näed seda kõike füüsiliselt. Ja 
programmeerija peab ka nägema. Aga ta peab nägema seda vaimusilmas, sest seda  
füüsilises maailmas  silmadega ei näe. Ja see vaimusilmas nägemise oskus on 
ülivajalik programmeerijale. Tagantjärele vaadates võib öelda, et eks ema mulle 
selle tegelikult õpetaski  see kolm päeva.

\question{Kas \texttt{goto} käib nii- või naapidi või tehete järjekord on 
selline või teine, on teisejärguline}

Just. Tegelikult oleks põhiliselt vaja teada, et sa \verb|goto| tegelikult teeb 
või et selles masinas, millise hammasratta, millise kujuga asja, see 
\verb|goto| seal teeb.

\question{See kolm päeva tekitas huvi, sa said enam-vähem aru, kuidas arvuti 
töötab, aga mis edasi sai?}

Siis läksime kuskil õhtul emaga siis sinna arvuti juurde, ema tööle. Ja siis me 
tippisime selle programmi sisse. Ja kui ma õieti mäletan, siis seda sisse 
tippimist võis juba mitu päeva olla. See oli ikka mitu lehekülge, see minu 
programm ja mõnikord läks midagi valesti ka ja nii edasi. Ema aitas mul siis 
seal mõned vead ära parandada ja siis tuli välja, et  tegelikult see programm 
töötas. See lahendas ühte väikest sihukest matemaatik  keerdülesannet, kus  
loogika oli  selles, et kui sul on  näiteks sada ühikut raha ja sa lähed 
raamatupoodi ja sa tahad seda sada ühikut raha ära kulutada. Siis sa pead 
kombineerima, et osta üks raamat, mis maksab viiskümmend seitse ja teine raamat 
nüüd maksab kolmkümmend, selline klassikaline  matemaatika keerdülesanne, 
kuidas kombineerida niimoodi, et kokku saada  summa, mis on võimalikult 
lähedane sajale aga mitte üle selles. Ja sellist ülesannet lahendas see minu 
programm. Ei ole nagu kõige triviaalsem asi, see ei ole nagu päris niimoodi, et 
vajutad nuppu ja programm ütleb lihtsalt \enquote{tere}. Tänapäeval ikkagi 
pigem kõik asjad üritatakse, ka heal põhjusel, ehitada niimoodi, et sul on 
selline nagu hästi kiire rahuldus või et sa nagu näed kaks minutit vaeva ja 
juba midagi hästi väikest nagu töötab ja siis sa näed veel viis minutit vaeva 
sealt tuleb veel midagi. Siis mina pidin vaeva nägema alguses kolm päeva, enne 
kui tulemust oli. Enne seda oli kõik ainult vaimusilmas.  Aga, tõepoolest, kui 
sul kogu aeg Facebook taskus ei hüppa, siis on nagu natuke lihtsam ka seda kolm 
päeva leida. 

\question{Mis tolle arvuti nimi oli?}

Ausalt öelda ma isegi ei mäleta, ei pruukinud isegi nõukogude masin olla,  seal 
oli tegelikult ka lääne aparaate.

Isegi minul sedasama seda ühte programmi, mis ma kirjutasin, seda minu meelest 
sai isegi  mitmel arvutil käitatud. Et see ei olnudki niimoodi, et 
\enquote{kuule Ahti see on nüüd sinu arvuti, millega nüüd sina  mitu päeva nii 
nagu tegeled}. See isegi vist nägi niimoodi välja, et ma pool programmi 
tippisin ühel arvutil sisse, mis oli sihuke suur must kapp ja siis järgmisel 
õhtul läksime ühe hoopis siukse läänelikuma välimusega siukse nagu nõtkema 
välimusega moodsama asja taha ja tippisin teise osa sisse. Et ma juba sain ka 
natuke kogemusi sellest, et see programm on ikka hoopis midagi muud, see ei ole 
see füüsiline arvuti, millega ma tegelen. Ma võin istuda ühe arvuti taha ja 
siis ma võin minna teisele kodusele teise arvuti taha, mis on terve toa suurune 
ja see minus seesama programm jookseb selles ühes jookseb selles teises.

\question{Mille peale sa vahepeal kirjutasid selle programmi? Kaartide peale?}

Siis olid ikka juba diskid olemas. Mitte need sihukesed, kolme tollised 
disketid, vaid sihukesed  kaheksa või viie tollised või mingid sellised asjad, 
pigem kaheksa tollised ilmselt. Aga kindlasti see esimene programm oli ainult 
selline algus, eks ole, sellest tuli mingisugune  oskus ja huvi asja vastu. 
Edasi hakkasin siis nüüd ise vaatama ja  sattusin kokku juba teiste poistega, 
kes siis analoogse asja vastu huvi tundsid. Lähemate aastate jooksul hakkasid 
tekkima ka personaalarvutid ja enam ei olnud alati niimoodi, et sa pead õhtul 
ema töö juurde minema tingimata vaid on kuskil juba muid kohti ka olemas.

\question{Kust sellised tutvused tekkisid, internetti ju polnud?}

Internetti ei olnud, aga küll oli olemas näiteks kaheksakümnendatel tekkinud 
selline asi, nagu Raaliklubi\index{Arvutiklubi!Raaliklubi}, mida vedas selline tegelane 
nagu Jaak Loonde\index[ppl]{Loonde, Jaak}. Mina olin ka selle klubi liige seal 
vahepeal ja see koondas sihukesi huvilisi poisikesi. Ega mul on raske öelda 
täpselt,  ise nagu poisikesena tol ajal süstemaatiliselt ei pannud tähele ja ei 
jätnud meelde ka täpselt, mida täpselt Jaak Loonde tegi ja kas ta üldse midagi 
tegi peale selle, et lihtsalt need poisid kokku tuua. Aga täiesti võimalik, et 
sellest piisabki, et need poisid kokku tuua, kellel on sama huvid ja siis nad 
omavahel juba vahetavad kogemusi, kellel kuskil jälle ema või isa töötab 
kuskil. 

Minul oli näiteks üks niisugune reliikvia, mille ema mulle andis: ta õpetas 
mind kolm päeva ja siis ta andis mulle ühe ingliskeelse raamatu, mis oli 
põhimõtteliselt Pascali programmeerimiskeele õpik. See oli inglise keeles, ehk 
siis ma ei saanud sellest eriti midagi aru. Ma koolis õppisin saksa keelt, 
mitte inglise keelt\sidenote{Tol ajal jagunesid koolid kaheks: kas lisaks vene 
keelele õpetatakse läbivalt inglise või saksa keelt}. Küll aga ma sain aru 
nendest  programmi näidetest, mis seal oli, eks ole, ja seal oli asjad ikkagi 
mingisugused loogilises  järjekorras. Tegelikult, kuigi ma inglise keelt ei 
osanud, ma siiski suutsin sellest raamatust kindlasti midagi õppida ja sealt 
tuli ideid, mida katsetada. Seal oli kuskil mingi, piltlikult öeldes, mingi 
\verb|goto| käsk, oletame. Ma ei teadnud, mis see tähendas, eksole, aga ma sain 
selle \verb|goto| käsu kuskil hiljem mingisse arvutisse sisse tippida ja 
vaadata, mis teeb ja küsida kelleltki teiselt, et mis see \verb|goto| tähendab. 
See on midagi muud kui lihtsalt öelda, et õpetaja mulle programmeerimist, et 
sul on ka konkreetne küsimus juba. Niimoodi läbi lukuaugu piltlikult öeldes see 
õppimine käis. Internetti ei olnud, aga  inimestel ei olnud ka internetti, siis 
kui nad  lennukeid ja autosid ehitasid, ja sai hakkama.

\question{Seal arvutiklubis sa käisid seepärast, et programmeerimine pakkus 
huvi?}

Jah, mulle pakkus see programmeerimise pool huvi. Mul tegelikult oli niimoodi, 
et isegi enne programmeerimist sattus kätte mingisugune lastele mõeldud 
elektroonika raamat ja ma  natukene nagu harjutasin või mõtlesin selle 
elektroonika peale ka, et kuidas näiteks transistorid töötavad ja muu selline. 
Ja see pakkus ka mulle kindlasti huvi. Aga tagantjärele vaadates siis ma 
ütleks, et minu elektroonika tegemine sel ajal oli  ülialgelisel tasemel. Ma 
nii-öelda  kuidagi nagu hästi õrnalt natuke nagu kõditasin elektroonikat ja 
üritasin sellest midagi aru saada, aga samas programmeerimisega ma tegelesin  
tegelikult. Selles mõttes oli seal väga suur vahe ja loomulikult oli ka väga 
suur vahe siis minu  professionaalsuse tasemes, mis  tekkis.

\question{Kas sa oskad öelda, kas see oli pigem eeskuju või midagi muud, mis 
sind pigem programmeerimise poole suunas?}

Üks asi oli kindlasti eeskuju, aga teine asi oli ka kindlalt puhtalt ju see, et 
selle jaoks, et elektroonikaga tegeleda, sul on vaja ikkagi mingisuguseid 
teatud füüsilisi asju. Sul on vaja elektroonikakomponente, sul on vaja 
tööriistu ja nii edasi, mida ju ei olnud. Isegi tänapäeval on ju poes  kõik 
olemas, aga sa pead minema ja üldse mitte vähe raha kulutama ja ostma need 
endale. Ma olen natukene ka hobi korras, elektroonikaga nagu tegelenud vahepeal 
kodus, tinutan üht-teist seal ja nii edasi. Ja noh, praegu, kui kõik on nagu 
justkui valla, kõik on olemas ja kahe päevaga tuuakse koju ära, raha ikka kulub 
selle jaoks. Mingi  üks jootekolb ja teine suurendusklaas, takistite 
komplektid, väiksed mikroprotsessorid, igasugused kivid ja sensoreid ja andurid 
ja displeid ja nii edasi. Sellega ei ole lihtne alustada. Programmeerimine on 
niimoodi, et sul tegelikult on vaja seda kohta, kus nii-öelda arvutis käia, eks 
ole, paber ja pliiats ja kolm päeva on täitsa piisavad alustamiseks.

Nagu mul sõber ja töökaaslane Jaan Tallinn\index[ppl]{Tallinn, Jaan} on  
öelnud, et programmeerimine on selline naljakas asi, et  enamikes muudes  
valdkondades on niimoodi, et kui sa hakkad   õppima, siis sa saad mingisuguse 
algse  taseme kätte ja siis sa pead rohkem õppima, et saada järgmisele 
tasemele,  sa pead veel rohkem õppima. Ja sa ei saa iseseisvalt õppida, vaid 
sul on vaja kedagi, kes õpetab. Kui sa õpid klaverimängu näiteks, siis sul on 
vaja tegelikult seda, et keegi sulle pidevalt õpetaks  klaverimängu, sa ei saa 
ise õppida klaverit mängima. Sul on mingisugused teatud käelised asjad, et 
mismoodi sa seda teed, parimal juhul sa saad või mingist YouTube'i, videost või 
mingist õpikust õppida. Aga sul on vaja seda YouTube'i videod või õpikut. 
Programmeerimine, aga, on tegelikult selline asi, et kui sa oled selle algse oa 
kätte saanud ja sind siis suletakse üksikule saarele aastaks niimoodi, et sul 
on arvuti käes, siis sa tegelikult suudad ise ilma ühegi õppevahendita õppida 
ennast väga heale tasemele, kui tahad. Puhtalt ise katsetamise,  ise mõtlemise  
teel. Ja eks tegelikult täpselt seda ma tegingi, sel ajal, kui ma teismeline 
olin.

\question{Kas sel ajal hakkas ka juba personaalarvuti moodi arvuteid liikuma?}

Jah, personaalarvuteid hakkas täiesti tulema ja meile koju tekkis ka üks Apple 
II\index{Arvutid!Apple II}. Sellega siis mina hakkasin toimetama, aga see oli 
üsna  kaheksakümnendate lõpus kuskil. Ma ei oska täpselt aastanumbrit öelda, 
aga ju ta võib olla võis juba 1988 olla või midagi niimoodi. Ma juba ikkagi  
nagu täiesti oskasin sel ajal programmeerida, ma ei olnud nagu päris enam kümne 
aastane, ma olin juba viisteist või kuusteist või midagi niimoodi. Inimestel, 
kes on seitsmeteist ja kaheksateist aasta vanused, enamikel inimestel on üsna 
kõvasti nagu meri põlvini  ja peod ja seltsielu ja asjad käivad. Aga minul on 
nagu paar asja teisiti. Esiteks ma olen üldiselt introvertne inimene ja mitte 
üli seltsiv, see seltsielu mul kuidagi nii hästi nagu välja ei tulnud. See on 
üks asi. Teine lihtne tõsiasi oli see, et vist kuni viimaste aastateni, umbes 
kolmekümne viienda neljakümnenda eluaastani oli mul elus selline asi, et kui ma 
joon kaks klaasi veini ära, siis hakkab mul pea valutama. Ma lihtsalt ei pea 
ühel  korralikul peol kaua vastu, lihtsalt ei pea,   ma lähen koju hiljemalt 
keskööks. Ja niimoodi on  kogu aeg olnud ja oli ka siis, kui ma olin 
kaheksateist, eks ole. Aga alates kella kaheteistkümnest ju nagu tegelik 
\emph{action} hakkab pihta, nagu mulle räägitud, ma nagu väga palju ise kogenud 
ei ole. Ja siis ongi see, et kui  ülejäänud inimesed avastavad seal seltsielu 
ja ja pidusid, siis osad teised avastavad arvutiasju ja siis avastavad 
seltsielu natukene hiljem lihtsalt.

\question{Aga mida sa programmeerisid? Sellise jõukohase aga samas huvitava 
ülesande leidmine ei ole ju üldse lihtne?}

No eks poisikesi ikkagi mängud huvitavad üsna palju ja kindlasti ma arvan  
minul ja minu kaasvõitlejatel kindlasti kõigil oli ju üks esimesi unistusi, et 
kirjutada oma üks arvuti mängida näiteks. Sel ajal oldi juba vahepeal need 
Yamaha arvutid juba tekkinud, eks ole, ja juba ka Apple II peal oli täitsa 
korralikke  mänge olemas. Need kommertsiaalsed mängud, need olid ikka sellisel 
tasemel, mida üks mingi hobistist poisikene ikkagi nädalaga valmis ei viska. Ja 
ega see tähelepanu ulatus  kolmeteistaastasel või viieteistaastasel ei ole väga 
selline, et suudaks midagi väga palju pikemat ette võtta. Selliseid väga 
lihtsaid mängukesi sai kindlasti ehitatud ja kindlasti ka proovitud üritada 
siis niimoodi häkkerlikult natukene läheneda sellele arvutile, et mida on 
võimalik arvutit tegema panna, mis hääli on võimalik arvutit tegema panna ja  
igasuguseid lollakaid visuaalseid kujundeid ette ette manada, ja muu selline. 
See pool kindlasti ka huvitas. 

Aga hiljem, tegelikult teismeeas sai igasuguseid asju proovitud. Ega täpselt ei 
teadnud, mida võiks  teha, aga valdkond kindlasti huvitas. Aga järjekindlamalt 
hakkasime mänge programmeerima Jaan Tallinna\index[ppl]{Tallinn, Jaan} ja Priit 
Kasesaluga\index[ppl]{Kasesalu, Priit}. Siis kui me olime keskkoolis. Siis oli 
juba niimoodi, et tähelepanu ulatus on juba nagu natukene inimesel juba 
kasvanud, eks ole, ja võtsime ette ühe mängu kirjutamise projekti. See sai 
natukene pikema vinnaga, et paneme nüüd kõik oma seni õpitud väiksed kogemused 
ja oskused kokku ja kohe mitu inimest paneme, mitte niimoodi, et igaüks oma 
nurgas pusib mingeid oma mängu, vaid teeme ikka sellise tiimi töö. Jagame 
ülesanded omavahel ära ja kuude viisi töötame selle kallal. 

\question{Kust selline mõtte üldse tuli või arvamus, et selline asi üldse 
võimalik võiks olla?}

Kuskilt iseenesest tuli, ma ei oska täpselt mõelda. Meil isegi ei olnud isegi 
mingit arutelu sel teemal. Lihtsalt sündis, et proovime midagi, midagi sellist. 

\question{Mis keskkool see oli?}

Mina õppisin Gustav Adolfi Gümnaasiumis\index{Koolid!Gustav Adolfi Gümnaasium} 
ja Jaan Tallinn\index[ppl]{Tallinn, Jaan} oli minu pinginaaber. Ja Priit 
Kasesalu\index[ppl]{Kasesalu, Priit} oli Jaan Tallinna pinginaaber eelmisest 
koolist, kus Jaan käis. Nii et me olime mõlemad, Jaani pinginaabrid olnud. Ja 
siis viimase keskkooliaasta jooksul niimoodi kolmekesi kirjutasimegi ühe mängu, 
millel oli nimeks Kosmonaut\index{Mängud!Kosmonaut}. Mina küll kogu selle 
kirjutasin seda kui hobiprojekti, aga Jaan ikka ütles, et see asi tuleb teha 
nagu äriks või see asi tuleks maha müüa ja selle eest  raha saada. 

\question{See oli nõukogude aeg ju veel, selle eest võis kinni minna ju?}

Peaaegu. See oli nõukogude aja lõpp küll, sel ajal, kus juba igasuguseid 
metalliärikaid juba juba käis ringi ja niisugune nagu väikene üle piiri  
kaubandus käis ja kooperatiivid ja asjad ja selline värk juba täitsa toimis. Me 
muidugi ei teadnud tuhkagi sellest, kuidas  see  ettevõtluse või selline maailm 
üldse käib. Ja ega tegelikult ei teadnud seda ka, need suured inimesed, kes sel 
ajal ettevõtluses nagu  olid. Aga mingi tegevus toimus ja siis osade  
täiskasvanud näiteks metalliäri alal mõningast mõningase kogemusega või 
sidemetega inimeste abil õnnestus meil tõesti see Kosmonaudi mäng müüa Rootsi. 
See oli selles mõttes muidugi pöördeline sündmus, et me saime selle eest 
lõppkokkuvõttes ikkagi, kui ma õigesti mäletan, siis oli see viis tuhat 
dollarit. See oli täiesti kosmiline number,  aasta oli mingi 1990 ja ja  rubla 
kurss oli seal selline, et vist kui ma õieti mäletan, ühe dollari eest sai 
kolmkümmend rubla juba. Ja siis kui arvutad kokku, siis see viis tuhat dollarit 
oli ikkagi umbes selline summa, mis meie vanemad olid elu jooksul teeninud või 
midagi umbes sellist. Loomulikult inflatsiooniga võib seal seal korrutada ja 
korrigeerida, aga ikkagi  oluline number, ikka väga-väga oluline number. Kas 
nüüd mõelda, et kui õigesti me seda summat  kasutasime, kokkuvõttes sa ikkagi 
ka valuutapoes käidud ja Coca-Colat ostetud, selle peale kulus ka ikkagi 
märgatav osa sellest ära. Aga mina ja Jaan ostsime enda endale näiteks kahe 
peale arvuti. Selle peale läks pool sellest  minu ja Jaani osas sellest rahast 


Selle raha me saime kätte kuskil, see oli juba üheksakümnendate alguses,  Eesti 
kroon oli just tulnud või tulemas. Sellega on selline lugu kohe, et see on just 
täpselt see aeg, kus Eesti kroon tuli niimoodi, et minul oli see raha rubladena 
käes. Ja siis oli mingi kooperatiiv või firmakene, kust me siis olime kokku 
lepitud ja välja valitud mingi 386SX protsessoriga arvuti ja me olime seda siis 
ostmas. Ja siis ma mäletan, see oli see hetk, kus meil oli teada, et järgmine 
päev on siis see rahareform ja minul olid need rublad käes, kümnete tuhandete 
kaupa neid rublasid, mis oli selle arvuti jaoks mõeldud. Ja meil oli kokku 
lepitud, eks ole, et me anname need nii palju rublasid ja saame siis selle 
arvuti. See kooperatiivitegelane, kellele helistasin, siis ütles midagi, et too 
homme see taha või midagi niimoodi. Aga siis mul ikka nii palju oidu oli, et ma 
ütlesin, et ei, on kokku lepitud, ma toon täna selle raha. Ja ma tõingi täna 
selle raha ja ta võttis selle täna vastu ja me saime selle arvuti kätte umbes 
homme või midagi, või veel samal päeval. Nii et jah, ma ei tea, mis oleks 
juhtunud, kui oleks tegelikult üritanud homme selle raha  maksta. 

\question{Eks ajalugu oleks läinud tonks teistmoodi. Aga see oli juba 386, mis 
oli juba päris korralik aparaat. Sinna vahele jääb ju õige mitu aastat 
puselemist mingite teiste inimeste arvutite juures. Kus te selle mängu 
kirjutasite? Kodus kellegi juures?}

Mängu me kirjutasime suurel määral tegelikult Jaani\index[ppl]{Tallinn, Jaan} 
ja Priidu\index[ppl]{Kasesalu, Priit} töökohas. Sest Jaan ja Priit keskkooli 
kõrvalt töötasid programmeerijatena ühes kooperatiivis. Mina tegelikult ka 
töötasin keskkooli kõrvalt programmeerijana poole kohaga minu vanemate töökohas 
ehk Küberneetika Instituudis\index{Küberneetika Instituut}. Aga  ütleme 
niimoodi, et ma arvan, et  minu vanemate tööandja oli selles mõttes mõistlik. 
Kui ma ise tööandjana mõtlen, et kui mingisugune seitsmeteistaastane poiss 
tahab tööle tulla,  alles õpib programmeerima või niimoodi, et ega esiteks ma 
ei maksaks talle väga palju või ma ei võtaks teda nagunii väga tõsiselt.  
Samuti ma võib-olla ei annaks talle nii palju mingeid võimalusi, ma vast ei 
annaks talle missioonikriitilisi asju. 

Aga Jaan ja Priit olid, olid tööl ühes kooperatiivis, kus nemad olid vaata et 
sihukesed peaaegu et juhtprogrammeerijad või midagi niimoodi.  Ja neil oli 
tunduvalt paremad võimalused  käes. Mis on noh, tänapäeval vaadates, ma ütleks, 
ikkagi küllalt ebamõistlik, aga need olidki  ebamõistlikud. See tähendas, et 
nad ei saanud oma arvuteid nii-öelda töölt koju kaasa võtta, aga neil oli 
tegelikult töökoht, kus nad said päeval  olla koolis, aga õhtud-ööd said olla 
arvutis. Ja sel ajal,  kui sa oled kuusteist ja seitseteist, siis võid vastu 
pidada niimoodi, et magad kuus tundi päevas, siis kui vaja.

\question{Kui ma nüüd kokku loen, siis te käisite Gustav Adolfi Gümnaasiumis, 
mis polnud lihtne asi, te töötasite programmeerijatena ja takkapihta 
kirjutasite mängu, mille kannatas pärast maha müüa. Kõike samal ajal?}

Jah, peab ütlema küll,  et vähemalt siis, kui mina töötasin programmeerijana, 
ma töötajana ei ole uhke tööpanuse üle, mille ma Küberneetika Instituudile 
andsin\index{Küberneetika Instituut}. Tõsi küll,  ma sain ikkagi midagi valmis 
ja mu tööandja oli rahul sellega. Ma ei olnud ka tegelikult ainus, oli natukene 
teisigi selliseid õppijaid ja mõni üliõpilane, kes oli seal niimoodi tööl ja ma 
sain isegi aru, et mu tööandja isegi oli pigem minuga rohkem rahul kui seal 
mõnede teistega. Aga ma arvan, et see ütleb rohkem nende teiste kohta kui 
minu kohta. Mina ikkagi kulutasin enamiku ajast selle mängu ja koolis käimise 
peale.

\question{Sel ajal hakkasid tekkima esimesed BBSid ka?}

BBSid hakkasid tekkima ja nii-öelda minu tutvusringkonnast siis Priit 
Kasesalu\index[ppl]{Kasesalu, Priit} oli see põhiline, kes meie kambas tegeles 
BBSidega ja ühe ka püsti pani, mille nimi oli \emph{Dark Corner}\index{BBS!Dark 
Corner}, kui ma õigesti mäletan. Ja mille Fido, kuidas seda siis nimetati, 
\emph{node} number või midagi sellist, oli, kui ma õieti mäletan, neliteist. Ja 
teda tõmbas nagu see pool kuidagi rohkem või kuidagi väga palju ja eks 
kindlasti mind ka, sest BBSiga tekkis järsku  võimalus  ekraani kaudu suhelda 
hästi paljude teiste inimestega, kellega sa võib-olla füüsiliselt ei istu koos. 
Teatud mõttes võiks isegi öelda, et järsku nendele inimestele anti natuke nagu 
Facebook kätte. Mitte taskusse otseselt, aga ikkagi kätte või niimoodi, et 
järsku tekkis hulk sõpru, kellega ma olin suhelnud ainult interneti teel. Ja 
Fidos vahetati mõtteid  kõikide asjade üle, mitte ainult arvutite üle ja tekkis 
järsku üks mingisugune  täiesti isevärki sotsiaalne seltskond. Tolle aja aja 
kohta oli see väga isevärki sotsiaalne seltskond. Tänapäeval on niimoodi, et 
sotsiaalne seltskond, kes on mingi Facebookis mingisuguse grupi, olgu mingi 
MMSi klubi või ma ei tea mis, liige,  siis nad võivad aeg-ajalt kokku saada. 
Netiajastul on see tegelikult väga-väga tavaline. Aga selline selline, kuidas 
öelda elustiil või tutvusringkonna ülesehitus järsku tekkis  meile kätte, kui  
aasta oli umbes 1990 või umbes kuskil sealkandis.

\question{See seltskond pidi siis olema ka teatavas mõttes homogeenne, sest 
Fido külge saamise barjäärid olid kõrged?}

Jah, eks muidugi oli palju ka inimesi, kes nii-öelda jõlkusid kaasas. Olid 
sellised entusiastid nagu näiteks Priit Kasesalu\index[ppl]{Kasesalu, Priit} 
või Tarmo Mamers\index[ppl]{Mamers, Tarmo} näiteks no nende muud sõbrad  
aeg-ajalt tekkisid ju ka sinna sisse, kellele siis  Tarmo või Priit võimaldasid 
ligipääsu. Ja see oli kindlasti väga huvitav. Tekkis selline  sotsiaalne 
distants-suhtlus. 

Ma mäletan ühte juhtumit, oli juba tegelikult siis, kui vaikselt Internet juba 
hakkas Eestisse tekkima. Internet kui selline tehniliselt oli juba olemas juba 
ju kuskil seitsmekümnendatel kaheksakümnendatel, aga  Eestisse  ta umbes sel 
ajal niimoodi natukene juba tekkima. Mul oli selline sõber, siiamaani väga hea 
sõber, nimega Sulo Kallas\index[ppl]{Kallas, Sulo}, kellel oli ka BBS ja kes 
töötab minuga koos Starshipis\index{Starship Technologies} praegu. Tema andis 
mulle kasutada ühte oma kontot ühes Unixi arvutis. Ja Unixis oli olemas selline 
programm nagu \verb|talk|, kus sai siis omavahel ekraani kaudu suhelda 
inimesed, kes olid sisse loginud samasse masinasse. Ja ma mäletan, et  minu 
jaoks oli üks ikkagi täiesti selline silmi avav  elamus.  Mul ei olnud sel ajal 
kodus telefonigi. Ja siis ma midagi toimetasin selle Sulo kontoga Sulo nime alt 
selles ühes arvutis ja järsku selle \verb|talk|iga  hakkab minuga keegi 
rääkima.  Ütleb, et minu nimi on Epp. Nii, ja mina siis esimese asjana, kuna ma 
teadsin, ma kasutatud Sulo kontot, eks ole, keegi Epp tahab Suloga rääkida. 
Siis ma selgitasin talle, et kuule, mina ei ole Sulo, et mina olen hoopis üks 
teine inimene. Tema ütleb vastu, et  sellest pole midagi, räägime ikka. Ma ei 
saanud täpselt aru, mis värk on nagu, mis mõttes, ta ju tahab Suloga rääkida, 
eks ole. Aga siis ma sain aru, et ta tahab tegelikult lihtsalt kellelegi 
rääkida, et tal  tegelikult on täitsa okei, et ta räägib  minuga. Sihuke 
jutuajamine tekkis sealt, ja ma sain teada selle jutuajamise käigus, et  
tegemist on ühe Eesti tüdrukuga, kelle nimi on Epp ja kes hetkel füüsiliselt 
asub Ameerikas. Ja ta läks Ameerikasse  ülikooli õppima, ta oli Ameerikas 
üliõpilane. Ja mina istun Eestis, eks ole, ja ma reaalajas räägin arvuti 
ekraani vahendusel  temaga juttu, eks ole. Me rääkisime maast ja ilmast 
mingisugune tund aega, see oli  väga-väga kummaline kogemus. Sa  suhtled 
kellegagi reaalajas, kes on nagu sinust väga-väga kaugel. Ma siiamaani ei tea, 
kes Epp täpselt oli, ta ütles oma perekonnanime ka, ma ei ole seda nime mitte 
kunagi hiljem kuulnud, mitte kunagi hiljem selle inimesega suhelnud. Aga see 
oli ikka väga kummaline kogemus minu jaoks. Ongi naljakas tegelikult et, 
tänapäeval ju selline asi on ju niivõrd tavaline, kõigil mingid Snapchatid ja 
asjad on kuskil taskus, eks ole. Ja tol ajal oli sotsiaalses mõttes see, et sa 
võid suhelda inimestega kuskil üle maailma,  oli nendele interneti häkkerite 
võimalik ja teistele inimestele ei olnud.

\question{Sa rääkisid, et BBSides räägiti igasugustel teemadel. Näiteks, 
millest räägiti?}

Kui ma õieti mäletan, seal oli igasugust, sellist elulist, nagu tänapäeva 
internetifoorumid, eks ole. Kõigest võidakse seal rääkida. Seal oli mingisugune 
filosoofiateemaline  vestlusgrupp, kus  inimesed olid ju enamasti sellised 
kaheksateist aastased, kes veel mõtestavad oma elu. Ongi selline aeg inimeste 
elus, kus kõik mõtlevad, mida tähendavad mingisugused asjad ja kas ikka inimene 
peaks panustama sellele või tollele. Tänapäeval neljakümneaastasena väga 
võib-olla ei viitsi sel teemal juttu vesta väga, kõigil on juba oma elu 
tõekspidamised välja kujunenud, aga tol ajal minul kindlasti ei olnud ja enamik 
sellest ülejäänud BBSi seltskonnast oli ka umbes sama vanad, eks ole. Siis oli 
seal igasuguseid psühholoogiateemalisi, neid vestlusi oli igasuguseid, see 
kindlasti ei olnud sugugi mitte ainult tehnoloogiateemaline. 

\question{See, mis sa ütled, kõlab väga oluliselt. Sest see tähendab, et 
mingisugune ports nutikaid inimesi mitte üksinda ja mitte juhuslike inimestega 
vaid koos sama moodi mõtlevate ja samade oskustega inimestega mõtestasid seda, 
mida tähendab olla inimene kõige laiemas mõttes}

Absoluutselt. See oli tegelikult üks niisugune virtuaalne sõpruskond.  Võib 
olla võib öelda, et see Fido seltskond oli kõige esimene virtuaalne sõpruskond 
Eestis üldse. Tänapäeval on  igaühel virtuaalseid sõpruskondi taskus sada tükki 
aga see võis olla võib olla täiesti esimene.

\question{Kas selle kõige juurde käis ka mingi spetsiifiline raamatu-, muusika- 
või filmihuvi?}

Ahaa, muusikakanaleid oli loomulikult ka, muusikateemalisi  vestlusgruppe. 
Minul ei käinud. Võib-olla natukene. Ma arvan, et  selles ringkonnas pigem olid 
populaarsed sihukesed elektroonilise muusika bändid. Nii, ja naa, ütleme. 
Kraftwerk mulle ei meeldinud ja ei meeldi siiamaani, Jean-Michel Jarre samuti 
mitte nii väga palju aga Tangerine Dream näiteks meeldis mulle väga ja 
siiamaani meeldib, mul on ikka mingi viisteist nende plaati ja nii edasi. Aga 
samas jälle ma olen inimene, kes ei ole kunagi vaadanud Star Warsi, ma ei ole 
kunagi lugenud \emph{Hitchhiker's Guide to The Galaxy}'t. Minu jaoks  on 
esteetiline subkultuur ja arvutid natukene lahus seisnud.

\question{Endal sul BBSi ei olnud?}

Minul endal BBS-i olnud. Ma vist nagu kuidagi ei tahtnud ka, see oli ikka hull 
jahmerdamine, mis oli vajalik selle BBSi üleval hoidmiseks ja sellega pidevalt 
toimetada. Mul oli väga hea meel, et ma sain  Priidu BBS-i kasutada.

\question{Selge. Aga siis te müüsite selle mängu maha, mis edasi sai?}

Noh, kui üheksateistaastasele inimesele anda nii palju raha, nagu tema vanemad 
on kogu elu jooksul teeninud, eks ole, siis tal karjäärivalik on nagu selge 
kohe, eks ole ju. Et noh, sellist küsimust nagu ei olnud, et mida ma siis 
tulevikus professionaalselt tegema hakkan. Loomulikult programmeerija.  Ja  mul 
oli ka selline mõtlemine, ma ei tea, kui õigustatud see oli, aga ma arvasin, et 
et noh, eriti üheksakümnendate alguses Eesti ülikoolides eriti midagi väga 
kasulikku sel teemal ei õpetatud. Ma ei tea,  kui õige või vale see on. 
Kindlasti vastas tõele see, et meil keskkoolis oli  arvutiõpetus ka ja üldiselt 
ikkagi meie klassist pigem paljud teadsid rohkem kui meie õpetaja. Ma 
miskipärast oletasin, et ülikool siin samamoodi, ma ei tea, kas see on tõsi või 
mitte. Tänapäeval see kindlasti ei ole tõsi aga  tol ajal  võib-olla pigem oli. 
Igatahes ma tegin selle otsuse, et ma ei lähe ülikooli õppima midagi 
programmeerimise või arvutitega seotut, vaid ma läksin hoopis õppima füüsikat. 
Füüsika oli kindlasti mul  niisugune teine selline huviala,  ma olin 
füüsikaolümpiaadidel käinud  ja mulle see kindlasti kindlasti väga meeldis. 

\question{Aga mis sulle füüsika juures meeldis?}

No võib-olla natuke sihuke filosoofiline aspekt, et ma sain kuidagi aru, kuidas 
nii-öelda maailm toimib teatud mõttes. See oli põnev. Mingid sihukesed asjad, kui 
 mingid tuumafüüsikad ja mingid planeedid, kuidas liiguvad ja niimoodi, see 
natukene andis võib just sellist filosoofilist mõõdet. Mis see maailm meie 
ümber on ja kui suured või väikesed meie, inimesed, selles maailmas  oleme.  Ja 
noh, pigem ikkagi väga väikesed oleme. 

\question{Kuhu sa läksid seda füüsikat õppima?}

Ma läksin  füüsikat õppima Tartusse\index{Tartu Ülikool}, koos Jaan 
Tallinnaga\index[ppl]{Tallinn, Jaan}. Pinginaabrid läksid mõlemad õppima 
füüsikat. Sellega läks niimoodi, ma kindlasti  tegelikult ei väärtustanud seda, 
et piltlikult öeldes mul oleks paber taskus, et mul ülikooli oleks  kuidagi 
edukalt lõpetanud. Ja kui ma olin ühe aasta või poolteist aastat ülikoolis ära 
olnud, siis mulle hakkas veel rohkem kohale jõudma see, et tegelikult ma ju 
tegelen programmeerimisega kogu aeg, töötan professionaalse programmeerijana. 
Samal ajal tegi mind järgmist mängu, mille me kavatsesime maha müüa ja nii 
edasi ja nii edasi. Ja ma kunagi ei kavatsenud füüsikuna töötada, ma hobi 
korras õppisin füüsikat. Kui esimese aasta sai hobi korras  füüsikat õppida 
siis teisel aastal hakkad aru saama, et tegelikult  õppejõud ikkagi eeldavad, 
et sa tõsiselt tegeled selle asjaga, panustanud enamiku oma ajast füüsiku 
õppimisse. 

Ja siis ma tulin ülikoolist ära. Ma sain aru, et see asi lihtsalt nõuab rohkem 
tööd, kui ma olen nõus sinna sisse parema ja siiamaani ma ülikooli lõpetanud ei 
ole. Jaan Tallinn\index[ppl]{Tallinn, Jaan} käis ülikooli lõpuni ja õppis 
füüsika siis siis lõpuni. Tegi oma oma lõputöö, kui ma õieti mäletan, 
relatiivsusteooriast. Sellest, kuidas ruumi painutada selle jaoks, et reisida 
valguse kiirusest suuremate kiirusega ühest kohast teise. Ma küll oletan, et 
tõenäoliselt  ta mingisugust väga suurt teadmist ühiskonnale sellega juurde nii 
väga ei lisanud selle nelja aastaga, mis ta õppis aga sellise töö ta tegi. Ta 
on rääkinud, et ükskord, kui ta kuskil seltskonnas kirjeldas oma seda tööd, 
mida ta tegi, siis tema vestluskaaslane küsis  vastu, et kas see oli nagu 
rohkem teoreetiline töö või tuli seal ka mingeid praktilisi laboratoorseid 
katseajale.

\question{Selle asja nimi, mida te tol hetkel kampas pidasite, oli juba 
Bluemoon\index{Bluemoon}?}\label{sisu!bluemoon}

Jah. See mängutegijate punt, me hakkasime ennast nimetama nimega Bluemoon 
Software ja Bluemoon Interactive. Inimesed ikka tahavad panna mingisuguseid 
kõlavaid firmanimesid.

\question{Aga miks just Bluemoon>}

Lihtsalt oli üks nimi. Ma arvan, et me ei osanud nimesid üldse välja mõelda ja 
ma olen kogu aeg pidanud ennast väga halvaks nimede väljamõtlejaks ja et ma ei 
valda seda valdkonda üldse ja niimoodi, aga kui Starshipile\index{Starship 
Technologies} nime panin, siis ikkagi osalesin selles kõvasti ja  lõpuks oli 
ikkagi minu pakutud nimi, mis selleks lõpuks sai.

\question{Programmeerimise juures pidi olema täpselt üks raske asi, nimede 
välja mõtlemine}\sidenote{Eksin tsitaadiga. Täpne tsitaat on 
Netscape\index{Netscape} arhitekti Philip Karltoni\index[ppl]{Karlton, Philip} 
poolt ja kõlab nii: \enquote{\emph{There are only two hard things in Computer 
Science: cache invalidation and naming things}}}

Ma olen täitsa nõus sellega, võib-olla nüüd neljakümne aastasena on juba 
natukene rohkem käppa seda saadud. 

\question{Mis sa praegu teed?}

Praegu ma olen sellises firmas nagu Starship Technologies ja ehitan 
pakiroboteid. Asutasime selle selle firma koos Skype'i\index{Skype} kaasasutaja 
Janus Friisiga\index[ppl]{Friis, Janus}  neli pool aastat 
tagasi\sidenote{Intervjuu Ahtiga toimus jaanuaris 2019}. Ja meil oli selline 
visioon, et asjad võiksid ju maailmas liikuda automaatselt samamoodi, nagu 
elekter tuleb meile stepslisse seina ise sisse ja veevärk on olemas ja 
informatsioon tuleb läbi interneti. Aga asjad liiguvad ikkagi  läbi meie maja 
või korteri ukse, tulevad füüsiliselt kohale ja alati mingisugune inimene toob 
seda, kas sa ise tood või siis sa maksad kellelegi inimesele, kes toob. Ja see 
on hirmus raiskav ja asjad võiksid liikuda automaatselt samamoodi nagu me 
lennukipileteid broneerime üle interneti nii öelda automaatselt, ilma et me 
läheksime füüsiliselt kohale kuskile reisibüroosse seda lennukipiletit ostma.

\question{Starshipi tegemine on ju juhtimise töö. Kuidas sa jõudsid 
programmeerimise juurest selle töö juurde, mida sa praegu teed ja kui erinevad 
nad sinu jaoks on?}

No need on ikka väga erinevad. Minu jaoks on see areng olnud selline, et ma 
olin programmeerija ja ma olin programmeerija üsna kaua aega, ilma et ma oleks 
üldse midagi kuskil juhtinud. Ja kui me hakkasime startuppe tegema koos Jaanus 
Friisi ja Niklas Zennströmiga\index[ppl]{Zennström, Niklas} siis ma olin 
nendest startuppides tehnilise arhitekti rollis. Arhitekti roll on juba rohkem 
natukene nagu  juhtimisega seotud, aga sa ei juhi nii väga  inimesi või 
organisatsioone või protsesse, vaid sa juhid just tehnilist arhitektuuri. Et 
milline see masin niisuguses suures plaanis kokku tuleb, mida siis terve suurem 
tiim inimesi ehitab. Nagu maja ehitamisega: osad inimesed ehitavad ja panevad 
kive üksteise peale ja on ka teisi inimesi, kes vaatavad seda projekti 
suuremalt, et kus peaks olema paneme aken ja mitu akent me üldse teeme ja kas 
me teeme rohkem ümmargused aknad või teeme kandilised aknad ja nii edasi ja nii 
edasi. Ja ma olin Skype'is, olin siis tehniline peaarhitekt alguses  ja 
mitmetest teistes startuppides samuti. Skype'is veel natukene pooleldi juhtisin 
ka ühte väikest tiimi, kus  ma tegelesin sellega, et mõelda umbes viiele 
inimesele välja seda, mida nad tegema peaksid ja koordineerida nende tööd. 
Mõtlesin välja, mis meie eesmärk peaks olema, kuhu poole me peaksime liikuma ja 
nii edasi, nii edasi. Sihukene viie inimeselise tiimi juhtimine oli selline 
nagu väike harjutus või  sissejuhatus, et mingisuguseid kogemusi natukene sain 
või natukene kujutasin ette. Hiljem olen juhtinud siis ka natuke suuremaid 
tiime, umbes kümneinimeselisi ja niimoodi. Aga Starship oli esimene koht, kus 
ma üsna kiiresti võtsin tööle kümme inimest, võtsin esimese kahe nädalaga tööle 
umbes ja esimese poole aastaga oli juba umbes kakskümmend inimest meil tööl ja 
nii edasi  läks juba natukene suuremaks see asi. Eks ma niimoodi käigu pealt 
natukene siis  õppisin, et  kuidas juhtimine käib. Ju ma olen kindlasti veel 
üsna  alguses seal, et me oleme siin Starshipis olnud sihukeses  naljakas 
olukorras, kus nagu juhtimises ikkagi üsna kogenematu juht on olnud sellel 
firmal. Neli aastat ma olin tegevjuht ja  nüüd jõudis pool aastat tagasi siis 
asi nii kaugele, et me palkasime  professionaalse tegevjuhi Lex 
Bayeri\index[ppl]{Bayer, Lex} Californiast. Ja mina olen CTO ehk 
tehnikadirektor, kus ka peab üsna palju juhtima, aga nüüd enam mitte kahtsadat 
inimest, vaid natukene väiksemat hulka inimesi.

\question{See on siis olnud pikk ja just vajadusest ja huvist kantud õppimine?}

Jah, absoluutselt.  Üldiselt ma ütleks niimoodi, et paljud programmeerijad, 
kaasa arvatud ka mina, meile programmeerimine meeldib nii palju see on niivõrd 
tore tegevus ja niivõrd äge tegevus, et selliseid masinaid ehitada, et tahaks  
muudkui eitada neid masinaid. Inimeste juhtimine on pigem selline asi, mida 
enamik programmeerijaid väga ei taha teha ja ma ei ole päris kindel ka ise, kui 
palju mina seda tegelikult teha tahan. Aga küll on lihtsalt asi selles, et kui 
sa oled  üksikprogrammeerija ja sul kogemus tekib ja sa oled  arhitekt ja sa 
oskad juba rohkem  arvata, mismoodi me seda tarkvara peaksime ehitama ja mis 
asjad on selle juures olulisi, mis need ei ole. Siis on nagu on kaks võimalust, 
kas sa  oled vait ja kellegi teise juhtimisel osaled selles protsessis või siis 
sa üha rohkem nagu vaatad seda, et ei, ma teen ise, ma teeksin seda paremini 
kui see juht, kes meil on. Ma tahaks ise seda asja juhtida või mul on juba nii 
hea ettekujutus, kuidas seda teha, et ma ei suuda pealt vaadata, kui 
mingisugune teine inimene, kes on võib-olla väiksema kogemusega kui mina,  
kuidagi seda asja juhib ja mitte selles suunas, kus mina olen täiesti 
veendunud, et  õige oleks. Ehk siis see on tulnud justkui nagu vajadusest. Kui 
sa oled üksikprogrammeerija, siis sa aja jooksul ikkagi saad aru, et sa saad 
tegelikult lõppkokkuvõttes rohkem tehtud, kui sa piltlikult öeldes palkad 
endale tiimi ja hakkad juhtima mingisuguseid suuremaid seltskondi. 

Minu jaoks küll nii-öelda raketiga lendamine nagu Starshipis, kümneinimeselise 
tiimi juhtimisest kuni selleni, et ma juhtisin üle kahesaja inimesega firmat 
tükk aega, see ikkagi võttis pea ringi käima. Et ma kindlasti kindlasti 
edutasin ennast  oma ebakompetentsuse tasemele. Aga eks kohati öeldaksegi, et 
starupid ongi asjad, mis on väga sageli on  klassikalise sellise juhtimise 
distsipliini ja teooria ja juhtimispraktikate mõttes väga halvasti juhitud 
organisatsioonid. Mis ei ole siiski tihti takistuseks olnud nende edule, 
sellepärast et nad on olnud nii piisavalt värske mõtlemisega, nende toode on 
olnud piisavalt selline värske ja revolutsiooniline, et sellest ei ole olnud 
hullu, et nad on olnud halvasti juhitud. Tegelikult ikkagi need kakssada 
inimest, kes meie Starshipis töötavad,  ma ikkagi vaatan nende peale küll nagu 
niimoodi, et palun vabandust nende ees, et nad on osalenud sellises loomkatses, 
et mina olen neid juhtinud mitu aastat. See ei ole võib-olla olnud aus nende 
suhtes. Aga samas nad ei ole ka sugugi mitte meil siin firmast minema jooksnud 
ja tunduvad olevat rahul, et võib olla väga hullusti, siis ei olegi läinud.


\chapter{Madis Kaal}
\label{cptr:mast}
\index[ppl]{Kaal, Madis}
\index[ppl]{Mast|see{Kaal, Madis}}

\question{Kuidas arvuti Saaremaale sai?}

Arvuti ei saanudki Saaremaale. Minu esimene kokkupuude päris arvutitega oli 
Rahvamajanduse Saavutuste Näitusel\sidenote{Tänapäeva mõistes oli tegu 
messikeskusega, kus ajutistel või püsinäitustel demonstreeriti 
kas liiduvabariigi (nagu Tallinnas asunud näituse puhul) või kogu Nõukogude 
Liidu majanduslikku võimekust. NSVLi Rahvamajanduse Saavutuste Näitusest arenes 
välja Eesti Näituste Messikeskus.}, praeguses Pirita näitusehallis. Käisin 
seal koos oma emaga. Ühes nurgas olid üles pandud 
terminalid, mida manageeris kaks imeilusat tüdrukut. Seda siis ajaloolisest perspektiivist, tõenäoliselt oli tegemist üsna keskmiste 
operaatoritega, aga siis tundusid nad imeilusad ja targad. Terminalide peal oli 
nõukogudeaegne venekeelne raamatukogude otsingu andmebaas. Terminalid ise olid ka 
venekeelsed. See oli esimene kord, kui ma reaalselt nägin, et ekraanil olid 
tähed ja klaviatuuril sai kirjutada. 

\question{Mis aastal see oli?}

Arvatavasti 1983. Ja need terminalid jätsid kustumatu mulje. 

\question{Kas pärast seda tekkis sul selge soov terminalide 
juurde pääseda?}

Pärast seda tekkis väga selge mõte, et see asi huvitab mind. 
Seejärel sattusin Tartusse ja ostsin sealt venekeelse 
raamatu \enquote{Programmeerimine keeles PL/I\index{PL/I}} ning lugesin 
seda. Ma ei teadnud arvutitest veel midagi, aga tasapisi hakkas selgeks saama, 
misasi on programmeerimine ja näiteks \verb|for|. See oli mingi imeline struktuurkeel, mitte päris vene, 
vaid kõlas nagu piraatversioon.

Järgmine kord nägin arvuteid Tehnikaülikooli\index{Tallinna 
Tehnikaülikool}, tolleaegse TPI\index{TPI} lahtiste uste päeval, kus me käisime 
pinginaabriga, kellega koos pärast ka kooli sisse astusime. 
Meile tehti ekskursioon automaatikateaduskonna kõigis 
kateedrites\index{Tallinna Tehnikaülikool!Automaatikateaduskond} neljal 
korrusel ja mõnes kohas olid arvutid. Mäletan selgelt, et Indrek 
Saul\index[ppl]{Saul, Indrek}, kes oli minu meelest sel ajal tudeng ja hiljem 
kinnisvaraärimees, näitas meile analoogarvutit. Sellega
sai analoogpingete ja skeemiga diferentsiaalvõrrandeid lahendada.

\question{Vanasti sihiti ju õhutõrjekahureid analoogarvutitega.}

See masin võis täiesti olla sedalaadi projekti osa. Igatahes mul tekkis kindel soov seda valdkonda
õppima minna, aga pinginaaber veenis mind ümber, et lähme parem 
raadiotehnikasse, ikkagi sama maja.

\question{Kas esimest korda arvuti nägemise ja ülikooli sisseastumise vahele jäi veel 
midagi arvutitega tegelemise mõttes?}

Ainult see üks raamat. Otsus arvuteid õppima minna sündis esimesel korral ja raamat tuli 
pärast seda. Ainuke imelik asi oli otsus raadiotehnikasse 
minna, aga selle vea parandasin ruttu ära. Ülikooli teise korruse otsas oli arvutussaal, kus oli kaks või 
kolm SM-4\index{SM EVM!SM-4}. Need olid PDP-11\index{PDP-11} vene 
versioonid. Pärast seda, kui sain aru, kuidas sinna sisse saab, ma enam 
tundidesse ei jõudnud. Ja kuna olin maalt tulnud poiss ja raha ka üldse ei olnud, 
käisin lihtsalt kõik kateedrid läbi ja küsisin iga ukse vahelt, kas neil on tööd anda. Raadiotehnika kateedris\index{Tallinna 
Tehnikaülikool!Automaatikateaduskond!Raadiotehnika kateeder}\label{sisu!mast_raadiotehnikas} oli, 
mind võeti sinna laborandiks tööle ja nii see läks. Kool jäi pooleli, kateedrisse
jäin seitsmeks aastaks paika.

\question{Mitmendal kursusel kool pooleli jäi?}

Esimesel kursusel. Algul olin raadiotehnika kateedris laborant ja pärast 
tehnik. Sattusin tuppa, kus olid väga toredad inimesed: Mart 
Palmas\index[ppl]{Palmas, Mart}, kes õpetas mulle peaaegu kõike, mida ma 
programmeerimisest tean, ja Villem 
Vannas\index[ppl]{Vannas, Villem}, kes praegu töötab Datelis\index{Datel}. Tema 
õpetas mulle enam-vähem kõike, mida ma rauast tean.

\question{Siis ei jäänud ju haridus pooleli.}

Formaalselt siiski jäi. Tol ajal oli
laborant rohkem nagu abitööline. 
Parandasin seda, mida vaja, aitasin seal, kus vaja. Mu esimene töö oli 
kolikamber tühjaks tõsta.
Algusaegadel oli üsna suvalisi projekte, hiljem tekkis
suund kommunikatsiooni poole, mis tundus mulle sel ajal huvitav. 

1990. aasta paiku tekkis Eestis 
mitu huvitavat suunda. Kõigepealt hakkas tulema 
personaalarvuteid. Sinnasamasse, kus oli kunagi SM-4 arvutiklass, tekkisid 
personaalarvutite klassid. Neid oli mitu tükki ja erinevate portsudena 
toodi Austraaliast MicroBeesid\index{MicroBee}\sidenote{1982. aastal 
Austraalias algselt komponentide komplektina müügile tulnud koduarvuti. Tuntud 
huvitava videolahenduse ja patareitoitel mälu poolest, mis 
võimaldas arvutit teisaldada mälu seisu kaotamata.}. Kuskilt tuli terve klassi jagu MSXi 
arvuteid\index{Yamaha MSX} ja siis mõned 
Robotronid\index{Robotron}\sidenote{Robotron (originaalis VEB Kombinat 
Robotron) oli Ida-Saksamaa suurim arvutitootja.}. 
Raadiotehnika kateedris oli juba siis, kui mina sinna sattusin, olemas 
Apple II\index{Apple II} ja mõned aastad hiljem tekkis sinna IBM 
PC\index{IBM PC}. See oli omapärane kogemus. Apple II peal olid 
harjunud, et lülitad sisse ja pilt on ees. IBMi sisse lülitades ei juhtunud midagi. Ühel hommikul tööle tulles vaatasin, et uus 
arvuti, ja lülitasin sisse. Midagi ei juhtunud. Ootasin natuke aega ja lülitasin välja, ise 
tegin näo, et midagi pole toimunud. Hiljem selgus, et masin tegi \emph{self 
test}'i. Seal oli tublisti mälu sees ja testimine võttis palju aega -- ma ei 
suutnud nii kaua oodata. 

\question{Midagi pidi see ju ekraanil senikaua näitama?}

IBMil oli roheline long-fosfor\sidenote[][-1.3cm]{Katoodkiirtel põhinevates monitorides suunati laetud osakeste kiir fosforühendiga kaetud ekraanile. Kasutatud ühendi tüübist sõltus nii elektronkiire mõjul tekkinud värv kui ka see, kui kauaks ekraan peale kiire edasi liikumist helendama jäi. Selle viimase järgi liigitataksegi ekraanides kasutatavaid fosforühendeid  \enquote{pikkadeks} ja \enquote{lühikesteks} (ingl. \emph{long} ja \emph{short}), seda Mast ilmselt silmas peabki.} monitor, mis läks tükk 
aega käima, ja ma ei jõudnud esimese \emph{boot}'imise ajal ära oodata, millal 
midagi toimuma hakkab.

Üheksakümnendate paiku tekkis meile tuhande kahesajane modem, mis läks 
PC sisse. Sel ajal olid just tulnud esimesed BBSid ja umbes samal ajal otsustas TPI 
automaatikateaduskond\index{Tallinna Tehnikaülikool!Automaatikateaduskond} 
ehitada arvutivõrgu. Toodi kohale viiesajameetrine kaablirull 
kollast sõrmejämedust Etherneti kaablit ja umbes kümmekond komplekti 
kobakaid kaabli peale, mille külge käis teine jäme kaabel, 
mis läks võrgukaardi taha. See oli nagu esimene Etherneti tehnoloogia. 
Mäletan selgelt, et meile toodi ainult kaabel ja kobakad, ei mingeid tööriistu, pistikuid ega terminaatoreid.

Kateedris oli sel ajal eterniiditahvlitest lagi, mille peale me selle Etherneti kaabli tõmbasime. Et kobakad külge saada, tegime
naaskliga kaabli kesta sisse augud, ajasime nõela läbi ja ühendasime
arvutite külge ning tinutasime otsa terminaatorid ja takistid. 

\question{Tarmo Mamers\index[ppl]{Mamers, Tarmo} rääkis, kuidas te PC ja Maci 
vahele traati vedasite. Kas too kaabeldamine oli enne või pärast seda?}

See oli meil kahe PC -- sellesama raadiotehnika kateedri PC ja Tarmo oma -- vahel, Tarmol oli 
veidi vägevam AT arvuti. Ühendasime need 
kaabliga ja tegime väikese 
\emph{chat}'i programmi, et teineteisega suhelda. 

Arvutivõrk tekkis sellest hiljem. Tehnikaülikooli toodi Novell 2.15\index{Novell} 
server, mille ma installisin ja mis oli üks esimesi väheseid asju, millel oli manuaal 
olemas, nii et kõik oli justnagu ametlik. Novelli serveri peal panin käima Pegasuse 
Maili\index{Pegasus Mail}-nimelise asja, kuhu külge kirjutasin \emph{gateway}, 
millega sai UUCP meili, mida toimetati Küberi 
majja Soomest (ma ei mäleta, kas Soomest siiapoole lükates või siit 
üle telefoniliini tõmmates). Tõmbasime selle oma pisikese modemiga 
Tehnikaülikooli majja ja jagasime kasutajate vahel laiali.

\question{Siin tundub jälle suuremat sorti lünk olema selle vahel, kuidas sulle 
hakati programmeerimist õpetama ja kuidas sa naaskliga kaablit torkisid ja 
\emph{gateway}'sid programmeerisid.}

Mõned aastad tuli õppida asjade kirjutamist lihtsalt erinevaid asju tehes ja ehitades, aega katsetamiseks oli palju. 
Olin noor inimene, peret polnud ja praktiliselt elasin raadiotehnika 
kateedris\index{Tallinna Tehnikaülikool!Automaatikateaduskond!Raadiotehnika 
kateeder}. Meil oli seal omamoodi seltskond: arvutussaali 
kamp, Tarmo\index[ppl]{Mamers, Tarmo} kohe kõrval sama 
koridori peal ja mina üleval raadiotehnikas. Vana kooli mees 
Lõvi\index[ppl]{Lõvi} oli kõrvalkorpuses ja käis aeg-ajalt Apple II peal 
oma projekte arendamas.

\question{Kas meetodiks oli siis peamiselt katsetamine, mitte 
manuaalide tudeerimine?}

Manuaale ega dokumentatsiooni ei olnud üldse. Riiklikul 
tasemel tarkvara varastamise programm pakkus küll ägedat tarkvara, aga 
enamasti ilma dokumentatsioonita. See oli nagu infovaakumis 
tegutsemine ja disassembler\sidenote{Programm, mis teeb masinkoodist 
oluliselt loetavamat Assemblerit.} oli justkui sõber.

\question{Keegi pidi sulle ju ometi ütlema, et selline asi nagu disassembler 
on olemas.}

Jaa, seda tegid head vanemad kolleegid, kes hoidsid kätt ja 
juhendasid. Lõviga\index[ppl]{Lõvi} tegutsesime pikalt koos, temal oli kindlasti 
väga suur mõju minu arengule. Aga see lünk, kuidas ma BBSideni 
jõudsin, sai täidetud nii, et mul oli raadiotehnika kateedris\index{Tallinna 
Tehnikaülikool!Automaatikateaduskond!Raadiotehnika kateeder} 
arvuti, mille sees oli modem ja millega sai helistada. Lähim BBS
asus Küberneetika Instituudi otsas, kus tollal asus 
Proekspert\index{Proekspert} ja kus nüüd on Tehnopoli kontor. Andrus Suitsu\index[ppl]{Suitsu, Andrus} 
oli BBSi mees, käisin tema juures oskusteavet ja tarkvara 
hankimas. Panin algul BBSi ja peatselt pärast seda ka Fido, algul vist
\emph{point}'i, ja käitasin seda üsna mitu aastat. 

\question{Miks sa seda tegid?}

Huvist kommunikatsiooni vastu.

\question{Kas sa mõtled kommunikatsiooni masinate või inimeste vahel?}

Mõlemat. See moment, kui täielikust infopuudusest saab järsku täielik 
infovabadus, on väga ergastav. Tänapäeva inimestel, kellel on internet olemas, ei 
kujuta ette, kuidas saab olla ilma, aga ilma oli väga pime.

Üks asi oli tehniline info, aga Fidoneti ja Useneti grupid
(UUCP meiliga koos toodi ka Useneti gruppe) olid ka muidu väga 
huvitavad. Sealsed diskussioonid olid väga 
informatiivsed. Suurem osa 
juttudest olid muidugi tehnilised, sest seal käisid tehnikud ehk need, kes 
said kanalile ligi.

\question{Kas too kollase kaabliga võrk hakkas tööle ka?}

Ikka, see töötas uhkelt. Novelli server käis veel 1992. aastal, kui ma 
sealt ära läksin. Inimesed said omavahel meilida ja ka välismaailmaga 
suhelda. Ainukene probleem oli see, et arvuteid, millel oli see 
Etherneti äge \emph{interface}, oli suhteliselt vähe, paar tükki kateedri 
peale vist suudeti tekitada.

\question{Kas Etherneti kaart oli defitsiit?}

Tol ajal oli kõik defitsiit, siis oli veel rublaaeg. Millise projekti 
raames see toodi, ei tea. Avo Ots\index[ppl]{Ots, Avo} tegi minu meelest 
doktoritöö selle kohta, kuidas ehitada arvutivõrku. See oli 
oluline kogemus, et toimuks järgmine samm. Pärast tehnikaülikooli 
töötasin lühikest aega Microlinkis\index{Microlink}, kus ma olingi 
arvutivõrkude installeerija ja ühtlasi .EXE\index{.EXE} kirjutaja.

\question{Miks sa sinna läksid?}

Ühel päeval astus uksest sisse Margus 
Kliimask\index[ppl]{Kliimask, Margus}, keda ma teadsin Rainer 
Nõlvaku\index[ppl]{Nõlvak, Rainer} kaudu, ja tegi ettepaneku hakata 
tegema ajakirja. Sellest sai .EXE.

\question{Miks ikkagi? Jälle kõlab suure muutusena, et ühel päeval tõmbasid kollast 
kaablit ja järgmisel päeval tegid ajakirja.}

Täpselt nii oligi. Ma arvan, et Rainer tahtis Microlinki promo teha. 
See võis olla suur motivaator, aga seda peab Rainerilt endalt küsima.

\question{Kust sul üldse tuli mõte, et ajakirja tegemine võiks huvi pakkuda?}

Tundus huvitav. Mul ei olnud siis rohkem kõrgeid eesmärke kui see, et elu oleks 
huvitav.

\question{See on tegelikult kõige kõrgem eesmärk, mis üldse saab olla.}

Algul oli jutt, et teeme ajakirja, ja siis selgus, 
et mul oleks ka uut töökohta vaja. Nii sattusingi korraks Microlinki\index{Microlink}. Olin seal aga loetud kuud, sest 
siis hakati tegema Eesti Forekspanka\index{Eesti Forekspank}\sidenote{Eesti 
Forekspank sündis 1992. aastal ja ühines 1995. aastal Raepangaga\index{Raepank} 
1995.}. Pangal olid oma sidevajadused ja mind kutsuti sinna tööle.

\question{Üheksakümnendate algus oli Eesti panganduses ju hull aeg!}

Jah, ja Forekspank oli sel ajal pisikene valuutavahetuskontor, mis opereeris 
rubla-dollari börsi.
See tegutses tolleaegses hulgifirmas Abestok\index{Abestok}. Selle ühes toas olid 
inimesed, kes otsustasid panga teha. Margus 
Kliimask\index[ppl]{Kliimask, Margus} oli nendega seotud, vist IT-poisi 
staatuses. Temaga läksimegi Rävala puiesteele, istusime koos 
ehitusjuhiga ühte tuppa, mille ühes nurgas hoidsid
ehitajad oma tööriistu, ja ehitasime panka.

\question{Kust tekkis mõte, et panga tegemiseks ei piisa kilekottidega 
sularaha edasi-tagasi lohistamisest?}

Need mehed, kes panga tegid, olid piisavalt targad, mõistmaks, et pank käib 
teistmoodi. Kui palju teistmoodi, sai alles siis selgeks, kui 
Inglismaalt osteti pangatarkvara ja konsultandid rääkisid, kuidas panka 
tehakse. Aga see ei olnud kohe esimesel aastal. Esimestel aastatel ehitasime, 
tõmbasime kaablit ja panime laua alla püsti serveri. Ühel ilusal päeval lükkas
Margus Kliimask\index[ppl]{Kliimask, Margus} kogemata varbaga 
toite välja ja pank jäi seisma. Aga mitte kauaks. 

Nii Rein Usin\index[ppl]{Usin, Rein}, Ivar Lukk\index[ppl]{Lukk, Ivar} kui ka Margus Kliimask\index[ppl]{Kliimask, Margus} olid 
visiooniga inimesed. See pidi olema suhteliselt algusaastatel -- BBSid ja 
Fidonet olid siis veel kuum teema --, kui Margus Kliimask ütles, et teeme 
modemipanga. Tal oli kindel mõte, et see peab olema Norton 
Commanderi\index{Norton Commander} F2 menüüs\sidenote{1986. aastal turule 
tulnud ja 1998. aastal viimase versiooni saanud Norton Commander oli 
ülipopulaarne failihaldur MS-DOSi platvormile. Ekraanil oli korraga kaks 
nimekirja faile ja käsurida, allservas nimekiri saadaolevatest 
klahvivajutusega käivitatavatest käskudest. Nii oli kasutajal ilma suurema 
koolituseta kohe selge, mida ja kuidas teha. Ohtralt kasutati F-klahve 
ja neist olulisemate funktsioonid on inimestel siiani peas (F3 -- faili sisu 
vaatamine, F5 -- faili kopeerimine).}. Kõik kasutasid Norton Commanderit 
ja kõigil oli see olemas, aga keegi ei ostnud, sest tol ajal tarkvara ei ostetud. 

\question{Jah, ma mäletan poes karpe, aga ei mäleta, et keegi oleks neid kunagi 
ostnud.}

Hämmastav oli see, kuidas mõtte väljakäimisest 
modemipanga \emph{launch}'ini läks umbes kaks kuud.

\question{Tegite kahe kuuga nullist modemipanga?}

See oli programm, mis oli mingil määral Norton Commanderiga integreeritud: 
läks sealt menüüst käima, nägi välja nagu Norton Commanderi 
osa, võimaldas makseid ette valmistada, kontoväljavõtteid ja panga teateid 
saada ning enda makseid panka saata.

\question{Ja teisel pool võttis mingi asi kõned vastu, suhtles panga 
tuumaga ja tegi arveldused ära?}

Just. Panga tuumaga suhtlemine oli üsna traagelniitidega asi, kuna selleks 
ajaks oli juba toodud Inglismaalt panga tarkvara, millel ei olnud ühtegi head 
liidest peale terminali.

\question{Ja siis tegite terminali emulaatori?}

Mina jah kirjutasin terminali emulaatori ja üks kolleeg kirjutas programmi, mis 
lükkas emulaatorist maksed pangasüsteemi, ning see toimis 
aastaid niimoodi, enne kui tekkisid tehnilised vahendid, et seda 
natukene viisakamalt teha. \emph{Launch} toimus 
tolle aja kohta suure pressikäraga: tehti korralik meediaüritus, imekenad 
Hansapanga\index{Hansapank} tüdrukud istusid ka seal ja tegid märkmeid. Ja läks mööda vaid
mõni kuu, kui Hansapangal tuli välja
Telehansa\index{Telehansa}.

\question{See kamp, kes tollal
BBSides suhtles, võis olla kokku paarsada inimest. Kust tulid
kliendid modemipangale?}

Kliendid jagunesid umbes pooleks. Forekspanga klientuurist arvestatav 
protsent oli Venemaal, sest suur 
raha oligi tol ajal Venemaal, aga ka Eesti klientuur ei olnud sugugi kehv. Pank 
müüs seda suhteliselt suure summa eest ja Eesti firmad 
ostsid. Käisin seda ise Tallinnas installeerimas. Küsimus ei olnud 
selles, et inimesed ei saanud tulla maalt linna pangaasju ajama, vaid nad 
lihtsalt ei tahtnud kontorist välja tulla. Pangas sai mugavalt ära käia 
laua tagant püsti tõusmata.

\question{Ja see kõik tasus ära, et hakata isegi arvutiga 
makseid ette valmistama?}

Sel ajal oli igas firmas raamatupidamiseks arvuti olemas ja raamatupidajate 
arvutites maksed olidki. Ilmselt mugavus ja aja kokkuhoid tõukasid
Eesti firmad sinnapoole.

\question{Kui palju seal telefoniliine küljes oli?}

Alustasime kahega ja lõpus oli vist kuus. Kuna 
sideseanss oli nii lühike, mahtus enamik sideseanssidest paari minuti 
sisse. Kõik pakiti kohapeal kokku ja saadeti ühe portsuna ära -- Fidonetist õpitud tehnoloogia. Alguses tegin mina kliendipoole ja 
Margus Kliimask\index[ppl]{Kliimask, Margus} kirjutas serveripoole. Hiljem kirjutasin serveripoole veidi paremaks, et see oleks paremini eskaleeritav.

\question{Mida see tähendab?}

Ühe masina taha sai panna mitu modemit.

\question{Kas sa oma BBSi hoidsid siis veel püsti?}

Minu meelest oli meil pangas ka BBS veel mõnda aega, 
Microlinkis\index{Microlink} oli kindlasti. Kuna Forekspank asus Rävala puiesteel, siis kohe, kui 
üheksakümnendate alguses tekkis internet, oli selge, et meil on ka 
seda vaja. Tõmbasime koos Andrus Aaslaiuga\index[ppl]{Aaslaid, 
Andrus} oma valgete käekestega mööda majakatuseid Forekspanga kõrvale KBFI\index{KBFI} majja, 
kus sündis Uninet\index{Uninet}, Etherneti kaabli.

\question{Te olite siis otse Unineti küljes?}

Otse Unineti küljes, olime ühed esimesed kliendid, kodukootud 
ruuteri softiga, mis läks flopi pealt käima. Mõlemas otsas oli üks 
arvutikast ja nii me ennast internetti panime. Muide, ükskord 
lõi meil sinna välk sisse.

\question{Mida te internetis tegite?}

Algul õppisime, mis see on. Ja pangas oli hädavajalik meilivahetus, et suhelda. Üks esimesi asju, mis pangas sai 
ehitatud, oli teleksi \emph{gateway} Pegasus Maili\index{Pegasus Mail}. 

\question{Misasi on teleks?}

Teleks oli viiekümneboodine\sidenote{\emph{Baud rate}, eesti keeles lihtsalt \emph{boodid}, 
näitab, mitu korda sekundis signaal liinil muutub andes indikatsiooni side kiirusest.}  telegraafisüsteem. Kahtlustan, et paljud pangad maailmas kasutavad seda endiselt. Suhtlus ei käi telefoniliini 
pidi, vaid selleks on eraldi teleksivõrk, mis toimib mööda telefonitraate 
hoopis teistsuguste signaalidega kui tavaline telefon.

\question{Kas see oli \emph{circuit switched}\sidenote{Ahelkommuteeritud. 
On ju ilus eestikeelne sõna?}, eks? Siis see vajas eraldi keskjaama?}

Jah. Põhimõtteliselt tuli ikkagi kõne teha ja ühendus püsti seada. 
See ehitati veel sel ajal, kui olid teletaibid -- klaviatuur ja 
paberirull.

\question{See \emph{gateway} ei saanud siis ju olla ainult tarkvaraline, vaid
oli ka riistvara vaja?}

Jah. Seal oli üks kast vahel, mis tegi sellest jadapordi. Esimese kasti tegi minu meelest
Küberneetika Instituudi\index{Küberneetika Instituut} majas üks Sass, Aleksander\index[ppl]{Reitsakas, 
Aleksander}.

See oli väga keeruline kast, tegin hiljem sellest peopesasuuruse 
versiooni flopikarpi.

\question{Mind hämmastab see, et sa ehitasid järjest keerulisemaid asju, aga kust sul tulid selleks teadmised, seda ei selgu.}

See on nagu Youtube'i videot vaadates -- tundub, et kõik asjad juhtuvad ise. 
Vahepeale mahtus siiski kuude kaupa õppimist, häkkimist ja katsetamist.

\question{Sul pidi hirmus kihu seda teha olema.}

Kindlasti, peaasi, et oli huvitav. 
Pangas töötades hakkas esimest korda ka kohusetunne vaevama, sest kui pank hommikul ei toiminud, olin ju mina paha.
Töötunde kulus kõvasti, aga üksiku inimesema ei olnud mul eriti muid kohustusi.

\question{Lisaks rääkisid muudkui teistega juttu BBSides.}

Panga ajal enam mitte, siis võttis töö kogu aja ära. Varem toimus jah BBSides suhtlus, aga kui tuli internet, võttis meilindus asja üle. Meiliga tuli kohe ka  \emph{gateway} 
kohe panga serverisse. Pank oli selles mõttes väga hästi kommunikeeruv.

\question{Legend räägib, et sina kirjutasid esimese eestikeelse klaviatuuri draiveri, 
on see tõsi?}

Nii ja naa. Rainer Nõlvak\index[ppl]{Nõlvak, 
Rainer} leidis esimesena, et klaviatuuril võiks eestikeelne \emph{layout} 
olla. Veel enne, kui infotehnoloogid jaole said, tellis Rainer eestikeelse 
klaviatuuri ära.  Nii et pärast, kui kehtestati  uus standard (EVS 8:1993),  
olid olemas klaviatuur ja oli kirja pandud standard. Lisaks klaviatuurile oli aga vaja ka standardile vastavat 
lokalisatsiooni. Eriti hull lugu oli Windowsi fontidega -- sel ajal oli olemas
Windows 3\index{Windows}. Ja siis korraldati konkurss, kus kõik lähenemised 
olid lubatud.

\question{Kes konkursi korraldas?}

Ma ei mäleta organisatsiooni nimesidenote{Tegemist oli Eesti Informaatikafondiga\index{Eesti Informaatikafond}, sellest sai hiljem Eesti Informaatikakeskus\index{Eesti Informaatikakeskus}, Riigi Infosüsteemi Ameti\index{Riigi Infosüsteemi Amet} eelkäia.}, aga see oli riiklik 
konkurss, mille auhind oli tolle aja kohta täitsa korralik, vist kakskümmend 
tuhat krooni. Olime selleks ajaks Raineriga juba natuke sel alal 
koostööd teinud -- Microlink pani enda klaviatuure müües kaasa draiveri, mis seda 
\emph{layout}'i toetas ka, nii et osa tööd oli juba tehtud. Kui konkurss 
välja kuulutati, ütles Margus Kliimask\index[ppl]{Kliimask, Margus}, visiooniga mees,
et teeme nii, nagu Microsoft teeb. Me \emph{reverse 
engineer}'isime kogu selle DOSi lokalisatsiooni ja klaviatuuri draiverid ning 
tegime installeerimisprogrammi, mis paigaldas 
standadkomponendid: \verb|KEYBOARD.SYS|i, \verb|COUNTRY.SYS|i ja muud
sellised asjad. Kuskilt õnnestus hankida soft, mis tegi Windowsi 
fonte, ja ma joonistasin fondid ka. See ei olnud küll kuigi hea soft, 
ei teinud TrueType'i \emph{hint}'ingut; \emph{kerning} vist 
on see teine, mis teeb fondid ilusaks, kui need väikseks muudad. Eesti 
fondid paistsid ekraanil karvased, aga me ei saanud sinna kahjuks midagi parata. Igal juhul
oli meie lähenemine teistega võrreldes nii palju parem, et võitsime konkursi.

\question{Kas pank läks konkursile osalema?}

Ei, ainult meie Margus Kliimaskiga\index[ppl]{Kliimask, Margus}. 
Meil oli pisike OÜ, koos pangaga tehtud ühisfirma Forex Communications modemipanga müümiseks. 
Selle firma alt osalesimegi. 

\question{Ja osalesite seepärast, et tundus huvitav?}

Sinna läksime ilmselt raha pärast ja võibolla ka 
Näitusepaviljonis toimunud joomingu pärast, mille seesama riiklik asutus piduliku sündmuse puhul 
korraldas.

\question{Kas sul sellepärast saigi panga aeg otsa, et pank sai valmis?}

Pigem pean olema tänulik pangajuhtidele, kes andsid meile 
hämmastavalt vabad käed igasugust tehnoloogiat katsetada ja uurida ning mõelda 
uusi asju. Tänu sellele oli Forekspank ka üks esimesi internetipanga tegijaid -- meil oli olemas internetiühendus ja me juba mõistsime, mis toimub. 

\question{Millega tollast internetipanka tehti?}

Forekspanga esimene internetipank oli minu meelest 
IISi\sidenote{1995. aastal turule toodud \emph{Internet Information 
Server (IIS)} oli Microsofti veebiserver, mis üritas (mõnevõrra tulutult) 
konkurentsi pakkuda tol ajal domineerinud Apache'i veebiserverile.} peal ja töötas
Windowsis\index{Windows}. 

\question{Eksootiline valik tolle aja kohta ...}

Oli küll imelik valik. Aga sel ajal olid meil juba arendus- ja 
hooldusmeeskonnad eraldi. Margus Kliimask\index[ppl]{Kliimask, Margus} oli 
arendusmeeskonnas. 

\question{Ehk te olite \emph{DevOpsist}\sidenote{Arendusmetoodika, kus tarkvara 
ehitamine ja selle edasine käitamine korraga nime kaotavad ehk omavahel 
lahutamatult kokku saavad.} astunud sammu tagasi?}

Panga käigushoidmine ongi natuke omapärane tegevus. Margus 
\index[ppl]{Kliimask, Margus} juhtis internetipanga arendust, tema meeskonnas
oli ka Pronto\index[ppl]{Pronto|see{Raja, Tanel}}\index[ppl]{Pronto}\sidenote{Vt lk\pageref{sisu:pronto}.} ja veel paar 
hakkajat selli. 

\question{Kas sina olid ka sellega seotud?}

Mina ei olnud internetipangaga peaaegu üldse seotud. Sel ajal oli
modemipank veel põhikanal, kuna internet oli siis vähestel. Forekspank oli juba üsna suureks kasvanud, 
hooldusmeeskonnas oli kümmekond inimest.

\question{See kõlab juba nagu terve organisatsioon, kahe telefoniliiniga ei saanud enam 
hakkama?}

Sel ajal tekkisid teised probleemid. 
Pangale ostetud tarkvara käis kummalise IBMi platvormi peal, mida aeg-ajalt 
tuli \emph{upgrade}'ida. Selle tarkvara jaoks oli COBOL uus keel. 
Tarkvara oli kirjutatud imelikus keeles nimega \emph{Report Generator Language}, mis 
oli pärit System/36\index{System/36}\sidenote{System/36 oli IBMi poolt 
1983. aastal turule toodud väike mitme kasutaja jaoks mõeldud mitmetegumiline 
server, mida programmeeriti peamiselt platvormipõhises RPG II\index{Report Program Generator} (\emph{Report Program Generator - RPG}) keeles.} ajast. Sellest keelest 
kumasid perfokaardid ikka veel kõvasti läbi.

\question{Vähe sellest, et teil oli visioon, aga raha pidi ju ka olema, et brittide juurde 
minna.}

Server maksis sel ajal meeletu raha. Algul ei olnud pangal jaksu õiget masinat osta, hangiti üks karm 
PC ja selle peal käis System/36 emulaator, millel jooksis 
panga tarkvara. Õnneks kasvasime sellest üsna ruttu välja. Pärast oli meil selline unikaalne platvorm nagu
AS/400\index{AS/400}\sidenote{AS/400, hiljem tuntud kui 
\enquote{System i}, oli IBMi keskmise suurusega serveriplatvorm, mis 
toodi turule 1988. aastal.}, mida ka korduvalt uuendati.

Ilmselt sai pank tarkvara ostes 
ka teadmise sellest, kuidas panka teha. See oli võibolla 
rohkem väärt.

\question{Teil oli Margusega juba siis kahe peale pisike OÜ, aga mõni 
veedab terve elu oma huvi üksnes akadeemilistes sfäärides rahuldades. Kust sul tekkis
arusaam ärist?}

Nagu ma mainisin, siis OÜ sündis modemipanka tehes ja pean jällegi kiitma 
tolleaegseid pangajuhte, kellega koos me ühisfirma lõime. Otseselt äritegemist kui sellist ei olnud: meie 
kirjutasime tarkvara ja inimesed maksid selle eest OÜ-le, pärast 
jagasime pangaga raha ära. Klassikalise äri mõistes ei pidanud meie midagi 
müüma, pank müüs. Muidugi tekkis ettekujutus näiteks
raamatupidamisest, aga erilist ärisoont see minus ei arendanud.
OÜ käigushoidmine mingit tähelepanu ei nõudnud, kogu fookus oli tehnoloogial.

\question{Tõnu Samuel\index[ppl]{Samuel, Tõnu}\sidenote{Vt lk \pageref{sisu:tonu}.}  rääkis mulle, et Mastsidenote{Ehk siis käesoleva loo kangelane.} oli see mees, kelle juurde sai minna riskantsete 
asjadega. Kui oli vaja emaplaadi peal vaibanoaga radu lahti kratsida 
ja sinna relee vahele panna, siis Tõnu teadis, mida teha, aga ei 
julgenud. Seevastu Mast julges.}

Ilmselt oli abiks raadiotehnika kateedri kool. Kui saad aru, 
mida teed, siis sa ei karda lõigata.

\question{Nii et sul sellist aukartust masina ees ei olnud?}

See kadus suhteliselt vara ära, kuna raadiotehnika kateedri Apple 
II\index{Apple II}s oli mitu laienduskaarti sees. Kui sellel oli
kaas peal, siis kuumenes üle, aga kaas ei olnud kunagi peal. Seal võis 
vabalt näppupidi sees sobrada ja mitte keegi ei öelnud, et sa ei tohi seda kivi 
välja võtta. Kõik oli pesades, kõike võis välja võtta. Kui katki läks, siis 
tuligi võtta. 

\question{Kas läks katki ka?}

Ikka läks, aga Apple II\index{Apple II} oli 
lihtsa loogika järgi ehitatud, Vene kivid läksid sinna asemele ja taktsagedus oli üks 
megaherts. Seda sai parandada ja see oli väga õpetlik. Ka 
esimese IBM PC\index{IBM PC}ga tulid kaasa (meil olid 
kõik juhendid olemas) BIOSi \emph{listing}'ud ja skeemid. Kõik olid 
standardtükid, kõike sai parandada ja parandatigi. 

\question{Mida sa pärast panka tegid?}

ITd ühele väikesele investeerimiskontorile. Kirjutasin 
Exceli Visual Basicus\index{Visual Basic} väärtpaberite 
kauplemise programmi. Tol ajal tehti paljusid asju Excelis, näiteks arvutati intressi. Tegin suured Exceli makrod, millega sai 
väärtpaberiportfelle hallata ja tehinguid jagada. 

\question{Kas jälle selle pärast, et oli huvitav?}

See oli rohkem vajaduspõhine. Meie enda investeerimiskontoril oli seda 
vaja ja ühe koopia müüsin maha ka. 

\question{Nii et tegelesid siiski ka müügiga?}

Ma ei tegelenud müügiga. Enamasti oli nii, et keegi tuli ja ütles, et tal 
oleks ka vaja. 

\question{Kui on väärt asi, siis lõpuks ikka tullakse.}

Jah, kui hind sobis, siis miks mitte.

\question{Sa oled BBSummeri\index{BBSummer} kuulsa grupipildi peal. Kas käisid tolle seltskonnaga läbi, kuigi töö võttis enamiku ajast ära?}

BBSummerid algasid siis, kui olin alles tehnikaülikoolis, ja neid ei olnud üldse palju. See grupipilt, mida sina vist 
mõtled\sidenote{Memcpy podcast'i kaanepildiks olev foto, kus on peal 
hämmastavalt paljude suurte asjade toonased või hilisemad algatajad.}, ei ole esimesest BBSummerist, vaid teisest või kolmandast, kus käisid ka FidoNeti tublid mehed Soomest. Seal 
pildil on üks habemega mees nimega Ron Dwight\index[ppl]{Dwight, Ron}, kes 
oli FidoNeti kunn Euroopas, regiooni pealik. Ron 
oli väga tore mees, ma olen tal isegi paar korda külas käinud ja tema juures Soomes 
ööbinud, kui piirid lahti läksid. Ja ma ei ole Eesti kambast ainukene, kes tal
külas käis. 
Soomlased, kes FidoNeti Soomes vedasid, olid tol ajal üldiselt väga toetavad. 
Sa oled teistega rääkinud, kuidas te Soome helistasite, ja keegi ei ole 
maininud, et tegelikult algusaegadel helistasid soomlased siia. Ei olnud nii, 
et ainult sealt oleks tõmmatud. Hiljem, kui BBSid ja firmad said siin jalad 
alla, saime "rinnapiima" otsast lahti, aga algusaegadel 
soomlased toetasid meid tublisti. 

\question{Kas puhtalt missioonitundest? Hõimuvelled ja nii?}

Ma ei tea, kui palju hõimuvendlus rolli mängis, pigem arusaam, et tehnoloogiat tuleb huvitatud inimestega jagada. 

Mul on nendest aegadest väga head mälestused ja sellepärast kutsusimegi neid ka BBSummeritele\index{BBSummer}. Ron käis minu meelest kahel. Igatahes oli
soomlasi esimestel BBSummeritel palju ja ma mäletan, kuidas nad olid selle grupipildi aegsel BBSummeril äärmiselt
hämmastunud sellest, et kõik võivad õlut juua ja et teisel päeval ei toimunud mingeid 
kaklusi!

BBSummeri korraldamise juures oli veel tore see, et korraldustasu
tagas söögi ja joogi kõigiks päevadeks. Ja õlut pidi kõigile jätkuma. Ühele BBSummerile toodi küll õlut Fanta tünnides, nii et 
õllel oli kerge Fanta mekk juures.

\question{Tundub, et sul on inimestega vedanud.}

Mul on jah sõpradega vedanud. Kui ma üksi elasin ja 
tehnikaülikoolis\index{Tallinna Tehnikaülikool} 
vabakutseline olin, siis suhtlesin väga paljudega. 
Hiljem võttis perekond nii palju aega ära, et kahjuks ei jõudnud enam kõigiga 
kontakti hoida.

\question{Aga kriitilisel hetkel olid nad olemas?}

Nad on siiamaani olemas. Näiteks 
Lõvi\index[ppl]{Lõvi} kohtasin ma umbes viis aastat tagasi Selveri 
parklas, nüüd käisin tal hiljuti tehnikaülikoolis külas.

\question{Ahti\index[ppl]{Heinla, Ahti}\sidenote{Vt lk \pageref{sisu:ahti}.}  ütles 
väga targasti, et seltskond noori inimesi sai
omavahel suheldes inimeseks koos Eesti riigiga. Kas sul on ka selline 
tunne?}

Jah, me olime kõik suhteliselt üheealised. Täpselt selles 
vanuses, kui oli huvi teha midagi uut ja selleks tekkis võimalus ning ka omavaheline klapp. Oli ka erandeid, näiteks Henn Ruukel\index[ppl]{Ruukel, Henn} 
oli esimesel BBSummeril selgelt alaealine, aga õlletünni juures passi ei 
küsitud.

\question{Mida sa praegu teed?}

Pean pausi. Aitan ülikoolil satelliiti\sidenote{Masti panusega satelliit lendas 
2020. aastal ka edukalt kosmosesse.} ehitada. 

\question{Sellepärast, et on huvitav?}

Sellepärast, et on huvitav. Kosmos on huvitav.

\question{Kosmos on suur ka, seal ei ole karta, et huvitavad asjad 
saavad otsa.}

Praegu käib sebimine enamjaolt Maale väga lähedal. Orbiidid, kuhu 
väikseid satelliite lastakse, on viie- kuni seitsmesaja kilomeetri kaugusel.

\question{Kas sul üldse on kunagi juhtunud, et järgmist huvitavat asja ei ole 
silmapiiril?}

Ei.

\question{Kuidas see sul on õnnestunud?}

Isegi kui päevatööl ei ole huvitav, siis mul 
kodus käib kogu aeg mõni projekt. Kui üks saab valmis või läheb 
sahtlisse (sinna läheb enamik, sest huvi kaob ära), on 
järgmine kohe laual. Sellist asja ei ole, et mul ei ole midagi teha.

\question{Kas sul sahtel juba täis ei saa?}

Saab. Jube täis on. 

\question{Mida sa siis teed?}

Viskan ära. Suur osa neist on ju eksperimendid. Võtan ära tükid, mis lähevad  
järgmise eksperimendi peale, ja ülejäänu on prügi. Teadmised jäävad alles.


\chapter{Kain Kalju}
\index[ppl]{Kalju, Kain}
\question{Kuidas sina arvutite juurde jõudsid?}
See oli umbes aastal 1990--1991, kui mu sõpradele tekkisid esimesed arvutid, 
olime kaksteist kuni neliteist aastat vanad. Ühele sõbrale tekkis selline 
imelik asi nagu Texas Instruments TI-99\sidenote{Täpsemalt Texas Instruments 
TI-99/4\index{Texas Instruments TI-99/4}. Ärilistel ja arhitektuursetel 
põhjustel lühikese elueaga koduarvutite perekond. Oli koos samal 1979. aastal 
turule tulnud Atari 8-bitiste arvutitega üks esimesi omataolisi, millel oli 
audio- ja videoülesanneteks omaette protsessorid.}, see oli Commodore ja Apple 
II sarnane riistapuu selles mõttes, et ta oli 16-bitise protsessoriga ja 
\emph{boot}is otse BASICusse\index{BASIC}. 

See arvuti oli telekaga ühendatud ja seal olid mingisugused primitiivsed mängud 
Space Invaders\index{Space Invaders} ja muud sarnased. Ja siis 
loomulikult ka BASIC. Kogu programmi kood tuli kassetilindilt, nii nagu tollel 
ajal kombeks, mingeid flopisid polnud olemas. See oli minu esimene kokkupuude 
sellise arvutiga, millel oli klaviatuur, kuhu sai sisestada programmi koodi, 
kus me siis katsetasime ka esimest korda ise programme teha BASICUS toksides 
arvutisse ajakirjades ilmunud koodi ja mõeldes neid ka ise välja. 

\question{Mis linnas see oli?}

Ma olen Keilast pärit. Mul ei ole nagu kunagi olnud mingisugust 
sellist spetsiaalset ligipääsu kuhugi  teadusasutustele, koolidele ja nii 
edasi. Minu ligipääs arvutitele oli selles mõttes suhteliselt  piiratud 
võrreldes mõnede teistega.

\question{Kas sul seejuures mingit reaalainete huvi ka oli taustal?}

Koolis ma käisin reaalkallakuga klassis. Meil oli väga vahva lend 
gümnaasiumis\index{Keila Gümnaasium}, meil praktiliselt kõik poisid olid 
mingisuguse arvutihuviga ja nii palju kui ma nende elukäiku jälginud olen, on 
praktiliselt kõik  arvutimaailmas miskit pidi tegevad.

\question{Aga kust see tuli? Teil oli koolis nii korralik tase?}

Selles mõttes ongi väga huvitav, et gümnaasiumi esimestes klassides (me just 
olime läinud kaheteistkümne klassi süsteemile), meil olid kooli arvutiklassis  
Jukud\index{Juku}. Need loomulikult absoluutselt meid ei huvitanud, 
seal oli Pascal\index{Pascal}, meil oli juba ligipääs PC-dele tollel 
hetkel. 

\question{Juku oli ju igavesti äge aparaat omas ajas?}

Jah, aga nad tulid  selles mõttes  hilisemas faasis, pärast seda, kui meil oli 
juba PC ligipääs olemas ja kui mul endal oli ka kodus juba PC. Minu  kõige 
suurema arvutihuvi läkski sellest hetkest lahti, kui vanemad otsustasid mulle 
PC osta. Seda lugu peab natuke tagasi kerima selles mõttes, et seesama sõber, 
kellel oli see Texas Instrumentsi imepill, sai aasta hiljem 
  monokroomekraaniga 286-e. Tal isa käis Ameerikas ja  tõi 
sealt. Naljakas oli veel see, mis näitab seda ajastut, nad elasid esimesel 
korrusel kortermajas ja PC oli raudkapis, mis käis kinni. Oli nii suur hirm, et 
keegi murrab sisse ja varastab ära.

\question{See arvuti ju maksis rohkem kui korter tollel ajal. Mõni ime, et PC 
kappi pandi!}

Mu vanematele, käis see kohutavalt pinnale, et ma üldse ei viibi kodus, olen 
kogu aeg sõbra juures külas, hilisööni välja. Millalgi üheksakümnendatel, 
vahetult enne Eesti krooni tulekut oli aeg, kui rubla devalveerus hästi 
kiiresti. Ma isa käest olen küsinud, kuidas see täpselt oli, ja ta meenutas, et 
tollel hetkel tema sai millegipärast palka juba Ameerika dollarites  ja siis 
mingisugusest kooperatiivist või mis iganes tollel hetkel äriühingud olid, sai 
ostetud üks 286 dollarite eest.  Hinnaklass oli umbes tuhat dollarit. See oli 
siis VGA ekraaniga ja nii edasi. Täis mats, täiesti uus, väga äge, kuigi 286 
ilmselt oli tolleks ajaks juba \emph{outdated}  natukene,  oli juba 386-te 
ajastu.

\question{Ikkagi, võrreldes nende XT-dega, mille abil Tartu Ülikoolis 
programmeerimist õpetati, oli see ikkagi väga kõva sõna. Mis sa tegid tolle 
arvutiga?}

Nagu noor poiss ikka, tõenäoliselt mängisin, mind huvitasid kõikvõimalikud 
tarkvarad. 

Üks huvitav seik on veel, et me käisime sama sõbraga 1993. aastal Ameerikas. 
See oli umbes aasta pärast seda kui ma omale arvuti sain. See Ameerikasse 
minek oli väga kummaline. Ma mäletan seda, et meil oli  kolmene punt, kes me 
elasime üksteise lähedal ja  kõikidel oli juba kodus arvuti. Kas vanemate 
tööarvutid või siis isiklikud. Me sõitsime rongiga Tallinnast Keilasse ja 
millegipärast rongis hakkasime rääkima, et kuule, jube lahe oleks minna 
Ameerikasse. Ühel sõbral on tädi Ameerikas, et ta võtaks hea meelega vastu, aga 
kuidas me sinna saaksime. Minu isa töötas tollel hetkel Muuga 
sadamas. Kuidagi sai räägitud, et põhimõtteliselt saaks ka 
laevaga minna. Mina ei tea, kust see tuli, noored poisid, me olime kuskil 
viisteist, kuusteist aastat vanad. Kodus rääkisin sellest ja kuidagi hakkas see 
pall veerema niimoodi, et üks hetk me olime USA saatkonnas viisat taotlemas, 
järgmisel hetkel isal oli juba kokku lepitud, et me saame minna kaasreisijateks 
Ameerika suurele kaubalaevale ja me sõitsime üle Atlandi ookeani laevaga Muuga 
sadamast New Orleansi. Seal pani laevakompanii meid lennuki peale ja sealt 
edasi lendasime JFK lennuväljale New Yorki, kus siis sõbra tädi meid vastu 
võttis. 

Kusjuures me saime laeva peal palka, sest laevafirmale oli palju odavam 
vormistada meid töötajateks. Muidu oleks olnud vaja tasuda suuri 
kindlustusmakseid. Selles mõttes täiesti kreisi.

\question{Sellist asja ma kuulen esimest korda! Kaua te sõitsite sinna?}

Kaks nädalat, umbes neliteist-viisteist päeva võttis see laevasõit üle
ookeani.

\question{Kas te midagi kasulikku ka seal laeva peal tegite või sõitsite 
lihtsalt kaasa?}

Midagi kasulikku me tegelikult ei teinud. Hängisime ohvitseride nii-öelda 
piirkonnas. Meile küll näidati, kuidas laev töötab, aga me ei teinud selles 
mõttes mingit kasulikku tööd, et me oleks  koristanud tekki või midagi sellist. 
Ei, me lihtsalt hängisime. Võib-olla heal juhul saime mingisugust sellist  
ülevaatlikku õpet, umbes nagu sa muuseumis käid, et näed, siin on see asi, siin 
mootoriruumis on sellised nupud. Loomulikult keegi meil midagi teha ei lasknud 
välja arvatud see, et võib-olla ava-ookeanil me saime rooli keerata ja natukene 
nii öelda laeva juhtida.

\question{Millega  te tagasi tulite?}

Tagasi me tulime lennukiga juba. Aga miks ma sellest üldse räägin on see, et 
kui Ameerika pinnale astusime, oli meil päris palju raha, saime ju laevast 
palka. Meil oli stiilis tuhat viissada dollarit, mis oli tolle aja kohta üüratu 
summa. Ja siis mina isiklikult kulutasin selle raha loomulikult ära arvutipoes. 
Ma tõin endale Ameerikast, ma arvan, et elu ühe kõige tähtsama riistapuu, 
milleks oli modem. 

Ja vot pärast seda läheks elu lahti. 

See oli mingisugune 2400 boodine modem, ma täpselt ei mäleta tüüpi enam. Lisaks 
tõin veel Sound Blaster 16\index{Sound Blaster} helikaardi, mis oli täiesti 
tipp tollel hetkel\sidenote{Sound Blaster oli Singapuri firma Creative 
Technology (tuntud USAs kui Creative Labs) helikaartide perekond. Need kaardid 
olid PC-maailmas \emph{de facto} standardiks, kuni Windows 95 vastavad liidesed 
standardiseeris ja PC audio kommoditiseerus.}. See oli just välja tulnud, 
stiilis paar kuud varem. 

Üks asi, mille ma  hiljem avastasin, mis BBSides levisid, olid  helimoodulid,  
mul oli neid hästi palju, ma mingil hetkel kogusin neid. Ma arvan, et tollel 
ajastul paljud tegid seda. Need moodulid on  sellised helifailid, mida tollel 
ajal Amiga arvutites kokku pandi, koosnesid sämplitest. Põhimõtteliselt sul oli 
 mingisugune kaheksa \emph{track}i, kuhu sa siis miksid sämpleid niimoodi 
kokku, et sellest tekkis mingisugune \emph{meaningful} muusika. 

\question{Need liikusid siis BBSides?}

Jah. Loomulikult  sai üritatud ise ka neid teha, aga mul erilist muusikalist 
tausta ei ole, nii et sellest midagi välja ei tulnud.

\question{Tulid Ameerikast tagasi ja panid kohe BBSi püsti?}

Ei. Kui ma tulin Ameerikast tagasi, siis ma hakkasin avastama enda jaoks  BBSi 
maailma. Vanu asju üle vaadates selgus, et mu üks lemmik-BBS oli Dark 
Corner\index{Dark Corner}, mis oli Priit Kasesalu\index[ppl]{Kasesalu, 
Priit} veetud. Esmalt loomulikult sa üritad  alla laadida kõike, mida saad. 
Kõik on ju puhas kuld, kõik tarkvara, mida sul pole veel kunagi olnud ja nii 
edasi. Siis huvitav oli veel see, et tollel ajal eksisteeris selline asi nagu 
Kadaka Turg\sidenote[][-4cm]{Aastal 1991 avatud ja 2002. aastal 
kaubanduskeskusega asendatud Mustamäel asunud turg oli küllalt metsik 
müügikeskkond, kust oli võimalik hankida kõike alates karvamütsidest ja 
Nõukogude aurahadest kõikvõimaliku piraatkaubani. Sisuliselt oli tegemist 
endise Nõukogude Liidu territooriumil toiminud varimajanduse väljundiga 
Eestisse. Turg oli turistide seas hinnas, parematel aegadel käisid sinna 
Tallinna Sadamast eribussid.}, seal müüdi piraattarkvara. Ma arvan, et ma sain 
ka väga palju sealt tarkvara. BBSides, kusjuures minu mäletamist mööda 
tegelikult otseselt piraattarkvara väljas ei olnud. Seal oli rohkem sellist 
häkkimise stiilis tarkvara, aga mitte nii otseselt.

\question{Windowsi sealt vist keegi ei laadinud endale}

Jah, just, selliseid asju otse faililistides ei olnud, need olid taha 
nurkadesse ära peidetud. Aga seda ma mäletan küll, et mul oli kodus 
telefoniliin ja minu meelest ei olnud minutitasu tollel hetkel või see  
minutitasu oli nii odav. Igal juhul, mul oli kodus liin praktiliselt 
ööpäevaringselt kinni kogu aeg, sinna ei olnud võimalik helistada, sest minu 
arvuti helistas kogu aeg, laadis midagi alla.

\question{Kuidas sa alguses rea peale said? Kuidas sa teada said, mis numbri 
peale helistada?}

Väga võimalik, et see tuli stiilis .EXE ajakirjast\index{.EXE}, ma ei suuda 
seda enam meenutada. Aga kui sa oled ühte BBSi juba sisse pääsenud, siis kogu 
see maailm juba avaneb. Üks teema, mida BBS levitas, oli teiste BBSide 
aadressidega failid. Mingil hetkel Priit Kasesalu\index[ppl]{Kasesalu, Priit} 
pani kogu oma BBSi viimase versiooni veebi üles. Ma laadisin selle alla ja 
avastasin selle  ketta pealt, vaatasin just hiljuti läbi, oli  päris huvitav. 

\question{Mis seal siis leidus?}

Kõikvõimalikke häkkimisvahendeid, C-programmide näiteid, mingisuguseid raamatuid 
stiilis \enquote{Terrorist Handbook}\sidenote{Ilmselt peab Kain silmas William 
Powelli raamatut \emph{The Anarchist Cookbook}. Vietnami sõja vastaste 
protestide laineharjal 1971. aastal USAs ilmunud (ja mitmel pool keelatud 
olnud) raamat sisaldas kõikvõimalikku vastandkultuuriga seotud sisu 
\emph{Thermite}-i ja LSD  valmistusõpetustest õpetusteni telefonisüsteemide 
murdmiseks. Raamat levis tekstifailina laialt ülikoolide serverite ja FidoNeti 
kaudu ning teda täiendati aja jooksul pidevalt: eriti kuulsad on anonüümse 
autori \enquote{\emph{The Jolly Roger}} täiendused.} ja muud sarnased. Igasugune 
selline kraam, mis  noortele inimestele põnevust pakkus.

\question{Tulles korra veel sinu arvutihuvi alguse juure. Kas sa olid pigem 
seda tüüpi mees, kes mängis arvutiga, võrgutas arvutit või programmeeris 
arvutiga?}

Ma olen mõelnud selle üle, et kuidas see siis täpselt oli. Mulle tundub, et mul 
on olnud  mitu  ajajärku. Koduse 286 ja BBSide ajajärk oli pigem selline, et sa 
lihtsalt üritad endale sisse krahmata kõike, mida sa näed. Seal leidus ka 
arvutimänge, aga ma ei mäleta, et ma oleksin väga  kohutavalt mänginud. Siis, 
kui mul endal veel arvutit ei olnud,  sõbra juures me mängisime loomulikult 
kõik need ajad täis. Me ei tegelenud programmeerimisega, vaid pigem ikkagi 
mängimisega. Aga hiljem jäi see mängimine pigem taha taustale ja ikkagi 
üritasid aru saada, kuidas arvuti töötab. Näiteks üks teema, mis mind 
kohutavalt paelus, olid viirused. Mul oli alati kõige viimane viirusetõrje 
tarkvara. Ma usun, et mul oli selleks hetkeks juba ka mitu kõvaketast, ehk mul 
oli võimalus katsetada, mida viirused teevad. Ka  viiruse nii-öelda 
kollektsioone levitati BBSides. Ja siis sai uuritud, et kuidas selline asi 
põhimõtteliselt töötab. 

Ja siis järgmine ajastu tuli siis, kui ma avastasin enda jaoks 
Linuxi\index{Linux}, samal ajal tuli ka Internet. Sealtsamast gümnaasiumi 
kõrvalt kaheteistkümnendas klassis ma sattusin tööle Riigi Elektriside 
Inspektsiooni\index{Riigi Elektriside Inspektsioon|see{Tehnilise Järelevalve 
Amet}}, mis on täna Tehnilise Järelevalve Amet\index{Tehnilise Järelevalve 
Amet}. Sattusin selliseks, noh,  patsiga või arvutipoisiks, mul patsi pole 
kunagi olnud. Olin selline arvutipoiss nagu ikka, kellele antakse mingisugused 
arvutid, et palun seadista nüüd need ära, tee seda ja teist.

\question{Kuidas sa sinna sattusid niimoodi kooli kõrvalt?}

Seesama sõber, töötas Pennus\index{Pennu} ja kuidagi tema kaudu tuli kontakt, 
et otsitakse sellist arvutitüüpi, kes oskab arvutitega midagi teha. Ma läksin 
kohale  ja kuidagi võeti tööle poole kohaga.

\question{Teil klassist ikka mitmed töötasid siis keskkooli ajal?}

Jah, meil mitmed töötasid. Üks klassivend näiteks töötas Keila Linnavalitsuse 
juures. Ta oli selline kõva programmeerija juba tollel ajal, kes kinkis mulle 
mu esimese programmeerimisraamatu C Programming Language\index{The C 
Programming Language}, Brian Kernighan and Dennis Ritchie.

\question{See on seesama salapärane väljaanne\sidenote{\label{sisu:richie_vene}Kainil oli raamat 
jutuajamisel kaasas, selles puudus igasugune märge väljaandja ning trükkimise 
aja ning koha osas. Raamat oli 
korralikult köidetud ja kopeeris isegi värvilist kaanekujundust täpselt. 
Isegi peidetud \enquote{üllatusmunad} olid kopeeritud: indeksis viitas mõiste \enquote{recursion}  
samale indeksi lehele. Mart Palmas\index[ppl]{Palmas, Mart} mäletab, et raamatut 
olla trükitud Novosibirskis.}, mis minul oli}

Just, kui praegu minna Amazoni vaatama, siis täpselt selline raamat on müügil. 
See oli mu esimene programmeerimisraamat, aga see leidis kasutamist ikkagi 
aastaid hiljem, kui ma juba netit\index{neti.ee} tegin ja mul oli praktiline 
vajadus programmeerida otsingusüsteemi mis oleks suurema jõudlusega.

\question{Tahaks ikkagi aru saada, et kuidas teil juhtus selline klass olema, 
kus mitmed juba keskkooli ajal töötasid, kodudes olid arvutid ja inimesed 
programmeerisid}

Aga võib-olla see oligi see aeg, kus need arvutid ilmusidki rohkem koju ja 
kontorisse ja oli tohutu puudus sellisest nii-öelda oskusteabest. Vanemad 
inimesed võib-olla julgenud arvuteid veel kasutada ja noored julgesid nendega 
igasuguseid asju teha.

\question{Aga igas keskkooliklassis ei olnud see asi niimoodi, et neli-viis 
poissi töötasid arvutispetsialistidena, miks teil oli?}

Ma ei oska seda tagantjärgi öelda. Küll aga mäletan sellist huvitavat seika, et 
meil üks  eksam oli põhimõtteliselt arvutieksam ja see ei seisnenud meie puhul 
programmeerimises. Meie puhul tähendas see eksam, et  me sisustasime 
arvutiklassi.  Kool sai Tiigrihüppe või  mis iganes programmi kaudu peaaegu 
klassitäie arvuteid ja siis R-klassi\sidenote{R nagu Reaal.} poiste ülesanne 
oli võrgutada see klass 
füüsiliselt Etherneti kaabliga, installeerida need arvutid, installeerida 
võrguserver, milleks oli  Linuxi server. Serveri \emph{task} jäi minu peale, 
kuna ma olin tollel hetkel kõige suurem Linuxi\index{Linux} käpp võrreldes 
siis teiste poistega.  Meie kaheteistkümnenda klassi arvutieksam seisnes 
selles, et me põhimõtteliselt seadistasime koolile esimese PC klassi. See oli 
1995. aastal.

\question{Linux ei olnud selleks ajaks ju kuigi vana, kuidas sa selle otsa 
komistasid?}

Linuxi otsa ma komistasin siis, kui ma juba Riigi Elektriside 
Inspektsioonis\index{Riigi Elektriside Inspektsioon} töötasin. Kui ma sinna 
läksin, siis seal veel Internetti ei olnud, aga see tekkis sinna üsna pea, ma 
arvan, et mingisugune kuu-paar hiljem. See oli siis 1994. aasta lõpp.  
Elektriside Inspektsioon asus aadressil Ädala 4d, mis on ka siis selline 
legendaarne internetihoone.  Meie allkorrusel oli 
Valitsusside\index{Valitsusside}, kus toimetas Taavi Talvik\index[ppl]{Talvik, 
Taavi}. Ja Taavi andis Riigi Elektriside Inspektsioonile juhtmeotsa kätte, 
milleks oli  tolleaegne kümnemegabitine koaksiaalkaabel ja, palun, siin on 
Internet. See koaksiaalkaabel sai siis veetud kõikidesse ruumidesse. Ei mingeid 
hub'e ega täht-topoloogiat.

Siis ma avastasin enda jaoks Interneti. Koolis loomulikult poistele rääkisin, et 
see FidoNet on nüüd mingi  vana jama, aeglane, toimib üle modemi, et siin on 
üks palju uuem ja huvitavam asi. Kusjuures  Valitsussidest edasi olid kanalid 
üsna kiired. Mäletan, et Tartu Ülikooli FTP-serverist sai kahemegabitise 
kiirusega faile alla laadida, see oli  meeletu kiirus. Välislink oli 
loomulikult kuskil 64 või 128 kilobitti. 

\question{Mis sealt Tartu Ülikoolist siis tõmmata oli nii väga?}

Vot seda ma täpselt ei mäleta, aga ju seal midagi oli, sest mul on väga selgelt 
meeles kadri.ut.ee\index{kadri.ut.ee}\sidenote{Tartu Ülikooli masinad kadri.ut.ee ja madli.ut.ee said Toomas Soome\index[ppl]{Soome, Toomas} andmetel nimed Otto Telleri\index[ppl]{Teller, Otto} tütarde järgi.}  FTP-server. 

Aga see Valitsusside\index{Valitsusside} ja see ethernetikaabel, see oli nagu 
huvitav. Tollel ajal, nagu teisedki on rääkinud, arvutiturvalisus ei olnud 
eriti teemaks. 

FidoNet oli selles mõttes tohutu kulla-auk, et ta avas loomulikult kõik oma  
\emph{echo} kanalid. Aga Internet avas meililistid ja kusagilt meililistist ma 
lugesin, et Anto Veldre\index[ppl]{Veldre, Anto} teeb 43. 
Keskkoolis\index{Tallinna 43. Keskkool} mingisuguseid selliseid 
\emph{introduction} kursuseid. Tollel ajal ilmus ka ajakiri .EXE\index{.EXE}, 
kus Anto artikleid kirjutas. Ma ei mäleta, kumb kummale täpselt eelnes, aga 
igal juhul mäletan seda, et üks hetk olin ma seal 43. Keskkoolis, et 
\enquote{siin ma olen ma tahan teadmisi saada}. Seal olid koha peal veel tol 
ajal sellised legendaarsed koolipoisid nagu Indrek Mandre\index[ppl]{Mandre, 
Indrek} ja Heno Ivanov\index[ppl]{Ivanov, Heno} vist. Tagasi tulin ma sealt 
juba Slackware\index{Slackware} distributsiooni installeerimisflopidega, mida 
oli stiilis kuus tükki. Installeerimise protsess käis ikkagi niimoodi, et 
esimene flopi, teine flopi, kolmas-neljas ja nii edasi lõpuks sai 
installeeritud. 

\question{Aga siis jääb lisaks kõigele muule Anto peale ka Linuxi pisiku 
levitamine Eestis?}

Ma usun küll, jah. Ma arvan, et temal on väga suur roll selles osas, Linux 
Eestis käima läks. Igal juhul mina selle pisiku sealt sain. Kuna ma olin tollel 
hetkel juba mõnda aega Elektriside Inspektsioonis\index{Riigi Elektriside 
Inspektsioon} töötanud ja ka palka saanud, oli mul päris korralik nii-öelda  
taskuraha. Ja ma ehitasin  endale uue arvuti, 286 sai FidoNetis  kuskil maha 
müüdud (FidoNetis  käis ka suur riistvaraga hangeldamine) ja ehitasin endale 
486 arvuti. Kusjuures see ei olnud mitte lihtsalt 486 vaid 486DX4 
100 MHz\sidenote{Inteli nomenklatuuris olid \enquote{DX} tähistusega protsessorid 
need, millel oli kiibil eraldi matemaatika kaasprotsessor, see andis olulise 
jõudlusvõidu.}, see oli siis absoluutne tipp. 

See oli kõige kõvem 486, mis üldse kunagi tehti. See oli  juba siis see aeg, 
kui mul oli juba \emph{node} registreeritud. Sealtsamast Dark Corner 
BBSist\index{Dark Corner} sain ma esimese FidoNeti \emph{point}i, kus ma 
pääsesin ligi FidoNeti uudistekanalitele. Mingil hetkel tundus, et aga palju 
ägedam oleks \emph{node}. Sai kirjutatud Tarmo Mamersile\index[ppl]{Mamers, 
Tarmo} (sest ta oli Eesti regiooni \emph{manager}, tema neid aadresse jagas), 
et kas oleks võimalik registreerida \emph{node} number kuuskümmend kuus ja 
Tarmo vastas, et \enquote{tehtud}. Sealt edasi oli mul \emph{node}, mis mõnda 
aega eksisteeris mul kodus. 

Aga siis mingil hetkel sai seadistatud Elektriside Inspektsioonis 
Linuxi\index{Linux} server, sest meil on praktiline vajadus serveerida 
printerit, faksi ja faile. Ehk siis sai nurka tekitatud Linuxi server, kes siis 
šeeris faile üle Samba teenuse ja võttis vastu fakse. Mul õnnestus ka enda 
FidoNeti \emph{node} sellesse samasse serverisse sokutada. Kui muidu FidoNeti 
tarkvara oli MS-DOSi peal, siis  oli ka alternatiiv Unixitele 
Ifmaili\index{Ifmail} nimelise programmi näol.

\question{Räägime korra sellest riigiametist. Miks seal üldse Internetti vaja 
oli? Kas see oli puhas sinu huvi või nad tegid midagi kasulikku ka sellega?}

Jah, selleks oli praktiline vajadus  olemas, sellepärast et Elektriside 
Inspektsioon\index{Riigi Elektriside Inspektsioon} tegi tihedat koostööd 
ITUga\sidenote{\emph{International Telecommunications Union (ITU)}.}, kes siis 
juhib kõiki neid sageduste jaotust ja protokolle ja  kõike muud sellist. 
Nendega oli inspektsioonil tihe kirjavahetus ja ma arvan, et e-maili teel. Ma 
ei suuda meenutada, kuidas see meilivahetus enne kaabliga Interneti käis, aga  
pärast  seesama Linuxi masin oli ka loomulikult mailiserver. Sellest hetkest 
tekkis ka meil oma domeen nimega rei.ee. Või äkki domeen oli juba varem olemas, 
igal juhul pärast Linuxi server tuli, hakkas ta rei.ee domeeni  kirju vastu 
võtma ja ka mina sain endale isikliku esimese ülilühikese e-maili aadressi, mis 
oli tollel ajal ülikõva, kain@rei.ee\sidenote{Lühikesed meili ja muud aadressid 
olid staatusesümboliks, need näitasid kuulumist kas serveri-administraatorite 
kõrgesse kasti või neile väga lähedasse ringkonda.}.

\question{Ehk sa avastasid ennast suhteliselt õrnas eas Linuxi ruuduna 
riigiasutuses?}

Just. Miks ma seda kümne megabitist ethernetikaablit mainisin oli see, et seal 
kõik liiklus oli ju näha. Ja kui ma külastasin siis Anto 
Veldre\index[ppl]{Veldre, Anto} arvutiklassi 43. 
Keskkoolis\index{Tallinna 43. Keskkool},  jäi mulle sealt üks asi elu 
lõpuni meelde. Kuidas kõik need noored tüübid, kes seal siis 
siil.edu.ee\index{siil.edu.ee} nimelise SCO\index{SCO UNIX} masina 
taga istusid, oli tohutu kõvad häkkerid. Nad demonstreerisid, mida nad siin 
teevad, näitasid, et kuidas nad  suudavad \emph{exploit}-ida mingisuguseid Tartu 
Ülikoolis olevaid masinaid\label{sisu!ylikooli_root}, mingeid professoreid seal jälgida ja nii edasi. See 
avaldas mulle nii kohutavalt mulje, et mind hakkas lisaks sellele varasemale 
viiruste teemale huvitama ka arvutiturvalisus.

Ma arvan, et see on esimest korda, kui ma avalikult sellest räägin aga ma 
\emph{sniff}isin loomulikult ka meie  võrku ja \emph{sniff}isin, mida siis 
Valitsusside\index{Valitsusside} insenerid seal tegid. Ega seal vahel midagi 
olnud,  sellesama  kaabli otsas oli kaks ametit: oli Riigi Elektriside 
Inspektsioon\index{Riigi Elektriside Inspektsioon} kõigi oma töötajatega ja 
\emph{Valitsusside}. Et kui Valitsusside insenerid käisid oma ruutereid või 
keskjaamu üle telneti konfimas, siis loomulikult see liiklus levis lahtise 
tekstina võrgus. Seda oli päris huvitav jälgida, mida nad siis seal teevad.  
Loomulikult ma seda kunagi pahatahtlikult ära ei kasutanud, see oli lihtsalt 
selline puhas  noore mehe huvi.

\question{Eks see seik iseloomustab suurepäraselt toonast aega. Ma usun, et kui 
praegu keegi lahtise traadi peal lahtist kanalit kasutaks, korraldataks umbes 
poole tunni jooksul mingi jama}

Jah, ma arvan ka. See mulje jäi niivõrd meelde, et kogu see võrguvärk on 
niivõrd ebaturvaline, et nii kui Soomest keegi härrasmees\sidenote{Tatu 
Ylönen, Helsingi Tehnoloogiaülikooli teadlane.} tegi 
\emph{secure shell}i esimese versiooni, siis ma hakkasin seda praktiliselt kohe 
kasutama, kui ma sellest teada sain. 

Veel Keila Gümnaasiumi\index{Keila Gümnaasium} juurde tagasi tulles. 
Pärast seda, kui ma kooli olin juba ära lõpetanud,  jäin ma edasi 
administreerima serverit, mis sinna maha jäi. Nagu tollel ajal ikka,  pidid 
kõikidel Unixi masinatel  olema ilusad nimed. Kodus rääkisin sellest teemast ja 
isa pakkus välja, et aga \enquote{kratt} oleks jube hea nimi. Ja praegu 
vaatasin nimeserverist järgi, et siiamaani on Keila Gümnaasiumis oleva serveri 
nimi kratt.keila.edu.ee\index{kratt.keila.edu.ee}.

\question{Hakka seda nime siis takkajärgi muutma. Loodetavasti riistvara ei ole 
päris seesama?}

Riistvara kindlasti ei ole seesama, sest seda koolimaja füüsiliselt enam alles 
ei ole. Keilas on nüüd uus koolimaja, kus mu enda lapsed käivad, sest ma elan 
siiamaani Keilas. Aga aadress on olemas.

\question{Seepärast ongi asjade nimetamine oluline, et need nimed võivad pikalt 
kesta}

Just. FidoNeti ajast veel üks huvitav \emph{impact} minu meelest, mis mul  
hiljem on väga kasulikuks osutunud oli see, et modemid töötasid
AT-käsustikuga\sidenote{Hayes käsustik, tuntud ka kui
AT-käsustik, on käsukeel, mille Dennis Hayes lõi 1981. 
aastal omanimelise ettevõtte 300-boodise Smartmodem modemi juhtimise tarbeks.}. 
Too käsustik oli selles mõttes universaalne asi, et seda kasutati hiljem 
erinevates muudes rakendustes. Loomulikult BBSidesse sissehelistamine toimus 
lihtsa terminaliga ehk et sa pidid nagu häkker käsustikku teadma. Enne 
helistamist pidi sisestama  ATDT, telefoninumber ja nii edasi, võib-olla veel 
seadistama protokolli. Loomulikult, tolleaegsed inimesed teavad täpselt, 
missuguse protokolliga vilistab  memcpy intro. See oli ka võib-olla selline 
asi, mis edaspidiselt  mõnes mõttes kaasa aitas.

\question{BBSil oli kliendisoft ka?}

Ei olnud. Helistasid terminaliga, kliendisoft oli ainult FidoNetil. Oli soft 
nimega FrontDoor, mis helistas, ja oli soft, mis pakkis kokku FidoNeti 
\emph{echo}d ja  saatis  selle paki edasi. Aga BBSil kui sellisel ei olnud 
kliendisofti. Läksid lihtsalt \emph{telnet}iga külge ja hakkasid seal edasi 
tegutsema.

\question{See läheb minu mäletamisega kokku küll. Oleks ju olnud loogiline, et 
keegi oleks mingisuguse tarkvara teinud BBSide ette, \emph{cache} jaoks 
näiteks?}

Jah, kui vaadata, mis Ameerikamaal toimus, kus siis olid need nii-öelda 
\emph{Online Service Provider}id  nagu AOL  ja 
CompuServe\index{CompuServe}\sidenote[][-1.7cm]{Interneti-eelsel ajal domineerisid USA 
turul agressiivsete turunduskampaaniatega (ühel hetkel oli pool \emph{kõigist} 
toodetud CDdest AOLi logoga) teenusepakkujad, kes pakkusid kummalist segu 
BBS-laadsetest ja Interneti-teenustest. Neist suurimad olid CompuServe, Prodigy 
ja America Online.} ja nii edasi, siis neil oli tarkvara. Ma mäletan seda, et 
kui ma USAs modemi  olin ostnud, siis loomulikult noorte poistena meil tuli 
seda proovida. Ja, kujuta ette, meil oli julgus kruvikeerajaga lahti keerata 
üks selline suur soliidne arvuti, see oli vist Computer 2000\sidenote{Computer 2000 
oli küll ka siinmail tegutsenud arvutiäri, kuid ilmselt peab Kain silmas Gateway 
2000 nimelist ettevõtmist, mis sama nime all personaalarvuteid tootis.} või mis iganes 
tolleaegne  selline hästi kõva valge PC bränd oli. Sõbra tädimees oli 
arhitekt, tal oli selline väike arhitektibüroo, ning meil oli julgus  
omavoliliselt kruvikeerajaga  lahti keerata üks nende suur \emph{tower} ja 
sinna sisse proovida seda sisemist modemit. Modemiga oli kaasas kas 
CompuServe'i või mingi muu sarnase teenuse CD plaat või flopi, äkki mäletan 
valesti. Ja siis sai helistatud Ameerika BBSi.  

\question{Kui sa nendes BBSides kolasid, kas sulle midagi muud peale tarkvara 
ka silma jäi? Raamatuid ja MODe sa mainisid?}

Raamatud mind eriti  tollel hetkel ei köitnud, BBSidest mina ikkagi laadisin 
peaasjalikult tarkvara ja siis muusika MODe. Aga kogu infovoog 
tuli FidoNetist. FidoNet oli minu jaoks täiesti puhas kulla-auk. Nagu varem 
mainisin,  mul ei ole olnud ligipääsu sellistesse teadusasutustesse või 
ülikoolidesse,  mul ei ole olnud piltlikult öeldes mentorit. Meil oli kamp 
poisse, kes omavahel  infot vahetasid. Meil ei olnud nagu sellist vanemat, kes 
teab, kuidas asjad käivad,  kõik käis katse-eksituse meetodil.

\question{Isegi hästi, et te kuidagi paha peale ei läinud selle kambaga. Noored 
poisid, tont teab, mida hakkavad tegema}

Ju siis me olime piisavalt mõistlikud. Ma arvan, et sellest ajast saadik on mul 
selline ise õppimise  oskus. Võib-olla see sai ka saatuslikuks, miks ma 
Tehnikaülikoolis ei suutnud väga kaua õppida,  ainult ühe aasta nagu tollel 
ajal võib-olla paljudel teistelgi kombeks.

Peale gümnaasiumi ma läksin kohe Tehnikaülikooli informaatikasse\index{Tallinna 
Tehnikaülikool!Informaatika}, aga kuna ma juba tollel hetkel töötasin, siis 
igasuguseid huvipakkuvaid projekte oli  palju kõrval. Mina eeldasin seda, et 
nüüd ma saan hakata programmeerimist ja igasugust muud sellist huvitavat asja 
õppima, aga siis tuli välja, et ei,  sa pead kõigepealt läbima füüsikad ja 
matemaatikad. Matemaatikast mul oli juba nagu natukene \enquote{kopp ees}, kuna 
meie meil oli selline väga püüdlik matemaatikaõpetaja gümnaasiumi ajal. Me 
tegelesime väga põhjaliku matemaatikaga, mingit sisse saamise probleemi 
Tehnikaülikoolis  absoluutselt ei olnud,  matemaatika eksamist lihtsalt 
lendasid läbi.

Ja nii see ülikool järgmisel aastal pooleli jäi.

\question{Kuidas sul kaitseväega on?}

Siis tuligi Kaitsevägi. Kui ülikoolis ei ole, siis varem või 
hiljem leitakse sind üles. Aga  Kaitseväkke ma läksin 1997. aasta suvel, ehk et 
ma olin siis juba aasta otsa Netit teinud. 

Ahjaa, et kuidas ma sinna sattusin. Töö Elektriside Inspektsioonis\index{Riigi 
Elektriside Inspektsioon}  hakkas natuke nagu ära tüütama. Nagu ikka,  tahad 
edasi areneda. Hakkasin otsima, et tahaks kuhugi  huvitavasse kohta tööle 
minna.  Mul mingil hetkel oli soov kindla peale töötada arvutifirmas, sest kuhu 
sa ikka lähed. Arvutifirmasse, seal olen kindel, et saan arvutitele väga 
lähedale.

Vanu \emph{backup}e läbi kammides jäi silma, mingil hetkel kandideerisin isegi 
Helmesesse\index{Helmes}, aga sinna ma ei saanud. Õnneks, tagantjärgi ma 
mõtlen. Keskkooli ja ülikooli vahelisel ajal suvel ma töötasin poolteist kuud 
Tõnu Samueli\index[ppl]{Samuel, Tõnu} IT-firmas nimega Eramees\index{Eramees} 
ja ma istusin samale kohale, kust oli just lahkunud Pronto\index[ppl]{Pronto}.  
 Tõnu ütles mulle, et Pronto müüs siin  neid Gravis 
Ultrasound\sidenote{Üheksakümnendatel väga populaarsed helikaardid, mis 
esimesena omataoliste hulgas suutsid toimetada päris instrumentide 
sämplingutega.} kaarte, et kuule, hakka nüüd sina sellega tegelema. Aga ma olin 
noor koolipoiss, ma ei  ei teadnud kaubandusest mitte essugi. Ma ma ei usu, et 
minust seal ettevõttes erilist kasu oli muidu kui nii-öelda patsiga poisist. 

\question{Ära ütle, päris mitmed inimesed kuni Tarmo Talini\index[ppl]{Tali, 
Tarmo} välja on mingil hetkel tegelenud müügitööga ja seejuures üldse mitte 
halvasti}

Eramehes üks asi, mis mul on veel eredalt meeles on, et Tõnu BBS oli siis 
kontoris. Kontor asus Eesti Talleksi majas, Mustamäe tee 1 vist, kui ma ei 
eksi. Ja BBS oli põhimõtteliselt  laiali laotatud arvutijupid  aknalaual. Seal 
oli siis USR Courier\index{US Robotics!Courier} modem\sidenote{US Roboticsi 
ülemise otsa Courier tooteliin oli oma töökindluse ja suurte kiiruste tõttu 
BBSide ja varaste Internetipakkujate lemmik, ka Eestis.},  emaplaat, toiteplokk  
ja nii edasi, lihtsalt hunnik juppe ja juhtmed, mis oli aknalauale laiali 
laotatud. Ja see oli siis Tõnu BBS või \emph{node}.

Pärast Erameest ma kandideerisin Estpak Datasse\index{Estpak Data}, sest mulle 
tundus, et ISP, et see on tegelikult veel huvitavam asi, sellepärast et nad 
tegelevad ju Internetiga.

\question{Kas Estpak oli tol ajal juba Eesti Telefoni oma või oli veel eraldi?}

Ta oli tollel ajal eraldi. Kui õieti mäletan, siis Estpak Data omanikuks oli 
siis Eesti Telekom\sidenote{Eesti Telekom pika nimega Riigiettevõte Eesti 
Telekommunikatsioonid oli Teede- ja Sideministeeriumi haldusalas töötav 
\emph{holding}-ettevõte, mis valdas Eesti Telefoni, Eesti Mobiiltelefoni, Eesti 
Kaugotsingu, EsData, Estpak Data ja TeleMedia aktsiaid. Hiljem viidi ettevõte 
börsile ja sealtkaudu sai tema ainuomanikuks Telia.}, mitte  Eesti Telefon, ta 
oli täiesti eraldiseisev ettevõte Eesti Telefonist. Huvitaval kombel kellelgi 
oli tulnud selline idee, et meil on kuidagi vaja edendada veebi  
virtuaalhostimist. Keegi oli välja mõeldud neti.ee\index{neti.ee} nimelise 
domeeni ja selle domeeni alt siis üritati müüa sellist traditsioonilist 
veebihostingut. Tollel ajal ta veel traditsiooniline ei olnud, aga ütleme, et 
siis tänapäeva mõistes. Ja  Estpak Data palkas mind kui nii öelda webmasterit, 
kes pidi hoolitsema veebi hostinguserveri ja teenuse eest. Ja siis muu seas oli 
neil selline idee, et kuidas me seda veebi hostingu äri ikka muud moodi 
edendame, kui meil on vaja mingit kataloogi. Inimesed peavad ju leidma üles 
need veebilehed, mida  kliendid sinna panevad.

\question{Kas tol ajal Meediamaa oli juba olemas?}

Meediamaa\index{Meediamaa} startis umbes samal ajal. Enne seda oli olemas  
Eesti veebisaitide nimekiri, mis oli nlibi ehk siis 
Rahvusraamatukogu\index{Rahvusraamatukogu} domeenis, kus Toomas 
Mölder\index[ppl]{Mölder, Toomas} tegutses. Ja Toomas Mölder kolis, ma arvan, 
sellesama nimekirja Meediamaasse ja sealt www.ee\index{www.ee}'sse. Kuna 
Meediamaa üks tegelane oli Tarvi Martens\index[ppl]{Martens, Tarvi}, siis neil 
õnnestus kuidagi EENetilt\index{EENet} välja meelitada domeen nimega 
www.ee\sidenote[][-1.5cm]{Alates oma asutamisest 1993. aastal kuni 2013. aastani oli 
EENet .ee domeeni registrar ja sellisena rakendas mitmeid suhteliselt rangeid 
reegleid. Näiteks oli domeeni registreerimine küll tasuta, kuid ühel 
organisatsioonil tohtis olla vaid üks domeen.}. Ma arvan, et mitte kellelegi 
teisele kui Tarvile ei oleks sellist domeeni elu sees välja antud.

\question{Seda ma kujutan ette küll. Kas sa seda kataloogi tegid siis käsitsi 
alguses?}

Jah, alguses alguses sai seda kataloogi käsitsi tehtud. Ta oligi selline väga 
algeline ja puine. Aga asi hakkas lendama siis, kui kui ma kutsusin kataloogi 
puhul endale appi Jaanus Vainu\index[ppl]{Vainu, Jaanus}, kellega ma olin kokku 
saanud Riigi Elektriside Inspektsioonis\index{Riigi Elektriside Inspektsioon}. 
Jaanus on ka omamoodi huvitav tegelane. Elektriside Inspektsioonis tema mõtles 
välja kogu meie FM 108 sageduse plaani, ehk kõik Eesti raadiojaamade 
sagedusnumbrid on tema tehtud. Nõukogude ajal oli meil teistsugune FM 
sagedusala -- kuidas saab nii, et sa saad poest osta raadio, millega saab 
välismaa raadiojaama kuulata. See ei sobinud kuidagi, Eesti Vabariigi alguses 
koliti Lääne sagedustele üle. Jaanus oli üks nendest, kes käis mööda Eestit  
mõõtmas ja tegi sagedusplaani. Tal oli väga detailselt Corel Draw's\index{Corel 
Draw} joonistatud kõik need nii-öelda sagedusringid Eesti kaardi peale. Eesmärk 
oli  planeerida sagedused niimoodi, et üle Eesti saatjatel oleksid sagedused, 
millel on võimalikult vähe häireid naaberriikidega ja omavahel.  

\question{Kogu seda teadust tehti Corel Draw abil?}

Jah. Jaanus on selline tohutu pedant,  tohutu  töövõimega katalogiseerija. 
Tema enda isiklik huvi on \emph{bluegrass}. Mäletan seda, et tema oli esimene 
inimene, keda mina tean, kes välismaalt e-poest asju tellis.  Tema tellis 
CDNow'st\sidenote{CDNow oli 1994. aastal asutatud Interneti-põhine muusikamüüja, 
kes paraku esimest dot-com-mulli üle ei elanud ja sajandivahetusel uksed 
sulges.}  plaate endale. Mina väga imestasin, et kuidas selline asi üldse 
võimalik on. Et ta tellib kuskilt, Jumal teab kust ja tulebki pakiga kohale CD 
muusikaga.

\question{Jaa, isegi üheksakümnendate lõpus oli Amazonist raamatute tellimine 
suhteliselt eksootiline tegevus. Aga mis hetkel ja kuidas te 
neti.ee\index{neti.ee} ära automatiseerisite?}

See meie tandem Jaanusega töötas selles mõttes ülihästi, et mina olin  
programmeerija ja arendasin tarkvara ja Jaanus oli  katalogiseerija.  Kui Jaanus 
selle projektiga liitus, siis võiks öelda, et projekt hakkas täielikult 
lendama. Ma arvan, et meil läks võib-olla paar kuud aega, kui me olime 
Meediamaast\index{Meediamaa} igatpidi kõikide näitajate poolest mööda läinud. 
Me olime tollel ajal võib-olla isegi natukene liiga ebaviisakad noored mehed. 
Näiteks me reklaamisime netit spämmides, tegime  ühe korra sellise 
masspostituse, saates kõikvõimalikele meiliaadressidele teate, et \enquote{nüüd 
on selline huvitav teenus olemas nagu neti.ee, tulge, külastage}. Midagi 
sarnast. Kusjuures huvitav on see, et kui ma vaatasin enda \emph{backupe}, siis 
ma nimetasin enda \emph{crawlerit} Nuhiks, seda otsingurobotit, kes mööda lehti 
ringi kolab. 

Ja huvitaval kombel ma olin selle Nuhi programmeerimist alustanud juba mitu kuud 
varem ehk nagu nagu miski oleks suunanud mind sellele teele, et seda võib vaja 
minna. Ja  otsingumootoreid ma olin ka natuke varem teinud. Kui ma pärast 
Erameest  ülikooli läksin, siis  üks sealt saadud kontaktidest kutsus mind 
tegema ühte ärikataloogi sarnast teenust,  mille pealkiri oli Bartanet. See 
asus EsData\index{EsData} serveris, oli mingi Suni server Akadeemia tee 21  
teisel korrusel, samas majas, kus me hetkel viibime. Ja selles Suni serveris, 
ma ei tea, mis asjaoludel, aga ma millegipärast sain seal teha FTP-serverite 
otsingut. Ma panin seal püsti otsinguteenuse nimega Filerix, mis töötas umbes 
kolm-neli kuud, mille ainukeseks sisuks oli see, et ta võimaldas väga hõlpsasti 
faile üles leida igasugustest kohalikest FTP \emph{mirror}itest. Tollel ajal  
Marek Tiits\index[ppl]{Tiits, Marek} IBSist\index{Institute of Baltic Studies} 
hostis sellist asja nagu TuCows\sidenote{TuCows (\emph{The Ultimate Collection 
Of Winsock Software}) keskendus oma algusaegadel tasuta tarkvarale. Kuna 
Interneti kiirus sõltus veel väga suurel määral geograafiast, opereeris 
ettevõte skeemi, kus  huvilised võisid jooksutada TuCows.com lehekülje 
lokaalseid peegleid. Ühte sellist Marek pidaski.}. Minu  otsingumootor 
võimaldas hõlpsasti failinimede järgi üles leida tarkvara tolleaegsele Windows 
95'le, vanadele Windowsidele ja nii edasi. Tollest pooleaastasest projektist nii-öelda kõrvalprojektina ma tegin failiotsingut.

\question{Suure hulga failide indekseerimine ei ole enam päris naljaasi ja 
eeldab programmeerimisoskust. Kust sa selle üles oled korjanud?}

Tollel hetkel ma oskasin programmeerida Perli\index{Perl} ja siis kõike 
seda, mis  Unixi \emph{shell}is saada on. See tuligi  sellest ajaperioodist, 
kui ma uurisin, mis on nii-öelda Unixil  kõhus.

\question{Ise korjasidki üles selle algoritmika ja muu sellise?}

Jah, mis puudutab veebi \emph{crawl}imist,  siis jah, selle peale tuli juba 
mõelda.

\question{Puhtalt konteksti pärast, kaua su \emph{crawler}il aega läks, et 
kogu Eesti veeb üle käia?}

Ma arvan, et see oli mingi stiilis ööpäev või midagi sellist, sest veeb oli 
tollel ajal väga väike. Ma täpselt pole vaadanud, aga ma usun, et selle 
kataloogi suurus oli võib-olla paar tuhat linki ja mitte rohkem tollel ajal. Ja 
keskmine koduleht oli ka selline kolm kuni viis lehekülge, et see ei  olnud 
eriline teema. Huvitavamaks läks pärast, siis kui see linkide hulk juba 
miljonitesse läks, siis mingil hetkel oli ikka selline \emph{crawler}, mis 
töötas paralleelselt  paljudes \emph{thread}ides ja nii. Aga noh, see oli kõik 
selline loomulik evolutsioon. 

Aga jah, ma mäletan tegelikult, et miks ma arvan, et miks mind Estpak 
Datasse\index{Estpak Data} tööle võeti. Ühe sellise kõrvalprojektina ma olin 
teinud HTML-i tutvustuse. Ma arvan, et mul oli vist koolis olnud vaja seda 
kellelegi õpetada. Ehk siis gümnaasiumis tollel perioodil, kui ma siis Keila 
seda serverit administreerisin. Ja siis mulle tundus, et ma seda ikka õpetan, 
mingit eestikeelset materjali pole ja siis ma tegin ühe esimese  eestikeelse 
HTML-i tutvustuse, mis võttis läbi kõiki üksikuid elemendid. 

\question{Millegi pärast tuleb see maru tuttav ette, ma arvan, et ma olen sealt 
mingeid asju otsinud}

Kusjuures seesama HTML tutvustus on sellel samal aadressil täna ka üleval ja ma 
olen üsna kindel, et see on üks kõige vanemaid veebilehti, mis leidub täna 
Eesti veebiruumis, mis on originaalkujul originaalaadressil. 

\question{Mis aastast see on?}

1996. Ja siis veel ühe projektina ma olin teinud veebi pokkeri, sellise  
veebipõhise mängu. Selles mõttes,  ei saa öelda, et  mul pole kunagi huvi olnud 
ka mänge teha, aga ma olen  rohkem  oma elus programmeerinud nii-öelda 
veebi-asju, kui \emph{desktop}is või masinas töötavaid rakendusi. 

Nende teadmiste baasil mind sinna Estpaki  siis tööle võeti. Tõenäoliselt ma  
näitasingi seda, et vaadake, ma olen teinud sellise veebi pokkerimängu, ma olen 
teinud HTML-i tutvustuse ja võib-olla ma rääkisin ka seda, et ma olen selle 
\emph{crawler}i teinud. Igal juhul mind võeti sinna tööle ja ma sain jätkata 
sellesama koha pealt, kus ma juba olin.

\question{Kes teil seda toote poolt tegi, või polnud niisugust mõistet, nagu 
tootejuht?}

Ei olnudki. Piltlikult öeldes pandi mind  istuma, et palun istu siia ja tee. 
Tegelikult  see oli ikkagi läbi mõeldud mõnes mõttes. Estpak Data\index{Estpak 
Data} tegi koostööd ühe reklaamiagentuuriga, mis rentis ruume Kullo majas 
Mustamäe teel. Nii et tegelikult minu füüsiline töökoht  asus selles 
reklaamiagentuuris Kullo majas. Minul oli arvuti, millel oli püsiühendus 19.2 
kilobitti sekundis ja sealt ma töötasin. Noh, noore mehena nagu ikka, et sind 
ei huvita, kuidas rahad liiguvad ja nii edasi,  sind huvitab ainult see 
tehniline pool. Idee siis seisnes selles, et reklaamiagentuur aitab 
potentsiaalsetel Estpak Data klientidel teha kodulehti, aitab teha neile 
reklaami ja nii edasi, umbes selline kokkulepe oli. PRC Nord Decor
oli tolle agentuuri nimi, ma ei tea, kas see kellelegi midagi ütleb. Aga, aga 
selles mõttes oli huvitav, et üks kolleeg, kes Nord Decoris töötas, oli kunagise 
OK Jutuka\index{OK jutukas}\sidenote{OK Jutukas oli üks esimesi tõeliselt massidesse läinud 
sotsiaalvõrgustiku laadne rakendus Eestis. Jututube - kohti, kus sai üle 
telneti kaaskodanikega suhelda - oli  veel, aga 1996. aastal käivitatud OK oli 
üks esimesi veebipõhiseid jutukaid ja tõenäoliselt omataolistest siinkandis 
suurim. Üheaegselt lobises omavahel kuni 300 inimest ja jutuka esimese 
aastapäeva pidu kajastas isegi toonane Päevaleht.} üks asutajatest. Mitte  
Kaupo Kalda\index[ppl]{Kalda, Kaupo}, aga Tiit Sermann\index[ppl]{Sermann, 
Tiit}. Kusjuures oligi nagu naljakas, et tema alias oli Ott\sidenote{OK Jutuka nimi tulenes siis asutajate nimedest: Ott ja Kaups.}, aga tegelikult tema 
päris nimi oli Tiit. Lihtsalt selline huvitav asi. Kuidagi tundub, et kogu see 
maailm oli tollel ajal nii pisikene, et kui sa natukene selles maailmas ringi 
käisid, siis sa puutusid paratamatult kuidagi kõikide nende inimestega kokku, 
kes tollel ajal toimetasid.

\question{Räägi korra palun sellest, kuidas te Hoti tegite?}

Ja,see oli tegelikult ka päris huvitav. Kaitseväest tagasi tulles oli Eesti 
Telefon\index{Eesti Telefon}  Estpak Data\index{Estpak Data} ära söönud, Estpak 
Data lakkas olemast. Mingil hetkel ma töötasin siis Lasnamäel Koorti 15, kus 
Estpak Data enne oli, vana Eesti Telefoni maja, aga siis õite pea koliti meid 
sealt siis päris Eesti Telefoni muudesse ruumidesse ära. Ma olin Eesti Telefoni 
sellises allüksuses, mille pealkiri oli Teleteenuste Arendus.  Eesti Telefon 
oli teleteenuseid pakkuv ettevõte ja too üksus oli siis Eesti Telefoni 
arendusüksus, kelle eesmärk oligi välja töötada uusi teenuseid. Ja siis oma 
neti.ee tegemisega me sinna sattusime. 

Kontoriruumi jagasin ma ühe teise noormehega, kes arendas 
sissehelistamisteenust. Ja  huvitaval kombel meil vedeles kapi peal üks pisike 
Ascendi sissehelistamiskeskus, seal väga palju liine ei olnud. Ma küsisin, et 
kas ma võin seda uurida.

\question{See oli siis mingi tükk riistvara? Seal käisid tavalised modemid 
külge või oli ta juba valmis lahendus?}

Ei, ta oligi \emph{dedicated} sissehelistamiskeskus, et sa põhimõtteliselt 
installeerisid ta \emph{rack}i, panid  juhtmed külge ja ta hakkaski numbreid 
kuulama ja teenust osutama. Aga miks ma seda räägin, on see, et tolle keskuse 
uurimise käigus ma avastasin selle, et  sissehelistamiskeskus autendib ennast  
vastu sellist autentimisserverit nagu Radius. Sealt edasi uurisin, et mis asi 
see Radius on, sain teada, et see on \emph{dictionary}-põhine protokoll, 
üldsegi mitte keeruline ja ma programmeerisin siis Radiuse serveri, kes  suutis 
sissehelistamiskeskust juhtida. Avastasin, et selle sissehelistamiskeskusse 
\emph{firmware} võimaldab igasuguseid huvitavaid asju, mis tundusid olevat nagu 
seni kasuta. Näiteks see, et sa võid kohe Radiuse serverist öelda 
sissehelistamiskeskusele, kui kaua see kasutaja võib ühenduses olla. Ja sellest 
teadmisest näiteks sündis selline toode nagu Atlas Surf\index{Atlas Surf}, mida 
Eesti Telefon \emph{prepaid} Internetina \sidenote{Sarnane kontseptsioon nagu mobiili kõnekaardid.} müüs. Ühe sõnaga, see toode sündis 
puhtalt sellest, et mina häkkisin  seda väikest sissehelistamiskeskust, mis oli 
tegelikult üldse mõeldud  mobiilidega sisse helistamiseks. Ta toetas sellist 
huvitavat protokolli nagu V.35. Paljud pole sellest ilmselt mitte kunagi 
kuulnud, aga see oli selline \emph{wideband} protokoll, mis töötas üle GSMi. 
Kui sul oli selline GSM telefon, mida sai arvutiga ühendada, siis ta võimaldas 
sisse helistada selle V.35 protokolliga ja sa said veidi suurema kiiruse kui 
tavalist modemit vilistades üle  mobiili. 


Võib olla korraks hüppan natuke tulevikku. Oli aasta kaks tuhat, kõik mäletavad 
Y2K\sidenote{\enquote{Sinu lapselapsed neavad päeva, mil sa otsustasid oma 
koodi optimeerida}. Kuna pikka aega leiti aastaarvu hoidmiseks kahekohaline 
number piisav olevat, tehti sajandivahetuse paiku üüratus koguses tööd ja raha 
tagamaks, et aasta 2000 ei oleks arvutite arvates võrdne aastaga 1900. Vaata ka 
\enquote{Aasta 2038 probleem}.} probleem, kohutavalt hirmus, sest arvutid 
lähevad katki sest nende kell  lakkab töötamast ja nii. Ja ka Eesti Telefonis 
kardeti seda,  \emph{legacy} süsteeme oli tohutu palju. Kõik insenerid, kes 
olid mingisuguste süsteemidega seotud, pidid jääma  valvesse. Ma ei tea kuidas, 
minul õnnestus sellest ära nihverdada niimoodi, et tol hetkel ma olin Soomes 
suusatamas, sõpradega lumelauaga mäest alla laskmas. Stiilis paar päeva enne 
aastavahetust tuleb mulle  klienditeenindusest kõne, et kuuled, et nüüd sisse 
helistada enam ei saa, et mingi jama on. Läksin siis autoni, mul oli läptop 
kaasas. Olles Soome Vabariigis, panin telefoni läptopile järgi, helistasin 
sellesse meie enda privaatkeskusse sisse sellesama V.35  protokolliga ja 
hakkasin  vaatama, et mille pärast siis Hoti kliendid sisse helistada ei saa. 
Tuli välja, et keegi oli veel viimasel hetkel mingisuguse turvapaiga peale 
laadinud Y2K hirmus ja see muutis natukene seda teadet, mis Radiuse serverile 
saadeti ja siis Radiuse server läks selle peale katki, kuna talle tuli tundmatu 
sisuga \emph{dictionary}.

Surfist edasi tekkis selline olukord, kus Eesti Telefoni kontsessioonileping 
oli juba lõppenud või lõppemas, ja turule tuli Tele2\index{Tele2} Rootsist. 
Tele2  idee oli korrata Eestis täpselt sama, mida ta tegi Rootsis, ehk et ta 
soovis sellelt suurelt \emph{telco}lt palju raha välja imeda. Kuna Eesti 
Telefon üüris ruume, liine ja nii edasi, oli meil teada, et Tele2 paneb oma 
sissehelistamiskeskuseid püsti. Eesti Telefoni juhtkond oli sellest paanikas, 
ma ise ka külastasin mingisugust sellist laiendatud juhatuse koosolekut, kus 
sellest arutati. Ma mäletan, et ma tulin sealt üsna mornilt tagasi. Mulle 
tundus, et vanemad kolleegid ei suuda  midagi otsustada või ära teha. Mina 
sellise noore mehena oleks tahtnud kohe rauh-rauh, et läksime. Ma ei mäleta, 
mis asjaoludel ma olin kodus, aga ma pidasin siis telefonikõne Priit 
Pirsoga\index[ppl]{Pirso, Priit}, kes oli tollel hetkel selle valdkonna juht 
Eesti Telefonis. Ja selle telefonikõne käigus me otsustasime, et me teeme Eesti 
Telefoni osutatavale Atlas Starter teenusele alternatiivse teenuse, sellepärast 
et Atlas Starter absoluutselt ei sobi Tele2'ga konkureerimiseks. Meil on vaja 
sellist teenust, kus kasutajate  registreerimise protseduur ja selline on 
automaatne. \emph{Self-service}, kasutaja saab ennast ise registreerida ja nii 
edasi. Kuna  kuutasu niikuinii pärast seda Tele2 jampsi enam ei ole, siis 
ainukesed, mis maksavad, on kõneminuti hinnad. Tele2  lootis  raha teenida 
sellest, et ta termineerib kõnet ja Eesti Telefon on sunnitud talle 
nii vahendama kliendi käest küsitud kõneminuti hinda. 

Selle telefonikõne käigus me leppisime kokku, kes mida teeb, kuidas teeb ja ma 
olen ka üsna kindel, et selle kõne käigus me leppisime kokku, et toote saab 
nimeks saab Hot\index{hot.ee}. Sest ma muu seas juba arvutist vaatasin, et 
millised huvitavad domeenid on meil vabad. Kusjuures tollel ajal oli veel see 
aeg, kui EENet\index{EENet}  ei nõustunud andma ühele ettevõttele mitut 
domeeni. Mina ei tea, kuidas, aga minu üks tänane kolleeg, Guido 
Kõiv\index[ppl]{Kõiv, Guido}, temal õnnestus kuidagi EENetist saada 
hot.ee domeen meile, ma ei tea, kuidas. Sarnane \emph{inside}, nagu nagu 
Tarvil\index[ppl]{Martens, Tarvi} oli www.ee'le. Igal juhul me saime selle ühe 
või paari telefonikõnega, väga lühikese ajaga kokku lepitud, kes mida teeb. Ja 
kujutad sa ette, kahe nädala pärast me olime \emph{live}'is. See tähendab, et 
meil toimus teenuse \emph{launch} ja meil hakkas kasutajaid registreeruma 
tempoga tuhat tükki päevas.

Sealt sai siis hot.ee alguse. Minu teha jäi  seesama Radiuse pool. 
Hoti\index{hot.ee} omaaegne sisu tegelikult oli järgmine. Meie huvi oli see, et 
inimesed helistaksid meile sisse.  Tollel ajal hakkas ka juba olema kombeks, et 
anname ka kasutajale e-maili. Aga kuna varasemalt küsi e-maili eest raha, siis 
meile tundus, et lihtsalt nii samas neid e-maile jagada ei tahaks. Ja siis sai 
tehtud  selline kriuks, et sa saad küll veebipõhiselt konto luua (kusjuures 
imelik \emph{chicken-and-egg} probleem, et sul on Internetiühenduse konto 
loomiseks Internetti vaja, aga tundus, et see ei olnud takistuseks), aga see  
meilikonto ja ka kodulehekonto ei hakanud enne tööle kui sa olid selle 
registreeritud kontoga vähemalt ühe telefonikõne teinud 
sissehelistamiskeskusesse. Seda loogikat võimaldas siis minu \emph{custom} 
Radius, kes kasutajatel järge pidas. 

\question{Ühel hetkel oli hot.ee-s veebimeil ka, eks ole?}

Veebimailer oli, ma arvan, et suhteliselt algusest kohe juba sellesama esimese 
kujunduse osa juba. Aga see ei olnud minu programmeeritud, see oli Internetist 
leitud vabavara, mida me saime kasutada. Ma arvan, et me isegi seda nii-öelda 
ei \emph{re-brand}inud enda värvidesse, vaid see oli lihtsalt meie lehelt 
lingitud. Me ise hostisime teda.

\question{See seletab, miks meil mõned aastad hiljem veebimeileri tegemine 
Hansapangas\index{Hansapank} nurja läks, meil miskipärast ei tulnud pähe mõtet 
see lahendus Internetist lihtsalt alla laadida}

Mulle ei tulnud selline mõte pähe, et seda ise teha. Küll aga mäletan seda, et 
hiljem kui keegi mäletab, oli selline huvitav protokoll nagu WAP. Ehk siis 
mobiilivariant Internetist\sidenote{\emph{\enquote{WAP - Wireless Application 
Protocol}} oli sajandivahetuse paiku tekkinud ja põgusalt ka kasutusel olnud 
katse luua toona kasinate sidevõimalustega mobiiltelefonide jaoks lihtsam pinu 
internetiprotokolle 4. kuni 7. OSI kihini. Muu hulgas sisaldas standard ka 
erilist \emph{markup}-keelt toonase mobiiltelefoni mõnerealisele ekraanile 
sobivate kasutajaliideste loomiseks}. Vot selle WAP-meili ma küll tegin täiesti 
nullist sellelesamale Hotile.

\question{Mis õnneks ei olnud väga pika elueaga sest WAP ei olnud väga pika 
elueaga}

Mina mäletan ka seda legendaarset väidet Ando Meentalolt\index[ppl]{Meentalo, 
Ando}, kes oli tollel ajal EMT üks arendusjuhte, kes kommenteeris minu WAP 
meili nii, et \enquote{noh, sa võid ju sinna suahiili keele ka panna, aga 
ilmselt pole sellest väga palju kasu}. Aga kogu see WAP sai minul isiklikult 
alguse sellest, et ma olin saanud endale WAPi-võimelise telefoni. Ma arvan, et 
see oli üks ainukesi telefone, mida ma olen iialgi tööandjalt saanud. See oli 
siis Nokia 7110\sidenote{Tegu oli 1999. aastal uskumatult innovatiivse 
telefoniga: mitut tekstirida näitav ekraan, rullikuga kasutajaliides, T9 
ennustav tekstisisestus sõnumite puhul, vedruga uhkelt lahti hüppav klapp, WAP, 
ebamaiselt küütlev korpus jne. Oma isepärase kuju tõttu sai aparaat rahva seas 
hüüdnimeks \enquote{banaan}}, klapiga telefon, millel oli suur ekraan. 

\question{See telefon oli muide suurepärane põhjus Hansapangale WAP-i põhine 
internetipank teha. Sest selle testimiseks pidi ju pank ometigi väljastama ka 
sobiliku seadme}

Mul oli \emph{vice-versa} selles mõttes, et ma sain kõigepealt telefoni, siis 
mul tuli idee, et mul telefon nüüd on, aga mida ma sellega teen  ja et jube äge 
oleks enda postkasti sisse vaadata sellisel mugaval moel.  Ja siis ma tegin 
WAP-meili.

\question{Aga sellega algab juba uus sajand ja sellest me räägime võib olla 
mõni teine kord. Lõpetuseks küsin veel, et mis sa praegu teed?}

Praegu ma olen Bolt serveri infra peal.  Minu üks kauaaegseid 
kolleege sealt samast Eesti Telefonist Tarmo Kople\index[ppl]{Kople, Tarmo} on 
üks nendest inseneridest, kellega me alustasime Bolti kogu seda 
serverimajandust praktiliselt juba algusest peale. Kui meie alustasime Tarmoga 
serverite poole majandamist Boltis, siis meil oli stiilis tuhandeid kliente ja 
tuhandeid sõitusid kuus ja nüüd see on siis asendunud miljonite klientide ja 
miljonite sõitudega.


\chapter{Andres Kütt}
%!TEX TS-program = arara
% arara: myindex

\textbf{\enquote{Kuidas sa arvutite juurde jõudsid?}}

Sündisin 1975. aastal Võrus\index{Võru}. Millestki midagi aru saama hakkasin mälu järgi kaheksakümnendate teisel poolel. See oli mitmes mõttes üsna kole aeg. Noorukile kõige arusaadavam neist koledustest oli lihtlabane praktiline puudus. Päris nälga ei olnud aga midagi vähegi leivast ja piimast edevamat saada ei olnud. Kui linnakeses levis kuuldus, et olla toodud kast jäätist, oli poes veerand tunniga saba ning poole tunni pärast kõik otsas. Muu hulgas oli kaubandusvõrgus saada kahte tüüpi meeste talvejopesid. Mitte kahtekümmet ja mitte kahtesadat vaid kahte. Ühed olid hallid ja neid said lihtsurelikud osta\sidenote{Huvitaval kombel oli tolle jope põuetasku 5.25" lai, sinna mahtus üks flopi täpselt sisse} ja teised olid punase suure a-tähega ja neid said osta ainult inimesed, kes teadsid kedagi, kes teadis kedagi. Ajad olid sellised. Kõige hämmastamaval moel käisid ka seda viletsust inimesed Pihkvast bussidega uudistamas ja viimastki kaupa ära ostmas. 

Aga kogu selle halluse keskel suutis Nõukogude Liit meie Võru Kreutzwaldi Gümnaasiumile\index{Koolid!Võru Kreutzwaldi Gümnaasium} tarnida arvutiklassitäie arvuteid Agat\index{Arvutid!Agat}\sidenote{Agat oli Nõukogudemaal valmistatud arvuti, mis oli küll Apple II\index{Arvutid!Apple II}'st inspireeritud, kuid siiski mitte täpne kloon}. Kust nad tulid ja kes seda asja ajas, ei tea. Küll aga mäletan, et nende saabumine oli pikalt oodatud ja edasi lükatud. Miks ja mida oodatud sai, ei oska öelda. Tean ainult seda, et kui klass tekkis, läksin ma sinna sisse ja enam välja ei tulnud. 

Ega tolle purgiga palju teha ei olnud. Olid mõned mängud ja programmeerimiseks Basic. Tolles meid programmeerima õpetatigi. Esimese hooga ei õpetatud seejuures mitte kõiki käske, näiteks for-tsükkel oli tükk aega saladus. Kui aga nohikud aru said, et nende eest tarkust varjatakse, kadus igasugune respekt ja läks lahti suuremaks isepusimiseks. Kõik muutus, kui kooli saabus noor, minu meelest värskelt ülikoolist tulnud, arvutiõpetaja Aivar Halapuu\index[ppl]{Halapuu, Aivar}. Temaga tekkis kohe mingisugune pool-kamraadlik side, mis siiski alati suurt kogust meiepoolest lugupidamist sisaldas. Tolleks ajaks oli meil tekkinud väiksem seltskond poisse, kes seal klassis toimetas ja kes kohe end \emph{in corpore} Aivarile sappa haakis. Aivar viitsis meiega tegeleda ja, kuigi ta meile suurt midagi arvutite mõttes ei õpetanud, sai tema käest midagi, mida vist kultuuriks nimetatakse. Meiega üritati bridži mängida, räägiti mänguteooriast ja nii. 

Kuna me seal klassis sisuliselt elasime, siis usaldati meie kätte üsna pea ka arvutiklassi võti. Aga \emph{kooli} võtit meie kätte keegi ei andnud. Seetõttu oli oluline hoida järjepidevust: keegi oli alati klassis olemas ja hõikamise või kivikese viske peale lasi tulija sisse. Mingitel tingimustel oli meie käes siiski ka välisukse võti aga tihti roniti ka aknast. 

Ühel hetkel avanesid kraanid ja saabus humantiaarabi. Võrus oli vist seoses rahvamuusikaga igasugu põnevaid suhteid välismaa asutustega, kes hakkasid meie suunas igasugu põnevat kola saatma. Saabus klassitäis mingeid rootsikeelsete paberite ja tarkvaraga masinaid, millega me mitte midagi teha ei osanud. Mis neist sai, ei tea. Aga tuli ka mingi iidne aparaat, mille külge käis neli-viis terminali ja kaks kokku külmkapisuurust kettaseadet. Seadmete sisse käisid hiigelsuured plastkarbis kettad. Tegu oli industriaalseadmega: kui tuurid sisse võttis, siis oli alla tänavale kuulda, et \enquote{arvuti töötab}. Tolle masina peal midagi tarka teha ei osanud keegi, tarkvara polnud. Sai mingeid mänge mängitud ja see oli ka kõik. Mäletan siiski, et seal puutusin esimest korda kokku Zorki\index{Mängud!Zork} nimelise mänguga\sidenote{Zork on üks varasemaid teksipõhiseid arvutimänge. Mängija sisestas teksti ja talle ka vastati tekstiga vastavalt sellele, mis mängus parasjagu juhtus. Kuna mängu alguses sattuti lagendikule valge maja ette, oli meie puhul ilmselt tegemist Zork I-ga}.

Lõpuks tulid meile Jukud\index{Arvutid!Juku} ja üheksakümnendatel lõpuks ka pc-d. Jukusid oodati väga, sest Agat oli päris jube aparaat\sidenote{Ma ei ole kunagi hiljem kohanud arvutit, mis suudab flopiketta füüsiliselt ära rikkuda}. Ja Jukud olidki väga ägedad, ainsaks nõrgaks kohaks oli minu mälu järgi klaviatuur. Ainus, mis palju ei muutunud, oli tarkvara. Võru ei ole Tartu ega Tallinn. Meie seltskond ei suhelnud õieti kellegagi, ei uut tarkvara ega teadmist ei tulud eriti kuskilt peale. Ajakirjast \enquote{Arvutustehnika \& Andmetöötlus} võis küll lugeda Unicode võludest aga programmeerida tuli ikkagi kas assembleris või basicus. Seejuures sain alles hiljem teada, et eksisteeris ka asi nimega makro-assembler. Tavalises pidi JMP käsule argumendiks andma suhtelise aadressi (mis muidugi kohe valeks osutus, kui kuskile mingi rea vahele panid)\sidenote{See oli probleemi minusugustele surelikele. Inimesed nagu klassivend Vallo Trell\index[ppl]{Trell, Vallo} suutsid ka otse BIOSi prompti peal mällu baite kirjutades masinkoodis programmeerida} aga tolles uuemas sai silte kasutada. Mingitel üritustel sai Tallinnas käidud (mäletan Pedas\index{Pedagoogikaülikool} asunud MSXide\index{Arvutid!Yamaha MSX} klassi) ja sealt ka mingit tarkvara kaasa toodud aga üldiselt olime üsna omaette. Isegi flopisid käisime ostmas Tallinnas, seal oli teada üks komisjonipood, kust selliseid sai. Tavaliselt kasutati ära mõnda käiku teatrisse, reeglina jäi kuhugi paar tundi linnas kolamise aega. 

Olin ka üks õnnelikest, kellele lõpuks arvuti suveks koju usaldati. Esmalt Agat, siis Juku. Kuna ekraanid olid mõlemal nigelad, veetsin kaks või kolm suve ette tõmmatud kardinate taga arvutiga toimetades. Juku peal mäletan kahte suuremat projekti. Esimene oli Norton Commanderi moodi failihaldur ja teine fondiredaktor. Jukul sai tähekujusid suhteliselt lihtsasti ümber teha, mälus olid vist kaheksabaidised bitimaatriksid ning teksti kuvamine käis kiiremini, kui muu graafika. Mõlemat kirjutasin assembleris ja kumbki päris valmis ei saanudki, sest teatud mahust alates muutus kood hoomamatuks. Sel ajal omandasin ka pärast palju vaeva põhjustanud kombe \enquote{tunde järgi} koodi kirjutada. Teed muutuse, kompileerid, proovid, muudad pikalt mõtlemata uuesti. Kood oli nii kole, et selle iga kord uuesti läbi mõtlemine oli liiga keeruline ja mingid \emph{off-by-one} vead olid sagedased, reeglina sai mingi konstandi ühe võrra nihutamise peale koodi käima. Sellest rumalast kombest pole ma siiani lõpuni vabanenud. 

Aga Juku peal sai ka andmebaase teha, täitsa oli olemas dBASE\index{dBASE}. Selle abil õnnestus maik suhu saada kellelegi arvuti abil kasulik olemisest. Koolivend Aini dieedi-teemalise uurimistöö jaoks tegin andmestiku ja kirjutasin ka programmi kassetiümbriste trükkimiseks. Tollal käibis muusika kassettidel, mida ohtralt kopeeriti\sidenote{Eksisteeris ka tänapäeval mõeldamatu täiesti põrandapealne muusikakopeerimise asutus, selline oli ka Tartus. Läksid kohale, valisid kataloogist albumi välja, jätsid tühja kasseti maha ja mõni päev hiljem sai sobiva summa vastu muusikaga kasseti tagasi}. Seetõttu kirjutati lugude nimesid käsitsi ning see oli tüütu. Minu tarkvara võimaldas aga kiiresti eri plaatide jaoks kassetiümbrised trükkida. Selle teenuse eest sai vist ühelt klassivennalt isegi raha küsitud.

Linna peal eri kohtades sai ka PCdega tutvust teha. Mööblivabrikus oli kellelgi tutvusi, seal toimus isegi mõned korrad mingisugune õpe. Istusime ilmselt raamatupidamise masinate taga ja meile näidati, kuidas FoxPros\index{FoxPro} vorme joonistada ja andmeid hoida. 

Keskkoolis õnnestus käia väga murdelistel aastatel 1990-1993. Võrus möllas punkar Saare Ain\index[ppl]{Saar, Ain}\sidenote{Kodanikunimega Ain Saar, asutas Vaba Sõltumatu Noortekolonni number 1 ja tegi muid tükke}, Võru surnuaial taastati Vabadussõja mälestussammas ja miilits ajas koertega üritusi laiali. Ühe sellise intsidendi järel oli koolis näha kummalistes ülikondades seltsimehi, kes pingsalt vanemate klasside õpilaste nägusid jälgisid ilmses lootuses tuttavaid kohata. Aga tekkis ka äri. Leidsime sõpradega mingist ajalehest kuulutuse, milles otsiti meie jaoks ulmeliste palkadega (mahus umbes meie vanemate aasta palk paarinädalase projekti eest) meelitades C programmeerijaid. Kandideerimise tähtaeg oli suurusjärgus kaks nädalat, see tundus täiesti mõistlik aeg, millega omale C selgeks teha. Kuskilt sai hangitud klassikaline Brian Kernighan ja Dennis Ritchie \enquote{The C Programming Language}\sidenote{Paraku läks mu koopia hiljem kaotsi. Kust ma selle raamatu sain, ei oska öelda, aga kindlasti ei tulnud see kuskilt välismaalt. Ilmselt oli tegu mingi kvaliteetse piraat-väljaandega, millel isegi kaanekujundus õige oli. Hiljem järele uurides selgub näiteks, et raamatus puudus igasugune märge trükkimise koha ja väljaandja kohta}. Seda sai siis kampas tudeeritud ja tundus sihuke loogiline. Kuna puudus juurdepääs C kompilaatorile, siis päris koodi kirjutada ei saanud. See meid ei heidutanud ja mingid kirjad me isegi välja saatsime. Vastust muidugi ei tulnud. Hiljem olen mõelnud, kas võis tegu olla tollesama legendaarse lehekuulutusega, mis viis kokku Bluemooni\index{Bluemoon} poisid ja Stefan Obergi\index[ppl]{Oberg, Stefan} aga ajastus vist ei klapi. 

Siiski saavad kõik head asjad otsa, nii ka keskkool. Tol hetkel sai mingites piirides omale lõpueksamit valida ning oleks olnud kummaline, kui meie seltskond ei oleks valinud arvutieksamit. Tolleks hetkeks olime Aivarist kaugel ees, sest meil sõna tõsises mõttes ei olnud mitte midagi muud teha, kui arvutit torkida. Laulsin kül ka kooris\sidenote{Kooriga välisreisile (kas Saksamaale või Soome) minek oli ka põhjuseks, miks ma ei ole kunagi vabariiklikul informaatikaolümpiaadil käinud. Tol ühel kevadel, kui sinna õnnestus välja murda, oli ka reis plaanis. Otsustavaks sai, et ma ei tahtnud koori hätta jätta. Mitte, et ma seal mingit kandvat rolli oleksin mänginud, aga siiski.} aga põhimõtteliselt kogu muu vaba aeg oli arvutite päralt. Isegi õppetöö ei seganud, sest põhikoolis tegin endale kõva põhja alla. Aga see kõik ei vähendanud sugugi eksami pidulikkust. Sisenesime ruumi, võtsime pileti, lahendasime, vastasime komisjonile, kõik oli nii nagu peab. Aivar oleks võinud meile kõigile viied välja kirjutada aga ometi viidi eksam täie tõsidusega läbi. 

Kuna õnnestus kool nibin-nabin kullaga lõpetada, sain Tartu Ülikooli Matemaatikateaduskonda\index{Tartu Ülikool!Matemaatikateaduskond} eksamiteta sisse. Sinna minek tundus loogiline, sest Tallinn oli kaugel ja tundmata ning arvuti-värki tahtsin kindlasti õppida. Sõjaväega probleeme ei olnud. Esiteks olid segased ajad ning Eesti riik polnud veel päriselt välja mõelnud, mis moodi oleks mõistlik väeteenistust korraldada. Teiseks oli mu silmanägemine nii paha, et mulle öeldi Kaitseväe tohtrite poolt: \enquote{Kui venelane peale tuleb, siis paneme su laipu vedama, seniks mine koju}. Nii veetsingi suve Võru ja Tartu vahel hääletades, käisin näiteks ka Steni\index[ppl]{Tamkivi, Sten} juures\sidenote{Tema ema ja minu tädi olid juba ülikooli aegsed sõbrannad, Steni vanaisa elas Võrus ja nii me juba üsna õrnas eas tuttavaks saimegi.} Primexis\index{Primex Data} külas. 

Sügisest algas ülikool ja jäin pidevalt Tartusse. Kuna jäin paberite ajamisega töllerdama, siis teiste matemaatikutega Tiigi ühikasse kohta saada ei õnnestunud. Ühe või kaks talve olin sugulase juures üüriliseks, ühe talve elasime kambaga Tartu Kurtide Ühingus (!)\index{Tartu Kurtide Ühing}, kes tudengitele tuba välja üüris. Küll aga sai külas käidud klassivendadel, kes läksid enamuses Tartusse majandust õppima, ja kelle ühikaks olid Narva Maantee Tornid. Toona Tartu ühikates toimunu on omaette lugu, millesse süvenemine viiks meid teemast kõrvale.

Ülikoolis sain kohe piltlikult öeldes ägeda laksu silmade vahele. Esmalt selgus, et, erinevalt keskkoolist, on ülikoolis vaja päriselt õppida. Aga oskus selleks oli juba kadunud ja tuli uuesti tekitada. Teiseks selgus, et puhtast ropust tööst enam heade hinnete saamiseks ei piisanud, vaja oli ka annet. Aga seda on mul kogu aeg nappinud. Kolmandaks selgus, et teistel seda annet jagus ning see tegi egole haiget. Inimesed nagu Meelis Roos\index[ppl]{Roos, Meelis} ja Rene Prillop\index[ppl]{Prillop, Rene} seilasid igasugu matemaatikast läbi ilma nähtava pingutuseta ja kirjutasid koodi, nagu jumalad. Margus Sutt\index[ppl]{Sutt, Margus} teadis arvutitest nähtavasti kõike ja oli tolleks ajaks juba tegelenud täiesti müstilisena tunduvate asjadega. Asko Seeba\index[ppl]{Seeba, Asko}, oli kõike seda \emph{ja} oli seejuures veel setskondlik ning tüdrukute hulgas popp. Ei jäänud midagi üle, tuli tasapisi inimeseks õppima hakata. 

Igatahes oli vaja tööle minna, sest ema käest ei saanud ju jääda raha küsima. Proovisin saada baarmaniks, vast avatud Atlantise ööklubi valgustajaks ja isegi arvutigraafikuks aga asjata. Lõpuks sattusin kuidagi ettevõttesse Korel IN\index{Korel IN} programmeerijaks, mu esimene tööpäev oli detsembri alguses aastal 1993. Mind ja kamraad Veljot\index[ppl]{Hagu, Veljo} võeti palgale eesmärgiga luua firmale arvetega majandamiseks vajalik tarkvara. Keeleks oli Visual Basic\index{Keeled!Visual Basic} ja ei läinud palju aega, kui meil mingid asjad juba töötasid. \enquote{Programmeerija} kõlab märkimisväärselt galmuursemalt, kui asi tegelikult välja nägi. Tegime kõike alates kauba tassimisest (kontor asus viiendal või kuuendal korrusel, kahekümnetolline CRT-monitor on päris raske), kuni isegi mõningase müügitööni. Toonasele arvutiärile iseloomulikult ei teadnud eales, mis seisus su töökoht kontorisse jõudes oli. Mõnikord oli ära müüdud mälu, mõnikord võrgu kaart või monitor. Mäletan end kirjutamas koodi üheksatollise must-valge kassamonitori ees taburetil istudes. 

Tartu ei ole suur linn ja nii puutusime Korelis töötades kokku suure osaga toonasest arvutiseltskonnast. Tarmo Tali\index[ppl]{Tali, Tarmo} oli meil müügimeheks ja aeg-ajalt käis tal külas Asko Oja\index[ppl]{Oja, Asko}, keda hellitavalt \enquote{Tarmo blondiiniks} kutsuti. Vahel astus Sorose sajalisi tuulutades läbi Marek Tiits\index[ppl]{Tiits, Marek}, kellele mingi ime läbi õnnestus isegi üks Suni tööjaam müüa. Kui ütlen, et puutusime, siis tegelikult mina ei puutunud eriti kellegagi kokku, olin toona ja olen siiani küllalt asotsiaalne. Igasugu toredat rahvast käis poest läbi, enamasti sai lihtsalt silmad punnis peas spetsialistide jutte kuulatud ilma nende nimesidki teadmata. 

Kuidagi tekkis Koreli lähedale aktiivne kodanik nimega Tanel Urbanik\index[ppl]{Urbanik, Tanel}. Ta pandi meile alguses ülemuseks aga üsna varsti vedas ta meid Korelist minema asutades uue ettevõtmise nimega HClub. Nimi tuli sellest, et meie tuba Koreli päris-ärimeeste hulgas veidi põlastavalt häkkeriklubiks kutsuti. Tanel tahtis tarkvaraäri teha, küllap seetõttu tal Koreliga teed lahku läksidki. Meie peamiseks leivanumbriks sai kassasüsteemide ehitamine, peamisteks klientideks erinevad tanklad, näiteks Favora omad. Kirjutasin muu hulgas ka näiteks Ravimiametile\index{Ravimiamet} nende ühe esimestest andmebaasidest. Selguse mõttes olgu üle korratud, et toona mingist klient-server arhitektuurist juttu ei olnud. Kõik lahendused hoidsid andmeid võrguketta peal Microsoft Accessi\index{Microsoft Access} andmebaasis ja selle poole pöördumine käis kliendi juurde paigaldatud \enquote{paksu} kliendi abil. 

Tollele ajale tagasi mõeldes tundub hämmastav, et meie tarkvara töötas. Meid olid ainult mõned inimesed, mingist testimisest või versioneerimisest ei teadnud keegi midagi. Mäletan, et korra pidin Tartust Võrru tanklasse tagasi sõitma, sest värsket versiooni flopi peal kohale viies olin midagi valesti teinud. Vähemalt minu kood püsis kindlasti koos peamiselt tati ja teibiga. Veljo oli märkimisväärselt pädevam programmeerija aga tarkvaratehnikast polnud ilmselt palju aimu temalgi. 

See mind lõpuks HClubist (päris suure tüliga, tuleb tunnistada) ära viiski. Ma ei jaksanud enam kõige selle kokku punutud ja päris kliente teenindava tarkvara eest vastutada. Põlesin läbi ja kõndisin Tanelit pipramaale saates ära. Toonaseid seiku nägin veel aastaid unes ja ärkasin keset ööd. Oma rolli mängis ilmselt ka see, et just tol ajal, kui õigesti mäletan, läksid põhja mu unistused saada arvutialane haridus. Nimelt oli toona matemaatikateaduskonnas esimesed paar aastat kõigile ühised, seejärel tuli valida kas arvutiteaduse, statistika või rakendusmatemaatika vahel. Valik käis seejuures õpitulemuste alusel. Minu õpitulemused võimaldasid napilt ennast arvutiteadlaseks pidada ja nii esitasin vajaliku avalduse ning asusin järgmisest semestrist hoogsalt arvutiteaduse aineid kuulama. Neid loeti enamasti Liivi tänava õppehoones. Dekanaat oma teadetetahvliga asus aga Vanemuise õppehoones. Ja kuna ma ka oma ut.ee meiliaadressi ei jälginud, läks minust täiesti mööda dekanaadi mõte, et peaks ikka veel mingeid pabereid küsima. Kui ma ükskord jaole sain, olid arvutiteaduse õppekohad täis ja minust sai statistikaüliõpilane. See oli päris valus hoop. Kuigi arvutiteaduse ained olid minu jaoks rasked (mäletan end kolm korda kompileerimismeetodite eksamit tegemas), oli mul siiski mingi lootus sealtkaudu kuidagi paremaks programmeerijaks saada ning kamraadidele järgi jõuda. Toonane ülikooliharidus oli tänasest väga erinev ja asus praktilisest elust valgusaastate kaugusel, aga lootus jäi. Statistikast huvitusin ma vähe ja ei näinud mingit võimalust sellest oma töises elus kasu saada (masinõppe-revolutsioonini jäi veel paarkümmend aastat). Seetõttu tegin edaspidi minimaalse, et kuidagi koolist läbi saada ja keskendusin tööl käimisele. 

Kogu BBSindus läks minust üsna suure kaarega mööda. Võrus ei olnud kohalikku BBSi ja kaugekõne ei tulnud kõne allagi. Sten Primexis küll vist näitas kuhugi helistamist, aga tuhka ma aru sain. Korelis oli küll väline modem ja aegajalt sai kuhugi sisse helistatud, aga seda väga sporaadiliselt. Peamiseks selleteemalise info allikaks oli kursavend Mati Muts\index[ppl]{Muts, Mati}, peamiselt sai käidud Lucifer BBSis\index{BBS!Lucifer BBS}. Küll aga oli ülikoolil tol ajal juba täiesti korralik internetiühendus ja palju aega kulus Vanemuise õppehoones\index{Tartu Ülikool!Vanemusise tänava õppehoone} terminali taga FTPd pidi ringi kolades. Mäletan, et tõmbasin kas ftp.funet.fi või ftp.sunet.se serverist tükk aega mingi Metallica albumi kaanepilti ja olin väga rahul, kui see ka päriselt kohale jõudis. 

Selgelt mäletan ka seda, kuidas ma kohtusin HTMLiga. See oli Liivi tänaval\index{Tartu Ülikool!Liivi õppehoone}, seal oli mingi Suni klass\sidenote{Need pidid olema Sunid, sest mäletan ruudulist hiirepatja. Mis muidugi ei olnud mingi padi. Kuna Sun kasutas toona levinud palli asemel hiire liikumise lugemiseks eesrindlikke optilisi sensoreid aga tehnoloogia polnud veel kuigi arenenud, pidi sensoritele teadaolevate vahedega ruudustikku näitama. Seetõttu töötas hiir ainult spetsiaalse metallist mati peal, kuhu oli joonistatud peen ruudustik} ning seal sukeldusin ma veebilehe tegemise võrratusse maailma. Pärast pikka pusimist suutsin omale tekitada kodulehe, kus asju õiges kohas hoidis tabel! Ega sinna kodulehele midagi kirjutada ei olnud aga tabeli ridade ja lahtrite saladuste lahti pusimine oli põnev.

Ja kõik see osutus kasulikuks, sest HClubi järel võttis mu oma juurde tehnikuks klassivend Meelis Mäeots\index[ppl]{Mäeots, Meelis}. Ta tegeles tol ajal igasugu imelike asjadega, kuid muu hulgas asutas ka internetifirma. See koosnes alguses peamiselt minust ja temast. Firma tegeles Unineti\index{Uninet} \emph{dial-up} ühenduste edasi müümisega, tegi kodulehekülgi ja pidas isegi Infomeistri nimelist interneti infokataloogi. See viimane oli täiesti hämmastav äri. Meelis käis ja rääkis mingitele firmadele augu pähe. Mina kirjutasin firma andmed kuskil serveris asunud staatilisse (!) HTMLi. Mis kasu sellest kellelegi ammu enne otsingumootorite laia levikut tõusta võis, on mulle siiani arusaamatu. Ma ka ei mäleta, et seal lehel keegi väga käinud oleks. Ometi sealt mingi kopika sai ja ma väga loodan, et tolle tegevuse käigus antud lubadused ikka enam-vähem täidetud said. 

Kuna teadsin Steni juba varasemast ja Meelis vist ka puutus temaga kokku, lõpetasime ühel hetkel modemitega jantimise ja infokataloogi pidamise ning asusime Halo\index{Halo Interactive DDB} nime all kodulehekülgi tegema. Kampa võeti ka mõned kunstnikud (näiteks väga andekas Oliver Reitalu\index[ppl]{Reitalu, Oliver} ja mitte vähem andekas Alar Koort\index[ppl]{Koort, Alar}, keda ilmselt tema rajude elukommete tõttu Helbekeseks kutsuti) ja projektijuhiks Priit Sasi\index[ppl]{Sasi, Priit}, keda kõik tema joviaalse oleku ja suure habeme tõttu Sasuks kutsusid. Sasu õpetas mind briti punki ja Alar kurjemat sorti hiphoppi kuulama ja elu oli päris tore. Miskipärast mäletan, et minu käe alt tuli Eesti esimene kommertsalustel tehtud (st. ettevõte maksis kellelegi lehe tegemise eest raha) kodulehekülg, see sai tehtud Tartu Raadiole\index{Tartu Raadio}, kui mälu ei peta. Kunstnik joonistas pildid valmis ja lõikas tükkideks, mina kirjutasin Notepadiga HTMLi ja nii see töö käis. 

Mingil hetkel hakkasime lehekülgede tekitamist automatiseerima, kirjutasime Perli skripte. Mõnda aega ei olnud meil ei oma serverit ega üldse kuskil Perli jooksutada. Siis sai programmeeritud nii, et skript läks e-mailiga Unineti\index{Uninet} süsadminile, see kopeeris faili õigesse kohta, meie vajutasime brauseris nuppu, saime veateate, admin saatis e-mailiga konsooli veateated, mina parandasin koodi ja saatsin uue versiooni. Admini kannatus lõppes enne, kui minu oma. 

Siiski jõudsime lõpuks päris kaugele oma tegemistega. Perli skriptid läksid järjest pikemaks ja, kuna andmebaasi pidamiseks ei olnud meil serverites piisavalt õigusi, hoiti andmeid enamasti lihtsalt tekstifailis. Üllataval moel kattis see ära päris suure hulga vajadusi. Perlilt liikusime ühel hetkel PHPle ja ühel hetkel tekkis ka levinud kui seetõttu mitte vähem rumal mõte endale ise oma sisuhaldussüsteem kirjutada. See vist sai isegi valmis aga konkreetsed mälestused tollest elukast puuduvad. 

Ma ei mäleta, et see äri kuidagi tänapäevases mõistes äri moodi välja oleks näinud. Raha oli alati vähe ja nii tuli teha kõike, mille eest maksti. Kuidagi müüs Sten Ühispangale maha mõtte anda nende aastaraamat välja CDl. Mis muud, õppisime selgeks Macromedia Director'i kasutamise ja video redigeerimise ja andsime minna. Ainus asi, millega hakkama ei saanud, oli heli. Õnneks oli Sten hea sõber Lauri Liivakuga\index[ppl]{Liivak, Lauri}, kelle Forwards Studio\index{Forwards Studio} asus meiega tol ajal sama koridori peal. Lauri tegi kenad kõllid ja plõnnid ja aitas selle kõik visuaaliga ära sünkroniseerida. Tulemus sai päris kena. 

Igatahes hakkas meile järjest rohkem Tallinna kliente sigima. Samuti müüs Sten suure tüki ettevõttest Brand Sellers DDBle\index{Brand Sellers DDB}. Too oli minusugusele Tartu nohikule täiesti müstiline kamp inimesi. Intelligentsed, säravad, jõukad (nii mulle tundus) ning andekad. Bruno Lill\index[ppl]{Lill, Bruno} oma terava ütlemisega on siiani meeles.  Nii tehti kampas otsus kolida Tallinna. Olin tegelikult ligi aasta üsna kahepaikne pendeldades Tartu ja Tallinna vahel. Ülikoolis olid veel viimased sabad lõpetada ja Mari\index[ppl]{Kütt, Maria}, kellega toona juba koos elasime, käis samuti veel koolis. Lõpuks sai lõputöö kaitstud ja, kuna selliseks triviaalseks asjaks ei hakanud ju keegi Tartusse sõitma, käis Mari diplomit dekanaadist ära toomas. Prouad nõudsid allkirjastatud volitust, mis ukse taga ka kohe valmis tehtud sai ning nii omandasin ma oma esimese teaduskraadi. Tartu Ülikooli peahoone sammaste vahelt ei ole ma kunagi välja astunud ja, kuigi toonaseid õppejõude hindan siiani kõrgelt, pean oma alma materiks siiski Massachusettsi Tehnoloogiainstituuti. 

Tallinnasse kolimisega sai läbi üks etapp Halo kasvu loost. Senise boheemliku mis-võib-ikka-valesti-minna mentaliteedi asemel tuli hakata käibenumbritest rääkima. Samuti oli meeskond kasvanud. Veel Tartu päevadel olin saanud omale oma elu esimese alluva olles samal ajal ka tema esimeseks ülemuseks. Vist veel keskkooli lõpetav noor nutikas tüüp aitas mul koodi kirjutada ja hängis niisama ringi, ei mina teadnud, kuidas inimesi juhitakse või mida üks ülemus tegema peaks. Nimeks oli tüübil Taavet Hinrikus\index[ppl]{Hinrikus, Taavet}. Inimesi lisandus veelgi ja ma ei saanud enam aru, miks ja kuidas asju tehakse. Nii leidsingi ühel ilusal päeval kuskilt kuulutuse, et Hansapank\index{Hansapank} otsib oma internetipanga meeskonda inimesi. Läksin intervjuule. Mäletan siiani seda tunnet, kui Liivalaia tänava pangahoone tolle aja kohta ülišiki lifti uksed kaheksandal korrusel avanesid ja minu ees avanes hurmav vaade vanalinnale. Olin müüdud mees, õnneks arvas Vilve Vene\index[ppl]{Vene, Vilve}, kes toona arendust vedas, samuti. Nii sai minust veidi enne sajandivahetust Hansapankur. Mul vedas kohutavalt, pank oli praeguses mõistes ulmeliselt dünaamiline asutus. Vägesid juhatas Indrek Neivelt\index[ppl]{Neivelt, Indrek}. Vaata Maailma programm oli just käima minemas ja sellega tegeles Tiit Pekk\index[ppl]{Pekk, Tiit}. Marketsi tiim eesotsas Erkki Raasukesega\index[ppl]{Raasuke, Erkki} pidas ülejäänud panka talumatuteks venivillemiteks ja tootis Erik Jõgi\index[ppl]{Jõgi, Erik} juhtimisel imeilusat koodi. Aga see, nagu öeldakse, on juba üks teine jutt.

\chapter{Meelis Roos}
%!TEX TS-program = arara
% arara: myindex

\index[ppl]{Roos, Meelis}
\textbf{\enquote{Kuidas sina arvutite juurde said?}}

Kõige esimene mälestus millestki arvutitega seoses on koolieelsest ajast, kui mõnikord läksime emaga lasteaiast koju mööda Liivi tänavat. Paremat kätt mäe otsas oli üks neljakordne maja. Ema ütles, et see on arvutuskeskus, ja see kõlas aukartustäratavalt ja põnevalt.
Päris arvutite juurde sattusin isa töö juures kaheksakümnendate lõpus. Füüsikud ostsid omale mõned arvutid elektromeetria laborisse, eksperimendi juhtimiseks. Arvutid olid CAMAC\sidenote{\emph{Computer-Aided Measurement And Control (CAMAC)} (elektroonikastandard andmete kogumiseks ja seadmete kontrolliks; kasutusel (osakeste) füüsikas aga ka tööstuses}) kontrolleriga vene DVK-d\index{Arvutid!DVK}\sidenote{\begin{russian}ДВК, Диалоговый вычислительный комплекс\end{russian}. Nõukogude personaalarvuti, ühilduv DECi PDP-11\index{PDP-11} perekonnaga. Varasemad mudelid on tuntud ka kui Elektronika MS-0501\index{Arvutid!Elektronika} ja Elektronika MS-0502}.

\textbf{\enquote{Kus see kõik sündis?}}

See juhtus Tartus\index{Tartu}, Tartu Ülikooli\index{Tartu Ülikool} juures. Isa oli Tartu Ülikooli Füüsika Osakonnas\index{Tartu Ülikool!Füüsika Osakond}\sidenote{Täpsemalt oli tegu Füüsika-Keemiateaduskonna Füüsika osakonnaga} füüsik. Nad tegelesid elektroonika mõõteseadmete välja töötamisega ja said isegi mingisuguse auhinna elektromeetri eest, mis eriti väikesi laenguid registreeris. Näiteks visati pastaka kuul, millel oli mingi laeng, kusagilt läbi ja mõõdeti see laeng mööda minnes ära. Neil oli seal elektromeetria sektoris lahe töögrupp, noored ülikoolist tulnud mehed tegid koos lahedaid asju. Nende katsete juures oli vaja andmeid töödelda ja katseid juhtida, selleks käis arvuti külge spetsiaalne lisaplokk. Ploki sees oli analoog-digitaaalmuundur (võibolla vastupidi ka aga igatahes niipidi neid kasutati). Füüsikud õppisid programmeerima, et suuta oma eksperimendi andmeid reaalajas kätte saada. 

\textbf{\enquote{Aga mis arvuti see selline oli, mis suutis andmeid niimoodi reaalajas kätte saada?}}

DVK-2M. Vene LSI-11\sidenote{DECi PDP-11 perekonna liige, tuntud ka kui PDP-11/03. Masinat tutvustati 1975. aastal ja ta oli oma sarjas esimene, mille CPU oli integreeritud. Mitte küll ühele, vaid neljale Western Digitali poolt toodetud \emph{Large Scale Integraton (LSI)} kiibile). Meelise sõnul: \enquote{PDP-11 oli legendaarne DECi masin iidsel ajal enne meie aega}} analoogid. Peaaegu täpne kloon aga natuke kohapeal ka täiendatud. Programmide poolt ühilduv aga mitte identne. DVK peal jooksis näiteks DECi originaal-opsüsteem RT-11\index{RT-11}. RT-11SJ oli igapäevane opsüsteem, see oli \emph{single job} ja RT-11FB'l oli \emph{foreground} ja \emph{background}, millega sai taustal jooksutada mingisugust teist tegevust. 

\textbf{\enquote{Kui vana sa olid, kui su isa need arvutid omale hankis?}}

Põhikooli teises pooles. Ega mul ei olnud põhjalikku teadmist, mida selle arvutiga teha saab. Kui ma tegin isale tekstisisestustööd, näiteks sugupuu andmete sisestamiseks, siis sain ma pärast seda kuni õhtuni mängida. Lemmikmäng oli Wall\index{Mängud!Wall}, seina pommitamine reketiga. Isa pani mind arvuti taga kohe tööle, et mu huvist miskit miskit kasu oleks, mis ma niisama aega raiskan. Programmeerima õpetati ka, eks nad ise ka õppisid. Isa rühmas programmeeriti BASICus\index{Keeled!BASIC}, FORTRANis\index{Keeled!FORTRAN} ja CASICus\index{Keeled!CASIC}. See viimane oli CAMACi kontrollerite programmeerimiseks mõeldud BASICu ja Pascali\index{Keeled!Pascal} vaheline keel\sidenote{Ilmselt peetakse silmas keelt formaalse nimega \emph{ANSI Standard Real-Time BASIC}, mille spetsifitseerib IEEE standard \enquote{726-1982 - IEEE Standard Real-Time BASIC for CAMAC}}. Selles viimases mina ei sattunud programmeerima, küll aga BASICus. Minu parim programm oli programm, mis ajas inimesega eesti keeles juttu. Programm ütles ühe lause, kasutaja ütles lause ja programm valis juhuslikult vastuse sisseprogrammeeritud lausete hulgast. Ta suutis mõnikord teemas ka püsida. Näiteks kui programm ütles \enquote{Osta elevant ära}, siis järgmised kaks lauset olid, et \enquote{Kõik ütlevad nii, aga osta elevant ära}. Enne ta ei läinud järgmist lauset valima kui ta oli kaks vastust saanud. Seda mängu teiste töötajate lapsed mängisid ja neil oli lõbus. See oli lahe emotsioon, et ma tegin midagi, mis teistele lahe oli. 

\textbf{\enquote{Huvitav, et sa kohe hakkasid mängu tegema ja seejuures kohe midagi AI-sarnast}}

See tundus kõige lahedam asi mida teha! 

\textbf{\enquote{Need füüsikud pidid ju kähku õppima, sest reaalajas riistvarast andmeid lugeda on ju keeruline?}}

Neid oli seal rühmas vähemasti kolm meest, kes programmeerimist õppisid. Neil oli üks natuke noorem pundis, kes oli nende põhiline arvuti-mees ja kes seda vist paremini jagas kui teised. Tema juures oli see CAMAC kontroller, millest enne juttu oli. Arvuteid oli selle labori peale vähemasti kolm tükki, aga üks oli see põhiline eksperimendi juhtimise oma. Mina kasutasin arvutit, mis oli niisama masinakirjutaja toas ja mida kasutati programmide sisestamiseks ja muidu andmetöötluse jaoks. Näiteks isa tegi selle abil sugupuu üles joonistamist, neid puid sai rullpaberile\sidenote{Toonaste printerite puhul oli tavaline, et paberi jooksis printerisse perforeeritud servadega rullist, nii sai paberit kiiremini liigutada} välja trükkida. Kui hiljem koolis tulid mingid tudengid ja andsid igaühele paberi, et joonistage oma sugupuu üles, siis mina palusin isal lihtsalt ühe koopia välja trükkida.

\textbf{\enquote{Aga miks sa lasid ennast sellesse suhteliselt igavasse andmesisestaja rolli suruda? Lihtsalt, et saaks mängida?}}

Algul selleks, et saaks mängida. Aga kui selgus, et ise programmeerida saab ka ja see on täitsa lahe, siis ma pigem mängimise asemel keskendusin rohkem sellele. Ma ei jätnud mängimist päris maha, mängisin ikka ka vahel. 

Mind köitis programmeerimise juures, et programm võis vähendada käsitööd. Näiteks ESC koodidega printerile õigeid asju saates\sidenote{\emph{Epson Standard Code for Printers, ESC/P\index{ESC/P}} on Epsoni poolt maatriksprinterite jaoks välja töötatud (ja termoprinteritel siiani kasutusel olev) keel, mis võimaldab juhtida rastrivõimekuseta printerit. Keel sai oma nime sellest, et tema käsud algasid sümboliga ESC (ASCII 27). Näiteks ESC E lülitas sisse ja ESC F välja rasvase trüki} trükkisin oma õpiku silte, kus oli rasvases ja suuremas või väiksemas kirjas kõik vajalik erinevatel ridadel kirjas. Üks ema tuttav tahtis oma firma logo visiitkaartidele, see logo tuli siis teisendada Epsoni printeri graafika ESC-jadadeks. Ma joonistasin selle \emph{bitmap}ina üles aga siis leidsime, et ei tasu vaeva ja seda logo ma ei teinud. See oli näiteks koht, kus ma leidsin, et programmist võiks oluliselt kasu olla. Ja üheksandas klassis oli seik, kus ma jäin füüsika tunnis programmeerimisega vahele -- kirjutasin oma vihikusse mingit BASIC-programmi ja õpetaja läks mööda ja ütles midagi stiilis, et siin tunnis tegeleme füüsikaga, mitte programmeerimisega. Ja keskkoolis tegin programmi, mis otsis lähendusi kaheteistkümnendale juurele kahest, nii et saaks isaga süntesaatori ehitamisel sagedusjagaja täpse teha -- oli esimene kasulik programm, mida ma mäletan.

\textbf{\enquote{Kuidas õppimine käis?}}

Ma sain mingisuguseid venekeelseid raamatuid. Osalt raamatukogust isa tõi, osalt oli ehk mõni raamat tal töö juures olemas.  Need olid enamasti kusagilt laenatud. Näiteks mul oli segadus ASCII koodi ja \emph{Escape} koodidega, mida sai printerile ja terminalile saata. Siis ma mäletan, et küsisin isalt nõu, et mis neil vahet on et kas see on seesama asi. Ja siis oli erinevaid raamatuid . Näiteks oli üks raamat BASICu kohta, kus mingisuguse käsu kohta on mul siia maani meeles kirjeldus, mis minu meelest ei sobinud niisugusesse raamatusse: \begin{russian}\enquote{эта команда работает хорошо}\end{russian}. See käsk töötab hästi. Minu meelest oli see lati liiga madalale laskmine. Minu meelest peaks kõik hästi töötama, asjad tuleks nii teha. 

\textbf{\enquote{Sul oli ju siis päris korralik vene keele oskus?}}

Jah, ma olin üheksandas klassis umbes kui ma programmeerimist õppisin ja kannatas venekeelset raamatut lugeda küll. Meil oli põhikoolis selline vene keele õpetaja, kellega pidi õppima, mul tõenäoliselt oli üsna normaalne vene keele oskus selle vanuse kohta. Ma käisin Tartu 12. Keskkoolis\index{Koolid!Tartu 12. Keskkool}. Meil oli üks ukrainlanna, Zinaida Tovkatš vene keele õpetajaks. Tema kohta meie kirusime, et ta on väga range ja isegi haige ei ole kunagi. Muudkui peab õppima ja muidu ei pääse. 

\textbf{\enquote{Kas keegi sind õpetas ka või käis ainult raamatu järgi see asi?}}

Isa õpetas mulle neid asju, mida tema teadis. Näiteks õpetas ta mulle plokkskeeme, sest ta ise õppis nende abil. See kestis kuni keskkooli ajani välja, et kui mina tegin programmi ja see ei töötanud, siis oli kaks viisi silumiseks. Üks oli see, et ma trükin ta rullpaberil välja ja loen õhtul kodus. Teine võimalus on see, et ma joonistan selle asja plokkskeemiks ja lähen näitan isale. Sealt pealt tema oskas vigu leida küll. Ja plokkskeemiks joonistamisel leidsin ma tihti vead ise ka üles. Ja isegi kui ma Pascal-keeles kirjutasin, mida isa ei osanud, ma sain temalt ikkagi plokkskeemide tasemel abi. Sest isal oli hea loogiline mõtlemine ja ta seletas mulle minu vead ära küll. 

Minu ülesanne oli kodus keskkütte katla alla tuli teha. Selle süütamiseks oli füüsikaosakonnast toodud vanapaberit, mille hulgas oli teinekord mingeid arvuti väljatrükke, mida ma lugesin. Panin need kõrvale samal ajal kui ajalehed ja muud läksid katla süütamiseks. Näiteks ma leidsin Minsk 32\index{Arvutid!Minsk-32}-e\sidenote{Minsk-32 loodi kuuekümnendatel, nagu nimigi ütleb, Minskis. Tegu oli mitmest mudelist koosneva Minsk suurarvutite sarja kõige võimekama esindajaga. Oli laialdasel kasutusel, kuni asendati seitsmekümnendatel IBM 360 kloonidega} mingisugused 32-bitised krahhi- või muidu mälutõmmised. Ma olin üllatunud, et minul on 16-bitised PCd (see oli tol hetkel hiljem vist kui ma juba PC taga olin) aga nendel oli juba siis 32-bitine arvuti. Ja seal olid FORTRAN-programmid, mida ma huviga lugesin. Isa kõrvalt ütles, et ah, need ei ole suurt midagi väärt, et see mees, kelle programmid need on, ei oska veel eriti programmeerida, tema programmide pealt pole eriti mõtet eeskuju võtta. Aga põnev oli neid lugeda sellegi poolest. FORTRANit õppisin keldris katla kütmise juures!

\textbf{\enquote{Miski pani sind tulehakatust lugema, mis see oli?}}

Seal olid uued põnevad asjad!

\textbf{\enquote{Kas sa peale tulehakatuse midagi muud ka lugesid? Või oli näiteks muusika huvi?}}

Ulme huvi natuke oli. Mul õnnestus saada venekeelsed Asumi\sidenote{Isaac Asimovi poolt kirjutatud sari. Ilmus esmakordselt triloogiana 1951. aastal, tunnustati 1966. aastal Hugo auhinnaga \enquote{\emph{Best All-Time Series}}. Alates 1981. aastast lisandus triloogiale veel köiteid} seeria raamatud, neid oli rohkem kui kaks esimest\sidenote{Eesti keeles ilmusid kaks esimest Asumi raamatut \enquote{Asum} ning \enquote{Asum ja impeerium} vastavalt 1985. ja 1989. aastal Linda Ariva tõlkes}. Asumid mulle meeldisid ja ühe isa sõbra käest laenasime venekeelsed ülejäänud Asumi raamatud. Mul õnnestus vene keeles raamatut lugeda, ma olin selle üle sügavalt üllatunud. Isa luges neid algul ise, hiljem mina. Nii et ulme huvi oli küll, aga see ei olnud väga sügav. Seda oli valdavalt nii palju, kui kodus sattus Mirabilia sarja ulmekaid olema. Need said kõik läbi loetud. See ei olnud esialgu eriti seotud arvutitega, arvutid olid asi, mis tuli reaalsest maailmast. Näiteks sõitsin bussiga koju ja ükskord Pärmivabriku peatusest mööda sõites parajasti ema seletas mulle arvutiviiruste kohta, mida ta oli kuskilt Horisondist või mõnest niisugusest kohast lugenud. Väga põnev oli. Parajasti sõitsime Pärmivabriku peatusest mööda, kui ma esimest korda arvutiviirustest kuulsin. Seda ma mäletan. 

\textbf{\enquote{Kas sa olümpiaadidel ka käisid?}}

Jaa, käisin. Matemaatikaolümpiaadil käisin neljandast klassist saadik. Oli naljakas korrelatsioon: lastest, kellega ma olin koos käinud ülikooli töötajate lasteaias, neist nii mõndagi sai seal olümpiaadidel kohatud. Järgmine laine olümpiaadidega oli keskkooli minnes. 

Miks ma vanast koolist ära läksin? Vanas koolis oli nii, et keskkoolis pidi tulema kaks klassi. Reaalkallakuga ja humanitaarkallakuga. Ja humanitaarkallakuga pidi see \enquote{A} ja eliitklass tulema, kuhu paremad õpilased lähevad ja ülejäänud võinuksid minna sinna reaalkallakuga klassi. Ma leidsin, et see on lati alla laskmine, et ma tahaksin ikka paremat. Mind kutsuti Nõkku\index{Koolid!Nõo Keskkool}. Hilisem ülemus Cyberneticast\index{Cybernetica}, toonane Nõo kooli direktor Uuno Puus\index[ppl]{Puus, Uuno} saatis laiali kõikidele olümpiaadikutele Nõo kooli kutseid. Sain ka. Kaalusin. Oli kaugel. Raske. Siis selgus, et esimene keskkool Tartus\index{Koolid!Tartu 1. Keskkool} on ka täitsa kõva tasemega. Helistasin kooli ja küsisin, et kas teil arvutiklass on. Direktor võttis vastu ja reklaamis, et neil on väga hea arvutiklass. Selle peale ma otsustasin sinna minna. Viisin paberid Esimesse Keskkooli, kui 1990. aastal sisse astusin, oli see juba Hugo Treffneri Gümnaasium\index{Koolid!Hugo Treffneri Gümnaasium|see{Tartu 1. Keskkool}}. Olid tõesti väga head arvutid (Yamaha MSX-II), lisaks arvutiklassile ka Juku\index{Arvutid!Juku}-klass. 

\textbf{\enquote{Sul oli siis selge arusaam, et sa just sinna kooli tahad minna?}}

Jah, ma läksin nimelt sinna. Selle kohta tegi ajaloo õpetaja meil kunagi pisikese kiire küsitluse üheksanda klassi kevadel. Et paljud teist siia jäävad ja paljud lähevad kuhugi mujale. Ja siis ta küsis kolme tema nina all oleva tegelase käest. Esimeses pingis sattusin mina istuma ja minu tagant kahe tüdruku käest, kes olid ka kätt tõstnud, et lähevad mujale, küsiti, mis nad teevad. Need oli täpselt need kolm, kes läksid Esimesse Keskkooli. Nii et kõik, mis ta küsis, sai vastuseks, et lähme ära esimesse keskkooli. Tüdrukud läksid teise paralleeli, bioloogia-keemia harusse. See tundus olevat umbes see vanus, kus mõned hakkasid ise mõtlema oma tulevikule ning seda planeerima ja mõned lasid asjadel isevoolu teed minna. Mina olin nende hulgas, kes leidis, et ma tahan ise oma tulevikku kujundada.

\textbf{\enquote{See oli see aeg, kui ühiskonnas hakkas juba muutus tulema, eks ole}}

Natuke oli juba varem selles mõttes, et kooperatiivid\sidenote{Nõukogude Liidu lõpuaastatel lubatud spetsiifiline ettevõtlusvorm, neid kasutati esimesel võimalusel massiliselt väike-ettevõtluse alustamiseks} olid juba varem olemas ja asjadest tohtis rääkida. Selle sama üheksanda klassi jooksul ma jõudsin kaks korda kirjutada ühele õpetajale referaate, millest võib olla aasta varem oleks vanematel pahandus tulnud. Aga siis juba tohtis. Selle õpetaja kohta oli teada, et ta on üks paras punane. Aga sain nende referaatide eest isegi kiita, mis oli üllatav. Ma mõtlesin, et tuleb kuidagi oma seisukohti kaitsta, sain hoopis kiita. 

\textbf{\enquote{Kas sind keskkooli ajal tööle ei tõmmatud kuhugi?}}

Ainult natukene. Tiražeerisin isa töö juures elektromeetrite trükkplaate. Joonistasin ahjulakiga ja risti ära lõigatud otsaga süstlaga rajad, söövitasin plaadi ära, tinatasin ära ja jootsin sinna peale kõik elemendid vastavalt skeemile. 

\textbf{\enquote{Aga see tahab ju käelist oskust ja elektroonikahuvi, kust sul see?}}

Seitsmeaastaselt oli mulle vist isa töö juures jootekolb esimest korda kätte sattunud, kui ma suvalisi tükke kokku jootsin. Eks ma oskasin kolbi hoida ja elektroonikahuvi mul oli. Aga elektroonikat ma ei osanud, analoogelektroonikat ei ole ma kunagi ära õppinud. Üldisi põhimõtteid tean aga ise midagi teha ei ole osanud.

Digielektroonika oli seal kõrval. Kui keskkool hakkas läbi saama ja oli vaja ülikooli minna, siis mina olin neljandast klassist peale kindel olnud, et ma lähen füüsikat ja nimelt elektroonikat õppima. Aga siis tulid arvutid, kah põnev elektroonika värk, neid sai matemaatikateaduskonnas ka õppida. Mul oli kuhugi ilma eksamiteta sisse saamised, äkki matemaatikasse ja füüsikasse olümpiaadi tulemuste pärast või midagi. Otsustasin matemaatika kasuks, sest füüsikaosakonnas ma olin kogu aeg kohal ja mulle ei meeldinud see. Tundus, et kui midagi ära tahta teha, siis peab ainult endale lootma. Oli nihukesi saarekesi, kes tegelesid oma kitsa erialaga, aga laiemat kandepinda ma ei märganud. Oli töögruppe, kes olid vingel tasemel ja tegelesid oma asjaga. Võib olla, et ma ei sattunud õigete inimestega kokku, aga tundus, et pigem on füüsika nihukene seisev konnatiik. Igaüks on seal kinni, kus on, ja nii on. 

Ega seal oli huvitavaid ja põnevaid asju ka. Näiteks olid füüsikapäevad, kus mu isa käis kuulamas Undo Uus\index[ppl]{Uus, Undo}i, kes rääkis materialismi ümber lükkamisest filosoofiliselt. Isa tuli koju, jutustas. Mina panin kõrva taha. Selliseid asju oli sealt ikka päris mitmeid. Füüsikalist maailmapilti tuli vanemate kõrvalt üksjagu, see oli mul olemas. 

\textbf{\enquote{Kuidas sa siis ikkagi matemaatikat sattusid õppima? Lihtsalt seepärast, et sai eksamiteta sisse?}}

Füüsikasse ma oleks vist ka saanud ilma eksamiteta, need ei oleks probleem ka olnud, ma arvan. Olin lihtsalt laisk, laisad me olime kõik. Keskkoolis klassijuhatajal tuli kaheteistkümnendas klassis üritada meile ikka auku pähe rääkida, et poisid olge tublid ja võtke tehke need eksamid ikka ära, siis saab medalile pretendeerida, muidu ei saa. Aga medaleid oleks ju vaja. Siis me tegime vist kolm medalit klassi peale või midagi. Mina sain hõbeda. Ma täpselt ei mäletanudki, kunagi hiljem kooli koduleheküljelt lugesin. Seda ma mäletasin, et medal oli, aga mis medal, seda ei mäletanud. Polnud oluline, see tuli iseenesest. 

\textbf{\enquote{Ühesõnaga, matemaatikasse sa läksid seepärast, et füüsika tundus natuke seisev vesi olevat?}}

Jah. Ja ma olin kuu aega enne paberite sisse andmist kindel, et matemaatikasse ma küll ei lähe. Me käisime koolist tiimiga Moskva\index{Moskva} lahtisel olümpiaadil matemaatikas. Seal olid mingid doktorandid, kes meiega tegelesid. Ühtlasi toimus seal ka \begin{russian}Международная конференция старшикласников "Наука, природа, человек"\end{russian}\sidenote{\enquote{Rahvusvaheline vanemate klasside konverents \enquote{Teadus, loodus, inimene}}} kus keskkooliõpilased said ise tehtud asju esitada. Keegi oli teinud kiiret vektorgraafikat, et voldime siin kuubikut kiiremini kui AutoCAD, või mis iganes. Ägedaid asju oli tehtud. Seal oli mingit Hollandi rahvast ka, oli rahvusvaheline küll. Seal need doktorandid, kes meiega tegelesid, olid nihukesed parajad uhuud. Näiteks tuleb tegelane hommikul tahvli ette, triiksärk on lükatud alukate sisse, alukad ulatuvad kümme sentimeetrit pikkade pükste pealt välja ja tuleb niimoodi tahvli ette. Ma leidsin, et vot matemaatikuks mina küll ei lähe. Aga siis ma mõtlesin ikkagi ümber. Matemaatikuks ma ei tahtnudki, ma läksin neid arvuteid õppima matemaatikateaduskonna\index{Tartu Ülikool!matemaatikateaduskond} poolt. Mitte elektroonika poolt aga programmeerimise poolt. 

\textbf{\enquote{Kuidas sulle ülikooli üleminek tundus? Sa ütlesid, et olla laisk olnud. Minu mälu järgi pidi ülikoolis kohe hakkama tööd tegema?}}

Jaa. Keskkoolis ma sain endale lubada laisk olemist isegi seal eliitkoolis, no vähemalt mingil tasemel. Ja ma sain keskkoolis arvutimängude mängimise isu täis mängida. Ostsin omale üheksanda klassi lõpus ZX Spectrum-i\index{Arvutid!ZX Spectrum}\sidenote{ZX Spectrum oli Sinclair Research'i poolt 1982. aastal Ühendkuningriigi turule lastud 8-bitine personaalarvuti, mõeldud peamiselt koduseks kasutamiseks. Selle kloone liikus Nõukogude Liidus hulganisti, skeemid olid koguni hobiajakirjades avaldatud} Leningradi turu klooni 1500 rubla eest, kui rubla juba kukkus. Siis oli suur rahanumber, aga ma sain oma isu täis mängida. Joystick\sidenote{Eesti keeles \enquote{juhtkang}. Eelmise sajandi algul Ameerika Ühendriikides patenteeritud, Teises Maailmasõjas Saksa vägede poolt laialt kasutatud ja kuuekümnendate lõpus arvutimängude külge jõudnud kaheteljeline juhtimisvahend. 21. sajandil kaotas ta mängude juhtimisel kiiresti populaarsust hiirtele ja on praegu peamiselt kasutusel lennunduses} sai peeneks mängitud, plastmassi paikasin alumiiniumiga. Tuttav treial tegi talle uue varre, pärast kippusid kontaktid läbi põhja tulema. Aga Spectrum oli nii hea arvuti, sellest sai aru igat pidi! Sai programmeerida BASICus ja Z80 Assembleris\index{Keeled!Assembler}. Sellest arvutist võis lõpuni aru saada. Elektroonikast peaaegu ka, välja arvatud videopildi genereerimise osa. Originaalis kasutati ULA kivi, vene variandis realiseeriti see laus-elektroonikana\sidenote{Originaalne ZX Spectrum sisaldas kahte suurt 40-jalaga mikroskeemi - Z80 protsessor ja üks eelprogrammeeritud loogikamassiiv (ULA - Uncommitted Logic Array). N-liidus tehtud Sinclairi koopiad kasutasid viimase asemel tervet trükkplaaditäit lihtloogikaelemente.}, sest seda kivi ei olnud kloonina võtta. Nii et ma sain sõbra Sinclairi diagnoosimisega hakkama. Näiteks, et sul on ROMi see ja see jalg lahti ja ei anna kontakti, seetõttu on tähtedel vertikaalsed kriipsud läbi, nagu dollarimärgid. Tähtede tabel oli ROMis ja kui seal bitt oli maas, siis joonistati selle biti koha peale alati täpp ja tekkis püstkriips. Järelikult pidi sellel ROMi kivil selle biti jalg mitte kontaktis olema.

\textbf{\enquote{See tähendab, seda, et sa pidid neid asju põhjalikumalt uurima?}}

Skeeme ma ikka kuskilt raamatutest ja mujalt nägin. Keskkooli lõpus, kui Venemaal käisin, ostsin metroost raamatu \begin{russian}Введение в схемотехники IBM PC / AT\end{russian}\sidenote{Eesti keeles \enquote{Sissejuhatus IBM PC/AT skeemitehnikasse}. Ilmselt peab Meelis silmas kodanike \begin{russian} Г. Н. Левкин\end{russian} ja \begin{russian}В. Е. Левкин\end{russian} 1991. aastal ilmutatud raamatut}. Venelased olid 286 skeemid välja ajanud arvuti järgi ja üles joonistanud. Neil oli seal viga, minu mälu järgi. Mingi reset signaali puhul oli aktiivne null ja aktiivne üks kusagil segamini, niisugust asja trükitud raamatus avastada oli igatahes lõbus. See Venemaal käik oli seesama kord, kui me olümpiaadil ja konverentsil käisime. Konverentsi osast ei teadnud me enne midagi, kui me sinna kohale sattusime. Meil ei olnud mingeid ettekandeid, kuulasime niisama, mis räägitakse. Ja vaatasime, mihukesed on kenamad tüdrukud. Üks vene Maša oli kõige kenam. 

Olümpiaadil me eriti hiilgavaid tulemusi keegi ei saanud. Mina sain meie pundist kõige parema tulemuse, sest ma ei joonud eelmisel õhtul alkoholi. Seda oli seal saada ja siis järgmisel hommikul pohmakaga inimesed ei esinenud oma võimete tasemel. Nii tuligi välja, et mina olin meie omadest parim, kuigi vähemasti üks kaasas olnud meestest oli parema peaga. Minu jaoks oli õppetund, mida rõõmsalt teistele edasi jagada: et näe, olümpiaadi tulemus sõltus selgelt sellest, kes ja mida eelmisel õhtul jõi. 

\textbf{\enquote{Räägi palun ülikoolist, me sattusime seal 1993. aastal kokku. Kuidas sulle see matemaatika tundus, mida me kohe esimese semestri alguses saama hakkasime?}}

See oli üks suur kukkumine. Ma näiteks mõtlesin ülikooli tulles, et ma tean, mis on reaalarv. Siis tuli matemaatilise analüüsi esimene loeng, kus hakati neid defineerima. Kõike hakati algusest peale defineerima, kõik muu ehitati ainult nende definitsioonide otsa. See kõik tahtis palju harjumist ja palju tööd aga mina ei olnud harjunud tööd tegema. 

Ma mõtlesin, et ma oskan programmeerida, kui ma ülikooli tulin. Aga Rein Pranki\index[ppl]{Prank, Rein} matemaatilise loogika õppeprogrammid näitasid, et on veel palju asju, millest ma aru ei saa. Seal joonistati näiteks ekraanile tõestuspuu ja ma mõtlesin, et \enquote{Vau, puud ma niimoodi joonistada ei oska}. Me õppisime seda küll hiljem umbes kolmandal kursusel Varmo Vene\index[ppl]{Vene, Varmo} Funktsionaalses Programmeerimises, kus me mingi \emph{minimax}i\sidenote{\emph{Minimax} on algselt nullsummamängude analüüsiks formuleeritud otsustusalgoritm, kuid mida on hiljem oluliselt laiendatud ning mis leiab laiemalt kasutust tehisintellekti puhul, statistikas, filosoofias ja mujal. Algoritm minimeerib võimalikku kahju halvimal, maksimaalse kahjuga, juhul andes optimaalse mängustrateegia eeldades, et ka oponent mängib optimaalselt} ülesande tüübi näiteülesandeks puu paigutust tegime. Esimese kursuse järel oleks seda ehk rekursiooniga ka kuidagi teha saanud, aga see oli jah näide sellest, et kõik ei ole ikka triviaalne. Ei saa igale asjale jõuga peale minna. 

\textbf{\enquote{matemaatiline analüüs, eriti matemaatiline analüüs II, võttis meil kursuse peal palju rahvast hõredamaks, see tahtis harjumist saada}}

Algebra tahtis ka. Kogu see matemaatiline lähenemine, et me ehitame asju üles mingite definitsioonide ja aksioomide otsa. Kogu see asi tahtis kõvasti tööd. Lisaks kukkusin ma esimesel kursusel haiglasse. Eksamisessiooni ajal ei jõudnud ma mõnesid eksameid tehtudki, tegin neid alles järgmise semestri sees. Käisin dekaanilt küsimas sessi pikendust, sest vanemad õpetasid, et nii tuleb teha. Siis dekaan ütles, et meie ajal enam niisugust asja pole, lihtsalt tehke need eksamid ära, kuidas saate. 

\textbf{\enquote{Mis hetkel oli võimalik minna arvutiteadust õppima?}} 

Mingid põhimoodulid oli vaja ära teha ja siis vist esimese aasta järel sai spetsialiseeruda. Kuna ma need moodulid sain kokku, siis kaldusin üldisest õppekavast kõrvale sellega, et läksin võtsin koos aasta vanematega põnevaid arvutiteaduse aineid. Käisin aasta vanema rahvaga koos lahedaid asju kuulamas. Ja siis järgmine aasta tuli võtta need ained ka, mis õppekavast tegemata olid. Minu oma kursus oli need ära teinud, mina tegin neid siis koos aasta noorematega. Mingeid tõenäosusteooriaid ja mingisuguseid matemaatikaaineid.

Juhtus ka seda, et ma kodutöö programme teiste pealt maha kirjutasin. Meil oli Algebra ja Analüüsi Numbrilised Meetodid, kus me arvutusmeetoditega numbriliselt tegelesime. Ma sain algoritmidest aru, nad ei pakkunud mulle algoritmi tasemel pinget ja ma ei viitsinud neid teha. Piisas, kui ma olin aru saanud, mis seal tehakse. Leidus üks lahke kaastudeng Jane, kelle programme ma esitamiseks kasutasin. Muutsin vist natuke treppimist ja muutujate nimesid. Mäletan, ma kirjutasin ühele kommentaaridesse üles \enquote{Viimati modifitseerinud Meelis Roos}\sidenote{Enne, kui vabavaralised tsentraliseeritud ja hajutatud koodirepositooriumid laialt levima hakkasid, hoiti koodi enamasti lihtsalt kettal. Seetõttu oli levinud praktikaks faili päisesse lisada kommentaar faili autori, viimase muutmise kuupäeva ja muu tarvilikuga}. Eks see praktikumi juhendaja teadis, et neid programme üksteise pealt üksjagu maha võetakse. Seepärast lasi ta endale ette seletada, mida see programm täpselt teeb, sellega polnud probleemi ja nii sain kõik asjad ilusti tehtud. Kirjutasin programme tüdrukute pealt maha, sest ma ei viitsinud programmeerida. 

\textbf{\enquote{Kas see ülikooli arvutuskeskus seal Liivi tänaval ei neelanud sind kuidagi endasse, nagu ta nii mõnedki neelas?}}\index{Tartu Ülikool!matemaatikateaduskond!Liivi õppehoone} 

Neelas ka mind aga natuke teistel viisidel. Mina ei kadunud ära Muda\index{Mängud!Muda}\sidenote{Originaalis \enquote{Multi User Dungeon (MUD)}. Paljude osapooltega reaalajaline tekstipõhine seiklusmäng. Täpsemalt siiski mängude alaliik, sest leidus mitmeid eri rõhuasetusega eri koodibaase kasutavaid versioone, mida jooksutati mitmetes eri serverites. Kuna Muda pakkus toona ainulaadset koos mängimise ja suhtlemise viisi, tekkis paljudel kiiresti sõltuvus ja liigne Mudas veedetud aeg oli sagedane ülikoolist välja langemise põhjus.} mängima. Muda oli küll tore: kui ma oma telneti klienti kirjutasin, sai seda Muda serveri vastu testida näiteks. Selleks oli Muda tore. 

\textbf{\enquote{Miks sa kirjutasid oma telneti kliendi?}} 

Võrguprogrammeerimise harjutamiseks. Tahtsin osata igasuguseid sokliühendusi teha. Ma kirjutasin oma netcati laadset asja, mis ei teinud mingisugust telneti \emph{handshake}'i  ja ei osanud \verb|echo off|i ja selliseid keerulisemaid asju, vaid lihtsalt sokli kuhugi ühendas. Sellise asja kirjutasin endale igasuguste asjade torkimiseks. Seal olid mingid mured stiilis et kui pikkade pakettidega asju saata ja vastu võtta võis. TCP võis andmed ju suvalise koha pealt ära hakkida. Ei saanud eeldada, et kui teiselt poolt rida sisse kirjutatakse, et sa selle täpselt rea suuruste tükkidena kätte saad. See oli põnev.

Aga mind neelas see arvutuskeskus natuke teistmoodi. Teisel korrusel Ülo Kaasiku\index[ppl]{Kaasik, Ülo} kabineti kõrval oli magistrantide arvutiklass, kus olid värvilised Sun'id. See oli ette nähtud magistrantidele, aga kellelgi ei olnud eriti probleeme, kui mina ka sinna imbusin. Aegajalt seal ei olnud kohti ja tuli ette, et ma kellelegi oma koha pidin loovutama, aga enamasti ei pidanud. Aasta vanema Raul Tölbiga\index[ppl]{Tölp, Raul} istusime seal koos ja seal sai õpitud ära Unix. 

Kuidas ma üldse sinna Unixit kasutama sattusin, oli omakorda lõbus. Seda ma võin lausa rääkida, kust on pärit minu kasutajanimi \enquote{mroos}. Minu esimene online konto oli masinas vask.ut.ee\index{Masinad!vask.ut.ee}. See oli VAX\index{Arvutid!VAX}\sidenote{Arvutisari, mille töötas DEC välja seitsmekümnendate keskel. Siiani üks kõige tuntumaid omalaadseid arhitektuure, oli ta PDP-11\index{PDP-11} edasiarendus, peamiselt mälu virtuaalse adresseerimise suunas. \emph{VAX - Virtual Address Extension}} tüüpi arvuti VMS\sidenote{VAX arvutite \enquote{kohalik} operatsioonisüsteem} opsüsteemiga. Selline umbes kuupmeetrine kast pluss kettad seal kõrval. Teine VAX oli rubiin.physic.ut.ee\index{Masinad!rubiin.physic.ut.ee} füüsikamajas. See oli MicroVAX, ainult sahtlitumba suurune masin. Vot need olid VMSid. Esimesel kursusel, selle asemel, et sessi ajal õppida, olin mina raamatukogust võtnud omale VAX/VMSi raamatu ja õppisin VMSi. Seal oli huvitavaid asju! Näiteks olid struktuursed failid. Sa võisid tekitada tühja faili, millel on ette antud kirjestruktuur. Opsüsteemi tasemel oli \emph{Record Management System}, millega mingis keeles kirjeldati struktuur ära ja tekitati selle kirjelduse järgi fail. Fail võis olla ka tühi, aga tal oli struktuur olemas. 

Kogu õiguste süsteem selles operatsioonisüsteemis oli keeruline. Windows NT\index{OS!Windows NT} on selle sisemiselt pärinud või umbes niimoodi. Nii keerukas ei ole minu meelest kui VMSis aga kui ma nägin Windows \emph{syscalli} \verb|CreateProcess| koos portsu argumentidega, siis tuli tuttav ette, sest VMSi SYS\$CREATEPROCESS oli umbes samasuguste argumentidega. SYS\$ käis syscallide funktsioonide nimede ette lihtsalt. 

Sealt ma käisin näiteks Lynxiga veebis surfamas. Tõmbasin FTP-ga mingeid faile, mida kuskilt kolmandat teed mööda kuidagi flopi peale sain. Käisin Internetis ka igasugu asju lugemas. Ma eriti ei programmeerinud VMSis. Kui vaja oli kursaõele Pascalis programmeerimist õpetada, aga ainult VAXu klass vaba oli, siis ma näitasin talle Pascalis programmeerimist VAXu peal. Ta oli väga üllatunud, et seda arvutit saab ka programmeerida. Aga sai. 

Seal oli lahe programm nimega SWIM, mis lasi ühe terminali peale multipleksida mitu akent, sai lausa akende suurusi muuta. Sellega ma kasutasin kolme rakendust korraga. Aga SWIM kippus ajama terminali hanguma, kõditas vist mingit VMSi terminali draiveri bugi või mida iganes. Siis tuli leida administraator, keda tihti majas ei olnud, või siis keegi sõber tudeng logis üle võrgu rubiini\index{Masinad!rubiin.physic.ut.ee} ja talk-is Ville Hallikuga\index[ppl]{Hallik, Ville}, kes oli sealne VMSi admin. Villel oli juurdepääs vaske olema ja ta sai tulla ja terminali päästa - hangunud terminali tagant ei saanud keegi enam midagi kasutada. Tappis SWIMi ja mingid asjad ära seal, nii et terminal sai jälle vabaks. Nii et SWIM oli tülikas. Keegi rääkis, et arvutiteaduse instituudi Sun'ides on Unixis programm nimega screen, millega sedasama teha saab. Ja siis tekkis mõte seda kasutada. Ma olin Unixit seni juba korra kasutanud. Math.ut.ee-s\index{Masinad!math.ut.ee}, kui tekkis online võrk, tuli 386BSD\index{OS!386BSD}. Ja see uuendati 93. aasta lõpus mingile uuele tundmatule opsüsteemile. Sinna osteti 486 arvuti asemele, suure kahe-gigase\sidenote{Meelis peab silmas kahte gigabaiti. Konteksti mõttes on oluline märkida, et tol ajal piisas keskmist sorti arvutifirma failiserveri kõvakettaks ühest gigabaidist üsna pikaks ajaks. Aastal 2020 täidab keskmine koduinternetiühendus selle mahu umbes minutiga} SCSI vindiga. Selle SCSI kaardi jaoks 386BSD enam ei sobinud ja pandi asemele mingi uus tundmatu asi nimega Linux\index{OS!Linux}. Versioon 0.99pl3 või midagi, kui õigesti mäletan. 

\textbf{\enquote{Kust selline asi sattus Tartu linna?}} 

No aga kust 386BSD sai? Internet oli ju olemas. Kasutajad koliti 386BSDst Linuxisse siuhti üle ja mul oli mingis Linuxis kasutaja. Jaanuaris umbes uuendati see Linux ära kerneli versioonile 1.0.2. Ma olin natukene nuusutanud Linuxit. Kui ma tahtsin seal Liivi tänaval Unixi screeni, siis math.ut.ee ühendus oli päris aeglane\sidenote{math.ut.ee asus füüsiliselt matemaatikateaduskonna hoones Vanemuise tänaval. Seega peetakse järgnevas silmas internetiühendust kahe, linnulennult 550 meetrise vahega, hoone vahel Tartu linnas}. 9600ne ühendus jagatud paljude kasutajate ja meilide ja muude vahel. Siis ma küsisin omale cs3-e (hilisem romulus.cs.ut.ee\index{Masinad!romulus.cs.ut.ee}) konto ja põhjendasin seda, et tahaksin näppida mõnda mitte-Linux Unixit. Seal oli Solaris\index{OS!Solaris}. Ja see tundus Toomas Soomele\index[ppl]{Soome, Toomas} piisavalt hea põhjendus. Toomas Soome kasutajanimi oli \enquote{tsoome}, ma mõtlesin, et ahaa, et eks Unixis käib see niimoodi. Küsisin siis omale tema süsteemi sama skeemi järgi kasutajanimeks \enquote{mroos}. Antigi. Seda ma olen sellest ajast edaspidi kasutanud igal pool. Isegi kui mul on kodus testarvuti, seal olen ma ka seal harjumusest mroos. Et tsoome mulle kasutajanime teeks, tuli öelda, et ma tahan Solarist kasutada ja kasutajanimi peaks ka samas formaadis olema, et võimalikult vähe küsimusi oleks. 

Mul möödunud aastal \sidenote{Intervjuu Meelisega toimus 2020. aasta kevadel} oli väga sürr kogemus, kui kevadel võttis minuga ühendust Toomas Soome, kellel oli siiamaani magistrikraad tegemata. Ta tahtis, et ma juhendaksin tema magistritööd. Ma mõtlesin, et muna õpetab kana, et mida mina siin teen. Aga tal oli korralik tehniline töö olemas ja mina teadsin, mismoodi üks magistritöö peab enam-vähem välja nägema. Sellest teadmisest oli kasu, see töö sai tal vormistatud magistritööks ja ta kaitses selle edukalt. Aga algul lihtsalt oli väga sürr reaktsioon. Arvutiteaduste Instituudis\index{Tartu Ülikool!matemaatikateaduskond!Arvutiteaduste Instituut} oli terve hulk rahvast, kes tegid oma magistrikraadi hiljem.

\textbf{\enquote{Kas sind teadust ei tõmmanud tegema?}} 

Ei, vot teadust tegema ei ole mind kunagi eriti tõmmanud ja keegi ei suutnud mulle ka auku pähe rääkida sel teemal. Väga ei proovitud ka. Meelitati erinevate viisidega, mingeid materjale ette söötes. Materjalid olid nii teadusega kui mitte-teadusega seotud. Näiteks Jaanus Pöial\index[ppl]{Pöial, Jaanus} jagas mulle omal algatusel kunagi \emph{Java Language Specification}i, et näe üks uus moodne asi. Selliseid asju ülikoolist ikka sattus. 

Ma mäletan, ma olin rebane ning ei olnud veel spetsialiseerunud Arvutiteaduse Instituuti informaatika erialale. Aga mul oli vaja kusagil välja trükkida viietollise flopi pealt mingit tekstifaili, raamatukogust mingisuguse kataloogi otsingu tulemus mingite raamatute otsimiseks. Äkitselt tekkis vajadus laupäevasel päeval trükkida. Ma lihtsalt vajusin kohale Liivi tänavale ja käisin mööda uksi koputamas. Oli vist laupäev ka või muidu õhtune aeg ja seal ei olnud palju rahvast. Sattusin Mati Tombaku\index[ppl]{Tombak, Mati} ukse taha, kes lahkelt lasi trükkida. Ja sellest tekkis nihukene tänutunne kogu selle instituudi vastu, et siin on lahked inimesed. See oli minu esimene isiklikul tasemel kontakt instituudi inimestega.

\textbf{\enquote{Millal sa tööle läksid?}} 	

Minu esimene ametlik töökoht oli Tartu Ülikooli Täppisteaduste Koolis\index{Tartu Ülikool!Täppisteaduste Kool} metoodik. See oli tegelikult postmasteri töö. Aga postmasteri nimelist ametinimetust ei olnud, oli metoodik. Korraldati programmeerimise kursust e-mailitsi koolides. Mina olin see, kes pidas arvet selle üle, kellel olid mis ülesanded lahendatud, ja saatis neile järgmisi. Arvutiõpetajad, kellele vastused saadeti ja kes neid parandasid, saatsid minule seisu ja mina siis selle järgi saatsin järgmisi ülesandeid. Mina olen laisk inimene. Esimesel tööpäeval võtsin nägin pool päeva vaeva ja kirjutasin skripti. Panin kuhugi tekstifaili valmis nimed. Programm võttis sealt järjest nimesid ja saatis neile ülesande ja pidas arvestust, et kellele on juba saadetud, et kellelegi topelt ei saaks. Ja kui ma selle skripti käima panin, siis rubiin.physic.ut.ee\index{Masinad!rubiin.physic.ut.ee}, tollane füüsikamaja Unixi server, kõristas umbes pool tundi. Pärastpoole ma õppisin \verb|nice| käsu\sidenote{Võimaldab Unixi keskkonnas kontrollida, kas kogu programm kasutab ära kogu saadaoleva arvutusressursi või jätab midagi ka teistele arvutikasutajatele} ka ära. Aga see tähendas, et kogu minu edasine töö pärast selle skripti kirjutamist oli copy-paste meili seest sinna sisendfaili ja skript tööle lükata. Automatiseerisin oma töö lihtsalt ära. 

\textbf{\enquote{Aga kuidas sa sinna sattusid?}}

Ma arvan, et Indrek Jentson\index[ppl]{Jentson, Indrek} Täppisteaduste koolist kutsus mind. Indrek oli matemaatikateaduskonnas vanem tegelane ja olümpiaadidega tegelenud. Ma läksin Täppisteaduste Kooli ukse taha, tuli Viire Sepp\index[ppl]{Sepp, Viire} vastu, kes juhataja oli, ütlesin, et tere, tulin töölepingut tegema. \enquote{Mis töölepingut?}, küsis tema. Ma siis seletasin, et Indrek Jentson saatis mind siia postmasteri töölepingut tegema. Kuskil 95. või 96. aasta algul, täpselt ei mäleta. 

\textbf{\enquote{See oli üsna vara ju? Tuleb häbiga tunnistada, ma läksin 93. aastal tööle juba}}

Te olite Veljo Haguga\index[ppl]{Hagu, Veljo} Korelis\index{Korel IN}, eks? Ma käisin Veljo töö juures vahel. Seal olid mingid mängud. Dune'i\index{Mängud!Dune} mängis Veljo näiteks õhtul näiteks millalgi kui ma sinna sattusin, vaatasin, kuidas see käib. Mängimisega ei olnud mul erilist suhet. Ma sain keskkooli ajal oma mängimise isu täis mängida Sinclairi peal ja lülituda juba programmeerimisele sellega, et ma tean, et see on palju põnevam asi. Ma kirjutasin näiteks oma \emph{binary editor}i, millega mängudest järgmiste levelite paroole välja nuuskida ja muid nihukesi asju. See oli juba keskkoolis, et sai igasugustel arvutiturva teemadel nuusitud ja huvi tuntud. 

Arvutiturva teema on mul keskkoolist saadik sees tõesti. Meil olid keskkoolis väga põnevad võidujooksud arvutiõpetajaga. Väga harivad. Näiteks oli õpetaja arvuti klaviatuur parooli all. Aegajalt tehti sellega meilivahetust, nii et masinal klaviatuur oli lukus aga muidu masin töötas edasi. IBM PS/2\index{Arvutid!IBM PS/2}\sidenote{PS/2 oli IBMi kolmas personaalarvutite põlvkond, mida tutvustati 1987. aastal. Paljud tolle masina innovatsioonid nagu näiteks VGA video muutusid \emph{de facto} standardiks pikkadeks aastateks}tedel oli mingi selline klaviatuuriluku võimalus. Küll ma üritasin leida meetodeid sellest mööda hiilimaks. Kui ma sain mingeid skeeme kuskilt näha, siis mul tekkis idee, kuidas i8042 klaviatuurikontrolleri kaudu teha masinale sobivat \emph{warm booti}, et sealt mööda hiilida, aga klaviatuurikontroller oli lukus edasi. Kirusin, et IBMi omad on kavalad olnud. See oli algul. 

Lõpuks selle arvuti parool saadi teada lihtsal viisil. Vaadati üle selle arvutiõpetaja õla, kes aeglasemalt tippis. Kui see oli teada saadud, ega me sellega midagi ei teinud, see ei olnud eesmärk. Aga minul oli edasi põnevam see, kui keskkoolis viimasel aastal oli 386d kohale jõudnud ja nende C: ketas, kõvaketas, pandi kirjutuskaitse alla nii, et mingi spetsiaalne draiver laaditi \verb|config.sys|-ist, mis tegi virtuaalse D: ketta ja keeras kogu C: \emph{read-only}'ks. Ja ma avastasin selle niimoodi, et mul oli mingi enda softi katsetamiseks see asi autoexec.bati või \verb|config.sys|i panna või sealt midagi välja kommenteerida, et minu asi ära mahuks või täpselt ei mäleta mis. Igatahes oli mul vaja sinna sekkuda. Kui ma sekkutud sain, siis ma pärast alati taastasin endise olukorra. 

\textbf{\enquote{Ka tol ajal mingit võrgu häkkimist ei toimunud?}}

Anto Veldre\index[ppl]{Veldre, Anto} rääkis jah\sidenote{Meelis peab ilmselt silmas varem eetrisse läinud memcpy episoodi Anto Veldrega}, kuidas tema poisid ülikooli adminidel ruutusid käest ära võtsid\sidenote{Unixi-laadsetel süsteemidel on root (mis eesti kõnekeeles mungandub tihtipeale sõnaks \enquote{ruut}) ees süsteemi täielike õigustega peakasutaja. Seega tähendab termin \enquote{ruutu võtma} arvutisüsteemi üle täieliku kontrolli saavutamist, tihti algset peakasutajat virtuaalse ukse taha jättes}. Tema jagas oma poistele modemeid ja terminale, mis tulid kuskilt humanitaarabina. Meil oli üks modem õpetaja arvuti küljes. Ühel poisil oli oma modem korra koolis kaasas, mida ta näitas, aga me ei osanud nendega midagi teha ja kohalikku võrku meil ei olnud. LAN\sidenote{\emph{LAN - Local Area Network}, kohtvõrk} tekkis meile alles 12. klassi kevadel, kui ma enam väga ei tegelenud sellega. OK, ma häkkisin LANtasticu\sidenote{LANtastic oli \emph{peer-to-peer} LANi operatsioonisüsteem, mida arendas Artisoft ja mis jäi hiljem Novelli ja Microsofti toodete varju} lahti \emph{social engineering}u meetodil. Sügisel pärast minu ära minekut oli kellelgi vaja saada LANtasticule juurdepääsu. Servermasinas oli nihuke koht nagu \emph{network control directory}. Seal olid andmebaasid binaarsena. Ja vot minu programm oskas käia ja binaarselt andmebaasi modifitseerida ja tekitada ühe administraatori juurde või panna kellelegi õigusi juurde või midagi. Ehk siis tuli meelitada noorem arvutiõpetaja flopi pealt ühte programmi käivitama seal masinas, viisakalt tänada ja puha. Tema poolt oli ka kõik OK. 

Aga varem oli see C:-ketta kirjutuskaitse. Algul me käisime Nortoni \emph{Disk Editor}iga kuskil seal \verb|config.sys| algust ära sodimas, et seda ei loetaks. Järgmisel tarkvara versioonil oli see koht paremini kaitstud ja siis oli vaja ikka flopi pealt bootida. Aga BIOS oli parooli all. A: ja C: vs C: ja A:. Noh, siis järelikult muugime BIOSi paroolid lahti. Need on obfuskeeritud kujul kirjutatud kuhugi CMOS-mälusse ja masina ROM oli välja loetav. Ma võtsin ja disassembleerisin selle Sourcereri-nimelise disassembleriga ja matemaatika tunni ajal kirjutasin omale matemaatika vihikusse kõrvallehe peale programmi, mis seda obfuskeeritud asja lahti võtab. Järgmine tund oli ajaloo tund. Läksin ajaloo tunnist ära arvutiklassi, realiseerisin selle programmi ära ja muukisin BIOSi paroolid lahti. Mul tuli suur pahandus, sest see oli ajaloo tund, kust väga paljud olid puudunud, õpetaja oli väga kuri ja keeras käkki. Mul oli pärast vaja see tund järgi teha ja õnnestus ikkagi. Põhjendasime ikka kui väga hea programmi me tegime spetsifitseerimata, mis see oli. Et väga hea idee oli ja tuli lihtsalt minna arvutiklassi ja kohe ära teha. Parool oli obfuskeeritud jadašifrina või baithaaval võibolla isegi, et otsast proovides järjest tähthaaval sai selle ära arvata. Ma kunagi arvutiõpetajalt küsisin, et miks teil nii imelik parool on. Siis ta lahendas selle turvaprobleemi niimoodi, et delegeeris osa vastutust arvutiklassi haldamises ja võttis appi arvutiklassi haldama. Väga hea pedagoogiline meetod, töötas. Ei häkitud enam, ei olnud enam huvi edasi jagada paroole, mida ma kätte saan. 

\textbf{\enquote{Aga kust sul see krüpto huvi?}}

Seda läks sealsamas kandis ka vaja. Näiteks meie õpetaja ässitas Norton Diskreet'i\sidenote{Diskreet oli tarkvarapaketi Norton Utilities 6.0 osa ning sisaldas paljuski kurikuulsat (Kevin Mitnicku\index[ppl]{Mitnick, Kevin} andmetel kasutati väidetud 56 biti asemel 30 bitist võtit, ka teised uurijad on osundanud mitmetele olulistele nõrkustele) DESi implementatsiooni} DESi\sidenote{\emph{DES - Data Encryption Standard} on sümmeetriline algoritm andmete krüpteerimiseks. Algoritm on oma väikese võtmeruumi tõttu tänapäeval kasutamiseks sobimatu (murti avalikult jaanuaris 1999), kuid oli siiski alates 1977. aastast USA föderaalse andmetöötlusstandardi (FIPS) osa.} kallale. DESist ma ei saanud jagu, ma ei saanud DESist arugi tol hetkel. Aga tema suunas. Ta oli üldse sedasorti kaval mees, et kui ta näiteks kuulis kunagi, kui meil pinginaaber Veljo Haguga\index[ppl]{Hagu, Veljo} oli plaan kirjutada viirus, siis ta suutis meid sellest eemal hoida. Me olime mingeid olemasolevaid viirusi disassembleerinud ja vaadanud, kuidas need käivad. Õpetaja sattus pealt kuulma, kui me rääkisime viiruse tegemisest ja ütles, et kui teha, siis teha kohe selline \enquote{stealth}-viirus. Me olime väga nõus, aga seda me ei viitsinud teha, ja nii jäi viirus tegemata. 

Ta leidis meile muidu ka rakendust. Keskkoolis üldine taustaülesanne oli midagi arvutada. Minu arvutusülesanne oli arvutada arvu $e$ kahe tuhande komakohaga 30 sekundiga 10 MHz 286 peal. Üks klassivend arvutas $\pi$-d tuhande komakohaga 60 sekundiga, sest see koondus aeglasemalt. Ja kust tulid ajapiirangud? Õpetaja oli vaadanud, kui kiiresti temal vastus tuleb selle arvuti peal. Ma sain 35-sekundilise programmiga juba viie kätte, sest vastus oli õigem kui õpetajal. Kuna need erinesid, siis ta võttis targa raamatu ja siis selgus, et minul oli õige. Mul oli selleks hetkeks 21 sekundiline programm, mis käigu pealt suurendas mingi hetk arvutüübi pikkust. Algul tegi lühema tüübiga ja hiljem pikemaga, et kiiremini saaks. Aga see oli veel bugine ja ei töötanud õigesti. Ma kontrollisin oma enda programmi vastu. Ma olin minut aega töötava programmiga algul tulemuse välja arvutanud tulemuse faili kirjutanud. Siis oli mul ka näiteks variant programmist, mis küsis, et kui mitme sekundiga oli vaja arvutada ja siis ütles \emph{hard-coded} vastuse. Aga see ei sobinud õpetajale. Aga 35-sekundiline juba sobis, kui vastus oli õigem tema oma. Minu 21-sekundiline ei läinud tööle, aga õpetaja seepeale võttis ja kirjutas ise asja haljas assembleris\index{Keeled!Assembler} ja sai kolme sekundiga. Muidu me kirjutasime Pascalis\index{Keeled!Pascal}. 

Teine asi, mida me tegime, millega oli keskkooli ajal hulga nuputamist, oli interferentsi simuleerimine arvuti ekraanil. On kaks punktlaineallikat ringlainetega ja tuleb arvutada, kuidas lained liituvad, et tekiks interferentspilt. Seal ma nägin ka vaeva, arvutasin ruutjuurt assembleris\index{Keeled!Assembler} Newtoni meetodil. Ma arvutasin iga ekraani punkti kohta pimesi selle faasi välja nii et ühtegi punkti näha ei olnud aga ma sättisin pikslite väärtused nii, et palett oli seatud üleni mustaks. Arvutasin kõik väärtused ära assembleris optimeeritud arvutusvalemiga ja õpetaja õpetas Newtoni meetodit sinna juurde. Oli abiks. Assembleris sai Newtoni meetodit tehtud! Oli väga hariv. 

Ja lõpuks ma siis ketrasin VGA paletti. Tehnilise dokumentatsiooni failid liikusid. Seal oli kirjas, kuidas VGA paletti muuta ja ma seadsin siis paletti niimoodi, et need värvid, mis mul on, liikusid sujuvalt heleduse järgi. Ja siis tulemus oli see, nagu oleks liikunud lained ekraanil. Ja see oli minu meelest tippsaavutus, see oli väga ilus sujuv liikumine selle kümne megahertsi juures, punkte üle arvutada poleks kuidagi jõudnud. Pärast viis õpetaja mind ühe teise õpetaja tehtud programmi vaatama. See teine õpetaja ütles, et tema ideest see alguse saigi, et interferentsi simuleerida. Tema tegi Juku peal \verb|circle| käsuga valgeid rõngaid üksteise ümber viiemillimeetrise vahega. Need läksid mida edasi seda aeglasemaks ja minu reaalajalise sujuva pildi vastu ei olnud see midagi. Mul oli tükk tegu, et mitte naerma hakata. Aga kiitsin siis takka. Meie õpetaja suutis anda sellise ülesande, mille peale mul kulus ikka kaua ja sain palju targemaks. Õpetaja oli Tarmo Ainsaar\index[ppl]{Ainsaar, Tarmo}. Seesama, kes suunas meid viiruse kirjutamiselt ära ja kes lahendas selle BIOSi paroolide haldusteema probleemi meiega nii, et probleemi ei tulnud. Väga hea õpetaja. Ta suutis meid suunata tegema õigeid asju nii, et me seejuures õpime ja paha peale ei lähe. 

\textbf{\enquote{Kuidas sa Cyberisse sattusid?}}

Ma töötasin HClubis\index{HClub} (sattusin HClubi tööle seoses sellega, et ma installisin sinna Linuxi serveri, \emph{gateway} veebi ja meili jaoks) ja mõtlesin, et mida võiks magistriks teha. Seal tegeldi hajusate andmebaasidega. Me saatsime SQL-käsk-haaval andmebaaside \emph{diff}e üle võrgu mitmes suunas. See oli põnev, me saime selle tehniliselt lahendatud. Algul käis see mul üle UUCP, hiljem üle PPP ja POP3 ja SMTP. Mina ehitasin Internetti sinna alla, oli ka põnev. Ja neid \emph{diff}e siis saatsime ja tekkis küsimus andmebaaside konsistentsusest: mis tingimustel jääb ja mis tingimustel ei jää andmestik konsistentseks. Et kas me saame mingi \emph{eventually consistent} mudeli sealt või mitte. Ma mõtlesin, et ma hakkan sel teemal magistrit tegema.

Aga HClubis Interneti teemal, mis mind huvitas tol hetkel, ei olnud mul eriti kuhugi areneda. Seal ei olnud kellegi teise käest sedasorti asju õppida. Kui siis, ise õppida ja ehitada. Asju, mida oleks võinud --- pisi ISP-na veel ehitada sinna ISDN sissehelistamiskeskus, kui oleks leitud raha ja et see rentaabel on. Nihukesi asju oleks ehk saanud. 

Samal ajal ma käisin mõnes koolis abiks Linuxit installimas ja käisin laenamas RedHati installplaati 1997. a. suvel Elmer Joandi\index[ppl]{Joandi, Elmer} käest Tartu lähedalt maalt. Tal oli see plaadina kohe olemas ja ei pidanud flopidega mässama. Elmer ütles, et Tarvil\index[ppl]{Martens, Tarvi} olla plaan Tartusse meiesuguste jaoks pesa teha. Ja siis mina käisin juunikuus umbes Tallinnas Cyberneticas\index{Cybernetica} Helger Lipmaa \index[ppl]{Lipmaa, Helger} juures, et tuleks magistrit tegema hoopis krüpto teemal. Ma mõtlesin, et näiteks pordiks OpenSSLi\sidenote{Teek arvutite omavaheliseks turvaliseks suhtluseks krüptograafia abil. Tuntuim ja levinuim omataoline} Windowsile, sest mul oli Windowsi all krüptot vaja olnud aga seda polnud kuskilt võtta. Sellest konkreetsest ideest arvati kehvasti, et keegi vist juba on portinud ka midagi. Aga Tarvi kutsus niisama mitte-krüptot progema, mitte-krüptot. Oleks peaaegu Küberisse tulemata jäänud. Helger kutsus mu ikka turva asju tegema. Ja siis kutsuti mind 1997. aasta sügisel Arula motelli Küberi\sidenote{Kõnekeeles on \enquote{Küber}, \enquote{Küberneetika AS}, \enquote{Küberneetika Instituut} ja \enquote{AS Cybernetica} sisuliselt sünonüümid. Aastal 1960 asutati Eesti Teaduste Akadeemia Küberneetika Instituut. 1997. aastal reorganiseeriti see Küberneetika AS-iks ja hiljem nimetati ümber Cybernetica AS-iks. Aga sisu jäi suuresti samaks} väljasõiduistungile ehk \enquote{kvartalnajale}\sidenote{Nii kutsutakse Cybernetica töötajate regulaarseid (algselt kvartaalseid, sellest nimi) ja legendaarsed meeskonnaüritusi.}. Seal oli kutsutud kogu tulevane Küberi Tartu Andmeturbelabor\index{Cybernetica!Andmeturbelabor}. Viljar Tulit\index[ppl]{Tulit, Viljar} seal oma habemese diktsiooniga ütles, et seda sa pead ikka ise suutma ära otsustada millegi järgi, kas sa tahad siia tulla või ei taha, kui ma ütlesin, et segane on veel, kas ma tulen või ei tule.

Aga läksin ma Küberisse tarkade inimeste juurde. Seal olid Arne Ansper\index[ppl]{Ansper, Arne} ja Viljar Tulit, kes oli kogenud süsadmin (kogenum, kui mina). Kui mina tegin näiteks tükk aega FTP otsingumootorit Nuuskur\index{Nuuskur} koos teiste tudengitega, siis Arne oli selle stiilis nädala otsa õhtutega ära teinud. Arnel oli Vosa\index{Vosa} nimeline FTP otsingumootor Eesti FTP serverite kohta. Vosa nagu \enquote{Vanaisa Oli Sulle Archie\sidenote{Archie oli üks esimesi Interneti otsingumootoreid, mis võimaldas otsingut üle FTP arhiivide}}. Tal oli ainult veebiliides, meil oli muid liideseid ka. Meil oli telneti liides ja archie prospero protokolli\sidenote{Archie kataloogides navigeerimiseks loodud protokoll, mida võib pidada tänapäevase www protokolli eellaseks. Prosperot kasutades võis terve Internet välja näha nagu üks suur ühine kataloogipuu} liides, millega vana archie klient töötaks, ja meililiides ka. Meil oli võimas vinge süsteem tehtud kamba peale. Kõike ei teinud mina, teised tegid ka. Ma olin lihtsalt üks vedajaid ning lõpuks see, kes tegi kõige rohkem tükke. Ja sellega selgus, et Arne on tark. Seal oli veel asju, millega see selgus. Näiteks tal oli Fido ja Interneti vaheline gateway. Ma olin selle kaudu Fido\index{FidoNet}\sidenote{FidoNet oli ülemaailmne arvutivõrk, mida kasutati BBSide omavaheliseks suhtluseks} lugeja. Ma pole päris Fidonetti kunagi näinudki. Minu jaoks Fido oli lihtsalt järjekordne NNTP\index{NNTP}\sidenote{\emph{Network News Transfer Protocol (NNTP). Useneti} uudiste vahetamiseks kasutatud protokoll} server stiilis keeks.ioc.ee\index{Masinad!keeks.ioc.ee}. Sinna tuli kasutajanime ja parooliga läheneda ja sai tavalise \emph{newsreader}iga lugeda ja kirjutada. Minu jaoks oli Fido teenus üle Interneti, mida vahendas Arne tehtud süsteem. 

\textbf{\enquote{Mis sa praegu teed?}}
Praegu ma olen Küberis\index{Cybernetica} turvainsener. Praktikas ka tarneinsener, kes pakendab lahendusi ja ehitab nende jaoks automatiseeritult mingeid keskkondi. Õpetan ülikoolis\sidenote{Tartu Ülikool}, olen ülikoolis hajussüsteemide külalislektor, õpetan operatsioonisüsteeme baaskursusena, andmeturvet baaskursusena ja magistrantidele õpetan turvalist programmeerimist. See viimane tegeleb sellega, kuidas teha nii, et koodis poleks auke. Mõni auk ikka kuskil leidub aga eks neid ole aja jooksul endale piisavalt vastu tulnud. Andmeturbe kursus sai tehtud siis, kui ma olin magistrant Helger Lipmaa\index[ppl]{Lipmaa, Helger} juhendamisel. Helger ütles, et kuule, et sa võiks ülikooli  sellise andmeturbe kursuse teha. Mõeldud tehtud, tegingi. Kellegagi eriti nõu ei pidanud. Küberi turvaraamatu\sidenote{Hanson, V., Lipmaa H., Buldas A., Ansper A., Tulit V., Martens V. \enquote{Infosüsteemide turve 1. osa}, 1997. Hanson, V., Lipmaa H., Buldas A., Ansper A., Tulit V., Martens V. \enquote{Infosüsteemide turve 2. osa}, 1998.} võtsin vihjete jaoks aluseks, vist esimene valdavalt esimese köite. 


\textbf{\enquote{See tundub olevat nii sinu moodi. Võtad, teed ja saab väga hea?}}
Parim kiitus, mis ma andmeturbe ainele kuulnud olen oli siis, kui hakati küberkaitse magistrikava tegema. Tallinnas oli sel teemal koosolek. Ja oli häda, et kui me tahame neile kõike seda õpetada, mida tahaks, ei mahu see meil ainetesse ära. Selle peale oli vist Enn Tõugu\index[ppl]{Tõugu, Enn}, kes ütles, et \enquote{Kuidas, Meelis jõuab andmeturbe kursuses neist kõigist asjadest rääkida, mahutame ikka magistrikavasse ka ära}. Mis sest, et põhjalikumalt, aga küll me mahutame. See oli hea kompliment kursusele, et Meelis räägib sellest kõigest. 





\chapter{Võrgustiku analüüs}
\begin{itemize}
	\item Eemaldame kõik ühe seosega tipud
	\item Suured asutused konsolideerime prefixi alusel (TTÜ Arvutuskeskus -> TTÜ)
	\item Kesksed tipud (eraldi organisatsioonid ja inimesed)
	\item Degreede jaotuse graafik
\end{itemize}

\chapter{Kokkuvõte}
\begin{itemize}
	\item Paljud läksid keskkooli kõrvalt tööle, eriti kaheksakümnendatel
	\item Läbivaks jooneks on iseõppimine - ka näiteks Anto ei oska öelda, kust ta elektroonika-alase teadmise üles korjas. Andrus väidab, et \enquote{Programmeerimine sünnib vajadusest}
	\begin{itemize}
		\item Arne ütleb, et see oli \enquote{katsetamine ja kõlakad}. Kuna materjale ei olnud, siis õpiti üksteise käest ja inimsuhted muutusid lihtsalt praktiliselt hädavajalikuks: kui kellegi pealt ei olnud šnitti võtta, siis sa lithsalt ei arenenud. Järelikult tippudel \emph{pidi} olema väga hea suhtevõrgustik
		\item Ahti ütleb Jaani tsiteerides, et programmeerimine, erinevalt muudest distsipliinidest on ise õpitav: pärast algse iva kätte saamist on edasi kõik võimalik ise katsetamise ja mõtlemise teel kätte saada
	\end{itemize}
	\item Mitmel puhul öeldakse, et "ei programmeeri, sest nägin väga vara tõeliselt häid programmeerijaid" Järeldus: arvutihuvi tuleb teisiti realiseerida?
	\item Tundub, et mingil hetkel oli ajalehekuulutuse kaudu omale tech-teami hankimine täiesti levinud praktika (vt. Vilve, Bluemoon, Kütt)
	\item \enquote{Kõik, mis on võimalik inimese peas, on võimalik ka päriselt}. Priit Raspel, Jaan Tallinn
	\item \enquote{Kitsendused sunnivad inimesi tegema õigeid otsuseid} Sergei. Ka Meelis ja tema interferents
	\item Vene kogukond on täiesti tundmata ja, tundub, elas mingit täiesti oma elu. Kahju, nutikad inimesed, nii palju, kui näha oli
	\item Mis asi see oli, et nii Jaan kui Andres P. tegid esimese asjana tekstiredaktori? Seotud mitmel puhul läbi käinud mõttega sellest, et vanasti oli arvuti ühest küljest kättesaadavam ja teisalt vaesem. Kogu soft tuli ise kirjutada aga selle ka \emph{sai} ise kirjutada
	\item Õpetajatelt (Meelis, AK, Anto jt.) sai pigem üldist juhendamist ja suunamist kui konkreetset õpet. Aga ka lihtsalt kriitilise massi huviliste kokku toomist (Jaak Loonde Ahti jutus)
	\item Blokkskeemid jooksevad väga mitmest kohast (Priit, mrx, Ahti, Jaan jt.) läbi kui oluline vahend vigade leidmiseks. Paberil programmeerimine, kui selline samuti Jaan, Sergei, Ahti
	\item Raadiosõlm, kui maagiline paik (Andrus Aaslaid jt.)
	\item Segavate faktorite puudus: Ahti, Arne
	\item Lihtsus
	\begin{itemize}
		\item Arne: ma sain oma Atarist lõpuni aru
		\item Mast ja Margus kirjutasid kahe kuuga modemipanga
		\item Jaani tekstiredaktor
	\end{itemize}
	\item Ülim usaldus
	\begin{itemize}
		\item Andrus (kuidas tööle mindi, kuidas internetti jagati)
		\item Kuidas arvutiturve toimis
	\end{itemize}
\end{itemize}

%%
% The back matter contains appendices, bibliographies, indices, glossaries, etc.



\backmatter

\bibliography{sample-handout}
\bibliographystyle{plainnat}


\printindex[ppl]
\printindex


\end{document}

