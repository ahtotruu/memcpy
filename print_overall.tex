%!TEX TS-program = arara
% arara: latexmk: { clean: partial }
% arara: xelatex: { shell: true, synctex: true} 
% arara: makeindex
% arara: xelatex: { shell: true, synctex: true} 
% arara: xelatex: { shell: true, synctex: true} 
% arara: latexmk: { clean: partial }

\documentclass{tufte-book}
\usepackage[
    type={CC},
    modifier={by-nc-nd},
    version={4.0},
]{doclicense}


\ifxetex
  \newcommand{\textls}[2][5]{%
    \begingroup\addfontfeatures{LetterSpace=#1}#2\endgroup
  }
  \renewcommand{\allcapsspacing}[1]{\textls[15]{#1}}
  \renewcommand{\smallcapsspacing}[1]{\textls[10]{#1}}
  \renewcommand{\allcaps}[1]{\textls[15]{\MakeTextUppercase{#1}}}
  \renewcommand{\smallcaps}[1]{\smallcapsspacing{\scshape\MakeTextLowercase{#1}}}
  \renewcommand{\textsc}[1]{\smallcapsspacing{\textsmallcaps{#1}}}
\fi


\usepackage[T1]{fontenc}
%\usepackage[utf8]{inputenc}
\usepackage{polyglossia}
\setmainlanguage{estonian} 
\setotherlanguage{russian}
\newfontfamily\russianfont[Script=Cyrillic]{Linux Libertine}

\hypersetup{colorlinks}% uncomment this line if you prefer colored hyperlinks (e.g., for onscreen viewing)

%%
% Book metadata
%\title{print(memcpy[])\thanks{Thanks to Edward R.~Tufte for his inspiration.}}
\title{memcpy.print()}
\author[Andres Kütt]{Andres Kütt}
\publisher{TeamConsulting}

%%
% If they're installed, use Bergamo and Chantilly from www.fontsite.com.
% They're clones of Bembo and Gill Sans, respectively.
%\IfFileExists{bergamo.sty}{\usepackage[osf]{bergamo}}{}% Bembo
%\IfFileExists{chantill.sty}{\usepackage{chantill}}{}% Gill Sans

%\usepackage{microtype}

%%
% Just some sample text
\usepackage{lipsum}

%%
% For nicely typeset tabular material
\usepackage{booktabs}

%%
% For graphics / images
\usepackage{graphicx}
\setkeys{Gin}{width=\linewidth,totalheight=\textheight,keepaspectratio}
\graphicspath{{graphics/}}

% The fancyvrb package lets us customize the formatting of verbatim
% environments.  We use a slightly smaller font.
\usepackage{fancyvrb}
\fvset{fontsize=\normalsize}

%%
% Prints argument within hanging parentheses (i.e., parentheses that take
% up no horizontal space).  Useful in tabular environments.
\newcommand{\hangp}[1]{\makebox[0pt][r]{(}#1\makebox[0pt][l]{)}}

%%
% Prints an asterisk that takes up no horizontal space.
% Useful in tabular environments.
\newcommand{\hangstar}{\makebox[0pt][l]{*}}

%%
% Prints a trailing space in a smart way.
\usepackage{xspace}

%%
% Some shortcuts for Tufte's book titles.  The lowercase commands will
% produce the initials of the book title in italics.  The all-caps commands
% will print out the full title of the book in italics.
\newcommand{\vdqi}{\textit{VDQI}\xspace}
\newcommand{\ei}{\textit{EI}\xspace}
\newcommand{\ve}{\textit{VE}\xspace}
\newcommand{\be}{\textit{BE}\xspace}
\newcommand{\VDQI}{\textit{The Visual Display of Quantitative Information}\xspace}
\newcommand{\EI}{\textit{Envisioning Information}\xspace}
\newcommand{\VE}{\textit{Visual Explanations}\xspace}
\newcommand{\BE}{\textit{Beautiful Evidence}\xspace}

\newcommand{\TL}{Tufte-\LaTeX\xspace}

% Prints the month name (e.g., January) and the year (e.g., 2008)
\newcommand{\monthyear}{%
  \ifcase\month\or January\or February\or March\or April\or May\or June\or
  July\or August\or September\or October\or November\or
  December\fi\space\number\year
}


% Prints an epigraph and speaker in sans serif, all-caps type.
\newcommand{\openepigraph}[2]{%
  %\sffamily\fontsize{14}{16}\selectfont
  \begin{fullwidth}
  \sffamily\large
  \begin{doublespace}
  \noindent\allcaps{#1}\\% epigraph
  \noindent\allcaps{#2}% author
  \end{doublespace}
  \end{fullwidth}
}

% Inserts a blank page
\newcommand{\blankpage}{\newpage\hbox{}\thispagestyle{empty}\newpage}

\usepackage{units}

% Typesets the font size, leading, and measure in the form of 10/12x26 pc.
\newcommand{\measure}[3]{#1/#2$\times$\unit[#3]{pc}}

% Macros for typesetting the documentation
\newcommand{\hlred}[1]{\textcolor{Maroon}{#1}}% prints in red
\newcommand{\hangleft}[1]{\makebox[0pt][r]{#1}}
\newcommand{\hairsp}{\hspace{1pt}}% hair space
\newcommand{\hquad}{\hskip0.5em\relax}% half quad space
\newcommand{\TODO}{\textcolor{red}{\bf TODO!}\xspace}
\newcommand{\ie}{\textit{i.\hairsp{}e.}\xspace}
\newcommand{\eg}{\textit{e.\hairsp{}g.}\xspace}
\newcommand{\na}{\quad--}% used in tables for N/A cells
\providecommand{\XeLaTeX}{X\lower.5ex\hbox{\kern-0.15em\reflectbox{E}}\kern-0.1em\LaTeX}
\newcommand{\tXeLaTeX}{\XeLaTeX\index{XeLaTeX@\protect\XeLaTeX}}
% \index{\texttt{\textbackslash xyz}@\hangleft{\texttt{\textbackslash}}\texttt{xyz}}
\newcommand{\tuftebs}{\symbol{'134}}% a backslash in tt type in OT1/T1
\newcommand{\doccmdnoindex}[2][]{\texttt{\tuftebs#2}}% command name -- adds backslash automatically (and doesn't add cmd to the index)
\newcommand{\doccmddef}[2][]{%
  \hlred{\texttt{\tuftebs#2}}\label{cmd:#2}%
  \ifthenelse{\isempty{#1}}%
    {% add the command to the index
      \index{#2 command@\protect\hangleft{\texttt{\tuftebs}}\texttt{#2}}% command name
    }%
    {% add the command and package to the index
      \index{#2 command@\protect\hangleft{\texttt{\tuftebs}}\texttt{#2} (\texttt{#1} package)}% command name
      \index{#1 package@\texttt{#1} package}\index{packages!#1@\texttt{#1}}% package name
    }%
}% command name -- adds backslash automatically
\newcommand{\doccmd}[2][]{%
  \texttt{\tuftebs#2}%
  \ifthenelse{\isempty{#1}}%
    {% add the command to the index
      \index{#2 command@\protect\hangleft{\texttt{\tuftebs}}\texttt{#2}}% command name
    }%
    {% add the command and package to the index
      \index{#2 command@\protect\hangleft{\texttt{\tuftebs}}\texttt{#2} (\texttt{#1} package)}% command name
      \index{#1 package@\texttt{#1} package}\index{packages!#1@\texttt{#1}}% package name
    }%
}% command name -- adds backslash automatically
\newcommand{\docopt}[1]{\ensuremath{\langle}\textrm{\textit{#1}}\ensuremath{\rangle}}% optional command argument
\newcommand{\docarg}[1]{\textrm{\textit{#1}}}% (required) command argument
\newenvironment{docspec}{\begin{quotation}\ttfamily\parskip0pt\parindent0pt\ignorespaces}{\end{quotation}}% command specification environment
\newcommand{\docenv}[1]{\texttt{#1}\index{#1 environment@\texttt{#1} environment}\index{environments!#1@\texttt{#1}}}% environment name
\newcommand{\docenvdef}[1]{\hlred{\texttt{#1}}\label{env:#1}\index{#1 environment@\texttt{#1} environment}\index{environments!#1@\texttt{#1}}}% environment name
\newcommand{\docpkg}[1]{\texttt{#1}\index{#1 package@\texttt{#1} package}\index{packages!#1@\texttt{#1}}}% package name
\newcommand{\doccls}[1]{\texttt{#1}}% document class name
\newcommand{\docclsopt}[1]{\texttt{#1}\index{#1 class option@\texttt{#1} class option}\index{class options!#1@\texttt{#1}}}% document class option name
\newcommand{\docclsoptdef}[1]{\hlred{\texttt{#1}}\label{clsopt:#1}\index{#1 class option@\texttt{#1} class option}\index{class options!#1@\texttt{#1}}}% document class option name defined
\newcommand{\docmsg}[2]{\bigskip\begin{fullwidth}\noindent\ttfamily#1\end{fullwidth}\medskip\par\noindent#2}
\newcommand{\docfilehook}[2]{\texttt{#1}\index{file hooks!#2}\index{#1@\texttt{#1}}}
\newcommand{\doccounter}[1]{\texttt{#1}\index{#1 counter@\texttt{#1} counter}}

% Generates the index
\usepackage{imakeidx}
\makeindex[name=ppl, title={Nimede register}]
\makeindex[title={Indeks}]

\usepackage{csquotes}



\begin{document}

% Front matter
\frontmatter

% r.1 blank page
\blankpage

% v.2 epigraphs
\newpage\thispagestyle{empty}
\openepigraph{%
Design and programming are human activities; forget that and all is lost.
}{Bjarne Stroustrup%, {\itshape Design, Form, and Chaos}
}
\vfill
\begin{fullwidth}
\sffamily\large
\begin{doublespace}
\noindent\allcaps{Ärge valetage isad }\\ % The quote
\noindent\allcaps{ära hoia kinni ema mind}\\ % The quote
\noindent\allcaps{Need ei ole halvad sõbrad}\\ % The quote
\noindent\allcaps{see on minu Vennaskond ja ring}\\ % The quote
\noindent\allcaps{Vennaskond. \enquote{Jumal kaitse vennaskonda}} % The author
\end{doublespace}
\end{fullwidth}
%\vfill
%\openepigraph{% 
%Ärge valetage isad ära hoia kinni ema mind Need ei ole halvad sõbrad see on minu Vennaskond ja ring}{Vennaskond. \enquote{Jumal kaitse vennaskonda}}
%\vfill
%\openepigraph{%
%\ldots the designer of a new system must not only be the implementor and the first 
%large-scale user; the designer should also write the first user manual\ldots 
%If I had not participated fully in all these activities, 
%literally hundreds of improvements would never have been made, 
%because I would never have thought of them or perceived 
%why they were important.
%}{Donald E. Knuth}


% r.3 full title page
\maketitle


% v.4 copyright page
\newpage
\begin{fullwidth}
~\vfill
\thispagestyle{empty}
\setlength{\parindent}{0pt}
\setlength{\parskip}{\baselineskip}
Copyright \copyright\ \the\year\ \thanklessauthor

\par\smallcaps{Published by \thanklesspublisher}

%\par\smallcaps{tufte-latex.googlecode.com}

\par \doclicenseThis \index{license}

\par\textit{First printing, \monthyear}
\end{fullwidth}

% r.5 contents
\tableofcontents

%\listoffigures

%\listoftables

% r.7 dedication
\cleardoublepage
~\vfill
\begin{doublespace}
\noindent\fontsize{18}{22}\selectfont\itshape
\nohyphenation
Pühendatud kõigile Eesti arvuti-inimestele, patsiga ja ilma.
\end{doublespace}
\vfill
\vfill


% r.9 introduction
\cleardoublepage
\chapter*{Sissejuhatus}
Juhatame sisse. 
\begin{itemize}
	\item Miks ma seda teen
	\begin{itemize}
		\item \enquote{Tahan kord saada selliseks, nagu on Villu või Freddy või Rott või Striit.}\sidenote{Villu Tamme\index[ppl]{Tamme, Villu}, "Paneme punki"}
	\end{itemize}
	\item Eesmärk: kujutada inimesi ja nende suhteid (mitte näiteks kurioosseid hetki või ettevõtteid)
	\item Lühike ajalugu: idee, otsing, podcast, siis analüüs ja raamat\index{Ajalugu}
	\item Sisu kohta
	\begin{itemize}
		\item Kõik ei mahtunud raamatusse, kõik ei soovinud rääkida ja kõik ei tulnud pähe. Andestust!
		\begin{itemize}
			\item Välja on jäänud näiteks Mainor ja natuke vanema põlvkonna (näiteks kadunud Ahto Kalja) tegemised
		\end{itemize}
		\item Lood lähevad omavahel vastuollu, see on OK
		\item Kõik on isiksused. Mõned kergemad, mõned raskemad. Olen üritanud suhte-taagast üle olla
		\item Nii \enquote{läbipaistev} vaade, kui võimalik
		\item Inimesed tähestikulises järjekorras
		\item Oma jutt on ka, sest muidu jääks juttudesse kummaline auk, lisaks tuleks ju anda aimu, mis prisma läbi ülejäänud asjad on kirjutatud. Intervjueerisin ennast ise
	\end{itemize}
	\item Kuidas lugeda
	\begin{itemize}
		\item On indeks, eraldi inimeste oma
		\item On lühikesed selgitused mainitud riistvara ja arvutite osas
		\item Detailsema jutu leiab igaüks ise internetist
	\end{itemize}
	\item Tänuavaldused
	\begin{itemize}
		\item Rein Rüüsak, A\&A ajaloo välja uurimine
		\item Ott Köstner, memcpy kaanepilt
	\end{itemize}
 \end{itemize}

%%
% Start the main matter (normal chapters)
\mainmatter


\chapter{Andrus Aaslaid}
%!TEX TS-program = arara
% arara: myindex

\index[ppl]{Aaslaid, Andrus}
\question{Kuidas sa arvutite juurde jõudsid?}

Tihti on nii, et me ei mäleta, kuidas me oma elu muutvad otsused  
tegime. Aga seda juhust ma mäletan täpselt. Mul oli juba toona 
raadiohobi. Olin põhikooli juntsu ja mulle meeldis hirmsasti mööda 
lühilainet ringi kammida. Meil oli kodus Melodija 101 stereo, Riia 
raadiotehase\sidenote{A. S. Popovi nimeline Riia Raadiotehas, alates 1951 Rigas 
Radio Rupnica.} toodang. Sellega ma siis seiklesin suviti, kui midagi targemat 
teha ei olnud, mööda eetrit. Tegelikult oli mul kaks raadiot: lisaks Melodijale 
detektorvastuvõtja, mille mu poolvend 
oli mulle ehitanud. Sellega ma istusin pööningul. Vanemad tegelesid 
põllumajandusega ja neil oli 
üks konkreetne põllumajandusnipp: raamatukogudest toodi vanu ajakirju, 
need rebiti lehtedeks, keerati ümber õõnsa 
põhjaga pudeli väikesteks pottideks, mille sisse istutati taimed. 
Paber lagunes mulla sees ära, taim pääses põllul vabaks. Neid ajakirju oli 
pööningul tohutu hunnik, muu hulgas mitu aastakäiku 
\begin{russian}Техника - молодёжи\end{russian}'t\sidenote{Aastast 1933 ilmuv 
algselt Nõukogude ja nüüd Vene populaarteaduslik ajakiri.}. Lappasin siis 
pööningul neid ajakirju, detektoriklapid peas. 

Igatahes ükskord astusin ma tuppa, lülitasin Melodija sisse ja sealt öeldi, et 
Tallinna 43. Keskkool\index{Tallinna 43. Keskkool}\sidenote{Praegune 
tehnikagümnaasium.\index{Tehnikagümnaasium|see{Tallinna 43. Keskkool}}} 
on otsustanud hakata 
eksperimentaalseks tehnikaülikooli\index{Tallinna 
Tehnikaülikool}\sidenote{Tallinna Polütehniline Instituut, praegune Tallinna 
Tehnikaülikool.} ettevalmistuskooliks ja 
nad võtavad kümnendasse klassi vastu õpilasi, kes tahaksid TPIsse edasi õppima 
minna. Kuulasin uudise ära, lülitasin raadio välja, läksin vanemate juurde ja 
teatasin, et lähen Tallinnasse kooli. Ma olin siis 14.


\question{Kust sa pärit oled, et tahtsid Tallinna kooli 
minna?}

Pärit olen ma tegelikult kahesaja meetri kauguselt sealt, kus ma täna elan, 
ehk siis Tallinnast. Aga kuna mu perekond otsustas evakueeruda 
Muhusse, kui ma olin kahe- või kolmeaastane, siis mind 
deporteeriti sinna. Nii et oma põrsapõlve veetsin Muhus ja siis ühel 
hetkel panin sealt tagasi tehnoloogia juurde putku. 

\question{Mõni ime, et te Mastiga\index[ppl]{Kaal, 
Madis}\index[ppl]{Mast}\sidenote{Vt lk \pageref{cptr:mast}.} 
hästi läbi saate!}

Me oleme Mastiga ühe kooli poisid, Mast oli keskkoolis, kui mina olin 
põhikoolis. Me oleme mõnda aega isegi sama 
raadiosõlme väisanud. Aga ega tollel ajal nooremad ja vanemad väga läbi käinud, 
eriti veel 
maakohtades. Mast oli hea 
poiss, ei peksnud nooremaid ega midagi. 

\question{Mis sealt lühilaine pealt kostis? Muusikat?}

Ei, muusikat kuulati Radio Luxembourgist. Lühilaine pealt tuli erinevaid 
hääli: morset, huvitavaid kahinaid ja sahinaid, keegi 
luges numbreid. Lühilaine on tegelikult siiamaani päris hea tervise 
juures, eeter on maast laeni sodi täis ja olemus 
ei ole väga palju muutunud. Võibolla propagandasaateid on vähemaks 
jäänud ja Hiina raadiojaamu vaikselt kinni pandud  
interneti pealetulekuga. Üldiselt on lühilaine ilmselt ikka samasugune nagu 
nelikümmend aastat tagasi.

\question{Kas nende ajakirjade hulgas oli arvutiajakirju ka?}

Esimest arvutit nägin tänu poolvennale. Ta tundis Guido 
Tammissaart\index[ppl]{Tammissaar, Guido} Eesti Energia 
arvutuskeskusest\index{Eesti Energia!Arvutuskeskus}. Ühel 
päeval tuli poolvend maale ja ütles: \enquote{Tule kaasa paariks päevaks, näed, 
mis asi 
see arvuti on. Sind see tehnikaasi huvitab.} Ja lubatigi mind paariks päevaks 
maalt 
linna. Estonia puiestee arvutuskeskuses olid tollal veel põhiliselt 
SMid\index{SM EVM}\sidenote{\begin{russian}Система Малых ЭВМ (СМ 
ЭВМ)\end{russian} oli mitut tüüpi Nõukogude Liidus toodetud, enamasti lääne 
analoogidel põhinevate arvutite üldnimetus.}. Ja 
üks CP/M\sidenote{CP/M oli 1974. 
aastal Inteli 8080/85 protsessorisarja tarvis turule toodud 
operatsioonisüsteem, mille 1980ndatel asendas mitmes mõttes sarnane MS-DOS.} 
masin, mis tagantjärele tundub oma sotsmaa disaini poolest täiesti kosmiline. 
Küllap Bulgaarias toodetud. Olen mõelnud, et 
peaks üles otsima, mis masin see selline võis olla. 

Sellel CP/M masinal ma klõbistasin niisama, aga 
SM-4\index{SM EVM!SM-4}\sidenote{SM-4 oli PDP-11/40\index{PDP-11} 
ühilduv 
Nõukogude päritolu ja terves idablokis toodetud arvutisüsteem.} peal 
kirjutasin selsamal päeval oma esimese BASICu\index{BASIC} programmi. 
See oli derivaat mingist asjast, mida mulle näidati, et näed, umbes nii 
käib. Ja edasi ma olin \emph{hooked}. Sellest ühest päevast piisas, et sõltlane 
tekitada. 

\question{See oli enne seda, kui otsustasid, et nüüd oled 
neliteist ja lähed Tallinnasse kooli?}

Ma ei oskagi öelda, ma ei ole sada protsenti kindel, kumb oli enne, kumb 
pärast, ja kas huvi tulla Tallinnasse mängis rolli. Ega nad ju 
arvutikallakut tegelikult ei propageerinud, suurem rõhk oli elektroonikal. 
Tarkvara osa nad väga ei reklaaminud. Minust pidi tegelikult elektroonik saama 
ja see minust ka sai, aga tollal tundusid ikkagi arvutid 
see päris asi. 

\question{Kas 43. keskkoolis valmistati päriselt ka ette 
ülikooliks? Oli sellest kasu?}

See oli selline kahe teraga mõõk -- valmistati ette ja 
väga hästi. Keskkooliprogrammi olid kokku pannud 
tollaste inseneride õpetajad, kes teadsid suhteliselt hästi, mida tuleks 
õpetada, et põhi alla saada. Saime 
läbisegi tavalisi keskkooliaineid ja siis ühel hetkel tuli härra 
Tiidemann\index[ppl]{Tiidemann, Tiit} meile rääkima võllide 
epüüridest\sidenote{Epüür (pr \emph{épure}) on teatava suuruse asukohast 
olenevate väärtuste graafiline esitus.}. Sisuliselt tegime käsitsi võllidele 
rakendavate jõudude arvutusi, näiteks kust läheb võll katki, kui see on siit
sellise ja sealt säärase jämedusega. Vahelduseks loeti meile 
teise kursuse elektrotehnikat ja 
inseneripsühholoogiat, mida andis Toomsalu\index[ppl]{Toomsalu, Arvo} ja mis ei 
olnud vist üldse TPI õppekavas. Meie õppekava lühinimetusega ETEK\index{ETEK}, 
mille koostasid Ants Reili\index[ppl]{Reili, Ants} ja 
Peeter Grossberg\index[ppl]{Grossberg, Peeter}, oli kõikide jaoks äge 
eksperiment ja täielik \emph{greenfield}, eriti kuna 
olime esimene lend.\sidenote{vt ka lk \pageref{sisu:43kool}.} 

Lahe oli ka see, et enne meid oli keskkool tühjaks löödud ja me olime kolm 
aastat keskkooli kõige 
vanem klass. Olime koolis nagu jumalad ja 
tänu sellele jäid olemata mitmed probleemid, mida tavalistes 
keskkoolides tol ajal veel eksisteeris. Keegi kedagi ei toginud ega 
nüginud ja samal ajal tekkis kõigil mingisugune väärikus. 

Kahe teraga mõõk oli see aga sellepärast, et nii kõva põhja pealt läksid paljud 
otse tööle. Me saime ju keskkooli lõpetades kõik  
automaatselt TPIsse sisse, sisseastumiseksamit ei olnud vaja teha. Nii et kõik 
meie vist kaheksateist õpilast marssis otse TPIsse. Nendest 
nominaalajaga lõpetas kooli vist paar inimest. Paljud läksid tööle, kuna aeg 
oli 
selline, et see, mida TPIs tollal arvutiteadusena õpetati, ei jõudnud 
päris elule veel järele. See pidi olema aasta 1991 või 1992, kui 
see \enquote{kambriumiplahvatus} siin Eestis toimus.

Mina istusin ööd-päevad arvuti taga ja kirjutasin 
ihuüksi tarkvara, mis pidi 
üleval hoidma tervet suurt autoparki. Samal ajal üritasin ennast kuidagi nügida 
läbi SuperCalci\index{SuperCalc}\sidenote{Varajane tabelarvutussüsteem, 
algselt loodud CP\textbackslash M operatsioonisüsteemile.} arvestusest TPIs, 
kus 
aeg-ajalt tuli õppejõule näidata, et \enquote{ära nii tee, nii see asi päris ei 
käi}. 
Mitte et nad oleksid rumalad olnud, nad õpetasid seda, mida olid kogu 
aeg õpetanud. Nüüd aga tekkis selline seis, kus reaalne elu liikus edasi palju 
kiiremini kui õppekava.

\question{Kuidas sa ikkagi programmeerimise juurde jõudsid? Sa 
pidid seda ju saama kuskil harjutada?}

Tänu 43. keskkoolile see eksperiment kestis ja kestab mõnes mõttes tänaseni. 
Seal oli 
põhimõtteliselt esimest korda selline päris arvutiinimese elu. Kuna 
IT-spetsialiste liiga palju ei olnud, siis juhtus selline hämar lugu, et meile 
Eero Tohvriga\index[ppl]{Tohver, Eero} ulatati kümnendas klassis arvutiklassi 
võtmed ja hakati 
koolist palka maksma. Tegelikult oli see vist seotud 
kerge koolipoolse kaastundega. Peale 
kaheksandat klassi tööstuskooli tulemise traditsiooni ei olnud enam juba 
mõnikümmend aastat ja kõigile tundus see kangesti hirmus, et laps tuleb üksi 
Tallinnasse. Ma arvan, et see oli pigem koolipoolne stipendium. 
Kahe peale maksti meile täisõppejõu palka, mis 
ei olnud ilmselt palju väiksem kui õpetajad 
ise said. Nii hästi kui keskkooli ajal ei ole ma kunagi ei varem ega 
hiljem elanud. 

\question{Mida te selle rahaga tegite?}

Käisime restoranis söömas ja mida ikka lapsed rahaga teevad. Aga kool sai 
selle, et nad ei pidanud rohkem arvutiklassiga tegelema. Klassis oli kolm-neli  
Iskrat\index{Iskra}\sidenote{\begin{russian}Искра\end{russian} oli 
mitmel pool Nõukogude Liidus eri modifikatsioonides toodetud arvutiseeria, mis
 omakorda jagunes erinevaid lääne süsteeme kopeerivateks mudeliperekondadeks.}, 
mida me püsti hoidsime. Meie asi 
oli hoolitseda, et masinad töötaksid ja nendel saaks midagi õpetada. Ühel
hetkel, kui olime ise juba natuke vanemad, tekkis arvutiklassi 
kamp nooremaid huvilisi, kes seal pidevalt hängisid. Arenes
tüüpiline arvutiklassi ökosüsteem. Ühel suvel ka remontisime 
klassi: värvisime ja panime uued põrandakatted. Ühesõnaga käitusime 
loodetavasti heaperemehelikult. 

\question{Tollal ilmselt ei olnud sarnastes situatsioonides hea{\-}peremeheliku 
käitumisega eriti 
probleeme?}

Aeg oli selline, inimeste usaldus oli suur. Arvuti oli müstiline ja teistmoodi 
asi, vanem 
generatsioon justkui kartis seda. Kunagi asus Rävala puiesteel, seal, kus 
praegu on Sakala 3 teatrimaja, turismibüroo Sarved ja Sõrad\index{Sarved ja 
Sõrad} (ma ei tea siiamaani, kellele see kuulus). Juhtusin nende akna alt mööda 
minema ja nägin, et neil on 
seal arvuti. See oli vist aastal 1991, igatahes ma veel ei töötanud 
Skriiningus\index{Skriining}. Keskkool 
oli läbi, sinna mind enam sisse ei lastud arvutit kasutama. Eks sõltlane käis 
mööda linna ja järsku nägi arvutit. Tundmatu värske keskkoolilõpetaja 
marssis tundmatusse firmasse hooga sisse, et 
\enquote{teil on siin arvuti, ma tahaksin seda kasutada}. Ja ilma mingisuguse 
tänapäeval heaks kiidetud taustauuringuta ja töövestluseta ütles firma omanik 
oma kirjutuslaua tagant: \enquote{Jah, loomulikult, me tahaksime seda ise ka 
kasutada.}  Pikema jututa anti mulle kontorivõtmed ja öeldi: \enquote{Tee 
see korda, et meie saaksime ka arvutit kasutada}. Ja avastasingi end
arvuti tagant, ilma et keegi oleks isegi dokumenti vaadanud või mõelnud, kas 
tegu on
vargaga, kes tahab terve firma ära varastada või 
ainult arvuti. Usaldus, mis tollal valitses inimeste vastu, kes 
oskasid arvuti sisse lülitada ja sellega midagi teha, oli 
\emph{enormous}.\phantomsection\label{sisu:andrus_usaldus} 
Tänapäeval ei ole võimalik seda ette kujutada. Värskel keskkoolilõpetajal oli 
põhimõtteliselt võimalik küsida ükskõik millise firma ükskõik millise  
arvuti \enquote{võtmed}. 

Noh, see lõppes muidugi sellega, et lõpuks tuli Imre Perli\index[ppl]{Perli, 
Imre}\sidenote{Imre Perli oli pehmelt öeldes raju elulooga Eesti 
arvutispetsialist, kes sai kuulsaks \enquote{Perli andmebaasi} koostajana. 
Kasutades ära ligipääsu mitmele andmebaasile, lõi ta üheksakümnendate keskel 
althõlma levinud \enquote{superandmebaasi}, mis sisaldas isikustatud andmeid autode, (toona üsna 
haruldaste) mobiiltelefonide, aadresside jms kohta.
 Perli hukkus segastel asjaoludel 15. aprillil 2000 
politseioperatsiooni käigus.} ja kopeeris kellelegi andmebaasid. Eks iga 
aeg saab lõpuks otsa. 

\question{Kuidas see programmeerima õppimise protsess ikkagi käis?}

See on eelmisel sajandil tekkinud paradigma, et 
programmeerimine on midagi, mida peab õppima ja millega tuleb 
spetsiaalselt vaeva näha. Programmeerimine juhtub. Vajadusest ja tahtmisest. 
Keegi ei ole mulle mitte kunagi õpetanud 
ridagi C-d ega assemblerit. 

\question{Ometi said ju kuskilt teada, kuidas \texttt{malloc} käib.}

See sündis tahtmisest teha. Mina hakkasin  
Pascalit\index{Pascal} õppima seepärast, et 
mulle sattus kätte Jürgensoni pruunide kaantega Pascali  
raamat\sidenote{Rein Jürgensoni 
\enquote{Programmeerimine Pascal-keeles} (1985), mis 
huviliste hulgas laialt levis.}, mis on tagantjärele mõeldes 
päris õudne algus programmeerimisele. Kui 
Turbo Pascal hakkas ära tüütama (selles 
keeles midagi normaalset teha oli väga keeruline), siis ühel hetkel 
leidsin, et assembler\index{assembler} on see päris asi. Kuna tol 
ajal oli popp kirjutada igasuguseid demosid ja häkkida kõiki tarkvarasid, mis 
kätte sattus, siis\ldots{ }Kuidas õppida x86 
assemblerit? Võtad raamatu ühte kätte ja AT86 teise kätte ning hakkad tegema.

\question{Kust sa selle raamatu said? Neid ju ei liikunud.}

Liikus küll. Selle eest tuleb tõenäoliselt varem või hiljem anda
presidendi auraha Tarmo Mamersile\index[ppl]{Mamers, 
Tarmo}\index[ppl]{Mamers, Tarmo}, kes oli tollal 
TTÜ-s\index{Tallinna Tehnikaülikool} üks arvutiasjanduse püstihoidjatest. 
Tarmo kaudu materjalid liikusidki, käest kätte. Tema oli raudselt minu varane 
mentor ja veel pikka aega ka siis, kui ma 
juba tööl käisin. Hiljem tuli 
FidoNet. Kui ma oma esimese FidoNeti \emph{point}'i 
püsti panin, siis oli kõik juba palju lihtsam, sest aken maailma oli
olemas. \emph{Point}'i püstipanemine käis ka loomulikult läbi TPI. Seal käis 
põhiline elu ja \emph{action}   
Aare Tali\index[ppl]{Tali, Aare}\phantomsection\label{sisu!aare_tali} ja Tõnu 
Raimla\index[ppl]{Raimla, Tõnu} toas. Tarmo juures teises ruumis oli natuke 
rahulikum 
õhkkond. 

Ühel  
hetkel (töötasin siis Skriiningus\index{Skriining}) tekkis mul kinnisidee teha 
endale FidoNeti \emph{point}, et 
lõpuks olla osa maailmast. Läksin Aare juurde: \enquote{Noh, Aare, 
sa oled siin \emph{sysop} ja värk} ning Aare ütles talle omase abivalmidusega: 
\enquote{Jah, masin on seal.} Leidsingi ennast seepeale BBSi masina tagant ja 
asusin
valmistama FidoNeti \emph{node}'i. Ilmselt Tõnu või keegi 
lõpuks halastas mu peale ja näitas, kuidas seda päriselt teha. 

Edaspidi oli materjal palju kättesaadavam, sai 
igasuguseid dokumente risti-rästi alla laadida. 

\question{Mida sa TPIsse õppima läksid?}

Ma läksin LIsse. Tollal nimetati seda vist informaatikaks. 
Kuna sain suhteliselt ruttu aru, et ma ei ole võimeline hommikul loengutes 
käima, siis läksime pundi inimestega, kes olid ka otsustanud, et nemad 
peavad õhtuõppes käima, dekanaati ja nõudsime õhtust vahetust. Kateedris öeldi, 
et jaa, väga tore mõte, aga 
meie kogemus näitab, et kui te juba sihukese jutuga tulete, siis vaevalt keegi 
teist seal õhtuses ka käima hakkab. Me ei hakka teie jaoks  
eraldi rühma püsti panema, käite ehitajatega esimese aasta koos koolis. Ja kui 
teisel aastal veel siin olete, siis vaatame seda asja. Kas nüüd osalt selle 
pärast või et dekanaadil oli õigus, nii või teisiti kukkusime 
sealt kõik robinal kolmanda kuu lõpuks välja ja läksime tööle. Nii 
et TPI on mul siiamaani lõpetamata. 

\question{Sa mainisid, et kirjutasid autobaasi softi. Kuidas 
sa seda tegema sattusid?}

Tol ajal \emph{start-up}-kultuuri ja ettevõtluse ehitamist veel ei 
eksisteerinud. Me lõpetasime kooli ajal, kui esimesi 
arvutikooperatiive oli väike käputäis. Minu esimene ametlik töökoht pidi 
olema tegelikult Noorsooteatri valgustaja. Kuna mulle juba 
tollel ajal meeldis audioga tegeleda, siis tahtsin sinna helimeheks minna, 
aga helimees oli värskelt tööle võetud ja valgustaja koht oli vaba. Paar päeva 
enne seda, kui pidin lepingu alla kirjutama, 
küsis Tarmo Mamers\index[ppl]{Mamers, Tarmo}, kas ma ei tahaks ikkagi päris 
tööd teha, kuna Skriining\index{Skriining} otsis programmeerijat. 

Nii sattusingi Skriiningusse Kalle Lotamõisa\index[ppl]{Lotamõis, Kalle} juurde 
tööle. Seal öeldi mulle esimese ülesandena, et \enquote{autopark on sellel 
aadressil}. 
Neil oli mingi eriti eksootilise asja peal jooksev andmebaasisüsteem, 
isegi mitte \emph{mainframe}, vaid mingi mini. Ja see tuli 
moodsale vahendile ümber kirjutada. Moodne vahend tähendas tol ajal Novell 
Netware'i\index{Novell} ja Paul Leis\index[ppl]{Leis, Paul} oli värskelt toonud 
Eestisse sellise asja nagu DataFlex\index{DataFlex}. Tegu oli päris 
 korraliku objektorienteeritud kõrgkeelega. Hakkasin ühest otsast õppima, 
kuidas DataFlexis programmeeritakse, ja 
teisest otsast, kuidas autopark töötab. 

\question{Ahaa, läksid kohe äriprotsessi ka sisse!}

Äriprotsessid olid seal paljuski olemas, st töötav tarkvara 
oli olemas. Pigem oli seal äriprotsesside seisukohast hea lastetuba, et ära 
kunagi eelda midagi. Näiteks mina oma IT-inimese mõistusega tegin oma 
arust mõned asjad paremaks ja siis selgus, et päris nii ei sobinud, nagu 
mina olin mõelnud. Raamatupidaja vaatas mind nagu idiooti ja küsis: 
\enquote{Kas sa ikka saad aru, kui palju ma pean numbreid siia päevas sisestama 
ja seda \emph{enter}'it, mille sa siia vahele toppisid, vajutama? Need arvud 
on neljakohalised. Ma sisestan neli numbrit ära ja 
need lähevad ise järgmisele väljale, mitte ma ei pea vajutama. Ma ei saa
vajutada \emph{tab}'i, mis on teises klaviatuuri otsas. Saad aru? Mul on ühes 
käes
paberid ja teise käega vajutan klaviatuuri. Kuidas ma sinna \emph{tab}'i juurde 
sinu 
meelest saan, kui mul on teine käsi kinni?} 

Nad olid väga innovatiivsed tegelikult selles mõttes, et nad olid 
sedasama andmetöötlust selleks ajaks juba aastat kuus-seitse kasutanud. See oli 
 meditsiinitehnika autobaas, Termak\index{Termak}, siiamaani elu ja tervise 
juures. 


\question{Kas nad olid juba nõukogude ajal arvutiasjandusega alustanud?}

Nad olid jah juba sügaval nõukogude ajal end täiesti ära automatiseerinud. 
Selleks 
ajaks, kui mina aastal 1992 sinna jõudsin, oli nende esimene IT-süsteem 
jõudnud moraalselt nii ära vananeda, et see tuli PCde peale ümber 
kirjutada. Neil oli siis juba \emph{legacy}, nad olid nii palju ajast 
ees.

\question{Kuidas Skriining jõudis selleni, et neil on programmeerijat 
vaja? Lihtsalt kasti sai ju ka edukalt müüa?}

Kalle\index[ppl]{Lotamõis, Kalle} hammustaski selle läbi, et kuna nad olid kogu 
aeg meditsiinitehnika ümber sebinud ja proovinud meditsiinisüsteemi arvuteid 
müüa, oli seal ka arendusvõimalusi. Nii saigi Skriiningust\index{Skriining} 
üheksakümnendate alguses arendusfirma. Arvutimüük käis ka, aga mina  
noore inimesena ei süüvinud sellesse, kust raha tuleb. Ilmselt päris palju tuli 
arendusest.


\question{Kas sa tehnikaülikoolis ka veel ringi hängisid?}

Ma hängisin seal pikalt, kuigi ma ei õppinud seal. Seal oli 
elu epitsenter, kuna seal töötasid kõik olulised inimesed: 
Mast\index[ppl]{Kaal, Madis} ülemisel korrusel, Tõnu\index[ppl]{Raimla, Tõnu}, 
Aare\index[ppl]{Tali, Aare} ja Tarmo\index[ppl]{Mamers, Tarmo} alumisel 
korrusel. Lisaks veel 
Martin Rinne\index[ppl]{Rinne, Martin}, Merle Alliksoo\index[ppl]{Alliksoo, 
Merle} ja kõik teised, kes hiljem MicroLinkis\index{MicroLink} lõpetasid. Tegu
oli sotsiaalse elu keskusega. 

\question{Mulle tundub see variant, et sa ei õpi, aga hängid, palju 
mõnusam, kui et õpid, aga ei hängi.}

Eks ma ise ikka soovitan teistele kool kohe 
ära lõpetada, sest pärast osutub see palju raskemaks. Mina ja mu sõbrad oleme 
hakanud 
neljakümnendates oma haridusega lõpuks tegelema. On 
tekkinud natuke rohkem vaba aega ja ka moraalne vajadus --
kuidas sa oled kõige väiksemate pagunitega mees ruumis \ldots

\question{Tol ajal ülikool kuigi palju praktiliselt 
kasulikku ei andnud. Tänapäeval on teistmoodi.}

Paljud ütlevad, et diplom ei olnud mitte tempel selle 
kohta, et tuled koolist välja targemana, vaid tõestus, et 
oled võimeline järjepidevalt, mitu aastat asjaga tegelema. See on pigem 
vastupidavuse ja hoolsuse proov kui koolitus.

\question{Räägi palun BBSidest. Kuidas sa selle \emph{node}'i ikkagi püsti 
said? 
Selleks tuli ju ennast kuskil registreerida?}

BBS oli varane arvutivõrk, mille mõte oli selles, et helistad 
kuhugi oma modemiga ja teises otsas on modem, kes vastab. Modemid saavad 
omavahel andmeühenduse ja siis saab teises arvutis, mille 
küljes teine modem on, ringi sobrada. Kusjuures tollal tõepoolest
sobrati, arvutiturvalisus oli pigem kokkuleppe 
küsimus. Üks suvaline BBSi omanik oleks võinud teise 
omaniku BBSi ilma mingi 
probleemita kaks korda tunnis neljaks tükiks lasta, aga seda lihtsalt ei 
tehtud. See oli nagu 
saarlase ukselukk: kui oled luku ukse ette paika pannud, siis kõik 
teavad, et sind ei ole kodus ja nii on. Ei ole vaja katsuda, kas uks 
on lahti või kinni, kedagi ei ole kodus. BBSidega turvalisusega oli sama lugu. 

BBSi teine ja palju kasulikum omadus oli see, et kui 
oli olemas modem ja arvuti, siis sai ennast FidoNeti 
\emph{node}'iks registreerida. BBS iseenesest ei eeldanud midagi sellist, vaja 
oli vaid
modemi ja vastava tarkvara olemasolu. Mingeid hämaraid teid pidi levisid 
telefoninumbrid, kuhu helistada ja end kohapeal ära registreerida.

FidoNet oli esimene üleilmne arvutivõrk selles 
mõttes, et modemid helistasid üksteisele automaatselt. See oli ka kaunikesti 
hästi toimiv elektronpostiteenus, mille üks eriline omadus 
oli veel see, et see liikus väljaspool KGB huviala. Eks küll 
kahtlustati, et seda kuulatakse pealt, ja aeg-ajalt mingid imelikud modemid 
üritasid sinu modemiga poole jutu pealt rääkida, aga üldiselt seda vist väga ei 
jälgitud. Ma vähemalt ei tea, et kellelgi oleks 
kaheksakümnendatel olnud modem-modemiga sidepidamisega probleeme, ei Eestis ega 
välismaaga. Mis on selles mõttes eriti huvitav, et kui kaugekõneliinid läksid 
nii palju lahti, et oli võimalik kuhugi automaatvalida, siis me ju helistasime 
igale poole välja.  
FidoNeti \emph{mail}, mis tuli Eestisse umbes aastal 1988 või 1989, oli esimene 
vaba ja 
demokraatlik sidekanal väljapoole.

Mina olin siis keskkoolis, esimese \emph{node}'i panin püsti umbes 1991. 
aastal. Ma olingi vist Aare \emph{point}. 
Omaenda \emph{point}'i numbrit ma enam ei mäleta, võibolla oli 
kaksteist-kakstest. \emph{Node}'i number oli
kolmkümmend viis. Eesti oli sel ajal ülemineku vabariik. 
Registreeritud postiaadress andis võimaluse foorumites 
kaasa rääkida. Eestis oli kümmekond gruppi, kus käis jutt erinevatel 
teemadel. Mõnes mõttes oli elu selline, nagu oleme täna 
harjunud, kuigi natuke teistsuguste tehniliste vahenditega. Post oli aeglasem 
ja 
saabus paar korda 
päevas, mitte reaalajas. Ei olnud nii, et kirjutan kirja ja see läheb kohe 
kõigile laiali. Samas täitis see kõik need ülesanded, millega täna tegeleme, 
ära. Nii et kaheksakümnendate lõpus, üheksakümnendate alguses oli see 
\enquote{ökosüsteem}, millega täna oleme harjunud, täiesti olemas ning 
väike käputäis inimesi Eestis omasid selle kasutamise privileegi. 

\question{Kas see väike käputäis olid pigem entusiastid, akadeemiline 
seltskond või kes?}

FidoNeti ökosüsteem koosnes sada 
protsenti entusiastidest. Akadeemilised inimesed läksid ärisse, panid püsti 
esimesed arvutifirmad ja üritasid raha teha. 

\question{Kas eksisteeris ka mõningane spetsialiseerumine, et siit saab 
tarkvara ja seal on huvitavaid jutte-raamatuid?}

BBSidel väike spetsialiseerumine oli, aga mitte eriti suur. Eks 
enam-vähem kõik proovisid endale kõhu alla korjata, mida vähegi said. 
See oli aeg, kus tekkisid esimesed suuremad kõvakettad. 
Lühikest aega valitses olukord, kus tarkvara 
oli vähem kui ruumi. Ruumi mõiste oli ka muidugi tollal huvitav. Kõige 
rohkem ruumi võtsid Sierra\sidenote{1979. aastal 
asutatud Sierra Entertainment (varem On-Line Systems ja Sierra On-Line) 
disainis paljud toonased hittmängud. Eriti populaarsed olid 
seiklusmängude sarjad \enquote{King's Quest}, \enquote{Space Quest} ja \enquote{Leisure 
Suit Larry.}\index{Larry (mängusari)}} mängud, mis olid flopiketaste peal. Neist suuremad, Space 
Questid\index{Space Quest} ja muud, tulid viie-kuue flopi 
kaupa. Mäletan, kuidas arutasime Eeroga\index[ppl]{Tohver, Eero}, et 
kui oleks võimalik panna kokku oma unelmate masin, siis kui suur kõvaketas sel 
peaks olema. Jõudsime järeldusele, et kui oleks umbes kaheksakümmend megabaiti, 
siis ilmselt jätkuks eluajaks, sinna saaks kõik mängud ja
tööasjad peale panna ning umbes pool jääks veel üle.

\question{Sierra oli omaette fenomen, seda mängiti palju. Kas keegi
seda müüs ka?}

Küsime laiemalt, kas Eestis üldse keegi tol ajal tarkvara müüs. 
Äritarkvara, nagu Novell, oli võimalik osta. Teoreetiliselt oli 
Windowsi või DESQview'd\index{DESQview}\sidenote{DESQview oli kaheksakümnendate 
lõpus ja üheksakümnendate algul populaarne tekstipõhine mitme{\-}tegumiline 
keskkond, mis toimis DOSi peal ja võimaldas korraga mitut programmi eri akendes 
käimas hoida.} kindlasti kuskilt võimalik osta. Aga peale Novelli serveri ja 
DataFlexi 
litsentside ei mäleta ma, et oleks üheksakümnendatel kellelgi 
legaalset tarkvara näinud. 

\question{Tuleme tagasi BBSinduse juurde. Kas selle sisu hulk, 
mida enda kõhu alla õnnestus kokku kuhjata, oli ka staatuse 
sümbol?}

Ma ei oska öelda, oskan ainult enda BBSide kohta rääkida. \mbox{Mina} 
korjasin kokku kõik, mida kätte sain, ja pakendasin ringi. See 
oli selline kultuuriküsimus, et tarkvara skaneeriti viiruste vastu 
kõige värskema skanneriga, mis parasjagu käeulatuses oli, ja see käis 
muidugi automaatselt. Siis lisasid arhiivi väikese faili, 
mis sisaldas sinu \emph{header}'it -- väikest 
failijuppi, kus oli graafiliselt (või tollal pseudo{\-}graafiliselt) sinu 
logo sisse punnitatud. Ja siis panid selle välja ja oma faililisti nupukese, 
millega tegu. 

See oli nagu \emph{basic housekeeping}. Kui sinu fail läks 
järgmisse BBSi, siis see viskas sinu logo välja ja pani enda oma 
asemele, \emph{tag}'iti ära nagu grafitiga, et see on 
minu käest tulnud asi. Vähemalt mul oli küll tunne, et välja läks 
kõik, mida olid ise endale mingil põhjusel hankinud. Mitte küll nii, et 
tõmbasid öösel HNSi\index{HNS} tühjaks ja 
panid enda lehekülje peale välja, küll aga mõned asjad, mille olid kätte 
saanud. Duplikaate ei olnud väga palju üllataval kombel.

\question{Tahtsingi küsida, et sedasi oleks pidanud ühel hetkel ju kõigil 
kõik olemas olema, aga seda siis ei tekkinud?}

Seda ei tekkinud. Kuna BBSid olid väga stabiilselt üleval, siis enda jaoks 
vajalikud asjad tõmmati
ära ja pandi omakorda enda juurde üles. Mõttetut \emph{leach}'imist ja püüet 
iga hinna eest oma failiandmebaas kõige suuremaks saada 
väga ei olnud. 

\question{Too mõni näide, mis laadi asjad sulle toona huvi pakkusid.}

Olin siis juba vihane \emph{nerd}, minu spetsialiteet oli 
programmeerimismaterjalid ja -vahendid, käsiraamatud ja
tööriistakesed. 
Kahjuks mul ei ole seda vana faililisti alles, sest kui ma Skriiningust ära 
läksin, lendas see vana SCSI-ketas, 
mille peal BBS jooksis, õhku. \emph{Backup}'i sellest ei olnud ja 
kogu FidoNeti \emph{node} koos failibaasiga läks hingusele.

Ma ise seda järgmisse kohta kaasa ei võtnud, sest läksin Skriiningust panka, 
kus 
olid ees sellised kõvad mehed nagu Mast\index[ppl]{Mast} ja 
Marx\index[ppl]{Marx|see{Kliimask, Margus}}\index[ppl]{Kliimask, Margus}, kes 
olid oma ökosüsteemi püsti pannud. Ühele BBSile seal rohkem ruumi ei olnud. 

\question{Mis panka sa läksid?}

Mina läksin sellesse panka, mille lõpupidu kohe 
kätte jõuab\sidenote{Intervjuu Andrusega toimus 2019. aasta novembri algul.} -- 
praegune Danske\index{Danske Pank}\index{Danske 
Pank|see{Forekspank}}, toona Forekspank\index{Forekspank|see{Eesti 
Forekspank}}. 

\question{Miks sa sinna läksid? 
Skriiningus said ju programmi kirjutada ja BBSi pidada.}

Nagu ma paljudesse kohtadesse olen läinud -- sellepärast et kutsuti. Ja 
kuna parasjagu jooksis Eestis teleseriaal Capital City, mis 
näitas panganduselu väga glamuurse \emph{highroller}'ina, siis mulle tundus, 
et mina tahan ka nii elada. Tuleb tunnistada, et üheksakümnendate panganduses  
ei pidanud väga pettuma, elu oli täitsa lill. Päris nii nagu 
teleseriaalis \enquote{Pank} elu meie majas küll ei käinud. 
Päris hulle pidusid sai peetud, aga et keegi oleks kokaiinise  
ninaga ringi käinud, seda mina ei tea. Meie kandis oli kokaiin täiesti 
tundmatu või ehk tehti seda salaja, mina küll
narkootikumidega pidusid ei näinud.

\question{Kas mäletan õigesti, et tollal tõmbasite panka 
püsiühenduse\sidenote{Enamik varasest internetiühendusest Eestis toimis kuhugi 
sisse helistades. See tähendas, et pidev side puudus ja side 
kvaliteet sõltus suuresti analoogtehnoloogial põhinevatest 
telefonikeskjaamadest. 
Püsiühenduseks kutsuti seda, kui asutusest jooksis füüsiline kaabel interneti 
külge 
ja kaabli olemasolu oli IT-inimeste unelmates kesksel kohal.} sisse?}

Püsiühenduse tõmbasime sisse väga konkreetsel päeval. 
Modemitega oli n-ö poolpüsiühendus juba pikemat aega olemas.  
Forekspank asus Rävala puiesteel, nagu 
juhtumisi ka KBFI\index{KBFI}\sidenote{Keemilise ja Bioloogilise 
Füüsika Instituut\index{Keemilise ja Bioloogilise Füüsika Instituut|see{KBFI}}
 (KBFI). 1979. aastal Endel Lippmaa\index[ppl]{Lippmaa, Endel} 
loodud teadusasutus, tuntud ka kui \enquote{Lippmaa instituut}. Just 
Lippmaade perekonna aktiivse ja laiahaardelise tegutsemise tõttu mängis 
instituut rolli paljudes toonastes olulistes protsessides (sh kohaliku 
interneti arengus).}. Baumaniga\index[ppl]{Bauman, Andres} 
oli läbi räägitud, kuidas internetti saab, ja meil oli suhteliselt 
rivitu ligipääs. Samas tundus ühel hetkel, et see võiks ikka päriselt 
permanentne olla. Võtsime Mastiga\index[ppl]{Mast} kaablirulli ja 
hakkasime üle Rävala puiestee katuste KBFI poole liikuma. Tähelepanuväärne oli, 
et see juhtus päeval, mil Eestit väisas esimest korda paavst.
\sidenote{Paavst Johannes Paulus II külastas Tallinna 10. septembril 1993.} 
Kõik katused olid snaipreid täis, kehtestati tohutu 
\emph{lockdown}, et keegi paavsti käigu pealt ära ei tapaks.  
Seletasime kõigile, et meil on vaja kaablit vedada ja paneme interneti 
püsiühendust. See oli maagiline valem, mis võimaldas ligipääsu 
kõikidele kesklinna katustele, ilma et keegi oleks midagi küsinud. Me küll 
otseselt snaiperitega samale katusele ei sattunud. Natukene tuli häkkida ka, 
et ühest koodlukust läbi minna, aga see ei olnud suur takistus. 

\question{Toona oli maailm järelikult teistsugune. Internet ei 
olnud veel kommertsiaalne, vaid pigem kogukondlik nähtus.}

Selle eest vististi keegi maksis ka kellelegi midagi, aga kui palju, 
seda jällegi ei mäleta. Eks see käis paljuski inimsuhete baasil. 
Kuna me tundsime Andres Baumani\index[ppl]{Bauman, Andres}, siis kuidas raha
seal tegelikult liikus, seda ma ei tea. Mast\index[ppl]{Mast} ajas seda asja. 
Millegipärast ma arvan, et maksime KBFI-le midagi. 
Tegelikult oli meil alates
üheksakümne viiendast aastast
Forekspangas\index{Forekspank} infotehnoloogiliselt selline elu nagu 
tänapäeval. 
Suhteliselt samal ajal tuli Mosaici\index{Mosaic}\sidenote{NCSA Mosaic oli üks 
esimesi internetibrausereid ja mängis WWW populariseerimisel olulist rolli. 
Sama meeskond lõi hiljem Netscape\index{Netscape} Navigatori, mis oli  
Firefoxi eelkäija.} brauser, hakkas veeb arenema ja 
tekkisid meile kõigile e-posti aadressid (need olid 
küll juba pisut varem KBFI kaudu korraks olnud, aga siis tekkisid need 
meie oma foreks.ee domeeni külge). Kogu see ökosüsteem, miinus Facebook, oli 
meil siis juba olemas. 

Tollal me ka täitsa tõsimeeli arutasime, 
et KBFI ühendus on ikkagi nii aeglane, et ehk peaks kogu 
veebi kohalikku serverisse kopeerima. Ja 
kuna see mahuks tõenäoliselt ühele DVD-le ära, siis ehk peaks tegema 
äri ja hakkama müüma internetiga DVDd. 

\question{Ka teistest intervjuudest käib läbi, et toonane maailm põhines 
suuresti 
inimsuhetel. Ometigi ei hakka inimesed arvutitega tegelema, kuna neile 
meeldib tegeleda inimestega. Samas tunduvad Eesti arvutiinimesed küllaltki 
suhtealtid ja -osavad. Miks see nii on?}

Kui inimesel on arvutihuvi, siis on ta
terve keskkooli ja pool ülikooliaega olnud sotsiopaat ning tal ei ole eriti 
olnud kellegagi millestki rääkida. Ja ühel hetkel leiab ta üles omasugused, 
samasuguste huvidega. Puhas \emph{nerd}'i ja nohiku käitumine, eks ole! 
Kui panna nohikud kõik ühte tuppa kinni, siis nad leiavad 
üksteist ja kõigil on järsku lõbus, sest kõik lõpuks ometi naeravad samade 
naljade üle. Pidudega on sama lugu. Kõige karmimad peod, kus ma 
olen osalenud, on ikkagi olnud inimestel, kelle igapäevatöö on kaunikesti 
\emph{boring}. Ma ei taha anda hinnangut teatud inimgruppidele, aga kui näiteks
 raamatupidajad ja andmesisestajad käima lähevad, siis see on 
ikka täiesti teine tase. Keskmised lõbusad inimesed on lõbusad 
kogu aeg. Aga kui nohkarid lõpuks lõbusaks muutuvad, siis juhtub asju.

Nii et see ökosüsteem toimis tänu sellele, et inimestel oli hea meel üksteist 
leida. 
Algul oli neid alla saja, 
võibolla isegi alla viie{\-}kümne inimese. Tegu oli uue 
laine arvutitegelastega, kelle seast suur osa meie tänasest 
\emph{start-up}-ettevõtlusest 
ongi välja kasvanud. Tänu tihedale suhtlusele hakkasid 
toimuma ka legendaarsed BBSummeri\index{BBSummer}-nimelised üritused. 

\question{Räägi lõpetuseks, mida sa praegu teed.}

See on võibolla masendav tõdemus, aga elu pole mind sellelt kursilt
kaugemale ega kuhugi mujale viinud. Laias laastus 
tegelen täna täpselt sama asjaga, millega kakskümmend viis aastat 
tagasi. Olen pendeldanud elektroonika ja tarkvara vahel, 
olnud mitme firma CTO, asutanud firmasid ja neid kihva keeranud, töötanud 
teiste juures ja endale. Ja kui keegi küsib, millega ma tegelen, 
siis tavaliselt ütlen, et annan masinatele hinge. 

\question{See on ilus ütlemine ja läheb kokku küsimusega, mis jäi enne 
küsimata. Tavaliselt inimesed tegelevad kas riist- või tarkvaraga, aga sinul 
tundub olevat üks jalg ühes ja teine teises?}

Mõeldes oma elu peale, siis ma muidugi tahaksin, et tarkvara oleks mu tõmmanud 
endasse. See on mõnes mõttes nii palju lihtsam ala. \mbox{Vigu} on palju 
lihtsam parandada ja katkiseid asju ei tule peaaegu üldse ära visata. 
Kettaruum ei maksa täna eriti palju erinevalt elektroonika valmistamisest ja
utiliseerimisest.

Mul on kuidagi juhtunud niimoodi, et kui panen 
tule vilkuma ja näen, kuidas minu tehtu manifesteerub päris asjades, 
siis mul läheb tuju paremaks. Mul tuleb elektroonika disain välja ka. Kuna ma 
olen ikkagi ka
programmeerija, siis olen sattunud sinna omamoodi side{\-}meheks. Ma suudan tõlkida 
riistvara tarkvara jaoks ja vastupidi. Selle konkreetne töönimetus on 
\emph{embedded engineering}. Vaadates, mis meil täna koolidest 
saabub, siis on see täiesti väljasurev kunst. Neid tegelasi, kes suudavad nii 
riistvara valmistada kui ka sellele tarkvara peale kirjutada, 
nimetatakse mehhatroonikuteks või kelleks iganes, aga fakt on see, et nende 
juurdekasv on järsult pidurdunud ja varem või hiljem hakkab see 
probleemiks muutuma. Tõsi küll, ka töömeetodid muutuvad. Me kasutame täna
töövahendeid, mis annavad näiteks tarkvaratiimile parema 
ettekujutuse riistvarast kui vanasti. Kirjeldused ja 
\emph{markup language}'id, millega seda tehakse, on paremad. Masinale hinge 
andmine tähendab seda, et kui sa näiteks lülitad oma pesumasina sisse, siis on 
oluline, mida see oskab või ei oska sinu 
heaks teha. Hea kasutajakogemus tuleb sellest, kui hästi raua ja tarkvara 
kooslus 
on välja mõeldud. 

\question{Sa ütlesid enne, et sa oled ka CTOna toimetanud. Järelikult tuleb
kolmas element juurde -- sa pead suutma selle kõik ka äriks tõlkida.}

CTO ametit on kaht sorti. Tavaliselt väikestes firmades tähendab 
CTO olek seda, et koosolekule on vaja kedagi kaasa võtta, ja kuidas sa 
ütled, et ta on mul programmeerija, eks. Sa pead talle andma 
visiitkaardi, millega ta näeb presentaabel välja. Väikefirma CTO 
teeb kõike, millel on tehnika maitse 
küljes. Suurema firma CTO tähendab, et ta ongi CTO. Tänases 
\emph{start-up}-maailmas on \emph{customer fit} ja \emph{market fit} kõva 
teema. 
Vanasti sellega väga ei tegeletud, aga nüüd, kus on tohutu kuhi 
investorite raha põlema pandud, ilma et sellest oleks isegi sooja saadud, on 
hakatud rääkima sellest, et toodetut peaks kellelegi päriselt ka
tarvis olema. See paistab olevat uus asi, viimase paari aasta 
paradigma. Kaks-kolm aastat tagasi hakkas Silicon Valleys 
pihta see kultuur, et laste kätte ei taheta raha enam hästi anda. Ehk 
nende kaheksateistaastaste imeettevõtjate aeg, kes suudavad väga suure kuhja 
raha 
korraga põlema panna, nii et sooja ei saa, on läbi saanud. Nüüd on selgunud 
innovatiivne lähenemine, et toodet peab 
kellelegi tarvis olema. See tähendab, et projektidele on erakordselt raske raha 
saada, sest kõik 
on järsku pirtsakas muutunud ja nõudnud, kust raha tagasi tuleb. 

\question{See läheb ju kokku sinu kunagise ettevõtte uksest sisse minekuga: 
seal pidid ka kohe kasulik olema ja ei tohtinud asju tuksi 
keerata.}

Kasumlikkus on tegelikult õudselt valus teema. Riistvaraga on  
asi selles mõttes selgem, et riistvara ei skaleeru, kui keegi seda ei osta. Sa 
ei 
saa valmistada sedasama \emph{recorder}'it, millega me siin praegu salvestame, 
miljon tükki, kui keegi ei osta. Sa lähed 
pankrotti. Tarkvara tiražeerimine ei maksa aga midagi. Ja täpselt samamoodi 
võib  
juhtuda, et tarkvara, millest mitte kellelegi mitte pennigi ei teki, on 
tegelikult väga kasulik. Seega kasulikkus ja ärimudel ei tähenda veel mitte 
midagi. Dotcomi- ja igasugu tarkusemullidega kipub tavaliselt juhtuma, et väga 
raske on tõmmata piiri selle vahel, kus asi ei teeni 
raha sellepärast, et väga head mõtet ei ole veel õpitud rahaks  
tegema, ja nende asjade vahel, mis ongi täiesti mõttetud. 
Seetõttu on väga palju tegelasi, kes suudavad maha müüa täiesti kasutu idee, 
öeldes, et tegu ongi monetariseerimiseelse faasiga ja see ei peagi midagi 
tootma. 
Unustades ära, et tegu on ühtlasi täieliku kräpiga. 
Viimasel ajal on tekkinud paar niisugust suuremat skandaali, näiteks
õnnetu Theranose \emph{case}, kus suudetakse endale nii veenvalt 
valetada. Terve ökosüsteem on üles ehitatud väga kasulikest 
asjadest, mille ainus viga on see, et fundamentaalne eeldus, millele süsteem 
rajati, oli täiesti vale. 

\question{Nii et selle kahekümne viie aastaga ei ole maailm väga muutunud,
aga toimib siiski natuke teisti?}

Üks asi on oluliselt erinev. Tollal valmistati tarkvara kahel põhjusel. 
Esiteks oli seda tarvis, mis tähendas tugevat kliendipoolset 
tõmmet. Teiseks taheti, et midagi sellist eksisteeriks 
maailmas, mis tähendab, et võeti lihtsalt kätte ja kirjutati tarkvara kas enda 
või teiste rõõmuks ning lasti maailma. Hästi palju väikesi ja 
kasulikke 
utiliite oli ju tegelikult kirjutatud kellelgi 
enda jaoks, siis pakendatud ja laiali saadetud. Eestis seda kontseptsiooni 
polnud, et teha tarkvaraga 
raha: kirjutada mõni vidin ja küsida selle eest tasu. \emph{Corporate} maailmas 
tollal 
küll juba osteti-müüdi igasuguseid raamatupidamissüsteeme väga 
edukalt ja see kõik töötas. Mujal maailmas tegeleti utiliitide 
pealt raha teenimisega ka väikest viisi. Eestis üldse mitte. Tänapäeval on 
tarkvara tootmine läinud niimoodi, et kellelgi tuleb mõni väga
hea idee ja ta tahab sellest teha raha tootmise masina. Asi on vastupidine: 
mitte 
vajadus-, vaid unistuspõhine. Nagu me 
aeg-ajalt Ivar Zaransiga\index[ppl]{Zarans, Ivar} naerame, et kui vanasti 
otsiti probleemidele lahendust, siis tänapäeva 
maailmas otsitakse probleeme neid vajavatele lahendustele. See on viimase 
kahekümne viie aasta jooksul kõige suurem paradigma muutus.
\chapter{Ahti Heinla}
\chapter{Madis Kaal}
\chapter{Andres Kütt}
%!TEX TS-program = arara
% arara: myindex

\textbf{\enquote{Kuidas sa arvutite juurde jõudsid?}}

Sündisin 1975. aastal Võrus\index{Võru}. Millestki midagi aru saama hakkasin mälu järgi kaheksakümnendate teisel poolel. See oli mitmes mõttes üsna kole aeg. Noorukile kõige arusaadavam neist koledustest oli lihtlabane praktiline puudus. Päris nälga ei olnud aga midagi vähegi leivast ja piimast edevamat saada ei olnud. Kui linnakeses levis kuuldus, et olla toodud kast jäätist, oli poes veerand tunniga saba ning poole tunni pärast kõik otsas. Muu hulgas oli kaubandusvõrgus saada kahte tüüpi meeste talvejopesid. Mitte kahtekümmet ja mitte kahtesadat vaid kahte. Ühed olid hallid ja neid said lihtsurelikud osta\sidenote{Huvitaval kombel oli tolle jope põuetasku 5.25" lai, sinna mahtus üks flopi täpselt sisse} ja teised olid punase suure a-tähega ja neid said osta ainult inimesed, kes teadsid kedagi, kes teadis kedagi. Ajad olid sellised. Kõige hämmastamaval moel käisid ka seda viletsust inimesed Pihkvast bussidega uudistamas ja viimastki kaupa ära ostmas. 

Aga kogu selle halluse keskel suutis Nõukogude Liit meie Võru Kreutzwaldi Gümnaasiumile\index{Koolid!Võru Kreutzwaldi Gümnaasium} tarnida arvutiklassitäie arvuteid Agat\index{Arvutid!Agat}\sidenote{Agat oli Nõukogudemaal valmistatud arvuti, mis oli küll Apple II\index{Arvutid!Apple II}'st inspireeritud, kuid siiski mitte täpne kloon}. Kust nad tulid ja kes seda asja ajas, ei tea. Küll aga mäletan, et nende saabumine oli pikalt oodatud ja edasi lükatud. Miks ja mida oodatud sai, ei oska öelda. Tean ainult seda, et kui klass tekkis, läksin ma sinna sisse ja enam välja ei tulnud. 

Ega tolle purgiga palju teha ei olnud. Olid mõned mängud ja programmeerimiseks Basic. Tolles meid programmeerima õpetatigi. Esimese hooga ei õpetatud seejuures mitte kõiki käske, näiteks for-tsükkel oli tükk aega saladus. Kui aga nohikud aru said, et nende eest tarkust varjatakse, kadus igasugune respekt ja läks lahti suuremaks isepusimiseks. Kõik muutus, kui kooli saabus noor, minu meelest värskelt ülikoolist tulnud, arvutiõpetaja Aivar Halapuu\index[ppl]{Halapuu, Aivar}. Temaga tekkis kohe mingisugune pool-kamraadlik side, mis siiski alati suurt kogust meiepoolest lugupidamist sisaldas. Tolleks ajaks oli meil tekkinud väiksem seltskond poisse, kes seal klassis toimetas ja kes kohe end \emph{in corpore} Aivarile sappa haakis. Aivar viitsis meiega tegeleda ja, kuigi ta meile suurt midagi arvutite mõttes ei õpetanud, sai tema käest midagi, mida vist kultuuriks nimetatakse. Meiega üritati bridži mängida, räägiti mänguteooriast ja nii. 

Kuna me seal klassis sisuliselt elasime, siis usaldati meie kätte üsna pea ka arvutiklassi võti. Aga \emph{kooli} võtit meie kätte keegi ei andnud. Seetõttu oli oluline hoida järjepidevust: keegi oli alati klassis olemas ja hõikamise või kivikese viske peale lasi tulija sisse. Mingitel tingimustel oli meie käes siiski ka välisukse võti aga tihti roniti ka aknast. 

Ühel hetkel avanesid kraanid ja saabus humantiaarabi. Võrus oli vist seoses rahvamuusikaga igasugu põnevaid suhteid välismaa asutustega, kes hakkasid meie suunas igasugu põnevat kola saatma. Saabus klassitäis mingeid rootsikeelsete paberite ja tarkvaraga masinaid, millega me mitte midagi teha ei osanud. Mis neist sai, ei tea. Aga tuli ka mingi iidne aparaat, mille külge käis neli-viis terminali ja kaks kokku külmkapisuurust kettaseadet. Seadmete sisse käisid hiigelsuured plastkarbis kettad. Tegu oli industriaalseadmega: kui tuurid sisse võttis, siis oli alla tänavale kuulda, et \enquote{arvuti töötab}. Tolle masina peal midagi tarka teha ei osanud keegi, tarkvara polnud. Sai mingeid mänge mängitud ja see oli ka kõik. Mäletan siiski, et seal puutusin esimest korda kokku Zorki\index{Mängud!Zork} nimelise mänguga\sidenote{Zork on üks varasemaid teksipõhiseid arvutimänge. Mängija sisestas teksti ja talle ka vastati tekstiga vastavalt sellele, mis mängus parasjagu juhtus. Kuna mängu alguses sattuti lagendikule valge maja ette, oli meie puhul ilmselt tegemist Zork I-ga}.

Lõpuks tulid meile Jukud\index{Arvutid!Juku} ja üheksakümnendatel lõpuks ka pc-d. Jukusid oodati väga, sest Agat oli päris jube aparaat\sidenote{Ma ei ole kunagi hiljem kohanud arvutit, mis suudab flopiketta füüsiliselt ära rikkuda}. Ja Jukud olidki väga ägedad, ainsaks nõrgaks kohaks oli minu mälu järgi klaviatuur. Ainus, mis palju ei muutunud, oli tarkvara. Võru ei ole Tartu ega Tallinn. Meie seltskond ei suhelnud õieti kellegagi, ei uut tarkvara ega teadmist ei tulud eriti kuskilt peale. Ajakirjast \enquote{Arvutustehnika \& Andmetöötlus} võis küll lugeda Unicode võludest aga programmeerida tuli ikkagi kas assembleris või basicus. Seejuures sain alles hiljem teada, et eksisteeris ka asi nimega makro-assembler. Tavalises pidi JMP käsule argumendiks andma suhtelise aadressi (mis muidugi kohe valeks osutus, kui kuskile mingi rea vahele panid)\sidenote{See oli probleemi minusugustele surelikele. Inimesed nagu klassivend Vallo Trell\index[ppl]{Trell, Vallo} suutsid ka otse BIOSi prompti peal mällu baite kirjutades masinkoodis programmeerida} aga tolles uuemas sai silte kasutada. Mingitel üritustel sai Tallinnas käidud (mäletan Pedas\index{Pedagoogikaülikool} asunud MSXide\index{Arvutid!Yamaha MSX} klassi) ja sealt ka mingit tarkvara kaasa toodud aga üldiselt olime üsna omaette. Isegi flopisid käisime ostmas Tallinnas, seal oli teada üks komisjonipood, kust selliseid sai. Tavaliselt kasutati ära mõnda käiku teatrisse, reeglina jäi kuhugi paar tundi linnas kolamise aega. 

Olin ka üks õnnelikest, kellele lõpuks arvuti suveks koju usaldati. Esmalt Agat, siis Juku. Kuna ekraanid olid mõlemal nigelad, veetsin kaks või kolm suve ette tõmmatud kardinate taga arvutiga toimetades. Juku peal mäletan kahte suuremat projekti. Esimene oli Norton Commanderi moodi failihaldur ja teine fondiredaktor. Jukul sai tähekujusid suhteliselt lihtsasti ümber teha, mälus olid vist kaheksabaidised bitimaatriksid ning teksti kuvamine käis kiiremini, kui muu graafika. Mõlemat kirjutasin assembleris ja kumbki päris valmis ei saanudki, sest teatud mahust alates muutus kood hoomamatuks. Sel ajal omandasin ka pärast palju vaeva põhjustanud kombe \enquote{tunde järgi} koodi kirjutada. Teed muutuse, kompileerid, proovid, muudad pikalt mõtlemata uuesti. Kood oli nii kole, et selle iga kord uuesti läbi mõtlemine oli liiga keeruline ja mingid \emph{off-by-one} vead olid sagedased, reeglina sai mingi konstandi ühe võrra nihutamise peale koodi käima. Sellest rumalast kombest pole ma siiani lõpuni vabanenud. 

Aga Juku peal sai ka andmebaase teha, täitsa oli olemas dBASE\index{dBASE}. Selle abil õnnestus maik suhu saada kellelegi arvuti abil kasulik olemisest. Koolivend Aini dieedi-teemalise uurimistöö jaoks tegin andmestiku ja kirjutasin ka programmi kassetiümbriste trükkimiseks. Tollal käibis muusika kassettidel, mida ohtralt kopeeriti\sidenote{Eksisteeris ka tänapäeval mõeldamatu täiesti põrandapealne muusikakopeerimise asutus, selline oli ka Tartus. Läksid kohale, valisid kataloogist albumi välja, jätsid tühja kasseti maha ja mõni päev hiljem sai sobiva summa vastu muusikaga kasseti tagasi}. Seetõttu kirjutati lugude nimesid käsitsi ning see oli tüütu. Minu tarkvara võimaldas aga kiiresti eri plaatide jaoks kassetiümbrised trükkida. Selle teenuse eest sai vist ühelt klassivennalt isegi raha küsitud.

Linna peal eri kohtades sai ka PCdega tutvust teha. Mööblivabrikus oli kellelgi tutvusi, seal toimus isegi mõned korrad mingisugune õpe. Istusime ilmselt raamatupidamise masinate taga ja meile näidati, kuidas FoxPros\index{FoxPro} vorme joonistada ja andmeid hoida. 

Keskkoolis õnnestus käia väga murdelistel aastatel 1990-1993. Võrus möllas punkar Saare Ain\index[ppl]{Saar, Ain}\sidenote{Kodanikunimega Ain Saar, asutas Vaba Sõltumatu Noortekolonni number 1 ja tegi muid tükke}, Võru surnuaial taastati Vabadussõja mälestussammas ja miilits ajas koertega üritusi laiali. Ühe sellise intsidendi järel oli koolis näha kummalistes ülikondades seltsimehi, kes pingsalt vanemate klasside õpilaste nägusid jälgisid ilmses lootuses tuttavaid kohata. Aga tekkis ka äri. Leidsime sõpradega mingist ajalehest kuulutuse, milles otsiti meie jaoks ulmeliste palkadega (mahus umbes meie vanemate aasta palk paarinädalase projekti eest) meelitades C programmeerijaid. Kandideerimise tähtaeg oli suurusjärgus kaks nädalat, see tundus täiesti mõistlik aeg, millega omale C selgeks teha. Kuskilt sai hangitud klassikaline Brian Kernighan ja Dennis Ritchie \enquote{The C Programming Language}\sidenote{Paraku läks mu koopia hiljem kaotsi. Kust ma selle raamatu sain, ei oska öelda, aga kindlasti ei tulnud see kuskilt välismaalt. Ilmselt oli tegu mingi kvaliteetse piraat-väljaandega, millel isegi kaanekujundus õige oli. Hiljem järele uurides selgub näiteks, et raamatus puudus igasugune märge trükkimise koha ja väljaandja kohta}. Seda sai siis kampas tudeeritud ja tundus sihuke loogiline. Kuna puudus juurdepääs C kompilaatorile, siis päris koodi kirjutada ei saanud. See meid ei heidutanud ja mingid kirjad me isegi välja saatsime. Vastust muidugi ei tulnud. Hiljem olen mõelnud, kas võis tegu olla tollesama legendaarse lehekuulutusega, mis viis kokku Bluemooni\index{Bluemoon} poisid ja Stefan Obergi\index[ppl]{Oberg, Stefan} aga ajastus vist ei klapi. 

Siiski saavad kõik head asjad otsa, nii ka keskkool. Tol hetkel sai mingites piirides omale lõpueksamit valida ning oleks olnud kummaline, kui meie seltskond ei oleks valinud arvutieksamit. Tolleks hetkeks olime Aivarist kaugel ees, sest meil sõna tõsises mõttes ei olnud mitte midagi muud teha, kui arvutit torkida. Laulsin kül ka kooris\sidenote{Kooriga välisreisile (kas Saksamaale või Soome) minek oli ka põhjuseks, miks ma ei ole kunagi vabariiklikul informaatikaolümpiaadil käinud. Tol ühel kevadel, kui sinna õnnestus välja murda, oli ka reis plaanis. Otsustavaks sai, et ma ei tahtnud koori hätta jätta. Mitte, et ma seal mingit kandvat rolli oleksin mänginud, aga siiski.} aga põhimõtteliselt kogu muu vaba aeg oli arvutite päralt. Isegi õppetöö ei seganud, sest põhikoolis tegin endale kõva põhja alla. Aga see kõik ei vähendanud sugugi eksami pidulikkust. Sisenesime ruumi, võtsime pileti, lahendasime, vastasime komisjonile, kõik oli nii nagu peab. Aivar oleks võinud meile kõigile viied välja kirjutada aga ometi viidi eksam täie tõsidusega läbi. 

Kuna õnnestus kool nibin-nabin kullaga lõpetada, sain Tartu Ülikooli Matemaatikateaduskonda\index{Tartu Ülikool!Matemaatikateaduskond} eksamiteta sisse. Sinna minek tundus loogiline, sest Tallinn oli kaugel ja tundmata ning arvuti-värki tahtsin kindlasti õppida. Sõjaväega probleeme ei olnud. Esiteks olid segased ajad ning Eesti riik polnud veel päriselt välja mõelnud, mis moodi oleks mõistlik väeteenistust korraldada. Teiseks oli mu silmanägemine nii paha, et mulle öeldi Kaitseväe tohtrite poolt: \enquote{Kui venelane peale tuleb, siis paneme su laipu vedama, seniks mine koju}. Nii veetsingi suve Võru ja Tartu vahel hääletades, käisin näiteks ka Steni\index[ppl]{Tamkivi, Sten} juures\sidenote{Tema ema ja minu tädi olid juba ülikooli aegsed sõbrannad, Steni vanaisa elas Võrus ja nii me juba üsna õrnas eas tuttavaks saimegi.} Primexis\index{Primex Data} külas. 

Sügisest algas ülikool ja jäin pidevalt Tartusse. Kuna jäin paberite ajamisega töllerdama, siis teiste matemaatikutega Tiigi ühikasse kohta saada ei õnnestunud. Ühe või kaks talve olin sugulase juures üüriliseks, ühe talve elasime kambaga Tartu Kurtide Ühingus (!)\index{Tartu Kurtide Ühing}, kes tudengitele tuba välja üüris. Küll aga sai külas käidud klassivendadel, kes läksid enamuses Tartusse majandust õppima, ja kelle ühikaks olid Narva Maantee Tornid. Toona Tartu ühikates toimunu on omaette lugu, millesse süvenemine viiks meid teemast kõrvale.

Ülikoolis sain kohe piltlikult öeldes ägeda laksu silmade vahele. Esmalt selgus, et, erinevalt keskkoolist, on ülikoolis vaja päriselt õppida. Aga oskus selleks oli juba kadunud ja tuli uuesti tekitada. Teiseks selgus, et puhtast ropust tööst enam heade hinnete saamiseks ei piisanud, vaja oli ka annet. Aga seda on mul kogu aeg nappinud. Kolmandaks selgus, et teistel seda annet jagus ning see tegi egole haiget. Inimesed nagu Meelis Roos\index[ppl]{Roos, Meelis} ja Rene Prillop\index[ppl]{Prillop, Rene} seilasid igasugu matemaatikast läbi ilma nähtava pingutuseta ja kirjutasid koodi, nagu jumalad. Margus Sutt\index[ppl]{Sutt, Margus} teadis arvutitest nähtavasti kõike ja oli tolleks ajaks juba tegelenud täiesti müstilisena tunduvate asjadega. Asko Seeba\index[ppl]{Seeba, Asko}, oli kõike seda \emph{ja} oli seejuures veel setskondlik ning tüdrukute hulgas popp. Ei jäänud midagi üle, tuli tasapisi inimeseks õppima hakata. 

Igatahes oli vaja tööle minna, sest ema käest ei saanud ju jääda raha küsima. Proovisin saada baarmaniks, vast avatud Atlantise ööklubi valgustajaks ja isegi arvutigraafikuks aga asjata. Lõpuks sattusin kuidagi ettevõttesse Korel IN\index{Korel IN} programmeerijaks, mu esimene tööpäev oli detsembri alguses aastal 1993. Mind ja kamraad Veljot\index[ppl]{Hagu, Veljo} võeti palgale eesmärgiga luua firmale arvetega majandamiseks vajalik tarkvara. Keeleks oli Visual Basic\index{Keeled!Visual Basic} ja ei läinud palju aega, kui meil mingid asjad juba töötasid. \enquote{Programmeerija} kõlab märkimisväärselt galmuursemalt, kui asi tegelikult välja nägi. Tegime kõike alates kauba tassimisest (kontor asus viiendal või kuuendal korrusel, kahekümnetolline CRT-monitor on päris raske), kuni isegi mõningase müügitööni. Toonasele arvutiärile iseloomulikult ei teadnud eales, mis seisus su töökoht kontorisse jõudes oli. Mõnikord oli ära müüdud mälu, mõnikord võrgu kaart või monitor. Mäletan end kirjutamas koodi üheksatollise must-valge kassamonitori ees taburetil istudes. 

Tartu ei ole suur linn ja nii puutusime Korelis töötades kokku suure osaga toonasest arvutiseltskonnast. Tarmo Tali\index[ppl]{Tali, Tarmo} oli meil müügimeheks ja aeg-ajalt käis tal külas Asko Oja\index[ppl]{Oja, Asko}, keda hellitavalt \enquote{Tarmo blondiiniks} kutsuti. Vahel astus Sorose sajalisi tuulutades läbi Marek Tiits\index[ppl]{Tiits, Marek}, kellele mingi ime läbi õnnestus isegi üks Suni tööjaam müüa. Kui ütlen, et puutusime, siis tegelikult mina ei puutunud eriti kellegagi kokku, olin toona ja olen siiani küllalt asotsiaalne. Igasugu toredat rahvast käis poest läbi, enamasti sai lihtsalt silmad punnis peas spetsialistide jutte kuulatud ilma nende nimesidki teadmata. 

Kuidagi tekkis Koreli lähedale aktiivne kodanik nimega Tanel Urbanik\index[ppl]{Urbanik, Tanel}. Ta pandi meile alguses ülemuseks aga üsna varsti vedas ta meid Korelist minema asutades uue ettevõtmise nimega HClub. Nimi tuli sellest, et meie tuba Koreli päris-ärimeeste hulgas veidi põlastavalt häkkeriklubiks kutsuti. Tanel tahtis tarkvaraäri teha, küllap seetõttu tal Koreliga teed lahku läksidki. Meie peamiseks leivanumbriks sai kassasüsteemide ehitamine, peamisteks klientideks erinevad tanklad, näiteks Favora omad. Kirjutasin muu hulgas ka näiteks Ravimiametile\index{Ravimiamet} nende ühe esimestest andmebaasidest. Selguse mõttes olgu üle korratud, et toona mingist klient-server arhitektuurist juttu ei olnud. Kõik lahendused hoidsid andmeid võrguketta peal Microsoft Accessi\index{Microsoft Access} andmebaasis ja selle poole pöördumine käis kliendi juurde paigaldatud \enquote{paksu} kliendi abil. 

Tollele ajale tagasi mõeldes tundub hämmastav, et meie tarkvara töötas. Meid olid ainult mõned inimesed, mingist testimisest või versioneerimisest ei teadnud keegi midagi. Mäletan, et korra pidin Tartust Võrru tanklasse tagasi sõitma, sest värsket versiooni flopi peal kohale viies olin midagi valesti teinud. Vähemalt minu kood püsis kindlasti koos peamiselt tati ja teibiga. Veljo oli märkimisväärselt pädevam programmeerija aga tarkvaratehnikast polnud ilmselt palju aimu temalgi. 

See mind lõpuks HClubist (päris suure tüliga, tuleb tunnistada) ära viiski. Ma ei jaksanud enam kõige selle kokku punutud ja päris kliente teenindava tarkvara eest vastutada. Põlesin läbi ja kõndisin Tanelit pipramaale saates ära. Toonaseid seiku nägin veel aastaid unes ja ärkasin keset ööd. Oma rolli mängis ilmselt ka see, et just tol ajal, kui õigesti mäletan, läksid põhja mu unistused saada arvutialane haridus. Nimelt oli toona matemaatikateaduskonnas esimesed paar aastat kõigile ühised, seejärel tuli valida kas arvutiteaduse, statistika või rakendusmatemaatika vahel. Valik käis seejuures õpitulemuste alusel. Minu õpitulemused võimaldasid napilt ennast arvutiteadlaseks pidada ja nii esitasin vajaliku avalduse ning asusin järgmisest semestrist hoogsalt arvutiteaduse aineid kuulama. Neid loeti enamasti Liivi tänava õppehoones. Dekanaat oma teadetetahvliga asus aga Vanemuise õppehoones. Ja kuna ma ka oma ut.ee meiliaadressi ei jälginud, läks minust täiesti mööda dekanaadi mõte, et peaks ikka veel mingeid pabereid küsima. Kui ma ükskord jaole sain, olid arvutiteaduse õppekohad täis ja minust sai statistikaüliõpilane. See oli päris valus hoop. Kuigi arvutiteaduse ained olid minu jaoks rasked (mäletan end kolm korda kompileerimismeetodite eksamit tegemas), oli mul siiski mingi lootus sealtkaudu kuidagi paremaks programmeerijaks saada ning kamraadidele järgi jõuda. Toonane ülikooliharidus oli tänasest väga erinev ja asus praktilisest elust valgusaastate kaugusel, aga lootus jäi. Statistikast huvitusin ma vähe ja ei näinud mingit võimalust sellest oma töises elus kasu saada (masinõppe-revolutsioonini jäi veel paarkümmend aastat). Seetõttu tegin edaspidi minimaalse, et kuidagi koolist läbi saada ja keskendusin tööl käimisele. 

Kogu BBSindus läks minust üsna suure kaarega mööda. Võrus ei olnud kohalikku BBSi ja kaugekõne ei tulnud kõne allagi. Sten Primexis küll vist näitas kuhugi helistamist, aga tuhka ma aru sain. Korelis oli küll väline modem ja aegajalt sai kuhugi sisse helistatud, aga seda väga sporaadiliselt. Peamiseks selleteemalise info allikaks oli kursavend Mati Muts\index[ppl]{Muts, Mati}, peamiselt sai käidud Lucifer BBSis\index{BBS!Lucifer BBS}. Küll aga oli ülikoolil tol ajal juba täiesti korralik internetiühendus ja palju aega kulus Vanemuise õppehoones\index{Tartu Ülikool!Vanemusise tänava õppehoone} terminali taga FTPd pidi ringi kolades. Mäletan, et tõmbasin kas ftp.funet.fi või ftp.sunet.se serverist tükk aega mingi Metallica albumi kaanepilti ja olin väga rahul, kui see ka päriselt kohale jõudis. 

Selgelt mäletan ka seda, kuidas ma kohtusin HTMLiga. See oli Liivi tänaval\index{Tartu Ülikool!Liivi õppehoone}, seal oli mingi Suni klass\sidenote{Need pidid olema Sunid, sest mäletan ruudulist hiirepatja. Mis muidugi ei olnud mingi padi. Kuna Sun kasutas toona levinud palli asemel hiire liikumise lugemiseks eesrindlikke optilisi sensoreid aga tehnoloogia polnud veel kuigi arenenud, pidi sensoritele teadaolevate vahedega ruudustikku näitama. Seetõttu töötas hiir ainult spetsiaalse metallist mati peal, kuhu oli joonistatud peen ruudustik} ning seal sukeldusin ma veebilehe tegemise võrratusse maailma. Pärast pikka pusimist suutsin omale tekitada kodulehe, kus asju õiges kohas hoidis tabel! Ega sinna kodulehele midagi kirjutada ei olnud aga tabeli ridade ja lahtrite saladuste lahti pusimine oli põnev.

Ja kõik see osutus kasulikuks, sest HClubi järel võttis mu oma juurde tehnikuks klassivend Meelis Mäeots\index[ppl]{Mäeots, Meelis}. Ta tegeles tol ajal igasugu imelike asjadega, kuid muu hulgas asutas ka internetifirma. See koosnes alguses peamiselt minust ja temast. Firma tegeles Unineti\index{Uninet} \emph{dial-up} ühenduste edasi müümisega, tegi kodulehekülgi ja pidas isegi Infomeistri nimelist interneti infokataloogi. See viimane oli täiesti hämmastav äri. Meelis käis ja rääkis mingitele firmadele augu pähe. Mina kirjutasin firma andmed kuskil serveris asunud staatilisse (!) HTMLi. Mis kasu sellest kellelegi ammu enne otsingumootorite laia levikut tõusta võis, on mulle siiani arusaamatu. Ma ka ei mäleta, et seal lehel keegi väga käinud oleks. Ometi sealt mingi kopika sai ja ma väga loodan, et tolle tegevuse käigus antud lubadused ikka enam-vähem täidetud said. 

Kuna teadsin Steni juba varasemast ja Meelis vist ka puutus temaga kokku, lõpetasime ühel hetkel modemitega jantimise ja infokataloogi pidamise ning asusime Halo\index{Halo Interactive DDB} nime all kodulehekülgi tegema. Kampa võeti ka mõned kunstnikud (näiteks väga andekas Oliver Reitalu\index[ppl]{Reitalu, Oliver} ja mitte vähem andekas Alar Koort\index[ppl]{Koort, Alar}, keda ilmselt tema rajude elukommete tõttu Helbekeseks kutsuti) ja projektijuhiks Priit Sasi\index[ppl]{Sasi, Priit}, keda kõik tema joviaalse oleku ja suure habeme tõttu Sasuks kutsusid. Sasu õpetas mind briti punki ja Alar kurjemat sorti hiphoppi kuulama ja elu oli päris tore. Miskipärast mäletan, et minu käe alt tuli Eesti esimene kommertsalustel tehtud (st. ettevõte maksis kellelegi lehe tegemise eest raha) kodulehekülg, see sai tehtud Tartu Raadiole\index{Tartu Raadio}, kui mälu ei peta. Kunstnik joonistas pildid valmis ja lõikas tükkideks, mina kirjutasin Notepadiga HTMLi ja nii see töö käis. 

Mingil hetkel hakkasime lehekülgede tekitamist automatiseerima, kirjutasime Perli skripte. Mõnda aega ei olnud meil ei oma serverit ega üldse kuskil Perli jooksutada. Siis sai programmeeritud nii, et skript läks e-mailiga Unineti\index{Uninet} süsadminile, see kopeeris faili õigesse kohta, meie vajutasime brauseris nuppu, saime veateate, admin saatis e-mailiga konsooli veateated, mina parandasin koodi ja saatsin uue versiooni. Admini kannatus lõppes enne, kui minu oma. 

Siiski jõudsime lõpuks päris kaugele oma tegemistega. Perli skriptid läksid järjest pikemaks ja, kuna andmebaasi pidamiseks ei olnud meil serverites piisavalt õigusi, hoiti andmeid enamasti lihtsalt tekstifailis. Üllataval moel kattis see ära päris suure hulga vajadusi. Perlilt liikusime ühel hetkel PHPle ja ühel hetkel tekkis ka levinud kui seetõttu mitte vähem rumal mõte endale ise oma sisuhaldussüsteem kirjutada. See vist sai isegi valmis aga konkreetsed mälestused tollest elukast puuduvad. 

Ma ei mäleta, et see äri kuidagi tänapäevases mõistes äri moodi välja oleks näinud. Raha oli alati vähe ja nii tuli teha kõike, mille eest maksti. Kuidagi müüs Sten Ühispangale maha mõtte anda nende aastaraamat välja CDl. Mis muud, õppisime selgeks Macromedia Director'i kasutamise ja video redigeerimise ja andsime minna. Ainus asi, millega hakkama ei saanud, oli heli. Õnneks oli Sten hea sõber Lauri Liivakuga\index[ppl]{Liivak, Lauri}, kelle Forwards Studio\index{Forwards Studio} asus meiega tol ajal sama koridori peal. Lauri tegi kenad kõllid ja plõnnid ja aitas selle kõik visuaaliga ära sünkroniseerida. Tulemus sai päris kena. 

Igatahes hakkas meile järjest rohkem Tallinna kliente sigima. Samuti müüs Sten suure tüki ettevõttest Brand Sellers DDBle\index{Brand Sellers DDB}. Too oli minusugusele Tartu nohikule täiesti müstiline kamp inimesi. Intelligentsed, säravad, jõukad (nii mulle tundus) ning andekad. Bruno Lill\index[ppl]{Lill, Bruno} oma terava ütlemisega on siiani meeles.  Nii tehti kampas otsus kolida Tallinna. Olin tegelikult ligi aasta üsna kahepaikne pendeldades Tartu ja Tallinna vahel. Ülikoolis olid veel viimased sabad lõpetada ja Mari\index[ppl]{Kütt, Maria}, kellega toona juba koos elasime, käis samuti veel koolis. Lõpuks sai lõputöö kaitstud ja, kuna selliseks triviaalseks asjaks ei hakanud ju keegi Tartusse sõitma, käis Mari diplomit dekanaadist ära toomas. Prouad nõudsid allkirjastatud volitust, mis ukse taga ka kohe valmis tehtud sai ning nii omandasin ma oma esimese teaduskraadi. Tartu Ülikooli peahoone sammaste vahelt ei ole ma kunagi välja astunud ja, kuigi toonaseid õppejõude hindan siiani kõrgelt, pean oma alma materiks siiski Massachusettsi Tehnoloogiainstituuti. 

Tallinnasse kolimisega sai läbi üks etapp Halo kasvu loost. Senise boheemliku mis-võib-ikka-valesti-minna mentaliteedi asemel tuli hakata käibenumbritest rääkima. Samuti oli meeskond kasvanud. Veel Tartu päevadel olin saanud omale oma elu esimese alluva olles samal ajal ka tema esimeseks ülemuseks. Vist veel keskkooli lõpetav noor nutikas tüüp aitas mul koodi kirjutada ja hängis niisama ringi, ei mina teadnud, kuidas inimesi juhitakse või mida üks ülemus tegema peaks. Nimeks oli tüübil Taavet Hinrikus\index[ppl]{Hinrikus, Taavet}. Inimesi lisandus veelgi ja ma ei saanud enam aru, miks ja kuidas asju tehakse. Nii leidsingi ühel ilusal päeval kuskilt kuulutuse, et Hansapank\index{Hansapank} otsib oma internetipanga meeskonda inimesi. Läksin intervjuule. Mäletan siiani seda tunnet, kui Liivalaia tänava pangahoone tolle aja kohta ülišiki lifti uksed kaheksandal korrusel avanesid ja minu ees avanes hurmav vaade vanalinnale. Olin müüdud mees, õnneks arvas Vilve Vene\index[ppl]{Vene, Vilve}, kes toona arendust vedas, samuti. Nii sai minust veidi enne sajandivahetust Hansapankur. Mul vedas kohutavalt, pank oli praeguses mõistes ulmeliselt dünaamiline asutus. Vägesid juhatas Indrek Neivelt\index[ppl]{Neivelt, Indrek}. Vaata Maailma programm oli just käima minemas ja sellega tegeles Tiit Pekk\index[ppl]{Pekk, Tiit}. Marketsi tiim eesotsas Erkki Raasukesega\index[ppl]{Raasuke, Erkki} pidas ülejäänud panka talumatuteks venivillemiteks ja tootis Erik Jõgi\index[ppl]{Jõgi, Erik} juhtimisel imeilusat koodi. Aga see, nagu öeldakse, on juba üks teine jutt.
\chapter{Jaanus Lillenberg}
\chapter{Kaspar Loit}
\chapter{Tarmo Mamers}
\chapter{Tarvi Martens}
\chapter{Peeter Marvet}
\chapter{Jaan  Priisalu}
\chapter{Tanel Raja}
\chapter{Tõnis Reimo}
\chapter{Henn  Ruukel}
\chapter{Tõnu Samuel}
\chapter{Margus Sutt}
\chapter{Taavi  Talvik}
\chapter{Veiko Tamm}
\chapter{Vilve Vene}

\chapter{Võrgustiku analüüs}
\begin{itemize}
	\item Eemaldame kõik ühe seosega tipud
	\item Suured asutused konsolideerime prefixi alusel (TTÜ Arvutuskeskus -> TTÜ)
	\item Kesksed tipud (eraldi organisatsioonid ja inimesed)
	\item Degreede jaotuse graafik
\end{itemize}

\chapter{Kokkuvõte}
\begin{itemize}
	\item Paljud läksid keskkooli kõrvalt tööle, eriti kaheksakümnendatel
	\item Läbivaks jooneks on iseõppimine - ka näiteks Anto ei oska öelda, kust ta elektroonika-alase teadmise üles korjas
	\item Mitmel puhul öeldakse, et "ei programmeeri, sest nägin väga vara tõeliselt häid programmeerijaid" Järeldus: arvutihuvi tuleb teisiti realiseerida?
	\item Tundub, et mingil hetkel oli ajalehekuulutuse kaudu omale tech-teami hankimine täiesti levinud praktika (vt. Vilve, Bluemoon, Kütt)
	\item \enquote{Kõik, mis on võimalik inimese peas, on võimalik ka päriselt}. Priit Raspel, Jaan Tallinn
	\item Vene kogukond on täiesti tundmata ja, tundub, elas mingit täiesti oma elu. Kahju, nutikad inimesed, nii palju, kui näha oli
	\item Mis asi see oli, et nii Jaan kui Andres P. tegid esimese asjana tekstiredaktori? Seotud mitmel puhul läbi käinud mõttega sellest, et vanasti oli arvuti ühest küljest kättesaadavam ja teisalt vaesem. Kogu soft tuli ise kirjutada aga selle ka _sai_ ise kirjutada
\end{itemize}

%%
% The back matter contains appendices, bibliographies, indices, glossaries, etc.



\backmatter

\bibliography{sample-handout}
\bibliographystyle{plainnat}


\printindex[ppl]
\printindex

\end{document}

